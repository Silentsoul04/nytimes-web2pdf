Sections

SEARCH

\protect\hyperlink{site-content}{Skip to
content}\protect\hyperlink{site-index}{Skip to site index}

\href{https://www.nytimes.com/section/politics}{Politics}

\href{https://myaccount.nytimes.com/auth/login?response_type=cookie\&client_id=vi}{}

\href{https://www.nytimes.com/section/todayspaper}{Today's Paper}

\href{/section/politics}{Politics}\textbar{}Jared Kushner, a Trump
In-Law and Adviser, Chases a Chinese Deal

\url{https://nyti.ms/2jeNWch}

\begin{itemize}
\item
\item
\item
\item
\item
\item
\end{itemize}

Advertisement

\protect\hyperlink{after-top}{Continue reading the main story}

Supported by

\protect\hyperlink{after-sponsor}{Continue reading the main story}

\hypertarget{jared-kushner-a-trump-in-law-and-adviser-chases-a-chinese-deal}{%
\section{Jared Kushner, a Trump In-Law and Adviser, Chases a Chinese
Deal}\label{jared-kushner-a-trump-in-law-and-adviser-chases-a-chinese-deal}}

\includegraphics{https://static01.nyt.com/images/2017/01/08/us/08KUSHNER/08KUSHNER-articleLarge-v2.jpg?quality=75\&auto=webp\&disable=upscale}

By \href{http://www.nytimes.com/by/susanne-craig}{Susanne Craig},
\href{http://www.nytimes.com/by/jo-becker}{Jo Becker} and
\href{https://www.nytimes.com/by/jesse-drucker}{Jesse Drucker}

\begin{itemize}
\item
  Jan. 7, 2017
\item
  \begin{itemize}
  \item
  \item
  \item
  \item
  \item
  \item
  \end{itemize}
\end{itemize}

\href{http://cn.nytimes.com/asia-pacific/20170108/jared-kushner-trump-business/}{阅读简体中文版}

On the night of Nov. 16, a group of executives gathered in a private
dining room of the restaurant La Chine at the Waldorf Astoria hotel in
Midtown Manhattan. The table was laden with Chinese delicacies and
\$2,100 bottles of Château Lafite Rothschild. At one end sat Wu Xiaohui,
the chairman of the Waldorf's owner, Anbang Insurance Group, a
\href{https://www.nytimes.com/2016/09/02/business/dealbook/anbang-global-shopping-spree-china-mystery-ownership.html?_r=0}{Chinese
financial behemoth} with estimated assets of \$285 billion and an
\href{https://www.nytimes.com/2016/09/02/world/asia/china-anbang-insurance.html}{ownership
structure shrouded in mystery}. Close by sat Jared Kushner, a major New
York real estate investor whose father-in-law, Donald J. Trump, had just
been elected president of the United States.

It was a mutually auspicious moment.

Mr. Wu and Mr. Kushner --- who is married to Mr. Trump's daughter Ivanka
and is one of his closest advisers --- were nearing agreement on a joint
venture in Manhattan: the redevelopment of 666 Fifth Avenue, the fading
crown jewel of the Kushner family real-estate empire. Anbang, which has
close ties to the Chinese state, has seen its aggressive efforts to buy
up hotels in the United States slowed amid concerns raised by Obama
administration officials who review foreign investments for national
security risk.

Now, according to two people with knowledge of the get-together, Mr. Wu
toasted Mr. Trump and declared his desire to meet the president-elect,
whose ascension, he was sure, would be good for global business.

Since the election, intense scrutiny has been trained on Mr. Trump's
company and the
\href{https://www.nytimes.com/2016/11/26/us/politics/donald-trump-international-business.html}{potential
conflicts of interest} he will face. But with Mr. Kushner laying the
groundwork for his own White House role, the meeting at the Waldorf
shines a light on his family's multibillion-dollar business, Kushner
Companies, and on the ethical thicket he would have to navigate while
advising his father-in-law on policy that could affect his bottom line.

Unlike the
\href{https://www.nytimes.com/2016/12/25/us/politics/trump-organization-business.html}{Trump
Organization}, which has shifted its focus from acquisition to branding
of the Trump name, the Kushner family business, led by Mr. Kushner, is a
major real estate investor across the New York area and beyond. The
company has participated in roughly \$7 billion in acquisitions in the
last decade, many of them backed by opaque foreign money, as well as
financial institutions Mr. Kushner's father-in-law will soon have a hand
in regulating.

The Anbang talks, which have not previously been reported, began roughly
six months ago --- ``Well before the president-elect's victory,'' Mr.
Kushner's spokeswoman, Risa Heller, noted. That was, however, just as
Mr. Trump clinched the Republican nomination. While the talks are far
along, representatives for Mr. Kushner said some points remained
unresolved. Ms. Heller declined to outline the financial terms under
discussion.

Mr. Kushner, who declined to be interviewed for this article, has hired
a leading Washington law firm, WilmerHale, to advise him on how to
comply with federal ethics laws should he join the White House staff as
an adviser to the president. The firm has concluded that one potential
sticking point, a federal anti-nepotism law, is not applicable, though
not all ethics experts agree. While the law prohibits federal officials
from hiring relatives for agencies they lead, Mr. Kushner's lawyers
argue, among other things, that the White House is not an agency and is
therefore exempt.

As for conflicts of interest, Mr. Kushner would be required to make
limited financial disclosures, which could give the public a clearer
picture of his holdings. And, unlike Mr. Trump, who as president will be
exempt from conflict-of-interest laws, he would have to recuse himself
from decisions with a ``direct and predictable effect'' on his financial
interests.

Jamie S. Gorelick, a WilmerHale partner who served in the Clinton
administration, said that while plans were not final, Mr. Kushner was
taking significant steps to extricate himself from the family business.
``Mr. Kushner is committed to complying with federal ethics laws, and we
have been consulting with the Office of Government Ethics regarding the
steps he would take,'' she said.

He will resign as chief executive of Kushner Companies, and though the
law does not require it, she said he would divest ``substantial
assets.'' She did not name them, but Ms. Heller said they would include
his stake in 666 Fifth Avenue.

Just how meaningful that plan is remains to be seen. Mr. Kushner's
representatives declined to detail his personal financial interest in
Kushner Companies' properties, and they said he intended to keep his
interest in other properties beyond 666 Fifth Avenue. He also has a
stake, through a family investment vehicle, in a private equity firm run
by his brother, Joshua, with far-flung investments of its own.

Mr. Kushner, who turns 36 on Tuesday, has emerged as one of the most
powerful figures in Mr. Trump's orbit. Already he is involved in
steering policy, making personnel choices and serving as the middleman
between foreign leaders, the White House and the president-elect in ways
that could affect his business, even as companies like Anbang see
opportunity in entering into new ventures with the president-elect's
son-in-law.

Mr. Kushner played a pivotal role in persuading Mr. Trump, who made the
Wall Street powerhouse Goldman Sachs a bête noire of his presidential
campaign, to appoint the firm's president,
\href{https://www.nytimes.com/2016/12/09/business/dealbook/goldman-sachs-no-2-seen-as-a-top-economic-adviser-to-trump.html}{Gary
D. Cohn}, as his chief economic adviser, according to several people
involved in the transition. (Like a number of people interviewed for
this article, they spoke on the condition of anonymity because they were
not authorized to discuss internal matters.) Goldman Sachs has lent the
Kushner Companies money and is an investor in a real estate technology
company co-founded by Mr. Kushner and his brother.

\includegraphics{https://static01.nyt.com/images/2016/12/29/us/politics/04KUSHNER2/04KUSHNER2-articleLarge-v2.jpg?quality=75\&auto=webp\&disable=upscale}

Mr. Trump has said that his son-in-law, an Orthodox Jew, will
\href{https://www.nytimes.com/2016/11/23/world/middleeast/jared-kushner-cast-as-potential-player-on-israel-is-little-known-there.html}{play
a central role in dealings with Israel}, describing him as so talented
that he could help ``do peace in the Middle East.'' Mr. Kushner's
company has received multiple loans from Israel's largest bank, Bank
Hapoalim. The incoming Trump administration will inherit a
\href{http://www.reuters.com/article/us-bank-hapoalim-investigation-idUSKCN1260D7}{Justice
Department} investigation into allegations that the bank helped wealthy
Americans evade taxes.

Indeed, despite a lack of foreign policy experience, Mr. Kushner is
emerging as an important figure at a crucial moment for some of
America's most complicated diplomatic relationships. Such is his
influence in the geopolitical realm that transition officials have told
the Obama White House that foreign policy matters that need to be
brought to Mr. Trump's attention should be relayed through his
son-in-law, according to a person close to the transition and a
government official with direct knowledge of the arrangement.

So when the Chinese ambassador to the United States called the White
House in early December to express what one official called China's
``deep displeasure'' at Mr. Trump's break with longstanding diplomatic
tradition by
\href{https://www.nytimes.com/2016/12/02/us/politics/trump-speaks-with-taiwans-leader-a-possible-affront-to-china.html}{speaking
by phone with the president of Taiwan}, the White House did not call the
president-elect's national security team. Instead, it relayed that
information through Mr. Kushner, whose company was not only in the midst
of discussions with Anbang but also has Chinese investors.

Ethics experts said that while the conflict-of-interest law is narrowly
drawn, Mr. Kushner's mix of roles leads inevitably to ethical questions.

Matthew T. Sanderson, a lawyer at Caplin \& Drysdale and former general
counsel to Senator Rand Paul's presidential campaign, said deals like
the one with Anbang ``might not be illegal under the
conflict-of-interest rules, but raise a strong appearance that a foreign
entity is using Mr. Kushner's business to try to influence U.S.
policy.''

Without knowing details of Mr. Kushner's holdings and divestiture plans,
he said, the merits of his proposal are hard to assess. Even if he
divests his stake in certain properties, Mr. Sanderson added, ``it
strikes me as a half-measure'' that ``still poses a real
conflict-of-interest issue and would be a drag on Mr. Trump's presidency
and cause the American people to question Mr. Kushner's role in policy
making.''

Image

From left, Josh, Charles and Jared Kushner, in 2014. The family has
extensive holdings in real estate and private equity
firms.Credit...Patrick McMullan

\hypertarget{the-family-business}{%
\subsection{The Family Business}\label{the-family-business}}

Like the president-elect, Mr. Kushner built on the fortune of a
successful father.

In the 1980s, his father, Charles Kushner, took over the New
Jersey-based construction business started by his own father, a
Holocaust survivor from Poland. Charles expanded into office buildings
and apartments, eventually assembling a \$1 billion real estate business
and becoming
\href{http://www.nytimes.com/2004/07/15/nyregion/democratic-donor-is-known-for-short-temper-and-big-heart.html}{a
leading Democratic donor}, contributing to politicians in New Jersey and
New York and
\href{http://www.nytimes.com/2003/02/27/nyregion/mcgreevey-s-nominee-to-lead-port-authority-withdraws.html}{winning
appointment to the board} of the Port Authority of New York and New
Jersey.

But the company was upended when Charles became engulfed in a nasty
family feud over how the business's proceeds were to be distributed. The
fight, which played out in a federal courthouse in Newark, resulted in a
plea deal for Charles, who in 2005 was
\href{http://www.nytimes.com/2005/03/05/nyregion/democratic-donor-receives-twoyear-prison-sentence.html}{sentenced
to two years in prison} for tax evasion, witness tampering and making
illegal campaign donations. The family infighting was so bitter that, at
one point, Charles hired a prostitute to seduce his brother-in-law,
videotaped the encounter and sent the footage to his sister.

Jared, 23 at the time of his father's conviction, had recently graduated
from Harvard. He was studying for an M.B.A. and law degree at New York
University in 2006 when he bought The New York Observer, at the time an
influential weekly newspaper known for its coverage of the city's elite
and high-end real estate.

It is unclear exactly when he assumed control of the family business.
The company now says he became chief executive in 2008, but
contemporaneous news accounts rarely describe him that way until 2012.
Nevertheless, Mr. Kushner quickly became the company's public face as it
expanded across the Hudson River into Manhattan, much as Mr. Trump had
left Queens for the big city decades before.

Charles Kushner was released from federal custody in August 2006. He
immediately resumed a significant role in the business and remains
heavily involved today. Still, it was with Jared as headliner that the
company soon made its biggest play ever: \$1.8 billion for the
skyscraper at 666 Fifth Avenue that would remain at the center of its
story to this day. It was the highest price ever paid for a single
office building in the United States --- and more than three times what
its seller had paid six years earlier.

Around this time, Mr. Kushner met the woman he would marry: Ivanka
Trump. ``J-Vanka,''
\href{http://nymag.com/daily/intelligencer/2007/06/on_the_hunt_for_jvanka_at_the.html}{the
headlines blared}, as the New York tabloids celebrated a match made in
real estate heaven.

Everything was looking up, until suddenly it wasn't. Within a year after
the deal, the overheated lending market seized up and Kushner Companies
struggled to repay its considerable loans --- and to hold on to 666
Fifth Avenue. To the rescue over the next few years came the Carlyle
Group, a giant private equity firm; Vornado Realty Trust, then a
co-owner of two of Mr. Trump's largest properties; and Inditex, owner of
Zara, the fashion retailer founded by Amancio Ortega, the Spanish tycoon
who is one of the world's wealthiest men.

In the end, Mr. Kushner's company survived, and he and Ms. Trump became
fixtures on the international boldface-name circuit.

In August, they were spotted with Wendi Deng, an ex-wife of Rupert
Murdoch, on the 453-foot yacht Rising Sun, owned by the entertainment
mogul David Geffen. Several weeks later, they were photographed watching
the United States Open tennis finals with the art collector Dasha
Zhukova, wife of the Russian oligarch Roman Abramovich, a member of
President Vladimir V. Putin's inner circle.

Since 2012, Kushner Companies has been on a buying spree. It has
acquired at least 120 properties, mostly a mix of existing commercial
and residential buildings in New York and New Jersey, according to data
compiled by Real Capital Analytics, a research firm.

Recent deals include the \$340 million acquisition of the Jehovah's
Witnesses' headquarters in the shadow of the Brooklyn Bridge, and \$345
million for a nearby plot of undeveloped land. Mr. Kushner's company
also bought several floors of the old New York Times building for \$295
million in 2015 from Lev Leviev, an Israeli who is chairman of one of
the largest real estate development companies in Russia.

Increasingly, the company is branching out across the country --- to
Philadelphia; Baltimore; Toledo, Ohio; and Kansas City, Mo. In Chicago,
it owns the building that houses the Midwest headquarters of AT\&T. In
all, the company owns more than 20,000 apartments and approximately 14
million square feet of office space.

Image

The crowd at the men's final of the United States Open in September
included, clockwise from top left, Jared Kushner, Leon Black, Karlie
Kloss, Dasha Zhukova, Ivanka Trump, David Geffen and Wendi
Deng.Credit...Jean Catuffe/GC Images

\hypertarget{investors-and-creditors}{%
\subsection{Investors and Creditors}\label{investors-and-creditors}}

As the Kushners have expanded their businesses, they have also, by
necessity, expanded their universe of investors and creditors. Lenders
have included private equity giants like Blackstone, the French bank
Natixis and Goldman Sachs. Another lender is Deutsche Bank, which
recently
\href{https://www.nytimes.com/2016/12/22/business/dealbook/deutsche-bank-mortgage-securities-justice-department-homeowners.html}{reached
a \$7.2 billion settlement} with the Justice Department over its sale of
toxic mortgage securities. But it remains under investigation over
allegations that it disguised trades that helped Russian clients move
money offshore.

Beyond real estate, Mr. Kushner has moved into the Wall Street, health
care and tech spaces.

He has an indirect investment in Thrive Capital, a
\href{https://www.nytimes.com/2016/07/19/business/dealbook/thrive-capital-raises-700-million-for-fifth-fund.html}{venture
capital firm} valued at about \$1.5 billion that is run by his brother,
Joshua. The company has made more than 100 investments in dozens of
companies, both in the United States and abroad. Among them is
\href{https://www.nytimes.com/2016/06/20/business/struggling-for-profit-selling-health-insurance-in-state-marketplaces.html}{Oscar},
a health insurance company founded in 2012 to take advantage of the
Affordable Care Act, which Mr. Trump has vowed to dismantle. Oscar's
investors have included Li Ka-shing, who is one of Hong Kong's richest
men, and China's Ping An Insurance, which has close ties to relatives of
former Prime Minister Wen Jiabao of China.

Image

Gary D. Cohn, the president and chief operating officer of Goldman
Sachs, at Trump Tower in November. Mr. Cohn has been named Mr. Trump's
chief economic adviser.Credit...Sam Hodgson for The New York Times

The Kushner brothers have counted the Russian billionaire tech investor
Yuri Milner and the Chinese billionaire founder of Alibaba, Jack Ma, as
investors in another endeavor --- \href{https://cadre.com/}{Cadre}, a
tech-savvy real estate investment company they started with a friend.
Goldman Sachs has invested in both tech ventures.

But the money behind many of Mr. Kushner's real-estate investments
remains a mystery. While the company lists dozens of partners on its
website, it does not disclose the individuals behind those companies.

One of the newest Kushner projects --- a Trump-branded luxury apartment
tower that opened in November in Jersey City --- got nearly a quarter of
its financing, about \$50 million, from Chinese investors who are not
publicly identified.

The investors are beneficiaries of a federal program that grants
two-year visas and a path to permanent residency in exchange for
investments of \$500,000. The program, known as EB-5, has become popular
with real estate developers as a cheap form of financing; in fiscal year
2015, the State Department issued 9,764 of the visas ---
\href{https://fas.org/sgp/crs/homesec/R44475.pdf}{overwhelmingly to
applicants from China}.

But the program, which must be renewed periodically by Congress, has
lately
\href{https://www.nytimes.com/2016/03/16/us/politics/program-that-lets-foreigners-write-a-check-and-get-a-visa-draws-scrutiny.html}{come
under fire}. The Government Accountability Office has
\href{http://www.gao.gov/products/GAO-16-828}{issued several reports}
raising concerns about what it termed the program's insufficient
background checks and lax safeguards against illicit financing. One
applicant, the agency found, failed to report potential
financial\href{http://www.gao.gov/assets/680/671940.pdf}{ties to a
string of Chinese brothels}.

Then there are the Kushners' continuing negotiations with Anbang's Mr.
Wu, one of the most politically connected men in China.

Image

Kushner Companies' real estate holdings include, clockwise from top
left, 666 Fifth Avenue; the former Jehovah's Witnesses headquarters in
Brooklyn; the old New York Times building; and Trump Bay Street in
Jersey City.Credit...Photographs by Pablo Enriquez for The New York
Times, except top right by Edward Caruso for The New York Times

\hypertarget{anbang-draws-scrutiny}{%
\subsection{Anbang Draws Scrutiny}\label{anbang-draws-scrutiny}}

In 2015, Mr. Kushner began pursuing a grand vision for 666 Fifth Avenue.
The renowned architect Zaha Hadid was asked to come up with a design to
resculpt the 40-story, 1950s-era aluminum-clad office building, adding
apartments, a hotel and a mall and nearly tripling its height to 1,400
feet.

But the plan needed money, and while Mr. Kushner had managed to hang on
to his family's flagship building, it still had a lot of debt, with a
\$1.1 billion loan coming due in 2019, and a good portion of the
commercial office space vacant.

Anbang, which got its start as an auto insurance company in 2004, had
become one of the most aggressive Chinese buyers of United States real
estate, and had begun investing in hotels. But it had encountered
problems of its own; its byzantine ownership structure had given rise to
concern on Wall Street and in Washington.

\href{https://www.nytimes.com/2016/09/02/business/dealbook/anbang-global-shopping-spree-china-mystery-ownership.html}{The
Times reported last year} that Anbang is owned by a few dozen companies,
which in turn are owned by a number of shell companies that are
controlled by roughly 100 people, many of whom have ties to a county in
China that is the home of Mr. Wu, whose own power stems in part from
marriage. In his case he married Zhuo Ran, a granddaughter of Deng
Xiaoping, the leader who brought China out of the chaos of the Mao era.
Mr. Wu also counts as a central business partner the son of a People's
Liberation Army marshal, and he has recruited several former government
insurance regulators to serve on his board.

Image

Wu Xiaohui, the chairman and chief executive of Anbang Insurance, a
Chinese financial behemoth which owns the Waldorf Astoria hotel and
controls as much as \$285 billion in assets.Credit...Ben
Asen/International Insurance Society

Anbang's structure has stoked such suspicion about its true ownership
that some Wall Street firms, including Morgan Stanley, have opted not to
advise the company on United States mergers and acquisitions because
they cannot get the information needed to satisfy their ``know your
client'' guidelines.

Anbang's deep ties to the Chinese state have also led to a break in
presidential protocol. Presidents have long stayed at the Waldorf, but
when Mr. Obama visited New York for the opening of a session of the
United Nations General Assembly in September 2015, he
\href{https://www.nytimes.com/2015/09/12/us/politics/white-house-spurns-waldorf-astoria-out-of-security-concerns.html}{decided
to seek other accommodations}. American officials were vague about the
reasons for the change at the time; a senior national security official
cited security, counterintelligence and cybersurveillance concerns.

National security concerns have also complicated Anbang's efforts to
acquire other properties in the United States.

One deal, to buy the Hotel del Coronado in San Diego, fell apart in
October amid concerns from the Committee on Foreign Investment in the
United States, which comprises the heads of nine federal agencies and is
charged with reviewing the national security risks of transactions
involving foreign governments or state-connected companies. The Hotel
del Coronado is near a naval base, and deals involving proximity to
national security infrastructure typically receive heightened scrutiny.
Anbang was, however, able to acquire the other hotels in the same
collection.

Last year, Anbang tried to purchase the Starwood Hotels chain,
\href{https://www.nytimes.com/2016/03/29/business/dealbook/starwood-bidding-war-increases-with-higher-offer.html}{outbidding
Marriott with a \$14 billion offer}. It was widely reported that the
deal would be subject to review by the committee. But though the parties
expressed confidence that it would pass muster, ultimately
\href{https://www.nytimes.com/2016/04/01/business/dealbook/starwood-hotels-chinese-suitor-backs-out-of-bidding.html}{Anbang
walked away from the deal} before submitting the kind of detailed inside
information that process would entail.

And while Anbang's planned \$1.57 billion purchase of Des Moines-based
Fidelity \& Guaranty Life, first announced in November 2015, was cleared
by the committee, also known as Cfius, it
\href{http://www.nytimes.com/2016/06/02/business/dealbook/anbang-fidelity-guaranty-life.html}{stalled}
after the New York State Department of Financial Services demanded more
information about Anbang's shareholding structure.

Image

Mr. Kushner, left, met with President Obama's chief of staff, Denis
McDonough, at the White House on Nov. 10.Credit...Jim Watson/Agence
France-Presse --- Getty Images

But Anbang was nothing if not savvy. Company officials had cultivated a
relationship with Benjamin M. Lawsky, who had earlier led the financial
services agency, from May 2011 to June 2015. It was Mr. Lawsky, by then
a consultant, who introduced Anbang to Kushner Companies, according to
people with knowledge of how the discussions came about. Mr. Lawsky
declined to comment.

Mr. Kushner led the negotiations, his spokeswoman, Ms. Heller,
confirmed. Kushner Companies would disclose little else about the joint
venture, except to say that Anbang would become one of the equity
partners in the building's redevelopment if an agreement is finalized.
Anbang declined to comment.

It was just coincidence that Mr. Kushner's Nov. 16 dinner at the Waldorf
with Mr. Wu took place the week after the election, Ms. Heller said,
adding that it had been in the works for a while.

By the time of the meeting, Mr. Kushner had decided to hand off certain
business relationships, including the one with Anbang, to others at
Kushner Companies, according to Ms. Heller, and it was for that reason
that he invited his father and Laurent Morali, the president of Kushner
Companies. She said he planned to sell his stake in 666 Fifth before the
closing of any Anbang deal, but she declined to name the potential
buyers or the price Mr. Kushner hoped to get.

Ms. Heller stressed in her statement that the United States has ``not
found Anbang to be a state-owned enterprise'' --- an important technical
point, given that the Constitution's Emoluments Clause prohibits the
acceptance of payments and gifts from foreign governments.

Should it consummate its deal with Anbang, she said, Kushner Companies
will seek any necessary approvals from the federal government. She
expressed confidence that any deal would pass muster with the foreign
investment committee, citing the fact that it did not block the Chinese
company from buying the Waldorf Astoria.

Come Jan. 20, when Mr. Trump is scheduled to be inaugurated, that
committee will be made up of his cabinet members, and the process is
such that the president has the final say.

It is a process with which Mr. Trump has some familiarity. During the
campaign, he repeatedly criticized Hillary Clinton for supporting, as
secretary of state and member of the foreign investment committee, a
deal that
\href{https://www.nytimes.com/2015/04/24/us/cash-flowed-to-clinton-foundation-as-russians-pressed-for-control-of-uranium-company.html}{benefited
donors to her family's charitable foundation} while giving the Russians
control of about 20 percent of America's uranium-extraction capacity.

On China, Mr. Trump has talked a tough game, accusing Beijing of
currency manipulation and raising the possibility of a trade war. But
whether that is only a negotiating tactic remains to be seen. The
president-elect has his own financial entanglements with China: He owns
a 30 percent stake in a partnership that owes roughly \$950 million to a
group of lenders that includes the Bank of China, and one of his biggest
tenants at Trump Tower is another state-owned bank, the Industrial and
Commercial Bank of China.

With Anbang a magnet for controversy, Mr. Kushner has kept the
negotiations under wraps. But a week after the Nov. 16 dinner at the
Waldorf, Mr. Kushner's father and Mr. Wu met at the hotel for lunch.
After the elder Mr. Kushner departed, Mr. Wu was clearly elated.

``I love you guys,'' he exclaimed in English to his remaining entourage,
according to one person present.

Advertisement

\protect\hyperlink{after-bottom}{Continue reading the main story}

\hypertarget{site-index}{%
\subsection{Site Index}\label{site-index}}

\hypertarget{site-information-navigation}{%
\subsection{Site Information
Navigation}\label{site-information-navigation}}

\begin{itemize}
\tightlist
\item
  \href{https://help.nytimes.com/hc/en-us/articles/115014792127-Copyright-notice}{©~2020~The
  New York Times Company}
\end{itemize}

\begin{itemize}
\tightlist
\item
  \href{https://www.nytco.com/}{NYTCo}
\item
  \href{https://help.nytimes.com/hc/en-us/articles/115015385887-Contact-Us}{Contact
  Us}
\item
  \href{https://www.nytco.com/careers/}{Work with us}
\item
  \href{https://nytmediakit.com/}{Advertise}
\item
  \href{http://www.tbrandstudio.com/}{T Brand Studio}
\item
  \href{https://www.nytimes.com/privacy/cookie-policy\#how-do-i-manage-trackers}{Your
  Ad Choices}
\item
  \href{https://www.nytimes.com/privacy}{Privacy}
\item
  \href{https://help.nytimes.com/hc/en-us/articles/115014893428-Terms-of-service}{Terms
  of Service}
\item
  \href{https://help.nytimes.com/hc/en-us/articles/115014893968-Terms-of-sale}{Terms
  of Sale}
\item
  \href{https://spiderbites.nytimes.com}{Site Map}
\item
  \href{https://help.nytimes.com/hc/en-us}{Help}
\item
  \href{https://www.nytimes.com/subscription?campaignId=37WXW}{Subscriptions}
\end{itemize}
