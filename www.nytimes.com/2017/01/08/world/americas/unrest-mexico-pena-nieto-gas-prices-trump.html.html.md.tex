Sections

SEARCH

\protect\hyperlink{site-content}{Skip to
content}\protect\hyperlink{site-index}{Skip to site index}

\href{https://www.nytimes.com/section/world/americas}{Americas}

\href{https://myaccount.nytimes.com/auth/login?response_type=cookie\&client_id=vi}{}

\href{https://www.nytimes.com/section/todayspaper}{Today's Paper}

\href{/section/world/americas}{Americas}\textbar{}Peña Nieto Faces
Unrest in Mexico as Gas Prices Climb and Trump Ascends

\url{https://nyti.ms/2i73LUs}

\begin{itemize}
\item
\item
\item
\item
\item
\item
\end{itemize}

Advertisement

\protect\hyperlink{after-top}{Continue reading the main story}

Supported by

\protect\hyperlink{after-sponsor}{Continue reading the main story}

\hypertarget{peuxf1a-nieto-faces-unrest-in-mexico-as-gas-prices-climb-and-trump-ascends}{%
\section{Peña Nieto Faces Unrest in Mexico as Gas Prices Climb and Trump
Ascends}\label{peuxf1a-nieto-faces-unrest-in-mexico-as-gas-prices-climb-and-trump-ascends}}

\includegraphics{https://static01.nyt.com/images/2017/01/09/world/09MEXICO3/09MEXICO3-articleLarge.jpg?quality=75\&auto=webp\&disable=upscale}

By \href{http://www.nytimes.com/by/kirk-semple}{Kirk Semple} and
\href{https://www.nytimes.com/by/elisabeth-malkin}{Elisabeth Malkin}

\begin{itemize}
\item
  Jan. 8, 2017
\item
  \begin{itemize}
  \item
  \item
  \item
  \item
  \item
  \item
  \end{itemize}
\end{itemize}

MEXICO CITY --- Amid nationwide marches, highway blockades and looting
stemming from widespread outrage over an increase in gas prices,
President Enrique Peña Nieto of Mexico went on national television to
appeal for understanding.

With international oil prices rising and Mexico dependent on gasoline
imports, he argued in the speech on Thursday, the government had no
alternative but to raise prices at the pump. ``Here I ask you,'' he
said, gesturing at the camera, ``what would you have done?''

It did not take long for him to get an answer, as social media erupted
with suggestions and disgust.

Combat corruption and impunity. Eliminate gasoline vouchers for elected
officials. Collect more taxes from multinational corporations. Cut the
salaries and benefits of high-level government officials. Sell the
presidential plane. Reduce the first lady's wardrobe spending. Resign.

It was a tough week for the president, who seems to be trapped in a
slow, downward spiral of unpopularity, with two more years left in his
term and Mexico reeling from myriad problems including rampant
corruption, resurgent homicide rates, a thriving drug trafficking
industry, a sluggish economy and a plummeting peso.

The few voices of support for Mr. Peña Nieto --- in political circles
and among news commentators --- have been drowned out by his detractors,
and no more so than in the past week, when discontent over the gas price
increase boiled over into protests and looting, setting off clashes with
security forces that left several dead around the country.

The unrest comes as Mexico braces for the administration of
President-elect Donald J. Trump, who has threatened to introduce far
more restrictive immigration and trade policies, including canceling the
North American Free Trade Agreement, increasing deportations and
building a wall on the southern border of the United States.

Concern in Mexico about Mr. Trump's planned tack on trade has been so
great that he has been able to move the markets on the basis of his
Twitter posts.

The Mexican peso hit record lows last week after he
\href{https://twitter.com/realDonaldTrump/status/816260343391514624}{criticized
General Motors on Twitter} for exporting cars made in Mexico and Ford
Motors announced that it would cancel plans for a \$1.6 billion plant in
the country. Mexico's Central Bank was forced to intervene to bolster
the peso, but the currency took another hit after Mr. Trump
\href{https://twitter.com/realDonaldTrump/status/817071792711942145}{threatened
Toyota on Thursday} with a ``big border tax'' if it went ahead with a
new factory in Mexico.

Mexico's Economy Ministry issued a brief statement in response saying
the government ``rejects any attempt to influence investment decisions
by companies based on fear or threats.''

\includegraphics{https://static01.nyt.com/images/2017/01/09/world/09MEXICO2/09MEXICO2-articleLarge.jpg?quality=75\&auto=webp\&disable=upscale}

But in general the Peña Nieto administration seems to be struggling to
figure out how to respond to Mr. Trump. Mexicans have been clamoring for
a full-throated, chest-out defense of their country and sovereignty
against Mr. Trump's threats, but many say they have yet to hear it.

Confidence in Mr. Peña Nieto is so low --- approval ratings have sunk
below 25 percent --- that he appears to be struggling to sell anything
to the public, most recently the gas price increase last week.

``Such a low level of popularity reduces his capacity to gather support
or his margin for action to reactivate the economy,'' said Ignacio
Marván, a political analyst at CIDE, a Mexico City university.

Mr. Peña Nieto's efforts have been handicapped, analysts say, by a
seeming disconnect from the public mood.

The government looked unprepared for the violent responses to the price
increases, which took effect on New Year's Day, when most officials were
on vacation. Mr. Peña Nieto himself was in the middle of a golfing trip.
And as bloody unrest swept across the country, the president kept
silent, finally making a public statement on the issue on Wednesday.

Even then, his comments were buried in a news conference focused on
cabinet changes that included
\href{http://www.nytimes.com/2017/01/04/world/americas/mexico-united-states-trump-pena-nieto-videgaray.html}{the
return of Luis Videgaray}, a close confidant
\href{http://www.nytimes.com/2016/09/08/world/americas/mexico-finance-minister-luis-videgaray-resigns.html?_r=0}{who
resigned under pressure} as finance minister in September after
championing an unpopular visit by Mr. Trump to Mexico.

The administration's detached response to the upheaval contributed to
the impression of a president out of touch with the population, analysts
said, and gave a sense of a leadership that is adrift, blindsided by
events.

The gas price increases of about 20 percent are part of a broad overhaul
that ends the state's monopoly over the energy industry. The government
has long controlled and subsidized gasoline prices, but by the end of
the year it will allow gas prices to fluctuate according to the market,
a move intended to attract foreign investment to compete with the state
oil company, Pemex.

The government has argued that ending fuel subsidies will help the
country avoid spending cuts to social programs, and that the subsidies
have disproportionately benefited wealthier Mexicans who own cars. But
many fear that higher gasoline prices will increase costs for food and
public transportation, hitting the pocketbooks of even the poorest
Mexicans.

Though Mexico's opposition parties are now condemning the price
increase, most of them voted for it as part of the budget approved in
October. But Mexico imports more than half of its gasoline from the
United States, and Mr. Trump's election sent the peso to a historic low,
raising the price of imported gasoline in pesos greater than anybody
expected.

Analysts said the government could have forestalled the fallout by
designing measures that would have softened the blow for poorer
Mexicans, or by creating subsidies for truck drivers or owners of older
vehicles.

``They didn't think about it,'' said Vidal Romero, a political analyst
at the Autonomous Technological Institute of Mexico. ``There is no
compensation for citizens.''

Image

Residents stealing gasoline and diesel amid protests against an increase
in fuel prices in Allende, southern Veracruz State, Mexico, last
week.Credit...Erick Herrera/Associated Press

Ignited by the gas price increase, but fueled by broader discontent with
the government and uncertainty about the country's direction, citizens
took to the streets, staging marches throughout the country and blocking
key highways.

Criminals also used the cover of the protests to break into stores and
malls to strip shelves bare of home appliances, electronics, food and
toys.

By last weekend, hundreds of stores had been looted around the country
and more than 1,000 people detained, the authorities said, and at least
six people had been killed in clashes between looters and the police.

The administration has rejected calls to rescind the price increase, and
the ruling Institutional Revolutionary Party traded accusations with
opposition groups about responsibility for fomenting the disorder.

With Mr. Peña Nieto's credibility so diminished, it will be impossible
for the president to accomplish much before the 2018 presidential
election, analysts said. He is not eligible to run again.

The contrast with the early days of Mr. Peña Nieto's presidency is
remarkable. When he took office four years ago, his sharp political
instincts helped push a bold reform agenda through Congress with support
from the opposition. The reforms included the opening of the
traditionally closed energy sector to foreign investment, an exceptional
political accomplishment given the Mexican public's view of the oil
industry as a bedrock of the national patrimony.

But now those instincts appear to have been eroded by scandal and
mismanagement, and perhaps by an insulation from emotions on the street.

``He has lost his feeling for politics,'' Mr. Romero said.

The president's perceived weaknesses and his low approval ratings have
opened up space for opposition groups and cast the future of his party's
influence into doubt. Even the Institutional Revolutionary Party's
longstanding dominance in the president's home state, the populous State
of Mexico surrounding Mexico City, has been thrown into question, with
the state's governorship up for grabs this summer.

Perhaps the biggest political beneficiary of Mr. Peña Nieto's declining
popularity has been the populist politician Andrés Manuel López Obrador,
a former mayor of Mexico City, who is leading in polls of potential
candidates for the 2018 presidential contest. With each misstep of the
president, Mr. López Obrador seems to become even more popular; some
political observers refer to Mr. Peña Nieto as Mr. López Obrador's
``campaign manager.''

The looting and criminal unrest across the country had subsided by the
weekend, but protests continued, with thousands taking to the streets in
largely peaceful marches.

``We don't want this corrupt country any more,'' said Alicia Rios, 32, a
receptionist who joined thousands of protesters for a march through
downtown Mexico City on Saturday. ``The legislators get 10,000 pesos in
gasoline vouchers when the people can't afford to fill up their tanks.''

She added, ``If gasoline goes up, everything goes up.''

Advertisement

\protect\hyperlink{after-bottom}{Continue reading the main story}

\hypertarget{site-index}{%
\subsection{Site Index}\label{site-index}}

\hypertarget{site-information-navigation}{%
\subsection{Site Information
Navigation}\label{site-information-navigation}}

\begin{itemize}
\tightlist
\item
  \href{https://help.nytimes.com/hc/en-us/articles/115014792127-Copyright-notice}{©~2020~The
  New York Times Company}
\end{itemize}

\begin{itemize}
\tightlist
\item
  \href{https://www.nytco.com/}{NYTCo}
\item
  \href{https://help.nytimes.com/hc/en-us/articles/115015385887-Contact-Us}{Contact
  Us}
\item
  \href{https://www.nytco.com/careers/}{Work with us}
\item
  \href{https://nytmediakit.com/}{Advertise}
\item
  \href{http://www.tbrandstudio.com/}{T Brand Studio}
\item
  \href{https://www.nytimes.com/privacy/cookie-policy\#how-do-i-manage-trackers}{Your
  Ad Choices}
\item
  \href{https://www.nytimes.com/privacy}{Privacy}
\item
  \href{https://help.nytimes.com/hc/en-us/articles/115014893428-Terms-of-service}{Terms
  of Service}
\item
  \href{https://help.nytimes.com/hc/en-us/articles/115014893968-Terms-of-sale}{Terms
  of Sale}
\item
  \href{https://spiderbites.nytimes.com}{Site Map}
\item
  \href{https://help.nytimes.com/hc/en-us}{Help}
\item
  \href{https://www.nytimes.com/subscription?campaignId=37WXW}{Subscriptions}
\end{itemize}
