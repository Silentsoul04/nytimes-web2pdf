Sections

SEARCH

\protect\hyperlink{site-content}{Skip to
content}\protect\hyperlink{site-index}{Skip to site index}

\href{https://www.nytimes.com/section/politics}{Politics}

\href{https://myaccount.nytimes.com/auth/login?response_type=cookie\&client_id=vi}{}

\href{https://www.nytimes.com/section/todayspaper}{Today's Paper}

\href{/section/politics}{Politics}\textbar{}Trump Received
Unsubstantiated Report That Russia Had Damaging Information About Him

\url{https://nyti.ms/2jsp4xR}

\begin{itemize}
\item
\item
\item
\item
\item
\item
\end{itemize}

Advertisement

\protect\hyperlink{after-top}{Continue reading the main story}

Supported by

\protect\hyperlink{after-sponsor}{Continue reading the main story}

\hypertarget{trump-received-unsubstantiated-report-that-russia-had-damaging-information-about-him}{%
\section{Trump Received Unsubstantiated Report That Russia Had Damaging
Information About
Him}\label{trump-received-unsubstantiated-report-that-russia-had-damaging-information-about-him}}

\includegraphics{https://static01.nyt.com/images/2017/01/11/us/11intel/11intel-articleInline.jpg?quality=75\&auto=webp\&disable=upscale}

By \href{http://www.nytimes.com/by/scott-shane}{Scott Shane},
\href{https://www.nytimes.com/by/adam-goldman}{Adam Goldman} and
\href{http://www.nytimes.com/by/matthew-rosenberg}{Matthew Rosenberg}

\begin{itemize}
\item
  Jan. 10, 2017
\item
  \begin{itemize}
  \item
  \item
  \item
  \item
  \item
  \item
  \end{itemize}
\end{itemize}

WASHINGTON --- The chiefs of America's intelligence agencies last week
presented President Obama and President-elect
\href{http://www.nytimes.com/topic/person/donald-trump?inline=nyt-per}{Donald
J. Trump} with a summary of unsubstantiated reports that Russia had
collected compromising and salacious personal information about Mr.
Trump, two officials with knowledge of the briefing said.

The summary is based on memos generated by political operatives seeking
to derail Mr. Trump's candidacy. Details of the reports began
circulating in the fall and were widely known among journalists and
politicians in Washington.

The two-page summary, first
\href{http://edition.cnn.com/2017/01/10/politics/donald-trump-intelligence-report-russia/index.html}{reported}
by CNN, was presented as an appendix to the intelligence agencies'
\href{https://www.nytimes.com/2017/01/06/us/politics/donald-trump-wall-hack-russia.html}{report
on Russian hacking efforts during the election}, the officials said. The
material was not corroborated, and The New York Times has not been able
to confirm the claims. But intelligence agencies considered it so
potentially explosive that they decided Mr. Obama, Mr. Trump and
congressional leaders needed to be told about it and informed that the
agencies were actively investigating it.

Intelligence officials were concerned that the information would leak
before they informed Mr. Trump of its existence, said the officials, who
spoke on the condition of anonymity because they were not authorized to
speak about it publicly.

The author of the memos is Christopher Steele, a former British
intelligence officer with MI-6, who once served in Moscow. After Mr.
Steele retired in 2009, he founded a private firm called
\href{https://orbisbi.com/}{Orbis Business Intelligence} in London.
Former C.I.A. officials described him as an expert on Russia who is well
respected in the spy world.

On Tuesday night, Mr. Trump responded to the memos on Twitter:

In an appearance recorded for NBC's ``Late Night With Seth Meyers,'' Mr.
Trump's spokeswoman, Kellyanne Conway, said of the claims in the
opposition research memos, ``He has said he is not aware of that.''

On Wednesday, a spokesman for President Vladimir V. Putin of Russia
dismissed the allegations. ``The Kremlin has no compromising dossier on
Trump, such information isn't consistent with reality and is nothing but
an absolute fantasy,'' the spokesman, Dmitri S. Peskov, said at a news
conference.

Since the intelligence agencies' report on Friday that Mr. Putin of
Russia had ordered the hacking and leaks of Democratic emails in order
to hurt his Democratic opponent, Hillary Clinton, and help Mr. Trump,
the president-elect and his aides have said that Democrats are trying to
mar his election victory.

The decision of top intelligence officials to give the president, the
president-elect and the so-called Gang of Eight --- Republican and
Democratic leaders of Congress and the intelligence committees --- what
they know to be unverified, defamatory material was extremely unusual.

The appendix summarized opposition research memos prepared mainly by a
retired British intelligence operative for a Washington political and
corporate research firm. The firm was paid for its work first by Mr.
Trump's Republican rivals and later by supporters of Mrs. Clinton. The
Times has checked on a number of the details included in the memos but
has been unable to substantiate them.

The memos suggest that for many years, the Russian government of Mr.
Putin has looked for ways to influence Mr. Trump, who has traveled
repeatedly to Moscow to investigate real estate deals or to oversee the
Miss Universe competition, which he owned for several years. Mr. Trump
never completed any major deals in Russia, though he discussed them for
years.

Mr. Steele, who gathered the material about Mr. Trump, is considered a
competent and reliable operative with extensive experience in Russia,
American officials said. But he passed on what he heard from Russian
informants and others, and what they told him has not yet been vetted by
American intelligence.

The memos describe sex videos involving prostitutes with Mr. Trump in a
2013 visit to a Moscow hotel. The videos were supposedly prepared as
``kompromat,'' or compromising material, with the possible goal of
blackmailing Mr. Trump in the future.

The memos also suggest that Russian officials proposed various lucrative
deals, essentially as disguised bribes in order to win influence over
Mr. Trump.

The memos describe several purported meetings during the 2016
presidential campaign between Trump representatives and Russian
officials to discuss matters of mutual interest, including the Russian
hacking of the Democratic National Committee and Mrs. Clinton's campaign
chairman, John D. Podesta.

If some of the unproven claims in the memos are merely titillating,
others would amount to extremely serious, potentially treasonous acts.

One of the opposition research memos quotes an unidentified Russian
source as claiming that the hacking and leaking of Democratic emails was
carried out ``with the full knowledge and support of TRUMP and senior
members of his campaign team.'' In return, the memo said, ``the TRUMP
team had agreed to sideline Russian intervention in Ukraine as a
campaign issue'' because Mr. Putin ``needed to cauterize the subject.''

Michael Cohen, a lawyer and adviser to Mr. Trump, also went to Twitter
to deny a specific claim in the opposition research involving him. One
of the memos claims that Mr. Cohen went to Prague in August or September
to meet with Kremlin representatives and to talk about Russian hacking
of Democrats.

Mr. Cohen tweeted on Tuesday night:

In addition, in a recent interview with The Times, one of the Russian
officials named in the memo as having met with Mr. Cohen, Oleg
Solodukhin, denied that he had met with Mr. Cohen or any other Trump
representative.

``I don't know where that rumor came from,'' Mr. Solodukhin, of the
Russian organization Rossotrudnichestvo, which promotes Russian culture
and interests abroad, said in a telephone interview.

The Times
\href{https://www.nytimes.com/2016/11/01/us/politics/fbi-russia-election-donald-trump.html}{reported
before the election} that the F.B.I. was looking into possible evidence
of links between the Trump campaign and Russia. But the investigation
surfaced again at a Senate hearing on Tuesday in a series of questions
from Senator Ron Wyden, Democrat of Oregon, to the F.B.I. director,
James B. Comey.

Mr. Wyden, trying to draw Mr. Comey out on information he may have heard
during a classified briefing, asked if the F.B.I. had investigated the
Trump campaign's contacts with Russia. Mr. Comey demurred, saying he
could not discuss any investigations that might or might not be
underway. Mr. Wyden kept pressing, asking Mr. Comey to provide a written
answer to the question before Mr. Trump's inauguration on Jan. 20
because he feared there would be no declassification of the information
once Mr. Trump took office.

After the hearing, Mr. Wyden posted on Twitter:

The F.B.I. obtained the material long before the election, and some of
the memos in the opposition research dossier are dated as early as June.
But agents have struggled to confirm it, according to federal officials
familiar with the investigation.

Allies of Senator Harry Reid, the Senate Democratic leader from Nevada
who retired at the end of the year, said the disclosures validated his
call last summer for an investigation by the F.B.I. into Mr. Trump's
links to Russia.

Democrats on Tuesday night pressed for a thorough investigation of the
claims in the memos. Representative Eric Swalwell of California, a
member of the House Intelligence Committee, called for law enforcement
to find out whether the Russian government had had any contact with Mr.
Trump or his campaign.

``The president-elect has spoken a number of times, including after
being presented with this evidence, in flattering ways about Russia and
its dictator,'' Mr. Swalwell said. ``Considering the evidence of Russia
hacking our democracy to his benefit, the president-elect would do a
service to his presidency and our country by releasing his personal and
business income taxes, as well as information on any global financial
holdings.''

Advertisement

\protect\hyperlink{after-bottom}{Continue reading the main story}

\hypertarget{site-index}{%
\subsection{Site Index}\label{site-index}}

\hypertarget{site-information-navigation}{%
\subsection{Site Information
Navigation}\label{site-information-navigation}}

\begin{itemize}
\tightlist
\item
  \href{https://help.nytimes.com/hc/en-us/articles/115014792127-Copyright-notice}{©~2020~The
  New York Times Company}
\end{itemize}

\begin{itemize}
\tightlist
\item
  \href{https://www.nytco.com/}{NYTCo}
\item
  \href{https://help.nytimes.com/hc/en-us/articles/115015385887-Contact-Us}{Contact
  Us}
\item
  \href{https://www.nytco.com/careers/}{Work with us}
\item
  \href{https://nytmediakit.com/}{Advertise}
\item
  \href{http://www.tbrandstudio.com/}{T Brand Studio}
\item
  \href{https://www.nytimes.com/privacy/cookie-policy\#how-do-i-manage-trackers}{Your
  Ad Choices}
\item
  \href{https://www.nytimes.com/privacy}{Privacy}
\item
  \href{https://help.nytimes.com/hc/en-us/articles/115014893428-Terms-of-service}{Terms
  of Service}
\item
  \href{https://help.nytimes.com/hc/en-us/articles/115014893968-Terms-of-sale}{Terms
  of Sale}
\item
  \href{https://spiderbites.nytimes.com}{Site Map}
\item
  \href{https://help.nytimes.com/hc/en-us}{Help}
\item
  \href{https://www.nytimes.com/subscription?campaignId=37WXW}{Subscriptions}
\end{itemize}
