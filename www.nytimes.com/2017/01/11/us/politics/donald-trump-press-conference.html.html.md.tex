Sections

SEARCH

\protect\hyperlink{site-content}{Skip to
content}\protect\hyperlink{site-index}{Skip to site index}

\href{https://www.nytimes.com/section/politics}{Politics}

\href{https://myaccount.nytimes.com/auth/login?response_type=cookie\&client_id=vi}{}

\href{https://www.nytimes.com/section/todayspaper}{Today's Paper}

\href{/section/politics}{Politics}\textbar{}Trump Says `I Think It Was
Russia' That Hacked the Democrats

\url{https://nyti.ms/2juiqHk}

\begin{itemize}
\item
\item
\item
\item
\item
\end{itemize}

Advertisement

\protect\hyperlink{after-top}{Continue reading the main story}

Supported by

\protect\hyperlink{after-sponsor}{Continue reading the main story}

Transition Briefing

\hypertarget{trump-says-i-think-it-was-russia-that-hacked-the-democrats}{%
\section{Trump Says `I Think It Was Russia' That Hacked the
Democrats}\label{trump-says-i-think-it-was-russia-that-hacked-the-democrats}}

By \href{http://www.nytimes.com/by/michael-d-shear}{Michael D. Shear},
\href{http://www.nytimes.com/by/jonathan-weisman}{Jonathan Weisman} and
\href{http://www.nytimes.com/by/matthew-rosenberg}{Matthew Rosenberg}

\begin{itemize}
\item
  Jan. 11, 2017
\item
  \begin{itemize}
  \item
  \item
  \item
  \item
  \item
  \end{itemize}
\end{itemize}

■ For the first time, President-elect Donald J. Trump concedes Russia
probably meddled in the election.

■ Mr. Trump's evocation of Nazi persecution prompts a fierce blowback.

■ Representative Mark Sanford, a Republican, to the president-elect: I
too want to see your tax returns.

■ Mr. Trump renews vow to empower the federal government to negotiate
drug prices, breaking with Republican orthodoxy.

\includegraphics{https://static01.nyt.com/images/2017/01/12/us/12transitionbriefing-ALTSUBALT/12transitionbriefing-ALTSUBALT-articleLarge.jpg?quality=75\&auto=webp\&disable=upscale}

\hypertarget{trump-concedes-russia-behind-election-interference}{%
\subsection{Trump concedes Russia behind election
interference}\label{trump-concedes-russia-behind-election-interference}}

President-elect Trump conceded for the first time that Russia was behind
the hacking of Democrats during the presidential election, saying at a
news conference that, ``I think it was Russia'' --- though a few minutes
later he said perhaps it was another country.

But he vigorously denied the swirl of allegations about his behavior
that was published online Tuesday, calling it ``fake news'' and praising
Russian President Vladimir V. Putin for saying it was false. ``I
respected the fact that he said that,'' Mr. Trump told reporters.

``It's all fake news. Its phony stuff. It didn't happen,'' Mr. Trump
said. ``It was gotten by opponents of ours. It was a group of opponents
that got together. Sick people and they got together and put that crap
together.''

Asked what Mr. Trump would say to Mr. Putin about the election hacking,
he responded, ``He shouldn't have done it. I don't believe he will be
doing it more.''

Mr. Trump disputed the allegations that he was involved in salacious
behavior in a Russian hotel room, saying that he is always highly aware
that there are ``cameras in the strangest places'' in hotel rooms when
he travels around the world.

``You can't see them and you won't know,'' he said. ``You better be
careful or you will be watching yourself on nightly televisions.''

\includegraphics{https://static01.nyt.com/images/2017/01/11/us/12TRUMP5-hp/12TRUMP5-hp-videoSixteenByNineJumbo1600-v5.jpg}

\hypertarget{trumps-evocation-of-nazism-provokes-blowback}{%
\subsection{Trump's evocation of Nazism provokes
blowback}\label{trumps-evocation-of-nazism-provokes-blowback}}

President-elect Trump should have known that comparing himself to a
victim of Nazi persecution would not go over well --- and it didn't.

On Wednesday morning, amid a furor over the publishing of an
unsubstantiated, explosive dossier, Mr. Trump posted on Twitter:

In 140 characters, he compared American intelligence agencies to Nazis,
called himself a target, and asked whether his travails were comparable
to those of the victims of Adolf Hitler.

The response was swift from organizations that combat anti-Semitism and
guard the memories of Nazi victims.

``It is a despicable insult to Holocaust survivors around the world, and
to the nation he is about to lead, that Donald Trump compares America to
Nazi Germany. The president-elect has denigrated our nation and its
commitment to freedom on the eve of his inauguration. He must retract
his tweet and apologize to survivors and to our entire nation,'' said
Steven Goldstein, executive director of The Anne Frank Center for Mutual
Respect, the American wing of an international organization that fights
discrimination in Anne Frank's name.

The Anti-Defamation League also joined the denunciations.

The blowback was a reminder to Mr. Trump of what is known as
\href{https://en.m.wikipedia.org/wiki/Godwin's_law}{Godwin's Law}: ``As
an online discussion grows longer, the probability of a comparison
involving Hitler approaches.'' And once it happens, the conversation
generally stops.

\hypertarget{it-wasnt-particularly-presidential-but-it-was-surprising}{%
\subsection{It wasn't particularly presidential but it was
surprising}\label{it-wasnt-particularly-presidential-but-it-was-surprising}}

\includegraphics{https://static01.nyt.com/images/2017/01/11/us/trump-cnn/trump-cnn-videoSixteenByNine3000.jpg}

President-elect Trump angrily accused CNN of being ``fake news'' as he
repeatedly refused to take a question from Jim Acosta, the network's
reporter at his news conference.

In a testy exchange between the two men, Mr. Acosta repeatedly asked for
a question in order to respond to the president-elect's accusations
against the network. Mr. Trump angrily said, ``I am not going to give
you a question. You are fake news.''

Mr. Trump also lashed out at
\href{https://www.buzzfeed.com/news}{Buzzfeed News}, which published a
document of allegations Tuesday night, calling the news organization a
``failing pile of garbage'' and saying that ``I think they are going to
suffer the consequences.''

\hypertarget{no-mr-trump-it-isnt-only-reporters-who-are-interested-in-your-tax-returns}{%
\subsection{No, Mr. Trump, it isn't only reporters who are interested in
your tax
returns}\label{no-mr-trump-it-isnt-only-reporters-who-are-interested-in-your-tax-returns}}

During his news conference, the president-elect was pressed to release
his tax returns to prove that his businesses have no relationships with
Russian oligarchs. Mr. Trump refused, once again, and then made a novel
claim: ``You know, the only one that cares about my tax returns are the
reporters, O.K.?''

Fact check: False.

Source: Representative Mark Sanford, Republican of South Carolina. When
he was that state's governor, Mr. Sanford famously disappeared to
Argentina to pursue an extramarital affair under the guise of hiking the
Appalachian Trail.

\begin{quote}
``I think it's a grave mistake for President-elect Trump to assume that
no one - except reporters - cares about whether or not he releases his
tax returns.

``I care - or more aptly said, I still care. I cared five months ago
when I wrote
\href{https://www.nytimes.com/2016/08/15/opinion/i-support-you-donald-trump-now-release-your-tax-returns.html}{an
op-ed published
in}\emph{\href{https://www.nytimes.com/2016/08/15/opinion/i-support-you-donald-trump-now-release-your-tax-returns.html}{The
New York Times}} on this very issue. And I care now because the
consequences of not doing so will echo into future campaigns.

``This is not about invading his privacy or looking for avenues into
opposition research. This is about upholding precedent and encouraging
transparency. Refusing to release his tax returns now will likely lead
to much the same from future presidential candidates, and I predict that
a trickle-down effect will occur where down-ballot candidates won't feel
pressured to do so either.''
\end{quote}

When Mr. Sanford gets to lecture on ethics, well, that's notable.

\hypertarget{yes-emirates-really-did-offer-trump-a-2-billion-deal}{%
\subsection{Yes, Emirates really did offer Trump a \$2 billion
deal}\label{yes-emirates-really-did-offer-trump-a-2-billion-deal}}

The Trump Organization may not be looking for new foreign deals, but
that doesn't mean big businesses overseas aren't looking to hook up with
the president-elect's family business.

At his news conference on Wednesday, Mr. Trump said that
\href{https://www.damacproperties.com/en}{Damac}, a large developer in
the United Arab Emirates, had recently tried to reach a \$2 billion deal
with the Trump Organization, which declined the offer.

\href{https://www.nytimes.com/interactive/2017/01/11/us/politics/fact-check-trump-press-conference.html}{}

\includegraphics{https://static01.nyt.com/images/2017/01/11/us/politics/fact-check-trump-press-conference-1484112495667/fact-check-trump-press-conference-1484112495667-thumbLarge-v5.jpg}

\hypertarget{fact-check-trumps-first-news-conference-since-the-election}{%
\subsection{Fact Check: Trump's First News Conference Since the
Election}\label{fact-check-trumps-first-news-conference-since-the-election}}

Reporters from The New York Times fact checked President-elect Donald J.
Trump's news conference in real time.

Niall McLoughlin, a senior vice president at Damac, confirmed Mr.
Trump's account, saying in an email that ``the discussions took place as
stated in the media briefing.''

``But the proposals were declined. These proposals were for a variety of
different properties deals,'' he added, without elaborating.

Damac is a major developer in the Persian Gulf region. It is building
the Trump International Golf Club Dubai and an adjacent luxury housing
development, which does not bear the Trump name. The project has been in
the works since 2013, with the Trump Organization managing the golf club
after it opens, scheduled for February.

Damac has some ties to the Emirati government --- any big business in
the Emirates does --- but its founder, Hussain Sajwani, is not part of
the social and political elite. His father owned a small shop in Dubai's
market that sold midrange watches and pens, and the family was solidly
middle class, and even today Mr. Sajwani remains an outsider among the
tightly knit old families that control the Emirates.

In many ways, Mr. Sajwani is a lot like Mr. Trump. He has a flair for
showmanship --- he once tried to attract buyers for a new development by
offering a free Bentley with every apartment --- and his best customers
tend to be rich outsiders who are trying to buy their way into the
elite, like moneyed Iranians and well-off Indians.

\hypertarget{this-time-big-pharma-gets-to-panic}{%
\subsection{This time, Big Pharma gets to
panic}\label{this-time-big-pharma-gets-to-panic}}

Admittedly, the president-elect was a little vague, but for the nation's
pharmaceutical giants, he was clear enough: The drug industry, Mr. Trump
said, is ``getting away with murder.''

``Pharma has a lot of lobbies and a lot of lobbyists and a lot of power,
and there's very little bidding on drugs. We're the largest buyer of
drugs in the world and yet we don't bid properly and we're going to
start bidding and we're going to save billions of dollars over a period
of time,'' he said.

During the early months of the campaign, before less substantive issues
swamped the debate, Mr. Trump broke with his party and said the federal
government should be using its purchasing power to negotiate drug prices
for Medicare and Medicaid.

Democrats have been pressing that for years --- it was in an early
version of the Affordable Care Act. But Republicans, backed by the drug
industry, have adamantly opposed, and thwarted, every effort to give the
Centers for Medicare and Medicaid Services that authority.

But the Republican Party of Mr. Trump has found a way to change sides on
a lot of issues.

\hypertarget{trump-nominates-veterans-affairs-secretary}{%
\subsection{Trump nominates Veterans Affairs
secretary}\label{trump-nominates-veterans-affairs-secretary}}

Image

Dr. David J. Shulkin at Trump Tower in Manhattan on
Monday.Credit...Kevin Hagen for The New York Times

David J. Shulkin, the current undersecretary at the Department of
Veterans Affairs and the head of its sprawling hospital system, will
become the secretary of the beleaguered agency, President-elect Trump
said Wednesday.

Mr. Trump made the comments during his first full news conference since
being elected president two months ago and just nine days before moving
into the White House. He answered questions from some of the more than
250 reporters crammed into the lobby of Trump Tower in Manhattan.

Even as Mr. Trump took questions in New York, his nominee for secretary
of state was being grilled on Capitol Hill by lawmakers about his
connections to Russia. And news continued to swirl regarding allegations
about Mr. Trump presented by intelligence officials to the
president-elect.

\hypertarget{a-defense-on-twitter}{%
\subsection{A defense on Twitter}\label{a-defense-on-twitter}}

From the moment the
\href{https://www.nytimes.com/2017/01/10/us/politics/donald-trump-russia-intelligence.html}{unsubstantiated
but explosive intelligence report} hit the internet, the questions
arose: When and what would Mr. Trump post on Twitter?

The initial volley came shortly after 8:15 p.m. on Tuesday.

Then on Wednesday morning, the president-elect decided to press his war
with the intelligence agencies he will soon oversee --- and rely on for
pressing matters of state.

He also seconded reports from Moscow that Russian government officials
deny having any compromising material on Mr. Trump.

It is true that the Kremlin denied holding any material that it could
use to blackmail the incoming president. Dmitri S. Peskov, a spokesman
for President Vladimir V. Putin, told reporters: ``The Kremlin has no
compromising dossier on Trump. Such information isn't consistent with
reality and is nothing but an absolute fantasy.''

Needless to say, Russian officials would not tell the world they had
such information if it was meant for blackmail. Also needless to say, an
election won with nearly three million votes fewer than your opponent
was not won easily.

A
\href{https://poll.qu.edu/national/release-detail?ReleaseID=2415}{Quinnipiac
University poll} on Tuesday put Mr. Trump's approval rating at 37
percent.

It should be an interesting news conference at Trump Tower on Wednesday
morning.

\hypertarget{a-problem-for-trump-you-cant-unsee-something}{%
\subsection{A problem for Trump: You can't unsee
something}\label{a-problem-for-trump-you-cant-unsee-something}}

Questions continue to swirl about the validity of the dossier saying
that Russian intelligence has deeply compromising material on the man
who is about to be commander in chief. But its wide circulation has
already had an impact.

On Tuesday night, Representative Brad Sherman, Democrat of California,
\href{http://sherman.house.gov/sites/sherman.house.gov/files/January\%2010\%20-\%20Letter\%20to\%20Obama.pdf}{asked
President Obama} for a classified briefing on ``the compromising
information Russia has obtained regarding President-elect Donald
Trump.''

On Wednesday, the Democratic National Committee's interim chairwoman,
Donna Brazile, pressed for a bipartisan, independent investigation.

\begin{quote}
``Senate Majority Leader Mitch McConnell must allow the establishment of
an independent bipartisan commission to immediately investigate the full
range of foreign interference in the 2016 election and determine what we
need to do to keep our democracy safe from foreign interference. As part
of that investigation, the commission must investigate President-elect
Trump's personal and financial ties to Russia, ties between his aides
and Russia, and the existence of allegedly compromising material that
has allegedly been obtained by Russia in order to blackmail him.''
\end{quote}

Senator Richard Durbin of Illinois, the No. 2 Democrat, said he too
wants a special investigation into the dossier.

\begin{quote}
\end{quote}

Advertisement

\protect\hyperlink{after-bottom}{Continue reading the main story}

\hypertarget{site-index}{%
\subsection{Site Index}\label{site-index}}

\hypertarget{site-information-navigation}{%
\subsection{Site Information
Navigation}\label{site-information-navigation}}

\begin{itemize}
\tightlist
\item
  \href{https://help.nytimes.com/hc/en-us/articles/115014792127-Copyright-notice}{©~2020~The
  New York Times Company}
\end{itemize}

\begin{itemize}
\tightlist
\item
  \href{https://www.nytco.com/}{NYTCo}
\item
  \href{https://help.nytimes.com/hc/en-us/articles/115015385887-Contact-Us}{Contact
  Us}
\item
  \href{https://www.nytco.com/careers/}{Work with us}
\item
  \href{https://nytmediakit.com/}{Advertise}
\item
  \href{http://www.tbrandstudio.com/}{T Brand Studio}
\item
  \href{https://www.nytimes.com/privacy/cookie-policy\#how-do-i-manage-trackers}{Your
  Ad Choices}
\item
  \href{https://www.nytimes.com/privacy}{Privacy}
\item
  \href{https://help.nytimes.com/hc/en-us/articles/115014893428-Terms-of-service}{Terms
  of Service}
\item
  \href{https://help.nytimes.com/hc/en-us/articles/115014893968-Terms-of-sale}{Terms
  of Sale}
\item
  \href{https://spiderbites.nytimes.com}{Site Map}
\item
  \href{https://help.nytimes.com/hc/en-us}{Help}
\item
  \href{https://www.nytimes.com/subscription?campaignId=37WXW}{Subscriptions}
\end{itemize}
