Sections

SEARCH

\protect\hyperlink{site-content}{Skip to
content}\protect\hyperlink{site-index}{Skip to site index}

\href{https://www.nytimes.com/section/politics}{Politics}

\href{https://myaccount.nytimes.com/auth/login?response_type=cookie\&client_id=vi}{}

\href{https://www.nytimes.com/section/todayspaper}{Today's Paper}

\href{/section/politics}{Politics}\textbar{}How a Sensational,
Unverified Dossier Became a Crisis for Donald Trump

\url{https://nyti.ms/2jx9scj}

\begin{itemize}
\item
\item
\item
\item
\item
\end{itemize}

Advertisement

\protect\hyperlink{after-top}{Continue reading the main story}

Supported by

\protect\hyperlink{after-sponsor}{Continue reading the main story}

\hypertarget{how-a-sensational-unverified-dossier-became-a-crisis-for-donald-trump}{%
\section{How a Sensational, Unverified Dossier Became a Crisis for
Donald
Trump}\label{how-a-sensational-unverified-dossier-became-a-crisis-for-donald-trump}}

\includegraphics{https://static01.nyt.com/images/2017/01/12/us/12intel/12intel-articleLarge.jpg?quality=75\&auto=webp\&disable=upscale}

By \href{http://www.nytimes.com/by/scott-shane}{Scott Shane},
\href{http://www.nytimes.com/by/nicholas-confessore}{Nicholas
Confessore} and
\href{http://www.nytimes.com/by/matthew-rosenberg}{Matthew Rosenberg}

\begin{itemize}
\item
  Jan. 11, 2017
\item
  \begin{itemize}
  \item
  \item
  \item
  \item
  \item
  \end{itemize}
\end{itemize}

WASHINGTON --- Seven months ago, a respected former British spy named
Christopher Steele won a contract to build a file on Donald J. Trump's
ties to Russia. Last week, the explosive details --- unsubstantiated
accounts of frolics with prostitutes, real estate deals that were
intended as bribes and coordination with Russian intelligence of the
hacking of Democrats --- were summarized for Mr. Trump in an appendix to
a
\href{https://www.nytimes.com/2017/01/06/us/politics/donald-trump-wall-hack-russia.html}{top-secret
intelligence report}.

The consequences have been incalculable and will play out long past
Inauguration Day. Word of the summary, which was also given to President
Obama and congressional leaders, leaked to CNN Tuesday, and the rest of
the media followed with sensational reports.

Mr. Trump denounced the unproven claims Wednesday as a fabrication, a
Nazi-style smear concocted by ``sick people.'' It has further undermined
his relationship with the intelligence agencies and cast a shadow over
the new administration.

Late Wednesday night, after speaking with Mr. Trump, James R. Clapper
Jr., the director of national intelligence, issued a statement decrying
leaks about the matter and saying of Mr. Steele's dossier that the
intelligence agencies have ``not made any judgment that the information
in this document is reliable.'' Mr. Clapper suggested that intelligence
officials had nonetheless shared it to give policy makers ``the fullest
possible picture of any matters that might affect national security.''

Parts of the story remain out of reach --- most critically the basic
question of how much, if anything, in the dossier is true. But it is
possible to piece together a rough narrative of what led to the current
crisis, including lingering questions about the ties binding Mr. Trump
and his team to Russia. The episode also offers a glimpse of the hidden
side of presidential campaigns, involving private sleuths-for-hire
looking for the worst they can find about the next American leader.

The story began in September 2015, when a wealthy Republican donor who
strongly opposed Mr. Trump put up the money to hire a Washington
research firm run by former journalists, Fusion GPS, to compile a
dossier about the real estate magnate's past scandals and weaknesses,
according to a person familiar with the effort. The person described the
opposition research work on condition of anonymity, citing the volatile
nature of the story and the likelihood of future legal disputes. The
identity of the donor is unclear.

Fusion GPS, headed by a former Wall Street Journal journalist known for
his dogged reporting, Glenn Simpson, most often works for business
clients. But in presidential elections, the firm is sometimes hired by
candidates, party organizations or donors to do political ``oppo'' work
--- shorthand for opposition research --- on the side.

It is routine work and ordinarily involves creating a big, searchable
database of public information: past news reports, documents from
lawsuits and other relevant data. For months, Fusion GPS gathered the
documents and put together the files from Mr. Trump's past in business
and entertainment, a rich target.

After Mr. Trump emerged as the presumptive nominee in the spring, the
Republican interest in financing the effort ended. But Democratic
supporters of Hillary Clinton were very interested, and Fusion GPS kept
doing the same deep dives, but on behalf of new clients.

In June, the tenor of the effort suddenly changed.
\href{https://www.washingtonpost.com/world/national-security/russian-government-hackers-penetrated-dnc-stole-opposition-research-on-trump/2016/06/14/cf006cb4-316e-11e6-8ff7-7b6c1998b7a0_story.html}{The
Washington Post reported} that the Democratic National Committee had
been hacked, apparently by Russian government agents, and a mysterious
figure
\href{https://www.nytimes.com/2016/07/28/us/politics/is-dnc-email-hacker-a-person-or-a-russian-front-experts-arent-sure.html}{calling
himself ``Guccifer 2.0''} began to publish the stolen documents online.

Mr. Simpson hired Mr. Steele, a former British intelligence officer with
whom he had worked before. Mr. Steele, in his early 50s, had served
undercover in Moscow in the early 1990s and later was the top expert on
Russia at the London headquarters of Britain's spy service, MI6. When he
stepped down in 2009, he started his own commercial intelligence firm,
Orbis Business Intelligence.

The former journalist and the former spy, according to people who know
them, had similarly dark views of President Vladimir V. Putin of Russia,
a former K.G.B. officer, and the varied tactics he and his intelligence
operatives used to smear, blackmail or bribe their targets.

As a former spy who had carried out espionage inside Russia, Mr. Steele
was in no position to travel to Moscow to study Mr. Trump's connections
there. Instead, he hired native Russian speakers to call informants
inside Russia and made surreptitious contact with his own connections in
the country as well.

Mr. Steele wrote up his findings in a series of memos, each a few pages
long, that he began to deliver to Fusion GPS in June and continued at
least until December. By then, the election was over, and neither Mr.
Steele nor Mr. Simpson was being paid by a client, but they did not stop
what they believed to be very important work. (Mr. Simpson declined to
comment for this article, and Mr. Steele did not immediately reply to a
request for comment.)

The memos described two different Russian operations. The first was a
yearslong effort to find a way to influence Mr. Trump, perhaps because
he had contacts with Russian oligarchs whom Mr. Putin wanted to keep
track of. According to Mr. Steele's memos, it used an array of familiar
tactics: the gathering of ``kompromat,'' compromising material such as
alleged tapes of Mr. Trump with prostitutes in a Moscow hotel, and
proposals for business deals attractive to Mr. Trump.

The goal would probably never have been to make Mr. Trump a knowing
agent of Russia, but to make him a source who might provide information
to friendly Russian contacts. But if Mr. Putin and his agents wanted to
entangle Mr. Trump using business deals, they did not do it very
successfully. Mr. Trump has said he has no major properties there,
though one of his sons said at a real estate conference in 2008 that ``a
lot of money'' was ``pouring in from Russia.''

The second Russian operation described was recent: a series of contacts
with Mr. Trump's representatives during the campaign, in part to discuss
the hacking of the Democratic National Committee and Mrs. Clinton's
campaign chairman, John D. Podesta. According to Mr. Steele's sources,
it involved, among other things, a late-summer meeting in Prague between
Michael Cohen, a lawyer for Mr. Trump, and Oleg Solodukhin, a Russian
official who works for Rossotrudnichestvo, an organization that promotes
Russia's interests abroad.

By all accounts, Mr. Steele has an excellent reputation with American
and British intelligence colleagues and had done work for the F.B.I. on
the investigation of bribery at FIFA, soccer's global governing body.
Colleagues say he was acutely aware of the danger he and his associates
were being fed Russian disinformation. Russian intelligence had mounted
a complex hacking operation to damage Mrs. Clinton, and a similar
operation against Mr. Trump was possible.

But much of what he was told, and passed on to Fusion GPS, was very
difficult to check. And some of the claims that can be checked seem
problematic. Mr. Cohen, for instance,
\href{https://twitter.com/MichaelCohen212/status/818991277685567489}{said
on Twitter} on Tuesday night that he has never been in Prague; Mr.
Solodukhin, his purported Russian contact,
\href{https://www.nytimes.com/2017/01/10/us/politics/donald-trump-russia-intelligence.html}{denied
in a telephone interview} that he had ever met Mr. Cohen or anyone
associated with Mr. Trump. The president-elect on Wednesday cited news
reports that a different Michael Cohen with no Trump ties may have
visited Prague and that the two Cohens might have been mixed up in Mr.
Steele's reports.

But word of a dossier had begun to spread through political circles.
Rick Wilson, a Republican political operative who was working for a
super PAC supporting Marco Rubio, said he heard about it in July, when
an investigative reporter for a major news network called him to ask
what he knew.

By early fall, some of Mr. Steele's memos had been given to the F.B.I.,
which was already investigating Mr. Trump's Russian ties, and to
journalists. An MI6 official, whose job does not permit him to be quoted
by name, said that in late summer or early fall, Mr. Steele also passed
the reports he had prepared on Mr. Trump and Russia to British
intelligence. Mr. Steele was concerned about what he was hearing about
Mr. Trump, and he thought that the information should not be solely in
the hands of people looking to win a political contest.

After the election, the memos, still being supplemented by his
inquiries, became one of Washington's worst-kept secrets, as reporters
--- including from The New York Times --- scrambled to confirm or
disprove them.

Word also reached Capitol Hill. Senator John McCain, Republican of
Arizona, heard about the dossier and obtained a copy in December from
David J. Kramer, a former top State Department official who works for
the McCain Institute at Arizona State University. Mr. McCain passed the
information to James B. Comey, the F.B.I. director.

Remarkably for Washington, many reporters for competing news
organizations had the salacious and damning memos, but they did not
leak, because their contents could not be confirmed. (Mother Jones
magazine was an exception,
\href{http://www.motherjones.com/politics/2016/10/veteran-spy-gave-fbi-info-alleging-russian-operation-cultivate-donald-trump}{publishing
a story on Oct. 31} that described the dossier, its origin and
significance, while omitting the titillating details.) That changed only
this week, after the heads of the C.I.A., the F.B.I. and the National
Security Agency added a summary of the memos, along with information
gathered from other intelligence sources, to their report on the Russian
cyberattack on the election.

Now, after the most contentious of elections, Americans are divided and
confused about what to believe about the incoming president. And there
is no prospect soon for full clarity on the veracity of the claims made
against him.

``It is a remarkable moment in history,'' said Mr. Wilson, the Florida
political operative. ``What world did I wake up in?''

Advertisement

\protect\hyperlink{after-bottom}{Continue reading the main story}

\hypertarget{site-index}{%
\subsection{Site Index}\label{site-index}}

\hypertarget{site-information-navigation}{%
\subsection{Site Information
Navigation}\label{site-information-navigation}}

\begin{itemize}
\tightlist
\item
  \href{https://help.nytimes.com/hc/en-us/articles/115014792127-Copyright-notice}{©~2020~The
  New York Times Company}
\end{itemize}

\begin{itemize}
\tightlist
\item
  \href{https://www.nytco.com/}{NYTCo}
\item
  \href{https://help.nytimes.com/hc/en-us/articles/115015385887-Contact-Us}{Contact
  Us}
\item
  \href{https://www.nytco.com/careers/}{Work with us}
\item
  \href{https://nytmediakit.com/}{Advertise}
\item
  \href{http://www.tbrandstudio.com/}{T Brand Studio}
\item
  \href{https://www.nytimes.com/privacy/cookie-policy\#how-do-i-manage-trackers}{Your
  Ad Choices}
\item
  \href{https://www.nytimes.com/privacy}{Privacy}
\item
  \href{https://help.nytimes.com/hc/en-us/articles/115014893428-Terms-of-service}{Terms
  of Service}
\item
  \href{https://help.nytimes.com/hc/en-us/articles/115014893968-Terms-of-sale}{Terms
  of Sale}
\item
  \href{https://spiderbites.nytimes.com}{Site Map}
\item
  \href{https://help.nytimes.com/hc/en-us}{Help}
\item
  \href{https://www.nytimes.com/subscription?campaignId=37WXW}{Subscriptions}
\end{itemize}
