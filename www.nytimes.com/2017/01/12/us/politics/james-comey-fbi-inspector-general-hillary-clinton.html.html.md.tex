Sections

SEARCH

\protect\hyperlink{site-content}{Skip to
content}\protect\hyperlink{site-index}{Skip to site index}

\href{https://www.nytimes.com/section/politics}{Politics}

\href{https://myaccount.nytimes.com/auth/login?response_type=cookie\&client_id=vi}{}

\href{https://www.nytimes.com/section/todayspaper}{Today's Paper}

\href{/section/politics}{Politics}\textbar{}Comey Letter on Clinton
Email Is Subject of Justice Dept. Inquiry

\url{https://nyti.ms/2ioVpHX}

\begin{itemize}
\item
\item
\item
\item
\item
\item
\end{itemize}

Advertisement

\protect\hyperlink{after-top}{Continue reading the main story}

Supported by

\protect\hyperlink{after-sponsor}{Continue reading the main story}

\hypertarget{comey-letter-on-clinton-email-is-subject-of-justice-dept-inquiry}{%
\section{Comey Letter on Clinton Email Is Subject of Justice Dept.
Inquiry}\label{comey-letter-on-clinton-email-is-subject-of-justice-dept-inquiry}}

\includegraphics{https://static01.nyt.com/images/2017/01/13/us/13comey/13comey-articleLarge.jpg?quality=75\&auto=webp\&disable=upscale}

By \href{https://www.nytimes.com/by/adam-goldman}{Adam Goldman},
\href{http://www.nytimes.com/by/eric-lichtblau}{Eric Lichtblau} and
\href{http://www.nytimes.com/by/matt-apuzzo}{Matt Apuzzo}

\begin{itemize}
\item
  Jan. 12, 2017
\item
  \begin{itemize}
  \item
  \item
  \item
  \item
  \item
  \item
  \end{itemize}
\end{itemize}

WASHINGTON --- The Justice Department's inspector general said Thursday
that he would open a broad investigation into how the F.B.I. director,
James B. Comey, handled the case over Hillary Clinton's emails,
including his decision to discuss it at a news conference and to
disclose 11 days before the election that he had new information that
could lead him to reopen it.

The inspector general, Michael E. Horowitz, will not look into the
decision not to prosecute Mrs. Clinton or her aides. But he will review
actions Mr. Comey took that Mrs. Clinton and many of her supporters
believe cost her the election.

They are: the news conference in July at which he announced he was not
indicting Mrs. Clinton but described her behavior as ``extremely
careless''; the letter to Congress in late October in which he said that
newly discovered emails could potentially change the outcome of the
F.B.I.'s investigation; and the letter three days before the election in
which he said that he was closing it again.

The inspector general's office said that it was initiating the
investigation in response to complaints from members of Congress and the
public about actions by the F.B.I. and the Justice Department during the
campaign that could be seen as politically motivated.

For Mr. Comey and the agency he heads, the Clinton investigation was
politically fraught from the moment the F.B.I. received a referral in
July 2015 to determine whether Mrs. Clinton and her aides had mishandled
classified information. Senior F.B.I. officials believed there was never
going to be a good outcome, since it put them in the middle of a
bitterly partisan issue.

Whatever the decision on whether to charge Mrs. Clinton with a crime,
Mr. Comey, a Republican former Justice Department official appointed by
President Obama, was going to get hammered. And he was.

Republicans, who made her use of a private email server a centerpiece of
their campaign against Mrs. Clinton, attacked Mr. Comey after he decided
there was not sufficient evidence she had mishandled classified
information to prosecute her.

\href{https://www.nytimes.com/interactive/2016/11/02/us/elections/James-Comey-options-classified-information-Hillary-Clinton-elections.html}{}

\includegraphics{https://static01.nyt.com/images/2016/11/02/us/elections/James-Comey-options-classified-information-Hillary-Clinton-elections-1478129558108/James-Comey-options-classified-information-Hillary-Clinton-elections-1478129558108-largeHorizontalJumbo.png}

\hypertarget{these-are-the-bad-and-worse-options-james-comey-faced}{%
\subsection{These Are the Bad (and Worse) Options James Comey
Faced}\label{these-are-the-bad-and-worse-options-james-comey-faced}}

When federal officials concluded their investigation into Hillary
Clinton's use of a private email server as secretary of state, the
F.B.I. director, James B. Comey, had a decision to make on how to
announce that news. The choices he made in July set the F.B.I. on the
path toward the predicament it faces today.

The Clinton campaign believed the F.B.I. investigation was overblown and
seriously damaged her chances to win the White House and resented Mr.
Comey's comments about Mrs. Clinton at his news conference. But the
campaign was particularly upset about Mr. Comey's two letters, which
created a wave of damaging news stories at the end of the campaign, when
Mrs. Clinton and her supporters thought they had put the email issue
behind them.

In the end, the emails that the F.B.I. reviewed --- which came up during
an unrelated inquiry into Anthony D. Weiner, the estranged husband of a
top Clinton aide, Huma Abedin --- proved irrelevant to the
investigation's outcome.

The Clinton campaign said Mr. Comey's actions quite likely caused a
significant number of undecided voters to cast ballots for
President-elect Donald J. Trump.

F.B.I. officials said Thursday that they welcomed the scrutiny. In a
statement, Mr. Comey described Mr. Horowitz as ``professional and
independent'' and promised to cooperate with his investigation. ``I hope
very much he is able to share his conclusions and observations with the
public because everyone will benefit from thoughtful evaluation and
transparency,'' Mr. Comey said.

Brian Fallon, the former press secretary for the Clinton campaign and
the former top spokesman for the Justice Department, said the inspector
general's investigation was long overdue.

``This is highly encouraging and to be expected, given Director Comey's
drastic deviation from Justice Department protocol,'' he said. ``A probe
of this sort, however long it takes to conduct, is utterly necessary in
order to take the first step to restore the F.B.I.'s reputation as a
nonpartisan institution.''

Mr. Horowitz has the authority to recommend a criminal investigation if
he finds evidence of illegality, but there has been no suggestion that
Mr. Comey's actions were unlawful. Rather, the question has been whether
he acted inappropriately, showed bad judgment or violated Justice
Department guidelines. It is not clear what the consequences would be
for Mr. Comey if he was found to have done any of those things.

The Justice Department and the F.B.I. have a longstanding policy against
discussing criminal investigations. Another
\href{https://www.justice.gov/sites/default/files/oip/legacy/2014/07/23/ag-memo-election-year-sensitivities.pdf}{Justice
Department policy} declares that politics should play no role in
investigative decisions. Both Democratic and Republican administrations
have interpreted that policy broadly to prohibit taking any steps that
might even hint at an impression of partisanship.

Inspectors general have investigated F.B.I. directors before, but
rarely. The most high-profile example was the investigation of William
S. Sessions, who was fired by President Bill Clinton after an internal
inquiry cited him for financial misconduct. In recent years, the
inspector general has investigated accusations of wrongdoing by the
F.B.I. involving some of its most sensitive operations, including a
number of surveillance and counterterrorism programs.

As part of the review, the inspector general will examine other issues
related to the email investigation that Republicans have raised. They
include whether the deputy director of the F.B.I., Andrew G. McCabe,
should have recused himself from any involvement in it.

In 2015, Mr. McCabe's wife ran for a State Senate seat in Virginia as a
Democrat and accepted nearly \$500,000 in political contributions from
Gov. Terry McAuliffe, a key ally of the Clintons. Though Mr. McCabe did
not assume his post until February 2016, months after his wife was
defeated, critics both within the agency and outside of it felt that he
should have recused himself.

The F.B.I. has said Mr. McCabe played no role in his wife's campaign. He
also told his superiors she was running and sought ethics advice from
F.B.I. officials.

Mr. Horowitz said he would also investigate whether the Justice
Department's top congressional liaison, Peter Kadzik, had improperly
provided information to the Clinton campaign.
\href{https://wikileaks.org/podesta-emails/emailid/43150}{A hacked
email} posted by WikiLeaks showed that Mr. Kadzik alerted the campaign
about a coming congressional hearing that was likely to raise questions
about Mrs. Clinton.

Investigators will be helped in gathering evidence by a law that
Congress passed just last month, which ensures that inspectors general
across the government will have access to all relevant agency records in
their reviews.

The law grew out of skirmishes between the F.B.I. and the Justice
Department inspector general over attempts by the F.B.I. to keep grand
jury material and other records off limits. The new law means Mr.
Horowitz's investigators should have access to any records deemed
relevant.

Mr. Trump has not indicated whether he intends to keep Mr. Comey in his
job. When he cleared Mrs. Clinton of criminal wrongdoing during the
campaign, Mr. Trump accused him of being part of a rigged system.

Although the president does not need cause to fire the F.B.I. director,
a critical inspector general report could provide justification to do so
if Mr. Trump is looking for some.

Advertisement

\protect\hyperlink{after-bottom}{Continue reading the main story}

\hypertarget{site-index}{%
\subsection{Site Index}\label{site-index}}

\hypertarget{site-information-navigation}{%
\subsection{Site Information
Navigation}\label{site-information-navigation}}

\begin{itemize}
\tightlist
\item
  \href{https://help.nytimes.com/hc/en-us/articles/115014792127-Copyright-notice}{©~2020~The
  New York Times Company}
\end{itemize}

\begin{itemize}
\tightlist
\item
  \href{https://www.nytco.com/}{NYTCo}
\item
  \href{https://help.nytimes.com/hc/en-us/articles/115015385887-Contact-Us}{Contact
  Us}
\item
  \href{https://www.nytco.com/careers/}{Work with us}
\item
  \href{https://nytmediakit.com/}{Advertise}
\item
  \href{http://www.tbrandstudio.com/}{T Brand Studio}
\item
  \href{https://www.nytimes.com/privacy/cookie-policy\#how-do-i-manage-trackers}{Your
  Ad Choices}
\item
  \href{https://www.nytimes.com/privacy}{Privacy}
\item
  \href{https://help.nytimes.com/hc/en-us/articles/115014893428-Terms-of-service}{Terms
  of Service}
\item
  \href{https://help.nytimes.com/hc/en-us/articles/115014893968-Terms-of-sale}{Terms
  of Sale}
\item
  \href{https://spiderbites.nytimes.com}{Site Map}
\item
  \href{https://help.nytimes.com/hc/en-us}{Help}
\item
  \href{https://www.nytimes.com/subscription?campaignId=37WXW}{Subscriptions}
\end{itemize}
