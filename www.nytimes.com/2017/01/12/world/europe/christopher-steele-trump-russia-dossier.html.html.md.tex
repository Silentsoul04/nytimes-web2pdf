Sections

SEARCH

\protect\hyperlink{site-content}{Skip to
content}\protect\hyperlink{site-index}{Skip to site index}

\href{https://www.nytimes.com/section/world/europe}{Europe}

\href{https://myaccount.nytimes.com/auth/login?response_type=cookie\&client_id=vi}{}

\href{https://www.nytimes.com/section/todayspaper}{Today's Paper}

\href{/section/world/europe}{Europe}\textbar{}Christopher Steele, Ex-Spy
Who Compiled Trump Dossier, Goes to Ground

\url{https://nyti.ms/2inTvrk}

\begin{itemize}
\item
\item
\item
\item
\item
\end{itemize}

Advertisement

\protect\hyperlink{after-top}{Continue reading the main story}

Supported by

\protect\hyperlink{after-sponsor}{Continue reading the main story}

\hypertarget{christopher-steele-ex-spy-who-compiled-trump-dossier-goes-to-ground}{%
\section{Christopher Steele, Ex-Spy Who Compiled Trump Dossier, Goes to
Ground}\label{christopher-steele-ex-spy-who-compiled-trump-dossier-goes-to-ground}}

\includegraphics{https://static01.nyt.com/images/2017/01/13/world/13Steele/13Steele-articleLarge.jpg?quality=75\&auto=webp\&disable=upscale}

By \href{http://www.nytimes.com/by/steven-erlanger}{Steven Erlanger}

\begin{itemize}
\item
  Jan. 12, 2017
\item
  \begin{itemize}
  \item
  \item
  \item
  \item
  \item
  \end{itemize}
\end{itemize}

LONDON --- Christopher Steele, the former British intelligence agent who
prepared the dossier on Donald J. Trump's supposed activities in Russia,
has gone underground.

The strange story of
\href{https://www.nytimes.com/2017/01/11/us/politics/trump-intelligence-report-explainer.html}{the
dossier}, which United States intelligence agencies, the F.B.I., Senator
John McCain and many journalists have had for weeks, if not months, and
which Mr. Trump presumably must have known about, appears to have had
personal consequences for Mr. Steele.

According to neighbors and news reports, Mr. Steele hurriedly left his
home in Surrey, a county southwest of London, on Wednesday to avoid
attention or possible retribution once his identity as the author of the
dossier was revealed,
\href{http://www.wsj.com/articles/christopher-steele-ex-british-intelligence-officer-said-to-have-prepared-dossier-on-trump-1484162553}{first
by The Wall Street Journal}. The Journal reported that Mr. Steele had
declined its interview requests because the subject was ``too hot.''

Mr. Steele, 52, was a longstanding officer with MI6, the British
equivalent of the C.I.A., serving in Paris and Moscow in the 1990s
before retiring. In 2009, he started a private research firm,
\href{https://orbisbi.com/}{Orbis Business Intelligence Ltd.}, with
Christopher Burrows, now 58. Mr. Burrows has refused to confirm or deny
that Mr. Steele and Orbis wrote
\href{https://www.nytimes.com/2017/01/11/us/politics/donald-trump-russia-intelligence.html?hp\&action=click\&pgtype=Homepage\&clickSource=story-heading\&module=a-lede-package-region\&region=top-news\&WT.nav=top-news}{the
memos that made up the dossier}, initially under contract to a
Washington firm paid to dig into harmful matters from Mr. Trump's past.

Mr. Burrows's profile page on LinkedIn describes him as a former
counselor in the Foreign Office, with postings in Brussels and New Delhi
in the early 2000s. Diplomatic postings are sometimes used to provide
cover for intelligence agents. Mr. Steele's profile on LinkedIn gives no
specifics about his career.

He is known in British intelligence circles for his knowledge of the
intricate web of Kremlin-tied companies and associates that control
Russia.

Mr. Steele, as a known former MI6 agent, was thought not to have gone to
Russia in his investigations but to have used contacts inside and
outside the country to prepare the dossier, which United States
intelligence agencies have said they cannot substantiate. But the file
was used to prepare a two-page appendix to the intelligence presentation
American officials
\href{https://www.nytimes.com/2017/01/10/us/politics/donald-trump-russia-intelligence.html?rref=collection\%2Ftimestopic\%2FTrump\%2C\%20Donald\%20J.\&action=click\&contentCollection=timestopics\&region=stream\&module=stream_unit\&version=latest\&contentPlacement=57\&pgtype=collection}{gave
to Mr. Trump last Friday}.

Mr. Trump has denied the allegations in the dossier in the sharpest
terms, and called them ``fake news.''
\href{https://www.nytimes.com/2016/09/22/world/europe/russia-syria-aid-convoy.html}{Russia
has denied} that it holds any compromising material on Mr. Trump.

John Sipher, who retired from the C.I.A. in 2014 after 28 years with the
agency, described Mr. Steele as having a good reputation and ``some
credibility.'' Mr. Sipher was stationed in Moscow in the 1990s, and then
ran the C.I.A.'s Russia program for three years, according to
\href{http://www.pbs.org/newshour/bb/credible-reports-alleged-russian-dossier-trump/}{an
interview he gave to PBS NewsHour}. He now works at
\href{https://www.crosslead.com/our-team/}{CrossLead}, a
Washington-based technology company.

``I have confidence that the F.B.I. is going to follow this through,''
Mr. Sipher said. ``My nervousness is that these kind of things are going
to dribble and drabble out for the next several years and cause a real
problem for this administration going forward.''

An investigator for a business research firm in London similar to Orbis,
who knows the work of the company but who has met Mr. Steele only
briefly, said he was not impressed by the dossier.

``I have a lot of experience in this world,'' he said. ``If I were the
client, I would throw it back and say, `Where's the evidence guys? I
can't use this.' ''

The investigator, who asked for anonymity because he did not want to
discuss publicly the work of a competitor, said that ``all intel has to
be caveated.''

``Maybe they went to a usually reliable source,'' he added, ``but
there's no explanation about the credibility of these sources.''

He continued, ``Maybe sometimes sources want to tell the investigators
what their clients want to hear.''

Referring to companies like Orbis and his own, he said: ``Usually your
job would be to stop clients from dealing with corrupt, questionable
counterparts in a high-risk country like Russia, but this same network
could be put to use'' to compile reports like the one on Mr. Trump.

``There's a risk that maybe the sources fed questionable intelligence,
knowing that it would do more damage to Trump's enemies than to Trump,''
the investigator suggested.

Orbis's website says that it was ``founded by former British
intelligence professionals.'' Based in Grosvenor Gardens, near Victoria
Station in London, the company says it has a ``sophisticated
investigative capability'' and mounts ``intelligence-gathering
operations'' and ``complex, often cross-border investigations.''

According to the website, it also offers ``real-time source reporting on
business and politics at all levels,'' and ``draws on extensive
experience at boardroom level in government, multilateral diplomacy and
international business to develop bespoke solutions for clients.''

Mr. Steele and Orbis have previously investigated corruption at FIFA,
the governing body of world soccer.

In October, David Corn of Mother Jones magazine
\href{http://www.motherjones.com/politics/2016/10/veteran-spy-gave-fbi-info-alleging-russian-operation-cultivate-donald-trump}{wrote
about} the dossier and described his conversations with Mr. Steele, whom
he did not identify by name or nationality.

According to the British newspaper The Telegraph, a friend of Mr.
Steele's
\href{http://www.telegraph.co.uk/news/2017/01/11/former-mi6-officer-produced-donald-trump-russian-dossier-terrified/}{said}
that after his name and nationality were revealed, he had become
``terrified for his and his family's safety.''

Mr. Steele's wife and children also were not at home.

Advertisement

\protect\hyperlink{after-bottom}{Continue reading the main story}

\hypertarget{site-index}{%
\subsection{Site Index}\label{site-index}}

\hypertarget{site-information-navigation}{%
\subsection{Site Information
Navigation}\label{site-information-navigation}}

\begin{itemize}
\tightlist
\item
  \href{https://help.nytimes.com/hc/en-us/articles/115014792127-Copyright-notice}{©~2020~The
  New York Times Company}
\end{itemize}

\begin{itemize}
\tightlist
\item
  \href{https://www.nytco.com/}{NYTCo}
\item
  \href{https://help.nytimes.com/hc/en-us/articles/115015385887-Contact-Us}{Contact
  Us}
\item
  \href{https://www.nytco.com/careers/}{Work with us}
\item
  \href{https://nytmediakit.com/}{Advertise}
\item
  \href{http://www.tbrandstudio.com/}{T Brand Studio}
\item
  \href{https://www.nytimes.com/privacy/cookie-policy\#how-do-i-manage-trackers}{Your
  Ad Choices}
\item
  \href{https://www.nytimes.com/privacy}{Privacy}
\item
  \href{https://help.nytimes.com/hc/en-us/articles/115014893428-Terms-of-service}{Terms
  of Service}
\item
  \href{https://help.nytimes.com/hc/en-us/articles/115014893968-Terms-of-sale}{Terms
  of Sale}
\item
  \href{https://spiderbites.nytimes.com}{Site Map}
\item
  \href{https://help.nytimes.com/hc/en-us}{Help}
\item
  \href{https://www.nytimes.com/subscription?campaignId=37WXW}{Subscriptions}
\end{itemize}
