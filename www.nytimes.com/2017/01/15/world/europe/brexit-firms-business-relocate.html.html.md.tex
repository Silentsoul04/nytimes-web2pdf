Sections

SEARCH

\protect\hyperlink{site-content}{Skip to
content}\protect\hyperlink{site-index}{Skip to site index}

\href{https://www.nytimes.com/section/world/europe}{Europe}

\href{https://myaccount.nytimes.com/auth/login?response_type=cookie\&client_id=vi}{}

\href{https://www.nytimes.com/section/todayspaper}{Today's Paper}

\href{/section/world/europe}{Europe}\textbar{}British Firms Await Brexit
Plans, Poised to Relocate

\url{https://nyti.ms/2izQvrI}

\begin{itemize}
\item
\item
\item
\item
\item
\end{itemize}

Advertisement

\protect\hyperlink{after-top}{Continue reading the main story}

Supported by

\protect\hyperlink{after-sponsor}{Continue reading the main story}

\hypertarget{british-firms-await-brexit-plans-poised-to-relocate}{%
\section{British Firms Await Brexit Plans, Poised to
Relocate}\label{british-firms-await-brexit-plans-poised-to-relocate}}

\includegraphics{https://static01.nyt.com/images/2017/01/13/world/13Brexit1/13Brexit1-articleLarge.jpg?quality=75\&auto=webp\&disable=upscale}

By \href{http://www.nytimes.com/by/stephen-castle}{Stephen Castle}

\begin{itemize}
\item
  Jan. 15, 2017
\item
  \begin{itemize}
  \item
  \item
  \item
  \item
  \item
  \end{itemize}
\end{itemize}

READING, England --- With its state-of-the-art robotics, Magal
Engineering's factory in the colorless outskirts of this busy commercial
town west of London is an important component of Britain's booming car
industry, churning out millions of sophisticated vehicle parts every
year.

Yet as Britain prepares to leave the European Union, Magal is having to
rethink its operations here.

The company has sister factories in France and Germany, imports
components from them to assemble in Reading, and sends many finished
products back across the English Channel --- all with minimal paperwork
and no tariffs.

Britain's departure from the European Union could change all that,
ending the tariff-free trade that Magal enjoys, potentially disrupting
the hiring of crucial engineers from abroad and requiring new hires to
keep up with the added paperwork.

Britain's exit is a ``big worry'' and ideally would be stopped, says
Gamil Magal, Magal Engineering's chief executive. But as a realist, he
is preparing to live with it. That means suspending investment in
Britain and contemplating production cuts here, even as the company
continues to put money into France.

With Britain, the policy is ``wait and see,'' Mr. Magal said, speaking
in his office above this bustling factory in a building that once
produced World War II aircraft.

\includegraphics{https://static01.nyt.com/images/2017/01/13/world/13Brexit2/13Brexit2-articleLarge.jpg?quality=75\&auto=webp\&disable=upscale}

Although Britain's economy has so far not suffered the economic blowback
that many predicted would follow the vote in a referendum last year to
quit the bloc, British companies that import or export goods and
services are anxiously assessing the potential costs of departure.

Complicating matters is continuing uncertainty over the terms of
departure that the government wants to negotiate. Prime Minister Theresa
May plans to start two years of formal talks with the European Union on
Britain's withdrawal, widely known as Brexit, by the end of March, and
on Tuesday she is expected to give some more detail about her
objectives.

``We don't know what is going to happen, the government doesn't know
what is going to happen, and I am not sure that the government even
knows what it is asking for,'' said David Bailey, professor of
industrial strategy at Aston University.

There are signs that British-based companies might be taking action
already, ahead of the actual Brexit. In
\href{http://www.thetimes.co.uk/article/banks-and-insurers-begin-to-move-workers-after-eu-vote-gnrtfqgzs}{a
survey} of 233 financial services companies published in The Sunday
Times of London last week, 39 said they would reduce staffing because of
the Brexit vote, and half of that number had already started to do so.

For British banks, a big worry is whether they will able to continue
offering services to clients across the bloc.

``Most international banks now have project teams working out which
operations they need to move to ensure they can continue serving
customers, the date by which this must happen, and how best to do it,''
Anthony Browne, the chief executive of the British Bankers Association,
wrote in The Observer last year. ``Their hands are quivering over the
relocate button,'' he added.

Sectors like agriculture, hospitality and construction are worried about
labor. They all rely heavily on workers from Southern or Eastern Europe,
whose right to employment here may be curbed to achieve Mrs. May's
objective of reducing immigration.

\href{https://www.nytimes.com/interactive/2016/business/international/brexit-uk-what-happens-business.html}{}

\includegraphics{https://static01.nyt.com/images/2017/03/29/business/27BREXIT/27BREXIT-articleLarge.jpg}

\hypertarget{how-brexit-could-change-business-in-britain}{%
\subsection{How `Brexit' Could Change Business in
Britain}\label{how-brexit-could-change-business-in-britain}}

Britain has started the clock on leaving the European Union, and will be
out of the bloc by March 2019. Here is how ``Brexit'' has affected
business so far.

Mrs. May has not said explicitly whether she wants Britain to remain in
Europe's Customs Union, which provides for tariff-free trade across the
28-nation bloc, and includes some nonmembers like Turkey. Nor has she
detailed what kind of relationship she wants with the European Union's
single market, which aims to eliminate nontariff barriers and to forge a
single services market. But she said in a
\href{http://www.bbc.co.uk/news/uk-politics-38546820}{recent interview}
that when Britain quits the union, it will not be keeping ``bits'' of
its membership, suggesting that she will be seeking something closer to
a clean break, or ``hard Brexit.'' And with the clock now ticking toward
a planned departure from the union in 2019, companies are already having
to contemplate that outcome, even before Ms. May lays out her intentions
on Tuesday.

For Mr. Magal, the threat of trade tariffs is forcing him to rethink the
structure of his business. The company assembles thermostatic control
units for car manufacturers, including Jaguar Land Rover in Britain and
Daimler in Germany.

Tariffs could add anything up to 10 percent to the price of some of his
products, an increase he can neither afford to absorb nor pass on. ``We
don't make 10 percent profit --- that's for sure,'' he said, adding,
``We won't be able to increase the price, because the customer will say,
`We will buy from the competition.' ''

The logic is to reduce the amount of products that are made in Britain
for continental Europe, and vice versa. Mr. Magal said that European
clients were asking him to consider moving production.

``With all the new projects we are getting now from Europe, they are
saying, `Why can't you put it in your plant in Europe? Why do you need
to do it in the U.K.?' '' he said.

If there is a hard Brexit, as seems increasingly likely, some production
might be shifted to Britain from the Continent to serve the British
market, but more would shift the other way, Mr. Magal said, adding that
some parts could be made only in Germany.

It is unclear what the impact will be on the Reading factory and its 230
jobs, or on another British factory that employs about 100 people.

Image

The Nissan car plant in Sunderland. British car manufacturing, which has
been booming in recent years after a period of decline, is particularly
vulnerable to tariffs that might follow Britain's exit from the
E.U.Credit...Scott Heppell/Agence France-Presse --- Getty Images

Mr. Magal is also worried about recruiting engineers, who are in short
supply in Britain, if new immigration restrictions are put in place. And
he thinks he may have to hire more administrative staff to deal with
possible tariffs and other red tape.

British car manufacturing, which has been booming in recent years after
a period of decline, is particularly vulnerable to tariffs. Key
components flow back and forth across the English Channel before a car
is built, and then the assembled vehicle itself is often exported. That
could mean paying tariffs repeatedly at different stages along the way.

Around three-quarters of the 1.6 million vehicles built in Britain in
2015 were sold abroad; about 57 percent of British car exports go to the
European Union.

Last year, Gareth Jones, president of Britain's Society of Motor
Manufacturers and Traders, called for continued ``membership of the
single market,'' and the ability to trade ``free from barriers and red
tape.'' He appealed to the government, ``Don't screw it up!''

British officials say, however, that the country imports more vehicles
from the European Union (800,000 a year from Germany alone) than it
exports to it, so would be a net beneficiary, in revenue terms, from a
tariff war.

Yet Germany has a bigger economic interest in keeping the 27 other
European Union nations together in a single market than it does in
cutting Britain a favorable deal that would undermine that market. A
hard-fought negotiation seems inevitable.

And the longer Brexit uncertainty prevails, the less attractive Britain
is likely to be for many manufacturers, including those from the auto
industry.

``The value chain crosses borders numerous times, so anything that gets
in the way --- tariff or nontariff barriers --- is likely to mean less
investment,'' Mr. Bailey said.

Advertisement

\protect\hyperlink{after-bottom}{Continue reading the main story}

\hypertarget{site-index}{%
\subsection{Site Index}\label{site-index}}

\hypertarget{site-information-navigation}{%
\subsection{Site Information
Navigation}\label{site-information-navigation}}

\begin{itemize}
\tightlist
\item
  \href{https://help.nytimes.com/hc/en-us/articles/115014792127-Copyright-notice}{©~2020~The
  New York Times Company}
\end{itemize}

\begin{itemize}
\tightlist
\item
  \href{https://www.nytco.com/}{NYTCo}
\item
  \href{https://help.nytimes.com/hc/en-us/articles/115015385887-Contact-Us}{Contact
  Us}
\item
  \href{https://www.nytco.com/careers/}{Work with us}
\item
  \href{https://nytmediakit.com/}{Advertise}
\item
  \href{http://www.tbrandstudio.com/}{T Brand Studio}
\item
  \href{https://www.nytimes.com/privacy/cookie-policy\#how-do-i-manage-trackers}{Your
  Ad Choices}
\item
  \href{https://www.nytimes.com/privacy}{Privacy}
\item
  \href{https://help.nytimes.com/hc/en-us/articles/115014893428-Terms-of-service}{Terms
  of Service}
\item
  \href{https://help.nytimes.com/hc/en-us/articles/115014893968-Terms-of-sale}{Terms
  of Sale}
\item
  \href{https://spiderbites.nytimes.com}{Site Map}
\item
  \href{https://help.nytimes.com/hc/en-us}{Help}
\item
  \href{https://www.nytimes.com/subscription?campaignId=37WXW}{Subscriptions}
\end{itemize}
