Sections

SEARCH

\protect\hyperlink{site-content}{Skip to
content}\protect\hyperlink{site-index}{Skip to site index}

\href{https://www.nytimes.com/section/politics}{Politics}

\href{https://myaccount.nytimes.com/auth/login?response_type=cookie\&client_id=vi}{}

\href{https://www.nytimes.com/section/todayspaper}{Today's Paper}

\href{/section/politics}{Politics}\textbar{}For Trump, Three Decades of
Chasing Deals in Russia

\url{https://nyti.ms/2jRJFLZ}

\begin{itemize}
\item
\item
\item
\item
\item
\end{itemize}

Advertisement

\protect\hyperlink{after-top}{Continue reading the main story}

Supported by

\protect\hyperlink{after-sponsor}{Continue reading the main story}

\hypertarget{for-trump-three-decades-of-chasing-deals-in-russia}{%
\section{For Trump, Three Decades of Chasing Deals in
Russia}\label{for-trump-three-decades-of-chasing-deals-in-russia}}

\includegraphics{https://static01.nyt.com/images/2017/01/14/us/15RUSSIA1/15RUSSIA1-articleLarge.jpg?quality=75\&auto=webp\&disable=upscale}

By \href{https://www.nytimes.com/by/megan-twohey}{Megan Twohey} and
\href{http://www.nytimes.com/by/steve-eder}{Steve Eder}

\begin{itemize}
\item
  Jan. 16, 2017
\item
  \begin{itemize}
  \item
  \item
  \item
  \item
  \item
  \end{itemize}
\end{itemize}

It was 2005, and Felix Sater, a Russian immigrant, was back in Moscow
pursuing an ambitious plan to build a Trump tower on the site of an old
pencil factory along the Moscow River that would offer hotel rooms,
condominiums and commercial office space.

Letters of intent had been signed and square footage was being analyzed.
``There was an opportunity to explore building Trump towers
internationally,'' said Mr. Sater, who worked for a New York-based
development company that was a partner with Donald J. Trump on a variety
of deals during that decade. ``And Russia was one of those countries.''

The president-elect's favorable comments about President Vladimir V.
Putin of Russia and the conclusion of United States intelligence
officials that Moscow acted to help Mr. Trump's campaign have focused
attention on Mr. Trump's business interests in Russia. Asked about the
issue at his news conference last week, Mr. Trump was emphatic on one
point: ``I have no dealings with Russia.'' And he repeated, ``I have no
deals that could happen in Russia because we've stayed away.''

The project on the old pencil factory site ultimately fizzled. And by
the time Mr. Trump entered the presidential race, he had failed to get
any real estate development off the ground in Russia. But it was not for
lack of trying.

Mr. Trump repeatedly sought business in Russia as far back as 1987, when
he traveled there to explore building a hotel. He applied for his
trademark in the country as early as 1996. And his children and
associates have appeared in Moscow over and over in search of joint
ventures, meeting with developers and government officials.

During a trip in 2006, Mr. Sater and two of Mr. Trump's children, Donald
Jr. and Ivanka, stayed at the historic Hotel National Moscow opposite
the Kremlin, connecting with potential partners over the course of
several days.

As recently as 2013, Mr. Trump himself was in Moscow. He had sold
Russian real estate developers the right to host his Miss Universe
pageant that year, and he used the visit as a chance to discuss
development deals, writing on Twitter at the time: ``TRUMP TOWER-MOSCOW
is next.''

As the Russian market opened up in the post-Soviet era, Mr. Trump and
his partners pursued Russians who were newly flush with cash to buy
apartments in Trump Towers in New York and Florida, sales that he
boasted about in a 2014 interview. ``I know the Russians better than
anybody,'' Mr. Trump told Michael D'Antonio, a Trump
\href{http://www.michaeldantonio.net/?page_id=160}{biographer} who
shared unpublished interview transcripts with The New York Times.

Seeking deals in Russia became part of a broader strategy to expand the
Trump brand worldwide. By the mid-2000s, Mr. Trump was transitioning to
mostly licensing his name to hotel, condominium and commercial towers
rather than building or investing in real estate himself. He discovered
that his name was especially attractive in developing countries where
the rising rich aspired to the type of ritzy glamour he personified.

While he nailed down ventures in the Philippines, India and elsewhere,
closing deals in Russia proved challenging. In 2008, Donald Trump Jr.
praised the opportunities in Russia, but also called it a ``scary
place'' to do business because of corruption and legal complications.

Mr. Sater said that American hotel chains that had moved into Russia did
so with straightforward agreements to manage hotels that other partners
owned. Mr. Trump, by contrast, was pursuing developments that included
residential or commercial offerings in which he would take a cut of
sales, terms that Russians were reluctant to embrace.

Even so, Mr. Trump said his efforts put him in contact with powerful
people there. ``I called it my weekend in Moscow,'' Mr. Trump said of
his 2013 trip to Moscow during a September 2015 interview on ``The Hugh
Hewitt Show.'' He added: ``I was with the top-level people, both
oligarchs and generals, and top of the government people. I can't go
further than that, but I will tell you that I met the top people, and
the relationship was extraordinary.''

When asked about Mr. Trump's claim that he had ``stayed away'' from
Russia, Alan Garten, general counsel for the Trump Organization, said it
was a fair characterization given that none of the development
opportunities ever materialized. Mr. Trump's interest in Russia, he
said, was no different from his attraction to other emerging markets in
which he investigated possible ventures. Mr. Garten did not respond to
questions about whom Mr. Trump met with in Moscow in 2013 and what was
discussed.

\hypertarget{stalking-deals}{%
\subsection{Stalking Deals}\label{stalking-deals}}

Ted Liebman, an architect based in New York, got the call in 1996. Mr.
Trump and Liggett-Ducat, an American tobacco company that owned property
in Moscow, wanted to build a high-end residential development near an
old Russian Olympic stadium. As they prepared to meet with officials in
Moscow, they needed sketches of the Trump tower they envisioned.

The architect scrambled to meet the request, handing over plans to Mr.
Trump at his Manhattan office. ``I hope we can do this,'' Mr. Liebman
recalled Mr. Trump telling him.

Soon after, Mr. Trump was in Russia, promoting the proposal and singing
the praises of the Russian market.

``I've seen cities all over the world. Some I've liked, some I
haven't,'' Mr. Trump said at a news conference in Moscow in 1996,
according to The Moscow Times. But he added that he didn't think he had
ever been ``as impressed with the potential of a city as I have been
with Moscow.''

Mr. Trump had been eyeing the potential for nearly a decade, expressing
interest to government officials ranging from the Soviet leader Mikhail
S. Gorbachev (they first met in Washington in 1987) to the military
figure Alexander Lebed.

\includegraphics{https://static01.nyt.com/images/2017/01/14/us/17RUSSIA1/15RUSSIA5-articleLarge.jpg?quality=75\&auto=webp\&disable=upscale}

The 1996 project never materialized, but by then Mr. Trump was already
well known in Russia. Moscow was in the midst of a construction boom,
which transformed the capital from a drab, post-Soviet expanse into a
sparkly modern city.

Yuri M. Luzhkov, Moscow's mayor at the time, said in an interview that
he had met with Mr. Trump and showed him plans for a massive underground
shopping mall just outside the Kremlin gates. Mr. Trump suggested
connecting it to the Metro, ``a very important observation,'' Mr.
Luzhkov said. Today, visitors to the Okhotny Ryad shopping center can go
straight from the Metro to the Calvin Klein store without venturing into
the cold.

In the following years, Mr. Trump's pursuit of Russia was strengthened
by a growing circle of partners and associates in Canada and the United
States who had roots in the region. Among them were Tevfik Arif, a
former Soviet-era commerce official originally from Kazakhstan who
founded a development company called the Bayrock Group, and Mr. Sater, a
partner in the firm, who had moved to New York from Russia as a child.

Bayrock was in Trump Tower, two floors below the Trump Organization.
While working to take Trump-branded towers to Arizona, Florida and New
York's SoHo neighborhood, Bayrock also began scouting for deals in
Russia and other countries.

``We looked at some very, very large properties in Russia,'' Mr. Sater
said. ``Think of a large Vegas high-rise.''

When Mr. Sater traveled to Moscow with Ivanka and Donald Trump Jr. to
meet with developers in 2006, he said their attitude could be summarized
as ``nice, big city, great. Let's do a deal here.''

Mr. Trump continued to work with Mr. Sater even after his role in a huge
stock manipulation scheme involving Mafia figures and Russian criminals
was revealed; Mr. Sater pleaded guilty and served as a government
informant.

Image

Mr. Trump with Tevfik Arif, center, and Felix Sater at the Trump SoHo
launch party in 2007. Mr. Trump discussed deals in Russia with their
development company, the Bayrock Group, but they never
materialized.Credit...Mark Von Holden/WireImage

In 2007, Mr. Trump discussed a deal for a Trump International Hotel and
Tower in Moscow that Bayrock had lined up with Russian investors.

``It would be a nonexclusive deal, so it would not have precluded me
from doing other deals in Moscow, which was very important to me,'' Mr.
Trump said in a deposition in an unsuccessful libel suit he brought
against Tim O'Brien, a journalist.

He claimed the development had fallen apart after Mr. O'Brien wrote a
book saying that Mr. Trump was worth far less than he claimed. But Mr.
Trump said he was close to striking another real estate deal in Moscow.

``We're going to do one fairly soon,'' he said. Moscow, he insisted
``will be one of the cities where we will be.''

\hypertarget{making-a-mark}{%
\subsection{Making a Mark}\label{making-a-mark}}

The Trump brand did appear in Russia, but not quite as the grand edifice
the real estate mogul had envisioned.

Trump Super Premium Vodka, with the shine of bottles glazed with
24-karat gold, was presented at the Millionaire's Fair in Moscow in
2007, and large orders for the spirits followed. The vodka was sold in
Russia as late as 2009, but eventually fizzled out. In a news release,
Mr. Trump heralded it as a ``tremendous achievement.''

He tried --- and failed --- to start a reality show in St. Petersburg in
2008 starring a Russian mixed martial arts fighter.

But real estate developments remained a constant goal. From 2006 to
2008, his company applied for several trademarks in Russia, including
Trump, Trump Tower, Trump International Hotel and Tower, and Trump Home,
according to a record search by Sojuzpatent, a Russian intellectual
property firm.

Image

In 2006, Mr. Sater and two of Mr. Trump's children, Donald Jr. and
Ivanka, stayed at the Hotel National Moscow across from the Kremlin,
connecting with potential partners over several days.Credit...Yuri
Kadobnov/Agence France-Presse --- Getty Images

Donald Trump Jr. became a regular presence in Russia. Speaking at a 2008
Manhattan real estate conference, he confessed to fears of doing
business in Russia, saying there is ``an issue of `Will I ever see my
money back out of that deal or can I actually trust the person I am
doing the deal with?''' according to coverage
\href{http://www.eturbonews.com/5008/executive-talk-donald-trump-jr-bullish-russia-and-few-emerging-ma}{of
his remarks} in eTurboNews.

But he told the Manhattan audience that ``I really prefer Moscow over
all cities in the world'' and that he had visited Russia a half-dozen
times in 18 months.

In 2011, he was still at it. ``Heading to the airport to go to Moscow
for business,'' he tweeted that year.

Mr. Trump himself was back in Moscow in 2013, attending the Miss
Universe pageant, which he owned with NBC.

Earlier that year, at the Miss USA pageant in Las Vegas, he had
announced that Aras and Emin Agalarov, father and son real estate
developers in Russia, would host the worldwide competition.

Erin Brady, that year's Miss USA winner, who watched the announcement
from backstage of the auditorium at Planet Hollywood Resort and Casino,
said the news was a surprise. She was expecting one of the Latin
American countries where beauty pageants are widely celebrated.

``I was like, `Wow, Russia, I never thought of that,''' she said.

Phil Ruffin, Mr. Trump's partner in the Trump International Hotel and
Tower in Las Vegas, said he was happy to lend him his new Global 5000
private plane for the trip. He and his wife met Mr. Trump in Moscow,
also checking into the Ritz-Carlton. Mr. Ruffin said he and Mr. Trump
had lunch at the hotel with the Agalarovs.

The Agalarovs also reportedly hosted a dinner for Mr. Trump the night of
the pageant, along with Herman Gref, a former Russian economy minister
who serves as chief executive of the state-controlled Sberbank PJSC,
according to Bloomberg News.

Talk of development deals swirled around the visit, and Mr. Trump sent
out his tweet, promising that Trump Tower Moscow was coming.

But the tower never appeared on the skyline.

Advertisement

\protect\hyperlink{after-bottom}{Continue reading the main story}

\hypertarget{site-index}{%
\subsection{Site Index}\label{site-index}}

\hypertarget{site-information-navigation}{%
\subsection{Site Information
Navigation}\label{site-information-navigation}}

\begin{itemize}
\tightlist
\item
  \href{https://help.nytimes.com/hc/en-us/articles/115014792127-Copyright-notice}{©~2020~The
  New York Times Company}
\end{itemize}

\begin{itemize}
\tightlist
\item
  \href{https://www.nytco.com/}{NYTCo}
\item
  \href{https://help.nytimes.com/hc/en-us/articles/115015385887-Contact-Us}{Contact
  Us}
\item
  \href{https://www.nytco.com/careers/}{Work with us}
\item
  \href{https://nytmediakit.com/}{Advertise}
\item
  \href{http://www.tbrandstudio.com/}{T Brand Studio}
\item
  \href{https://www.nytimes.com/privacy/cookie-policy\#how-do-i-manage-trackers}{Your
  Ad Choices}
\item
  \href{https://www.nytimes.com/privacy}{Privacy}
\item
  \href{https://help.nytimes.com/hc/en-us/articles/115014893428-Terms-of-service}{Terms
  of Service}
\item
  \href{https://help.nytimes.com/hc/en-us/articles/115014893968-Terms-of-sale}{Terms
  of Sale}
\item
  \href{https://spiderbites.nytimes.com}{Site Map}
\item
  \href{https://help.nytimes.com/hc/en-us}{Help}
\item
  \href{https://www.nytimes.com/subscription?campaignId=37WXW}{Subscriptions}
\end{itemize}
