Sections

SEARCH

\protect\hyperlink{site-content}{Skip to
content}\protect\hyperlink{site-index}{Skip to site index}

\href{https://www.nytimes.com/section/politics}{Politics}

\href{https://myaccount.nytimes.com/auth/login?response_type=cookie\&client_id=vi}{}

\href{https://www.nytimes.com/section/todayspaper}{Today's Paper}

\href{/section/politics}{Politics}\textbar{}Trump National Security Team
Gets a Slow Start

\url{https://nyti.ms/2jy023F}

\begin{itemize}
\item
\item
\item
\item
\item
\end{itemize}

Advertisement

\protect\hyperlink{after-top}{Continue reading the main story}

Supported by

\protect\hyperlink{after-sponsor}{Continue reading the main story}

\hypertarget{trump-national-security-team-gets-a-slow-start}{%
\section{Trump National Security Team Gets a Slow
Start}\label{trump-national-security-team-gets-a-slow-start}}

\includegraphics{https://static01.nyt.com/images/2017/01/19/us/19nsc1/19nsc1-videoSixteenByNine3000-v3.jpg}

By \href{http://www.nytimes.com/by/mark-landler}{Mark Landler}

\begin{itemize}
\item
  Jan. 18, 2017
\item
  \begin{itemize}
  \item
  \item
  \item
  \item
  \item
  \end{itemize}
\end{itemize}

WASHINGTON --- The Obama administration has written 275 briefing papers
for the incoming Trump administration: nearly 1,000 pages of classified
material on North Korea's nuclear program, the military campaign against
the Islamic State, tensions in the South China Sea, and every other kind
of threat the new team could face in its first weeks in office.

Nobody in the current administration knows whether anyone in the next
has read any of it.

Less than three days before President Obama turns the keys to the White
House, and the nuclear codes, over to President-elect Donald J. Trump,
Mr. Trump's transition staff has barely engaged with the National
Security Council below the most senior levels. His designated national
security adviser, Lt. Gen. Michael T. Flynn, has met four times with his
Obama counterpart, Susan E. Rice, most recently on Tuesday afternoon.

But the chronic upheaval in Mr. Trump's transition, a delay in
appointing senior National Security Council staff members, and a dearth
of people with security clearances have deprived the Trump team of weeks
of prep work on some of the most complex national security issues facing
the country.

``We really wanted to make sure there was nothing a new team needed to
know that we hadn't told them,'' Ms. Rice said in an interview. ``It
took them more time than we expected for them to be ready to engage with
us.'' Now, she added, ``we're racing to make up lost time.''

Ms. Rice insisted that she was confident the Trump administration would
have the information it needed by the time Mr. Trump was sworn in. The
process has settled down in recent days with the arrival of Keith
Kellogg, a retired three-star Army general whom Mr. Trump named as chief
of staff of the N.S.C. last month, and who is now running the
transition.

In a statement, Mr. Flynn said, ``Members of our incoming team have held
extensive meetings with their N.S.C. counterparts.'' He thanked Ms. Rice
for her ``cooperation and assistance.'' Last week, the two engaged in a
public display of harmony, shaking hands at a ``pass the baton''
conference sponsored by the United States Institute of Peace.

Still, officials from both the Obama and Trump teams acknowledged that
the transition had been rocky, in no small part because Mr. Trump's
defeat of Hillary Clinton caught both the outgoing and incoming
administrations so completely by surprise. Had Mrs. Clinton won, her
staff planned to place a transition team in the N.S.C. within a couple
of days.

In Mr. Trump's case, the first contact with the National Security
Council did not come until Nov. 22, two weeks after Election Day. That
delay was caused by
\href{http://www.nytimes.com/2016/11/12/us/politics/trump-cabinet.html}{the
purge of the original transition team} led by Gov. Chris Christie of New
Jersey. Among
\href{https://www.nytimes.com/2016/11/16/us/politics/trump-transition.html}{those
swept out} was Matthew Freedman, who had been chosen to run the N.S.C.
transition but quickly came under scrutiny because of his foreign
lobbying ties. Mr. Freedman's replacement, Marshall Billingslea, a
former Pentagon and State Department official, arrived in the West Wing
with six people, only two of whom had security clearances.

The Obama administration began meeting with that team after
Thanksgiving, but its lack of clearances meant that Mr. Trump's
emissaries could not read the materials that the Obama people had
prepared for them. The N.S.C. began creating unclassified versions of
the papers.

Then, in mid-December, there was another shake-up when Mr. Billingslea
was replaced by General Kellogg, who began meeting with his Obama
counterparts this month.

\includegraphics{https://static01.nyt.com/images/2017/01/19/us/19nsc2/19nsc2-articleInline.jpg?quality=75\&auto=webp\&disable=upscale}

By some accounts, the situation is even more fluid in the State
Department and the Pentagon, where the Trump team has devoted much of
its attention to lining up cabinet secretaries and will now have to win
Senate confirmation for dozens of other senior officials. Officials in
the State Department's transition office said that they had had only
intermittent contact with Trump representatives, and that those people
often changed.

But the snags in the National Security Council transition may be more
problematic, current and former officials said, because that is the
organization that advises the president on the most urgent foreign
policy issues, drawing together recommendations from the State
Department, the Pentagon, the Central Intelligence Agency and other
agencies.

``This is the nerve center of the White House,'' said David J. Rothkopf,
the chief executive and editor of the FP Group, who has written a
history of the N.S.C., ``Running the World: The Inside Story of the
National Security Council and the Architects of American Power.'' ``If
your brain isn't functioning, your arms and legs aren't going to
function.''

Because none of the jobs on the National Security Council require Senate
confirmation, it can in theory be staffed more quickly than other
government agencies. Much of the organization's policy staff is composed
of career civil servants, who are lent by their agencies to the White
House for a limited term of service. The Trump administration plans to
keep on most of these people, as the Obama administration did.

But that still leaves a cadre of senior directors, who run departments
that develop policy on Asia, Western Europe, Russia, nonproliferation
and other areas. While the Trump team has chosen several senior
directors, an official said, it has not yet announced them. And the
efforts have been complicated by outside distractions.

On Monday, Monica Crowley, a Fox News commentator and writer whom Mr.
Trump had named deputy national security adviser for strategic
communications,
\href{https://www.nytimes.com/2017/01/16/us/politics/monica-crowley-plagiarism.html}{announced}
that she would not take the post after
\href{http://money.cnn.com/interactive/news/kfile-trump-monica-crowley-plagiarized-multiple-sources-2012-book/}{CNN
disclosed} that she had plagiarized passages of her 2012 book, ``What
the (Bleep) Just Happened?'' Later,
\href{http://www.politico.com/magazine/story/2017/01/monica-crowley-plagiarism-phd-dissertation-columbia-214612}{Politico
reported} that she had done the same in her doctoral dissertation.

Mr. Trump's advisers initially defended Ms. Crowley, labeling the
plagiarism charges ``a politically motivated attack.'' But her position
became untenable because part of her job would have involved writing
speeches. HarperCollins, Ms. Crowley's publisher, said it
\href{https://www.nytimes.com/2017/01/10/business/harpercollins-pulls-monica-crowley-book-for-plagiarism.html}{would
withdraw the digital edition} of her book until she revised the copied
passages.

``The Monica Crowley episode is a sign of failure on several levels,''
Mr. Rothkopf said. ``She wasn't vetted enough, and they waited too long
after the plagiarism scandal broke to act.''

Ms. Rice said that ensuring a smooth transition had been one of her
major priorities for 2016, and that she had set a goal to ``meet or
exceed what the Bush administration did for us.''

N.S.C. officials began drafting briefing papers last summer. Some
focused on nuts and bolts: How do you arrange meetings? How do you
circulate information to the agencies? Others discussed the evolution of
administration policies or contingency planning for crises. Most were
three to five pages to make them easy to digest.

A Trump official said that members of the team had read some of the
memos and praised their quality. But there is an inherent tension in
transitions, particularly between two administrations with starkly
different political views. Officials in the Bush administration said
they doubted that the incoming Obama people read all the briefing papers
they prepared.

``It's difficult, because you campaigned on how these guys drove the car
into the ditch,'' said Peter D. Feaver, who served on the Bush N.S.C.
``Now, here are memos from the guys who were driving the car, and they
drove the car into a ditch.''

Advertisement

\protect\hyperlink{after-bottom}{Continue reading the main story}

\hypertarget{site-index}{%
\subsection{Site Index}\label{site-index}}

\hypertarget{site-information-navigation}{%
\subsection{Site Information
Navigation}\label{site-information-navigation}}

\begin{itemize}
\tightlist
\item
  \href{https://help.nytimes.com/hc/en-us/articles/115014792127-Copyright-notice}{©~2020~The
  New York Times Company}
\end{itemize}

\begin{itemize}
\tightlist
\item
  \href{https://www.nytco.com/}{NYTCo}
\item
  \href{https://help.nytimes.com/hc/en-us/articles/115015385887-Contact-Us}{Contact
  Us}
\item
  \href{https://www.nytco.com/careers/}{Work with us}
\item
  \href{https://nytmediakit.com/}{Advertise}
\item
  \href{http://www.tbrandstudio.com/}{T Brand Studio}
\item
  \href{https://www.nytimes.com/privacy/cookie-policy\#how-do-i-manage-trackers}{Your
  Ad Choices}
\item
  \href{https://www.nytimes.com/privacy}{Privacy}
\item
  \href{https://help.nytimes.com/hc/en-us/articles/115014893428-Terms-of-service}{Terms
  of Service}
\item
  \href{https://help.nytimes.com/hc/en-us/articles/115014893968-Terms-of-sale}{Terms
  of Sale}
\item
  \href{https://spiderbites.nytimes.com}{Site Map}
\item
  \href{https://help.nytimes.com/hc/en-us}{Help}
\item
  \href{https://www.nytimes.com/subscription?campaignId=37WXW}{Subscriptions}
\end{itemize}
