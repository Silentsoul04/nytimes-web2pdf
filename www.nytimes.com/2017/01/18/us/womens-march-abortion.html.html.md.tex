Sections

SEARCH

\protect\hyperlink{site-content}{Skip to
content}\protect\hyperlink{site-index}{Skip to site index}

\href{https://www.nytimes.com/section/us}{U.S.}

\href{https://myaccount.nytimes.com/auth/login?response_type=cookie\&client_id=vi}{}

\href{https://www.nytimes.com/section/todayspaper}{Today's Paper}

\href{/section/us}{U.S.}\textbar{}Views on Abortion Strain Calls for
Unity at Women's March on Washington

\url{https://nyti.ms/2jZygdk}

\begin{itemize}
\item
\item
\item
\item
\item
\end{itemize}

Advertisement

\protect\hyperlink{after-top}{Continue reading the main story}

Supported by

\protect\hyperlink{after-sponsor}{Continue reading the main story}

\hypertarget{views-on-abortion-strain-calls-for-unity-at-womens-march-on-washington}{%
\section{Views on Abortion Strain Calls for Unity at Women's March on
Washington}\label{views-on-abortion-strain-calls-for-unity-at-womens-march-on-washington}}

\includegraphics{https://static01.nyt.com/images/2017/01/18/us/18ABORTION-01/18ABORTION-01-articleLarge.jpg?quality=75\&auto=webp\&disable=upscale}

By \href{http://www.nytimes.com/by/sheryl-gay-stolberg}{Sheryl Gay
Stolberg}

\begin{itemize}
\item
  Jan. 18, 2017
\item
  \begin{itemize}
  \item
  \item
  \item
  \item
  \item
  \end{itemize}
\end{itemize}

WASHINGTON --- As a self-described feminist and law student who wants to
correct racial wrongs in the criminal justice system, Maria Lyon agrees
with Hillary Clinton that ``women's rights are human rights.'' But when
thousands of women
\href{https://www.nytimes.com/interactive/2017/01/10/us/politics/womens-march-guide.html}{march
on the capital} the day after Donald J. Trump is inaugurated as
president, she will not be there.

The reason: She opposes abortion.

``It's hard, because right now it feels like if you're pro-life, you're
anti-woman,'' said Ms. Lyon, 23, who studies law at the University of
Wisconsin. ``That's kind of the traditional rhetoric. It's like if you
care about women and you care about women's rights then you should be
pro-choice.''

Ms. Lyon is not the only feminist agonizing. Across the country, women
who oppose abortion --- including one in six women who supported Hillary
Clinton, according to a
\href{http://www.pewresearch.org/fact-tank/2016/11/03/women-drive-increase-in-democratic-support-for-legal-abortion/}{recent
survey by the Pew Research Center} \textbf{---} are demanding to be
officially included in Saturday's Women's March on Washington. But those
requests have been spurned, creating a
\href{https://www.nytimes.com/2017/01/09/us/womens-march-on-washington-opens-contentious-dialogues-about-race.html?_r=0}{bitter
rift among women's organizations}, and raising thorny questions about
what it means to be a feminist in 2017.

``If you want to come to the march you are coming with the understanding
that you respect a woman's right to choose,'' Linda Sarsour, a
Brooklyn-born Palestinian-American Muslim racial justice and
civil-rights activist, and one of four
\href{https://www.womensmarch.com/team/}{co-chairwomen of the march},
said in an interview on Tuesday.

Now these tensions, which have simmered behind the scenes, are spilling
out into the open.

On Monday, march organizers revoked so-called partnership status -- a
kind of official recognition --- for a Texas anti-abortion group, New
Wave Feminists. (Ms. Sarsour called the initial decision to include the
group ``a mistake.'')

Many anti-abortion women like Ms. Lyon are simply staying home. But one
group, Students for Life of America, which organizes college students to
oppose abortion, does plan to march, banners and all. Kristan Hawkins,
its president, a 31-year-old mother of four, said she, too, made a
request for partnership status, which she said was ignored.

She said her marchers will wear Go-Pro cameras: ``When they start
spitting and screaming at us it will be helpful.''

The march comes as abortion foes are newly emboldened by the election of
Mr. Trump. He won the support of 53 percent of white women, who flocked
to him primarily for economic reasons, pollsters said, but also because
he made explicit promises to overturn Roe v. Wade, the 1973 Supreme
Court decision that found a right to abortion within the privacy
protections of the Constitution.

\includegraphics{https://static01.nyt.com/images/2017/01/18/us/18ABORTION-02/18ABORTION-02-articleLarge.jpg?quality=75\&auto=webp\&disable=upscale}

Abortion opponents will mark the 44th anniversary of the landmark ruling
with their own event, the \href{http://marchforlife.org/}{March for
Life}, next week. Kellyanne Conway, who will be Mr. Trump's White House
counselor after becoming the first woman to manage a successful
presidential campaign,
\href{https://www.nytimes.com/2017/01/11/us/kellyanne-conway-anti-abortion-march.html}{will
speak} at that march.

At the same time, Republicans on Capitol Hill are threatening to
withdraw taxpayer funding for Planned Parenthood, the reproductive
rights behemoth. The group is the lead financial sponsor of the Women's
March and is helping to organize the event, along with Naral Pro-Choice
America and other abortion-rights organizations. Their participation
makes it clear: Abortion opponents are out of step with the march.

``Reproductive freedom or reproductive justice means that women decide
the fate of our own bodies,'' Gloria Steinem, an honorary co-chairwoman
of the march, wrote in an email message. She said if women ``want to
make decisions over their own bodies themselves, and want other women to
have the same power, then they should feel very welcome at the march.''

Yet many women do not. Among them is Charmaine Yoest, a vocal opponent
of abortion who is a senior fellow at American Values, a conservative
organization here.

``This is what we conservative women live with all the time, this idea
that we somehow aren't really women and we just reflect internalized
misogyny,'' she said. Of the march, she added: ``I don't think they
represent women. I think they are a wholly owned subsidiary of the
abortion movement.''

The relationship, and sometimes tension, between reproductive rights and
feminism goes back decades, to the inception of so-called second wave
feminism in the 1960s; the first wave culminated in women winning the
right to vote. Its pioneers, echoing the American left's fight for
racial justice and its opposition to the Vietnam War, embraced the birth
control pill as a way to give women more power over their own lives.
Thus a sexual revolution was born.

In 1973, just as middle-class women were abandoning homemaking for the
work force, inspired by writers like Ms. Steinem and Betty Friedan, the
Supreme Court handed down Roe v. Wade. That drove a wedge in the women's
movement, said Carole Joffe, a sociologist and reproductive rights
advocate at the University of California, San Francisco.

While many feminists saw the right to abortion as essential to women's
empowerment, she said, the decision also galvanized ``what we now call
the religious right'' --- Catholics, Protestants and evangelical
Christians, as well as churchgoing African-Americans, a number of whom
considered themselves liberal on other issues.

``I think this march will be discussed for a very long time, because
this march raises in a very powerful way the question of who can
rightfully be called a feminist, what does feminist organizing mean in
the 21st century,'' Ms. Joffe said. ``Is it even possible to have a
conception of American feminism that does not involve pro-choice and
pro-contraception?''

Image

Junghye Kim, the president of New Hampshire College Democrats and a
student at Dartmouth, has been organizing a group of students to come to
Washington for the March for Women. Naral is helping sponsor
them.Credit...Jacob Hannah for The New York Times

Public sentiment on abortion has remained remarkably consistent over
time. According to
\href{http://www.pewresearch.org/fact-tank/2017/01/03/about-seven-in-ten-americans-oppose-overturning-roe-v-wade/}{a
Pew survey} conducted after the election, 7 in 10 Americans oppose
overturning Roe v. Wade. Democrats have historically been more in favor
of abortion rights than Republicans, but despite a perception that women
are more supportive than men, there is often little daylight between the
sexes on the issue.

Celinda Lake, a Democratic pollster, said that is because the question
is inextricably tied to faith, and women tend to be more religious than
men.

With so many causes being represented --- from family leave legislation
to gay rights to better police treatment of minorities to access to
abortion \textbf{---} some disagreements were inevitable. Some Facebook
pages for the various marches around the country have broken out in
\href{https://www.nytimes.com/2017/01/09/us/womens-march-on-washington-opens-contentious-dialogues-about-race.html?_r=0}{contentious
conflicts over ethnicity and race}, including whether it was necessary
for white women to take a back seat in leading the march.

Officials of Planned Parenthood and Naral, however, say that as soon as
word of the march started spreading, their members across the country
began asking what they could do. Both groups have a powerful presence
here in the capital and are expert at organizing protests.

``There have been some stories about anti-choice folks trying to show up
in protest; I'm not worried about them --- they're going to be drowned
out by hundreds of thousands of folks,'' Mitchell Stille, Naral's
national political director, said in an interview. Asked if those women
could march in solidarity with the march, Mr. Stille sounded irked.

``Anti-choice women voted overwhelmingly for Donald Trump,'' he said.
``They got exactly what they wanted, so I'm not sure exactly what the
solidarity of that would be.''

Ms. Lyon, the law student, did not vote for Mr. Trump; she said she
could not support either him or Mrs. Clinton, and picked a third-party
candidate.

Living in the liberal-leaning city of Madison, she often feels isolated
for her views, which she said she came to through research that
reinforced her Catholic beliefs. She also accepts the church's ban on
contraception, and sees her position as consistent with her pursuit of
racial justice in prisons.

Asked to define feminism, she paused for a moment and said, ``I would
define it as the right to live out my womanhood.''

Advertisement

\protect\hyperlink{after-bottom}{Continue reading the main story}

\hypertarget{site-index}{%
\subsection{Site Index}\label{site-index}}

\hypertarget{site-information-navigation}{%
\subsection{Site Information
Navigation}\label{site-information-navigation}}

\begin{itemize}
\tightlist
\item
  \href{https://help.nytimes.com/hc/en-us/articles/115014792127-Copyright-notice}{©~2020~The
  New York Times Company}
\end{itemize}

\begin{itemize}
\tightlist
\item
  \href{https://www.nytco.com/}{NYTCo}
\item
  \href{https://help.nytimes.com/hc/en-us/articles/115015385887-Contact-Us}{Contact
  Us}
\item
  \href{https://www.nytco.com/careers/}{Work with us}
\item
  \href{https://nytmediakit.com/}{Advertise}
\item
  \href{http://www.tbrandstudio.com/}{T Brand Studio}
\item
  \href{https://www.nytimes.com/privacy/cookie-policy\#how-do-i-manage-trackers}{Your
  Ad Choices}
\item
  \href{https://www.nytimes.com/privacy}{Privacy}
\item
  \href{https://help.nytimes.com/hc/en-us/articles/115014893428-Terms-of-service}{Terms
  of Service}
\item
  \href{https://help.nytimes.com/hc/en-us/articles/115014893968-Terms-of-sale}{Terms
  of Sale}
\item
  \href{https://spiderbites.nytimes.com}{Site Map}
\item
  \href{https://help.nytimes.com/hc/en-us}{Help}
\item
  \href{https://www.nytimes.com/subscription?campaignId=37WXW}{Subscriptions}
\end{itemize}
