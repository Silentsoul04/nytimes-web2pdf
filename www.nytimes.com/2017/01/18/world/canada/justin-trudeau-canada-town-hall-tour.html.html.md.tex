Sections

SEARCH

\protect\hyperlink{site-content}{Skip to
content}\protect\hyperlink{site-index}{Skip to site index}

\href{https://www.nytimes.com/section/world/canada}{Canada}

\href{https://myaccount.nytimes.com/auth/login?response_type=cookie\&client_id=vi}{}

\href{https://www.nytimes.com/section/todayspaper}{Today's Paper}

\href{/section/world/canada}{Canada}\textbar{}Trudeau Hits the Road, and
Crowds and Questions Follow

\url{https://nyti.ms/2k12OeK}

\begin{itemize}
\item
\item
\item
\item
\item
\end{itemize}

Advertisement

\protect\hyperlink{after-top}{Continue reading the main story}

Supported by

\protect\hyperlink{after-sponsor}{Continue reading the main story}

\hypertarget{trudeau-hits-the-road-and-crowds-and-questions-follow}{%
\section{Trudeau Hits the Road, and Crowds and Questions
Follow}\label{trudeau-hits-the-road-and-crowds-and-questions-follow}}

\includegraphics{https://static01.nyt.com/images/2017/01/18/world/18Canada/18Canada-articleInline.jpg?quality=75\&auto=webp\&disable=upscale}

By \href{http://www.nytimes.com/by/ian-austen}{Ian Austen}

\begin{itemize}
\item
  Jan. 18, 2017
\item
  \begin{itemize}
  \item
  \item
  \item
  \item
  \item
  \end{itemize}
\end{itemize}

PETERBOROUGH, Ontario --- This time last year, Prime Minister Justin
Trudeau was both a novelty and a star at
\href{https://www.nytimes.com/2016/01/22/business/dealbook/justin-trudeau-carries-his-look-to-canada-image-to-davos.html}{the
World Economic Forum} in Davos, Switzerland. But rather than rub
shoulders with the billionaires, bankers and potentates again this year,
Mr. Trudeau has set off on a nationwide tour of banquet halls, cafes and
highway rest stops.

His decision to hang around with Canadian voters and schoolchildren
instead of the high and mighty comes as Mr. Trudeau's image as a
nothing-to-hide champion of the middle class is showing a few dents.

Some of those dings have come from his holiday plans: He and his family
spent part of the recent Christmas holidays on a private island in the
Bahamas as guests of the Aga Khan, the billionaire philanthropist and
spiritual leader of Ismaili Muslims. The parliamentary commissioner for
ethics and conflicts of interest said on Monday that she was
\href{https://www.nytimes.com/2017/01/16/world/americas/justin-trudeau-canada-aga-khan-ethics-investigation-mary-dawson.html}{investigating
the propriety of the trip}.

Mr. Trudeau was under fire even before the trip, for serving as the main
attraction at
\href{https://www.nytimes.com/2016/11/24/world/americas/canada-justin-trudeau-donors.html}{Liberal
party fund-raisers} where small crowds, often dominated by wealthy
investors and business executives, paid 1,500 Canadian dollars (\$1,147)
apiece for the chance to socialize with him. Events like that, common in
American politics, are seen as less savory in Canada, where campaign
contributions are tightly restricted.

``Justin Trudeau believes that laws don't apply to people like him,''
Rona Ambrose, the interim leader of the opposition Conservative Party,
\href{https://twitter.com/RonaAmbrose/status/821007536853807104}{wrote
on Twitter}on Monday. ``He's wrong. It's time he starts putting ordinary
Canadians first.''

Several polls show that Mr. Trudeau remains very popular, but even so,
his January road show has all the earmarks of a campaign to reconnect
with voters. At the first few stops, it has also shown Mr. Trudeau
practicing politics without a safety net.

The large crowds he draws have generally been friendly, but there have
also been tears, anger and pointed questions, including some for which
Mr. Trudeau acknowledges he does not have good answers.

Above all, there has been keen interest.

The setting booked for Mr. Trudeau's appearance in Peterborough,
Ontario, last Friday was the Evinrude Centre, a cavernous banquet hall
attached to a hockey rink complex. Even so, it was filled to its legal
capacity well before Mr. Trudeau arrived, and organizers had to start
turning people away. Several dozen latecomers stood outside in the
biting cold to listen to the proceedings on loudspeakers.

Mr. Trudeau's town-hall-style event in London, Ontario, later that day
had to be relocated three times to successively larger sites, and people
were still shut out. And his appearance in Halifax, Nova Scotia, on
Monday was eventually moved to a 3,000-seat arena.

The unemployment rate in Peterborough, a city of 74,600 in a popular
vacation area about an hour northeast of Toronto, is unusually low right
now at 4.5 percent. But many of its traditional large industrial
employers have either cut payrolls or moved work elsewhere since Canada
signed free trade agreements with the United States and Mexico.

The building across the street from the Evinrude Centre is an
illustration. These days it houses the
\href{http://www.canoemuseum.ca}{Canadian Canoe Museum,} which, as Mr.
Trudeau noted, includes a birch bark canoe and a buckskin jacket that
belonged to his father, Pierre Elliott Trudeau, who was also a Liberal
prime minister. But until 1990, the building was a factory producing
Evinrude and Johnson outboard motors. Bombardier Recreational Products
of Valcourt, Quebec, owns the company now, but the motors it sells in
Canada are made in the United States.

A university, a college, government and health care employment and
Peterborough's growing popularity with retirees have helped offset the
city's lost manufacturing jobs. But a sense of economic unease was still
evident in the questions Mr. Trudeau fielded from his audience.

He got an earful, for instance, about steep rises in electricity bills,
even though the province of Ontario sets the rates, not the national
government. One woman, who said she was a single mother and spoke on the
verge of tears, produced a recent monthly bill for 1,085 Canadian
dollars. Cries of ``shame, shame, shame'' echoed through the hall.

Standing in front of an oversize Canadian flag with the sleeves of his
light-blue dress shirt rolled up and the top button undone behind his
necktie, Mr. Trudeau acknowledged that many Canadians feel squeezed
financially.

``We are facing a big challenge,'' he said. ``The government should be
helping you, not harming you.''

In Halifax on Monday, he heard from voters about another looming
economic worry: a Trump administration in Washington that dislikes free
trade agreements and has hinted at new border taxes on imports from
Canada. That could be devastating for Canada's export-dependent economy,
especially its auto industry, which has been integrated with American
factories since 1965.

Mr. Trudeau said that he did not plan to change Canada's course to suit
Mr. Trump, but he avoided criticizing the incoming president.

``We both got elected on a commitment to help the middle class, and
we're going to be able to find common ground on doing the kinds of
things that will help ordinary families right across the continent,'' he
said.

Mr. Trudeau was challenged on a variety of issues, including Canada's
official policy of English and French bilingualism, water problems in
indigenous communities, the country's electoral system and immigration
--- the last from someone who wanted more refugees admitted more
quickly.

Conspicuously, no one in the crowd asked about the fund-raisers, and
references to the vacation with the Aga Khan were limited to mild
heckling from two middle-aged men at the back of the room. But those
matters continue to dominate news coverage in Canada, and they are what
the reporters covering Mr. Trudeau's tour are asking him about.

Mr. Trudeau has known the Aga Khan since childhood. The billionaire was
a friend of his father and served as a pallbearer at his father's
funeral, along with Fidel Castro. Canada took in many Ismaili Muslims
when they were expelled from Uganda by Idi Amin in 1972, and since then
the Aga Khan and his organizations have set up several institutions in
Canada, including
\href{http://www.akdn.org/where-we-work/north-america/canada/aga-khan-museum}{a
museum of Muslim art in Toronto}.

Mr. Trudeau's political opponents say that the vacation created a
conflict of interest because the government of Canada has donated
millions of dollars over the years to humanitarian aid programs
organized by charities headed by the Aga Khan. Under an agreement with
the previous Conservative government, the Canadian treasury also matched
a donation of 30 million Canadian dollars by the Aga Khan to set up a
\href{http://www.akdn.org/where-we-work/north-america/canada/global-centre-pluralism}{Global
Center for Pluralism}in Ottawa.

Once Mr. Trudeau had posed for his last selfie and lifted up the last
baby at the Peterborough stop, several attendees said they were pleased
with the event, including some who had questions about certain policies.

``I really enjoyed listening to Mr. Trudeau, love the fact that he's
very accessible to us. That's the big thing,'' said Laura Scaife, who
had hoped to ask the prime minister to give Canada's provinces more
money for health care. ``But he does have a tendency to wander just a
bit. I wish his answers had been a bit shorter, so that more people
could ask questions.''

After the public event came a short news conference and a brief session
with volunteers; then Mr. Trudeau waved goodbye. A crew took to the
stage to fold up the giant flag, and the prime minister's caravan of
seven black sport utility vehicles and vans set off for the next stop.

Advertisement

\protect\hyperlink{after-bottom}{Continue reading the main story}

\hypertarget{site-index}{%
\subsection{Site Index}\label{site-index}}

\hypertarget{site-information-navigation}{%
\subsection{Site Information
Navigation}\label{site-information-navigation}}

\begin{itemize}
\tightlist
\item
  \href{https://help.nytimes.com/hc/en-us/articles/115014792127-Copyright-notice}{©~2020~The
  New York Times Company}
\end{itemize}

\begin{itemize}
\tightlist
\item
  \href{https://www.nytco.com/}{NYTCo}
\item
  \href{https://help.nytimes.com/hc/en-us/articles/115015385887-Contact-Us}{Contact
  Us}
\item
  \href{https://www.nytco.com/careers/}{Work with us}
\item
  \href{https://nytmediakit.com/}{Advertise}
\item
  \href{http://www.tbrandstudio.com/}{T Brand Studio}
\item
  \href{https://www.nytimes.com/privacy/cookie-policy\#how-do-i-manage-trackers}{Your
  Ad Choices}
\item
  \href{https://www.nytimes.com/privacy}{Privacy}
\item
  \href{https://help.nytimes.com/hc/en-us/articles/115014893428-Terms-of-service}{Terms
  of Service}
\item
  \href{https://help.nytimes.com/hc/en-us/articles/115014893968-Terms-of-sale}{Terms
  of Sale}
\item
  \href{https://spiderbites.nytimes.com}{Site Map}
\item
  \href{https://help.nytimes.com/hc/en-us}{Help}
\item
  \href{https://www.nytimes.com/subscription?campaignId=37WXW}{Subscriptions}
\end{itemize}
