Sections

SEARCH

\protect\hyperlink{site-content}{Skip to
content}\protect\hyperlink{site-index}{Skip to site index}

\href{https://www.nytimes.com/section/politics}{Politics}

\href{https://myaccount.nytimes.com/auth/login?response_type=cookie\&client_id=vi}{}

\href{https://www.nytimes.com/section/todayspaper}{Today's Paper}

\href{/section/politics}{Politics}\textbar{}Inauguration Protesters and
Police Clash on Washington's Streets

\url{https://nyti.ms/2jHq84m}

\begin{itemize}
\item
\item
\item
\item
\item
\item
\end{itemize}

Advertisement

\protect\hyperlink{after-top}{Continue reading the main story}

Supported by

\protect\hyperlink{after-sponsor}{Continue reading the main story}

\hypertarget{inauguration-protesters-and-police-clash-on-washingtons-streets}{%
\section{Inauguration Protesters and Police Clash on Washington's
Streets}\label{inauguration-protesters-and-police-clash-on-washingtons-streets}}

\includegraphics{https://static01.nyt.com/images/2017/01/21/us/21protests/21protests-videoSixteenByNine3000.jpg}

By \href{http://www.nytimes.com/by/michael-s-schmidt}{Michael S.
Schmidt}, \href{http://www.nytimes.com/by/nick-corasaniti}{Nick
Corasaniti} and \href{http://www.nytimes.com/by/matt-flegenheimer}{Matt
Flegenheimer}

\begin{itemize}
\item
  Jan. 20, 2017
\item
  \begin{itemize}
  \item
  \item
  \item
  \item
  \item
  \item
  \end{itemize}
\end{itemize}

WASHINGTON --- A spate of violence erupted on Friday in the nation's
capital, as protesters damaged storefronts, threw rocks and bricks at
police officers and lit a limousine on fire.

Phalanxes of police officers used pepper spray, flash grenades and other
nonlethal crowd-control tools to disperse the protesters. By the end of
the day, six police officers had sustained minor injuries and more than
200 people had been arrested.

Many of the protesters were dressed in black, wore face masks and
carried flags associated with anti-fascist groups. They congregated on a
series of streets just blocks from the parade where Donald J. Trump
passed as he made his way to the White House for the first time as
president, their activities creating a distraction as television
networks played live footage of the clashes.

\includegraphics{https://static01.nyt.com/images/2017/01/21/us/21protests-2/21protests-2-articleInline.jpg?quality=75\&auto=webp\&disable=upscale}

The violence was focused not only on the police. Richard B. Spencer, a
leader of the so-called alt-right, a far-right fringe movement that
embraces white nationalism and a range of racist and anti-immigrant
positions, was punched in the face by a protester as Mr. Spencer gave an
interview on the street, according to a video posted on Twitter.

``We're not peaceful,'' said one of the masked protesters who, like many
others who clashed with the police, ran away after being approached by
reporters.

While the clashes occurred, thousands of peaceful protesters marched
across the country as they voiced anti-Trump slogans. In New York, seven
people were arrested when they blocked the sidewalk outside Trump Tower.

\href{https://www.nytimes.com/interactive/2017/01/17/us/inauguration-protests.html}{}

\includegraphics{https://static01.nyt.com/images/2017/01/17/us/trump-inauguration-protests-1484347704033/trump-inauguration-protests-1484347704033-thumbLarge.png}

\hypertarget{where-protests-are-happening-on-inauguration-day}{%
\subsection{Where Protests Are Happening on Inauguration
Day}\label{where-protests-are-happening-on-inauguration-day}}

President-elect Donald J. Trump's inauguration is expected to draw
thousands of protesters to Washington.

Protesters in San Francisco formed a blockade across train tracks,
leading to eight arrests, and chained themselves to the downtown offices
of Uber and Wells Fargo. In the central business district of Portland,
Ore., banks, clothing retailers and a jewelry store boarded up windows
and covered walls to limit vandalism. Chants in English and Spanish
broke out in front of the Capitol in Phoenix.

``This is one of the darkest days in the history of our country,'' said
Adelle Wallace, 75, during a rain-soaked march in Los Angeles.

The violence in Washington began about an hour before Mr. Trump was
sworn in at noon. Storefront windows at a Bank of America and a
Starbucks several blocks from the parade route were smashed, leading to
many arrests.

\includegraphics{https://static01.nyt.com/images/2017/01/21/us/21inauguration-corasiniti/21inauguration-corasiniti-videoSixteenByNineJumbo1600.jpg}

Around 2 p.m., as Mr. Trump ate lunch on Capitol Hill with lawmakers and
supporters, the protests expanded and turned violent. Protesters hurled
rocks and bricks at police officers several blocks from the parade
route. Officers with helmets and riot shields tried to disperse the
protesters by using flash grenades and pepper spray.

After being pushed back a block, protesters outside the Washington Post
building lit a fire in the middle of the street, smashed the windows of
a limousine and then lit it on fire. The police, using more flash
grenades, cleared a path for fire trucks as protesters retreated to a
park.

The violence attracted a throng of onlookers, journalists and peaceful
protesters who had marched earlier in the day.

Reed Arahood, 34, of Massachusetts, who had come to Washington to be
part of the peaceful protests, said she was ambivalent about the
violence.

``I don't think I have words to accurately describe how I feel about
them,'' she said about the protesters. She added that she felt
``solidarity'' but also was ``absolutely'' concerned about the image of
violent protests.

But over all the scene of thousands of protesters gave her hope.

``I feel pretty proud of the number who showed up today,'' she said.
``Looking strangers in the eye and knowing that we're together and
talking with people from all over the country who have come here to
express their concerns about what is going to happen in the next four
years and what is already happening in our country. I feel really good
about that.''

\includegraphics{https://static01.nyt.com/images/2017/01/21/universal/21inaugurationphotos10/21inaugurationphotos10-videoSixteenByNineJumbo1600.jpg}

Along several access points to view the inauguration, the protesters
hoped simply to put themselves in the way, locking arms, forming human
blockades in front of both public and ticketed entrances. The police
directed attendees around the corners of blockades, sometimes in single
file, forcing some ticketed attendees to wait nearly an hour in line to
trickle past the protests.

Content with their success disrupting the flow of attendees, about 150
protesters gathered downtown in McPherson Square, breaking off in groups
to march along I Street. An organizer advised two dozen people on the
day's aims: to disrupt Mr. Trump's celebration as much as possible ---
an objective, he predicted, that would rankle ``mainly police officers
and Trump supporters.''

``Police officers,'' a woman in the crowd grumbled, ``are Trump
supporters.''

A few attendees drummed on buckets, nodding at the instructions. Some
wondered about divine intervention as the day turned rainy. ``It's the
earth crying about the climate-denial president,'' said Elodie Huttner,
52.

Rallies have been planned all over the country all weekend, cresting
with a women's march in Washington on Saturday.

Despite the disruptions, some in Washington were able to find moments of
normalcy.

Molly Schwizer, 52, a government employee who had the day off, left the
quiet neighborhood in northwestern Washington where she lives to check
out the chaotic and heavily policed streets of downtown, where
protesters vied with Trump supporters.

``I wanted to see what this was all about,'' she said, motioning toward
Saks Off 5th, the discount branch of the New York department store.
``And,'' she added, ``I had some shopping to do.''

By midafternoon, she had seen enough of the protesters (``they should
clean up,'' she said) and the Trump supporters (no comment --- she does
have a government job, after all). Also, it was raining --- and so it
was time to shop.

Advertisement

\protect\hyperlink{after-bottom}{Continue reading the main story}

\hypertarget{site-index}{%
\subsection{Site Index}\label{site-index}}

\hypertarget{site-information-navigation}{%
\subsection{Site Information
Navigation}\label{site-information-navigation}}

\begin{itemize}
\tightlist
\item
  \href{https://help.nytimes.com/hc/en-us/articles/115014792127-Copyright-notice}{©~2020~The
  New York Times Company}
\end{itemize}

\begin{itemize}
\tightlist
\item
  \href{https://www.nytco.com/}{NYTCo}
\item
  \href{https://help.nytimes.com/hc/en-us/articles/115015385887-Contact-Us}{Contact
  Us}
\item
  \href{https://www.nytco.com/careers/}{Work with us}
\item
  \href{https://nytmediakit.com/}{Advertise}
\item
  \href{http://www.tbrandstudio.com/}{T Brand Studio}
\item
  \href{https://www.nytimes.com/privacy/cookie-policy\#how-do-i-manage-trackers}{Your
  Ad Choices}
\item
  \href{https://www.nytimes.com/privacy}{Privacy}
\item
  \href{https://help.nytimes.com/hc/en-us/articles/115014893428-Terms-of-service}{Terms
  of Service}
\item
  \href{https://help.nytimes.com/hc/en-us/articles/115014893968-Terms-of-sale}{Terms
  of Sale}
\item
  \href{https://spiderbites.nytimes.com}{Site Map}
\item
  \href{https://help.nytimes.com/hc/en-us}{Help}
\item
  \href{https://www.nytimes.com/subscription?campaignId=37WXW}{Subscriptions}
\end{itemize}
