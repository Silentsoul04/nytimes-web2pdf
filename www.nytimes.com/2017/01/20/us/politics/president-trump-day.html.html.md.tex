Sections

SEARCH

\protect\hyperlink{site-content}{Skip to
content}\protect\hyperlink{site-index}{Skip to site index}

\href{https://www.nytimes.com/section/politics}{Politics}

\href{https://myaccount.nytimes.com/auth/login?response_type=cookie\&client_id=vi}{}

\href{https://www.nytimes.com/section/todayspaper}{Today's Paper}

\href{/section/politics}{Politics}\textbar{}Trump Embraces the Look of
the Presidency

\url{https://nyti.ms/2jITgZ0}

\begin{itemize}
\item
\item
\item
\item
\item
\end{itemize}

Advertisement

\protect\hyperlink{after-top}{Continue reading the main story}

Supported by

\protect\hyperlink{after-sponsor}{Continue reading the main story}

\hypertarget{trump-embraces-the-look-of-the-presidency}{%
\section{Trump Embraces the Look of the
Presidency}\label{trump-embraces-the-look-of-the-presidency}}

\includegraphics{https://static01.nyt.com/images/2017/01/21/us/21trumpday/21trumpday-articleInline-v2.jpg?quality=75\&auto=webp\&disable=upscale}

By \href{http://www.nytimes.com/by/maggie-haberman}{Maggie Haberman} and
\href{https://www.nytimes.com/by/glenn-thrush}{Glenn Thrush}

\begin{itemize}
\item
  Jan. 20, 2017
\item
  \begin{itemize}
  \item
  \item
  \item
  \item
  \item
  \end{itemize}
\end{itemize}

WASHINGTON --- Donald J. Trump, proudly unpresidential in word and tweet
during the transition, readily embraced the pomp-and-patriotism visuals
of the presidency on Friday.

There were some images that even an appearance-obsessed branding
impresario could not control: a less-than-packed inaugural parade route,
an inauguration crowd markedly smaller than the one Barack Obama drew in
2009, and protesters who set a limousine on fire in the afternoon.

But Mr. Trump glided through hours of ceremony, intent on projecting the
image of a confident and unifying leader, even as scores of protesters
were being arrested on the streets of Washington and opponents tried to
heckle his Inaugural Address.

\href{https://www.nytimes.com/interactive/2017/01/20/us/politics/trump-inauguration-photo-zoomer.html}{}

\includegraphics{https://static01.nyt.com/images/2017/01/20/us/politics/trump-inauguration-photo-zoomer-1484950544134/trump-inauguration-photo-zoomer-1484950544134-square640.jpg}

\hypertarget{the-58th-presidential-inaugural-seating-chart}{%
\subsection{The 58th Presidential Inaugural Seating
Chart}\label{the-58th-presidential-inaugural-seating-chart}}

A who's who of President Donald J. Trump's Washington.

``I see my generals; they are central casting,'' Mr. Trump said during a
luncheon with legislators and military leaders in the Capitol shortly
after he was sworn in, paying them one of his highest compliments ---
one he has used to describe Vice President Mike Pence and Mitt Romney.

``If I'm doing a movie, I'd pick you general, General Mattis,'' Mr.
Trump added, gesturing to James N. Mattis, whom the Senate
\href{https://www.nytimes.com/2017/01/20/us/politics/trump-cabinet-confirmation-mattis-kelly.html}{confirmed
as defense secretary} later in the day.

Mr. Trump, who often watches his televised interviews with the sound
off, took pains to look like a central casting version of a president on
his first day in office.

``Trump loves optics, and he loves brands, and he has been burnishing
the optics of the presidency, which he thinks need to be dramatically
improved,'' said Christopher Ruddy, the chief executive of the
conservative Newsmax Media and a longtime friend of Mr. Trump.

The new president is a creature of habit who loves to control his
working environment, and the day of pageantry obscured the change from
private citizen to leader of the free world, which will begin in earnest
on Saturday. One of the biggest shifts: Staff members have swapped out a
new encrypted phone for his Android cellphone --- his electronic Linus
blanket --- which had hundreds of numbers of associates from whom he
seeks advice.

One friend of Mr. Trump said that his aides and security officials did
not want him to text, and that some are urging him to forgo his personal
Twitter account --- a staple of his campaign --- to use only the
official @POTUS handle created by Mr. Obama's team and controlled by
staff members.

Mr. Trump is resisting it, the friend said. The president polled the
crowd from the stage at the second inaugural ball about what he should
do and delighted in reminding people that he had defied his critics. At
the first ball, he danced with the first lady, Melania Trump, to the
Frank Sinatra song ``My Way'' after saying he found a moment at his
swearing-in ceremony ``like from a movie set, so beautiful.''

The day began for the new president with the ceremonial signing of the
guest book at Blair House, the White House guest residence, before he
crossed Pennsylvania Avenue for tea at the Executive Mansion with Mr.
Obama, whose legitimacy as president he had questioned for years.

\includegraphics{https://static01.nyt.com/images/2017/01/21/us/21trumpday-2/21trumpday-2-articleInline.jpg?quality=75\&auto=webp\&disable=upscale}

The greeting between the Obamas and the Trumps was camera-ready cordial,
if stiff, as Mrs. Trump gave Michelle Obama a flat blue Tiffany box ---
\href{https://www.nytimes.com/2017/01/20/us/politics/tiffany-melania-trump-michelle-obama.html}{its
contents undisclosed} --- and stood for a photograph of the couples.
Mrs. Obama, one of Hillary Clinton's most forceful surrogates during the
campaign, kept the new president at arm's length.

Mr. Trump, who is as allergic to confrontation in private as he is
bellicose in public, was surrounded on the inaugural platform at the
Capitol's east front by politicians of both parties whom he has derided,
including George W. Bush, Bill Clinton and Mrs. Clinton. In the farther
reaches of the crowd, some of his followers reprised his campaign rally
chant of ``lock her up.''

``Every four years, we gather on these steps to carry out the orderly
and peaceful transfer of power, and we are grateful to President Obama
and first lady Michelle Obama for their gracious aid throughout this
transition,'' Mr. Trump said. ``They have been magnificent.''

The tone then quickly shifted, as he excoriated both parties and vowed
to wrench power away from Washington elites. But later, he also
playfully teased the top Democrats in the House and in the Senate,
Representative Nancy Pelosi and Senator Chuck Schumer, the type of
back-patting and banter that amasses chits on Capitol Hill and that Mr.
Obama viewed with a whiff of contempt.

\href{https://www.nytimes.com/interactive/2017/01/20/us/politics/trump-inauguration-crowd.html}{}

\includegraphics{https://static01.nyt.com/images/2017/01/20/us/politics/trump-inauguration-crowd-1484943564224/trump-inauguration-crowd-1484943564224-square640.jpg}

\hypertarget{trumps-inauguration-vs-obamas-comparing-the-crowds}{%
\subsection{Trump's Inauguration vs. Obama's: Comparing the
Crowds}\label{trumps-inauguration-vs-obamas-comparing-the-crowds}}

Estimates put the crowd gathered for President Donald J. Trump's
inauguration at far less than President Obama's in 2009.

At the congressional luncheon, Mr. Trump basked in the ritualized
speeches and formalized bonhomie of the city he had pilloried in his
inaugural address. And he seemed genuinely excited when Mr. Schumer
presented him with two framed photos of the inauguration.

Then it was time to make nice with the Clintons, in a show of
new-president magnanimity that left many in the room wincing.

``There is something that I wanted to say because I was very honored,
very, very honored when I heard that President Bill Clinton and
Secretary Hillary Clinton was coming today, and I think that's
appropriate to say,'' said Mr. Trump, who, before the second
presidential debate, appeared with several women who had accused Mr.
Clinton of sexual misconduct.

``I'd like you to stand up,'' he said to the former first couple. ``I'd
like you to stand up.''

Later, Mr. Trump got about the business of erasing etchings of Mr. Obama
on government and in the White House. He signed executive orders,
including one that took aim at the Affordable Care Act. He installed new
carpet in the Oval Office that resembled the pattern there when George
W. Bush was in office. He restored to the office a bust of Winston
Churchill, controversially moved by Mr. Obama in the early days of his
presidency.

\includegraphics{https://static01.nyt.com/images/2017/01/21/us/21MEDIA-01p/21inauguration-trump-highlights-videoSixteenByNineJumbo1600-v2.jpg}

Throughout the day, images and words often clashed, as is often the case
with Mr. Trump. His America-in-decline speech, aimed at his base, echoed
the dark, antiglobalist pitch he delivered at the Republican convention
last summer.

``He wants to redevelop the brand of the presidency of the United
States,'' Mr. Ruddy said. ``He says the brand is tarnished. He's a brand
guy. His business was significantly about his brand. He built some
buildings, he ran some casinos, but Trump was all about the Trump brand,
and I think this is a brand job.''

But the presidency, as Mr. Trump himself emphasized on Friday, is a job
where performance matters.

The new president did little to tamp down expectations, telling the
audience that ``we will no longer accept politicians who are all talk
and no action, constantly complaining but never doing anything about it.
The time for empty talk is over.''

Some of Mr. Trump's most vocal supporters are intent on making sure he
keeps his word.

``If he delivers, they are going to build a statue for Donald Trump
that's bigger and more impressive than the one they did for Lincoln at
the memorial,'' the conservative radio host Alex Jones said as he walked
out of the inauguration ceremony.

Advertisement

\protect\hyperlink{after-bottom}{Continue reading the main story}

\hypertarget{site-index}{%
\subsection{Site Index}\label{site-index}}

\hypertarget{site-information-navigation}{%
\subsection{Site Information
Navigation}\label{site-information-navigation}}

\begin{itemize}
\tightlist
\item
  \href{https://help.nytimes.com/hc/en-us/articles/115014792127-Copyright-notice}{©~2020~The
  New York Times Company}
\end{itemize}

\begin{itemize}
\tightlist
\item
  \href{https://www.nytco.com/}{NYTCo}
\item
  \href{https://help.nytimes.com/hc/en-us/articles/115015385887-Contact-Us}{Contact
  Us}
\item
  \href{https://www.nytco.com/careers/}{Work with us}
\item
  \href{https://nytmediakit.com/}{Advertise}
\item
  \href{http://www.tbrandstudio.com/}{T Brand Studio}
\item
  \href{https://www.nytimes.com/privacy/cookie-policy\#how-do-i-manage-trackers}{Your
  Ad Choices}
\item
  \href{https://www.nytimes.com/privacy}{Privacy}
\item
  \href{https://help.nytimes.com/hc/en-us/articles/115014893428-Terms-of-service}{Terms
  of Service}
\item
  \href{https://help.nytimes.com/hc/en-us/articles/115014893968-Terms-of-sale}{Terms
  of Sale}
\item
  \href{https://spiderbites.nytimes.com}{Site Map}
\item
  \href{https://help.nytimes.com/hc/en-us}{Help}
\item
  \href{https://www.nytimes.com/subscription?campaignId=37WXW}{Subscriptions}
\end{itemize}
