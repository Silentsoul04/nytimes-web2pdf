Sections

SEARCH

\protect\hyperlink{site-content}{Skip to
content}\protect\hyperlink{site-index}{Skip to site index}

\href{https://www.nytimes.com/section/politics}{Politics}

\href{https://myaccount.nytimes.com/auth/login?response_type=cookie\&client_id=vi}{}

\href{https://www.nytimes.com/section/todayspaper}{Today's Paper}

\href{/section/politics}{Politics}\textbar{}With Echoes of the '30s,
Trump Resurrects a Hard-Line Vision of `America First'

\url{https://nyti.ms/2k9FtaX}

\begin{itemize}
\item
\item
\item
\item
\item
\end{itemize}

Advertisement

\protect\hyperlink{after-top}{Continue reading the main story}

Supported by

\protect\hyperlink{after-sponsor}{Continue reading the main story}

Washington Memo

\hypertarget{with-echoes-of-the-30s-trump-resurrects-a-hard-line-vision-of-america-first}{%
\section{With Echoes of the '30s, Trump Resurrects a Hard-Line Vision of
`America
First'}\label{with-echoes-of-the-30s-trump-resurrects-a-hard-line-vision-of-america-first}}

\includegraphics{https://static01.nyt.com/images/2017/01/21/us/21americafirst/21americafirst-videoSixteenByNineJumbo1600.jpg}

By \href{http://www.nytimes.com/by/david-e-sanger}{David E. Sanger}

\begin{itemize}
\item
  Jan. 20, 2017
\item
  \begin{itemize}
  \item
  \item
  \item
  \item
  \item
  \end{itemize}
\end{itemize}

WASHINGTON --- America, and the world, just found out what ``America
First'' means.

President Trump could have used his inaugural address to define one of
the touchstone phrases of his campaign in the most inclusive way,
arguing, as did many of his predecessors, that as the world's greatest
superpower rises, its partners will also prosper.

Instead, he chose a dark, hard-line alternative, one that appeared to
herald the end of a 70-year American experiment to shape a world that
would be eager to follow its lead. In Mr. Trump's vision, America's new
strategy is to win every transaction and confrontation. Gone are the
days, he said, when America extended its defensive umbrella without
compensation, or spent billions to try to lift the fortune of foreign
nations, with no easy-to-measure strategic benefits for the United
States.

``From this day forward, it's going to be only America first,'' he said,
in a line that resonated around the world as soon as he uttered it from
the steps of the Capitol. ``We must protect our borders from the ravages
of other countries making our products, stealing our companies and
destroying our jobs.''

The United States, he said, will no longer subsidize ``the armies of
other countries while allowing for the very sad depletion of our
military.''

While all American presidents pledge to defend America's interests first
--- that is the core of the presidential oath --- presidents of both
parties since the end of World War II have wrapped that effort in an
expansion of the liberal democratic order. Until today, American policy
has been a complete rejection of the America First rallying cry that the
famed flier Charles Lindbergh championed when, in the late 1930s, he
became one of the most prominent voices to keep the United States out of
Europe's wars, even if it meant abandoning the country's closest allies.

Mr. Trump has rejected comparisons with the earlier movement, with its
taint of Nazism and anti-Semitism.

\includegraphics{https://static01.nyt.com/images/2017/01/21/us/21MEDIA-01p/21inauguration-trump-highlights-videoSixteenByNineJumbo1600-v2.jpg}

After World War II, the United States buried the Lindbergh vision of
America First. The United Nations was born in San Francisco and raised
on the East River of Manhattan, an ambitious, if still unfulfilled,
experiment in shaping a liberal order. Lifting the vanquished nations of
World War II into democratic allies was the idea behind the Marshall
Plan, the creation of the World Bank and institutions to spread American
aid, technology and expertise around the world. And NATO was created to
instill a commitment to common defense, though Mr. Trump has accurately
observed that nearly seven decades later, many of its member nations do
not pull their weight.

Mr. Trump's defiant address made abundantly clear that his threat to
pull out of those institutions, if they continue to take advantage of
the United States' willingness to subsidize them, could soon be
translated into policy. All those decades of generosity, he said,
punching the air for emphasis, had turned America into a loser.

``We've made other countries rich,'' he said, ``while the wealth,
strength and confidence of our country has disappeared over the
horizon.'' The American middle class has suffered the most, he said,
finding its slice of the American dream ``redistributed across the
entire world.''

To those who helped build that global order, Mr. Trump's vow was at best
shortsighted. ``Truman and Acheson, and everyone who followed, based our
policy on a `world-first,' not an `America-first,' basis,'' said Richard
N. Haass, whose new book, ``A World in Disarray,'' argues that a more
granular, short-term view of American interests will ultimately fail.

``A narrow America First posture will prompt other countries to pursue
an equally narrow, independent foreign policy,'' he said after Mr.
Trump's speech, ``which will diminish U.S. influence and detract from
global prosperity.''

To Mr. Trump and his supporters, it is just that view that put America
on the slippery slope to obsolescence. As a builder of buildings, Mr.
Trump's return on investment has been easily measurable. So it is
unsurprising that he would grade America's performance on a scorecard in
which he totals up wins and losses.

Curiously, among the skeptics are his own appointees. His nominee for
defense secretary, Gen. James N. Mattis, strongly defended the
importance of NATO during his confirmation hearing. Both Rex W.
Tillerson, the nominee for secretary of state, and Nikki R. Haley, the
choice for ambassador to the United Nations, offered up paeans to the
need for robust American alliances, though Mr. Tillerson periodically
tacked back to concepts echoing Mr. Trump's.

And there is a question about whether the exact meaning of America First
will continue to evolve in Mr. Trump's mind.

He first talked about it in a March interview with The New York Times,
when asked whether that phrase was a good summation of his
foreign-policy views.

He thought for a moment. Then he agreed with this reporter's summation
of Mr. Trump's message that the world had been ``freeloading off of us
for many years'' and that he fundamentally mistrusted many foreigners,
both adversaries and some allies.

``Correct,'' he responded. Then he added, in his staccato style: ``Not
isolationist. I'm not isolationist, but I am `America First.' So I like
the expression.'' He soon began using it at almost every rally.

In another interview with The Times, on the eve of the Republican
National Convention, he offered a refinement. He said he did not mean
for the slogan to be taken the way Lindbergh meant it. ``It was used as
a brand-new, very modern term,'' he said. ``Meaning we are going to take
care of this country first before we worry about everybody else in the
world.''

\includegraphics{https://static01.nyt.com/images/2017/01/21/universal/21inaugurationphotos10/21inaugurationphotos10-videoSixteenByNineJumbo1600.jpg}

As Walter Russell Mead, a professor at Bard College and a scholar at the
conservative Hudson Institute, put it the other day, ``The fact that he
doesn't have a grounding in the prior use of the term is liberating.''

``If you said to the average American voter, `Do you think it's the job
of the president to put America first,' they say, `Yes, that's the
job.'''

But Mr. Mead said that formulation disregarded the reality that
``sometimes to achieve American interests, you have to work
cooperatively with other countries.'' And any such acknowledgment was
missing from Mr. Trump's speech on Friday.

Mr. Trump cast America's new role in the world as one of an aggrieved
superpower, not a power intent on changing the globe. There was no
condemnation of authoritarianism or fascism, no clarion call to defend
human rights around the world --- one of the commitments that John F.
Kennedy made in his famed address, delivered 56 years ago to the day, to
protect human rights ``at home and around the world.''

That was, of course, the prelude to Kennedy's most famous line: that
America would ``bear any burden, meet any hardship, support any friend,
oppose any foe to assure the survival and the success of liberty.''

But the America that elected Mr. Trump had concluded that it was no
longer willing to bear that burden --- or even to make the spread of
democracy the mission of the nation, as George W. Bush, who was sitting
behind Mr. Trump, vowed 12 years ago. Mr. Trump views American democracy
as a fine import for those who like it.

``We do not seek to impose our way of life on anyone,'' he said, ``but
rather to let it shine as an example for everyone to follow.''

Advertisement

\protect\hyperlink{after-bottom}{Continue reading the main story}

\hypertarget{site-index}{%
\subsection{Site Index}\label{site-index}}

\hypertarget{site-information-navigation}{%
\subsection{Site Information
Navigation}\label{site-information-navigation}}

\begin{itemize}
\tightlist
\item
  \href{https://help.nytimes.com/hc/en-us/articles/115014792127-Copyright-notice}{©~2020~The
  New York Times Company}
\end{itemize}

\begin{itemize}
\tightlist
\item
  \href{https://www.nytco.com/}{NYTCo}
\item
  \href{https://help.nytimes.com/hc/en-us/articles/115015385887-Contact-Us}{Contact
  Us}
\item
  \href{https://www.nytco.com/careers/}{Work with us}
\item
  \href{https://nytmediakit.com/}{Advertise}
\item
  \href{http://www.tbrandstudio.com/}{T Brand Studio}
\item
  \href{https://www.nytimes.com/privacy/cookie-policy\#how-do-i-manage-trackers}{Your
  Ad Choices}
\item
  \href{https://www.nytimes.com/privacy}{Privacy}
\item
  \href{https://help.nytimes.com/hc/en-us/articles/115014893428-Terms-of-service}{Terms
  of Service}
\item
  \href{https://help.nytimes.com/hc/en-us/articles/115014893968-Terms-of-sale}{Terms
  of Sale}
\item
  \href{https://spiderbites.nytimes.com}{Site Map}
\item
  \href{https://help.nytimes.com/hc/en-us}{Help}
\item
  \href{https://www.nytimes.com/subscription?campaignId=37WXW}{Subscriptions}
\end{itemize}
