Sections

SEARCH

\protect\hyperlink{site-content}{Skip to
content}\protect\hyperlink{site-index}{Skip to site index}

\href{https://www.nytimes.com/section/politics}{Politics}

\href{https://myaccount.nytimes.com/auth/login?response_type=cookie\&client_id=vi}{}

\href{https://www.nytimes.com/section/todayspaper}{Today's Paper}

\href{/section/politics}{Politics}\textbar{}With False Claims, Trump
Attacks Media on Turnout and Intelligence Rift

\url{https://nyti.ms/2kd6EBo}

\begin{itemize}
\item
\item
\item
\item
\item
\end{itemize}

Advertisement

\protect\hyperlink{after-top}{Continue reading the main story}

Supported by

\protect\hyperlink{after-sponsor}{Continue reading the main story}

\hypertarget{with-false-claims-trump-attacks-media-on-turnout-and-intelligence-rift}{%
\section{With False Claims, Trump Attacks Media on Turnout and
Intelligence
Rift}\label{with-false-claims-trump-attacks-media-on-turnout-and-intelligence-rift}}

\includegraphics{https://static01.nyt.com/images/2017/01/23/us/politics/23inauguration-crowds-vid/23inauguration-crowds-vid-videoSixteenByNine3000.jpg}

By \href{https://www.nytimes.com/by/julie-hirschfeld-davis}{Julie
Hirschfeld Davis} and
\href{http://www.nytimes.com/by/matthew-rosenberg}{Matthew Rosenberg}

\begin{itemize}
\item
  Jan. 21, 2017
\item
  \begin{itemize}
  \item
  \item
  \item
  \item
  \item
  \end{itemize}
\end{itemize}

WASHINGTON --- President Trump used his first full day in office on
Saturday to unleash a remarkably bitter attack on the news media,
falsely accusing journalists of both inventing a rift between him and
intelligence agencies and deliberately understating the size of his
inauguration crowd.

In a visit to the Central Intelligence Agency intended to showcase his
support for the intelligence community, Mr. Trump ignored his own
repeated public statements criticizing the intelligence community, a
group he compared to Nazis just over a week ago.

He also called journalists ``among the most dishonest human beings on
earth,'' and he said that up to 1.5 million people had attended his
inauguration, a claim that photographs disproved.

Later, at the White House, he dispatched Sean Spicer, the press
secretary, to the briefing room in the West Wing, where Mr. Spicer
scolded reporters and made a series of false statements.

He said news organizations had deliberately misstated the size of the
crowd at Mr. Trump's inauguration on Friday in an attempt to sow
divisions at a time when Mr. Trump was trying to unify the country,
warning that the new administration would hold them to account.

The statements from the new president and his spokesman came as
\href{https://www.nytimes.com/2017/01/21/us/women-march-protest-president-trump.html}{hundreds
of thousands of people protested against Mr. Trump}, a crowd that
\href{https://www.nytimes.com/interactive/2017/01/22/us/politics/womens-march-trump-crowd-estimates.html}{appeared
to dwarf} the one that gathered the day before when he was sworn in. It
was a striking display of invective and grievance at the dawn of a
presidency, usually a time when the White House works to set a tone of
national unity and to build confidence in a new leader.

\href{https://www.nytimes.com/interactive/2017/01/22/us/politics/womens-march-trump-crowd-estimates.html}{}

\includegraphics{https://static01.nyt.com/images/2017/01/22/us/politics/womens-march-trump-crowd-estimates-1485071976042/womens-march-trump-crowd-estimates-1485071976042-square640-v2.jpg}

\hypertarget{crowd-scientists-say-womens-march-in-washington-had-3-times-as-many-people-as-trumps-inauguration}{%
\subsection{Crowd Scientists Say Women's March in Washington Had 3 Times
as Many People as Trump's
Inauguration}\label{crowd-scientists-say-womens-march-in-washington-had-3-times-as-many-people-as-trumps-inauguration}}

Estimates by crowd scientists of attendance at events on Friday and
Saturday and how they calculated it.

Instead, the president and his team appeared embattled and defensive,
signaling that the pugnacious style Mr. Trump employed as a candidate
will persist now that he has ascended to the nation's highest office.

Saturday was supposed to be a day for Mr. Trump to mend fences with the
intelligence community, with an appearance at the C.I.A.'s headquarters
in Langley, Va. While he was lavish in his praise, the president focused
in his 15-minute speech on his complaints about news coverage of his
criticism of the nation's spy agencies, and meandered to other topics,
including the crowd size at his inauguration, his level of political
support, his mental age and his intellectual heft.

\includegraphics{https://static01.nyt.com/images/2017/01/21/us/trump-cia-1/trump-cia-1-videoSixteenByNine3000.jpg}

``I just want to let you know, I am so behind you,'' Mr. Trump told more
than 300 employees assembled in the lobby for his remarks.

In recent weeks, Mr. Trump has
\href{https://www.nytimes.com/2016/12/11/us/politics/trump-russia-democrats.html}{questioned
the intelligence agencies' conclusion}that Russia meddled in the United
States election on his behalf. After the
\href{https://www.nytimes.com/2017/01/11/us/politics/donald-trump-russia-intelligence.html}{disclosure
of a dossier with unsubstantiated claims} about Mr. Trump, he accused
the intelligence community of allowing the leak and
\href{https://twitter.com/realDonaldTrump/status/819164172781060096}{wrote
on Twitter}, ``Are we living in Nazi Germany?''

On Saturday, he said journalists were responsible for any suggestion
that he was not fully supportive of intelligence agencies' work.

``I have a running war with the media,'' Mr. Trump said. ``They are
among the most dishonest human beings on earth, and they sort of made it
sound like I had a feud with the intelligence community.''

``The reason you're the No. 1 stop is, it is exactly the opposite,'' Mr.
Trump added. ``I love you, I respect you, there's nobody I respect
more.''

Mr. Trump also took issue with news reports about the number of people
who attended his inauguration, complaining that the news media used
photographs of ``an empty field'' to make it seem as if his inauguration
did not draw many people.

``We caught them in a beauty,'' Mr. Trump said of the news media, ``and
I think they're going to pay a big price.''

Mr. Spicer said that Mr. Trump had drawn ``the largest audience to ever
witness an inauguration,'' a statement that photographs clearly show to
be false. Mr. Spicer said photographs of the inaugural ceremonies were
deliberately framed ``to minimize the enormous support that had gathered
on the National Mall,'' although he provided no proof of either
assertion.

Photographs of Barack Obama's inauguration in 2009 and of Mr. Trump's
\href{https://www.nytimes.com/interactive/2017/01/20/us/politics/trump-inauguration-crowd.html}{plainly
showed that the crowd on Friday was significantly smaller}, but Mr.
Spicer attributed that disparity to new white ground coverings he said
had caused empty areas to stand out and to security measures that had
blocked people from entering the Mall.

\href{https://www.nytimes.com/interactive/2017/01/20/us/politics/trump-inauguration-crowd.html}{}

\includegraphics{https://static01.nyt.com/images/2017/01/20/us/politics/trump-inauguration-crowd-1484943564224/trump-inauguration-crowd-1484943564224-square640.jpg}

\hypertarget{trumps-inauguration-vs-obamas-comparing-the-crowds}{%
\subsection{Trump's Inauguration vs. Obama's: Comparing the
Crowds}\label{trumps-inauguration-vs-obamas-comparing-the-crowds}}

Estimates put the crowd gathered for President Donald J. Trump's
inauguration at far less than President Obama's in 2009.

``These attempts to lessen the enthusiasm of the inauguration are
shameful and wrong,'' Mr. Spicer said. He also admonished a journalist
for erroneously reporting on Friday that Mr. Trump had removed a bust of
the Rev. Dr. Martin Luther King Jr. from the Oval Office, calling the
mistake --- which was corrected quickly --- ``egregious.''

And he incorrectly claimed that ridership on Washington's subway system
was higher than on Inauguration Day in 2013. In reality, there were
782,000 riders that year, compared with 571,000 riders this year,
according to figures from the Washington-area transit authority.

Mr. Spicer also said that security measures had been extended farther
down the National Mall this year, preventing ``hundreds of thousands of
people'' from viewing the ceremony. But the Secret Service said the
measures were largely unchanged this year, and there were few reports of
long lines or delays.

\includegraphics{https://static01.nyt.com/images/2017/01/22/us/22whbriefing-2/22whbriefing-2-videoSixteenByNine3000.jpg}

Commentary about the size of his inauguration crowd made Mr. Trump
increasingly angry on Friday, according to several people familiar with
his thinking.

On Saturday, Mr. Trump told his advisers that he wanted to push back
hard on ``dishonest media'' coverage --- mostly referring to a Twitter
post from a New York Times reporter showing side-by-side frames of Mr.
Trump's crowd and Mr. Obama's in 2009. But most of Mr. Trump's advisers
urged him to focus on the responsibilities of his office during his
first full day as president.

However, in his remarks at the C.I.A., he wandered off topic several
times, at various points telling the crowd he felt no older than 39 (he
is 70); reassuring anyone who questioned his intelligence by saying,
``I'm, like, a smart person''; and musing out loud about how many
intelligence workers backed his candidacy.

\includegraphics{https://static01.nyt.com/images/2017/01/22/us/22whbriefing/21inaugurationphotos11-articleInline.jpg?quality=75\&auto=webp\&disable=upscale}

``Probably almost everybody in this room voted for me, but I will not
ask you to raise your hands if you did,'' Mr. Trump said. ``We're all on
the same wavelength, folks.''

But most of his remarks were devoted to attacking the news media. And
Mr. Spicer picked up the theme later in the day in the White House
briefing room. But his appearance, according to the people familiar with
Mr. Trump's thinking, went too far, in the president's opinion.

Mr. Trump's appearance at the C.I.A. touched off a fierce reaction from
some current and former intelligence officials.

Nick Shapiro, who served as chief of staff to John O. Brennan, who
resigned Friday as the C.I.A. director, said Mr. Brennan ``is deeply
saddened and angered at Donald Trump's despicable display of
self-aggrandizement in front of C.I.A.'s Memorial Wall of Agency heroes.

``Brennan says that Trump should be ashamed of himself,'' Mr. Shapiro
added.

``I was heartened that the president gave a speech at C.I.A.,'' said
Michael V. Hayden, a former director of the C.I.A. and the National
Security Agency. ``It would have been even better if more of it had been
about C.I.A.''

Representative Adam B. Schiff of California, the ranking Democrat on the
House Intelligence Committee, said that he had had high hopes for Mr.
Trump's visit as a step to begin healing the relationship between the
president and the intelligence community, but that Mr. Trump's
meandering speech had dashed them.

``While standing in front of the stars representing C.I.A. personnel who
lost their lives in the service of their country --- hallowed ground ---
Trump gave little more than a perfunctory acknowledgment of their
service and sacrifice,'' Mr. Schiff said. ``He will need to do more than
use the agency memorial as a backdrop if he wants to earn the respect of
the men and women who provide the best intelligence in the world.''

Mr. Trump said nothing during the visit about how he had mocked the
C.I.A. and other intelligence agencies as ``the same people that said
Saddam Hussein had weapons of mass destruction.'' He did not mention his
apparent willingness to believe Julian Assange, the founder of
WikiLeaks, who is widely detested at the C.I.A., over his own
intelligence agencies.

He also did not say whether he would start receiving the daily
intelligence briefs that are prepared for the president. The agency sees
the president as its main audience, and his dismissal of the need for
daily briefings from the intelligence community has raised concerns
about morale among people who believe their work will not be respected
at the White House.

Since the election, hopes at the C.I.A. that the new administration
would bring an infusion of energy and ideas have given way to
trepidation about what Mr. Trump and his loyalists have planned. But the
nomination of Mike Pompeo, a former Army infantry officer who is well
versed in issues facing the intelligence community, to lead the C.I.A.
has been received positively at the agency.

``He has left the strong impression that he doesn't trust the
intelligence community and that he doesn't have tremendous regard for
their work,'' Mark M. Lowenthal, a retired C.I.A. analyst, said of Mr.
Trump. ``The obvious thing to do is to counter that by saying, `I value
you. I look forward to working with you.'''

``He called them Nazis,'' Mr. Lowenthal added, referring to Mr. Trump's
characterization of the intelligence community. Mr. Lowenthal said
Saturday's visit should have been ``a stroking expedition.''

Advertisement

\protect\hyperlink{after-bottom}{Continue reading the main story}

\hypertarget{site-index}{%
\subsection{Site Index}\label{site-index}}

\hypertarget{site-information-navigation}{%
\subsection{Site Information
Navigation}\label{site-information-navigation}}

\begin{itemize}
\tightlist
\item
  \href{https://help.nytimes.com/hc/en-us/articles/115014792127-Copyright-notice}{©~2020~The
  New York Times Company}
\end{itemize}

\begin{itemize}
\tightlist
\item
  \href{https://www.nytco.com/}{NYTCo}
\item
  \href{https://help.nytimes.com/hc/en-us/articles/115015385887-Contact-Us}{Contact
  Us}
\item
  \href{https://www.nytco.com/careers/}{Work with us}
\item
  \href{https://nytmediakit.com/}{Advertise}
\item
  \href{http://www.tbrandstudio.com/}{T Brand Studio}
\item
  \href{https://www.nytimes.com/privacy/cookie-policy\#how-do-i-manage-trackers}{Your
  Ad Choices}
\item
  \href{https://www.nytimes.com/privacy}{Privacy}
\item
  \href{https://help.nytimes.com/hc/en-us/articles/115014893428-Terms-of-service}{Terms
  of Service}
\item
  \href{https://help.nytimes.com/hc/en-us/articles/115014893968-Terms-of-sale}{Terms
  of Sale}
\item
  \href{https://spiderbites.nytimes.com}{Site Map}
\item
  \href{https://help.nytimes.com/hc/en-us}{Help}
\item
  \href{https://www.nytimes.com/subscription?campaignId=37WXW}{Subscriptions}
\end{itemize}
