Sections

SEARCH

\protect\hyperlink{site-content}{Skip to
content}\protect\hyperlink{site-index}{Skip to site index}

\href{https://www.nytimes.com/section/business}{Business}

\href{https://myaccount.nytimes.com/auth/login?response_type=cookie\&client_id=vi}{}

\href{https://www.nytimes.com/section/todayspaper}{Today's Paper}

\href{/section/business}{Business}\textbar{}Galaxy Note 7 Fires Caused
by Battery and Design Flaws, Samsung Says

\url{https://nyti.ms/2jOKHM1}

\begin{itemize}
\item
\item
\item
\item
\item
\end{itemize}

Advertisement

\protect\hyperlink{after-top}{Continue reading the main story}

Supported by

\protect\hyperlink{after-sponsor}{Continue reading the main story}

\hypertarget{galaxy-note-7-fires-caused-by-battery-and-design-flaws-samsung-says}{%
\section{Galaxy Note 7 Fires Caused by Battery and Design Flaws, Samsung
Says}\label{galaxy-note-7-fires-caused-by-battery-and-design-flaws-samsung-says}}

\includegraphics{https://static01.nyt.com/images/2017/01/23/business/23samsung_web1/23samsung_web1-articleInline.jpg?quality=75\&auto=webp\&disable=upscale}

By \href{https://www.nytimes.com/by/paul-mozur}{Paul Mozur}

\begin{itemize}
\item
  Jan. 22, 2017
\item
  \begin{itemize}
  \item
  \item
  \item
  \item
  \item
  \end{itemize}
\end{itemize}

HONG KONG --- Samsung, the South Korean conglomerate, blamed battery
manufacturing problems and design flaws for the embarrassing and
\href{https://www.nytimes.com/2016/10/12/business/international/samsung-galaxy-note7-terminated.html?_r=0}{costly
failure of its Galaxy Note 7 smartphone} and apologized to its customers
and suppliers.

In a news conference that took place on Monday morning in South Korea,
Samsung and outside experts said batteries made by two suppliers
contained flaws that allowed the phones to overheat and in several cases
catch fire. But they also cited what they said were flaws in the design
of the phone, including an unusually thin lining between the electrodes
of the battery.

Samsung said it would form an outside battery advisory group and add
teams focused on the quality assurance of each core component of the
device.

``We are taking responsibility for our failure to ultimately identify
and verify the issues arising out of the battery design and
manufacturing process prior to the launch of the Note 7,'' said Koh
Dong-jin, president of Samsung's mobile communications business.

During the almost hourlong presentation, Samsung offered an extensive
technical explanation of the problems with the battery but little
insight into the breakdowns that caused the company to fail to identify
the problems. Mr. Koh said the lessons the company had learned had been
integrated into its processes and culture, yet offered no explanation of
how the culture would change or what the problems with the culture were.

The cancellation of the Galaxy Note 7 has been an unprecedented public
relations disaster for Samsung, the world's largest maker of
smartphones. It has also cost the company billions of dollars and, for
some critics in South Korea, even called into question the very business
model that has made Samsung so successful.

The way the company handled the recall also angered regulators and led
to confusion as it tried to get back millions of phones around the
world.

Part of the problem was that with the Note 7, Samsung had pushed itself
to the limit, company officials said. It rushed the Note 7 to the market
before Apple rolled out its iPhone 7. The accelerated production was
also driven by fear; Huawei, Xiaomi and other Chinese cellphone makers
were fast catching up. By packing the Note 7 with new features, like
waterproofing and iris-scanning for added security, Samsung also wanted
to prove that it was more than a fast follower.

\includegraphics{https://static01.nyt.com/images/2017/01/23/business/23samsung_web2/23samsung_web2-articleInline.jpg?quality=75\&auto=webp\&disable=upscale}

Although Samsung mostly pointed to manufacturing failures on Monday,
battery scientists say aggressive design decisions made problems more
likely. In the Note 7, Samsung opted for an exceptionally thin separator
in its battery. This critical component, which sits between the two
electrodes in a battery, can cause fires if it breaks down, varies in
thickness or is damaged by outside pressure. Samsung's choice to push
the limits of battery technology left little safety margin in the event
of a problem, like pressure on a smartphone casing, two battery
scientists said.

``The management pushed their engineers to make the battery separator
really thin,'' said Qichao Hu, founder of the battery start-up
SolidEnergy Systems. He added that doing so could increase the
likelihood of fires or explosions in batteries.

The same design flaw was identified by UL, a safety science company that
Samsung brought in to do an outside analysis. UL also said the high
energy density of the battery design meant more severe problems when a
breakdown occurred.

In addition to the design flaws, Samsung and outside experts said
manufacturing problems were often directly to blame. For example, the
initial fires were caused in part by a pinching of the top corner of the
battery by the pouch that held it. The batteries that came from a second
supplier in phones issued after the recall had defects in the welding,
and some also lacked protective tape.

If Samsung's overzealous insistence on speed and internal pressures to
outdo rivals were partly to blame for the Note 7's flaws, others said
the way the company had handled the situation indicated much broader
management problems.

``The rather poor way they handled the first recall suggests that they
have trouble accepting problems until they become quite big and they
have no choice but to face them,'' said Willy C. Shih, a professor at
the Harvard Business School. ``This time, it will really call into
question how they communicate problems, whether management is open to
hearing things from the front line.''

When reports of Note 7s catching fire began accumulating, Samsung
quickly blamed faulty batteries from one of its two suppliers, Samsung
SDI. In early September, it made a bold decision to recall 2.5 million
devices globally. It continued to ship Note 7s with batteries from the
other supplier, ATL, offering them as safe replacements.

But some of those began catching fire, too.

Officials from the United States' Consumer Product Safety Commission
were angered; the commission had approved Samsung's initial recall,
trusting Samsung's assurance that the replacement model was safe and
knowing that Samsung had no other Note 7 battery supplier. Consumers
began ridiculing the Samsung device as the ``Death Note 7.'' A video
clip quickly spread online featuring a game character throwing Note 7s
as explosives in an urban battle zone.

On Oct. 6, a Southwest Airlines plane was evacuated after a Note 7 began
smoking.

The decision to ditch the Note 7 cost Samsung an estimated 7 trillion
won, or \$6.2 billion.

An editorial in South Korea's leading daily newspaper, The Chosun Ilbo,
pointed to even greater costs. ``The Galaxy Note disaster,'' the article
read, ``shows that the business model that brought Samsung success after
success has reached its limit.''

Advertisement

\protect\hyperlink{after-bottom}{Continue reading the main story}

\hypertarget{site-index}{%
\subsection{Site Index}\label{site-index}}

\hypertarget{site-information-navigation}{%
\subsection{Site Information
Navigation}\label{site-information-navigation}}

\begin{itemize}
\tightlist
\item
  \href{https://help.nytimes.com/hc/en-us/articles/115014792127-Copyright-notice}{©~2020~The
  New York Times Company}
\end{itemize}

\begin{itemize}
\tightlist
\item
  \href{https://www.nytco.com/}{NYTCo}
\item
  \href{https://help.nytimes.com/hc/en-us/articles/115015385887-Contact-Us}{Contact
  Us}
\item
  \href{https://www.nytco.com/careers/}{Work with us}
\item
  \href{https://nytmediakit.com/}{Advertise}
\item
  \href{http://www.tbrandstudio.com/}{T Brand Studio}
\item
  \href{https://www.nytimes.com/privacy/cookie-policy\#how-do-i-manage-trackers}{Your
  Ad Choices}
\item
  \href{https://www.nytimes.com/privacy}{Privacy}
\item
  \href{https://help.nytimes.com/hc/en-us/articles/115014893428-Terms-of-service}{Terms
  of Service}
\item
  \href{https://help.nytimes.com/hc/en-us/articles/115014893968-Terms-of-sale}{Terms
  of Sale}
\item
  \href{https://spiderbites.nytimes.com}{Site Map}
\item
  \href{https://help.nytimes.com/hc/en-us}{Help}
\item
  \href{https://www.nytimes.com/subscription?campaignId=37WXW}{Subscriptions}
\end{itemize}
