Sections

SEARCH

\protect\hyperlink{site-content}{Skip to
content}\protect\hyperlink{site-index}{Skip to site index}

\href{https://www.nytimes.com/section/business}{Business}

\href{https://myaccount.nytimes.com/auth/login?response_type=cookie\&client_id=vi}{}

\href{https://www.nytimes.com/section/todayspaper}{Today's Paper}

\href{/section/business}{Business}\textbar{}Samsung Galaxy Note 7 Crisis
Signals Problems at Korea Inc.

\url{https://nyti.ms/2jPHhZD}

\begin{itemize}
\item
\item
\item
\item
\item
\end{itemize}

Advertisement

\protect\hyperlink{after-top}{Continue reading the main story}

Supported by

\protect\hyperlink{after-sponsor}{Continue reading the main story}

\hypertarget{samsung-galaxy-note-7-crisis-signals-problems-at-korea-inc}{%
\section{Samsung Galaxy Note 7 Crisis Signals Problems at Korea
Inc.}\label{samsung-galaxy-note-7-crisis-signals-problems-at-korea-inc}}

\includegraphics{https://static01.nyt.com/images/2017/01/24/world/24SAMSUNG-1/24SAMSUNG-1-articleInline.jpg?quality=75\&auto=webp\&disable=upscale}

By \href{http://www.nytimes.com/by/choe-sang-hun}{Choe Sang-Hun} and
\href{https://www.nytimes.com/by/paul-mozur}{Paul Mozur}

\begin{itemize}
\item
  Jan. 23, 2017
\item
  \begin{itemize}
  \item
  \item
  \item
  \item
  \item
  \end{itemize}
\end{itemize}

SEOUL, South Korea --- Before a world audience watching online, Samsung
on Monday
\href{https://www.nytimes.com/2017/01/22/business/samsung-galaxy-note-7-battery-fires-report.html}{offered}
details and schematics showing how its Galaxy Note 7 smartphone went
from cutting-edge technology to a
\href{https://www.nytimes.com/2016/10/28/business/samsung-galaxy-note-7-profit.html}{costly},
\href{https://www.nytimes.com/2016/10/12/business/international/samsung-galaxy-note7-terminated.html}{combustible
failure}.

But for many of the company's critics, the most interesting part of the
presentation was what Samsung did not say: How could such a
technologically advanced titan --- a symbol of South Korea's
considerable industrial might --- allow the problems to happen to begin
with?

The answer to that question gets to deep shortfalls that former
employees, suppliers and others who watch the company say may have
contributed to the incident. Samsung, like South Korea as a whole,
fosters a top-down, hidebound culture that stifles innovation and buries
festering problems, they say.

For those critics, these problems have come to light through another
front: politics. Samsung
\href{https://www.nytimes.com/2017/01/15/world/asia/south-korea-samsung-arrest-jay-lee-park-geun-hye.html}{has
been caught up} in a scandal surrounding the country's president, which
they say illustrates a hierarchical culture that tends to micromanage
away creativity and insulate family-run business empires from
accountability and competition.

``The Korean economy as a whole has reached a kind of limit,'' said Park
Sang-in, a Seoul National University economics professor who thinks that
the cozy relationship between the government and business is stifling
innovation in South Korea.

Mr. Park pointed to a decision by a South Korean court last week
\href{https://www.nytimes.com/2017/01/18/world/asia/samsung-korea-president-impeachment.html}{to
block} the arrest of Jay Y. Lee, Samsung's de facto leader, after a
prosecutor sought a warrant accusing Mr. Lee of bribery in relation to
the presidential scandal. A string of major South Korean executives,
including Mr. Lee's own father, have been
\href{http://www.nytimes.com/2009/12/30/business/global/30samsung.html}{pardoned}
or had
\href{http://www.nytimes.com/2008/07/16/business/worldbusiness/16iht-samsung.1.14535273.html}{sentences
suspended} after being convicted of wrongdoing over the past decade.

Samsung has said Mr. Lee did nothing illegal. But corporate executives
in recent years have said Mr. Lee has been working
\href{https://www.nytimes.com/2017/01/16/business/lee-jae-yong-samsung.html}{to
loosen up} a top-down corporate culture. Under him, Samsung began a
campaign to root out harsh and often violent language supervisors used
with their staff, a practice common in the South Korean corporate world.
Samsung officials described Mr. Lee as a polite, casual leader who
encouraged employees to use English more widely in internal
communications.

Though Samsung did not address its corporate culture directly in its
discussion of the Note 7, it apologized and detailed new steps it would
take to stop future problems, including naming a board of battery
advisers.

``To produce an innovative Galaxy Note 7, we set the goals on battery
specifications,'' D. J. Koh, Samsung's mobile chief, said on Monday.
``We now feel a painful responsibility for failing to test and confirm
that there were problems in the design and manufacturing of batteries
before we put the product out to the market.''

Despite the troubled handset, the company said on Tuesday that profit
more than doubled in the last three months of 2016, to about \$7.93
billion. The increase was driven by chip sales, and gave Samsung the
financial firepower to announce a nearly \$8 billion share buyback
program.

Over the past decade Samsung and South Korea have been widely viewed as
a model of forward-thinking, technological prowess. In Asia it was
viewed as the exception to the sluggish economic growth of nearby Taiwan
and Japan, whose once world-beating electronic makers have been in
decline.

So large and influential is Samsung that some worried South Koreans call
their own country the
``\href{https://www.nytimes.com/2016/10/23/world/asia/galaxy-note-7-recall-south-korea-samsung.html}{Republic
of Samsung}.'' The company is responsible for 20 percent of South Korean
exports, and any blow to its success often raises anxiety about the
overall health of the country's economic prospects.

But as with the rest of the country, a shake-up has been slow to come.
In recent years the company has run initiatives to push back against
what is widely described as a rigid, top-down management system. Samsung
engineers and midlevel managers are seldom allowed to second-guess
management goals set by top bosses, former employees say.

Pressures rose after Mr. Lee's father, Lee Kun-hee, its chairman,
suffered a stroke in 2014 and fell into a coma. One engineer in the
United States who works with Samsung suppliers on projects, including
the Note 7, said Samsung's no-questions-asked corporate culture had
grown more inflexible in recent years.

``In the Samsung culture, managers constantly feel pressured to prove
themselves with short-term achievements,'' said Kim Jin-baek, who worked
at Samsung until 2010 before becoming a professor at the business school
of Chung-Ang University in Seoul. ``Executives fret that they may not be
able to meet the goals and lose their jobs, even when they know the
goals are excessive.''

With the Note 7, Samsung pushed its business model, as well as its
technology, to the limit, according to Samsung officials, who spoke on
the condition of anonymity while the Note 7 investigation was being
completed. Driven by the desire to prove it was more than a fast
follower of Apple, Samsung rushed the Note 7 to market ahead of Apple's
iPhone 7. To fend off Chinese competitors like Huawei and Xiaomi, it
packed the phone with new features, like waterproof technology and
iris-scanning for added security.

But battery scientists, including those who spoke at the announcement,
said aggressive decisions in designing the batteries made problems more
likely. Pushing to make the battery thinner and more powerful, Samsung
opted for an exceptionally thin separator in its battery. As the
critical component that separates the positive and negative electrodes
in a battery, separators can cause fires if they break down or contain
flaws.

UL, a safety science company contracted by Samsung to investigate the
battery, said the thin separator offered poor protection against
defects. It also said that the high energy density of the battery could
increase the severity of failures when they occurred.

Samsung's insistence on speed and internal pressures to outdo rivals in
part signal a breakdown in the ability to truly innovate and push out
new ideas, critics say. In place of big new ideas, Samsung focused on
maxing out the capability of components like the battery. That
philosophy, which worked to keep Samsung on the heels of the likes of
Apple, simply is not as effective as Samsung tries to push ahead, they
argue.

A similar strain can be felt in other parts of South Korea's economy.
Even the symbols held up as signs of the country's forward thinking,
like speedy internet and eye-popping creative mobile apps, come in part
from the support of the government. While that has helped create new
companies, workers in the tech industry argue that approach also ensures
start-ups do not challenge the country's biggest companies.

``The government is trying to pick the next Steve Jobs,'' said Mr. Park,
the Seoul National University professor. ``You cannot pick the next
Steve Jobs. You need to allow somebody to achieve that.''

Tim Chae, a venture capital investor in South Korea and partner at 500
Startups, a venture fund, cites another government program called TIPS.
The government program is designed to support new companies with
funding. Yet with the money come rules, he said, such as funding that
can be used to pay only certain types of employees --- engineers are
included, but sales personnel are not.

While he acknowledged that the government should keep checks on taxpayer
money, Mr. Chae said it should lean toward more freedom: ``Just give
them the money and trust them to use it right.''

Mr. Chae highlighted other problems. Many of South Korea's most talented
workers are sucked up into companies like Samsung, where they imbibe its
corporate culture, making it harder for them to work elsewhere or
develop the right mentality to start their own company, he said.

That top-down culture that pervades both Samsung and South Korea hints
at a deeper question embedded in the accusations about the company's
involvement in the unfolding graft scandal: Would Samsung itself not be
better off if South Korean regulators treated it more harshly?

In October, at Samsung's request, the Korea Testing Laboratory
investigated a Note 7 that had caught fire. Within a few hours, it
issued a report indicating the fire was due to an ``external shock.''

Its chief, Lee Won-bok, now regrets that decision, calling it ``too
hasty.''

``They did the test for Samsung and produced the findings that Samsung
wanted,'' Woo Won-shik, an opposition lawmaker, said during a
parliamentary hearing. ``This Republic of Samsung phenomenon actually
hurts its international credibility.''

Advertisement

\protect\hyperlink{after-bottom}{Continue reading the main story}

\hypertarget{site-index}{%
\subsection{Site Index}\label{site-index}}

\hypertarget{site-information-navigation}{%
\subsection{Site Information
Navigation}\label{site-information-navigation}}

\begin{itemize}
\tightlist
\item
  \href{https://help.nytimes.com/hc/en-us/articles/115014792127-Copyright-notice}{©~2020~The
  New York Times Company}
\end{itemize}

\begin{itemize}
\tightlist
\item
  \href{https://www.nytco.com/}{NYTCo}
\item
  \href{https://help.nytimes.com/hc/en-us/articles/115015385887-Contact-Us}{Contact
  Us}
\item
  \href{https://www.nytco.com/careers/}{Work with us}
\item
  \href{https://nytmediakit.com/}{Advertise}
\item
  \href{http://www.tbrandstudio.com/}{T Brand Studio}
\item
  \href{https://www.nytimes.com/privacy/cookie-policy\#how-do-i-manage-trackers}{Your
  Ad Choices}
\item
  \href{https://www.nytimes.com/privacy}{Privacy}
\item
  \href{https://help.nytimes.com/hc/en-us/articles/115014893428-Terms-of-service}{Terms
  of Service}
\item
  \href{https://help.nytimes.com/hc/en-us/articles/115014893968-Terms-of-sale}{Terms
  of Sale}
\item
  \href{https://spiderbites.nytimes.com}{Site Map}
\item
  \href{https://help.nytimes.com/hc/en-us}{Help}
\item
  \href{https://www.nytimes.com/subscription?campaignId=37WXW}{Subscriptions}
\end{itemize}
