Sections

SEARCH

\protect\hyperlink{site-content}{Skip to
content}\protect\hyperlink{site-index}{Skip to site index}

\href{https://www.nytimes.com/section/world/europe}{Europe}

\href{https://myaccount.nytimes.com/auth/login?response_type=cookie\&client_id=vi}{}

\href{https://www.nytimes.com/section/todayspaper}{Today's Paper}

\href{/section/world/europe}{Europe}\textbar{}`Brexit' Ruling Reveals
Cracks in Britain's Centuries-Old Institutions

\url{https://nyti.ms/2knupXI}

\begin{itemize}
\item
\item
\item
\item
\item
\end{itemize}

Advertisement

\protect\hyperlink{after-top}{Continue reading the main story}

Supported by

\protect\hyperlink{after-sponsor}{Continue reading the main story}

\hypertarget{brexit-ruling-reveals-cracks-in-britains-centuries-old-institutions}{%
\section{`Brexit' Ruling Reveals Cracks in Britain's Centuries-Old
Institutions}\label{brexit-ruling-reveals-cracks-in-britains-centuries-old-institutions}}

\includegraphics{https://static01.nyt.com/images/2017/01/25/world/25Brexit1/25Brexit1-videoSixteenByNine3000.jpg}

By \href{http://www.nytimes.com/by/katrin-bennhold}{Katrin Bennhold}

\begin{itemize}
\item
  Jan. 24, 2017
\item
  \begin{itemize}
  \item
  \item
  \item
  \item
  \item
  \end{itemize}
\end{itemize}

LONDON --- It remains unclear whether Prime Minister Theresa May's plans
or timetable for taking Britain out of the European Union will be
altered by the Supreme Court's ruling on Tuesday that she must secure
Parliament's approval before beginning the process. Most analysts, even
those who opposed ``Brexit,'' as the departure from the bloc is known,
doubt that it will.

And Mrs. May had already said in her
\href{https://www.nytimes.com/2017/01/17/world/europe/brexit-theresa-may-uk-eu.html?_r=0}{speech
on Brexit} last week that Parliament would have a vote on whether to
accept the final deal negotiated with the European Union.

But the ruling Tuesday --- which included a decision to deny the
Scottish, Welsh and Northern Irish legislatures a veto in the matter ---
has brought to the fore some ancient tensions in Britain's democracy,
which has somehow made it through the centuries with an unequal union of
four nations, an unelected upper house of Parliament and without a
written constitution. These tensions may ultimately have far greater
impact than a ruling that was widely anticipated.

``There are some fairly serious questions about how the U.K.'s
constitutional settlement operates, not least the lack of democracy at
the heart of the houses of Parliament,'' Stephen Gethins, the Scottish
National Party's spokesman on Europe in the British Parliament, said in
an interview. ``All this raises quite substantial questions about the
future of the union.''

The strongest words of protest arose from the nations of the United
Kingdom whose voters opposed Brexit. In Scotland, the first minister,
Nicola Sturgeon, said Tuesday that the case for a second referendum on
independence was growing ever stronger.

``It's becoming clearer by the day that Scotland's voice is simply not
being heard or listened to within the U.K.,'' Ms. Sturgeon said.

In Northern Ireland, where the fragile 1998 Good Friday Agreement that
ended decades of sectarian conflict is predicated on membership in the
European Union and an open border with Ireland, the decision not to give
its Parliament a vote risks aggravating the sectarian divide, officials
said.

``Brexit will undermine the institutional, constitutional and legal
integrity of the Good Friday Agreement,'' said Gerry Adams, the leader
of Sinn Fein, which represents the Catholic nationalist community in
Northern Ireland. ``Our stability and economic progress,'' he said,
``are regarded as collateral damage.''

\href{https://www.nytimes.com/interactive/2016/business/international/brexit-uk-what-happens-business.html}{}

\includegraphics{https://static01.nyt.com/images/2017/03/29/business/27BREXIT/27BREXIT-articleLarge.jpg}

\hypertarget{how-brexit-could-change-business-in-britain}{%
\subsection{How `Brexit' Could Change Business in
Britain}\label{how-brexit-could-change-business-in-britain}}

Britain has started the clock on leaving the European Union, and will be
out of the bloc by March 2019. Here is how ``Brexit'' has affected
business so far.

In its
\href{https://www.supremecourt.uk/cases/docs/uksc-2016-0196-judgment.pdf}{ruling},
which
\href{https://www.nytimes.com/2016/11/04/world/europe/uk-brexit-vote-parliament.html}{upholds
an earlier decision} by the High Court in London, the Supreme Court
noted that Parliament had approved the 1972 legislation that enabled the
country to join the European Union and incorporated European law into
British law. Leaving the bloc would take away from British citizens a
number of rights that had been granted by the bloc.

As a result, ``the government cannot trigger Article 50 without an act
of Parliament authorizing that course,'' David Neuberger, the Supreme
Court's president, said in announcing the decision, which was approved,
8 to 3.

Although a majority of lawmakers had campaigned to stay in the European
Union before the referendum last year, most political observers said it
was unlikely that legislators would reject the will of the voters.

The prime minister's office is expected to submit a tightly worded bill
to Parliament as early as this week, and if all goes well Mrs. May will
begin
\href{https://www.nytimes.com/2017/01/17/world/europe/brexit-theresa-may-uk-eu.html}{a
two-year, irreversible process} of exit negotiations with the European
Union by the end of March.

``There's no going back,'' David Davis, the British official assigned to
oversee the withdrawal, told Parliament later Tuesday. ``The point of no
return was June 23,'' he said, referring to the date of the referendum.

That is not going to stop some from trying. The Scottish National Party
is likely to vote against the measure, and has vowed to submit 50
``serious and substantive'' amendments in an effort to slow the process
and, if possible, soften or reverse the outcome.

But with the leader of the opposition Labour Party, Jeremy Corbyn,
pledging not to stand in the way, and the Conservatives holding a
majority in Parliament, Mrs. May is widely expected to prevail.

Some expect a little more pushback in the House of Lords, not enough to
stop the bill's passage but possibly enough to delay it, surely enraging
the most vocal cheerleaders for Brexit --- the tabloid press and English
nationalists, from the right wing of the Conservative Party to the U.K.
Independence Party.

That the Lords could intervene in the process stems from another ancient
quirk of the British political system.

Unelected, overcrowded and with an age profile similar to that of many
retirement homes,
\href{http://topics.nytimes.com/top/news/international/countriesandterritories/unitedkingdom/index.html?inline=nyt-geo}{Britain}'s
upper chamber of Parliament has survived a century of debate over its
purpose, while becoming the largest legislative assembly in the world
outside of China. Unlike the elected House of Commons, members of the
House of Lords are mostly appointed, and many were named by the Labour
governments in power from 1997 to 2010. At the least, they could throw
sand in the legislative gears on Brexit.

\includegraphics{https://static01.nyt.com/images/2017/01/25/world/25Brexit2/25Brexit2-articleInline.jpg?quality=75\&auto=webp\&disable=upscale}

The court case has also underscored the generally polarizing nature of
the June referendum, in which 52 percent voted to leave the European
Union.

One of the plaintiffs, Gina Miller, an investment fund manager, has said
she was threatened with murder and rape by Brexit supporters, who have
accused her of trying to sabotage the withdrawal. A lawyer by training,
Ms. Miller has said she was merely standing up for the rights of
Parliament.

``This case was about the legal process, not about politics,'' Ms.
Miller said in a news conference outside the Supreme Court, where she
thanked her law firm,
\href{https://www.mishcon.com/news/firm_news/article_50_legal_challenge_supreme_court_upholds_high_court_decision_01_2017}{Mishcon
de Reya}, for fighting her case.

Ms. Miller said she was ``shocked by the levels of personal abuse that I
have received from many quarters over the last several months for simply
bringing and asking a legitimate question.''

At times it felt that the judges themselves were on trial. Members of
the High Court who ruled against the government in November, setting the
stage for the Supreme Court decision, were described by one tabloid
newspaper as ``enemies of the people.''

But lawmakers from across the political spectrum have made clear that
they want to be involved from the start.

``I and many others did not exercise our vote in the referendum so as to
restore the sovereignty of this Parliament only to see what we regarded
as the tyranny of the European Union replaced by that of a government,''
Stephen Phillips, a member of Mrs. May's Conservative Party, said when
the case was first brought.

There is still an outside chance that Parliament will reassert control
of Brexit talks, said
\href{https://www.cer.org.uk/personnel/simon-tilford}{Simon Tilford}, a
Britain and Europe specialist at the Center for European Reform in
London.

``What this does is it opens the way for much greater parliamentary
scrutiny of the whole process,'' he said. ``But we're only going to get
this if the Labour Party is willing to push back. That is not likely,
but not impossible.''

``It opens up the way for a kind of democratization of what is
happening, for Parliament to hold the government to account,'' he added.
``So far we had a vote to leave the E.U., but it certainly wasn't a vote
to take Britain out of the single market, out of the customs union and
to make people poorer.''

Advertisement

\protect\hyperlink{after-bottom}{Continue reading the main story}

\hypertarget{site-index}{%
\subsection{Site Index}\label{site-index}}

\hypertarget{site-information-navigation}{%
\subsection{Site Information
Navigation}\label{site-information-navigation}}

\begin{itemize}
\tightlist
\item
  \href{https://help.nytimes.com/hc/en-us/articles/115014792127-Copyright-notice}{©~2020~The
  New York Times Company}
\end{itemize}

\begin{itemize}
\tightlist
\item
  \href{https://www.nytco.com/}{NYTCo}
\item
  \href{https://help.nytimes.com/hc/en-us/articles/115015385887-Contact-Us}{Contact
  Us}
\item
  \href{https://www.nytco.com/careers/}{Work with us}
\item
  \href{https://nytmediakit.com/}{Advertise}
\item
  \href{http://www.tbrandstudio.com/}{T Brand Studio}
\item
  \href{https://www.nytimes.com/privacy/cookie-policy\#how-do-i-manage-trackers}{Your
  Ad Choices}
\item
  \href{https://www.nytimes.com/privacy}{Privacy}
\item
  \href{https://help.nytimes.com/hc/en-us/articles/115014893428-Terms-of-service}{Terms
  of Service}
\item
  \href{https://help.nytimes.com/hc/en-us/articles/115014893968-Terms-of-sale}{Terms
  of Sale}
\item
  \href{https://spiderbites.nytimes.com}{Site Map}
\item
  \href{https://help.nytimes.com/hc/en-us}{Help}
\item
  \href{https://www.nytimes.com/subscription?campaignId=37WXW}{Subscriptions}
\end{itemize}
