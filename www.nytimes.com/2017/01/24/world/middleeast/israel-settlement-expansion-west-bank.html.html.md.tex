Sections

SEARCH

\protect\hyperlink{site-content}{Skip to
content}\protect\hyperlink{site-index}{Skip to site index}

\href{https://www.nytimes.com/section/world/middleeast}{Middle East}

\href{https://myaccount.nytimes.com/auth/login?response_type=cookie\&client_id=vi}{}

\href{https://www.nytimes.com/section/todayspaper}{Today's Paper}

\href{/section/world/middleeast}{Middle East}\textbar{}Emboldened by
Trump, Israel Approves a Wave of West Bank Settlement Expansion

\url{https://nyti.ms/2knKAse}

\begin{itemize}
\item
\item
\item
\item
\item
\end{itemize}

Advertisement

\protect\hyperlink{after-top}{Continue reading the main story}

Supported by

\protect\hyperlink{after-sponsor}{Continue reading the main story}

\hypertarget{emboldened-by-trump-israel-approves-a-wave-of-west-bank-settlement-expansion}{%
\section{Emboldened by Trump, Israel Approves a Wave of West Bank
Settlement
Expansion}\label{emboldened-by-trump-israel-approves-a-wave-of-west-bank-settlement-expansion}}

\includegraphics{https://static01.nyt.com/images/2017/01/25/world/middleeast/25Israel/25Israel-articleInline-v2.jpg?quality=75\&auto=webp\&disable=upscale}

By \href{https://www.nytimes.com/by/isabel-kershner}{Isabel Kershner}

\begin{itemize}
\item
  Jan. 24, 2017
\item
  \begin{itemize}
  \item
  \item
  \item
  \item
  \item
  \end{itemize}
\end{itemize}

JERUSALEM --- In a pointed act of defiance against international
pressure, Israel on Tuesday approved a huge new wave of settlement
construction in the occupied West Bank.

The announcement made clear that just a few days into the Trump
presidency, the Israeli government feels emboldened to shake off the
constraints imposed by the Obama administration and more willing to
disregard international condemnation.

Leaders from 70 countries met in Paris more than a week ago and issued a
\href{https://www.nytimes.com/2017/01/15/world/middleeast/missing-at-israel-palestinian-peace-conference-israelis-or-palestinians.html}{warning
that the two-state peace solution was imperiled} by Israel's expanding
of settlements in Palestinian-claimed territory in the West Bank and
East Jerusalem, as well as violence against Israelis. But even though
Prime Minister Benjamin Netanyahu has endorsed the principle of
side-by-side states, in the past few days Israel's campaign of
settlement building has only accelerated.

The first step came on Sunday, when the Jerusalem City Council approved
566 new housing units in East Jerusalem that had been delayed over
President Barack Obama's objections.

Then on Tuesday, the Israeli government announced that 2,500 new housing
units would be built in the West Bank. Officials said most would be
built in ``settlement blocs,'' referring to areas of the West Bank that
Israel has long intended to keep under any future agreement with the
Palestinians, possibly in return for land swaps along the boundary that
separated Israel from the West Bank before the 1967 war. But in years of
failed negotiations, the Israelis and Palestinians have never agreed on
the size or location of such blocs.

The Israeli Ministry of Defense said 900 of the newly announced homes
were being planned for Ariel, an urban settlement of about 20,000
residents that Israel considers a ``bloc,'' but is strategically --- and
problematically --- located in the heart of the West Bank. It also said
it would bring to the cabinet a plan to build a large industrial zone to
create work for Palestinians in the southern West Bank.

``We are going back to normal life in Judea and Samaria,'' Avigdor
Lieberman, Israel's hard-line defense minister, said in a statement
announcing the new settlement building, referring to the West Bank by
its biblical names.

Asked about the Israeli move, the White House spokesman, Sean Spicer,
said that Mr. Trump was still getting his team together and that there
would be discussions with Mr. Netanyahu. ``Israel continues to be a huge
ally of the United States,'' Mr. Spicer said. ``He wants to grow closer
with Israel to make sure that it gets the full respect that it deserves
in the Middle East, and that's what he's going to do.

Palestinian officials immediately denounced the new plans.

``Once again, the Israeli government has proved that it is more
committed to land theft and colonialism than to the two-state solution
and the requirements for peace and stability,'' Hanan Ashrawi, a member
of the Palestine Liberation Organization's executive committee, said in
a statement.

``It is evident that Israel is exploiting the inauguration of the new
American administration to escalate its violations and the prevention of
any existence of a Palestinian state,'' she added, calling on the United
States and other international players to take concrete measures against
Israeli settlement activities.

Israel's campaign of settlement construction has brought widespread
criticism. A month ago, the United Nations Security Council
\href{https://www.nytimes.com/2016/12/23/world/middleeast/israel-settlements-un-vote.html?hp\&action=click\&pgtype=Homepage\&clickSource=story-heading\&module=first-column-region\&region=top-news\&WT.nav=top-news}{passed
a resolution} condemning Israel's settlements in the West Bank and East
Jerusalem as having no legal validity and constituting a ``flagrant
violation under international law'' after the Obama administration
decided not to veto the measure.

Days later, the departing secretary of state, John Kerry, rebuked
Israel's settlement activities in an impassioned speech, saying, ``The
status quo is leading toward one state and perpetual occupation.''

But with Israel's occupation of the West Bank in its 50th year, the
Israeli government, dominated by right-wing and religious parties, is
clearly expecting a friendlier approach from the White House after years
of tension with the Obama administration.

David M. Friedman, the bankruptcy lawyer President Trump has nominated
as his ambassador to Israel, has
\href{https://www.nytimes.com/2016/12/16/world/middleeast/david-friedman-us-ambassador-israel.html}{led
a fund-raising arm} of the settlement movement and has dismissed the
idea of a Palestinian state alongside Israel. He has declared that he
intends to work in Jerusalem, not Tel Aviv, where the American Embassy
has been for decades, under the State Department's insistence that the
holy city's status be determined as part of a broader deal between
Israel and the Palestinians.

It was not immediately clear whether the Israeli announcement had been
coordinated in advance with Mr. Trump's team. But beyond Mr. Netanyahu's
apparent attempt to chart a new course with Mr. Trump, he is also under
intense pressure from the right flank of his governing coalition to
demonstrate where his domestic loyalties lie.

Naftali Bennett, the education minister and leader of the staunchly
pro-settlement Jewish Home party, has been goading the prime minister to
seize the moment and take the extreme step of beginning a process of
annexing the West Bank settlements to Israel.

``Netanyahu is facing a historic decision: sovereignty or Palestine,''
Mr. Bennett said on Monday. ``We urge Netanyahu, don't miss an
opportunity that comes along once every 50 years.''

Mr. Netanyahu appeared to postpone any discussion of annexation: ``This
is no time for off-the-cuff decisions or political dictations, and this
is no time for surprises.'' This, he added, ``is the time for
considered, responsible diplomacy among friends.''

The prime minister's office said that in a phone conversation with Mr.
Trump on Sunday, Mr. Netanyahu discussed the peace process and hoped to
forge a ``common vision'' with Mr. Trump ``to advance peace and security
in the region, with no daylight between the United States and Israel.''
No more details were given.

The peace process has been at an impasse since the last round of
American-brokered
\href{https://www.nytimes.com/2014/04/29/world/middleeast/arc-of-a-failed-deal-how-nine-months-of-mideast-talks-ended-in-dissarray.html}{talks
collapsed in the spring of 2014}. During the nine months of talks, Mr.
Netanyahu attempted to appease Israel's right wing by advancing plans
for about 13,000 new housing units in the West Bank and East Jerusalem,
infuriating the Palestinian side. The weakened Palestinian president,
Mahmoud Abbas, who appeared reluctant to take risks of his own, never
responded to the ideas that Mr. Kerry's team had formulated for a
framework to guide further negotiations.

Now, with the change of American administrations, some Israeli analysts
have recommended that Mr. Netanyahu take the opportunity to try to
reinstate understandings that Israel had with President George W. Bush,
who wrote in a
\href{https://georgewbush-whitehouse.archives.gov/news/releases/2004/04/20040414-3.html}{2004
letter} that ``already existing major Israeli population centers''
should be taken into consideration in redrawing the borders between
Israel and the West Bank --- a reference to settlement blocs.

But that came in the context of Israel's plans to unilaterally withdraw
from Gaza and from a section of the northern West Bank. And the case of
Ariel serves to illustrate the contentiousness of unilaterally defining
the blocs.

Israelis have long labeled Ariel part of their national ``consensus,''
meaning that it would be included in Israel's borders under any peace
deal, and it often appears as one of the regular dots on Israeli weather
maps. But Palestinian negotiators have always rejected that idea,
arguing that Israeli control over Ariel would preclude the territorial
contiguity of a Palestinian state. They also note that Ariel sits on a
major aquifer.

According to Tuesday's announcement, 20 of the new units are to be built
in Beit El, a settlement deep in the West Bank that has particularly
benefited from Mr. Friedman's fund-raising activities. The government
promised in 2012 to build 300 units in Beit El, a settlement of about
7,000 residents, to compensate for the court-ordered evacuation of part
of a neighborhood there that was illegally built on private Palestinian
land. So far, the promise has remained unfulfilled.

According to Israel's Ministry of Defense, bids will now be solicited
for the construction of about 900 of the 2,500 new units around the West
Bank. But the rest, including most of those planned for Ariel, still
have to go through additional planning phases, a bureaucratic process
that can take months, if not years, and requires additional government
approval at each stage.

Oded Revivi, the chief foreign envoy of the Yesha Council, an umbrella
organization representing the more than 400,000 settlers in the West
Bank, said in a statement, ``We hope that this is just the beginning of
a wave of new building across our ancestral homeland after eight very
difficult years.''

But some in the settler camp played down the construction plans and
expressed suspicions about Mr. Netanyahu's intentions.

``We are not stupid,'' Bezalel Smotrich, a legislator from the Jewish
Home party, \href{https://www.facebook.com/Bezazelsmotrich/}{wrote} in a
post on his Facebook page. Objecting to the government announcement
mostly describing the advancement of existing plans in settlement blocs,
Mr. Smotrich accused Mr. Netanyahu of ``throwing a candy'' to the
settlers and playing ``public relations tricks.''

Advertisement

\protect\hyperlink{after-bottom}{Continue reading the main story}

\hypertarget{site-index}{%
\subsection{Site Index}\label{site-index}}

\hypertarget{site-information-navigation}{%
\subsection{Site Information
Navigation}\label{site-information-navigation}}

\begin{itemize}
\tightlist
\item
  \href{https://help.nytimes.com/hc/en-us/articles/115014792127-Copyright-notice}{©~2020~The
  New York Times Company}
\end{itemize}

\begin{itemize}
\tightlist
\item
  \href{https://www.nytco.com/}{NYTCo}
\item
  \href{https://help.nytimes.com/hc/en-us/articles/115015385887-Contact-Us}{Contact
  Us}
\item
  \href{https://www.nytco.com/careers/}{Work with us}
\item
  \href{https://nytmediakit.com/}{Advertise}
\item
  \href{http://www.tbrandstudio.com/}{T Brand Studio}
\item
  \href{https://www.nytimes.com/privacy/cookie-policy\#how-do-i-manage-trackers}{Your
  Ad Choices}
\item
  \href{https://www.nytimes.com/privacy}{Privacy}
\item
  \href{https://help.nytimes.com/hc/en-us/articles/115014893428-Terms-of-service}{Terms
  of Service}
\item
  \href{https://help.nytimes.com/hc/en-us/articles/115014893968-Terms-of-sale}{Terms
  of Sale}
\item
  \href{https://spiderbites.nytimes.com}{Site Map}
\item
  \href{https://help.nytimes.com/hc/en-us}{Help}
\item
  \href{https://www.nytimes.com/subscription?campaignId=37WXW}{Subscriptions}
\end{itemize}
