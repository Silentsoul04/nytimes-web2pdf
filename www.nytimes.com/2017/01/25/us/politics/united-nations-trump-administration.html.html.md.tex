Sections

SEARCH

\protect\hyperlink{site-content}{Skip to
content}\protect\hyperlink{site-index}{Skip to site index}

\href{https://www.nytimes.com/section/politics}{Politics}

\href{https://myaccount.nytimes.com/auth/login?response_type=cookie\&client_id=vi}{}

\href{https://www.nytimes.com/section/todayspaper}{Today's Paper}

\href{/section/politics}{Politics}\textbar{}Trump Prepares Orders Aiming
at Global Funding and Treaties

\url{https://nyti.ms/2ktQ7cF}

\begin{itemize}
\item
\item
\item
\item
\item
\end{itemize}

Advertisement

\protect\hyperlink{after-top}{Continue reading the main story}

Supported by

\protect\hyperlink{after-sponsor}{Continue reading the main story}

\href{/column/the-interpreter}{The Interpreter}

\hypertarget{trump-prepares-orders-aiming-at-global-funding-and-treaties}{%
\section{Trump Prepares Orders Aiming at Global Funding and
Treaties}\label{trump-prepares-orders-aiming-at-global-funding-and-treaties}}

\includegraphics{https://static01.nyt.com/images/2017/01/26/us/26orders/26orders-articleInline.jpg?quality=75\&auto=webp\&disable=upscale}

By \href{https://www.nytimes.com/by/max-fisher}{Max Fisher}

\begin{itemize}
\item
  Jan. 25, 2017
\item
  \begin{itemize}
  \item
  \item
  \item
  \item
  \item
  \end{itemize}
\end{itemize}

WASHINGTON --- The Trump administration is preparing executive orders
that would clear the way to drastically reduce the United States' role
in the United Nations and other international organizations, as well as
begin a process to review and potentially abrogate certain forms of
multilateral treaties.

The first of the two draft orders, titled ``Auditing and Reducing U.S.
Funding of International Organizations'' and obtained by The New York
Times, calls for terminating funding for any United Nations agency or
other international body that meets any one of several criteria.

Those criteria include organizations that give full membership to the
Palestinian Authority or Palestine Liberation Organization, or support
programs that fund abortion or any activity that circumvents sanctions
against Iran or North Korea. The draft order also calls for terminating
funding for any organization that ``is controlled or substantially
influenced by any state that sponsors terrorism'' or is blamed for the
persecution of marginalized groups or any other systematic violation of
human rights.

The order calls for then enacting ``at least a 40 percent overall
decrease'' in remaining United States funding toward international
organizations.

The order establishes a committee to recommend where those funding cuts
should be made. It asks the committee to look specifically at United
States funding for peacekeeping operations; the International Criminal
Court; development aid to countries that ``oppose important United
States policies''; and the United Nations Population Fund, which
oversees maternal and reproductive health programs.

If President Trump signs the order and its provisions are carried out,
the cuts could severely curtail the work of United Nations agencies,
which rely on billions of dollars in annual United States contributions
for missions that include caring for refugees.

The second executive order, ``Moratorium on New Multilateral Treaties,''
calls for a review of all current and pending treaties with more than
one other nation. It asks for recommendations on which negotiations or
treaties the United States should leave.

The order says this review applies only to multilateral treaties that
are not ``directly related to national security, extradition or
international trade,'' but it is unclear what falls outside these
restrictions.

For example, the Paris climate agreement or other environmental treaties
deal with trade issues but could potentially fall under this order.

An explanatory statement that accompanies the draft order mentions two
United Nations treaties for review: the Convention on the Elimination of
All Forms of Discrimination Against Women and the Convention on the
Rights of the Child.

Taken together, the orders suggest that Mr. Trump intends to pursue his
\href{https://www.nytimes.com/2017/01/20/us/politics/trump-resurrects-dark-definition-of-america-first-vision.html}{campaign
promises} of withdrawing the United States from international
organizations. He has expressed heavy skepticism of multilateral
agreements such as the Paris climate agreement and of the United
Nations.

The draft orders, which are only a few pages each, leave several
unanswered questions. For example, it is unclear whether they call for
cutting 40 percent of United States contributions to each international
agency separately, or to the overall federal funding budget.

The orders call for reviewing any funding that could go toward the
International Criminal Court, though the United States currently
provides no funding to that body. They also call for terminating funding
to United Nations bodies that include full Palestinian membership,
though this is already United States law. Under former President Barack
Obama, the United States cut funding to the United Nations Educational,
Scientific and Cultural Organization when
\href{http://www.nytimes.com/2011/11/01/world/middleeast/unesco-approves-full-membership-for-palestinians.html}{it
accepted Palestinians as full members}.

The United States provides about a quarter of all funding to United
Nations peacekeeping operations, of which there are more than a dozen,
in Europe, Africa, Latin America, the Middle East and Asia. At least one
of these, the operation in southern Lebanon, directly serves Israeli
interests by protecting the country's northern border, though the draft
order characterizes the funding cuts as serving Israeli interests.

Advertisement

\protect\hyperlink{after-bottom}{Continue reading the main story}

\hypertarget{site-index}{%
\subsection{Site Index}\label{site-index}}

\hypertarget{site-information-navigation}{%
\subsection{Site Information
Navigation}\label{site-information-navigation}}

\begin{itemize}
\tightlist
\item
  \href{https://help.nytimes.com/hc/en-us/articles/115014792127-Copyright-notice}{©~2020~The
  New York Times Company}
\end{itemize}

\begin{itemize}
\tightlist
\item
  \href{https://www.nytco.com/}{NYTCo}
\item
  \href{https://help.nytimes.com/hc/en-us/articles/115015385887-Contact-Us}{Contact
  Us}
\item
  \href{https://www.nytco.com/careers/}{Work with us}
\item
  \href{https://nytmediakit.com/}{Advertise}
\item
  \href{http://www.tbrandstudio.com/}{T Brand Studio}
\item
  \href{https://www.nytimes.com/privacy/cookie-policy\#how-do-i-manage-trackers}{Your
  Ad Choices}
\item
  \href{https://www.nytimes.com/privacy}{Privacy}
\item
  \href{https://help.nytimes.com/hc/en-us/articles/115014893428-Terms-of-service}{Terms
  of Service}
\item
  \href{https://help.nytimes.com/hc/en-us/articles/115014893968-Terms-of-sale}{Terms
  of Sale}
\item
  \href{https://spiderbites.nytimes.com}{Site Map}
\item
  \href{https://help.nytimes.com/hc/en-us}{Help}
\item
  \href{https://www.nytimes.com/subscription?campaignId=37WXW}{Subscriptions}
\end{itemize}
