Sections

SEARCH

\protect\hyperlink{site-content}{Skip to
content}\protect\hyperlink{site-index}{Skip to site index}

\href{https://www.nytimes.com/section/world/americas}{Americas}

\href{https://myaccount.nytimes.com/auth/login?response_type=cookie\&client_id=vi}{}

\href{https://www.nytimes.com/section/todayspaper}{Today's Paper}

\href{/section/world/americas}{Americas}\textbar{}As Trump Orders Wall,
Mexico's President Considers Canceling U.S. Trip

\url{https://nyti.ms/2ktMun5}

\begin{itemize}
\item
\item
\item
\item
\item
\item
\end{itemize}

Advertisement

\protect\hyperlink{after-top}{Continue reading the main story}

Supported by

\protect\hyperlink{after-sponsor}{Continue reading the main story}

\hypertarget{as-trump-orders-wall-mexicos-president-considers-canceling-us-trip}{%
\section{As Trump Orders Wall, Mexico's President Considers Canceling
U.S.
Trip}\label{as-trump-orders-wall-mexicos-president-considers-canceling-us-trip}}

\includegraphics{https://static01.nyt.com/images/2017/01/26/world/26mexico/26mexico-articleInline.jpg?quality=75\&auto=webp\&disable=upscale}

By \href{http://www.nytimes.com/by/azam-ahmed}{Azam Ahmed}

\begin{itemize}
\item
  Jan. 25, 2017
\item
  \begin{itemize}
  \item
  \item
  \item
  \item
  \item
  \item
  \end{itemize}
\end{itemize}

MEXICO CITY --- When Donald J. Trump called some Mexican immigrants
rapists, threatened to deport millions of them and promised to build a
wall to keep others out, Mexican officials counseled caution, saying it
was merely bluster from an unlikely candidate who, if elected, would
never follow through.

Now, after just five days in office, President Trump is looking a lot
like Candidate Trump --- and the Mexicans are furious.

With just a few strokes of the pen on Wednesday, the new American
president signed an executive order to beef up the nation's deportation
force and start construction on a new wall between the nations. Adding
to the perceived insult was the timing of the order: It came on the
first day of talks between top Mexican officials and their counterparts
in Washington, and just days before a meeting between the two countries'
presidents.

The action was enough to prompt President Enrique Peña Nieto of Mexico
to consider scrapping his plans to visit the White House on Tuesday,
according to Mexican officials. In a video message delivered over
Twitter on Wednesday night, Mr. Peña Nieto did not address whether he
would cancel the meeting, saying only that future steps would be taken
in consultation with the country's lawmakers. Instead, he reiterated his
commitment to protect the interests of Mexico and the Mexican people,
and chided the move in Washington to continue with the wall.

``I regret and condemn the United States' decision to continue with the
construction of a wall that, for years now, far from uniting us, divides
us,'' he said.

It mattered little to Mexicans whether Mr. Trump's order would receive
congressional approval or the funding required to fulfill it.

The perceived insults endured during the campaign had finally turned
into action. Decades of friendly relations between the nations --- on
matters involving trade, security and migration --- seemed to be
unraveling.

\includegraphics{https://static01.nyt.com/images/2017/08/24/multimedia/24dc-wall/merlin-to-scoop-126242489-947823-videoSixteenByNine3000.jpg}

Calls began to come in from across the political spectrum for Mr. Peña
Nieto to cancel his visit, and to respond with greater fortitude to the
perceived menace from President Trump. On Twitter, Mr. Trump's action
was referred to by politicians and historians as a ``an offense to
Mexico,'' a ``slap in the face'' and a ``monument to lies.''

Historians said that not since President Calvin Coolidge threatened to
invade a ``Soviet Mexico'' had the United States so deeply antagonized
the Mexican populace.

``It is an unprecedented moment for the bilateral relationship,'' said
Genaro Lozano, a professor at the Iberoamerican University in Mexico
City. ``In the 19th century, we fought a war with the U.S.; now we find
ourselves in a low-intensity war, a commercial one over Nafta and an
immigration war due to the measures he just announced.''

The fear of many in Mexico is that Mr. Trump's remarks will force Mr.
Peña Nieto's hand. While the Mexican president has opted for dialogue
over confrontation, the building pressure, and perceived slights, could
complicate his agenda.

That could ultimately lead to disastrous consequences. With Nafta gone,
both countries would be likely to suffer significant damage to their
economies, and Mexico could see a sharp decline in foreign investment.
The United States, meanwhile, might find its southern neighbor far less
willing to cooperate on issues of security.

Pressure from the political opposition could also color the Mexican
government's response. During a rally, the leftist politician Andrés
Manuel López Obrador asked for Mr. Peña Nieto to appeal to the United
Nations and accused Mr. Trump of ``violating human rights'' and of
racism.

Mr. López Obrador, who ran a close race against Mr. Peña Nieto for
president in 2012, said the actions by Mr. Trump on Wednesday were ``a
great opportunity for Peña to begin his defense of Mexican immigrants,
with steadiness and dignity.''

And while Mexico does not want a war of words --- or a trade war ---
with Washington, the dynamic has already started to change. In recent
days, top government officials said they would be
\href{https://www.nytimes.com/2017/01/24/world/americas/trump-mexico-nafta.html}{willing
to walk away from Nafta} if the negotiations did not suit Mexico's
interests. That position was unthinkable even a few months ago.

For a new American president, Mr. Trump is both well known in Mexico and
widely disliked. His remarks have unified an otherwise fractious
country, where corruption, a shaky rule of law and economic struggles
have alienated many and kept them from becoming politically engaged. As
Mr. Trump's promises have begun to materialize into action, the public
discourse is changing.

\includegraphics{https://static01.nyt.com/images/2017/01/26/world/26mexico2/26mexico2-articleInline.jpg?quality=75\&auto=webp\&disable=upscale}

Some fear the worst.

``Diplomatic relations are ever changing, never static, but there is a
real concern that this wall, as a powerful symbol of national resolve,
will unleash a blunt, raw nationalism in Mexico,'' said Allert
Brown-Gort, a professor of international relations at the Autonomous
Technological Institute of Mexico. ``It would be similar to the one the
American government is showing, and would undermine all efforts for the
relations to go back to their normal state.''

The Trump broadside on Wednesday, coming as the Mexican foreign
minister, Luis Videgaray, was to begin preliminary talks in Washington,
was a new low in bilateral relations.

Mr. Videgaray orchestrated Mr. Trump's visit to Mexico during the
campaign last year, when he was finance minister. The move drew such
widespread condemnation that
\href{https://www.nytimes.com/2016/09/08/world/americas/mexico-finance-minister-luis-videgaray-resigns.html}{Mr.
Videgaray resigned.}

After Mr. Trump's victory, Mr. Videgaray's fortunes revived and he
became foreign minister. But any hope that his invitation to Washington
could be leveraged in Mexico's favor now seem dashed.

Many Mexicans said conciliatory gestures were no longer an option.

``It's like we are Charlie Brown and they are Lucy with the football,''
said Jorge Castañeda, a former foreign minister. ``Peña is a weak
president in a weak country at a weak moment, but he has to find a way
to get some official backbone.''

Still, despite the outrage, there was a recognition that the action was
more political than substantive, like the wall itself. A meaningful
stretch of the United States border is already lined with a wall, and
illegal immigration persists.

``Symbolic gestures are not going to change the fact that we're
neighbors and that our countries will affect each other in perpetuity,''
said Carlos Pascual, a former United States ambassador to Mexico. ``We
have no choice but to keep working for better solutions.''

Others took talk of a wall with a grain of salt, figuring Mexican
creativity, or ingenuity, would render it useless anyway.

``This is just politics, it won't upend life in Mexico,'' said Ariel
Najum, 39, who runs a family business. ``You know how Mexicans are: If
they go high, we go underneath, with tunnels.''

Advertisement

\protect\hyperlink{after-bottom}{Continue reading the main story}

\hypertarget{site-index}{%
\subsection{Site Index}\label{site-index}}

\hypertarget{site-information-navigation}{%
\subsection{Site Information
Navigation}\label{site-information-navigation}}

\begin{itemize}
\tightlist
\item
  \href{https://help.nytimes.com/hc/en-us/articles/115014792127-Copyright-notice}{©~2020~The
  New York Times Company}
\end{itemize}

\begin{itemize}
\tightlist
\item
  \href{https://www.nytco.com/}{NYTCo}
\item
  \href{https://help.nytimes.com/hc/en-us/articles/115015385887-Contact-Us}{Contact
  Us}
\item
  \href{https://www.nytco.com/careers/}{Work with us}
\item
  \href{https://nytmediakit.com/}{Advertise}
\item
  \href{http://www.tbrandstudio.com/}{T Brand Studio}
\item
  \href{https://www.nytimes.com/privacy/cookie-policy\#how-do-i-manage-trackers}{Your
  Ad Choices}
\item
  \href{https://www.nytimes.com/privacy}{Privacy}
\item
  \href{https://help.nytimes.com/hc/en-us/articles/115014893428-Terms-of-service}{Terms
  of Service}
\item
  \href{https://help.nytimes.com/hc/en-us/articles/115014893968-Terms-of-sale}{Terms
  of Sale}
\item
  \href{https://spiderbites.nytimes.com}{Site Map}
\item
  \href{https://help.nytimes.com/hc/en-us}{Help}
\item
  \href{https://www.nytimes.com/subscription?campaignId=37WXW}{Subscriptions}
\end{itemize}
