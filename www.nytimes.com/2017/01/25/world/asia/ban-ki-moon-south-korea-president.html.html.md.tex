Sections

SEARCH

\protect\hyperlink{site-content}{Skip to
content}\protect\hyperlink{site-index}{Skip to site index}

\href{https://www.nytimes.com/section/world/asia}{Asia Pacific}

\href{https://myaccount.nytimes.com/auth/login?response_type=cookie\&client_id=vi}{}

\href{https://www.nytimes.com/section/todayspaper}{Today's Paper}

\href{/section/world/asia}{Asia Pacific}\textbar{}Skepticism and Support
in South Korea as Ban Ki-moon Weighs Presidential Bid

\url{https://nyti.ms/2ksaEOG}

\begin{itemize}
\item
\item
\item
\item
\item
\end{itemize}

Advertisement

\protect\hyperlink{after-top}{Continue reading the main story}

Supported by

\protect\hyperlink{after-sponsor}{Continue reading the main story}

\hypertarget{skepticism-and-support-in-south-korea-as-ban-ki-moon-weighs-presidential-bid}{%
\section{Skepticism and Support in South Korea as Ban Ki-moon Weighs
Presidential
Bid}\label{skepticism-and-support-in-south-korea-as-ban-ki-moon-weighs-presidential-bid}}

\includegraphics{https://static01.nyt.com/images/2017/01/24/world/24BAN-1/24BAN-1-articleInline.jpg?quality=75\&auto=webp\&disable=upscale}

By \href{http://www.nytimes.com/by/choe-sang-hun}{Choe Sang-Hun}

\begin{itemize}
\item
  Jan. 25, 2017
\item
  \begin{itemize}
  \item
  \item
  \item
  \item
  \item
  \end{itemize}
\end{itemize}

HAENGCHI VILLAGE, South Korea --- Each day hundreds of visitors, many
with young children, make a pilgrimage to Haengchi Village, where Ban
Ki-moon was born 72 years ago. They wander through a replica of Mr.
Ban's old thatched-roof house. They learn about his personal journey to
the United Nations, where he was secretary general for 10 years.

Despite criticism of his tenure there, Mr. Ban is seen as a role model
by vast numbers of South Koreans. School textbooks, for example,
celebrate him as a ``man who made South Korea proud.'' And many South
Koreans want Mr. Ban to be their next president, succeeding Park
Geun-hye, whom the National Assembly voted to impeach last month on
corruption charges.

When Mr. Ban arrived home on Jan. 12, crowds of well-wishers turned out
at the airport, waving flags and shouting, ``Ban Ki-moon, please save
this country!''

Yet there is also deep skepticism about his potential presidential bid,
especially among the nation's progressives. They say he is trying to be
part of the establishment yet against it at the same time --- a ``Mr.
Half-Half,'' in the words of critics. (The word ``ban'' in Korean means
``half.'')

Mr. Ban calls himself ``a child of the United Nations,'' part of a
generation of South Koreans who remembered United Nations handouts in
the destitute years after the 1950-53 Korean War, as well as
American-led United Nations Forces who fought in the war. Many of his
contemporaries view the United States as South Korea's savior and
protector.

``I am ready to give my all to uniting the divided country and making
South Korea a first-rate nation,'' Mr. Ban said. ``As United Nations
secretary general, I have seen why some nations prosper and why some
fail.''

\includegraphics{https://static01.nyt.com/images/2017/01/24/world/24BAN-2/24BAN-2-articleInline.jpg?quality=75\&auto=webp\&disable=upscale}

His advocates say he is a seasoned, pro-American diplomat who can best
deal with both North Korea's advancing nuclear weapons program and
President Trump, who has raised questions about Washington's trade and
defense commitments to its allies.

One of the first things Mr. Ban did after his homecoming was support the
deployment of an
\href{https://www.nytimes.com/2016/07/08/world/asia/south-korea-and-us-agree-to-deploy-missile-defense-system.html}{American
missile defense system} that has angered North Korea and China.

Critics say his place in the establishment makes him unsuitable as a
figure who can restore trust in government. They believe the political
class has been disgraced by Ms. Park's corruption scandal and yet is
also desperate for a candidate it can support in an election that could
take place as early as this spring. The Constitutional Court is
\href{https://www.nytimes.com/2016/12/22/world/asia/south-korea-president-park-impeachment.html}{expected
to rule}in the coming weeks whether Ms. Park should be formally
unseated.

As Mr. Ban crisscrossed the country after his return, paying homage to
the dead at national cemeteries and shaking hands with street vendors,
his detractors trailed him, holding signs that called him ``an
opportunist,'' or worse.

``He has spent his entire life on the sunny side,'' said Moon Jae-in, an
opposition leader who comes in ahead of Mr. Ban in surveys on
presidential hopefuls. ``He is not the kind who shares the people's
desperate desire for change.''

Lee Hae-chan, who served as South Korea's prime minister when Mr. Ban
was its foreign minister from 2004 to 2006, called Mr. Ban ``a diplomat
who looks twice but does not leap.''

Mr. Ban won his United Nations job 10 years ago with the support of Roh
Moo-hyun, then the president, a progressive who handpicked him as a
candidate. Critics called Mr. Ban a turncoat when he later appeared to
align himself closely with conservatives, including Ms. Park. His
popularity rating as a presidential contender has plummeted in the wake
of Ms. Park's scandal.

Image

A statue of Mr. Ban near the home where he once lived in the city of
Chungju.Credit...Jean Chung for The New York Times

Since he has returned home, Mr. Ban has defined himself as a
``progressive conservative'' who can mend an ideologically fractured
country. But some local news media suspect him of fence-sitting while he
tries to find an ally among the existing political parties or to woo
enough lawmakers away to form his own. They also call him a ``slippery
eel,'' accusing him of being notoriously vague on tough questions, a
trait that has sometimes served him well as a diplomat but now is under
harsher scrutiny as he considers a presidential bid.

As United Nations secretary general, he praised the deeply unpopular
agreement Ms. Park struck with Japan to end a dispute over ``comfort
women,'' or Korean sex slaves for Japan's World War II army. But as a
presidential hopeful, he began raising questions about the deal, saying
that an agreement that failed to satisfy the surviving victims was not
enough.

With his popularity ratings stuck behind Mr. Moon's, the usually
mild-mannered Mr. Ban began bridling at criticism. When journalists
recently dogged him with hard questions and then wrote articles that
accused him of being double-faced over the comfort women issue,
he\href{http://tv.naver.com/v/1388672}{called them names}. (He later
apologized.)

He has also called himself a Mr. Clean, responding to the outcry over
Ms. Park's scandal. But to people weary of recurring corruption scandals
among political leaders and their families, Mr. Ban's claim has already
lost some of its luster; this month, the
\href{https://www.justice.gov/usao-sdny/pr/four-individuals-charged-foreign-bribery-and-fraud-scheme-involving-potential-800}{United
States indicted} his nephew, who is a New York real estate broker, and
his younger brother in South Korea on charges of attempting to pay
bribes to facilitate a Korean company's sale of a 72-story commercial
building in Vietnam. Mr. Ban denies involvement.

Yet here in his home province of Chungcheong, pride in Mr. Ban is
compared to a personality cult by his critics. Streets, marathons and
English-speaking contests are named after him. Songs are written about
him, including one that calls him ``Korea's favorite son who embraced
five oceans and six continents.''

In Eumseong, the seat of the county that includes this village, a park
displays a circle of bronze busts of Mr. Ban and other former United
Nations secretaries general.

And this hamlet, which has only a dozen households, is a veritable Ban
Ki-moon theme park.

A monument erected by the local Ban clan calls him a ``sacred peak of
the world'' whose ``warm smile dissipated international conflicts.''
Visitors stroll around the ``Ban Ki-moon Peace Land,'' a small park with
a granite monument in the shape of the United Nations headquarters
surrounded by flags of member states.

Image

At Haengchi Village, a replica of Mr. Ban's birthplace was built on the
site of the original.Credit...Jean Chung for The New York Times

In the ``Ban Ki-moon Memorial Hall,'' biographical sketches and video
clippings tell how Mr. Ban, with his quiet tenacity and ``warm
charisma,'' overcame his humble origin and became the ``president of the
world.''

``If I sleep now, I may dream, but if I study now, my dream will come
true,'' goes one of the 19 Ban Ki-moon sayings in a museum handout.

Older villagers remember the young Mr. Ban walking on a dirt road with
his eyes fixated on an English textbook. (His English skills gave him
his first big break: As a teenager in 1962, he excelled at an
English-language contest, winning a Red Cross-sponsored trip to the
White House, where he met President John F. Kennedy and resolved to
become a diplomat.)

A roadside motel here added more rooms because so many newlyweds
believed that if their firstborns were conceived with the blessing of
the energy of the mountains surrounding the village, they would grow up
to be luminaries like Mr. Ban.

``I brought my children here so they can learn from Secretary General
Ban's life that there is no easy way in life but that if you try hard,
your dream comes true,'' said Lee Dae-won, 42, who recently visited here
with his daughter and son.

Kim Young-gu, 65, a motorcycle-shop owner who recently visited Mr. Ban's
birthplace, said Mr. Ban's ``vast experiences and self-control'' will
make him a great president, enabling him to avoid the kind of mistakes
that led to Ms. Park's scandal.

``It's an honor to have him in our country,'' he said.

But another visitor, Kim Ki-tae, also a Ban fan, feared that Mr. Ban
might not survive the thrust and parry of domestic politics.

``I wonder why he risks ruining his image by entering domestic
politics,'' he said. ``It's a mud pit, and he could end up losing all.''

Advertisement

\protect\hyperlink{after-bottom}{Continue reading the main story}

\hypertarget{site-index}{%
\subsection{Site Index}\label{site-index}}

\hypertarget{site-information-navigation}{%
\subsection{Site Information
Navigation}\label{site-information-navigation}}

\begin{itemize}
\tightlist
\item
  \href{https://help.nytimes.com/hc/en-us/articles/115014792127-Copyright-notice}{©~2020~The
  New York Times Company}
\end{itemize}

\begin{itemize}
\tightlist
\item
  \href{https://www.nytco.com/}{NYTCo}
\item
  \href{https://help.nytimes.com/hc/en-us/articles/115015385887-Contact-Us}{Contact
  Us}
\item
  \href{https://www.nytco.com/careers/}{Work with us}
\item
  \href{https://nytmediakit.com/}{Advertise}
\item
  \href{http://www.tbrandstudio.com/}{T Brand Studio}
\item
  \href{https://www.nytimes.com/privacy/cookie-policy\#how-do-i-manage-trackers}{Your
  Ad Choices}
\item
  \href{https://www.nytimes.com/privacy}{Privacy}
\item
  \href{https://help.nytimes.com/hc/en-us/articles/115014893428-Terms-of-service}{Terms
  of Service}
\item
  \href{https://help.nytimes.com/hc/en-us/articles/115014893968-Terms-of-sale}{Terms
  of Sale}
\item
  \href{https://spiderbites.nytimes.com}{Site Map}
\item
  \href{https://help.nytimes.com/hc/en-us}{Help}
\item
  \href{https://www.nytimes.com/subscription?campaignId=37WXW}{Subscriptions}
\end{itemize}
