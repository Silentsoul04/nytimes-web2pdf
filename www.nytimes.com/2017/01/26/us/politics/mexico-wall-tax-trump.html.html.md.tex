Sections

SEARCH

\protect\hyperlink{site-content}{Skip to
content}\protect\hyperlink{site-index}{Skip to site index}

\href{https://www.nytimes.com/section/politics}{Politics}

\href{https://myaccount.nytimes.com/auth/login?response_type=cookie\&client_id=vi}{}

\href{https://www.nytimes.com/section/todayspaper}{Today's Paper}

\href{/section/politics}{Politics}\textbar{}Tax Plan Sows Confusion as
Tensions With Mexico Soar

\url{https://nyti.ms/2jCoY7b}

\begin{itemize}
\item
\item
\item
\item
\item
\item
\end{itemize}

Advertisement

\protect\hyperlink{after-top}{Continue reading the main story}

Supported by

\protect\hyperlink{after-sponsor}{Continue reading the main story}

\hypertarget{tax-plan-sows-confusion-as-tensions-with-mexico-soar}{%
\section{Tax Plan Sows Confusion as Tensions With Mexico
Soar}\label{tax-plan-sows-confusion-as-tensions-with-mexico-soar}}

\includegraphics{https://static01.nyt.com/images/2017/01/27/us/27tax/27tax-articleInline.jpg?quality=75\&auto=webp\&disable=upscale}

By \href{http://www.nytimes.com/by/michael-d-shear}{Michael D. Shear},
\href{http://www.nytimes.com/by/binyamin-appelbaum}{Binyamin Appelbaum}
and \href{https://www.nytimes.com/by/alan-rappeport}{Alan Rappeport}

\begin{itemize}
\item
  Jan. 26, 2017
\item
  \begin{itemize}
  \item
  \item
  \item
  \item
  \item
  \item
  \end{itemize}
\end{itemize}

PHILADELPHIA --- President Trump's decision to build a wall along the
length of the United States' southern border with Mexico erupted into a
diplomatic standoff on Thursday, leading to the cancellation of a White
House visit by Mexico's president and sharply rising tensions over who
would pay for the wall.

With the conflict escalating, Mr. Trump appeared to embrace a proposal
by House Republicans that would impose a 20 percent tax on all imported
goods. The White House press secretary, Sean Spicer, told reporters that
the proceeds would be used to pay for the border wall, estimated to cost
as much as \$20 billion.

But a furious uproar prompted Mr. Spicer to temper his earlier remarks,
saying the plan was simply ``one idea'' that might work to finance the
wall. Mr. Spicer said it was not the job of the White House to ``roll
something out'' on tax policy, while Mr. Trump's chief of staff, Reince
Priebus, said the administration was considering ``a buffet of
options.''

If Mr. Trump does eventually announce his support for the tax plan, it
could have a broad impact on the American economy, and its consumers and
workers, by sharply increasing the prices of imported goods or reducing
profits for the companies that produce them. Other nations could
retaliate, prompting a trade war that could hit consumers around the
globe.

Retail businesses could see their tax bills surge, said David French of
the National Retail Federation, who predicted that those costs would be
passed on to consumers. He called the idea ``very counter to the way
consumers are feeling at the moment.''

If nothing else, the rapid-fire developments showed Mr. Trump that
international diplomacy and a top-to-bottom overhaul of the tax code
would not be as easy as an announcement before a campaign microphone.
The events unfolded after Mr. Trump signed an executive order on
Wednesday to strengthen the nation's deportation force and start
construction on a new wall along the border.

Adding to Mexico's perception of an insult was the timing of the order:
It came on the first day of talks between top Mexican officials and
their counterparts in Washington, and just days before a scheduled
meeting between Mr. Trump and the Mexican president, Enrique Peña Nieto.

The sense of chaos and confusion about the tax issue added to the
fallout from Mr. Trump's conflict with Mr. Peña Nieto, his first direct
clash with a world leader since becoming president a week ago. The
Mexican peso bounced sharply with each new development.

Tensions between the two have been simmering for months, despite
comments by both men that they were trying to work together. Mr. Trump's
immigration and border-wall decisions on Wednesday appeared to shatter
the remaining good will between them.

In a \href{https://twitter.com/EPN/status/824447050066468865}{video
message} delivered on Twitter on Wednesday night, Mr. Peña Nieto
reiterated his commitment to protect the interests of Mexico and the
Mexican people, and pledged to devote the resources of Mexico's
consulates in the United States to protecting its citizens.

``I regret and condemn the United States' decision to continue with the
construction of a wall that, for years now, far from uniting us, divides
us,'' Mr. Peña Nieto said.

Mr. Trump responded on Twitter, ``If Mexico is unwilling to pay for the
badly needed wall, then it would be better to cancel the upcoming
meeting.''

Within hours, that is just what happened. Blasting Mr. Trump for sowing
division between the countries, Mr. Peña Nieto angrily backed out of the
White House meeting, which had been scheduled for next week.

In remarks at congressional Republicans' retreat in Philadelphia, Mr.
Trump portrayed the decision to cancel the meeting as his own and issued
a stern warning to Mr. Peña Nieto about the consequences of refusing to
cooperate with him on financing the wall.

``Unless Mexico is going to treat the United States fairly, with
respect, such a meeting would be fruitless, and I want to go a different
route,'' Mr. Trump said. ``We have no choice.''

In the same remarks, Mr. Trump alluded to the idea of a border tax,
saying, ``We're working on a tax reform bill that will reduce our trade
deficits, increase American exports, and will generate revenue from
Mexico that will pay for the wall if we decide to go that route.''

After the speech, in a brief, impromptu news conference as Mr. Trump
flew back to Washington, Mr. Spicer told reporters that the president
now favored the plan to impose a 20 percent border tax as part of a
sweeping overhaul of corporate taxation. Only last week, Mr. Trump had
dismissed the tax as too complicated, favoring his own plan to impose a
35 percent tariff on manufactured goods made by American corporations in
overseas factories.

Mr. Spicer said that the plan for the tax was ``taking shape'' and that
it was ``really going to provide the funding'' for the wall.

Mr. Spicer said that was a direct reference to the centerpiece of House
Republicans' proposal to overhaul the tax code. They have been pushing
the idea for months, but with little evidence, until Thursday, that Mr.
Trump was interested in it.

\includegraphics{https://static01.nyt.com/images/2017/01/27/us/orders-vid/orders-vid-videoSixteenByNine3000.jpg}

But by the time Mr. Spicer returned to the White House two hours later,
he had already recanted. In another hastily arranged conversation with
reporters, he called the proposal ``one idea'' that might work and said
it was not the job of the White House to ``roll something out'' on tax
policy.

``We've been asked over and over again: `How could you possibly do this?
There's no way that Mexico will pay for it,' '' Mr. Spicer said.
``Here's one way. Boom. Done. We could go in another direction. We could
talk about tariffs. We could talk about other custom user fees. There
are a hundred other things.''

The White House and House Republicans have been hashing out their
respective tax proposals as they press forward with Mr. Trump's agenda
to revive American manufacturing and increase exports.

The House proposal would replace the current system of corporate
taxation with one that more closely resembles the approach taken by many
other developed nations. The government would impose a 20 percent tax on
corporate income earned in the United States, which would have the
effect of taxing imports while exempting exports.

The approach, known as border adjustment, creates the appearance of
taxing trade deficits. The goods that the United States imported from
Mexico in 2015 were worth about \$60 billion more than the goods it
exported to Mexico, so federal revenue in the short term would increase
by roughly \$12 billion.

But the House plan would offset that revenue by reducing the 35 percent
corporate income tax rate, and would thus generate no new federal
revenue over all. It was unclear how that fit with Mr. Spicer's repeated
contention Thursday afternoon that revenue from the tax adjustment would
help finance construction of the border wall.

By siphoning off that revenue, Mr. Trump would make it impossible to
reduce the tax rate as far as Republicans wish. He is pressing for a 15
percent corporate tax rate.

Moreover, the tax would not be paid by Mexico. It would be paid by
companies selling Mexican goods in the United States. Some might raise
prices, imposing the cost on consumers, while others might be forced by
competitive pressures to absorb the tax, reducing their profits. Many
economists also doubt that the change would end up penalizing imports or
encouraging exports. They predict that the value of the dollar would
rise, offsetting those effects.

Nonetheless, many businesses in industries such as retail and energy,
which rely heavily on imports, were in a panic.

Representative Kevin Brady, the Texas Republican who wrote the plan,
told Fox News on Thursday afternoon that he was pleased that Mr. Trump
appeared to be on board with it after his appearance in Philadelphia.

``What I heard today from this president was that in tax reform, that
they would level the playing field for imports around the world and
level it with the U.S. products here in America at the exact same
rate,'' Mr. Brady said.

Advertisement

\protect\hyperlink{after-bottom}{Continue reading the main story}

\hypertarget{site-index}{%
\subsection{Site Index}\label{site-index}}

\hypertarget{site-information-navigation}{%
\subsection{Site Information
Navigation}\label{site-information-navigation}}

\begin{itemize}
\tightlist
\item
  \href{https://help.nytimes.com/hc/en-us/articles/115014792127-Copyright-notice}{©~2020~The
  New York Times Company}
\end{itemize}

\begin{itemize}
\tightlist
\item
  \href{https://www.nytco.com/}{NYTCo}
\item
  \href{https://help.nytimes.com/hc/en-us/articles/115015385887-Contact-Us}{Contact
  Us}
\item
  \href{https://www.nytco.com/careers/}{Work with us}
\item
  \href{https://nytmediakit.com/}{Advertise}
\item
  \href{http://www.tbrandstudio.com/}{T Brand Studio}
\item
  \href{https://www.nytimes.com/privacy/cookie-policy\#how-do-i-manage-trackers}{Your
  Ad Choices}
\item
  \href{https://www.nytimes.com/privacy}{Privacy}
\item
  \href{https://help.nytimes.com/hc/en-us/articles/115014893428-Terms-of-service}{Terms
  of Service}
\item
  \href{https://help.nytimes.com/hc/en-us/articles/115014893968-Terms-of-sale}{Terms
  of Sale}
\item
  \href{https://spiderbites.nytimes.com}{Site Map}
\item
  \href{https://help.nytimes.com/hc/en-us}{Help}
\item
  \href{https://www.nytimes.com/subscription?campaignId=37WXW}{Subscriptions}
\end{itemize}
