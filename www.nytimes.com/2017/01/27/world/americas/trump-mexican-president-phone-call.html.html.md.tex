Sections

SEARCH

\protect\hyperlink{site-content}{Skip to
content}\protect\hyperlink{site-index}{Skip to site index}

\href{https://www.nytimes.com/section/world/americas}{Americas}

\href{https://myaccount.nytimes.com/auth/login?response_type=cookie\&client_id=vi}{}

\href{https://www.nytimes.com/section/todayspaper}{Today's Paper}

\href{/section/world/americas}{Americas}\textbar{}Trump and Mexican
President Speak by Phone Amid Dispute Over Wall

\url{https://nyti.ms/2kbwdGE}

\begin{itemize}
\item
\item
\item
\item
\item
\end{itemize}

Advertisement

\protect\hyperlink{after-top}{Continue reading the main story}

Supported by

\protect\hyperlink{after-sponsor}{Continue reading the main story}

\hypertarget{trump-and-mexican-president-speak-by-phone-amid-dispute-over-wall}{%
\section{Trump and Mexican President Speak by Phone Amid Dispute Over
Wall}\label{trump-and-mexican-president-speak-by-phone-amid-dispute-over-wall}}

\includegraphics{https://static01.nyt.com/images/2017/01/27/world/27mexicomeeting/27mexicomeeting-articleInline.jpg?quality=75\&auto=webp\&disable=upscale}

By \href{http://www.nytimes.com/by/kirk-semple}{Kirk Semple}

\begin{itemize}
\item
  Jan. 27, 2017
\item
  \begin{itemize}
  \item
  \item
  \item
  \item
  \item
  \end{itemize}
\end{itemize}

\href{https://www.nytimes.com/es/2017/01/27/trump-y-pena-nieto-acuerdan-no-hablar-mas-sobre-el-muro-fronterizo-en-publico/}{Leer
en español}

MEXICO CITY --- Amid the diplomatic showdown over President Trump's
order to build a border wall --- and who should pay for it --- Mr. Trump
and President
\href{https://www.nytimes.com/2017/01/08/world/americas/unrest-mexico-pena-nieto-gas-prices-trump.html}{Enrique
Peña Nieto}of
\href{https://www.nytimes.com/topic/destination/mexico?8qa}{Mexico}
spoke by telephone for an hour on Friday and agreed to proceed with
negotiations on a range of bilateral issues, both leaders said.

The presidents made an effort to say that the call had been productive,
implying that tensions had cooled, but they stood their ground on their
positions, with no suggestion that the conversation had resolved their
disagreement over fundamental issues, most notably the payment for
construction of a wall between the United States and Mexico.

``With respect to payment for the border wall, both presidents
recognized their clear and very public differences in their stances on
this very sensitive issue, and agreed to solve these differences as part
of a comprehensive discussion of all aspects of the bilateral
relationship,'' according to a statement issued by Mr. Peña Nieto's
office.

The statement added: ``The presidents also agreed for now not to speak
publicly about this controversial issue.''

The White House issued an almost identical statement, calling it ``a
joint statement.'' But it differed in one key respect: It did not
include any mention of an agreement to refrain from speaking publicly
about the wall or its financing.

The standoff with Mexico's president was the first full-blown foreign
policy clash with a foreign leader of the Trump administration, and was
the culmination of months of tension between the two men surrounding Mr.
Trump's pledge to build the wall and renegotiate Nafta, the trade
accord.

The rancor came to a head this week after Mr. Trump issued an executive
order for
\href{https://www.nytimes.com/2017/01/25/us/politics/refugees-immigrants-wall-trump.html?_r=0}{construction
of the border wall}. On Thursday, Mr. Peña Nieto, who has long urged
diplomacy in the face of Mr. Trump's disparaging comments about Mexico,
canceled a meeting with the new American leader that had been scheduled
for Tuesday.

``We had a very good call,'' Mr. Trump said Friday at a news conference
in Washington. ``The border is soft and weak, drugs are pouring in, and
I'm not going to let that happen.''

``I have great respect for Mexico,'' he continued. ``I love the Mexican
people.'' But he added that Mexico ``has outnegotiated us and beat us to
a pulp through our past leaders.''

``They've made us look foolish,'' he added.

Neither side disclosed who had initiated the phone conversation.

Mr. Peña Nieto did not comment beyond the statement issued by his
office. But a Mexican government official described the conversation as
``cordial'' and said it went ``relatively well.''

The presidents did not set another date to meet, in part to allow the
relationship to ``cool off,'' said the official, who asked not to be
identified to discuss a diplomatic issue.

But they agreed to allow their teams to push forward with negotiations
on an array of bilateral issues, including trade, immigration and the
cross-border trafficking of drugs and weapons, the official said.

They also agreed to leave discussion about the wall and its funding out
of the negotiations, in effect quarantining it so that it did not
contaminate conversations about other issues, the official said.

Mr. Peña Nieto was widely applauded in Mexico for canceling the meeting
--- a rare moment of public approval for a wildly unpopular leader.

He received additional backing on Friday from
\href{https://www.nytimes.com/topic/person/carlos-slim-helu?8qa}{Carlos
Slim}, the Mexican billionaire, who held a rare news conference to
criticize Mr. Trump and publicly throw his support behind the Mexican
government as it entered what he called ``arduous, difficult''
negotiations with the United States, specifically referring to Nafta.

``We have to support the president of Mexico to defend the national
interest,'' Mr. Slim told a packed news conference at his business
headquarters in Mexico City.

Mr. Slim said ``there are those'' who ``apparently don't have pluralism,
diversity, liberty, human rights, globalization, productivity, the
environment, competition, on their radar,'' in a clear reference to Mr.
Trump, whom he dined with in Palm Beach last month.

The White House team, he continued, is ``negotiating from a position of
strength to see if we weaken.'' Like a growing number of Mexican
business leaders and policy makers, Mr. Slim said that Mexico should be
prepared to allow the United States to withdraw from Nafta if the
Americans' demands were too high.

But he said that Mexican negotiators must try to convince the Americans
that ``it's better to do more business with us.''

The relationship between Presidents Peña Nieto and Trump --- not to
mention Mr. Trump's with the rest of Mexico --- had been on a downward
trajectory since Mr. Trump took aim at Mexican immigrants during the
2015 speech that kicked off his campaign, and proceeded to build his
White House run on promises to build a border wall, deport millions of
Mexicans and pull out of Nafta.

On Monday, in advance of the meeting between the presidents that had
been scheduled for next week, Mr. Peña Nieto released a set of
principles that he said would guide any negotiations between the two
countries. ``Mexico doesn't believe in walls,'' he said.

The next day, however, Mr. Trump issued a Twitter message reaffirming
his vow to build the wall, to which the Mexican president, responding in
a televised speech distributed on social media, once again criticized
the project and reiterated his longstanding vow that Mexico would never
pay for it.

But on Thursday morning, Mr. Trump seemed to escalate the matter further
with a threat to cancel the meeting should Mexico not be willing to pay
for the wall, immediately prompting Mr. Peña Nieto's cancellation of the
meeting, a gambit that was widely applauded here by a population that
has been clamoring for him to take a forcible stand against Mr. Trump's
provocations.

Mr. Slim, in his rambling 90-minute news conference, extolled the
national unity he witnessed this week as Mexicans overwhelmingly
rejected Mr. Trump's treatment of Mexico.

``This national unity is very important to allow the government to go
with dignity and strength to negotiate,'' he said.

Advertisement

\protect\hyperlink{after-bottom}{Continue reading the main story}

\hypertarget{site-index}{%
\subsection{Site Index}\label{site-index}}

\hypertarget{site-information-navigation}{%
\subsection{Site Information
Navigation}\label{site-information-navigation}}

\begin{itemize}
\tightlist
\item
  \href{https://help.nytimes.com/hc/en-us/articles/115014792127-Copyright-notice}{©~2020~The
  New York Times Company}
\end{itemize}

\begin{itemize}
\tightlist
\item
  \href{https://www.nytco.com/}{NYTCo}
\item
  \href{https://help.nytimes.com/hc/en-us/articles/115015385887-Contact-Us}{Contact
  Us}
\item
  \href{https://www.nytco.com/careers/}{Work with us}
\item
  \href{https://nytmediakit.com/}{Advertise}
\item
  \href{http://www.tbrandstudio.com/}{T Brand Studio}
\item
  \href{https://www.nytimes.com/privacy/cookie-policy\#how-do-i-manage-trackers}{Your
  Ad Choices}
\item
  \href{https://www.nytimes.com/privacy}{Privacy}
\item
  \href{https://help.nytimes.com/hc/en-us/articles/115014893428-Terms-of-service}{Terms
  of Service}
\item
  \href{https://help.nytimes.com/hc/en-us/articles/115014893968-Terms-of-sale}{Terms
  of Sale}
\item
  \href{https://spiderbites.nytimes.com}{Site Map}
\item
  \href{https://help.nytimes.com/hc/en-us}{Help}
\item
  \href{https://www.nytimes.com/subscription?campaignId=37WXW}{Subscriptions}
\end{itemize}
