Sections

SEARCH

\protect\hyperlink{site-content}{Skip to
content}\protect\hyperlink{site-index}{Skip to site index}

\href{https://www.nytimes.com/section/world/europe}{Europe}

\href{https://myaccount.nytimes.com/auth/login?response_type=cookie\&client_id=vi}{}

\href{https://www.nytimes.com/section/todayspaper}{Today's Paper}

\href{/section/world/europe}{Europe}\textbar{}Trump Meets Theresa May of
Britain as He Weighs Lifting Russia Sanctions

\url{https://nyti.ms/2kbs1Ha}

\begin{itemize}
\item
\item
\item
\item
\item
\end{itemize}

Advertisement

\protect\hyperlink{after-top}{Continue reading the main story}

Supported by

\protect\hyperlink{after-sponsor}{Continue reading the main story}

\hypertarget{trump-meets-theresa-may-of-britain-as-he-weighs-lifting-russia-sanctions}{%
\section{Trump Meets Theresa May of Britain as He Weighs Lifting Russia
Sanctions}\label{trump-meets-theresa-may-of-britain-as-he-weighs-lifting-russia-sanctions}}

\includegraphics{https://static01.nyt.com/images/2017/01/28/us/28trumpmay-clips/28trumpmay-clips-videoSixteenByNine3000.jpg}

By \href{http://www.nytimes.com/by/peter-baker}{Peter Baker}

\begin{itemize}
\item
  Jan. 27, 2017
\item
  \begin{itemize}
  \item
  \item
  \item
  \item
  \item
  \end{itemize}
\end{itemize}

WASHINGTON ---
\href{https://www.nytimes.com/topic/person/donald-trump}{President
Trump} plans to open a dialogue with Russia on Saturday that could lead
to lifting American sanctions, even as Britain's visiting prime minister
and leading senators from his own party urged him not to let up the
pressure on the Kremlin until it reverses its armed intervention in
Ukraine.

In what will be their first conversation since Mr. Trump took office, he
and President Vladimir V. Putin of Russia intended to
\href{http://www.nytimes.com/aponline/2017/01/27/world/europe/ap-eu-russia-us.html}{talk
by telephone} on Saturday about areas of possible cooperation,
particularly in fighting terrorism in the Middle East, a collaboration
that would represent a significant turnabout from years of friction
between the two countries.

At the direction of the White House, American officials in recent days
have been preparing memos outlining possible common ground, including
the prospect of removing some or all of the sanctions imposed by former
President Barack Obama, according to officials briefed on the matter.
Mr. Trump has suggested lifting the punitive measures in exchange for
nuclear arms cuts and Russian cooperation in fighting the Islamic State.

Asked about sanctions on Friday, Mr. Trump played down the possibility
of quick action, but did not rule it out.

``As for sanctions, very early to be talking about that,'' Mr. Trump
said Friday at a White House news conference with Prime Minister Theresa
May of Britain. ``But we look to have a great relationship with all
countries, ideally.''

As for Mr. Putin, he offered a far more distant assessment after months
of praising the Russian president for his leadership. ``I don't know the
gentleman,'' Mr. Trump said. ``I hope we have a fantastic relationship.
That's possible, and it's also possible that we won't. We will see what
happens.''

\href{https://www.nytimes.com/interactive/2017/01/27/us/politics/donald-trump-theresa-may-live.html}{}

\includegraphics{https://static01.nyt.com/images/2017/01/28/world/28trumpmay4_hp/28trumpmay4_hp-thumbLarge.jpg}

\hypertarget{president-trump-and-theresa-may-joint-news-conference-video-and-analysis}{%
\subsection{President Trump and Theresa May Joint News Conference: Video
and
Analysis}\label{president-trump-and-theresa-may-joint-news-conference-video-and-analysis}}

Join Times reporters for live updates and analysis from President
Trump's joint news conference with Prime Minister Theresa May of Great
Britain.

Standing at Mr. Trump's side, Mrs. May warned against easing sanctions
unless Russia abides by a peace settlement for Ukraine negotiated in
Minsk, the capital of Belarus. ``We believe the sanctions should
continue until we see that Minsk Agreement fully implemented, and we've
been continuing to argue that inside the European Union,'' she said.

Any talk of lifting sanctions is all but certain to spark the first
serious conflagration between Mr. Trump and congressional Republicans,
who have largely given the president a pass on
\href{https://mobile.nytimes.com/2017/01/26/us/politics/trump-republican-retreat.html?hp=\&action=click\&pgtype=Homepage\&clickSource=story-heading\&module=b-lede-package-region\&region=top-news\&WT.nav=top-news\&smid=tw-nytpolitics\&smtyp=cur\&referer=https://t.co/MklSBYHS66}{myriad
policy areas} where they disagree. Republican lawmakers have been
bracing for Mr. Trump to make this move, and their concerns deepened
Friday when his counselor, Kellyanne
Conway,\href{https://twitter.com/foxandfriends/status/824966414456852480}{said
in an interview} that removing sanctions was under consideration.

Senator John McCain of Arizona warned Mr. Trump on Friday against
lifting sanctions and vowed to push legislation reinstating them if he
does, a measure that already has strong bipartisan support, including
from Republicans like Senators Rob Portman of Ohio and Ben Sasse of
Nebraska. Senator Mitch McConnell of Kentucky, the majority leader, who
has largely shunned confrontation with Mr. Trump, has been a
longstanding opponent of lifting sanctions, a position he
forcefully\href{http://www.politico.com/story/2017/01/mitch-mcconnell-trump-russia-sanctions-234281}{reiterated}on
Friday.

In a scathing statement against Mr. Putin, Mr. McCain cataloged all of
Russia's controversial actions in Ukraine, Syria and elsewhere and said
it could not be trusted as a partner.

``President Trump should remember this when he speaks to Vladimir
Putin,'' Mr. McCain said. ``He should remember that the man on the other
end of the line is a murderer and a thug who seeks to undermine American
national security interests at every turn. For our commander in chief to
think otherwise would be naïve and dangerous.''

Mr. Portman concurred, saying in a statement, ``We must stand by our
allies in the region, including Ukraine.''

Mr. Trump's meeting with Mrs. May was his first with a visiting foreign
leader since taking office with a promise to pursue an ``America First''
foreign policy. For Mr. Trump, it was a debut on the world stage that
took on additional meaning after a scheduled White House visit by
Mexico's president next week fell apart in a dispute over the border
wall Mr. Trump wants to build.

Mr. Trump appeared comfortable and confident with Mrs. May standing to
his right. He offered a brief opening statement that referred twice to
the ``special relationship'' between the two countries, a phrase Britons
take seriously. He offered crisp answers, in contrast to Mr. Obama, who
tended to talk at length. While Mr. Trump did not demonstrate detailed
policy knowledge, he went out of his way to emphasize commonalities with
Mrs. May.

He also tried to reassure Europeans who view him with deep skepticism.
When a British reporter referred to him as a ``brash TV extrovert,'' Mr.
Trump replied, ``Actually, I'm not as brash as you might think.''

Mrs. May, eager to
\href{https://www.nytimes.com/2017/01/27/world/europe/theresa-may-trump.html}{forge
a relationship with him} akin to Margaret Thatcher's alliance with
Ronald Reagan, reciprocated the warm sentiments, praising his ``stunning
election victory'' and conveying an invitation from Queen Elizabeth II
for the president to make a state visit, which he accepted.

Addressing one area of disagreement, Mrs. May said that the president
had privately expressed his support for NATO, despite past comments
disparaging the alliance as ``obsolete.'' ``Mr. President,'' she said,
``I think you said, you confirmed that you're 100 percent behind NATO.''

Mr. Trump embraced the decision by British voters to exit the European
Union, a referendum known as ``Brexit'' that he and others have seen as
a precursor to his own election. ``I think Brexit's going to be a
wonderful thing for your country,'' he said. ``I think when it irons
out, you're going to have your identity and you're going to have the
people that you want in your country.''

Mr. Trump and Mrs. May talked about
\href{https://www.nytimes.com/2017/01/26/business/trump-trade-theresa-may-uk-britain.html}{negotiating
a new free-trade agreement} between the two countries.

Ms. May is not the only European leader worried about Mr. Trump's
blossoming friendship with Russia, which United States intelligence
agencies have concluded hacked Democratic email accounts to influence
the American election last year. Chancellor Angela Merkel of Germany has
been a leading voice of keeping the pressure on the Kremlin, and Mr.
Trump is scheduled to talk with her by telephone, too.

The United States and Europe have imposed sanctions on Russian officials
and companies, mainly in response to the seizure and annexation of
Crimea and the separatist war fomented in eastern Ukraine. Before
leaving office, Mr. Obama imposed
\href{https://www.nytimes.com/2016/12/29/us/politics/russia-election-hacking-sanctions.html}{additional
sanctions} in response to the Russian election hacking.

In an
\href{http://www.thetimes.co.uk/article/full-transcript-of-interview-with-donald-trump-5d39sr09d}{interview
with The Times of London} shortly before taking office, Mr. Trump
suggested a bargain that would ease sanctions on Russia in exchange for
nuclear arms cuts and cooperation in the fight against the Islamic
State.

One topic that may come up on Saturday's call is
\href{https://www.nytimes.com/2015/03/27/world/middleeast/us-and-syria-discuss-missing-journalist.html}{the
fate of Austin Tice}, a freelance journalist who disappeared in Syria in
2012. Mr. Trump may ask Mr. Putin for help in pressuring Russia's ally,
President Bashar al-Assad of Syria, to release Mr. Tice, according to an
official briefed on the matter. Syria has never acknowledged holding
him, but Mr. Trump has considered dropping support for the Syrian
opposition.

Advertisement

\protect\hyperlink{after-bottom}{Continue reading the main story}

\hypertarget{site-index}{%
\subsection{Site Index}\label{site-index}}

\hypertarget{site-information-navigation}{%
\subsection{Site Information
Navigation}\label{site-information-navigation}}

\begin{itemize}
\tightlist
\item
  \href{https://help.nytimes.com/hc/en-us/articles/115014792127-Copyright-notice}{©~2020~The
  New York Times Company}
\end{itemize}

\begin{itemize}
\tightlist
\item
  \href{https://www.nytco.com/}{NYTCo}
\item
  \href{https://help.nytimes.com/hc/en-us/articles/115015385887-Contact-Us}{Contact
  Us}
\item
  \href{https://www.nytco.com/careers/}{Work with us}
\item
  \href{https://nytmediakit.com/}{Advertise}
\item
  \href{http://www.tbrandstudio.com/}{T Brand Studio}
\item
  \href{https://www.nytimes.com/privacy/cookie-policy\#how-do-i-manage-trackers}{Your
  Ad Choices}
\item
  \href{https://www.nytimes.com/privacy}{Privacy}
\item
  \href{https://help.nytimes.com/hc/en-us/articles/115014893428-Terms-of-service}{Terms
  of Service}
\item
  \href{https://help.nytimes.com/hc/en-us/articles/115014893968-Terms-of-sale}{Terms
  of Sale}
\item
  \href{https://spiderbites.nytimes.com}{Site Map}
\item
  \href{https://help.nytimes.com/hc/en-us}{Help}
\item
  \href{https://www.nytimes.com/subscription?campaignId=37WXW}{Subscriptions}
\end{itemize}
