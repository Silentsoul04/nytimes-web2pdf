Sections

SEARCH

\protect\hyperlink{site-content}{Skip to
content}\protect\hyperlink{site-index}{Skip to site index}

\href{https://www.nytimes.com/section/us}{U.S.}

\href{https://myaccount.nytimes.com/auth/login?response_type=cookie\&client_id=vi}{}

\href{https://www.nytimes.com/section/todayspaper}{Today's Paper}

\href{/section/us}{U.S.}\textbar{}Judge Blocks Trump Order on Refugees
Amid Chaos and Outcry Worldwide

\url{https://nyti.ms/2ke1qJe}

\begin{itemize}
\item
\item
\item
\item
\item
\item
\end{itemize}

Advertisement

\protect\hyperlink{after-top}{Continue reading the main story}

Supported by

\protect\hyperlink{after-sponsor}{Continue reading the main story}

\hypertarget{judge-blocks-trump-order-on-refugees-amid-chaos-and-outcry-worldwide}{%
\section{Judge Blocks Trump Order on Refugees Amid Chaos and Outcry
Worldwide}\label{judge-blocks-trump-order-on-refugees-amid-chaos-and-outcry-worldwide}}

\href{https://www.nytimes.com/slideshow/2017/01/28/us/refugees-stopped-at-airports-after-trumps-order.html}{}

\hypertarget{refugees-stopped-at-airports-after-trumps-order}{%
\subsection{Refugees Stopped at Airports After Trump's
Order}\label{refugees-stopped-at-airports-after-trumps-order}}

9 Photos

View Slide Show ›

\includegraphics{https://static01.nyt.com/images/2017/01/28/us/29doorcloses-slide-6F8Z/29doorcloses-slide-6F8Z-articleLarge.jpg?quality=75\&auto=webp\&disable=upscale}

Justin Lane/European Pressphoto Agency

By \href{http://www.nytimes.com/by/michael-d-shear}{Michael D. Shear},
\href{http://www.nytimes.com/by/nicholas-kulish}{Nicholas Kulish} and
\href{http://www.nytimes.com/by/alan-feuer}{Alan Feuer}

\begin{itemize}
\item
  Jan. 28, 2017
\item
  \begin{itemize}
  \item
  \item
  \item
  \item
  \item
  \item
  \end{itemize}
\end{itemize}

WASHINGTON --- A federal judge in Brooklyn came to the aid of scores of
refugees and others who were trapped at airports across the United
States on Saturday after an executive order signed by President Trump,
which sought to keep many foreigners from entering the country, led to
chaotic scenes across the globe.

The judge's ruling blocked part of the president's actions, preventing
the government from deporting some arrivals who found themselves
ensnared by the presidential order. But it stopped short of letting them
into the country or issuing a broader ruling on the constitutionality of
Mr. Trump's actions.

The high-stakes legal case played out on Saturday amid global turmoil,
as the executive order signed by the president slammed shut the borders
of the United States for an Iranian scientist headed to a lab in
Massachusetts, a Syrian refugee family headed to a new life in Ohio and
countless others across the world.

The president's order, enacted with the stroke of a pen at 4:42 p.m.
Friday, suspended entry of all refugees to the United States for 120
days, barred Syrian refugees indefinitely, and blocked entry into the
United States for 90 days for citizens of seven predominantly Muslim
countries: Iran, Iraq, Libya, Somalia, Sudan, Syria and Yemen.

The Department of Homeland Security said that the order also barred
green card holders from those countries from re-entering the United
States. In a briefing for reporters, White House officials said that
green card holders from the seven affected countries who are outside the
United States would need a case-by-case waiver to return.

\href{https://www.nytimes.com/interactive/2017/01/28/us/politics/trump-darweesh-decision-stay-refugee-ban.html}{}

\includegraphics{https://static01.nyt.com/images/2017/01/28/us/politics/image-Darweesh-v-Trump-Order-on-Emergency-Motion-For/image-Darweesh-v-Trump-Order-on-Emergency-Motion-For-thumbLarge.gif}

\hypertarget{darweesh-v-trump-order}{%
\subsection{Darweesh v Trump Order}\label{darweesh-v-trump-order}}

A federal judge blocked part of President Trump's executive order on
immigration, ordering that refugees and others trapped at airports
across the United States should not be sent back to their home
countries.

Mr. Trump --- in office just a week --- found himself accused of
constitutional and legal overreach by two Iraqi immigrants, defended by
the American Civil Liberties Union. Meanwhile, large crowds of
protesters turned out at airports around the country to denounce Mr.
Trump's ban on the entry of refugees and people from seven predominantly
Muslim countries.

Lawyers who sued the government to block the White House order said the
judge's decision could affect an estimated 100 to 200 people who were
detained upon arrival at American airports.

Judge Ann M. Donnelly of Federal District Court in Brooklyn, who was
nominated by former President Barack Obama, ruled just before 9 p.m.
that implementing Mr. Trump's order by sending the travelers home could
cause them ``irreparable harm.'' She said the government was ``enjoined
and restrained from, in any manner and by any means, removing
individuals'' who had arrived in the United States with valid visas or
refugee status.

The ruling does not appear to force the administration to let in people
otherwise blocked by Mr. Trump's order who have not yet traveled to the
United States.

The judge's one-page ruling came swiftly after lawyers for the A.C.L.U.
testified in her courtroom that one of the people detained at an airport
was being put on a plane to be deported back to Syria at that very
moment. A government lawyer, Gisela A. Westwater, who spoke to the court
by phone from Washington, said she simply did not know.

\includegraphics{https://static01.nyt.com/images/2017/01/29/us/29jfkprotests-vid-2/29jfkprotests-vid-2-videoSixteenByNine3000.jpg}

Hundreds of people waited outside of the courthouse chanting, ``Set them
free!'' as lawyers made their case. When the crowd learned that Judge
Donnelly had ruled in favor of the plaintiffs, a rousing cheer went up
in the crowd.

Minutes after the judge's ruling in New York City, another judge, Leonie
M. Brinkema of Federal District Court in Virginia, issued a temporary
restraining order for a week to block the removal of any green card
holders being detained at Dulles International Airport.

In a statement released early Sunday morning, the Department of Homeland
Security said it would continue to enforce all of the president's
executive orders, even while complying with judicial decisions.
``Prohibited travel will remain prohibited,'' the department said in a
statement, adding that the directive was ``a first step towards
re-establishing control over America's borders and national security.''

Around the nation, security personnel at major international airports
had new rules to follow, though the application of the order appeared
chaotic and uneven. Humanitarian organizations delivered the bad news to
overseas families that had overcome the bureaucratic hurdles previously
in place and were set to travel. And refugees already on flights when
the order was signed on Friday found themselves detained upon arrival.

``We've gotten reports of people being detained all over the country,''
said Becca Heller, the director of the International Refugee Assistance
Project. ``They're literally pouring in by the minute.''

\includegraphics{https://static01.nyt.com/images/2017/01/29/us/29Doorclosed-1/29Doorclosed-1-articleLarge.jpg?quality=75\&auto=webp\&disable=upscale}

Earlier in the day, at the White House, Mr. Trump shrugged off the sense
of anxiety and disarray, suggesting that there had been an orderly
rollout. ``It's not a Muslim ban, but we were totally prepared,'' he
said. ``It's working out very nicely. You see it at the airports, you
see it all over.''

But to many, the government hardly seemed prepared for the upheaval that
Mr. Trump's actions put into motion.

There were numerous reports of students attending American universities
who were blocked from returning to the United States from visits abroad.
One student said in a Twitter post that he would be unable to study at
Yale. Another who attends the Massachusetts Institute of Technology was
refused permission to board a plane. A Sudanese graduate student at
Stanford University was blocked for hours from entering the country.

Human rights groups reported that legal permanent residents of the
United States who hold green cards were being stopped in foreign
airports as they sought to return from funerals, vacations or study
abroad. There was widespread condemnation of the order, from religious
leaders, business executives, academics, political leaders and others.
Mr. Trump's supporters offered praise, calling it a necessary step on
behalf of the nation's security.

Homeland Security officials said on Saturday night that 109 people who
were already in transit to the United States when the order was signed
were denied access; 173 were stopped before boarding planes heading to
America. Eighty-one people who were stopped were eventually given
waivers to enter the United States, officials said.

\href{https://www.nytimes.com/interactive/2017/01/28/us/politics/ACLU-Complaint.html}{}

\includegraphics{https://static01.nyt.com/images/2017/01/28/us/politics/ACLU-Complaint/ACLU-Complaint-articleLarge.gif}

\hypertarget{aclu-complaint-on-trump-immigration-order}{%
\subsection{A.C.L.U. Complaint on Trump Immigration
Order}\label{aclu-complaint-on-trump-immigration-order}}

The A.C.L.U. and other legal organizations filed a lawsuit on Saturday
on behalf of individuals subject to President Trump's executive order.
The lead plaintiffs were detained by the U.S. government and threatened
with deportation.

Legal residents who have a green card and are currently in the United
States should meet with a consular officer before leaving the country, a
White House official, who spoke on the condition of anonymity, told
reporters. Officials did not clarify the criteria that would qualify
someone for a waiver, other than that it would be granted ``in the
national interest.''

But the week-old administration appeared to be implementing the order
chaotically, with agencies and officials around the globe interpreting
it in different ways.

The Stanford student, Nisrin Omer, a legal permanent resident, said she
was held at Kennedy International Airport in New York for about five
hours but was eventually allowed to leave the airport. Others who were
detained appeared to be still in custody or sent back to their home
countries.

White House aides claimed on Saturday that there had been consultations
with State Department and homeland security officials about carrying out
the order. ``Everyone who needed to know was informed,'' one aide said.

But that assertion was denied by multiple officials with knowledge of
the interactions, including two officials at the State Department.
Leaders of Customs and Border Protection and of Citizenship and
Immigration Services --- the two agencies most directly affected by the
order --- were on a telephone briefing on the new policy even as Mr.
Trump signed it on Friday, two officials said.

Image

Protesters gathered at Dulles International Airport outside
Washington.Credit...Paul J. Richards/Agence France-Presse --- Getty
Images

The A.C.L.U.'s legal case began with two Iraqis detained at Kennedy
Airport, the named plaintiffs in the case. One was en route to reunite
with his wife and son in Texas. The other had served alongside Americans
in Iraq for a decade.

Shortly after noon on Saturday, Hameed Khalid Darweesh, an interpreter
who worked for more than a decade on behalf of the United States
government in Iraq, was released. After nearly 19 hours of detention,
Mr. Darweesh began to cry as he spoke to reporters, putting his hands
behind his back and miming handcuffs.

``What I do for this country? They put the cuffs on,'' Mr. Darweesh
said. ``You know how many soldiers I touch by this hand?''

The other man the lawyers are representing, Haider Sameer Abdulkhaleq
Alshawi, who was en route to Houston, was released Saturday night.

Before the two men were released, one of the lawyers, Mark Doss, a
supervising attorney at the International Refugee Assistance Project,
asked an official, ``Who is the person we need to talk to?''

\href{https://www.nytimes.com/interactive/2017/01/28/us/politics/document-Virgina-Ruling-TRO-Order-Signed.html}{}

\includegraphics{https://static01.nyt.com/images/2017/01/28/us/politics/image-Virgina-Ruling-TRO-Order-Signed/image-Virgina-Ruling-TRO-Order-Signed-articleLarge.gif}

\hypertarget{virginia-ruling-on-trump-order}{%
\subsection{Virginia Ruling on Trump
Order}\label{virginia-ruling-on-trump-order}}

A federal judge in Virginia issued a temporary stay on part of President
Trump's immigration order.

``Call Mr. Trump,'' said the official, who declined to identify himself.

While the judge's ruling means that none of the detainees will be sent
back immediately, lawyers for the plaintiffs in the case expressed
concern that all those at the airports would now be put in detention,
pending a resolution of the case.

The White House said the restrictions would protect ``the United States
from foreign nationals entering from countries compromised by
terrorism'' and allow the administration time to put in place ``a more
rigorous vetting process.'' But critics condemned Mr. Trump over the
collateral damage on people who had no sinister intentions in trying to
come to the United States.

Peaceful protests began forming Saturday afternoon at Kennedy Airport,
where nine travelers had been detained upon arrival at Terminal 7 and
two others at Terminal 4, an airport official said. Similar scenes were
playing out at other airports across the nation.

An official message to all American diplomatic posts around the world
provided instructions about how to treat people from the countries
affected: ``Effective immediately, halt interviewing and cease issuance
and printing'' of visas to the United States.

Internationally, confusion turned to panic as travelers found themselves
unable to board flights bound for the United States. In Dubai and
Istanbul, airport and immigration officials turned passengers away at
boarding gates and, in at least one case, ejected a family from a flight
it had boarded.

Seyed Soheil Saeedi Saravi, a promising young Iranian scientist, had
been scheduled to travel in the coming days to Boston, where he had been
awarded a fellowship to study cardiovascular medicine at Harvard,
according to Thomas Michel, the professor who was to supervise the
research fellowship.

But Professor Michel said the visas for the student and his wife had
been indefinitely suspended.

``This outstanding young scientist has enormous potential to make
contributions that will improve our understanding of heart disease, and
he has already been thoroughly vetted,'' Professor Michel wrote to The
New York Times.

A Syrian family of six who have been living in a Turkish refugee camp
since fleeing their home in 2014
\href{http://www.cleveland.com/metro/index.ssf/2017/01/president_donal_trump_acts_to.html}{had
been scheduled to arrive on Tuesday in Cleveland}. Instead, the family's
trip has been called off.

``Everyone is just so heartbroken, so angry, so sad,'' said Danielle
Drake, the community manager for US Together, an agency that resettles
refugees.

A Christian family of six from Syria said in an email to Representative
Charlie Dent, Republican of Pennsylvania, that they were being detained
on Saturday morning at Philadelphia International Airport despite having
legal paperwork, green cards and visas that had been approved.

In the case of the two Iraqis held at Kennedy Airport, the legal filings
by his lawyers say that Mr. Darweesh was granted a special immigrant
visa on Jan. 20, the same day Mr. Trump was sworn in as president.

A husband and father of three, Mr. Darweesh arrived at Kennedy Airport
with his family. Mr. Darweesh's wife and children made it through
passport control and customs, but agents of Customs and Border
Protection detained him.

In Istanbul, during a stopover on Saturday, passengers reported that
security officers had entered a plane after everyone had boarded and
ordered a young Iranian woman and her family to leave the aircraft.

Iranian green card holders who live in the United States were blindsided
by the decree while on vacation in Iran, finding themselves in a legal
limbo and unsure whether they would be able to return to America.

``How do I get back home now?'' said Daria Zeynalia, a green card holder
who was visiting family in Iran. He had rented a house and leased a car,
and would be eligible for citizenship in November. ``What about my job?
If I can't go back soon, I'll lose everything.''

Advertisement

\protect\hyperlink{after-bottom}{Continue reading the main story}

\hypertarget{site-index}{%
\subsection{Site Index}\label{site-index}}

\hypertarget{site-information-navigation}{%
\subsection{Site Information
Navigation}\label{site-information-navigation}}

\begin{itemize}
\tightlist
\item
  \href{https://help.nytimes.com/hc/en-us/articles/115014792127-Copyright-notice}{©~2020~The
  New York Times Company}
\end{itemize}

\begin{itemize}
\tightlist
\item
  \href{https://www.nytco.com/}{NYTCo}
\item
  \href{https://help.nytimes.com/hc/en-us/articles/115015385887-Contact-Us}{Contact
  Us}
\item
  \href{https://www.nytco.com/careers/}{Work with us}
\item
  \href{https://nytmediakit.com/}{Advertise}
\item
  \href{http://www.tbrandstudio.com/}{T Brand Studio}
\item
  \href{https://www.nytimes.com/privacy/cookie-policy\#how-do-i-manage-trackers}{Your
  Ad Choices}
\item
  \href{https://www.nytimes.com/privacy}{Privacy}
\item
  \href{https://help.nytimes.com/hc/en-us/articles/115014893428-Terms-of-service}{Terms
  of Service}
\item
  \href{https://help.nytimes.com/hc/en-us/articles/115014893968-Terms-of-sale}{Terms
  of Sale}
\item
  \href{https://spiderbites.nytimes.com}{Site Map}
\item
  \href{https://help.nytimes.com/hc/en-us}{Help}
\item
  \href{https://www.nytimes.com/subscription?campaignId=37WXW}{Subscriptions}
\end{itemize}
