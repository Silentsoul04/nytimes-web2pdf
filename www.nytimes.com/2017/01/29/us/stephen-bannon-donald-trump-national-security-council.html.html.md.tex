Sections

SEARCH

\protect\hyperlink{site-content}{Skip to
content}\protect\hyperlink{site-index}{Skip to site index}

\href{https://www.nytimes.com/section/us}{U.S.}

\href{https://myaccount.nytimes.com/auth/login?response_type=cookie\&client_id=vi}{}

\href{https://www.nytimes.com/section/todayspaper}{Today's Paper}

\href{/section/us}{U.S.}\textbar{}Bannon Is Given Security Role Usually
Held for Generals

\url{https://nyti.ms/2jM9qOj}

\begin{itemize}
\item
\item
\item
\item
\item
\end{itemize}

Advertisement

\protect\hyperlink{after-top}{Continue reading the main story}

Supported by

\protect\hyperlink{after-sponsor}{Continue reading the main story}

\hypertarget{bannon-is-given-security-role-usually-held-for-generals}{%
\section{Bannon Is Given Security Role Usually Held for
Generals}\label{bannon-is-given-security-role-usually-held-for-generals}}

\includegraphics{https://static01.nyt.com/images/2017/01/30/us/30BANNON-01/30NSC-01-articleLarge.jpg?quality=75\&auto=webp\&disable=upscale}

By \href{https://www.nytimes.com/by/glenn-thrush}{Glenn Thrush} and
\href{http://www.nytimes.com/by/maggie-haberman}{Maggie Haberman}

\begin{itemize}
\item
  Jan. 29, 2017
\item
  \begin{itemize}
  \item
  \item
  \item
  \item
  \item
  \end{itemize}
\end{itemize}

WASHINGTON --- The whirlwind first week of Donald J. Trump's presidency
had all the bravura hallmarks of a Stephen K. Bannon production.

It started with the doom-hued
\href{https://www.nytimes.com/2017/01/20/us/politics/trump-inauguration-day.html}{inauguration
homily}to ``American carnage'' in United States cities co-written by Mr.
Bannon, followed a few days later by his
\href{https://www.nytimes.com/2017/01/26/business/media/stephen-bannon-trump-news-media.html}{``shut
up'' message} to the news media. The week culminated with a blizzard of
executive orders, mostly hatched by Mr. Bannon's team and the White
House policy adviser, Stephen Miller, aimed at disorienting the
``enemy,'' fulfilling campaign promises and distracting attention from
Mr. Trump's less than flawless debut.

But the defining moment for Mr. Bannon came Saturday night in the form
of an executive order giving the rumpled right-wing agitator a full seat
on the ``principals committee'' of the National Security Council ---
while downgrading the roles of the chairman of the Joint Chiefs of Staff
and the director of national intelligence, who will now attend only when
the council is considering issues in their direct areas of
responsibilities. It is a startling elevation of a political adviser, to
a status alongside the secretaries of state and defense, and over the
president's top military and intelligence advisers.

In theory, the move put Mr. Bannon, a former Navy surface warfare
officer, admiral's aide, investment banker, Hollywood producer and
Breitbart News firebrand, on the same level as his friend, Michael T.
Flynn, the national security adviser, a former Pentagon intelligence
chief who was Mr. Trump's top adviser on national security issues before
a series of missteps reduced his influence.

But in terms of real influence, Mr. Bannon looms above almost everyone
except the president's son-in-law, Jared Kushner, in the Trumpian
pecking order, according to interviews with two dozen Trump insiders and
current and former national security officials. The move involving Mr.
Bannon, as well as the boost in status to the White House homeland
security adviser, Thomas P. Bossert, and Mr. Trump's relationships with
cabinet appointees like Defense Secretary Jim Mattis, have essentially
layered over Mr. Flynn.

Sean Spicer, the White House press secretary, said Mr. Bannon --- whose
Breitbart website was a magnet for white nationalists, antiglobalists
and conspiracy theorists --- always planned to participate in national
security. Mr. Flynn welcomed his participation, Mr. Spicer said, but the
general ``led the reorganization of the N.S.C.'' in order to streamline
an antiquated and bloated bureaucracy.

Former White House officials in both parties were shocked by the move.

``The last place you want to put somebody who worries about politics is
in a room where they're talking about national security,'' said Leon E.
Panetta, a former White House chief of staff, defense secretary and
C.I.A. director in two Democratic administrations.

``I've never seen that happen, and it shouldn't happen. It's not like he
has broad experience in foreign policy and national security issues. He
doesn't. His primary role is to control or guide the president's
conscience based on his campaign promises. That's not what the National
Security Council is supposed to be about.''

That opinion was shared by President George W. Bush's last chief of
staff, Josh Bolten, who barred Karl Rove, Mr. Bush's political adviser,
from N.S.C. meetings. A president's decisions made with those advisers,
he told a conference audience in September, ``involve life and death for
the people in uniform'' and should ``not be tainted by any political
decisions.''

Susan E. Rice, President Barack Obama's last national security adviser,
called the arrangement ``stone cold crazy''
\href{https://twitter.com/AmbassadorRice/status/825612583159803904}{in a
tweet posted Sunday}.

Mr. Spicer said the language the Trump White House used in its N.S.C.
executive order is, with the exception of Mr. Bannon's position ---
which was created during the transition --- almost identical in content
to one the Bush administration drafted in 2001. And Mr. Obama's top
political operative, David Axelrod, sat in on some N.S.C. meetings, he
added.

There were key differences. Mr. Axelrod never served as a permanent
member as Mr. Bannon will now, though he sat in on some critical
meetings, especially as Mr. Obama debated strategy in Afghanistan and
Pakistan. ``It's a profound shift,'' Mr. Axelrod said. ``I don't know
what his bona fides are to be the principal foreign policy adviser to
the president.''

But Mr. Bannon's elevation does not merely reflect his growing influence
on national security. It is emblematic of Mr. Trump's trust on a range
of political and ideological issues.

During the campaign, the sly and provocative Mr. Bannon played a
paradoxical role --- calming the easily agitated candidate during his
frequent rough patches and egging him on when he felt Mr. Trump needed
to fire up the white working-class base. The president respects Mr.
Bannon because he is independently wealthy and therefore does not need
the job, and both men ascribe to a shoot-the-prisoners credo when put on
the defensive, according to the former Trump campaign manager Corey
Lewandowski.

Mr. Bannon is a deft operator within the White House, and he has been
praised by Republicans who view him skeptically as the most
knowledgeable on policy around the president. But his stated preference
for blowing things up --- as opposed to putting them back together ---
may not translate to his new role.

The hasty drafting of the immigration order, and its scattershot
execution, brought a measure of Mr. Bannon's chaotic and hyperaggressive
political style to the more predictable administration of the federal
government. Within hours of the edict, airport customs and border agents
were detaining or blocking dozens of migrant families, some of whom had
permanent resident status, until John F. Kelly, the new homeland
security secretary, intervened.

Mr. Kelly's department had suggested green card holders be exempted from
the order, but Mr. Bannon and Mr. Miller, a hard-liner on immigration,
overruled him, according to two American officials.

Mr. Priebus, speaking on NBC's ``Meet the Press'' on Sunday, indicated a
softening of the stance, saying the order would not block ``green card
holders moving forward'' --- but said anyone seeking to enter the
country from the listed countries would be subjected to tighter
scrutiny.

People close to Mr. Bannon said he is not accumulating power for power's
sake, but is instead helping to fill a staff leadership vacuum created,
in part, by Mr. Flynn's stumbling performance as national security
adviser.

Mr. Flynn still communicates with Mr. Trump frequently, and his staff
has been assembling a version of the Presidential Daily Briefing for Mr.
Trump, truncated but comprehensive, to be the president's main source of
national security information. During the campaign, he often had
unfettered access to the candidate, who appreciated his brash style and
contempt for Hillary Clinton, but during the transition, Mr. Flynn
privately complained about having to share face time with others.

Mr. Flynn ``has the full confidence of the president and his team,''
Hope Hicks, a spokeswoman for Mr. Trump, said in an email. Emails and
phone calls to Mr. Flynn and his top aide were not returned.

A president who likes generals and abhors political correctness, Mr.
Trump found in Mr. Flynn --- who joined Trump backers in an anti-Clinton
``lock her up!'' chant during the campaign --- perhaps the most
politically incorrect general this side of his hero, Gen. George S.
Patton.

But Mr. Flynn, a lifelong Democrat sacked as head of the Pentagon's
intelligence arm after clashing with Obama administration officials in
2014, has gotten on the nerves of Mr. Trump and other administration
officials because of his sometimes overbearing demeanor, and has further
diminished his internal standing by presiding over a
\href{https://www.nytimes.com/2017/01/18/us/politics/trump-team-has-barely-engaged-with-national-security-council.html}{chaotic
and opaque N.S.C. transition process} that prioritized the hiring of
military officials over civilian experts recommended to him by his own
team.

Mr. Flynn's penchant for talking too much was on display on Friday in a
meeting with Theresa May, the British prime minister, according to two
people with direct knowledge of the events.

When Mrs. May said that she understood wanting a dialogue with Mr. Putin
but stressed the need to be careful, Mr. Trump asked Mr. Flynn when the
two were scheduled to speak.

Mr. Flynn replied it was Saturday --- he had delayed it to fit in Mrs.
May's meeting for ``protocol'' as a United States ally, adding at length
that Mr. Putin was impatient to chat.

Mr. Trump, the person said, appeared irritated by the response.

Still, the episode that did the most damage to the Trump-Flynn
relationship occurred in early December when Mr. Flynn's son, also named
Michael, unleashed a series of tweets pushing a discredited conspiracy
theory that Clinton associates had run a child sex-slave ring out of a
Washington pizza restaurant.

Mr. Trump told his staff to get rid of the younger Mr. Flynn, who had
been hired by his father to help during the transition. But Mr. Trump
did so reluctantly because of his loyalty during the campaign, when
dozens of former military officials were dismissing Mr. Trump as too
unstable to command.

``I want him fired immediately,'' Mr. Trump said in a muted rendition of
his ``You're fired!'' line in ``The Apprentice,'' according to two
people with knowledge of the interaction.

That has not stopped the general's son from spouting off: On Saturday,
at a time when Trump surrogates were pushing back on the idea that the
executive order did not discriminate against any religion, the younger
Mr. Flynn tweeted his approval of the policy, adding ``\#MuslimBan.''
The tweet was subsequently deleted; his entire account disappeared later
in the day.

Still, the national security adviser has also continued to dabble in the
kind of bomb-throwing behavior that concerns Mr. Trump's allies, such as
planning to attend an anti-Clinton ``Deploraball'' event at the time of
the inauguration. He was urged to skip it by Trump allies, and
ultimately agreed.

Both Mr. Trump and Mr. Bannon still regard Mr. Flynn as an asset. ``In
the room and out of the room, Steve Bannon is General Flynn's biggest
defender,'' said Kellyanne Conway, another top adviser to the president.

But it is unclear when the maneuvers to reduce Mr. Flynn's role began.
Two Obama administration officials said Trump transition officials
inquired about expanded national security roles for Mr. Bannon and Mr.
Kushner at the earliest stages of the transition in November --- before
the younger Mr. Flynn became a liability --- but after Mr. Flynn had
begun to chafe on the nerves of his colleagues on the team.

Mr. Flynn's reputation has raised questions among some in the cabinet.
Two weeks ago, both men held a meeting with Rex W. Tillerson, Mr.
Trump's pick to run the State Department, Mr. Mattis and Mike Pompeo,
now the C.I.A. director, to discuss coordination --- Mr. Flynn was
invited but did not attend.

Part of the meeting was devoted to discussing concerns about Mr. Flynn,
according to an official with knowledge of it.

Advertisement

\protect\hyperlink{after-bottom}{Continue reading the main story}

\hypertarget{site-index}{%
\subsection{Site Index}\label{site-index}}

\hypertarget{site-information-navigation}{%
\subsection{Site Information
Navigation}\label{site-information-navigation}}

\begin{itemize}
\tightlist
\item
  \href{https://help.nytimes.com/hc/en-us/articles/115014792127-Copyright-notice}{©~2020~The
  New York Times Company}
\end{itemize}

\begin{itemize}
\tightlist
\item
  \href{https://www.nytco.com/}{NYTCo}
\item
  \href{https://help.nytimes.com/hc/en-us/articles/115015385887-Contact-Us}{Contact
  Us}
\item
  \href{https://www.nytco.com/careers/}{Work with us}
\item
  \href{https://nytmediakit.com/}{Advertise}
\item
  \href{http://www.tbrandstudio.com/}{T Brand Studio}
\item
  \href{https://www.nytimes.com/privacy/cookie-policy\#how-do-i-manage-trackers}{Your
  Ad Choices}
\item
  \href{https://www.nytimes.com/privacy}{Privacy}
\item
  \href{https://help.nytimes.com/hc/en-us/articles/115014893428-Terms-of-service}{Terms
  of Service}
\item
  \href{https://help.nytimes.com/hc/en-us/articles/115014893968-Terms-of-sale}{Terms
  of Sale}
\item
  \href{https://spiderbites.nytimes.com}{Site Map}
\item
  \href{https://help.nytimes.com/hc/en-us}{Help}
\item
  \href{https://www.nytimes.com/subscription?campaignId=37WXW}{Subscriptions}
\end{itemize}
