Sections

SEARCH

\protect\hyperlink{site-content}{Skip to
content}\protect\hyperlink{site-index}{Skip to site index}

\href{https://www.nytimes.com/section/politics}{Politics}

\href{https://myaccount.nytimes.com/auth/login?response_type=cookie\&client_id=vi}{}

\href{https://www.nytimes.com/section/todayspaper}{Today's Paper}

\href{/section/politics}{Politics}\textbar{}Jeff Sessions Approved as
Attorney General by Senate Committee

\url{https://nyti.ms/2jYnYgo}

\begin{itemize}
\item
\item
\item
\item
\item
\end{itemize}

Advertisement

\protect\hyperlink{after-top}{Continue reading the main story}

Supported by

\protect\hyperlink{after-sponsor}{Continue reading the main story}

\hypertarget{jeff-sessions-approved-as-attorney-general-by-senate-committee}{%
\section{Jeff Sessions Approved as Attorney General by Senate
Committee}\label{jeff-sessions-approved-as-attorney-general-by-senate-committee}}

\includegraphics{https://static01.nyt.com/images/2017/02/02/us/02sessions2/02sessions2-articleInline.jpg?quality=75\&auto=webp\&disable=upscale}

By \href{http://www.nytimes.com/by/eric-lichtblau}{Eric Lichtblau}

\begin{itemize}
\item
  Feb. 1, 2017
\item
  \begin{itemize}
  \item
  \item
  \item
  \item
  \item
  \end{itemize}
\end{itemize}

WASHINGTON --- A divided Senate Judiciary Committee approved the
nomination of Senator Jeff Sessions as attorney general Wednesday,
despite a fierce pushback from Democrats over President Trump's firing
this week of the Justice Department's acting chief, who had objected to
the administration's refugee policy.

The action came on a straight party-line vote, with 11 Republicans
supporting their former colleague from Alabama and nine Democrats
opposing him.

The full, Republican-controlled Senate now appears ready to approve Mr.
Sessions' nomination next week, which would give the Justice Department
and its 113,000 employees a full-time boss after a tumultuous few days
that called its independence into question.

The judiciary committee vote came two days after
\href{https://www.nytimes.com/2017/01/30/us/politics/trump-immigration-ban-memo.html}{Mr.
Trump ousted Sally Q. Yates}, a holdover from the Obama administration
who was acting attorney general. Ms Yates had refused to defend the
president's order on refugees, saying its legality was unclear. The
White House accused her of having ``betrayed'' her department.

Democrats zeroed in Ms. Yates's dismissal, and said Mr. Sessions, an
early supporter of Mr. Trump's long-shot campaign who went on to become
an influential adviser, would not have the independence to challenge the
White House on questions of the law and policy. At his confirmation
hearing last month, Mr. Sessions pledged repeatedly that he would be
able to ``say no'' to Mr. Trump if needed and would not be ``a mere
rubber stamp'' on issues like immigration and national security.

The debate turned ugly as committee members traded personal barbs and
accusations of ``untoward'' behavior. The jousting started when Senator
Al Franken, a Minnesota Democrat, began challenging Mr. Sessions's civil
rights record and claimed that another committee member, Senator Ted
Cruz, Republican of Texas, had ``misrepresented'' the issue.

Mr. Cruz was not at the committee hearing at the time, but his fellow
Texan, Senator John Cornyn, rushed to his defense and cut off Mr.
Franken midspeech.

``I object to the senator disparaging a fellow member of the committee
here in his absence,'' Mr. Cornyn interjected angrily.

``Well, he should be here --- first of all --- and, secondly, he
disparaged me,'' Mr. Franken responded.

``I would hope he would do it to his face,'' Mr. Cornyn said. The attack
on Mr. Cruz, he said a moment later, was ``untoward and it's
inappropriate and I object.''

Senator Charles E. Grassley, the chairman of the committee, told the two
senators that ``we'd be better off if we just let it go.''

But the two senators did not, and Mr. Grassley eventually complained
that they were putting him ``in an awful bad position.''

``Could you please leave personalities out of it?'' he implored.

Mr. Franken went right on talking --- with Mr. Cruz at the center of his
attacks.

Advertisement

\protect\hyperlink{after-bottom}{Continue reading the main story}

\hypertarget{site-index}{%
\subsection{Site Index}\label{site-index}}

\hypertarget{site-information-navigation}{%
\subsection{Site Information
Navigation}\label{site-information-navigation}}

\begin{itemize}
\tightlist
\item
  \href{https://help.nytimes.com/hc/en-us/articles/115014792127-Copyright-notice}{©~2020~The
  New York Times Company}
\end{itemize}

\begin{itemize}
\tightlist
\item
  \href{https://www.nytco.com/}{NYTCo}
\item
  \href{https://help.nytimes.com/hc/en-us/articles/115015385887-Contact-Us}{Contact
  Us}
\item
  \href{https://www.nytco.com/careers/}{Work with us}
\item
  \href{https://nytmediakit.com/}{Advertise}
\item
  \href{http://www.tbrandstudio.com/}{T Brand Studio}
\item
  \href{https://www.nytimes.com/privacy/cookie-policy\#how-do-i-manage-trackers}{Your
  Ad Choices}
\item
  \href{https://www.nytimes.com/privacy}{Privacy}
\item
  \href{https://help.nytimes.com/hc/en-us/articles/115014893428-Terms-of-service}{Terms
  of Service}
\item
  \href{https://help.nytimes.com/hc/en-us/articles/115014893968-Terms-of-sale}{Terms
  of Sale}
\item
  \href{https://spiderbites.nytimes.com}{Site Map}
\item
  \href{https://help.nytimes.com/hc/en-us}{Help}
\item
  \href{https://www.nytimes.com/subscription?campaignId=37WXW}{Subscriptions}
\end{itemize}
