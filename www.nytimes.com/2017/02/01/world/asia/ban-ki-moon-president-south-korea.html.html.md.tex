Sections

SEARCH

\protect\hyperlink{site-content}{Skip to
content}\protect\hyperlink{site-index}{Skip to site index}

\href{https://www.nytimes.com/section/world/asia}{Asia Pacific}

\href{https://myaccount.nytimes.com/auth/login?response_type=cookie\&client_id=vi}{}

\href{https://www.nytimes.com/section/todayspaper}{Today's Paper}

\href{/section/world/asia}{Asia Pacific}\textbar{}Ban Ki-moon Says He
Won't Run for President of South Korea

\url{https://nyti.ms/2jVIXk5}

\begin{itemize}
\item
\item
\item
\item
\item
\end{itemize}

Advertisement

\protect\hyperlink{after-top}{Continue reading the main story}

Supported by

\protect\hyperlink{after-sponsor}{Continue reading the main story}

\hypertarget{ban-ki-moon-says-he-wont-run-for-president-of-south-korea}{%
\section{Ban Ki-moon Says He Won't Run for President of South
Korea}\label{ban-ki-moon-says-he-wont-run-for-president-of-south-korea}}

\includegraphics{https://static01.nyt.com/images/2017/02/02/world/02korea-web1/02korea-web1-articleInline.jpg?quality=75\&auto=webp\&disable=upscale}

By \href{http://www.nytimes.com/by/choe-sang-hun}{Choe Sang-Hun}

\begin{itemize}
\item
  Feb. 1, 2017
\item
  \begin{itemize}
  \item
  \item
  \item
  \item
  \item
  \end{itemize}
\end{itemize}

SEOUL, South Korea --- Ban Ki-moon, the former United Nations chief,
said on Wednesday that he would not run for the presidency of South
Korea, a surprising announcement that deprived beleaguered conservatives
of their likeliest candidate to succeed the country's sidelined leader,
\href{https://www.nytimes.com/topic/person/park-geunhye?inline=nyt-per}{Park
Geun-hye}.

Mr. Ban, who returned to his native South Korea last month after 10
years as the United Nations secretary general,
\href{https://www.nytimes.com/2017/01/25/world/asia/ban-ki-moon-south-korea-president.html}{had
been touted as a viable contender} to replace Ms. Park, a conservative
whose presidential powers have been suspended since the National
Assembly voted to impeach her in December amid accusations of
corruption. But Mr. Ban's approval ratings have been falling, and he has
been the subject of negative news coverage about his policy positions
and a scandal involving his relatives.

``I have decided to fold my pure-hearted plan to lead an effort to
achieve political reform and national unity,'' Mr. Ban said at a news
conference at the National Assembly. He apologized to South Koreans who
had supported his tentative presidential bid, including former diplomats
and politicians.

The Constitutional Court is expected to decide in the coming weeks
\href{https://www.nytimes.com/2016/12/22/world/asia/south-korea-president-park-impeachment.html}{whether
to end Ms. Park's presidency}, and political parties have been gearing
up for an election that could take place as early as this spring. If Ms.
Park survives in office, an election will be held in December to decide
who will succeed her when her five-year term ends next February.

Mr. Ban had acted like a candidate since coming home, paying homage to
the dead at national cemeteries, meeting with politicians to discuss
election strategy and holding news conferences, where he had tirelessly
explained why he would make a good president. He said his experience at
the United Nations would help him lead the country through tough
problems, like
\href{https://www.nytimes.com/2017/01/09/world/asia/north-korea-trump-icbm-test.html?rref=collection\%2Ftimestopic\%2FNuclear\%20Weapons\&action=click\&contentCollection=science\&region=stream\&module=stream_unit\&version=search\&contentPlacement=4\&pgtype=collection}{the
growing nuclear threat from North Korea} and rising discontent over
economic inequality.

But skepticism has abounded over his presidential bid, especially among
progressives. And recent polls indicated that the gap between him and
the front-runner, the opposition leader Moon Jae-in, was only widening.

Detractors portrayed Mr. Ban as a weak diplomat who would be unable to
institute badly needed reforms, or as a stooge for a conservative
establishment desperate for a candidate. And domestic news outlets have
hounded him over what they called his changing stances on the issue of
``comfort women,'' the euphemistic term for the Korean women forced into
sexual slavery for Japanese soldiers during World War II. A bribery
scandal that involved his younger brother and nephew also drew
considerable news coverage.

But the main argument made by his critics was that Mr. Ban, essentially
an outsider with no political faction of his own, would not survive the
thrust and parry of domestic politics. And that was what did him in, Mr.
Ban indicated during his news conference on Wednesday.

``I was deeply disappointed by outdated and narrow-minded egoism among
some politicians,'' Mr. Ban said. ``I have determined that it is
meaningless to try to work with them.'' He said he had been subjected to
``slander and fake news that bordered on character assassination.''

Mr. Ban's elevation to the top United Nations job a decade ago, after a
stint as South Korea's foreign minister, made him one of the country's
most celebrated role models for the young. School textbooks, for
example, refer to him as a ``man who made South Korea proud.''

Some conservative politicians have suggested that Prime Minister Hwang
Kyo-ahn, who is serving as acting president while Ms. Park is on trial
before the Constitutional Court, should run as a conservative candidate.
But Mr. Hwang has not committed to doing so, and he has ranked a distant
third or fourth in recent polls surveying the popularity of potential
candidates.

Advertisement

\protect\hyperlink{after-bottom}{Continue reading the main story}

\hypertarget{site-index}{%
\subsection{Site Index}\label{site-index}}

\hypertarget{site-information-navigation}{%
\subsection{Site Information
Navigation}\label{site-information-navigation}}

\begin{itemize}
\tightlist
\item
  \href{https://help.nytimes.com/hc/en-us/articles/115014792127-Copyright-notice}{©~2020~The
  New York Times Company}
\end{itemize}

\begin{itemize}
\tightlist
\item
  \href{https://www.nytco.com/}{NYTCo}
\item
  \href{https://help.nytimes.com/hc/en-us/articles/115015385887-Contact-Us}{Contact
  Us}
\item
  \href{https://www.nytco.com/careers/}{Work with us}
\item
  \href{https://nytmediakit.com/}{Advertise}
\item
  \href{http://www.tbrandstudio.com/}{T Brand Studio}
\item
  \href{https://www.nytimes.com/privacy/cookie-policy\#how-do-i-manage-trackers}{Your
  Ad Choices}
\item
  \href{https://www.nytimes.com/privacy}{Privacy}
\item
  \href{https://help.nytimes.com/hc/en-us/articles/115014893428-Terms-of-service}{Terms
  of Service}
\item
  \href{https://help.nytimes.com/hc/en-us/articles/115014893968-Terms-of-sale}{Terms
  of Sale}
\item
  \href{https://spiderbites.nytimes.com}{Site Map}
\item
  \href{https://help.nytimes.com/hc/en-us}{Help}
\item
  \href{https://www.nytimes.com/subscription?campaignId=37WXW}{Subscriptions}
\end{itemize}
