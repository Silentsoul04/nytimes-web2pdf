Sections

SEARCH

\protect\hyperlink{site-content}{Skip to
content}\protect\hyperlink{site-index}{Skip to site index}

\href{https://www.nytimes.com/section/world/middleeast}{Middle East}

\href{https://myaccount.nytimes.com/auth/login?response_type=cookie\&client_id=vi}{}

\href{https://www.nytimes.com/section/todayspaper}{Today's Paper}

\href{/section/world/middleeast}{Middle East}\textbar{}Iran Is
Threatened With U.S. Reprisals Over Missile Test

\url{https://nyti.ms/2jV4ruO}

\begin{itemize}
\item
\item
\item
\item
\item
\end{itemize}

Advertisement

\protect\hyperlink{after-top}{Continue reading the main story}

Supported by

\protect\hyperlink{after-sponsor}{Continue reading the main story}

\hypertarget{iran-is-threatened-with-us-reprisals-over-missile-test}{%
\section{Iran Is Threatened With U.S. Reprisals Over Missile
Test}\label{iran-is-threatened-with-us-reprisals-over-missile-test}}

\includegraphics{https://static01.nyt.com/images/2017/02/02/world/02iran/02Geneva-articleInline.jpg?quality=75\&auto=webp\&disable=upscale}

By \href{http://www.nytimes.com/by/mark-landler}{Mark Landler} and
\href{http://www.nytimes.com/by/thomas-erdbrink}{Thomas Erdbrink}

\begin{itemize}
\item
  Feb. 1, 2017
\item
  \begin{itemize}
  \item
  \item
  \item
  \item
  \item
  \end{itemize}
\end{itemize}

WASHINGTON --- The Trump administration on Wednesday fired a warning
shot at a perennial adversary, declaring that it was ``putting Iran on
notice'' after a recent ballistic missile launch, and threatening the
Iranian government with unspecified reprisals.

``As of today, we are officially putting Iran on notice,'' said Michael
T. Flynn, the national security adviser, making his debut in the White
House briefing room to read a terse statement that was almost as
critical of the Obama administration as it was of Iran.

``The Trump administration condemns such actions by Iran that undermine
security, prosperity and stability throughout and beyond the Middle
East, and place American lives at risk,'' he said.

Mr. Flynn said the missile test was the latest in a series of
provocative actions by Iran and violated a United Nations Security
Council resolution restricting its ballistic missile program ---
something the Iranians deny. Mr. Flynn did not specify how the United
States would respond, although other officials have said the White House
is weighing sanctions and other measures to counter Iranian initiatives
throughout the Middle East and the Persian Gulf.

His blunt tone --- and lack of specifics --- offered an early sign of
how President Trump plans to deal with Iran: pushing back against Tehran
on multiple fronts and leaving all options, including military action,
on the table.

Mr. Flynn singled out Iran's support for Houthi rebels in Yemen, who
recently attacked a Saudi naval vessel.

To that end, Defense Department officials said they have been directed
to explore ways the United States can challenge Iran in Yemen, where the
Houthis have been
\href{https://www.nytimes.com/interactive/2016/10/14/world/middleeast/yemen-saudi-arabia-us-airstrikes.html}{battling
Saudi Arabia and other American allies}.

``In these and other similar activities,'' Mr. Flynn said, ``Iran
continues to threaten U.S. friends and allies in the region.''

At a subsequent official briefing, a senior administration official said
the White House was considering a range of options --- and he did not
rule out military force. But he also said the administration, in its
second week, did not want to be premature or rash in how it confronted
Tehran.

The challenge for the administration in contemplating economic pressure
is that it would be all but impossible to reassemble the international
coalition that imposed draconian sanctions on Iran's oil and banking
industries --- and drew Iran into negotiations that resulted in the
\href{https://www.nytimes.com/2015/07/15/world/middleeast/iran-nuclear-deal-is-reached-after-long-negotiations.html}{agreement
limiting its nuclear program}.

Mr. Flynn pinned much of the blame for Iran's aggressiveness on former
President Barack Obama, saying his administration ``failed to respond
adequately to Tehran's malign actions --- including weapons transfers,
support for terrorism and other violations of international norms.''

He also noted that Mr. Trump had criticized agreements between Iran and
the Obama administration as ``weak and ineffective.'' During the
campaign, Mr. Trump spoke of ripping up the Iran nuclear agreement,
though his aides now say their focus is less on abrogating that deal
than on constraining Iran's behavior in the region.

In Yemen, for example, the Pentagon is considering stepped-up patrols
and perhaps even airstrikes, aimed at preventing
\href{https://www.nytimes.com/2017/01/10/world/middleeast/yemen-iran-weapons-houthis.html}{Iranian
weapons from getting to the Houthis}. In addition, Saudi officials are
pushing for more support for their air campaign in Yemen, an
administration official said. But officials said on Wednesday that there
had been no change in the military's posture.

While the Obama administration targeted Houthis and conducted airstrikes
against forces aligned with Al Qaeda in Yemen, current and former
officials say Mr. Obama was wary of deepening American support for the
Saudi air campaign because of concerns about the accuracy of targeting
and the large number of civilian casualties.

``Obama said all the time, `Aim before you shoot,''' said Derek Chollet,
who served in the White House, the Pentagon and the State Department
during the Obama administration. ``Anytime there was one of these heated
discussions, and people said, `We've got to do something,' he said,
`O.K., what does the intel say, and where will this take us?'''

The Trump administration, however, said it would continue to criticize
and draw distinctions with its predecessor.

``This president is seeking to make the country safer, stronger, more
prosperous,'' Sean Spicer, the White House press secretary, said. ``I
think the president, when it comes --- came --- to the Iran nuclear
deal, was very, very adamant in his opposition to the deal and to its
implications.''

Mr. Flynn's tough words left some Iran analysts troubled.

Cliff Kupchan, a political risk analyst at the Eurasia Group in
Washington, said the tone was ``very worrisome.'' He and others also
questioned how Iran's missile test had violated the Security Council
resolution in question, in which Iran is ``called upon'' to refrain from
missile tests but is not forbidden to conduct them.

``It's all Michael Flynn, Steve Bannon and Stephen Miller right now,''
Mr. Kupchan said in an email, referring to the national security adviser
and two other hard-line Trump aides. ``The `revolutionaries' are running
the Trump administration.''

Other analysts, however, said the stiffer tone was overdue.

``It was very sensible for the administration to early on warn Iran of
its malign activities,'' said Ray Takeyh, a senior fellow at the Council
on Foreign Relations. ``The fact is that Iran is probably testing the
administration to see if there is any pushback. Over the past few years
they have not been given too many stern warnings.''

Earlier on Wednesday, Iran confirmed that it had recently conducted a
missile test, but it rejected accusations that the launch had violated a
Security Council resolution.

The confirmation by the Iranian defense minister, Hossein Dehghan, was
the first by an official there since the country was accused of
violating the 2015 resolution because the test involved a ballistic
missile that could theoretically carry a nuclear warhead.

His remarks came a day after President Hassan Rouhani disparaged Mr.
Trump for his order barring refugees, as well as citizens of seven
predominantly Muslim countries including Iran. ``Banning visas for other
nations is the act of newcomers to the political scene,'' Mr. Rouhani
said.

Mr. Dehghan emphasized that the missile test did not, in Iran's view,
violate the resolution, or the 2015 nuclear agreement that preceded it.
No country will be allowed to interfere in Iranian domestic affairs, he
said, adding that tests would definitely continue. ``Our nation has
tested itself in this path,'' Mr. Dehghan said.

The United States called an urgent meeting of the Security Council on
Tuesday to discuss the matter.

``You're going to see us call them out as we said we would, and you are
also going to see us act accordingly,'' Nikki R. Haley, the new United
States ambassador to the United Nations, said on Tuesday.

Advertisement

\protect\hyperlink{after-bottom}{Continue reading the main story}

\hypertarget{site-index}{%
\subsection{Site Index}\label{site-index}}

\hypertarget{site-information-navigation}{%
\subsection{Site Information
Navigation}\label{site-information-navigation}}

\begin{itemize}
\tightlist
\item
  \href{https://help.nytimes.com/hc/en-us/articles/115014792127-Copyright-notice}{©~2020~The
  New York Times Company}
\end{itemize}

\begin{itemize}
\tightlist
\item
  \href{https://www.nytco.com/}{NYTCo}
\item
  \href{https://help.nytimes.com/hc/en-us/articles/115015385887-Contact-Us}{Contact
  Us}
\item
  \href{https://www.nytco.com/careers/}{Work with us}
\item
  \href{https://nytmediakit.com/}{Advertise}
\item
  \href{http://www.tbrandstudio.com/}{T Brand Studio}
\item
  \href{https://www.nytimes.com/privacy/cookie-policy\#how-do-i-manage-trackers}{Your
  Ad Choices}
\item
  \href{https://www.nytimes.com/privacy}{Privacy}
\item
  \href{https://help.nytimes.com/hc/en-us/articles/115014893428-Terms-of-service}{Terms
  of Service}
\item
  \href{https://help.nytimes.com/hc/en-us/articles/115014893968-Terms-of-sale}{Terms
  of Sale}
\item
  \href{https://spiderbites.nytimes.com}{Site Map}
\item
  \href{https://help.nytimes.com/hc/en-us}{Help}
\item
  \href{https://www.nytimes.com/subscription?campaignId=37WXW}{Subscriptions}
\end{itemize}
