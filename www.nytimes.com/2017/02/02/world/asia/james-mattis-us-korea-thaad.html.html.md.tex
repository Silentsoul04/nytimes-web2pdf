Sections

SEARCH

\protect\hyperlink{site-content}{Skip to
content}\protect\hyperlink{site-index}{Skip to site index}

\href{https://www.nytimes.com/section/world/asia}{Asia Pacific}

\href{https://myaccount.nytimes.com/auth/login?response_type=cookie\&client_id=vi}{}

\href{https://www.nytimes.com/section/todayspaper}{Today's Paper}

\href{/section/world/asia}{Asia Pacific}\textbar{}Jim Mattis, in South
Korea, Tries to Reassure an Ally

\url{https://nyti.ms/2jY3i5u}

\begin{itemize}
\item
\item
\item
\item
\item
\end{itemize}

Advertisement

\protect\hyperlink{after-top}{Continue reading the main story}

Supported by

\protect\hyperlink{after-sponsor}{Continue reading the main story}

\hypertarget{jim-mattis-in-south-korea-tries-to-reassure-an-ally}{%
\section{Jim Mattis, in South Korea, Tries to Reassure an
Ally}\label{jim-mattis-in-south-korea-tries-to-reassure-an-ally}}

\includegraphics{https://static01.nyt.com/images/2017/02/03/world/03military-web1/03military-web1-videoSixteenByNineJumbo1600.jpg}

By \href{http://www.nytimes.com/by/michael-r-gordon}{Michael R. Gordon}
and \href{http://www.nytimes.com/by/choe-sang-hun}{Choe Sang-Hun}

\begin{itemize}
\item
  Feb. 2, 2017
\item
  \begin{itemize}
  \item
  \item
  \item
  \item
  \item
  \end{itemize}
\end{itemize}

SEOUL, South Korea --- On his first mission to reassure an important
American ally, Defense Secretary Jim Mattis met on Thursday with top
South Korean officials, who agreed to push ahead with the deployment of
a new missile defense system.

``Thaad is for defense of our allies' people, of our troops who are
committed to their defense,'' Mr. Mattis told reporters, using the
acronym for \href{https://www.mda.mil/system/thaad.html}{Terminal
High-Altitude Area Defense}, the American antimissile system. It is
meant to intercept North Korea's medium-range missiles.

``Were it not for the provocative behavior of North Korea, we would have
no need for Thaad out here,'' Mr. Mattis added. ``There is no other
nation that needs to be concerned about Thaad.''

South Korea was a logical first stop for Mr. Mattis, who will also visit
Japan on the trip. Tensions have risen in the region after Kim Jong-un,
the North Korean leader, proclaimed during his New Year's Day address
that his military
\href{https://www.nytimes.com/2017/01/01/world/asia/north-korea-intercontinental-ballistic-missile-test-kim-jong-un.html}{was
preparing to conduct} its first test launch of an intercontinental
ballistic missile.

Asian nations have also been concerned about the conflicting signals
from President Trump about the United States' posture in the region. The
most recent one was a contentious phone call between Mr. Trump and the
Australian prime minister, which was disclosed just as Mr. Mattis began
his Asia trip.

Mr. Trump mused during his election campaign that the United States
could save money if nations like South Korea and Japan
\href{http://cnnpressroom.blogs.cnn.com/2016/03/29/full-rush-transcript-donald-trump-cnn-milwaukee-republican-presidential-town-hall/}{developed
their own nuclear weapons} --- comments that ran counter to decades of
American nonproliferation policy.

Mr. Trump
\href{https://twitter.com/realDonaldTrump/status/816057920223846400}{said
on Twitter} last month that North Korea would be prevented from
developing the ability to reach the United States with a nuclear weapon.
But he did not say whether he was referring to military or diplomatic
actions. ``It won't happen,'' he
\href{https://www.nytimes.com/2017/01/02/world/asia/trump-twitter-north-korea-missiles-china.html}{tersely
declared} of a North Korean missile test.

One of Mr. Trump's first acts as president was
\href{https://www.nytimes.com/2017/01/23/us/politics/tpp-trump-trade-nafta.html}{to
formally withdraw from the Trans-Pacific Partnership trade agreement},
which had been an important pillar of the Obama administration's policy
in the region. Critics say the withdrawal by the United States will give
China an opportunity to expand its influence.

More recently, in a phone call on Saturday, Mr. Trump reassured the
Japanese prime minister, Shinzo Abe, of the United States' ``ironclad''
commitment to Japan's security, according to a statement from the White
House. Mr. Trump made
\href{https://www.nytimes.com/2017/01/30/world/asia/trump-north-korea-south.html}{a
similar assurance} to South Korea on Monday in a call with the country's
acting president, Hwang Kyo-ahn.

The various messages --- some spontaneous, some premeditated --- have
turned Mr. Mattis's otherwise traditional statements of support for
South Korea and Japan into messages with strategic importance.

\includegraphics{https://static01.nyt.com/images/2017/02/03/world/03military-web2/03military-web2-articleInline.jpg?quality=75\&auto=webp\&disable=upscale}

``It is a priority for President Trump's administration to pay attention
to the northwest Pacific,'' Mr. Mattis said. ``I am going to get current
by listening to them, finding out where their issues are, and then we
are going to work together and strengthen our alliance.''

Mr. Mattis met with an array of officials in Seoul, including Mr. Hwang,
who is the country's prime minister as well as serving as acting
president during the impeachment trial of President Park Geun-hye. If
she is removed, a new presidential election may be held as early as the
spring, so the fraught political situation in South Korea poses a
challenge for the United States.

``Mattis is going to meet with people who probably aren't going to be in
office in a few months,'' said
\href{http://uskoreainstitute.org/research/visiting-scholars/joel-wit/}{Joel
S. Wit}, a Korea expert at the School of Advanced International Studies
at Johns Hopkins University.

The Thaad system is designed to intercept missiles like the Rodong,
which is believed to have enough range to reach all of South Korea and
some parts of Japan. The United States and South Korea initially said
they wanted to deploy the Thaad system by the end of the year, but given
the North's bellicose behavior, there has been some speculation that it
may be deployed sooner.

Under its
\href{https://www.nytimes.com/2016/07/14/world/asia/south-korea-thaad-us.html}{arrangement
with Washington}, South Korea would provide land and build a base for a
Thaad battery, while the United States would pay for the missile system,
which would be built by Lockheed Martin, and then cover its operational
costs.

During Mr. Mattis's meeting with Mr. Hwang, the allies confirmed that
they would deploy the Thaad system as planned.

``Secretary Mattis reaffirmed the United States' firm defense commitment
to South Korea, including the provision of extended deterrence, and said
that the Trump administration will be treating the North Korean nuclear
threat as a top-priority security issue,'' the office of Kim Kwan-jin,
South Korea's director of national security, said in a statement.

Mindful of the possible early election, crucial opposition leaders in
Seoul are opposing the deployment of the Thaad system. They say it would
do little to defend South Korea from the North's plentiful short-range
missiles but would anger China, which might retaliate economically. The
Chinese have long objected to any deployment of limited missile
defenses, out of concern that it would lead to a more comprehensive
antimissile shield that could fend off Beijing's own nuclear missiles.

Moon Jae-in, an opposition leader who is considered the front-runner
among potential presidential candidates, has argued that South Korea
should use the Thaad program as diplomatic leverage with China, keeping
open the possibility that it would not be deployed if China helped rein
in North Korea.

``Given our standoff with North Korea and its nuclear program, our
security and the alliance with the United States are our top priority,''
Mr. Moon told reporters recently. ``But the best scenario for us is when
the U.S. and China get along well. If there is friction between the two,
it's not going to be easy for us.''

Mr. Hwang, in contrast, has said that the Thaad deployment is
``inevitable'' because of the North's rapidly growing missile threat.

``Thaad is a defense tool whose deployment should not be delayed any
more,'' he said at a recent news conference. ``We are explaining our
position in various ways to neighboring countries like China, who are
concerned about the Thaad deployment.''

Advertisement

\protect\hyperlink{after-bottom}{Continue reading the main story}

\hypertarget{site-index}{%
\subsection{Site Index}\label{site-index}}

\hypertarget{site-information-navigation}{%
\subsection{Site Information
Navigation}\label{site-information-navigation}}

\begin{itemize}
\tightlist
\item
  \href{https://help.nytimes.com/hc/en-us/articles/115014792127-Copyright-notice}{©~2020~The
  New York Times Company}
\end{itemize}

\begin{itemize}
\tightlist
\item
  \href{https://www.nytco.com/}{NYTCo}
\item
  \href{https://help.nytimes.com/hc/en-us/articles/115015385887-Contact-Us}{Contact
  Us}
\item
  \href{https://www.nytco.com/careers/}{Work with us}
\item
  \href{https://nytmediakit.com/}{Advertise}
\item
  \href{http://www.tbrandstudio.com/}{T Brand Studio}
\item
  \href{https://www.nytimes.com/privacy/cookie-policy\#how-do-i-manage-trackers}{Your
  Ad Choices}
\item
  \href{https://www.nytimes.com/privacy}{Privacy}
\item
  \href{https://help.nytimes.com/hc/en-us/articles/115014893428-Terms-of-service}{Terms
  of Service}
\item
  \href{https://help.nytimes.com/hc/en-us/articles/115014893968-Terms-of-sale}{Terms
  of Sale}
\item
  \href{https://spiderbites.nytimes.com}{Site Map}
\item
  \href{https://help.nytimes.com/hc/en-us}{Help}
\item
  \href{https://www.nytimes.com/subscription?campaignId=37WXW}{Subscriptions}
\end{itemize}
