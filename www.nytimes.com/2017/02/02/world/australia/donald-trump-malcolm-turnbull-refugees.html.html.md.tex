Sections

SEARCH

\protect\hyperlink{site-content}{Skip to
content}\protect\hyperlink{site-index}{Skip to site index}

\href{https://www.nytimes.com/section/world/australia}{Australia}

\href{https://myaccount.nytimes.com/auth/login?response_type=cookie\&client_id=vi}{}

\href{https://www.nytimes.com/section/todayspaper}{Today's Paper}

\href{/section/world/australia}{Australia}\textbar{}Trump's Harsh Talk
With Malcolm Turnbull of Australia Strains Another Alliance

\url{https://nyti.ms/2k4xz5s}

\begin{itemize}
\item
\item
\item
\item
\item
\end{itemize}

Advertisement

\protect\hyperlink{after-top}{Continue reading the main story}

Supported by

\protect\hyperlink{after-sponsor}{Continue reading the main story}

\hypertarget{trumps-harsh-talk-with-malcolm-turnbull-of-australia-strains-another-alliance}{%
\section{Trump's Harsh Talk With Malcolm Turnbull of Australia Strains
Another
Alliance}\label{trumps-harsh-talk-with-malcolm-turnbull-of-australia-strains-another-alliance}}

\includegraphics{https://static01.nyt.com/images/2017/02/03/world/03Australia1/03Australia1-articleInline.jpg?quality=75\&auto=webp\&disable=upscale}

By \href{http://www.nytimes.com/by/jane-perlez}{Jane Perlez}

\begin{itemize}
\item
  Feb. 2, 2017
\item
  \begin{itemize}
  \item
  \item
  \item
  \item
  \item
  \end{itemize}
\end{itemize}

BEIJING --- President Trump's combative phone call with Australia's
prime minister over a refugee agreement has set off a political storm in
that country, one that threatens to weaken support for a seven-decade
alliance with the United States just as many Australians say they want
closer ties with China.

Enthusiasm for the alliance in Australia, one of America's closest
partners, which hosts American spy facilities and rotations of American
Marines, had already been under pressure from China, with which
Australia conducts the most trade. Reports that Mr. Trump had scolded
Prime Minister Malcolm Turnbull on Saturday, before
\href{https://www.nytimes.com/2017/02/02/us/politics/us-australia-trump-turnbull.html?hp\&action=click\&pgtype=Homepage\&clickSource=story-heading\&module=a-lede-package-region\&region=top-news\&WT.nav=top-news}{abruptly
ending the call}, are likely to further undermine confidence in the
United States, Australian analysts said.

``Trump is needlessly damaging the deep trust that binds one of
America's closest alliances,'' said Professor Rory Medcalf, head of the
National Security College at the Australian National University in
Canberra. ``China and those wishing to weaken the strongest alliance in
the Pacific will see opportunity in this moment.''

In less than two weeks in office, Mr. Trump's actions have strained
alliances and alienated potential partners of the United States, and his
phone call with Mr. Turnbull seemed to be one more example, this time
with a country that has fought on America's side since World War I.

His administration's
\href{https://www.nytimes.com/2017/02/01/world/middleeast/iran-missile-test.html}{confrontational
stance on Iran} has undermined liberal voices in that country; his
restrictions on
\href{https://www.nytimes.com/2017/01/27/us/politics/trump-syrian-refugees.html}{immigration
from some predominantly Muslim countries} have been widely criticized by
allies; and his rejection of the
\href{https://www.nytimes.com/2017/01/23/us/politics/tpp-trump-trade-nafta.html}{Trans-Pacific
Partnership trade deal} threatens to push countries in the Asia-Pacific
region, including Australia, closer to China.

Like many countries in the region, Australia depends on the United
States for its security but looks to China for its economic well-being,
and it does not want to choose definitively between the two as they wage
a global contest for power.

Experts said the American and Australian militaries were sufficiently
intertwined --- the Royal Australian Air Force has flown in Syria, and
Australian soldiers have helped train the Iraqi Army --- that the
countries' security arrangements would endure. But the trust and
confidence underlying the longstanding alliance will be harmed by Mr.
Trump's apparent lack of respect, and his remarks will be very costly in
the public domain, they said.

\includegraphics{https://static01.nyt.com/images/2017/02/03/world/03Australia2/03Australia2-articleInline.jpg?quality=75\&auto=webp\&disable=upscale}

The phone call on Saturday became contentious after Mr. Turnbull pressed
Mr. Trump to honor a deal in which the United States had agreed to take
in up to 1,250 refugees
\href{https://www.nytimes.com/2016/12/09/opinion/sunday/australia-refugee-prisons-manus-island.html}{being
held by Australia} at offshore detention centers.

Under the terms of the deal, hurriedly worked out by Mr. Turnbull and
former President Barack Obama in New York last year, Australia would
also accept Central American refugees staying in a Costa Rican detention
facility.

Australia has been harshly criticized for its offshore detention policy,
and the issue is politically delicate at home. Many of the refugees it
holds, on the Pacific island-nation of Nauru and on the island of Manus
in Papua New Guinea, are from Iran and Iraq. Both countries are among
the seven whose citizens are barred from entering the United States for
at least 90 days under the executive order Mr. Trump signed on Friday.

In his conversation with Mr. Turnbull the next day, Mr. Trump said the
deal with Australia was going to hurt him politically, according to a
senior official in the Trump administration.

Late Wednesday, hours after
\href{https://www.washingtonpost.com/world/national-security/no-gday-mate-on-call-with-australian-pm-trump-badgers-and-brags/2017/02/01/88a3bfb0-e8bf-11e6-80c2-30e57e57e05d_story.html}{details
of the call} were reported, Mr. Trump
\href{https://twitter.com/realDonaldTrump/status/827002559122567168}{wrote
on Twitter} that the agreement was ``dumb.'' He said he would need to
``study'' it, leaving the door open to renege or to accept fewer
refugees.

The White House press secretary, Sean Spicer, told reporters in
Washington on Thursday that Mr. Trump would allow the deal to proceed as
long as the refugees were subjected to ``extreme vetting.'' Mr. Spicer
also said Mr. Trump remained ``extremely upset'' over the deal arranged
by his predecessor.

Mr. Turnbull, whose popularity has been sagging over domestic issues,
struggled on Thursday to cast the call in a positive light, fending off
demands from the opposition Labor Party that he detail exactly what Mr.
Trump had said.

Even after Mr. Trump made his remarks on Twitter, Mr. Turnbull insisted
in a radio interview that he had ``a clear commitment from the
president'' that the resettlement plans would proceed.

``The alliance is absolutely rock solid,'' Mr. Turnbull said. ``It is so
strong.''

The United States Embassy in Canberra had tried to help Mr. Turnbull
with his predicament earlier Thursday, saying the White House had
confirmed that the agreement would be honored. But after Mr. Trump wrote
about the refugee deal on Twitter, the embassy referred questions about
the agreement to the White House.

\includegraphics{https://static01.nyt.com/images/2017/02/03/world/03manus-diary/03manus-diary-videoSixteenByNine3000-v2.jpg}

Mr. Turnbull's Liberal Party is the more conservative of Australia's two
major parties, and it has been a stalwart supporter of close ties with
the United States.

The Labor Party has leaned slightly more toward China, but as the debate
over relations with Washington has intensified in recent months, former
leaders of the party have become outspoken critics of the United States
and have argued for shifting toward Beijing.

Australian attitudes toward relations with the United States, which have
historically been favorable, are now under pressure from China and its
trading weight, according to polls. That is largely because the enormous
Chinese demand for
\href{http://dfat.gov.au/trade/resources/Documents/chin.pdf}{Australia's
resources} --- particularly iron ore, natural gas and coal --- has
bolstered Australia's economy for more than a decade.

Chinese students also contribute substantially to Australian
universities and schools, so much so that many of the institutions are
dependent on the fees for survival. And under a new trade agreement,
Australia is exporting large quantities of wine and meat to China.

After Mr. Trump, in office just a few days, scrapped the Trans-Pacific
Partnership --- a regional trade pact that the Obama administration had
hoped would be an economic counterweight to China --- Mr. Turnbull
announced that he would seek to reconstitute the deal without the United
States, but possibly including China, another indication of Beijing's
clout.

A 2016 survey conducted by the public policy group Lowy Institute asked
respondents to identify the country that was more important to
Australia; 43 percent chose the United States, and 43 percent China. In
2014, 48 percent had answered the United States, and only 37 percent had
chosen China, said
\href{https://www.lowyinstitute.org/people/experts/publication/sam-roggeveen}{Sam
Roggeveen}, a senior fellow at the institute.

According to the 2016 survey, 45 percent said that Australia should
distance itself from the United States if Mr. Trump became president,
Mr. Roggeveen said.

Image

Asylum seekers held up their identity cards after landing on the island
of Manus in Papua New Guinea. Australia has been harshly criticized for
its offshore detention policy.Credit...Eoin Blackwell/Australian
Associated Press, via Associated Press

``We can assume that number will now rise,'' said Mr. Medcalf, of
Australian National University. ``This incident will only intensify the
damage done by Trump's abandonment of the T.P.P., which would have been
a pillar of strategic partnership as well as of trade.''

The United States operates signals intelligence and radar facilities in
remote corners of Australia that are becoming more important as North
Korea's nuclear threat expands, said Peter Jennings, executive director
of the Australian Strategic Policy Institute.

``Australia shares raw and finished intelligence in the closest possible
collaboration you can imagine,'' Mr. Jennings said. ``It is extremely
high tech. It is something no countries can equal, and it is part of
what is known as the Five Eyes intelligence operation. It has created
the highest possible level of trusted collaboration between the five
countries.''

The Five Eyes countries are Australia, Britain, Canada, New Zealand and
the United States.

Mr. Trump visited Australia in 2011, expressing a great liking for the
country and entering the fray with China. On Twitter, he called it a
``beautiful country with terrific people who love America.'' Then he
added that Australia should ``screw'' China by raising its commodity
prices.

James Goldrick, a former rear admiral in the Australian Navy who served
in Afghanistan, said that if Mr. Trump's current abrasive attitude
continued, it would probably make Australia a less publicly cooperative
partner of the United States.

``If the Trump administration pursues a less consultative approach,
Australia will need to become more public in expressing its opinion on
American initiatives that it does not wholly support,'' Mr. Goldrick
said.

China's role in Asia and North Korea's nuclear efforts are far more
important to the United States-Australia alliance than any one telephone
call, said Peter Hayes, an Australian who is the director of the
Nautilus Institute, a California-based research institute on security
issues. But Mr. Trump's ``ability to create chaos'' has placed those
important strategic questions under a ``dark cloud,'' he said.

``The United States has invested five decades in creating that system of
alliances, and it is evident Trump is pretty ignorant of it and doesn't
care about it, to the extent he knows about it,'' Mr. Hayes said.

Advertisement

\protect\hyperlink{after-bottom}{Continue reading the main story}

\hypertarget{site-index}{%
\subsection{Site Index}\label{site-index}}

\hypertarget{site-information-navigation}{%
\subsection{Site Information
Navigation}\label{site-information-navigation}}

\begin{itemize}
\tightlist
\item
  \href{https://help.nytimes.com/hc/en-us/articles/115014792127-Copyright-notice}{©~2020~The
  New York Times Company}
\end{itemize}

\begin{itemize}
\tightlist
\item
  \href{https://www.nytco.com/}{NYTCo}
\item
  \href{https://help.nytimes.com/hc/en-us/articles/115015385887-Contact-Us}{Contact
  Us}
\item
  \href{https://www.nytco.com/careers/}{Work with us}
\item
  \href{https://nytmediakit.com/}{Advertise}
\item
  \href{http://www.tbrandstudio.com/}{T Brand Studio}
\item
  \href{https://www.nytimes.com/privacy/cookie-policy\#how-do-i-manage-trackers}{Your
  Ad Choices}
\item
  \href{https://www.nytimes.com/privacy}{Privacy}
\item
  \href{https://help.nytimes.com/hc/en-us/articles/115014893428-Terms-of-service}{Terms
  of Service}
\item
  \href{https://help.nytimes.com/hc/en-us/articles/115014893968-Terms-of-sale}{Terms
  of Sale}
\item
  \href{https://spiderbites.nytimes.com}{Site Map}
\item
  \href{https://help.nytimes.com/hc/en-us}{Help}
\item
  \href{https://www.nytimes.com/subscription?campaignId=37WXW}{Subscriptions}
\end{itemize}
