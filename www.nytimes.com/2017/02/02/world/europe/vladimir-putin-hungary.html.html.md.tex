Sections

SEARCH

\protect\hyperlink{site-content}{Skip to
content}\protect\hyperlink{site-index}{Skip to site index}

\href{https://www.nytimes.com/section/world/europe}{Europe}

\href{https://myaccount.nytimes.com/auth/login?response_type=cookie\&client_id=vi}{}

\href{https://www.nytimes.com/section/todayspaper}{Today's Paper}

\href{/section/world/europe}{Europe}\textbar{}Putin Swaggers Into
Hungary as Europe Wonders About U.S.

\url{https://nyti.ms/2jZaM8o}

\begin{itemize}
\item
\item
\item
\item
\item
\end{itemize}

Advertisement

\protect\hyperlink{after-top}{Continue reading the main story}

Supported by

\protect\hyperlink{after-sponsor}{Continue reading the main story}

\hypertarget{putin-swaggers-into-hungary-as-europe-wonders-about-us}{%
\section{Putin Swaggers Into Hungary as Europe Wonders About
U.S.}\label{putin-swaggers-into-hungary-as-europe-wonders-about-us}}

\includegraphics{https://static01.nyt.com/images/2017/02/02/world/03Putin1/02Putin1-articleInline.jpg?quality=75\&auto=webp\&disable=upscale}

By \href{https://www.nytimes.com/by/rick-lyman}{Rick Lyman}

\begin{itemize}
\item
  Feb. 2, 2017
\item
  \begin{itemize}
  \item
  \item
  \item
  \item
  \item
  \end{itemize}
\end{itemize}

BUDAPEST --- When President Vladimir V. Putin of Russia
\href{https://www.nytimes.com/2015/02/18/world/hungary-keeps-visit-by-putin-low-key-as-it-seeks-to-repair-relations-with-west.html}{last
paid a visit} to Hungary, Prime Minister Viktor Orban was under siege
for his autocratic style, Russia was isolated for its seizure of Crimea,
and both men were called xenophobes for their
\href{https://www.nytimes.com/2016/11/09/world/europe/hungary-refugee-crisis-ban.html}{hard-line
stance on immigration}.

Two years later, as Mr. Putin landed on Thursday for his first foray
into Europe in the Trump era, it was a different story. Both men feel
vindicated. There is talk of lifting the economic sanctions placed on
Russia for its land grab in Ukraine. Their brand of nationalism has
moved from the fringe to the mainstream.

There was a note of triumphalism, even a bit of swagger, in the air.

``We all sense, it's in the air, that the world is in the process of a
substantial realignment,'' Mr. Orban said in a news conference after
Thursday's meeting. ``We believe this will create favorable conditions
for stronger Russian-Hungarian relations.''

Even so, beneath the triumph lies a strain of uneasiness. The visit is
expected to be fairly low-key, an indication of the uncertainty
surrounding the new Trump administration, analysts say. President
Trump's intentions remain unclear, and the prospects of a grand bargain
between Washington and the Kremlin are highly uncertain.

In the meantime, leaders across Europe have been forced to recalculate
the best way to balance pressures in the East and West. Nowhere is that
challenge felt more keenly than in Central and Eastern Europe,
historically torn between Russia and the West.

That means European and global leaders are closely scrutinizing the
visit. They are looking for hints of how aggressive Mr. Putin and
populist leaders like Mr. Orban will be in capitalizing on this new
international climate and on Mr. Trump's stated desire for better
relations with Moscow.

If Thursday's post-meeting news conference is any indication, any hints
of aggression are well buried. Both leaders focused on economic issues,
such as Russian energy deals, and emphasized the need for international
cooperation.

``I provided information in great detail on our assessment of what is
happening in eastern Ukraine and what, in our opinion, is happening in
Syria,'' Mr. Putin said --- which, he added, underlines the need for
more global cooperation to fight terrorism.

Many here, skeptical that the Americans and Russians will actually
bridge the chasm of interests dividing them, are injecting a note of
caution about the balancing act ahead for leaders like Mr. Orban and his
governing right-wing party, Fidesz.

Andras Racz, a Russia expert and associate professor at Pazmany Peter
Catholic University in Budapest, predicted that the reset in relations
between the United States and Russia would result in ``a brief
honeymoon, but nothing else, soon overwritten by conflicting
interests.''

As for Hungary, ``there is no trust on the Russian side towards Orban,''
Mr. Racz said. The Hungarian leader has been seen mostly as a useful
tool for weakening European Union unity, he said.

And the feeling is mutual, said Balazs Orban, director of research for
the Szazadveg Foundation, a think tank that advises the Fidesz party.

``Fidesz doesn't feel chemistry with the Russians,'' he said. ``They
don't think they are friends of Hungary, necessarily.''

The warmer relations of recent years, he said, had more to do with
economic necessity and Hungary's dependence on Russian energy.

Indeed, Zoltan Kovacs, Viktor Orban's spokesman, said in an interview
that both nations would treat Mr. Putin's visit as ``business as
usual,'' with energy policy and a Russian deal to build a nuclear power
plant in Hungary at the top of the agenda.

\includegraphics{https://static01.nyt.com/images/2017/02/02/world/03Putin2/02Putin2-articleInline.jpg?quality=75\&auto=webp\&disable=upscale}

It was not clear how significant a role, if any, the thorniest issue
between Russia and the West --- the sanctions imposed by the European
Union and the United States after the seizure of Crimea --- would play
in the meeting. But Mr. Putin is clearly eager to have the sanctions
lifted, and to sow divisions in the European Union on that policy and
others.

Hungary may be among the nations most susceptible to Mr. Putin's
maneuvering to remove the sanctions. Mr. Orban has voted with other
European nations to support them, as a show of solidarity.

When Hungary's foreign minister, Peter Szijjarto, visited Moscow last
week to prepare for Mr. Putin's visit, he described the sanctions as
``counterproductive and harmful'': an indicator, some thought, of
weakening Hungarian resolve.

But since then, Mr. Trump has said that it is ``too early'' to revisit
the issue, but that he remains open to easing sanctions down the road.
In
\href{https://www.nytimes.com/2017/01/28/us/politics/trump-putin-russia-sanctions.html}{separate
phone conversations} he had last weekend with Mr. Putin and Chancellor
Angela Merkel of Germany, who strongly supports the sanctions, the
subject did not even come up.

And that pattern held on Thursday, when neither leader mentioned the
word ``sanctions'' in their public statements.

Mr. Orban, though, did allude to the sanctions, saying that some nations
in ``the western side of the Continent have shown very anti-Russian
policies,'' which have harmed the Hungarian economy ``for reasons which
are beyond us.''

Mr. Orban's hosting of Mr. Putin is the first part of a busy year of
global outreach. Efforts are underway to arrange a meeting with Mr.
Trump --- the timing and location are still under discussion --- and Mr.
Orban is also planning a visit to Beijing and a meeting with Turkey's
increasingly autocratic leader, Recep Tayyip Erdogan.

``Orban has collected some credits in the international sphere,'' said
Balazs Orban, the researcher, who is not related to the prime minister.
``He forecast everything correctly, like immigration.''

Now, seeing a potential ally in Washington to balance the one in Moscow,
the prime minister intends to cash those credits.

``He understands geopolitics is changing,'' Balazs Orban said: The
notion that all nations need to embrace globalism and ``the liberal
world order'' is no longer automatically accepted.

The Hungarian prime minister's chief opposition comes from the far-right
Jobbik Party. Its leader, Gabor Vona, said in an interview this week
that he had ``very mixed feelings about Donald J. Trump's election,''
and that he was unsure how seriously to take Mr. Trump's talk. He said
he would wait ``to see what will be unfurled.''

Russia has been accused of
\href{https://www.nytimes.com/2016/12/24/world/europe/intent-on-unsettling-eu-russia-taps-foot-soldiers-from-the-fringe.html}{backing
fringe parties} in an effort to destabilize the European Union and NATO,
but Mr. Vona denied persistent rumors that Jobbik received money from
the Kremlin, calling it government propaganda.

Nevertheless, Mr. Vona said Jobbik would welcome a grand bargain between
Mr. Trump and Mr. Putin.

``We will only be happy if relations between the U.S. and Russia
improve,'' he said. If that bargain includes the creation of new
``spheres of influence'' for Russia and the West, as Mr. Putin dearly
wishes, so much the better.

In such a world, the prime minister's spokesman, Mr. Kovacs, made clear
that Hungary would be working for more latitude to pursue its own
interests, even while staying in the European Union.

``We don't want to step out of the European Union,'' he said. ``We want
to reform it,'' turning it from a ``United States of Europe'' into an
alliance of more independent, sovereign nations whose leaders can govern
without what Mr. Kovacs characterized as undue influence from the
organization's bureaucrats in Brussels.

``At the same time, we all sense there is going to be a resetting of the
relationship with Moscow, and Hungary would like to be there,'' Mr.
Kovacs said. ``It is not a bipolar world anymore. It is a multipolar
world that is emerging.''

Advertisement

\protect\hyperlink{after-bottom}{Continue reading the main story}

\hypertarget{site-index}{%
\subsection{Site Index}\label{site-index}}

\hypertarget{site-information-navigation}{%
\subsection{Site Information
Navigation}\label{site-information-navigation}}

\begin{itemize}
\tightlist
\item
  \href{https://help.nytimes.com/hc/en-us/articles/115014792127-Copyright-notice}{©~2020~The
  New York Times Company}
\end{itemize}

\begin{itemize}
\tightlist
\item
  \href{https://www.nytco.com/}{NYTCo}
\item
  \href{https://help.nytimes.com/hc/en-us/articles/115015385887-Contact-Us}{Contact
  Us}
\item
  \href{https://www.nytco.com/careers/}{Work with us}
\item
  \href{https://nytmediakit.com/}{Advertise}
\item
  \href{http://www.tbrandstudio.com/}{T Brand Studio}
\item
  \href{https://www.nytimes.com/privacy/cookie-policy\#how-do-i-manage-trackers}{Your
  Ad Choices}
\item
  \href{https://www.nytimes.com/privacy}{Privacy}
\item
  \href{https://help.nytimes.com/hc/en-us/articles/115014893428-Terms-of-service}{Terms
  of Service}
\item
  \href{https://help.nytimes.com/hc/en-us/articles/115014893968-Terms-of-sale}{Terms
  of Sale}
\item
  \href{https://spiderbites.nytimes.com}{Site Map}
\item
  \href{https://help.nytimes.com/hc/en-us}{Help}
\item
  \href{https://www.nytimes.com/subscription?campaignId=37WXW}{Subscriptions}
\end{itemize}
