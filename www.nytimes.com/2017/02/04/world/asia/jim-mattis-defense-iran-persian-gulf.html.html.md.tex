Sections

SEARCH

\protect\hyperlink{site-content}{Skip to
content}\protect\hyperlink{site-index}{Skip to site index}

\href{https://www.nytimes.com/section/world/asia}{Asia Pacific}

\href{https://myaccount.nytimes.com/auth/login?response_type=cookie\&client_id=vi}{}

\href{https://www.nytimes.com/section/todayspaper}{Today's Paper}

\href{/section/world/asia}{Asia Pacific}\textbar{}No Extra Forces Needed
in Gulf Now, Defense Chief Says

\url{https://nyti.ms/2kyM0jh}

\begin{itemize}
\item
\item
\item
\item
\item
\end{itemize}

Advertisement

\protect\hyperlink{after-top}{Continue reading the main story}

Supported by

\protect\hyperlink{after-sponsor}{Continue reading the main story}

\hypertarget{no-extra-forces-needed-in-gulf-now-defense-chief-says}{%
\section{No Extra Forces Needed in Gulf Now, Defense Chief
Says}\label{no-extra-forces-needed-in-gulf-now-defense-chief-says}}

By \href{http://www.nytimes.com/by/michael-r-gordon}{Michael R. Gordon}
and \href{http://www.nytimes.com/by/motoko-rich}{Motoko Rich}

\begin{itemize}
\item
  Feb. 4, 2017
\item
  \begin{itemize}
  \item
  \item
  \item
  \item
  \item
  \end{itemize}
\end{itemize}

TOKYO --- Defense Secretary Jim Mattis described Iran as the world's
greatest sponsor of terrorism on Saturday, but he emphasized that there
was no pressing need for the United States to beef up its military
presence in the Persian Gulf region.

``I do not see any need to increase the number of forces we have in the
Middle East at this time,'' Mr. Mattis said, speaking in Tokyo at a news
conference as he wound up his visits to Japan and South Korea, his first
foreign trip as defense secretary.

Michael T. Flynn, President Trump's national security adviser, said this
week that the United States was
\href{https://www.nytimes.com/2017/02/01/world/middleeast/iran-missile-test.html}{putting
Iran ``on notice''} because of its recent missile test and support for
Houthi rebels in Yemen, whom the United States has accused of
threatening American vessels in the Red Sea and attacking a Saudi Navy
patrol boat.

The Trump administration imposed economic sanctions on Friday against 25
Iranians and companies that it said were connected with Iran's missile
program and the country's Islamic Revolutionary Guard Corps. But so far,
the White House has not announced military steps to strengthen its
presence in the region. No American aircraft carrier is currently
deployed in the Persian Gulf, though the Navy was expected to rotate one
into the area.

Mr. Mattis defended the decision to put a spotlight on Iran's behavior,
saying that it was important to make Iran recognize that ``it is getting
the attention of a lot of people.''

But Mr. Mattis said that the United States did not need to deploy
additional military resources to signal its concern. ``Right now, I do
not think that is necessary,'' he said.

Mr. Mattis also signaled restraint on another hot spot: the South China
Sea. Mr. Mattis said that China's territorial claim to almost all of its
waters ``has shredded the trust of nations in the region.'' But he
emphasized that he saw no need for more military maneuvers in the area.

``What we have to do is exhaust all diplomatic efforts to try to resolve
this properly,'' he said.

In a meeting on Saturday morning with Japan's defense minister, Tomomi
Inada, Mr. Mattis reiterated the United States' commitment to defend
Japan in any confrontation with China over disputed islands in the East
China Sea, known in Japan as the Senkaku and in China as the Diaoyu.

For the Japanese government, that reassurance was perhaps the most
important message of Mr. Mattis's visit, aside from his confirmation of
the United States' broader commitment to the security of its allies in
Asia. During last year's presidential campaign, Mr. Trump suggested he
might pull back from those commitments unless countries like Japan
contributed more to the cost of their defense.

\includegraphics{https://static01.nyt.com/images/2017/02/04/world/MILITARY/MILITARY-articleInline.jpg?quality=75\&auto=webp\&disable=upscale}

On Friday, after Mr. Mattis wound up a two-day visit to South Korea ---
where he sought to reassure officials that the U.S. commitment to that
country's defense against North Korea had not changed --- the defense
secretary told Prime Minister Shinzo Abe of Japan that the United States
would stand by the countries' mutual defense treaty.

``I want there to be no misunderstanding during the transition in
Washington that we stand firmly, 100 percent, shoulder to shoulder with
you and the Japanese people,'' Mr. Mattis said at the start of a meeting
with Mr. Abe.

During his meeting Saturday with Ms. Inada, Mr. Mattis did not bring up
the matter of Japan's financial commitment to its military defense,
according to briefings from both countries.

But in response to a question from a reporter at the Ministry of Defense
on Saturday, Mr. Mattis described Japan as ``a model of cost sharing and
burden sharing'' and praised the Abe administration for spending more on
the military. Under Mr. Abe, Japan has
\href{https://www.nytimes.com/2016/08/31/world/asia/japan-defense-military-budget-shinzo-abe.html}{increased
its annual defense budget five years in a row}.

Still, the foreign policy community in Japan has begun discussing
further increases in military spending, which currently stands at about
1 percent of the country's economy. ``It would not be such a bad thing
for Japan to become more self-reliant in terms of security,'' wrote the
authors of a
\href{http://www.iips.org/en/research/usjr2017en.pdf}{report} from the
Institute for International Policy Studies released in Tokyo this week.

The American security presence in Japan is most visibly represented by
its military bases across the country, with the largest number of troops
concentrated in Okinawa, a chain of islands south of the Japanese
mainland.

In his meeting with Ms. Inada, Mr. Mattis confirmed that the United
States would proceed with relocating one of its bases, the Futenma Air
Base in the south of Okinawa's main island, to a much less populated
area in Nago, also on the main island.

That announcement is likely to anger residents in Okinawa who want the
base moved off the island altogether and have long complained about
noise and violence. This week, the governor of Okinawa, Takeshi Onaga,
visited several members of Congress in Washington to lobby them to
persuade the Trump administration to withdraw the base from Okinawa.

On Saturday, Mr. Onaga blasted Mr. Mattis for sticking to the relocation
plan, calling it ``regrettable.''

As a result, Mr. Onaga said in remarks to reporters in Washington,
objections from Okinawans ``could be intensified and eventually turned
into a protest against the entire U.S. forces, and it could impact the
stability of base operations. This could cause serious problems for the
U.S.-Japan alliance.''

Advertisement

\protect\hyperlink{after-bottom}{Continue reading the main story}

\hypertarget{site-index}{%
\subsection{Site Index}\label{site-index}}

\hypertarget{site-information-navigation}{%
\subsection{Site Information
Navigation}\label{site-information-navigation}}

\begin{itemize}
\tightlist
\item
  \href{https://help.nytimes.com/hc/en-us/articles/115014792127-Copyright-notice}{©~2020~The
  New York Times Company}
\end{itemize}

\begin{itemize}
\tightlist
\item
  \href{https://www.nytco.com/}{NYTCo}
\item
  \href{https://help.nytimes.com/hc/en-us/articles/115015385887-Contact-Us}{Contact
  Us}
\item
  \href{https://www.nytco.com/careers/}{Work with us}
\item
  \href{https://nytmediakit.com/}{Advertise}
\item
  \href{http://www.tbrandstudio.com/}{T Brand Studio}
\item
  \href{https://www.nytimes.com/privacy/cookie-policy\#how-do-i-manage-trackers}{Your
  Ad Choices}
\item
  \href{https://www.nytimes.com/privacy}{Privacy}
\item
  \href{https://help.nytimes.com/hc/en-us/articles/115014893428-Terms-of-service}{Terms
  of Service}
\item
  \href{https://help.nytimes.com/hc/en-us/articles/115014893968-Terms-of-sale}{Terms
  of Sale}
\item
  \href{https://spiderbites.nytimes.com}{Site Map}
\item
  \href{https://help.nytimes.com/hc/en-us}{Help}
\item
  \href{https://www.nytimes.com/subscription?campaignId=37WXW}{Subscriptions}
\end{itemize}
