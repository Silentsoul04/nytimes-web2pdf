Sections

SEARCH

\protect\hyperlink{site-content}{Skip to
content}\protect\hyperlink{site-index}{Skip to site index}

\href{https://www.nytimes.com/section/world/middleeast}{Middle East}

\href{https://myaccount.nytimes.com/auth/login?response_type=cookie\&client_id=vi}{}

\href{https://www.nytimes.com/section/todayspaper}{Today's Paper}

\href{/section/world/middleeast}{Middle East}\textbar{}Israel Passes
Provocative Law to Retroactively Legalize Settlements

\url{https://nyti.ms/2jVHYwI}

\begin{itemize}
\item
\item
\item
\item
\item
\end{itemize}

Advertisement

\protect\hyperlink{after-top}{Continue reading the main story}

Supported by

\protect\hyperlink{after-sponsor}{Continue reading the main story}

\hypertarget{israel-passes-provocative-law-to-retroactively-legalize-settlements}{%
\section{Israel Passes Provocative Law to Retroactively Legalize
Settlements}\label{israel-passes-provocative-law-to-retroactively-legalize-settlements}}

\includegraphics{https://static01.nyt.com/images/2017/02/07/world/07israel1/07israel1-articleInline.jpg?quality=75\&auto=webp\&disable=upscale}

By \href{http://www.nytimes.com/by/ian-fisher}{Ian Fisher}

\begin{itemize}
\item
  Feb. 6, 2017
\item
  \begin{itemize}
  \item
  \item
  \item
  \item
  \item
  \end{itemize}
\end{itemize}

JERUSALEM --- Israel's Parliament passed a provocative law late Monday
that would retroactively legalize Jewish settlements on privately owned
Palestinian land, pressing ahead with a statement of right-wing
assertiveness despite the likelihood that the country's high court will
nullify the legislation.

It was a defining --- opponents said frightening --- moment in Israel's
ever-more-distant relations with Palestinians and amid fading hopes of
ending decades of conflict through a two-state solution.

While polls consistently show that most Israelis still support two
states, their leaders and the reality of what is happening on the ground
are consistently heading in the opposite direction: Fifty years after
Israel defeated Jordan and captured the West Bank and East Jerusalem,
many right-wing politicians say that now --- with negotiations with the
Palestinians frozen --- is the moment Israel must decide what it wants
and act decisively on it.

The new law is ``deteriorating Israel's democracy, making stealing an
official policy and bringing us one step closer to annexation'' of more
land Palestinians claim for a future state, said Anat Ben Nun, the
director of external relations for Peace Now, an anti-settlement group.

Only a few months ago, the law was believed to have little chance of
coming up for a vote. Even Prime Minister Benjamin Netanyahu, who was
flying back from a meeting with Britain's leaders as the law was being
debated, seemed to oppose its passage for fear of further international
censure.

The bill had been so contentious that the nation's attorney general, who
described it as unconstitutional and in contravention of international
law, said he would not defend it in the high court, which seemed in any
case likely to nullify it.

That is partly because the law applies to Palestinians and their
property rights. Since Palestinians in the occupied West Bank are not
Israeli citizens and cannot vote for candidates for Israel's Parliament,
or Knesset, critics of the legislation say it is inherently
anti-democratic. Under the law, Palestinian landowners will be offered
compensation for the long-term use of their property but will not be
able to reclaim it.

But the bill gained internal momentum through several forces: Mr.
Netanyahu is determined to show his support to the powerful settler
movement, and is under pressure from hard-liners on the right and from
corruption investigations that even his supporters say appear serious.
That pressure intensified last week after Mr. Netanyahu's government
carried out a court order to
\href{https://www.nytimes.com/2017/02/01/world/middleeast/amona-west-bank-settlers-eviction.html}{evacuate
about 40 settler families} at the Amona outpost, declared illegal a
decade ago.

``Today Israel decreed that developing settlement in Judea and Samaria
is an Israeli interest,'' said Bezalel Smotrich, a right-wing lawmaker,
using the biblical names for the West Bank. ``From here we move on to
expanding Israeli sovereignty and continuing to build and develop
settlements across the land.''

At the same time, Mr. Netanyahu and the right --- some allies, some
opponents --- have taken into account that they have more leeway under
President Trump than under President Barack Obama, who regularly
condemned settlement building.

It is uncertain, however, just how firm the support from the new
administration in Washington is: Last week, the White House issued a
statement, amid announcements here about thousands of units of housing
for settlers, saying that further expansion ``may not be helpful'' in
achieving a deal with the Palestinians, which Mr. Trump has said he
wants.

A clearer sense of how Mr. Trump differs from Mr. Obama and from nearly
50 years of American opposition to settlement building is expected to
emerge from a meeting between Mr. Trump and Mr. Netanyahu on Feb. 15 in
Washington.

The vote on Monday, which passed, 60 to 52, retroactively legalized
several thousand housing units in 16 settlements on about 2,000 acres of
Palestinian-owned land. The law provides for compensation to Palestinian
landowners.

Opponents said the law would encourage more settlements on Palestinian
land, with the expectation that they, too, would be legalized.

``Looting is illegal,'' Saeb Erekat, the Palestinians' chief negotiator,
said in a statement after the vote. ``The Israeli settlement enterprise
negates peace and the possibility of the two-state solution.''

Yair Lapid, the opposition politician seeking to succeed Mr. Netanyahu,
said before the vote: ``It's unjust, it's not smart, and it's a law
which damages the state of Israel, the security of Israel, governance in
Israel and our ability to fight back against those who hate Israel.''

He added, ``They are passing a law which endangers our soldiers, will
undermine our international standing and undermine us as a country of
law and order.''

Israel's settlement activity has come under intense international
criticism. In December, the United Nations --- with the tacit support of
the outgoing Obama administration ---
\href{https://www.nytimes.com/2016/12/23/world/middleeast/israel-settlements-un-vote.html}{condemned
Israeli settlements} in the occupied West Bank and East Jerusalem as an
impediment to a two-state solution. Settlers and right-wing Israelis say
the West Bank and East Jerusalem, captured from Jordan in the
Arab-Israeli War of 1967, belong to the Jewish people.

The international significance of the vote on Monday was underscored
during Mr. Netanyahu's quick trip to visit Prime Minister Theresa May of
Britain. On one hand, she noted that her first meeting with Mr.
Netanyahu came 100 years after the Balfour Declaration, in which the
British government governing the area supported the creation of Jewish
state. She said, however, that Britain remained ``committed to a
two-state solution,'' adding, ``It's the best way of building stability,
peace and prosperity in the future.''

Appearing before reporters with Mrs. May in London, Mr. Netanyahu, who
has in the past tepidly supported a two-state solution, did not do so on
Monday.

As voting neared, tensions rose in the divided Knesset. ``You are only
passing this law so that the Supreme Court will later overturn it, and
then you'll be in the position to blame the judges,'' Revital Swid, a
member of the Zionist Union Party, told the governing Likud Party's
science minister, Ofir Akunis.

``The land of Israel is ours, and this cannot be disputed or be
divided,'' Mr. Akunis responded. ``The concept of settlement blocs is no
longer relevant because there are no Arabs to negotiate with anymore.''

The vote came on the same day as a rocket fired from Gaza landed near
the Israeli city of Ashkelon. No one was hurt. The Israeli military
responded with artillery fire and airstrikes in northern Gaza. It was
unclear if the rocket attack was related to Monday's vote.

Advertisement

\protect\hyperlink{after-bottom}{Continue reading the main story}

\hypertarget{site-index}{%
\subsection{Site Index}\label{site-index}}

\hypertarget{site-information-navigation}{%
\subsection{Site Information
Navigation}\label{site-information-navigation}}

\begin{itemize}
\tightlist
\item
  \href{https://help.nytimes.com/hc/en-us/articles/115014792127-Copyright-notice}{©~2020~The
  New York Times Company}
\end{itemize}

\begin{itemize}
\tightlist
\item
  \href{https://www.nytco.com/}{NYTCo}
\item
  \href{https://help.nytimes.com/hc/en-us/articles/115015385887-Contact-Us}{Contact
  Us}
\item
  \href{https://www.nytco.com/careers/}{Work with us}
\item
  \href{https://nytmediakit.com/}{Advertise}
\item
  \href{http://www.tbrandstudio.com/}{T Brand Studio}
\item
  \href{https://www.nytimes.com/privacy/cookie-policy\#how-do-i-manage-trackers}{Your
  Ad Choices}
\item
  \href{https://www.nytimes.com/privacy}{Privacy}
\item
  \href{https://help.nytimes.com/hc/en-us/articles/115014893428-Terms-of-service}{Terms
  of Service}
\item
  \href{https://help.nytimes.com/hc/en-us/articles/115014893968-Terms-of-sale}{Terms
  of Sale}
\item
  \href{https://spiderbites.nytimes.com}{Site Map}
\item
  \href{https://help.nytimes.com/hc/en-us}{Help}
\item
  \href{https://www.nytimes.com/subscription?campaignId=37WXW}{Subscriptions}
\end{itemize}
