Sections

SEARCH

\protect\hyperlink{site-content}{Skip to
content}\protect\hyperlink{site-index}{Skip to site index}

\href{https://www.nytimes.com/section/politics}{Politics}

\href{https://myaccount.nytimes.com/auth/login?response_type=cookie\&client_id=vi}{}

\href{https://www.nytimes.com/section/todayspaper}{Today's Paper}

\href{/section/politics}{Politics}\textbar{}Republican Senators Vote to
Formally Silence Elizabeth Warren

\url{https://nyti.ms/2k0Qdrp}

\begin{itemize}
\item
\item
\item
\item
\item
\item
\end{itemize}

Advertisement

\protect\hyperlink{after-top}{Continue reading the main story}

Supported by

\protect\hyperlink{after-sponsor}{Continue reading the main story}

\hypertarget{republican-senators-vote-to-formally-silence-elizabeth-warren}{%
\section{Republican Senators Vote to Formally Silence Elizabeth
Warren}\label{republican-senators-vote-to-formally-silence-elizabeth-warren}}

\includegraphics{https://static01.nyt.com/images/2017/02/07/us/08warren-mobile/08warren-mobile-videoSixteenByNine3000-v3.jpg}

By \href{http://www.nytimes.com/by/matt-flegenheimer}{Matt Flegenheimer}

\begin{itemize}
\item
  Feb. 7, 2017
\item
  \begin{itemize}
  \item
  \item
  \item
  \item
  \item
  \item
  \end{itemize}
\end{itemize}

WASHINGTON --- Republican senators voted on Tuesday to formally silence
a Democratic colleague for impugning a peer, Senator Jeff Sessions of
Alabama, by condemning his nomination for attorney general while
\href{https://www.documentcloud.org/documents/3259988-Scott-King-1986-Letter-and-Testimony-Signed.html\#document/p1}{reading
a letter from Coretta Scott King}.

Senator Elizabeth Warren, Democrat of Massachusetts, had been holding
forth on the Senate floor on the eve of Mr. Sessions's expected
confirmation vote, reciting a 1986 letter from Mrs. King that criticized
Mr. Sessions's record on civil rights.

Sensing a stirring beside her a short while later, Ms. Warren stopped
herself and scanned the chamber.

Across the room, Senator Mitch McConnell, the majority leader, had
stepped forward with an objection, setting off an extraordinary
confrontation in the Capitol and silencing a colleague, procedurally, in
the throes of a contentious debate over President Trump's cabinet
nominee.

``The senator has impugned the motives and conduct of our colleague from
Alabama, as warned by the chair,'' Mr. McConnell began, alluding to Mrs.
King's letter, which accused Mr. Sessions of using ``the awesome power
of his office to chill the free exercise of the vote by black
citizens.''

Mr. McConnell called the Senate to order under what is known as
\href{http://www.rules.senate.gov/public/index.cfm?p=RuleXIX}{Rule XIX},
which prohibits debating senators from ascribing ``to another senator or
to other senators any conduct or motive unworthy or unbecoming a
senator.''

When Mr. McConnell concluded, Ms. Warren said she was ``surprised that
the words of
\href{http://www.nytimes.com/2006/01/31/national/coretta-scott-king-78-widow-of-dr-martin-luther-king-jr-dies.html}{Coretta
Scott King} are not suitable for debate in the United States Senate.''
She asked to continue her remarks.

Mr. McConnell objected.

``Objection is heard,'' said Senator Steve Daines, Republican of
Montana, who was presiding in the chamber at the time. ``The senator
will take her seat.''

The debate appeared to center, in part, on whether the rule allowed
exemptions for quoted remarks --- Ms. Warren had been reading directly
from the letter from Mrs. King, the widow of the Rev. Dr. Martin Luther
King Jr. --- to demean a sitting senator.

In a party-line vote, 49 to 43, senators upheld Mr. Daines's decision,
forcing Ms. Warren into silence, at least on the Senate floor, until the
showdown over Mr. Sessions's nomination is complete. He is expected to
be confirmed on Wednesday.

Immediately, Democrats took up Ms. Warren's cause, urging on social
media for Republicans to ``\#LetLizSpeak.'' Ms. Warren
\href{https://twitter.com/SenWarren/status/829140554109820928}{said on
Twitter} that Mr. McConnell had ``silenced Mrs. King's voice'' on the
Senate floor, to say nothing of ``millions who are afraid \& appalled by
what's happening in our country.'' Within hours of being shut down on
the Senate floor, Ms. Warren read the letter from Mrs. King on
\href{https://www.facebook.com/senatorelizabethwarren/videos/vb.131559043673264/724337794395383/?type=2\&theater\&notif_t=live_video_interaction\&notif_id=1486526408091711}{Facebook},
attracting more than two million views --- an audience she would have
been unlikely to match on C-Span, if she had been permitted to continue
speaking in the chamber.

Democrats argued that Mr. McConnell was enforcing the rule selectively,
citing examples of Republicans appearing to test the boundaries of Rule
XIX. In one instance from 2015, Senator Ted Cruz of Texas accused Mr.
McConnell of lying ``over and over and over again.'' In another, last
year, Senator Tom Cotton of Arkansas described the ``cancerous
leadership'' of Senator Harry Reid, the former Democratic leader.

Republicans accused Ms. Warren of violating the rule repeatedly, saying
she had been warned before Mr. McConnell's objection. Senator John
Cornyn, Republican of Texas, suggested that Ms. Warren had been rebuked
over ``a quotation from Senator Ted Kennedy that called the nominee a
disgrace to the Justice Department.''

``Our colleagues want to try to make this all about Coretta Scott King,
and it is not,'' he said.

But when Senator Chuck Schumer, the Democratic leader, sought
clarification, he was informed that while a warning was issued over the
letter from Mr. Kennedy, the ruling itself hinged on Mrs. King's letter.
That judgment came from Senator Mike Rounds, Republican of South Dakota,
who had taken over as the presiding officer.

In either event, Republicans suggested, the episode spoke to Democrats'
inability to accept the results of the 2016 election --- and, more
narrowly, to adhere to the rules of a body where decorum has often
fallen away.

``She was warned,'' Mr. McConnell said of Ms. Warren. ``She was given an
explanation. Nevertheless, she persisted.''

Democrats planned to hold the floor into the wee hours of Wednesday to
protest Mr. Sessions's nomination.

Advertisement

\protect\hyperlink{after-bottom}{Continue reading the main story}

\hypertarget{site-index}{%
\subsection{Site Index}\label{site-index}}

\hypertarget{site-information-navigation}{%
\subsection{Site Information
Navigation}\label{site-information-navigation}}

\begin{itemize}
\tightlist
\item
  \href{https://help.nytimes.com/hc/en-us/articles/115014792127-Copyright-notice}{©~2020~The
  New York Times Company}
\end{itemize}

\begin{itemize}
\tightlist
\item
  \href{https://www.nytco.com/}{NYTCo}
\item
  \href{https://help.nytimes.com/hc/en-us/articles/115015385887-Contact-Us}{Contact
  Us}
\item
  \href{https://www.nytco.com/careers/}{Work with us}
\item
  \href{https://nytmediakit.com/}{Advertise}
\item
  \href{http://www.tbrandstudio.com/}{T Brand Studio}
\item
  \href{https://www.nytimes.com/privacy/cookie-policy\#how-do-i-manage-trackers}{Your
  Ad Choices}
\item
  \href{https://www.nytimes.com/privacy}{Privacy}
\item
  \href{https://help.nytimes.com/hc/en-us/articles/115014893428-Terms-of-service}{Terms
  of Service}
\item
  \href{https://help.nytimes.com/hc/en-us/articles/115014893968-Terms-of-sale}{Terms
  of Sale}
\item
  \href{https://spiderbites.nytimes.com}{Site Map}
\item
  \href{https://help.nytimes.com/hc/en-us}{Help}
\item
  \href{https://www.nytimes.com/subscription?campaignId=37WXW}{Subscriptions}
\end{itemize}
