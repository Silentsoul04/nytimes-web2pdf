Sections

SEARCH

\protect\hyperlink{site-content}{Skip to
content}\protect\hyperlink{site-index}{Skip to site index}

\href{https://www.nytimes.com/section/us}{U.S.}

\href{https://myaccount.nytimes.com/auth/login?response_type=cookie\&client_id=vi}{}

\href{https://www.nytimes.com/section/todayspaper}{Today's Paper}

\href{/section/us}{U.S.}\textbar{}Flynn Is Said to Have Talked to
Russians About Sanctions Before Trump Took Office

\url{https://nyti.ms/2k9Vzks}

\begin{itemize}
\item
\item
\item
\item
\item
\end{itemize}

Advertisement

\protect\hyperlink{after-top}{Continue reading the main story}

Supported by

\protect\hyperlink{after-sponsor}{Continue reading the main story}

\hypertarget{flynn-is-said-to-have-talked-to-russians-about-sanctions-before-trump-took-office}{%
\section{Flynn Is Said to Have Talked to Russians About Sanctions Before
Trump Took
Office}\label{flynn-is-said-to-have-talked-to-russians-about-sanctions-before-trump-took-office}}

\includegraphics{https://static01.nyt.com/images/2017/02/10/world/10FLYNN/10FLYNN-articleLarge.jpg?quality=75\&auto=webp\&disable=upscale}

By \href{http://www.nytimes.com/by/matthew-rosenberg}{Matthew Rosenberg}
and \href{http://www.nytimes.com/by/matt-apuzzo}{Matt Apuzzo}

\begin{itemize}
\item
  Feb. 9, 2017
\item
  \begin{itemize}
  \item
  \item
  \item
  \item
  \item
  \end{itemize}
\end{itemize}

WASHINGTON --- Weeks before President Trump's inauguration, his national
security adviser, Michael T. Flynn, discussed American sanctions against
Russia, as well as areas of possible cooperation, with that country's
ambassador to the United States, according to current and former
American officials.

Throughout the discussions, the message Mr. Flynn conveyed to the
ambassador, Sergey I. Kislyak --- that the Obama administration was
Moscow's adversary and that relations with Russia would change under Mr.
Trump --- was unambiguous and highly inappropriate, the officials said.

The accounts of the conversations raise the prospect that Mr. Flynn
violated a law against private citizens' engaging in diplomacy, and
directly contradict statements made by Trump advisers. They have said
that Mr. Flynn spoke to Mr. Kislyak a few days after Christmas merely to
arrange a phone call between President Vladimir V. Putin of Russia and
Mr. Trump after the inauguration.

But current and former American officials said that conversation ---
which took place the day before the Obama administration imposed
sanctions on Russia over accusations that it used cyberattacks to help
sway the election in Mr. Trump's favor --- ranged far beyond the
logistics of a post-inauguration phone call. And they said it was only
one in a series of contacts between the two men that began before the
election and also included talk of cooperating in the fight against the
Islamic State, along with other issues.

The officials said that Mr. Flynn had never made explicit promises of
sanctions relief, but that he had appeared to leave the impression it
would be possible.

Mr. Flynn could not immediately be reached for comment about the
conversations, details of which were first reported by The Washington
Post. Despite Mr. Flynn's earlier denials, his spokesman told the Post
that ``while he had no recollection of discussing sanctions, he couldn't
be certain that the topic never came up.''

During the Christmas week conversation, he urged Mr. Kislyak to keep the
Russian government from retaliating over the coming sanctions --- it was
an open secret in Washington that they were in the works --- by telling
him that whatever the Obama administration did could be undone, said the
officials, who spoke on the condition of anonymity because they were
discussing classified material.

\includegraphics{https://static01.nyt.com/images/2017/01/19/us/19nsc1/19nsc1-videoSixteenByNine3000-v3.jpg}

Days before Mr. Trump's inauguration, Vice President-elect Mike Pence
also denied that Mr. Flynn had discussed sanctions with Mr. Kislyak. He
said he had personally spoken to Mr. Flynn, who assured him that the
conversation was an informal chat that began with Mr. Flynn extending
Christmas wishes.

``They did not discuss anything having to do with the United States'
decision to expel diplomats or impose censure against Russia,'' Mr.
Pence said on the CBS News program ``Face the Nation.''

Some officials regarded the conversation as a potential violation of the
Logan Act, which prohibits private citizens from negotiating with
foreign governments in disputes involving the American government,
according to one current and one former American official familiar with
the case.

Federal officials who have read the transcript of the call were
surprised by Mr. Flynn's comments, since he would have known that
American eavesdroppers closely monitor such calls. They were even more
surprised that Mr. Trump's team publicly denied that the topics of
conversation included sanctions.

The call is the latest example of how Mr. Trump's advisers have come
under scrutiny from American counterintelligence officials. The F.B.I.
is also investigating Mr. Trump's former campaign chairman, Paul
Manafort; Carter Page, a businessman and former foreign policy adviser
to the campaign; and Roger Stone, a longtime Republican operative.

Prosecutions in these types of cases are rare, and the law is murky,
particularly around people involved in presidential transitions. The
officials who had read the transcripts acknowledged that while the
conversation warranted investigation, it was unlikely, by itself, to
lead to charges against a sitting national security adviser.

But, at the very least, openly engaging in policy discussions with a
foreign government during a presidential transition is a remarkable
breach of protocol. The norm has been for the president-elect's team to
respect the sitting president, and to limit discussions with foreign
governments to pleasantries. Any policy discussions, even with allies,
would ordinarily be kept as vague as possible.

``It's largely shunned, period. But one cannot rule it out with an ally
like the U.K.,'' said Derek Chollet, who was part of the Obama
transition in 2008 and then served in senior roles at the State
Department, White House and Pentagon.

``But it's way out of bounds when the said country is an adversary, and
one that has been judged to have meddled in the election,'' he added.
``It's just hard to imagine anyone having a substantive discussion with
an adversary, particularly if it's about trying to be reassuring.''

Advertisement

\protect\hyperlink{after-bottom}{Continue reading the main story}

\hypertarget{site-index}{%
\subsection{Site Index}\label{site-index}}

\hypertarget{site-information-navigation}{%
\subsection{Site Information
Navigation}\label{site-information-navigation}}

\begin{itemize}
\tightlist
\item
  \href{https://help.nytimes.com/hc/en-us/articles/115014792127-Copyright-notice}{©~2020~The
  New York Times Company}
\end{itemize}

\begin{itemize}
\tightlist
\item
  \href{https://www.nytco.com/}{NYTCo}
\item
  \href{https://help.nytimes.com/hc/en-us/articles/115015385887-Contact-Us}{Contact
  Us}
\item
  \href{https://www.nytco.com/careers/}{Work with us}
\item
  \href{https://nytmediakit.com/}{Advertise}
\item
  \href{http://www.tbrandstudio.com/}{T Brand Studio}
\item
  \href{https://www.nytimes.com/privacy/cookie-policy\#how-do-i-manage-trackers}{Your
  Ad Choices}
\item
  \href{https://www.nytimes.com/privacy}{Privacy}
\item
  \href{https://help.nytimes.com/hc/en-us/articles/115014893428-Terms-of-service}{Terms
  of Service}
\item
  \href{https://help.nytimes.com/hc/en-us/articles/115014893968-Terms-of-sale}{Terms
  of Sale}
\item
  \href{https://spiderbites.nytimes.com}{Site Map}
\item
  \href{https://help.nytimes.com/hc/en-us}{Help}
\item
  \href{https://www.nytimes.com/subscription?campaignId=37WXW}{Subscriptions}
\end{itemize}
