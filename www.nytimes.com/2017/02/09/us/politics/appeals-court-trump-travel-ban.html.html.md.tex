Sections

SEARCH

\protect\hyperlink{site-content}{Skip to
content}\protect\hyperlink{site-index}{Skip to site index}

\href{https://www.nytimes.com/section/politics}{Politics}

\href{https://myaccount.nytimes.com/auth/login?response_type=cookie\&client_id=vi}{}

\href{https://www.nytimes.com/section/todayspaper}{Today's Paper}

\href{/section/politics}{Politics}\textbar{}Court Refuses to Reinstate
Travel Ban, Dealing Trump Another Legal Loss

\url{https://nyti.ms/2kUaROJ}

\begin{itemize}
\item
\item
\item
\item
\item
\item
\end{itemize}

Advertisement

\protect\hyperlink{after-top}{Continue reading the main story}

Supported by

\protect\hyperlink{after-sponsor}{Continue reading the main story}

\hypertarget{court-refuses-to-reinstate-travel-ban-dealing-trump-another-legal-loss}{%
\section{Court Refuses to Reinstate Travel Ban, Dealing Trump Another
Legal
Loss}\label{court-refuses-to-reinstate-travel-ban-dealing-trump-another-legal-loss}}

\includegraphics{https://static01.nyt.com/images/2017/02/10/us/10courts/10courts-articleInline.jpg?quality=75\&auto=webp\&disable=upscale}

By \href{http://www.nytimes.com/by/adam-liptak}{Adam Liptak}

\begin{itemize}
\item
  Feb. 9, 2017
\item
  \begin{itemize}
  \item
  \item
  \item
  \item
  \item
  \item
  \end{itemize}
\end{itemize}

WASHINGTON --- A federal appeals panel on Thursday unanimously rejected
President Trump's bid to reinstate his ban on travel into the United
States from seven largely Muslim nations, a sweeping rebuke of the
administration's claim that the courts have no role as a check on the
president.

The three-judge panel, suggesting that the ban did not advance national
security, said the administration had shown ``no evidence'' that anyone
from the seven nations --- Iran, Iraq, Libya, Somalia, Sudan, Syria and
Yemen --- had committed terrorist acts in the United States.

The ruling also rejected Mr. Trump's claim that courts are powerless to
review a president's national security assessments. Judges have a
crucial role to play in a constitutional democracy, the court said.

``It is beyond question,'' the decision said, ``that the federal
judiciary retains the authority to adjudicate constitutional challenges
to executive action.''

\includegraphics{https://static01.nyt.com/images/2017/02/06/us/06travel-web/06travel-web-videoSixteenByNine3000.jpg}

The decision was handed down by the United States Court of Appeals for
the Ninth Circuit, in San Francisco. It upheld a ruling last Friday by a
federal district judge, James L. Robart, who
\href{http://cdn.ca9.uscourts.gov/datastore/general/2017/02/03/17-141_TRO_order.pdf}{blocked
key parts} of the travel ban, allowing thousands of foreigners to enter
the country.

The appeals court acknowledged that Mr. Trump was owed deference on his
immigration and national security policies. But it said he was claiming
something more --- that ``national security concerns are
\emph{unreviewable}, even if those actions potentially contravene
constitutional rights and protections.''

Within minutes of the ruling, Mr. Trump angrily vowed to fight it,
presumably in an appeal to the Supreme Court.

``SEE YOU IN COURT, THE SECURITY OF OUR NATION IS AT STAKE!'' Mr. Trump
\href{https://twitter.com/realDonaldTrump/status/829836231802515457}{wrote
on Twitter}.

At the White House, the president told reporters that the ruling was ``a
political decision'' and predicted that his administration would win an
appeal ``in my opinion, very easily.'' He said he had not yet conferred
with his attorney general, Jeff Sessions, on the matter.

The Supreme Court remains short-handed and could deadlock. A 4-to-4 tie
there would leave the appeals court's ruling in place. The
administration has moved fast in the case so far, and it is likely to
file an emergency application to the Supreme Court in a day or two. The
court typically asks for a prompt response from the other side, and it
could rule soon after it received one. A decision next week, either to
reinstate the ban or to continue to block it, is possible.

\href{https://www.nytimes.com/interactive/2017/02/09/us/document-Ninth-Circuit-s-Decision-on-Trump-s-Travel-Ban.html}{}

\includegraphics{https://static01.nyt.com/images/2017/02/09/us/image-Ninth-Circuit-s-Decision-on-Trump-s-Travel-Ban/image-Ninth-Circuit-s-Decision-on-Trump-s-Travel-Ban-largeHorizontalJumbo-v2.gif}

\hypertarget{ninth-circuits-decision-on-trumps-travel-ban}{%
\subsection{Ninth Circuit's Decision on Trump's Travel
Ban}\label{ninth-circuits-decision-on-trumps-travel-ban}}

Read the text of the Ninth Circuit Court of Appeals refusal to reinstate
President Trump's travel ban.

The travel ban, one of the first executive orders Mr. Trump issued after
taking office, suspended worldwide refugee entry into the United States.
It also barred visitors from seven Muslim-majority nations for up to 90
days to give federal security agencies time to impose stricter vetting
processes.

Immediately after it was issued, the ban spurred chaos at airports and
protests nationwide as foreign travelers found themselves stranded at
immigration checkpoints by a policy that critics derided as un-American.
The State Department said up to 60,000 foreigners' visas were canceled
in the days immediately after the ban was imposed.

The World Relief Corporation, one of the agencies that resettles
refugees in the United States, called the ruling ``fabulous news'' for
275 newcomers who are scheduled to arrive in the next week, many of whom
will be reunited with family.

``We have families that have been separated for years by terror, war and
persecution,'' said Scott Arbeiter, the president of the organization,
which will arrange for housing and jobs for the refugees in cities
including Seattle; Spokane, Wash.; and Sacramento.

``Some family members had already been vetted and cleared and were
standing with tickets, and were then told they couldn't travel,'' Mr.
Arbeiter said. ``So the hope of reunification was crushed, and now they
will be admitted.''

\includegraphics{https://static01.nyt.com/images/2017/02/10/us/politics/10travel-ban-ruling-video/10travel-ban-ruling-video-videoSixteenByNineJumbo1600.jpg}

Several Democrats said they hoped the appeals court ruling would cow Mr.
Trump into rescinding the ban. Representative Karen Bass, Democrat of
California, said in a statement that the ban ``is rooted in bigotry and,
most importantly, it's illegal.''

``We will not stop,'' Ms. Bass said.

But some Republicans cast aspersions on the Ninth Circuit's decision and
predicted that it would not withstand a challenge in the Supreme Court.

``Courts ought not second-guess sensitive national security decisions of
the president,'' Senator Tom Cotton, Republican of Arkansas, said in a
statement.

``This misguided ruling is from the Ninth Circuit, the most notoriously
left-wing court in America, and the most-reversed court at the Supreme
Court,'' he said. ``I'm confident the administration's position will
ultimately prevail.''

Trial judges nationwide have blocked aspects of
\href{https://www.nytimes.com/2017/01/27/us/politics/refugee-muslim-executive-order-trump.html}{Mr.
Trump's executive order}, but no other case has yet reached an appeals
court. The case in front of Judge Robart, in Seattle, was filed by the
states of Washington and Minnesota and is still at an early stage. The
appeals court order issued Thursday ruled only on the narrow question of
whether to stay a lower court's temporary restraining order blocking the
travel ban.

The appeals court said the government had not justified suspending
travel from the seven countries. ``The government has pointed to no
evidence,'' the decision said, ``that any alien from any of the
countries named in the order has perpetrated a terrorist attack in the
United States.''

The three members of the panel were Judge Michelle T. Friedland,
appointed by President Barack Obama; Judge William C. Canby Jr.,
appointed by President Jimmy Carter; and Judge Richard R. Clifton,
appointed by President George W. Bush.

They said the states were likely to succeed at the end of the day
because Mr. Trump's order appeared to violate the due process rights of
lawful permanent residents, people holding visas and refugees.

The court said the administration's legal position in the case had been
a moving target. It noted that Donald F. McGahn II, the White House
counsel, had issued ``authoritative guidance'' several days after the
executive order came out, saying it did not apply to lawful permanent
residents. But the court said that ``we cannot rely'' on that statement.

``The White House counsel is not the president,'' the decision said,
``and he is not known to be in the chain of command for any of the
executive departments.`` It also mentioned ``the government's shifting
interpretations'' of the executive order.

\includegraphics{https://static01.nyt.com/images/2017/02/10/us/10courts_web1/10courts_web1-articleLarge.jpg?quality=75\&auto=webp\&disable=upscale}

In its briefs and in the arguments before the panel on Tuesday, the
Justice Department's position evolved. As the case progressed, the
administration offered a backup plea for at least a partial victory.

At most, a Justice Department brief said, ``previously admitted aliens
who are temporarily abroad now or who wish to travel and return to the
United States in the future'' should be allowed to enter the country
despite the ban.

The appeals court ultimately rejected that request, however, saying that
people in the United States without authorization have due process
rights, as do citizens with relatives who wish to travel to the United
States.

The court discussed, but did not decide, whether the executive order
violated the First Amendment's ban on government establishment of
religion by disfavoring Muslims.

It noted that the states challenging the executive order ``have offered
evidence of numerous statements by the president about his intent to
implement a `Muslim ban.''' And it said, rejecting another
administration argument, that it was free to consider evidence about the
motivation behind laws that draw seemingly neutral distinctions.

\href{https://www.nytimes.com/interactive/2017/01/31/us/politics/trump-immigration-ban-groups.html}{}

\includegraphics{https://static01.nyt.com/images/2017/01/30/us/trump-immigration-ban-groups-1485813876433/trump-immigration-ban-groups-1485813876433-largeHorizontalJumbo-v5.png}

\hypertarget{trumps-immigration-ban-who-is-barred-and-who-is-not}{%
\subsection{Trump's Immigration Ban: Who Is Barred and Who Is
Not}\label{trumps-immigration-ban-who-is-barred-and-who-is-not}}

A wide array of people are affected by President Trump's order.

But the court said it would defer a decision on the question of
religious discrimination.

``The political branches are far better equipped to make appropriate
distinctions,'' the decision said. ``For now, it is enough for us to
conclude that the government has failed to establish that it will likely
succeed on its due process argument in this appeal.''

The court also acknowledged ``the massive attention this case has
garnered at even the most preliminary stages.''

``On the one hand, the public has a powerful interest in national
security and in the ability of an elected president to enact policies,''
the decision said. ``And on the other, the public also has an interest
in free flow of travel, in avoiding separation of families, and in
freedom from discrimination.''

``These competing public interests,'' the court said, ``do not justify a
stay.''

The court ruling did not affect one part of the executive order: the cap
of 50,000 refugees to be admitted in the 2017 fiscal year. That is down
from the 110,000 ceiling put in place under President Barack Obama. The
order also directed the secretary of state and the secretary of homeland
security to prioritize refugee claims made by persecuted members of
religious minorities.

As of Thursday, that means the United States will be allowed to accept
only about 16,000 more refugees this fiscal year. Since Oct. 1, the
start of the fiscal year, 33,929 refugees have been admitted, 5,179 of
them Syrians.

Advertisement

\protect\hyperlink{after-bottom}{Continue reading the main story}

\hypertarget{site-index}{%
\subsection{Site Index}\label{site-index}}

\hypertarget{site-information-navigation}{%
\subsection{Site Information
Navigation}\label{site-information-navigation}}

\begin{itemize}
\tightlist
\item
  \href{https://help.nytimes.com/hc/en-us/articles/115014792127-Copyright-notice}{©~2020~The
  New York Times Company}
\end{itemize}

\begin{itemize}
\tightlist
\item
  \href{https://www.nytco.com/}{NYTCo}
\item
  \href{https://help.nytimes.com/hc/en-us/articles/115015385887-Contact-Us}{Contact
  Us}
\item
  \href{https://www.nytco.com/careers/}{Work with us}
\item
  \href{https://nytmediakit.com/}{Advertise}
\item
  \href{http://www.tbrandstudio.com/}{T Brand Studio}
\item
  \href{https://www.nytimes.com/privacy/cookie-policy\#how-do-i-manage-trackers}{Your
  Ad Choices}
\item
  \href{https://www.nytimes.com/privacy}{Privacy}
\item
  \href{https://help.nytimes.com/hc/en-us/articles/115014893428-Terms-of-service}{Terms
  of Service}
\item
  \href{https://help.nytimes.com/hc/en-us/articles/115014893968-Terms-of-sale}{Terms
  of Sale}
\item
  \href{https://spiderbites.nytimes.com}{Site Map}
\item
  \href{https://help.nytimes.com/hc/en-us}{Help}
\item
  \href{https://www.nytimes.com/subscription?campaignId=37WXW}{Subscriptions}
\end{itemize}
