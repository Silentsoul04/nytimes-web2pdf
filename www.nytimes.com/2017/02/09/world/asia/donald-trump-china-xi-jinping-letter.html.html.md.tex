Sections

SEARCH

\protect\hyperlink{site-content}{Skip to
content}\protect\hyperlink{site-index}{Skip to site index}

\href{https://www.nytimes.com/section/world/asia}{Asia Pacific}

\href{https://myaccount.nytimes.com/auth/login?response_type=cookie\&client_id=vi}{}

\href{https://www.nytimes.com/section/todayspaper}{Today's Paper}

\href{/section/world/asia}{Asia Pacific}\textbar{}Trump Tells Xi Jinping
U.S. Will Honor `One China' Policy

\url{https://nyti.ms/2kRYq5F}

\begin{itemize}
\item
\item
\item
\item
\item
\item
\end{itemize}

Advertisement

\protect\hyperlink{after-top}{Continue reading the main story}

Supported by

\protect\hyperlink{after-sponsor}{Continue reading the main story}

\hypertarget{trump-tells-xi-jinping-us-will-honor-one-china-policy}{%
\section{Trump Tells Xi Jinping U.S. Will Honor `One China'
Policy}\label{trump-tells-xi-jinping-us-will-honor-one-china-policy}}

\includegraphics{https://static01.nyt.com/images/2017/02/10/world/10chinatrump/10chinatrump-articleLarge.jpg?quality=75\&auto=webp\&disable=upscale}

By \href{http://www.nytimes.com/by/mark-landler}{Mark Landler} and
\href{http://www.nytimes.com/by/michael-forsythe}{Michael Forsythe}

\begin{itemize}
\item
  Feb. 9, 2017
\item
  \begin{itemize}
  \item
  \item
  \item
  \item
  \item
  \item
  \end{itemize}
\end{itemize}

WASHINGTON --- President Trump told President Xi Jinping of China on
Thursday evening that the United States would honor the ``One China''
policy, reversing his earlier expressions of doubt about the longtime
diplomatic understanding and removing a major source of tension between
the United States and China since shortly after he was elected.

In a statement, the White House said Mr. Trump and Mr. Xi ``discussed
numerous topics, and President Trump agreed, at the request of President
Xi, to honor our One China policy.'' It described the call as
``extremely cordial'' and said the leaders had invited each other to
visit.

The concession was clearly designed to put an end to an extended chill
in the relationship between China and the United States. Mr. Xi, stung
by Mr. Trump's
\href{https://www.nytimes.com/2016/12/02/us/politics/trump-speaks-with-taiwans-leader-a-possible-affront-to-china.html}{unorthodox
telephone call} with the president of Taiwan in December and his
subsequent assertion that the United States might no longer abide by the
One China policy, had not spoken to Mr. Trump since Nov. 14, the week
after he was elected.

Administration officials concluded that Mr. Xi would take a call only if
Mr. Trump publicly committed to upholding the 44-year-old policy, under
which the United States recognized a single Chinese government in
Beijing and severed its diplomatic ties with Taiwan.

Given the domestic political stakes of this issue for Mr. Xi, the fact
that both sides went ahead with a call -- and that the White House
statement afterward acknowledged Mr. Trump's acquiescence -- suggested
that the agreement on ``one China'' had been worked out beforehand.

The Chinese state news media, in its readout of the call, said Mr. Trump
had ``stressed that he fully understood the great importance for the
U.S. government to respect the One China policy,'' and that ``the U.S.
government adheres to the One China policy.''

It also said the two leaders had agreed on the ``necessity and urgency
of strengthening cooperation between China and the United States'' and
noted that Beijing wants to work with Washington on a range of issues,
including the economy and trade, science, energy, communications and
global stability.

The timing of the conversation was significant, as Mr. Trump is about to
welcome Japan's prime minister, Shinzo Abe, for an extravagant three-day
visit that will include a weekend of golf in Florida --- a visit that
will be closely monitored in China.

Among the issues Mr. Trump is expected to discuss with Mr. Abe, is the
president's commitment to a mutual defense treaty with Japan, which
surfaced during the campaign. At the time, Mr. Trump said he was
prepared to pull back from the pact unless Tokyo did more to reimburse
the United States for defending Japanese territory.

On Thursday, Secretary of State Rex W. Tillerson met with officials at
the White House to discuss issuing a statement about relations with
China. His involvement was noteworthy because he had pledged, in written
answers to questions after his Senate confirmation hearing, to uphold
the One China policy.

Mr. Tillerson specifically rejected the idea, advanced by Mr. Trump,
that Taiwan
\href{https://www.nytimes.com/2016/12/11/us/politics/trump-taiwan-one-china.html}{be
used as a bargaining chip} in a broader negotiation with China on trade,
security and other issues.

On Wednesday, the White House sent a letter from Mr. Trump to Mr. Xi
wishing him a happy Chinese New Year, which administration officials
described as an effort to keep the relationship from unraveling further
while they sought to resolve the tensions.

Relations between Washington and Beijing had been frozen since December,
when Mr. Trump took a congratulatory phone call from Taiwan's president,
Tsai Ing-wen. The United States has not had diplomatic relations with
Taiwan since 1979, and Mr. Trump defended the call by saying he did not
know why the United States should be bound by the One China policy.

To lay the groundwork for a better relationship, Mr. Trump's national
security adviser, Michael T. Flynn, spoke last Friday with China's top
foreign policy official, Yang Jiechi. That call produced only a vague
commitment to ``reinforce high-level exchanges,'' suggesting that Mr.
Trump's statements on China sill precluded a direct leader-to-leader
exchange.

As a gesture of conciliation, Mr. Flynn and his deputy, K. T. McFarland,
hand-delivered Mr. Trump's letter to China's ambassador to the United
States, Cui Tiankai. Mr. Trump wrote that he wished ``the Chinese people
a happy Lantern Festival and prosperous Year of the Rooster.'' He also
said he ``looks forward to working with President Xi to develop a
constructive relationship that benefits both the United States and
China.''

``This letter means they're looking for creative ways to stabilize this
relationship when Trump and Xi can't talk due to differences over Taiwan
policy,'' said Evan S. Medeiros, who was senior director for Asia on the
National Security Council under President Barack Obama.

But there were indications that the administration recognized it needed
to do more. Mr. Tillerson, officials said, suggested that Mr. Trump
publicly reaffirm his commitment to the One China policy as a way of
breaking the deadlock and getting the two presidents back on the phone.

For Mr. Trump, it was a significant reversal. In an interview with Fox
News in December, he said the policy should be contingent on extracting
concessions from Beijing.

``We're being hurt very badly by China with devaluation; with taxing us
heavy at the borders when we don't tax them; with building a massive
fortress in the middle of the South China Sea, which they shouldn't be
doing; and, frankly, with not helping us at all with North Korea,'' he
said.

Since his inauguration, Mr. Trump has spoken by phone with about 20
foreign leaders. Although these are usually highly scripted affairs, Mr.
Trump's have been anything but. His conversation last week with Prime
Minister Malcolm Turnbull of Australia
\href{https://www.nytimes.com/2017/02/02/us/politics/us-australia-trump-turnbull.html}{turned
contentious} when Mr. Turnbull urged Mr. Trump to honor an agreement
made under Mr. Obama to accept 1,250 refugees from an offshore detention
center.

But arguably, no bilateral relationship is more important than the one
between Beijing and Washington, and the fact that Mr. Trump and Mr. Xi
had not talked since Mr. Trump took office in January has drawn
increasing scrutiny.

``The U.S.-China relationship only works if the two leaders have a
serious relationship and use their contact to do real business,'' Mr.
Medeiros said. ``Given the rigidity of the Chinese system, leader-level
contact provides essential stability, direction and momentum to
U.S.-China ties.''

Administration officials are also keenly aware that the Chinese will be
closely watching the visit of Mr. Abe, which begins here Friday with an
Oval Office meeting, a White House lunch and a joint news conference.
Then Mr. Trump will take Mr. Abe to Palm Beach, Fla., on Air Force One.
The leaders plan to play golf at the Trump International Golf Club and
then they and their wives will have dinner at Mar-a-Lago, another Trump
club.

To the status-conscious Chinese, this red-carpet treatment will not go
unnoticed. Analysts say it may reinforce their suspicion that the Trump
administration is making Japan the centerpiece of its Asia strategy.

In Beijing, Lu Kang, a spokesman for China's Foreign Ministry, expressed
thanks for Mr. Trump's letter. He dismissed as ``senseless'' speculation
the idea that Mr. Trump had been snubbing Mr. Xi by not scheduling a
phone call earlier. ``The two countries share wide common interests, and
cooperation is the only correct path for both,'' Mr. Lu told reporters
on Thursday.

Even before the phone call, Mr. Trump and his advisers had markedly
shifted their tone toward China since the inauguration.

During the campaign, Mr. Trump advocated a 45 percent tariff on Chinese
exports to the United States, complaining that China manipulated the
value of its currency. This month, however, Ivanka Trump, the
president's daughter, attended a Lunar New Year celebration at the
Chinese Embassy in Washington. Her daughter, Arabella, sang a New Year's
greeting in Mandarin that was widely viewed in China.

Ms. Trump's husband, Jared Kushner, who is a senior adviser to Mr.
Trump, met with Mr. Cui before the embassy event, part of a blossoming
dialogue between the two men.

The business relationships between some of Mr. Trump's advisers and
leading Chinese companies with close links to the Communist Party may
also be strengthening ties. Mr. Kushner
\href{https://www.nytimes.com/2017/01/07/us/politics/jared-kushner-trump-business.html}{took
part in talks last year} with the Chinese billionaire Wu Xiaohui to help
redevelop the Kushner family's crown jewel, a commercial building on
Fifth Avenue.

``It's an expression of good will,'' Jia Qingguo, dean of the School of
International Studies at Peking University, said of Mr. Trump's letter.
``It's necessary to handle this relationship with practical
cooperation.''

Advertisement

\protect\hyperlink{after-bottom}{Continue reading the main story}

\hypertarget{site-index}{%
\subsection{Site Index}\label{site-index}}

\hypertarget{site-information-navigation}{%
\subsection{Site Information
Navigation}\label{site-information-navigation}}

\begin{itemize}
\tightlist
\item
  \href{https://help.nytimes.com/hc/en-us/articles/115014792127-Copyright-notice}{©~2020~The
  New York Times Company}
\end{itemize}

\begin{itemize}
\tightlist
\item
  \href{https://www.nytco.com/}{NYTCo}
\item
  \href{https://help.nytimes.com/hc/en-us/articles/115015385887-Contact-Us}{Contact
  Us}
\item
  \href{https://www.nytco.com/careers/}{Work with us}
\item
  \href{https://nytmediakit.com/}{Advertise}
\item
  \href{http://www.tbrandstudio.com/}{T Brand Studio}
\item
  \href{https://www.nytimes.com/privacy/cookie-policy\#how-do-i-manage-trackers}{Your
  Ad Choices}
\item
  \href{https://www.nytimes.com/privacy}{Privacy}
\item
  \href{https://help.nytimes.com/hc/en-us/articles/115014893428-Terms-of-service}{Terms
  of Service}
\item
  \href{https://help.nytimes.com/hc/en-us/articles/115014893968-Terms-of-sale}{Terms
  of Sale}
\item
  \href{https://spiderbites.nytimes.com}{Site Map}
\item
  \href{https://help.nytimes.com/hc/en-us}{Help}
\item
  \href{https://www.nytimes.com/subscription?campaignId=37WXW}{Subscriptions}
\end{itemize}
