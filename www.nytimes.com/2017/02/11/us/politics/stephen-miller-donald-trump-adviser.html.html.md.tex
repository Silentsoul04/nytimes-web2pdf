Sections

SEARCH

\protect\hyperlink{site-content}{Skip to
content}\protect\hyperlink{site-index}{Skip to site index}

\href{https://www.nytimes.com/section/politics}{Politics}

\href{https://myaccount.nytimes.com/auth/login?response_type=cookie\&client_id=vi}{}

\href{https://www.nytimes.com/section/todayspaper}{Today's Paper}

\href{/section/politics}{Politics}\textbar{}Stephen Miller Is a `True
Believer' Behind Core Trump Policies

\url{https://nyti.ms/2l1gw5y}

\begin{itemize}
\item
\item
\item
\item
\item
\end{itemize}

Advertisement

\protect\hyperlink{after-top}{Continue reading the main story}

Supported by

\protect\hyperlink{after-sponsor}{Continue reading the main story}

\hypertarget{stephen-miller-is-a-true-believer-behind-core-trump-policies}{%
\section{Stephen Miller Is a `True Believer' Behind Core Trump
Policies}\label{stephen-miller-is-a-true-believer-behind-core-trump-policies}}

\includegraphics{https://static01.nyt.com/images/2017/02/12/us/12-miller/11miller-2-articleLarge.jpg?quality=75\&auto=webp\&disable=upscale}

By \href{https://www.nytimes.com/by/glenn-thrush}{Glenn Thrush} and
\href{http://www.nytimes.com/by/jennifer-steinhauer}{Jennifer
Steinhauer}

\begin{itemize}
\item
  Feb. 11, 2017
\item
  \begin{itemize}
  \item
  \item
  \item
  \item
  \item
  \end{itemize}
\end{itemize}

WASHINGTON --- Staff members on Capitol Hill recall Stephen Miller, the
31-year-old White House adviser behind many of President Trump's most
contentious executive orders, as the guy from Jeff Sessions's office who
made their inboxes cry for mercy.

As a top aide to Mr. Sessions, the conservative Alabama senator, Mr.
Miller dispatched dozens and dozens of bombastic emails to congressional
staff members and reporters in early 2013 when the Senate was
considering a big bipartisan immigration overhaul. Mr. Miller slammed
the evils of ``foreign labor'' and pushed around nasty news articles on
proponents of compromise, like Senator Marco Rubio of Florida.

One exhausted Senate staff member, forwarding a Miller-gram to a
reporter at the time, wrote: ``His latest. And it's only 11:45 a.m.''

The ascent of Mr. Miller from far-right gadfly with little policy
experience to the president's senior policy adviser came as a shock to
many of the staff members who knew him from his seven years in the
Senate. A man whose emails were, until recently, considered spam by many
of his Republican peers is now shaping the Trump administration's core
domestic policies with his economic nationalism and hard-line positions
on immigration.

\includegraphics{https://static01.nyt.com/images/2017/02/23/us/miller-vid/miller-vid-videoSixteenByNine3000.jpg}

But his unlikely rise is emblematic of a White House where
unconventional résumés rule --- where the chief strategist is Stephen K.
Bannon, until recently the head of the flame-throwing right-wing website
Breitbart News, and the president himself is a former reality television
star who before winning the nation's highest office had never shown much
interest in the arcana of governing.

Yet all three men are bound by a belief in an America-first economic
policy that has suddenly moved from the fringes of American politics to
the Oval Office.

``Stephen was the kind of guy who would make a passionate ideological
argument to a roomful of people who were there to make pragmatic
decisions,'' said Alex Conant, a former aide to Mr. Rubio who remembers
squaring off against Mr. Miller at a routine Republican messaging
meeting that turned into a full-dress immigration debate.

Mr. Miller has been at the epicenter of some of the administration's
most provocative moves, from pushing hard for the construction of a wall
along the border with Mexico to threatening decades-long trade deals at
the heart of Republican economic orthodoxy, to rolling out Mr. Trump's
travel ban on seven largely Muslim nations, whose bungled introduction
he oversaw.

Working in an administration that ``didn't come here to do small
things,'' as Mr. Bannon has put it, is a role that Mr. Miller ---
universally known as a tireless worker --- has been preparing for much
of his life. From his days at a
\href{http://www.latimes.com/politics/la-na-pol-trump-speechwriter-santamonica-20170117-story.html}{public
high school in Southern California}, where he preached against
``political correctness'' and liberalism and called in to conservative
radio shows, to his time at Duke University, where he was known for
\href{http://www.dukechronicle.com/staff/stephen-miller}{controversial
writings} in the student newspaper and a failed attempt at a run for
dorm president, he has delighted in challenging prevailing orthodoxies.

At a freshman mixer, recalled a college classmate who spoke on the
condition of anonymity, Mr. Miller announced, ``My name is Stephen
Miller, I am from Los Angeles, and I like guns.''

Mr. Miller, known for his skinny ties, so-outdated-they're-chic pants
and his recently abandoned chain-smoking habit, enjoyed a relatively
turbulence-free ascent in Mr. Trump's orbit until the travel ban. His
eagerness to keep a tight lid on key details of executive orders to
prevent leaks --- as well as his inexperience --- has at times hampered
coordination between the West Wing and agencies that would have to carry
them out, several White House officials said.

In part to deal with the confusion that surrounded the travel ban,
Reince Priebus, the White House chief of staff, recently created a
10-point protocol that requires all major executive actions to be
cleared with the communications department and other senior White House
staff members before implementation.

But Mr. Miller's peacock confidence has served him well with Mr. Trump,
who first got to know him in 2015, when Mr. Miller helped bring Mr.
Sessions, now the attorney general, into the Trump fold.

After the Republican National Convention in July, Mr. Miller became Mr.
Trump's principal day-to-day speechwriter once the candidate had
switched from handwritten notes to a teleprompter in the middle of the
campaign.

\includegraphics{https://static01.nyt.com/images/2017/02/11/us/11miller1/03MILLER-articleLarge.jpg?quality=75\&auto=webp\&disable=upscale}

The message in those speeches was so reflective of Mr. Trump's views
that it earned Mr. Miller a spot as the warm-up act for Mr. Trump's
campaign rallies. His words became Mr. Trump's --- ``We're going to
build that wall, and we're going to build it out of love,'' Mr. Miller
often said.

``Steve is a true believer in every sense of the word, not just in this
message of economic populism but in President Trump as a leader,'' said
Jason Miller, who worked with him in the Trump campaign and is not
related. ``Steve's fiercely loyal and has a better understanding of the
president's vision than almost anyone.''

It is sometimes hard to tell Mr. Trump's voice from that of Mr. Miller,
who suppressed his own orotund speech to capture the president's more
visceral, off-the-cuff style. Not that he has had much choice: As one of
three or four staff members to fly around with Mr. Trump during the last
few months of the campaign, Mr. Miller was summoned to speechwriting
tasks by a bark of ``Ready!'' from Mr. Trump, who insisted on dictating
practically every word --- and laced into staff members who changed a
word or inserted an overly complex policy point.

Mr. Miller's flexibility as a speechwriter is offset by the consistent
stridency of his political philosophy, which has remained much the same
since he was the distinct minority at Santa Monica High School, a
liberal outpost where he often railed against fellow students and the
school administration. Mexican heritage celebrations and Iraq war
protests were things of particular offense. He produced a 2003 essay,
\href{http://archive.frontpagemag.com/readArticle.aspx?ARTID=17409}{``How
I Changed My Left Wing High School,''} that capped a high school career
steeped in political activism.

At Duke, Mr. Miller, who is Jewish, cut a similarly confrontational
swath, and was briefly friendly with Richard Spencer, who later became a
prominent white supremacist, when both were members of the university's
Young Conservatives chapter.

From there, it was straight to Capitol Hill, where Mr. Miller worked for
Representatives Michele Bachmann of Minnesota and John Shadegg of
Arizona before ending up with Mr. Sessions in 2009.

Mr. Sessions and Mr. Miller worked tirelessly against the 2013-14
congressional effort at an immigration overhaul. The bill passed the
Senate easily in spite of Mr. Sessions's vociferous objections, but
failed in the House.

``We knew we were taking on the establishment, and Steve was an
incredibly hard worker and had no second thoughts about it,'' Mr.
Sessions said in an interview.

Mr. Miller wrote many of the incendiary speeches that Mr. Sessions gave
about the bill, including one in which he suggested that a
Cuban-American aide to Senator Chuck Schumer, Democrat of New York, had
been the author of a measure that he believed to be ``amnesty'' in
disguise.

Ultimately, it was Mr. Miller's dour views on illegal immigration that
endeared him to Mr. Bannon and a small team of like-minded economic
nationalists that included Julia Hahn, a former Breitbart writer. The
group came together during the 2014 campaign of the far-right Republican
candidate Dave Brat, whose upset win over the House majority leader,
Eric Cantor, in a suburban Richmond, Va., district augured Mr. Trump's
success.

Image

Mr. Miller, right, was a spokesman for Senator Jeff Sessions, center,
now attorney general, before joining Mr. Trump.Credit...Douglas
Graham/CQ Roll Call, via Associated Press

Even then, Mr. Miller had his eye on Mr. Trump, who had flirted with a
run for president in 2012. Soon after Mr. Brat's victory in the
Republican primary in July 2014, Mr. Miller sent his friends a Breitbart
interview in which Mr. Trump declared, ``Everybody is vulnerable because
what's happening in the country is very sad, and the world is
watching.'' Mr. Miller added a comment: ``Trump gets it. I wish he'd run
for president.''

Mr. Bannon, Mr. Miller and the lesser-known head of the Domestic Policy
Council, Andrew Bremberg, spent the later part of the transition period
mapping out a shock-and-awe protocol of executive orders, sending more
than 200 to federal agencies for review. The trouble came when they sent
some of them to Obama-appointed officials at federal agencies for
review, leading to leaks that prompted Mr. Miller to restrict the
circulation of the plans.

Mr. Trump, initially pleased by the bold series of executive actions
orchestrated by the team, was stung by the fallout from Mr. Miller's
execution of the immigration order, and expressed frustration about not
being fully briefed on an order reorganizing the National Security
Council to give Mr. Bannon additional power.

Despite the internal finger-pointing, Mr. Miller remains close to the
still-powerful Mr. Bannon, who described him in an email as ``a loyal
and faithful soldier in the Trump movement, a warrior for the working
class.''

In recent days, Mr. Miller has been working on what is expected to be
another contentious order: an as-yet-uncirculated rewriting of the guest
worker program that is likely to impose new restrictions on the cheap
foreign labor that Mr. Miller deplored in many of his 2013 emails,
according to two officials familiar with their planning.

Mr. Miller, one of the officials said, is working closely with
Department of Homeland Security aides to avoid a repeat of the travel
ban fiasco.

Advertisement

\protect\hyperlink{after-bottom}{Continue reading the main story}

\hypertarget{site-index}{%
\subsection{Site Index}\label{site-index}}

\hypertarget{site-information-navigation}{%
\subsection{Site Information
Navigation}\label{site-information-navigation}}

\begin{itemize}
\tightlist
\item
  \href{https://help.nytimes.com/hc/en-us/articles/115014792127-Copyright-notice}{©~2020~The
  New York Times Company}
\end{itemize}

\begin{itemize}
\tightlist
\item
  \href{https://www.nytco.com/}{NYTCo}
\item
  \href{https://help.nytimes.com/hc/en-us/articles/115015385887-Contact-Us}{Contact
  Us}
\item
  \href{https://www.nytco.com/careers/}{Work with us}
\item
  \href{https://nytmediakit.com/}{Advertise}
\item
  \href{http://www.tbrandstudio.com/}{T Brand Studio}
\item
  \href{https://www.nytimes.com/privacy/cookie-policy\#how-do-i-manage-trackers}{Your
  Ad Choices}
\item
  \href{https://www.nytimes.com/privacy}{Privacy}
\item
  \href{https://help.nytimes.com/hc/en-us/articles/115014893428-Terms-of-service}{Terms
  of Service}
\item
  \href{https://help.nytimes.com/hc/en-us/articles/115014893968-Terms-of-sale}{Terms
  of Sale}
\item
  \href{https://spiderbites.nytimes.com}{Site Map}
\item
  \href{https://help.nytimes.com/hc/en-us}{Help}
\item
  \href{https://www.nytimes.com/subscription?campaignId=37WXW}{Subscriptions}
\end{itemize}
