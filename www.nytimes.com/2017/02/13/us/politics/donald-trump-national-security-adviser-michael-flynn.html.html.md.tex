Sections

SEARCH

\protect\hyperlink{site-content}{Skip to
content}\protect\hyperlink{site-index}{Skip to site index}

\href{https://www.nytimes.com/section/politics}{Politics}

\href{https://myaccount.nytimes.com/auth/login?response_type=cookie\&client_id=vi}{}

\href{https://www.nytimes.com/section/todayspaper}{Today's Paper}

\href{/section/politics}{Politics}\textbar{}Michael Flynn Resigns as
National Security Adviser

\url{https://nyti.ms/2l8YoXo}

\begin{itemize}
\item
\item
\item
\item
\item
\item
\end{itemize}

Advertisement

\protect\hyperlink{after-top}{Continue reading the main story}

Supported by

\protect\hyperlink{after-sponsor}{Continue reading the main story}

\hypertarget{michael-flynn-resigns-as-national-security-adviser}{%
\section{Michael Flynn Resigns as National Security
Adviser}\label{michael-flynn-resigns-as-national-security-adviser}}

\includegraphics{https://static01.nyt.com/images/2017/02/14/us/14flynn-JP/14flynn-JP-articleLarge.jpg?quality=75\&auto=webp\&disable=upscale}

By \href{https://www.nytimes.com/by/maggie-haberman}{Maggie Haberman},
\href{https://www.nytimes.com/by/matthew-rosenberg}{Matthew Rosenberg},
\href{https://www.nytimes.com/by/matt-apuzzo}{Matt Apuzzo} and
\href{https://www.nytimes.com/by/glenn-thrush}{Glenn Thrush}

\begin{itemize}
\item
  Feb. 13, 2017
\item
  \begin{itemize}
  \item
  \item
  \item
  \item
  \item
  \item
  \end{itemize}
\end{itemize}

Michael T. Flynn, the national security adviser,
\href{http://www.nytimes.com/interactive/2017/02/13/us/politics/document-Michael-Flynn-Resignation-Letter.html}{resigned
on Monday night} after it was revealed that he had misled Vice President
Mike Pence and other top White House officials about his conversations
with the Russian ambassador to the United States.

Mr. Flynn, who served in the job for less than a month, said he had
given ``incomplete information'' regarding a telephone call he had with
the ambassador in late December about American sanctions against Russia,
weeks before President Trump's inauguration. Mr. Flynn previously had
denied that he had any substantive conversations with Ambassador Sergey
I. Kislyak, and Mr. Pence repeated that claim in television interviews
as recently as this month.

But on Monday, a former administration official said the Justice
Department warned the White House last month that Mr. Flynn had not been
fully forthright about his conversations with the ambassador. As a
result, the Justice Department feared that Mr. Flynn could be vulnerable
to blackmail by Moscow.

In his resignation letter, which the White House emailed to reporters,
Mr. Flynn said he had held numerous calls with foreign officials during
the transition. ``Unfortunately, because of the fast pace of events, I
inadvertently briefed the vice president-elect and others with
incomplete information regarding my phone calls with the Russian
ambassador,'' he wrote. ``I have sincerely apologized to the president
and the vice president, and they have accepted my apology.''

``I am tendering my resignation, honored to have served our nation and
the American people in such a distinguished way,'' Mr. Flynn wrote.

\includegraphics{https://static01.nyt.com/images/2017/02/15/us/15assess/15assess-videoSixteenByNineJumbo1600-v2.jpg}

The White House said in the statement that it was replacing Mr. Flynn
with retired Lt. Gen. Joseph K. Kellogg Jr. of the Army, a Vietnam War
veteran, as acting national security adviser.

\includegraphics{https://static01.nyt.com/images/2017/02/15/us/15flynn/15flynn-videoSixteenByNineJumbo1600-v2.jpg}

Mr. Flynn was an early and ardent supporter of Mr. Trump's candidacy,
and in his resignation he sought to praise the president. ``In just
three weeks,'' Mr. Flynn said, the new president ``has reoriented
American foreign policy in fundamental ways to restore America's
leadership position in the world.''

But in doing so, he inadvertently illustrated the brevity of his
tumultuous run at the National Security Council, and the chaos that has
gripped the White House in the first weeks of the Trump administration
--- and created a sense of uncertainty around the world.

Earlier Monday, Sean Spicer, the White House press secretary, told
reporters that ``the president is evaluating the situation'' about Mr.
Flynn's future. By Monday evening, Mr. Flynn's fortunes were rapidly
shifting --- his resignation came roughly seven hours after Kellyanne
Conway, a counselor to the president, said on MSNBC that Mr. Trump had
``full confidence'' in the retired general.

And when he did step down, it happened so quickly that his resignation
does not appear to have been communicated to National Security Council
staff members, two of whom said they learned about it from news reports.

Officials said Mr. Pence had told others in the White House that he
believed Mr. Flynn lied to him by saying he had not discussed the topic
of sanctions on a call with the Russian ambassador in late December.
Even the mere discussion of policy --- and the apparent attempt to
assuage the concerns of an American adversary before Mr. Trump took
office --- represented a
\href{http://www.dod.mil/dodgc/defense_ethics/resource_library/summary_emoluments_clause_restrictions.pdf}{remarkable
breach of protocol}.

The F.B.I. had been examining Mr. Flynn's phone calls as he came under
growing questions about his interactions with Russian officials and his
management of the National Security Council. The blackmail risk
envisioned by the Justice Department would have stemmed directly from
Mr. Flynn's attempt to cover his tracks with his bosses. The Russians
knew what had been said on the call; thus, if they wanted Mr. Flynn to
do something, they could have threatened to expose the lie if he
refused.

The Justice Department's warning to the White House was first reported
on Monday night by
\href{https://www.washingtonpost.com/world/national-security/justice-department-warned-white-house-that-flynn-could-be-vulnerable-to-russian-blackmail-officials-say/2017/02/13/fc5dab88-f228-11e6-8d72-263470bf0401_story.html?hpid=hp_rhp-top-table-main_flynn-0818pm\%3Ahomepage\%2Fstory\&utm_term=.5bdf89b8ea34}{The
Washington Post}.

\href{https://www.nytimes.com/interactive/2017/02/13/us/politics/document-Michael-Flynn-Resignation-Letter.html}{}

\includegraphics{https://static01.nyt.com/images/2017/02/13/us/politics/image-Michael-Flynn-Resignation-Letter/image-Michael-Flynn-Resignation-Letter-thumbLarge.gif}

\hypertarget{michael-flynns-resignation-letter}{%
\subsection{Michael Flynn's Resignation
Letter}\label{michael-flynns-resignation-letter}}

Michael T. Flynn, under scrutiny for his communication with Russia,
resigned as President Trump's national security adviser late Monday.

In addition, the Army has been investigating whether Mr. Flynn received
money from the Russian government during a trip he took to Moscow in
2015, according to two defense officials. Such a payment might violate
the Emoluments Clause of the Constitution, which prohibits former
military officers from receiving money from a foreign government without
consent from Congress. The defense officials said there was no record
that Mr. Flynn, a retired three-star Army general, filed the required
paperwork for the trip.

Representative Adam B. Schiff of California, the top Democrat on the
House Intelligence Committee, said in a statement late Monday that Mr.
Flynn's resignation would not close the question of his contact with
Russian officials.

``General Flynn's decision to step down as national security adviser was
all but ordained the day he misled the country about his secret talks
with the Russian ambassador,'' said Mr. Schiff, noting that the matter
is still under investigation by the House committee.

Two other Democratic lawmakers --- Representative John Conyers Jr. of
Michigan and Representative Elijah E. Cummings of Maryland --- called
for an immediate briefing by the Justice Department and the F.B.I. over
the ``alarming new disclosures'' that Mr. Flynn was a blackmail risk.
``We need to know who else within the White House is a current and
ongoing risk to our national security,'' they said in a statement.

Representative Devin Nunes, Republican of California and the chairman of
the House intelligence committee, was supportive of Mr. Flynn until the
end. ``Washington, D.C., can be a rough town for honorable people, and
Flynn --- who has always been a soldier, not a politician --- deserves
America's gratitude and respect,'' Mr. Nunes said in a statement.

The White House had examined a transcript of a wiretapped conversation
that Mr. Flynn had with Mr. Kislyak in December, according to
administration officials. Mr. Flynn originally told Mr. Pence and others
that the call was limited to small talk and holiday pleasantries.

But the conversation, according to officials who saw the transcript of
the wiretap, also included a discussion about sanctions imposed on
Russia after intelligence agencies determined that President Vladimir V.
Putin's government tried to interfere with the 2016 election on Mr.
Trump's behalf. Still, current and former administration officials
familiar with the call said the transcript was ambiguous enough that Mr.
Trump could have justified either firing or retaining Mr. Flynn.

\includegraphics{https://static01.nyt.com/images/2017/01/19/us/19nsc1/19nsc1-videoSixteenByNine3000-v3.jpg}

Mr. Trump, however, had become increasingly concerned about the
continued fallout over Mr. Flynn's behavior, according to people
familiar with his thinking, and told aides that the media storm around
Mr. Flynn would damage the president's image on national security
issues.

Stephen K. Bannon, the president's chief strategist, asked for Mr.
Flynn's resignation --- a move that he has been pushing for since
Friday, when it became clear that the national security adviser had
misled Mr. Pence.

Around 8:20 p.m. Monday, a sullen Mr. Flynn was seen in the Oval Office,
just as preparations were being made for the swearing-in of newly
\href{https://www.nytimes.com/2017/02/13/us/politics/steven-mnuchin-confirmed-treasury-secretary.html}{confirmed}
Treasury Secretary Steven T. Mnuchin. Soon after, Mr. Flynn's
resignation letter started making the rounds.

Administration officials said it was unlikely that Mr. Kellogg would be
asked to stay on as Mr. Flynn's permanent replacement. Mr. Flynn brought
Mr. Kellogg into the Trump campaign, according to a former campaign
adviser, and the two have remained close. K. T. McFarland, the deputy
national security adviser who also was brought on by Mr. Flynn, is
expected to leave that role, a senior official said.

One person close to the administration, who was not authorized to
discuss the personnel moves and spoke on the condition of anonymity,
said that retired Vice Admiral Robert S. Harward is the leading
candidate to replace Mr. Flynn, although Mr. Kellogg and David H.
Petraeus are being discussed. It was not clear whether Mr. Petraeus is
still expected to appear at the White House this week, as initially
discussed by advisers to the president.

Mr. Flynn's concealment of the call's content, combined with
\href{https://www.nytimes.com/2017/02/12/us/politics/national-security-council-turmoil.html}{questions
about his management of his agency} and reports of a demoralized staff,
put him in a precarious position less than a month into Mr. Trump's
presidency.

Few members of Mr. Trump's team were more skeptical of Mr. Flynn than
the vice president, numerous administration officials said. Mr. Pence,
who used the false information provided by Mr. Flynn to defend him in a
series of television appearances, was incensed at Mr. Flynn's lack of
contrition for repeatedly embarrassing him by withholding the
information, according to three administration officials familiar with
the situation.

Mr. Flynn and Mr. Pence spoke twice in the past few days about the
matter, but administration officials said that rather than fully
apologize and accept responsibility, the national security adviser
blamed his faulty memory --- which irked the typically slow-to-anger Mr.
Pence.

The slight was compounded by an episode late last year when Mr. Pence
went on television to deny that Mr. Flynn's son, who had posted
conspiracy theories about Hillary Clinton on social media, had been
given a security clearance by the transition team. The younger Mr. Flynn
had, indeed, been given such a clearance, even though his father had
told Mr. Pence's team that he had not.

Officials said classified information did not appear to have been
discussed during the conversation between Mr. Flynn and the ambassador,
which would have been a crime. The call was captured on a routine
wiretap of diplomats' calls, the officials said.

But current Trump administration officials and former Obama
administration officials said that Mr. Flynn did appear to be reassuring
the ambassador that Mr. Trump would adopt a more accommodating tone on
Russia once in office.

Former and current administration officials said that Mr. Flynn urged
Russia not to retaliate against any sanctions because an overreaction
would make any future cooperation more complicated. He never explicitly
promised sanctions relief, one former official said, but he appeared to
leave the impression that it would be possible.

During his 2015 trip to Moscow, Mr. Flynn was paid to attend the
anniversary celebration of Russia Today, a television network controlled
by the Kremlin. At the banquet, he sat next to Mr. Putin.

Mr. Flynn had notified the Defense Intelligence Agency, which he once
led, that he was taking the trip. He received a security briefing from
agency officials before he left, which is customary for former top
agency officials when they travel overseas.

Still, some senior agency officials were surprised when footage of the
banquet appeared on RT, and believed that Mr. Flynn should have been
more forthcoming with the agency about the nature of his trip to Russia.

Advertisement

\protect\hyperlink{after-bottom}{Continue reading the main story}

\hypertarget{site-index}{%
\subsection{Site Index}\label{site-index}}

\hypertarget{site-information-navigation}{%
\subsection{Site Information
Navigation}\label{site-information-navigation}}

\begin{itemize}
\tightlist
\item
  \href{https://help.nytimes.com/hc/en-us/articles/115014792127-Copyright-notice}{©~2020~The
  New York Times Company}
\end{itemize}

\begin{itemize}
\tightlist
\item
  \href{https://www.nytco.com/}{NYTCo}
\item
  \href{https://help.nytimes.com/hc/en-us/articles/115015385887-Contact-Us}{Contact
  Us}
\item
  \href{https://www.nytco.com/careers/}{Work with us}
\item
  \href{https://nytmediakit.com/}{Advertise}
\item
  \href{http://www.tbrandstudio.com/}{T Brand Studio}
\item
  \href{https://www.nytimes.com/privacy/cookie-policy\#how-do-i-manage-trackers}{Your
  Ad Choices}
\item
  \href{https://www.nytimes.com/privacy}{Privacy}
\item
  \href{https://help.nytimes.com/hc/en-us/articles/115014893428-Terms-of-service}{Terms
  of Service}
\item
  \href{https://help.nytimes.com/hc/en-us/articles/115014893968-Terms-of-sale}{Terms
  of Sale}
\item
  \href{https://spiderbites.nytimes.com}{Site Map}
\item
  \href{https://help.nytimes.com/hc/en-us}{Help}
\item
  \href{https://www.nytimes.com/subscription?campaignId=37WXW}{Subscriptions}
\end{itemize}
