Sections

SEARCH

\protect\hyperlink{site-content}{Skip to
content}\protect\hyperlink{site-index}{Skip to site index}

\href{https://www.nytimes.com/section/world/asia}{Asia Pacific}

\href{https://myaccount.nytimes.com/auth/login?response_type=cookie\&client_id=vi}{}

\href{https://www.nytimes.com/section/todayspaper}{Today's Paper}

\href{/section/world/asia}{Asia Pacific}\textbar{}Relief in Japan After
Shinzo Abe's Visit With Trump

\url{https://nyti.ms/2kCfyck}

\begin{itemize}
\item
\item
\item
\item
\item
\end{itemize}

Advertisement

\protect\hyperlink{after-top}{Continue reading the main story}

Supported by

\protect\hyperlink{after-sponsor}{Continue reading the main story}

\hypertarget{relief-in-japan-after-shinzo-abes-visit-with-trump}{%
\section{Relief in Japan After Shinzo Abe's Visit With
Trump}\label{relief-in-japan-after-shinzo-abes-visit-with-trump}}

\includegraphics{https://static01.nyt.com/images/2017/02/14/world/14Japan2/14Japan2-articleLarge.jpg?quality=75\&auto=webp\&disable=upscale}

By \href{http://www.nytimes.com/by/motoko-rich}{Motoko Rich}

\begin{itemize}
\item
  Feb. 13, 2017
\item
  \begin{itemize}
  \item
  \item
  \item
  \item
  \item
  \end{itemize}
\end{itemize}

TOKYO --- In many respects, Prime Minister Shinzo Abe's trip to
Washington and Florida to meet and play golf with President Trump went
as well as the Japanese leader could have hoped.

Sure, there was an
\href{https://www.youtube.com/watch?v=GWbP8eC-SIw\&feature=player_embedded}{awkward
handshake} between the two leaders that may have gone on for too long
(and unleashed a meme of Mr. Abe's uncomfortable facial expression). But
after an election campaign in which Mr. Trump frequently criticized
Japan on trade issues and accused the country of not paying enough for
its military defense, he assured Japan that the relationship between the
two countries
``\href{https://www.nytimes.com/2017/02/10/world/asia/trump-shinzo-abe-meeting.html?smid=tw-share}{runs
very, very deep},'' and showed a deference to Mr. Abe that belied his
previous remarks.

Before the visit last week, some in the Japanese news media had gibed
Mr. Abe for his apparent eagerness to foster a friendship with Mr.
Trump, and some joked that the American president would
\href{https://twitter.com/motokorich/status/829946184269656065}{take
advantage} of the Japanese leader during their bout of golf diplomacy at
the president's Mar-a-Lago resort in Florida. But in a
\href{http://english.kyodonews.jp/news/2017/02/458432.html}{Kyodo News
poll} taken after the meeting, 70 percent of the Japanese public said
they were satisfied with the talks between the two leaders, and Mr.
Abe's approval ratings rose slightly from a month earlier to close to 62
percent.

``In a basic sense, Prime Minister Abe got almost everything he
wanted,'' said Fumiaki Kubo, a professor of political science at the
University of Tokyo. Mr. Trump's statements in a joint news conference
with Mr. Abe were ``totally different from what he has been saying about
Japan since the 1980s,'' Mr. Kubo said. ``That is surprising as well as
remarkable. In a sense he showed us, including the American public, that
he is capable of changing his position on such an important issue as
Japan.''

Mr. Trump, who as a candidate and president-elect assailed Japan as one
of the countries that
``\href{https://www.nytimes.com/2016/09/27/us/politics/transcript-debate.html?_r=0}{do
not pay us}'' for defense and repeatedly called for an ``America First''
economy, ended up thanking the people of Japan for hosting United States
troops and called for a trading relationship ``that is free, fair and
reciprocal, benefiting both of our countries.''

Perhaps most significant to the Japanese, Mr. Trump promised that the
United States was ``committed to the security of Japan and all areas
under its administrative control,'' a reference to the American
guarantee to defend Japan in any confrontation with China over disputed
islands, known in Japan as the Senkaku and in China as the Diaoyu, in
the East China Sea.

The remarks drew swift criticism from China, where an
\href{http://paper.people.com.cn/rmrbhwb/html/2017-02/13/content_1749422.htm}{editorial}
in the overseas edition of People's Daily, an official newspaper of the
Communist Party, said Mr. Abe had made a ``fetish'' of Japan's alliance
with the United States.

Mr. Abe had ``exaggerated the threat from China to create momentum for
America and Japan to join hands and contain China's rise,'' said the
editorial, written by Su Xiaohui, a senior researcher at a state-run
foreign policy think tank in Beijing.

Analysts said it was not surprising that Mr. Abe could get along so well
with Mr. Trump. He was the first world leader to
\href{https://www.nytimes.com/2016/11/17/world/asia/shinzo-abe-donald-trump.html}{meet
with the president-elect} in November at Trump Tower in New York after
the election and the second, after
\href{https://www.nytimes.com/2017/01/27/world/europe/theresa-may-britain-trump.html}{Prime
Minister Theresa May of Britain}, to meet with President Trump after the
inauguration.

Mr. Abe has also been adept at developing relationships with other
potentially difficult heads of state.

``Abe himself is just very good at dealing with strong-willed
authoritarian leaders,'' said
\href{https://www.csis.org/people/michael-j-green}{Michael J. Green}, a
former Asia adviser to President George W. Bush and now at the Center
for Strategic and International Studies in Washington, referring to Mr.
Abe's relationships with leaders including President Recep Tayyip
Erdogan of Turkey, Prime Minister Narendra Modi of India and President
Vladimir V. Putin of Russia.

While the majority of the Japanese public approved of Mr. Abe's meeting
with Mr. Trump, there was some criticism, particularly from the left.
The leader of the Japanese Communist Party, Kazuo Shii, denounced Mr.
Abe for not objecting to Mr. Trump's barring of refugees and foreign
visitors from seven predominantly Muslim countries. Others cautioned
that the mercurial Mr. Trump could easily change his mind if offended,
and that Mr. Abe had made himself vulnerable by seeming too eager to
align himself with the American president.

In an editorial,
\href{http://mainichi.jp/english/articles/20170212/p2a/00m/0na/009000c}{the
centrist Mainichi Shimbun said} that Mr. Trump might have taken a
strategy of ``first giving away what Japan desires and then making it
impossible for Japan to turn down U.S. demands,'' while
\href{http://www.asahi.com/ajw/articles/AJ201702120030.html}{Hirotoshi
Sako}, political news editor of the Asahi Shimbun, wrote that ``not
being able to voice views that may run counter to the will of the other
side cannot equate to a mature bilateral relationship.''

Some analysts said that by pursuing a close friendship with Mr. Trump,
Mr. Abe was betraying the moral foundation of the alliance that has
endured since the end of World War II.

``The idea was that the United States is really the standard-bearer of
the liberal international order,'' said
\href{http://www.fla.sophia.ac.jp/professors/nakanokoichi}{Koichi
Nakano}, a political scientist at Sophia University in Tokyo. ``Now,
obviously with Trump there's a very big question mark whether the
American commitments to these so-called shared values remain the same.''

From a strategic security standpoint, Japan may have few choices other
than to continue a strong alliance with the United States, given
\href{https://www.nytimes.com/2017/02/13/world/asia/north-korea-missile-launch-success.html?ref=asia}{threats
from North Korea}, which tested a ballistic missile while Mr. Abe was
still in Florida with Mr. Trump, and China, which has become more
assertive in the East and South China Seas.

``What exactly are Mr. Abe's or Japan's options?'' said Grant Newsham,
senior research fellow at the
\href{http://www.jfss.gr.jp/english/aboutus-en.htm}{Japan Forum for
Strategic Studies}. ``What exactly is to be gained if he was to be
standoffish or even pick a fight, which doesn't take much when Mr.
Trump's involved? Japan, really, by itself, cannot handle the security
problems in Asia.''

Mr. Trump may also be coming to realize that with a rising China, Japan
is the United States' best ally in Asia.

``The logical conclusion is that, from the strategic point of view,
Japan and the United States are still in full sync,'' said Kunihiko
Miyake, a former Japanese diplomat now teaching at
\href{http://en.ritsumei.ac.jp/}{Ritsumeikan University} in Kyoto.

Before the American election, the Japanese government had clearly
preferred Hillary Clinton to Mr. Trump. But now that Mr. Trump is the
president, some analysts believe he could inadvertently help Mr. Abe's
ambitions to strengthen Japan's military and increase defense spending.

Although he has been assiduous about currying favor with Mr. Trump, Mr.
Abe has also pursued a program of diplomacy throughout Southeast Asia,
visiting countries like the Philippines and Vietnam in recent months.

One Achilles' heel: South Korea. Relations between Tokyo and Seoul are
fraught after Japan
\href{https://www.nytimes.com/2017/01/06/world/asia/japan-south-korea-ambassador-comfort-woman-statue.html}{recalled
its envoy} to South Korea last month to protest a statue commemorating
Korean women who were forced into sexual slavery for Japanese soldiers
during World War II.

Analysts said that South Korea is Japan's most important ally in the
region. ``It's so close and bigger than all of them, except India,'' Mr.
Green said. For Japan, he said, South Korea remains ``a blind spot.''

Advertisement

\protect\hyperlink{after-bottom}{Continue reading the main story}

\hypertarget{site-index}{%
\subsection{Site Index}\label{site-index}}

\hypertarget{site-information-navigation}{%
\subsection{Site Information
Navigation}\label{site-information-navigation}}

\begin{itemize}
\tightlist
\item
  \href{https://help.nytimes.com/hc/en-us/articles/115014792127-Copyright-notice}{©~2020~The
  New York Times Company}
\end{itemize}

\begin{itemize}
\tightlist
\item
  \href{https://www.nytco.com/}{NYTCo}
\item
  \href{https://help.nytimes.com/hc/en-us/articles/115015385887-Contact-Us}{Contact
  Us}
\item
  \href{https://www.nytco.com/careers/}{Work with us}
\item
  \href{https://nytmediakit.com/}{Advertise}
\item
  \href{http://www.tbrandstudio.com/}{T Brand Studio}
\item
  \href{https://www.nytimes.com/privacy/cookie-policy\#how-do-i-manage-trackers}{Your
  Ad Choices}
\item
  \href{https://www.nytimes.com/privacy}{Privacy}
\item
  \href{https://help.nytimes.com/hc/en-us/articles/115014893428-Terms-of-service}{Terms
  of Service}
\item
  \href{https://help.nytimes.com/hc/en-us/articles/115014893968-Terms-of-sale}{Terms
  of Sale}
\item
  \href{https://spiderbites.nytimes.com}{Site Map}
\item
  \href{https://help.nytimes.com/hc/en-us}{Help}
\item
  \href{https://www.nytimes.com/subscription?campaignId=37WXW}{Subscriptions}
\end{itemize}
