Sections

SEARCH

\protect\hyperlink{site-content}{Skip to
content}\protect\hyperlink{site-index}{Skip to site index}

\href{https://www.nytimes.com/section/politics}{Politics}

\href{https://myaccount.nytimes.com/auth/login?response_type=cookie\&client_id=vi}{}

\href{https://www.nytimes.com/section/todayspaper}{Today's Paper}

\href{/section/politics}{Politics}\textbar{}Flynn's Downfall Sprang From
`Eroding Level of Trust'

\url{https://nyti.ms/2lMoxsm}

\begin{itemize}
\item
\item
\item
\item
\item
\item
\end{itemize}

Advertisement

\protect\hyperlink{after-top}{Continue reading the main story}

Supported by

\protect\hyperlink{after-sponsor}{Continue reading the main story}

\hypertarget{flynns-downfall-sprang-from-eroding-level-of-trust}{%
\section{Flynn's Downfall Sprang From `Eroding Level of
Trust'}\label{flynns-downfall-sprang-from-eroding-level-of-trust}}

\includegraphics{https://static01.nyt.com/images/2017/02/15/us/15xp-flynntimeline/15xp-flynntimeline-videoSixteenByNine3000.jpg}

By \href{http://www.nytimes.com/by/peter-baker}{Peter Baker},
\href{https://www.nytimes.com/by/glenn-thrush}{Glenn Thrush},
\href{http://www.nytimes.com/by/maggie-haberman}{Maggie Haberman},
\href{https://www.nytimes.com/by/adam-goldman}{Adam Goldman} and
\href{https://www.nytimes.com/by/julie-hirschfeld-davis}{Julie
Hirschfeld Davis}

\begin{itemize}
\item
  Feb. 14, 2017
\item
  \begin{itemize}
  \item
  \item
  \item
  \item
  \item
  \item
  \end{itemize}
\end{itemize}

WASHINGTON --- Just days into his new position as President Trump's
national security adviser, Michael T. Flynn found himself in a meeting
that any White House official would dread. Face to face with F.B.I.
agents, he was grilled about a phone call he had had with Russia's
ambassador.

What exactly Mr. Flynn said has not been disclosed, but current and
former government officials said on Tuesday that investigators had come
away believing that he was not entirely forthcoming. Soon after, the
acting attorney general decided to notify the White House, setting in
motion a chain of events that cost Mr. Flynn his job and thrust Mr.
Trump's fledgling administration into a fresh crisis.

Mr. Flynn's rise and fall followed familiar patterns in Washington,
where ambitious figures secure positions of great authority only to lose
them in a blizzard of contradictions, recriminations and scandal. But
rarely has an official at such a high level risen and fallen in such a
dizzyingly short time, in this case just 24 days after Mr. Flynn arrived
in the West Wing to take his corner office.

Given his short stay at the top, Mr. Flynn's case might be quickly
forgotten as an isolated episode if it did not raise other questions,
particularly about what the president knew and when. Even more broadly,
it underscores lingering uncertainty about the relationship between the
Trump administration and Vladimir V. Putin's Russia, a subject of great
interest given American intelligence reports of Moscow's intervention in
last year's elections in the United States.

As leaders of both parties said on Tuesday that they expected the Senate
to investigate and probably even summon Mr. Flynn to testify, more
details emerged about a drama that played out largely in secret inside a
White House riven by competing power centers. Sean Spicer, the White
House press secretary, revealed that Mr. Trump had known about concerns
that Mr. Flynn lied for more than two weeks before
\href{https://www.nytimes.com/2017/02/13/us/politics/donald-trump-national-security-adviser-michael-flynn.html}{demanding
his resignation} on Monday night. But Vice President Mike Pence was kept
in the dark and did not learn that Mr. Flynn had misled him about his
Russia contacts until reading news accounts late last week.

\includegraphics{https://static01.nyt.com/images/2017/02/15/us/15flynn-1487118237258/15flynn-1487118237258-articleLarge.jpg?quality=75\&auto=webp\&disable=upscale}

Mr. Spicer described a deliberative process in which a new president
took his time deciding what to do with Mr. Flynn, a retired three-star
general who played a major role in his campaign. The issue, Mr. Spicer
said, was not about legality but credibility.

``The evolving and eroding level of trust as a result of this situation
and a series of other questionable instances is what led the president
to ask for General Flynn's resignation,'' he said.

But other aides privately said that Mr. Trump, while annoyed at Mr.
Flynn, might not have pushed him out had the situation not attracted
such attention from the news media. Instead, according to three people
close to Mr. Trump, the president made the decision to cast aside Mr.
Flynn in a flash, the catalyst being a news alert of a coming article
about the matter.

``Yeah, it's time,'' Mr. Trump told one of his advisers.

Until around that point, Mr. Flynn seemed to think he was going to keep
his job. He told The Daily Caller, a conservative news site, on Monday
that he had not violated the law. ``If I did, believe me, the F.B.I.
would be down my throat, my clearances would be pulled,'' he said.
``There were no lines crossed.''

But by that evening, he was writing a resignation letter, admitting no
deception, only that he had ``inadvertently'' passed along ``incomplete
information.''

The issue traced back to a call last December between Mr. Flynn, then on
tap to become Mr. Trump's national security adviser, and Sergey I.
Kislyak, the Russian ambassador to the United States. President Barack
Obama was imposing new sanctions on Russia and expelling 35 diplomats
after the election meddling.

The day after the sanctions were announced, Mr. Putin said Russia would
not retaliate in kind, as has been the custom in the long, tortured
history of Russian-American relations, instead waiting for a new
administration that he assumed would be friendlier.

Inside the Obama administration, officials were stunned. Mr. Trump
publicly welcomed the decision. ``Great move on delay (by V. Putin),''
he wrote on Twitter. ``I always knew he was very smart!''

Around the same time, Obama advisers heard separately from the F.B.I.
about Mr. Flynn's conversation with Mr. Kislyak, whose calls were
routinely monitored by American intelligence agencies that track Russian
diplomats. The Obama advisers grew suspicious that perhaps there had
been a secret deal between the incoming team and Moscow, which could
violate the rarely enforced, two-century-old
\href{https://www.nytimes.com/2017/02/14/us/politics/logan-act-flynn.html}{Logan
Act} barring private citizens from negotiating with foreign powers in
disputes with the United States.

The Obama officials asked the F.B.I. if a quid pro quo had been
discussed on the call, and the answer came back no, according to one of
the officials, who like others asked not to be named discussing delicate
communications. The topic of sanctions came up, they were told, but
there was no deal.

\includegraphics{https://static01.nyt.com/images/2017/02/15/us/15flynn/15flynn-videoSixteenByNineJumbo1600-v2.jpg}

On Jan. 12, David Ignatius, a columnist for The Washington Post,
reported that Mr. Flynn had called Mr. Kislyak, setting off news media
interest in what was said. Mr. Spicer, then the spokesman for Mr.
Trump's transition team, went to Mr. Flynn, who he said told him that
sanctions had not come up during the call. Briefing reporters the next
day, Mr. Spicer repeated the misinformation, saying that the
conversation had ``never touched on the sanctions.''

Mr. Flynn told the same thing to Mr. Pence and Reince Priebus, the
incoming White House chief of staff, who were scheduled to go on the
Sunday talk shows and expected that they would be asked about the
matter, according to the two men. On Jan. 15, Mr. Pence went on ``Face
the Nation'' on CBS and on ``Fox News Sunday'' and repeated that
sanctions had not been discussed, while Mr. Priebus said much the same
on ``Meet the Press'' on NBC.

The topic came up again after Mr. Trump and his team moved into the
White House. At his first full briefing on Jan. 23, Mr. Spicer said that
Mr. Flynn's conversation had touched on only four subjects, none of them
sanctions. That caught the attention of the F.B.I. and the Justice
Department.

Sally Q. Yates, an Obama appointee held over as acting attorney general
until Mr. Trump's choice was confirmed, concluded that the disparity
between what was said on the call and what Mr. Flynn had evidently told
the vice president and others about it might make the new national
security adviser vulnerable to blackmail. When foreign governments hold
information that could prove embarrassing, it is considered a potential
leverage point.

Soon after the Jan. 23 briefing, James B. Comey, the F.B.I. director,
sent agents to interview Mr. Flynn. If he told the agents what he said
publicly for more than a week after that interview --- that his
conversations with the ambassador had been innocuous and did not involve
sanctions --- then he could face legal trouble. If the authorities
concluded that he knowingly lied to the F.B.I., it could expose him to a
felony charge.

\href{https://www.nytimes.com/interactive/2017/02/13/us/politics/document-Michael-Flynn-Resignation-Letter.html}{}

\includegraphics{https://static01.nyt.com/images/2017/02/13/us/politics/image-Michael-Flynn-Resignation-Letter/image-Michael-Flynn-Resignation-Letter-thumbLarge.gif}

\hypertarget{michael-flynns-resignation-letter}{%
\subsection{Michael Flynn's Resignation
Letter}\label{michael-flynns-resignation-letter}}

Michael T. Flynn, under scrutiny for his communication with Russia,
resigned as President Trump's national security adviser late Monday.

It was not clear whether Mr. Flynn had a lawyer for his interview or
whether anyone at the White House knew the interview was happening. But
they knew afterward because Ms. Yates, with the support of Mr. Comey,
reached out to Donald F. McGahn II, the new White House counsel, on Jan.
26 to give him what Mr. Spicer called a ``heads up'' about the
discrepancy.

Mr. Trump was told ``immediately,'' Mr. Spicer said, and directed Mr.
McGahn to look into the matter. After an ``extensive review'' that
lasted several days, Mr. McGahn concluded that nothing in the
conversation had violated federal law, Mr. Spicer said.

But the president then set out to determine whether he could still trust
Mr. Flynn. Mr. Spicer said Mr. Flynn stuck to his original account,
making matters worse.

``We got to a point not based on a legal issue, but based on a trust
issue, with the level of trust between the president and General Flynn
had eroded to the point where he felt he had to make a change,'' Mr.
Spicer said. ``The president was very concerned that General Flynn had
misled the vice president and others.''

Asked if Mr. Trump had instructed Mr. Flynn to talk about sanctions with
Mr. Kislyak, Mr. Spicer said, ``No, absolutely not.'' Asked if Mr. Trump
knew that the issue had come up before the Justice Department told the
White House, Mr. Spicer said, ``No, he was not aware.''

Image

Vice President Mike Pence, left, learned that Mr. Flynn had misled him
about a call to the Russian ambassador after reading news media reports
last week.Credit...Gabriella Demczuk for The New York Times

Mr. Spicer emphasized that there was ``nothing wrong'' with Mr. Flynn's
talking with representatives of other countries to prepare for the new
president taking office, and that, in fact, Mr. Trump wanted him to.

By that point, Mr. Trump's relationship with Mr. Flynn had grown more
awkward. One person close to the president, who asked to remain
anonymous to describe private discussions, said Mr. Trump had been
``uncomfortable'' with Mr. Flynn for weeks. Jared Kushner, the
president's son-in-law and senior adviser, had expressed concern about
Mr. Flynn's appointment even before the inauguration, according to
another person briefed on the discussions.

Mr. Trump's views were coming around to the same point. ``What he knew
was that Flynn was too much about Flynn, versus Mattis,'' the person
close to the president said. Defense Secretary Jim Mattis was seen as
deferential to the chain of command. ``He loves Mattis because Mattis is
respectful and self-confident.''

Another key figure with growing concerns about Mr. Flynn was Stephen K.
Bannon, the president's chief strategist whom Mr. Flynn perceived as a
rival for control over national security. Mr. Trump began asking Mr.
Mattis about two weeks ago for suggestions of possible replacements for
Mr. Flynn. The defense secretary recommended retired Vice Adm.
\href{https://www.nytimes.com/2017/02/14/us/politics/robert-harward-national-security-adviser.html}{Robert
S. Harward}. Mr. Bannon reached out to Mr. Harward last week, two senior
officials said.

The situation escalated late Thursday when word reached the White House
that The Washington Post was reporting that the transcript of Mr.
Flynn's call showed that he had discussed sanctions, contrary to his
assurances to Mr. Pence and others.

White House officials confronted Mr. Flynn, who only then said that it
was possible they had come up, but that he did not remember. ``His story
remained the same until that night,'' Mr. Spicer said. ``That's when his
response changed.''

That was also when Mr. Pence first learned that the Justice Department
had proof that Mr. Flynn had not told the truth and had warned the White
House two weeks earlier, according to Marc Lotter, his spokesman. ``He
did an inquiry based on those media accounts,'' Mr. Lotter added,
without elaborating.

Another person who speaks frequently with him said Mr. Pence went
``ballistic,'' or at least what qualifies as ballistic for the
coolheaded vice president.

Mr. Pence, Mr. Priebus and Mr. Bannon urged Mr. Trump to fire the
national security adviser, according to officials, but the president
could not bring himself to do it, in part for fear of losing face. When
a reporter on Air Force One heading to Florida on Friday asked him about
The Post's report, Mr. Trump said he had not read it. ``I don't know
about that,'' he said. ``I haven't seen it.''

As late as Monday, he was sticking by Mr. Flynn. He sent his counselor,
Kellyanne Conway, to tell a television interviewer that he had ``full
confidence'' in Mr. Flynn. And Mr. Flynn phoned a reporter for The Daily
Caller on Monday to say the president had ``expressed confidence'' in
him and urged him to ``go out and talk more.''

In that interview, posted on Tuesday, Mr. Flynn said he had discussed
the Russian diplomats' expulsion with Mr. Kislyak. ``It wasn't about
sanctions,'' he said. ``It was about the 35 guys who were thrown out.''
Mr. Flynn added: ``It was basically, `Look, I know this happened. We'll
review everything.' I never said anything such as, `We're going to
review sanctions,' or anything like that.''

Either way, it was too late. When the matter came to overshadow the
president's glitch-free meeting with Prime Minister Justin Trudeau of
Canada and word arrived of another Post article on Ms. Yates's warning
to the White House, Mr. Trump ordered an end to the situation. ``He made
a determination late in the day,'' Mr. Spicer said, ``and he executed on
it.''

Advertisement

\protect\hyperlink{after-bottom}{Continue reading the main story}

\hypertarget{site-index}{%
\subsection{Site Index}\label{site-index}}

\hypertarget{site-information-navigation}{%
\subsection{Site Information
Navigation}\label{site-information-navigation}}

\begin{itemize}
\tightlist
\item
  \href{https://help.nytimes.com/hc/en-us/articles/115014792127-Copyright-notice}{©~2020~The
  New York Times Company}
\end{itemize}

\begin{itemize}
\tightlist
\item
  \href{https://www.nytco.com/}{NYTCo}
\item
  \href{https://help.nytimes.com/hc/en-us/articles/115015385887-Contact-Us}{Contact
  Us}
\item
  \href{https://www.nytco.com/careers/}{Work with us}
\item
  \href{https://nytmediakit.com/}{Advertise}
\item
  \href{http://www.tbrandstudio.com/}{T Brand Studio}
\item
  \href{https://www.nytimes.com/privacy/cookie-policy\#how-do-i-manage-trackers}{Your
  Ad Choices}
\item
  \href{https://www.nytimes.com/privacy}{Privacy}
\item
  \href{https://help.nytimes.com/hc/en-us/articles/115014893428-Terms-of-service}{Terms
  of Service}
\item
  \href{https://help.nytimes.com/hc/en-us/articles/115014893968-Terms-of-sale}{Terms
  of Sale}
\item
  \href{https://spiderbites.nytimes.com}{Site Map}
\item
  \href{https://help.nytimes.com/hc/en-us}{Help}
\item
  \href{https://www.nytimes.com/subscription?campaignId=37WXW}{Subscriptions}
\end{itemize}
