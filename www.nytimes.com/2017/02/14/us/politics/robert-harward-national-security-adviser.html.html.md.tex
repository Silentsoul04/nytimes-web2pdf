Sections

SEARCH

\protect\hyperlink{site-content}{Skip to
content}\protect\hyperlink{site-index}{Skip to site index}

\href{https://www.nytimes.com/section/politics}{Politics}

\href{https://myaccount.nytimes.com/auth/login?response_type=cookie\&client_id=vi}{}

\href{https://www.nytimes.com/section/todayspaper}{Today's Paper}

\href{/section/politics}{Politics}\textbar{}Ex-Admiral and Member of
Navy SEALs Is Top Choice to Replace Flynn

\url{https://nyti.ms/2lh6EnP}

\begin{itemize}
\item
\item
\item
\item
\item
\end{itemize}

Advertisement

\protect\hyperlink{after-top}{Continue reading the main story}

Supported by

\protect\hyperlink{after-sponsor}{Continue reading the main story}

\hypertarget{ex-admiral-and-member-of-navy-seals-is-top-choice-to-replace-flynn}{%
\section{Ex-Admiral and Member of Navy SEALs Is Top Choice to Replace
Flynn}\label{ex-admiral-and-member-of-navy-seals-is-top-choice-to-replace-flynn}}

\includegraphics{https://static01.nyt.com/images/2017/02/15/us/15harward/15harward-articleLarge.jpg?quality=75\&auto=webp\&disable=upscale}

By \href{https://www.nytimes.com/by/julie-hirschfeld-davis}{Julie
Hirschfeld Davis}

\begin{itemize}
\item
  Feb. 14, 2017
\item
  \begin{itemize}
  \item
  \item
  \item
  \item
  \item
  \end{itemize}
\end{itemize}

WASHINGTON --- Robert S. Harward, the retired vice admiral who is
President Trump's top choice to replace his ousted national security
adviser, is a hard-charging member of the Navy SEALs who rose through
the ranks to top military positions and is close with Jim Mattis, the
new secretary of defense.

Mr. Harward, 60, is a former deputy commander of the United States
Central Command, the military's busiest with management of the wars in
Iraq and Afghanistan, and served on the National Security Council under
President George W. Bush, where he was responsible for counterterrorism
issues. He is currently a top executive at Lockheed Martin, the weapons
and aerospace company, overseeing business with the United Arab
Emirates.

Mr. Harward's career has closely tracked that of Mr. Mattis, from the
time the two worked together in Afghanistan after the Sept. 11 terrorist
attacks of 2001 to his tour at Central Command from 2011 to 2013, which
included an assignment heading detainee operations in Kabul.

``He has faced down and defeated the world's most ruthless and deadly
enemies, and he has done all that by Mattis's side,'' said Fran
Townsend, Mr. Bush's former homeland security adviser, for whom Mr.
Harward worked from 2003 to 2005. ``He has been in tougher knife fights
than this, and won.''

Still, it is not clear whether Mr. Harward would be willing to surrender
his lucrative position and comfortable existence in Abu Dhabi to step
into the tumult of the Trump White House. The new administration has
been troubled by an unusual level of infighting, disorganization and
grievance --- including
\href{http://www.nytimes.com/2017/02/12/us/politics/national-security-council-turmoil.html}{within
the ranks} of the National Security Council --- capped off on Monday by
the
\href{https://www.nytimes.com/2017/02/13/us/politics/donald-trump-national-security-adviser-michael-flynn.html?hp\&action=click\&pgtype=Homepage\&clickSource=story-heading\&module=a-lede-package-region\&region=top-news\&WT.nav=top-news\&_r=0}{resignation}
of Michael T. Flynn as national security adviser.

Raised in prerevolutionary Tehran, Mr. Harward was known to startle his
Afghan counterparts during his tours there by conversing with them
fluently in Farsi, which is similar to their native Dari. Trained as an
elite Special Operations officer, Mr. Harward is also known for his
bravado and obsession with physical fitness. As the head of detainee
operations in Afghanistan, he would lead weekly hikes in the mountains
outside Kabul, outpacing colleagues who were 20 years younger, and has
been known to challenge them to push-up contests that left them
vomiting.

Mr. Harward looks the part of a battle-hardened military man, a crucial
factor for a president who has made clear that he considers appearance
an important indicator of a job candidate's suitability for a role.

With his bald head, ice-blue eyes and long scar of mysterious provenance
down his cheek, Mr. Harward has the bearing of an officer who once
carried out risky secret operations. But he is also an effective inside
player, according to people who know him, having worked for some of the
military's top policy leaders and at the White House.

James G. Stavridis, a retired admiral and former NATO commander, said
Mr. Harward was ``someone who will find a way to succeed no matter how
daunting the task.''

``I have known him well for two decades, and have boundless admiration
for his ingenuity, integrity and ability to navigate choppy seas ---
both operationally in the field and in the battlefield of Washington,
D.C.,'' said Mr. Stavridis, currently dean of the Fletcher School of Law
and Diplomacy at Tufts University.

``The real question,'' he added, ``is whether he wants to take the
job.''

Some question whether Mr. Harward's decades of military experience are
the right preparation for a senior policy-making post.

In a
\href{https://twitter.com/MaxBoot/status/831522476140195843}{Twitter
post} Tuesday,
\href{http://www.cfr.org/experts/national-security-warfare-terrorism/max-boot/b5641}{Max
Boot}, a senior fellow in national security studies at the Council on
Foreign Relations, called Mr. Harward ``a great SEAL,'' but said it was
``not clear that running detainee ops in Afghanistan or being No 2 at
Centcom is right background for this.''

Friends say Mr. Harward has experience with high-stakes military special
operations, but also in navigating the arcane world of the National
Security Council, which is charged with synthesizing recommendations
from national security and intelligence agencies and advising the
president on policy. The process frequently involves managing turf
battles and balancing competing interests.

``He'll bring a buffering calm and balance, as well as his using his
previous experience at the N.S.C.,'' said Douglas H. Wise, a former
deputy director at the Defense Intelligence Agency.

Mr. Mattis is widely seen as a force for steadiness within the Trump
administration, and some Republicans who have expressed misgivings about
the president's policies and his attitude toward national security
matters have looked to his defense secretary as an island of reliability
in a sea of unpredictability.

Senator John McCain, Republican of Arizona and chairman of the Armed
Services Committee, said on Tuesday that Mr. Trump should name a new
national security adviser ``who is empowered by clear lines of authority
and responsibility and possesses the skills and experience necessary to
organize the national security system across our government.''

Mr. McCain said he looked forward ``to working with the president's
administration, especially Secretary Mattis, to defend the nation and
support our military service members.''

If he is chosen for the post and accepts, Mr. Harward would be reunited
with his old boss and mentor, Mr. Mattis. But he would also have to
contend with Mr. Trump's inner circle, populated by political advisers
with whom Mr. Harward is not familiar, including Steven K. Bannon, the
chief White House strategist.

In an executive order last month --- which Mr. Trump later
\href{http://www.nytimes.com/2017/02/05/us/politics/trump-white-house-aides-strategy.html}{complained
privately} that he had not been fully briefed on --- the president
placed Mr. Bannon on the principals committee of the National Security
Council, giving a political adviser a position of parity with the
secretaries of state and defense, and with the national security
adviser.

Two former national security officials who have worked closely with Mr.
Harward said he would be unlikely to take the position without strong
assurances from Mr. Trump and his team that the council would not be
driven by partisan considerations on national security policy, and that
he would have the autonomy to provide principled counsel. They spoke on
the condition of anonymity because they were not authorized to speak on
Mr. Harward's behalf.

Mr. Harward's name surfaced briefly in 2015 in connection to
\href{http://www.nytimes.com/2015/04/24/us/david-petraeus-to-be-sentenced-in-leak-investigation.html}{the
scandal} involving David H. Petraeus, the former general who was forced
to resign as director of the Central Intelligence Agency after admitting
that he had provided classified information to his lover, and who is now
also said to be in the running to be Mr. Trump's national security
adviser.

Jill Kelley, a Tampa socialite who had befriended Mr. Petraeus and then
become a target of threatening emails from Mr. Petraeus' lover, had also
written gushing notes to Mr. Harward and Mr. Mattis.

``You ROCK!!!'' Ms. Kelley wrote to Mr. Harward in 2012 regarding his
dealings with foreign heads of state at a social gathering, according to
emails obtained by The Washington Post. ``YOU ROCK MORE!,'' Mr. Harward
replied.

There was no evidence of impropriety in the friendly correspondence,
which would have been routine between top military commanders and civic
leaders in the communities in which they were stationed.

Advertisement

\protect\hyperlink{after-bottom}{Continue reading the main story}

\hypertarget{site-index}{%
\subsection{Site Index}\label{site-index}}

\hypertarget{site-information-navigation}{%
\subsection{Site Information
Navigation}\label{site-information-navigation}}

\begin{itemize}
\tightlist
\item
  \href{https://help.nytimes.com/hc/en-us/articles/115014792127-Copyright-notice}{©~2020~The
  New York Times Company}
\end{itemize}

\begin{itemize}
\tightlist
\item
  \href{https://www.nytco.com/}{NYTCo}
\item
  \href{https://help.nytimes.com/hc/en-us/articles/115015385887-Contact-Us}{Contact
  Us}
\item
  \href{https://www.nytco.com/careers/}{Work with us}
\item
  \href{https://nytmediakit.com/}{Advertise}
\item
  \href{http://www.tbrandstudio.com/}{T Brand Studio}
\item
  \href{https://www.nytimes.com/privacy/cookie-policy\#how-do-i-manage-trackers}{Your
  Ad Choices}
\item
  \href{https://www.nytimes.com/privacy}{Privacy}
\item
  \href{https://help.nytimes.com/hc/en-us/articles/115014893428-Terms-of-service}{Terms
  of Service}
\item
  \href{https://help.nytimes.com/hc/en-us/articles/115014893968-Terms-of-sale}{Terms
  of Sale}
\item
  \href{https://spiderbites.nytimes.com}{Site Map}
\item
  \href{https://help.nytimes.com/hc/en-us}{Help}
\item
  \href{https://www.nytimes.com/subscription?campaignId=37WXW}{Subscriptions}
\end{itemize}
