\href{/pages/business/yourtaxes/index.html}{Your
Taxes}\textbar{}Republicans Agree on Cutting Taxes, but Not on How to Do
It

\url{https://nyti.ms/2lmY3Ap}

\begin{itemize}
\item
\item
\item
\item
\item
\end{itemize}

\includegraphics{https://static01.nyt.com/images/2017/02/19/business/19OVERHAUL/19OVERHAUL-articleLarge.png?quality=75\&auto=webp\&disable=upscale}

Sections

\protect\hyperlink{site-content}{Skip to
content}\protect\hyperlink{site-index}{Skip to site index}

\hypertarget{republicans-agree-on-cutting-taxes-but-not-on-how-to-do-it}{%
\section{Republicans Agree on Cutting Taxes, but Not on How to Do
It}\label{republicans-agree-on-cutting-taxes-but-not-on-how-to-do-it}}

Republicans control the White House and Congress, and remain united in
their desire to change the tax code. But they are far from united in how
to achieve that goal.

Credit...Ryan Huddle

Supported by

\protect\hyperlink{after-sponsor}{Continue reading the main story}

By \href{http://www.nytimes.com/by/patricia-cohen}{Patricia Cohen}

\begin{itemize}
\item
  Feb. 16, 2017
\item
  \begin{itemize}
  \item
  \item
  \item
  \item
  \item
  \end{itemize}
\end{itemize}

For anyone who thinks that Republican control of the presidency, the
Senate and the House of Representatives means that an overhaul of the
tax code will zip through at bobsled speed, there is an important phrase
to remember:

\href{https://www.nytimes.com/2014/02/27/us/politics/sweeping-tax-overhaul-plan-would-bring-big-changes.html?_r=0}{``Blah,
blah, blah, blah.''}

That was the reaction of the Republican speaker of the House, John A.
Boehner of Ohio, in 2014 to a sweeping overhaul of the federal tax code
proposed by the chairman of the Ways and Means Committee --- and a
member of his own party.

Republicans remain united in their desire to cut taxes, simplify the
system and recover some of the
\href{https://www.nytimes.com/2016/11/06/your-money/strategies-corporate-cash-repatriation-bipartisan-consensuss.html}{\$2
trillion in untaxed profits that American companies have stashed
overseas,} out of the reach of the Internal Revenue Service. And the
possibility of a once-in-a-generation makeover of the rules governing
both business and personal taxes provides powerful motivation. Still,
despite their grip on power, they are far from united in how to
accomplish those goals.

``Just because there's an `R' in the White House, an `R' in the House
and an `R' in the Senate doesn't mean they're from the same alphabet,''
said Douglas Duncan, the chief economist at Fannie Mae.

This time around, the speaker, Paul D. Ryan of Wisconsin, and the Ways
and Means chairman, Kevin Brady of Texas, have both taken ownership of
\href{https://abetterway.speaker.gov/_assets/pdf/ABetterWay-Tax-PolicyPaper.pdf}{a
House blueprint}for transforming the tax code, currently a masterpiece
of complexity, inefficiencies and inequities.

But several Republicans have misgivings about specific provisions, not
to mention the cost. As for the White House, the most predictable
response is unpredictability.

The president said his administration would release a
\href{https://www.nytimes.com/video/business/dealbook/100000004922130/trump-on-phenomenal-tax-plan.html}{``phenomenal''
tax plan} in a couple of weeks. Mr. Brady has indicated that he wants to
introduce comprehensive legislation by March and pass a bill before the
August recess --- perhaps suggesting the existence of an alternate
congressional universe with no gridlock.

Yet the controversy that has already exploded over how to tax imports
and exports ---
\href{https://www.nytimes.com/reuters/2017/02/03/us/politics/03reuters-usa-tax-border.html}{the
border adjustment tax}, for short --- offers a taste of how difficult
getting to ``yes'' will be on any comprehensive bill.

As it turns out, Republicans at 1600 Pennsylvania Avenue and in the
House are not that far apart when it comes to the individual side of the
tax code. They agree on the goal of lowering income tax rates for
everybody. The House blueprint reduces the number of brackets to three
from seven and drops the top rate on the highest incomes to 33 percent
from 39.6 percent. The middle range rate would be 25 percent and the
lower rate, 12 percent.

Taxes on capital gains (which tend to affect higher-income households
the most) would be slashed, and the alternative minimum tax, devised to
limit the use of loopholes, would be eliminated. The estate tax, which
affects the top 0.2 percent sliver of households, would disappear, as
would most itemized deductions, including one that allows the deduction
of state and local taxes --- dear to anyone who lives where there is a
state or city income tax.

But bigger standard deductions would take away much of the sting,
reducing the incentives for most taxpayers to itemize at all. (Instead
of 30 percent, just 5 percent would still find it worth their while.)
During his campaign, President Trump talked about capping deductions for
mortgage interest and charitable donations, but the House plan promises
to retain them (with vague references to making them more efficient).
Given how popular these two costly deductions are, however, there is not
much appetite in Congress for going after them.

Other proposals, like tax credits for children, have also been widely
embraced.

What could trip up the reform of personal taxes is the promise by the
Treasury secretary, Steven T. Mnuchin, of
\href{https://www.nytimes.com/2017/02/09/business/economy/mnuchin-rule-tax-cut.html}{``no
absolute tax cut for the upper class.''} Republicans have repeatedly
characterized their plan as a big cut for the middle class, but
independent analyses have concluded that the wealthy are clearly the
biggest winners.

``Three-quarters of the tax cuts would benefit the top 1 percent of
taxpayers,'' the nonpartisan Tax Policy Center said of the House plan.
The highest-income households --- the top 0.1 percent --- would get ``an
average tax cut of about \$1.3 million, 16.9 percent of after-tax
income.'' By contrast, those in the middle fifth of incomes would get a
0.5 percent tax cut, worth about \$260.

Such problems are trifles when compared with divisions caused by
proposed alterations to corporate and business taxes.

On Friday, Senator Mitch McConnell, the majority leader and a Republican
from Kentucky, announced that Congress would move ahead on health care
and tax reform without the Democrats.

That is not a surprising decision these days, when ``bipartisanship''
has turned into a punch line and party-line voting has become the rule
rather than the exception. Still, this strategy has risks .

Laws that lack a single vote from the opposition party are by nature
more controversial, unstable and vulnerable to political attack: The
Affordable Care Act was passed that way. And when the budget is
involved, bills passed with slim majorities can be subject to time
limits. (When you hear Washington insiders talking about ``budget
reconciliation,'' that's part of what they are referring to.)

\includegraphics{https://static01.nyt.com/images/2017/02/19/business/19OVERHAUL2/19OVERHAUL2-articleLarge.jpg?quality=75\&auto=webp\&disable=upscale}

Although Republicans unilaterally pushed through personal tax cuts when
they controlled Congress in the past, the majority's fast-track weapon
has never been used with comprehensive tax reform. Tax provisions can
dictate many businesses' long-term planning and investment decisions,
and no one wants to restructure his or her operation only to find that
the rule book will be thrown away in a decade.

Senator Orrin G. Hatch of Utah, chairman of the Finance Committee and
now\href{http://www.cnn.com/2017/02/09/politics/orrin-hatch-senate-honor/}{the
longest-serving Republican in the chamber's history}, is also keenly
attuned to the Senate's traditions of decorum and the veneer of
statesmanship.

Earlier this month,
he\href{https://www.finance.senate.gov/chairmans-news/hatch-unveils-finance-committee-agenda-for-115th-congress-}{told
the U.S. Chamber of Commerce,} ``My preference would be to find a
bipartisan path forward.'' He added, ``Historically speaking, that's
what's worked best.''

The last major
\href{http://www.nytimes.com/2012/11/23/business/a-starting-point-for-tax-reform-what-reagan-did.html}{revamping
of the tax code, in 1986,} was led by a Republican president (Ronald
Reagan), a Democratic speaker (Tip O'Neill of Massachusetts) and a
Democratic senator (Bill Bradley of New Jersey). Now, the question is
whether Republicans can do it on their own. Republicans, after all, have
just a two-vote margin in the Senate. And some of the most bitter tax
face-offs do not necessarily fall along partisan or ideological lines.

Ideas intended to discourage companies from going abroad to hire cheaper
foreign workers instead of Americans have
split\href{http://fortune.com/2017/02/01/walmart-target-border-tax-trump/}{importers
and exporters}. The border adjustment tax --- a complex arrangement that
would have the effect of adding a 20 percent tax to any imports sold in
the United States and nothing on exports --- has
\href{http://www.americanmadecoalition.org/?utm_source=google\&utm_campaign=border\%20tax\&utm_medium=search}{drawn
support from} agriculture, manufacturing and technology businesses like
General Electric, Boeing, Dow Chemical and Oracle. The companies are
exporters and argue that the move would ``support American jobs and
American-made products.''

Sectors that rely heavily on imports --- retailers, foreign carmakers
and oil interests led by the billionaires Charles G. and David H. Koch
--- have
\href{https://www.keepamericaaffordable.com/content.aspx?page=About}{declared
war on the idea}, saying it would hurt consumers by raising prices.

\href{http://www.forbes.com/forbes/welcome/?toURL=http://www.forbes.com/sites/kellyphillipserb/2017/01/27/lindsey-graham-tweets-concern-over-trumps-mucho-sad-border-tax-proposal/\&refURL=https://www.google.com/\&referrer=https://www.google.com/}{``Mucho
sad,''} Senator Lindsey Graham, Republican of South Carolina, warned on
Twitter, writing that an import tax was ``a big-time bad idea'' because
it would raise the cost of Mexican products.

Other proposals --- like ending the business deduction for interest
payments and allowing immediate
\href{https://www.irs.gov/businesses/small-businesses-self-employed/a-brief-overview-of-depreciation}{depreciation}
--- could set heavily leveraged industries like farming, finance and
real estate (Mr. Trump's backyard) against those that primarily invest
in equipment that loses value over time, like manufacturers.

``There's a lot of proposals floating around, and it's clear there are
going to be some winners and losers,'' said Ken Esch, a partner at the
consulting and accounting firm PricewaterhouseCoopers. ``Many companies
are struggling to figure if they're going to be a winner or a loser.''

Then there is the cost. Supporters of the House plan say it aims to
bring in the same revenue as the existing code. But independent analyses
by both the
\href{http://www.taxpolicycenter.org/publications/analysis-house-gop-tax-plan}{Tax
Policy Center} and the conservative-leaning
\href{https://taxfoundation.org/details-and-analysis-2016-house-republican-tax-reform-plan/}{Tax
Foundation} point to a shortfall that could run into the trillions over
the next decade.

That red ink will irk Republican deficit hawks.

``If we don't get any Democrats on board, we'll basically need universal
Republican support to pass anything,'' Mr. Hatch said. ``That means the
Senate will have to work through its own tax reform process if we're
going to have any chance of succeeding.''

Several Democrats --- most important, the Senate's minority leader,
Chuck Schumer of New York --- also want to lower corporate tax rates
from their current 35 percent. (They would demand concessions in return,
potentially putting other treasured Republican proposals at risk.)

With each passing day, partisan rancor has ratcheted up, including in
the Finance Committee. When the Democrats boycotted a committee hearing
on Mr. Mnuchin, Mr. Hatch suspended the rule that required a least one
member of the minority party to be in attendance. ``I don't care what
they want at this point,'' Mr. Hatch said, calling the Democrats
``idiots.''

Ignoring Democrats still doesn't patch up the divisions among
Republicans. Mr. Hatch also told the Chamber of Commerce that ``no one
should expect the Senate to simply take up and pass a House tax reform
bill,'' signaling to Mr. Ryan and Mr. Brady that their blueprint was
unlikely to emerge intact.

At the same time, Mr. Hatch seemed to deliver a subtle dig to Mr. Trump,
who has boasted that when it comes to solving the country's problems,
\href{http://video.cnbc.com/gallery/?video=3000536691}{``I alone can fix
it.''}

In regard to lackluster economic growth, a crippling national debt and
many other problems, Mr. Hatch told the chamber, ``I certainly don't
believe they can all be fixed by a single party or president.''

The veteran Republican's warning, however, is unlikely to stop the
Republicans from trying.

Advertisement

\protect\hyperlink{after-bottom}{Continue reading the main story}

\hypertarget{site-index}{%
\subsection{Site Index}\label{site-index}}

\hypertarget{site-information-navigation}{%
\subsection{Site Information
Navigation}\label{site-information-navigation}}

\begin{itemize}
\tightlist
\item
  \href{https://help.nytimes.com/hc/en-us/articles/115014792127-Copyright-notice}{©~2020~The
  New York Times Company}
\end{itemize}

\begin{itemize}
\tightlist
\item
  \href{https://www.nytco.com/}{NYTCo}
\item
  \href{https://help.nytimes.com/hc/en-us/articles/115015385887-Contact-Us}{Contact
  Us}
\item
  \href{https://www.nytco.com/careers/}{Work with us}
\item
  \href{https://nytmediakit.com/}{Advertise}
\item
  \href{http://www.tbrandstudio.com/}{T Brand Studio}
\item
  \href{https://www.nytimes.com/privacy/cookie-policy\#how-do-i-manage-trackers}{Your
  Ad Choices}
\item
  \href{https://www.nytimes.com/privacy}{Privacy}
\item
  \href{https://help.nytimes.com/hc/en-us/articles/115014893428-Terms-of-service}{Terms
  of Service}
\item
  \href{https://help.nytimes.com/hc/en-us/articles/115014893968-Terms-of-sale}{Terms
  of Sale}
\item
  \href{https://spiderbites.nytimes.com}{Site Map}
\item
  \href{https://help.nytimes.com/hc/en-us}{Help}
\item
  \href{https://www.nytimes.com/subscription?campaignId=37WXW}{Subscriptions}
\end{itemize}
