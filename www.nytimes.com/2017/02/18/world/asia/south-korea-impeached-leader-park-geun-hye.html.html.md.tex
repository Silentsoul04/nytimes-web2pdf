Sections

SEARCH

\protect\hyperlink{site-content}{Skip to
content}\protect\hyperlink{site-index}{Skip to site index}

\href{https://www.nytimes.com/section/world/asia}{Asia Pacific}

\href{https://myaccount.nytimes.com/auth/login?response_type=cookie\&client_id=vi}{}

\href{https://www.nytimes.com/section/todayspaper}{Today's Paper}

\href{/section/world/asia}{Asia Pacific}\textbar{}South Korean Right Is
Frozen, as Impeached Leader's Loyalists Won't Let Go

\url{https://nyti.ms/2luwepB}

\begin{itemize}
\item
\item
\item
\item
\item
\end{itemize}

Advertisement

\protect\hyperlink{after-top}{Continue reading the main story}

Supported by

\protect\hyperlink{after-sponsor}{Continue reading the main story}

\hypertarget{south-korean-right-is-frozen-as-impeached-leaders-loyalists-wont-let-go}{%
\section{South Korean Right Is Frozen, as Impeached Leader's Loyalists
Won't Let
Go}\label{south-korean-right-is-frozen-as-impeached-leaders-loyalists-wont-let-go}}

\includegraphics{https://static01.nyt.com/images/2017/02/19/world/19korea1/15conservatives-1-articleLarge.jpg?quality=75\&auto=webp\&disable=upscale}

By \href{http://www.nytimes.com/by/choe-sang-hun}{Choe Sang-Hun}

\begin{itemize}
\item
  Feb. 18, 2017
\item
  \begin{itemize}
  \item
  \item
  \item
  \item
  \item
  \end{itemize}
\end{itemize}

SEOUL, South Korea --- Chung Kwang-yong choked up in describing how much
he missed Park Geun-hye, the South Korean president, who has been
cloistered in her official residence since her impeachment in December
on corruption charges.

``Dear President Park Geun-hye, please come out. We miss you so much,''
Mr. Chung said before a large crowd that rallied in central Seoul on a
recent Saturday to demand her immediate reinstatement. ``You have done
nothing wrong.''

Few South Korean leaders have ever been as besieged as Ms. Park, whose
presidential powers have been suspended since the National Assembly
\href{https://www.nytimes.com/2016/12/09/world/asia/south-korea-president-park-geun-hye-impeached.html}{voted
to impeach her} on Dec. 9. Recent surveys have ranked her as one of the
least popular presidents ever, with about 80 percent of respondents
wanting her removed from office.

But Ms. Park still commands an almost cultlike following among people
like Mr. Chung, and that lingering devotion is fragmenting the country's
conservative bloc as it struggles to find a viable replacement candidate
in an election that could take place as early as May.

In South Korean elections, conservatives have usually rallied around a
single presidential candidate, propelling them to victory as progressive
voters split among rival opposition candidates. Now, it is the divisions
in the conservative ranks that are providing the progressives with an
opportunity to return to power after a decade away from the presidential
palace.

In December, a group of conservative lawmakers, disillusioned by the
accusations of corruption and abuse of power made against Ms. Park,
joined the opposition in passing the bill to impeach her. They then
bolted from her governing Saenuri Party and created the new Bareun
Party.

Its approval rating plunging and desperate to rebrand itself, Saenuri
changed its name Monday to the Liberty Korea Party. But it has been
unable to redefine its relationship with Ms. Park.

Many conservatives, including some Liberty Korea lawmakers, want to
distance themselves from Ms. Park and regroup around a new leader to
have a fighting chance against the progressive opposition leader Moon
Jae-in in the election.

\includegraphics{https://static01.nyt.com/images/2017/02/19/world/19korea2/15conservatives-2-articleLarge.jpg?quality=75\&auto=webp\&disable=upscale}

But other party members and right-wing groups, like Mr. Chung's
Parksamo, or ``People Who Love Park Geun-hye,'' want Ms. Park, 65, to
finish the final year of her five-year term.

These groups have organized increasingly large rallies in central Seoul
in recent weeks, calling any conservative politician who turns against
Ms. Park a ``betrayer.'' Their rallies attract not only Park loyalists
but also older Koreans who share, if not their loyalty to Ms. Park,
their belief that the country's progressive opposition is too
sympathetic toward North Korea to be trusted.

``I have always voted conservative and always will, as long as North
Korea exists,'' said Kim Myong-soo, 65, whose family fled Communist rule
in the North during the Korean War. ``But frankly, if an election is
held now, I don't know which conservative candidate to vote for. There
is none who can win.''

To her critics, Ms. Park has come to symbolize everything wrong with the
country's conservative elite, as she stands accused of conspiring with a
longtime friend to extort tens of millions of dollars from big
businesses in return for political favors. Prosecutors also accuse her
of ordering a government
\href{https://www.nytimes.com/2017/01/12/world/asia/south-korea-president-park-blacklist-artists.html}{blacklisting
of artists}, writers and movie directors deemed progressive, blocking
them from government support programs.

But according to flag-waving, military uniform-clad conservatives at the
rallies, Ms. Park was an innocent victim of a ``sedition'' masterminded
by politically biased prosecutors, a ``fake-news media'' and
``Communists.''

Their rallies feature military parade songs and chants for Ms. Park to
``mobilize the military'' to regain power, an echo of how her father,
the dictator Park Chung-hee, took power in a military coup in 1961. Some
participants carried signs that said: ``It's O.K. to kill Commies!''

``They want to overthrow the government and establish a pro-North Korean
regime,'' Kim Chul-hong, a theology professor and vocal supporter of Ms.
Park, said of the opposition during a news conference this month.
``South Korea is now in a civil war.''

Few South Koreans believe that another military coup is possible. Mr.
Chung's Parksamo is considered by many to be little more than a
personality cult and an overzealous ideological outlier. (The group
recently helped pay for a large newspaper advertisement that said:
``Please don't cry, Park Geun-hye!'')

But its Red-baiting campaign, a traditional vote-gathering tool for
South Korean conservatives, has intensified as the country's
Constitutional Court prepares to rule on
\href{https://www.nytimes.com/2017/01/03/world/asia/south-korea-president-impeachment-trial.html}{whether
to reinstate Ms. Park} or formally end her presidency.

Local news media have reported that a ruling could come as early as next
month, and some protesters like Mr. Chung said they would ``rebel'' if
the court did not reinstate Ms. Park.

Image

Yoo Seong-min, a leader of the new Bareun Party, said he expected
conservatives to eventually form an alliance.Credit...Yonhap, via
European Pressphoto Agency

Alarmed by the conservative pushback, pro-impeachment groups have begun
rebuilding their weekend rallies, which once attracted more than a
million but shrank after Ms. Park's impeachment. They urged the
Constitutional Court to oust Ms. Park quickly to end the political
uncertainty.

As her supporters' rallies have grown bigger, Ms. Park has become
increasingly defiant. Once tearfully apologetic about her scandal, she
has recently begun claiming that she is a victim of a plot by her
enemies to ``frame'' her with ``a mountain of lies.''

``My heart aches when I think of those who come out to the streets to
defend free democracy and rule of law,'' she said in an interview with a
right-wing podcast station late last month, referring to her supporters.

If the court decides to end Ms. Park's presidency, it will leave the
fissured conservative camp little time to regroup. By law, an election
to select her successor
\href{https://www.nytimes.com/2016/11/27/world/asia/impeaching-south-korea-president.html?_r=0}{must
be held within 60 days}.

Many conservatives
\href{https://www.nytimes.com/2017/01/25/world/asia/ban-ki-moon-south-korea-president.html}{had
looked to Ban Ki-moon}, the former United Nations secretary general, to
become their candidate. But Mr. Ban
\href{https://www.nytimes.com/2017/02/01/world/asia/ban-ki-moon-president-south-korea.html}{pulled
out of the race} this month after he failed to narrow the gap in polls
with Mr. Moon, the progressive.

Highlighting the fractures among conservatives, as many as 10
politicians affiliated with the two conservative parties have declared
their presidential ambitions, but none has a popularity rating higher
than the low single digits.

In Myung-jin, the leader of Liberty Korea, said he favored Prime
Minister Hwang Kyo-ahn, who is serving as acting president, as his
party's candidate. Mr. Hwang, who has no party affiliation, is the only
conservative with a popularity rating of more than 10 percent, ranking
third in recent surveys after Mr. Moon and a provincial governor, Ahn
Hee-jung, also a progressive.

But Mr. Hwang has not committed to running yet, and critics deride his
close ties to Ms. Park. Mr. Hwang also has never served in the military,
an often fatal strike against men seeking the presidency in South Korea,
which is technically still at war with North Korea.

Yoo Seong-min, a leader of the Bareun Party and the second-most popular
conservative candidate after Mr. Hwang, said conservatives would
eventually form an alliance.

``The South Korean conservatives face a crisis they had never
experienced before,'' said Mr. Yoo, who supported Ms. Park's
impeachment. ``But once the Constitutional Court rules, conservatives
will settle for the verdict, whatever it may be, and will start
unifying.''

Advertisement

\protect\hyperlink{after-bottom}{Continue reading the main story}

\hypertarget{site-index}{%
\subsection{Site Index}\label{site-index}}

\hypertarget{site-information-navigation}{%
\subsection{Site Information
Navigation}\label{site-information-navigation}}

\begin{itemize}
\tightlist
\item
  \href{https://help.nytimes.com/hc/en-us/articles/115014792127-Copyright-notice}{©~2020~The
  New York Times Company}
\end{itemize}

\begin{itemize}
\tightlist
\item
  \href{https://www.nytco.com/}{NYTCo}
\item
  \href{https://help.nytimes.com/hc/en-us/articles/115015385887-Contact-Us}{Contact
  Us}
\item
  \href{https://www.nytco.com/careers/}{Work with us}
\item
  \href{https://nytmediakit.com/}{Advertise}
\item
  \href{http://www.tbrandstudio.com/}{T Brand Studio}
\item
  \href{https://www.nytimes.com/privacy/cookie-policy\#how-do-i-manage-trackers}{Your
  Ad Choices}
\item
  \href{https://www.nytimes.com/privacy}{Privacy}
\item
  \href{https://help.nytimes.com/hc/en-us/articles/115014893428-Terms-of-service}{Terms
  of Service}
\item
  \href{https://help.nytimes.com/hc/en-us/articles/115014893968-Terms-of-sale}{Terms
  of Sale}
\item
  \href{https://spiderbites.nytimes.com}{Site Map}
\item
  \href{https://help.nytimes.com/hc/en-us}{Help}
\item
  \href{https://www.nytimes.com/subscription?campaignId=37WXW}{Subscriptions}
\end{itemize}
