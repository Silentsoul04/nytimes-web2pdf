Sections

SEARCH

\protect\hyperlink{site-content}{Skip to
content}\protect\hyperlink{site-index}{Skip to site index}

\href{https://www.nytimes.com/section/world/americas}{Americas}

\href{https://myaccount.nytimes.com/auth/login?response_type=cookie\&client_id=vi}{}

\href{https://www.nytimes.com/section/todayspaper}{Today's Paper}

\href{/section/world/americas}{Americas}\textbar{}Rex Tillerson Arrives
in Mexico Facing Twin Threats to Relations

\url{https://nyti.ms/2m9GO6A}

\begin{itemize}
\item
\item
\item
\item
\item
\end{itemize}

Advertisement

\protect\hyperlink{after-top}{Continue reading the main story}

Supported by

\protect\hyperlink{after-sponsor}{Continue reading the main story}

\hypertarget{rex-tillerson-arrives-in-mexico-facing-twin-threats-to-relations}{%
\section{Rex Tillerson Arrives in Mexico Facing Twin Threats to
Relations}\label{rex-tillerson-arrives-in-mexico-facing-twin-threats-to-relations}}

\includegraphics{https://static01.nyt.com/images/2017/02/23/world/23Mexico1/23Mexico1-articleInline.jpg?quality=75\&auto=webp\&disable=upscale}

By \href{http://www.nytimes.com/by/gardiner-harris}{Gardiner Harris} and
\href{http://www.nytimes.com/by/kirk-semple}{Kirk Semple}

\begin{itemize}
\item
  Feb. 22, 2017
\item
  \begin{itemize}
  \item
  \item
  \item
  \item
  \item
  \end{itemize}
\end{itemize}

\href{https://www.nytimes.com/es/2017/02/23/entre-amenazas-por-deportaciones-y-el-muro-rex-tillerson-visita-mexico/}{Leer
en español}

MEXICO CITY --- The Trump administration calls the visit a step toward
mutual understanding, a way to move the relationship forward.

But as Secretary of State Rex W. Tillerson arrived in Mexico on
Wednesday, twin threats hung over the frayed relationship between the
two nations: President Trump's
\href{https://www.nytimes.com/2017/02/21/us/politics/dhs-immigration-trump.html?_r=0}{new
orders to round up and deport} immigrants who are in the United States
illegally, and a separate effort to take a hard look at all American aid
to Mexico, possibly using it to pay for a border wall instead.

By Friday, American officials are required to finish calculating all the
money and grants that the United States provides to Mexico, a task that
Mr. Trump first demanded in the
\href{https://www.nytimes.com/2017/01/25/us/politics/refugees-immigrants-wall-trump.html}{executive
order he signed last month} directing the construction of a border wall.

The Trump administration, which set the Friday deadline in an internal
State Department memo this month, has not explicitly said why it ordered
the review. But its inclusion in the executive order mandating that a
wall be built suggests that Mr. Trump has linked the two issues --- and
may be looking for more leverage in negotiations with Mexico.

The timing adds to the
\href{https://www.nytimes.com/2017/01/26/world/americas/mexico-pena-nieto-donald-trump.html}{deep
tensions between the two countries}. Mr. Tillerson, the top American
official to visit Mexico since Mr. Trump's inauguration, arrived with
John F. Kelly, the secretary of Homeland Security, only a day after the
Trump administration released documents ordering a crackdown on
immigration in the United States.

Newspapers here have described the Trump administration's new
deportation policies in apocalyptic terms, saying in some cases that
they represented ``war'' on the millions of Mexicans in the United
States.

Mexico's foreign minister, Luis Videgaray, said Wednesday that the
package of immigration directives is ``something that, without doubt,
worries all of us Mexicans'' and will be ``the first point on the
agenda'' when he meets with his American counterpart.

Nothing about the meetings this week is likely to be easy, for either
side. Mr. Tillerson met with Mr. Trump in the Oval Office just before
his departure, but there have been
\href{https://www.nytimes.com/2017/02/15/world/europe/germany-rex-tillerson.html}{few
signs that the secretary of state plays a pivotal role} in setting the
administration's foreign policy agenda. He has largely been absent from
important White House meetings with foreign leaders, has uttered few
words in public since his confirmation and was not even allowed his
choice of a top deputy.

Instead, Mr. Tillerson has largely been assigned to tidy up the
confrontations Mr. Trump has had with longtime allies. Last week,
\href{https://www.nytimes.com/2017/02/15/world/europe/jim-mattis-nato-trump.html}{he
went to Germany} to reassure his European counterparts that Mr. Trump
valued NATO and the European Union, despite the president's
\href{https://www.nytimes.com/2017/01/15/world/europe/donald-trump-nato.html}{statements
to the contrary}.

Mr. Trump's rift with Mexico is not only deeper, but also is likely to
worsen.

For the Mexicans, the meetings will be an important step toward deciding
whether to battle or appease an administration that has consistently
excoriated their country.

It is a choice leaders around the world are grappling with. Japan's
prime minister, Shinzo Abe, courted and flattered Mr. Trump,
\href{https://www.nytimes.com/2017/02/13/world/asia/trump-japan-shinzo-abe.html}{seeming
to succeed} in reversing decades of Mr. Trump's criticisms of Japan.
China's president, Xi Jinping, seemed to publicly ignore Mr. Trump for
weeks before
\href{https://www.nytimes.com/2017/02/09/world/asia/donald-trump-china-xi-jinping-letter.html}{Mr.
Trump reversed himself} on questioning the ``One China'' policy that
nation holds so dear.

The Mexicans seem to be using a combination of outreach and complaint
that has so far proved ineffective, as the twin blows this week
demonstrated.

\includegraphics{https://static01.nyt.com/images/2017/02/23/world/23Mexico2/23Mexico2-articleInline.jpg?quality=75\&auto=webp\&disable=upscale}

The review of American aid due on Friday, for instance, is likely to
highlight about \$1 billion that has been allocated but not yet spent
under the Merida Initiative, a bilateral partnership begun in 2007 that
focuses on fighting organized criminal groups, re-engineering the
judicial system, modernizing the border between the two countries and
strengthening civil society groups.

Most of the American foreign aid to Mexico is provided under the aegis
of the initiative. Since it was signed, Congress has appropriated more
than \$2.8 billion for those programs, of which at least \$1.6 billion
has been delivered to Mexico, according to a report in January by the
Congressional Research Service.

Some Mexican officials and civil society leaders have been alarmed by
the suggestion that Mr. Trump could cut assistance to key initiatives
that bolster community-building and the rule of law to help pay for a
wall that many on both sides of the border say would probably fail in
stop the flow of illegal drugs, weapons and immigration.

But perhaps even more worrisome to Mexico is the threat to deport to
millions of its citizens who, with settled lives and jobs in the United
States, provide most of the nearly \$25 billion in remittance payments
to Mexican families every year.

The Trump administration also said it planned to detain non-Mexicans who
had crossed the southwest border with the United States and send them
back to Mexico to await the outcome of their deportation proceedings.

Though American officials said that this measure would be done only
after discussions with the Mexican government, Mexican officials and
legal experts rejected the idea as a violation of Mexican law and
international accords.

At an event in Mexico City on Wednesday, Mr. Videgaray said, ``I want to
make clear, and in the most emphatic way, that the Mexican government
and the Mexican people do not have to accept orders that a government
seeks to impose unilaterally on another.''

That threat to saddle Mexico with other countries' migrants is one
reason Mexican officials could emerge from their meetings this week
deciding to fight rather than appease the Americans. For months, in the
face of a hostile stance by Mr. Trump, President Enrique Peña Nieto
adopted a largely conciliatory strategy, not allowing himself to be
provoked by the American president despite increasing calls from the
Mexican electorate for a tougher stance.

Then last month,
\href{https://www.nytimes.com/2017/01/26/world/americas/mexico-pena-nieto-donald-trump.html}{Mr.
Peña Nieto canceled his meeting}in Washington with Mr. Trump, prompting
a rare uptick in his
\href{https://www.nytimes.com/2017/01/25/world/americas/trump-mexico-border-wall.html}{woeful
approval ratings}. The Trump administration responded by accusing Mexico
of burdening the United States with undocumented immigrants, criminals
and a trade deficit.

If relations worsen significantly, Mr. Peña Nieto could make life
difficult for Mr. Trump by limiting or stopping Mexican cooperation on a
range of fronts, analysts said.

Beyond the billions in trade, the two countries cooperate on many
security issues. Mexico could limit its sharing of information, like the
lists of passengers aboard international flights, and loosen visa rules
for citizens of nations suspected of harboring terrorists.

It could also limit its cooperation in the realm of migration by, for
example, detaining fewer unauthorized migrants traveling from Central
America and allowing more people to reach American borders. Mexico,
which has long provided a militarized buffer against the flow of drugs
to the United States, could also relax its prosecution of the drug war.

Mexico ``has many cards to play,'' said Carlos Heredia, a professor at
CIDE, a Mexican research center. ``Mexico must approach these
conversations knowing the issue of bilateral cooperation and security is
deeply intertwined with immigration issues and regional, commercial
integration.''

Advertisement

\protect\hyperlink{after-bottom}{Continue reading the main story}

\hypertarget{site-index}{%
\subsection{Site Index}\label{site-index}}

\hypertarget{site-information-navigation}{%
\subsection{Site Information
Navigation}\label{site-information-navigation}}

\begin{itemize}
\tightlist
\item
  \href{https://help.nytimes.com/hc/en-us/articles/115014792127-Copyright-notice}{©~2020~The
  New York Times Company}
\end{itemize}

\begin{itemize}
\tightlist
\item
  \href{https://www.nytco.com/}{NYTCo}
\item
  \href{https://help.nytimes.com/hc/en-us/articles/115015385887-Contact-Us}{Contact
  Us}
\item
  \href{https://www.nytco.com/careers/}{Work with us}
\item
  \href{https://nytmediakit.com/}{Advertise}
\item
  \href{http://www.tbrandstudio.com/}{T Brand Studio}
\item
  \href{https://www.nytimes.com/privacy/cookie-policy\#how-do-i-manage-trackers}{Your
  Ad Choices}
\item
  \href{https://www.nytimes.com/privacy}{Privacy}
\item
  \href{https://help.nytimes.com/hc/en-us/articles/115014893428-Terms-of-service}{Terms
  of Service}
\item
  \href{https://help.nytimes.com/hc/en-us/articles/115014893968-Terms-of-sale}{Terms
  of Sale}
\item
  \href{https://spiderbites.nytimes.com}{Site Map}
\item
  \href{https://help.nytimes.com/hc/en-us}{Help}
\item
  \href{https://www.nytimes.com/subscription?campaignId=37WXW}{Subscriptions}
\end{itemize}
