Sections

SEARCH

\protect\hyperlink{site-content}{Skip to
content}\protect\hyperlink{site-index}{Skip to site index}

\href{https://www.nytimes.com/section/world/asia}{Asia Pacific}

\href{https://myaccount.nytimes.com/auth/login?response_type=cookie\&client_id=vi}{}

\href{https://www.nytimes.com/section/todayspaper}{Today's Paper}

\href{/section/world/asia}{Asia Pacific}\textbar{}In Kim Jong-nam's
Death, North Korea Lets Loose a Weapon of Mass Destruction

\url{https://nyti.ms/2mfuGkg}

\begin{itemize}
\item
\item
\item
\item
\item
\end{itemize}

Advertisement

\protect\hyperlink{after-top}{Continue reading the main story}

Supported by

\protect\hyperlink{after-sponsor}{Continue reading the main story}

\hypertarget{in-kim-jong-nams-death-north-korea-lets-loose-a-weapon-of-mass-destruction}{%
\section{In Kim Jong-nam's Death, North Korea Lets Loose a Weapon of
Mass
Destruction}\label{in-kim-jong-nams-death-north-korea-lets-loose-a-weapon-of-mass-destruction}}

\includegraphics{https://static01.nyt.com/images/2017/02/25/world/25kim-1/25kim-1-articleLarge.jpg?quality=75\&auto=webp\&disable=upscale}

By \href{https://www.nytimes.com/by/richard-c-paddock}{Richard C.
Paddock}, \href{http://www.nytimes.com/by/choe-sang-hun}{Choe Sang-Hun}
and \href{http://www.nytimes.com/by/nicholas-wade}{Nicholas Wade}

\begin{itemize}
\item
  Feb. 24, 2017
\item
  \begin{itemize}
  \item
  \item
  \item
  \item
  \item
  \end{itemize}
\end{itemize}

KUALA LUMPUR, Malaysia --- For years, North Korea has rattled the world
with its nuclear tests and its threats to visit a nuclear holocaust upon
the United States. Now, the finding by the Malaysian police that Kim
Jong-nam was assassinated with
\href{https://www.nytimes.com/2017/02/24/world/asia/vx-nerve-agent-kim-jong-nam.html?rref=collection\%2Fsectioncollection\%2Fasia\&action=click\&contentCollection=asia\&region=stream\&module=stream_unit\&version=latest\&contentPlacement=5\&pgtype=sectionfront}{VX
nerve agent} is a stark reminder of the North's lesser-known weapons of
mass destruction: a stockpile of chemical and biological weapons.

Mr. Kim, the estranged elder brother of North Korea's leader, Kim
Jong-un, was killed on Feb. 13 when
\href{https://www.nytimes.com/2017/02/22/world/asia/kim-jong-nam-assassination-korea-malaysia.html?rref=collection\%2Fsectioncollection\%2Fasia\&action=click\&contentCollection=asia\&region=stream\&module=stream_unit\&version=latest\&contentPlacement=6\&pgtype=sectionfront}{two
women rubbed his face with the nerve agent} at Kuala Lumpur
International Airport, the police said on Friday.

If North Korean citizens were behind the killing, as Malaysian officials
suggest, the use of VX raises several questions: Was the North Korean
government using the attack to signal to the world its fearsome arsenal
of such dangerous weapons? Or was the toxin simply an attempt to avoid
detection in carrying out a brazen killing at one of the world's busiest
airports?

``By using VX in an international airport in the heart of Asia, North
Korea has sent a very clear message to the world that it will strike its
enemies anywhere in the world,'' said Rohan Gunaratna, an expert on
terrorism at the S. Rajaratnam School of International Studies in
Singapore. ``It also demonstrates the North Korean response in the event
of an attack against North Korea.''

North Korea's nuclear program has long been the most urgent concern of
the United States and its allies, and the now-dormant six-party talks to
curb the program did not address chemical and biological weapons.

``The reported use of VX reminds us that not only is the North's
nuclear-missile threat serious but so are its asymmetric threats,
including biochemical weapons and cyber that are all part of the
regime's W.M.D. tool kit,'' said Duyeon Kim, a Seoul-based nonresident
fellow at Georgetown University's Institute for the Study of Diplomacy.

South Korea's Foreign Ministry issued a statement on Friday expressing
``shock'' at the use of a chemical weapon and vowed to work with the
international society to deal ``strongly'' with the violation of the
Chemical Weapons Convention.

The deadly use of a chemical weapon banned by international conventions
in such a public manner could strengthen calls for the United States to
put North Korea back on a list of terrorism-sponsoring countries,
analysts said.

Image

Security footage shows Kim Jong-nam being accosted by a woman in a white
shirt at Kuala Lumpur International Airport.Credit...Fuji TV, via
Reuters

The North was first put on the terrorist list after its bombing of a
South Korean airliner near Myanmar in 1987, killing all 115 people
onboard. But the United States delisted the country in 2008 as part of
an agreement aimed at ending North Korea's nuclear programs --- a deal
that has since disintegrated with the North's subsequent missile and
nuclear weapons tests.

After his announcement that
\href{https://www.nytimes.com/2017/02/23/world/asia/kim-jong-nam-vx-nerve-agent-.html?rref=collection\%2Fsectioncollection\%2Fasia\&action=click\&contentCollection=asia\&region=stream\&module=stream_unit\&version=latest\&contentPlacement=7\&pgtype=sectionfront}{Mr.
Kim had been killed by VX nerve agent}, Khalid Abu Bakar, the inspector
general of the Malaysian police, said on Friday that small amounts of
the poison could have been brought into the country without being
discovered.

``If the amount of the chemical brought in was small, it would be
difficult for us to detect,'' Mr. Khalid told reporters.

The airport terminal, which handles more than two million passengers a
month, will be decontaminated despite the passage of time since the
killing, he said.

Two women have been arrested in the killing, one from Indonesia and the
other from Vietnam. Their defenders say they were duped into carrying
out the attack and thought it was a prank, but Mr. Khalid said they had
trained for it and practiced at two major shopping malls. The women used
their bare hands to apply the poison on Mr. Kim's face and washed them
immediately afterward, he said.

One drop of VX, or about 10 milligrams, can be fatal. But the attackers
could have used a safety-enhancing battlefield form of the agent. Known
as VX2, it is divided into two compounds that are harmless individually
but become lethal when mixed together.

Each component also could have been made in slow-release form, as is
done with many drugs.

If Mr. Kim's two assassins had each applied one component of VX, this
would explain why two people were needed, how they survived the attack,
and perhaps why it took 15 minutes or more for Mr. Kim to die.

``Use of a binary nerve agent lends itself to this method and allows for
a potentially highly targeted hit,'' said Vipin Narang, an associate
professor of political science at the Massachusetts Institute of
Technology who has two degrees in chemical engineering.

The woman who applied the second compound would have risked exposing
herself to the first component, which could explain why, as Mr. Khalid
said on Friday, one of the women became ill and began vomiting after the
attack.

This scenario raises the possibility that Mr. Kim could have saved his
own life by immediately washing his face rather than going to airport
staff members, as he did, to report the attack.

\includegraphics{https://static01.nyt.com/images/2017/02/25/world/25kim-3/25kim-3-articleLarge.jpg?quality=75\&auto=webp\&disable=upscale}

Professor Narang said it was clear that North Korea wanted the West to
know what it is capable of --- but without causing mass casualties.

``They wanted everyone, especially the U.S., to know it was VX and that
they can make it or have it,'' he said. ``Doing it publicly but not
killing anyone else is a pretty good way to reveal that capability and
deterrent.''

In 2014, the South Korean Defense Ministry said the North had stockpiled
2,500 to 5,000 tons of chemical weapons and had a capacity to produce a
variety of biological weapons.

Kim Jong-un has a history of resorting to extreme measures against his
enemies.

Since taking power after the death of his father, Kim Jong-il, in 2011,
he has executed at least 140 senior officials, sometimes killing them
with antiaircraft machine guns and even incinerating some of their
bodies with flamethrowers, according to the Institute for National
Security Strategy, a think tank affiliated with South Korea's National
Intelligence Service. Such measures were designed as a warning to
others, South Korean officials said.

Lee Byong-chul, a nonproliferation expert at the Institute for Peace and
Cooperation in Seoul, said the use of VX nerve agent highlights the
proliferation threat posed by North Korea, noting that it has been
accused of providing chemical weapons technology to Syria since the
1990s.

Shipments of gas masks, gas detectors and other protective gear bound
for Syria from North Korea were intercepted in 2009 and 2013.

If confirmed, Mr. Lee said, the use of VX nerve agent by North Korea
will very likely weaken the Trump administration's appetite for
reopening nuclear disarmament talks, especially after its recent test of
what it called a new type of intermediate-range ballistic missile.

China has been the most vocal proponent of new negotiations, but its
relations with North Korea have
\href{https://www.nytimes.com/2017/02/24/world/asia/china-north-korea-relations-kim-jong-un.html}{deteriorated
sharply}. Pyongyang criticized Beijing this week as ``dancing to the
tune of the U.S.''

Steve Vickers, a security consultant based in Hong Kong, said that Mr.
Kim's assassination would be seen as a further insult to China, which
had protected him for years by allowing him to live in the Chinese
territory of Macau.

``This is clearly an embarrassment for the Chinese state security and to
a lesser extent to the Malaysian government,'' Mr. Vickers said.

Advertisement

\protect\hyperlink{after-bottom}{Continue reading the main story}

\hypertarget{site-index}{%
\subsection{Site Index}\label{site-index}}

\hypertarget{site-information-navigation}{%
\subsection{Site Information
Navigation}\label{site-information-navigation}}

\begin{itemize}
\tightlist
\item
  \href{https://help.nytimes.com/hc/en-us/articles/115014792127-Copyright-notice}{©~2020~The
  New York Times Company}
\end{itemize}

\begin{itemize}
\tightlist
\item
  \href{https://www.nytco.com/}{NYTCo}
\item
  \href{https://help.nytimes.com/hc/en-us/articles/115015385887-Contact-Us}{Contact
  Us}
\item
  \href{https://www.nytco.com/careers/}{Work with us}
\item
  \href{https://nytmediakit.com/}{Advertise}
\item
  \href{http://www.tbrandstudio.com/}{T Brand Studio}
\item
  \href{https://www.nytimes.com/privacy/cookie-policy\#how-do-i-manage-trackers}{Your
  Ad Choices}
\item
  \href{https://www.nytimes.com/privacy}{Privacy}
\item
  \href{https://help.nytimes.com/hc/en-us/articles/115014893428-Terms-of-service}{Terms
  of Service}
\item
  \href{https://help.nytimes.com/hc/en-us/articles/115014893968-Terms-of-sale}{Terms
  of Sale}
\item
  \href{https://spiderbites.nytimes.com}{Site Map}
\item
  \href{https://help.nytimes.com/hc/en-us}{Help}
\item
  \href{https://www.nytimes.com/subscription?campaignId=37WXW}{Subscriptions}
\end{itemize}
