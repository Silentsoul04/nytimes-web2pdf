Sections

SEARCH

\protect\hyperlink{site-content}{Skip to
content}\protect\hyperlink{site-index}{Skip to site index}

\href{https://www.nytimes.com/section/world/europe}{Europe}

\href{https://myaccount.nytimes.com/auth/login?response_type=cookie\&client_id=vi}{}

\href{https://www.nytimes.com/section/todayspaper}{Today's Paper}

\href{/section/world/europe}{Europe}\textbar{}Geert Wilders, Reclusive
Provocateur, Rises Before Dutch Vote

\url{https://nyti.ms/2muI6pw}

\begin{itemize}
\item
\item
\item
\item
\item
\item
\end{itemize}

Advertisement

\protect\hyperlink{after-top}{Continue reading the main story}

Supported by

\protect\hyperlink{after-sponsor}{Continue reading the main story}

\hypertarget{geert-wilders-reclusive-provocateur-rises-before-dutch-vote}{%
\section{Geert Wilders, Reclusive Provocateur, Rises Before Dutch
Vote}\label{geert-wilders-reclusive-provocateur-rises-before-dutch-vote}}

\includegraphics{https://static01.nyt.com/images/2017/02/28/world/28Wilders01/28Wilders01-articleInline-v2.jpg?quality=75\&auto=webp\&disable=upscale}

By \href{https://www.nytimes.com/by/alissa-j-rubin}{Alissa J. Rubin}

\begin{itemize}
\item
  Feb. 27, 2017
\item
  \begin{itemize}
  \item
  \item
  \item
  \item
  \item
  \item
  \end{itemize}
\end{itemize}

SPIJKENISSE, the Netherlands --- He wants to end immigration from Muslim
countries, tax head scarves and ban the Quran. He is partly of
Indonesian heritage, and dyes his hair bright blond. He is omnipresent
on social media but lives as a political phantom under police
protection, rarely campaigning in person and reportedly sleeping in a
different location every night.

He has structured his party so that he is the only official, giving him
the liberty to remain, above all things, in complete control, and a
provocateur and an uncompromising verbal bomb thrower.

Geert Wilders, far-right icon, is one of Europe's unusual politicians,
not least because he comes from the Netherlands, one of Europe's most
socially liberal countries, with a centuries-long tradition of promoting
religious tolerance and welcoming immigrants.

How he and his party fare in the March 15 elections could well signal
how the far right will do in pivotal elections in France, Germany and
possibly Italy later this year, and ultimately determine the future of
the European Union. Mr. Wilders (pronounced VIL-ders) has promised to
demand a ``Nexit'' referendum on whether the Netherlands should follow
Britain's example and leave the union.

``The Netherlands is kind of a bellwether, a lot of trends manifest
themselves here first,'' said Hans Anker, a Dutch political strategist
who has worked both in the Netherlands and the United States.

``I wouldn't rule out that Wilders could be prime minister,'' he added.
``This one is fundamentally unpredictable.''

Remarkably, Mr. Wilders, 53, has managed to build a movement despite his
infrequent public appearances. Living under threat since the police
discovered plots against him in 2004 has turned him into a politician
ahead of his time, using the internet and later social media to talk to
voters without the filter of journalists.

It has proved a particularly effective means of reaching disillusioned
citizens. Other politicians have followed his lead but almost none have
done it as effectively, Dutch experts said.

\href{https://www.nytimes.com/interactive/2017/02/28/world/europe/netherlands-immigration-election.html}{}

\includegraphics{https://static01.nyt.com/images/2017/02/28/world/europe/28netherlands-callout/28netherlands-callout-thumbLarge.jpg}

\hypertarget{are-you-in-the-netherlands-share-your-thoughts-on-immigration}{%
\subsection{Are You in the Netherlands? Share Your Thoughts on
Immigration}\label{are-you-in-the-netherlands-share-your-thoughts-on-immigration}}

We want to hear from people in the Netherlands about how immigration has
affected their lives and views.

``He's the most strategic, smartest politician out there,'' said Sarah
de Lange, a political science professor at the University of Amsterdam.
``He's very skilled. He's a very good debater. He has media savvy.
Internationally, he's compared to Trump. But with Wilders every tweet is
thought through, calculated. With Trump it's emotional.''

Right now Mr. Wilders's party looks set to win more seats than any other
or to come in second. However, he has historically polled better before
elections than he has performed in them. Still, after pollsters
underestimated the likelihood of both Brexit and the victory of Donald
Trump last year, no one is relying on predictions.

But whether Mr. Wilders's party wins the most votes, or enters a
government, hardly matters. He has already succeeded in one of his main
ambitions --- to push politics in the Netherlands to the right and make
possible a conversation about shutting out immigrants and dismantling
the European Union that was unthinkable not long ago.

Mr. Wilders is close ideologically to Marine Le Pen of France, the
far-right National Front leader who is set to make it to a runoff in
presidential elections this spring. He was also close to Mr. Trump's
campaign, and is sometimes even called the ``Dutch Trump,'' though he
has a far longer political history and as many differences as
similarities.

Like Mr. Trump, Mr. Wilders is unafraid to say things in the most
direct, divisive, dismissive, and often disparaging and insulting of
ways. Similar to Mr. Trump, many of his supporters feel buoyed and
relieved that he is giving voice to what they cannot say, or feel they
are not supposed to say.

Last week Mr. Wilders
\href{https://www.nytimes.com/2017/02/18/world/europe/geert-wilders-netherlands-freedom-party-moroccan-immigrants.html?rref=collection\%2Ftimestopic\%2FWilders\%2C\%20Geert\&action=click\&contentCollection=timestopics\&region=stream\&module=stream_unit\&version=latest\&contentPlacement=1\&pgtype=collection\&_r=0}{delighted
in publicly referring to ``Moroccan scum''} before a gaggle of
reporters. He has called the hijab a ``useless piece of cloth.'' He has
been
\href{https://www.nytimes.com/2016/12/09/world/europe/geert-wilders-netherlands-trial.html}{convicted}
of inciting discrimination and insulting an ethnic group, but was let
off without a penalty.

The one time Mr. Wilders was in government, in 2010, he had an informal
liaison with the mainstream conservative party's coalition, but he
bolted when it wanted to cut back pension benefits. Those in his
parliamentary group are not technically members of his party, allowing
Mr. Wilders to entirely control his party's platform and
decision-making.

Being a party of one also allows him to avoid most campaign finance and
disclosure rules, leaving the sources of his money murky, though he
receives funding from at least one American conservative group.

\includegraphics{https://static01.nyt.com/images/2017/03/13/world/00wilders-video/00wilders-video-videoSixteenByNine3000.jpg}

Mr. Wilders describes himself as an outsider. Yet he is the
third-longest-sitting member of the Dutch Parliament and has spent his
life in politics since he was about 28.

In recent years, because of the apparent threats against him, Mr.
Wilders has become progressively more isolated. He sees his wife once or
twice a week and has cut off his brother, who disagrees with him
politically, the brother has said in media interviews. He maintains the
image of being present through carefully dispensing Twitter posts,
videos and television interviews. His rare public appearances guarantee
that every time he ventures out he attracts a media circus.

Last week, he suspended his campaign appearances altogether after
reports that a member of his police security detail was suspected of
leaking his movements to a Dutch-Moroccan criminal gang.

Still, he manages to travel to give speeches outside the Netherlands,
including at the Republican convention in Cleveland, where
\href{https://www.youtube.com/watch?v=5GJC17c2sfk}{he spoke} at the
``Milo Yiannopoulos Wake Up Party,'' a gathering of lesbians, gays,
bisexuals and transgender people for Mr. Trump.

He has also traveled on many occasions to Israel, for which he developed
a deep affection after spending months on a kibbutz as a young man. He
is described by political compatriots as friendly with Benjamin
Netanyahu, the right-wing Israeli prime minister.

Over time, Mr. Wilders's own positions have hardened, his colleagues
said. He arrived in The Hague, the Netherlands, in 1991-92 as a
parliamentary assistant in the mainstream conservative party then led by
Frits Bolkestein.

Today Mr. Bolkestein likens Mr. Wilders to ``the sorcerer's
apprentice,'' who, the story goes, uses one of his newly learned spells
to enchant a broom into washing the floor for him.

Soon the water is all over, and he realizes that he does not know how to
stop the broom. He tries splitting it in two with an ax, but then there
are two brooms, then four. ``The apprentice can't stop,'' Mr. Bolkestein
said.

\includegraphics{https://static01.nyt.com/images/2017/02/28/world/28Wilders03/28Wilders03-articleInline.jpg?quality=75\&auto=webp\&disable=upscale}

It is an apt description of Mr. Wilders, who sometimes seems to try to
outdo himself more for shock value and to grab attention than for
practical effect, particularly on immigration.

``In 2012 his position was no new mosques in the Netherlands; now it is
`close all the mosques,' '' said Michiel Servaes, a Labor Party member
in Parliament who has served with him. ``In 2012 it was limit asylum
seekers to 1,000 a year; now it's `no new asylum seekers.' ''

Yet Mr. Wilders's stands have brought the mainstream right to advocate
strict limits on aid for immigrants and helped spawn new small
right-wing parties, all with strong positions against immigration and in
support of stricter rules to push immigrants to accept Dutch culture,
Mr. Servaes said.

E. C. Hendriks, a political sociologist who is allied with a new
far-right party, the Forum for Democracy, says unease with immigration
and disillusionment with the European Union are rife. ``Certain groups
in Dutch society have had trouble integrating,'' he said.

Yet immigrants are costly for the Dutch since the country has a generous
social welfare system and pays for newcomers' education, health care,
housing and food.

``So if certain groups come and do not speak Dutch and do not share our
values, if they don't integrate, it's a bigger problem,'' he said.

The Netherlands, with its religious tolerance and relative prosperity
--- unemployment is among the lowest in Europe --- may seem an unlikely
place for the far right to take hold.

Yet it is another measure of Mr. Wilders's idiosyncratic appeal that in
a country with such an open lifestyle, some Dutch are turning to him to
safeguard their liberal social values.

Image

Dutch Party for Freedom supporters at a rally. Mr. Wilders has managed
to build a movement despite his rare public appearances.Credit...Sergey
Ponomarev for The New York Times

A longtime parliamentary colleague, Harry van Bommel, who came to office
in 1998, the same year as Mr. Wilders, said it was difficult to deny
that the political winds had shifted in his favor. ``In this country
there is an underestimation of the number of people who are afraid of
Islam,'' he said.

That turn was spurred in part by the sense of shock
\href{http://www.nytimes.com/2002/05/07/world/rightist-candidate-in-netherlands-is-slain-and-the-nation-is-stunned.html}{at
the assassination} in 2002 of Pim Fortuyn, a right-leaning politician,
and two years later, that of
\href{http://www.nytimes.com/2004/11/03/world/europe/dutch-filmmaker-an-islam-critic-is-killed.html?_r=0}{the
filmmaker Theo Van Gogh}, both of whom provoked and spoke out harshly
against immigrants.

It was against this backdrop that Mr. Wilders formed his own party and
began to find a wider audience. Today he has pockets of strength in
almost every part of the country.

As Mr. Wilders opened his campaign this month in one of his strongholds
in the Rotterdam suburb of Spijkenisse, supporters stood with arms
folded in the cold gray morning in the central square, as vendors hawked
fresh herring and the police and security guards tried to keep back a
media scrum.

Among the crowd was Ieg Van Haperen, 66, a former postal worker who
complained that prices had gone up but pensions had not. Like many
Dutch, she feels Mr. Wilders goes too far in his condemnations of Islam,
but ``the borders indeed have to close,'' she said.

In his hometown, Venlo, a stone's throw from the German border, where
locals speak a dialect all their own, Carnival was in full swing as Mr.
Wilders opened his campaign miles away.

There was speculation that Mr. Wilders might visit, though it seems he
has distanced himself from the one place where his Indonesian heritage
on his mother's side is generally known. His mother is half Indonesian;
her parents returned to Venlo around the time Holland gave up its
colonies in Indonesia, according to a Dutch anthropologist and
journalist, Lizzy van Leeuwen.

Few seemed even to have met the hometown boy in a place that remains a
seemingly tolerant bastion in perhaps a waning Dutch tradition.

``I asked myself what I would say if a journalist asked me if Geert
Wilders was welcome here,'' said Roel Versleijen, the president of
Jocus, the association that coordinates Venlo's celebration.

He was indeed asked, by this one.

``Well, everyone is welcome to celebrate,'' he said, ``but his political
views are not widely supported.''

Advertisement

\protect\hyperlink{after-bottom}{Continue reading the main story}

\hypertarget{site-index}{%
\subsection{Site Index}\label{site-index}}

\hypertarget{site-information-navigation}{%
\subsection{Site Information
Navigation}\label{site-information-navigation}}

\begin{itemize}
\tightlist
\item
  \href{https://help.nytimes.com/hc/en-us/articles/115014792127-Copyright-notice}{©~2020~The
  New York Times Company}
\end{itemize}

\begin{itemize}
\tightlist
\item
  \href{https://www.nytco.com/}{NYTCo}
\item
  \href{https://help.nytimes.com/hc/en-us/articles/115015385887-Contact-Us}{Contact
  Us}
\item
  \href{https://www.nytco.com/careers/}{Work with us}
\item
  \href{https://nytmediakit.com/}{Advertise}
\item
  \href{http://www.tbrandstudio.com/}{T Brand Studio}
\item
  \href{https://www.nytimes.com/privacy/cookie-policy\#how-do-i-manage-trackers}{Your
  Ad Choices}
\item
  \href{https://www.nytimes.com/privacy}{Privacy}
\item
  \href{https://help.nytimes.com/hc/en-us/articles/115014893428-Terms-of-service}{Terms
  of Service}
\item
  \href{https://help.nytimes.com/hc/en-us/articles/115014893968-Terms-of-sale}{Terms
  of Sale}
\item
  \href{https://spiderbites.nytimes.com}{Site Map}
\item
  \href{https://help.nytimes.com/hc/en-us}{Help}
\item
  \href{https://www.nytimes.com/subscription?campaignId=37WXW}{Subscriptions}
\end{itemize}
