Sections

SEARCH

\protect\hyperlink{site-content}{Skip to
content}\protect\hyperlink{site-index}{Skip to site index}

\href{https://www.nytimes.com/section/business/economy}{Economy}

\href{https://myaccount.nytimes.com/auth/login?response_type=cookie\&client_id=vi}{}

\href{https://www.nytimes.com/section/todayspaper}{Today's Paper}

\href{/section/business/economy}{Economy}\textbar{}Set to Lift Interest
Rate, Fed Embraces Investors' Optimism

\url{https://nyti.ms/2lDx1Rm}

\begin{itemize}
\item
\item
\item
\item
\item
\item
\end{itemize}

Advertisement

\protect\hyperlink{after-top}{Continue reading the main story}

Supported by

\protect\hyperlink{after-sponsor}{Continue reading the main story}

\hypertarget{set-to-lift-interest-rate-fed-embraces-investors-optimism}{%
\section{Set to Lift Interest Rate, Fed Embraces Investors'
Optimism}\label{set-to-lift-interest-rate-fed-embraces-investors-optimism}}

\includegraphics{https://static01.nyt.com/images/2017/03/04/business/04-SUB-FED/04-SUB-FED-videoSixteenByNineJumbo1600.jpg}

By \href{http://www.nytimes.com/by/binyamin-appelbaum}{Binyamin
Appelbaum}

\begin{itemize}
\item
  March 3, 2017
\item
  \begin{itemize}
  \item
  \item
  \item
  \item
  \item
  \item
  \end{itemize}
\end{itemize}

The Federal Reserve is poised to raise its benchmark interest rate in
mid-March, significantly sooner than investors had expected, as it moves
to keep pace with a wave of economic optimism that started with the
election of President Trump.

In an unusually clear statement about a pending decision, the Fed
chairwoman, Janet L. Yellen,
\href{https://www.federalreserve.gov/newsevents/speech/yellen20170303a.htm}{said
on Friday in Chicago} that the central bank was likely to act at its
next policy-making meeting --- barring any unpleasant economic
surprises.

Ms. Yellen added that the Fed still expected to raise rates twice more
later in the year, which she said would bring the benchmark rate close
to a level that the Fed regards as neutral, with low rates no longer
providing an inducement for borrowing and risk-taking. That outlook
signals that an end is finally in sight for the Fed's economic stimulus
campaign, devised during the depths of the financial crisis more than
eight years ago.

Stanley Fischer, the Fed's vice chairman, delivered the same message at
the same time at a conference in New York. ``We've seen a lot of
substantial change in a relatively short time,'' Mr. Fischer said of the
postelection shift in economic conditions. ``There is almost no economic
indicator that has come in badly in the last three months.''

Asked whether Fed officials were delivering a coordinated message, Mr.
Fischer responded wryly, ``If there has been a conscious effort, I'm
about to join it.''

The impending rate increase could heighten tensions with the White
House, which wants to stimulate growth by cutting taxes, reducing
regulation and increasing defense and infrastructure spending. Fed
officials have concluded the economy is already growing at something
close to the maximum sustainable pace, meaning faster growth should be
offset by faster rate increases.

Financial markets, however, are taking the prospect of higher rates in
stride. The Standard \& Poor's 500-stock index, which is up more than 11
percent since Election Day, ended trading on Friday mostly flat.

The prospective Fed move has modest short-term implications for
consumers. Interest rates on car loans and some kinds of credit card
debt will tick upward, but remain at low levels by historical standards.
Rates on 30-year mortgages are up by about half a percentage point over
the past year.

The broader consequences depend on the Fed's ability to raise interest
rates without slowing economic growth. The Fed's goal is to return rates
to a level that neither encourages nor impedes economic activity. Over
the past century, however, most of the central bank's attempts to strike
that balance have ended in economic recessions.

The American economy is in the midst of one of the longest expansions in
the nation's history, but it is also one of the weakest. The economy
expanded by 1.6 percent in 2016, compared with 2.6 percent in 2015,
\href{https://www.bea.gov/newsreleases/national/gdp/gdpnewsrelease.htm}{according
to the government's most recent estimate}.

Fed officials have concluded, however, that monetary policy cannot
deliver faster growth. The Fed's job is to minimize unemployment and
moderate inflation. The unemployment rate, at 4.8 percent in January, is
in a range Fed officials regard as healthy, and prices rose 1.9 percent
in the 12 months ending in January, the closest the Fed has come since
2012 to hitting its target of 2 percent annual inflation.

In December, the Fed
\href{https://www.nytimes.com/2016/12/14/business/economy/fed-interest-rates-janet-yellen.html}{raised
its benchmark rate} for just the second time since the financial crisis,
to a range of 0.5 percent to 0.75 percent, and predicted three increases
in 2017.

At the beginning of the week, however, Wall Street analysts and
investors did not expect the Fed to raise rates again any earlier than
June. The Fed issued a measured statement after its
\href{https://www.nytimes.com/2017/02/01/business/economy/fed-interest-rates-trump-yellen.html}{policy
meeting} in early February, and the meeting minutes, published three
weeks later, conveyed little sense of urgency.

Now, after a week of discussions, analysts regard a March increase as
highly likely.

Michael Feroli, the chief United States economist at JPMorgan Chase,
described the shift in Fed language as ``remarkably swift and
decisive.'' Investors put the chances at almost 80 percent in trading on
Friday, according to an analysis of asset prices
by\href{https://www.cmegroup.com/}{CME Group}.

Some Fed officials appear particularly focused on the rise of the stock
market. William C. Dudley, the president of the Federal Reserve Bank of
New York, who described markets as ``very buoyant'' on Tuesday, has said
in the past that if markets did not respond to rate increases, the Fed
might need to act more forcefully to tighten financial conditions.

It is also getting harder to dismiss the market's reaction to Mr.
Trump's victory as a bout of temporary euphoria. Mr. Fischer noted on
Friday that the stock market boom was creating wealth that people would
begin to spend.

Ms. Yellen pointed to an improvement in the global context. ``The
prospects for further moderate economic growth look encouraging,
particularly as risks emanating from abroad appear to have receded
somewhat,'' she said.

\href{https://www.nytimes.com/interactive/2015/12/16/upshot/fed-interest-rates-rube-goldberg-machine.html}{}

\includegraphics{https://static01.nyt.com/images/2015/12/16/upshot/fed-interest-rates-rube-goldberg-machine-1450212792328/fed-interest-rates-rube-goldberg-machine-1450212792328-videoLarge.jpg}

\hypertarget{what-happens-when-the-fed-raises-rates-in-one-rube-goldberg-machine}{%
\subsection{What Happens When the Fed Raises Rates, in One Rube Goldberg
Machine}\label{what-happens-when-the-fed-raises-rates-in-one-rube-goldberg-machine}}

Exactly seven years ago, the Federal Reserve cut interest rates to
almost zero in order to nurse the ailing economy back to health.
Recently it changed direction. This is how it works.

The shift in the Fed's language over the last week also may reflect a
recognition that market expectations were not keeping pace with the
Fed's evolving view of the economy. Ms. Yellen, in a February appearance
before Congress, hinted that the Fed might be providing a little too
much stimulus, describing the Fed's policy as ``accommodative.'' But at
the start of this week, investors still put a low probability on a March
increase.

Markets are wary of the Fed's flirtations with interest rate increases,
as the central bank in recent years has often found reasons for
last-minute postponements.

This time, the Fed chose to overwhelm any lingering doubts.

On Tuesday, Mr. Dudley
\href{http://www.cnbc.com/2017/02/28/case-for-fed-rate-hike-now-a-lot-more-compelling-dudley.html}{told
CNBC} that the case for a rate increase ``has become a lot more
compelling.''

On Wednesday, Lael Brainard, a Fed governor who has been one of the most
consistent supporters of raising interest rates slowly,
\href{https://www.federalreserve.gov/newsevents/speech/brainard20170301a.htm}{suggested}that
she too was ready to act.

``We are closing in on full employment, inflation is moving gradually
toward our target, foreign growth is on more solid footing, and risks to
the outlook are as close to balanced as they have been in some time,''
Ms. Brainard said at Harvard's Kennedy School of Government. ``Assuming
continued progress, it will likely be appropriate soon to remove
additional accommodation, continuing on a gradual path.''

Fed officials often bury their latest views on monetary policy at the
end of their speeches. Ms. Brainard's remarks came at the beginning, so
that no one missed the point.

On Thursday, another Fed governor, Jerome H. Powell, issued a similarly
blunt notice of intent in
\href{http://www.cnbc.com/2017/03/02/feds-powell-the-case-for-a-rate-increase-for-march-has-come-together.html}{an
interview} with CNBC. ``I think the case for a rate increase for March
has come together, and I think it's on the table for discussion,'' he
said.

Then came Friday, the last day on which Fed rules allowed officials to
comment on monetary policy before the March meeting, and Ms. Yellen
delivered the last word.

``At our meeting later this month,'' she said, ``the committee will
evaluate whether employment and inflation are continuing to evolve in
line with our expectations, in which case a further adjustment of the
federal funds rate would likely be appropriate.''

The committee is scheduled to meet in Washington on March 14 and 15.

Advertisement

\protect\hyperlink{after-bottom}{Continue reading the main story}

\hypertarget{site-index}{%
\subsection{Site Index}\label{site-index}}

\hypertarget{site-information-navigation}{%
\subsection{Site Information
Navigation}\label{site-information-navigation}}

\begin{itemize}
\tightlist
\item
  \href{https://help.nytimes.com/hc/en-us/articles/115014792127-Copyright-notice}{©~2020~The
  New York Times Company}
\end{itemize}

\begin{itemize}
\tightlist
\item
  \href{https://www.nytco.com/}{NYTCo}
\item
  \href{https://help.nytimes.com/hc/en-us/articles/115015385887-Contact-Us}{Contact
  Us}
\item
  \href{https://www.nytco.com/careers/}{Work with us}
\item
  \href{https://nytmediakit.com/}{Advertise}
\item
  \href{http://www.tbrandstudio.com/}{T Brand Studio}
\item
  \href{https://www.nytimes.com/privacy/cookie-policy\#how-do-i-manage-trackers}{Your
  Ad Choices}
\item
  \href{https://www.nytimes.com/privacy}{Privacy}
\item
  \href{https://help.nytimes.com/hc/en-us/articles/115014893428-Terms-of-service}{Terms
  of Service}
\item
  \href{https://help.nytimes.com/hc/en-us/articles/115014893968-Terms-of-sale}{Terms
  of Sale}
\item
  \href{https://spiderbites.nytimes.com}{Site Map}
\item
  \href{https://help.nytimes.com/hc/en-us}{Help}
\item
  \href{https://www.nytimes.com/subscription?campaignId=37WXW}{Subscriptions}
\end{itemize}
