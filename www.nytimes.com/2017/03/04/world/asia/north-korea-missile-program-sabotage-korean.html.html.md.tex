Sections

SEARCH

\protect\hyperlink{site-content}{Skip to
content}\protect\hyperlink{site-index}{Skip to site index}

\href{https://www.nytimes.com/section/world/asia}{Asia Pacific}

\href{https://myaccount.nytimes.com/auth/login?response_type=cookie\&client_id=vi}{}

\href{https://www.nytimes.com/section/todayspaper}{Today's Paper}

\href{/section/world/asia}{Asia Pacific}\textbar{}트럼프가 물려받은
유산: 북한 미사일에 대응하는 비밀 사이버전(戰)

\url{https://nyti.ms/2lHSeJV}

\begin{itemize}
\item
\item
\item
\item
\item
\end{itemize}

Advertisement

\protect\hyperlink{after-top}{Continue reading the main story}

Supported by

\protect\hyperlink{after-sponsor}{Continue reading the main story}

\hypertarget{uxd2b8uxb7fcuxd504uxac00-uxbb3cuxb824uxbc1buxc740-uxc720uxc0b0-uxbd81uxd55c-uxbbf8uxc0acuxc77cuxc5d0-uxb300uxc751uxd558uxb294-uxbe44uxbc00-uxc0acuxc774uxbc84uxc804ux6230}{%
\section{트럼프가 물려받은 유산: 북한 미사일에 대응하는 비밀
사이버전(戰)}\label{uxd2b8uxb7fcuxd504uxac00-uxbb3cuxb824uxbc1buxc740-uxc720uxc0b0-uxbd81uxd55c-uxbbf8uxc0acuxc77cuxc5d0-uxb300uxc751uxd558uxb294-uxbe44uxbc00-uxc0acuxc774uxbc84uxc804ux6230}}

\includegraphics{https://static01.nyt.com/images/2017/01/14/science/14MISSILE1-ko/14MISSILE1-articleInline.jpg?quality=75\&auto=webp\&disable=upscale}

By \href{http://www.nytimes.com/by/david-e-sanger}{David E. Sanger} and
\href{http://www.nytimes.com/by/william-j-broad}{William J. Broad}

\begin{itemize}
\item
  March 4, 2017
\item
  \begin{itemize}
  \item
  \item
  \item
  \item
  \item
  \end{itemize}
\end{itemize}

\href{https://www.nytimes.com/2017/03/04/world/asia/north-korea-missile-program-sabotage.html}{Read
in
English}\href{http://cn.nytimes.com/usa/20170304/north-korea-missile-program-sabotage/}{阅读简体中文版}

3년 전, 버락 오바마 대통령은 펜타곤(국방부) 각료들에게 북한의 미사일
개발과 관련, 미국의 사이버 및 전자 공격력의 수준을 끌어올리라고
지시했다. 북한이 시험발사를 할 경우 몇 초 안에 그 시도를 무력화하기 위한
의도였다.

곧 수많은 북한의 군사용 로켓들은 시험발사 후 폭발해 궤도를 이탈하며
공중에서 분해되고 바다로 떨어졌다. 이러한 조치를 옹호하는 이들은 이런
목표 공격 덕에 미국의 대(對)미사일 방어 능력이 한층 강화됐으며 북한이
미국 본토의 도시를 핵무기를 탑재한 대륙간탄도미사일(ICBM)로 타격할 수
있는 능력을 갖출 수 있는 시점을 수년 이후로 미룰 수 있었다고 말한다.

그러나 다른 전문가들은 이러한 새로운 시도에 회의적이다. 그들은 미사일
개발과 제조 과정에서의 오류나 불만을 품은 내부 분자들, 혹은 순수하게
능력이 없기 때문에 미사일들이 엉망이 된 거라고 주장한다. 지난 8개월간
북한은 세 개의 중거리 로켓을 발사할 수 있었다는 데 그들은 주목한다.
그리고 북한의 지도자인 김정은 노동당 위원장은 지금 그의 정권이 최초 ICBM
시험 발사의 ``준비 사업이 마감 단계''에 있다고 주장했다. 허풍일 수도
있지만 아닐 수도 있다.

오바마 정부와 트럼프 정부의 관료들과의 인터뷰뿐 아니라 광범위하지만 잘
알려지지는 않은 공개 기록 검토를 통해 펜타곤의 방해 공작을 들여다본
결과, 미국은 북한의 핵·미사일 프로그램에 효과적으로 대응할 만한 능력을
아직 갖추지 못하고 있음이 드러났다. 이런 위협은 많은 전문가들이 예상했던
것보다 훨씬 더 끈질긴 것으로 뉴욕타임스(NYT) 취재 결과 드러났다. 이런
위험 때문에 오바마 대통령이 임기를 마치며 트럼프 대통령에게 북한의
핵·미사일 위협이 그가 직면하게 될 가장 급박한 문제가 될 것이라고 경고한
것이다.

트럼프 대통령은 북한의 위협에 대해 공격적으로 대처하는 방법을
우선시하겠다는 신호를 보냈다. 김 위원장이 신년사에서 첫 경고를 보낸 뒤
트럼프가 ``그럴 일은 없을 것!''이라고 트위터 계정에서 밝히면서다. 하지만
전임 대통령이었던 오바마와 마찬가지로, 트럼프 역시 꽤나 불완전한
선택지에서 결정을 내려야 한다는 점을 신속히 깨닫고 있다.

그는 펜타곤의 사이버 및 전자전(戰) 능력을 향상시키라고 지시를 내릴 수도
있을 터이지만, 그렇다고 문제 해결이 보장되는 것은 아니다. 그는 북한이
핵과 미사일 개발을 중단하도록 북한과 협상을 시작할 수도 있겠지만, 그렇게
되면 꿈틀거리는 위협은 그대로 놔둔 채 일을 진행하게 된다. 그는 시험발사
장소를 직접 미사일로 타격할 준비를 할 수도 있는데, 이는 오바마 역시
고려했던 바이지만, 모든 목표물을 명중시킬 가능성은 작다. 그는 중국을
압박해 교역을 끊고 지원을 중단하도록 요구할 수 있겠지만, 중국은 항상
북한 체제의 붕괴로 이어질 수 있는 단계까지는 움직이지 않았다.

백악관 상황실에서 트럼프 대통령의 국가안보 팀의 2인자들이 회의를 연 건
두 번이었는데, 이 중 가장 최근에 회의가 열린 건 화요일이었다. 이
회의들에선 모든 위의 옵션들이 논의되었고 한국에 핵무기를 재배치함으로써
극적인 경고 효과를 낼 수도 있다는 가능성도 함께 거론됐다. 트럼프 정부
관계자들은 이러한 이슈들이 곧 트럼프 대통령과, 그의 최고위급 국가안보
담당자들에게 전달될 것이라고 말했다.

\includegraphics{https://static01.nyt.com/images/2017/01/14/science/14MISSILE2-ko/14MISSILE2-articleInline.jpg?quality=75\&auto=webp\&disable=upscale}

사이버 및 전자 공격 강화를 추진하기 위한 결정은 지난 2014년 초, 오바마
대통령이 미국이 아이젠하워 대통령 시절부터 지금까지 3000억달러를 들여
개발해왔으며, 때로 `총알을 맞히는 총알'이라고 비유되는 대(對)미사일
체계가 미국 본토를 보호하는 핵심 목표를 달성하는 데 실패했다는 결론을
내리면서 시작됐다. 요격기의 비행 실험이 알래스카와 캘리포니아에서
진행됐는데 전반적인 실패율은 56\% 정도였는데, 거의 완벽한 환경에서도
그랬다. 많은 전문가들은 마음속으로 이 시스템은 실제 전투 상황에선 더
비효율적이라고 우려했다.

그리하여 오바마 정부는 미사일을 파괴하기 위한 새로운 방식을 찾아나섰다.
그리고 오바마 정부가 찾은 것은 펜타곤이 ``발사의 왼편'' 공격이라는
별칭으로 불린 지시문을 전제로 오랜 기간 해온 실험이었다. 카운트다운
이전의 부분, 혹은 미사일이 발사대에 올려지거나 발사가 막 됐을 때 그
왼쪽을 조준하기에 이런 별칭이 붙었다. 수년간, 펜타곤의 최고위급 국장과
당국자들은 상대적으로 덜 관심을 끈 의회 증언이나 국방 관련 회의에서 이런
종류의 치밀한 공격을 공개적으로 옹호해왔다.

NYT의 취재는 지난봄에 시작됐다. 북한의 미사일 실패율이 치솟았을 때였다.
NYT는 취재 과정에서 북한을 겨냥한 새로운 대(對)미사일 접근법을 호평하는
군 내 문서들을 발굴했는데, 일부의 경우 사진과 도표까지 첨부돼 있었으며,
가장 급박한 목표물이라고 적혀 있었다.

트럼프 대통령의 국가안보 당국자들과 지난해뿐 아니라 지난 며칠간 논의한
결과, NYT는 입수한 조치들의 세부사항은 밝히지 않기로 합의했다. 북한이
미국이 그들을 어떻게 패배시킬지를 알 수 있도록 하는 것은 막자는 취지다.
지난가을, 김 위원장이 미국이 북한의 발사를 막고 있는 것은 아닌지
알아보라고 지시를 했다는 소문이 파다했으며, 지난주 그는 고위급 안보 관련
담당자를 처형하기도 했다.

북한 미사일을 목표로 한 접근법은 미국과 이스라엘이 주도해 이란의 핵
개발에 대해 방해공작을 폈던 것과 일정 부분 비슷한 점이 있는데, 사이버
무기가 핵 위협을 무력화시키는 데 쓰였던 가장 치밀한 방법이었다는 점에서
그러하다. 하지만 `스턱스넷(Stuxnet)'이라는 웜바이러스를 이란에 사용한
것은 곧 한계에 부딪혔다. 수년간 효과를 봤지만, 이란이 곧 눈치를 채고
복구를 했기 때문이다. 이후 이란은 상대적으로 다루기 더 쉬운 방법을
찾아냈다. 반복해서 공격을 받아도 견딜 수 있는 지하 핵농축 시설이다.

북한의 목표물은 미국에 더 큰 도전적인 존재다. 미사일은 복수의
발사장소에서 발사되며 이동식 발사대에 옮겨져 이동하는데 이는 적대국들을
기만하기 위한 치밀한 전략이다. 그들을 타격하기 위해서는 타이밍이
중요하다.

Image

2013년 7월, 평양에서 열린 열병식에서 이동식 발사대 차량에 실려 등장한
탄도 미사일 KN-08. 이동식 발사대는 동굴 속이나 지하에 숨기기 쉬워서
위치를 추적하거나 타격 목표를 잡기 어렵다.Credit...사진: 교도통신

북한 내부의 미사일 체계의 데이터를 조작하자는 치밀한 계획을 옹호하는
이들의 논리는 이랬다. 미국은 북한이 핵무기를 만들 수 있는 비밀을 알지
못하도록 막을 수 있는 노력을 했으나 이미 실패했고, 따라서 실제적 대안이
없다는 것이다. 이 나라가 대륙간 미사일을 개발하고 국제사회에 파괴적
무기를 선보이는 것을 막는 게 이제 남은 유일한 희망이라는 것이었다.

클린턴 정부 시절 국방장관을 지낸 윌리엄 페리는 최근 워싱턴에서 한
발표에서 ``그들의 실험을 방해하는 것은 그들의 ICBM 개발을 중단시키는 데
있어서 상당히 효율적인 방법''이 될 것이라고 말했다.

\hypertarget{uxac1cuxbc1cuxc5d0uxb9cc-uxc218uxc2eduxb144}{%
\subsection{개발에만
수십년}\label{uxac1cuxbc1cuxc5d0uxb9cc-uxc218uxc2eduxb144}}

김씨 일가는 3대에 걸쳐 그들의 고장난, 혹은 실패한 국가가 스스로 핵무기를
갖고, 그 핵무기를 운반할 수 있는 수단인 미사일을 가질 수 있기를 꿈꿨다.
그들에게 이는 궁극적인 생존 전략이었다. 손에 핵을 쥔다면 그들은 한국에
의해 무너지거나 미국에 의해 침략을 당하거나, 중국에 의해 경제적으로 지배
당하지 않을 것이라는 게 김씨 일가의 계산이었다.

북한은 ICBM 개발을 수십 년 전부터 추진해왔다. 이것은 북한을 세운
김일성의 꿈이었다. 김일성은 6·25 전쟁 당시 미국이 북한에 핵무기를
쓰겠다는 위협을 가했던 것을 쓰라린 기억으로 가지고 있었다.

그의 꿈을 이룰 절호의 기회는 소비에트연방의 붕괴로 찾아왔다. 더 이상
쓸모없어진 러시아제 로켓 과학자들이 북한에서 일자리를 찾기 시작한
것이다. 곧 북한 미사일의 새 세대가 열리기 시작했고 이는 모두 소련
미사일의 복제품이었다. 실제로 발사 실험을 하는 경우는 드물었으나 미국의
전문가들은 북한이 어떻게 새로운 로켓 프로그램을 개발하는 데 있어서
부딪히기 마련인 실패를 피해 나가는 것처럼 보이는지 신기해했다. 미국 역시
1950년대 말 미사일 실험을 하면서 많은 실패를 맛보았다.

그 성공은 너무도 두드러졌기에 몬테레이의 미들베리 국제문제연구소는
2001년의 한 분석 보고서에서 평양의 기록은 ``미사일 개발과 생산의 역사에
있어서 완벽하게 특별해 보인다.''고 썼을 정도다.

이에 대응해 조지 W. 부시 대통령은 2002년 말 대(對) 미사일 요격부대를
알래스카와 캘리포니아에 배치하겠다고 발표했다. 이와 동시에 부시 대통령은
북한 미사일 부품의 기나긴 공급망에 침투하는 프로그램에 속도를 내면서
단점과 약점을 가미시켰다. 이는 미국이 이란에 대해서도 수년간 써온
방식이다.

\hypertarget{uxc624uxbc14uxb9c8-uxc2dcuxb300-uxb354-uxcee4uxc9c4-uxc704uxd611}{%
\subsection{오바마 시대, 더 커진
위협}\label{uxc624uxbc14uxb9c8-uxc2dcuxb300-uxb354-uxcee4uxc9c4-uxc704uxd611}}

오바마 대통령이 취임한 2009년 1월, 북한은 수백 개의 단거리와 중거리
미사일을 배치했는데 이는 러시아의 디자인을 따온 것이었다. 이어 북한은
이집트·리비아·파키스탄·시리아·아랍에미리트연합(UAE)·예멘에 스커드
미사일을 판매하면서 수십억 달러를 벌어들였다. 그러나 북한의 꿈은 탄두를
탑재한 채 훨씬 더 먼 거리를 날아갈 수 있는 차세대 미사일을 개발하는
것이었다.

오바마 정부 출범 첫해에 보고된 비밀 외교 전문에 의하면 힐러리 클린턴
국무장관은 고개를 드는 위협에 주목했다.

위키리크스에 의해 공개된 것 중 가장 놀라운 것 중 하나인 이 전문은 북한이
장거리 미사일 개발이라는 목표를 위해 수십 년 전 소련이 잠수함 용으로
개발한 원자핵융합반응 탄두 기술을 새롭게 연구하고 있음을 보고하고 있다.

그 이름은 R-27이다. 북한의 느리고 구식인 로켓이나 미사일과 달리, 이들은
크기가 작아서 땅굴에 숨길 수도 있고 트럭으로 위치를 이동시킬 수도
있었다. 이로써 취할 수 있는 이점은 분명했다. 이 미사일은 미국이 찾아내서
요격하는 것이 더 힘들 것이라는 점이다.

``북한의 다음 목표는 이동이 가능한 ICBM을 개발해 세계 각 지역의 목표물을
위협할 수 있는 능력을 갖추는 것일 수 있다''는 것은 클린턴 장관이
사인하고 ``기밀''이라고 적은 2009년 10월 전문의 내용이다.

그 다음해, 북한의 군사 퍼레이드에서 새로운 미사일 중 한 가지가 등장했다.
정보당국이 경고한 그대로였다.

2013년까지 북한 로켓은 새롭게 정기적으로 발사됐다. 그리고 그해 2월,
북한은 워싱턴을 경각시키는 핵실험을 감행했다. 이 핵실험을 모니터한
기록에 따르면 이 핵실험은 히로시마에 투하한 원자폭탄의 위력에 거의
맞먹는 폭발력을 과시했다.

Image

중거리 탄도 미사일 무수단이 2015년 평양에서 열린 열병식을 지나고
있다.Credit...사진: 교도통신

그 핵실험이 진행된 지 수일 후, 펜타곤은 캘리포니아와 알래스카의
대(對)미사일 요격기를 증강하겠다고 발표했다. 펜타곤은 또 ``발사의
왼편(left of launch)''이라고 이름 붙인 프로그램을 개발하기 시작했는데,
이는 미사일이 발사되기 전에 미사일을 무력화하기 위한 프로그램으로, 그
미사일들을 파괴할 수 있는 가능성을 강화하는 것이 목표다. 합참의장인 마틴
뎀프시 장군이 이 프로그램을 발표했는데 그는 악성소프트웨어, 레이저 및
신호 교란 등을 의미하는 ``사이버전과 에너지 및 전자 공격''이라는 표현을
썼다. 이런 모든 종류의 신기술은 적을 공격하는 기존의 방식에 덧붙여서
중요하고도 새로운 추가 사항이 됐다.

그는 북한을 한 번도 직접 언급하진 않았다. 그러나 뎀프시 장군의 관련 정책
보고서에 첨부된 지도 한 장은 북한의 미사일이 미국 본토를 향해 사정거리를
좁히고 있음을 보여준다. 곧, 의회에서의 증언과 워싱턴에서의 공개 세미나
등을 통해, 전·현직 당국자들과 주요 계약자 - 레이시언(Raytheon) - 이라는
군수기업은 ``발사의 왼편'' 기술에 대해 공개적으로 얘기하기 시작했는데,
특히 발사 순간에 사이버 및 전자 공격을 가하는 방식을 많이 언급했다.

한편 북한은 나름 특이한 무기를 개발하고 있었다. 북한은 미국과 한국의
군사훈련을 미사일에 대한 방해 공작을 펼쳤는데, 미사일을 포함한 유도 무기
전자파를 교란하는 방법을 썼다. 그리고 북한은 자신들의 사이버 능력을
특이한 장소에서 과시했는데, 바로 할리우드였다. 지난 2014년, 북한은
소니픽처스를 공격해 이 회사 컴퓨터 시스템의 70\%가량을 파괴했다. 당시
전문가들은 북한의 기술적 진보에 대해 놀라움을 표했다.

지난달, 국방과학위원회가 오바마 정부 시절 펜타곤의 지시를 받고 사이버
공간의 취약성에 대해 작성한 보고서는 북한이 미국의 전력망을 무력화시킬
수 있는 능력을 습득하고 있을 수 있으며, ``필수적인 미국의 공격 시스템을
위협하도록'' 허용되어서는 안 될 것이라고도 경고했다.

\hypertarget{uxbe44uxbc00uxc2a4uxb7ecuxc6b4-uxc555uxb825-uxadf8uxb9acuxace0-uxc0c8uxb85cuxc6b4-uxc758uxd639uxb4e4}{%
\subsection{비밀스러운 압력, 그리고 새로운
의혹들}\label{uxbe44uxbc00uxc2a4uxb7ecuxc6b4-uxc555uxb825-uxadf8uxb9acuxace0-uxc0c8uxb85cuxc6b4-uxc758uxd639uxb4e4}}

뎀프시 장군이 공개 발표를 한 지 얼마 되지 않아 오바마 대통령과 그의
국방장관인 애쉬튼 카터는 회의들을 소집했다. 이 회의들은 하나의 질문에
초점이 맞춰졌다. `크래쉬(crash) 프로그램'이라고 불리는 이 프로그램이
ICBM 개발을 위해 전진하는 북한의 발걸음을 늦출 수 있을 것인가?

선택할 수 있는 가능성은 많았다. 일부는 뎀프시 장군이 만든 목록에서
나왔다. 오바마 대통령은 결국 펜타곤과 정보당국을 압박하면서 모든 노력을
경주하도록 했고, 이를 당국자들은 테스트가 완료되지 않은 기술에까지 손을
댈 수 있도록 격려하는 뜻으로 받아들였다.

북한의 미사일 개발은 곧 놀라운 속도로 실패하기 시작했다. 일부 미사일은
물론 우연하게, 또는 의도적으로 파괴됐다. 북한이 개발하려는 기술은 새로운
디자인과 새 엔진을 장착한 다단식 로켓으로, 잘못될 경우 상상할 수 있는
모든 종류의 재앙을 가져올 수 있었다. 그러나 미국이 그러한 실패를 더
두드러지게 했다는 것이 대부분의 평가다.

증거는 숫자에서 찾을 수 있다. 북한이 클린턴 국무장관의 경고 이후 보란 듯
공개했던 무기인 `무수단'이라는 이름의 중장거리 미사일 발사실험의
대부분은 실패로 끝났고 로켓은 불에 타버렸다. 실패율은 대략 88\%에
달한다.

그럼에도 불구하고 김정은은 그의 주요 목표를 향해 계속 나아갔다. 그
목표는 바로 대륙간탄도미사일(ICBM)이다. 지난 4월, 그는 거대한 시험발사대
옆에 서서 기술자들이 러시아에서 디자인한 R-27 엔진 한 쌍을 성공리에
발사한 것을 축하하는 모습을 사진으로 공개했다. 그 의도는 분명했다.
하나의 미사일에 두 개의 엔진을 같이 묶을 수 있다는 것은 미국에 탄두를
실은 ICBM을 날려보낼 수 있다는 의미다.

이어 지난 9월, 그는 북한 핵무기 사상 가장 성공적이라고 평가되는 실험을
했다. 이는 히로시마에 떨어진 원자탄보다 두 배는 더 큰 폭발력을 가진
것으로 평가됐다.

그의 다음 목표는 전문가들에 따르면 이 두 기술을 결합하는 것이며,
핵탄두를 소형화해서 대륙간 미사일에 탑재할 수 있도록 하는 것이다. 그렇게
되어야만 그는 그의 고립된 정부가 수천 마일 떨어진 미국의 도시들을 때릴
수 있는 노하우를 가지고 있다는 신빙성 있는 주장을 할 수 있게 되는
것이다.

오바마 대통령은 그의 임기 마지막 해에 공개적으로 북한이 모든 핵과 미사일
실험을 통해 - 실패로 끝난 실험을 포함해서 - 그의 목표에 조금씩
가까워지고 있다고 자주 말했다. 그의 참모들은 사적인 자리에서 그가 점점
더 북한의 개발 속도에 불쾌감을 표시했다고 전했다.

임기 말, 단지 몇 개월을 남겨두고 그는 그의 참모들에게 새로운 접근법을
시도해볼 것을 종용했다. 한 회의에서 그는 만약 효과가 있다고 생각이
된다면 북한의 지도부와 무기 관련 장소를 직접 목표로 삼았을 수도 있다고
말했다. 그러나 오바마 대통령과 그의 참모들이 알고 있었듯, 그것은 소용
없는 위협이었다. 북한의 지도자들이나 무기들의 소재를 제때 파악하는 것은
거의 불가능에 가까우며, 목표를 놓칠 경우, 한반도의 또 다른 전쟁이 발발할
가능성을 포함해 미국이 감수해야 할 위험은 막대했다.

\hypertarget{uxd2b8uxb7fcuxd504uxac00-uxb0b4uxb824uxc57c-uxd560-uxc5b4uxb824uxc6b4-uxacb0uxc815uxb4e4}{%
\subsection{트럼프가 내려야 할 어려운
결정들}\label{uxd2b8uxb7fcuxd504uxac00-uxb0b4uxb824uxc57c-uxd560-uxc5b4uxb824uxc6b4-uxacb0uxc815uxb4e4}}

Image

북한 관영 통신사가 지난해 6월 무수단 미사일의 성공적인 발사 소식을
보도하며 내건 사진. 무수단 미사일은 총 여덟 차례 시험발사 결과 실패율이
88\%로 상당히 높았다. 미군의 교란 작전이 성과가 있었음을 암시한다.
이와는 대조적으로, 무수단 미사일의 원천 기술로 제작된 구소련 잠수함
미사일의 발사시험 실패율은 13\%에 불과하다.Credit...사진: 조선중앙통신,
로이터

대통령 후보 시절, 트럼프는 ``우리의 사이버 대응은 너무 구식이다''라는
불만을 터뜨린 적이 있는데, 이 말은 수십억 달러를 들여 대통령에게
정보수집 및 사이버공격 관련 새로운 선택지를 제공해온 미국사이버사령부와
국가안전국(NSA)의 관리들을 초조하게 만들었다. 이제, 트럼프는 이러한
당국의 노력을 더 지원할 것인가, 아니면 지원 규모를 줄일 것인가라는
문제에 즉시 답을 내놓아야 한다.

적대국의 발사 능력을 추적하겠다는 결정은 향후 의도하지 않은 결과를
가져올 수 있다고 전문가들은 경고한다. 만약 미국이 핵 발사 시스템에
대항하기 위한 방법으로 사이버 무기를 사용한다면, 북한과 같은 위협적인
곳에 대한 것이라고 하더라도, 러시아와 중국도 같은 조치를 취할 수 있는
자유가 있다고 느낄 것이며, 미국의 미사일을 목표로 삼을 수도 있다. 일부
전략가들은 모든 핵 시스템은 사이버 공격이 닿을 수 없는 곳에 있다고도
주장한다. 그렇지 않다면, 만약 핵보유국이 만약 비밀리에 적국의 원자력
통제력을 무력화하려고 할 경우, 선제공격을 감행할 위험을 감수하는 것에 더
큰 유혹을 느낄 수도 있다.

스탠포드 대학의 정보 및 사이버안보 전문가인 에이미 지가트는 자신이 현재
미국이 어떤 방향으로 노력을 하고 있는지는 잘 모른다고 전제하면서 이렇게
말했다. ``위협 수위가 높기 때문에 긴박한 상황이라는 점은 이해된다.
하지만 지금으로부터 30년 후, 우리는 그 결정이 매우, 매우 위험한
것이었다고 결론을 내릴지도 모를 일이다.''

트럼프 대통령의 참모들은 모든 옵션이 테이블 위에 있다고 말한다. 중국은
최근 북한으로부터의 석탄 수입을 중단하는 조치를 했지만 미국 정부는
중국의 영향력 아래 있는 은행들에 은닉된 것으로 알려져 있는 김씨 일가의
자산을 동결시킬 방법도 검토하고 있다. 중국은 이미 한국에
고고도미사일방어(THAAD·사드) 체계를 배치하는 것에 반대하고 나섰다.
트럼프의 참모들은 그러나 그런 방어체계의 추가 배치를 요구할지도 모르는
상황이다.

백악관은 또한 선제타격 옵션도 검토하고 있다고 트럼프 정부의 고위급
관계자는 전했다. 이는 물론 북한에 산악지대가 많고 땅 속 깊이 묻힌 터널과
벙커들이 상당수라는 점을 고려할 때 상당히 위험 수위가 높은 옵션이다. 약
25년 전 한국에서 철수시켰던 미국의 전략적 핵무기를 한국에 재배치하는 것
역시 북한과의 무기 배치 경쟁을 촉발시키는 조치일 수도 있지만, 검토되고
있다.

트럼프 대통령은 북한이 ICBM 위협에 대해 ``그럴 일은 없다!''고 트위터에
올린 것은 어쩌면 더 큰 종류의 갈등이 꿈틀대고 있다는 것을 의미한다.

카네기국제평화재단의 핵 문제 전문가인 제임스 M. 액튼은 최근 ``트럼프가
실제로 어떤 의도를 갖고 있는지는 모르겠다.''고 전제하면서도 ``그 트윗은
`레드라인'으로 보일 수도 있는데, 그렇다고 한다면 그의 신뢰도에 대한
잠재적인 테스트가 될 수도 있을 것''이라고 말했다.

Advertisement

\protect\hyperlink{after-bottom}{Continue reading the main story}

\hypertarget{site-index}{%
\subsection{Site Index}\label{site-index}}

\hypertarget{site-information-navigation}{%
\subsection{Site Information
Navigation}\label{site-information-navigation}}

\begin{itemize}
\tightlist
\item
  \href{https://help.nytimes.com/hc/en-us/articles/115014792127-Copyright-notice}{©~2020~The
  New York Times Company}
\end{itemize}

\begin{itemize}
\tightlist
\item
  \href{https://www.nytco.com/}{NYTCo}
\item
  \href{https://help.nytimes.com/hc/en-us/articles/115015385887-Contact-Us}{Contact
  Us}
\item
  \href{https://www.nytco.com/careers/}{Work with us}
\item
  \href{https://nytmediakit.com/}{Advertise}
\item
  \href{http://www.tbrandstudio.com/}{T Brand Studio}
\item
  \href{https://www.nytimes.com/privacy/cookie-policy\#how-do-i-manage-trackers}{Your
  Ad Choices}
\item
  \href{https://www.nytimes.com/privacy}{Privacy}
\item
  \href{https://help.nytimes.com/hc/en-us/articles/115014893428-Terms-of-service}{Terms
  of Service}
\item
  \href{https://help.nytimes.com/hc/en-us/articles/115014893968-Terms-of-sale}{Terms
  of Sale}
\item
  \href{https://spiderbites.nytimes.com}{Site Map}
\item
  \href{https://help.nytimes.com/hc/en-us}{Help}
\item
  \href{https://www.nytimes.com/subscription?campaignId=37WXW}{Subscriptions}
\end{itemize}
