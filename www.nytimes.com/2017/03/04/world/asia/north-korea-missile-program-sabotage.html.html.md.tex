Sections

SEARCH

\protect\hyperlink{site-content}{Skip to
content}\protect\hyperlink{site-index}{Skip to site index}

\href{https://www.nytimes.com/section/world/asia}{Asia Pacific}

\href{https://myaccount.nytimes.com/auth/login?response_type=cookie\&client_id=vi}{}

\href{https://www.nytimes.com/section/todayspaper}{Today's Paper}

\href{/section/world/asia}{Asia Pacific}\textbar{}Trump Inherits a
Secret Cyberwar Against North Korean Missiles

\url{https://nyti.ms/2lHz4E9}

\begin{itemize}
\item
\item
\item
\item
\item
\item
\end{itemize}

Advertisement

\protect\hyperlink{after-top}{Continue reading the main story}

Supported by

\protect\hyperlink{after-sponsor}{Continue reading the main story}

\hypertarget{trump-inherits-a-secret-cyberwar-against-north-korean-missiles}{%
\section{Trump Inherits a Secret Cyberwar Against North Korean
Missiles}\label{trump-inherits-a-secret-cyberwar-against-north-korean-missiles}}

\includegraphics{https://static01.nyt.com/images/2017/01/14/science/14MISSILE1/14MISSILE1-articleLarge.jpg?quality=75\&auto=webp\&disable=upscale}

By \href{http://www.nytimes.com/by/david-e-sanger}{David E. Sanger} and
\href{http://www.nytimes.com/by/william-j-broad}{William J. Broad}

\begin{itemize}
\item
  March 4, 2017
\item
  \begin{itemize}
  \item
  \item
  \item
  \item
  \item
  \item
  \end{itemize}
\end{itemize}

\href{https://www.nytimes.com/2017/03/04/world/asia/north-korea-missile-program-sabotage-korean.html}{한국어로
읽기}\href{http://cn.nytimes.com/usa/20170304/north-korea-missile-program-sabotage/}{阅读简体中文版}

WASHINGTON --- Three years ago, President Barack Obama ordered Pentagon
officials to step up their cyber and electronic strikes against North
Korea's missile program in hopes of sabotaging test launches in their
opening seconds.

Soon a large number of the North's military rockets began to explode,
veer off course, disintegrate in midair and plunge into the sea.
Advocates of such efforts say they believe that targeted attacks have
given American antimissile defenses a new edge and delayed by several
years the day when North Korea will be able to threaten American cities
with nuclear weapons launched atop intercontinental ballistic missiles.

But other experts have grown increasingly skeptical of the new approach,
arguing that manufacturing errors, disgruntled insiders and sheer
incompetence can also send missiles awry. Over the past eight months,
they note, the North has managed to successfully launch three
medium-range rockets. And Kim Jong-un, the North Korean leader, now
claims his country is in ``the final stage in preparations'' for the
inaugural test of his intercontinental missiles --- perhaps a bluff,
perhaps not.

An examination of the Pentagon's disruption effort, based on interviews
with officials of the Obama and Trump administrations as well as a
review of extensive but obscure public records, found that the United
States still does not have the ability to effectively counter the North
Korean nuclear and missile programs. Those threats are far more
resilient than many experts thought, The New York Times's reporting
found, and pose such a danger that Mr. Obama, as he left office, warned
President Trump they were likely to be the most urgent problem he would
confront.

Mr. Trump has signaled his preference to respond aggressively against
the North Korean threat.
\href{https://www.nytimes.com/2017/01/02/world/asia/trump-twitter-north-korea-missiles-china.html?_r=0}{In
a Twitter post} after Mr. Kim first issued his warning on New Year's
Day, the president wrote, ``It won't happen!'' Yet like Mr. Obama before
him, Mr. Trump is quickly discovering that he must choose from highly
imperfect options.

He could order the escalation of the Pentagon's cyber and electronic
warfare effort, but that carries no guarantees. He could open
negotiations with the North to freeze its nuclear and missile programs,
but that would leave a looming threat in place. He could prepare for
direct missile strikes on the launch sites, which Mr. Obama also
considered, but there is little chance of hitting every target. He could
press the Chinese to cut off trade and support, but Beijing has always
stopped short of steps that could lead to the regime's collapse.

In two meetings of Mr. Trump's national security deputies in the
Situation Room, the most recent on Tuesday, all those options were
discussed, along with the possibility of reintroducing nuclear weapons
to South Korea as a dramatic warning. Administration officials say those
issues will soon go to Mr. Trump and his top national security aides.

The decision to intensify the cyber and electronic strikes, in early
2014, came after Mr. Obama concluded that
\href{http://wmdjunction.com/120413_missile_defense_costs.htm}{the \$300
billion spent} since the Eisenhower era on traditional antimissile
systems, often compared to hitting ``a bullet with a bullet,'' had
failed the core purpose of protecting the continental United States.
Flight tests of interceptors based in Alaska and California had an
overall
\href{https://www.mda.mil/global/documents/pdf/testrecord.pdf}{failure
rate} of 56 percent, under near-perfect conditions. Privately, many
experts warned the system would fare worse in real combat.

So the Obama administration searched for a better way to destroy
missiles. It reached for techniques the Pentagon had long been
experimenting with under the rubric of
``\href{http://missiledefenseadvocacy.org/alert/3132/}{left of
launch},'' because the attacks begin before the missiles ever reach the
launchpad, or just as they lift off. For years, the Pentagon's most
senior officers and officials have publicly advocated these kinds of
sophisticated attacks in little-noticed testimony to Congress and at
defense conferences.

\includegraphics{https://static01.nyt.com/images/2017/01/14/science/14MISSILE2/14MISSILE2-articleInline.jpg?quality=75\&auto=webp\&disable=upscale}

The Times inquiry began last spring as the number of the North's missile
failures soared. The investigation uncovered the military documents
praising the new antimissile approach and found some pointing with
photos and diagrams to North Korea as one of the most urgent targets.

After discussions with the office of the director of national
intelligence last year and in recent days with Mr. Trump's national
security team, The Times agreed to withhold details of those efforts to
keep North Korea from learning how to defeat them. Last fall, Mr. Kim
was
\href{https://www.nknews.org/2016/10/kim-jong-un-to-investigate-espionage-linked-to-failed-missile-launch-report/}{widely
reported} to have ordered an investigation into whether the United
States was sabotaging North Korea's launches, and over the past week he
has
\href{https://www.nytimes.com/aponline/2017/02/27/world/asia/ap-as-north-korea-executions.html}{executed
senior security officials}.

The approach taken in targeting the North Korean missiles has distinct
echoes of the American- and Israeli-led
\href{http://www.nytimes.com/2011/01/16/world/middleeast/16stuxnet.html}{sabotage
of Iran's nuclear program}, the most sophisticated known use of a
cyberweapon meant to cripple a nuclear threat. But even that use of the
``Stuxnet'' worm in Iran quickly ran into limits. It was effective for
several years, until the Iranians figured it out and recovered. And Iran
posed a relatively easy target: an underground nuclear enrichment plant
that could be attacked repeatedly.

In North Korea, the target is much more challenging. Missiles are fired
from multiple launch sites around the country and moved about on mobile
launchers in an elaborate shell game meant to deceive adversaries. To
strike them, timing is critical.

Advocates of the sophisticated effort to remotely manipulate data inside
North Korea's missile systems argue the United States has no real
alternative because the effort to stop the North from learning the
secrets of making nuclear weapons has already failed. The only hope now
is stopping the country from developing an intercontinental missile, and
demonstrating that destructive threat to the world.

Image

KN-08 ballistic missiles were paraded through Pyongyang in July 2013 on
mobile launch vehicles that can be hidden in caves or underground,
making the missiles hard to track and target.Credit...Kyodo News

``Disrupting their tests,'' William J. Perry, secretary of defense in
the Clinton administration,
\href{http://38north.org/wp-content/uploads/2017/01/2017-0109-38-North-Press-Briefing-Transcript.pdf}{said
at a recent presentation} in Washington, would be ``a pretty effective
way of stopping their ICBM program.''

\hypertarget{decades-in-the-making}{%
\subsection{Decades in the Making}\label{decades-in-the-making}}

Three generations of the Kim family have dreamed that their broken,
otherwise failed nation could build its own nuclear weapons, and the
missiles to deliver them, as the ultimate survival strategy. With nukes
in hand, the Kims have calculated, they need not fear being overrun by
South Korea, invaded by the United States or sold out by China.

North Korea began seeking an intercontinental ballistic missile decades
ago: It was the dream of Kim Il-sung, the country's founder, who
bitterly remembered the American threats to use nuclear weapons against
the North during the Korean War.

His break came after the collapse of the Soviet Union, when
\href{https://www.nytimes.com/2016/09/10/science/north-korea-nuclear-weapons.html}{out-of-work
Russian rocket scientists began seeking employment} in North Korea.
Soon, a new generation of North Korean missiles began to appear, all
knockoffs of Soviet designs. Though flight tests were sparse, American
experts marveled at how the North seemed to avoid the kinds of failures
that typically strike new rocket programs, including those of the United
States in the late 1950s.

The success was so marked that Timothy McCarthy of the Middlebury
Institute of International Studies at Monterey
\href{http://calhoun.nps.edu/bitstream/handle/10945/40268/inc_barletta_op6.pdf?sequence=1\&isAllowed=y}{wrote
in a 2001 analysis}that Pyongyang's record ``appears completely unique
in the history of missile development and production.''

In response, President George W. Bush
\href{http://www.nytimes.com/2002/12/18/world/threats-responses-defense-antimissile-system-limited-form-ordered-bush.html}{in
late 2002 announced} the deployment of antimissile interceptors in
Alaska and California. At the same time, Mr. Bush accelerated programs
to get inside the long supply chain of parts for North Korean missiles,
lacing them with defects and weaknesses, a technique also used for years
against Iran.

\hypertarget{threat-grows-in-obama-era}{%
\subsection{Threat Grows in Obama Era}\label{threat-grows-in-obama-era}}

By the time Mr. Obama took office in January 2009, the North had
\href{https://www.nonproliferation.org/wp-content/uploads/npr/npr_18-2_pollack_ballistic-trajectory.pdf}{deployed
hundreds of short- and medium-range missiles} that used Russian designs,
and had made billions of dollars selling its Scud missiles to Egypt,
Libya, Pakistan, Syria, the United Arab Emirates and Yemen. But it
aspired to a new generation of missiles that could fire warheads over
much longer distances.

In secret cables written in the first year of the Obama administration,
Secretary of State Hillary Clinton laid out the emerging threat. Among
the most alarming
\href{http://www.nytimes.com/interactive/world/statessecrets.html}{released
by WikiLeaks}, the cables described a new path the North was taking to
reach its long-range goal, based on a missile designed by the Soviets
decades ago for their submarines that carried thermonuclear warheads.

It was \href{http://www.navweaps.com/Weapons/WMRUS_R-27.php}{called the
R-27}. Unlike the North's lumbering, older rockets and missiles, these
would be small enough to hide in caves and move into position by truck.
The advantage was clear: This missile would be far harder for the United
States to find and destroy.

``North Korea's next goal may be to develop a mobile ICBM that would be
capable of threatening targets around the world,''
\href{https://wikileaks.org/plusd/cables/08STATE105029_a.html}{said an
October 2009 cable} marked ``Secret'' and signed by Mrs. Clinton.

The next year, one of the new missiles
\href{https://en.wikipedia.org/wiki/Hwasong-10}{showed up} in a North
Korean military parade, just as the intelligence reports had warned.

By 2013, North Korean rockets thundered with new regularity. And that
February, the North
\href{http://www.nytimes.com/2013/02/12/world/asia/north-korea-nuclear-test.html?ref=global-home}{set
off a nuclear test} that woke up Washington: The monitoring data told of
an explosion roughly the size of the bomb that had leveled Hiroshima.

Days after the explosion, the
\href{http://archive.defense.gov/Speeches/Speech.aspx?SpeechID=1759}{Pentagon
announced} an expansion of its force of antimissile interceptors in
California and Alaska. It also began to unveil its ``left of launch''
program to disable missiles before liftoff --- hoping to bolster its
chances of destroying them. Gen. Martin E. Dempsey, the chairman of the
Joint Chiefs of Staff,
\href{http://www.jcs.mil/Portals/36/Documents/Publications/JointIAMDVision2020.pdf}{announced
the program}, saying that ``cyberwarfare, directed energy and electronic
attack,'' a reference to such things as malware, lasers and signal
jamming, were all becoming important new adjuncts to the traditional
ways of deflecting enemy strikes.

Image

Intermediate-range Musudan missiles rolled through Pyongyang during a
parade in 2015.Credit...Kyodo News

He never mentioned North Korea. But a map accompanying General Dempsey's
policy paper on the subject showed one of the North's missiles streaking
toward the United States. Soon, in testimony before Congress and at
public panels in Washington, current and former officials and a major
contractor --- Raytheon --- began talking openly about ``left of
launch'' technologies, in particular cyber and electronic strikes at the
moment of launch.

The North, meanwhile, was developing its own exotic arsenal. It tried
repeatedly to disrupt American and South Korean military exercises by
\href{http://www.upi.com/North-Korea-jams-Souths-guided-missiles/49341299621609/}{jamming
electronic signals} for guided weapons, including missiles. And it
demonstrated its cyberpower in the oddest of places --- Hollywood. In
2014,
\href{https://www.nytimes.com/2014/12/18/world/asia/us-links-north-korea-to-sony-hacking.html?_r=0}{it
attacked} Sony Pictures Entertainment with a strike that destroyed about
70 percent of the company's computing systems, surprising experts with
its technical savvy.

Last month, a
\href{http://www.acq.osd.mil/dsb/reports/2010s/DSB-CyberDeterrenceReport_02-28-17_Final.pdf}{report
on cybervulnerabilities by the Defense Science Board}, commissioned by
the Pentagon during the Obama administration, warned that North Korea
might acquire the ability to cripple the American power grid, and
cautioned that it could never be allowed to ``hold vital U.S. strike
systems at risk.''

\hypertarget{secret-push-and-new-doubts}{%
\subsection{Secret Push, and New
Doubts}\label{secret-push-and-new-doubts}}

Not long after General Dempsey made his public announcement, Mr. Obama
and his defense secretary, Ashton B. Carter, began calling meetings
focused on one question: Could a crash program slow the North's march
toward an intercontinental ballistic missile?

There were many options, some drawn from General Dempsey's list. Mr.
Obama ultimately pressed the Pentagon and intelligence agencies to pull
out all the stops, which officials took as encouragement to reach for
untested technologies.

The North's missiles soon began to
\href{http://www.globalsecurity.org/wmd/world/dprk/nd-b5.htm}{fail at a
remarkable pace}. Some were destroyed, no doubt, by accident as well as
by design. The technology the North was pursuing, using new designs and
new engines, involved multistage rockets, introducing all kinds of
possibilities for catastrophic mistakes. But by most accounts, the
United States program accentuated the failures.

The evidence was in the numbers. Most flight tests of an
intermediate-range missile called the Musudan, the weapon that the North
Koreans showed off in public just after Mrs. Clinton's warning, ended in
flames: Its overall failure rate is 88 percent.

Nonetheless Kim Jong-un has pressed ahead on his main goal: an
intercontinental ballistic missile. Last April, he was photographed
standing next to a giant test-stand, celebrating after engineers
\href{http://www.globalsecurity.org/wmd/world/dprk/kn-14-first-stage-main-engine-cluster-static-test-firing-4-9-2016.htm}{successfully
fired off a matched pair} of the potent Russian-designed R-27 engines.
The implication was clear: Strapping two of the engines together at the
base of a missile was the secret to building an ICBM that could
ultimately hurl warheads at the United States.

In September, he celebrated the
\href{http://www.nti.org/analysis/articles/north-koreas-nuclear-year-reviewand-whats-next/}{most
successful test yet} of a North Korean nuclear weapon --- one that
exploded with more than twice the destructive force of the Hiroshima
bomb.

His next goal, experts say, is to combine those two technologies,
shrinking his nuclear warheads to a size that can fit on an
intercontinental missile. Only then can he credibly claim that his
isolated country has the know-how to hit an American city thousands of
miles away.

In the last year of his presidency, Mr. Obama often noted publicly that
the North was learning from every nuclear and missile test --- even the
failures --- and getting closer to its goal. In private, aides noticed
he was increasingly disturbed by North Korea's progress.

With only a few months left in office, he pushed aides for new
approaches. At one meeting, he declared that he would have targeted the
North Korean leadership and weapons sites if he thought it would work.
But it was, as Mr. Obama and his assembled aides knew, an empty threat:
Getting timely intelligence on the location of North Korea's leaders or
their weapons at any moment would be almost impossible, and the risks of
missing were tremendous, including renewed war on the Korean Peninsula.

\hypertarget{hard-decisions-for-trump}{%
\subsection{Hard Decisions for Trump}\label{hard-decisions-for-trump}}

Image

The single successful test flight in a run of Musudan missile failures
came in June, shown in this image from North Korea's state-run news
agency. The Musudan had an overall failure rate of 88 percent, much
higher than the 13 percent failure rate of the Soviet-era missile on
which it was based.Credit...Korean Central News Agency, via Reuters

As a presidential candidate, Mr. Trump complained that
``\href{https://www.nytimes.com/2016/03/27/us/politics/donald-trump-transcript.html?_r=0}{we're
so obsolete in cyber},'' a line that grated on officials at the United
States Cyber Command and the National Security Agency, where billions of
dollars have been spent to provide the president with new options for
intelligence gathering and cyberattacks. Now, one of the immediate
questions he faces is whether to accelerate or scale back those efforts.

A decision to go after an adversary's launch ability can have unintended
consequences, experts warn.

Once the United States uses cyberweapons against nuclear launch systems
--- even in a threatening state like North Korea --- Russia and China
may feel free to do the same, targeting fields of American missiles.
Some strategists argue that all nuclear systems should be off limits for
cyberattack. Otherwise, if a nuclear power thought it could secretly
disable an adversary's atomic controls, it might be more tempted to take
the risk of launching a pre-emptive attack.

``I understand the urgent threat,'' said Amy Zegart, a Stanford
University intelligence and cybersecurity expert, who said she had no
independent knowledge of the American effort. ``But 30 years from now we
may decide it was a very, very dangerous thing to do.''

Mr. Trump's aides say everything is on the table. China recently
\href{https://www.nytimes.com/2017/02/24/world/asia/china-north-korea-relations-kim-jong-un.html}{cut
off coal imports} from the North, but the United States is also looking
at ways to freeze the Kim family's assets, some of which are believed
held in Chinese-controlled banks. The Chinese have already
\href{https://www.nytimes.com/2016/07/08/world/asia/south-korea-and-us-agree-to-deploy-missile-defense-system.html}{opposed
the deployment} of a high-altitude missile defense system known as Thaad
in South Korea; the Trump team may call for even more such systems.

The White House is also looking at pre-emptive military strike options,
a senior Trump administration official said, though the challenge is
huge given the country's mountainous terrain and deep tunnels and
bunkers. Putting American tactical nuclear weapons back in South Korea
--- they were withdrawn a quarter-century ago --- is also under
consideration, even if that step could accelerate an arms race with the
North.

Mr. Trump's ``It won't happen!'' post on Twitter about the North's ICBM
threat suggests a larger confrontation could be looming.

``Regardless of Trump's actual intentions,'' James M. Acton, a nuclear
analyst at the Carnegie Endowment for International Peace,
\href{https://www.theatlantic.com/international/archive/2017/01/trump-twitter-north-korea/512450/}{recently
noted}, ``the tweet could come to be seen as a `red line' and hence set
up a potential test of his credibility.''

Advertisement

\protect\hyperlink{after-bottom}{Continue reading the main story}

\hypertarget{site-index}{%
\subsection{Site Index}\label{site-index}}

\hypertarget{site-information-navigation}{%
\subsection{Site Information
Navigation}\label{site-information-navigation}}

\begin{itemize}
\tightlist
\item
  \href{https://help.nytimes.com/hc/en-us/articles/115014792127-Copyright-notice}{©~2020~The
  New York Times Company}
\end{itemize}

\begin{itemize}
\tightlist
\item
  \href{https://www.nytco.com/}{NYTCo}
\item
  \href{https://help.nytimes.com/hc/en-us/articles/115015385887-Contact-Us}{Contact
  Us}
\item
  \href{https://www.nytco.com/careers/}{Work with us}
\item
  \href{https://nytmediakit.com/}{Advertise}
\item
  \href{http://www.tbrandstudio.com/}{T Brand Studio}
\item
  \href{https://www.nytimes.com/privacy/cookie-policy\#how-do-i-manage-trackers}{Your
  Ad Choices}
\item
  \href{https://www.nytimes.com/privacy}{Privacy}
\item
  \href{https://help.nytimes.com/hc/en-us/articles/115014893428-Terms-of-service}{Terms
  of Service}
\item
  \href{https://help.nytimes.com/hc/en-us/articles/115014893968-Terms-of-sale}{Terms
  of Sale}
\item
  \href{https://spiderbites.nytimes.com}{Site Map}
\item
  \href{https://help.nytimes.com/hc/en-us}{Help}
\item
  \href{https://www.nytimes.com/subscription?campaignId=37WXW}{Subscriptions}
\end{itemize}
