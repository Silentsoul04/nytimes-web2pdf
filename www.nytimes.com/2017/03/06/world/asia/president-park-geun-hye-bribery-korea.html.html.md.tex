Sections

SEARCH

\protect\hyperlink{site-content}{Skip to
content}\protect\hyperlink{site-index}{Skip to site index}

\href{https://www.nytimes.com/section/world/asia}{Asia Pacific}

\href{https://myaccount.nytimes.com/auth/login?response_type=cookie\&client_id=vi}{}

\href{https://www.nytimes.com/section/todayspaper}{Today's Paper}

\href{/section/world/asia}{Asia Pacific}\textbar{}Prosecutor Pushes for
Indictment of South Korean President in Samsung Scandal

\url{https://nyti.ms/2n59p9q}

\begin{itemize}
\item
\item
\item
\item
\item
\end{itemize}

Advertisement

\protect\hyperlink{after-top}{Continue reading the main story}

Supported by

\protect\hyperlink{after-sponsor}{Continue reading the main story}

\hypertarget{prosecutor-pushes-for-indictment-of-south-korean-president-in-samsung-scandal}{%
\section{Prosecutor Pushes for Indictment of South Korean President in
Samsung
Scandal}\label{prosecutor-pushes-for-indictment-of-south-korean-president-in-samsung-scandal}}

\includegraphics{https://static01.nyt.com/images/2017/03/07/business/07park-2/07park-2-articleInline.jpg?quality=75\&auto=webp\&disable=upscale}

By \href{http://www.nytimes.com/by/choe-sang-hun}{Choe Sang-Hun}

\begin{itemize}
\item
  March 6, 2017
\item
  \begin{itemize}
  \item
  \item
  \item
  \item
  \item
  \end{itemize}
\end{itemize}

SEOUL, South Korea --- A special prosecutor in South Korea asked state
prosecutors on Monday to indict President Park Geun-hye on bribery
charges, saying that Ms. Park and her secretive confidante conspired to
take \$38 million in bribes from Samsung, one of the world's largest
technology companies.

The special prosecutor, Park Young-soo, recommended the indictment as he
announced the results of his team's 90-day investigation into a
corruption scandal surrounding Ms. Park, who was
\href{https://www.nytimes.com/2016/12/09/world/asia/south-korea-president-park-geun-hye-impeached.html}{impeached
by a parliamentary vote} in December.

The inquiry resulted in the indictments of 30 people, including several
former aides to Ms. Park, on criminal charges, including the abuse of
official power. But the prosecutor could not bring any charge against
Ms. Park because she is protected from indictment while in office.

His mandate now over, Mr. Park said he was leaving the task of indicting
Ms. Park once she is out of office to state prosecutors.

Ms. Park's presidential powers have been suspended since the impeachment
vote in December. The Constitutional Court is expected to rule in the
coming weeks on whether she should be reinstated or formally removed
from office. Even if she resumes the presidency, her five-year term ends
in February, after which she can face criminal charges.

On Monday, Ms. Park's lawyer, Yu Young-ha, rejected the special
prosecutor's findings, saying his investigation was ``politically
biased'' and ``lacking in fairness.'' He called the bribery allegation
``an absurd fiction.''

But on Monday, Mr. Park, the special prosecutor, who is not related to
Ms. Park, said his team found enough evidence that Ms. Park and her
confidante, Choi Soon-sil, conspired to collect bribes from Samsung.

On Feb. 28, he
\href{https://www.nytimes.com/2017/02/28/world/asia/lee-jae-yong-samsung.html}{indicted
Lee Jae-yong}, the third-generation scion of the family that runs
Samsung, on charges of giving or promising \$38 million in bribes to Ms.
Park and Ms. Choi. He also added a bribery charge to the case against
Ms. Choi, who is already on trial.

Mr. Lee offered the bribes in return for political favors from Ms. Park,
most notably government support for a merger of two Samsung affiliates
in 2015 that helped him inherit corporate control of the Samsung
conglomerate from his incapacitated father, Lee Kun-hee, the prosecutor
said.

Acting on Ms. Park's order, her aides forced the government-controlled
National Pension Service, a major shareholder at the two Samsung
companies, to vote for the merger, though it was opposed by many
minority shareholders and devalued the pension fund's own stocks there,
the prosecutor said.

\includegraphics{https://static01.nyt.com/images/2017/03/07/business/07park-3/07park-3-articleInline.jpg?quality=75\&auto=webp\&disable=upscale}

On Monday, Samsung denied the special prosecutor's findings.

``Samsung has not paid bribes nor made improper requests seeking
favors,'' it said in a statement. ``Future court proceedings will reveal
the truth.''

On Monday, Mr. Park, the special prosecutor, said that the president
should also face a criminal charge of abusing official power, saying she
conspired with aides
\href{https://www.nytimes.com/2017/01/12/world/asia/south-korea-president-park-blacklist-artists.html}{to
blacklist thousands of artists, writers and movie directors} deemed
unfriendly to her government and exclude them from government-funded
support programs.

Ms. Park also fired three senior Culture Ministry officials who had been
reluctant to discriminate against some of the 9,473 names on the list,
the prosecutor said. She demoted and later fired another senior ministry
official who had angered Ms. Choi, her friend, by
\href{https://www.nytimes.com/2016/11/12/world/asia/south-korea-park-geun-hye.html}{investigating
allegations of corruption} involving her family, the prosecutor said.

While blackballing unfriendly artists, Ms. Park's office ensured that
pro-government civic groups received special favors, he said.

It asked the Federation of Korean Industries, which lobbies on behalf of
Samsung and other big businesses, to provide \$5.9 million for those
groups between 2014 and 2016, the special prosecutor said. Some of those
groups, like the right-wing Korea Parent Federation, have held noisy
protests in downtown Seoul calling the critics of Ms. Park ``commies.''

Besides Samsung, scores of other South Korean companies were found to
have made payments to two foundations controlled by Ms. Choi. But on
Monday, the special prosecutor did not recommend further actions against
them, and state prosecutors had earlier said that those companies were
coerced to donate and were not engaged in bribery.

Ms. Park has repeatedly denied any legal wrongdoing, insisting that she
was framed by hostile political forces and that she was not aware of any
criminal conspiracy by Ms. Choi. She said she only let Ms. Choi edit
some of her speeches and run her personal errands.

On Monday, the special prosecutor said Ms. Park and Ms. Choi had 573
phone conversations between April and October last year using cellphones
issued under borrowed names. Of these calls, 127 took place between
September, when Ms. Choi left for Germany, and October, when she
returned home to be arrested.

The prosecutor accused Ms. Park of impeding his investigation. She
refused to be questioned by his investigators and also did not allow
them to search her office. As a result, he said his team could not fully
determine what she was doing at her residence for seven hours in April
2014, when a
\href{https://www.nytimes.com/2014/04/17/world/asia/south-korean-ferry-accident.html}{ferry
loaded with hundreds of schoolchildren sank}, killing more than 300.

Ms. Park said she was working at the time, getting reports on the
disaster. But she has been
\href{https://www.nytimes.com/2015/12/18/world/asia/south-korea-park-geun-hye-defamation-verdict.html}{haunted
by lurid rumors}, some of them claiming that she was having a romantic
encounter or undergoing plastic surgery.

On Monday, the prosecutor said a cosmetic surgeon gave Ms. Park at least
five simple face-lifting operations at her residence between 2013 and
2016. Even unlicensed people visited her there to give her nutritional
shots and help her with kinesiotherapy and reiki, a form of traditional
healing. But investigators could not find evidence that such things took
place on the day of the ferry disaster.

Advertisement

\protect\hyperlink{after-bottom}{Continue reading the main story}

\hypertarget{site-index}{%
\subsection{Site Index}\label{site-index}}

\hypertarget{site-information-navigation}{%
\subsection{Site Information
Navigation}\label{site-information-navigation}}

\begin{itemize}
\tightlist
\item
  \href{https://help.nytimes.com/hc/en-us/articles/115014792127-Copyright-notice}{©~2020~The
  New York Times Company}
\end{itemize}

\begin{itemize}
\tightlist
\item
  \href{https://www.nytco.com/}{NYTCo}
\item
  \href{https://help.nytimes.com/hc/en-us/articles/115015385887-Contact-Us}{Contact
  Us}
\item
  \href{https://www.nytco.com/careers/}{Work with us}
\item
  \href{https://nytmediakit.com/}{Advertise}
\item
  \href{http://www.tbrandstudio.com/}{T Brand Studio}
\item
  \href{https://www.nytimes.com/privacy/cookie-policy\#how-do-i-manage-trackers}{Your
  Ad Choices}
\item
  \href{https://www.nytimes.com/privacy}{Privacy}
\item
  \href{https://help.nytimes.com/hc/en-us/articles/115014893428-Terms-of-service}{Terms
  of Service}
\item
  \href{https://help.nytimes.com/hc/en-us/articles/115014893968-Terms-of-sale}{Terms
  of Sale}
\item
  \href{https://spiderbites.nytimes.com}{Site Map}
\item
  \href{https://help.nytimes.com/hc/en-us}{Help}
\item
  \href{https://www.nytimes.com/subscription?campaignId=37WXW}{Subscriptions}
\end{itemize}
