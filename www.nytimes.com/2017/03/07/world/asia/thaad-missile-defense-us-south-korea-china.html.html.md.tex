Sections

SEARCH

\protect\hyperlink{site-content}{Skip to
content}\protect\hyperlink{site-index}{Skip to site index}

\href{https://www.nytimes.com/section/world/asia}{Asia Pacific}

\href{https://myaccount.nytimes.com/auth/login?response_type=cookie\&client_id=vi}{}

\href{https://www.nytimes.com/section/todayspaper}{Today's Paper}

\href{/section/world/asia}{Asia Pacific}\textbar{}China Warns of Arms
Race After U.S. Deploys Missile Defense in South Korea

\url{https://nyti.ms/2naes8D}

\begin{itemize}
\item
\item
\item
\item
\item
\item
\end{itemize}

Advertisement

\protect\hyperlink{after-top}{Continue reading the main story}

Supported by

\protect\hyperlink{after-sponsor}{Continue reading the main story}

\hypertarget{china-warns-of-arms-race-after-us-deploys-missile-defense-in-south-korea}{%
\section{China Warns of Arms Race After U.S. Deploys Missile Defense in
South
Korea}\label{china-warns-of-arms-race-after-us-deploys-missile-defense-in-south-korea}}

\includegraphics{https://static01.nyt.com/images/2017/03/08/world/08thaad-4/08thaad-4-articleInline.jpg?quality=75\&auto=webp\&disable=upscale}

By \href{http://www.nytimes.com/by/gerry-mullany}{Gerry Mullany} and
\href{http://www.nytimes.com/by/chris-buckley}{Chris Buckley}

\begin{itemize}
\item
  March 7, 2017
\item
  \begin{itemize}
  \item
  \item
  \item
  \item
  \item
  \item
  \end{itemize}
\end{itemize}

\href{http://cn.nytimes.com/china/20170308/thaad-missile-defense-us-south-korea-china/}{阅读简体中文版}

HONG KONG --- The United States said on Tuesday that it had
\href{https://www.nytimes.com/2017/03/06/world/asia/north-korea-thaad-missile-defense-us-china.html}{begun
deploying} an advanced and contentious missile defense system in South
Korea, prompting China to warn of a new atomic arms race in a region
increasingly on edge over North Korea's drive to build a nuclear
arsenal.

The American announcement came a day after the
\href{https://www.nytimes.com/2017/03/05/world/north-korea-ballistic-missiles.html}{simultaneous
launch of four missiles} by North Korea into waters off the Japanese
coast, which Pyongyang said was a drill for striking American bases in
Japan. The feat, footage of which was broadcast on state television,
raised concern about the North's ability to
\href{https://www.nytimes.com/2017/03/06/world/asia/north-korea-missiles-japan.html}{overwhelm
the new defense system} being deployed.

Hours later, North Korea further unnerved the region by declaring it was
blocking all
\href{https://www.nytimes.com/2017/03/07/world/asia/kim-jong-nam-north-korea-malaysia-travel-ban.html?hp\&action=click\&pgtype=Homepage\&clickSource=story-heading\&module=first-column-region\&region=top-news\&WT.nav=top-news\&_r=0}{Malaysians
from leaving its soil}, sharply escalating a dispute over last month's
\href{https://www.nytimes.com/2017/02/22/world/asia/kim-jong-nam-assassination-korea-malaysia.html}{assassination
of Kim Jong-nam}, the half brother of North Korea's dictator, Kim
Jong-un.

Malaysia has accused several North Korean citizens of using
\href{https://www.nytimes.com/2017/02/24/world/asia/vx-nerve-agent-kim-jong-nam.html}{VX
nerve agent}to kill Mr. Kim in a case that has reminded the world of
Pyongyang's access to a stockpile of banned chemical weapons on top of
its nuclear program --- and its willingness to take extreme measures.

\includegraphics{https://static01.nyt.com/images/2017/03/07/world/asia/07thaad-1/07thaad-1-videoSixteenByNine3000.jpg}

The flurry of developments heightened anxiety in Asia over signs that
Pyongyang is closing in on its goal of developing an intercontinental
missile that can deliver a nuclear payload to the United States --- and
what the new Trump administration might do to prevent it. And they came
as the United States and South Korea participated in large-scale
military exercises that North Korea has condemned.

The
\href{https://www.nytimes.com/2017/03/04/world/asia/north-korea-missile-program-sabotage.html}{New
York Times reported Sunday} that President Trump's national security
deputies have discussed both the possibility of pre-emptive strikes that
would almost certainly provoke an attack on South Korea and a
reintroduction of nuclear weapons to the South. Intelligence officials
say North Korea is already able
\href{https://www.nytimes.com/2016/05/07/world/asia/north-korea-nuclear-us-strategy.html}{to
hit much of South Korea and Japan} with a nuclear-tipped missile.

A spokesman for the Chinese Ministry of Foreign Affairs, Geng Shuang,
denounced the United States' decision to deploy the Terminal High
Altitude Area Defense system, or Thaad, and vowed that Beijing would
``take the necessary steps to safeguard our own security interests.''

``The consequences will be shouldered by the United States and South
Korea,'' Mr. Geng added, warning that the two countries should not ``go
further and further down the wrong road.''

For days, the official Chinese news media has warned that deployment of
Thaad could lead to a ``de facto'' break in relations with South Korea
and urged consumers to boycott South Korean products. The Chinese
authorities recently forced the closing of 23 stores owned by Lotte, a
South Korean conglomerate that agreed to turn over land that it owned
for use in the Thaad deployment, and hundreds of Chinese protested at
Lotte stores over the weekend, some holding banners that read, ``Get out
of China.''

Xinhua, the official Chinese news agency, warned that Thaad ``will bring
an arms race in the region,'' likening the defensive system to a shield
that would prompt the development of new spears. ``More missile shields
of one side inevitably bring more nuclear missiles of the opposing side
that can break through the missile shield,'' it said.

But in another article, the news agency rebuked North Korea, saying it
must ``face the reality that it can neither thwart Washington and Seoul
nor consolidate its security in a breeze with its immature nuclear
technology.''

The United States' decision to deploy the missile technology brought new
scrutiny to China's policies toward North and South Korea and suggested
that its attempts to please both countries in hopes of averting a crisis
had fallen short.

``To put it bluntly using a common Chinese expression, it has wanted to
have a foot in two boats,'' said Deng Yuwen, a current affairs
commentator in Beijing who has sharply criticized North Korea.

Yang Xiyu, a former senior Chinese official who once oversaw talks with
North Korea, said China was worried that the deployment of the system
would open the door to a broader American network of antimissile systems
in the region, possibly in places like Japan and the Philippines, to
counter China's growing military as much as North Korea.

``China can see benefits only for a U.S. regional plan, not for South
Korea's national security interest,'' he said.

The developments come as South Korea is consumed by turmoil over the
impeachment of President
\href{http://topics.nytimes.com/top/reference/timestopics/people/p/park_geunhye/index.html?inline=nyt-per}{Park
Geun-hye}, whose administration agreed to the Thaad deployment. But with
the president facing
\href{https://www.nytimes.com/2016/12/09/world/asia/south-korea-president-park-geun-hye-impeached.html}{possible
removal from office} over a corruption scandal, the fate of the system
has been in doubt. Its accelerated deployment could make it harder, if
not impossible, for her successor to head off its installation.

Moon Jae-in, an opposition leader who is the front-runner in the race to
replace President Park, acknowledged that it would be difficult to
overturn South Korea's agreement to deploy the system. But he has
insisted that the next South Korean government should have the final say
on the matter, saying that Ms. Park's government never allowed a full
debate on it.

Under its deal with Washington, South Korea is providing the land for
the missile system and will build the base, but the United States will
pay for the system, to be built by Lockheed Martin, as well as its
operational costs.

A C-17 cargo plane landed at the United States military's Osan Air Base,
about 40 miles south of Seoul, on Monday evening, carrying two trucks,
each mounted with a Thaad launchpad. More equipment and personnel will
start arriving in the coming weeks, South Korean military officials
said.

The South Korean Defense Ministry declined to specify when the system
would be operational. But the South Korean news agency Yonhap reported
that the deployment was likely to be completed in one or two months,
with the system ready for use by April.

Paul Haenle, director of the Carnegie-Tsinghua Center at Tsinghua
University in Beijing, said that policy makers in China had failed to
grasp how Washington and its allies regarded North Korea's nuclear
program as getting closer to a dangerous threshold of being able to
place a warhead on an intercontinental ballistic missile that could hit
American cities.

``That's a game-changer,'' said Mr. Haenle, who was director for China
on the National Security Council under Presidents George W. Bush and
Barack Obama.

China has long opposed American missile defenses, in part because of
fears that they might embolden American decision-makers to consider a
first strike to destroy China's relatively small nuclear arsenal.
Chinese strategists warn that the United States might consider such an
attack if it was confident a defense system could intercept Chinese
weapons that escaped destruction.

China is believed to have already embarked on a program to modernize its
arsenal and develop new weapons designed to avoid missile defenses, and
analysts said the deployment of Thaad could prompt it to accelerate
those efforts.

Takashi Kawakami, a professor of international politics and security at
Takushoku University in Tokyo, said the deployment of Thaad could put
the United States in a stronger position to consider a pre-emptive
strike on North Korea. If the United States took such action, he said,
``North Korea is going to make a counterattack on the U.S. or Japan or
another place, so in this case they will use Thaad'' to defend against
the North's missiles.

The Japanese prime minister, Shinzo Abe, said he spoke for 25 minutes on
Tuesday with Mr. Trump, who reiterated his pledge to stand by Japan
``100 percent,'' according to the public broadcaster NHK. ``I appreciate
that the United States is showing that all the options are on the
table,'' Mr. Abe said, adding that Japan was ``ready to fulfill larger
roles and responsibilities'' to deter North Korea.

Advertisement

\protect\hyperlink{after-bottom}{Continue reading the main story}

\hypertarget{site-index}{%
\subsection{Site Index}\label{site-index}}

\hypertarget{site-information-navigation}{%
\subsection{Site Information
Navigation}\label{site-information-navigation}}

\begin{itemize}
\tightlist
\item
  \href{https://help.nytimes.com/hc/en-us/articles/115014792127-Copyright-notice}{©~2020~The
  New York Times Company}
\end{itemize}

\begin{itemize}
\tightlist
\item
  \href{https://www.nytco.com/}{NYTCo}
\item
  \href{https://help.nytimes.com/hc/en-us/articles/115015385887-Contact-Us}{Contact
  Us}
\item
  \href{https://www.nytco.com/careers/}{Work with us}
\item
  \href{https://nytmediakit.com/}{Advertise}
\item
  \href{http://www.tbrandstudio.com/}{T Brand Studio}
\item
  \href{https://www.nytimes.com/privacy/cookie-policy\#how-do-i-manage-trackers}{Your
  Ad Choices}
\item
  \href{https://www.nytimes.com/privacy}{Privacy}
\item
  \href{https://help.nytimes.com/hc/en-us/articles/115014893428-Terms-of-service}{Terms
  of Service}
\item
  \href{https://help.nytimes.com/hc/en-us/articles/115014893968-Terms-of-sale}{Terms
  of Sale}
\item
  \href{https://spiderbites.nytimes.com}{Site Map}
\item
  \href{https://help.nytimes.com/hc/en-us}{Help}
\item
  \href{https://www.nytimes.com/subscription?campaignId=37WXW}{Subscriptions}
\end{itemize}
