Sections

SEARCH

\protect\hyperlink{site-content}{Skip to
content}\protect\hyperlink{site-index}{Skip to site index}

\href{https://www.nytimes.com/section/world/asia}{Asia Pacific}

\href{https://myaccount.nytimes.com/auth/login?response_type=cookie\&client_id=vi}{}

\href{https://www.nytimes.com/section/todayspaper}{Today's Paper}

\href{/section/world/asia}{Asia Pacific}\textbar{}U.S. and South Korea
Rebuff China's Proposal to Defuse Korea Tensions

\url{https://nyti.ms/2mEuafC}

\begin{itemize}
\item
\item
\item
\item
\item
\end{itemize}

Advertisement

\protect\hyperlink{after-top}{Continue reading the main story}

Supported by

\protect\hyperlink{after-sponsor}{Continue reading the main story}

\hypertarget{us-and-south-korea-rebuff-chinas-proposal-to-defuse-korea-tensions}{%
\section{U.S. and South Korea Rebuff China's Proposal to Defuse Korea
Tensions}\label{us-and-south-korea-rebuff-chinas-proposal-to-defuse-korea-tensions}}

\includegraphics{https://static01.nyt.com/images/2017/03/09/world/09korea-1/09korea-1-articleInline.jpg?quality=75\&auto=webp\&disable=upscale}

By \href{http://www.nytimes.com/by/chris-buckley}{Chris Buckley} and
\href{http://www.nytimes.com/by/somini-sengupta}{Somini Sengupta}

\begin{itemize}
\item
  March 8, 2017
\item
  \begin{itemize}
  \item
  \item
  \item
  \item
  \item
  \end{itemize}
\end{itemize}

BEIJING --- China tried unsuccessfully to calm newly volatile tensions
on the Korean Peninsula on Wednesday, proposing that North Korea freeze
nuclear and missile programs in exchange for a halt to major military
exercises by American and South Korean forces. The proposal was rejected
hours later by the United States and South Korea.

``We have to see some sort of positive action by North Korea before we
can take them seriously,'' Nikki R. Haley, the United States ambassador
to the United Nations, told reporters after a Security Council meeting
in New York on the escalating Korea crisis. Standing beside her, Cho
Tae-yul, the South Korean ambassador, said, ``This is not the time for
us to talk about freezing or dialogue with North Korea.''

The statements by Ms. Haley and her South Korean counterpart came hours
after China's foreign minister, Wang Yi, proposed the suspensions during
a Beijing news conference, describing them as a way to create the basis
for talks that would end North Korea's nuclear ambitions.

The alternative to talks, he said, would be an increasingly perilous
standoff that threatened the entire region.

``The two sides are like two accelerating trains coming toward each
other, and neither side is willing to give way,'' Mr. Wang said. ``The
question is: Are both sides really prepared for a head-on collision?''

But in what appeared to be a hardening American position on North Korea,
Ms. Haley said the United States was re-evaluating its approach to the
country and its unpredictable young leader, Kim Jong-un, whom she
described as ``not rational.''

``I can tell you we're not ruling anything out, and we're considering
every option,'' Ms. Haley said after the Security Council meeting,
flanked by Mr. Cho and the Japanese ambassador to the United Nations,
Koro Bessho.

At the same time, Ms. Haley sought to reassure China publicly that the
United States meant no harm by moving ahead with the deployment of a
defensive missile shield system in South Korea, after North Korea's
\href{https://www.nytimes.com/2017/03/05/world/north-korea-ballistic-missiles.html?rref=collection\%2Fsectioncollection\%2Fasia\&action=click\&contentCollection=asia\&region=stream\&module=stream_unit\&version=latest\&contentPlacement=16\&pgtype=sectionfront}{missile
launch} on Monday. China has condemned the missile shield as a
provocation by the Americans that risked a
\href{https://www.nytimes.com/2017/03/07/world/asia/thaad-missile-defense-us-south-korea-china.html?rref=collection\%2Fsectioncollection\%2Fasia}{new
arms race} in the region.

Developments this week have abruptly escalated regional tensions over
the isolated North's nuclear arms development.

Image

The Chinese foreign minister, Wang Yi, said at a news conference on
Tuesday that the priority in the dispute over North Korea's nuclear
program was ``to flash the red light and apply brakes.''Credit...Mark
Schiefelbein/Associated Press

The North is also in a diplomatic standoff with Malaysia after the Feb.
13
\href{https://www.nytimes.com/2017/02/22/world/asia/kim-jong-nam-assassination-korea-malaysia.html}{killing
of Kim Jong-nam}, the North Korean leader's estranged half brother, in
Kuala Lumpur. On Tuesday, Pyongyang --- angered by a police
investigation that has named several North Koreans as suspects --- said
that no Malaysians living in North Korea would be allowed to leave the
country, and Malaysia quickly responded in kind.

On Wednesday, Mr. Wang said the priority in the dispute over North
Korea's nuclear program was now ``to flash the red light and apply
brakes.'' China's ``suspension for suspension'' proposal ``can help us
break out of the security dilemma and bring the parties back to the
negotiating table,'' he said.

Doubts that the idea would gain traction were not surprising. North
Korea
\href{https://www.nytimes.com/2015/01/11/world/asia/north-korea-offers-us-deal-to-halt-nuclear-test-.html}{made
a similar offer} in 2015 that went nowhere.

Mr. Wang's proposal was China's latest attempt to regain the initiative
on the nuclear issue, which has bedeviled Beijing's efforts to stay
friends with both North and South Korea and prove itself a mature
regional power broker.

``The current situation is a challenge for the Chinese government's
diplomacy,'' said Cheng Xiaohe, an associate professor at Renmin
University in Beijing who specializes in North Korea. ``The situation in
the East Asian region is increasingly complicated, and the possibility
of a diplomatic solution to the nuclear missile issue is increasingly
slim,'' he said, referring to North Korea's nuclear arms program.

Reining in North Korea has also become a focus for the Trump
administration's dealings with China. Starting next week, Secretary of
State Rex W. Tillerson is to visit Japan, South Korea and China for
talks that will focus on ``the advancing nuclear and missile threat''
from North Korea, the State Department said.

North Korea's weapons advancements have reached a point where ``we do
need to look at other alternatives,'' Mark C. Toner, a spokesman for the
State Department,
\href{https://www.state.gov/r/pa/prs/dpb/2017/03/268288.htm}{told
reporters} in Washington on Tuesday. ``And that's part of what this trip
is about, that we're going to talk to our allies and partners in the
region to try to generate a new approach to North Korea.''

But bringing the countries into agreement over initial steps toward
peace will not be easy, especially while China is also in a deepening
dispute with South Korea and the Trump administration. At the same news
conference where he laid out his proposal on Tuesday, Mr. Wang stuck to
China's fierce opposition to the missile defense system the United
States began assembling in South Korea this week, known as Thaad, or
Terminal High-Altitude Area Defense.

The Chinese government says the system goes far beyond its declared
purpose of warding off potential attacks by North Korea and could
undermine China's military security. American and South Korean officials
say that that is untrue, and that China should instead focus on halting
North Korea's threats.

``It's common knowledge that the monitoring and early warning radius of
Thaad reaches far beyond the Korean Peninsula and compromises China's
strategic security,'' Mr. Wang said at the news conference, which was
part of a regular round of briefings during China's annual legislative
session. ``It's not the way that neighbors should treat each other, and
it may very well make South Korea less secure.''

\includegraphics{https://static01.nyt.com/images/2017/03/09/world/09korea-2/09korea-2-articleInline.jpg?quality=75\&auto=webp\&disable=upscale}

Mr. Wang's proposal for mutual suspensions was an attempt to give new
life to China's long-running efforts to tamp down confrontation between
North and South Korea. China is the North's only major economic and
security partner, but it has also developed strong economic and
political ties with South Korea that the missile defense system
threatens to rupture.

For years, China hosted six-country talks on North Korea's nuclear
program, which brought together North and South Korea, China, Japan,
Russia and the United States.

But those talks fell apart in 2009, and North Korea has continued to
test nuclear weapons and refine missiles that could eventually carry
nuclear warheads as far as the continental United States. North Korea
described its launch on Monday of four ballistic missiles as practice
for hitting American military bases in Japan.

American officials, and many Chinese experts, have grown skeptical that
North Korea would ever seriously contemplate giving up its nuclear
weapons.

China's rift with South Korea and the United States over the missile
defense system is likely to embolden North Korea, making it more
confident that Beijing would not turn on it, said Shen Dingli, a
professor at Fudan University in Shanghai who specializes in nuclear
proliferation issues.

``The deployment of Thaad has led to a serious deterioration in
Chinese-South Korean relations, so North Korea is delighted with that,''
Dr. Shen said in an interview. North Korea appears to have passed the
point where it would abandon its nuclear arms, he said. ``There's no
solution to this, because North Korea won't give up its nuclear
weapons.''

But Mr. Wang said negotiations were the only acceptable way to resolve
the dispute.

``To resolve the nuclear issue, we have to walk on both legs,'' he said,
``which means not just implementing sanctions, but also restarting
talks.''

North Korea's ties to the global financial system are also under renewed
pressure. On Wednesday, the Society for Worldwide Interbank Financial
Telecommunication, or Swift, issued a statement saying it had recently
moved to ban North Korean banks from accessing its platform.

Swift operates as part of the backbone of global bank payment processing
by providing a communication platform used by central banks and
financial institutions around the world.

Several North Korean banks that were subject to sanctions by both the
United Nations and the United States had continued as recently as last
year to find ways to access the Swift network, according to a report by
a United Nations expert panel that was published last week. Swift said
it was responding to an enforcement action by the authorities in
Belgium, where Swift is based, but it did not say when it moved to block
the North Korean banks from its service.

Advertisement

\protect\hyperlink{after-bottom}{Continue reading the main story}

\hypertarget{site-index}{%
\subsection{Site Index}\label{site-index}}

\hypertarget{site-information-navigation}{%
\subsection{Site Information
Navigation}\label{site-information-navigation}}

\begin{itemize}
\tightlist
\item
  \href{https://help.nytimes.com/hc/en-us/articles/115014792127-Copyright-notice}{©~2020~The
  New York Times Company}
\end{itemize}

\begin{itemize}
\tightlist
\item
  \href{https://www.nytco.com/}{NYTCo}
\item
  \href{https://help.nytimes.com/hc/en-us/articles/115015385887-Contact-Us}{Contact
  Us}
\item
  \href{https://www.nytco.com/careers/}{Work with us}
\item
  \href{https://nytmediakit.com/}{Advertise}
\item
  \href{http://www.tbrandstudio.com/}{T Brand Studio}
\item
  \href{https://www.nytimes.com/privacy/cookie-policy\#how-do-i-manage-trackers}{Your
  Ad Choices}
\item
  \href{https://www.nytimes.com/privacy}{Privacy}
\item
  \href{https://help.nytimes.com/hc/en-us/articles/115014893428-Terms-of-service}{Terms
  of Service}
\item
  \href{https://help.nytimes.com/hc/en-us/articles/115014893968-Terms-of-sale}{Terms
  of Sale}
\item
  \href{https://spiderbites.nytimes.com}{Site Map}
\item
  \href{https://help.nytimes.com/hc/en-us}{Help}
\item
  \href{https://www.nytimes.com/subscription?campaignId=37WXW}{Subscriptions}
\end{itemize}
