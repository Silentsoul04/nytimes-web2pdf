Sections

SEARCH

\protect\hyperlink{site-content}{Skip to
content}\protect\hyperlink{site-index}{Skip to site index}

\href{https://www.nytimes.com/section/business/economy}{Economy}

\href{https://myaccount.nytimes.com/auth/login?response_type=cookie\&client_id=vi}{}

\href{https://www.nytimes.com/section/todayspaper}{Today's Paper}

\href{/section/business/economy}{Economy}\textbar{}Steady U.S. Job
Growth Sets Stage for Fed to Raise Interest Rates

\url{https://nyti.ms/2msKMWV}

\begin{itemize}
\item
\item
\item
\item
\item
\item
\end{itemize}

Advertisement

\protect\hyperlink{after-top}{Continue reading the main story}

Supported by

\protect\hyperlink{after-sponsor}{Continue reading the main story}

\hypertarget{steady-us-job-growth-sets-stage-for-fed-to-raise-interest-rates}{%
\section{Steady U.S. Job Growth Sets Stage for Fed to Raise Interest
Rates}\label{steady-us-job-growth-sets-stage-for-fed-to-raise-interest-rates}}

By \href{http://www.nytimes.com/by/patricia-cohen}{Patricia Cohen}

\begin{itemize}
\item
  March 10, 2017
\item
  \begin{itemize}
  \item
  \item
  \item
  \item
  \item
  \item
  \end{itemize}
\end{itemize}

A wave of hiring in February --- President Trump's first full month in
office --- pointed to a strong foundation for the nation's economy,
providing further evidence for the Federal Reserve that the moment to
raise interest rates has come.

The \href{https://www.bls.gov/news.release/empsit.nr0.htm}{Labor
Department reported} a gain of 235,000 jobs and healthy wage growth in a
month when even the weather cooperated. It was the last major data
release before Fed policy makers meet Tuesday and Wednesday, when they
have signaled their intent to
\href{https://www.nytimes.com/2017/03/03/business/economy/federal-reserve-interest-rates.html}{increase
the benchmark interest rate}.

``The economy is riding a wave of bullish sentiment postelection,'' said
Andrew Chamberlain, chief economist at Glassdoor, a career website.
``We're seeing strong labor demand across the board and no sign of
slowing right now.''

Republicans and Democrats quickly jostled for credit.

\href{http://www.cbsnews.com/live/video/sean-spicer-speaks-on-tweets-on-job-reports/}{Sean
Spicer,} the White House press secretary, said Mr. Trump had
``jump-started job creation, not only through his executive action but
because of the surge in economic confidence and optimism that has been
inspired since his election.''

Mr. Trump, who, as a candidate, repeatedly
\href{https://www.nytimes.com/2016/11/05/opinion/donald-trumps-denial-of-economic-reality.html}{dismissed}
\href{https://www.nytimes.com/2016/11/05/opinion/donald-trumps-denial-of-economic-reality.html}{the
official jobs reports as ``phony},'' reposted a
\href{https://twitter.com/realDonaldTrump}{comment on Twitter} from the
conservative website Drudge Report that said, ``GREAT AGAIN: +235,000.''
Mr. Spicer later quoted Mr. Trump on his faith in the report, ``They may
have been phony in the past, but it's very real now.''

The Labor Department repeated that it had
\href{https://www.nytimes.com/2016/11/04/business/economy/unemployment-labor-department-data-politics.html}{not
changed the way it collected}and analyzed jobs data since Mr. Trump took
office. ``It's business as usual,'' said Megan Kindelan, director of
public affairs at the Bureau of Labor Statistics.

The Republican self-congratulation clearly irked Democrats. Tom Perez,
labor secretary in the Obama administration and now chairman of the
Democratic National Committee, countered that Mr. Trump had ``absolutely
nothing'' to do with the job gains. ``Trump inherited an economy from
Barack Obama with the longest streak of private sector job growth in
history,'' he said.

Although the economic anxiety that helped put Mr. Trump in the White
House remains, the official jobless rate is near what the Fed considers
\href{http://www.economist.com/blogs/economist-explains/2017/01/economist-explains-19}{full
employment} --- a threshold where, in theory at least, everyone who
wants a job at the going rate can find one. The official jobless rate
fell to 4.7 percent, from 4.8 percent in January, even as the overall
labor force grew.

At the same time, jobless claims are near a 44-year low, and the stock
market is surging. Revisions to previous estimates raised the
three-month average of monthly job gains to 209,000 and annual wage
growth to 2.8 percent, further bolstering the case for those who argue
the economy is strong enough to withstand a rate increase.

The overall economic momentum received a push
from\href{https://www.nytimes.com/2017/02/24/nyregion/warm-weather-hurts-winter-businesses.html}{February's
unusually warm weather}, with almost a quarter of the jobs --- about
58,000 --- coming from construction. Manufacturing and mining rose too.

Also significant was the increase in the labor participation rate to 63
percent, a result of rising employment even among people without a high
school diploma. ``There's got to be some optimism that these people are
feeling they finally have a chance,'' said Diane Swonk, founder and
chief executive of DS economics in Chicago.

On the other end are employers who are seeing acute labor shortages.
``They're offering training programs now,'' Ms. Swonk said. ``They're
complaining about it. But that's what tight labor markets do. It forces
you to invest more to work with less.''

Bigger paychecks are something that most Americans are particularly
eager to see, after years of stagnant wage growth. The
\href{https://www.nytimes.com/2017/03/03/business/economy/federal-reserve-interest-rates.html}{Fed},
too, has been waiting for an increase, but it is also wary of wages
rising too fast. Its members want to head off incipient inflation
without putting the brakes on hiring, especially because the benefits of
the eight-year-old recovery have been so unevenly distributed.

Balancing those two goals is tricky.

Lauren Griffin, senior vice president at Adecco Staffing USA, said the
scarcity of qualified workers had compelled employers to raise wages,
strengthen benefits and improve amenities at the office. ``We've got
people in orientation classes,'' Ms. Griffin said, ``and they get up and
leave because they're contacted about another job that might be more
money.''

At the same time, a broader measure of unemployment --- which includes
the millions of Americans who have given up looking for work or who are
working part time but would prefer full-time jobs --- dropped to 9.2
percent last month but is still high given how tight the labor market
looks otherwise.

Cautioning the Fed against moving too quickly with a rate increase,
Elise Gould, an economist at the left-leaning
\href{http://www.epi.org/people/josh-bivens/}{Economic Policy
Institute}, noted that, ``Workers throughout the economy, including
young workers, workers of color, and low-wage workers, need a chance to
make up lost ground on wage growth.''

Many Americans who live outside urban centers also have been excluded
from most of the rewards of the recovery.

Large metropolitan counties have had more than twice the annual wage
growth of nonmetropolitan areas, according to the
\href{https://www.bls.gov/news.release/srgune.nr0.htm}{latest figures}
from the Bureau of Labor Statistics.

``Higher-wage jobs might be following educated, young workers, who are
increasingly living in dense, urban neighborhoods as other demographic
groups move to the suburbs,'' said Jed Kolko, chief economist at Indeed,
a job-search site. ``Broader economic shifts also favor big cities: The
occupations projected to grow tend to be more urban, while shrinking
sectors like manufacturing and farming tend to be located outside large
metros.''

That is disappointing for people with longstanding ties to smaller, more
rural communities. ``A lot of this has to do with mobility,'' said
Steven W. Rick, chief economist at CUNA Mutual Group, an insurance
company. ``People are going to have to move where the jobs are and not
expect the jobs to come where they are.''

Although the Trump administration has had little time to make any
substantial policy changes, the expectation of a reduction in taxes and
regulations and the possibility of vast infrastructure spending have
created optimism among employers and blue-collar workers.

Mr. Trump has promised to expand the economy by 4 percent a year, create
25 million jobs in the next decade, revive manufacturing and reduce the
trade deficit.

Achieving all that would be difficult in the best of circumstances, let
alone with the potential headwinds facing the White House. Dissension
among Republicans and the unpredictability of Mr. Trump's course in
several policy areas could dampen job growth.

The future of the Affordable Care Act and
\href{https://www.nytimes.com/2017/03/07/upshot/why-even-some-republicans-are-rejecting-the-replacement-bill.html}{a
possible replacement} is making hospitals and community health centers
cautious about adding workers. A strong dollar and a potential backlash
against the White House's travel ban could slow tourism and hiring in
the sector. And Mr. Trump's across-the-board hiring freeze on federal
government jobs, combined with declines at the state level, will
probably reduce the total number of public sector employees.

The
\href{https://www.nytimes.com/2017/02/16/business/yourtaxes/tax-code-republicans.html}{uncertainty
extends to prospects for tax cuts}. Some Wall Street analysts, expecting
delays, have pared their growth forecasts for 2017, after recently
raising them.

Certainly the snapshot of February's labor market is good. The question
is, if the economy does slow, whether Mr. Trump will accept the
legitimacy of weak reports as enthusiastically as he does good ones.

Mr. Spicer suggested the president would. ``Numbers are going to go up
and down,'' he said. ``We recognize that.''

Advertisement

\protect\hyperlink{after-bottom}{Continue reading the main story}

\hypertarget{site-index}{%
\subsection{Site Index}\label{site-index}}

\hypertarget{site-information-navigation}{%
\subsection{Site Information
Navigation}\label{site-information-navigation}}

\begin{itemize}
\tightlist
\item
  \href{https://help.nytimes.com/hc/en-us/articles/115014792127-Copyright-notice}{©~2020~The
  New York Times Company}
\end{itemize}

\begin{itemize}
\tightlist
\item
  \href{https://www.nytco.com/}{NYTCo}
\item
  \href{https://help.nytimes.com/hc/en-us/articles/115015385887-Contact-Us}{Contact
  Us}
\item
  \href{https://www.nytco.com/careers/}{Work with us}
\item
  \href{https://nytmediakit.com/}{Advertise}
\item
  \href{http://www.tbrandstudio.com/}{T Brand Studio}
\item
  \href{https://www.nytimes.com/privacy/cookie-policy\#how-do-i-manage-trackers}{Your
  Ad Choices}
\item
  \href{https://www.nytimes.com/privacy}{Privacy}
\item
  \href{https://help.nytimes.com/hc/en-us/articles/115014893428-Terms-of-service}{Terms
  of Service}
\item
  \href{https://help.nytimes.com/hc/en-us/articles/115014893968-Terms-of-sale}{Terms
  of Sale}
\item
  \href{https://spiderbites.nytimes.com}{Site Map}
\item
  \href{https://help.nytimes.com/hc/en-us}{Help}
\item
  \href{https://www.nytimes.com/subscription?campaignId=37WXW}{Subscriptions}
\end{itemize}
