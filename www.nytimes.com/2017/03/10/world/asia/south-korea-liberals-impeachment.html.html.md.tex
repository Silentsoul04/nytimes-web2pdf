Sections

SEARCH

\protect\hyperlink{site-content}{Skip to
content}\protect\hyperlink{site-index}{Skip to site index}

\href{https://www.nytimes.com/section/world/asia}{Asia Pacific}

\href{https://myaccount.nytimes.com/auth/login?response_type=cookie\&client_id=vi}{}

\href{https://www.nytimes.com/section/todayspaper}{Today's Paper}

\href{/section/world/asia}{Asia Pacific}\textbar{}Ouster of South Korean
President Could Return Liberals to Power

\url{https://nyti.ms/2mbOVf4}

\begin{itemize}
\item
\item
\item
\item
\item
\end{itemize}

Advertisement

\protect\hyperlink{after-top}{Continue reading the main story}

Supported by

\protect\hyperlink{after-sponsor}{Continue reading the main story}

\hypertarget{ouster-of-south-korean-president-could-return-liberals-to-power}{%
\section{Ouster of South Korean President Could Return Liberals to
Power}\label{ouster-of-south-korean-president-could-return-liberals-to-power}}

\includegraphics{https://static01.nyt.com/images/2017/03/11/world/11Korea1/11Korea1-articleInline.jpg?quality=75\&auto=webp\&disable=upscale}

By \href{http://www.nytimes.com/by/choe-sang-hun}{Choe Sang-Hun}

\begin{itemize}
\item
  March 10, 2017
\item
  \begin{itemize}
  \item
  \item
  \item
  \item
  \item
  \end{itemize}
\end{itemize}

SEOUL, South Korea --- The foes of South Korea's likely new leaders have
called them blindly naïve, closet North Korea followers and
anti-American --- an unsettling accusation in a country where the
alliance with Washington has been the military bedrock for seven
decades.

Now, after being out of power for almost 10 years, the South Korean
liberal opposition is on the verge of retaking the presidency with the
historic court ruling on Friday that
\href{https://www.nytimes.com/2017/03/09/world/asia/park-geun-hye-impeached-south-korea.html}{ousted
its conservative enemy, President Park Geun-hye}, who had been impeached
in a corruption scandal.

The liberals' presidential hopeful, Moon Jae-in, wants a profound change
in the country's tense relations with North Korea, pushing outreach and
dialogue. He also is deeply skeptical of the hawkish stance embraced by
the conservatives and South Korea's most important defender, the United
States.

Mr. Moon and his liberal partners are especially worried about a new
antimissile shield the Americans are installing in South Korea, citing
China's fury over it and warning of a
\href{https://www.nytimes.com/2017/03/07/world/asia/korea\%2Dmissile\%2Ddefense\%2Dchina\%2Dtrump.html}{standoff
reminiscent of the Cuban Missile Crisis.}

Much has changed since South Korea's liberals were running the country.
The North not only has amassed nuclear weapons but claims it can fit
them atop missiles. It has
\href{https://www.nytimes.com/2017/03/05/world/north-korea-ballistic-missiles.html}{successfully
launched missiles that could hit Japan}, and could soon perfect models
that could reach the United States.

North Korea now has an unpredictable 33-year-old leader, Kim Jong-un,
who styles himself in the godlike image of his grandfather, the hermetic
country's Communist revolutionary founder. He appears even more obsessed
with nuclear weapons than his father and predecessor,
\href{http://www.nytimes.com/2011/12/19/world/asia/kim-jong-il-is-dead.html?pagewanted=all}{Kim
Jong-il}, and has threatened South Korea and its allies with pre-emptive
nuclear attack.

The challenges for Mr. Moon and his liberal partners as they push to
reclaim power in elections now scheduled for May will be how to engage
with a far more dangerous North Korea, maintain close ties with the
United States and repair relations with China, which increasingly
mistrusts the American military's intentions.

Mr. Moon has called himself ``America's friend,'' grateful that the
United States protected South Korea from communism and supported its
economic growth and democratization. The alliance with Washington is ``a
pillar of our diplomacy,'' he said in an interview on the eve of Ms.
Park's court-ordered ouster.

But he also said in a recently published book that South Korea should
learn to ``say no to the Americans.''

Mr. Moon's ascent could seriously complicate the American rush this past
week to deploy the new advanced missile-defense system, known as
\href{https://www.mda.mil/system/thaad.html}{Terminal High Altitude Area
Defense}, or Thaad, in the South.

He and his liberal associates have questioned the deployment, calling it
an unnecessary escalation of tensions on the Korean Peninsula. The
missile system would put sophisticated American weaponry on China's
doorstep, and has infuriated the Chinese so much they are
\href{https://www.nytimes.com/2017/03/09/world/asia/china-lotte-thaad-south-korea.html}{boycotting
South Korean brands} and may now be less willing to use economic
leverage to rein in the North.

Ms. Park's government considered Thaad a centerpiece in defending
against the growing North Korean missile threat. But Mr. Moon vowed to
review the deployment if elected.

``I cannot understand why there should be such a hurry with this,'' he
said. ``I suspect that they are trying to make it a fait accompli, make
it a political issue to be used in the election.''

As of Friday, the Trump administration had said nothing publicly about
Ms. Park's removal and its implications for South Korea's relations with
the United States. But Mr. Trump's United Nations envoy,
\href{https://www.nytimes.com/2017/03/08/world/asia/china-north-korea-thaad-nuclear.html}{Nikki
R. Haley, made clear this week that she sees no point in dialogue with
North Korea's leader}, whom she described as ``not rational,'' and that
the Thaad deployment was not directed at China.

Mr. Moon said he abhorred ``the ruthless dictatorial regime of North
Korea.'' But he also said sanctions that the United States had enforced
with the conservatives in South Korea for a decade had failed to stop
North Korea's nuclear weapons program, so it was time to try something
less confrontational.

``We must embrace the North Korean people as part of the Korean nation,
and to do that, whether we like it or not, we must recognize Kim Jong-un
as their ruler and as our dialogue partner,'' Mr. Moon said.

\includegraphics{https://static01.nyt.com/images/2017/03/11/world/11Korea2/11Korea2-articleInline.jpg?quality=75\&auto=webp\&disable=upscale}

That idea is not entirely new. The last time the liberals were in power,
from 1998 to 2008, they pushed the so-called Sunshine Policy of
promoting aid and exchanges with North Korea in the hopes of building
trust and guiding it toward openness and nuclear disarmament.

The result was an unprecedented détente on the divided Korean Peninsula.
But the North persisted in its nuclear weapons pursuits, conducting its
first nuclear test in 2006.

It has since conducted four more. It also has demonstrated advances in
missile technology that have deeply worried not only the United States
but all countries in the region, including China, leading to multiple
sets of sanctions by the United Nations Security Council.

Analysts have said that engagement with the North is now riskier, given
the United Nations sanctions, and is likely costlier, given how much
North Korea's nuclear stockpile has grown. But Mr. Moon said the
strategy pursued by the conservatives was simply not working.

``What have the conservative governments done, except badmouthing the
North?'' he said. ``If necessary, we will have to strengthen sanctions
even further, but the goal of sanctions must be to bring North Korea
back to the negotiating table.''

The possibility that Mr. Moon could become South Korea's next president
in a few months also comes as the Trump administration is formulating a
new policy on North Korea. Ms. Haley said on Tuesday that ``we're
considering every option.''

``I hope that Mr. Trump will come to the same conclusion as I did,'' Mr.
Moon said.

As part of his outreach to the North, Mr. Moon said he would reopen the
joint-venture factory complex the two Koreas had run in the North Korean
town of Kaesong. After Mr. Kim's government conducted a nuclear test in
January last year, Ms. Park
\href{https://www.nytimes.com/2016/02/11/world/asia/north-south-korea-kaesong.html}{closed
the Kaesong venture}, saying it violated United Nations sanctions.

Analysts of the conflict said it was premature to assess the outcome of
Ms. Park's removal.

``A giant leap forward for South Korean democracy, a major step backward
for taming Pyongyang,'' said
\href{http://fletcher.tufts.edu/Fletcher_Directory/Directory/Faculty\%20Profile?personkey=43424CD6-91DF-469E-83CC-E9DD75B6B913}{Lee
Sung-yoon}, a North Korea expert at Tufts University's Fletcher School
of Law and Diplomacy.

For the past decade, South Korea has been governed by conservative
leaders who regard anything less than unequivocal support for the
alliance with Washington and the downfall of the Kim Jong-un government
as ideologically suspect.

Their liberal adversaries would bring in a new mentality.

Mr. Moon grew up in a family that hated the Communist rule in the North
but also had roots there. His parents were among tens of thousands of
Korean War refugees on the run from Communists who were evacuated from
the North Korean port of Hungnam by retreating American Navy vessels in
the winter of 1950.

But Mr. Moon also belonged to the generation of South Koreans who shaped
their perspectives by fighting the anti-North Korean, pro-American
military dictators, like Park Chung-hee, Ms. Park's father, who ruled
South Korea from 1961 to 1979.

In college in the 1970s, Mr. Moon said he was profoundly influenced by
Rhee Young-hee, a dissident journalist who wrote a book criticizing the
Vietnam War. The military government, which sent its troops to fight for
the Americans in Vietnam, banned the book and arrested the writer.

``Until then, we were taught to think that whatever the United States
did was justice, whatever the United States said was truth, and that
whoever argued otherwise was an evil to repel,'' Mr. Moon wrote in his
2011 memoir, ``Fate.'' ``The book helped lift the veil of falsehood.''

Some of the old student activists, like Mr. Moon, from the 1970s and
'80s form the mainstay of the political opposition of today.

They do not want their country dragged into what they see as a hegemonic
struggle among big powers, while conservatives have no qualms about
siding with the United States in its rivalry with China. The Thaad
antimissile deployment showcased those opposing perspectives.

As the Americans pushed ahead with the deployment this week in an
apparent attempt to embed the system before a new government takes
power, China retaliated by forcing South Korean businesses there to
close and banning Chinese tourists from visiting the South.

``The United States is pushing us to the West unnecessarily, and China
is shoving us to the East unnecessarily,'' said Kim Ki-jung, a political
scientist at Yonsei University in Seoul and foreign policy adviser for
Mr. Moon. ``They should not push us too much; part of the Korean DNA is
resistance against big powers.''

Advertisement

\protect\hyperlink{after-bottom}{Continue reading the main story}

\hypertarget{site-index}{%
\subsection{Site Index}\label{site-index}}

\hypertarget{site-information-navigation}{%
\subsection{Site Information
Navigation}\label{site-information-navigation}}

\begin{itemize}
\tightlist
\item
  \href{https://help.nytimes.com/hc/en-us/articles/115014792127-Copyright-notice}{©~2020~The
  New York Times Company}
\end{itemize}

\begin{itemize}
\tightlist
\item
  \href{https://www.nytco.com/}{NYTCo}
\item
  \href{https://help.nytimes.com/hc/en-us/articles/115015385887-Contact-Us}{Contact
  Us}
\item
  \href{https://www.nytco.com/careers/}{Work with us}
\item
  \href{https://nytmediakit.com/}{Advertise}
\item
  \href{http://www.tbrandstudio.com/}{T Brand Studio}
\item
  \href{https://www.nytimes.com/privacy/cookie-policy\#how-do-i-manage-trackers}{Your
  Ad Choices}
\item
  \href{https://www.nytimes.com/privacy}{Privacy}
\item
  \href{https://help.nytimes.com/hc/en-us/articles/115014893428-Terms-of-service}{Terms
  of Service}
\item
  \href{https://help.nytimes.com/hc/en-us/articles/115014893968-Terms-of-sale}{Terms
  of Sale}
\item
  \href{https://spiderbites.nytimes.com}{Site Map}
\item
  \href{https://help.nytimes.com/hc/en-us}{Help}
\item
  \href{https://www.nytimes.com/subscription?campaignId=37WXW}{Subscriptions}
\end{itemize}
