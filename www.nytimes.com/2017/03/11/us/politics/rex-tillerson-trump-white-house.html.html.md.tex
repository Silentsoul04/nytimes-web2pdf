Sections

SEARCH

\protect\hyperlink{site-content}{Skip to
content}\protect\hyperlink{site-index}{Skip to site index}

\href{https://www.nytimes.com/section/politics}{Politics}

\href{https://myaccount.nytimes.com/auth/login?response_type=cookie\&client_id=vi}{}

\href{https://www.nytimes.com/section/todayspaper}{Today's Paper}

\href{/section/politics}{Politics}\textbar{}Tillerson Leads From State
Dept. Shadows as White House Steps In

\url{https://nyti.ms/2mdmtcU}

\begin{itemize}
\item
\item
\item
\item
\item
\end{itemize}

Advertisement

\protect\hyperlink{after-top}{Continue reading the main story}

Supported by

\protect\hyperlink{after-sponsor}{Continue reading the main story}

Diplomatic Memo

\hypertarget{tillerson-leads-from-state-dept-shadows-as-white-house-steps-in}{%
\section{Tillerson Leads From State Dept. Shadows as White House Steps
In}\label{tillerson-leads-from-state-dept-shadows-as-white-house-steps-in}}

\includegraphics{https://static01.nyt.com/images/2017/03/12/us/12-Tillerson/12-Tillerson-articleInline.jpg?quality=75\&auto=webp\&disable=upscale}

By \href{http://www.nytimes.com/by/david-e-sanger}{David E. Sanger}

\begin{itemize}
\item
  March 11, 2017
\item
  \begin{itemize}
  \item
  \item
  \item
  \item
  \item
  \end{itemize}
\end{itemize}

WASHINGTON --- Henry A. Kissinger slipped into the State Department last
week for a quiet lunch in his old office with Rex W. Tillerson, the
former Exxon Mobil chief executive, who has all but covered himself in a
cloak of invisibility in his first six weeks as secretary of state.

Describing his impressions, Mr. Kissinger, perhaps America's most famous
diplomatic strategist, chose his words judiciously. ``The normal
tendency when you come into that job is to increase your visibility and
to show that you are present and in charge,'' he said in an interview.
``He wanted to first inform himself of all the nuances. I was impressed
by the confidence and self-assurance that he showed.''

But in the Washington of Donald J. Trump, where foreign policy
proclamations often appear first on Twitter, and where White House
advisers are still battling for dominance, this approach can be seen as
brilliant, mystifying or a prescription for powerlessness.

Mr. Tillerson has skipped every opportunity to define his views or give
guidance to American diplomats abroad, limiting himself to terse,
scripted statements, taking no questions from reporters and offering no
public protest when the White House proposed cutting the State
Department budget by 37 percent without first consulting him.

He suffered in silence, State Department officials said, when President
Trump called, in a matter-of-fact way, to reject Mr. Tillerson's choice
for deputy secretary of state. He has been absent from the White House
meetings with key world leaders, and when the State Department issued
its annual report on human rights --- usually a major moment for the
United States to stand up against repression around the world --- he
skipped the announcement.

Defenders say Mr. Tillerson has been accomplishing far more behind the
scenes, including arranging for the first trip of a Saudi foreign
minister to Iraq in more than a quarter-century --- his first foray into
the sinkhole of Middle East politics.

``He's already developing plans to begin ratcheting back Putin's
nefarious behavior,'' Senator Bob Corker, the chairman of the Senate
Foreign Relations Committee, said in an interview --- steps that would
represent the first known effort by the new administration to face off
against President Vladimir V. Putin of Russia.

``He's won status and respect of the president, of McMaster, and talks
all the time to Jared,'' the senator said, referring to the national
security adviser, Lt. Gen. H. R. McMaster, and Jared Kushner, Mr.
Trump's son-in-law, who has emerged as a prominent voice on American
foreign policy.

``He doesn't mind at all that these stories are being written about him
being missing,'' Mr. Corker, a Tennessee Republican, said about Mr.
Tillerson. ``When he's ready to talk, you will be very highly
impressed.''

On Tuesday, Mr. Tillerson will leave for his first truly fraught
diplomatic mission: a trip to Japan, South Korea and China, at a moment
when open conflict with North Korea is a growing possibility, and when
the administration is planning Mr. Trump's first meeting with President
Xi Jinping of China. The trip is so vital that the ``principals''
committee of the National Security Council is set to convene on Monday
to discuss the North Korean threat and how to deal with China, so that
Mr. Tillerson speaks from a consensus strategy.

But do not expect to hear much about that strategy from the secretary
before he arrives in Asia: The State Department has told reporters that
they cannot ride on the plane. The decision appears to be unprecedented
for a major diplomatic trip --- even four decades ago, when Mr.
Kissinger was conducting shuttle diplomacy in the Middle East and
opening up China, he was delivering spin to reporters on the plane and
offering up diplomatic tutorials.

``All his predecessors have traveled with press,'' said R. Nicholas
Burns, who served as spokesman, ambassador and under secretary of state
in both Republican and Democratic administrations. Failing to do so, he
noted, creates the risk that the secretary of state will be defined by
the country he is visiting --- especially a place like China.

Within the State Department, Mr. Tillerson, 64, got off to a promising
start with a warm, humble greeting to staff members in the drab
headquarters' flag-draped foyer on his first day on the job. He talked
about his upbringing and his wife's belief that he had been preparing
for this job his whole life, even if he had not known it.

But few have heard from him since. Those who have say they regard him as
an impressive manager who knows how to run a crisp meeting, take in a
variety of views and give little away about his own.

``He forces everyone to boil their memos down to a page or two, so they
really have to think about what their message is,'' said one official
who has dealt with him frequently in recent weeks. ``He's already met
with two of the most important Chinese officials. He knows a lot about
some countries many secretaries don't know about,'' including Indonesia
and others that have energy assets. He understands what embassies do,
because Exxon Mobil often relied on them for help.

But he is also introverted, a bit standoffish. He never met in person
with John Kerry, his predecessor. ``These guys came in to drain the
swamp,'' one career State Department official said, ``and it's clear
that they are under orders not to cooperate or deal with swamp
creatures.''

So, for thousands in the State Department, Mr. Tillerson has come to be
viewed as the phantom of Foggy Bottom, scarcely glimpsed and known
mostly for his directives to wipe out some of the department's top jobs.

Longtime career officials who expected to stay in their jobs or remain
long enough to show their successors the ropes were ousted in Mr.
Tillerson's first two weeks. He is talking to members of Congress about
further cuts, and while there are plenty of opportunities in a
department that has not exactly embraced technological change, the major
reductions proposed by the administration are a nonstarter with many
lawmakers.

The biggest concern among diplomats and many in Congress is that when
Mr. Trump talks about bolstering America's commitment to its national
security, he does not have diplomacy in mind. Longtime diplomats often
cite --- or email to reporters --- a line uttered four years ago by the
new defense secretary, Jim Mattis, when he was in charge of Central
Command.

``If you don't fund the State Department fully, then I need to buy more
ammunition,'' Mr. Mattis said at the time. As one diplomat who has met
frequently with Mr. Tillerson since he took office noted recently, ``Rex
clearly agrees with that. He just won't say it.''

(A senior State Department official said Mr. Tillerson did say it, to
Mr. Trump, over dinner a little more than a week ago.)

On his first trip, to Europe, Mr. Tillerson went out of his way to
reassure allies of the United States' commitment to NATO, doing little
to repeat the ``America First'' notion that Mr. Trump has promoted. In
Asia, Mr. Tillerson is scheduled to visit the Demilitarized Zone on the
border with North Korea, and it seems almost unimaginable that he would
repeat Mr. Trump's warning as a candidate that the United States may
pull back from the region.

So why is the man many in the State Department call T. Rex so quiet?
Secretaries of state from both parties have relished their role as chief
spokesman for American values.

Madeleine K. Albright made her name describing the United States as the
``indispensable nation'' that needed to intervene in the Balkans. Colin
L. Powell took the lead in making the case for invading Iraq (words he
later regretted). Hillary Clinton, under President Barack Obama,
highlighted human and women's rights in particular. Mr. Kerry narrated
his own role as relentless negotiator, sometimes using background
briefings to read aloud from copious handwritten notes he had taken
while haggling over the Iran nuclear deal and the removal of chemical
weapons from Syria. In indiscreet moments, he talked about his
differences of view, mostly on Syria, with Mr. Obama.

There are several theories about Mr. Tillerson's reticence.

One is that his silence is highly strategic: He wants to cement key
relationships in private, make sure he is aligned with a mercurial
president and let the policy process at the National Security Council
play out before making any grand pronouncements.

The second is that he is waiting for the battles at the White House to
burn out. In short, he wants to sidestep Stephen K. Bannon, the
president's top strategist, who believes that China's rise can be halted
and that Iran should be vigorously confronted, and work with Mr. Mattis,
Mr. Kushner and Mr. McMaster. Mr. Corker said that ``he's already
reached an agreement with Mattis to come to agreement and present ideas
together,'' something that Condoleezza Rice and Mrs. Clinton often did
with their defense counterpart, Robert M. Gates.

The third is that he sees the job as more akin to what he did at Exxon
Mobil: Cut your deals, say as little as possible and take the heat.

One of the first tests may come in the arena of human rights, where he
caused alarm during his confirmation hearings in January by declining to
criticize the killings ordered by the Philippine president, Rodrigo
Duterte, in an antidrug campaign.

Speaking out against repression has always been fraught; Mr. Kerry often
danced around the topic when visiting Egypt. But Mr. Tillerson took up
the issue on a recent call with Senator Benjamin L. Cardin, Democrat of
Maryland, who was pressing him to link arms sales to Bahrain --- home of
the Fifth Fleet of the United States Navy --- to measurable human rights
improvements.

``When we get the notifications of arms sales, it will be interesting to
see if he has something to say on the issues we talked about,'' Mr.
Cardin said in an interview. He attributed some of Mr. Tillerson's
problems to the fact that ``he doesn't have his team in place.''

``Having a subcabinet to back up your views,'' Mr. Cardin added, ``gives
you the confidence to be more direct.''

Clearly, Mr. Tillerson will not have much of a staff for a while; not a
single under secretary or assistant secretary --- the people who make
the policy wheels turn --- has been nominated, and only a couple of
ambassadors have been named.

Some say the problem is not with Mr. Tillerson, but those he works for.

``Rex Tillerson is off to a slow start, but the White House is partly to
blame,'' said Richard Haass, the president of the Council on Foreign
Relations, whom the administration briefly considered for a top post.
``The president needs to give the secretary the staff he wants; protect,
not decimate, his budget; and make clear to the world that it is the
secretary and no one else who speaks for the administration when it
comes to foreign policy.''

Mr. Kissinger, at 93, is philosophical about most things, including Mr.
Tillerson. ``I would expect that as foreign policy evolves, Rex
Tillerson will become an increasingly prominent exponent of it,'' he
said. ``When I first came to Washington'' as national security adviser
to President Richard M. Nixon, ``you would find me mentioned in The New
York Times maybe 10 times.''

A computer index suggests that the actual number in his first year was
around 228, but who's counting? Mr. Tillerson, in a sign of progress,
has already exceeded that figure this year.

Advertisement

\protect\hyperlink{after-bottom}{Continue reading the main story}

\hypertarget{site-index}{%
\subsection{Site Index}\label{site-index}}

\hypertarget{site-information-navigation}{%
\subsection{Site Information
Navigation}\label{site-information-navigation}}

\begin{itemize}
\tightlist
\item
  \href{https://help.nytimes.com/hc/en-us/articles/115014792127-Copyright-notice}{©~2020~The
  New York Times Company}
\end{itemize}

\begin{itemize}
\tightlist
\item
  \href{https://www.nytco.com/}{NYTCo}
\item
  \href{https://help.nytimes.com/hc/en-us/articles/115015385887-Contact-Us}{Contact
  Us}
\item
  \href{https://www.nytco.com/careers/}{Work with us}
\item
  \href{https://nytmediakit.com/}{Advertise}
\item
  \href{http://www.tbrandstudio.com/}{T Brand Studio}
\item
  \href{https://www.nytimes.com/privacy/cookie-policy\#how-do-i-manage-trackers}{Your
  Ad Choices}
\item
  \href{https://www.nytimes.com/privacy}{Privacy}
\item
  \href{https://help.nytimes.com/hc/en-us/articles/115014893428-Terms-of-service}{Terms
  of Service}
\item
  \href{https://help.nytimes.com/hc/en-us/articles/115014893968-Terms-of-sale}{Terms
  of Sale}
\item
  \href{https://spiderbites.nytimes.com}{Site Map}
\item
  \href{https://help.nytimes.com/hc/en-us}{Help}
\item
  \href{https://www.nytimes.com/subscription?campaignId=37WXW}{Subscriptions}
\end{itemize}
