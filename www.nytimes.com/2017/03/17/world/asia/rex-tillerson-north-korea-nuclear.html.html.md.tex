Sections

SEARCH

\protect\hyperlink{site-content}{Skip to
content}\protect\hyperlink{site-index}{Skip to site index}

\href{https://www.nytimes.com/section/world/asia}{Asia Pacific}

\href{https://myaccount.nytimes.com/auth/login?response_type=cookie\&client_id=vi}{}

\href{https://www.nytimes.com/section/todayspaper}{Today's Paper}

\href{/section/world/asia}{Asia Pacific}\textbar{}Rex Tillerson Rejects
Talks With North Korea on Nuclear Program

\url{https://nyti.ms/2m9yTqn}

\begin{itemize}
\item
\item
\item
\item
\item
\item
\end{itemize}

Advertisement

\protect\hyperlink{after-top}{Continue reading the main story}

Supported by

\protect\hyperlink{after-sponsor}{Continue reading the main story}

\hypertarget{rex-tillerson-rejects-talks-with-north-korea-on-nuclear-program}{%
\section{Rex Tillerson Rejects Talks With North Korea on Nuclear
Program}\label{rex-tillerson-rejects-talks-with-north-korea-on-nuclear-program}}

\includegraphics{https://static01.nyt.com/images/2017/03/18/world/18tillerson-1/18tillerson-1-videoSixteenByNineJumbo1600.jpg}

By \href{http://www.nytimes.com/by/david-e-sanger}{David E. Sanger}

\begin{itemize}
\item
  March 17, 2017
\item
  \begin{itemize}
  \item
  \item
  \item
  \item
  \item
  \item
  \end{itemize}
\end{itemize}

SEOUL, South Korea --- Secretary of State Rex W. Tillerson ruled out on
Friday opening any negotiation with North Korea to freeze its nuclear
and missile programs and said for the first time that the Trump
administration might be forced to take pre-emptive action ``if they
elevate the threat of their weapons program'' to an unacceptable level.

Mr. Tillerson's comments in Seoul, a day before he travels to Beijing to
meet Chinese leaders, explicitly rejected any return to the bargaining
table in an effort to buy time by halting North Korea's
\href{https://www.nytimes.com/2017/02/17/world/asia/north-korea-nuclear-threat.html}{accelerating
testing program}. The country's leader, Kim Jong-un, said on New Year's
Day that North Korea was in the
\href{https://www.nytimes.com/2017/01/01/world/asia/north-korea-intercontinental-ballistic-missile-test-kim-jong-un.html}{``final
stage''} of preparation for the first launch of an intercontinental
ballistic missile that could reach the United States.

The secretary of state's comments were the Trump administration's first
public hint at the options being considered, and they made clear that
none involved a negotiated settlement or waiting for the North Korean
government to collapse.

``The policy of strategic patience has ended,'' Mr. Tillerson said, a
reference to the term used by the Obama administration to describe a
policy of waiting out the North Koreans, while gradually ratcheting up
sanctions and covert action.

Negotiations ``can only be achieved by denuclearizing, giving up their
weapons of mass destruction,'' he said --- a step to which the North
committed in 1992, and again in subsequent accords, but has always
violated. ``Only then will we be prepared to engage them in talks.''

His warning on Friday about new ways to pressure the North was far more
specific and martial sounding than
\href{https://www.nytimes.com/2017/03/16/world/asia/rex-tillerson-asia-trump-us-japan.html}{during
the first stop of his three-country tour}, in Tokyo on Thursday. His
inconsistency of tone may have been intended to signal a tougher line to
the Chinese before he lands in Beijing on Saturday. It could also
reflect an effort by Mr. Tillerson, the former
\href{https://www.nytimes.com/interactive/2017/01/11/us/politics/rex-tillerson-exxon-maverick-oil-diplomacy.html}{chief
executive of Exxon Mobil}, to issue the right diplomatic signals in a
region where American commitment is in doubt.

Almost exactly a year ago, when Donald J. Trump was still a presidential
candidate, he threatened
\href{https://www.nytimes.com/2016/03/27/us/politics/donald-trump-foreign-policy.html}{in
an interview with The New York Times} to pull troops back from the
Pacific region unless South Korea and Japan paid a greater share of the
cost of keeping them there. During Mr. Tillerson's stops in South Korea
and Japan, there was no public talk of that demand.

On Friday afternoon, after visiting the Demilitarized Zone and peering
into North Korean territory in what has become a ritual for American
officials making a first visit to the South, Mr. Tillerson explicitly
rejected
\href{https://www.nytimes.com/2017/03/08/world/asia/china-north-korea-thaad-nuclear.html}{a
Chinese proposal} to get the North Koreans to freeze their testing in
return for the United States and South Korea suspending all annual joint
military exercises, which are now underway.

Mr. Tillerson argued that a freeze would essentially enshrine ``a
comprehensive set of capabilities'' North Korea possesses that already
pose too great a threat to the United States and its allies, and he said
there would be no negotiation until the North agreed to dismantle its
programs.

Mr. Tillerson ignored a question about whether the Trump administration
would double down on the use of cyberweapons against the North's missile
development,
\href{https://www.nytimes.com/2017/03/04/world/asia/north-korea-missile-program-sabotage.html}{a
covert program that President Barack Obama accelerated early in 2014}
and that so far has yielded mixed results.

\includegraphics{https://static01.nyt.com/images/2017/03/18/world/18tillerson-2/18tillerson-2-articleLarge.jpg?quality=75\&auto=webp\&disable=upscale}

Instead, Mr. Tillerson referred vaguely to a ``number of steps'' the
United States could take --- a phrase that seemed to embrace much more
vigorous enforcement of sanctions, ramping up missile defenses, cutting
off North Korea's oil, intensifying the cyberwar program and striking
the North's known missile sites.

The rejection of negotiations on a freeze would be consistent with the
approach taken by Mr. Obama, who declined Chinese offers to restart the
so-called six-party talks that stalled several years ago unless the
North agreed at the outset that the goal of the negotiations was the
``complete, verifiable, irreversible'' dismantling of its program.

But classified assessments of the North that the Obama administration
left for its successors included a grim assessment by the intelligence
community: that North Korea's leader, Mr. Kim, believes his nuclear
weapons program is the only way to guarantee the survival of his regime
and will never trade it away for economic or other benefits.

The assessment said that the example of what happened to Col. Muammar
el-Qaddafi, the longtime leader of Libya, had played a critical role in
North Korean thinking.
\href{http://www.nytimes.com/2003/12/20/world/libya-to-give-up-arms-programs-bush-announces.html}{Colonel
Qaddafi gave up the components of Libya's nuclear program} in late 2003
--- most of them were still in crates from Pakistan --- in hopes of
economic integration with the West. Eight years later, when the Arab
Spring broke out, the United States and its European allies joined
forces to depose Colonel Qaddafi, who was eventually found hiding in a
ditch and
\href{http://www.nytimes.com/2011/10/21/world/africa/qaddafi-killed-as-hometown-falls-to-libyan-rebels.html}{executed
by Libyan rebels}.

Among many experts, the idea of a freeze has been favored as the least
terrible of a series of bad options. Jon Wolfsthal, a nuclear expert who
worked on Mr. Obama's National Security Council, and Toby Dalton
\href{http://www.politico.com/magazine/story/2017/03/can-trump-stop-kim-jong-un-214910}{wrote
recently in Politico}: ``A temporary freeze on missile and nuclear
developments sounds better than an unconstrained and growing threat. It
is also, possibly, the most logical and necessary first step toward an
overall agreement between the U.S. and North Korea. But the risk that
North Korea will cheat or hide facilities during a negotiated freeze is
great.''

William J. Perry, who was secretary of defense under President Bill
Clinton, argued on Friday that it was no longer realistic to expect
North Korea to commit to dismantling or surrendering its nuclear
arsenal. The Trump administration, he said, should instead focus on
persuading the North to commit to a long-term freeze in which it
suspends testing of nuclear weapons and long-range missiles and pledges
not to sell or transfer any of its nuclear technology.

``I see very little prospect of a collapse,'' he added. ``For eight
years in the Obama administration and eight years in the Bush
administration, they were expecting that to happen. As a consequence,
their policies were not very effective.''

In Asia, on his first major trip overseas as secretary of state, Mr.
Tillerson has been heavily scripted in his few public comments, and he
has gone out of his way to make sure he is not subject to questions
beyond highly controlled news conferences, at which his staff chooses
the questioners. In a breach of past practice, he traveled without the
usual State Department press corps, which has flown on the secretary's
plane for roughly half a century.

That group of reporters, many of them veterans of foreign policy and
national security coverage, use the plane rides to try to get the
secretary and other top State Department officials to explain American
policy. Mr. Tillerson's aides first said their plane was too small to
accommodate the press corps and later said they were experimenting with
new forms of coverage; then they opened a seat for a reporter from the
web-based Independent Journal Review, which is aimed at younger,
conservative-leaning readers. The site's reporters have never traveled
with the secretary before.

That decision is a striking departure for the State Department. Last
May, department officials protested when Egypt's military leader, Abdel
Fattah el-Sisi,
\href{https://www.nytimes.com/2016/05/19/insider/egypt-john-kerry-abdel-fattah-el-sisi.html}{blocked
pool reporters traveling with Secretary John Kerry from entering the
presidential palace}, and China frequently imposes similar restrictions
to avoid unwanted questions to the Chinese leadership.

Mr. Tillerson appears to be using similar tactics during his travels,
though the two news conferences he held on the trip were his first since
taking office at the beginning of February.

Advertisement

\protect\hyperlink{after-bottom}{Continue reading the main story}

\hypertarget{site-index}{%
\subsection{Site Index}\label{site-index}}

\hypertarget{site-information-navigation}{%
\subsection{Site Information
Navigation}\label{site-information-navigation}}

\begin{itemize}
\tightlist
\item
  \href{https://help.nytimes.com/hc/en-us/articles/115014792127-Copyright-notice}{©~2020~The
  New York Times Company}
\end{itemize}

\begin{itemize}
\tightlist
\item
  \href{https://www.nytco.com/}{NYTCo}
\item
  \href{https://help.nytimes.com/hc/en-us/articles/115015385887-Contact-Us}{Contact
  Us}
\item
  \href{https://www.nytco.com/careers/}{Work with us}
\item
  \href{https://nytmediakit.com/}{Advertise}
\item
  \href{http://www.tbrandstudio.com/}{T Brand Studio}
\item
  \href{https://www.nytimes.com/privacy/cookie-policy\#how-do-i-manage-trackers}{Your
  Ad Choices}
\item
  \href{https://www.nytimes.com/privacy}{Privacy}
\item
  \href{https://help.nytimes.com/hc/en-us/articles/115014893428-Terms-of-service}{Terms
  of Service}
\item
  \href{https://help.nytimes.com/hc/en-us/articles/115014893968-Terms-of-sale}{Terms
  of Sale}
\item
  \href{https://spiderbites.nytimes.com}{Site Map}
\item
  \href{https://help.nytimes.com/hc/en-us}{Help}
\item
  \href{https://www.nytimes.com/subscription?campaignId=37WXW}{Subscriptions}
\end{itemize}
