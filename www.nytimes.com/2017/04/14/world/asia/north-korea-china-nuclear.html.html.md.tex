Sections

SEARCH

\protect\hyperlink{site-content}{Skip to
content}\protect\hyperlink{site-index}{Skip to site index}

\href{https://www.nytimes.com/section/world/asia}{Asia Pacific}

\href{https://myaccount.nytimes.com/auth/login?response_type=cookie\&client_id=vi}{}

\href{https://www.nytimes.com/section/todayspaper}{Today's Paper}

\href{/section/world/asia}{Asia Pacific}\textbar{}China Warns of `Storm
Clouds Gathering' in U.S.-North Korea Standoff

\url{https://nyti.ms/2pACxqq}

\begin{itemize}
\item
\item
\item
\item
\item
\item
\end{itemize}

Advertisement

\protect\hyperlink{after-top}{Continue reading the main story}

Supported by

\protect\hyperlink{after-sponsor}{Continue reading the main story}

\hypertarget{china-warns-of-storm-clouds-gathering-in-us-north-korea-standoff}{%
\section{China Warns of `Storm Clouds Gathering' in U.S.-North Korea
Standoff}\label{china-warns-of-storm-clouds-gathering-in-us-north-korea-standoff}}

\includegraphics{https://static01.nyt.com/images/2017/04/15/world/15fears_web1/15fears_web1-articleLarge-v2.jpg?quality=75\&auto=webp\&disable=upscale}

By \href{http://www.nytimes.com/by/gerry-mullany}{Gerry Mullany},
\href{http://www.nytimes.com/by/chris-buckley}{Chris Buckley} and
\href{http://www.nytimes.com/by/david-e-sanger}{David E. Sanger}

\begin{itemize}
\item
  April 14, 2017
\item
  \begin{itemize}
  \item
  \item
  \item
  \item
  \item
  \item
  \end{itemize}
\end{itemize}

HONG KONG ---
\href{https://www.nytimes.com/topic/destination/china?inline=nyt-geo}{China}
warned on Friday that tensions on the Korean Peninsula could spin out of
control, as
\href{https://www.nytimes.com/topic/destination/north-korea?8qa}{North
Korea} said it could test a nuclear weapon at any time and a United
States naval group neared the peninsula --- an American effort to sow
doubt in Pyongyang over how President Trump might respond.

``The United States and South Korea and North Korea are engaging in tit
for tat, with swords drawn and bows bent, and there have been storm
clouds gathering,'' China's foreign minister, Wang Yi, said in Beijing,
\href{http://news.xinhuanet.com/politics/2017-04/14/c_1120812558.htm}{according
to Xinhua}, the state news agency.

``If they let war break out on the peninsula, they must shoulder that
historical culpability and pay the corresponding price for this,'' Mr.
Wang said.

The comments were unusually blunt from China, which has been trying to
steer between the Trump administration's demands for it to do more to
stop North Korea's nuclear weapons program and its longstanding
reluctance to risk a rupture with the North. The remarks also reflected,
American experts said, an effort by the Chinese to throw responsibility
for what happens back on Washington, after Mr. Trump declared, in
several Twitter messages, that it was up to the Chinese to contain their
neighbor and sometime partner.

In a telephone conversation with Mr. Trump on Wednesday, China's
president, Xi Jinping, also
\href{http://www.fmprc.gov.cn/mfa_eng/zxxx_662805/t1453741.shtml}{called
for restraint}. But behind the scenes, officials said, Mr. Trump and Mr.
Xi had reached some preliminary understandings, during their meeting at
the president's Mar-a-Lago resort a week ago, about what the Chinese
might do to change the behavior of the North's leader, Kim Jong-un.

\includegraphics{https://static01.nyt.com/images/2017/04/15/world/15fears-1/15fears-1-articleLarge.jpg?quality=75\&auto=webp\&disable=upscale}

According to officials who have seen notes of the conversations, the
Chinese have agreed to crack down on their second-tier banks that have
helped finance the North's trade. But it is unclear what that crackdown
would look like: While much has been made by Mr. Trump about North
Korean ``boats'' of coal that have been turned away by China, the most
recent statistics show a significant increase in overall trade between
the two countries.

American officials contend that the two countries have also agreed to
share some intelligence --- a highly unusual step --- about suspected
North Korean shipments of arms and other illicit goods. That would
improve the chances that those shipments can be intercepted, perhaps
when they make port calls. The Bush administration began such a program,
called the Proliferation Security Initiative, more than a decade ago,
but attention to it has waxed and waned.

The North Korean military issued a statement on Friday threatening to
attack major American military bases in
\href{https://www.nytimes.com/topic/destination/south-korea?8qa}{South
Korea}, as well as the presidential Blue House, warning that it could
annihilate those targets ``within minutes.''

Administration officials flatly denied a report on NBC News that the
United States was planning for a pre-emptive strike ahead of any nuclear
test. It was unclear what American forces would strike, and the nuclear
test site where the North has conducted its five previous tests would
make a hard-to-hit target. Moreover, they noted, Vice President Mike
Pence is scheduled to visit Seoul this weekend, and it is almost
impossible to imagine a strike occurring while he was consulting with
the South's acting president about how to respond to the crisis.

Even if a nuclear test occurs this weekend or in coming weeks, officials
say, the response is likely to be diplomatic, with a ramping up of
economic pressure and the deployment of more military assets. A carrier
group, led by the aircraft carrier Carl Vinson, is headed to the waters
off the peninsula. It includes Aegis cruisers with antimissile ability.

\href{https://www.nytimes.com/interactive/2017/04/12/world/asia/north-korea-nuclear-test.html}{}

\includegraphics{https://static01.nyt.com/images/2017/04/12/world/asia/north-korea-nuclear-test-1492043970939/north-korea-nuclear-test-1492043970939-square640-v3.jpg}

\hypertarget{north-korea-may-be-preparing-its-6th-nuclear-test}{%
\subsection{North Korea May Be Preparing Its 6th Nuclear
Test}\label{north-korea-may-be-preparing-its-6th-nuclear-test}}

Growing evidence suggests that North Korea may soon conduct another
underground detonation in its effort to make nuclear arms.

That is notable because administration officials say they are more
concerned about a test of an intercontinental ballistic missile that
could reach the United States --- a feat the North has never come close
to accomplishing --- than another nuclear test. According to one
official, it's the combination of a missile and a warhead that is most
worrisome.

The official, and others cited in this article, asked for anonymity to
discuss a matter of national security.

The North, for its part, issued a statement that denounced what it
called the Trump administration's ``maniacal military provocations,''
including the deployment of the carrier group.

``Nothing will be more foolish if the United States thinks it can deal
with us the way it treated Iraq and Libya, miserable victims of its
aggression, and Syria, which did not respond immediately even after it
was attacked,'' a spokesman for the general staff of the North's
People's Army said in a statement carried by Pyongyang's official Korean
Central News Agency.

North Korea's vice foreign minister, Han Song-ryol, said on Friday that
the United States was ``becoming more vicious and aggressive'' under Mr.
Trump and that ``we will go to war if they choose.''

Image

A preflight operations check on the United States aircraft carrier Carl
Vinson in the South China Sea last week. The carrier, along with other
warships, neared the Korean Peninsula this week.Credit...Matt Brown/U.S.
Navy, via Getty Images

Mr. Han told
\href{http://bigstory.ap.org/article/7db133ff3bf94648be84d48a1babf4e3/n-korean-official-us-more-vicious-aggressive-under-trump}{The
Associated Press} that whether North Korea holds another nuclear test
would be ``something that our headquarters decides.'' But he added an
ominous coda: ``At a time and at a place where the headquarters deems
necessary, it will take place.''

The speculation about an imminent underground detonation arises from
satellite photographs that show the test site is fully prepared, and
that holes into the site have been plugged, usually a last step to
contain radiation. But at times the North, knowing it is under space
surveillance, has readied the site but waited to conduct the test.

On Saturday, with Kim Jong-un watching from a raised platform, North
Korea began a military parade in central Pyongyang to celebrate the
105th anniversary of the birth of his grandfather, Kim Il-sung, the
North's founding president. The North sometimes uses such occasions to
show off its military advances.

Japan is clearly examining worst-case scenarios.

The Japanese news media reported that the government's National Security
Council had been discussing the possible evacuation of an estimated
57,000 Japanese citizens in South Korea, should war break out. ``We will
take all necessary steps to protect our people's lives and assets,''
said Yoshihide Suga, Japan's chief cabinet secretary. The Kyodo News
agency said the council was concerned about the possibility of North
Korean refugees arriving in boats on its shores.

Prime Minister Shinzo Abe expressed concern on Thursday that North Korea
could have the ability to deliver missiles equipped with sarin, the
nerve agent whose recent use against civilians in Syria prompted Mr.
Trump to
\href{https://www.nytimes.com/2017/04/07/us/politics/syria-strike-trump-timeline.html}{order
a missile strike} there.

Russia, another neighbor of North Korea, echoed China on Friday in
urging all parties to exercise caution. A Kremlin spokesman, Dmitri S.
Peskov, called on ``all the countries to refrain from any actions that
could amount to provocative steps,''
\href{http://www.reuters.com/article/us-northkorea-russia-idUSKBN17G0RB}{Reuters
reported}.

In a phone call on Friday between the Russian foreign minister, Sergey
V. Lavrov, and his Chinese counterpart, Mr. Wang, both said they would
try to revive talks over North Korea, according to the
\href{http://www.mfa.gov.cn/web/zyxw/t1453861.shtml}{Chinese Foreign
Ministry}.

In South Korea, whose people have lived through saber-rattling involving
the North for decades, there were few signs of panic. Nonetheless, the
South Korean Foreign Ministry warned on Friday that if the North
conducted another nuclear test or launched an intercontinental ballistic
missile, it would suffer an ``unbearably strong punishment.'' All the
major candidates in the presidential election set for next month have
called on the United States not to do anything that might initiate war
on the peninsula without first seeking the consent of South Korea, its
military ally.

In his remarks in Beijing, Mr. Wang said there was still hope for
renewed negotiations with North Korea on its weapons program. ``There
can also be flexibility about the form of renewed talks,'' he said.

Secretary of State Rex Tillerson, visiting Seoul in March, said the
United States would
\href{https://www.nytimes.com/2017/03/17/world/asia/rex-tillerson-north-korea-nuclear.html}{not
negotiate} with the North unless it first gave up both its nuclear and
missile programs. That was essentially a rejection of talks, since the
purpose of the negotiation would be to end those programs. But the North
has said it will never surrender what it calls its ``deterrent'' against
American aggression, and President Barack Obama never engaged in
prolonged discussions out of concern the North was just playing for
time.

The Chinese made much of their announcement two months ago that they
were suspending coal imports from North Korea. But while those shipments
seem to have dried up, Chinese overall trade rose 37.4 percent in the
first quarter of 2017, compared with the same period in 2016.

Chinese news outlets
\href{http://news.163.com/17/0414/19/CI0OMCIA0001875O.html}{reported on
Friday} that Air China, the country's main international airline, would
suspend flights to Pyongyang starting on Monday, leaving only Air Koryo
of North Korea operating regular flights between Pyongyang and Beijing
or other Chinese cities. The move appeared to have been in the works for
some time; NK News, a website about North Korea,
\href{https://www.nknews.org/2017/03/air-china-to-cancel-2017-north-korea-service-following-marathon/}{reported
last month} that the suspension was likely, saying that Air China's
services were underused and that its flights were often canceled.

Later, Air China issued a statement saying that it had not entirely
abandoned the route from Beijing to Pyongyang, and would arrange flights
if there was enough demand.

Advertisement

\protect\hyperlink{after-bottom}{Continue reading the main story}

\hypertarget{site-index}{%
\subsection{Site Index}\label{site-index}}

\hypertarget{site-information-navigation}{%
\subsection{Site Information
Navigation}\label{site-information-navigation}}

\begin{itemize}
\tightlist
\item
  \href{https://help.nytimes.com/hc/en-us/articles/115014792127-Copyright-notice}{©~2020~The
  New York Times Company}
\end{itemize}

\begin{itemize}
\tightlist
\item
  \href{https://www.nytco.com/}{NYTCo}
\item
  \href{https://help.nytimes.com/hc/en-us/articles/115015385887-Contact-Us}{Contact
  Us}
\item
  \href{https://www.nytco.com/careers/}{Work with us}
\item
  \href{https://nytmediakit.com/}{Advertise}
\item
  \href{http://www.tbrandstudio.com/}{T Brand Studio}
\item
  \href{https://www.nytimes.com/privacy/cookie-policy\#how-do-i-manage-trackers}{Your
  Ad Choices}
\item
  \href{https://www.nytimes.com/privacy}{Privacy}
\item
  \href{https://help.nytimes.com/hc/en-us/articles/115014893428-Terms-of-service}{Terms
  of Service}
\item
  \href{https://help.nytimes.com/hc/en-us/articles/115014893968-Terms-of-sale}{Terms
  of Sale}
\item
  \href{https://spiderbites.nytimes.com}{Site Map}
\item
  \href{https://help.nytimes.com/hc/en-us}{Help}
\item
  \href{https://www.nytimes.com/subscription?campaignId=37WXW}{Subscriptions}
\end{itemize}
