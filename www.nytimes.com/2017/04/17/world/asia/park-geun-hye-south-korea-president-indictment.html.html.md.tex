Sections

SEARCH

\protect\hyperlink{site-content}{Skip to
content}\protect\hyperlink{site-index}{Skip to site index}

\href{https://www.nytimes.com/section/world/asia}{Asia Pacific}

\href{https://myaccount.nytimes.com/auth/login?response_type=cookie\&client_id=vi}{}

\href{https://www.nytimes.com/section/todayspaper}{Today's Paper}

\href{/section/world/asia}{Asia Pacific}\textbar{}Park Geun-hye, Ousted
President of South Korea, Is Formally Indicted

\url{https://nyti.ms/2oOgzDj}

\begin{itemize}
\item
\item
\item
\item
\item
\end{itemize}

Advertisement

\protect\hyperlink{after-top}{Continue reading the main story}

Supported by

\protect\hyperlink{after-sponsor}{Continue reading the main story}

\hypertarget{park-geun-hye-ousted-president-of-south-korea-is-formally-indicted}{%
\section{Park Geun-hye, Ousted President of South Korea, Is Formally
Indicted}\label{park-geun-hye-ousted-president-of-south-korea-is-formally-indicted}}

\includegraphics{https://static01.nyt.com/images/2017/04/18/world/18skorea-1/18skorea-1-articleLarge.jpg?quality=75\&auto=webp\&disable=upscale}

By \href{http://www.nytimes.com/by/choe-sang-hun}{Choe Sang-Hun}

\begin{itemize}
\item
  April 17, 2017
\item
  \begin{itemize}
  \item
  \item
  \item
  \item
  \item
  \end{itemize}
\end{itemize}

SEOUL, South Korea --- South Korea's recently impeached and ousted
president, Park Geun-hye, was formally indicted on Monday on charges of
collecting or demanding \$52 million in bribes, becoming the first
leader put on criminal trial since the mid-1990s, when two former
military-backed presidents were imprisoned for corruption and mutiny.

Prosecutors
\href{https://www.nytimes.com/2017/03/30/world/asia/park-geun-hye-south-korea-arrest.html}{arrested
Ms. Park} on 13 criminal charges in March. They have questioned her five
times in her jail cell outside Seoul. In the indictment on Monday, the
number of criminal charges against Ms. Park increased to 18, including
bribery, coercion, abuse of office and illegal leaking of government
secrets.

The indictment, a widely expected follow-up to Ms. Park's arrest, will
prompt the Seoul Central District Court to open a trial. The court is
expected to assign the case to a three-judge panel soon.

The judges will then set the date for the first hearing in what will
become the biggest court trial since the former military dictator Chun
Doo-hwan was
\href{http://www.nytimes.com/1995/12/22/world/south-korea-indicts-2-former-presidents-in-staging-of-1979-coup.html}{sentenced
to death} and his friend and successor, Roh Tae-woo, was sentenced to
22½ years in prison on bribery, mutiny and sedition charges in 1996.
(Their sentences were later reduced, and they were pardoned and released
in 1997.)

Months of political turmoil and intrigue, set in motion when huge crowds
began gathering in central Seoul in the fall to
\href{https://www.nytimes.com/2016/11/26/world/asia/korea-park-geun-hye-protests.html}{demand
Ms. Park's resignation}, were capped by a Constitutional Court ruling in
early March that
\href{https://www.nytimes.com/2017/03/09/world/asia/park-geun-hye-impeached-south-korea.html}{formally
removed her} from office.

The National Assembly had
\href{https://www.nytimes.com/2016/12/09/world/asia/south-korea-president-park-geun-hye-impeached.html}{voted
in December to impeach her} on charges of bribery, extortion and abuse
of power.

The sprawling corruption scandal implicated the leadership of Samsung,
the nation's largest conglomerate, and other big businesses, rekindling
public furor over decades-old ties between government and corporations
in one of Asia's most vibrant democracies.

The coercion charge against Ms. Park stems from \$68 million she and a
\href{https://www.nytimes.com/2016/11/06/world/asia/south-koreans-ashamed-over-les-secretive-adviser.html}{longtime
confidante, Choi Soon-sil}, were accused of extorting from big
businesses in the form of ``donations'' to two foundations that Ms. Choi
controlled.

But a more damning charge against Ms. Park and Ms. Choi was bribery. The
\$52 million they were accused of collecting or demanding in bribes from
businesses included \$38 million in bribes or promised bribes from
Samsung. Ms. Choi and the company's top executive, Lee Jae-yong, were
also under arrest and on trial.

On Monday, prosecutors also charged Ms. Park and Ms. Choi with demanding
bribes worth \$6.2 million from the retail conglomerate Lotte and \$7.8
million from the telecommunications and semiconductor conglomerate SK.
Shin Dong-bin, the chairman of Lotte, was indicted on bribery charges on
Monday. But Mr. Shin, who was already on trial on tax evasion and
embezzlement charges stemming from
\href{https://www.nytimes.com/2016/10/20/business/international/south-korea-lotte-chaebol-conglomerate-indicted.html}{a
separate corruption scandal}, was not arrested.

Lotte had no immediate reaction on Monday.

Both Lotte and SK had lost valuable licenses to run duty-free shops in
2015 and lobbied to regain them last year. Mr. Shin was accused of
paying the \$6.2 million bribe in May while seeking government help to
regain the license. The money was later returned, but Lotte won back its
license in December.

Ms. Park and Ms. Choi were accused of demanding a \$7.8 million bribe
from SK, prosecutors said on Monday. But SK did not pay the money, and
its chairman, Chey Tae-won, was spared indictment on Monday. (SK did not
get back its duty-free shop license.)

In a
\href{https://www.nytimes.com/2017/03/09/business/jay-y-lee-samsung-trial.html}{trial
that began last month}, Mr. Lee, the third-generation scion of the
family that runs the Samsung conglomerate and the vice chairman of
Samsung Electronics, has vehemently denied the bribery and other charges
against him. He has said that he sought no favor from Ms. Park's
government in return for the money Samsung admitted contributing to
support Ms. Choi's foundations and her daughter.

Prosecutors said that what the company called ``donations'' were bribes
used to win government support for the contentious 2015 merger of two
Samsung affiliates, which they say helped Mr. Lee cement control of the
conglomerate.

Ms. Park, too, has denied the charges against her, arguing that she was
victimized by her political enemies.

The removal and arrest of Ms. Park, an icon of the conservative
establishment, have been a crushing blow to that camp.

\href{https://www.nytimes.com/2017/03/10/world/asia/south-korea-liberals-impeachment.html}{Moon
Jae-in} and
\href{https://www.nytimes.com/2017/04/14/world/asia/south-korea-election-ahn-cheol-soo.html}{Ahn
Cheol-soo}, leading contenders in the election next month to select Ms.
Park's successor, were both opposition politicians and vocal critics of
her four-year run in power, which they said symbolized a government that
served the privileged rather than the common good and that continued
corrupt ties with big businesses. Two conservative candidates are
polling in the single digits in pre-election surveys.

Ms. Park's downfall and the presidential election in South Korea also
have the potential to rattle the delicate balance of international
diplomacy in Asia at a time of heightened tensions with North Korea.

Both Mr. Moon and Mr. Ahn criticized the hard-line North Korean policy
of Ms. Park's government and Washington. They said that sanctions and
pressure alone had failed to stop the North's nuclear and missile
programs and that it was time to try dialogue.

Ms. Park is the daughter of the former military dictator Park Chung-hee,
who seized power in a coup in 1961 and ruled South Korea with martial
law and the arbitrary arrests and torture of dissidents before he was
\href{http://www.nytimes.com/1979/10/27/archives/president-park-is-slain-in-korea-by-intelligence-chief-seoul-says.html}{assassinated
by his spy chief in 1979}.

Ms. Park was the first leader of South Korea to be forced from office in
response to popular pressure since the founding president, Syngman Rhee,
fled into exile in Hawaii in 1960 after protests against his corrupt,
authoritarian rule.

If she is convicted of bribery, Ms. Park, 65, could face 10 years to
life in prison, although her successor has the power to free her with a
presidential pardon.

Advertisement

\protect\hyperlink{after-bottom}{Continue reading the main story}

\hypertarget{site-index}{%
\subsection{Site Index}\label{site-index}}

\hypertarget{site-information-navigation}{%
\subsection{Site Information
Navigation}\label{site-information-navigation}}

\begin{itemize}
\tightlist
\item
  \href{https://help.nytimes.com/hc/en-us/articles/115014792127-Copyright-notice}{©~2020~The
  New York Times Company}
\end{itemize}

\begin{itemize}
\tightlist
\item
  \href{https://www.nytco.com/}{NYTCo}
\item
  \href{https://help.nytimes.com/hc/en-us/articles/115015385887-Contact-Us}{Contact
  Us}
\item
  \href{https://www.nytco.com/careers/}{Work with us}
\item
  \href{https://nytmediakit.com/}{Advertise}
\item
  \href{http://www.tbrandstudio.com/}{T Brand Studio}
\item
  \href{https://www.nytimes.com/privacy/cookie-policy\#how-do-i-manage-trackers}{Your
  Ad Choices}
\item
  \href{https://www.nytimes.com/privacy}{Privacy}
\item
  \href{https://help.nytimes.com/hc/en-us/articles/115014893428-Terms-of-service}{Terms
  of Service}
\item
  \href{https://help.nytimes.com/hc/en-us/articles/115014893968-Terms-of-sale}{Terms
  of Sale}
\item
  \href{https://spiderbites.nytimes.com}{Site Map}
\item
  \href{https://help.nytimes.com/hc/en-us}{Help}
\item
  \href{https://www.nytimes.com/subscription?campaignId=37WXW}{Subscriptions}
\end{itemize}
