Sections

SEARCH

\protect\hyperlink{site-content}{Skip to
content}\protect\hyperlink{site-index}{Skip to site index}

\href{https://www.nytimes.com/section/world/middleeast}{Middle East}

\href{https://myaccount.nytimes.com/auth/login?response_type=cookie\&client_id=vi}{}

\href{https://www.nytimes.com/section/todayspaper}{Today's Paper}

\href{/section/world/middleeast}{Middle East}\textbar{}Trump Takes
Credit for Saudi Move Against Qatar, a U.S. Military Partner

\url{https://nyti.ms/2rQtuWJ}

\begin{itemize}
\item
\item
\item
\item
\item
\item
\end{itemize}

Advertisement

\protect\hyperlink{after-top}{Continue reading the main story}

Supported by

\protect\hyperlink{after-sponsor}{Continue reading the main story}

\hypertarget{trump-takes-credit-for-saudi-move-against-qatar-a-us-military-partner}{%
\section{Trump Takes Credit for Saudi Move Against Qatar, a U.S.
Military
Partner}\label{trump-takes-credit-for-saudi-move-against-qatar-a-us-military-partner}}

\includegraphics{https://static01.nyt.com/images/2017/06/07/world/07dc-prexy/merlin-to-scoop-122378063-99718-articleLarge.jpg?quality=75\&auto=webp\&disable=upscale}

By \href{http://www.nytimes.com/by/mark-landler}{Mark Landler}

\begin{itemize}
\item
  June 6, 2017
\item
  \begin{itemize}
  \item
  \item
  \item
  \item
  \item
  \item
  \end{itemize}
\end{itemize}

WASHINGTON --- President Trump thrust himself into a bitter Persian Gulf
dispute on Tuesday, taking credit for Saudi Arabia's move to isolate its
smaller neighbor, Qatar, and rattling his national security staff by
upending a critical American strategic relationship.

In a series of tweets, Mr. Trump said his call for an end to the
financing of radical groups had prompted Saudi Arabia and four other
countries
\href{https://www.nytimes.com/2017/06/05/world/middleeast/qatar-saudi-arabia-egypt-bahrain-united-arab-emirates.html}{to
act this week} against Qatar, a tiny, energy-rich emirate that is
arguably America's most important military outpost in the Middle East.

``During my recent trip to the Middle East I stated that there can no
longer be funding of Radical Ideology,'' he wrote in
\href{https://twitter.com/realDonaldTrump/status/872062159789985792}{a
midmorning post}. ``Leaders pointed to Qatar --- look!''

Qatar has long been accused of funneling money to the
\href{https://www.nytimes.com/2017/02/07/world/middleeast/muslim-brotherhood-terrorism-trump.html}{Muslim
Brotherhood} --- which has officially forsworn violence but is still
accused of terrorism by some countries --- as well as to radical groups
in Syria, Libya and other Arab nations. But it is also home to two major
American command posts, including a \$60 million center from which the
United States and its allies conduct their air war on Islamic State
militants in Iraq and Syria.

Those contradictory roles may explain the mixed signals the
administration sent after Saudi Arabia's unexpected move. Secretary of
State Rex W. Tillerson and Defense Secretary Jim Mattis initially tried
on Monday to smooth over the rift, with Mr. Tillerson offering to play
peacemaker and Mr. Mattis insisting it would have no effect on the
campaign against the Islamic State.

Less than 12 hours later, however, Mr. Trump discarded that approach by
putting his thumb on the scale firmly in Saudi Arabia's favor. His
tweets, which a senior White House official said were not a result of
any policy deliberation, sowed confusion about America's strategy and
its intentions toward a key military partner.

``So good to see the Saudi Arabia visit with the King and 50 countries
already paying off,''
\href{https://twitter.com/realDonaldTrump/status/872084870620520448}{Mr.
Trump wrote}. ``They said they would take a hard line on funding.''
\href{https://twitter.com/realDonaldTrump/status/872086906804240384}{He
added}, ``Perhaps this will be the beginning of the end to the horror of
terrorism!''

Additionally, officials in Jordan said on Tuesday that the country would
downgrade its diplomatic relations with Qatar and revoke the license of
the Doha-based television channel Al Jazeera, Reuters reported.

On Tuesday evening, the president appeared to be trying to ease
tensions. In a call with King Salman of Saudi Arabia, Mr. Trump said
that unity among gulf nations was ``critical to defeating terrorism and
promoting regional stability,'' according to a White House statement.

Administration officials said Mr. Trump was not trying to cause a
rupture among Sunni Muslim nations in the Middle East. Rather, they
said, he was expressing genuine frustration with Qatar's record and
making sure it followed through on the commitments it made in backing a
new joint Terrorist Financing Targeting Center, which the president
\href{https://www.nytimes.com/2017/05/21/world/middleeast/trump-saudi-arabia-islam-speech.html}{announced
last month} in Riyadh.

``The U.S. still wants to see this issue de-escalated and resolved
immediately, keeping with the principles that the president laid out in
terms of defeating terror financing,'' said Sean Spicer, the White House
press secretary.

Mr. Spicer denied that the president was taking sides. He said Mr. Trump
had had a ``very productive'' discussion with Sheikh Tamim bin Hamad
al-Thani, the 37-year-old emir of Qatar, during his visit to Riyadh. But
another person briefed on the conversation said it had been noticeably
colder than the president's meetings with other gulf leaders.

In Washington, Qatar's ambassador, Meshal bin Hamad al-Thani, expressed
surprise at Mr. Trump's tweets. ``No one approached us directly and
said, `Look, we have problems with this and this and this,''' he said in
an
\href{http://www.thedailybeast.com/qatar-ambassador-to-trump-whats-with-the-hate-tweets}{interview}
with The Daily Beast.

There was little immediate threat to American military facilities in
Qatar, administration officials and outside analysts said, not least
because Qatar views America's military presence as an insurance policy
against the aggressive moves of its neighbors.

But the mood there was jittery. Government officials and news outlets
described the cutoff of diplomatic relations, travel and trade as a
``siege'' and even as an attempt at a coup.

Those jitters have been intensified by suspicions that Russia was behind
a cyberattack that published fake information on Qatar's state news
agency --- a claim the United States is investigating, according to an
official briefed on the inquiry, who spoke on the condition of
anonymity. The official said it was unclear whether the hackers were
state-sponsored.

An American diplomat warned that there was a temptation to blame
malicious acts on the Russian government before the evidence had been
weighed. But the same diplomat noted that Russia had much to gain from
divisions among Iran's rivals in the region, particularly if they made
it more difficult for the United States to use Qatar as a major base.

``For sure, this is an attempt at regime change,'' said Jamal Elshayyal,
a senior producer for Al Jazeera, the Qatari-owned news channel that
many in the region accuse of spreading extremist ideas.

Pentagon officials said they, too, were taken aback by Mr. Trump's
tweets, particularly given the American military's deep ties to Qatar.
The military has been eager to avoid political quarrels with the
Qataris, a goal reflected in statements by its spokesmen.

``The United States and the coalition are grateful to the Qataris for
their longstanding support of our presence and their enduring commitment
to regional security,'' Lt. Col. Damien Pickart, spokesman for the Air
Force component of the Central Command, said on Monday.

Al Udeid Air Base, outside the Qatari capital, Doha, is home to more
than 11,000 American and coalition service members. Mr. Mattis made a
point of visiting in April, spending three nights in Doha, where he met
with the emir.

Mr. Trump's tweets also appeared to contradict the American ambassador
to Qatar, Dana Shell Smith, who this week
\href{https://twitter.com/AmbDana/status/871593197708787712}{retweeted a
post of hers} saying Qatar had made ``real progress'' in curbing
financial support for terrorists.

On Tuesday, an American diplomat in Doha said that Qatar's relationship
with the United States was ``strong'' and that it had made strides:
prosecuting people suspected of funding terrorist groups, freezing
assets and putting stringent controls on its banks.

Not for the first time, Mr. Trump's comments differed sharply from those
of his top national security aides.

``We certainly would encourage the parties to sit down together and
address these differences,'' Mr. Tillerson told reporters in Sydney,
Australia, where he and Mr. Mattis were meeting with officials on
Monday.

Mr. Mattis added, ``I am positive there will be no implications coming
out of this dramatic situation at all.''

In addition to hosting the air command center, Qatar is the home of the
forward headquarters of the United States Central Command and an
American intelligence hub in the Middle East.

It also has deep ties to American academia, providing funding and
property to build Middle Eastern campuses for six major universities,
including Cornell, Georgetown and Northwestern.

Qatar's financing of radical groups has long been a source of tension
with Washington. But the United States has generally avoided taking
sides in the regional feuds in the Persian Gulf, because it has
strategic partnerships with several countries and most of them,
including Saudi Arabia, have a record of financing extremist groups.

``Clearly, the Saudis and the Emiratis felt they had someone in the
White House who would take their side,'' said Robert Malley, who
coordinated Middle East policy in the Obama administration. ``This puts
Qatar in a tough position: Either make a dramatic policy shift or face
deeper isolation.''

Others analysts were more critical, saying the Saudis had exploited Mr.
Trump by seizing on the good will generated during his visit to carry
out a long-planned move against a smaller neighbor.

``The Saudis played Donald Trump like a fiddle,'' said Bruce O. Riedel,
a former intelligence analyst who advised Mr. Obama and now works at the
Brookings Institution. ``He unwittingly encouraged their worst instincts
toward their neighbors.''

It is not the first time the White House has struggled to explain Mr.
Trump's statements about a security partnership. Just last month, during
a visit to NATO headquarters in Belgium, he
\href{https://www.nytimes.com/2017/05/25/world/europe/donald-trump-eu-nato.html}{declined
to reaffirm America's commitment} to the alliance's principle of mutual
defense, after a senior administration official had told The New York
Times he would.

That line was deleted from Mr. Trump's speech shortly before he
delivered it,
\href{http://www.politico.com/magazine/story/2017/06/05/trump-nato-speech-national-security-team-215227}{according
to Politico}, to the surprise of officials including Mr. Tillerson and
Mr. Mattis.

On Tuesday, Mr. Spicer described persistent questions about the episode
as ``a bit silly,'' saying Mr. Trump's mere presence at the NATO
ceremony was evidence of the American commitment to mutual defense.

Advertisement

\protect\hyperlink{after-bottom}{Continue reading the main story}

\hypertarget{site-index}{%
\subsection{Site Index}\label{site-index}}

\hypertarget{site-information-navigation}{%
\subsection{Site Information
Navigation}\label{site-information-navigation}}

\begin{itemize}
\tightlist
\item
  \href{https://help.nytimes.com/hc/en-us/articles/115014792127-Copyright-notice}{©~2020~The
  New York Times Company}
\end{itemize}

\begin{itemize}
\tightlist
\item
  \href{https://www.nytco.com/}{NYTCo}
\item
  \href{https://help.nytimes.com/hc/en-us/articles/115015385887-Contact-Us}{Contact
  Us}
\item
  \href{https://www.nytco.com/careers/}{Work with us}
\item
  \href{https://nytmediakit.com/}{Advertise}
\item
  \href{http://www.tbrandstudio.com/}{T Brand Studio}
\item
  \href{https://www.nytimes.com/privacy/cookie-policy\#how-do-i-manage-trackers}{Your
  Ad Choices}
\item
  \href{https://www.nytimes.com/privacy}{Privacy}
\item
  \href{https://help.nytimes.com/hc/en-us/articles/115014893428-Terms-of-service}{Terms
  of Service}
\item
  \href{https://help.nytimes.com/hc/en-us/articles/115014893968-Terms-of-sale}{Terms
  of Sale}
\item
  \href{https://spiderbites.nytimes.com}{Site Map}
\item
  \href{https://help.nytimes.com/hc/en-us}{Help}
\item
  \href{https://www.nytimes.com/subscription?campaignId=37WXW}{Subscriptions}
\end{itemize}
