Sections

SEARCH

\protect\hyperlink{site-content}{Skip to
content}\protect\hyperlink{site-index}{Skip to site index}

\href{https://www.nytimes.com/section/business}{Business}

\href{https://myaccount.nytimes.com/auth/login?response_type=cookie\&client_id=vi}{}

\href{https://www.nytimes.com/section/todayspaper}{Today's Paper}

\href{/section/business}{Business}\textbar{}C.E.O.s React After Trump
Attacks Merck Chief

\url{https://nyti.ms/2uVHPyO}

\begin{itemize}
\item
\item
\item
\item
\item
\end{itemize}

Advertisement

\protect\hyperlink{after-top}{Continue reading the main story}

Supported by

\protect\hyperlink{after-sponsor}{Continue reading the main story}

\hypertarget{ceos-react-after-trump-attacks-merck-chief}{%
\section{C.E.O.s React After Trump Attacks Merck
Chief}\label{ceos-react-after-trump-attacks-merck-chief}}

By \href{http://www.nytimes.com/by/david-gelles}{David Gelles}

\begin{itemize}
\item
  Aug. 14, 2017
\item
  \begin{itemize}
  \item
  \item
  \item
  \item
  \item
  \end{itemize}
\end{itemize}

Kenneth C. Frazier, chief executive of Merck, the pharmaceuticals
company, on Monday resigned from President Trump's
\href{https://www.whitehouse.gov/the-press-office/2017/01/27/president-trump-announces-manufacturing-jobs-initiative}{American
Manufacturing Council} after the president failed to directly condemn
the white nationalist protesters at the center of violent protests in
Charlottesville, Va., over the weekend.

``America's leaders must honor our fundamental values by clearly
rejecting expressions of hatred, bigotry and group supremacy, which run
counter to the American ideal that all people are created equal,'' Mr.
Frazier said in a statement. ``As C.E.O. of Merck and as a matter of
personal conscience, I feel a responsibility to take a stand against
extremism.''

\begin{quote}
\href{https://t.co/a1PNQZism5}{pic.twitter.com/a1PNQZism5}

--- Merck (@Merck)
\href{https://twitter.com/Merck/status/897065338566791169?ref_src=twsrc\%5Etfw}{August
14, 2017}
\end{quote}

Less than an hour later, the president took aim at the Merck executive.

Mr. Trump replied swiftly to Mr. Frazier on Twitter, suggesting that
stepping down from the council would give him ``more time to LOWER
RIPOFF DRUG PRICES!''

Later in the day, President Trump
\href{https://www.nytimes.com/2017/08/14/us/politics/trump-charlottesville-protest.html}{explicitly
condemned hate groups} including neo-Nazis and the K.K.K., offering the
kind of sharp denunciation of racism that many critics believed was
lacking over the weekend.

But the president's remarks came too late for Mr. Frazier, who resigned
early on Monday.

By midday, Mr. Frazier was joined by a handful of other executives as a
voice of dissent in the business community, which has particular sway of
the president, himself a businessman.

Lloyd Blankfein, the chief executive of Goldman Sachs, posted a cryptic
message on Twitter, quoting Abraham Lincoln.

Tim Cook, the chief executive of Apple, also posted a statement opposing
white supremacy and racism.

And at least one other chief executive --- albeit of a foreign company
--- came to Mr. Frazier's defense. Paul Polman, the chief executive of
Unilever, signaled his support for the Merck boss on Twitter.

General Electric said in a statement that it ``has no tolerance for
hate, bigotry or racism,'' but added that its chairman and
recently-retired chief executive, Jeff Immelt, will remain on the
manufacturing council from which Mr. Frazier resigned.

The chief executive of Intel, Brian Krzanich, who is a member of one of
the president's advisory councils, said, ``There should be no hesitation
in condemning hate speech or white supremacy by name,'' and urged the
nation's leaders to do so.

Kevin Plank, Under Armour's chief executive, who is also on one of the
advisory councils, issued a statement denouncing racism and
discrimination. He stopped short of criticizing the president or his
comments about Mr. Frazier.

With the exception of Mr. Frazier, no other major business leaders were
explicitly distancing themselves from the president's advisory councils,
or criticizing his response to the events in Charlottesville. Both
Republican and Democratic lawmakers criticized the president's response,
which
\href{https://www.nytimes.com/2017/08/12/us/trump-charlottesville-protest-nationalist-riot.html}{some
viewed as muted and equivocal}.

But Richard Trumka, the president of AFL-CIO labor federation who is a
member of Mr. Trump's manufacturing council, said Monday afternoon that
the organization was ``assessing our role'' with the group.

\begin{quote}
``The AFL-CIO has unequivocally denounced the actions of bigoted
domestic terrorists in Charlottesville and called on the president to do
the same,'' the statement said. ``We are aware of the decisions of other
members of the president's manufacturing council, which has yet to hold
any real meeting, and are assessing our role.''
\end{quote}

Tom Glocer, the former chief executive of Thomson Reuters, offered his
support for Mr. Frazier on Twitter and called for other chief executives
to step down from the president's advisory councils. ``Ken has stood up
for true American values,'' he said.

This is not the first time the president's actions have caused a chief
executive to walk away from a seat on one of his advisory councils.

In February, as Mr. Trump began to implement stringent new immigration
rules, Travis Kalanick, then the chief executive of Uber Technologies,
\href{https://www.nytimes.com/2017/02/02/technology/uber-ceo-travis-kalanick-trump-advisory-council.html?_r=0}{left
the president's economic advisory council}.

And after President Trump said he would withdraw the United States from
the Paris climate accord, Elon Musk, Tesla's chief, and Robert Iger,
Disney's chief, stepped down from the President's Strategic and Policy
Forum.

Even after Mr. Musk and Mr. Iger walked away in the wake of the
president's withdrawal from the Paris agreement,
\href{https://www.nytimes.com/interactive/2017/06/02/opinion/trump-paris-climate-reacts-advisors.html}{most
other chief executives stuck with President Trump}.

Today, the same dynamic appears to be playing out.

Advertisement

\protect\hyperlink{after-bottom}{Continue reading the main story}

\hypertarget{site-index}{%
\subsection{Site Index}\label{site-index}}

\hypertarget{site-information-navigation}{%
\subsection{Site Information
Navigation}\label{site-information-navigation}}

\begin{itemize}
\tightlist
\item
  \href{https://help.nytimes.com/hc/en-us/articles/115014792127-Copyright-notice}{©~2020~The
  New York Times Company}
\end{itemize}

\begin{itemize}
\tightlist
\item
  \href{https://www.nytco.com/}{NYTCo}
\item
  \href{https://help.nytimes.com/hc/en-us/articles/115015385887-Contact-Us}{Contact
  Us}
\item
  \href{https://www.nytco.com/careers/}{Work with us}
\item
  \href{https://nytmediakit.com/}{Advertise}
\item
  \href{http://www.tbrandstudio.com/}{T Brand Studio}
\item
  \href{https://www.nytimes.com/privacy/cookie-policy\#how-do-i-manage-trackers}{Your
  Ad Choices}
\item
  \href{https://www.nytimes.com/privacy}{Privacy}
\item
  \href{https://help.nytimes.com/hc/en-us/articles/115014893428-Terms-of-service}{Terms
  of Service}
\item
  \href{https://help.nytimes.com/hc/en-us/articles/115014893968-Terms-of-sale}{Terms
  of Sale}
\item
  \href{https://spiderbites.nytimes.com}{Site Map}
\item
  \href{https://help.nytimes.com/hc/en-us}{Help}
\item
  \href{https://www.nytimes.com/subscription?campaignId=37WXW}{Subscriptions}
\end{itemize}
