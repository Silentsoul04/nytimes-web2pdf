Sections

SEARCH

\protect\hyperlink{site-content}{Skip to
content}\protect\hyperlink{site-index}{Skip to site index}

\href{https://www.nytimes.com/section/politics}{Politics}

\href{https://myaccount.nytimes.com/auth/login?response_type=cookie\&client_id=vi}{}

\href{https://www.nytimes.com/section/todayspaper}{Today's Paper}

\href{/section/politics}{Politics}\textbar{}New Outcry as Trump Rebukes
Charlottesville Racists 2 Days Later

\url{https://nyti.ms/2uUNOUU}

\begin{itemize}
\item
\item
\item
\item
\item
\item
\end{itemize}

Advertisement

\protect\hyperlink{after-top}{Continue reading the main story}

Supported by

\protect\hyperlink{after-sponsor}{Continue reading the main story}

\hypertarget{new-outcry-as-trump-rebukes-charlottesville-racists-2-days-later}{%
\section{New Outcry as Trump Rebukes Charlottesville Racists 2 Days
Later}\label{new-outcry-as-trump-rebukes-charlottesville-racists-2-days-later}}

\includegraphics{https://static01.nyt.com/images/2017/08/15/us/15dc-trump-statement/15dc-trump-statement-videoSixteenByNine3000.jpg}

By \href{https://www.nytimes.com/by/glenn-thrush}{Glenn Thrush}

\begin{itemize}
\item
  Aug. 14, 2017
\item
  \begin{itemize}
  \item
  \item
  \item
  \item
  \item
  \item
  \end{itemize}
\end{itemize}

WASHINGTON --- The crisis in Charlottesville, Va., presented President
Trump with a choice between adopting the unifying tone of a traditional
president or doubling down on the go-it-alone approach that got him
elected in 2016.

On Monday, Mr. Trump offered a glimpse of a more calming and
conventional president, but he ended the day with a flurry of angry
tweets that left little doubt he intended to govern on his own terms.

Mr. Trump, after two days of issuing equivocal statements, bowed to
overwhelming pressure that he personally condemn white supremacists who
incited bloody weekend demonstrations in Charlottesville.

``Racism is evil,'' said Mr. Trump, delivering a statement from the
White House at a hastily arranged appearance meant to halt the growing
political threat posed by the unrest. ``And those who cause violence in
its name are criminals and thugs, including the K.K.K., neo-Nazis, white
supremacists and other hate groups that are repugnant to everything we
hold dear as Americans.''

But before and after his conciliatory statement --- which called for
``love,'' ``joy'' and ``justice'' --- Mr. Trump issued classically
caustic Twitter attacks on Kenneth C. Frazier, the head of Merck
Pharmaceuticals and one of the country's top African-American
executives.

Mr. Frazier announced Monday morning that he was resigning from the
American Manufacturing Council --- the first of three chief executives
who quit the advisory panel on Monday --- to protest Mr. Trump's initial
equivocal statements on Charlottesville.

``Now that Ken Frazier of Merck Pharma has resigned from President's
Manufacturing Council, he will have more time to LOWER RIPOFF DRUG
PRICES!'' the
\href{https://twitter.com/realDonaldTrump/status/897079051277537280}{president
wrote at 8:54 a.m.}, as he departed his golf resort in Bedminster, N.J.,
for a day trip back to Washington.

Shortly before leaving the capital, Mr. Trump attacked the news media
for blowing the episode out of proportion.

``Made additional remarks on Charlottesville and realize once again that
the \#Fake News Media will never be satisfied...truly bad people!'' he
wrote Monday evening.

``Trump faced a fork in the road today, and he took it,'' said
Representative Nancy Pelosi, Democrat of California and the House
minority leader. ``He showed cowardice on Saturday by refusing to call
out the racists and neo-Nazis, and on Monday he showed how uncomfortable
he was in delivering another kind of message.''

Even Mr. Trump's allies worried that his measured remarks, delivered two
days after dozens of public figures issued more forceful denunciations
of the violence in Virginia, came too late to reverse the self-inflicted
damage on his moral standing as president.

On Saturday, Mr. Trump
\href{https://www.nytimes.com/2017/08/12/us/trump-charlottesville-protest-nationalist-riot.html}{said}
the rioting was initiated by ``many sides.'' His comments prompted
nearly universal criticism and spurred several of his top advisers,
including his new chief of staff, John F. Kelly, to press the president
to issue a more forceful rebuke.

Even after a wave of disapproval that included a majority of Senate
Republicans --- and stronger statements delivered by allies, including
Vice President Mike Pence and the president's daughter Ivanka Trump ---
Mr. Trump seemed reluctant to tackle the issue head-on when he appeared
Monday before the cameras.

He first offered a lengthy and seemingly out-of-place recitation of his
accomplishments on the economy, trade and job creation. When he did
address the violence in Charlottesville, he presented his stronger
language as an update on the Justice Department's civil rights
investigation into the death of a woman who was hit by a car that the
authorities said was driven by an
\href{https://www.nytimes.com/2017/08/13/us/james-alex-fields-charlottesville-driver-.html}{Ohio
protester} with ties to neo-Nazi groups.

``To anyone who acted criminally in this weekend's racist violence, you
will be held fully accountable. Justice will be delivered,'' said Mr.
Trump, who had just concluded a meeting with Attorney General Jeff
Sessions and Christopher A. Wray, the F.B.I. director.

Mr. Trump has had a career-long pattern of delaying and muting his
criticism of white nationalism. During the 2016 presidential campaign,
he refused to immediately denounce David Duke, a former Klansman who
supported his candidacy.

Some human rights activists, skeptical that Mr. Trump's latest remarks
on the issue represented a change of heart, called on him to fire
so-called nationalists --- a group of hard-right populists led by
Stephen K. Bannon, the White House chief strategist --- working in the
West Wing.

``The president should make sure that no one on his staff has ties to
white supremacists,'' Jonathan Greenblatt, the chief executive officer
of the Anti-Defamation League, said in a telephone briefing on Monday
afternoon. He added, ``Nor should they be on the payroll of the American
people.''

He said that the Justice Department and the Office of Government Ethics
should ``do an investigation and make that determination'' to see if
anyone in the White House has had links to hate groups.

Mr. Trump and his staff have consistently denied any connection to such
organizations, and the president called for racial harmony in his
remarks on Monday.

``As I have said many times before, no matter the color of our skin, we
all live under the same laws,'' he said. ``We all salute the same great
flag, and we are all made by the same almighty God. We must love each
other, show affection for each other and unite together in condemnation
of hatred, bigotry and violence.''

Far-right leaders, including Richard B. Spencer, who attended the
Charlottesville rally, said they did not take the president's remarks
seriously.

``The statement today was more `kumbaya' nonsense,'' Mr. Spencer told
reporters on Monday. ``He sounded like a Sunday school teacher.''

``I don't think that Donald Trump is a dumb person, and only a dumb
person would take those lines seriously,'' Mr. Spencer said.

As Mr. Trump was delivering the kind of statement his critics had
demanded over the weekend, Fox News reported that the president was
considering pardoning Joe Arpaio, the former sheriff of Maricopa County,
Ariz., a political ally who has been accused of federal civil rights
violations for allegedly mistreating prisoners, many of them black and
Hispanic.

The timing of the interview was especially striking, given that it came
at the height of the controversy over his tepid remarks about
Charlottesville.

``I am seriously considering a pardon for Sheriff Arpaio,'' the
president said in the interview on Sunday, speaking from his golf club
in Bedminster, N.J. ``He has done a lot in the fight against illegal
immigration. He's a great American patriot, and I hate to see what has
happened to him.''

Two themes --- uniting the country while defending himself --- collided
on Mr. Trump's Twitter feed earlier on Monday.

It is not unusual for Mr. Trump to attack, via Twitter, any public
figure who ridicules, criticizes or even mildly questions his actions.
But his decision to take on Mr. Frazier, a self-made multimillionaire
who rose from a modest childhood in Philadelphia to attend Harvard Law
School, was extraordinary given the wide-ranging criticism the president
faced from both parties for not forcefully denouncing the neo-Nazis and
Klan sympathizers who rampaged in Charlottesville.

Mr. Frazier's exit from the business council marks a mini-exodus of
business leaders from presidential advisory panels as a result of Mr.
Trump's stances on social issues and the environment. His recent
decision to leave the Paris climate accord prompted Elon Musk of Tesla
to resign, as did the chief executive of Disney, Bob Iger.

Additionally, the chief executives of the athletic clothing line Under
Armour and of Intel announced they too would step down from the American
Manufacturing Council --- the same panel from which Mr. Frazier
resigned.

Kevin Plank, the head of Under Armour, said he was resigning to focus on
``the power of sport which promotes unity, diversity and inclusion.''

Intel chief executive Brian Krzanich said he would be willing to serve
in the government again when ``those who have stood up for equality''
are honored. ``I resigned because I want to make progress, while many in
Washington seem more concerned with attacking anyone who disagrees with
them,'' Mr. Krzanich said in a statement.

Mr. Trump's shot at Mr. Frazier, one of the country's best-known black
executives, prompted an immediate outpouring of support for the Merck
chief executive from major figures in business, media and politics.

``Thanks \href{https://twitter.com/Merck}{@Merck} Ken Frazier for strong
leadership to stand up for the moral values that made this country what
it is,'' Paul Polman, the chief executive of Unilever,
\href{https://twitter.com/PaulPolman/status/897104757340467200}{wrote on
Twitter}.

Last month, Mr. Frazier appeared next to Mr. Trump at the White House to
announce an agreement among drug makers that would create 1,000 jobs.

He is only the second African-American executive to lead a major
pharmaceutical firm, and rose to prominence as Merck's general counsel
when he successfully defended the company against class-action lawsuits
stemming from complications involving the anti-inflammatory drug Vioxx.

``It took Trump 54 minutes to condemn Merck CEO Ken Frazier, but after
several days he still has not condemned murdering white supremacists,''
Keith Boykin, a former aide to President Bill Clinton who comments on
politics and race for CNN,
\href{https://twitter.com/keithboykin/status/897084712920985601}{wrote
in a tweet}.

Advertisement

\protect\hyperlink{after-bottom}{Continue reading the main story}

\hypertarget{site-index}{%
\subsection{Site Index}\label{site-index}}

\hypertarget{site-information-navigation}{%
\subsection{Site Information
Navigation}\label{site-information-navigation}}

\begin{itemize}
\tightlist
\item
  \href{https://help.nytimes.com/hc/en-us/articles/115014792127-Copyright-notice}{©~2020~The
  New York Times Company}
\end{itemize}

\begin{itemize}
\tightlist
\item
  \href{https://www.nytco.com/}{NYTCo}
\item
  \href{https://help.nytimes.com/hc/en-us/articles/115015385887-Contact-Us}{Contact
  Us}
\item
  \href{https://www.nytco.com/careers/}{Work with us}
\item
  \href{https://nytmediakit.com/}{Advertise}
\item
  \href{http://www.tbrandstudio.com/}{T Brand Studio}
\item
  \href{https://www.nytimes.com/privacy/cookie-policy\#how-do-i-manage-trackers}{Your
  Ad Choices}
\item
  \href{https://www.nytimes.com/privacy}{Privacy}
\item
  \href{https://help.nytimes.com/hc/en-us/articles/115014893428-Terms-of-service}{Terms
  of Service}
\item
  \href{https://help.nytimes.com/hc/en-us/articles/115014893968-Terms-of-sale}{Terms
  of Sale}
\item
  \href{https://spiderbites.nytimes.com}{Site Map}
\item
  \href{https://help.nytimes.com/hc/en-us}{Help}
\item
  \href{https://www.nytimes.com/subscription?campaignId=37WXW}{Subscriptions}
\end{itemize}
