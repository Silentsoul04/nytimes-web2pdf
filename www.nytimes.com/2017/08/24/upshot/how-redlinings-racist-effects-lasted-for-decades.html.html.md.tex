Sections

SEARCH

\protect\hyperlink{site-content}{Skip to
content}\protect\hyperlink{site-index}{Skip to site index}

\href{https://myaccount.nytimes.com/auth/login?response_type=cookie\&client_id=vi}{}

\href{https://www.nytimes.com/section/todayspaper}{Today's Paper}

\href{/section/upshot}{The Upshot}\textbar{}How Redlining's Racist
Effects Lasted for Decades

\url{https://nyti.ms/2vrx87D}

\begin{itemize}
\item
\item
\item
\item
\item
\item
\end{itemize}

Advertisement

\protect\hyperlink{after-top}{Continue reading the main story}

Supported by

\protect\hyperlink{after-sponsor}{Continue reading the main story}

Upshot

Self-Fulfilling Prophecies

\hypertarget{how-redlinings-racist-effects-lasted-for-decades}{%
\section{How Redlining's Racist Effects Lasted for
Decades}\label{how-redlinings-racist-effects-lasted-for-decades}}

\includegraphics{https://static01.nyt.com/images/2017/08/25/upshot/25up-redliningbk/25up-redliningbk-articleLarge.jpg?quality=75\&auto=webp\&disable=upscale}

By \href{https://www.nytimes.com/by/emily-badger}{Emily Badger}

\begin{itemize}
\item
  Aug. 24, 2017
\item
  \begin{itemize}
  \item
  \item
  \item
  \item
  \item
  \item
  \end{itemize}
\end{itemize}

The appraiser who went to Brooklyn in the 1930s to assess
Bedford-Stuyvesant for the government summarized the neighborhood's
prospects on a single page. Many brownstones in ``obsolescence and poor
upkeep.'' Clerks, laborers and merchants lived there, about 30 percent
of them foreign-born, Jews and Irish mostly.

Also, this: ``Colored infiltration a definitely adverse influence on
neighborhood desirability.''

The government-sponsored Home Owners' Loan Corporation drew a line
around Bedford-Stuyvesant on a map, colored the area red and gave it a
``D,'' the worst grade possible, denoting a hazardous place to
underwrite mortgages.

Lines like these, drawn in cities across the country to separate
``hazardous'' and ``declining'' from ``desirable'' and ``best,''
codified patterns of racial segregation and disparities in access to
credit. Now economists at the Federal Reserve Bank of Chicago,
\href{https://www.chicagofed.org/publications/working-papers/2017/wp2017-12}{analyzing
data} from \href{https://dsl.richmond.edu/panorama/redlining/}{recently
digitized copies} of those maps, show that the consequences lasted for
decades.

As recently as 2010, they find, differences in the level of racial
segregation, homeownership rates, home values and credit scores were
still apparent where these boundaries were drawn.

``Did the creation of these maps actually influence the development of
urban neighborhoods over the course of the 20th century to now?'' said
Bhash Mazumder, one of the Fed researchers, along with Daniel Aaronson
and Daniel Hartley. ``That was our primary question.''

The economists now believe that appraisers like the one in
Bedford-Stuyvesant weren't merely identifying disparities that already
existed in the 1930s, and that were likely to worsen anyway. The lines
they helped draw, based in large part on the belief that the presence of
blacks and other minorities would undermine property values, altered
what would happen in these communities for years to come. Maps alone
didn't create segregated and unequal cities today. But the role they
played was pivotal.

\includegraphics{https://static01.nyt.com/images/2017/08/25/upshot/25up-redliningatl/25up-redliningatl-articleLarge.jpg?quality=75\&auto=webp\&disable=upscale}

The maps became self-fulfilling prophesies, as ``hazardous''
neighborhoods --- ``redlined'' ones --- were starved of investment and
deteriorated further in ways that most likely also fed white flight and
rising racial segregation. These neighborhood classifications were later
used by the Veterans Administration and the Federal Housing
Administration to decide who was worthy of home loans at a time when
homeownership was rapidly expanding in postwar America.

``Housing policy can have a really long-lasting impact, since structures
last a long time,'' Mr. Hartley said.

The new research reaffirms the role of government policy in shaping
racial disparities in America in access to housing, credit and wealth
accumulation. And as the country grapples with the blurred lines between
past racism and present-day outcomes, this new data illustrates how such
history lives on.

``We now have evidence that is very systematic and nationwide that has
detailed that these borders did matter,'' said Leah Boustan, an economic
historian at Princeton familiar with the research, which she called
``pathbreaking.''

Historians have long pointed to the significance of the Home Owners'
Loan Corporation maps. But a large collection of the 239 cities that
were originally appraised was only recently digitized by a collaboration
of schools and housed at
\href{https://dsl.richmond.edu/panorama/redlining/\#loc=5/39.105/-94.583\&opacity=0.8}{the
University of Richmond}, making the underlying geographic data widely
available.

The Chicago Fed economists used that data to identify boundaries between
neighborhoods with different ratings. As of 1930, there were already
clear differences along some of the borders in racial demographics and
homeownership rates. Blacks were already more likely to be living in
``D'' neighborhoods than ``C'' neighborhoods, for example. But
differences in the black share of the population and homeownership rates
widened after the 1930s, reaching a peak in the 1970s, when federal laws
requiring equal access to housing and credit took effect.

Those patterns alone don't prove that the maps \emph{caused} widening
gaps in segregation or homeownership. To do that, the researchers drew
their own hypothetical boundaries to compare what might have happened
had the Home Owners' Loan Corporation placed the lines in other
locations where similar differences existed at the time. The disparities
along those simulated borders didn't widen; they disappeared.

The differences the researchers detected from the maps kept reappearing
whether they looked across whole neighborhoods or just at blocks
adjacent to these borders. They reappeared even when the researchers
looked at a subset of boundaries where the nearby demographics were
barely changing before the 1930s. By analyzing the differences in
several ways, the researchers say they feel confident they have picked
up on effects that were actually caused by the maps.

They estimate that the maps account for 15 to 30 percent of the overall
gaps in segregation and homeownership that they find between ``D'' and
``C'' neighborhoods from 1950 to 2010 (the gaps between ``D'' and ``A''
neighborhoods are clearly even wider).

People living in poorly rated neighborhoods would have had trouble
obtaining mortgages for homes there, regardless of their individual
creditworthiness. Other consequences most likely piled up from there.

``The availability of credit has really significant impacts on every
dimension of neighborhood life, in terms of the quality of real estate,
the willingness of investors to come in, the prices of property, the
emergence of predatory practices,'' said Thomas Sugrue, a historian at
New York University. ``These are all direct consequences of the lack of
affordable loans and affordable mortgages.''

Blacks who did not have access to conventional home loans had to turn to
\href{https://www.theatlantic.com/magazine/archive/2014/06/the-case-for-reparations/361631/}{schemes
like contract sales} that entailed steep interest rates (the
practice\href{https://www.nytimes.com/series/the-housing-trap?action=click\&contentCollection=DealBook\&module=Collection\&region=Marginalia\&src=me\&version=series\&pgtype=article}{is
returning today} in many of these same communities). Because those homes
could be frequently repossessed by predatory lenders, these
neighborhoods would experience more population instability.

Slumlords, too, would move in, squeezing value from subdivided rental
homes that otherwise might have been owned by families. Commercial
investors, meanwhile, would have stayed away. Blacks discriminated
against in the housing market elsewhere would have had limited options
to move away. And any existing homeowners would have struggled to obtain
credit for maintenance and repairs, leading to the further deterioration
of properties.

This process can be invisible to people who might look at these
communities, Mr. Sugrue said, and place blame for their disrepair on
residents who don't value their homes.

Image

The 1937 Home Owners' Loan Corporation map of Oakland,
Calif.Credit...National Archives and Records Administration, Mapping
Inequality

There would be long-term and invisible effects, too, on family wealth,
as people who weren't able to buy a home never developed the equity that
would allow their children (and grandchildren) to buy homes.

The black-white gap in homeownership in America has in fact changed
little over the last century,
\href{https://www.bu.edu/econ/files/2013/03/101122_sem777_Robert-Margo-Paper-1.pdf}{according
to research by Robert Margo and William Collins}. That pattern helps
explain why, as the income gap between the two groups has persisted, the
\emph{wealth} gap has widened by much more.

That these maps made a difference that's still visible today is
striking.

``It doesn't surprise me at all,'' said Richard Rothstein, a researcher
with the Economic Policy Institute who has written a new book,
``\href{http://www.epi.org/publication/the-color-of-law-a-forgotten-history-of-how-our-government-segregated-america/}{The
Color of Law},'' on how official policies like redlining fostered
segregation. These maps --- and their lingering effects --- derive from
a time when the American government, he writes, believed that
``inharmonious racial groups'' should be separated.

Advertisement

\protect\hyperlink{after-bottom}{Continue reading the main story}

\hypertarget{site-index}{%
\subsection{Site Index}\label{site-index}}

\hypertarget{site-information-navigation}{%
\subsection{Site Information
Navigation}\label{site-information-navigation}}

\begin{itemize}
\tightlist
\item
  \href{https://help.nytimes.com/hc/en-us/articles/115014792127-Copyright-notice}{©~2020~The
  New York Times Company}
\end{itemize}

\begin{itemize}
\tightlist
\item
  \href{https://www.nytco.com/}{NYTCo}
\item
  \href{https://help.nytimes.com/hc/en-us/articles/115015385887-Contact-Us}{Contact
  Us}
\item
  \href{https://www.nytco.com/careers/}{Work with us}
\item
  \href{https://nytmediakit.com/}{Advertise}
\item
  \href{http://www.tbrandstudio.com/}{T Brand Studio}
\item
  \href{https://www.nytimes.com/privacy/cookie-policy\#how-do-i-manage-trackers}{Your
  Ad Choices}
\item
  \href{https://www.nytimes.com/privacy}{Privacy}
\item
  \href{https://help.nytimes.com/hc/en-us/articles/115014893428-Terms-of-service}{Terms
  of Service}
\item
  \href{https://help.nytimes.com/hc/en-us/articles/115014893968-Terms-of-sale}{Terms
  of Sale}
\item
  \href{https://spiderbites.nytimes.com}{Site Map}
\item
  \href{https://help.nytimes.com/hc/en-us}{Help}
\item
  \href{https://www.nytimes.com/subscription?campaignId=37WXW}{Subscriptions}
\end{itemize}
