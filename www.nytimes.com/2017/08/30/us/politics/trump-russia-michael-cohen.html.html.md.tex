Sections

SEARCH

\protect\hyperlink{site-content}{Skip to
content}\protect\hyperlink{site-index}{Skip to site index}

\href{https://www.nytimes.com/section/politics}{Politics}

\href{https://myaccount.nytimes.com/auth/login?response_type=cookie\&client_id=vi}{}

\href{https://www.nytimes.com/section/todayspaper}{Today's Paper}

\href{/section/politics}{Politics}\textbar{}Trump Lawyer `Vehemently'
Denies Russian Collusion

\url{https://nyti.ms/2x6tNiX}

\begin{itemize}
\item
\item
\item
\item
\item
\end{itemize}

Advertisement

\protect\hyperlink{after-top}{Continue reading the main story}

Supported by

\protect\hyperlink{after-sponsor}{Continue reading the main story}

\hypertarget{trump-lawyer-vehemently-denies-russian-collusion}{%
\section{Trump Lawyer `Vehemently' Denies Russian
Collusion}\label{trump-lawyer-vehemently-denies-russian-collusion}}

\includegraphics{https://static01.nyt.com/images/2017/08/31/us/31DC-COHEN/31DC-COHEN-articleInline.jpg?quality=75\&auto=webp\&disable=upscale}

By \href{http://www.nytimes.com/by/maggie-haberman}{Maggie Haberman} and
\href{http://www.nytimes.com/by/matt-apuzzo}{Matt Apuzzo}

\begin{itemize}
\item
  Aug. 30, 2017
\item
  \begin{itemize}
  \item
  \item
  \item
  \item
  \item
  \end{itemize}
\end{itemize}

WASHINGTON --- President Trump's longtime lawyer, Michael D. Cohen, has
given Congress a point-by-point rebuttal of a dossier alleging that he
has deep ties to Russian officials --- an effort to clear his name as
the Justice Department and congressional committees investigate Russia's
attempts to disrupt last year's election.

Mr. Cohen encouraged lawmakers to investigate those who paid for the
salacious 35-page dossier, which surfaced online early this year and
alleges that President Trump and his campaign conspired with Russia in
the November election. The dossier, compiled by a retired British spy,
portrays Mr. Cohen as a central figure in the conspiracy.

In an eight-page letter to the House Intelligence Committee, a lawyer
for Mr. Cohen offered a full-throated rejection of any suggestion that
Mr. Cohen was involved in an effort to work with Russia to disrupt the
election.

``We have not uncovered a single document that would in any way
corroborate the dossier's allegations regarding Mr. Cohen, nor do we
believe that any such document exists,'' wrote the lawyer, Stephen M.
Ryan.

``Mr. Cohen vehemently denies the claims made in the dossier about him,
which are false and remain wholly unsubstantiated.''

The letter, which was obtained by The New York Times, follows a similar
denial by the president's son-in-law, Jared Kushner, who spoke to
congressional investigators in July.

Mr. Cohen produced records to Congress this week, including a series of
emails he had received in 2015 from Felix Sater, a real estate broker
with ties to the Kremlin. In the emails, Mr. Sater predicted that a
Trump Tower being planned for Moscow could be built with the help of the
Russian government, and that the project would help Mr. Trump win the
presidency.

The pitch began in the latter half of 2015, when Mr. Trump was already
running for president. The emails show that even then, some around him
believed that close ties to Russia were politically advantageous. But
the project failed to get funding or permits and was dropped shortly
before the Republican primaries. Mr. Cohen said that Mr. Sater, who
worked on and off for the Trump Organization over many years, was given
to boastful language and overstated his influence.

Mr. Cohen's name appears throughout the dossier compiled by the retired
British spy, Christopher Steele, who has deep expertise in Russia. The
dossier is a compendium of unsubstantiated allegations of questionable
real estate deals, secret coordination with Russian operatives who
hacked Democratic targets during the election, and evenings Mr. Trump
spent with prostitutes.

In the letter to Congress, Mr. Cohen denied the document's claims,
including one allegation that he had secret meetings in Prague with a
Russian official last summer.

The letter says that Mr. Cohen has never been to Prague and that his
passport shows no visits to the country. Mr. Cohen also denied being
part of an effort to cover up what the dossier called Mr. Trump's
relationship with Russia.

``Mr. Cohen is not aware of any impropriety related to Mr. Trump's
`relationship' with Russia, nor is he aware of Mr. Trump having an
improper political relationship with officials of the Russian
Federation,'' letter said.

Fusion GPS, the Washington research firm that commissioned Mr. Steele to
produce the dossier, declined to comment about Mr. Cohen's letter.

Mr. Steele's investigation was paid for by political operatives ---
first by Republicans and then by Democrats. But the document also piqued
the interest of the F.B.I., which was investigating Russian meddling and
possible ties to the Trump campaign.

``The committee should discern and publicly disclose the entity or
entities that paid for the 35-page dossier,'' Mr. Cohen's letter to the
House Intelligence Committee said.

American intelligence officials briefed Mr. Trump on the dossier in
January, and it surfaced online soon afterward.

No evidence has surfaced so far that Trump aides or campaign advisers
were involved in Russian efforts to disrupt the 2016 election, but flat
denials about contacts with Russia have caused political headaches for
Mr. Trump. The president originally denied that his campaign had any
contacts with Russian officials, only to see journalists and
investigators uncover one meeting after the next.

As he pushed for the proposed Moscow project, Mr. Cohen sent an email to
a spokesman for the Russian president, Vladimir V. Putin. But Mr. Cohen
sent the message to a general email inbox, not directly to the
spokesman, Dmitri S. Peskov.

Mr. Peskov on Wednesday confirmed to reporters that the Kremlin had
received the email, but he said he did not respond to it, and that his
office did not get involved in such matters.

Advertisement

\protect\hyperlink{after-bottom}{Continue reading the main story}

\hypertarget{site-index}{%
\subsection{Site Index}\label{site-index}}

\hypertarget{site-information-navigation}{%
\subsection{Site Information
Navigation}\label{site-information-navigation}}

\begin{itemize}
\tightlist
\item
  \href{https://help.nytimes.com/hc/en-us/articles/115014792127-Copyright-notice}{©~2020~The
  New York Times Company}
\end{itemize}

\begin{itemize}
\tightlist
\item
  \href{https://www.nytco.com/}{NYTCo}
\item
  \href{https://help.nytimes.com/hc/en-us/articles/115015385887-Contact-Us}{Contact
  Us}
\item
  \href{https://www.nytco.com/careers/}{Work with us}
\item
  \href{https://nytmediakit.com/}{Advertise}
\item
  \href{http://www.tbrandstudio.com/}{T Brand Studio}
\item
  \href{https://www.nytimes.com/privacy/cookie-policy\#how-do-i-manage-trackers}{Your
  Ad Choices}
\item
  \href{https://www.nytimes.com/privacy}{Privacy}
\item
  \href{https://help.nytimes.com/hc/en-us/articles/115014893428-Terms-of-service}{Terms
  of Service}
\item
  \href{https://help.nytimes.com/hc/en-us/articles/115014893968-Terms-of-sale}{Terms
  of Sale}
\item
  \href{https://spiderbites.nytimes.com}{Site Map}
\item
  \href{https://help.nytimes.com/hc/en-us}{Help}
\item
  \href{https://www.nytimes.com/subscription?campaignId=37WXW}{Subscriptions}
\end{itemize}
