Sections

SEARCH

\protect\hyperlink{site-content}{Skip to
content}\protect\hyperlink{site-index}{Skip to site index}

\href{https://www.nytimes.com/section/politics}{Politics}

\href{https://myaccount.nytimes.com/auth/login?response_type=cookie\&client_id=vi}{}

\href{https://www.nytimes.com/section/todayspaper}{Today's Paper}

\href{/section/politics}{Politics}\textbar{}Kaspersky Lab Antivirus
Software Is Ordered Off U.S. Government Computers

\url{https://nyti.ms/2eV8nKY}

\begin{itemize}
\item
\item
\item
\item
\item
\end{itemize}

Advertisement

\protect\hyperlink{after-top}{Continue reading the main story}

Supported by

\protect\hyperlink{after-sponsor}{Continue reading the main story}

\hypertarget{kaspersky-lab-antivirus-software-is-ordered-off-us-government-computers}{%
\section{Kaspersky Lab Antivirus Software Is Ordered Off U.S. Government
Computers}\label{kaspersky-lab-antivirus-software-is-ordered-off-us-government-computers}}

\includegraphics{https://static01.nyt.com/images/2017/09/14/us/14dc-hack/14dc-hack-articleInline.jpg?quality=75\&auto=webp\&disable=upscale}

By \href{http://www.nytimes.com/by/matthew-rosenberg}{Matthew Rosenberg}
and \href{http://www.nytimes.com/by/ron-nixon}{Ron Nixon}

\begin{itemize}
\item
  Sept. 13, 2017
\item
  \begin{itemize}
  \item
  \item
  \item
  \item
  \item
  \end{itemize}
\end{itemize}

WASHINGTON --- The federal government moved on Wednesday to wipe from
its computer systems any software made by a prominent Russian
cybersecurity firm, Kaspersky Lab, that is being investigated by the
F.B.I. for possible links to Russian security services.

The concerns surrounding Kaspersky, whose software is sold throughout
the United States, are longstanding. The F.B.I., aided by American
spies, has for years been trying to determine whether Kaspersky's senior
executives are working with Russian military and intelligence, according
to current and former American officials. The F.B.I. has also been
investigating whether Kaspersky software, including its well-regarded
antivirus programs, contain back doors that could allow Russian
intelligence access into computers on which it is running. The company
denies the allegations.

The officials, all of whom spoke on the condition of anonymity because
the inquiries are classified, would not provide details of the
information they have collected on Kaspersky. But on Wednesday, Elaine
C. Duke, the acting secretary of Homeland Security, ordered federal
agencies to develop plans to remove Kaspersky software from government
systems in the next 90 days.

Wednesday's announcement is the latest instance of the apparent
disconnect between the Trump White House, which has often downplayed the
threat of Russian interference to the country's infrastructure, and
front-line American law enforcement and intelligence officials, who are
engaged in a perpetual shadow war against Moscow-directed operatives.

Kaspersky's business in the United States now appears to be the latest
casualty in those spy wars. Best Buy, the electronics giant, announced
last week that it was pulling Kaspersky Lab's cybersecurity products
from its shelves and website, and the Senate is voting this week on a
defense-spending bill that would ban Kaspersky Lab products from being
used by American government agencies, effectively codifying Wednesday's
directive into law.

Kaspersky is considered one of the foremost cybersecurity research firms
in the world, and has considerable expertise in designing antivirus
software and tools to uncover spyware used by Western intelligence
services. The company was
\href{https://www.nytimes.com/2016/06/11/world/europe/kaspersky-lab-russia-cybercrime-internet.html?mcubz=3}{founded
by} Eugene V. Kaspersky, who attended a high school that trained Russian
spies, and later wrote software for the Soviet Army before going on to
found Kaspersky Lab in 1997. He has insisted that neither he nor his
company have active ties to the Russian military or intelligence
services.

Yet despite its prominence in the cybersecurity world, its origins in
Russia have for years fueled suspicions about its possible ties to
Russia's intelligence agencies. Federal officials have warned private
companies to avoid Kaspersky software, and earlier this year the firm
was removed from two lists of approved vendors used by government
agencies to purchase technology.

At a Senate hearing in May, a number of senior American security
officials, including the chiefs of the F.B.I. and the C.I.A., were even
more blunt when asked if they would be comfortable with Kaspersky
software running on their agencies' systems:
\href{https://www.youtube.com/watch?v=TJdEq8YqzIg}{``No,''} they said.

Still, Kaspersky's software is believed to be used in many federal
agencies, especially its antivirus products, though there is no reliable
estimate of its ubiquity --- government computer systems tend be a
jumbled-together collection of often-aging software and hardware, and no
central authority keeps track of who uses what.

Kaspersky's software is also widely used by state governments and
ordinary Americans. The company says it has more than 400 million users
around the world. It also has a robust business analyzing and
investigating cyberthreats.

``The risk that the Russian government, whether acting on its own or in
collaboration with Kaspersky, could capitalize on access provided by
Kaspersky products to compromise federal information and information
systems directly implicates U.S. national security,'' Ms. Duke said in a
statement.

Kaspersky said it was disappointed with Homeland Security's decision and
denied any ties to the Russian government.

Advertisement

\protect\hyperlink{after-bottom}{Continue reading the main story}

\hypertarget{site-index}{%
\subsection{Site Index}\label{site-index}}

\hypertarget{site-information-navigation}{%
\subsection{Site Information
Navigation}\label{site-information-navigation}}

\begin{itemize}
\tightlist
\item
  \href{https://help.nytimes.com/hc/en-us/articles/115014792127-Copyright-notice}{©~2020~The
  New York Times Company}
\end{itemize}

\begin{itemize}
\tightlist
\item
  \href{https://www.nytco.com/}{NYTCo}
\item
  \href{https://help.nytimes.com/hc/en-us/articles/115015385887-Contact-Us}{Contact
  Us}
\item
  \href{https://www.nytco.com/careers/}{Work with us}
\item
  \href{https://nytmediakit.com/}{Advertise}
\item
  \href{http://www.tbrandstudio.com/}{T Brand Studio}
\item
  \href{https://www.nytimes.com/privacy/cookie-policy\#how-do-i-manage-trackers}{Your
  Ad Choices}
\item
  \href{https://www.nytimes.com/privacy}{Privacy}
\item
  \href{https://help.nytimes.com/hc/en-us/articles/115014893428-Terms-of-service}{Terms
  of Service}
\item
  \href{https://help.nytimes.com/hc/en-us/articles/115014893968-Terms-of-sale}{Terms
  of Sale}
\item
  \href{https://spiderbites.nytimes.com}{Site Map}
\item
  \href{https://help.nytimes.com/hc/en-us}{Help}
\item
  \href{https://www.nytimes.com/subscription?campaignId=37WXW}{Subscriptions}
\end{itemize}
