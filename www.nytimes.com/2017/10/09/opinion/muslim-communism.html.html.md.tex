Sections

SEARCH

\protect\hyperlink{site-content}{Skip to
content}\protect\hyperlink{site-index}{Skip to site index}

\href{https://myaccount.nytimes.com/auth/login?response_type=cookie\&client_id=vi}{}

\href{https://www.nytimes.com/section/todayspaper}{Today's Paper}

\href{/section/opinion}{Opinion}\textbar{}What Killed the Promise of
Muslim Communism?

\href{https://nyti.ms/2hXjQPb}{https://nyti.ms/2hXjQPb}

\begin{itemize}
\item
\item
\item
\item
\item
\end{itemize}

Advertisement

\protect\hyperlink{after-top}{Continue reading the main story}

Supported by

\protect\hyperlink{after-sponsor}{Continue reading the main story}

\href{/section/opinion}{Opinion}

\href{/column/red-century}{Red Century}

\hypertarget{what-killed-the-promise-of-muslim-communism}{%
\section{What Killed the Promise of Muslim
Communism?}\label{what-killed-the-promise-of-muslim-communism}}

By John T. Sidel

\begin{itemize}
\item
  Oct. 9, 2017
\item
  \begin{itemize}
  \item
  \item
  \item
  \item
  \item
  \end{itemize}
\end{itemize}

LONDON --- For a brief moment after the Bolshevik uprisings of 1917, it
looked like revolution might be waged across vast swaths of the world
under the joint banner of Communism and Islam.

Pan-Islam had emerged in the final decades of the Ottoman Empire, with
the efforts of Sultan Abdulhamid II to lay claim to the title of caliph
among Muslims. New forms of Islamic schooling and associations began to
emerge across the Arab world and beyond. From Egypt and Iraq to India
and the Indonesian archipelago, Islam became a rallying call against
European colonialism and imperialism.

Islam's mobilizing power attracted Communist activists in the 1910s and
1920s. The Bolsheviks, who lacked organizational infrastructure in the
vast Muslim lands of the former Russian empire, allied with Islamic
reformers in those areas. They created a special Commissariat for Muslim
Affairs under the Tatar Bolshevik Mirsaid Sultan-Galiev, promising to
establish a distinctive ``Muslim Communism'' across the Caucasus and
Central Asia. During the 1920 Congress of the Peoples of the East in
Baku, in what is today Azerbaijan, the Comintern chairman Grigory
Zinoviev, a Ukrainian Jew, called for waging a ``holy war'' against
Western imperialism.

But as we now know, Communism and Islam failed to coalesce into a
lasting alliance. By the onset of the Cold War, they seemed irrevocably
opposed. Differing views about Communism divided Muslims across Asia,
Africa and the Middle East in their struggles for independence and
emancipation during the second half of the 20th century. An
anti-Communist jihad fundamentally remade Afghanistan in the 1980s and
helped set the stage for the rise of Al Qaeda and the emergence of a new
form of Islamist terrorism.

Yet around the time of the Russian Revolution, the prospects of
Communism and Islam joining forces seemed very bright. They were perhaps
no brighter than in the Indonesian archipelago, then under Dutch rule:
In 1918-21, left-wing labor organizers working hand in glove with
Islamic scholars and pious Muslim merchants built the biggest mass
movement in Southeast Asia.

Over the preceding decade, Indonesian labor activists had already
established a strong union representing workers on the extensive
railroad network servicing the vast plantation economy of Java and
Sumatra. By 1914, the Indische Sociaal-Democratische Vereeniging, or
Indies Social-Democratic Union, had expanded from labor organizing among
railroad workers into broader forms of social activism and political
action against colonial rule.

\includegraphics{https://static01.nyt.com/images/2017/10/09/opinion/09redcenturyWeb/09redcenturyWeb-articleLarge.jpg?quality=75\&auto=webp\&disable=upscale}

In particular, members began to join the Sarekat Islam, an organization
founded in 1912 as a Muslim batik traders' association that had morphed
into a broader popular movement and was staging mass rallies and strikes
across Java. Socialist influence within the Sarekat Islam was already
evident at the movement's congress in 1916, where the Prophet Muhammad
was proclaimed to be ``the father of Socialism and the pioneer of
democracy'' and ``the Socialist par excellence.''

The Russian Revolution further inspired the Sarekat Islam. By late 1917,
activists from the Indies Social-Democratic Union had begun agitating
and organizing among the lower ranks of the Dutch armed forces in the
Indies. Borrowing the successful tactics of the Bolsheviks in Russia,
hundreds of sailors and soldiers were recruited in the hope of staging
mutinies and uprisings. The Dutch colonial authorities promptly arrested
and imprisoned the activists and ordered their expulsion from the
Indies.

But by 1920, the Indies Social-Democratic Union had renamed itself the
Communist Union of the Indies, becoming the first Communist party in
Asia to join the Comintern. New unions were formed on Java and Sumatra.
Peasant villagers mobilized against landowners. A railway strike briefly
paralyzed the plantation belt in eastern Sumatra.

It was in this context that the legendary figure of Tan Malaka first
appeared. The scion of an aristocratic family from western Sumatra, Tan
Malaka had spent World War I as a student in the Netherlands. He came
into contact with Socialist activists and ideas, and witnessed the
short-lived Troelstra Revolution of late 1918, during which Dutch
social-democrats briefly tried to emulate an ongoing revolutionary
uprising in Germany. In early 1919, Tan Malaka returned to Indonesia,
where he was soon drawn into labor organizing. He joined the embryonic
local Communist Party, quickly ascending to its leadership --- before
the colonial government forced him into exile, and back to the
Netherlands, in early 1922.

And so it was with early experience of the revolutionary potential of
combining Communism and Islam that Tan Malaka made an appearance at the
Fourth Comintern Congress in Moscow and Petrograd in 1922. There, he
delivered a memorable speech about the similarities between Pan-Islamism
and Communism. Pan-Islamism was not religious per se, he argued, but
rather ``the brotherhood of all Muslim peoples, and the liberation
struggle not only of the Arab but also of the Indian, the Javanese and
all the oppressed Muslim peoples.''

``This brotherhood,'' he added, ``means the practical liberation
struggle not only against Dutch but also against English, French and
Italian capitalism, therefore against world capitalism as a whole.''

The official record of the proceedings notes that Tan Malaka's
impassioned plea for an alliance between Communism and Pan-Islamism was
met with ``lively applause.'' But his memoirs recall that after three
days of heated debate following his speech, he was formally prohibited
from further contributing to the proceedings. The official conclusions
of the Fourth Comintern Congress, including the ``Theses on the Eastern
Question,'' are notably ambiguous on the question of Pan-Islamism and
strikingly silent on Indonesia, even though the movement there was far
more successful than any other Communist mobilization in the so-called
East at the time.

An alliance between Communism and Islam was not to be, neither in
Indonesia nor elsewhere. The strength of Communism, as a movement, was
its ability to mobilize laborers to fight for better wages and working
conditions through unions, whether in oil boomtown Baku or the
plantations of Java and Sumatra. But as a form of government, Communism
meant one-party rule, a command economy with collectivized agriculture
and party-state control over all spheres of social life --- including
religion.

Islamism, by contrast, was a much broader and enduringly more open-ended
and ambiguous basis for political engagement. In Java and elsewhere,
``Islam'' provided a banner for Muslim merchants to contest economic
encroachment by non-Muslims and build an infrastructure for organizing
in the countryside, largely through Islamic schools. Politically, it was
a supple notion: Islamist scholars and activists could be for
colonialism, Communism or capitalism.

In Indonesia, tensions between Communists and Islamic leaders had
already begun to divide Sarekat Islam in the early 1920s. Communists
urged escalating strikes and protests, whereas Islamic leaders advocated
accommodation with the Dutch colonial authorities. Sarekat Islam
dissolved in the face of Dutch repression after failed rebellions in
1926-7.

In the late 1940s, Islamic parties opposed the Partai Komunis Indonesia
(P.K.I.), or Indonesian Communist Party, during the struggle for
independence. Islamic parties were uncomfortable with the Communists'
insistence that independence from Dutch colonial rule also upend
aristocratic privileges and bring about the establishment of Socialist
forms of ownership over land and industry. This conflict extended into
the early post-independence period. Islamic organizations actively
participated in the anti-Communist pogroms of 1965-66, which destroyed
the P.K.I. and left hundreds of thousands of casualties across
Indonesia.

By this time, the pattern of antagonism was well established across the
Muslim world, and it persisted throughout the Cold War. The
institutional and ideological boundaries of both Communism and Islamism
hardened, dashing prospects for renewed experiments in political
alliance-building.

In Muslim areas of the Soviet Union, the party-state suppressed
institutions of Islamic worship, education, association and pilgrimage,
which were viewed as obstacles to ideological and social transformation
along Communist lines. Where Islamic states were established, left-wing
politics was often associated with blasphemy, and outlawed. In countries
like Sudan, Yemen, Syria, Iraq and Iran, Communist and other left-wing
parties found themselves in bitter competition for power with Islamists.

One effect of the failure of revolutionary forces to mobilize under the
joint banner of Communism and Islam was to deeply divide Muslims,
weakening their capacity first to fight colonialism during the first
half of 20th century and then to resist the rise of authoritarianism
across the Muslim world. Another effect was to stimulate new forms of
U.S.-backed, anti-Soviet Islamist mobilization during the Cold War ---
including some that turned into the virulent anti-Western terrorist
groups that partly define the world today.

Divisions between leftists and Islamists in Egypt after the fall of
President Hosni Mubarak in 2011 also helped set the stage for the
country's return to military rule in mid-2013. Similar tensions divided
the opposition to President Bashar al-Assad in Syria, paving the way for
the country's descent into civil war over six years ago. A full century
after the Russian Revolution, the failed alliance between Communism and
Islam continues to shape the politics of the Muslim world.

Advertisement

\protect\hyperlink{after-bottom}{Continue reading the main story}

\hypertarget{site-index}{%
\subsection{Site Index}\label{site-index}}

\hypertarget{site-information-navigation}{%
\subsection{Site Information
Navigation}\label{site-information-navigation}}

\begin{itemize}
\tightlist
\item
  \href{https://help.nytimes.com/hc/en-us/articles/115014792127-Copyright-notice}{©~2020~The
  New York Times Company}
\end{itemize}

\begin{itemize}
\tightlist
\item
  \href{https://www.nytco.com/}{NYTCo}
\item
  \href{https://help.nytimes.com/hc/en-us/articles/115015385887-Contact-Us}{Contact
  Us}
\item
  \href{https://www.nytco.com/careers/}{Work with us}
\item
  \href{https://nytmediakit.com/}{Advertise}
\item
  \href{http://www.tbrandstudio.com/}{T Brand Studio}
\item
  \href{https://www.nytimes.com/privacy/cookie-policy\#how-do-i-manage-trackers}{Your
  Ad Choices}
\item
  \href{https://www.nytimes.com/privacy}{Privacy}
\item
  \href{https://help.nytimes.com/hc/en-us/articles/115014893428-Terms-of-service}{Terms
  of Service}
\item
  \href{https://help.nytimes.com/hc/en-us/articles/115014893968-Terms-of-sale}{Terms
  of Sale}
\item
  \href{https://spiderbites.nytimes.com}{Site Map}
\item
  \href{https://help.nytimes.com/hc/en-us}{Help}
\item
  \href{https://www.nytimes.com/subscription?campaignId=37WXW}{Subscriptions}
\end{itemize}
