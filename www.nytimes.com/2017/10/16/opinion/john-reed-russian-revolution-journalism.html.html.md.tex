Sections

SEARCH

\protect\hyperlink{site-content}{Skip to
content}\protect\hyperlink{site-index}{Skip to site index}

\href{https://myaccount.nytimes.com/auth/login?response_type=cookie\&client_id=vi}{}

\href{https://www.nytimes.com/section/todayspaper}{Today's Paper}

\href{/section/opinion}{Opinion}\textbar{}The Journalist and the
Revolution

\href{https://nyti.ms/2kT9518}{https://nyti.ms/2kT9518}

\begin{itemize}
\item
\item
\item
\item
\item
\item
\end{itemize}

Advertisement

\protect\hyperlink{after-top}{Continue reading the main story}

Supported by

\protect\hyperlink{after-sponsor}{Continue reading the main story}

\href{/section/opinion}{Opinion}

\href{/column/red-century}{Red Century}

\hypertarget{the-journalist-and-the-revolution}{%
\section{The Journalist and the
Revolution}\label{the-journalist-and-the-revolution}}

By Jack Shenker

\begin{itemize}
\item
  Oct. 16, 2017
\item
  \begin{itemize}
  \item
  \item
  \item
  \item
  \item
  \item
  \end{itemize}
\end{itemize}

At the house on the corner, past the cigarette kiosk and the laundry
place and the twisted metal barricade assembled to shield passers-by
from gunfire, I pitched in with a group of kids carrying rocks.

It was early February 2011. Egypt, where I had been working as a
reporter for several years, was engulfed in revolution, and rubble was
being ferried to the rooftop by protesters in an effort to defend Tahrir
Square from a counterrevolutionary assault. Journalists are often told
to stand separate from the events they are reporting on, to ensure their
notebooks are tidy organs of record, carefully sealed off from the
turmoil around them. The pages of mine were smeared with grime and dust,
and some were splotched with tears.

The historian Howard Zinn once noted that ``you can't be neutral on a
moving train,'' and no train moves faster or tilts more fiercely than a
nation consumed by popular rebellion. Exactly where and how reporters
should plant their feet at such a moment is a question that must be
asked anew by each correspondent, in every corner of the world, uprising
after uprising. Many of us who have been forced to grapple with it ---
as I was that afternoon --- have arrived at different answers. All of
them are messy. And for the past century all of them, consciously or
not, have been shaped to some degree by the work of John Reed, the
legendary chronicler of Russia's October Revolution in 1917.

Reed, a young American who arrived in Saint Petersburg with his wife,
Louise Bryant, just as Russia's fragile provisional government began to
buckle and the city's back streets were humming with whispers of
strikes, mutinies and sedition, made no claims to impartiality in his
coverage. ``This was his revolution, not an obscure event in a foreign
country,'' the British historian A.J.P. Taylor later wrote. Reed's book,
``Ten Days That Shook the World,'' explores the Communist insurgency not
as a scientist might analyze slides through a microscope but rather as a
lived experience, with all of a real life's hopes and fears.

\includegraphics{https://static01.nyt.com/images/2017/10/16/opinion/16redcenturyweb/16shenkerWeb-articleLarge.jpg?quality=75\&auto=webp\&disable=upscale}

Far from being an invisible presence in his narrative, Reed is
frequently the star of it: bluffing his way past guards, being
threatened with assault by suspicious demonstrators, narrowly avoiding
being shot by soldiers against a wall. At one point he finds himself
participating in the distribution of leaflets announcing the fall of the
ancien régime; a few (much mythologized) pages later, he pours through
the gates of the Winter Palace in the company of triumphant
revolutionaries. In the process, he imparts to his readers a sense of
how the thrill of revolution coursed through not just his subjects'
veins but also his own. ``It is still fashionable,'' Reed remarks
unapologetically, ``to speak of the Bolshevik insurrection as an
`adventure.' Adventure it was, and one of the most marvelous mankind
ever embarked upon.''

Reed's belief that personal passion and political engagement on the part
of a reporter are not antithetical to meaningful revolutionary
journalism but rather lie at the very core of it was not the only
feature of his work that resonated with me as I attempted to chart a
very different national transformation --- more than 2,000 miles away
and more than nine decades later. Just as striking was the way his prose
is littered with people and places that seem a long way from anywhere
but are actually at the center of everything.

Barely 72 hours after the Bolsheviks had seized power, for example, and
just as the civil war that would divide Russia for the next half decade
began to crystallize, Reed devotes several paragraphs to an ill-tempered
conversation between an uneducated member of the Red Guards and a
supercilious counterrevolutionary student, which took place by the door
of a provincial railway station.

The pair were arguing about the proletariat and the bourgeoisie; beyond
them, rival armies and ideas were on the march. In any other context,
the soldier would have deferred to the student, his class superior. But
the old orthodoxy was crumbling, and Reed shows both men navigating the
social terrain of something unknown, something new.

In Egypt, too, the real story of unrest lay not in Hosni Mubarak's
presidential palace but in the mundane spaces where norms were shifting:
in the tuk-tuks, previously confined to the informal settlements on the
margins of the capital, which now honked their way defiantly into the
city center; in the schoolchildren re-enacting battles against the
security forces on their playground; in the low-level insurgencies waged
in family dining rooms, college lecture halls and factory floors across
the country.

Trotsky would later write of 1917 that the history of revolution is
``first of all a history of the forcible entrance of the masses into the
realm of rulership over their own destiny.'' Reed understood this not as
an academic treatise in which the masses remain faceless but as a
practical reality, one that locates the essence of revolution as much in
the erratic widening of individual imaginations as it does in the
corridors of formal power or in the machinations of competing leaders.

Rereading ``Ten Days That Shook the World'' today, it is not the
near-verbatim accounts of interminable, overlapping Soviet committee
meetings that stand out, nor the alphabet soup of long-forgotten
organizational acronyms that requires a 10-page glossary. It is not even
the grand showpieces that Reed witnessed and relates in his work, like
the raucous smoke-filled meetings at Lenin's Smolny headquarters where
insurrection was hatched, or the mammoth funeral processions for the
martyrs of Moscow after the city was won. Rather it is the description
of a well-to-do young woman's hysterics after she is addressed as
``comrade'' by a streetcar conductor. It is the scene where an old
workman pilots an auto-truck back toward the capital after the
revolution is victorious, sweeping his arm across the urban haze: ``
`Mine!' he cried, his face all alight. `All mine now! My Petrograd!' ''
It is all the times when Reed trains his gaze on the irrefutably human
micro-dramas that are inevitable, and epic, when history is sloping; the
times when he homes in on the struggles that take place when every
person, with their own varying level of investment in yesterday, tries,
tentatively, to find a foothold in tomorrow.

Revolutions are by their nature make-do affairs with few maps to guide
either participants or observers. When people are making and doing
something radically transformative, and transforming themselves in the
process, it's impossible to interrogate what's happening if you're
relying solely on the templates that came before. Reed appreciated this.
Rather than fighting unknowability, he embraced it. He opens the main
body of his book by recounting the bafflement of a visiting sociology
professor who is informed that revolutionary sentiment is both rising
and on the wane. ``The professor was puzzled,'' Reed notes, ``but he
need not have been; both observations were correct.''

Reed is not afraid to convey the contradictions of revolution --- its
tangle of the tumultuous and the prosaic, its clouds of misinformation
and obscurity. He describes the trundling of armored vehicles in the
streets, the voices in the darkness, the fear and reckless daring from
which the new Russia was born. He captures, just as I tried to in Egypt,
that curious feature of rapid political change whereby the furniture and
accessories of the previous system remain dotted about the landscape,
suddenly shorn of their power, both unaltered and simultaneously absurd.
He probes the language of elites as they scrabble to keep up with
events: One tycoon tells him that revolution is a sickness and that
intervention is necessary to prevent it, just as ``one would intervene
to cure a sick child'' --- a foreshadowing of the infantilizing rhetoric
adopted by successive Egyptian leaders. ``The air was full of confused
sound,'' Reed reports, in a passage that could have been lifted straight
out of Cairo during its own uprising. ``The city stirred uneasily,
wakeful.''

Across time, place and context, revolutions occur when a whiff of
possibility appears, a broadening of horizons, tangible evidence that
the status quo is not immutable. Wherever we are, we are all capable of
picking up that scent. Of course, the full history of Russia's
revolution contains great shafts of darkness as well as light. In Egypt,
too, albeit under very different circumstances, the utopianism of 2011
has given way to suffocation and violence, as a new iteration of
military despots attempt to expunge collective memories of that brief
moment when the ability to shape the world around oneself had fallen
into collective hands. Far from invalidating the sort of reporting Reed
helped pioneer, though, the fragility of such moments reinforces its
worth. It is through the act of storytelling that revolution itself
becomes possible.

``Ten Days That Shook the World'' lives on, not because Reed got
everything right (he didn't) or because the revolution he covered was an
uncomplicated success story (it was anything but), but because he
understood the real force of revolutionary journalism: its potential to
rouse all who engage with it --- not least the reporters themselves.

Advertisement

\protect\hyperlink{after-bottom}{Continue reading the main story}

\hypertarget{site-index}{%
\subsection{Site Index}\label{site-index}}

\hypertarget{site-information-navigation}{%
\subsection{Site Information
Navigation}\label{site-information-navigation}}

\begin{itemize}
\tightlist
\item
  \href{https://help.nytimes.com/hc/en-us/articles/115014792127-Copyright-notice}{©~2020~The
  New York Times Company}
\end{itemize}

\begin{itemize}
\tightlist
\item
  \href{https://www.nytco.com/}{NYTCo}
\item
  \href{https://help.nytimes.com/hc/en-us/articles/115015385887-Contact-Us}{Contact
  Us}
\item
  \href{https://www.nytco.com/careers/}{Work with us}
\item
  \href{https://nytmediakit.com/}{Advertise}
\item
  \href{http://www.tbrandstudio.com/}{T Brand Studio}
\item
  \href{https://www.nytimes.com/privacy/cookie-policy\#how-do-i-manage-trackers}{Your
  Ad Choices}
\item
  \href{https://www.nytimes.com/privacy}{Privacy}
\item
  \href{https://help.nytimes.com/hc/en-us/articles/115014893428-Terms-of-service}{Terms
  of Service}
\item
  \href{https://help.nytimes.com/hc/en-us/articles/115014893968-Terms-of-sale}{Terms
  of Sale}
\item
  \href{https://spiderbites.nytimes.com}{Site Map}
\item
  \href{https://help.nytimes.com/hc/en-us}{Help}
\item
  \href{https://www.nytimes.com/subscription?campaignId=37WXW}{Subscriptions}
\end{itemize}
