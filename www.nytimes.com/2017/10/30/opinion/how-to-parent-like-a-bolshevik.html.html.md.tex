Sections

SEARCH

\protect\hyperlink{site-content}{Skip to
content}\protect\hyperlink{site-index}{Skip to site index}

\href{https://myaccount.nytimes.com/auth/login?response_type=cookie\&client_id=vi}{}

\href{https://www.nytimes.com/section/todayspaper}{Today's Paper}

\href{/section/opinion}{Opinion}\textbar{}How to Parent Like a Bolshevik

\href{https://nyti.ms/2yXHHm2}{https://nyti.ms/2yXHHm2}

\begin{itemize}
\item
\item
\item
\item
\item
\item
\end{itemize}

Advertisement

\protect\hyperlink{after-top}{Continue reading the main story}

Supported by

\protect\hyperlink{after-sponsor}{Continue reading the main story}

\href{/section/opinion}{Opinion}

\href{/column/red-century}{Red Century}

\hypertarget{how-to-parent-like-a-bolshevik}{%
\section{How to Parent Like a
Bolshevik}\label{how-to-parent-like-a-bolshevik}}

By Yuri Slezkine

\begin{itemize}
\item
  Oct. 30, 2017
\item
  \begin{itemize}
  \item
  \item
  \item
  \item
  \item
  \item
  \end{itemize}
\end{itemize}

\includegraphics{https://static01.nyt.com/images/2017/10/30/opinion/30slezkineWeb/30slezkineWeb-articleLarge.jpg?quality=75\&auto=webp\&disable=upscale}

The original Bolsheviks expected Communism in their lifetime. When that
began to appear unlikely, they moved the deadline to the lifetime of
their children.

``Fire cannot be contained,'' Nina Avgustovna Didrikil, an employee at
the Lenin Institute, wrote in her diary in 1920. ``It will burst forth,
and I am certain that if it does not burst forth within me, it will do
so through my children, who will make me immortal.''

The path to the parents' immortality was the children's happiness. ``You
are happy, and you will be even happier when you realize just how happy
you are,'' wrote Didrikil in 1933 to one of her daughters on her 17th
birthday. ``You are the youngest and strongest, and the whole life of
your society is young and strong. My wish for you, in your 17th spring,
is that you continue to move closer and closer, in all your interests,
feelings, and thoughts, to the camp of the youngest and strongest: to
Marx, Engels, Lenin and all the true Bolsheviks.''

The most celebrated interpretation of children's happiness in the Soviet
Union was Constantin Stanislavsky's production of Maurice Maeterlinck's
play ``The Blue Bird*,''* which had its premiere in 1908 and survived
the revolution to become a required rite of passage for the children of
the original Bolsheviks (and eventually the longest-running theater
production of all time).

In the play, a little girl named Mytyl and her brother Tyltyl find the
bird of happiness and release it into the world. In her diary, Didrikil
described the Soviet Union as ``that miracle-producing magic garden of
Communism, from which blue birds fly to every corner of the world,
spreading the news of Communist happiness.''

The key to finding the blue bird of happiness was education, and the
sacred center of Soviet education was Alexander Pushkin. ``We spoke of
Pushkin as if he were alive,'' wrote Lydia Libedinskaia, who attended
the Moscow Exemplаry School in the 1930s. ``We kept asking each other if
Pushkin would like our metro, our new bridges that spanned the Moskva,
the neon lights on Gorky Street.''

After ushering in the New Year of 1937, the 16-year-old Libedinskaia and
her friends went to the Pushkin Monument in the center of Moscow.
According to her memoirs, they gathered around Pushkin's statue and took
turns ``reading his poems to him --- one after another, on and on.''

\begin{quote}
Suddenly, in the frosty silence of that New Year's Eve, a boy's voice,
trembling with excitement, rang out:

While freedom kindles us, my friend,

While honor calls us and we hear it,

Come, to our country let us tend

The noble promptings of the spirit.

It sounded like a vow. That is how, in solemn silence, warriors take
their oaths. Happy are those who had such moments in their youth ...

The snow kept falling, melting on our flushed faces and silvering our
hair. Our hearts were overflowing with love for Pushkin, poetry, Moscow,
and our country. We yearned for great deeds and vowed silently to
accomplish them. My generation! The children of the 1920s, the men and
women of a happy and tragic age! You grew up as equal participants in
the building of the Soviet Union, you were proud of your fathers, who
had carried out an unheard-of revolution, you dreamed of becoming their
worthy successors ...
\end{quote}

Many of these boys and girls would be killed during World War II, better
known in the Soviet Union as the Great Patriotic War. Some would be
arrested and sent into exile. Some, like Libedinskaia, would go on to
welcome Khrushchev's thaw and then Gorbachev's Perestroika. Most would
continue to be proud of their fathers. None would consider themselves
their spiritual successors.

Bolshevism --- and Marxism in general --- had a remarkably flat
conception of human nature: A revolution in property relations was the
only necessary condition for a revolution in human hearts. The
dictatorship of the proletariat would automatically result in the
withering away of whatever got in the way of Communism, from the state
to the family. Accordingly, the Bolsheviks never worried much about the
family, never policed the home, and never connected the domestic rites
of passage --- childbirth, marriage and death --- to their sociology and
political economy.

No one knew what a good Communist home was supposed to look like, and no
one came to check whether Nina Avgustovna Didrikil and her husband, the
commander of the assault on the Winter Palace and later president of Red
Sports International, Nikolai Ilyich Podvoisky, were reading Marx,
Engels and Lenin to their children. They were not, and they were not
expected to. They were reading Goethe, Heine and Tolstoy instead.

Most millenarian sects that survive the death of the first generation of
believers are those that preserve the hope of salvation by maintaining a
strict separation --- physical, ritual and intellectual --- from the
outside world. The Bolsheviks, secure in their economic determinism,
assumed that the outside world would join them as a matter of course,
and embraced non-Communist art and literature as both prologue and
accompaniment to their own. Even at the height of fear and suspicion,
when anyone connected to the outside world might be subject to
sacrificial murder, Soviet readers were expected to learn from Dante,
Shakespeare and Cervantes.

The children of the Bolshevik millenarians never read Marx, Engels or
Lenin at home, and, after the educational system was rebuilt around
Pushkin, Soviet children stopped reading them in school, too. At home,
the children of the Bolsheviks read what they called the ``treasures of
world literature,'' with an emphasis on the Golden Ages analogous to
their own (the Renaissance, Romanticism and the realist novel,
especially Balzac, Dickens and Tolstoy).

What most of these books had in common was their anti-millenarian
humanism. Some particular favorites, including ``A Tale of Two Cities''
and Anatole France's ``The Gods Are Athirst,'' were expressly
anti-revolutionary; most did the opposite of what the Bolsheviks
preached by embracing the folly and pathos of human existence. The point
of the golden ages, as opposed to the silver ones and any number of
modernisms, is the affirmation of ``really existing'' humanity.

The books proclaimed as models at the First Congress of Soviet Writers
in 1934 and imbibed religiously by the children of the original
Bolsheviks were profoundly anti-Bolshevik, none more so than the one
routinely described as the best of them all: Tolstoy's ``War and
Peace.'' All rules, plans, grand theories and historical explanations
were vanity, stupidity or deception. Natasha Rostova, Tolstoy's
protagonist, ``did not deign to be intelligent.'' The meaning of life
was in living it.

Something else all those books had in common was that they were
``historical'' in the sense of being self-consciously concerned with the
passing of time and with the past as a foreign country. The children of
the Revolution did not only live in the past --- they loved it for being
the past and, like most readers and writers of historical fiction,
tended to focus on lost causes: Scott's Scots, Boussenard's Boers,
Cooper's Mohicans, Sienkiewicz's Poles, Mayne Reid's Seminoles,
Mérimée's Corsicans, Pushkin's Pugachev, Gogol's Taras Bulba, Stendhal's
Napoleon and everything Dumas's Musketeers pledged to preserve, from Her
Majesty's honor to the head of Charles I.

Even the great socialist classics, Raffaello Giovagnoli's ``Spartacus''
and Ethel Voynich's ``The Gadfly,'' were about Romantic self-sacrifice.
And, of course, no one doubted that the greatest of them all was the one
that focused on the most hopeless of lost causes: the pursuit of
historical causality. Tolstoy did not deign to be intelligent.

Revolutions do not devour their children; revolutions, like all
millenarian experiments, are devoured by the children of the
revolutionaries. The Bolsheviks, who did not fear the past and who
employed God-fearing peasant nannies to bring up their children, were
particularly proficient in creating their own gravediggers.

The parents had their faith; the children had their tastes and
knowledge. The parents had comrades (fellow saints who shared their
faith); the children had friends (pseudo-kin who shared their tastes and
knowledge). The parents started out as sectarians and ended up as
priestly rulers or sacred scapegoats; the children started out as poets
and ended up as professionals and intellectuals. The parents considered
their sectarianism the realization of humanism --- until their
interrogators forced them to choose, and to die, one way or the other.
The children never knew anything but humanism and never understood their
parents' final dilemma.

As Tyltyl and Mytyl discovered at the end of ``The Blue Bird'' (and at
the beginning of their self-aware lives), the truth they were seeking
had been with them all along: They were that truth.

Advertisement

\protect\hyperlink{after-bottom}{Continue reading the main story}

\hypertarget{site-index}{%
\subsection{Site Index}\label{site-index}}

\hypertarget{site-information-navigation}{%
\subsection{Site Information
Navigation}\label{site-information-navigation}}

\begin{itemize}
\tightlist
\item
  \href{https://help.nytimes.com/hc/en-us/articles/115014792127-Copyright-notice}{©~2020~The
  New York Times Company}
\end{itemize}

\begin{itemize}
\tightlist
\item
  \href{https://www.nytco.com/}{NYTCo}
\item
  \href{https://help.nytimes.com/hc/en-us/articles/115015385887-Contact-Us}{Contact
  Us}
\item
  \href{https://www.nytco.com/careers/}{Work with us}
\item
  \href{https://nytmediakit.com/}{Advertise}
\item
  \href{http://www.tbrandstudio.com/}{T Brand Studio}
\item
  \href{https://www.nytimes.com/privacy/cookie-policy\#how-do-i-manage-trackers}{Your
  Ad Choices}
\item
  \href{https://www.nytimes.com/privacy}{Privacy}
\item
  \href{https://help.nytimes.com/hc/en-us/articles/115014893428-Terms-of-service}{Terms
  of Service}
\item
  \href{https://help.nytimes.com/hc/en-us/articles/115014893968-Terms-of-sale}{Terms
  of Sale}
\item
  \href{https://spiderbites.nytimes.com}{Site Map}
\item
  \href{https://help.nytimes.com/hc/en-us}{Help}
\item
  \href{https://www.nytimes.com/subscription?campaignId=37WXW}{Subscriptions}
\end{itemize}
