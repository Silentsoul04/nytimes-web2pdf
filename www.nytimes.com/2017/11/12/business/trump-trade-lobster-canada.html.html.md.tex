Sections

SEARCH

\protect\hyperlink{site-content}{Skip to
content}\protect\hyperlink{site-index}{Skip to site index}

\href{https://www.nytimes.com/section/business}{Business}

\href{https://myaccount.nytimes.com/auth/login?response_type=cookie\&client_id=vi}{}

\href{https://www.nytimes.com/section/todayspaper}{Today's Paper}

\href{/section/business}{Business}\textbar{}Trump's Trade Policy Is
Lifting Exports. Of Canadian Lobster.

\url{https://nyti.ms/2hugJyu}

\begin{itemize}
\item
\item
\item
\item
\item
\item
\end{itemize}

Advertisement

\protect\hyperlink{after-top}{Continue reading the main story}

Supported by

\protect\hyperlink{after-sponsor}{Continue reading the main story}

\hypertarget{trumps-trade-policy-is-lifting-exports-of-canadian-lobster}{%
\section{Trump's Trade Policy Is Lifting Exports. Of Canadian
Lobster.}\label{trumps-trade-policy-is-lifting-exports-of-canadian-lobster}}

\includegraphics{https://static01.nyt.com/images/2017/11/13/us/13dc-lobster2/01lobster2-articleLarge.jpg?quality=75\&auto=webp\&disable=upscale}

By \href{https://www.nytimes.com/by/ana-swanson}{Ana Swanson}

\begin{itemize}
\item
  Nov. 12, 2017
\item
  \begin{itemize}
  \item
  \item
  \item
  \item
  \item
  \item
  \end{itemize}
\end{itemize}

CENTREVILLE, Nova Scotia --- This lobster factory on a windswept bay in
eastern Canada is so remote that its workers have to drive for miles
just to get cellphone service. But Gidney Fisheries is truly global,
with its lobsters landing on plates in Paris and Shanghai through trade
agreements hammered out in far-off capitals.

Of late, these trade pacts have been shifting in the factory's favor,
giving it an advantage over its American competitors.

A new trade agreement between Canada and the European Union has slashed
tariffs on imports of Canadian lobsters. That means more 747s filled
with Christmas-red crustaceans will depart from Nova Scotia for European
markets this winter --- and more revenue will flow to Gidney Fisheries.
The factory, which in the 1800s sent its lobsters to Boston by
steamship, is flush with potential as it gains access to new markets and
plans to increase its work force by roughly 50 percent, adding dozens of
positions to its current payroll of around 85 workers.

``For us, free trade is a good thing,'' said Robert MacDonald, the
president of Gidney Fisheries, which processes 10,000 to 15,000 lobsters
a day.

\includegraphics{https://static01.nyt.com/images/2017/11/01/us/01lobster3/01lobster3-articleLarge.jpg?quality=75\&auto=webp\&disable=upscale}

Image

Robert MacDonald, the president of Gidney Fisheries, with a catch. ``For
us, free trade is a good thing,'' he said.Credit...Stephanie Foden for
The New York Times

The Trump administration has adopted a skeptical view of trade deals,
promising to scrap or renegotiate global agreements that it believes put
American companies and workers at a disadvantage. Among them is the
North American Free Trade Agreement, which the United States is trying
to renegotiate. It will join its partners in the agreement, Canada and
Mexico, for a fifth round of talks in Mexico City that officially begin
on Friday.

Some trade experts, though, say America's get-tough approach is
dissuading foreign partners from jumping into talks. Other countries,
like Canada, are forging ahead with their own trade deals as they balk
at the tough terms the United States is demanding in its trade
negotiations. Over the weekend, a group of 11 countries including Canada
announced that they were committed to moving ahead with the
Trans-Pacific Partnership, a sweeping multinational trade deal
negotiated by the Obama administration.

As these deals progress, American companies, particularly exporters, are
finding themselves on the losing end of global trade as their
competitors abroad gain easier access to foreign markets.

``We live in such a low-margin world, where industry after industry is
engaged in fierce global competition,'' said John G. Murphy, senior vice
president for international policy at the U.S. Chamber of Commerce.
``There is a sense in which the United States is standing still, while
countries around us are moving forward.''

It's a historic shift for the United States, which has long led the
charge for free trade and open markets. The United States has
traditionally been the global leader in forging trade pacts, including
during the Obama administration, which negotiated an earlier version of
the Trans-Pacific Partnership and began talks with Europe on an
agreement known as the Trans-Atlantic Trade and Investment Partnership.

Skeptics in the current administration criticize these pacts as a global
race to the bottom that has cost American jobs and depressed wages.
President Trump condemned the Trans-Pacific Partnership as one of the
worst deals ever negotiated and officially withdrew the United States
from the pact on his fourth day in office. Talks with Europe over the
trans-Atlantic trade pact have stalled.

Image

The factory plans to add dozens of jobs now that a new trade deal has
slashed European tariffs on Canadian lobsters.Credit...Stephanie Foden
for The New York Times

When Americans think about lobster, Maine often comes to mind. But Nova
Scotia has emerged as a fierce competitor in exporting lobsters,
particularly to Europe. Last year, American lobstermen sold only
slightly more to Europe than their Canadian counterparts.

That balance could soon shift given the Canadian-European trade pact,
which eliminated an 8 percent European tariff on live lobster when it
went into effect in September. Tariffs on frozen and processed Canadian
lobster will be phased out in the next three to five years as part of
the agreement.

The elimination of European tariffs is ``the single most challenging
issue'' for the American lobster industry, said Annie Tselikis, the
executive director of the Maine Lobster Dealers' Association, which
represents companies that buy lobster from Maine fishermen. ``This trade
agreement does give Canada a huge leg up in the European marketplace,''
she said.

Ms. Tselikis said the pact was encouraging American companies to invest
in new facilities in Canada to qualify for the lower European tariff.

``If the argument is you're not going to develop this trade policy
because you're worried about outsourcing jobs --- well, here we are,
potentially outsourcing jobs due to an absence of trade policy,'' she
said.

Gidney Fisheries, which exports live and frozen lobster, is poised to
take advantage of the changing terms of trade. Last year, in
anticipation of increased demand, the factory invested in
state-of-the-art technology to set itself apart.

The company imported a German machine, sometimes used to make
cold-pressed juice, that creates pressure of up to 87,000 pounds per
square inch. The machine compresses the lobster in its shell, breaking
the connective tissue, killing the lobster in seconds and allowing the
meat to be extracted entirely raw --- a selling point for chefs and
consumers, as the process is considered relatively humane.

Image

``A decade or two ago, there would be very few players who would have
been shipping internationally,'' said Robert MacDonald, the president of
Gidney Fisheries. ``We now ship live lobsters all over the
world.''Credit...Stephanie Foden for The New York Times

Image

Refrigerated packages at Gidney Fisheries. Europe could soon be the
factory's fastest-growing market.Credit...Stephanie Foden for The New
York Times

On the factory floor in September, a worker in a gray smock covered by a
shiny rubber apron loaded lobsters into a plastic tube to feed into the
machine. A dozen workers smashed claws, used tiny air hoses to remove
entrails and sorted peachy-pink lobster meat into various packages to be
flash frozen.

``A decade or two ago, there would be very few players who would have
been shipping internationally,'' Mr. MacDonald said. ``We now ship live
lobsters all over the world.''

Once mostly confined to the plates of the rich, lobster has gone mass
market. A glut in the global catch roughly five years ago --- the
product of overfishing cod, a natural lobster predator --- caused the
price to plummet. Lobster rolls and lobster mac and cheese suddenly
appeared on menus of fast food chains like Pret a Manger, Au Bon Pain,
Quiznos and even McDonald's locations in New England.

Better packaging and faster freight services allowed American and
Canadian exporters to expand into Europe and Asia. Exporters found a
promising new market in China, where newly affluent diners were eagerly
adopting Western luxury products like wine, caviar and lobster as a
marker of taste and distinction.

Gidney Fisheries sought to tap into that market, with the help of Duan
Zeng, Mr. MacDonald's colleague. Ms. Zeng, who has a master's degree in
fish biology, does much of her work on WeChat, a Chinese mobile app,
where she sells its wares. Last December, the company teamed up with
Alibaba, the Chinese e-commerce company, to sell premium lobsters
online.

Gidney Fisheries' largest market by far is still the United States,
where the company supplies restaurants and hotel chains with Nova Scotia
lobsters. But given the changing dynamics of trade pacts, Europe could
soon be its fastest-growing market, much to the chagrin of American
lobstermen, who were hopeful the United States would sign its own
agreement with the European Union.

Image

The harbor near the factory, which has teamed up with Alibaba, the
Chinese e-commerce company, to sell premium lobsters
online.Credit...Stephanie Foden for The New York Times

The Trump administration has not said whether it will continue trade
talks with Europe. But other trade pacts under discussion, including
Nafta, have shown little progress. In mid-October, Jyrki Katainen, a
high-ranking European Union official, said Europe was ``negotiating with
all Nafta countries, and with all TPP countries, except one'' --- a
not-so-veiled reference to America.

John Weekes, Canada's Nafta negotiator in the 1990s, said he initially
believed Canadian companies might have just a narrow window of advantage
over their American competitors. Now, that window looks quite a bit
larger.

``It does open up a number of opportunities for Canadians that clearly
aren't going to be available to Americans in the foreseeable future,''
Mr. Weekes said.

Advertisement

\protect\hyperlink{after-bottom}{Continue reading the main story}

\hypertarget{site-index}{%
\subsection{Site Index}\label{site-index}}

\hypertarget{site-information-navigation}{%
\subsection{Site Information
Navigation}\label{site-information-navigation}}

\begin{itemize}
\tightlist
\item
  \href{https://help.nytimes.com/hc/en-us/articles/115014792127-Copyright-notice}{©~2020~The
  New York Times Company}
\end{itemize}

\begin{itemize}
\tightlist
\item
  \href{https://www.nytco.com/}{NYTCo}
\item
  \href{https://help.nytimes.com/hc/en-us/articles/115015385887-Contact-Us}{Contact
  Us}
\item
  \href{https://www.nytco.com/careers/}{Work with us}
\item
  \href{https://nytmediakit.com/}{Advertise}
\item
  \href{http://www.tbrandstudio.com/}{T Brand Studio}
\item
  \href{https://www.nytimes.com/privacy/cookie-policy\#how-do-i-manage-trackers}{Your
  Ad Choices}
\item
  \href{https://www.nytimes.com/privacy}{Privacy}
\item
  \href{https://help.nytimes.com/hc/en-us/articles/115014893428-Terms-of-service}{Terms
  of Service}
\item
  \href{https://help.nytimes.com/hc/en-us/articles/115014893968-Terms-of-sale}{Terms
  of Sale}
\item
  \href{https://spiderbites.nytimes.com}{Site Map}
\item
  \href{https://help.nytimes.com/hc/en-us}{Help}
\item
  \href{https://www.nytimes.com/subscription?campaignId=37WXW}{Subscriptions}
\end{itemize}
