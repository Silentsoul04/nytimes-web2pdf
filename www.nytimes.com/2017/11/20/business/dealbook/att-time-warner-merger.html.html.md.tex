Sections

SEARCH

\protect\hyperlink{site-content}{Skip to
content}\protect\hyperlink{site-index}{Skip to site index}

\href{https://myaccount.nytimes.com/auth/login?response_type=cookie\&client_id=vi}{}

\href{https://www.nytimes.com/section/todayspaper}{Today's Paper}

\href{/section/business/dealbook}{DealBook}\textbar{}Justice Department
Sues to Block AT\&T-Time Warner Merger

\url{https://nyti.ms/2jJLC2B}

\begin{itemize}
\item
\item
\item
\item
\item
\item
\end{itemize}

Advertisement

\protect\hyperlink{after-top}{Continue reading the main story}

Supported by

\protect\hyperlink{after-sponsor}{Continue reading the main story}

DealBook Business and Policy

\hypertarget{justice-department-sues-to-block-att-time-warner-merger}{%
\section{Justice Department Sues to Block AT\&T-Time Warner
Merger}\label{justice-department-sues-to-block-att-time-warner-merger}}

\includegraphics{https://static01.nyt.com/images/2017/11/16/business/16ATT1/merlin_126367256_9dae289a-25bb-4515-b492-6d88e020d906-articleLarge.jpg?quality=75\&auto=webp\&disable=upscale}

By \href{http://www.nytimes.com/by/cecilia-kang}{Cecilia Kang} and
\href{http://www.nytimes.com/by/michael-j-de-la-merced}{Michael J. de la
Merced}

\begin{itemize}
\item
  Nov. 20, 2017
\item
  \begin{itemize}
  \item
  \item
  \item
  \item
  \item
  \item
  \end{itemize}
\end{itemize}

WASHINGTON --- The Justice Department
\href{https://www.nytimes.com/interactive/2017/11/20/business/dealbook/att-time-warner-lawsuit.html}{sued
to block AT\&T's \$85.4 billion bid for Time Warner} on Monday, setting
up a showdown over the first blockbuster acquisition to be considered by
the Trump administration and drawing limits on corporate power in the
fast-evolving media landscape.

By challenging the deal, the Justice Department is taking an approach to
antitrust issues that is starkly different from the Obama
administration's. In 2011, for instance, the department
\href{http://www.nytimes.com/2011/01/19/business/media/19comcast.html}{approved}
a similar deal --- Comcast's acquisition of NBCUniversal --- after
imposing numerous conditions on the transaction.

If AT\&T's bid for Time Warner were to go through, the merger would
create a media and telecommunications behemoth. By itself, AT\&T is one
of the nation's largest internet and telephone providers. With its 2015
acquisition of DirecTV, the country's largest satellite company, it also
became the largest television distributor in the United States.

The combined company would have an unrivaled ability to reach consumers
through news and entertainment programming. Among Time Warner's
properties are HBO, the home to ``Game of Thrones''; Warner Bros., the
studio behind blockbusters like ``Wonder Woman'' and the Harry Potter
film series; and Turner Broadcasting, which includes the news channel
CNN and the sports-heavy TNT network.

Makan Delrahim, the Justice Department's top antitrust regulator, said a
union of the two companies would harm consumers and weaken competition.

``This merger would greatly harm American consumers,'' Mr. Delrahim, the
assistant attorney general for antitrust, said in a statement. ``It
would mean higher monthly television bills and fewer of the new,
emerging innovative options that consumers are beginning to enjoy.''

AT\&T said it would defend the proposed deal in court, arguing that
companies don't directly compete against each other and that the
government hasn't challenged a similar kind of merger in decades.

``It defies logic, and it's unprecedented,'' AT\&T's chief executive,
Randall L. Stephenson, said in a news conference on Monday after the
suit was filed.

The complaint was filed in Federal District Court for the District of
Columbia against AT\&T, DirecTV and Time Warner.

In a call with reporters on Monday, a Justice Department official said
the agency remained open to negotiating a settlement. To gain favor with
the antitrust division, the official said, the companies would have to
sell off some of their assets.

During the news conference, Mr. Stephenson said that the Justice
Department had a long history of approving similar mergers and that the
company was not willing to part with any assets to get the deal
approved.

In its complaint, the Justice Department said consumers would most
likely face higher prices for cable or satellite television
subscriptions because AT\&T would be able to charge more for licensing
of valuable programming like the N.C.A.A. men's basketball tournament,
which is
\href{http://www.ncaa.com/news/basketball-men/article/march-madness-2017-tv-schedule-how-watch-and-live-stream}{broadcast}
in large part on Turner networks.

The government argued further that AT\&T's acquisition of Time Warner
would stifle innovation from online streaming firms like SlingTV, which
competes with AT\&T's DirecTV Now service. AT\&T could withhold programs
from those online providers.

A combined AT\&T and Time Warner, the complaint said, would be more
harmful to consumers and larger in scope than Comcast-NBCUniversal.
Comcast's cable and broadband service reaches more than one-third of the
nation.

``Both AT\&T/DirecTV's video distribution services and Time Warner's TV
networks are available nationwide, so the harm would occur throughout
the country,'' the Justice Department said in its complaint.

Mr. Stephenson has argued that AT\&T needs media content in order to
compete against internet firms like Google and Facebook for digital
advertising dollars. With a big stock of television programming, it
would also compete more effectively for subscribers against companies
like Comcast and Verizon, which both own content.

While speaking with reporters on Monday, Mr. Stephenson obliquely raised
the issue of possible interference by the White House. President Trump,
a frequent critic of news coverage by CNN, said during the 2016
presidential campaign that the deal should be blocked. Mr. Stephenson
called the issue of CNN the ``elephant in the room'' and speculated
about its role in Mr. Delrahim's decision.

``Frankly, I don't know,'' Mr. Stephenson said.

\href{https://www.nytimes.com/2017/11/09/technology/justice-department-antitrust.html}{Mr.
Delrahim has spoken at length} about how Justice Department officials
should handle mergers involving two companies that don't compete against
each other, like the one between AT\&T and Time Warner. Problems with
those mergers have traditionally been resolved by adding conditions
known as consent decrees, which restrict the new company's behavior or
operations.

Mr. Delrahim has argued that those remedies are not effective. Instead,
he has spoken in favor of so-called structural remedies, like forcing a
company to sell assets before a merger or acquisition.

On Sept. 29, he was put in a position to act on that theory, when he was
sworn in at the Justice Department.

Throughout the summer, Justice Department officials and AT\&T lawyers
had discussed conditions that would allay antitrust concerns. During the
talks, AT\&T representatives said the combined company would abstain
from anticompetitive business practices. At least some Justice
Department staff members seemed open to the idea, according to two
people familiar with the government review.

The antitrust staff, which included holdovers from the Obama
administration, presented Mr. Delrahim with three options: Accept the
deal with conditions, accept the deal with divestitures, or block it
altogether.

Five weeks later, it became clear to AT\&T that the deal was in trouble.

An article by
\href{https://www.wsj.com/articles/u-s-weighs-suit-against-at-ts-deal-for-time-warner-1509633797}{The
Wall Street Journal on Nov. 2} reported that the Justice Department was
considering filing a lawsuit against the deal. The next day, according
to a company official, AT\&T asked for a meeting with Mr. Delrahim and
senior staff.

\includegraphics{https://static01.nyt.com/images/2017/11/16/business/16ATT2/merlin_115346318_89894072-ed0f-40d0-9f91-fb42eec53420-articleLarge.jpg?quality=75\&auto=webp\&disable=upscale}

Mr. Stephenson and the company's general counsel flew from Dallas to
Washington a few days later to meet with the officials. After the
meeting, the two sides were far from reaching a deal.

According to AT\&T sources, the Justice Department asked the company to
rid itself of Turner or DirecTV, the satellite TV provider.

Government officials, however, said AT\&T had offered to sell CNN. By
the end of the day's talks, Mr. Stephenson put out a statement saying he
had never offered to sell CNN.

The next day, on Nov. 9, Mr. Stephenson, speaking at The New York
Times's DealBook conference, said the company was prepared to defend
itself in court. He added that a significant merger of companies that
were not direct competitors with each other had not been challenged in
40 years.

In recent weeks, the Justice Department reached out to several state
attorneys general who had been investigating the transaction to see if
they would join the lawsuit as plaintiffs. None of them have yet joined,
leaving the Justice Department as the sole plaintiff.

Whether or not Mr. Trump played a role in the Justice Department's
attempt to block the deal will remain a subject of debate, fanned in
part by the president himself. On Nov. 15, he
\href{https://twitter.com/realdonaldtrump/status/930748627642998784?lang=en}{wrote}
on his @realDonaldTrump Twitter account: ``While in the Philippines I
was forced to watch @CNN, which I have not done in months, and again
realized how bad, and FAKE, it is. Loser!'' The tweet was retweeted on
the official @POTUS account.

In July, Senator Amy Klobuchar of Minnesota, the ranking Democrat on the
Senate Judiciary antitrust subcommittee, wrote a letter to Attorney
General Jeff Sessions that asked, ``Has any employee of the White House
or adviser to the president (either official or unofficial) had any
contact with any Department of Justice employee regarding the AT\&T/Time
Warner transaction?''

Mr. Delrahim has strongly denied there has been interference by the
White House or Mr. Trump.

Legal experts were divided on the merits of the suit.

Gene Kimmelman, the president of Public Knowledge, a nonprofit consumer
advocacy group and a former senior antitrust official at the Justice
Department, said, ``We believe the Justice Department has presented a
persuasive case that should satisfy any federal judge that this
transaction is illegal and should be blocked, regardless of any politics
surrounding the deal.''

Ryan Radia, a legal and regulatory expert at the Competitive Enterprise
Institute, took a different view. ``The AT\&T-Time Warner merger is a
vertical transaction that wouldn't reduce competition in any distinct
market,'' Mr. Radia said. ``Under established antitrust principles, the
government will have a difficult time showing a court that the deal is
likely to harm consumers.''

Anxiety over the merger has simmered in the CNN newsroom for months,
with staff members speculating about the role of the president's
animosity toward the network. On Monday, CNN anchors reminded viewers
that Mr. Trump had once tweeted a video that portrayed him body-slamming
a wrestler with a CNN logo for a head.

AT\&T's general counsel, David R. McAtee, said in a statement on Monday:
``Today's D.O.J. lawsuit is a radical and inexplicable departure from
decades of antitrust precedent. We are confident that the court will
reject the government's claims and permit this merger under longstanding
legal precedent.''

Advertisement

\protect\hyperlink{after-bottom}{Continue reading the main story}

\hypertarget{site-index}{%
\subsection{Site Index}\label{site-index}}

\hypertarget{site-information-navigation}{%
\subsection{Site Information
Navigation}\label{site-information-navigation}}

\begin{itemize}
\tightlist
\item
  \href{https://help.nytimes.com/hc/en-us/articles/115014792127-Copyright-notice}{©~2020~The
  New York Times Company}
\end{itemize}

\begin{itemize}
\tightlist
\item
  \href{https://www.nytco.com/}{NYTCo}
\item
  \href{https://help.nytimes.com/hc/en-us/articles/115015385887-Contact-Us}{Contact
  Us}
\item
  \href{https://www.nytco.com/careers/}{Work with us}
\item
  \href{https://nytmediakit.com/}{Advertise}
\item
  \href{http://www.tbrandstudio.com/}{T Brand Studio}
\item
  \href{https://www.nytimes.com/privacy/cookie-policy\#how-do-i-manage-trackers}{Your
  Ad Choices}
\item
  \href{https://www.nytimes.com/privacy}{Privacy}
\item
  \href{https://help.nytimes.com/hc/en-us/articles/115014893428-Terms-of-service}{Terms
  of Service}
\item
  \href{https://help.nytimes.com/hc/en-us/articles/115014893968-Terms-of-sale}{Terms
  of Sale}
\item
  \href{https://spiderbites.nytimes.com}{Site Map}
\item
  \href{https://help.nytimes.com/hc/en-us}{Help}
\item
  \href{https://www.nytimes.com/subscription?campaignId=37WXW}{Subscriptions}
\end{itemize}
