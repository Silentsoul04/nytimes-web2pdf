Sections

SEARCH

\protect\hyperlink{site-content}{Skip to
content}\protect\hyperlink{site-index}{Skip to site index}

\href{https://www.nytimes.com/section/books}{Books}

\href{https://myaccount.nytimes.com/auth/login?response_type=cookie\&client_id=vi}{}

\href{https://www.nytimes.com/section/todayspaper}{Today's Paper}

\href{/section/books}{Books}\textbar{}Will Democracy Survive President
Trump? Two New Books Aren't So Sure

\url{https://nyti.ms/2FozwSA}

\begin{itemize}
\item
\item
\item
\item
\item
\end{itemize}

Advertisement

\protect\hyperlink{after-top}{Continue reading the main story}

Supported by

\protect\hyperlink{after-sponsor}{Continue reading the main story}

\href{/column/books-of-the-times}{Books of The Times}

\hypertarget{will-democracy-survive-president-trump-two-new-books-arent-so-sure}{%
\section{Will Democracy Survive President Trump? Two New Books Aren't So
Sure}\label{will-democracy-survive-president-trump-two-new-books-arent-so-sure}}

\includegraphics{https://static01.nyt.com/images/2018/01/11/arts/11bookfrum-levitsky/11bookfrum-levitsky-articleLarge.jpg?quality=75\&auto=webp\&disable=upscale}

By Jennifer Szalai

\begin{itemize}
\item
  Jan. 10, 2018
\item
  \begin{itemize}
  \item
  \item
  \item
  \item
  \item
  \end{itemize}
\end{itemize}

Barely 12 months into the churn of the Trump presidency, when a week can
be as eventful as a year, David Frum has published his new book aware of
the perils he faces. ``My choice of timing imposes on this project many
risks of error and misunderstanding,'' he concedes on the first page of
``Trumpocracy: The Corruption of the American Republic.'' There's still
plenty of time until the next presidential election for Trump and his
sentries to do more: more fulminating, more tweeting, more
back-stabbing, more enabling. ``But if it's potentially embarrassing to
speak too soon,'' Frum writes, ``it can also be dangerous to wait too
long.''

If this sounds a bit much as justification for a release date, it's
worth remembering that the neoconservative Frum, once a speechwriter for
George W. Bush, was an early and ardent supporter of the Iraq war.
Neocons don't like to wait.

But Frum purports to offer more than a rushed assessment of the last
year. After all, he says, President Trump is not a cause but a symptom.
Like another new book, Steven Levitsky and Daniel Ziblatt's ``How
Democracies Die,'' ``Trumpocracy'' is, in part, an exploration of the
reasons for the president's electoral upset and the roots of his rule.
Trump, Frum writes, ``did not create the vulnerabilities he exploited.''

Despite sterling conservative credentials --- helping Bush coin the
phrase
``\href{http://www.nytimes.com/2003/01/20/us/white-house-letter-axis-of-evil-first-birthday-for-a-famous-phrase.html}{axis
of evil}''; co-writing
``\href{http://www.nytimes.com/2004/02/08/books/showing-them-who-s-boss.html}{An
End to Evil}'' with Richard Perle (evil being even more loathed than
waiting) --- Frum was developing a reputation as a Republican dissenter
before the election of 2016 put him in full Never Trump mode, when he
held his nose and voted for Hillary Clinton. (This was a real sacrifice,
considering he deemed Clinton
\href{https://www.theatlantic.com/politics/archive/2016/11/dont-gamble-on-trump/506207/}{``a
suspicious and vindictive personality.''}) As a columnist at The
Atlantic, Frum has spent the past year excoriating the president and the
Republican Party as a toxic mix of hot heads in the White House and lily
livers in Congress.

Image

David FrumCredit...Michael Bennett Kress

``The worse Trump behaved, the more frantically congressional
Republicans worked to protect him,'' he writes in ``Trumpocracy.'' The
government has ``imported the spirit of thuggery, crookedness and
dictatorship into the very core of the American state.'' While the
populace (or audience) gets swept along by the daily gusts of gossip and
palace intrigues, Frum wants to direct our ire at all those
mild-mannered functionaries who have allowed ``this new regime of deceit
and brutishness'' to take hold.

Among Frum's fellow Republicans who read this book, all but the most
determined Trump enthusiasts should feel pin pricks of recognition and,
depending on how much hypocrisy they can live with, a queasy discomfort.
Frum relishes going on the attack, and he castigates members of a
Republican establishment who have laid any pretensions to moral
rectitude on the altar of a tax cut.

The book seems to have been written in haste, a patchwork of bits and
pieces from his Atlantic columns, additional examples of Trumpian
malfeasance, and new ways of expressing old outrage. Frum has followed
the blogger's template, clogging his text with block quotes. Rambling
disquisitions by Steve Bannon and Trump prove little besides the
exceedingly hard time they have getting to a point. Paragraphs of wire
stories are reprinted verbatim. Frum's prescriptions for responding to
Trump amount to cut-rate self-help: ``Resistance begins by refusing to
let him corrupt you personally.''

Image

Credit...Alessandra Montalto/The New York Times

Frum has the pamphleteer's flair for the scathing epithet, which can be
energizing or enervating, depending on your tolerance for hyperbole.
Even sympathetic readers may feel besieged when he works himself up to
full throttle. ``No single person could possibly plumb the foulnesses of
the Trump presidency.'' ``The Trump White House is a mess of careless
slobs.'' Trump himself is ``the most shameless liar in the history of
the presidency.''

On that last note, Levitsky and Ziblatt might not disagree, though in
``How Democracies Die'' they are more methodical and less fervid in
their assessments. The most withering designation they offer for
President Trump is --- get ready for it --- ``serial norm breaker.''

Then again, Levitsky and Ziblatt are political scientists, for whom
being a serial norm breaker is serious stuff indeed. Norms are what have
sustained American democracy ``in ways we have come to take for
granted.'' They identify two in particular: ``mutual toleration,'' or
the understanding among competing parties and politicians that they are
legitimate rivals rather than existential enemies; and ``forbearance,''
or the understanding among politicians that just because they
technically have the power to do something doesn't mean they ought to
use it. The erosion of these two norms can lead to a partisan death
spiral. The authors argue that Trump has tried to eviscerate both.

``How Democracies Die'' is a lucid and essential guide to
\href{https://www.nytimes.com/2017/01/17/books/review/classic-novel-that-predicted-trump-sinclar-lewis-it-cant-happen-here.html}{what
can happen here}. Levitsky and Ziblatt show how democracies have
collapsed elsewhere --- not just through violent coups, but more
commonly (and insidiously) through a gradual slide into
authoritarianism. Autocrats often come to power through democratic
elections rather than at the point of a gun. Establishment elites can
inadvertently assist would-be despots, as insiders delude themselves
into believing they can invite an outsider into power and then pull the
puppet strings. As one German aristocrat boasted in 1933: ``Within two
months, we will have pushed Hitler so far into a corner that he'll
squeal.''

Image

Steven Levitsky, left, and Daniel Ziblatt.Credit...Stephanie Mitchell

A ``profound miscalculation,'' is how Levitsky and Ziblatt describe this
attempt to share power with the Nazis. When presenting the most
distressing historical analogies, the authors' understatement is so
subdued it verges on deadpan. But our current moment is so fraught that
``How Democracies Die'' is never dull, even if the writing can be. ``If
partisan animosity prevails over mutual toleration, those in control of
congress may prioritize defense of the president over the performance of
their constitutional duties.'' This might be blanched prose, but it also
sounds like a sly subtweet of the Republican Party.

In one of the most original turns in the book, Levitsky and Ziblatt
assiduously dismantle the myth of American exceptionalism. Even during
the supposed heyday of 20th-century bipartisan cooperation, ``the norms
sustaining our political system rested, to a considerable degree, on
racial exclusion.'' Jim Crow was allowed to flourish in a South that was
``profoundly undemocratic.'' The post-Confederate states had changed
their constitutions and laws to deprive African-Americans of the vote,
while Democrats and Republicans found common cause in a political system
that was largely restricted to white people.

The authors hazard that most of the norm-breaking in the last few
decades has been conducted by the Republican Party because, unlike its
rival, it ``has remained culturally homogeneous.'' The Democrats have
had to negotiate among varying interests in their ranks; the Republicans
have not, allowing them to be more single-minded --- and reckless, this
book suggests --- in their pursuit of power. But Levitsky and Ziblatt
oppose
\href{https://www.nytimes.com/2016/11/20/opinion/sunday/the-end-of-identity-liberalism.html}{those
liberals} who advise
\href{https://www.nytimes.com/2017/07/06/opinion/center-democrats-identity-politics.html}{compromising
the concerns of ethnic minorities} in order to make Democrats more
appealing to Trump's white working-class base: ``It would repeat some of
our country's most shameful mistakes.''

Each of these books ends on a characteristic note. Frum has prepared us
for a grand finale, full of clashing cymbals and swelling violins: ``We
are living through the most dangerous challenge to the free government
of the United States that anyone alive has encountered. What happens
next is up to you. Don't be afraid. This moment of danger can also be
your finest hour as a citizen and an American.'' This kind of
exhortation is as vague and bombastic as old calls for regime change in
Iraq. Levitsky and Ziblatt are drier and more circumspect. There is no
democratic paradise, no easy way out. Democracy, when it functions
properly, is hard, grinding work. This message may not be as loud and as
lurid as what passes for politics these days, but it might be the one we
need to hear.

Advertisement

\protect\hyperlink{after-bottom}{Continue reading the main story}

\hypertarget{site-index}{%
\subsection{Site Index}\label{site-index}}

\hypertarget{site-information-navigation}{%
\subsection{Site Information
Navigation}\label{site-information-navigation}}

\begin{itemize}
\tightlist
\item
  \href{https://help.nytimes.com/hc/en-us/articles/115014792127-Copyright-notice}{©~2020~The
  New York Times Company}
\end{itemize}

\begin{itemize}
\tightlist
\item
  \href{https://www.nytco.com/}{NYTCo}
\item
  \href{https://help.nytimes.com/hc/en-us/articles/115015385887-Contact-Us}{Contact
  Us}
\item
  \href{https://www.nytco.com/careers/}{Work with us}
\item
  \href{https://nytmediakit.com/}{Advertise}
\item
  \href{http://www.tbrandstudio.com/}{T Brand Studio}
\item
  \href{https://www.nytimes.com/privacy/cookie-policy\#how-do-i-manage-trackers}{Your
  Ad Choices}
\item
  \href{https://www.nytimes.com/privacy}{Privacy}
\item
  \href{https://help.nytimes.com/hc/en-us/articles/115014893428-Terms-of-service}{Terms
  of Service}
\item
  \href{https://help.nytimes.com/hc/en-us/articles/115014893968-Terms-of-sale}{Terms
  of Sale}
\item
  \href{https://spiderbites.nytimes.com}{Site Map}
\item
  \href{https://help.nytimes.com/hc/en-us}{Help}
\item
  \href{https://www.nytimes.com/subscription?campaignId=37WXW}{Subscriptions}
\end{itemize}
