The Mystery of the Exiled Billionaire Whistle-Blower

\url{https://nyti.ms/2Es4yrE}

\begin{itemize}
\item
\item
\item
\item
\item
\item
\end{itemize}

\includegraphics{https://static01.nyt.com/images/2018/01/14/magazine/14mag-wengui/14mag-wengui-articleLarge.jpg?quality=75\&auto=webp\&disable=upscale}

Sections

\protect\hyperlink{site-content}{Skip to
content}\protect\hyperlink{site-index}{Skip to site index}

Feature

\hypertarget{the-mystery-of-the-exiled-billionaire-whistle-blower}{%
\section{The Mystery of the Exiled Billionaire
Whistle-Blower}\label{the-mystery-of-the-exiled-billionaire-whistle-blower}}

From a penthouse on Central Park, Guo Wengui has exposed a phenomenal
web of corruption in China's ruling elite --- if, that is, he's telling
the truth.

Credit...Sasha Rudensky for The New York Times

Supported by

\protect\hyperlink{after-sponsor}{Continue reading the main story}

By Lauren Hilgers

\begin{itemize}
\item
  Jan. 10, 2018
\item
  \begin{itemize}
  \item
  \item
  \item
  \item
  \item
  \item
  \end{itemize}
\end{itemize}

\href{https://cn.nytimes.com/china/20180110/the-world-according-to-guo/}{阅读简体中文版}\href{https://cn.nytimes.com/china/20180110/the-world-according-to-guo/zh-hant/}{閱讀繁體中文版}

On a recent Saturday afternoon, an exiled Chinese billionaire named Guo
Wengui was holding forth in his New York apartment, sipping tea while an
assistant lingered quietly just outside the door, slipping in
occasionally to keep Guo's glass cup perfectly full. The tycoon's
Twitter account had been suspended again --- it was the fifth or sixth
time, by Guo's count --- and he blamed the Communist Party of China.
``It's not normal!'' he said, about this cycle of blocking and
reinstating. ``But it doesn't matter. I don't need anyone.''

Guo's New York apartment is a 9,000-square-foot residence along Central
Park that he bought for \$67.5 million in 2015. He sat in a
Victorian-style chair, his back to a pair of west-facing windows, the
sunset casting craggy shadows. A black-and-white painting of an
angry-looking monkey hung on the wall to Guo's right, a hat bearing a
star-and-wreath Soviet insignia on its head and a cigarette hanging from
its lips. Guo had arrived dressed entirely in black, except for two
silver stripes on each lapel. ``I have the best houses,'' he told me.
Guo had picked his apartment for its location, its three sprawling
balconies and the meticulously tiled floor in the entryway. He has the
best apartment in London, he said; the biggest apartment in Hong Kong.
His yacht is docked along the Hudson River. He is comfortable and,
anyway, Guo likes to say that as a Buddhist, he wants for nothing. If it
were down to his own needs alone, he would have kept his profile low.
But he has a higher purpose. He is going to save China.

Guo pitches himself as a former insider, a man who knows the secrets of
a government that tightly controls the flow of information. A man who,
in 2017, did the unthinkable --- tearing open the veil of secrecy that
has long surrounded China's political elite, lobbing accusations about
corruption, extramarital affairs and murder plots over Facebook and
Twitter. His
\href{https://www.youtube.com/channel/UCO3pO3ykAUybrjv3RBbXEHw/videos}{YouTube
videos} and tweets have drawn in farmers and shopkeepers, democracy
activists, writers and businesspeople. In China, people have been
arrested for chatting about Guo online and distributing T-shirts with
one of his slogans printed on the front (``This is only the
beginning!''). In New York, Guo has split a community of dissidents and
democracy activists down the middle. Some support him. Others believe
that Guo himself is a government spy.

Nothing in Guo's story is as straightforward as he would like it to
seem. Guo is 47 years old, or 48, or 49. Although he has captured the
attention of publications like The Guardian, The New York Times and The
Wall Street Journal, the articles that have run about him have offered
only hazy details about his life. This is because his biography varies
so widely from one source to the next. Maybe his name isn't even Guo
Wengui. It could be Guo Wugui. There are reports that in Hong Kong, Guo
occasionally goes by the name Guo Haoyun.

When pressed, Guo claims a record of unblemished integrity in his
business dealings, both in real estate and in finance (when it comes to
his personal life, he strikes a more careful balance between virility
and dedication to his family). ``I never took a square of land from the
government,'' he said. ``I didn't take a penny of investment from the
banks.'' If you accept favors, he said, people will try to exploit your
weaknesses. So, Guo claims, he opted to take no money and have no
weaknesses.

Yet when Guo left China in 2014, he fled in anticipation of corruption
charges. A former business partner had been detained just days before,
and his political patron would be detained a few days afterward. In
2015, articles about corruption in Guo's business dealings --- stories
that he claims are largely fabrications --- started appearing in the
media. He was accused of defrauding business partners and colluding with
corrupt officials. To hear Guo tell it, his political and business
opponents used a national corruption campaign as a cover for a personal
vendetta.

Whatever prompted Guo to take action, his campaign came during an
important year for China's president, Xi Jinping. In October, the
Communist Party of China (C.P.C.) convened its 19th National Congress, a
twice-a-decade event that sets the contours of political power for the
next five years. The country is in the throes of a far-reaching
anti-corruption campaign, and Xi has overseen a crackdown on dissidents
and human rights activists while increasing investment in censorship and
surveillance. Guo has become a thorn in China's side at the precise
moment the country is working to expand its influence, and its
censorship program, overseas.

In November 2017, the Tiananmen Square activist Wang Dan warned of the
growing influence of the C.P.C. on university campuses in the United
States. His own attempts to hold ``China salons'' on college campuses
had largely been blocked by the Chinese Students and Scholars
Association --- a group with ties to China's government. Around the same
time, the academic publisher Springer Nature agreed to block access to
hundreds of articles on its Chinese site, cutting off access to articles
on Tibet, Taiwan and China's political elite. Reports emerged last year
that China is spending hundreds of thousands of dollars quarterly to
purchase ads on Facebook (a service that is blocked within China's
borders). In Australia, concerns about China's growing influence led to
a ban on foreign political donations.

``That's why I'm telling the United States they should really be
careful,'' Guo said. China's influence is spreading, he says, and he
believes his own efforts to change China will have global consequences.
``Like in an American movie,'' he told me with unflinching
self-confidence. ``In the last minutes, we will save the world.''

\textbf{Propaganda, censorship} and rewritten histories have long been
specialties of authoritarian nations. The aim, as famously explained by
the political philosopher Hannah Arendt, is to confuse: to breed a
combination of cynicism and gullibility. Propaganda can leave people in
doubt of all news sources, suspicious of their neighbors, picking and
choosing at random what pieces of information to believe. Without a
political reality grounded in facts, people are left unmoored, building
their world on whatever foundation --- imaginary or otherwise --- they
might choose.

The tight grip that the C.P.C. keeps on information may be nothing new,
but China's leadership has been working hard to update the way it
censors and broadcasts. People in China distrusted print and television
media long before U.S. politicians started throwing around accusations
of ``fake news.'' In 2016, President Xi Jinping was explicit about the
arrangement, informing the country's media that it should be ``surnamed
Party.'' Likewise, while the West has only recently begun to grapple
with government-sponsored commenters on social media, China's government
has been manipulating online conversations for over a decade.

``They create all kinds of confusion,'' said Ha Jin, the National Book
Award-winning American novelist born in China's Liaoning Province, and a
vocal supporter of Guo. ``You don't know what information you have and
whether it's right. You don't know who are the informers, who are the
agents.''

Online, the C.P.C. controls information by blocking websites, monitoring
content and employing an army of commenters widely known as the 50-cent
party. The name was used as early as 2004, when a municipal government
in Hunan Province hired a number of online commenters, offering a
stipend of 600 yuan, or about \$72. Since then, the 50-cent party has
spread. In 2016, researchers from Harvard, Stanford and the University
of California-San Diego estimated that these paid commenters generated
448 million social-media comments annually. The posts, researchers
found, were conflict averse, cheerleading for the party rather than
defending it. Their aim seemed not to be engaging in argument but rather
distracting the public and redirecting attention from sensitive issues.

In early 2017, Guo issued his first salvos against China's ruling elite
through more traditional channels. He contacted a handful of
Chinese-language media outlets based in the United States. He gave
interviews to the Long Island-based publication Mingjing News and to
Voice of America --- a live event that was cut short by producers,
leading to speculation that V.O.A. had caved to Chinese government
pressure. He called The New York Times and spoke with reporters at The
Wall Street Journal. It did not take long, however, before the
billionaire turned to direct appeals through social media. The
accusations he made were explosive --- he attacked Wang Qishan, Xi
Jinping's corruption czar, and Meng Jianzhu, the secretary of the
Central Political and Legal Affairs Commission, another prominent player
in Xi's anti-corruption campaign. He talked about Wang's mistresses, his
business interests and conflicts within the party.

In one YouTube video, released on Aug. 4, Guo addressed the tension
between Wang and another anti-corruption official named Zhang Huawei. He
recounted having dinner with Zhang when ``he called Wang Qishan's
secretary and gave him orders,'' Guo said. ``Think about what Wang had
to suffer in silence back then. They slept with the same women, and
Zhang knew everything about Wang.'' In addition, Guo said, Zhang knew
about Wang's corrupt business dealings. When Zhang Huawei was placed
under official investigation in April, Guo claimed, it was a result of a
grudge.

``Everyone in China is a slave,'' Guo said in the video. ``With the
exception of the nobility.''

To those who believe Guo's claims, they expose a depth of corruption
that would surprise even the most jaded opponent of the C.P.C. ``The
corruption is on such a scale,'' Ha Jin said. ``Who could imagine that
the czar of anti-corruption would himself be corrupt? It is
extraordinary.''

Retaliation came quickly. A barrage of counteraccusations began pouring
out against Guo, most published in the pages of the state-run Chinese
media. Warrants for his arrest were issued on charges of corruption,
bribery and even rape. China asked Interpol to issue a red notice
calling for Guo's arrest and extradition. He was running out of money,
it was reported. In September, Guo recorded a video during which he
received what he said was a phone call from his fifth brother: Two of
Guo's former employees had been detained, and their family members were
threatening suicide. ``My Twitter followers are so important they are
like heaven to me,''
\href{https://www.youtube.com/watch?v=aGjBsAB46fI\&t=887s}{Guo said}.
But, he declared, he could not ignore the well-being of his family and
his employees. ``I cannot finish the show as I had planned,'' he said.
Later, Guo told his followers in a video that he was planning to divorce
his wife, in order to shield her from the backlash against him.

Guo quickly resumed posting videos and encouraging his followers. His
accusations continued to accumulate throughout 2017, and he recently
started his own YouTube channel (and has yet to divorce his wife). His
YouTube videos are released according to no particular schedule,
sometimes several days in a row, some weeks not at all. He has developed
a casual, talkative style. In some, Guo is running on a treadmill or
still sweating after a workout. He has demonstrated cooking techniques
and played with a tiny, fluffy dog, a gift from his daughter. He invites
his viewers into a world of luxury and offers them a mix of secrets,
gossip and insider knowledge.

Wang Qishan, Guo has claimed, is hiding the money he secretly earned in
the Hainan-based conglomerate HNA Group, a company with an estimated
\$35 billion worth of investments in the United States. (HNA Group
denies any ties to Wang and is suing Guo.) He accused Wang of carrying
on an affair with the actress Fan Bingbing. (Fan is reportedly suing Guo
for defamation.) He told stories of petty arguments among officials and
claimed that Chinese officials sabotaged Malaysia Airlines Flight 370,
which disappeared in 2014 en route to Beijing, in order to cover up an
organ-harvesting scheme. Most of Guo's accusations have proved nearly
impossible to verify.

``This guy is just covered in question marks,'' said Minxin Pei, a
professor at Claremont McKenna who specializes in Chinese governance.

The questions that cover Guo have posed a problem for both the United
States government and the Western journalists who, in trying to write
about him, have found themselves buffeted by the currents of propaganda,
misinformation and the tight-lipped code of the C.P.C. elite. His claims
have also divided a group of exiled dissidents and democracy activists
--- people who might seem like Guo's natural allies. For the most part,
the democracy activists who flee China have been chased from their
country for protesting the government or promoting human rights, not
because of corruption charges. They tell stories of personal
persecution, not insider tales of bribery, sex and money. And perhaps as
a consequence, few exiled activists command as large an audience as Guo.
``I will believe him,'' Ha Jin said, ``until one of his serious
accusations is proved to be false.''

Pei, the professor, warns not to take any of Guo's accusations at face
value. The reaction from the C.P.C. has been so extreme, however, that
Pei believes Guo must know something. ``He must mean something to the
government,'' he said. ``They must be really bothered by this
billionaire.'' In May,
\href{https://www.wsj.com/articles/chinas-hunt-for-guo-wengui-a-fugitive-businessman-kicks-off-manhattan-caper-worthy-of-spy-thriller-1508717977}{Chinese
officials visited Guo} on visas that did not allow them to conduct
official business, causing a confrontation with the F.B.I. A few weeks
later,
\href{https://www.washingtontimes.com/news/2017/oct/25/jeff-sessions-guo-wengui-deportation-to-china-woul/}{according
to The Washington Times}, China's calls for Guo's extradition led to a
White House showdown, during which Jeff Sessions threatened to resign if
Guo was sent back to China.

Guo has a history of cultivating relationships with the politically
influential, and the trend has continued in New York. He famously bought
5,000 copies of a book by Cherie Blair, Tony Blair's wife. (``It was to
give to my employees,'' Guo told me. ``I often gave my employees books
to read.'') Guo has also cultivated a special relationship with Steve
Bannon, whom he says he has met with a handful of times, although the
two have no financial relationship. Not long after one of their
meetings, Bannon
\href{http://www.breitbart.com/radio/2017/11/07/bannon-china-is-an-enemy-of-incalculable-power-not-a-strategic-partner-and-we-have-to-understand-that/}{appeared
on Breitbart Radio} and called China ``an enemy of incalculable power.''

Despite Guo's high-powered supporters and his army of online followers,
one important mark of believability has continued to elude him. Western
news organizations have struggled to find evidence that would
corroborate Guo's claims. When his claims appear in print, they are
carefully hedged --- delivered with none of his signature charm and
bombast. ``Why do you need more evidence?'' Guo complained in his
apartment. ``I can give them evidence, no problem. But while they're out
spending time investigating, I'm waiting around to get killed!''

\textbf{The details of} Guo's life may be impossible to verify, but the
broad strokes confirm a picture of a man whose fortunes have risen and
fallen with the political climate in China. To hear Guo tell it, he was
born in Jilin Province, in a mining town where his parents were sent
during the Cultural Revolution. ``There were foreigners there,'' Guo
says in a video recorded on what he claims is his birthday. (Guo was
born on Feb. 2, or May 10, or sometime in June.) ``They had the most
advanced machinery. People wore popular clothing.'' Guo, as a result,
was not ignorant of the world. He was, however, extremely poor.
``Sometimes we didn't even have firewood,'' he says. ``So we burned the
wet twigs from the mountains --- the smoke was so thick.'' Guo
emphasizes this history: He came from hardship. He pulled himself up.

The story continues into Guo's pre-teenage years, when he moved back to
his hometown in Shandong Province. He met his wife and married her when
he was only 15, she 14. They moved to Heilongjiang, where they started a
small manufacturing operation, taking advantage of the early days of
China's economic rise, and then to Henan. Guo got his start in real
estate in a city called Zhengzhou, where he founded the Zhengzhou Yuda
Property Company and built the tallest building the city had seen so
far, the Yuda International Trade Center. According to Guo, he was only
25 when he made this first deal.

The string of businesses and properties that Guo developed provide some
of the confirmable scaffolding of his life. No one disputes that Guo
went on to start both the Beijing Morgan Investment Company and Beijing
Zenith Holdings. Morgan Investment was responsible for building a
cluster of office towers called the Pangu Plaza, the tallest of which
has a wavy top that loosely resembles a dragon, or perhaps a precarious
cone of soft-serve ice cream. Guo is in agreement with the Chinese media
that in buying the property for Pangu Plaza, he clashed with the deputy
mayor of Beijing. The dispute ended when Guo turned in a lengthy sex
tape capturing the deputy mayor in bed with his mistress.

There are other details in Guo's biography, however, that vary from one
source to the next. Guo says that he never took government loans;
Caixin, a Beijing-based publication, quoted ``sources close to the
matter'' in
\href{https://www.caixinglobal.com/2015-03-26/101012577.html}{a 2015
article} claiming that Guo took out 28 loans totaling 588 million yuan,
or about \$89 million. Guo, according to Caixin, eventually defaulted.
At some point in this story --- the timeline varies --- Guo became
friends with the vice minister of China's Ministry of State Security, Ma
Jian. The M.S.S. is China's answer to the C.I.A. and the F.B.I.
combined. It spies on civilians and foreigners alike, conducting
operations domestically and internationally, amassing information on
diplomats, businessmen and even the members of the C.P.C. Describing Ma,
Guo leans back in his chair and mimes smoking a cigarette. ``Ma Jian! He
was fat and his skin was tan.'' According to Guo, Ma sat like this
during their first meeting, listening to Guo's side of a dispute. Then
Ma told him to trust the country. ``Trust the law,'' he told Guo. ``We
will treat you fairly.'' The older master of spycraft and the young
businessman struck up a friendship that would become a cornerstone in
Guo's claims of insider knowledge, and also possibly the reason for the
businessman's downfall in China.

Following the construction of Pangu Plaza in Beijing, Guo's life story
becomes increasingly hard to parse. He started a securities business
with a man named Li You. After a falling-out, Li was detained by the
authorities. Guo's company accused Li and his company of insider
trading. According to the 2015 article in Caixin, Li then penned a
letter to the authorities accusing Guo of ``wrongdoing.''

As this dispute was going on, China's anti-​corruption operation was
building a case against Ma Jian. In Guo's telling, Ma had long been
rumored to be collecting intelligence on China's leaders. As the
anti-corruption campaign gained speed and officials like Wang Qishan
gained power, Ma's well of intelligence started to look like a threat.
It was Guo's relationship with Ma, the tycoon maintains, that made
officials nervous. Ma was detained by the authorities in January 2015,
shortly after Guo fled the country. Soon after Ma's detention, accounts
began appearing in China's state-run media claiming that Ma had six
Beijing villas, six mistresses and at least two illegitimate sons. In a
2015 article that ran in the party-run newspaper The China Daily, the
writer added another detail: ``The investigation also found that Ma had
acted as an umbrella for the business ventures of Guo Wengui, a tycoon
from Henan Province.''

In the mix of spies, corrupt business dealings, mistresses and sex
scandals, Guo has one more unbelievable story to tell about his past. It
is one reason, he says, that he was mentally prepared to confront the
leaders of the Communist Party. It happened nearly 29 years ago, in the
aftermath of the crackdown on Tiananmen Square. According to Guo, he had
donated money to the students protesting in the square, and so a group
of local police officers came to find him at his home. An overzealous
officer fired off a shot at Guo's wife --- at which point Guo's younger
brother jumped in front of the bullet, suffering a fatal wound. ``That
was when I started my plan,'' he said. ``If your brother had been killed
in front of your eyes, would you just forget it?'' Never mind the fact
that it would take 28 years for him to take any public stand against the
party that caused his brother's death. Never mind that the leadership
had changed. ``I'm not saying everyone in the Communist Party is bad,''
he said. ``The system is bad. So what I need to oppose is the system.''

\textbf{On an unusually} warm Saturday afternoon in Flushing, Queens, a
group of around 30 of Guo's supporters gathered for a barbecue in
Kissena Park. They laid out a spread of vegetables and skewers of shrimp
and squid. Some children toddled through the crowd, chewing on hot dogs
and rolling around an unopened can of Coke. The adults fussed with a
loudspeaker and a banner that featured the name that Guo goes by in
English, Miles Kwok. ``Miles Kwok, NY loves U,'' it said, a heart
standing in for the word ``loves.'' ``Democracy, Justice, Liberty for
China.'' Someone else had carried in a life-size cutout of the
billionaire.

The revelers decided to hold the event in the park partly for the
available grills but also partly because the square in front of Guo's
penthouse had turned dangerous. A few weeks earlier, some older women
had been out supporting Guo when a group of Chinese men holding flags
and banners showed up. At one point, the men wrapped the women in a
protest banner and hit them. The park was a safer option. And the
protesters had learned from Guo --- it wasn't a live audience they were
hoping for. The group would be filming the protest and posting it on
social media. Halfway through, Guo would call in on someone's cellphone,
and the crowd would cheer.

Despite this show of support, Guo's claims have divided China's exiled
dissidents to such an extent that on a single day near the end of
September, two dueling meetings of pro-democracy activists were held in
New York, one supporting Guo, the other casting doubt on his
motivations. (``They are jealous of me,'' Guo said of his detractors.
``They think: Why is he so handsome? Why are so many people listening to
him?'') Some of Guo's claims are verifiably untrue --- he claimed in an
interview with Vice that he paid \$82 million for his apartment --- and
others seem comically aggrandized. (Guo says he never wears the same
pair of underwear twice.) But the repercussions he is facing are real.

In December, Guo's brother was sentenced to three years and six months
in prison for destroying accounting records. The lawsuits filed against
Guo for defamation are piling up, and Guo has claimed to be amassing a
``war chest'' of \$150 million to cover his legal expenses. In
September, a new set of claims against Guo were made in a 49-page
document circulated by a former business rival. For Ha Jin, Guo's
significance runs deeper than his soap-opera tales of scandal and
corruption. ``The grand propaganda scheme is to suppress and control all
the voices,'' Jin said. ``Now everybody knows that you can create your
own voice. You can have your own show. That fact alone is historical.''
In the future, Jin predicts, there will be more rebels like Guo. ``There
is something very primitive about this, realizing that this is a man, a
regular citizen who can confront state power.''

Ho Pin, the founder of Long Island's Mingjing News, echoed Jin.
Mingjing's reporters felt that covering Guo was imperative, no matter
the haziness of the information. ``In China, the political elite that
Guo was attacking had platforms of their own,'' Ho said. ``They have the
opportunity, the power and the ability to use all the government's
apparatus to refute and oppose Guo Wengui. So our most important job is
to allow Guo Wengui's insider knowledge reach the fair, open-minded
people in China.'' Still, people like Pei urge caution when dealing with
Guo's claims. Even Guo's escape raises questions. Few others have
slipped through the net of China's anti-corruption drive. ``How could he
get so lucky?'' Pei asked. ``He must have been tipped off long before.''

At the barbecue, a supporter named Ye Rong tucked one of his children
under his arm and acknowledged that Guo's past life is riddled with
holes. There was always the possibility that Guo used to be a thug, but
Ye didn't think it mattered. The rules of the conflict had been set by
the Communist Party. ``You need all kinds of people to oppose the
Chinese government,'' Ye said. ``We need intellectuals; we also need
thugs.''

Guo, of course, has his own opinions about his legacy. He warned of dark
times for Americans and for the world, if he doesn't succeed in his
mission to change China. ``I am trying to help,'' he told me. ``I am not
joking with you.'' He continued: ``I will change China within the next
three years. If I don't change it, I won't be able to survive.''

Advertisement

\protect\hyperlink{after-bottom}{Continue reading the main story}

\hypertarget{site-index}{%
\subsection{Site Index}\label{site-index}}

\hypertarget{site-information-navigation}{%
\subsection{Site Information
Navigation}\label{site-information-navigation}}

\begin{itemize}
\tightlist
\item
  \href{https://help.nytimes.com/hc/en-us/articles/115014792127-Copyright-notice}{©~2020~The
  New York Times Company}
\end{itemize}

\begin{itemize}
\tightlist
\item
  \href{https://www.nytco.com/}{NYTCo}
\item
  \href{https://help.nytimes.com/hc/en-us/articles/115015385887-Contact-Us}{Contact
  Us}
\item
  \href{https://www.nytco.com/careers/}{Work with us}
\item
  \href{https://nytmediakit.com/}{Advertise}
\item
  \href{http://www.tbrandstudio.com/}{T Brand Studio}
\item
  \href{https://www.nytimes.com/privacy/cookie-policy\#how-do-i-manage-trackers}{Your
  Ad Choices}
\item
  \href{https://www.nytimes.com/privacy}{Privacy}
\item
  \href{https://help.nytimes.com/hc/en-us/articles/115014893428-Terms-of-service}{Terms
  of Service}
\item
  \href{https://help.nytimes.com/hc/en-us/articles/115014893968-Terms-of-sale}{Terms
  of Sale}
\item
  \href{https://spiderbites.nytimes.com}{Site Map}
\item
  \href{https://help.nytimes.com/hc/en-us}{Help}
\item
  \href{https://www.nytimes.com/subscription?campaignId=37WXW}{Subscriptions}
\end{itemize}
