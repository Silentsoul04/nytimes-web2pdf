Sections

SEARCH

\protect\hyperlink{site-content}{Skip to
content}\protect\hyperlink{site-index}{Skip to site index}

\href{https://myaccount.nytimes.com/auth/login?response_type=cookie\&client_id=vi}{}

\href{https://www.nytimes.com/section/todayspaper}{Today's Paper}

\href{/section/opinion}{Opinion}\textbar{}A Conservative Case for
Identity Politics

\href{https://nyti.ms/2G9d3cM}{https://nyti.ms/2G9d3cM}

\begin{itemize}
\item
\item
\item
\item
\item
\end{itemize}

Advertisement

\protect\hyperlink{after-top}{Continue reading the main story}

Supported by

\protect\hyperlink{after-sponsor}{Continue reading the main story}

\href{/section/opinion}{Opinion}

\href{/column/on-campus}{On Campus}

\hypertarget{a-conservative-case-for-identity-politics}{%
\section{A Conservative Case for Identity
Politics}\label{a-conservative-case-for-identity-politics}}

By Jon A. Shields

\begin{itemize}
\item
  Jan. 23, 2018
\item
  \begin{itemize}
  \item
  \item
  \item
  \item
  \item
  \end{itemize}
\end{itemize}

\includegraphics{https://static01.nyt.com/images/2018/01/23/opinion/23Shields-web/23Shields-web-articleLarge.jpg?quality=75\&auto=webp\&disable=upscale}

How should professors respond to the trend of identity politics that is
now roiling American college campuses? Although I am a conservative
professor, I recommend making a concession to it by explicitly assigning
writers of different races and social backgrounds. Let me explain.

When I was in college, I took a class in logic. There I learned that one
should never reject an argument because of the characteristics of the
person making it. Instead, one should assess the argument itself on its
rational merits. And while I agree that the power of an argument
\emph{should not} depend on the person making it, nonetheless, \emph{it
does}.

I learned that lesson during my first year as a visiting professor at
Cornell University. I taught a course on American evangelicals, which
attracted a mix of secular and religious students. When we discussed
``The Scandal of the Evangelical Mind,'' a 1994 book by Mark A. Noll
about anti-intellectualism in the evangelical tradition, my evangelical
students were critical of it. But they were willing to take the book's
thesis seriously because the author was an evangelical.

Perhaps Mr. Noll's identity shouldn't have mattered. His historical
evidence and the power of his arguments would be worth considering even
if he were Catholic, Jewish or secular. But his identity did matter. It
mattered because my evangelical students could not simply assume bad
faith on the author's part. They knew Mr. Noll cared about evangelicals
as a group of people. Instead of dismissing Mr. Noll as a bigot, my
students thoughtfully engaged with his work.

Since then, I have taken identity into account every time I have
assigned new books for one of my courses. I currently teach a course
called Black Intellectuals, which is focused on debates around racial
inequality in the post-civil rights era. It tends to attract progressive
students who, in analyzing racial inequality, are drawn to arguments
that stress structural obstacles to equality and the enduring power of
white racism, especially in our criminal justice system. The course
features black authors who do defend that view, but I also teach the
work of others who depart from it in some measure, including heterodox
thinkers like Thomas Chatterton Williams and conservatives like Jason
Riley. Much like my conservative evangelical students at Cornell, my
progressive students at Claremont McKenna College are less likely to
assume these contrarian black thinkers are acting in bad faith or are
motivated by bigotry --- even when the thinkers criticize hip-hop
culture or defend white police officers. So the students engage the
challenging arguments and ideas instead.

As conservatives have long observed and psychologists have since
confirmed, human beings are hive-minded animals whose moral judgments
are shaped more by sentiments than by reason. Thus, when we are
confronted by arguments we disagree with, we can easily find reasons to
reject them. The search for disconfirming evidence, however, can
sometimes be short-circuited, especially when we feel close to the
person making an argument we disagree with. As the social psychologist
Jonathan Haidt concluded in his 2012 book, ``The Righteous Mind,'' if we
have ``affection, admiration, or desire to please'' other people, we
lean toward them and attempt to ``find the truth'' in their arguments.
Social proximity matters.

If we want our students to consider the work of authors they're inclined
to disagree with, we professors must take the identity of those authors
into account. This doesn't mean scrubbing all white men from our
syllabuses. But when we design an education for our students, we should
remember that humans are partial, tribal beings --- not rational
automatons.

Some readers --- especially those on the right --- may suspect that
embracing identity in this way will only embolden campus radicals. But
that objection ignores an important truth: Practicing the new identity
politics in the right way can subvert the dogmas that drive its
excesses. When students read books by a broad intellectual range of
evangelical or female or black authors, for example, they learn that
there is no single evangelical or female or black perspective.
Disagreements about ideas transcend these social categories.

The left has often placed too much faith in the power of human reason.
Conservatives make the same error when they insist that the identities
of intellectuals should never matter. The fact is, they do. And they
would, even absent new movements on campus.

Advertisement

\protect\hyperlink{after-bottom}{Continue reading the main story}

\hypertarget{site-index}{%
\subsection{Site Index}\label{site-index}}

\hypertarget{site-information-navigation}{%
\subsection{Site Information
Navigation}\label{site-information-navigation}}

\begin{itemize}
\tightlist
\item
  \href{https://help.nytimes.com/hc/en-us/articles/115014792127-Copyright-notice}{©~2020~The
  New York Times Company}
\end{itemize}

\begin{itemize}
\tightlist
\item
  \href{https://www.nytco.com/}{NYTCo}
\item
  \href{https://help.nytimes.com/hc/en-us/articles/115015385887-Contact-Us}{Contact
  Us}
\item
  \href{https://www.nytco.com/careers/}{Work with us}
\item
  \href{https://nytmediakit.com/}{Advertise}
\item
  \href{http://www.tbrandstudio.com/}{T Brand Studio}
\item
  \href{https://www.nytimes.com/privacy/cookie-policy\#how-do-i-manage-trackers}{Your
  Ad Choices}
\item
  \href{https://www.nytimes.com/privacy}{Privacy}
\item
  \href{https://help.nytimes.com/hc/en-us/articles/115014893428-Terms-of-service}{Terms
  of Service}
\item
  \href{https://help.nytimes.com/hc/en-us/articles/115014893968-Terms-of-sale}{Terms
  of Sale}
\item
  \href{https://spiderbites.nytimes.com}{Site Map}
\item
  \href{https://help.nytimes.com/hc/en-us}{Help}
\item
  \href{https://www.nytimes.com/subscription?campaignId=37WXW}{Subscriptions}
\end{itemize}
