Sections

SEARCH

\protect\hyperlink{site-content}{Skip to
content}\protect\hyperlink{site-index}{Skip to site index}

\href{https://www.nytimes.com/section/technology/personaltech}{Personal
Tech}

\href{https://myaccount.nytimes.com/auth/login?response_type=cookie\&client_id=vi}{}

\href{https://www.nytimes.com/section/todayspaper}{Today's Paper}

\href{/section/technology/personaltech}{Personal Tech}\textbar{}Meet the
\$800 Smartphone That You Probably Won't Buy

\url{https://nyti.ms/2F6wcLe}

\begin{itemize}
\item
\item
\item
\item
\item
\item
\end{itemize}

Advertisement

\protect\hyperlink{after-top}{Continue reading the main story}

Supported by

\protect\hyperlink{after-sponsor}{Continue reading the main story}

\href{/column/tech-fix}{Tech Fix}

\hypertarget{meet-the-800-smartphone-that-you-probably-wont-buy}{%
\section{Meet the \$800 Smartphone That You Probably Won't
Buy}\label{meet-the-800-smartphone-that-you-probably-wont-buy}}

\includegraphics{https://static01.nyt.com/images/2018/01/24/business/25TECHFIX-1/25TECHFIX-1-articleLarge.jpg?quality=75\&auto=webp\&disable=upscale}

By \href{http://www.nytimes.com/by/brian-x-chen}{Brian X. Chen}

\begin{itemize}
\item
  Jan. 24, 2018
\item
  \begin{itemize}
  \item
  \item
  \item
  \item
  \item
  \item
  \end{itemize}
\end{itemize}

\href{https://cn.nytimes.com/technology/20180126/huawei-mate-10-pro-smartphone-review/}{阅读简体中文版}\href{https://cn.nytimes.com/technology/20180126/huawei-mate-10-pro-smartphone-review/zh-hant/}{閱讀繁體中文版}

There's a smartphone that the United States does not want you to buy.
It's called the Mate 10 Pro, and it's made by Huawei, a Chinese
manufacturer that the American government has long suspected of
committing espionage for China.

The device, priced at \$800, was supposed to make a big splash this year
as the first high-end smartphone from Huawei in the United States. But
AT\&T, which intended to promote the Mate 10 Pro as a rival to premium
devices from Apple and Samsung,
\href{https://www.nytimes.com/2018/01/09/business/att-huawei-mate-smartphone.html}{abruptly
pulled out of the deal} this month, appearing to bend to pressure from
Washington over security concerns. Verizon Wireless, the country's
biggest carrier, may have also canceled a similar deal because of
\href{https://www.cnet.com/news/verizon-huawei-mate-10-pro-political-pressure-ces/}{political
pressure}, according to some reports. (Verizon declined to comment.)

The snub by AT\&T, the country's No. 2 carrier, aroused a
\href{https://www.theverge.com/2018/1/9/16871538/huawei-ces-2018-event-ceo-richard-yu-keynote-speech}{candid
diatribe from Richard Yu}, Huawei's chief executive, this month at
\href{https://www.nytimes.com/interactive/2018/01/08/technology/ces-2018-reader-questions.html}{CES},
the giant tech convention in Las Vegas.

``It's a big loss for us, and also for carriers,'' he said. ``But the
more big loss is for consumers, because consumers don't have the best
choice.''

Security issues aside, Mr. Yu may have a point. Based on a week of
testing, the Mate 10 Pro is a solid all-around Android smartphone. It
has an excellent camera that takes advantage of artificial intelligence
to shoot clear, rich photos of pets, plants, food and, of course,
people. The device has longer battery life than phones from Apple and
Samsung, and, with durability in mind, it comes with a protective case
and a screen protector.

\includegraphics{https://static01.nyt.com/images/2018/01/24/business/25TECHFIX-3/25TECHFIX-2-articleLarge.jpg?quality=75\&auto=webp\&disable=upscale}

Yet without the backing of a big American carrier, the risks of buying
the smartphone are high. While the Mate 10 Pro will still be available
online next month and on sale at Best Buy stores by the end of the
quarter, the lack of carrier buy-in means it will be tougher to get
device support if your screen shatters or if something goes wrong.

Here's what you need to know about the device.

\hypertarget{the-highlights}{%
\subsection{The Highlights}\label{the-highlights}}

The signature feature of the Mate 10 Pro is the processor, which has a
dedicated part of its silicon specifically designed for artificial
intelligence.

This allows the phone to crunch algorithms and do things like
automatically recognize an object so that the camera can be adjusted to
focus quickly and let in the right amount of light. Huawei also says
A.I. allows the phone to maximize its performance: Periodically, it will
automatically do
\href{https://www.nytimes.com/2016/04/21/technology/personaltech/choosing-to-skipthe-upgrade-and-care-for-the-gadget-youve-got.html}{maintenance},
like clearing out old system files that might otherwise
\href{https://www.nytimes.com/2017/11/15/technology/personaltech/new-iphones-slow-tech-myth.html}{slow
down the phone}.

The camera is notable as well. Huawei teamed up with
\href{http://www.nytimes.com/2013/10/17/technology/personaltech/paying-more-for-the-leica-name.html}{Leica},
a popular camera maker, to develop the phone's dual-lens setup. Like
phones from Apple and Samsung, the Mate 10 Pro's camera can create a
so-called bokeh effect, where the two cameras work together to show the
picture's main subject in sharp focus while gently blurring the
background.

Like other modern smartphones, the Mate 10 Pro is water and dust
resistant. But it also has an extra-large battery that Huawei says will
last longer than that in many other phones. That's partly because of its
A.I. processor, which examines how the battery is being used and changes
resource allocation to prolong its life.

Image

A photo taken with the Huawei Mate 10 Pro, left, compared with one taken
with Apple's iPhone X.

The Mate 10 Pro also ships with a screen protector applied to its
display, and inside the box there is a plastic protective case. These
are thoughtful additions. The case absorbs the impact of drops, and the
screen protector helps prevent scratches, which weaken the structural
integrity of a display.

\hypertarget{pros-and-cons}{%
\subsection{Pros and Cons}\label{pros-and-cons}}

In my tests, the two best features of the Mate 10 Pro were the camera
and battery. The least impressive was the display.

But let's start with the good stuff. In side-by-side comparisons with an
iPhone X and Samsung's Galaxy S8+, the Mate 10 Pro came in second to
Apple's offering in photo quality. All took nice photos, but the colors
in the Galaxy S8+'s pictures looked oversaturated, and while the Mate 10
Pro's photos appeared rich and clear, the shadow details looked better
on the iPhone X.

As for the bokeh effect, also known as portrait mode, the Mate 10 Pro
excelled at separating the subject from the background compared with the
Galaxy S8+, but I still preferred the iPhone X because it did a better
job at lighting up a person's face.

There was one area where the Mate 10 Pro was the clear winner: the
battery. In my tests browsing the web over a cellular connection,
Huawei's phone had roughly two hours more juice than Samsung's Galaxy
Note 8 and the iPhone X.

The display --- the biggest downside of the Mate 10 Pro --- had a lower
resolution than the Note 8, the Galaxy S8+ and the iPhone X, meaning
some graphics and text looked more pixelated. Over all, text appeared
crisper and websites more vibrant on the iPhone X and Samsung Galaxy
screens than they did on the Mate 10 Pro's display.

\hypertarget{bottom-line}{%
\subsection{Bottom Line}\label{bottom-line}}

The Mate 10 Pro is an impressive smartphone, but you probably aren't
going to buy it even if you get your hands on it. The lower-resolution
display is a major negative, as is the lack of carrier support.

Huawei said that to get technical support for the Mate 10 Pro, you can
call its hotline, and for repairs, you can ship your device to a center
in Texas. That's still not ideal compared with the ease of strolling
into an Apple store or your carrier's nearest location.

Privacy and trust are also important. In 2012, the House Intelligence
Committee
\href{http://www.nytimes.com/2012/10/09/us/us-panel-calls-huawei-and-zte-national-security-threat.html}{concluded}
that Huawei and ZTE, another Chinese telecommunications company, were a
national security threat because of their attempts to extract sensitive
data from American companies. And in 2016, security researchers
\href{https://www.nytimes.com/2016/11/16/us/politics/china-phones-software-security.html}{discovered
preinstalled software} on some Huawei and ZTE phones that included a
back door that sent all of a device's text messages to China every 72
hours. That feature was not intended for American phones, according to
the company that made the software. But American lawmakers have been
wary of Huawei.

Most important, you will have to decide whether you trust Huawei. The
onus is on you to
\href{https://consumer.huawei.com/en/legal/privacy-policy/}{carefully
read Huawei's privacy policy} and determine if you feel confident using
this phone. In a statement, Huawei said that privacy and security were
top priorities and that it complied with stringent privacy frameworks
and regulations.

At CES, Huawei's Mr. Yu described how the company had previously
overcome trust hurdles --- including at home in China, where Huawei's
smartphones were initially distrusted by Chinese carriers because the
company was a newcomer.

``It was very hard,'' he said. ``But we won the trust of the Chinese
carriers, we won the trust of the developing market and we also won the
global carriers, all the European and Japanese carriers. Over the last
30 years, we've proven our quality.''

Advertisement

\protect\hyperlink{after-bottom}{Continue reading the main story}

\hypertarget{site-index}{%
\subsection{Site Index}\label{site-index}}

\hypertarget{site-information-navigation}{%
\subsection{Site Information
Navigation}\label{site-information-navigation}}

\begin{itemize}
\tightlist
\item
  \href{https://help.nytimes.com/hc/en-us/articles/115014792127-Copyright-notice}{©~2020~The
  New York Times Company}
\end{itemize}

\begin{itemize}
\tightlist
\item
  \href{https://www.nytco.com/}{NYTCo}
\item
  \href{https://help.nytimes.com/hc/en-us/articles/115015385887-Contact-Us}{Contact
  Us}
\item
  \href{https://www.nytco.com/careers/}{Work with us}
\item
  \href{https://nytmediakit.com/}{Advertise}
\item
  \href{http://www.tbrandstudio.com/}{T Brand Studio}
\item
  \href{https://www.nytimes.com/privacy/cookie-policy\#how-do-i-manage-trackers}{Your
  Ad Choices}
\item
  \href{https://www.nytimes.com/privacy}{Privacy}
\item
  \href{https://help.nytimes.com/hc/en-us/articles/115014893428-Terms-of-service}{Terms
  of Service}
\item
  \href{https://help.nytimes.com/hc/en-us/articles/115014893968-Terms-of-sale}{Terms
  of Sale}
\item
  \href{https://spiderbites.nytimes.com}{Site Map}
\item
  \href{https://help.nytimes.com/hc/en-us}{Help}
\item
  \href{https://www.nytimes.com/subscription?campaignId=37WXW}{Subscriptions}
\end{itemize}
