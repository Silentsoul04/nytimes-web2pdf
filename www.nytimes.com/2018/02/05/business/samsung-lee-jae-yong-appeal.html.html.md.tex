Sections

SEARCH

\protect\hyperlink{site-content}{Skip to
content}\protect\hyperlink{site-index}{Skip to site index}

\href{https://www.nytimes.com/section/business}{Business}

\href{https://myaccount.nytimes.com/auth/login?response_type=cookie\&client_id=vi}{}

\href{https://www.nytimes.com/section/todayspaper}{Today's Paper}

\href{/section/business}{Business}\textbar{}Samsung Heir Freed, to
Dismay of South Korea's Anti-Corruption Campaigners

\url{https://nyti.ms/2GMvUdX}

\begin{itemize}
\item
\item
\item
\item
\item
\end{itemize}

Advertisement

\protect\hyperlink{after-top}{Continue reading the main story}

Supported by

\protect\hyperlink{after-sponsor}{Continue reading the main story}

\hypertarget{samsung-heir-freed-to-dismay-of-south-koreas-anti-corruption-campaigners}{%
\section{Samsung Heir Freed, to Dismay of South Korea's Anti-Corruption
Campaigners}\label{samsung-heir-freed-to-dismay-of-south-koreas-anti-corruption-campaigners}}

\includegraphics{https://static01.nyt.com/images/2018/02/06/world/06samsung-1sub/06samsung-1sub-articleLarge.jpg?quality=75\&auto=webp\&disable=upscale}

By \href{https://www.nytimes.com/by/choe-sang-hun}{Choe Sang-Hun} and
\href{https://www.nytimes.com/by/raymond-zhong}{Raymond Zhong}

\begin{itemize}
\item
  Feb. 5, 2018
\item
  \begin{itemize}
  \item
  \item
  \item
  \item
  \item
  \end{itemize}
\end{itemize}

SEOUL, South Korea --- When Lee Jae-yong, the de facto leader of
Samsung, walked free on Monday after spending barely a year in jail, it
reaffirmed a pattern South Koreans have fought for decades to break:
Business tycoons convicted of corruption here hardly spend any time
behind bars.

\href{https://www.nytimes.com/2017/02/16/world/asia/korea-samsung-lee-jae-yong.html}{Mr.
Lee's arrest} a year ago, and his subsequent conviction and sentencing,
were hard-won victories for millions of protesters who took to the
streets from late 2016. The demonstrations also felled President Park
Geun-hye,
\href{https://www.nytimes.com/2017/04/17/world/asia/park-geun-hye-south-korea-president-indictment.html}{impeached
on charges of collecting bribes} from family-controlled conglomerates,
\href{https://www.nytimes.com/2017/03/04/business/south-korea-samsung-bribery-lee.html}{known
as chaebol}, like Samsung.

But here in South Korea, Mr. Lee's detention was as big a piece of news
as Ms. Park's ouster. His father ---
\href{http://www.nytimes.com/2009/12/30/business/global/30samsung.html}{Lee
Kun-hee}, the son of Samsung's founder and the conglomerate's chairman
--- was twice convicted of bribery and other corruption charges but
never spent a day in jail, creating an image of Samsung as untouchable.
Many South Koreans hailed the lower-court ruling, which
\href{https://www.nytimes.com/2017/08/25/business/samsung-bribery-embezzlement-conviction-jay-lee-south-korea.html}{sentenced
the younger Mr. Lee to five years in prison} on corruption charges, as
an important milestone in their country's long-running campaign toward
greater transparency and accountability.

So when an appeals court freed Mr. Lee, 49, on Monday by reducing his
prison term to two and a half years and then suspending it, the scene
was dishearteningly familiar for many.

Over the decades, numerous chaebol executives have been paraded into
courts on bribery and other charges. But they have usually walked away
with light sentences
(\href{https://www.nytimes.com/2017/12/22/business/korea-lotte-corruption-conviction.html}{most
of them suspended}), free to manage their businesses, even as courts
routinely sentenced lesser-known white-collar criminals to far longer
terms for lesser offenses.

That led to criticism that the conglomerates were too powerful to tame
--- a problem President Moon Jae-in repeatedly decried when he rode
waves of popular discontent to win an election in May to replace Ms.
Park.

``This is a critical setback for the country,'' said Jun Sung-in, an
economist at Hongik University in Seoul. ``This case once again shows
why the South Korean judiciary does not have the people's trust when it
comes to cases involving chaebol chieftains.''

Samsung is arguably South Korea's brightest corporate success story, as
the country transformed from a war-torn agrarian economy into a global
export powerhouse.

In the span of a few decades, the company, once a copycat manufacturer
of clunky television sets, surpassed Sony and other global giants in
value and reach, offering high-end smartphones, computer chips and
flat-panel TVs. It is the biggest and most lucrative of a handful of
chaebol conglomerates that dominate South Korea's economy.

But it is not universally beloved. At home, Samsung is often seen as a
menace with unbridled power. Mr. Lee --- known as Jay Y. Lee in the West
--- is a third-generation tycoon whose qualifications as a top manager
are regarded with skepticism, if not downright scorn. Unlike their
fathers and grandfathers, this latest set of business leaders stand
accused of inheriting management control and wealth through opaque
bookkeeping and questionable trading among subsidiaries.

The verdict Monday will only deepen that unsavory image of Samsung,
while ``disappointing numerous people who have hoped that this case
would serve to end politics-business collusion,'' said Chung Sun-sup,
editor of \href{http://chaebul.com/}{chaebul.com}, a website that
specializes in monitoring the family conglomerates.

Mr. Lee's supporters say he has been made a scapegoat for politically
motivated prosecutors. They accused the authorities of pandering to
widespread anti-chaebol sentiments and building their case against Mr.
Lee with little evidence.

Prosecutors had originally indicted Mr. Lee on charges of giving or
promising \$27 million in bribes to foundations and business entities
controlled by Choi Soon-sil, a longtime confidante of Ms. Park, to win
Ms. Park's support for strengthening his control of Samsung, which he is
inheriting from his ailing father.

But the lower-court ruling recognized only about \$8 million as bribes.
In Monday's ruling, the size was reduced further to \$3.3 million. Park
Young-soo, the special prosecutor investigating the case, said his team
would appeal to the Supreme Court.

The latest ruling was closely monitored because it could also affect Ms.
Park, whose corruption trial is still underway.

Mr. Lee's lawyers acknowledged the payments but argued that Samsung did
not receive any favors or special treatment in return. Instead, Mr. Lee
said he was a victim of Ms. Park's extortion --- an argument Justice
Chung Hyong-sik sided with, to a degree, in his appeals court ruling.

``This is a case where President Park intimidated the management of
Samsung,'' Mr. Chung said, calling Mr. Lee a ``passive'' provider of
bribes.

Shortly afterward, Mr. Lee emerged from prison a free man.

``I want to say once again how sorry I am that I have failed to present
a good image of myself,'' Mr. Lee told reporters. ``The past year has
been a valuable time for me to reflect.''

Pro-business groups, investors and conservative politicians welcomed the
verdict.

``In a time of worsening business environments, incarcerating
businessmen for a long time does tremendous damage not only to their
companies but also to the national economy,'' said Kim Kyong-man, an
executive at the Korea Federation of Small and Medium Business.

The jewel of the Samsung group's empire, Samsung Electronics, grew into
a global force by identifying and getting ahead of big shifts in
technology. It charged into the semiconductor business, for instance,
when the company was still known as an assembler of cut-price TVs. The
company also began focusing on slick, astutely marketed consumer
products just as public enthusiasm for gadgets started ramping up in the
2000s.

Today, the semiconductor business accounts for the bulk of the company's
profit and, with demand for powerful servers and data centers continuing
to rise worldwide, it is expected to continue to rake in money. Samsung
is also the world's largest seller of smartphones.

Credit for those successes, however, goes largely to Mr. Lee's
grandfather and father. Industry observers have long questioned whether
the Harvard-educated Mr. Lee, who had not been known for running a
business by himself before he was thrust into his present role, can
safeguard the company's success.

His first high-profile venture was eSamsung, a start-up incubator that
went belly-up during the dot-com collapse of the early 2000s. He later
headed up a joint venture with Sony to
\href{http://www.nytimes.com/2011/12/27/technology/sony-sells-stake-in-lcd-panel-joint-venture.html}{make
flat-panel screens} and, in 2007, was made the company's first ``chief
customer officer.''

But he stepped down the next year to develop new businesses in emerging
markets. He then served as Samsung's president and chief operating
officer, before becoming vice chairman in 2012.

More recently, Mr. Lee has shown greater initiative in steering Samsung
toward the technologies of tomorrow.

In 2016, he oversaw the
\href{https://www.nytimes.com/2016/11/15/business/samsung-auto-industry-harman-automotive.html}{\$8
billion purchase of Harman International Industries}, an American maker
of car audio systems. Harman has become part of Samsung's push to create
an ecosystem of internet-enabled devices, from refrigerators and phones
to cars and televisions, with which users can conduct their digital
lives.

The big unknown on Samsung's horizon today, however, is China. The
government in Beijing is
\href{https://www.nytimes.com/2017/11/07/business/made-in-china-technology-trade.html}{pouring
billions of dollars} into local companies with the aim of dominating
advanced computer chips, and potential challengers to Samsung are
emerging.

``That's going to be a huge threat,'' said Nam Lee, a professor at
Chung-Ang University in Seoul. ``It's not imminent, I would say. But
nobody can guarantee the future.''

That means Samsung could still use a far-thinking leader to chart out
the company's next strategic moves.

``One thing is very clear,'' Professor Lee said, ``J. Y. Lee has not
proven if he's the one.''

Advertisement

\protect\hyperlink{after-bottom}{Continue reading the main story}

\hypertarget{site-index}{%
\subsection{Site Index}\label{site-index}}

\hypertarget{site-information-navigation}{%
\subsection{Site Information
Navigation}\label{site-information-navigation}}

\begin{itemize}
\tightlist
\item
  \href{https://help.nytimes.com/hc/en-us/articles/115014792127-Copyright-notice}{©~2020~The
  New York Times Company}
\end{itemize}

\begin{itemize}
\tightlist
\item
  \href{https://www.nytco.com/}{NYTCo}
\item
  \href{https://help.nytimes.com/hc/en-us/articles/115015385887-Contact-Us}{Contact
  Us}
\item
  \href{https://www.nytco.com/careers/}{Work with us}
\item
  \href{https://nytmediakit.com/}{Advertise}
\item
  \href{http://www.tbrandstudio.com/}{T Brand Studio}
\item
  \href{https://www.nytimes.com/privacy/cookie-policy\#how-do-i-manage-trackers}{Your
  Ad Choices}
\item
  \href{https://www.nytimes.com/privacy}{Privacy}
\item
  \href{https://help.nytimes.com/hc/en-us/articles/115014893428-Terms-of-service}{Terms
  of Service}
\item
  \href{https://help.nytimes.com/hc/en-us/articles/115014893968-Terms-of-sale}{Terms
  of Sale}
\item
  \href{https://spiderbites.nytimes.com}{Site Map}
\item
  \href{https://help.nytimes.com/hc/en-us}{Help}
\item
  \href{https://www.nytimes.com/subscription?campaignId=37WXW}{Subscriptions}
\end{itemize}
