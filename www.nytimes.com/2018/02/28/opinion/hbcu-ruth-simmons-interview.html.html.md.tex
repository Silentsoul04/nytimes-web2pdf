Sections

SEARCH

\protect\hyperlink{site-content}{Skip to
content}\protect\hyperlink{site-index}{Skip to site index}

\href{https://myaccount.nytimes.com/auth/login?response_type=cookie\&client_id=vi}{}

\href{https://www.nytimes.com/section/todayspaper}{Today's Paper}

\href{/section/opinion}{Opinion}\textbar{}Ruth Simmons on Cultivating
the Next Generation of College Students

\href{https://nyti.ms/2FFATgy}{https://nyti.ms/2FFATgy}

\begin{itemize}
\item
\item
\item
\item
\item
\item
\end{itemize}

Advertisement

\protect\hyperlink{after-top}{Continue reading the main story}

Supported by

\protect\hyperlink{after-sponsor}{Continue reading the main story}

\href{/section/opinion}{Opinion}

\href{/column/on-campus}{On Campus}

\hypertarget{ruth-simmons-on-cultivating-the-next-generation-of-college-students}{%
\section{Ruth Simmons on Cultivating the Next Generation of College
Students}\label{ruth-simmons-on-cultivating-the-next-generation-of-college-students}}

By Marguerite Joutz

\begin{itemize}
\item
  Feb. 28, 2018
\item
  \begin{itemize}
  \item
  \item
  \item
  \item
  \item
  \item
  \end{itemize}
\end{itemize}

\includegraphics{https://static01.nyt.com/images/2018/02/09/opinion/00oncampusWeb/merlin_132030296_0c920f3e-686e-45fb-8ca9-0bf59d159a34-articleLarge.jpg?quality=75\&auto=webp\&disable=upscale}

In 2012, after a distinguished career in higher education, which
included serving as the president of Smith College and Brown University
--- where she was the first black woman to lead an Ivy League
institution --- Ruth J. Simmons retired and moved back to her home
state, Texas.

She didn't think she'd ever work at a university again. But after
several years, Prairie View A\&M University, a historically black school
with about 9,000 students, came calling.

``I was quite surprised to be approached about the opportunity,'' Dr.
Simmons told me, but ``I also went to an H.B.C.U. --- Dillard University
--- and I'm very much aware of the ways in which these colleges play a
role for many students.''

After several months as interim president of the university, she
accepted the permanent position in October 2017.

I asked Dr. Simmons to talk about her new role and plans for Prairie
View because she has been at the forefront of many trends in higher
education. At Brown, she established need-blind admissions for
undergraduates (the last of the Ivy League to do so). In 2003, she
commissioned a study of the university's ties to the slave trade --- one
of the first university efforts to uncover the historical connection
between academia and slavery. At Smith, she created an engineering
program, the first at any all-women's college.

(Disclosure: I attended Brown and overlapped with Dr. Simmons my
freshman year.)

This conversation has been edited and condensed.

\textbf{Marguerite Joutz:} Hi, Dr. Simmons. I'd love to jump right in
and hear your aspirations for Prairie View. What do you hope to
accomplish in your time as president? I assume you had many offers to
lead other colleges. Why was Prairie View different?

\textbf{Ruth Simmons:} Well, I'm a native of Texas, and one of the
things that I did when I returned to Houston after I retired from Brown
was think about the ways I could use my experience to help the
community. I started with small ventures in the city, but I hadn't
thought really that another university would be in my life. But when I
was approached by Prairie View, I thought immediately about its
significant impact on my brother, who went there. And what Dillard did
for me.

I grew up in a segregated neighborhood in Grapeland, Tex., with very
little experience with the world. The transition that Dillard enabled
for me was extremely valuable. As an undergraduate, I began to grow a
sense of confidence that helped me move from a fairly insular
environment to a bigger stage when I went to graduate school.

My aspirations for Prairie View are to essentially make sure the
university is continuing to do the same thing for students today that it
did for my brother --- and Dillard did for me. And that is to offer the
advantage of a strong education that will prepare students for the
careers they want, in a social and cultural context that helps them
develop the confidence to perform after graduation.

As president, that means focusing on time-honored strategies to success
that apply to universities everywhere: worrying about the faculty who
are recruited here, the campus experience, and whether we are providing
the leadership and internship opportunities that students need. It means
worrying about the reputation of the university. It's obviously a much
more competitive world today than it was when I was a student, but the
underlying work to move the university to a level of achievement that
makes students and alumni proud is the same.

\textbf{M.J.: What do you see as some of the major challenges facing
historically black colleges and universities?}

\textbf{R.S.:} The general challenge for H.B.C.U.s is not unlike the
problem many universities face today, and that is the financial model.
Being tuition dependent has become a large problem for all universities.
H.B.C.U.s are dependent on raising money, but that is a reality that
most universities obviously face. So it very much depends on the
question of whether the schools have a good-enough foundation because of
secure financing. When it comes to the federal government's role in
supporting H.B.C.U.s, I think we don't yet understand the extent to
which they will be affected by the changes in Washington today. To say
that many are not hopeful in this moment is an understatement. We just
don't know.

Frankly, that's why I advocate for developing enough financial
independence to secure the future of our institutions. That means
looking for different revenue streams, looking to increase our ability
to raise funds and finding alternate programs that help to bring more
income into our university. The only strategy cannot be to persistently
raise tuition, making college outside the reach of many families.

\textbf{M.J.: You've led several institutions that are all very
different from each other, and the topic of leadership is something
you've talked a lot about. In another}
\href{http://www.nytimes.com/2011/12/04/business/ruth-simmons-of-brown-university-on-amiable-leadership.html}{\textbf{interview}}
\textbf{with The Times you said that it's much easier to lead people
``if you convey the underlying principles.'' Can you elaborate?}

\textbf{R.S.:} One thing that I've learned is that the perceptions of
what it takes to be a leader are often based on prototypical models that
don't have much truth in reality. People look at the institutions that I
have led and they see dissimilarities. I see similarities. When people
think in terms of leadership, they're often thinking about the kind of
specific skills needed for different types of enterprises. I think of
leadership as more of a disposition --- the ability to step into a
situation to learn about the history of the enterprise, the
opportunities that it faces, the culture that exists and the people who
are served by it. To look at all of that, to listen to stakeholders and
then to think about how that enterprise or institution should best be
served. There is no one model of leadership if you approach it that way.
What I have tried to do wherever I go is to start where the institution
is rather than try to import particularly rigid constructs from other
places. In that sense, I think a leader is more than anything else a
facilitator. A person who is able to come in to show a community a
picture of what it is, to provide some insight into what it could be ---
how it could be different or improved perhaps --- and then enlist the
help of people who are there and others who support that institution in
order to move forward together.

I don't subscribe to the model of hero leadership, which is identifying
somebody who can come in and have magical powers and then wield the wand
and fix things that have not been fixable before. I don't see that. I
think leadership is a community affair.

\textbf{M.J.: That makes sense. So how do you approach educating the
current generation of students at Prairie View to lead in the way you
described?}

\textbf{R.S.:} People today are fond of leadership programs that
theorize about the profile and tenets of leadership. And students
anxiously get involved in those programs. I rather think that our entire
campuses are incubators of leadership even without the formality of such
programs because if we're doing what we should be doing, we are
acclimating students to an environment in which they have to learn to
work with others who are very different from themselves. And that seems
to me to be the first requirement of leadership. To actually learn to
work with people in a respectful and inclusive way is inordinately
important. A campus provides one of the best opportunities for people to
be able to do that because you sit in class alongside people whom you
initially don't know. You are discussing your ideas, having people
respond to them and often rejecting what you're saying. You are joining
organizations that have certain aims that are being advanced by the
collective, so you're learning how to facilitate. You are asked to step
into leadership roles either through student government or
organizations. Even in the classroom context you might play a leadership
role. You have to learn to express yourself, to be convincing, to write
out your ideas, to be more thoughtful than you'd ordinarily be. All of
those things are components of leadership.

\textbf{M.J.: Some would argue that social media, or perhaps
disagreements over the concept of safe spaces, have made it more
challenging for people to come together and to work together.}

\textbf{R.S.:} Yeah, I would say that people frequently say that, but
that has certainly not been my experience. I would say that in my days
as a student, the tensions were higher, the disagreements were greater,
the separation among us was certainly more pronounced. And people are
kind of inventing a new narrative, that things are so much worse today
than they were in the past. I don't buy that. I think we have new
terminology, because the modern sciences of psychology and sociology and
so forth and modern media have enabled us to peer into areas that we
were not able to see as easily as before. And so when you have social
media and all of the ways in which people can now say whatever they want
to say --- in an unartful or offensive way --- that turns the volume up
to be sure. But it doesn't mean it wasn't there before.

I would say that the current situation ought to give us much more
practice in how to engage with each other because it is not underground.
It's out front, it's in the open. The media today and the things that
are being reported \emph{should} be upsetting or troubling to students,
my goodness. But it may be an even better opportunity to engage
students.

As a teacher, as an educator, I would say these times are especially
fertile for teaching the skills that are needed in a world in which
there are many types of people, many perspectives and so forth. Students
don't need to feel silenced by the current situation. When students at
Brown came to me and said, you know, they were hurt because somebody
asked them about their hair, and they were tired of that and they didn't
want any part of it, I would say, ``Well, that's just too bad!''
Somebody asked you about your hair, tell them about your hair. You are
in college not to hide from that. You're there to teach the world about
yourself, you're there to learn about others and to teach them about who
you are. So I'd rather think this whole notion of protecting people from
that is something that we have to take in hand and deal with. And help
students see the opportunity in that because if they engage with people
who have those kinds of perspectives, they are going to be much stronger
in the end than if they run away.

\textbf{M.J.: Do you see the situation differently for students at
H.B.C.U.s? You mentioned that Dillard gave you the space to learn and
build confidence before heading to graduate school.}

\textbf{R.S.:} There is a narrative that many people have for H.B.C.U.s.
People see them as a monolith, or only see them as filling a gap that
exists for African-Americans. H.B.C.U.s are not a brand. They are
institutions with a certain history --- they were originally set up
basically to insure that there would be two separate systems for
education --- but they are not all alike, and are very diverse in what
they offer.

Prairie View, for example, is the second-oldest public institution of
higher education in the state of Texas. It was set up basically to
insure that there would be two separate systems for ex-slaves and
whites. For a long time it provided educational opportunities for black
students that wouldn't have been open to them otherwise. In nursing,
architecture, engineering and other STEM fields at a time when many
blacks were not able to go into those areas, the university was able to
facilitate the entry of blacks into those professions.

\textbf{M.J.: Yes, this is something that Nikole Hannah-Jones has
written about for The New York Times Magazine, in particular how Xavier
University of Louisiana}
\textbf{\href{https://www.nytimes.com/2015/09/13/magazine/a-prescription-for-more-black-doctors.html}{sends
more black students to medical school}} \textbf{than almost any other
college in the country. I read that Prairie View has many more students
who graduate with degrees in architecture and engineering.}

\textbf{R.S.:} Yes, going to Dillard is not like going to Prairie View.
So the student who chooses to go to one H.B.C.U. may not find another
appealing. That has to do with the unique features of the place, the
academic culture and so on, the traditions. All of that makes a big
difference.

When I went off to college, I was just trying to find a place where I
could be a student without impediments that in that day and time were
fairly widely known to be substantial if you enrolled in a
majority-white institution. There was no question that in 1963, if you
went off to college in the South, you had to be very careful about where
you studied. The racial dynamic was very challenging at that time. One
of the things I'm learning now is that many of the students I speak to
at Prairie View today have the same kinds of thoughts that I had those
many years ago. They ask themselves, where can I feel comfortable so
that I don't have to worry about anything but my academic work? So
certainly that plays a factor in why students apply to H.B.C.U.s. And
for some H.B.C.U.s, applications are way up. But if you reach the point
where you think that a category of institutions are the right choice for
you --- going to an H.B.C.U. just to go to an H.B.C.U. --- then you're
not thinking broadly enough.

\textbf{M.J.: As you look across the landscape of higher education, what
other challenges do college students face today?}

\textbf{R.S.:} I guess I worry a lot about our students having the
skills and the experience to promote respectful interactions with a
wide-ranging group of people. Students can be quite passionate about
what they see on the national scene and how sometimes dangerous they
think it is --- how unpleasant it is and so forth. But we've got to try
to find a way on our campuses not to taint the national picture as
hopeless, or as being antithetical to anything that we can think of as
good and admirable. The last thing we want is for our students to bow
out and decide that it isn't worth trying to do anything about a
hopeless situation.

And I do think that civility goes hand in hand with being hopeful.
Hopeful that the next person you meet you will be able to converse with
you in a respectful way. Hopeful that the next election cycle is going
to give you a chance to be engaged and able to do something that makes
you feel a sense of your own agency. Hopeful that the things will get
better rather than worse.

So much of what we have to do on our campuses is really to hold open the
possibility for people that civil society will in the end right itself
--- if enough of us are engaged. I am so thankful every day for the fact
that I grew up in a time when nobody thought blacks would be able to be
integrated into American society, they would never be able to hold
important positions, they'd never be respected intellectually, they'd
never be able to do the things that they are now able to do. So why did
we persist? And why did we continue to work toward betterment? We did it
because we had people working with us in education who were ever hopeful
that things were going to change. So I like to say to educators, you
have to be always projecting to students that civility enables
hopefulness to be ever resurgent.

Advertisement

\protect\hyperlink{after-bottom}{Continue reading the main story}

\hypertarget{site-index}{%
\subsection{Site Index}\label{site-index}}

\hypertarget{site-information-navigation}{%
\subsection{Site Information
Navigation}\label{site-information-navigation}}

\begin{itemize}
\tightlist
\item
  \href{https://help.nytimes.com/hc/en-us/articles/115014792127-Copyright-notice}{©~2020~The
  New York Times Company}
\end{itemize}

\begin{itemize}
\tightlist
\item
  \href{https://www.nytco.com/}{NYTCo}
\item
  \href{https://help.nytimes.com/hc/en-us/articles/115015385887-Contact-Us}{Contact
  Us}
\item
  \href{https://www.nytco.com/careers/}{Work with us}
\item
  \href{https://nytmediakit.com/}{Advertise}
\item
  \href{http://www.tbrandstudio.com/}{T Brand Studio}
\item
  \href{https://www.nytimes.com/privacy/cookie-policy\#how-do-i-manage-trackers}{Your
  Ad Choices}
\item
  \href{https://www.nytimes.com/privacy}{Privacy}
\item
  \href{https://help.nytimes.com/hc/en-us/articles/115014893428-Terms-of-service}{Terms
  of Service}
\item
  \href{https://help.nytimes.com/hc/en-us/articles/115014893968-Terms-of-sale}{Terms
  of Sale}
\item
  \href{https://spiderbites.nytimes.com}{Site Map}
\item
  \href{https://help.nytimes.com/hc/en-us}{Help}
\item
  \href{https://www.nytimes.com/subscription?campaignId=37WXW}{Subscriptions}
\end{itemize}
