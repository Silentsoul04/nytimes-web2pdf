Sections

SEARCH

\protect\hyperlink{site-content}{Skip to
content}\protect\hyperlink{site-index}{Skip to site index}

\href{https://www.nytimes.com/section/politics}{Politics}

\href{https://myaccount.nytimes.com/auth/login?response_type=cookie\&client_id=vi}{}

\href{https://www.nytimes.com/section/todayspaper}{Today's Paper}

\href{/section/politics}{Politics}\textbar{}How 2 Gulf Monarchies Sought
to Influence the White House

\url{https://nyti.ms/2FTdrjc}

\begin{itemize}
\item
\item
\item
\item
\item
\item
\end{itemize}

Advertisement

\protect\hyperlink{after-top}{Continue reading the main story}

Supported by

\protect\hyperlink{after-sponsor}{Continue reading the main story}

\hypertarget{how-2-gulf-monarchies-sought-to-influence-the-white-house}{%
\section{How 2 Gulf Monarchies Sought to Influence the White
House}\label{how-2-gulf-monarchies-sought-to-influence-the-white-house}}

\includegraphics{https://static01.nyt.com/images/2018/03/22/us/politics/22dc-investigate1/22dc-investigate1-articleLarge-v2.jpg?quality=75\&auto=webp\&disable=upscale}

By \href{https://www.nytimes.com/by/david-d-kirkpatrick}{David D.
Kirkpatrick} and \href{https://www.nytimes.com/by/mark-mazzetti}{Mark
Mazzetti}

\begin{itemize}
\item
  March 21, 2018
\item
  \begin{itemize}
  \item
  \item
  \item
  \item
  \item
  \item
  \end{itemize}
\end{itemize}

A cooperating witness in the special counsel investigation worked for
more than a year to turn a top Trump fund-raiser into an instrument of
influence at the White House for the rulers of Saudi Arabia and the
United Arab Emirates, according to interviews and previously undisclosed
documents.

Hundreds of pages of correspondence between the two men reveal an active
effort to cultivate President Trump on behalf of the two oil-rich Arab
monarchies, both close American allies.

High on the agenda of the two men ---
\href{https://www.nytimes.com/2018/03/06/us/politics/george-nader-special-counsel-mueller-cooperating-seychelles.html}{George
Nader}, a political adviser to the de facto ruler of the U.A.E., and
Elliott Broidy, the deputy finance chairman of the Republican National
Committee --- was pushing the White House to remove Secretary of State
Rex W. Tillerson, backing confrontational approaches to Iran and Qatar
and repeatedly pressing the president to meet privately outside the
White House with the leader of the U.A.E.

\href{https://www.nytimes.com/2018/03/13/us/politics/trump-tillerson-pompeo.html?action=click\&module=Top\%20Stories\&pgtype=Homepage}{Mr.
Tillerson was fired last week}, and the president has adopted tough
approaches toward both Iran and Qatar.

Mr. Nader tempted the fund-raiser, Mr. Broidy, with the prospect of more
than \$1 billion in contracts for his private security company,
Circinus, and he helped deliver deals worth more than \$200 million with
the United Arab Emirates. He also flattered Mr. Broidy about ``how well
you handle Chairman,'' a reference to Mr. Trump, and repeated to his
well-connected friend that he told the effective rulers of both Saudi
Arabia and the U.A.E. about ``the Pivotal Indispensable Magical Role you
are playing to help them.''

Mr. Nader's cultivation of Mr. Broidy, laid out in documents provided to
The New York Times, provides a case study in the way two Persian Gulf
monarchies have sought to gain influence inside the Trump White House.
Mr. Nader has been granted immunity in a deal for his cooperation with
the special counsel, Robert S. Mueller III, according to people familiar
with the matter, and his relationship with Mr. Broidy may also offer
clues to the direction of that inquiry.

Mr. Nader has now been called back from abroad to provide additional
testimony, one person familiar with the matter said this week. Mr.
Mueller's investigators have already asked witnesses about Mr. Nader's
contacts with top Trump administration officials and about his possible
role in funneling Emirati money to Mr. Trump's political efforts, a sign
that the investigation has broadened to examine the role of foreign
money in the Trump administration.

The documents contain evidence not previously reported that Mr. Nader
also held himself out as intermediary for Saudi Arabia's crown prince,
Mohammed bin Salman,
\href{https://www.nytimes.com/2018/03/20/us/politics/saudi-crown-prince-arrives-at-white-house-to-meet-with-trump.html}{who
met with Mr. Trump on Tuesday in the Oval Office} at the beginning of a
tour of the United States to meet with political and business leaders.

A lawyer for Mr. Nader declined to comment. Two people close to Mr.
Broidy said he had not been contacted by the special counsel's
investigators. In a statement, Mr. Broidy said that his efforts ``aimed
to strengthen the national security of the United States, in full
coordination with the U.S. government.'' He added, ``I have always
believed strongly in countering both Iran and Islamic extremism, and in
working closely with our friends in the Arab world in order to do so.''

The documents, which included emails, business proposals and contracts,
were provided by an anonymous group critical of Mr. Broidy's advocacy of
American foreign policies in the Middle East. The Times showed Mr.
Broidy's representatives copies of all of the emails it intended to cite
in an article. In his statement, Mr. Broidy said he could not confirm
the authenticity of all of them, noting that The Times was able to show
him only printouts and not the original emails.

A spokesman for Mr. Broidy has said he believes the documents were
stolen by hackers working for Qatar in retaliation for his work critical
of the country --- a regional nemesis of the Saudis and Emiratis.

``We now possess irrefutable evidence tying Qatar to this unlawful
attack on, and espionage directed against, a prominent United States
citizen within the territory of the United States,'' Lee S. Wolosky, a
lawyer for Mr. Broidy, wrote this week in
\href{https://assets.documentcloud.org/documents/4417582/Elliott-Broidy-Letter-to-Qatari-Ambassador.pdf}{a
letter to the Qatari ambassador} in Washington. If Qatar was not
responsible, ``we expect your government to hold accountable the rogue
actors in Qatar who have caused Mr. Broidy substantial damages.''

\includegraphics{https://static01.nyt.com/images/2018/04/14/us/politics/14Broidy4/22dc-investigate2-articleLarge.jpg?quality=75\&auto=webp\&disable=upscale}

\hypertarget{forging-a-connection}{%
\subsection{Forging a Connection}\label{forging-a-connection}}

The two men first met during the crush of parties and other events
surrounding Mr. Trump's inauguration. Mr. Broidy, 60, a longtime
Republican donor and a vice chairman of the inaugural fund-raising
committee, got his start in business as an accountant and then as an
investment manager for
\href{https://www.nytimes.com/2010/01/19/business/19bell.html}{Glen
Bell}, the founder of Taco Bell.

Mr. Nader, 58, a United States citizen born in Lebanon, previously ran a
Washington-based journal called Middle East Insight, acted as an
informal emissary to Syria under the Clinton administration, and,
according to a short biography in the emails, later worked for Vice
President Dick Cheney.

The two became fast friends, and by February, they were exchanging
emails about potential contracts for Circinus with both the U.A.E. and
Saudi Arabia, and also about Saudi and Emirati objectives in Washington,
such as persuading the United States government to take action against
the Muslim Brotherhood or put pressure on its regional ally, Qatar.

Early in the Trump administration, the two men also noted with approval
a successful effort to block a top Pentagon position for Anne Patterson,
a former ambassador to Cairo whom the Emiratis and Saudis have long
criticized as too sympathetic to the deposed Egyptian president Mohamed
Morsi of the Muslim Brotherhood during his one year in office.

In one message to Mr. Nader in March 2017, Mr. Broidy referred to
\href{https://www.secureamericanow.org/home}{Secure America Now}, an
advocacy organization that he suggested had campaigned against Ms.
Patterson, as ``one of the groups I am working with.'' The two people
close to Mr. Broidy said he had not raised money for the group or
campaigned against Ms. Patterson.

The Saudis and Emiratis have had particularly warm relations with the
Trump administration. Mr. Trump at times has appeared to side with the
Arab monarchies against his own cabinet secretaries --- including in the
bitter regional dispute against neighboring Qatar. Also in concert with
the Saudis and Emiratis, Mr. Trump has taken a far more hawkish stance
toward Iran than either his cabinet or President Barack Obama,
threatening to ``rip up'' the Iran nuclear deal that
\href{https://www.nytimes.com/2015/07/15/world/middleeast/iran-nuclear-deal-is-reached-after-long-negotiations.html}{Mr.
Obama brokered in 2015}.

On March 25, Mr. Broidy emailed Mr. Nader a spreadsheet outlining a
proposed Washington lobbying and public relations campaign against both
Qatar and the Muslim Brotherhood. The proposed campaign's total cost was
\$12.7 million.

The two people close to Mr. Broidy said the plan was drafted by a third
party for circulation to like-minded American donors, and that only some
of its provisions were carried out.

Mr. Nader did, however, provide a \$2.7 million payment to Mr. Broidy
for ``consulting, marketing and other advisory services rendered,''
apparently to help pay for the cost of conferences at two Washington
think tanks, the Hudson Institute and the Foundation for Defense of
Democracies, that featured heavy criticism of Qatar and the Muslim
Brotherhood.

Hudson Institute policies prohibit donations from foreign governments
that are not democracies, and the Foundation for Defense of Democracies
bars donations from all foreign governments, so Mr. Nader's role as an
adviser to the U.A.E. may have raised concerns had he donated directly.

The foundation said in a statement that it was approached by Mr. Broidy
in 2017 seeking to fund a conference on Qatar and the Muslim
Brotherhood. ``As is our funding policy, we asked if his funding was
connected to any foreign governments or if he had business contracts in
the Gulf. He assured us that he did not,'' the statement said.

Documents show Mr. Nader's payment was made by an Emirati-based company
he controlled, GS Investments, to an obscure firm based in Vancouver,
British Columbia, controlled by Mr. Broidy, Xieman International. A
person close to Mr. Broidy said the money was passed through the
Canadian company at Mr. Nader's request, and the reason for its
circuitous path could not be determined.

Documents also appear to show that lawyers for Mr. Broidy discussed with
him a possible agreement to share with Mr. Nader a portion of the
profits from the first round of business his company did with the Saudis
and Emiratis --- an apparent reflection of his integral role in helping
the company, Circinus, negotiate for the lucrative security contracts.

Image

Mr. Trump hosted Crown Prince Mohammed bin Salman of Saudi Arabia at the
White House on Tuesday.Credit...Doug Mills/The New York Times

In his statement, Mr. Broidy said Mr. Nader ``is not a shareholder,
officer, director or employee of any of my companies.''

``He has not been paid any origination fee or any other fees in
connection with these matters,'' he said.

\hypertarget{influential-links}{%
\subsection{Influential Links}\label{influential-links}}

Months later, as Mr. Broidy was preparing for an Oval Office meeting
with Mr. Trump, Mr. Nader pressed him to try to line up a private
meeting outside the White House between Mr. Trump and the leader of the
United Arab Emirates, Crown Prince Mohammed bin Zayed, whom he referred
to as ``Friend.''

``Tell him that Friend would like to come ASAP to meet you SOONEST out
of official site, in New Jersey'' or Camp David, the presidential
retreat in Maryland, Mr. Nader wrote to Mr. Broidy on Oct. 1.

``Again, Again and Again, please try to be the ONE to fix a date for
Friend while you are there if at all possible,'' he added.

Six days later, Mr. Broidy did just that, repeatedly pressing Mr. Trump
to meet with the crown prince in a ``quiet'' setting outside the White
House --- perhaps in New York or New Jersey --- according to a detailed
report on the meeting that Mr. Broidy sent to Mr. Nader shortly after.
Mr. Trump's national security adviser, Lt. Gen. H. R. McMaster, blocked
the request, Mr. Broidy reported.

In a memorandum to Mr. Nader about the Oval Office meeting on Oct. 6,
Mr. Broidy reported that he personally urged Mr. Trump to fire Mr.
Tillerson, whom the Saudis and Emiratis saw as insufficiently tough on
Iran and Qatar.

Image

A portion of a memo that Elliott Broidy sent to George Nader, recounting
his meeting with President Trump in the Oval Office.

Later in the fall, Mr. Nader complained that the Secret Service had
stopped him from getting his picture taken with Mr. Trump at a
fund-raiser. Although the reasons he was kept at bay from the president
are unclear, Mr. Nader pleaded guilty in 1991 to a federal child
pornography charge and served six months at a halfway house after
videotapes were found in his luggage when he arrived at Washington
Dulles International Airport from a trip to Germany, according to court
records released last week. In 2003, he received a one-year prison
sentence in the Czech Republic after he was convicted there of 10 cases
of sexually abusing minors,
\href{http://hosted2.ap.org/PASHA/a5050f4ad4f44dafab85bb41a15281cf/Article_2018-03-15-US-Trump-Russia-Probe/id-93db9ed0b5054340b9acc851355ce56b}{The
Associated Press reported}, citing a court spokeswoman.

Mr. Broidy was puzzled by the Secret Service's objections. Mr. Nader, in
his capacity as an adviser to the ruler of United Arab Emirates, had met
several times with senior administration officials in the White House
during Mr. Trump's first weeks in office.

Mr. Broidy was apparently able to deliver: On Dec. 14, he emailed Mr.
Nader his photograph grinning next to Mr. Trump.

Despite the close relations between the White House and the two gulf
nations, there have been occasional hiccups, and in January, Mr. Nader
twice emailed his friend with another delicate request: The leader of
the U.A.E. asked that Mr. Trump call the crown prince of Saudi Arabia to
try to smooth over potential bad feelings created by the book ``Fire and
Fury,'' by Michael Wolff. It portrayed the president's views of the
Saudi prince in an unflattering light, Mr. Nader wrote.

``See what you can trigger and do and we can discuss more in person,''
Mr. Nader wrote, reiterating once again the ``genuine desire'' of the
ruler of the United Arab Emirates to meet alone with Mr. Trump.

Days later, Mr. Nader wrote to his friend that he was looking forward to
an upcoming trip to the United States. Mr. Broidy was arranging for him
to attend a gala dinner at Mar-a-Lago, the president's Florida estate,
to celebrate the anniversary of Mr. Trump's inauguration, and the two
men were considering a trip to Saudi Arabia to try to sell the kingdom's
young and powerful crown prince on a \$650 million contract with Mr.
Broidy's security company.

But those grand plans were interrupted. It was on that trip to the
United States that, as he touched down at Dulles Airport, Mr. Nader was
greeted by F.B.I. agents working for Mr. Mueller.

Advertisement

\protect\hyperlink{after-bottom}{Continue reading the main story}

\hypertarget{site-index}{%
\subsection{Site Index}\label{site-index}}

\hypertarget{site-information-navigation}{%
\subsection{Site Information
Navigation}\label{site-information-navigation}}

\begin{itemize}
\tightlist
\item
  \href{https://help.nytimes.com/hc/en-us/articles/115014792127-Copyright-notice}{©~2020~The
  New York Times Company}
\end{itemize}

\begin{itemize}
\tightlist
\item
  \href{https://www.nytco.com/}{NYTCo}
\item
  \href{https://help.nytimes.com/hc/en-us/articles/115015385887-Contact-Us}{Contact
  Us}
\item
  \href{https://www.nytco.com/careers/}{Work with us}
\item
  \href{https://nytmediakit.com/}{Advertise}
\item
  \href{http://www.tbrandstudio.com/}{T Brand Studio}
\item
  \href{https://www.nytimes.com/privacy/cookie-policy\#how-do-i-manage-trackers}{Your
  Ad Choices}
\item
  \href{https://www.nytimes.com/privacy}{Privacy}
\item
  \href{https://help.nytimes.com/hc/en-us/articles/115014893428-Terms-of-service}{Terms
  of Service}
\item
  \href{https://help.nytimes.com/hc/en-us/articles/115014893968-Terms-of-sale}{Terms
  of Sale}
\item
  \href{https://spiderbites.nytimes.com}{Site Map}
\item
  \href{https://help.nytimes.com/hc/en-us}{Help}
\item
  \href{https://www.nytimes.com/subscription?campaignId=37WXW}{Subscriptions}
\end{itemize}
