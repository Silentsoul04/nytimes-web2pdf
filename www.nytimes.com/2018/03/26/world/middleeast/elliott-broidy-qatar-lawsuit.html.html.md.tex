Sections

SEARCH

\protect\hyperlink{site-content}{Skip to
content}\protect\hyperlink{site-index}{Skip to site index}

\href{https://www.nytimes.com/section/world/middleeast}{Middle East}

\href{https://myaccount.nytimes.com/auth/login?response_type=cookie\&client_id=vi}{}

\href{https://www.nytimes.com/section/todayspaper}{Today's Paper}

\href{/section/world/middleeast}{Middle East}\textbar{}Trump Fund-Raiser
Files Hacking Lawsuit Against Qatar

\url{https://nyti.ms/2pMMe7R}

\begin{itemize}
\item
\item
\item
\item
\item
\end{itemize}

Advertisement

\protect\hyperlink{after-top}{Continue reading the main story}

Supported by

\protect\hyperlink{after-sponsor}{Continue reading the main story}

\hypertarget{trump-fund-raiser-files-hacking-lawsuit-against-qatar}{%
\section{Trump Fund-Raiser Files Hacking Lawsuit Against
Qatar}\label{trump-fund-raiser-files-hacking-lawsuit-against-qatar}}

\includegraphics{https://static01.nyt.com/images/2018/03/26/reader-center/27BROIDY/27BROIDY-articleLarge.jpg?quality=75\&auto=webp\&disable=upscale}

By \href{https://www.nytimes.com/by/david-d-kirkpatrick}{David D.
Kirkpatrick}

\begin{itemize}
\item
  March 26, 2018
\item
  \begin{itemize}
  \item
  \item
  \item
  \item
  \item
  \end{itemize}
\end{itemize}

LONDON --- Lawyers for Elliott Broidy, a Republican fund-raiser close to
President Trump, on Monday filed a lawsuit accusing the government of
Qatar of hacking into his emails and conspiring with Washington
lobbyists to besmirch his reputation.

The lawsuit is one of the first high-profile attempts to hold a foreign
government accountable in American courts for cyberespionage. It comes
at a time when hacking is becoming an increasingly common tool among a
growing number of states seeking to punish enemies or achieve political
goals.

``This suit is the first of its kind,'' said Lee Wolosky, a lawyer for
Mr. Broidy.

Mr. Broidy, a Los Angeles investor, has been an antagonist of Qatar in
Washington. He has accused it of supporting Islamist extremism, and he
has provided millions of dollars in financial support for think-tank
conferences amplifying those criticisms. He has made the same arguments
to Mr. Trump and Republican lawmakers.

At the same time, Mr. Broidy also owns a defense contractor, Circinus
L.L.C., that in the past year signed a contract worth more than \$200
million with the United Arab Emirates and is pursuing another large
contract with Saudi Arabia. Both countries are engaged in a bitter
dispute with Qatar, the home to a major American military base and vast
natural gas deposits.

Several recent news articles, including three on the front page of The
New York Times, have called attention to the overlap of Mr. Broidy's
political advocacy and his business interests. They describe what appear
to be his promises of access to the Trump administration or
congressional Republicans as he sought lucrative contracts with various
foreign governments.

Most of those articles, including those in The Times, relied in part on
copies of emails from Mr. Broidy's account that were provided to
journalists by an anonymous group critical of his views about the Middle
East.

Representatives of Mr. Broidy immediately suspected Qatar of stealing
his emails, in part because the private emails of at least one other
high-profile foe of Qatar --- Yousef al-Otaiba, the Emirati ambassador
to Washington --- have also been hacked and disseminated to journalists
in a similar fashion. The hack required a level of resources and
sophistication that suggested a state was responsible.

``This is a case about a hostile intelligence operation undertaken by a
foreign nation on the territory of the United States against successful,
influential United States citizens,'' the lawyers for Mr. Broidy charged
in a lawsuit filed in United States District Court for the Central
District of California.

In a statement, a spokesman for Qatar said the suit was ``without merit
or fact.'' The spokesman, Jassim al-Thani, of the Qatari Embassy in
Washington, called Mr. Broidy's lawsuit ``a transparent attempt to
divert attention from U.S. media reports about his activities.''

The lawsuit charges that the attack began last Dec. 27, when Mr.
Broidy's wife, Robin Rosenzweig, received an email that appeared to be a
security alert from Google. She entered her password as the alert
requested. It turned out to be a phishing attack, according to the
lawsuit, and the information she provided was used to get access to her
account, Mr. Broidy's and that of his company, Broidy Capital
Management.

After the emails began appeared in the news media, Mr. Broidy retained a
team of cyberforensic experts, including at least one former American
intelligence official. According to the lawsuit, their initial analysis
indicated that the attacks appeared to originate from computer servers
in Britain and the Netherlands, but the researchers later concluded that
the addresses of those servers had been used to mask another point of
origin.

``A more thorough review of the server data'' showed that for a brief
time on one day --- Feb. 14, 2018 --- ``problems with the attackers
obfuscation techniques'' had ``revealed that the attack originated in
Qatar.''

The lawsuit also claims that a Republican lobbyist, Nicolas D. Muzin of
Stonington Strategies, conspired with Qatar to exploit the hacked emails
to damage Mr. Broidy's reputation. Stonington Strategies is registered
as a foreign agent of Qatar, and the lawsuit says that Qatar pays him
\$300,000 a month. Qatar spent nearly \$5 million on Washington
lobbyists and media relations during the six months that ended last
October, according to the Center for Responsive Politics.

Mr. Muzin has reportedly courted Jewish leaders and others by offering
them trips to Doha, the capital of Qatar. ``Starting last year, the
State of Qatar, Muzin, and other foreign agents conspired in a strategic
campaign to retaliate against and discredit Plaintiff Broidy,'' the
lawsuit claims. It specifically accused Qatar of orchestrating the
hacking after Mr. Muzin ``identified Plaintiff Broidy as an individual
who was opposing the State of Qatar's efforts to improve its image and
relationships in Washington, D.C. and who was aligned with its regional
rivals, the UAE and Saudi Arabia.''

In a statement, Mr. Muzin said, ``Mr. Broidy's lawsuit is an obvious
attempt to draw attention away from his controversial work, and is as
flimsy as the promises he reportedly made to his clients.''

Advertisement

\protect\hyperlink{after-bottom}{Continue reading the main story}

\hypertarget{site-index}{%
\subsection{Site Index}\label{site-index}}

\hypertarget{site-information-navigation}{%
\subsection{Site Information
Navigation}\label{site-information-navigation}}

\begin{itemize}
\tightlist
\item
  \href{https://help.nytimes.com/hc/en-us/articles/115014792127-Copyright-notice}{©~2020~The
  New York Times Company}
\end{itemize}

\begin{itemize}
\tightlist
\item
  \href{https://www.nytco.com/}{NYTCo}
\item
  \href{https://help.nytimes.com/hc/en-us/articles/115015385887-Contact-Us}{Contact
  Us}
\item
  \href{https://www.nytco.com/careers/}{Work with us}
\item
  \href{https://nytmediakit.com/}{Advertise}
\item
  \href{http://www.tbrandstudio.com/}{T Brand Studio}
\item
  \href{https://www.nytimes.com/privacy/cookie-policy\#how-do-i-manage-trackers}{Your
  Ad Choices}
\item
  \href{https://www.nytimes.com/privacy}{Privacy}
\item
  \href{https://help.nytimes.com/hc/en-us/articles/115014893428-Terms-of-service}{Terms
  of Service}
\item
  \href{https://help.nytimes.com/hc/en-us/articles/115014893968-Terms-of-sale}{Terms
  of Sale}
\item
  \href{https://spiderbites.nytimes.com}{Site Map}
\item
  \href{https://help.nytimes.com/hc/en-us}{Help}
\item
  \href{https://www.nytimes.com/subscription?campaignId=37WXW}{Subscriptions}
\end{itemize}
