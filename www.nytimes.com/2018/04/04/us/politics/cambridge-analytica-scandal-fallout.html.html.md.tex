Sections

SEARCH

\protect\hyperlink{site-content}{Skip to
content}\protect\hyperlink{site-index}{Skip to site index}

\href{https://www.nytimes.com/section/politics}{Politics}

\href{https://myaccount.nytimes.com/auth/login?response_type=cookie\&client_id=vi}{}

\href{https://www.nytimes.com/section/todayspaper}{Today's Paper}

\href{/section/politics}{Politics}\textbar{}Cambridge Analytica and
Facebook: The Scandal and the Fallout So Far

\url{https://nyti.ms/2GBQ4Lm}

\begin{itemize}
\item
\item
\item
\item
\item
\end{itemize}

Advertisement

\protect\hyperlink{after-top}{Continue reading the main story}

Supported by

\protect\hyperlink{after-sponsor}{Continue reading the main story}

\hypertarget{cambridge-analytica-and-facebook-the-scandal-and-the-fallout-so-far}{%
\section{Cambridge Analytica and Facebook: The Scandal and the Fallout
So
Far}\label{cambridge-analytica-and-facebook-the-scandal-and-the-fallout-so-far}}

Revelations that digital consultants to the Trump campaign misused the
data of millions of Facebook users set off a furor on both sides of the
Atlantic. This is how The Times covered it.

\includegraphics{https://static01.nyt.com/images/2018/03/21/business/00cambridgestory6/merlin_135756423_8ac3aeb6-7781-4912-b9e5-fd7210e35170-articleLarge.jpg?quality=75\&auto=webp\&disable=upscale}

By \href{https://www.nytimes.com/by/nicholas-confessore}{Nicholas
Confessore}

\begin{itemize}
\item
  April 4, 2018
\item
  \begin{itemize}
  \item
  \item
  \item
  \item
  \item
  \end{itemize}
\end{itemize}

In March, The New York Times, working with The Observer of London and
The Guardian, obtained a cache of documents from inside Cambridge
Analytica, the data firm principally owned by the right-wing donor
\href{https://www.nytimes.com/2016/08/19/us/politics/robert-mercer-donald-trump-donor.html}{Robert
Mercer}. The documents proved that the firm, where the former Trump aide
\href{https://www.nytimes.com/2017/03/31/business/dealbook/how-some-top-trump-aides-made-their-fortunes.html}{Stephen
K. Bannon} was a board member, used data improperly obtained from
Facebook to build voter profiles. The news put Cambridge under
investigation and thrust Facebook into its biggest crisis ever. Here's a
guide to our coverage.

March 17

\hypertarget{harvesting-data-and-testing-election-law}{%
\subsection{Harvesting data and testing election
law}\label{harvesting-data-and-testing-election-law}}

The Times
\href{https://www.nytimes.com/2018/03/17/us/politics/cambridge-analytica-trump-campaign.html}{reported
that in 2014 contractors and employees} of Cambridge Analytica, eager to
sell psychological profiles of American voters to political campaigns,
acquired the private Facebook data of tens of millions of users --- the
largest known leak in Facebook history.

There was more. Our article first showed how Cambridge received warnings
from its own lawyer, Laurence Levy, as it employed European and Canadian
citizens on campaigns, potentially violating American election law. The
Times also found that tranches of raw data still existed beyond
Facebook's control.

\emph{Read}
\href{https://www.nytimes.com/2018/03/20/technology/facebook-cambridge-behavior-model.html}{\emph{how
researchers may have used your Facebook ``likes''}} \emph{to predict
your political views.}

\hypertarget{what-was-the-russia-link}{%
\subsection{What was the Russia link?}\label{what-was-the-russia-link}}

In
\href{https://www.nytimes.com/2018/03/17/us/politics/cambridge-analytica-russia.html}{a
companion piece}, The Times reported that people at Cambridge Analytica
and its British affiliate, the SCL Group, were in contact with
executives from Lukoil, the Kremlin-linked oil giant, as Cambridge built
its Facebook-derived profiles. Lukoil was interested in the ways data
was used to target American voters, according to two former company
insiders. SCL and Lukoil denied that the talks were political in nature
and said the oil giant never became a client.

\includegraphics{https://static01.nyt.com/images/2018/03/18/autossell/00cambridgestory5/merlin_85870796_417b83c0-39de-4e4f-a57c-a447901d734e-articleLarge.jpg?quality=75\&auto=webp\&disable=upscale}

March 18

\hypertarget{anger-on-both-sides-of-the-atlantic}{%
\subsection{Anger on both sides of the
Atlantic}\label{anger-on-both-sides-of-the-atlantic}}

The articles
\href{https://www.nytimes.com/2018/03/18/us/cambridge-analytica-facebook-privacy-data.html}{drew
an instant response} in Washington, where lawmakers demanded that Mark
Zuckerberg, Facebook's chief executive, testify before Congress.
Democrats looking into Russian interference in the 2016 election ---
already interested in Cambridge's role in providing analytics to the
Trump campaign --- said they would seek an investigation into the leak.
They were echoed by lawmakers in Britain investigating Cambridge
Analytica's role in disinformation and the country's referendum to leave
the European Union.

March 20

\hypertarget{bribes-entrapment-and-a-suspension}{%
\subsection{Bribes, entrapment and a
suspension}\label{bribes-entrapment-and-a-suspension}}

The Times reported that Cambridge
\href{https://www.nytimes.com/2018/03/20/world/europe/cambridge-analytica-ceo-suspended.html}{suspended}
its chief executive, Alexander Nix, after a British television channel
\href{https://www.nytimes.com/2018/03/19/us/cambridge-analytica-alexander-nix.html}{released
an undercover video} in which he suggested that the company had used
seduction and bribery to entrap politicians and influence foreign
elections. In Washington, the Federal Trade Commission
\href{https://www.nytimes.com/2018/03/20/business/ftc-facebook-privacy-investigation.html}{moved
to investigate} whether Facebook had violated an early agreement to
safeguard user data.

Image

Alexander Nix, the chief executive of Cambridge Analytica, left his
offices through the back door the day he was suspended.Credit...Facundo
Arrizabalaga/EPA, via Shutterstock

March 21

\hypertarget{facebook-faces-a-reckoning}{%
\subsection{Facebook faces a
reckoning}\label{facebook-faces-a-reckoning}}

The Times reported on a growing number of Facebook users, including the
singer Cher,
\href{https://www.nytimes.com/2018/03/21/technology/users-abandon-facebook.html}{deleting
their accounts} --- and broke news of the
\href{https://www.nytimes.com/2018/03/19/technology/facebook-alex-stamos.html}{departure
of Facebook's top security official}, who had clashed with other
executives on how to handling discontent over the platform's role in
spreading disinformation. The hashtag \#DeleteFacebook began trending on
Twitter.

After
\href{https://www.nytimes.com/2018/03/21/technology/mark-zuckerberg-facebook.html}{remaining
silent for days} --- spurring another social media hashtag,
\#WheresZuck? --- Mr. Zuckerberg
\href{https://www.nytimes.com/2018/03/21/technology/mark-zuckerberg-q-and-a.html}{spoke
with The Times} about steps Facebook was taking to address users' anger.

\emph{Our columnist Brian X. Chen}
\href{https://www.nytimes.com/2018/03/19/technology/personaltech/protect-yourself-on-facebook.html}{\emph{explains
how to protect your Facebook data}}\emph{.}

March 23

\hypertarget{new-trump-adviser-old-cambridge-connection}{%
\subsection{New Trump adviser, old Cambridge
connection}\label{new-trump-adviser-old-cambridge-connection}}

As Facebook reeled, The Times delved into the relationship between
\href{https://www.nytimes.com/2018/03/23/us/politics/bolton-cambridge-analyticas-facebook-data.html}{Cambridge
Analytica and John Bolton}, the conservative hawk
\href{https://www.nytimes.com/2018/03/22/us/politics/hr-mcmaster-trump-bolton.html}{named
national security adviser by President Trump}. The Times broke the news
that in 2014, Cambridge provided Mr. Bolton's ``super PAC'' with early
versions of its Facebook-derived profiles --- the technology's first
large-scale use in an American election.

\emph{What about 2016?}
\href{https://www.nytimes.com/2017/03/06/us/politics/cambridge-analytica.html}{\emph{We
examined}} \emph{the skepticism and evidence around the role Cambridge
claimed it played in Mr. Trump's win.}

Cambridge Analytica helped develop ads for candidates supported by John
Bolton's ``super PAC.''

March 24

\hypertarget{another-look-at-brexit}{%
\subsection{Another look at `Brexit'}\label{another-look-at-brexit}}

The Times and The Observer
\href{https://www.nytimes.com/2018/03/24/world/europe/uk-brexit-vote-leave-shahmir-sanni.html}{reported
allegations} that the 2016 ``Brexit'' campaign used a Cambridge
Analytica contractor to help skirt election spending limits. The story
implicated two senior advisers to Prime Minister Theresa May. Testifying
to Parliament
\href{https://www.nytimes.com/2018/03/27/world/europe/whistle-blower-data-mining-cambridge-analytica.html}{a
few days later}, a former Cambridge employee, Christopher Wylie,
contended that the company helped swing the results in favor of
Britain's withdrawal from the European Union.

March 28

\hypertarget{the-silicon-valley-spy-contractor}{%
\subsection{The Silicon Valley spy
contractor}\label{the-silicon-valley-spy-contractor}}

Image

The London offices of Cambridge Analytica, which had help from at least
one employee at Palantir Technologies, a company founded by the Trump
supporter Peter Thiel.Credit...Chris J Ratcliffe/Getty Images

In another report, The Times showed how an employee at Palantir
Technologies --- an intelligence contractor founded by the Trump backer
and tech investor Peter Thiel ---
\href{https://www.nytimes.com/2018/03/27/us/cambridge-analytica-palantir.html}{helped
Cambridge harvest Facebook data}. The article reported that Palantir and
Cambridge executives briefly considered a formal partnership to work on
political campaigns. Though the deal fell through, a Palantir employee
continued working with Cambridge to figure out how to obtain data for
psychographic profiles. Palantir officials said the employee did so in a
strictly personal capacity.

April 4

\hypertarget{how-many-were-affected}{%
\subsection{How many were affected?}\label{how-many-were-affected}}

The Times originally reported that Cambridge harvested data from over 50
million Facebook users. But at the bottom of a company announcement
about new privacy features, Facebook's chief technology officer, Mike
Schroepfer, issued a new estimate for the number of users who were
affected:
\href{https://www.nytimes.com/2018/04/04/technology/mark-zuckerberg-testify-congress.html}{as
many as 87 million}, most of them in the United States.

\emph{Facebook is responding to users' distrust. Read how the company}
\href{https://www.nytimes.com/2018/03/28/technology/facebook-privacy-security-settings.html}{\emph{introduced
a central privacy page}}\emph{. }

April 8

\hypertarget{you-are-the-product}{%
\subsection{`You are the product'}\label{you-are-the-product}}

Amid the crisis, one set of voices remained notably absent: Facebook
users whose data was harvested. So
\href{https://www.nytimes.com/2018/04/08/us/facebook-users-data-harvested-cambridge-analytica.html?rref=collection\%2Fbyline\%2Fmatthew-rosenberg\&action=click\&contentCollection=undefined\&region=stream\&module=stream_unit\&version=latest\&contentPlacement=1\&pgtype=collection}{The
Times found some}, and their reactions ranged from anger to resigned
annoyance at how tech giants use personal information. As one of the
affected Facebook users put it, ``You are the product on the internet.''

The Times also reported new details on the app used to collect data for
Cambridge Analytica. It was no simple Facebook quiz, as many had
assumed. Rather, it was attached to a lengthy psychology questionnaire
hosted by Qualtrics, a company that manages online surveys. The first
step for those filling out the questionnaire was to grant access to
their Facebook profiles. Once they did, an app then harvested their data
and that of their friends.

April 10

\hypertarget{zuckerberg-speaks-to-lawmakers}{%
\subsection{Zuckerberg speaks to
lawmakers}\label{zuckerberg-speaks-to-lawmakers}}

\includegraphics{https://static01.nyt.com/images/2018/04/11/us/politics/11zuck-highlights/11zuck-highlights-videoSixteenByNineJumbo1600.jpg}

Mr. Zuckerberg made
\href{https://www.nytimes.com/2018/04/10/us/politics/zuckerberg-facebook-senate-hearing.html}{his
first appearance before Congress}, testifying to Senate and House
committees. First up was the Senate, where
\href{https://www.nytimes.com/2018/04/10/us/politics/mark-zuckerberg-testimony.html}{he
faced tough questions} about the company's mishandling of data, and said
Facebook was investigating ``tens of thousands of apps'' to see what
information they harvested.

The next day, he faced
\href{https://www.nytimes.com/2018/04/11/business/zuckerberg-facebook-congress.html}{an
even tougher crowd} in the House. There, the consensus was that social
media technology --- and its potential for abuse --- had far outpaced
Washington, and that Congress may have to step in to close the gap. Even
Mr. Zuckerberg seemed to suggest he could be open to some regulation,
but neither he nor lawmakers seemed sure about how exactly to regulate
the new breed of companies.

\hypertarget{a-setback-for-the-mercers}{%
\subsection{A setback for the Mercers}\label{a-setback-for-the-mercers}}

Image

The conservative donor Robert Mercer invested \$15 million in Cambridge
Analytica, where his daughter Rebekah is a board member.Credit...Patrick
McMullan, via Getty Images

On the same day as the Senate hearing, The Times reported how the
Cambridge furor had
\href{https://www.nytimes.com/2018/04/10/us/politics/mercer-family-cambridge-analytica.html}{impacted
the Mercers}, particularly Mr. Mercer's daughter Rebekah, who leads the
family's political network. Shortly after the scandal broke, a friend of
hers visited Facebook headquarters to plead the case for Cambridge.
Though the Mercers were once Mr. Trump's leading patrons in conservative
politics, their standing in the president's circle has suffered.

April 19

\hypertarget{who-audits-the-auditors}{%
\subsection{Who audits the auditors?}\label{who-audits-the-auditors}}

The Times
\href{https://www.nytimes.com/2018/04/19/technology/facebook-audit-cambridge-analytica.html}{reported
on a series of assessments} of Facebook's privacy programs, conducted by
the consulting firm PwC on behalf of federal regulators. In the
assessments, mandated by a 2011 consent decree, PwC deemed Facebook's
internal controls effective at protecting users' privacy --- even after
the social media giant lost control of a huge trove of user data that
was improperly obtained by Cambridge Analytica.

\begin{center}\rule{0.5\linewidth}{\linethickness}\end{center}

Matthew Rosenberg contributed reporting.

Advertisement

\protect\hyperlink{after-bottom}{Continue reading the main story}

\hypertarget{site-index}{%
\subsection{Site Index}\label{site-index}}

\hypertarget{site-information-navigation}{%
\subsection{Site Information
Navigation}\label{site-information-navigation}}

\begin{itemize}
\tightlist
\item
  \href{https://help.nytimes.com/hc/en-us/articles/115014792127-Copyright-notice}{©~2020~The
  New York Times Company}
\end{itemize}

\begin{itemize}
\tightlist
\item
  \href{https://www.nytco.com/}{NYTCo}
\item
  \href{https://help.nytimes.com/hc/en-us/articles/115015385887-Contact-Us}{Contact
  Us}
\item
  \href{https://www.nytco.com/careers/}{Work with us}
\item
  \href{https://nytmediakit.com/}{Advertise}
\item
  \href{http://www.tbrandstudio.com/}{T Brand Studio}
\item
  \href{https://www.nytimes.com/privacy/cookie-policy\#how-do-i-manage-trackers}{Your
  Ad Choices}
\item
  \href{https://www.nytimes.com/privacy}{Privacy}
\item
  \href{https://help.nytimes.com/hc/en-us/articles/115014893428-Terms-of-service}{Terms
  of Service}
\item
  \href{https://help.nytimes.com/hc/en-us/articles/115014893968-Terms-of-sale}{Terms
  of Sale}
\item
  \href{https://spiderbites.nytimes.com}{Site Map}
\item
  \href{https://help.nytimes.com/hc/en-us}{Help}
\item
  \href{https://www.nytimes.com/subscription?campaignId=37WXW}{Subscriptions}
\end{itemize}
