Sections

SEARCH

\protect\hyperlink{site-content}{Skip to
content}\protect\hyperlink{site-index}{Skip to site index}

\href{https://www.nytimes.com/section/us}{U.S.}

\href{https://myaccount.nytimes.com/auth/login?response_type=cookie\&client_id=vi}{}

\href{https://www.nytimes.com/section/todayspaper}{Today's Paper}

\href{/section/us}{U.S.}\textbar{}Daniel Akaka, Former Democratic
Senator From Hawaii, Dies at 93

\url{https://nyti.ms/2Eu7V0Y}

\begin{itemize}
\item
\item
\item
\item
\item
\end{itemize}

Advertisement

\protect\hyperlink{after-top}{Continue reading the main story}

Supported by

\protect\hyperlink{after-sponsor}{Continue reading the main story}

\hypertarget{daniel-akaka-former-democratic-senator-from-hawaii-dies-at-93}{%
\section{Daniel Akaka, Former Democratic Senator From Hawaii, Dies at
93}\label{daniel-akaka-former-democratic-senator-from-hawaii-dies-at-93}}

\includegraphics{https://static01.nyt.com/images/2018/04/07/obituaries/07akaka-obit/akaka-obit-3-articleLarge.jpg?quality=75\&auto=webp\&disable=upscale}

By Adam Clymer

\begin{itemize}
\item
  April 6, 2018
\item
  \begin{itemize}
  \item
  \item
  \item
  \item
  \item
  \end{itemize}
\end{itemize}

Former Senator Daniel K. Akaka, a Democrat who represented Hawaii for 36
years in Congress and successfully fought for the belated recognition of
Asians and Asian-Americans who had fought for the United States in World
War II, died on Friday in Honolulu. He was 93.

Jon Yoshimura, the senator's former communications director, confirmed
the death, saying Mr. Akaka had been hospitalized for several months,
The Associated Press reported.

A World War II veteran, Mr. Akaka sponsored legislation in 1996 that led
to a re-evaluation of the service records of Asian-Americans who had
fought in the 442nd Regimental Combat Team and the 100th Infantry
Battalion during the war.

As a result, almost two dozen Medals of Honor, the military's highest
award, were ultimately bestowed belatedly, some posthumously, on
Asian-American veterans, most of them of Japanese heritage. Only one had
been awarded during the war itself.

\includegraphics{https://static01.nyt.com/images/2018/04/06/obituaries/akaka2/merlin_136484709_ab921f6f-9398-453b-9c9e-953198cfaa72-articleLarge.jpg?quality=75\&auto=webp\&disable=upscale}

After a White House
\href{https://www.nytimes.com/2000/05/14/us/21-asian-americans-receive-medal-of-honor.html}{awards-presentation
ceremony} led by President Bill Clinton in 2000, Senator Akaka said the
medals had dispelled apparent wartime discrimination against
Asian-American military personnel.

The most prominent recipient was Senator Daniel K. Inouye, Mr. Akaka's
much better-known colleague --- and Hawaii's senior senator --- for 22
years in the Senate. Mr. Inouye,
\href{https://www.nytimes.com/2012/12/18/us/daniel-inouye-hawaiis-quiet-voice-of-conscience-in-senate-dies-at-88.html}{who
died in 2012,} had lost his right arm while serving with the 442nd in
Europe.

Senator Akaka also successfully pursued legislation that provided
onetime compensation for members of the
\href{https://www.army.mil/article/39797/the_us_armys_philippine_scouts}{Phillipine
Scouts}, an American-led unit of mostly Filipino and Filipino-American
recruits who fought alongside United States troops but did not qualify
for Veterans Administration benefits.

And he secured a formal apology for the United States's role in the
overthrow of
\href{https://www.smithsonianmag.com/smart-news/five-things-know-about-liliuokalani-last-queen-hawaii-180967155/}{Queen
Lili'uokalani} of Hawaii in 1893 as well as a transfer of land that the
federal government had taken.

Image

Senator Daniel K. Akaka on Capitol Hill in 2011.Credit...Alex
Brandon/Associated Press

But he failed in repeated legislative efforts to have native Hawaiians
recognized as an indigenous people so that they might receive federal
benefits similar to those provided to American Indians and natives of
Alaska.

During his Senate years Mr. Akaka had stints as chairman of its
Committee on Veterans Affairs and of its Committee on Indian Affairs.

Mr. Akaka was an outspoken critic of the war in Iraq. On March 17, 2003,
three days before the United States attacked that country, he warned the
Senate:

``If we pursue our current path, we will have a war lacking in many
things essential to achieving complete success. It will be a war without
broad international support, without sufficient planning for
post-conflict reconstruction and stability, without a definite exit time
and strategy, and without a firm price tag.

``Moreover,'' he continued, ``it will be a war with serious
ramifications for our long-term readiness capabilities for homeland
security and for managing other crises.''

A steadfast liberal on most issues, he was known as a champion of
federal workers, complaining that his Senate colleagues too often
denigrated them and cheerfully froze their pay.

He chaired a Senate subcommittee on the federal work force and was the
chief sponsor of the 2012 Whistleblower Protection Act, which provided
safeguards against retaliation to federal workers who report waste,
fraud and abuse.

Daniel Kahikina Akaka was born in Honolulu on Sept. 11, 1924, the
youngest of eight children. His father was of Chinese and Hawaiian
descent; his mother was Hawaiian. He attended public schools.

After service with the Army Corps of Engineers, he graduated from the
University of Hawaii in 1952 with a degree in education and taught
music, social studies and math in elementary, middle and high schools.
He later became a school principal and earned a master's degree.

Image

Mr. Akaka arriving in Honolulu on Air Force One in 2010 with President
Barack Obama. The senator was a steadfast liberal on most
issues.Credit...Chris Carlson/Associated Press

After Hawaii was admitted into the union in 1959, he was an official in
the state's Department of Education and was named director of the Hawaii
Office of Economic Opportunity, an antipoverty program.

Mr. Akaka was first elected to the House in 1976 and easily re-elected
afterward. In 1990 he was appointed to fill a Senate vacancy caused by
the death of Spark Matsunaga. He was elected that fall and re-elected in
1994, 2000 and 2006. He announced in March 2011 that he would not run
again in 2012.

Mr. Akaka, who lived in Honolulu, is survived by his wife, Mary Mildred
Chong, whom he married in 1948; a daughter, Millannie Akaka Mattson;
four sons, Daniel Jr., Gerard, Alan and Nicholas; and many grandchildren
and great-grandchildren.

While he was never known as a key lawmaker, Mr. Akaka was familiar to
watchers of C-Span: his name came first whenever the Senate roll was
called and, in his early years, he relished presiding over that body, a
duty many of his colleagues regarded as tedious.

In 1992, the Senate presented him with its Golden Gavel Award for
presiding for at least 100 hours.

``I really was proud of being able to chair the Senate floor over the
years and really looked forward to it,'' he said in a 2011 interview for
this obituary.

Even in his final years, he left instructions with the Democratic
cloakroom that he would preside in a pinch, saying, ``Any time you can't
find somebody, call me.''

Advertisement

\protect\hyperlink{after-bottom}{Continue reading the main story}

\hypertarget{site-index}{%
\subsection{Site Index}\label{site-index}}

\hypertarget{site-information-navigation}{%
\subsection{Site Information
Navigation}\label{site-information-navigation}}

\begin{itemize}
\tightlist
\item
  \href{https://help.nytimes.com/hc/en-us/articles/115014792127-Copyright-notice}{©~2020~The
  New York Times Company}
\end{itemize}

\begin{itemize}
\tightlist
\item
  \href{https://www.nytco.com/}{NYTCo}
\item
  \href{https://help.nytimes.com/hc/en-us/articles/115015385887-Contact-Us}{Contact
  Us}
\item
  \href{https://www.nytco.com/careers/}{Work with us}
\item
  \href{https://nytmediakit.com/}{Advertise}
\item
  \href{http://www.tbrandstudio.com/}{T Brand Studio}
\item
  \href{https://www.nytimes.com/privacy/cookie-policy\#how-do-i-manage-trackers}{Your
  Ad Choices}
\item
  \href{https://www.nytimes.com/privacy}{Privacy}
\item
  \href{https://help.nytimes.com/hc/en-us/articles/115014893428-Terms-of-service}{Terms
  of Service}
\item
  \href{https://help.nytimes.com/hc/en-us/articles/115014893968-Terms-of-sale}{Terms
  of Sale}
\item
  \href{https://spiderbites.nytimes.com}{Site Map}
\item
  \href{https://help.nytimes.com/hc/en-us}{Help}
\item
  \href{https://www.nytimes.com/subscription?campaignId=37WXW}{Subscriptions}
\end{itemize}
