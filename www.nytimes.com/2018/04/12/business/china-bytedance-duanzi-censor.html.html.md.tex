Sections

SEARCH

\protect\hyperlink{site-content}{Skip to
content}\protect\hyperlink{site-index}{Skip to site index}

\href{https://www.nytimes.com/section/business}{Business}

\href{https://myaccount.nytimes.com/auth/login?response_type=cookie\&client_id=vi}{}

\href{https://www.nytimes.com/section/todayspaper}{Today's Paper}

\href{/section/business}{Business}\textbar{}Horns Honk, and Censors in
China Get a Headache

\url{https://nyti.ms/2JGTTww}

\begin{itemize}
\item
\item
\item
\item
\item
\end{itemize}

Advertisement

\protect\hyperlink{after-top}{Continue reading the main story}

Supported by

\protect\hyperlink{after-sponsor}{Continue reading the main story}

\hypertarget{horns-honk-and-censors-in-china-get-a-headache}{%
\section{Horns Honk, and Censors in China Get a
Headache}\label{horns-honk-and-censors-in-china-get-a-headache}}

\includegraphics{https://static01.nyt.com/images/2018/05/12/world/12china-honk-1/merlin_136774878_6c7d959b-86a2-4e25-a43f-99346feb1fba-articleLarge.jpg?quality=75\&auto=webp\&disable=upscale}

By \href{https://www.nytimes.com/by/raymond-zhong}{Raymond Zhong},
\href{https://www.nytimes.com/by/paul-mozur}{Paul Mozur} and Iris Zhao

\begin{itemize}
\item
  April 12, 2018
\item
  \begin{itemize}
  \item
  \item
  \item
  \item
  \item
  \end{itemize}
\end{itemize}

\href{https://cn.nytimes.com/business/20180413/china-bytedance-duanzi-censor/}{阅读简体中文版}\href{https://cn.nytimes.com/business/20180413/china-bytedance-duanzi-censor/zh-hant/}{閱讀繁體中文版}

SHANGHAI --- If you were driving in China recently, you might have
gotten in trouble had you tried honking your car horn like this:

Beep.

(Pause.)

Beep, beep.

The pattern is a secret code of sorts for loyal users of two Chinese
social media apps to identify themselves. Honk the signal while idling
at a red light, and if you hear it in response, then you know a fellow
fan is near.

This week, though, China's top media regulator closed one of those apps.
Officially, the app,
\href{https://www.nytimes.com/2018/04/11/technology/china-toutiao-bytedance-censor.html}{Neihan
Duanzi, was shut down} for hosting ``vulgar'' jokes and videos. But it
and another app, Douyin, which helps users make goofy music videos, have
brought together legions of fans who make themselves known to one
another in the real world.

That has led some to wonder whether the platform's tightknit user
community, with its own subculture and obscure vocabulary, had angered
the authorities. China's ruling Communist Party has a history of
cracking down on groups that seek to organize citizens outside its
sphere of control.

Users of both apps put decals with the apps' names on their car windows.
They hold meet-ups where they chant invented slogans --- ``Sky king
covers earth tiger, stewed chicken with mushrooms!'' one goes --- and do
things like arrange their cars to
\href{http://news.163.com/18/0412/02/DF5L0HNN000187R2.html}{spell the
name of their city}.

They also honk rhythmically at crowded intersections, prompting
\href{http://news.bandao.cn/news_html/201804/20180411/news_20180411_2820484.shtml}{rebukes
from the police} in
\href{http://news.cnwest.com/content/2018-04/07/content_15758620.htm}{several
Chinese cities} recently.

No ideology appears to link the apps' fans, although their online videos
suggest that sports cars and community service are common enthusiasms.

In one clip, a column of cars parades down an empty highway, headlights
flashing.

In another, a bronze sports car drifts and does doughnuts at an
abandoned intersection.

One popular Chinese variety show this year also chronicled some of the
good acts done by the fans. In China's sparsely populated west, one
group pitched in to help the elderly and entertain children. On Chinese
social media, enthusiasts from other parts of the country posted photos
after donating blood and sending secondhand goods to remote schools.

China has undergone dizzying social change over the past decades. The
internet has become a place for many people to seek out personal
connections and community.

``I'm truly sad. My friend and I both cried while talking about it,''
Liu Wei, 23, an aspiring R\&B composer in Shanghai, said of Duanzi's
closing. He had been a user of the app for more than five years.

``So many years of love,'' Mr. Liu said. ``It's like I lost a brother.''

Bytedance, the company behind both Duanzi and Douyin, has become one of
the most highly valued technology start-ups in the world. But it has to
manage the material on its platforms carefully in order
\href{https://www.nytimes.com/2018/01/02/business/china-toutiao-censorship.html}{to
stay on the right side of China's censors}. Two more of the company's
popular platforms have been
\href{https://www.nytimes.com/2018/04/06/technology/china-censor-teen-moms.html}{pulled
from app stores} in the past week.

Zhang Yiming, Bytedance's founder and chief executive, apologized in a
statement for failing to adequately police content, and for the
company's failure to respect ``core socialist values.''

``As a start-up developing rapidly in the wake of the 18th National
Congress, we understand deeply that our rapid development was an
opportunity afforded to us by this great era,'' Mr. Zhang wrote,
referring to the 2012 Communist Party confab at which President Xi
Jinping took office.

``I thank this era,'' Mr. Zhang wrote. ``I thank the historic
opportunity of China's economic reform and opening-up. I thank the
support the government has given to the technology industry's
development.''

China's highest media regulator, the State Administration of Radio and
Television, did not respond to a faxed request for comment about whether
Duanzi's fans were a factor in the app being shut down.

But in recent days, the authorities in China have instructed news
outlets to suppress information about fan gatherings, according to
censorship orders reviewed by The New York Times. Any mention of
``revolt'' in connection with Duanzi fans has been ordered scrubbed.
Outlets have also been told to censor photos, videos and articles
attacking the media regulator or calling for protests.

``It's almost like there's a war on humor and fun,'' said Joshua
Rosenzweig, a Hong Kong-based adviser at Amnesty International. ``And
the only things that are allowed are the things that fit into the
official propaganda messages: the
\href{https://www.nytimes.com/2017/10/13/sunday-review/xi-jinping-china.html}{`China
Dream,'} the
\href{https://www.nytimes.com/2017/05/13/business/china-railway-one-belt-one-road-1-trillion-plan.html}{`Belt
and Road.'} It has to be connected to what the party wants people to be
concerned about.''

Internet crackdowns are nothing new in China. But the latest seems more
specifically aimed at making sure there is not too much online that
might vie with the government's messaging for people's attention, Mr.
Rosenzweig said.

``They want to do more, I think, to proactively shape people's thinking
through the media,'' he said, ``and they can't have this kind of
competition.''

Advertisement

\protect\hyperlink{after-bottom}{Continue reading the main story}

\hypertarget{site-index}{%
\subsection{Site Index}\label{site-index}}

\hypertarget{site-information-navigation}{%
\subsection{Site Information
Navigation}\label{site-information-navigation}}

\begin{itemize}
\tightlist
\item
  \href{https://help.nytimes.com/hc/en-us/articles/115014792127-Copyright-notice}{©~2020~The
  New York Times Company}
\end{itemize}

\begin{itemize}
\tightlist
\item
  \href{https://www.nytco.com/}{NYTCo}
\item
  \href{https://help.nytimes.com/hc/en-us/articles/115015385887-Contact-Us}{Contact
  Us}
\item
  \href{https://www.nytco.com/careers/}{Work with us}
\item
  \href{https://nytmediakit.com/}{Advertise}
\item
  \href{http://www.tbrandstudio.com/}{T Brand Studio}
\item
  \href{https://www.nytimes.com/privacy/cookie-policy\#how-do-i-manage-trackers}{Your
  Ad Choices}
\item
  \href{https://www.nytimes.com/privacy}{Privacy}
\item
  \href{https://help.nytimes.com/hc/en-us/articles/115014893428-Terms-of-service}{Terms
  of Service}
\item
  \href{https://help.nytimes.com/hc/en-us/articles/115014893968-Terms-of-sale}{Terms
  of Sale}
\item
  \href{https://spiderbites.nytimes.com}{Site Map}
\item
  \href{https://help.nytimes.com/hc/en-us}{Help}
\item
  \href{https://www.nytimes.com/subscription?campaignId=37WXW}{Subscriptions}
\end{itemize}
