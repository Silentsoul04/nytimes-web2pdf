Sections

SEARCH

\protect\hyperlink{site-content}{Skip to
content}\protect\hyperlink{site-index}{Skip to site index}

\href{https://myaccount.nytimes.com/auth/login?response_type=cookie\&client_id=vi}{}

\href{https://www.nytimes.com/section/todayspaper}{Today's Paper}

5 New York Times Writers on What They Got Right and Wrong in the Early
'80s

\href{https://nyti.ms/2IRQFoQ}{https://nyti.ms/2IRQFoQ}

\begin{itemize}
\item
\item
\item
\item
\item
\end{itemize}

Advertisement

\protect\hyperlink{after-top}{Continue reading the main story}

Supported by

\protect\hyperlink{after-sponsor}{Continue reading the main story}

\hypertarget{5-new-york-times-writers-on-what-they-got-right-and-wrong-in-the-early-80s}{%
\section{5 New York Times Writers on What They Got Right and Wrong in
the Early
'80s}\label{5-new-york-times-writers-on-what-they-got-right-and-wrong-in-the-early-80s}}

What critic hasn't wondered if what she thinks now will endure five, 10
or even 50 years later? Plus: One Times journalist revisits one of the
most controversial stories of the decade.

\href{https://www.nytimes.com/by/thessaly-la-force}{\includegraphics{https://static01.nyt.com/images/2018/12/05/multimedia/author-thessaly-la-force/author-thessaly-la-force-thumbLarge.png}}

Edited by \href{https://www.nytimes.com/by/thessaly-la-force}{Thessaly
La Force}

\begin{itemize}
\item
  April 13, 2018
\item
  \begin{itemize}
  \item
  \item
  \item
  \item
  \item
  \end{itemize}
\end{itemize}

Opinions. They can be strong. And also fleeting. Anyone who has written
a review --- whether for Yelp or for a major newspaper --- understands
this. What critic hasn't wondered if what she thinks now will endure
five, 10 or even 50 years later? Here are four reviews by four critics
and writers for The New York Times who revisited either a restaurant, a
gallery, an album or a play from the early '80s, examining what effect,
if any, the passage of time has had on their opinion of it. Plus: One
Times journalist revisits one of the most controversial stories of the
decade.

\includegraphics{https://static01.nyt.com/images/2018/04/04/t-magazine/04tmag-reconsiderations-slide-T3RJ/04tmag-reconsiderations-slide-T3RJ-articleLarge.jpg?quality=75\&auto=webp\&disable=upscale}

\hypertarget{the-last-surviving-downtown-art-collective}{%
\subsection{The Last Surviving Downtown Art
Collective}\label{the-last-surviving-downtown-art-collective}}

\hypertarget{grace-glueck-was-an-arts-news-writer-for-the-new-york-times-in-the-summer-of-1983-she-started-working-at-the-paper-in-1964}{%
\subsubsection{Grace Glueck was an arts news writer for The New York
Times in the summer of 1983. She started working at the paper in
1964.}\label{grace-glueck-was-an-arts-news-writer-for-the-new-york-times-in-the-summer-of-1983-she-started-working-at-the-paper-in-1964}}

It was the summer of 1983, and I was by then a weekly columnist and
reviewer in the art news department of The Times. I rode my bicycle down
to the East Village to inspect the new art scene that had developed
there. As I approached the area around Tompkins Square Park, I was
assailed by a frenzied scrum of vendors offering products like angel
dust. For nearly two decades I had covered many art events, but never a
drug scene, and I was clueless, thinking maybe it was perfume or candy
they were peddling. But I was wised up at the Kenkeleba gallery, the
first and most important one I visited, where I was warned to store my
bike inside.

The budding art scene I visited back then has long since faded away.
Although the small, tenement-size rooms operated by these galleries
nurtured some of the era's icons, like Jean-Michel Basquiat and Keith
Haring, the spaces were no match for the king-size canvases that many
artists were beginning to produce. And the combative drug scene, worked
by dealers and users around the clock, with an agenda of murder and
mayhem, kept many potential visitors away. But spunk, determination and
a growing push for gentrification have brought major changes.

\href{https://www.facebook.com/kenkeleba.house}{The Kenkeleba}, a
staunch holdover from the 1980s scene (and mentioned in my 1983 story)
deserves plaudits for its role in those changes. Established by Corrine
Jennings, a would-be scenic designer barred from the union for being
black and a woman, and her partner, Joe Overstreet, who is still active
as an artist, the gallery took its name from a West African plant
thought to embody good mojo. Ensconced at 214 East Second Street in a
formerly abandoned building, rundown but roomy, the nonprofit gallery is
supported by New York City's Cultural Affairs Department, private
foundations and contributions from many individuals. Its exhibition
space is generous, and it displays a broad range of art, with emphasis
on African-American, Latino, Asian and Native American artists. To date,
the gallery claims to have shown the work of more than 7,000 exhibitors
(including the black Abstract Expressionist painter Norman Lewis, who
had a show at the Jewish Museum a few years ago; Edward Mitchell
Bannister, a black Canadian-American landscape painter of the 19th
century; Rose Piper, one of the first black female painters to be given
a solo show in New York City; and Arlan Huang, a Chinese-American glass
sculptor and painter.) It also has living and studio space for a small
cadre of working artists. Among its other laudable activities, the
gallery was heavily involved in the campaign that finally led
authorities to oust the junkies from seven abandoned city-owned
buildings. ``We're still here and looking toward an interesting
future,'' Ms. Jennings said.

\emph{``\href{https://www.nytimes.com/1983/06/26/arts/gallery-view-a-gallery-scene-that-pioneers-in-new-territories.html}{A
Gallery Scene That Pioneers in New Territories},'' by Grace Glueck,
published June 26, 1983.}

\begin{center}\rule{0.5\linewidth}{\linethickness}\end{center}

Image

Credit...Weichia Huang

\hypertarget{the-first-time-i-was-left-breathless-by-broadway}{%
\subsection{The First Time I Was Left Breathless By
Broadway}\label{the-first-time-i-was-left-breathless-by-broadway}}

\hypertarget{ben-brantley-has-been-a-chief-theater-critic-for-the-new-york-times-since-1996-in-1983-he-was-working-as-a-fashion-reporter-for-womens-wear-daily-and-just-beginning-to-keep-up-with-broadway}{%
\subsubsection{Ben Brantley has been a chief theater critic for The New
York Times since 1996. In 1983, he was working as a fashion reporter for
Women's Wear Daily, and just beginning to keep up with
Broadway.}\label{ben-brantley-has-been-a-chief-theater-critic-for-the-new-york-times-since-1996-in-1983-he-was-working-as-a-fashion-reporter-for-womens-wear-daily-and-just-beginning-to-keep-up-with-broadway}}

The script, I discovered later, said the 90-minute play that had so
raised my adrenaline level on one time-bending afternoon in 1983 was to
be performed ``relentlessly, without a break.'' And relentless it had
been. Sam Shepard's ``Fool for Love,'' starring Ed Harris and Kathy
Baker, was the sort of show that had you and your date staring at each
other in wonder and exhaustion afterward, with expressions that said,
``Did I really just see what I think I saw?''

Shepard,
\href{https://www.nytimes.com/2017/07/31/theater/sam-shepard-dead.html}{who
died} last year, was in peak form as a dramatist in the early 1980s and
blessed (or cursed) with a sort of dichotomous stardom: as a Pulitzer
Prize-winning playwright (for ``Buried Child'' four years earlier) and
as a sometimes manly movie god in the laconic, strapping mold of Gary
Cooper. In both incarnations, he was to me the coolest guy on the
planet.

But while I might take a pass on some of his film appearances, I never
missed a Sam Shepard play, even the obscure revivals of the chaotic head
trips from the 1960s. Not only were his shows steeped in a wild,
stinging poetry that ate at the very foundations of American identity;
they also had a pure visceral charge you seldom found outside of a
wrestling match.

Which in a way is what ``Fool for Love'' is --- a gloves-off,
love-and-hate fight between a man and a woman who are doomed to keep
running away from and toward each other for all eternity. That may sound
impossible, but Baker and Harris, both in their early 30s then, exerted
a magnetically charged ambivalence that seemed to defy physics.

By the way, even by the standards of Shepard productions, which are
notorious for tearing up the scenery, ``Fool for Love'' has always had a
particularly high injury rate. Somehow I too felt bruised by the end of
my first encounter with it --- and so energized that I wanted to run all
the way. A good fight can have that effect.

\emph{``\href{https://www.nytimes.com/1983/05/27/theater/stage-fool-for-love-sam-shepard-western.html}{Stage:
`Fool for Love,' Sam Shepard Western},'' by Frank Rich, published May
27, 1983.}

\begin{center}\rule{0.5\linewidth}{\linethickness}\end{center}

Image

Credit...Weichia Huang

\hypertarget{what-no-one-could-have-predicted}{%
\subsection{What No One Could Have
Predicted}\label{what-no-one-could-have-predicted}}

\hypertarget{lawrence-k-altman-md-joined-the-new-york-times-science-news-staff-in-1969-and-is-believed-to-be-the-first-licensed-physician-to-work-as-a-daily-newspaper-reporter}{%
\subsubsection{Lawrence K. Altman, M.D. joined The New York Times
science news staff in 1969, and is believed to be the first licensed
physician to work as a daily newspaper
reporter.}\label{lawrence-k-altman-md-joined-the-new-york-times-science-news-staff-in-1969-and-is-believed-to-be-the-first-licensed-physician-to-work-as-a-daily-newspaper-reporter}}

New York and the rest of the world, including The New York Times, were
ill-prepared for the arrival and spread of a new disease (later named
AIDS) when the first cases were recognized among gay men in 1981. For
three years, while fear tore through the gay community, the world's most
sophisticated laboratories could not identify the cause. Experts debated
various candidates (among them: toxins, drugs, sperm in the bowel) until
the discovery of the retrovirus H.I.V. Even after epidemiologists
documented that the virus spread only through sexual contact, childbirth
or injections of contaminated drugs and blood, disbelievers shunned
public restrooms and restaurants for fear that it could be passed
through food and flatware. Many hospital workers avoided AIDS patients
and left trays by the door where occupants were too weak to pick them
up. Some doctors refused to treat AIDS patients.

At the time, medicine was celebrating the eradication of a naturally
occurring disease (smallpox) from the world for the first time and
installing the first artificial hearts in humans. Earlier, I had written
about the discoveries of the Ebola and Lassa fever viruses and the
Legionnaire's disease bacterium, as well as the ways in which health
workers stopped outbreaks of these infections. Few thought --- and
certainly no one I know went on record to predict --- that AIDS would
become one of the worst pandemics in history, infecting an estimated 76
million people and killing 35 million of them.

At Bellevue Hospital in Manhattan, where, in the late 1970s, I was an
attending physician in addition to working as a reporter at The Times, I
saw a small number of young patients with unusual infections. Others had
enlarged lymph nodes throughout their body. Many had been injection drug
users. These cases baffled the medical staff.

My plans to write about the mysterious ailment before my first article
about AIDS appeared on July 3, 1981 were thwarted by assigned coverage
of the attempted assassinations of President Reagan and Pope John Paul
II and by two broken elbows, souvenirs of an accident in Italy. My
initial article focused on 41 cases of a rare cancer, Kaposi's sarcoma,
in homosexual men (The Times did not allow people to be described as
``gay'' at that time) who also had seriously abnormal immune systems.
But whether these immune issues were the cause of, or resulted from,
their ailment was unknown. Many in the gay community ridiculed the
article as a needless scare. Scientists who went on to become
international leaders in battling AIDS considered this collection of
cases to be an oddity, until a broader picture emerged in the advancing
months and years. In 1985, I traveled through Africa to report that in
many sub-Saharan countries, AIDS infected nearly as many women as men
through heterosexual intercourse. I went on to write several hundred
articles about scientific discoveries of the virus, its uncanny ability
to trick the immune system and the development of drugs that now make
AIDS a chronic disease. Despite these and other advances, it is still
transmitted in this country and the world.

\emph{``\href{https://www.nytimes.com/1981/07/03/us/rare-cancer-seen-in-41-homosexuals.html}{Rare
Cancer Seen in 41 Homosexuals},'' by Lawrence K. Altman, published July
3, 1981.}

\begin{center}\rule{0.5\linewidth}{\linethickness}\end{center}

Image

Credit...Double Fantasy record: courtesy of Yoko Ono

\hypertarget{the-review-im-relieved-never-ran}{%
\subsection{The Review I'm Relieved Never
Ran}\label{the-review-im-relieved-never-ran}}

\hypertarget{stephen-holden-first-joined-the-new-york-times-as-a-freelancer-in-1981-he-became-a-staff-critic-in-1988-retiring-from-the-paper-last-year}{%
\subsubsection{Stephen Holden first joined The New York Times as a
freelancer in 1981. He became a staff critic in 1988, retiring from the
paper last
year.}\label{stephen-holden-first-joined-the-new-york-times-as-a-freelancer-in-1981-he-became-a-staff-critic-in-1988-retiring-from-the-paper-last-year}}

Upon its release on Nov. 17, 1980,
\href{https://www.nytimes.com/topic/person/john-lennon}{John Lennon} and
Yoko Ono's ``Double Fantasy'' album was greeted by most pop critics with
a yawn. A freelancer at the time, I was asked to review the record for
the Arts and Leisure section and found the mostly sedate collection of
songs celebrating the couple's domestic bliss with their young son,
Sean, a slick, shallow self-advertisement that turned its back on the
post-punk moment.

Perceptions changed overnight when Lennon was shot to death outside the
Dakota apartment building in New York on Dec. 8. When my ho-hum
assessment was scrapped, I was so relieved I didn't even save a copy.

The mid-1970s had been a fallow period for Lennon, during which he and
Ono separated while Lennon and his pal, the singer-songwriter Harry
Nilsson, went on a notorious rampage known as Lennon's ``lost weekend.''
When the binge ended, Lennon and Ono reunited, while Nilsson continued
his downward spiral and died at 52 in 1994.

As a talent scout and staff writer for Nilsson's record label, I had
witnessed the chaos firsthand as Harry, coked to the gills, lurched
around the halls swigging from a bottle of cognac, sometimes with Lennon
by his side. He became the role model for a fading rock star in my
satirical novel, ``Triple Platinum,'' published in early 1980. Lennon,
having cleaned up his act, retreated to the apartment in the Dakota that
he shared with Ono and Sean, to play househusband.

Revisited more than three decades later, ``Double Fantasy,'' a record
whose songs celebrated new beginnings and starting over in unabashedly
sentimental songs with 1950s rock 'n' roll flavor, feels like a
deliberate farewell album, though it wasn't conceived as such. Since
1980, the hostility against Ono (whom many blamed for breaking up the
Beatles) has evaporated, while respect for her --- she contributed
several quirky songs to a collection that portrays their marriage as a
rock 'n' roll fairy tale --- has grown. ``Double Fantasy'' went on to
sell three million copies: triple platinum. In hindsight, an album that
seemed smugly out of touch nearly 40 years ago today sounds like a brave
statement of refusal to keep playing the adolescent rock 'n' roll game.
Embracing what might be called bohemian family values in songs like
``Cleanup Time'' and its celebrations of parenthood in ``Beautiful
Boy,'' it offered a persuasive and personal argument for growing up
while still keeping the faith.

\begin{center}\rule{0.5\linewidth}{\linethickness}\end{center}

Image

Credit...Weichia Huang

\hypertarget{the-austrian-restaurant-that-got-away}{%
\subsection{The Austrian Restaurant That Got
Away}\label{the-austrian-restaurant-that-got-away}}

\hypertarget{mimi-sheraton-was-the-new-york-timess-restaurant-critic-from-1976-to-1983}{%
\subsubsection{Mimi Sheraton was The New York Times's restaurant critic
from 1976 to
1983.}\label{mimi-sheraton-was-the-new-york-timess-restaurant-critic-from-1976-to-1983}}

In January of 1981, I awarded the top four-star rating to Vienna '79, a
suavely stylish Austrian restaurant that opened on East 79th Street in
1979. This just 10 months after I gave it three stars in March 1980.
Looking back, I might not have been more generous originally because I
was miffed at not hearing about this wonderful place until a year after
it opened. Especially as I have always had an abiding fondness for
schnitzels, tafelspitz, kuchen and, above all, Salzburger nockerl, those
airy, warm, mini-soufflés adrift on a vanilla-scented sea of creme
anglaise.

What stunned and won me at Vienna '79 were the achievements of the
owner, Peter Grünauer, and his chef, Thomas Ferlesch. Their food
reflected a nouvelle lightness thanks to reduced amounts of fats and
flour, and it was served in artfully restrained presentations.
Invariably they maintained completely authentic flavors of the most
traditional dishes such as the icy, seductively creamy cucumber salad;
the aromatic snails baked in a crusty kaiser roll; the magical
zwiebelrostbraten, rare beef steak topped with glassily crisp onion
filaments; and crisply rendered Wiener backhuhn, perhaps the world's
best fried chicken. And when most of the city's Austro-German
restaurants in Yorkville and even the legendary Luchow's near Union
Square dripped Gemütlichkeit, Vienna '79 suggested a smart, urbane
supper club, with dove-gray walls and stunning black-and-white prints of
antique Viennese scenes.

Strangely, however, as good as Vienna '79 was, like many excellent tries
at Austro-German cooking in New York, its popularity was relatively
short-lived, and it closed in 1986. Since then we have had other waves
of that cuisine, initially hot then quickly cooling for reasons I have
never understood. But the real hero of the tale is Kurt Gutenbrunner,
who in 2000 revived and still enchants with his lush and coddling
cuisine at
\href{https://www.kurtgutenbrunner.com/restaurants/wallse/}{Wallsé} in
the West Village and the enticing
\href{http://www.neuegalerie.org/cafes/sabarsky}{Cafe Sabarsky} at the
Neue Galerie, as well as the mecca of wurst and schnitzel that is his
very casual
\href{https://www.kurtgutenbrunner.com/restaurants/blaue-gans/}{Blaue
Gans} in TriBeCa. The other good news is that Peter Grünauer is back
with lusty food in his informal, taverny
\href{http://grunauernyc.com/}{Grünauer Bistro} on the Upper East Side,
and the gifted Thomas Ferlesch is wowing the young and what remains of
the old at \href{http://www.werkstattbrooklyn.com/}{Werkstatt}, a
ticklishly playful cafe in the Prospect Park South section of --- where
else? --- Brooklyn. With Austria as his inspiration, he puts some modern
spins on classics such as crisply baked mushrooms with tartar sauce, a
crunchy homemade pretzel and fragrantly juicy goulash with tiny
spaetzle. And, in a nod toward vegetarians, he creates a convincing
schnitzel of almost-meaty-tasting celery root. Such efforts are happily
tempting traditionalists as well as the gastronomically restless. It's
an invitation, perhaps, to raise a glass of beer and offer a toast to
``Gut Essen.''

\emph{``\href{https://www.nytimes.com/1981/01/09/arts/restaurants-by-mimi-sheraton-a-viennese-nouvelle-cuisine.html}{A
Viennese Nouvelle Cuisine},'' by Mimi Sheraton, published Jan. 8, 1981.}

\begin{center}\rule{0.5\linewidth}{\linethickness}\end{center}

\emph{\textbf{Read more:}}

\emph{\href{https://www.nytimes.com/2018/04/17/t-magazine/timeline-1981-to-1983.html}{What
Happened in New York Between 1981 and 1983}}

\emph{\href{https://www.nytimes.com/2018/04/17/t-magazine/24-hours-new-york-city-1980s-life.html}{What
New York Was Like in the Early '80s --- Hour by Hour}}

\emph{\href{https://www.nytimes.com/2018/04/12/t-magazine/asian-american-art-martin-wong-tseng-kwong-chi.html}{The
Artists Who Brought Asian-Americans Into the Annals of Contemporary
Art}}

Advertisement

\protect\hyperlink{after-bottom}{Continue reading the main story}

\hypertarget{site-index}{%
\subsection{Site Index}\label{site-index}}

\hypertarget{site-information-navigation}{%
\subsection{Site Information
Navigation}\label{site-information-navigation}}

\begin{itemize}
\tightlist
\item
  \href{https://help.nytimes.com/hc/en-us/articles/115014792127-Copyright-notice}{©~2020~The
  New York Times Company}
\end{itemize}

\begin{itemize}
\tightlist
\item
  \href{https://www.nytco.com/}{NYTCo}
\item
  \href{https://help.nytimes.com/hc/en-us/articles/115015385887-Contact-Us}{Contact
  Us}
\item
  \href{https://www.nytco.com/careers/}{Work with us}
\item
  \href{https://nytmediakit.com/}{Advertise}
\item
  \href{http://www.tbrandstudio.com/}{T Brand Studio}
\item
  \href{https://www.nytimes.com/privacy/cookie-policy\#how-do-i-manage-trackers}{Your
  Ad Choices}
\item
  \href{https://www.nytimes.com/privacy}{Privacy}
\item
  \href{https://help.nytimes.com/hc/en-us/articles/115014893428-Terms-of-service}{Terms
  of Service}
\item
  \href{https://help.nytimes.com/hc/en-us/articles/115014893968-Terms-of-sale}{Terms
  of Sale}
\item
  \href{https://spiderbites.nytimes.com}{Site Map}
\item
  \href{https://help.nytimes.com/hc/en-us}{Help}
\item
  \href{https://www.nytimes.com/subscription?campaignId=37WXW}{Subscriptions}
\end{itemize}
