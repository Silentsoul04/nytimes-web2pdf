Sections

SEARCH

\protect\hyperlink{site-content}{Skip to
content}\protect\hyperlink{site-index}{Skip to site index}

\href{https://www.nytimes.com/section/politics}{Politics}

\href{https://myaccount.nytimes.com/auth/login?response_type=cookie\&client_id=vi}{}

\href{https://www.nytimes.com/section/todayspaper}{Today's Paper}

\href{/section/politics}{Politics}\textbar{}Trump Is Said to Have Known
of Payment to Stormy Daniels Months Before He Denied It

\url{https://nyti.ms/2HOyp3v}

\begin{itemize}
\item
\item
\item
\item
\item
\end{itemize}

Advertisement

\protect\hyperlink{after-top}{Continue reading the main story}

Supported by

\protect\hyperlink{after-sponsor}{Continue reading the main story}

\hypertarget{trump-is-said-to-have-known-of-payment-to-stormy-daniels-months-before-he-denied-it}{%
\section{Trump Is Said to Have Known of Payment to Stormy Daniels Months
Before He Denied
It}\label{trump-is-said-to-have-known-of-payment-to-stormy-daniels-months-before-he-denied-it}}

\includegraphics{https://static01.nyt.com/images/2018/05/05/us/politics/05dc-trumpcohen1/merlin_137678598_5aaadfd7-9ae2-490e-a689-58b2350f8b38-articleLarge.jpg?quality=75\&auto=webp\&disable=upscale}

By \href{http://www.nytimes.com/by/michael-d-shear}{Michael D. Shear},
\href{http://www.nytimes.com/by/maggie-haberman}{Maggie Haberman},
\href{http://www.nytimes.com/by/jim-rutenberg}{Jim Rutenberg} and
\href{http://www.nytimes.com/by/matt-apuzzo}{Matt Apuzzo}

\begin{itemize}
\item
  May 4, 2018
\item
  \begin{itemize}
  \item
  \item
  \item
  \item
  \item
  \end{itemize}
\end{itemize}

WASHINGTON --- President Trump knew about a six-figure payment that
Michael D. Cohen, his personal lawyer, made to a pornographic film
actress several months before he denied any knowledge of it to reporters
aboard Air Force One in April, according to two people familiar with the
arrangement.

How much Mr. Trump knew about the payment to Stephanie Clifford, the
actress, and who else was aware of it have been at the center of a
swirling controversy for the past 48 hours touched off by a television
interview with Rudolph W. Giuliani, a new addition to the president's
legal team. The interview was the first time a lawyer for the president
had acknowledged that Mr. Trump had reimbursed Mr. Cohen for the
payments to Ms. Clifford, whose stage name is Stormy Daniels.

It was not immediately clear when Mr. Trump learned of the payment,
which Mr. Cohen made in October 2016, at a time when news media outlets
were poised to pay her for her story about an alleged affair with Mr.
Trump in 2006. But three people close to the matter said that Mr. Trump
knew that Mr. Cohen had succeeded in keeping the allegations from
becoming public at the time the president denied it.

Ms. Clifford signed a nondisclosure agreement, and accepted the payment
just days before Mr. Trump won the 2016 presidential election. Mr. Trump
has denied he had an affair with Ms. Clifford and insisted that the
nondisclosure agreement was created to prevent any embarrassment to his
family.

Mr. Giuliani said this week that the reimbursement to Mr. Cohen totaled
\$460,000 or \$470,000, leaving it unclear what else the payments were
for beyond the \$130,000 that went to Ms. Clifford. One of the people
familiar with the arrangement said that it was a \$420,000 total over 12
months.

Allen Weisselberg, the chief financial officer of the Trump
Organization, has known since last year the details of how Mr. Cohen was
being reimbursed, which was mainly through payments of \$35,000 per
month from the trust that contains the president's personal fortune,
according to two people with knowledge of the arrangement.

One person close to the Trump Organization said people with the company
were aware that Mr. Cohen was still doing ``legal work'' for the
president in 2017, but another person familiar with the situation said
that Mr. Weisselberg did not know that Mr. Cohen had paid Ms. Clifford
when the retainer agreement was struck and when the payments went
through.

Mr. Weisselberg's knowledge of the retainer agreement could draw Mr.
Trump's company deeper into the federal investigation of Mr. Cohen's
activities, increasing the president's legal exposure in a wide-ranging
case involving the lawyer often described as the president's ``fixer''
in New York City.

In interviews on Wednesday and Thursday, Mr. Giuliani insisted that the
president had reimbursed Mr. Cohen for the \$130,000 hush payment ---
and then paid him another \$330,000, if not more --- which was in direct
conflict with the longstanding assertion by Mr. Trump and the White
House that he did not know about the hush money or where it came from.

In an interview with The New York Times on Friday, Mr. Giuliani sought
to clarify his statements by saying that he did not know whether Mr.
Trump had known that some of the payments to Mr. Cohen had gone to Ms.
Clifford. ``It's not something I'm aware of, nor is it relevant to what
I'm doing, the legal part,'' Mr. Giuliani said.

Mr. Giuliani acknowledged that ``politically,'' it could be troublesome.
``Politically, everything matters, but I don't see a problem here, at
least not'' legally, he said.

A lawyer for the Trump Organization declined to comment, and a
spokeswoman for the organization did not respond to an email about Mr.
Weisselberg.

The president has said that he would view any investigation into his
finances or those of his family as ``a violation,'' though he was
referring to the investigation into Russia by the special counsel,
Robert S. Mueller III; the investigation into Mr. Cohen is being run by
federal prosecutors in the Southern District of New York.

The payment to Ms. Clifford is a part of that investigation. The
circumstances surrounding it had become all the murkier this week after
Mr. Giuliani gave an explanation of how the funds to Ms. Clifford were
accounted for that contradicted all those that came before it.

After initially appearing to back Mr. Giuliani's assertions in a series
of Twitter messages on Thursday, Mr. Trump reversed course on Friday,
after a series of headlines suggesting that the president had lied about
knowing of the hush payment. In remarks to reporters on Friday, Mr.
Trump criticized Mr. Giuliani and said he would eventually ``get his
facts straight.''

\includegraphics{https://static01.nyt.com/images/2018/05/05/us/politics/05dc-trumpcohen2/merlin_137361501_6de20352-91cc-4b5d-b9ac-ba1a4224d2b0-articleLarge.jpg?quality=75\&auto=webp\&disable=upscale}

``Virtually everything said has been said incorrectly, and it's been
said wrong, or it's been covered wrong by the press,'' Mr. Trump told
reporters, though he excused Mr. Giuliani by explaining he had ``just
started a day ago.''

In a written statement later in the day, Mr. Giuliani said that he had
not been ``describing my understanding of the president's knowledge.''
And he reversed a previous suggestion that the payment to Ms. Clifford
was motivated by the election. Mr. Giuliani said on Friday that the
payment was personal in nature and ``would have been done in any event,
whether he was a candidate or not.'' Mr. Giuliani told The Times that he
had ``confused'' the two factors, but that it was irrelevant since Mr.
Trump had repaid Mr. Cohen.

While some White House officials had insisted that Mr. Trump was pleased
with Mr. Giuliani's performance on Fox News in an interview with Sean
Hannity on Wednesday night, two people close to the president painted a
different picture. They said that Mr. Trump was displeased with how Mr.
Giuliani, a former New York mayor, conducted himself, and that he was
also unhappy with Mr. Hannity, a commentator whose advice the president
often seeks, in terms of the language he used to describe the payments
to Ms. Clifford.

The nature of the payments is significant because of campaign finance
laws that regulate who may contribute to candidates and how much they
can give.

If Mr. Cohen or others paid to silence Ms. Clifford primarily out of
fear that a public airing of her story would have harmed Mr. Trump's
election prospects --- rather than to keep it from his family for
personal reasons --- then the payment would most likely be viewed as an
illegal campaign expenditure. Mr. Giuliani told The Times on Friday that
the issue was ``primarily'' about keeping Mr. Trump's wife, Melania,
from being embarrassed by the claim, which Mr. Trump has maintained was
false.

But if investigators determine that the hush payment was in effect a
campaign expenditure, then how the funds were distributed could take on
added legal significance. Mr. Cohen had been careful to say that neither
the campaign nor the Trump Organization was involved in the deal or any
effort to reimburse him.

Under campaign finance law, Mr. Trump would have been within his rights
to pay Ms. Clifford himself as a way to protect his presidential
prospects --- though he would have had to have formally made note of it
in his public campaign filings, which had no accounting of the payment.
If he directed Mr. Cohen to pay it on his behalf, then that could
qualify as an illegal, coordinated campaign expenditure, even if Mr.
Trump later paid him back.

Any involvement by the Trump Organization would further complicate the
legal picture, given that American election law is strictest of all when
it comes to corporate involvement with political campaigns. Businesses
are not allowed to donate directly to campaigns or to coordinate with
them.

Ms. Clifford's lawyer, Michael J. Avenatti, has been arguing for months
that Mr. Trump's company was more involved in the arrangement than Mr.
Cohen had been letting on.

After filing a lawsuit on Ms. Clifford's behalf seeking to get out of
the deal --- which he has called invalid --- Mr. Avenatti showed that
Mr. Cohen had used his Trump Organization email at one point in
arranging the payment. He also pointed to a secret document in
California that a Trump Organization lawyer filed to force Ms. Clifford
into arbitration this year.

\href{https://www.wsj.com/articles/top-trump-company-lawyer-worked-to-silence-stormy-daniels-1521072252?tesla=y\&mod=breakingnews}{At
the time}, the Trump Organization said that the lawyer, Jill A. Martin,
who works in California, had acted in a personal capacity to help Mr.
Cohen, who needed assistance with the initial arbitration filing from
someone licensed in the state. The Trump Organization had said that
``the company has had no involvement in the matter.''

In an interview, Mr. Avenatti said that any indication that still more
executives at the Trump Organization knew about the effort to reimburse
Mr. Cohen for the payment to Ms. Clifford could lead to further
investigation of the Trump family business.

``There's no question it opens up another avenue of inquiry into the
depths of the involvement of the Trump Organization,'' he said.

Advertisement

\protect\hyperlink{after-bottom}{Continue reading the main story}

\hypertarget{site-index}{%
\subsection{Site Index}\label{site-index}}

\hypertarget{site-information-navigation}{%
\subsection{Site Information
Navigation}\label{site-information-navigation}}

\begin{itemize}
\tightlist
\item
  \href{https://help.nytimes.com/hc/en-us/articles/115014792127-Copyright-notice}{©~2020~The
  New York Times Company}
\end{itemize}

\begin{itemize}
\tightlist
\item
  \href{https://www.nytco.com/}{NYTCo}
\item
  \href{https://help.nytimes.com/hc/en-us/articles/115015385887-Contact-Us}{Contact
  Us}
\item
  \href{https://www.nytco.com/careers/}{Work with us}
\item
  \href{https://nytmediakit.com/}{Advertise}
\item
  \href{http://www.tbrandstudio.com/}{T Brand Studio}
\item
  \href{https://www.nytimes.com/privacy/cookie-policy\#how-do-i-manage-trackers}{Your
  Ad Choices}
\item
  \href{https://www.nytimes.com/privacy}{Privacy}
\item
  \href{https://help.nytimes.com/hc/en-us/articles/115014893428-Terms-of-service}{Terms
  of Service}
\item
  \href{https://help.nytimes.com/hc/en-us/articles/115014893968-Terms-of-sale}{Terms
  of Sale}
\item
  \href{https://spiderbites.nytimes.com}{Site Map}
\item
  \href{https://help.nytimes.com/hc/en-us}{Help}
\item
  \href{https://www.nytimes.com/subscription?campaignId=37WXW}{Subscriptions}
\end{itemize}
