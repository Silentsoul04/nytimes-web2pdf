Sections

SEARCH

\protect\hyperlink{site-content}{Skip to
content}\protect\hyperlink{site-index}{Skip to site index}

\href{https://www.nytimes.com/section/world/middleeast}{Middle East}

\href{https://myaccount.nytimes.com/auth/login?response_type=cookie\&client_id=vi}{}

\href{https://www.nytimes.com/section/todayspaper}{Today's Paper}

\href{/section/world/middleeast}{Middle East}\textbar{}Trump Abandons
Iran Nuclear Deal He Long Scorned

\url{https://nyti.ms/2KMeG2f}

\begin{itemize}
\item
\item
\item
\item
\item
\item
\end{itemize}

Advertisement

\protect\hyperlink{after-top}{Continue reading the main story}

Supported by

\protect\hyperlink{after-sponsor}{Continue reading the main story}

\hypertarget{trump-abandons-iran-nuclear-deal-he-long-scorned}{%
\section{Trump Abandons Iran Nuclear Deal He Long
Scorned}\label{trump-abandons-iran-nuclear-deal-he-long-scorned}}

\includegraphics{https://static01.nyt.com/images/2018/05/09/us/politics/09dc-prexy-new1/09dc-prexy-new1-videoSixteenByNine3000.jpg}

By \href{https://www.nytimes.com/by/mark-landler}{Mark Landler}

\begin{itemize}
\item
  May 8, 2018
\item
  \begin{itemize}
  \item
  \item
  \item
  \item
  \item
  \item
  \end{itemize}
\end{itemize}

WASHINGTON --- President Trump declared on Tuesday that he was
withdrawing from the
\href{https://www.nytimes.com/2020/01/14/world/europe/iran-nuclear-deal.html}{Iran
nuclear deal}, unraveling the signature foreign policy achievement of
his predecessor Barack Obama, isolating the United States from its
Western allies and sowing uncertainty before a risky nuclear negotiation
with North Korea.

The decision, while long anticipated and widely telegraphed, leaves the
2015 agreement reached by seven countries after more than two years of
grueling negotiations in tatters. The United States will now reimpose
the stringent sanctions it imposed on Iran before the deal and is
considering new penalties.

\href{https://www.nytimes.com/2020/01/14/world/europe/iran-nuclear-deal.html}{Iran}
said it will remain in the deal, which tightly restricted its nuclear
ambitions for a decade or more in return for ending the sanctions that
had crippled its economy.

So did France, Germany and Britain, raising the prospect of a
trans-Atlantic clash as European companies face the return of American
sanctions for doing business with Iran. China and Russia, also
signatories to the deal, are likely to join
\href{https://www.nytimes.com/2020/01/14/world/europe/iran-nuclear-deal.html}{Iran}
in accusing the United States of violating the accord.

Mr. Trump's move could embolden hard-line forces in Iran, raising the
threat of Iranian retaliation against Israel or the United States,
fueling an arms race in the Middle East and fanning sectarian conflicts
from Syria to Yemen.

The president, however, framed his decision as the fulfillment of a
bedrock campaign promise and as the act of a dealmaker dissolving a
fatally flawed agreement. He predicted his tough line with Iran would
strengthen his hand as he prepared to meet North Korea's leader, Kim
Jong-un, to begin negotiating the surrender of his nuclear arsenal.

``This was a horrible one-sided deal that should have never, ever been
made,'' a grim-faced Mr. Trump said in an 11-minute address from the
Diplomatic Reception Room of the White House. ``It didn't bring calm, it
didn't bring peace, and it never will.''

\emph{{[}Read the full transcript of}
\href{https://www.nytimes.com/2018/05/08/us/politics/trump-speech-iran-deal.html?action=click\&module=Intentional\&pgtype=Article}{\emph{President
Trump's remarks.}}\emph{{]}}

\href{https://www.nytimes.com/2019/04/08/world/middleeast/trump-iran-revolutionary-guard-corps.html}{Mr.
Trump's} announcement drew a chorus of opposition from European leaders,
several of whom lobbied him feverishly not to pull out of the agreement
and searched for fixes to it that would satisfy him.

It also drew a rare public rebuke by Mr. Obama, who said Mr. Trump's
withdrawal would leave the world less safe, confronting it with ``a
losing choice between a nuclear-armed Iran or another war in the Middle
East.''

The response from Iran itself, however, was muted. President Hassan
Rouhani declared that the Iranians intended to abide by the terms of the
deal, and he criticized Mr. Trump for his history of not honoring
international treaties. Mr. Trump won strong backing from Saudi Arabia
and Israel, whose leader, Prime Minister Benjamin Netanyahu, hailed him
for a ``historic move'' and ``courageous leadership.''

\href{https://www.nytimes.com/interactive/2018/05/07/world/middleeast/iran-deal-before-after.html}{}

\includegraphics{https://static01.nyt.com/images/2018/05/08/us/iran-deal-before-after-promo-1525810744376/iran-deal-before-after-promo-1525810744376-articleLarge.png}

\hypertarget{what-changes-and-what-remains-in-the-iran-nuclear-deal}{%
\subsection{What Changes and What Remains in the Iran Nuclear
Deal}\label{what-changes-and-what-remains-in-the-iran-nuclear-deal}}

The restrictions on Iran's nuclear program under the deal could survive.

Three times previously, the president's aides had persuaded him not to
dismantle the Iran deal. But Mr. Trump made clear that his patience had
worn thin, and with a new, more hawkish cohort of advisers --- led by
Secretary of State Mike Pompeo and the national security adviser, John
R. Bolton --- the president faced less internal resistance than earlier
in his administration.

While Mr. Trump had long scorned the Iran deal, threatening repeatedly
to rip it up during the 2016 presidential race, his impulse to act now
was reinforced by what he views as the success of his policy toward
North Korea. He has told aides and foreign leaders that his policy of
maximum pressure had forced Mr. Kim to the bargaining table, and that a
similar policy of overwhelming pressure would enable the United States
to extract a better deal from Iran.

As Mr. Trump abandoned one diplomatic project, he accelerated another
--- announcing that Mr. Pompeo was flying to Pyongyang, the capital of
North Korea, to continue discussions with Mr. Kim about the upcoming
summit meeting. He expressed hope that three Americans who are detained
in the North would be released soon.

``The message to North Korea,'' Mr. Bolton told reporters, ``is the
president wants a real deal.''

He rejected the suggestion that the United States could not be trusted
to keep its agreements when political winds change. ``Any nation
reserves the right to correct a past mistake,'' Mr. Bolton said, citing
President George W. Bush's decision to withdraw from the Antiballistic
Missile Treaty in 2001.

The Trump administration, he said, would continue to work with Europeans
to pressure the Iranians. He dismissed those who said the United States
was on a path to war with Iran, though he did not present any new
diplomatic initiatives. Another senior administration official
acknowledged that there was no Plan B.

Months of intense negotiations with the Europeans to keep the accord in
place collapsed over Mr. Trump's insistence that the limits placed by
the agreement on Iran's nuclear fuel production were inadequate. Under
the provisions of the deal, those limits, or ``sunset clauses,'' were to
expire in 2030 --- 15 years after the deal was signed.

As a result, the United States will reinstate all the sanctions it had
waived as part of the nuclear accord, and it will impose additional
economic penalties that are now being drawn up by the Treasury
Department.

Treasury Secretary Steven Mnuchin declined on Tuesday to specify what
additional sanctions the United States might impose, but he expressed
confidence that they would still be powerful even if other American
allies did not follow suit.

``We do not want to let Iran use the U.S. financial markets and
financial system and transact in dollars until they agree that not only
will they not have a nuclear weapon now, but we've put in provisions
that they will never have one,'' Mr. Mnuchin said.

In his announcement, Mr. Trump recited familiar arguments against the
deal: that it does not address the threat of Iran's ballistic missiles
or its malign behavior in the region, and that the expiration dates for
the sunset clauses open the door to an Iranian nuclear bomb down the
road.

Even if Iran was in compliance, he said, it could ``still be on the
verge of a nuclear breakout in just a short period of time.'' In fact,
under the deal, the limits on Iran's uranium enrichment and stockpiles
of nuclear fuel mean that Iran would not be on the verge of a nuclear
breakout until 2030.

Still, Mr. Trump said, the United States and its allies could not stop
Iran from building a nuclear weapon ``under the decaying and rotten
structure of the current agreement.''

``The Iran deal is defective at its core,'' he concluded.

Mr. Trump's announcement capped a frantic four-day period in which
American and European diplomats made a last-ditch effort to bridge their
differences and preserve the agreement.

That effort began Friday, when Mr. Pompeo called his counterparts in
Europe to tell them that Mr. Trump was planning to withdraw from the
deal, but that he was trying to win a two-week reprieve for the United
States and Europe to continue negotiating. Mr. Pompeo, people familiar
with the talks said, suggested that he favored a so-called soft
withdrawal, in which Mr. Trump would pull out of the deal but hold off
on reimposing some of the sanctions.

The next day, the State Department's chief negotiator, Brian H. Hook,
consulted with European diplomats to try to break a deadlock over the
sunset provision, under which the restrictions on Iran's ability to
produce nuclear fuel for civilian use expire after 15 years.

The Europeans had already agreed to a significant compromise: to
reimpose sanctions if there were a determination that the Iranians were
within 12 months of producing a nuclear weapon. But officials said that
still did not satisfy Mr. Trump, and the Europeans were not willing to
go any further.

By Monday, the White House began informing allies that Mr. Trump was
going to withdraw from the deal and reimpose sanctions on oil and impose
new sanctions against the Central Bank of Iran.

Under the financial sanctions, European companies will have 90 to 180
days to wind down their operations in Iran, or they will run afoul of
the American banking system. The sanctions on oil will require European
and Asian countries to reduce their imports from Iran.

Mr. Mnuchin insisted that the restrictions would not drive up oil prices
because other suppliers would pick up the slack. ``My expectation is not
that oil prices go higher,'' he said. ``To a certain extent, some of
this was already in the market on oil prices.''

Mr. Trump's decision will test his already frayed relationship with
European leaders. President Emmanuel Macron of France, whom the
president welcomed with a state dinner two weeks ago, learned of his
decision in a phone call with Mr. Trump on Tuesday morning. Later, he
said in a
\href{https://twitter.com/EmmanuelMacron/status/993921478998544384}{post
on Twitter} that the European allies ``regret'' his decision.

``The international regime against nuclear proliferation is at stake,''
he added.

In a joint statement, Mr. Macron, Chancellor Angela Merkel of Germany
and Prime Minister Theresa May of Britain noted pointedly that the
United Nations Security Council resolution endorsing the nuclear deal
remained the ``binding international legal framework for the resolution
of the dispute.'' That raises the possibility that the United States
will be found to be in violation in the Security Council.

Few people were more stung by Mr. Trump's decision than those who worked
for Mr. Obama. Though he has moved methodically to dismantle his
predecessor's legacy, his reversal of the Iran deal was particularly
painful, given the five years of effort that went into imposing
sanctions, and the more than two-year-long negotiation led by Secretary
of State John Kerry that yielded the accord.

``No rhetoric is required,'' Mr. Kerry said in a statement. ``The facts
speak for themselves. Instead of building on unprecedented
nonproliferation verification measures, this decision risks throwing
them away and dragging the world back to the brink we faced a few years
ago.''

Advertisement

\protect\hyperlink{after-bottom}{Continue reading the main story}

\hypertarget{site-index}{%
\subsection{Site Index}\label{site-index}}

\hypertarget{site-information-navigation}{%
\subsection{Site Information
Navigation}\label{site-information-navigation}}

\begin{itemize}
\tightlist
\item
  \href{https://help.nytimes.com/hc/en-us/articles/115014792127-Copyright-notice}{©~2020~The
  New York Times Company}
\end{itemize}

\begin{itemize}
\tightlist
\item
  \href{https://www.nytco.com/}{NYTCo}
\item
  \href{https://help.nytimes.com/hc/en-us/articles/115015385887-Contact-Us}{Contact
  Us}
\item
  \href{https://www.nytco.com/careers/}{Work with us}
\item
  \href{https://nytmediakit.com/}{Advertise}
\item
  \href{http://www.tbrandstudio.com/}{T Brand Studio}
\item
  \href{https://www.nytimes.com/privacy/cookie-policy\#how-do-i-manage-trackers}{Your
  Ad Choices}
\item
  \href{https://www.nytimes.com/privacy}{Privacy}
\item
  \href{https://help.nytimes.com/hc/en-us/articles/115014893428-Terms-of-service}{Terms
  of Service}
\item
  \href{https://help.nytimes.com/hc/en-us/articles/115014893968-Terms-of-sale}{Terms
  of Sale}
\item
  \href{https://spiderbites.nytimes.com}{Site Map}
\item
  \href{https://help.nytimes.com/hc/en-us}{Help}
\item
  \href{https://www.nytimes.com/subscription?campaignId=37WXW}{Subscriptions}
\end{itemize}
