Sections

SEARCH

\protect\hyperlink{site-content}{Skip to
content}\protect\hyperlink{site-index}{Skip to site index}

\href{https://www.nytimes.com/section/technology}{Technology}

\href{https://myaccount.nytimes.com/auth/login?response_type=cookie\&client_id=vi}{}

\href{https://www.nytimes.com/section/todayspaper}{Today's Paper}

\href{/section/technology}{Technology}\textbar{}Hundreds of Apps Can
Empower Stalkers to Track Their Victims

\url{https://nyti.ms/2GvyzaJ}

\begin{itemize}
\item
\item
\item
\item
\item
\end{itemize}

Advertisement

\protect\hyperlink{after-top}{Continue reading the main story}

Supported by

\protect\hyperlink{after-sponsor}{Continue reading the main story}

\hypertarget{hundreds-of-apps-can-empower-stalkers-to-track-their-victims}{%
\section{Hundreds of Apps Can Empower Stalkers to Track Their
Victims}\label{hundreds-of-apps-can-empower-stalkers-to-track-their-victims}}

\includegraphics{https://static01.nyt.com/images/2018/05/18/us/18stalkerware6/merlin_138354531_de298b5c-4501-4e07-9044-50a402a528be-articleLarge.jpg?quality=75\&auto=webp\&disable=upscale}

By Jennifer Valentino-DeVries

\begin{itemize}
\item
  May 19, 2018
\item
  \begin{itemize}
  \item
  \item
  \item
  \item
  \item
  \end{itemize}
\end{itemize}

\href{https://www.nytimes.com/es/2018/05/24/espia-aplicacion-celular}{Leer
en español}

KidGuard is a phone app that markets itself as a tool for keeping tabs
on children. But it has also promoted its surveillance for other
purposes and run blog posts with headlines like ``How to Read Deleted
Texts on Your Lover's Phone.''

A similar app, mSpy, offered advice to a woman on secretly monitoring
her husband. Still another, Spyzie, ran ads on Google alongside results
for search terms like ``catch cheating girlfriend iPhone.''

As digital tools that gather cellphone data for tracking children,
friends or lost phones have multiplied in recent years, so have the
options for people who abuse the technology to track others without
consent.

More than 200 apps and services offer would-be stalkers a variety of
capabilities, from basic location tracking to harvesting texts and even
secretly recording video, according to
\href{https://www.ipvtechresearch.org/pubs/spyware.pdf}{a new academic
study}. More than two dozen services were promoted as surveillance tools
for spying on romantic partners, according to the researchers and
reporting by The New York Times. Most of the spying services required
access to victims' phones or knowledge of their passwords --- both
common in domestic relationships.

Digital monitoring of a spouse or partner can constitute illegal
stalking, wiretapping or hacking. But laws and law enforcement have
struggled to keep up with technological changes, even though stalking is
a top warning sign for attempted homicide in domestic violence cases.

``We misunderstand and minimize this abuse,'' said Erica Olsen, director
of the \href{https://www.techsafety.org/}{Safety Net Project} at the
National Network to End Domestic Violence. ``People think that if
there's not an immediate physical proximity to the victim, there might
not be as much danger.''

Statistics on electronic stalking are hard to find because victims may
not know they are being watched, or they may not report it. Even if they
believe they are being tracked, hidden software can make confirmation
difficult.

But data breaches at two surveillance companies last year --- revealing
accounts of more than 100,000 users, according to the technology site
\href{https://motherboard.vice.com/en_us/article/53vm7n/inside-stalkerware-surveillance-market-flexispy-retina-x}{Motherboard}
--- gave some sense of the scale. The tracking app company mSpy told The
New York Times that it sold subscriptions to more than 27,000 users in
the United States in the first quarter of this year.

According to data published last year by the Centers for Disease Control
and Prevention, 27 percent of women and 11 percent of men in the United
States at some point endure stalking or sexual or physical violence by
an intimate partner that has significant effects. While comprehensive
numbers aren't available on domestic abuse cases involving digital
stalking in the United States, a small survey published in Australia in
2016 found that 17 percent of victims were tracked via GPS, including
through such apps.

In a Florida case involving abusive surveillance, a man named Luis
Toledo installed an app called SMS Tracker on his wife's phone in 2013
because he suspected she was having an affair. ``He said he was able to
see text messages and photos his wife was sending and receiving from
others,'' Sgt. A. J. Pagliari of the Volusia County Sheriff's Office
recalled.

This January, Mr. Toledo
\href{http://www.news-journalonline.com/news/20180119/luis-toledo-gets-3-consecutive-life-sentences-for-murders-of-wife-her-2-children}{was
sentenced} to three consecutive life terms after being convicted of
killing his wife, Yessenia Suarez, and her two children. Sergeant
Pagliari said Mr. Toledo told him he installed the app several days
before her death. ``With the use of the app, Toledo was able to confirm
his suspicion,'' the sergeant said.

Representatives for SMS Tracker, made by the Dallas-based Gizmoquip, did
not respond to requests for comment about the app's role in the case. A
recent review on the Google Play store for SMS Tracker tells potential
users: ``I would recommend if you think your partner is cheating.''

\hypertarget{an-opening-for-abuse}{%
\subsection{An Opening for Abuse}\label{an-opening-for-abuse}}

There is no federal law against location tracking, but such monitoring
can violate state laws on stalking. Spying on communications can break
statutes on wiretapping or computer crime. And knowingly selling illegal
wiretapping tools is a federal crime.

But it's not illegal to sell or use an app for tracking your children or
your own phone. And it can be difficult to tell whether the person being
surveilled has given consent, because abusers frequently coerce victims
into using such apps.

In Everson, Wash., for example, Brooks Owen Laughlin is accused of
beating his wife and using an app typically used for benign purposes,
Find My iPhone, to control her movements.

Image

Luis Toledo was convicted of killing his wife, Yessenia Suarez, and her
two children, Michael Otto, 8, left, and Thalia Otto, 9, right. Mr.
Toledo had installed an app called SMS Tracker on Ms. Suarez's phone
because he suspected she was having an affair.

``If she would turn it off, he would instantly call her or text her and
say, `Why did you turn that off? What are you doing?' That was pretty
much 24-7,'' Chief Daniel MacPhee of the Everson Police Department, said
in an interview. Mr. Laughlin pleaded not guilty in April to charges of
assault, harassment and stalking.

Such technical and legal ambiguity has created an environment in which
tools are marketed for both legal and illegal uses, without apparent
repercussion.

``There are definitely app makers that are complicit, seeking out these
customers and advertising this use,'' said Periwinkle Doerfler, a
doctoral student at New York University and an author of the study on
apps, which will be presented in the coming days. ``They're a little bit
under the radar about it, but they're still doing it.''

The researchers, from N.Y.U., Cornell University and Cornell Tech,
contacted customer support for nine companies with tracking services.
The researchers claimed to be women who wanted to secretly track their
husbands, and only one company, TeenSafe, refused to assist.

KidGuard, the app largely aimed at parents, also bought ads alongside
Google results for searches like ``catch cheating spouse app.'' A
spokesman for the business, based in Los Angeles, said in an email that
the company worked with third-party marketers and customer service reps
who had been ``testing new strategies.'' It deleted blog posts about
tracking romantic partners and said it did not support that activity.

Spyzie, another app that ran such ads, did not respond to requests for
comment.

On YouTube, dozens of videos provide tutorials on using several of the
apps to catch cheating lovers. The videos frequently link back to the
app makers' sites using a special code that ensures the promoter will
get a cut of the sale --- a type of deal known as affiliate marketing.

Affiliate marketing also appeared on multiple websites that discussed
using surveillance apps to track romantic partners. One site,
spyblog.ml, had posts about spying on ``loved ones'' and linked to mSpy.
The app company said that its terms of service prohibited illegal
activity and that it would block the site from its affiliate program.

\includegraphics{https://static01.nyt.com/images/2018/05/18/autossell/18stalkerware4/merlin_138093102_99f4c7e4-522e-4244-be2d-541cd7d327ef-articleLarge.jpg?quality=75\&auto=webp\&disable=upscale}

Reviews and online discussions about the apps suggest the market for
spying on spouses has been important to the businesses. FlexiSPY, an app
company, posted survey results on its site showing that 52 percent of
potential customers were interested because they thought their partners
might be cheating. Asked about the results, the company said the data
was five years old and ``no longer relevant.''

\hypertarget{different-phones-different-abilities}{%
\subsection{Different Phones, Different
Abilities}\label{different-phones-different-abilities}}

The proliferation of such tracking apps raises questions about the role
of businesses like Google and Apple in policing their services.

The two companies, which run nearly all smartphones in the United
States, have long taken different approaches to regulating apps.

Apple makes it difficult for iPhone users to download apps from outside
the company's App Store, and has many restrictions on what apps in its
store can do. After testing several programs available in the stores on
both platforms, the researchers found that Apple's strict rules resulted
in more limited surveillance capabilities on those apps than those
running Google's software.

Many App Store apps offered location tracking for phones. But for more
intrusive surveillance, spying companies had to work around Apple's
restrictions by using the victim's name and password to get data. To
combat misuse by predators, an Apple spokesman said, the company urges
people to use a tool called two-factor authentication to help protect
their accounts even if their passwords are stolen.

Google prides itself on being more open. Its smartphone software,
Android, allows people to install apps from anywhere, and the most
invasive ones were found outside the company's app store, Play.

The researchers found two apps in the Google Play store that allowed the
app icon to be hidden from victims and the camera to run without
notifications, as well as a handful of others that tracked users'
locations without telling them, all apparent violations of Google's
rules.

``They're not enforcing their own policies,'' Ms. Doerfler, the N.Y.U.
researcher, said. ``If someone reports it then they'll take it down, but
it's not something they are checking within their operating system.''

In response to the researchers' findings, Google tightened several
policies ``to further restrict the promotion and distribution'' of
surveillance apps, a company spokesman said. The company provides
funding to the N.Y.U. team that helped conduct the study.

Image

Tim Leslie, sheriff of Dakota County, Minn., with Derrick Warnecke, a
forensic specialist hired to help tackle the problem of digital stalking
in domestic violence.Credit...Jenn Ackerman for The New York Times

Google removed many spying and tracking apps and blocked advertising on
search results about spying on spouses and romantic partners. YouTube,
owned by Google, took down some videos about spying services, although
the company determined that others didn't violate its policies because
the services could be used with consent.

\hypertarget{enforcing-the-law}{%
\subsection{Enforcing the Law}\label{enforcing-the-law}}

Many law enforcement agencies don't have the computer skills to quickly
help survivors, or they don't devote forensic resources to domestic
abuse and stalking cases, which in many states are misdemeanors.

One sheriff's department, in Dakota County, Minn., is trying to tackle
the problem of abusive digital surveillance, and has used Justice
Department grants to hire a forensic specialist for the task.

The sheriff, Tim Leslie, said that from 2015 to 2017, the department
went to court in 198 cases involving technology and stalking or domestic
abuse, on par with earlier years. Its conviction rate rose to 94 percent
from 50 percent, with many more suspects pleading guilty instead of
contesting the charges, he said.

In one case, the specialist analyzed a woman's phone and found it had a
program on it called Mobile Spy, bought using her then-husband's email
address. The specialist could see that it had been launched 122 times.
The effect of the stalking was ``profound,'' the woman said.

Even though it had been more than a year since the app was last used,
the man was charged with misdemeanor stalking and pleaded guilty in
2015.

``We go after the misdemeanor stuff pretty hard, in the theory that if
you stop that, it doesn't escalate,'' Sheriff Leslie said.

Federal cases involving such spying are rare. The Justice Department in
2014 charged the maker of a spying program called StealthGenie under a
wiretap law that prohibits advertising and selling a device for
``surreptitious interception.'' The developer paid a \$500,000 fine,
shut down StealthGenie and was sentenced to time served.

Image

Phones that have been under investigation at the Dakota County Sheriff's
Office.Credit...Jenn Ackerman for The New York Times

Victims' advocates said they noticed after the case that makers of
surveillance tools changed their tactics, sometimes moving computer
servers overseas or scrubbing explicit language about spousal spying
from their websites. ``As soon as these companies caught wind that they
shouldn't be doing it, they just changed their marketing,'' Ms. Olsen
said.

One app maker told The Times that he hired a legal team after the
StealthGenie case to help him avoid running afoul of the law. ``There
were a few modifications we had to make,'' said Patrick Hinchy, the
founder of New York-based ILF Mobile Apps, which makes Highster Mobile
and other services. Several apps, he said, removed call recording and
delayed the availability of the data by 10 to 15 minutes. Mr. Hinchy
said the company only provided assistance to customers that it believed
was legal.

When a researcher recently contacted the company and asked, ``If I use
this app to track my husband, will he know that I am tracking him?'' the
representative responded: ``Our software is undetectable from the home
screen.''

Advertisement

\protect\hyperlink{after-bottom}{Continue reading the main story}

\hypertarget{site-index}{%
\subsection{Site Index}\label{site-index}}

\hypertarget{site-information-navigation}{%
\subsection{Site Information
Navigation}\label{site-information-navigation}}

\begin{itemize}
\tightlist
\item
  \href{https://help.nytimes.com/hc/en-us/articles/115014792127-Copyright-notice}{©~2020~The
  New York Times Company}
\end{itemize}

\begin{itemize}
\tightlist
\item
  \href{https://www.nytco.com/}{NYTCo}
\item
  \href{https://help.nytimes.com/hc/en-us/articles/115015385887-Contact-Us}{Contact
  Us}
\item
  \href{https://www.nytco.com/careers/}{Work with us}
\item
  \href{https://nytmediakit.com/}{Advertise}
\item
  \href{http://www.tbrandstudio.com/}{T Brand Studio}
\item
  \href{https://www.nytimes.com/privacy/cookie-policy\#how-do-i-manage-trackers}{Your
  Ad Choices}
\item
  \href{https://www.nytimes.com/privacy}{Privacy}
\item
  \href{https://help.nytimes.com/hc/en-us/articles/115014893428-Terms-of-service}{Terms
  of Service}
\item
  \href{https://help.nytimes.com/hc/en-us/articles/115014893968-Terms-of-sale}{Terms
  of Sale}
\item
  \href{https://spiderbites.nytimes.com}{Site Map}
\item
  \href{https://help.nytimes.com/hc/en-us}{Help}
\item
  \href{https://www.nytimes.com/subscription?campaignId=37WXW}{Subscriptions}
\end{itemize}
