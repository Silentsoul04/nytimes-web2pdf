Sections

SEARCH

\protect\hyperlink{site-content}{Skip to
content}\protect\hyperlink{site-index}{Skip to site index}

\href{https://myaccount.nytimes.com/auth/login?response_type=cookie\&client_id=vi}{}

\href{https://www.nytimes.com/section/todayspaper}{Today's Paper}

\href{/section/opinion}{Opinion}\textbar{}Finding Myself in Research

\href{https://nyti.ms/2LrktuG}{https://nyti.ms/2LrktuG}

\begin{itemize}
\item
\item
\item
\item
\item
\end{itemize}

Advertisement

\protect\hyperlink{after-top}{Continue reading the main story}

Supported by

\protect\hyperlink{after-sponsor}{Continue reading the main story}

\href{/section/opinion}{Opinion}

\href{/column/on-campus}{On Campus}

\hypertarget{finding-myself-in-research}{%
\section{Finding Myself in Research}\label{finding-myself-in-research}}

By Mya Roberson

Ms. Roberson is a graduate student at UNC-Chapel Hill.

\begin{itemize}
\item
  May 23, 2018
\item
  \begin{itemize}
  \item
  \item
  \item
  \item
  \item
  \end{itemize}
\end{itemize}

\includegraphics{https://static01.nyt.com/images/2018/05/23/opinion/23oncampus-Roberson/merlin_138535914_3514feb9-e12d-498d-b3b5-7d42351097d1-articleLarge.jpg?quality=75\&auto=webp\&disable=upscale}

I'm a black, first-generation college graduate from a low-income
Appalachian community in Pennsylvania. It was statistically unlikely
that I would complete a four-year bachelor's degree. It was even less
likely that I would further my education beyond that.

Now, as a graduate student in epidemiology at the University of North
Carolina, I spend my time studying health disparities, the variation in
rates of disease between socioeconomic and racial groups. Probability
--- as well as my lived experience --- says I'm an anomaly. People with
Ph.D.'s do not look like me, and do not come from where I come from.

I was drawn to study cancer because of the unknown. It is not like
diabetes, a disease about which much is known on treatment and
prevention. I also wanted to make the United States a healthier place
for groups like black women, who suffer disproportionately from diseases
like breast cancer.

As a child, I witnessed the reality of health disparities in my own
family. My maternal white grandmother received dialysis for diabetes
that extended her life, so she could witness the birth of my niece, her
great-granddaughter. My paternal black grandmother had a leg amputated
because of diabetes and ultimately died of it, when I was just a young
girl. Both of my grandmothers lived in cities in Florida, yet had vastly
different outcomes for the same disease.

Last fall, I flew to Atlanta to present my work at a conference called
the Science of Cancer Health Disparities in Racial/Ethnic Minorities and
the Medically Underserved, held by the American Association for Cancer
Research. This conference is a big deal in my field. Once a year, the
most notable names in this research are in attendance. This was my first
conference as a graduate student. Needless to say, I was nervous.

As I stood next to the poster detailing my work down to the level of
molecular characteristics of breast cancer, researchers from all over
the country stopped to question me. I took copious notes, eager to
capture a fraction of the ideas and inspiration in the room. I even met
some of my academic idols. I felt like a real scientist, engaged in the
process of scientific inquiry.

Later in the session, a black woman with silver hair and no
institutional affiliation listed on her name tag approached me. Before I
could begin my standard elevator pitch, she said she'd prefer to read my
entire poster first. I studied her face as she read. A few minutes
passed. The concentration in her face gradually shifted to raw emotion.

After she finished reading, she introduced herself as a survivor
advocate, someone who is not a scientist but interacts with them to give
the patient and survivor perspective. She shared her medical history ---
her diagnosis of severe endometriosis in her early 30s, the removal of
her ovaries and uterus in an attempt to allay the symptoms, and her more
recent diagnosis of breast cancer, which led to her advocacy in the
cancer research community.

Then, she reached out and touched my poster and said, ``I see myself in
this research. This was a study meant for women like me.''

Her personal revelations stood in stark contrast to my previous
interactions with fellow scientists, which were mechanical and
formulaic. This was different. It was two black women talking about our
resilience.

Graduate school is a notoriously isolating experience. Ph.D. candidates
at American research institutions have six times as much anxiety and
depression as the general population,
\href{http://www.sciencemag.org/careers/2018/03/graduate-students-need-more-mental-health-support-new-study-highlights}{according
to a study} published earlier this year. For people like me, the already
stressful experience takes a different shape. When you don't fit the
mold of a traditional graduate student, there exists an intricate
interplay between impostor syndrome, social support, sense of purpose
and mental health. Out of self-preservation, I immersed myself in the
scientific process in an earnest attempt to avoid the isolation that too
often accompanies graduate school.

Instead, I spent my first year of graduate school in front of a computer
screen with de-identified subject identification numbers. A screen full
of numbers indicated whether a woman smoked or not, had children or not,
had a family history of breast cancer or not, and whether she had
succumbed to breast cancer or not. Women's entire lives were distilled
into data; their health care life cycles and eventual deaths now were 0s
and 1s on my screen. Despite my deep sense of purpose, emotionally, I
felt removed from the work.

This made my experience with that woman at the poster session all the
more meaningful. Now, when I present my analyses of the binary numbers
representing women who gave part of themselves for the advancement of
cancer research, I include her story. To me, her interaction with the
humanity of science is just as important as the output of my statistical
models.

In graduate school, we are urged to publish in the most prominent
journals and pursue prestigious fellowships. The number of peer
citations or research dollars measures success. What is often absent is
the consideration of how research affects everyday individuals. My
experience with the woman at the poster session reminded me that I am
not just doing research to become a known scholar in my field. I
research for the sake of humanity. All researchers could use that
reminder.

Advertisement

\protect\hyperlink{after-bottom}{Continue reading the main story}

\hypertarget{site-index}{%
\subsection{Site Index}\label{site-index}}

\hypertarget{site-information-navigation}{%
\subsection{Site Information
Navigation}\label{site-information-navigation}}

\begin{itemize}
\tightlist
\item
  \href{https://help.nytimes.com/hc/en-us/articles/115014792127-Copyright-notice}{©~2020~The
  New York Times Company}
\end{itemize}

\begin{itemize}
\tightlist
\item
  \href{https://www.nytco.com/}{NYTCo}
\item
  \href{https://help.nytimes.com/hc/en-us/articles/115015385887-Contact-Us}{Contact
  Us}
\item
  \href{https://www.nytco.com/careers/}{Work with us}
\item
  \href{https://nytmediakit.com/}{Advertise}
\item
  \href{http://www.tbrandstudio.com/}{T Brand Studio}
\item
  \href{https://www.nytimes.com/privacy/cookie-policy\#how-do-i-manage-trackers}{Your
  Ad Choices}
\item
  \href{https://www.nytimes.com/privacy}{Privacy}
\item
  \href{https://help.nytimes.com/hc/en-us/articles/115014893428-Terms-of-service}{Terms
  of Service}
\item
  \href{https://help.nytimes.com/hc/en-us/articles/115014893968-Terms-of-sale}{Terms
  of Sale}
\item
  \href{https://spiderbites.nytimes.com}{Site Map}
\item
  \href{https://help.nytimes.com/hc/en-us}{Help}
\item
  \href{https://www.nytimes.com/subscription?campaignId=37WXW}{Subscriptions}
\end{itemize}
