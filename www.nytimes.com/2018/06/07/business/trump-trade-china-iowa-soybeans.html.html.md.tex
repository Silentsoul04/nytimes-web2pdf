Sections

SEARCH

\protect\hyperlink{site-content}{Skip to
content}\protect\hyperlink{site-index}{Skip to site index}

\href{https://www.nytimes.com/section/business}{Business}

\href{https://myaccount.nytimes.com/auth/login?response_type=cookie\&client_id=vi}{}

\href{https://www.nytimes.com/section/todayspaper}{Today's Paper}

\href{/section/business}{Business}\textbar{}Why Trump Might Cave to
China: Iowa Soybean Farmers

\url{https://nyti.ms/2M6CI8Z}

\begin{itemize}
\item
\item
\item
\item
\item
\end{itemize}

Advertisement

\protect\hyperlink{after-top}{Continue reading the main story}

Supported by

\protect\hyperlink{after-sponsor}{Continue reading the main story}

\href{/column/common-sense}{Common Sense}

\hypertarget{why-trump-might-cave-to-china-iowa-soybean-farmers}{%
\section{Why Trump Might Cave to China: Iowa Soybean
Farmers}\label{why-trump-might-cave-to-china-iowa-soybean-farmers}}

\includegraphics{https://static01.nyt.com/images/2018/06/08/business/08stewart-1/08stewart-1-articleLarge.jpg?quality=75\&auto=webp\&disable=upscale}

By \href{https://www.nytimes.com/by/james-b-stewart}{James B. Stewart}

\begin{itemize}
\item
  June 7, 2018
\item
  \begin{itemize}
  \item
  \item
  \item
  \item
  \item
  \end{itemize}
\end{itemize}

\href{http://cn.nytimes.com/business/20180608/trump-trade-china-iowa-soybeans/}{阅读简体中文版}\href{http://cn.nytimes.com/business/20180608/trump-trade-china-iowa-soybeans/zh-hant/}{閱讀繁體中文版}

For all his bluster about trade wars, President Trump seems willing to
push China only so far: Witness
\href{https://www.nytimes.com/2018/06/07/business/us-china-zte-deal.html}{the
deal on Thursday} to grant Chinese telecom giant ZTE a reprieve from
harsh American penalties. The reason is likely to lead straight to Iowa
soybean and corn farmers like Benjamin Schmidt.

Mr. Schmidt's forebears have farmed the same land outside Iowa City for
nearly 150 years. He and his father together till about 2,500 acres of
the fertile prairie that stretches from Ohio through Nebraska. When I
reached him last week, he was on his tractor, spreading fertilizer on
this year's corn crop.

Apart from the weather, hardly any issue looms larger for farmers than
the prospect of retaliatory tariffs against American agriculture
products. China has threatened a 25 percent tariff on soybeans and has
already sharply curtailed purchases from the United States. This week
Mexico imposed a 20 percent tariff on pork. The European Union and
Canada have said they, too, will slap tariffs on a variety of American
agricultural products.

``China is our most important export market for soybeans,'' Mr. Schmidt
said. ``When your most important customer hits you with tariffs, there
are going to be serious ramifications. My first reaction was this is
going to hit us pretty hard.''

Grant Kimberley, who with his father farms 4,000 acres near Maxwell,
Iowa, and is director of market development for the Iowa Soybean
Association, was even more emphatic: ``We want to sell to China, Mexico,
whoever. We should be part of the solution, which is bringing down the
trade imbalance.''

American farmers may be dwindling in absolute numbers, but they wield
outsize influence in the raging war between protectionists and free
traders in the Trump White House. That's because of both the importance
of their occupation to the balance of trade --- United States
agricultural exports have averaged nearly \$140 billion a year since
2010 --- and their geographical concentration in states that were
critical to Mr. Trump's 2016 electoral majority.

\includegraphics{https://static01.nyt.com/images/2018/06/08/business/08stewart-2/08stewart-2-articleLarge.jpg?quality=75\&auto=webp\&disable=upscale}

Much of the farm belt is solidly Republican. But Iowa, Wisconsin and
Minnesota are presidential battlegrounds, where even a small defection
of farmers could doom Mr. Trump's re-election prospects.

Later this year, hotly contested Senate races in a swath of farm states
--- Minnesota, Wisconsin, Missouri, Indiana and North Dakota --- will
determine whether Republicans can maintain their majority in the Senate.

Mr. Schmidt is one of those voters up for grabs. He voted for Mr. Trump
and leans Republican. ``But I'm more of an independent,'' he said. ``I'm
no Trumpeter. I'm still pondering whether he's the right person for the
job.''

He pointed out that Mr. Trump recently wrote on Twitter that China would
be buying ``massive amounts'' of United States agricultural products,
``one of the best things to happen to our farmers in many years,'' only
to renew the tariff threat days later.

``He says one thing and then two days later something else happens,''
Mr. Schmidt said. ``It's like North Korea. It's on, then it's off, then
it's on again.''

Mr. Schmidt and other Iowa farmers are turning for help to one of their
senators, Charles Grassley, a Republican who they note is himself a
farmer.

Steffen Schmidt, professor of political science at Iowa State University
in Ames, credited Mr. Grassley with being ``the single most powerful
senator'' right now, thanks to his seniority and chairmanship of the
powerful Judiciary Committee, which is investigating Russian
interference in Mr. Trump's election.

Image

Benjamin Schmidt with his father, David, and a friend at the family
farm. ``When your most important customer hits you with tariffs,''
Benjamin Schmidt said, ``there are going to be serious
ramifications.''Credit...Whitten Sabbatini for The New York Times

When I spoke to Mr. Grassley this week, he said he'd met with the
president three times --- and his advisers at least a dozen times ---
over the last year to discuss trade issues. ``I know Trump heard what we
said, but I don't know what impact it made on him,'' Mr. Grassley said.
``Because today there are going to be tariffs, and tomorrow there
aren't. I sincerely believe him when the president says he likes
farmers. But I don't feel he understands the economics of agriculture
the way he understands real estate.''

``Whenever there's tariff retaliation agriculture is the first thing
hit,'' he continued. ``We just hope the president knows what he's doing,
and we hope he's negotiating in good faith to get a better deal for us.
If he doesn't, it's going to be catastrophic for agriculture.''

Mr. Grassley said he didn't want to make any predictions about the
midterm or presidential elections, but said, ``If this turns out to be
catastrophic, there's obviously going to be real disappointment'' among
voters in farm states.

Mr. Trump clearly recognizes the high stakes. In a recent White House
meeting with lawmakers and governors from a number of farm states, he
pledged that the federal government would support agricultural prices
should retaliatory tariffs cause the price of soybeans, corn and other
major exports to fall.

``He wasn't specific, but he assured us Sonny Perdue has a plan,'' Mr.
Grassley said, referring to the secretary of agriculture. ``Our response
was unanimous. I'm paraphrasing, but the message was, we don't want help
from the Treasury. We want free and open markets.''

Mr. Kimberley had a similar reaction. And he asked a question that still
resonates in Iowa: ``Who's going to pay for this? Soybean exports alone
account for \$14 billion a year. You're talking billions of dollars.''

Everyone I spoke to in Iowa agreed that China engages in a range of
unfair trade practices that need to be addressed. But no one said a
tariff war is the way to do it. ``China is projected to be one of the
fastest growing export markets over the next decade, and we want to be
part of that growth,'' Mr. Kimberley said.

Image

American farmers may be dwindling in absolute numbers, but they wield
outsize influence in the raging war between protectionists and free
traders in the Trump White House.Credit...Whitten Sabbatini for The New
York Times

While some farmers ``want to hold China's feet to the fire,'' he said,
the Trump administration may be underestimating the ties that have
developed in recent decades between American farmers and Chinese
leaders.

Mr. Kimberley told me he's visited China on average more than once a
year over the last decade to promote soybeans. Iowa and China's Hebei
Province have been official sister states since 1985. In 2012, Xi
Jinping, then vice president and now president of China,
\href{https://www.nytimes.com/2016/12/09/world/asia/china-iowa-farm-branstad-xi.html}{visited
Mr. Kimberley's family farm}, where Mr. Kimberley greeted him in
Mandarin. It's not a coincidence that the United States ambassador in
Beijing is Iowa's former governor Terry Branstad.

So far, commodity markets have largely shrugged off threats of
agriculture tariffs, presumably because traders don't believe they'll go
into effect. ``When I run into my neighbors at the feed store, they're
saying they don't think there will be a trade war,'' Professor Schmidt
said. ``They say that Trump's bluffing; he's a deal maker. I hope
they're right.''

This week's ZTE deal is an example of the Trump administration making
concessions to China --- an encouraging sign for farmers like Mr.
Schmidt.

After ZTE violated American sanctions on Iran, the United States in
April threatened to bar it from doing business in the country. The
Chinese government was furious that the Trump administration was
essentially putting a major company out of business.

\emph{{[}}\href{https://www.nytimes.com/2018/06/07/business/what-is-zte.html}{\emph{Catch
up}} \emph{on what ZTE is and why President Trump wants to help it.{]}}

On Thursday, Commerce Secretary Wilbur Ross said the United States and
China had reached an agreement in which ZTE would pay a \$1 billion fine
and allow American officials into the company to monitor its compliance
with sanctions.

Mr. Ross said the ZTE deal wasn't connected to broader trade issues, but
Mr. Kimberley hopes the Trump administration will also compromise on
tariffs. ``There's going to be posturing on both sides, but there's a
lot more to be gained by meeting in the middle and finding a solution,''
he said. This week, for example,
\href{https://www.nytimes.com/2018/06/05/us/politics/china-trump-trade-tariffs.html}{China
offered} to buy nearly \$70 billion of American goods, including farm
exports, if the United States drops its tariff threat.

Mr. Schmidt said he, too, is optimistic a tariff war can be averted. ``I
can't say Trump has gone about this the way I would,'' he said from his
tractor. ``As a farmer, I try to stay positive.''

Advertisement

\protect\hyperlink{after-bottom}{Continue reading the main story}

\hypertarget{site-index}{%
\subsection{Site Index}\label{site-index}}

\hypertarget{site-information-navigation}{%
\subsection{Site Information
Navigation}\label{site-information-navigation}}

\begin{itemize}
\tightlist
\item
  \href{https://help.nytimes.com/hc/en-us/articles/115014792127-Copyright-notice}{©~2020~The
  New York Times Company}
\end{itemize}

\begin{itemize}
\tightlist
\item
  \href{https://www.nytco.com/}{NYTCo}
\item
  \href{https://help.nytimes.com/hc/en-us/articles/115015385887-Contact-Us}{Contact
  Us}
\item
  \href{https://www.nytco.com/careers/}{Work with us}
\item
  \href{https://nytmediakit.com/}{Advertise}
\item
  \href{http://www.tbrandstudio.com/}{T Brand Studio}
\item
  \href{https://www.nytimes.com/privacy/cookie-policy\#how-do-i-manage-trackers}{Your
  Ad Choices}
\item
  \href{https://www.nytimes.com/privacy}{Privacy}
\item
  \href{https://help.nytimes.com/hc/en-us/articles/115014893428-Terms-of-service}{Terms
  of Service}
\item
  \href{https://help.nytimes.com/hc/en-us/articles/115014893968-Terms-of-sale}{Terms
  of Sale}
\item
  \href{https://spiderbites.nytimes.com}{Site Map}
\item
  \href{https://help.nytimes.com/hc/en-us}{Help}
\item
  \href{https://www.nytimes.com/subscription?campaignId=37WXW}{Subscriptions}
\end{itemize}
