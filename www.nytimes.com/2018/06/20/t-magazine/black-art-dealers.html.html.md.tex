Sections

SEARCH

\protect\hyperlink{site-content}{Skip to
content}\protect\hyperlink{site-index}{Skip to site index}

\href{https://myaccount.nytimes.com/auth/login?response_type=cookie\&client_id=vi}{}

\href{https://www.nytimes.com/section/todayspaper}{Today's Paper}

Why Have There Been No Great Black Art Dealers?

\href{https://nyti.ms/2K5UvMh}{https://nyti.ms/2K5UvMh}

\begin{itemize}
\item
\item
\item
\item
\item
\end{itemize}

Advertisement

\protect\hyperlink{after-top}{Continue reading the main story}

Supported by

\protect\hyperlink{after-sponsor}{Continue reading the main story}

\hypertarget{why-have-there-been-no-great-black-art-dealers}{%
\section{Why Have There Been No Great Black Art
Dealers?}\label{why-have-there-been-no-great-black-art-dealers}}

How a small but influential group of black gallerists is correcting
history.

\includegraphics{https://static01.nyt.com/images/2018/06/21/t-magazine/art/art-dealers-slide-21LU/art-dealers-slide-21LU-articleLarge.jpg?quality=75\&auto=webp\&disable=upscale}

By \href{https://www.nytimes.com/by/janelle-zara}{Janelle Zara}

\begin{itemize}
\item
  June 20, 2018
\item
  \begin{itemize}
  \item
  \item
  \item
  \item
  \item
  \end{itemize}
\end{itemize}

IN 1966, TWO BROTHERS, Alonzo and Dale Davis, set out from Los Angeles
on a road trip across the United States, seeking out other artists of
color like them. They meant for the trip ``to broaden our limited art
history experience,''
\href{https://hammer.ucla.edu/now-dig-this/artists/alonzo-davis/}{Alonzo}
says, since African-American artists had been conspicuously absent from
his curriculum at Pepperdine University, or
\href{https://hammer.ucla.edu/now-dig-this/artists/dale-brockman-davis/}{Dale}'s
at the University of Southern California. ``We drove from L.A. to
Mississippi, up through New York and Chicago, and somewhere between all
those cornfields, we thought: it'd be interesting to own a gallery.''

The following year, using Alonzo's high school art teacher salary, they
opened Brockman Gallery, where artists of color ---
\href{http://www.jacktiltongallery.com/artists/outterbridge/}{John
Outterbridge}, \href{http://www.betyesaar.net/}{Betye Saar}, and
\href{http://www.noahpurifoy.com/about-noah/}{Noah Purifoy} among them
--- could show and sell their art. They opened in Leimert Park, then a
middle-class, predominantly black neighborhood of Los Angeles, and
called it Brockman Gallery, after their grandmother Della Brockman,
whose mother was a slave from Charleston, South Carolina. Their
upbringing in Tuskegee, Alabama, had exposed the Davis brothers to the
possibilities of successful black-owned businesses, but their gallery
treaded uncharted territory on a number of levels. The overall L.A.
gallery scene of the '60s was influential but small, and focused mostly
on white men who made Conceptual or Pop Art; Andy Warhol had his first
exhibition ever in 1962 at the Ferus Gallery on North La Cienega
Boulevard, which closed just before Brockman Gallery opened. Until then,
black artists had been relegated to showing their work at salons and
community centers, and both brothers had been specifically advised to
get teaching credentials rather than try to make a living as artists.
Only one full year had passed since the Watts riots had led to violent
clashes between black residents and the L.A.P.D., and the city was still
four years away from the Los Angeles County Museum of Art electing its
first black board member.

The art world of the '60s and '70s in general, ``was a hostile
environment for black folks,'' recalls Linda Goode Bryant, who, in 1974,
challenged the white establishment in New York by opening a gallery of
her own. Black artists then were still embarrassingly absent from
museums (a 1969 show at the Metropolitan Museum of Art, called ``Harlem
on My Mind: Cultural Capital of Black America, 1900-1968,'' infamously
contained no work by black artists) and there were wild discrepancies in
value between white artists and their non-white peers. Realtors,
unreceptive to the idea of showing work by ``black artists,'' would hang
up on Bryant, and it wasn't until she started calling her gallery a
place for ``emerging artists'' that she could secure a space. She went
directly to 57th Street, then the financial heart of the art world, and
Just Above Midtown (better known as JAM) was born.

\includegraphics{https://static01.nyt.com/images/2018/06/21/t-magazine/art/art-dealers-slide-LZNF/art-dealers-slide-LZNF-articleLarge.jpg?quality=75\&auto=webp\&disable=upscale}

``The art world was angry --- they were angry that I was there, and that
the realtor had leased me the space,'' says Bryant. ``Dealers would say
nasty things to me in the elevator. But, hell yeah, we succeeded on a
lot of levels.'' JAM became an interdisciplinary community of artists
and curators, including Lowery Stokes Sims, the first black curator at
the Metropolitan Museum of Art; and multimedia artists
\href{http://www.garthgreenan.com/artists/howardena-pindell}{Howardena
Pindell} and \href{http://lorraineogrady.com/}{Lorraine O'Grady},
fostering, Bryant says, ``an appreciation for what was possible, not
those little boxes we had contained ourselves in.''

What these art dealers understood is that the gallery, as an entrée into
the art market, is the sole platform for an artist to make a living. And
in many ways, galleries are where the hierarchy of power in the art
world begins and ends. They discover an artist's work and promote it to
both collectors and institutions; the work rises in value once it enters
a museum, and this ultimately leads to more gallery shows. It is an
unchanging cycle that for decades artists of color, lacking a commercial
outlet, ``couldn't even attempt to break into,'' according to Bryant.

In the last decade, major museums have amped up efforts to re-examine
the past, unearthing the work of artists who had previously been
excluded. For black artists in particular, MoMA hired a consulting
curator to broaden its collection in 2014, the same year that the
Guggenheim mounted
``\href{https://www.guggenheim.org/exhibition/carrie-mae-weems-three-decades-of-photography-and-video}{Carrie
Mae Weems: Three Decades of Photography and Video},'' the first
retrospective of a black female artist in the museum's history.
Institutions nationwide, including the Detroit Institute of Arts and the
Pérez Art Museum Miami, have set aside millions of dollars toward the
acquisition of African-American art, and in May, the Baltimore Museum of
Art made the controversial announcement that it would be deaccessioning
works by Andy Warhol and Robert Rauschenberg to make room for work by
women and artists of color. It wasn't until this year, at age 74, that
former JAM artist Howardena Pindell had her first major museum survey at
the Museum of Contemporary Art Chicago. The desire for museums to patch
the holes in art history is strong, but for so many artists, it comes
too late; LACMA's 2015 Noah Purifoy survey arrived 11 years after his
death, and the Pennsylvania Academy of the Fine Arts'
\href{https://www.pafa.org/normanlewis}{Norman Lewis} retrospective came
36 years after his.

During this newfound institutional interest, critics and historians have
described artists like these as ``overlooked,'' while the more difficult
truth is that they were willfully ignored. But if artists of color were,
until recently, effectively written out of art history, black dealers
have remained almost entirely absent from the narrative of contemporary
art. A black-owned gallery is to this day an exception, though in the
last few years, a small group of black gallery owners and directors ---
taking their cue from an even smaller group of forebears --- are working
hard to prevent the art world from repeating its mistakes.

Image

Mariane Ibrahim opened her Seattle gallery in 2012.Credit...Sean Donnola

CONTRARY TO THE assumption that society moves toward equality on its
own, the ascent of black artists into the status quo has been a result
of diligent actors. It has been helped enormously through a dedicated
group that includes Joeonna Bellorado-Samuels, a director at Jack
Shainman, New York gallery with a roster of largely black artists
including \href{http://www.jackshainman.com/artists/nick-cave/}{Nick
Cave}, \href{http://carriemaeweems.net/}{Carrie Mae Weems} and the
estate of \href{http://www.gordonparksfoundation.org/}{Gordon Parks};
Mariane Ibrahim, who founded her Seattle gallery in 2012; the San
Francisco-based Karen Jenkins-Johnson, who recently expanded to
Brooklyn; and a rising population of black staffers, who for so long
were not present in most galleries at all.

But in 2018, even as black artists enjoy growing acclaim, American art
continues to privilege the perspective of white men. While Shainman is a
longstanding champion of artists of color, Bellorado-Samuels --- who has
worked at the gallery for ten years --- is still one of the few black
dealers in Chelsea. This kind of perspective has marginalized black
artists in a way that is only just being reversed. All the way back in
1975, a young David Hammons, now one of the most famous and highly
valued living artists who would bring his early paintings into Brockman
Gallery while they were still wet, described the phenomenon of white
curators lumping black artists together in shows, no matter how
dissimilar their work, as if being black alone was their only
distinguishing virtue. ``Throwing everyone into a barrel --- that
bothers me, that that's still happening,'' he said. Almost 50 years
later, it is \emph{still} happening, though having more black gallerists
helps matters.

``If someone wants to do a `black art' show and put together several of
my artists who are only thematically linked only by a thread, we're
going to have a conversation about that,'' Bellorado-Samuels says.
``We're the artist's first line of defense; part of our responsibility
is to build their market, and another is to navigate how we talk about
them, and how we contextualize them.''

In 2017 Belloarado-Samuels opened \href{https://webuygold.wtf/}{We Buy
Gold}, her own roving exhibition space that began in Bedford-Stuyvesant
and currently resides in Chinatown before it moves on again. Unconfined
by the art world establishment, it provides space for emerging and
mid-career artists, of color or not. Working outside the limited
perspective of a predominantly white art world, spaces like hers
effectively broaden it.

Likewise, Ebony L. Haynes, director of
\href{http://www.martosgallery.com/}{Martos Gallery}, has put on two
group shows based on pivotal works by black authors: Ralph Ellison's
``Invisible Man,'' and, at the gallery's project space in New York
called Shoot the Lobster, Octavia Butler's ``Bloodchild.'' Each
showcased black artists without needing to bill itself as an ``all-black
show.'' Karen Jenkins-Johnson, who opened her
\href{http://www.jenkinsjohnsongallery.com/}{first gallery} in 1996, has
been proactively growing the small percentage of black collectors in the
art market, and in 2017 opened a second space in Lefferts Gardens
dedicated to providing artists of color space to curate. Mariane
Ibrahim, who grew up in Somaliland, opened her
\href{http://marianeibrahim.com/}{namesake Seattle gallery} with a
roster largely of African and diasporic artists as a corrective to all
the African folk art exhibitions she had seen growing up. To her, they
felt as though, ``Europe and America were holding a telescope to Africa
with white gloves on.''

The work of Brockman Gallery, JAM, and many of their artists rode the
activist momentum of the Civil Rights movement; it was through the
persistence of the Black Arts Council (BAC) that LACMA had its first
show of black artists,
``\href{http://www.lacma.org/sites/default/files/ThreeGraphicArtists.pdf}{Three
Graphic Artists: Charles White. David Hammons. Timothy Washington,}'' in
1971. The Davis brothers had to cross picket lines to get their artists
in --- BAC had subsequently staged a protest of the museum's exhibiting
a nationally recognized name like White's alongside two emerging
artists, inside a small prints and drawings department gallery, no less.

The one universal truism among good gallerists, says Joeonna
Bellorado-Samuels, a director at Jack Shainman gallery, is that, ``At
the end of the day, gallery work is advocacy work.'' Similar to the
Civil Rights movements of the '60s and '70s, more recent political
shifts that brought black issues to mainstream attention seep into the
art world and push it forward: ``Black Lives Matter trickled down into a
lot of day-to-day, regular life for people in different quadrants and
insular economies,'' says Haynes. The art world also recently witnessed
the organized protest against the depiction of Emmett Till by Dana
Schutz, a white artist, as well as the painting of
\href{https://www.nytimes.com/2018/02/12/arts/design/obama-portrait.html}{presidential
portraits} by two black artists, \href{http://kehindewiley.com/}{Kehinde
Wiley} and \href{http://www.amysherald.com/}{Amy Sherald}. All of these
follow the inauguration of America's first black president, which
Jenkins-Johnson describes as a ``sea change'': the normalization of a
black man as the most powerful figure in the country.

The pioneering black art dealers have all since moved on: JAM closed in
1986, after which Bryant became senior policy analyst for development
during the Mayor David Dinkins's administration and a Peabody
Award-winning documentary filmmaker; Brockman Gallery shuttered in 1990,
and both brothers continued their respective teaching careers and
artistic practices, and Alonzo moved to the East Coast. In the path they
cleared, black gallerists continue to disrupt the art world's
homogeneity, one show, one art fair, or one press release at a time.

Image

Peter A. Bradley is one of New York's original black art dealers and an
abstract painter.Credit...Sean Donnola

WHY HAVE THERE been so few black gallerists? Besides the legacy of their
historic exclusion, one reason is that starting a gallery takes
tremendous resources; Linda Goode Bryant had been lucky enough to find a
landlord who would turn a blind eye to her absent rent payments, and a
community of artists and curators who would install the parquet flooring
for free. Another reason: The predominantly white art world can still be
an uncomfortable space for a person of color to navigate, both in the
making of art and the selling of it.

The abstract painter \href{http://www.peterabradley.com/}{Peter A.
Bradley}, one of New York's original black art dealers, has known the
highs and lows of both. As a struggling artist he worked as an art
handler at Perls Gallery, an Uptown dealer of Picasso and Modigliani's,
until he was promoted to a sales position; as an artist himself, he knew
the language of paintings well enough to sell them. As the gallery's
associate director he experienced the joys of tailored suits, expensed
lunches with Alexander Calder and business trips to Europe; but still,
on two occasions, was assailed with an ugly racial epithet. Both times,
his boss verbally smacked down the culprits: The Perls, Bradley says,
``really protected me big time because I made them a lot of money.''

Bradley's position put him in close contact with the art world's high
society, which had been closed off to him as an artist; it's how he met
dealer André Emmerich, who would give Bradley his first show in 1972.
The Whitney had also invited Bradley to participate in the 1971
``Contemporary Black Artists in America,'' an exhibition in which 15 of
the 75 artists withdrew because of the white curator's minimal
consultation with experts of color. For his part, Bradley declined
because of its reduction of him to a \emph{black} artist. Instead, with
the support of Houston-based collector John de Menil, he put together
what is widely considered one of the first racially integrated art shows
in the country's history, in Houston's rundown DeLuxe Theater. Here, he
put artists of color --- including himself --- alongside some of the
most famous artists of the day, people like Kenneth Noland, Larry Poons
and numerous others who were free, due to the color of their skin, to
simply be artists, no modifying adjective necessary.

Somewhere between the DeLuxe Show and today, America was forced to
confront its longstanding misperceptions on the value of black culture.
Alongside the art world, Hollywood is re-examining its own history of
exclusion: it clung to the assumption that a film with a black cast
couldn't sell tickets until very recently, when the box office success
of ``Black Panther'' and ``Get Out'' proved that kind of myth untenable.
Similarly, the art world had reinforced its own assumption that no one
would buy black art by not selling it, only now reckoning with it as a
commercial force: In the last 10 years, according to a 2016
\href{https://news.artnet.com/market/racial-bias-art-auction-market-672518}{Artnet
analysis}, Glenn Ligon, Mark Bradford and Julie Mehretu --- three
acclaimed black artists --- all joined the ranks of top 10 most valuable
**** American artists born after 1955.

Haynes, happy about the overall climate, is still reticent to hand out
pats on the back. ``A lot of time has passed that no one paid attention,
and no one should be congratulated for paying attention now.'' And
during a time when the representation of blackness within the mainstream
has become an increasingly central civil rights issue, to be a black
gallerist in 2018 still entails the tedious sidestepping of both the
fetishization of black artists, as well as the assumptions of what a
gallerist should look like.

Bellorado-Samuels recalls how at a recent dinner, the daughter of a
prominent art collector turned to her and asked her if she was an
artist. ``I said no, and jokingly told her, `I'll give you three more
tries,''' she says. The collector's daughter went down the list: Writer?
Publicist? Curator? She finally cocked her head to the side and said,
``You're not a \emph{dealer}, are you?'' Incidents like this happen with
disheartening frequency. Jenkins-Johnson, who shows at fairs including
Art Basel, Untitled and Expo Chicago, regularly watches potential
collectors enter her booth and direct their questions to her white
employees. It's a symptom, she says, of the ``systemic understanding of
how bodies should be treated and valued'' based on their color. When
Haynes experiences these kinds of transgressions, she largely keeps them
to herself. ``I take the position that making it public might put us in
a different category of dealer,'' **** she says, noting the tenuous line
between leaning into one's blackness and being defined by it.

Progress, however, does reveal itself in small victories.
Jenkins-Johnson recalls participating in the 2017 edition of the
Association of International Photography Art Dealers fair in New York,
where she was exhibiting next to Ibrahim and Atlanta-based gallerist
Arnika Dawkins. ``That was a groundbreaking time,'' Jenkins-Johnson
says, ``to have three black-owned galleries in one fair.'' Her first
instinct was to pull out her camera. ``I said, `We gotta have a photo to
mark this occasion.'''

Advertisement

\protect\hyperlink{after-bottom}{Continue reading the main story}

\hypertarget{site-index}{%
\subsection{Site Index}\label{site-index}}

\hypertarget{site-information-navigation}{%
\subsection{Site Information
Navigation}\label{site-information-navigation}}

\begin{itemize}
\tightlist
\item
  \href{https://help.nytimes.com/hc/en-us/articles/115014792127-Copyright-notice}{©~2020~The
  New York Times Company}
\end{itemize}

\begin{itemize}
\tightlist
\item
  \href{https://www.nytco.com/}{NYTCo}
\item
  \href{https://help.nytimes.com/hc/en-us/articles/115015385887-Contact-Us}{Contact
  Us}
\item
  \href{https://www.nytco.com/careers/}{Work with us}
\item
  \href{https://nytmediakit.com/}{Advertise}
\item
  \href{http://www.tbrandstudio.com/}{T Brand Studio}
\item
  \href{https://www.nytimes.com/privacy/cookie-policy\#how-do-i-manage-trackers}{Your
  Ad Choices}
\item
  \href{https://www.nytimes.com/privacy}{Privacy}
\item
  \href{https://help.nytimes.com/hc/en-us/articles/115014893428-Terms-of-service}{Terms
  of Service}
\item
  \href{https://help.nytimes.com/hc/en-us/articles/115014893968-Terms-of-sale}{Terms
  of Sale}
\item
  \href{https://spiderbites.nytimes.com}{Site Map}
\item
  \href{https://help.nytimes.com/hc/en-us}{Help}
\item
  \href{https://www.nytimes.com/subscription?campaignId=37WXW}{Subscriptions}
\end{itemize}
