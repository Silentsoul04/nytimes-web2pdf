Sections

SEARCH

\protect\hyperlink{site-content}{Skip to
content}\protect\hyperlink{site-index}{Skip to site index}

\href{/section/climate}{Climate}\textbar{}Climate Change Brought a
Lobster Boom. Now It Could Cause a Bust.

\url{https://nyti.ms/2lnI4k7}

\begin{itemize}
\item
\item
\item
\item
\item
\item
\end{itemize}

\hypertarget{climate-and-environment}{%
\subsubsection{\texorpdfstring{\href{https://www.nytimes.com/section/climate?name=styln-climate\&region=TOP_BANNER\&variant=undefined\&block=storyline_menu_recirc\&action=click\&pgtype=Article\&impression_id=a9781a70-e108-11ea-b922-ef603cfaa1fb}{Climate
and
Environment}}{Climate and Environment}}\label{climate-and-environment}}

\begin{itemize}
\tightlist
\item
  \href{https://www.nytimes.com/2020/08/13/climate/trump-methane.html?name=styln-climate\&region=TOP_BANNER\&variant=undefined\&block=storyline_menu_recirc\&action=click\&pgtype=Article\&impression_id=a9784180-e108-11ea-b922-ef603cfaa1fb}{Methane
  Rule Rollback}
\item
  \href{https://www.nytimes.com/interactive/2020/climate/trump-environment-rollbacks.html?name=styln-climate\&region=TOP_BANNER\&variant=undefined\&block=storyline_menu_recirc\&action=click\&pgtype=Article\&impression_id=a9784181-e108-11ea-b922-ef603cfaa1fb}{Trump's
  Changes}
\item
  \href{https://www.nytimes.com/interactive/2020/04/19/climate/climate-crash-course-1.html?name=styln-climate\&region=TOP_BANNER\&variant=undefined\&block=storyline_menu_recirc\&action=click\&pgtype=Article\&impression_id=a9784182-e108-11ea-b922-ef603cfaa1fb}{Climate
  101}
\item
  \href{https://www.nytimes.com/interactive/2018/08/30/climate/how-much-hotter-is-your-hometown.html?name=styln-climate\&region=TOP_BANNER\&variant=undefined\&block=storyline_menu_recirc\&action=click\&pgtype=Article\&impression_id=a9784183-e108-11ea-b922-ef603cfaa1fb}{Is
  Your Hometown Hotter?}
\end{itemize}

\includegraphics{https://static01.nyt.com/images/2018/06/20/climate/20cli-lobster-cousensdock/20cli-lobster-cousensdock-articleLarge-v2.jpg?quality=75\&auto=webp\&disable=upscale}

\hypertarget{climate-change-brought-a-lobster-boom-now-it-could-cause-a-bust}{%
\section{Climate Change Brought a Lobster Boom. Now It Could Cause a
Bust.}\label{climate-change-brought-a-lobster-boom-now-it-could-cause-a-bust}}

Dave Cousens, a lobsterman, begins his day before sunrise in South
Thomaston, Me.Credit...Greta Rybus for The New York Times

Supported by

\protect\hyperlink{after-sponsor}{Continue reading the main story}

\href{https://www.nytimes.com/by/livia-albeck-ripka}{\includegraphics{https://static01.nyt.com/images/2018/06/12/multimedia/author-livia-albeck-ripka/author-livia-albeck-ripka-thumbLarge.png}}

By \href{https://www.nytimes.com/by/livia-albeck-ripka}{Livia
Albeck-Ripka}

\begin{itemize}
\item
  June 21, 2018
\item
  \begin{itemize}
  \item
  \item
  \item
  \item
  \item
  \item
  \end{itemize}
\end{itemize}

VINALHAVEN, Me. --- At 3:30 in the morning on a Friday in late May, the
lobstermen ate breakfast. Outside, their boats bobbed in the labradorite
water, lit only by the dull yellow of streetlamps across the bay. It was
windy, too windy for fishing, but one by one the island's fishermen
showed up at the Surfside cafe anyway. Over pancakes and eggs, they
grumbled about the season's catch to date.

Some of the lobstermen said it was just too early in the season. Others
feared that it was a sign of things to come. Since the early 1980s,
climate change had warmed the Gulf of Maine's cool waters to the ideal
temperature for lobsters, which has helped grow Maine's fishery fivefold
to
\href{https://www.maine.gov/dmr/commercial-fishing/landings/documents/2017ValueBySpecies.Pie.Graph.pdf}{a
half-billion-dollar industry}, among the most valuable in the United
States. But last year the state's lobster landings
\href{http://www.maine.gov/dmr/commercial-fishing/landings/documents/lobster.table.pdf}{dropped
by 22 million pounds}, to 111 million.

Now, scientists and some fishermen are worried that the waters might
eventually warm too much for the lobsters, and are asking how much
longer the boom can last.

``Climate change really helped us for the last 20 years,'' said Dave
Cousens, who stepped down as president of the Maine Lobstermen's
Association in March. But, he added, ``Climate change is going to kill
us, in probably the next 30.''

\includegraphics{https://static01.nyt.com/images/2018/06/20/climate/20cli-lobster-cousens/merlin_139597206_a9b0feef-4d87-49d1-9d53-64760bfbd9ee-articleLarge.jpg?quality=75\&auto=webp\&disable=upscale}

Scientists say a variety of factors have contributed to the boom,
including overfishing of predators like cod and the lobstermen's own
conservation efforts. But without climate change, Maine's lobster
fishery would not be anywhere near as successful as it is today, said
\href{https://umaine.edu/wahlelab/}{Richard A. Wahle}, a professor at
the University of Maine's School of Marine Sciences.

The Gulf of Maine
\href{http://science.sciencemag.org/content/350/6262/809}{has warmed
faster} than 99 percent of the world's oceans for much of this century,
driven by climate change in combination with natural variation. By 2050,
that warming could **** cut lobster populations in the gulf by
\href{http://www.pnas.org/content/115/8/1831}{up to 62 percent}, the
Gulf of Maine Research Institute says. That has left some lobstermen
feeling anxious.

Fishing off the coast of Spruce Head, Me., one crisp overcast morning,
Mr. Cousens, 60, hauled up trap after disappointing trap. It was early
in the season, so few lobsters were expected. Even so, Mr. Cousens was
disheartened. He said he worried that in the future, Maine's fishermen
might catch fewer lobsters during the peak summer season than they do
now in the spring.

``We're past the point of climate change helping us. We're on the
downward spiral,'' Mr. Cousens said, as he dragged up a kelp-entangled
trap. His crewman untied the trap's bait bag and tossed the
sour-smelling herring remains into the water, where a flock of sea gulls
scuffled.

Image

Mr. Cousens on his boat, Three Sons.Credit...Greta Rybus for The New
York Times

Image

In the 1990s, Mr. Cousens said, he could haul up to 80,000 pounds of
lobster per year. But last year, his earnings fell 30 percent. ``You
can't do that too many years in a row,'' he said.

\href{https://www.nytimes.com/section/climate?action=click\&pgtype=Article\&state=default\&region=MAIN_CONTENT_1\&context=storylines_keepup}{}

\hypertarget{climate-and-environment-}{%
\subsubsection{Climate and Environment
›}\label{climate-and-environment-}}

\hypertarget{keep-up-on-the-latest-climate-news}{%
\paragraph{Keep Up on the Latest Climate
News}\label{keep-up-on-the-latest-climate-news}}

Updated Aug. 17, 2020

Here's what you need to know this week:

\begin{itemize}
\item
  \begin{itemize}
  \tightlist
  \item
    The Trump administration
    \href{https://www.nytimes.com/2020/08/17/climate/alaska-oil-drilling-anwr.html?action=click\&pgtype=Article\&state=default\&region=MAIN_CONTENT_1\&context=storylines_keepup}{finalized
    a plan} to open the Arctic National Wildlife Refuge to oil and gas
    companies, which will likely spur a legal battle.
  \item
    Climate change leaders said
    \href{https://www.nytimes.com/2020/08/12/climate/kamala-harris-environmental-justice.html?action=click\&pgtype=Article\&state=default\&region=MAIN_CONTENT_1\&context=storylines_keepup}{the
    vice-presidential choice of Kamala Harris} signaled that Democrats
    will have a focus on environmental justice.
  \item
    This year is poised to be one of the hottest ever and millions are
    already feeling the pain, but the
    \href{https://www.nytimes.com/interactive/2020/08/06/climate/climate-change-inequality-heat.html?action=click\&pgtype=Article\&state=default\&region=MAIN_CONTENT_1\&context=storylines_keepup}{agony
    of extreme heat} is profoundly unequal across the globe.
  \end{itemize}
\end{itemize}

As temperatures in the gulf have increased, the favorable conditions for
lobster reproduction have shifted northeast, away from Mr. Cousens's
home on the coast and toward the islands of Vinalhaven and Stonington
--- and in the direction of Canadian waters.

``You don't have to be a rocket scientist to say this does not bode well
for us,'' Mr. Cousens said. He worries about younger fishermen who have
invested hundreds of thousands of dollars in boats, gear and trucks but
who have never experienced the fishery outside of these boom years.
``They're basing their financial future,'' Mr. Cousens said, on a
``fantasyland.''

That does not worry Mr. Cousens's 24-year-old son Samuel, even though
his boat, Adrenaline, has sent him more than \$200,000 into debt. ``I
just put my head down and work,'' he said.

Often, the younger Mr. Cousens will fish 14-hour days, 35 miles from the
mainland. This has become the norm for many younger fishermen, who are
venturing farther offshore in bigger, faster, more expensive boats.
Lobster populations are not only expanding northeast but are thriving in
deeper waters as coastal waters continue to heat up, scientists say.

Image

Maine exports more than 50,000 tons of lobster globally each
year.Credit...Greta Rybus for The New York Times

Offshore, the fishing is high-risk and high-reward, Mr. Cousens said.
When you haul a trap up into a boat, he said, the feeling is
exhilarating: You either see ``dollar signs or dirt.''

Lobstering has always been a boom-and-bust business, but the
\href{http://www.maine.gov/dmr/science-research/species/lobster/guide/index.html}{conservation
measures long enforced by Maine's lobstermen} may help stave off
complete collapse, scientists say.

The lobstermen clip the tails of egg-bearing female lobsters and release
them, a practice called V-notching that began voluntarily in the late
19th century and was later mandated by law. They throw back lobsters
that already have V-notches, alongside lobsters that are smaller than
3.25 inches or larger than five, measured from the eye socket to the
base of the tail. These measures help conserve the brood stock, ensuring
that the lobsters continue to repopulate.

\href{http://www.pnas.org/content/early/2018/01/12/1711122115}{A study}
published this year in the Proceedings of the National Academy of
Sciences found that these conservation measures had not only capitalized
on the favorable conditions created by climate change but could also
save the industry from sharp decline in the future.

``It allowed them to take advantage of the boom, and it's going to give
them some resiliency to the changes that we think are coming,'' said
Andrew Pershing, the chief scientist at the Gulf of Maine Research
Institute and a lead author of the study.

To understand the role that the conservation measures played in the
broader context of climate change, Dr. Pershing and his colleagues
modeled the Maine fishery against those in Long Island Sound and Rhode
Island, where such measures were not mandated. In those regions, warming
waters led to an almost 80 percent decline in the lobster stock and the
collapse of the fisheries.

Image

Each trap, with its rope and buoy, can cost around \$150. Maine
lobstermen usually work from a few hundred traps up to
800.Credit...Greta Rybus for The New York Times

In the afternoon, Mr. Cousens --- who campaigned to enforce and increase
conservation measures as president of the lobstermen's association ---
notched about 50 female lobsters, their abdomens ripe with pearly black
eggs. A seven- to eight-pound female can carry upward of 100,000 eggs,
roughly 1 percent of which are likely to survive. ``That's a big bang
for your buck,'' Mr. Cousens said, as he plopped one of the freshly
notched females overboard. ``You want to be gentle with them,'' he said.
``That's the future.''

James M. Acheson, a professor of anthropology at the University of Maine
who has written about lobstermen's
\href{https://www.jstor.org/stable/26268854}{attitudes toward
conservation}, said Maine lobstermen were ``strongly, strongly in
favor'' of the laws because they were in their best interest.
``Conservation works,'' Dr. Acheson said.

Image

Curtis Brown, a lobsterman and marine biologist for Ready Seafood, one
of the state's largest exporters.Credit...Greta Rybus for The New York
Times

Image

Equipment for measuring blood protein levels, an indicator that helps
specialists like Mr. Brown evaluate shell strength.

Still, there is only so much the measures can do to prevent the decline
of the fishery. The maximum water temperature that a lobster can
tolerate is about 70 degrees Fahrenheit. Beyond that, ``their system
starts shutting down, one organ after another,'' said Dr. Wahle.
Consecutive days above this limit in Southern New England, he said, had
lead to ``mass mortality.''

For lobsters in the earliest stage of their life cycle, however, the
impacts of warming waters are less well understood. And despite healthy
numbers of brood stock, scientists have seen a collapse in larval
lobsters in the Gulf of Maine in recent years. ``We have a
multimillion-dollar industry, and a woefully inadequate understanding,''
said Curtis Brown, a lobsterman and marine biologist for Ready Seafood,
one of the state's largest exporters of lobster. A shell disease, which
scientists have also attributed in part to warming waters, is another
threat.

Given the ominous signs, some lobstermen, and lobsterwomen,
\href{https://www.nytimes.com/interactive/2017/10/10/us/aquaculture.html}{are
trying to branch out}. This summer, Krista Tripp, 33, is buying a small
oyster farm in Spruce Head to complement her lobster fishing.

Image

Krista Tripp was on a waiting list for 12 years before getting her
lobster license. She said she felt like she was playing catch-up on the
tail end of a booming industry.Credit...Greta Rybus for The New York
Times

Diversifying is hard, Ms. Tripp said. She had always wanted to be a
lobsterwoman, ever since she watched her father and grandfather hauling,
measuring and banding the claws of the lobsters, in what she said almost
resembled a dance. ``They were so good, they were so fast,'' she said.
``I knew that that's what I wanted to do.''

But ``with fishing going downhill,'' Ms. Tripp said, she feels as though
she is ``playing catch-up'' on the tail end of a booming industry. ``I
don't want to put all my eggs in one basket,'' she said.

This summer, she plans to spend her mornings lobstering and her
afternoons on the farm, wading through the shallow mud flats. In the
meantime, the lobstermen on Vinalhaven will continue to rise in the dark
for breakfast at Surfside, just as they have every season for the past
two decades. Eventually, the sun will come up, and they will go out onto
the water. Whether they will always find their traps full, however, is
another question.

Image

The harbor in Vinalhaven, Me.Credit...Greta Rybus for The New York Times

Advertisement

\protect\hyperlink{after-bottom}{Continue reading the main story}

\hypertarget{site-index}{%
\subsection{Site Index}\label{site-index}}

\hypertarget{site-information-navigation}{%
\subsection{Site Information
Navigation}\label{site-information-navigation}}

\begin{itemize}
\tightlist
\item
  \href{https://help.nytimes.com/hc/en-us/articles/115014792127-Copyright-notice}{©~2020~The
  New York Times Company}
\end{itemize}

\begin{itemize}
\tightlist
\item
  \href{https://www.nytco.com/}{NYTCo}
\item
  \href{https://help.nytimes.com/hc/en-us/articles/115015385887-Contact-Us}{Contact
  Us}
\item
  \href{https://www.nytco.com/careers/}{Work with us}
\item
  \href{https://nytmediakit.com/}{Advertise}
\item
  \href{http://www.tbrandstudio.com/}{T Brand Studio}
\item
  \href{https://www.nytimes.com/privacy/cookie-policy\#how-do-i-manage-trackers}{Your
  Ad Choices}
\item
  \href{https://www.nytimes.com/privacy}{Privacy}
\item
  \href{https://help.nytimes.com/hc/en-us/articles/115014893428-Terms-of-service}{Terms
  of Service}
\item
  \href{https://help.nytimes.com/hc/en-us/articles/115014893968-Terms-of-sale}{Terms
  of Sale}
\item
  \href{https://spiderbites.nytimes.com}{Site Map}
\item
  \href{https://help.nytimes.com/hc/en-us}{Help}
\item
  \href{https://www.nytimes.com/subscription?campaignId=37WXW}{Subscriptions}
\end{itemize}
