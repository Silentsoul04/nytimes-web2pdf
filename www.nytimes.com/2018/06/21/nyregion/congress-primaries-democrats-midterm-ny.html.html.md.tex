Sections

SEARCH

\protect\hyperlink{site-content}{Skip to
content}\protect\hyperlink{site-index}{Skip to site index}

\href{https://www.nytimes.com/section/nyregion}{New York}

\href{https://myaccount.nytimes.com/auth/login?response_type=cookie\&client_id=vi}{}

\href{https://www.nytimes.com/section/todayspaper}{Today's Paper}

\href{/section/nyregion}{New York}\textbar{}For Democrats Challenging
Party Incumbents, Insurgency Has Its Limits

\url{https://nyti.ms/2KawWlc}

\begin{itemize}
\item
\item
\item
\item
\item
\item
\end{itemize}

Advertisement

\protect\hyperlink{after-top}{Continue reading the main story}

Supported by

\protect\hyperlink{after-sponsor}{Continue reading the main story}

\hypertarget{for-democrats-challenging-party-incumbents-insurgency-has-its-limits}{%
\section{For Democrats Challenging Party Incumbents, Insurgency Has Its
Limits}\label{for-democrats-challenging-party-incumbents-insurgency-has-its-limits}}

\includegraphics{https://static01.nyt.com/images/2018/06/21/nyregion/00nydems/merlin_139839702_ecdf6e21-894d-47c5-b444-12fb33ddc212-articleLarge.jpg?quality=75\&auto=webp\&disable=upscale}

By \href{https://www.nytimes.com/by/shane-goldmacher}{Shane Goldmacher}
and \href{https://www.nytimes.com/by/jeffery-c-mays}{Jeffery C. Mays}

\begin{itemize}
\item
  June 21, 2018
\item
  \begin{itemize}
  \item
  \item
  \item
  \item
  \item
  \item
  \end{itemize}
\end{itemize}

For Suraj Patel, running for Congress against an entrenched incumbent
has led to some awkward and unexpected moments.

Elected officials have asked him to delete their pictures from his
campaign's Facebook page. When he has tried to set up meetings with key
New York City leaders, some refused --- simply because they do not dare
to be seen with him in public. Many did not answer at all.

``You'll never eat lunch in this town again if you challenge Carolyn
Maloney,'' Mr. Patel recalled one political consultant warning about his
race against a 13-term Democrat.

For Alexandria Ocasio-Cortez, challenging Representative Joseph Crowley
has meant watching local Democratic officials bolt the other way from
her at parades, wary of appearing too close to her. Some have agreed to
meetings, but behind closed doors --- no cellphones allowed.

Mr. Patel, 34, and Ms. Ocasio-Cortez, 28, are among a group of energetic
Democratic insurgents across the country, many of them young or female
or people of color, who are seeking to knock off some of Congress's most
tenured Democrats.

They have not succeeded so far: No congressional Democrat in America has
lost a primary in 2018.

The establishment's winning streak will be tested again on Tuesday in
New York City, where primary contests feature four Democratic
challengers in one of the densest concentrations of intraparty battles
in the nation.

While national Democrats have celebrated the President Trump-inspired
surge of activist energy coursing through the party in their efforts to
take control of the House, many of those same leaders have moved to tame
that energy, from Colorado to Massachusetts to New York, when it has
turned against them.

Beyond Mr. Crowley and Ms. Maloney, Representatives Eliot L. Engel and
Yvette D. Clarke also face unusually spirited rivals. All four are
Democrats in safely Democratic districts, and all four are heavily
favored.

``Any time there's an infusion of new people it's a good thing,'' said
Mr. Engel, who was first elected in 1988 and faces his most serious
primary challenge in more than a decade. ``It doesn't mean you should
automatically elect the people.''

\includegraphics{https://static01.nyt.com/images/2018/06/22/nyregion/22nydems-02/merlin_139839693_44c948bb-7b49-4cde-9727-d04bd84cd951-articleLarge.jpg?quality=75\&auto=webp\&disable=upscale}

Ms. Ocasio-Cortez's campaign, in particular, has become a cause célèbre
for some on the left who seem set to put a scare into Mr. Crowley, the
head of the Queens Democratic Party, one of the last and most powerful
political machines remaining in New York. MoveOn.org and Our Revolution,
an outgrowth of the Sanders campaign, both endorsed her, and the news
site The Intercept has generated a drumbeat of negative stories on Mr.
Crowley.

Earlier this week, she showed up to debate Mr. Crowley,
\href{https://www.rollcall.com/news/hawkings/joseph-crowley-democratic-leadership}{the
potential future leader of House Democrats} in Washington --- only to
discover that he was a no-show, a Latina surrogate sent in his place.
(Mr. Crowley had debated her earlier this month.)

``We have a political culture of intimidation, of favoring, of patronage
and of fear and that is no way for a community to be governed,'' Ms.
Ocasio-Cortez said.

Lauren French, a spokeswoman for Mr. Crowley, predicted victory next
week because the congressman ``is unapologetically fighting for the
people of Queens and the Bronx --- communities that need health care,
affordable housing, gun safety laws, immigration reform and better jobs
with higher wages.''

Mr. Patel has waged the most millennial of campaigns. On a recent
Thursday evening, he was sitting in a former bar in the East Village
that he uses as his campaign headquarters. His campaign manager handed
him one of three phones that was logged into the dating app Tinder, and
Mr. Patel began furiously swiping right.

All around the bar --- adorned with blue-velvet booths and a sound
system that was playing Kanye West --- campaign volunteers, logged onto
Tinder, Grindr or Bumble, were doing the same thing. Mr. Patel calls it
Tinder banking: Participants set up an account with a picture of an
attractive person, usually not themselves, and begin seeking matches.
Mr. Patel uses a picture of his brother.

He compared it to the practice of creating a fake online persona to lure
someone into a relationship. ``It's kinda like catfishing,'' he
admitted, ``but you are telling people who you are.''

When someone responds, Mr. Patel replies with a political pickup line:
``Hi Sarah. Are you into civic engagement?'' He soon reveals who he
really is.

Mr. Patel, a hotel executive, made a splash by amassing \$1.2 million in
a few short months --- rivaling Ms. Maloney's haul. He said he is not
interested in kowtowing to the traditional Democratic machine or
methods; he has been canvassing for votes at yoga studios and printing
campaign materials on coffee sleeves and drink coasters across this
mostly affluent district that covers much of the east side of Manhattan
and parts of Brooklyn and Queens.

Image

Adem Bunkeddeko, a son of Ugandan immigrants who went on to attend
Harvard Business School, is challenging Representative Yvette Clarke in
the Ninth Congressional District in New York.Credit...Gabriella
Angotti-Jones/The New York Times

``No bar ever says no to free coasters,'' he noted.

Ms. Maloney did not directly address Mr. Patel's challenge, but in a
phone interview, she acknowledged ``a lot of energy on the Democratic
side.''

``That's a good thing for the country and the party,'' she said.

In Brooklyn, Adem Bunkeddeko, the 30-year-old son of Ugandan immigrants
who went on to attend Harvard Business School, is among those who were
told to wait their turn. He ignored that advice and is challenging Ms.
Clarke,
\href{https://www.nytimes.com/2006/09/14/nyregion/14yvette.html}{part of
a local dynasty}, the daughter of the former longtime New York City
Councilwoman Una S.T. Clarke.

``The mom was a city councilwoman and she went ahead and inherited her
mom's seat. When it came time for the congressional, the mom helped push
her in,'' Mr. Bunkeddeko said.

He has criticized Ms. Clarke sharply --- ``No one can credibly say this
community has been represented well,'' he said in an interview --- and
not surprisingly, their recent debate on NY1 was intense.

``I understand that Ms. Clarke is upset by the fact that she has a
competitive primary,'' Mr. Bunkeddeko said at one point.

``Upset?'' Ms. Clarke interrupted. ``I'm \emph{laughing}.''

In an interview, Ms. Clarke said she was annoyed at Mr. Bunkeddeko for
misrepresenting her record; his age was not an issue. ``My office is
full of millennials,'' Ms. Clarke said.

Ms. Clarke called Mr. Bunkeddeko a ``shiny new thing'' who wasn't part
of the ``bench of young people committed to their community and public
service.''

``He's a brilliant young man and I take nothing away from him, but I've
brought value to my district and the nation,'' Ms. Clarke said.

Among those who counseled Mr. Bunkeddeko against running was
Representative Hakeem Jeffries, also of Brooklyn. He recalled telling
Mr. Bunkeddeko that his future had two divergent paths: one was to be
like Shirley Chisholm, Barack Obama or Charles B. Rangel; the other
option was characterized by various obscure figures who had run for
Congress and lost. The difference, Mr. Jeffries said, is that Ms.
Chisholm, Mr. Obama and Mr. Rangel ran for lower office before Congress.

Image

Adem Bunkeddeko and Debora Aquino, a campaign field organizer, greet
potential voters in Brooklyn.Credit...Gabriella Angotti-Jones/The New
York Times

``If it's good enough for them,'' Mr. Jeffries, who served as a state
assemblyman for six years, recalled telling Mr. Bunkeddeko, ``it should
be good enough for anyone.''

In the Bronx and Westchester County, Mr. Engel faces a challenge from
Jonathan Lewis, 56, who co-founded a money management firm and has put
roughly \$650,000 of his own money into the contest and is airing three
different ads on cable.

Mr. Lewis, who lives in Scarsdale, said he is running in part because of
``the energy of the post-Trump era.''

``My diagnosis is many of us naïvely presumed our party would take care
of the issues but our party was taking of itself,'' he said in an
interview. He has taken particular aim at Mr. Engel's raising money with
groups with business before Congress.

Mr. Engel pushed back that Mr. Lewis ``apparently thinks only wealthy
people should run for office.''

Nancy Pelosi, the minority leader of the House of Representatives,
recently made an appearance with Mr. Engel in his district and praised
him profusely. ``We couldn't be better served than by Eliot Engel,'' she
said.

In recent months, Mr. Engel has paid for multiple negative mailers,
including
\href{https://int.nyt.com/data/documenthelper/52-engel-cia-mailer/56e35fd9f054abe789be/optimized/full.pdf\#page=1}{one
linking Mr. Lewis to the Central Intelligence Agency} --- an unusual
tactic for the top Democrat on the House Foreign Affairs Committee, who
would become chairman should Democrats take the House in November.

The mailer features a large seal of the C.I.A., and says Mr. Lewis'
record ``does include an award from the C.I.A.!''

In the interview, Mr. Engel initially said he was ``frankly not aware
that we did that,'' but then defended it. ``I don't believe we were
implying what you think we were implying,'' Mr. Engel said.

Ms. Ocasio-Cortez said there is ``camaraderie'' among congressional
challengers, and that she's spoken with Mr. Bunkeddeko and Mr. Patel to
``share best practices in dismantling this calcified machine.''

Among the common themes have been criticizing where Congress members get
their money and embracing unconventional tactics.

``What if Tinder banking works so well that we get 1,000 extra votes?
Shouldn't the party be like: `Cool, let's start doing this,''' said Mr.
Patel, who added that Democrats risk becoming a tired ``legacy
corporation'' without further innovation. ``The primary is a phenomenal
opportunity for us to test new ideas, new energy. I find the lack of
creativity in politics appalling.''

Advertisement

\protect\hyperlink{after-bottom}{Continue reading the main story}

\hypertarget{site-index}{%
\subsection{Site Index}\label{site-index}}

\hypertarget{site-information-navigation}{%
\subsection{Site Information
Navigation}\label{site-information-navigation}}

\begin{itemize}
\tightlist
\item
  \href{https://help.nytimes.com/hc/en-us/articles/115014792127-Copyright-notice}{©~2020~The
  New York Times Company}
\end{itemize}

\begin{itemize}
\tightlist
\item
  \href{https://www.nytco.com/}{NYTCo}
\item
  \href{https://help.nytimes.com/hc/en-us/articles/115015385887-Contact-Us}{Contact
  Us}
\item
  \href{https://www.nytco.com/careers/}{Work with us}
\item
  \href{https://nytmediakit.com/}{Advertise}
\item
  \href{http://www.tbrandstudio.com/}{T Brand Studio}
\item
  \href{https://www.nytimes.com/privacy/cookie-policy\#how-do-i-manage-trackers}{Your
  Ad Choices}
\item
  \href{https://www.nytimes.com/privacy}{Privacy}
\item
  \href{https://help.nytimes.com/hc/en-us/articles/115014893428-Terms-of-service}{Terms
  of Service}
\item
  \href{https://help.nytimes.com/hc/en-us/articles/115014893968-Terms-of-sale}{Terms
  of Sale}
\item
  \href{https://spiderbites.nytimes.com}{Site Map}
\item
  \href{https://help.nytimes.com/hc/en-us}{Help}
\item
  \href{https://www.nytimes.com/subscription?campaignId=37WXW}{Subscriptions}
\end{itemize}
