Sections

SEARCH

\protect\hyperlink{site-content}{Skip to
content}\protect\hyperlink{site-index}{Skip to site index}

\href{https://www.nytimes.com/section/world/europe}{Europe}

\href{https://myaccount.nytimes.com/auth/login?response_type=cookie\&client_id=vi}{}

\href{https://www.nytimes.com/section/todayspaper}{Today's Paper}

\href{/section/world/europe}{Europe}\textbar{}Trump, at Putin's Side,
Questions U.S. Intelligence on 2016 Election

\href{https://nyti.ms/2Jp88Vz}{https://nyti.ms/2Jp88Vz}

\begin{itemize}
\item
\item
\item
\item
\item
\item
\end{itemize}

Advertisement

\protect\hyperlink{after-top}{Continue reading the main story}

Supported by

\protect\hyperlink{after-sponsor}{Continue reading the main story}

\hypertarget{trump-at-putins-side-questions-us-intelligence-on-2016-election}{%
\section{Trump, at Putin's Side, Questions U.S. Intelligence on 2016
Election}\label{trump-at-putins-side-questions-us-intelligence-on-2016-election}}

\includegraphics{https://static01.nyt.com/images/2018/07/17/world/17Trump-live14/17Trump-live14-videoSixteenByNine3000.jpg}

By \href{https://www.nytimes.com/by/julie-hirschfeld-davis}{Julie
Hirschfeld Davis}

\begin{itemize}
\item
  July 16, 2018
\item
  \begin{itemize}
  \item
  \item
  \item
  \item
  \item
  \item
  \end{itemize}
\end{itemize}

HELSINKI, Finland --- President Trump stood next to President Vladimir
V. Putin of Russia on Monday and publicly challenged the conclusion of
his own intelligence agencies that Moscow interfered in the 2016
presidential election, wrapping up what he called a ``deeply
productive''
\href{https://www.nytimes.com/2018/07/16/world/europe/trump-putin-summit-helsinki.html?rref=collection\%2Fsectioncollection\%2Fworld\&action=click\&contentCollection=world\&region=rank\&module=package\&version=highlights\&contentPlacement=1\&pgtype=sectionfront}{summit
meeting} with an extraordinary show of trust for a leader accused of
attacking American democracy.

In a remarkable news conference, Mr. Trump did not name a single action
for which Mr. Putin should be held accountable. Instead, he saved his
sharpest criticism for the United States and the special counsel
investigation into the election interference, calling it a
``ridiculous'' probe and a ``witch hunt'' that has kept the two
countries apart.

Mr. Trump even questioned the determinations by his intelligence
officials that Russia had meddled in the election.

\emph{{[}}\href{https://www.nytimes.com/2018/07/17/podcasts/the-daily/trump-putin-russia-summit-meeting.html?rref=nytimes.com\%2Fthedaily}{\emph{Julie
Hirschfeld Davis joined ``The Daily'' to discuss how President Trump
avoided criticizing President Vladimir V. Putin, and instead aimed his
barbs at the U.S. Listen here.}}\emph{{]}}

``They said they think it's Russia,'' Mr. Trump said. ``I have President
Putin; he just said it's not Russia,'' the president continued, only
moments after Mr. Putin conceded that he had wanted Mr. Trump to win the
election because of his promises of warmer relations with Moscow.

``I will say this: I don't see any reason why it would be'' Russia that
was responsible for the election hacking, Mr. Trump added. ``President
Putin was extremely strong and powerful in his denial today.''

The 45-minute news conference offered the spectacle of the American and
Russian presidents both pushing back on the notion of Moscow's election
interference, with Mr. Putin demanding evidence of something he said had
never been proved, and Mr. Trump appearing to agree.

When asked directly whether he believed
\href{100000006003853/web/editing}{Mr. Putin} or his own intelligence
agencies about the election meddling, Mr. Trump said there were ``two
thoughts'' on the matter: one from American officials like Dan Coats,
his director of national intelligence, asserting Russia's involvement;
and one from Mr. Putin dismissing it.

``I have confidence in both parties,'' Mr. Trump said.

He then changed the subject, demanding to know why the F.B.I. never
examined the hacked computer servers of the Democratic National
Committee, and asking about the fate of emails missing from the server
of Hillary Clinton, his campaign rival.

``Where are Hillary Clinton's emails?'' Mr. Trump said.

His performance drew howls of protests from Democrats and some
Republicans, prompting John O. Brennan, who served as C.I.A. director
under President Barack Obama, to suggest that the remarks warranted Mr.
Trump's impeachment.

``Donald Trump's press conference performance in Helsinki rises to \&
exceeds the threshold of `high crimes \& misdemeanors,''' Mr. Brennan
\href{https://twitter.com/JohnBrennan/status/1018885971104985093}{wrote
on Twitter}, calling the president's behavior ``treasonous.'' ``Not only
were Trump's comments imbecilic, he is wholly in the pocket of Putin.''

The House speaker, Paul D. Ryan,
\href{https://www.speaker.gov/press-release/statement-russia}{released a
statement} saying, ``there is no question that Russia interfered in our
election and continues attempts to undermine democracy.''

And within hours, Mr. Coats issued his own strongly worded statement
that contained an implicit rebuke of Mr. Trump.

``We have been clear in our assessments of Russian meddling in the 2016
election and their ongoing, pervasive efforts to undermine our
democracy,'' Mr. Coats said. ``We will continue to provide unvarnished
and objective intelligence in support of our national security.''

Some of Mr. Trump's own advisers privately said they were shocked by the
president's performance, including his use of the phrase ``witch hunt''
to describe the special counsel investigation while standing beside Mr.
Putin.

\href{https://www.nytimes.com/interactive/2018/07/16/us/politics/republicans-trump-putin-russia-reaction.html}{}

\includegraphics{https://static01.nyt.com/images/2018/07/16/us/republicans-trump-putin-russia-reaction-promo-1531788003912/republicans-trump-putin-russia-reaction-promo-1531788003912-articleLarge.jpg}

\hypertarget{how-republican-lawmakers-responded-to-trumps-russian-meddling-denial}{%
\subsection{How Republican Lawmakers Responded to Trump's Russian
Meddling
Denial}\label{how-republican-lawmakers-responded-to-trumps-russian-meddling-denial}}

Reports of the president's comments prompted outcry from some lawmakers,
but they were followed by notable silence from others.

Aboard Air Force One back to Washington, Mr. Trump's mood grew foul as
the breadth of the critical reactions became clear, according to some
people briefed on the flight. Aides steered clear of the front of the
plane to avoid being tapped for a venting session with Mr. Trump.

Some political allies worried that the encounter with Mr. Putin would
linger over Republicans heading into the midterm elections this fall.

``President Trump must clarify his statements in Helsinki on our
intelligence system and Putin,'' Newt Gingrich, the former Republican
speaker of the House who has advised Mr. Trump,
\href{https://twitter.com/newtgingrich/status/1018967261418344450}{said}
on Twitter. ``It is the most serious mistake of his presidency and must
be corrected --- immediately.''

Both presidents said it was vital to talk to each other because, as
leaders of two major nuclear powers, they had a responsibility to engage
in dialogue and ensure global stability.

But Mr. Trump's statements at the news conference were a remarkable
break with his administration, which on Friday
\href{https://www.nytimes.com/2018/07/13/us/politics/mueller-indictment-russian-intelligence-hacking.html}{indicted
12 Russian intelligence officers} for cyberattacks intended to interfere
in the presidential contest. The indictment explained, in detail, how
Russian intelligence officers hacked the Democratic National Committee
and the Clinton presidential campaign, providing the most explicit
account to date of the Russian government's meddling in American
democracy.

Mr. Trump said he did not regard Mr. Putin as an adversary, but as a
``good competitor,'' adding that, ``the word competitor is a
compliment.''

When Mr. Putin was asked by an American reporter whether he had wanted
Mr. Trump to win and directed an effort intended to bring about that
result, the Russian president quickly answered, ``Yes I did, yes I did,
because he talked about bringing the U.S.-Russia relationship back to
normal.'' It was not clear whether he had heard the translation of the
second part of the question.

Mr. Putin said he would look into the possibility of having Russian law
enforcement authorities assist Robert S. Mueller III, the special
counsel investigating Moscow's election interference, in questioning the
12 people who were charged. Mr. Trump called it an ``incredible offer.''

But in return, Mr. Putin, who rolled his eyes and snickered at the
notion that he had compromising material on Mr. Trump or his family,
said that Russia would expect American assistance in cases of interest
to Moscow, including the ability to send Russian law enforcement
officials to work in the United States.

He singled out
\href{https://www.nytimes.com/2018/07/16/world/europe/putin-bill-browder-magnitsky-investor.html}{William
F. Browder}, a longstanding critic of the Kremlin whose associates Mr.
Putin accused of evading taxes and funneling millions of dollars to the
Clinton campaign, without providing evidence.

Mr. Putin also took solace in Mr. Trump's doubt-casting about who was
responsible for the hacking, saying the allegations that Russia had
directed the effort were ``utter nonsense, just like the president
recently mentioned.''

Emerging from his one-on-one meeting with Mr. Putin, which was followed
by a larger lunch meeting with advisers, Mr. Trump cited a litany of
factors that he said had stood in the way of better relations between
the United States and Russia. He blamed Democrats' bitterness over
having lost an election that they should have won, and Mr. Mueller's
investigation.

But Mr. Trump claimed to have defused all of that tension in a matter of
minutes.

``Our relationship has never been worse than it is now,'' Mr. Trump
said. ``However, that changed as of about four hours ago.''

Mr. Trump began his day on Monday on Twitter, blaming
\href{https://twitter.com/realDonaldTrump/status/1018738368753078273}{American
``foolishness and stupidity''} for years of escalating tension with
Russia, as well as the ``Rigged Witch Hunt.''

The comment appeared to absolve Moscow of many irritants in the
relationship with Russia, including the election hacking, the annexation
of Crimea, Russian backing for rebels in Ukraine and for the Assad
government in Syria, and Moscow's suspected
\href{https://www.nytimes.com/2018/07/15/world/europe/uk-skripal-russia-novichok.html}{use
of a nerve agent to poison people in Britain}.

In fact, Russia's Foreign Ministry recirculated the comment,
\href{https://twitter.com/mfa_russia/status/1018803468805566464}{chiming
in, ``We agree.''}

For much of the news conference, Mr. Trump appeared to be far more
focused on defending the legitimacy of his election victory than on
determining who was behind the election hacking.

\href{https://www.nytimes.com/interactive/2018/07/16/us/elections/russian-interference-statements-comments.html}{}

\includegraphics{https://static01.nyt.com/images/2018/07/17/us/russian-interference-statements-comments-promo-1531839768920/russian-interference-statements-comments-promo-1531839768920-articleLarge-v2.gif}

\hypertarget{8-us-intelligence-groups-blame-russia-for-meddling-but-trump-keeps-clouding-the-picture}{%
\subsection{8 U.S. Intelligence Groups Blame Russia for Meddling, but
Trump Keeps Clouding the
Picture}\label{8-us-intelligence-groups-blame-russia-for-meddling-but-trump-keeps-clouding-the-picture}}

The heads of the national security agencies on Thursday said that Russia
was still trying to influence United States elections, contradicting
statements made by President Trump.

``There was no collusion at all --- everybody knows it,'' Mr. Trump
said. ``That was a clean campaign. I beat Hillary Clinton easily.''

He added, ``We ran a brilliant campaign, and that's why I'm president.''

Later, in an interview with Fox News, Mr. Putin repeated his assertion
that Russia had not interfered in the American presidential race. But
then he suggested that the hacking itself should not be treated as such
an explosive issue, because the emails taken from Democratic officials
were accurate.

``The information that I am aware of, there's nothing false about it,''
Mr. Putin said in the interview. ``Every single grain of it is true. And
the Democratic leadership admitted it.''

In the United States, critics of Mr. Trump reacted angrily to his
contention that Russia and the United States shared blame for their
deteriorated relationship.

``This is bizarre and flat-out wrong,'' said Senator Ben Sasse,
Republican of Nebraska. ``The United States is not to blame. America
wants a good relationship with the Russian people, but Vladimir Putin
and his thugs are responsible for Soviet-style aggression. When the
president plays these moral equivalence games, he gives Putin a
propaganda win he desperately needs.''

Mr. Ryan said, ``There is no moral equivalence between the United States
and Russia, which remains hostile to our most basic values and ideas.''

The summit meeting capped a weeklong European trip in which Mr. Trump
disparaged NATO allies,
\href{https://www.nytimes.com/2018/07/11/world/europe/germany-merkel-russia-trump-nato.html}{castigated
Germany},
\href{https://www.nytimes.com/2018/07/12/world/europe/trump-brexit-theresa-may.html}{criticized
the British prime minister} on her own soil and
\href{https://www.nytimes.com/2018/07/15/world/europe/trump-putin-summit-meeting.html?rref=collection\%2Fsectioncollection\%2Fworld\&action=click\&contentCollection=world\&region=rank\&module=package\&version=highlights\&contentPlacement=7\&pgtype=sectionfront}{branded
the European Union a ``foe''} on trade --- while he mused about his wish
for warmer relations with Mr. Putin.

Many in Mr. Trump's own government consider Mr. Putin a potentially
dangerous adversary to be countered, not courted. On Friday, Mr. Coats
said of Russian cyberattacks on the United States,
``\href{https://www.nytimes.com/2018/07/13/us/politics/dan-coats-intelligence-russia-cyber-warning.html}{the
warning lights are blinking red again}.''

As Mr. Putin and Mr. Trump emerged from a longer-than-expected set of
talks, which included a 130-minute one-on-one session with no advisers
present, they said they had made progress in forging the bond that both
were seeking.

``We had direct, open, deeply productive dialogue,'' Mr. Trump said.
``It went very well.''

Mr. Putin said the two were ``glad with the outcome of our first
full-scale meeting,'' adding, ``I hope that we start to understand each
other better, and I'm grateful to Donald for it.''

Finland, the site of the meeting, shares a long border with Russia and
has been the site of past diplomatic breakthroughs. But this round of
talks did not appear to have produced any.

The two presidents said they would work together on nuclear arms
control, although neither mentioned a concrete set of actions on forging
a new treaty to replace the New Start Treaty, which is set to expire in
2021. They also did not address what American officials have said are
Russian violations of the Intermediate-range Nuclear Forces Treaty.

The two also said they would work together to secure Israel's border
with Syria, restoring a cease-fire in the Golan Heights, and cooperate
to bring humanitarian relief to address the toll of the civil war that
has raged in Syria for more than seven years.

Mr. Trump said he had directly confronted Mr. Putin on the election
interference, calling it ``a message best delivered in person.'' But he
pivoted almost immediately to criticism of Mr. Mueller's investigation
and a recitation of his success in the Electoral College, blaming the
investigation --- not Russia's actions --- for the inability of two
great nuclear powers to work together.

``I think that the probe is a disaster for our country,'' Mr. Trump
said. ``I think it's kept us apart; it's kept us separated. There is no
collusion at all.''

He twice called Mr. Putin's proposal to potentially allow Russian
officials to question the 12 indicted agents with American observers
present ``an incredible offer.''

Peter Carr, a spokesman for the special counsel, had no comment on the
idea.

Advertisement

\protect\hyperlink{after-bottom}{Continue reading the main story}

\hypertarget{site-index}{%
\subsection{Site Index}\label{site-index}}

\hypertarget{site-information-navigation}{%
\subsection{Site Information
Navigation}\label{site-information-navigation}}

\begin{itemize}
\tightlist
\item
  \href{https://help.nytimes.com/hc/en-us/articles/115014792127-Copyright-notice}{©~2020~The
  New York Times Company}
\end{itemize}

\begin{itemize}
\tightlist
\item
  \href{https://www.nytco.com/}{NYTCo}
\item
  \href{https://help.nytimes.com/hc/en-us/articles/115015385887-Contact-Us}{Contact
  Us}
\item
  \href{https://www.nytco.com/careers/}{Work with us}
\item
  \href{https://nytmediakit.com/}{Advertise}
\item
  \href{http://www.tbrandstudio.com/}{T Brand Studio}
\item
  \href{https://www.nytimes.com/privacy/cookie-policy\#how-do-i-manage-trackers}{Your
  Ad Choices}
\item
  \href{https://www.nytimes.com/privacy}{Privacy}
\item
  \href{https://help.nytimes.com/hc/en-us/articles/115014893428-Terms-of-service}{Terms
  of Service}
\item
  \href{https://help.nytimes.com/hc/en-us/articles/115014893968-Terms-of-sale}{Terms
  of Sale}
\item
  \href{https://spiderbites.nytimes.com}{Site Map}
\item
  \href{https://help.nytimes.com/hc/en-us}{Help}
\item
  \href{https://www.nytimes.com/subscription?campaignId=37WXW}{Subscriptions}
\end{itemize}
