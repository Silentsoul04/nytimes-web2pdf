Sections

SEARCH

\protect\hyperlink{site-content}{Skip to
content}\protect\hyperlink{site-index}{Skip to site index}

\href{https://www.nytimes.com/section/world/asia}{Asia Pacific}

\href{https://myaccount.nytimes.com/auth/login?response_type=cookie\&client_id=vi}{}

\href{https://www.nytimes.com/section/todayspaper}{Today's Paper}

\href{/section/world/asia}{Asia Pacific}\textbar{}Military's Influence
Casts a Shadow Over Pakistan's Election

\url{https://nyti.ms/2JHQDQk}

\begin{itemize}
\item
\item
\item
\item
\item
\end{itemize}

Advertisement

\protect\hyperlink{after-top}{Continue reading the main story}

Supported by

\protect\hyperlink{after-sponsor}{Continue reading the main story}

\hypertarget{militarys-influence-casts-a-shadow-over-pakistans-election}{%
\section{Military's Influence Casts a Shadow Over Pakistan's
Election}\label{militarys-influence-casts-a-shadow-over-pakistans-election}}

\includegraphics{https://static01.nyt.com/images/2018/07/22/world/22pakistan-election/merlin_141442734_f665cead-063f-49b8-a014-fef50b3ba767-articleLarge.jpg?quality=75\&auto=webp\&disable=upscale}

By \href{https://www.nytimes.com/by/maria-abi-habib}{Maria Abi-Habib}
and \href{https://www.nytimes.com/by/salman-masood}{Salman Masood}

\begin{itemize}
\item
  July 21, 2018
\item
  \begin{itemize}
  \item
  \item
  \item
  \item
  \item
  \end{itemize}
\end{itemize}

LAHORE, Pakistan --- The phone calls started last month, said Rana Iqbal
Siraj: intimidating, anonymous demands that he defect from the party
that governed Pakistan for the past five years and tried to curb the
power of the military. Soon, he was summoned by state security officials
who delivered the same message.

Mr. Siraj, a candidate for the legislature in Punjab Province, stayed
with his party, the Pakistan Muslim League-Nawaz, which was built
decades ago around former Prime Minister Nawaz Sharif. Then in June,
roughly a month before Election Day, security officials raided his
business at the behest of the military, Mr. Siraj said in an interview.

``They are trying to ruin me financially by raiding my warehouse and
beating my staff,'' he said, adding that he was considering moving his
family abroad for their safety. ``What am I at fault for? Just because
I'm running on the PML-N ticket?''

Mr. Siraj and fellow party members said the aim of the raid was to
weaken the former governing party's chances by forcing its candidates to
defect ahead of
\href{https://www.nytimes.com/2018/07/23/world/asia/pakistan-election-explainer.html}{national
elections on Wednesday} that are shaping up to be a referendum on the
military and its interference in Pakistan's democracy.

That military campaign has been likened by some candidates to
\href{https://www.nytimes.com/2018/06/06/world/asia/pakistan-military-dissent-censorship.html}{a
soft coup}, and has included sidelining candidates who are out of the
military's favor,
\href{https://www.nytimes.com/2018/04/06/world/asia/pakistan-geo-military-censorship.html}{censoring
major news outlets} and
\href{https://www.nytimes.com/2018/04/17/world/asia/pashtun-movement-pakistan-military.html}{persecuting
peaceful political movements}.

The most likely beneficiary of the military's manipulation is the party
led by the former cricket star Imran Khan, who has called the Taliban's
war against the United States military in Afghanistan justified, and is
seen as the military's favored candidate --- a notion he denies. Mr.
Khan has positioned himself as a fighter against corruption, taking aim
at the dynastic politics and nepotism of parties like the PML-N while
maintaining a good relationship with the military, which he credits with
protecting the country.

The military has ruled Pakistan, a nuclear-armed country, through
various coups for nearly half the country's history since it gained
independence in 1947. Even during civilian rule, the country's generals
have wielded enormous power, setting the agenda for the country's
foreign and security policies and tolerance of extremist groups ---
including the Afghan Taliban in its fight against the United
States-backed government in Afghanistan next door.

As prime minister, Mr. Sharif ran afoul of the military early on by
trying to assert control over foreign and defense policy, which is seen
as the army's domain. He also tried to improve ties with India,
Pakistan's archrival, and opposed the military's embrace of terrorist
groups, members of his party say.

In Wednesday's election, voters will choose provincial legislatures and
the country's Parliament, which will appoint the next prime minister.
Officially, it will be only the second democratic transition between
civilian governments in the Pakistan's history, after the last election
in 2013.

The PML-N accuses the army of pressuring the country's courts to
disqualify its top candidates, including Mr. Sharif, who was sentenced
to prison this month. At the same time, some
\href{https://www.nytimes.com/2018/07/17/world/asia/pakistan-election-extremists.html}{candidates
who are on the government's terrorism watch list} have been cleared to
run.

The main Pakistani Army spokesman, Maj. Gen. Asif Ghafoor, denied at a
news conference this month that Mr. Siraj was targeted because he
belonged to the PML-N, saying that he had been the subject of a
government investigation for a year and a half. General Ghafoor would
not specify the nature of the investigation, and he denied that
intelligence agencies had been involved in the raid on Mr. Siraj's
warehouse.

Other high-profile PML-N candidates have defected to Mr. Khan's party,
the Pakistan Tehreek-e-Insaf or P.T.I.

\includegraphics{https://static01.nyt.com/images/2018/07/22/world/22pakistan-election2/22pakistan-election2-articleLarge.jpg?quality=75\&auto=webp\&disable=upscale}

Mr. Khan said that while he has a productive relationship with the
military, he is not receiving any help from it. Candidates are joining
his centrist party because they are fed up with traditional parties that
have failed to deliver, he said.

``When you have poor-quality leadership without the moral standing, you
have a void and someone will always fill it,'' Mr. Khan said in an
interview at his home in Islamabad, referring to the military's track
record of coups and political interference.

The P.T.I. is popular with voters under 35 who are hungry for change and
make up 43 percent of the electorate.

But the military's influence over Pakistan's courts and its muzzling of
the news media have cast a shadow over Mr. Khan's party and its rallying
cries for change and transparency.

Mr. Sharif and his daughter and political heir Maryam
\href{https://www.nytimes.com/2018/07/13/world/asia/pakistan-sharif-arrrest.html}{returned
to Pakistan this month to face arrest} after being
\href{https://www.nytimes.com/2018/07/06/world/asia/pakistan-nawaz-sharif-corruption-verdict.html}{convicted
of corruption} and sentenced to lengthy prison terms. He had already
been
\href{https://www.nytimes.com/2017/07/28/world/asia/pakistan-prime-minister-nawaz-sharif-removed.html}{forced
to resign last year} by Pakistan's Supreme Court in a case involving
undisclosed luxury properties the Sharif family owns in London.

The Sharifs say those rulings were politicized, with the courts pressed
by the military to bar them from politics.

On Saturday, a judge of the Islamabad High Court accused the military's
spy agency, Inter-Services Intelligence, or ISI, of meddling in the
judiciary and forcing the justices to rule against Mr. Sharif and his
relatives.

The speech by Justice Shaukat Aziz Siddiqui to lawyers in Rawalpindi was
the latest public indictment of the military's interference in politics.
The parts of the speech that were critical of the ISI were not aired by
local television news networks but short video clips went viral on
social media.

The judge accused the ISI of influencing and pressuring the court that
convicted and sentenced Mr. Sharif and his relatives. On July 17, the
Islamabad High Court deferred the hearings of the appeals by Mr. Sharif
against the court verdict until after elections.

``In this election, what's at stake is the fate of Pakistan,'' said Hina
Rabbani, a former foreign minister who is running with the Pakistan
Peoples Party, a rival of the PML-N. ``I may hate Nawaz Sharif for his
political choices, but I believe the system needs to self-correct, and
we can no longer allow external forces to correct it. The only thing
that can correct the system is elections.''

The 2013 election was important because it was the first time power had
been transferred from one civilian government to another, Ms. Rabbani
said. ``But for the next 10 years, we'll be holding our breath with
every election.''

Mr. Khan, who made Mr. Sharif's removal from office almost a personal
mission, sees the situation differently.

``To say the army castrated Nawaz Sharif --- Nawaz Sharif was castrated
by his own corruption,'' he said. ``The unlevel playing field you see is
that they have minted this country,'' he said, referring to the endemic
corruption among Pakistan's top political parties.

Although Mr. Khan has a good chance of becoming prime minister, the
military is likely to insist on curbing the next government's ability to
shape defense and foreign policy, risking Pakistan's further
international isolation.

Image

Ousted Pakistani Prime Minister Nawaz Sharif, center, and his daughter,
Maryam Nawaz, in London this month. Mr. Sharif and his daughter returned
to Pakistan to face arrest after being convicted of corruption and
sentenced to lengthy prison terms.Credit...Tolga Akmen/Agence
France-Presse --- Getty Images

``The military finds itself in a tight corner,'' said Raza Rumi, the
editor of The Daily Times, an influential newspaper based in Lahore.
``They want a hung Parliament that doesn't focus on cutting the
military's budget or curtailing its foreign policy. Instead, they want a
government that focuses on cleaning the streets and planting trees.''

Whichever party forms the government will inherit a raft of problems:
domestic terrorism; terrible relationships with neighboring India and
Afghanistan; deteriorating ties with the United States, once a major
ally; and a sputtering economy.

Last month, Pakistan was
\href{https://www.nytimes.com/2018/03/01/world/asia/pakistan-terrorism-china-saudi-arabia.html}{returned
to a ``gray list''} by the Financial Action Task Force --- a global body
that fights terrorism financing --- for not doing enough to counter
Islamic extremists operating from its territory. The listing could
affect the country's ability to raise funds internationally. At the
beginning of this year, the Trump administration
\href{https://www.nytimes.com/2018/01/04/us/politics/trump-pakistan-aid.html}{cut
more than \$1 billion} in annual security aid over Pakistan's support
for terrorist groups. (The Pakistani military denies supporting
terrorists.)

The military believes it can weather the storm by turning to China,
which is spending some \$65 billion on infrastructure and other projects
in Pakistan, as well as doling out billions in loans.

``The question the whole nation is asking is what does the army want and
why this level of interference?'' said Ahmed Rasheed, a foreign-policy
analyst and author.

Like others interviewed, Mr. Rasheed said he believed the military
wanted a weak government, with the P.T.I. at the helm of an unwieldy
alliance in Parliament.

While the PML-N, which held a supermajority in the last Parliament, may
win the most votes, it will struggle to form a government if the
military pressures potential coalition partners. Analysts say Mr. Khan's
party is likely to be able to form the next government by cobbling
together a coalition with smaller parties and independents.

But the military risks a severe backlash, Mr. Rasheed said, in part
because social media has increased scrutiny of an institution once seen
as sacrosanct.

``For the first time, not just the elite, but the public is now aware of
the army's major role,'' Mr. Rasheed said. ``It's now talked about at
the village level.''

When Gul Bukhari, a journalist and vocal critic of the military, was
abducted in an army-controlled area of Lahore last month by unknown
assailants, including men in military uniform, the news spread quickly
online. Pakistanis took to social media, including Twitter, to demand
that Ms. Bukhari be freed, and within hours she was returned home.

Ms. Bukhari said the public outcry had played a large role in her quick
release.

``It was a demonstration of the immense power of social media in our
times,'' she said in an interview.

The traditional news media have also stood up to the military, as
happened this spring when the newspaper Dawn and the TV channel Geo News
complained that their distribution was being disrupted in parts of the
country that the military administers.

Many candidates are nervous about the military's unusual decision to
deploy some 371,000 soldiers to monitor the election, including inside
polling stations. But Khurram Dastgir Khan, a PML-N candidate who was
defense minister in the last government, said social media had made the
military and its allies more careful about overt interference.

``Things come out --- they can't be kept hidden anymore,'' he said.
``It's unfeasible to use the draconian measures of two decades ago.
Society has moved forward and technology has moved forward.''

Advertisement

\protect\hyperlink{after-bottom}{Continue reading the main story}

\hypertarget{site-index}{%
\subsection{Site Index}\label{site-index}}

\hypertarget{site-information-navigation}{%
\subsection{Site Information
Navigation}\label{site-information-navigation}}

\begin{itemize}
\tightlist
\item
  \href{https://help.nytimes.com/hc/en-us/articles/115014792127-Copyright-notice}{©~2020~The
  New York Times Company}
\end{itemize}

\begin{itemize}
\tightlist
\item
  \href{https://www.nytco.com/}{NYTCo}
\item
  \href{https://help.nytimes.com/hc/en-us/articles/115015385887-Contact-Us}{Contact
  Us}
\item
  \href{https://www.nytco.com/careers/}{Work with us}
\item
  \href{https://nytmediakit.com/}{Advertise}
\item
  \href{http://www.tbrandstudio.com/}{T Brand Studio}
\item
  \href{https://www.nytimes.com/privacy/cookie-policy\#how-do-i-manage-trackers}{Your
  Ad Choices}
\item
  \href{https://www.nytimes.com/privacy}{Privacy}
\item
  \href{https://help.nytimes.com/hc/en-us/articles/115014893428-Terms-of-service}{Terms
  of Service}
\item
  \href{https://help.nytimes.com/hc/en-us/articles/115014893968-Terms-of-sale}{Terms
  of Sale}
\item
  \href{https://spiderbites.nytimes.com}{Site Map}
\item
  \href{https://help.nytimes.com/hc/en-us}{Help}
\item
  \href{https://www.nytimes.com/subscription?campaignId=37WXW}{Subscriptions}
\end{itemize}
