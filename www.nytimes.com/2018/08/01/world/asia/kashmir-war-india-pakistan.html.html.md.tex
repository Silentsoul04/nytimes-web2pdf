Sections

SEARCH

\protect\hyperlink{site-content}{Skip to
content}\protect\hyperlink{site-index}{Skip to site index}

\href{https://www.nytimes.com/section/world/asia}{Asia Pacific}

\href{https://myaccount.nytimes.com/auth/login?response_type=cookie\&client_id=vi}{}

\href{https://www.nytimes.com/section/todayspaper}{Today's Paper}

\href{/section/world/asia}{Asia Pacific}\textbar{}In Kashmir, Blood and
Grief in an Intimate War: `These Bodies Are Our Assets'

\url{https://nyti.ms/2O3e3SN}

\begin{itemize}
\item
\item
\item
\item
\item
\item
\end{itemize}

Advertisement

\protect\hyperlink{after-top}{Continue reading the main story}

Supported by

\protect\hyperlink{after-sponsor}{Continue reading the main story}

\hypertarget{in-kashmir-blood-and-grief-in-an-intimate-war-these-bodies-are-our-assets}{%
\section{In Kashmir, Blood and Grief in an Intimate War: `These Bodies
Are Our
Assets'}\label{in-kashmir-blood-and-grief-in-an-intimate-war-these-bodies-are-our-assets}}

By \href{https://www.nytimes.com/by/jeffrey-gettleman}{Jeffrey
Gettleman}

\begin{itemize}
\item
  Aug. 1, 2018
\item
  \begin{itemize}
  \item
  \item
  \item
  \item
  \item
  \item
  \end{itemize}
\end{itemize}

\includegraphics{https://static01.nyt.com/images/2018/03/27/world/00kashmir-1/merlin_134725104_b5b87f80-0aef-428e-a143-a94cdb52d961-articleLarge.jpg?quality=75\&auto=webp\&disable=upscale}

QASBAYAR, Kashmir --- It was 9:30 p.m. when Sameer Tiger came to the
door, a rifle slung over his shoulder.

Most of the village of Qasbayar, a tucked-away hamlet surrounded by
apple orchards and framed by Kashmir's mountain peaks, was getting ready
for sleep. A few yellowish lights burned in windows, but otherwise the
village was dark.

``Is Bashir home?'' Sameer Tiger asked. ``Can we talk to him?''

Bashir Ahmad's family didn't know what to do. Mr. Ahmad wasn't a
fighter; he was a 55-year-old pharmacist. And Sameer Tiger was a bit of
mystery. He had grown up a skinny kid just down the road and used to
lift weights with Mr. Ahmad's sons at the neighborhood gym; they'd spot
each other with the barbells, all friends.

But Sameer Tiger had disappeared for a while and then resurfaced as a
bushy-haired militant, a member of an outlawed Kashmiri separatist group
that had killed many people, the vast majority of them fellow Kashmiris.

Kashmir's war, a territorial dispute between India and neighboring
Pakistan, has smoldered for decades. Now it is collapsing into itself.
The violence is becoming smaller, more intimate and harder to escape.

Years ago, Pakistan pushed thousands of militants across the border as a
proxy army to wreak havoc in the Indian-controlled parts of Kashmir.
Now, the resistance inside the Indian areas is overwhelmingly homegrown.

The conflict today is probably driven less by geopolitics than by
internal Indian politics, which have increasingly taken an anti-Muslim
direction. Most of the fighters are young men like Sameer Tiger from
quiet brick-walled villages like Qasbayar, who draw support from a
population deeply resentful of India's governing party and years of
occupation.

Anyone even remotely associated with politics is in danger. That
included Mr. Ahmad, who, when he wasn't sitting behind the counter of
the village pharmacy, was known to host events for a local Kashmiri
political party.

``Don't worry,'' Sameer Tiger said, standing at Mr. Ahmad's door,
seeming to sense the family's anxiety.

He looked Mr. Ahmad's son right in the eye.

``We don't mean any harm,'' he said. ``Your father is like our father.''

Mr. Ahmad rushed home from work and invited Sameer Tiger in for tea.
They sat on the living room carpet talking quietly, then Mr. Ahmad
nodded goodbye to his wife and son and left with the visitor.

He didn't have much choice. Sameer Tiger was armed, and insistent, and
had arrived with three others who were waiting in the road. The group
moved slowly down the unlit lane.

At a bend in the road, in front of a shuttered shop, Sameer Tiger and
Mr. Ahmad started arguing, a witness said. Four gun blasts rang out. Mr.
Ahmad screamed. The few remaining lights in the neighborhood were
suddenly extinguished.

\textbf{JUST THE NAME KASHMIR} conjures a set of very opposing images:
snowy mountain peaks and chaotic protests, fields of wildflowers and
endless deaths. It is a staggeringly beautiful place that lives up to
all its fabled charm, yet even the quietest moments here feel ominous.

Image

A mosque in Srinagar. The Kashmir Valley is predominantly Muslim, but it
is controlled by India, which is predominantly Hindu.Credit...Atul Loke
for The New York Times

Kashmir sits on the frontier of India and Pakistan, and both countries
have spilled rivers of blood over it. Three times, they have gone to
war, and tens of thousands of people have been killed in the conflict.
It is one of
\href{http://www.nytimes.com/1999/08/12/world/vale-tears-special-report-kashmir-crushed-jewel-caught-vise-hatred.html}{Asia's
most dangerous flash points}, where a million troops have squared off
along the disputed border. Both sides now wield nuclear arms. And the
two sides are divided by religion, with Kashmir stuck in the middle.

India, which has controlled most of the Kashmir Valley for the past 70
years, is predominantly Hindu. The valley itself is predominantly
Muslim, as is Pakistan. But as the days pass, the conflict has become
less of a religiously driven proxy war.

The rebellion, says Imran Khan, Pakistan's presumed new leader, is now
``indigenous.'' Mr. Khan, who clearly has a Pakistani perspective on the
conflict, says he is determined to negotiate an end to it. His
persuasive election victory last month --- and the fact that India's
prime minister, Narendra Modi, made a friendly phone call to
congratulate him --- suggests a breakthrough is possible.

But India still loves to blame Pakistan for all its Kashmir problems,
and Pakistan, according to Western intelligence agents, continues to
send some money and weapons to militants in Kashmir. Many Indian
politicians seem in denial that their own politics and policies might be
a factor.

India's swerve to the right in recent years, with the rise of the Hindu
nationalist Bharatiya Janata Party, has deeply alienated its Muslim
minority. Many top members of the ruling party have a very questionable
record when it comes to treating Muslims fairly. This has emboldened
Hindu supremacists across India, and in recent years, Hindu lynch mobs
have targeted and killed Muslims, often based on false rumors. Many of
the culprits are lightly punished, if at all, leaving India's Muslims
feeling exposed.

In the Indian-administered parts of Kashmir, where there was already a
history of bitter conflict, the new politics have spurred more people to
turn against the government. Some pick up guns, others rocks, but the
root emotion is the same: Many Kashmiris now hate India.

Image

Photos of people wanted by the police, at a station near Qasbayar.
Around 250 militants are operating in the Kashmir Valley, down from
thousands two decades ago.Credit...Atul Loke for The New York Times

``This is what's different,'' said Siddiq Wahid, a Kashmiri historian
who earned his Ph.D from Harvard. ``Before, in the 1990s, many Kashmiris
felt we can negotiate this, we can talk.''

``But nobody wants to be part of India now,'' he said. ``Every Kashmiri
is resisting today, in different ways.''

The latest are children and grandmothers. At almost every recent
security operation, as Indian officers closed in on houses where
militants were believed to be hiding, they have had to reckon with
seething crowds of residents of all ages acting as human shields.

Walk through Kashmiri villages, where little apples are ripening on the
trees and the air tastes clean and crisp, and ask people what they want.
The most common response is independence. Some say they want to join
Pakistan. None say anything good about India, at least not in public.

India's steely response has pushed away even moderates. Soldiers
manhandle residents, cut off roads and barge into homes, saying they are
looking for militants, who often hide among ordinary residents. When
violent protests erupt, the Indian security services blast live
ammunition and buckshot into the crowds,
\href{https://www.nytimes.com/2016/08/29/world/asia/pellet-guns-used-in-kashmir-protests-cause-dead-eyes-epidemic.html}{killing
or blinding many people}, including schoolchildren who are simply
bystanders, despite cries from human rights groups to stop.

Image

The police patrolling in Srinagar.Credit...Atul Loke for The New York
Times

But while protests against Indian rule have grown in number and size,
the armed militancy has become surprisingly small, partly because
Pakistan is not providing as much support as it used to. Security
officials say there are only around 250 armed militants operating in the
Kashmir Valley, down from thousands two decades ago. Most of them are
poorly trained and militarily lost. But still, the Indians can't stomp
them out.

``I'll be honest,'' said Mohammad Aslam, a seemingly forthright police
commander in southern Kashmir. ``For every militant we kill, more are
joining.''

\textbf{THE HUNT FOR SAMEER TIGER} began the night he killed Mr. Ahmad,
on April 15, 2017.

Back then, he wasn't widely known as Sameer Tiger. To most, he was still
Sameer Bhat, a 17-year-old high school dropout who had worked in a local
bakery. The Indian security forces give all the known militants a grade:
A through C, with A being the most wanted. Sameer Tiger was a C.

The first place the police searched was Drabgam, his village. The shops
are small, tucked into old brick buildings. The jobs are few. Like much
of southern Kashmir, Drabgam hangs on the apple business. After the last
of the apples have been picked in October and until the new crop is
tended in the spring, there is little to do.

Sameer Tiger's house is one of the more modest: one and a half stories
of crudely finished brick, a couple of naked electrical bulbs dangling
in the living room, some wet shawls flapping on a line outside. His
father is a laborer and farmer who tends just a few acres of orchards.
His mother, Gulshan, is chatty and welcoming. They live on a dirt road.

Image

Sameer Tiger's parents, Mohammed Maqbool Bhat, left, and Gulshan Begum,
center, at home in southern Kashmir. Sameer changed his last name to
Tiger in honor of an uncle known for his great strength.Credit...Atul
Loke for The New York Times

``Sameer loves these,'' she said, pressing a handful of coconut candies
into my palm and tugging me into their bare living room. The candies
were exceptionally sweet and left a milky taste on the tongue.

Sameer Tiger's parents said their son was a reluctant militant. One
afternoon in early 2016, he was accused of throwing rocks at police
officers. Sameer Tiger was working in the bakery at the time, his
parents said, and they insisted he was innocent.

But the police didn't listen and dragged him into a truck by his hair,
they said. He spent a few days in jail. After he was let out, he
disappeared.

Soon his face popped up on separatist websites, his piercing eyes
staring at the camera, his bushy hair now down to his shoulders, a
Kalashnikov in his hands.

``When we saw that,'' said Maqbool, his father, ``we said goodbye.''

More than 250,000 Indian Army soldiers, border guards, police officers
and police reservists are stationed in the valley, outnumbering the
militants 1,000 to one. Most militants don't last two years. One
fighter, a former college sociology professor, was
\href{https://www.firstpost.com/india/kashmir-university-professor-rafi-bhat-gets-killed-within-40-hrs-of-joining-militancy-students-colleagues-in-shock-4458433.html}{killed
in May just two days after he joined}.

Their attacks tend to be quixotic and they usually die in a hail of
automatic weapon fire. Their assassinations and killings are not
militarily significant, more acts of protest against Indian rule. Of the
approximately 250 known militants, police officials said, only 50 or so
came from Pakistan, and most of the rest, the locals, have never left
the valley.

Image

Protesters in Srinagar in February. ``For every militant we kill,'' one
officer said, ``more are joining.''Credit...Atul Loke for The New York
Times

Sameer Tiger's parents said he changed his last name from Bhat to Tiger
in honor of a brawny uncle with that nickname who was known for his
immense strength.

When I asked about the killing of Bashir Ahmad, his father looked down
at the carpet. For the first time, he seemed embarrassed about his son.

``Bashir was a good man,'' he mumbled. ``Sameer wasn't there to kill
him. It was an accident.''

It might have been. On this point, Sameer Tiger's family and a survivor
of the shooting seem to agree.

The night Mr. Ahmad was killed, the militants had also pulled another
village elder from his home, Mohamad Altaf, a first cousin of Mr. Ahmad.
Both were among Qasbayar's elite, landowners who supported the
\href{https://economictimes.indiatimes.com/news/politics-and-nation/jks-pdp-bjp-alliance-becoming-increasingly-untenable/articleshow/62897945.cms}{Peoples
Democratic Party}, Kashmir's dominant political organization.

The party used to sympathize with separatism, but to win control of the
state parliament, it joined hands with the Hindu-nationalist Bharatiya
Janata Party three years ago. Many Kashmiris accused it of selling out
to Indian rule.

In June, the alliance suddenly broke apart, leaving a vacuum in the
State Assembly. India's central government took over running the state.
Kashmiris are now terrified that the government will escalate military
operations; the sense of hopelessness is rising.

According to Mr. Altaf, as they walked through the unlit lanes of
Qasbayar with the militants, Sameer Tiger urged him and Mr. Ahmad to
renounce their party affiliation. When Mr. Ahmad started arguing, Sameer
Tiger ordered both men to lie facedown and close their eyes.

Mr. Altaf was shot once in the back of his right knee and not critically
hurt. He thinks the intent was to send a message.

But Mr. Ahmad was shot three times in his legs, the bullets moving
upward toward his waist, Mr. Altaf said. His cousin, a lifelong friend,
bled to death on the spot. Maybe the Kalashnikov jumped in Sameer
Tiger's hands. Maybe he squeezed a split second too long.

Mr. Altaf can't stop thinking about it. The betrayal haunts him.

**``**Bashir invited Sameer Tiger in for tea, \emph{tea,}'' he said.

His cousin's death seems so pointless. He wonders if Sameer Tiger didn't
set out that night to kill. Maybe, Mr. Altaf thinks, he just didn't know
how to use his gun.

Image

Mohamad Altaf, center, and his wife, Fareeda Akhtar, at home. Mr. Altaf
was shot once in the back of his right knee by Sameer
Tiger.Credit...Atul Loke for The New York Times

These days, the Kashmiri militants don't have many opportunities to
practice shooting, police officials said. It is not like the 1990s, when
thousands of young Kashmiri men slipped across the border to training
camps on the Pakistani side. The Indians have sealed much of the
contested frontier, which runs about 450 miles.

The
\href{http://indianexpress.com/article/india/israel-fence-systems-quick-response-team-at-pakistanbangladesh-borders-bsf-dg-4794770/}{Israelis
have been surreptitiously helping them, providing security cameras},
night vision gear, drones and other surveillance equipment along the
border to stop big infiltrations. All this, coupled with the fact that
\href{http://www.nytimes.com/2002/01/02/world/india-pakistan-tension-islamabad-pakistan-said-order-end-support-for-militant.html}{Pakistan
has closed most of its militant camps} under pressure from the United
States, has pushed the fighting away from the border, and deeper into
the villages.

Kashmiris speak of a psychological tension that divides communities,
individual families and sometimes even the same person. On one hand,
people want to support a functioning society --- to have their children
go to school, get jobs, see some economic development --- and Indian
control represents that. On the other, they feel real sympathy for a
cause, Kashmiri independence, that they consider just.

``Let's be realistic: India's never going to give up this land,'' said
one young Kashmiri who asked that his identity not be revealed because
he could be labeled a collaborator.

``I can say such things in my house. But as soon as I step outside, even
into my own street, I can't say that. It has to be `\emph{Azadi! Azadi!
Azadi!}' '' he said, using the word for freedom. ``It's like you have to
be two different people, all the time.''

He sighed.

``It's exhausting.''

\textbf{THE BIGGEST CHALLENGE IN KILLING MILITANTS}, Officer Ashiq Tak
explained, isn't finding them.

``Information is coming in all the time,'' he said. ``We know their
friends, their girlfriends, which houses they're using.

``The trick,'' he said, ``is laying the cordon.''

Officer Tak is another example of how this war is shrinking. He grew up
in Qasbayar, a couple of miles from Sameer Tiger. Mr. Ahmad was his
mother's brother. This winter he found himself, as the commanding
officer of a tactical police unit in southern Kashmir, hunting the man
who killed his uncle.

Sameer Tiger was emerging as a militant's militant. He was increasingly
active --- and not just on social media.

He attacked police stations, he recruited new fighters and he supplied
pistols to young men to carry out assassinations, Officer Tak said. The
police often discovered where he was hiding, and set up their security
cordons, but he was slippery.

Image

Sameer Tiger's image started popping up on separatist websites. ``When
we saw that,'' his father said, ``we said goodbye.''Credit...Atul Loke
for The New York Times

``We almost had him,'' Officer Tak said in February. ``But he escaped,
dressed like a girl.''

Officer Tak seemed dispirited by all the support for Sameer Tiger, and
the fact that many Kashmiris consider police officers like himself to be
traitors. Unlike soldiers in the Indian Army, which is recruited from
across the country, police officers in the region come from within the
state of Jammu and Kashmir, and dozens have been killed.

Many Kashmiris see them as collaborators and call them ``Modi's dogs,''
a reference to India's prime minister, who rose to power as part of the
Hindu right-wing movement.

Officer Tak said that Kashmiris had so little faith in the security
services that when a police officer or soldier killed a civilian, people
didn't even bother demanding justice.

``Anywhere else, they'd ask for an investigation,'' he said. ``Here,
they just take the body and go away.''

``That's a bad sign,'' he said. ``That's total alienation.''

\textbf{SAMEER TIGER RESURFACED} in late April, a year after Mr. Ahmad's
death. A few miles from his house, witnesses said, he stopped a car
carrying a local politician and shot him dead. The attack, conducted in
the daytime and on a busy road, was unusually audacious. India's
national news media seized upon it, and for the first time Sameer Tiger
was
\href{https://www.news18.com/news/india/ghulam-nabi-patel-slain-by-hizbul-mujahideen-disowned-by-pdp-congress-1729811.html}{front-page
news}.

The hunt for him intensified but more civilians were rallying to the
defense of militants, often barricading the roads as the police closed
in and pelting officers with rocks.

``It's getting very hard to do operations,'' Officer Tak grumbled.

Around this time \href{https://www.youtube.com/watch?v=Bo8KypsQST4}{a
mysterious video} appeared on Facebook in which Sameer Tiger issued a
threat to Maj. Rohit Shukla, one of the area's commanding army officers:
``Tell Shukla to come and face me.''

A few days later, on April 30, the army got a tip that Sameer Tiger was
hiding in a house in the center of Drabgam. Though he was now a highly
wanted militant, upgraded to an A rating, it seemed he had never strayed
far from home.

This time, the Indian Army didn't arrive en masse. They used mud-smeared
dump trucks packed with soldiers wearing traditional pheran cloaks, guns
hidden. The villagers thought they were laborers. The soldiers quietly
surrounded the house and called for backup.

The soldiers sent in two rounds of emissaries, including village elders,
to persuade Sameer Tiger to surrender. He replied with a burst of
bullets, hitting Major Shukla in the shoulder.

The sound of gunfire served as an alarm, setting off an eruption. The
village mobilized. Boys, girls, men and women scampered out of their
houses and rushed into the road with stones in their hands. Mosque
loudspeakers blared: ``Sameer Tiger is trapped! Go help him!'' The whole
town, quite openly, was rallying to an outlaw's side.

As additional army trucks rumbled in, packed with troops, more civilians
rushed forward, trying to insert themselves between the troops and
Sameer Tiger. One young man was shot dead; the crowd kept coming.

But the cordon had been well laid, growing to nearly 300 soldiers and
police officers. The civilians, however determined, couldn't break it.

Several police commanders said security officers then moved in, firing a
rocket at the house. Flames burst out. Sameer Tiger scampered onto a
rooftop. The soldiers opened up with automatic weapons from four
directions. He was hit several times.

\textbf{A CULTURE OF DEATH IS SPREADING} across Kashmir. The militants
have become the biggest heroes. People paint their names on walls. They
wear T-shirts showing their bearded faces. They speak of them
affectionately, as if they are close friends. The militants are
especially revered after they are dead.

On a Tuesday morning, May 1, Sameer Tiger's lifeless body, riddled with
holes and soaked in blood, was hoisted onto a makeshift wooden platform
in the yard of one of Drabgam's mosques. Thousands poured in from across
the valley. For hours they chanted his name: ``Tiger! Tiger! Sameer
Tiger!''

Image

Sameer Tiger's funeral procession. ``These bodies are our assets,'' said
a woman who identified herself as a separatist leader.Credit...Tauseef
Mustafa/Agence France-Presse --- Getty Images

Boys scrambled up trees and scurried across tin roofs, the light metal
popping beneath their gym shoes, to find any vantage point. Others
fought through the nearly impenetrable crowd to the funeral pyre, just
to gently stroke Sameer Tiger's beard or to kiss his pale face goodbye.
Many vowed to join the militants.

One woman who identified herself as a separatist leader looked out at
the sea of mourners and gravely smiled.

``We are winning,'' she said. ``These bodies are our assets.''

A few hundred yards away, on the rooftop where Sameer Tiger had been
cornered, a team of boys wearing religious skullcaps scrubbed a
rust-colored splotch. A crowd pressed in to watch.

``Young ones, tell me: What does the spilling of this blood mean?'' one
man shouted.

\emph{``Azadi!}'' the crowd roared back.

The boys worked fast, heads down, sweat trickling off their temples.
They used wet rags to mop up the splotch. They squeezed the blood-water
mixture into a copper urn, to be saved. An imam watching closely told
them to capture every last drop of blood.

Advertisement

\protect\hyperlink{after-bottom}{Continue reading the main story}

\hypertarget{site-index}{%
\subsection{Site Index}\label{site-index}}

\hypertarget{site-information-navigation}{%
\subsection{Site Information
Navigation}\label{site-information-navigation}}

\begin{itemize}
\tightlist
\item
  \href{https://help.nytimes.com/hc/en-us/articles/115014792127-Copyright-notice}{©~2020~The
  New York Times Company}
\end{itemize}

\begin{itemize}
\tightlist
\item
  \href{https://www.nytco.com/}{NYTCo}
\item
  \href{https://help.nytimes.com/hc/en-us/articles/115015385887-Contact-Us}{Contact
  Us}
\item
  \href{https://www.nytco.com/careers/}{Work with us}
\item
  \href{https://nytmediakit.com/}{Advertise}
\item
  \href{http://www.tbrandstudio.com/}{T Brand Studio}
\item
  \href{https://www.nytimes.com/privacy/cookie-policy\#how-do-i-manage-trackers}{Your
  Ad Choices}
\item
  \href{https://www.nytimes.com/privacy}{Privacy}
\item
  \href{https://help.nytimes.com/hc/en-us/articles/115014893428-Terms-of-service}{Terms
  of Service}
\item
  \href{https://help.nytimes.com/hc/en-us/articles/115014893968-Terms-of-sale}{Terms
  of Sale}
\item
  \href{https://spiderbites.nytimes.com}{Site Map}
\item
  \href{https://help.nytimes.com/hc/en-us}{Help}
\item
  \href{https://www.nytimes.com/subscription?campaignId=37WXW}{Subscriptions}
\end{itemize}
