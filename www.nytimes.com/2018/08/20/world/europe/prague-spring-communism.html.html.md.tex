Sections

SEARCH

\protect\hyperlink{site-content}{Skip to
content}\protect\hyperlink{site-index}{Skip to site index}

\href{https://www.nytimes.com/section/world/europe}{Europe}

\href{https://myaccount.nytimes.com/auth/login?response_type=cookie\&client_id=vi}{}

\href{https://www.nytimes.com/section/todayspaper}{Today's Paper}

\href{/section/world/europe}{Europe}\textbar{}50 Years After Prague
Spring, Lessons on Freedom (and a Broken Spirit)

\url{https://nyti.ms/2wjzVC8}

\begin{itemize}
\item
\item
\item
\item
\item
\item
\end{itemize}

Advertisement

\protect\hyperlink{after-top}{Continue reading the main story}

Supported by

\protect\hyperlink{after-sponsor}{Continue reading the main story}

\hypertarget{50-years-after-prague-spring-lessons-on-freedom-and-a-broken-spirit}{%
\section{50 Years After Prague Spring, Lessons on Freedom (and a Broken
Spirit)}\label{50-years-after-prague-spring-lessons-on-freedom-and-a-broken-spirit}}

\includegraphics{https://static01.nyt.com/images/2018/08/02/world/xxCzechoslovakia-invasion-slide-6M6I/xxCzechoslovakia-invasion-slide-6M6I-articleLarge-v3.jpg?quality=75\&auto=webp\&disable=upscale}

By \href{https://www.nytimes.com/by/marc-santora}{Marc Santora}

\begin{itemize}
\item
  Aug. 20, 2018
\item
  \begin{itemize}
  \item
  \item
  \item
  \item
  \item
  \item
  \end{itemize}
\end{itemize}

\href{https://www.nytimes.com/es/2018/08/23/primavera-praga-aniversario-50-urss/}{Leer
en español}

PRAGUE --- Could Soviet-style communism be reconciled with the dignity
and freedom of the individual?

In 1968, the question was put to the test when the leader of
Czechoslovakia's Communist Party,
\href{https://www.nytimes.com/1992/11/09/obituaries/alexander-dubcek-70-dies-in-prague.html}{Alexander
Dubcek}, initiated a project of liberalization that he said would offer
``socialism with a human face.''

What followed was a rebirth of political and cultural freedom long
denied by party leaders loyal to Moscow.

The free press flourished, artists and writers spoke their minds, and
Mr. Dubcek stunned Moscow by proclaiming that he wanted to create ``a
free, modern and profoundly humane society.''

A season when hope and optimism were in bloom, it became known as the
Prague Spring.

But nearly as soon as the movement came to life, it was crushed under
the treads of Soviet T-54 tanks.

On Aug. 21,
\href{https://archive.nytimes.com/www.nytimes.com/learning/general/onthisday/big/0820.html\#article}{50
years ago}, the Soviet-led invasion of Czechoslovakia killed the dreams
of the reformers, broke the spirit of a nation and ushered in an era of
oppression whose effects are still felt today.

Moscow succeeded in restoring the supremacy of the state, but the
ultimate cost of victory was high.

Perhaps more than any other event during the Cold War, the invasion laid
bare for the world to see the totalitarian nature of the Soviet regime.

\includegraphics{https://static01.nyt.com/images/2018/08/02/world/xxCzechoslovakia-invasion-slide-08BF/xxCzechoslovakia-invasion-slide-08BF-articleLarge-v3.jpg?quality=75\&auto=webp\&disable=upscale}

Image

Residents of Prague witnessing the invasion.Credit...Josef
Koudelka/Magnum Photos

Image

A young Czech man showing a news report about the invasion to a Soviet
soldier.Credit...Josef Koudelka/Magnum Photos

The photographs of unarmed citizens confronting columns of heavily armed
soldiers, pleading, ``Ivan, go home,'' made it clear to the world that
this was an ideology that needed to be enforced at the point of a gun.

Many of the most famous images were taken by Josef Koudelka, who was on
the streets with his Exakta camera loaded with film that he had cut from
the end of exposed movie reels.

Mr. Koudelka's pictures were smuggled out of Prague and published
anonymously, credited only to ``Prague Photographer.''

In their intimacy and vivid detail, putting viewers on the street with
shocked and horrified citizens, they showed the propaganda flowing from
Moscow --- that troops were sent to restore order and had been welcomed
by the people --- as utter lies.

Image

Soldiers abandoning a burning tank in Prague.Credit...Josef
Koudelka/Magnum Photos

Image

The body of a young Czech, killed for having tried to drape his flag
over a Soviet tank.Credit...Josef Koudelka/Magnum Photos

Image

A final farewell for a victim of the invasion.Credit...Josef
Koudelka/Magnum Photos

``It was a defining moment,'' said
\href{https://www.nytimes.com/2008/08/24/opinion/24pehe.html}{Jiri
Pehe}, a former political adviser to
\href{https://www.nytimes.com/2011/12/19/world/europe/vaclav-havel-dissident-playwright-who-led-czechoslovakia-dead-at-75.html}{Vaclav
Havel}, the first president of post-Communist Czechoslovakia and now the
director of New York University in Prague.

``For the country, it was a defining moment because after a huge rise in
the hopes of the people and an outburst of creative energy, the country
was crushed,'' Mr. Pehe said. ``It really broke the backbone of the
nation.''

Mr. Pehe was 13 at the time. He can still recall the shock of the moment
--- and not just the violence and chaos.

``I still remember people going to the tanks and going to the soldiers,
and talking to the soldiers who did not even know where they were, they
were saying: `This is a terrible mistake. What are you doing here? Why
did you come?' '' he recalled.

``We were young kids,'' he added. ``And like all of my other
schoolmates, we were raised with this idea that the system might have
problems, but that it was a humane system. This was drummed into us.
After 1968, this all ended. We realized this was all lies.''

Image

A protester standing in defiance atop a tank.Credit...Josef
Koudelka/Magnum Photos

With the benefit of hindsight, it may now seem obvious that the
countries that fell under the sphere of Soviet influence after World War
II were doomed to fall victim to Stalinist oppression.

But that was the bargain reached at the end of the war, when Europe was
essentially cut in half.

``From Stettin in the Baltic to Trieste in the Adriatic, an iron curtain
has descended across the Continent. Behind that line lie all the
capitals of the ancient states of Central and Eastern Europe,'' Winston
Churchill warned in a
\href{https://winstonchurchill.org/resources/speeches/1946-1963-elder-statesman/the-sinews-of-peace/}{1946
speech}.

``Warsaw, Berlin, Prague, Vienna, Budapest, Belgrade, Bucharest and
Sofia, all these famous cities and the populations around them lie in
what I must call the Soviet sphere,'' he said, ``and all are subject in
one form or another, not only to Soviet influence but to a very high
and, in many cases, increasing measure of control from Moscow.''

When citizens fought back in nations under the Soviet yoke, such as East
Germany in 1953 and Hungary in 1956, the rebellions were brutally
crushed.

But the Prague Spring was different.

Image

A general strike in Wenceslas Square.Credit...Josef Koudelka/Magnum
Photos

Image

Gathering around a radio for news of the uprising after the Soviet
invasion.Credit...Hulton-Deutsch Collection/Corbis, via Getty Images

Image

Czech students working to produce underground news reports during the
first days of the Soviet occupation.Credit...Mario De Biasi/Mondadori
portfolio, via Getty Images

It was not an attempt to overthrow the communist regime, but rather one
to transform it.

But Moscow viewed events in Czechoslovakia as something like a virus,
fearing they would spread and infect other Warsaw Pact nations,
according to documents unearthed by a committee of scholars with the
help of the National Security Archive, a nongovernmental group in
Washington, and published in ``The Prague Spring '68.''

Leonid I. Brezhnev, the Soviet leader, comes across as particularly
incensed by the criticisms leveled at the Soviet system by a newly free
and emboldened news media. Among the first targets of the invading
troops were the Prague radio and television stations.

Image

Soviet soldiers on their way to occupy the Czechoslovak Radio
building.Credit...Josef Koudelka/Magnum Photos

Image

A family leaving Prague at the central railway station to escape the
Soviet invasion.Credit...Stefan Tyszko/Getty Images

Image

Alexander Dubcek, the leader of Czechoslovakia's Communist Party, on his
way to meet the Soviets.Credit...Harry Redl/The LIFE Picture Collection,
via Getty Images

American intelligence agencies
\href{https://www.cia.gov/news-information/featured-story-archive/2008-featured-story-archive/a-look-back-the-prague-spring-the-soviet.html}{watched
with concern} as troops amassed near the borders of Czechoslovakia, but
the invasion caught the administration of President Lyndon B. Johnson by
surprise. There was little Western nations could do. Some 250,000 troops
from 20 Warsaw Pact divisions swept across Czechoslovakia, with 10
Soviet divisions filling the positions they vacated.

They were backed by thousands of nuclear weapons pointed at Western and
Central Europe.

``Nothing short of a world war was likely to get them out,'' was the
judgment of
\href{https://www.cia.gov/news-information/featured-story-archive/2008-featured-story-archive/a-look-back-the-prague-spring-the-soviet.html}{a
review} of the C.I.A.'s handling of the crisis. ``In 1938, the Western
powers had responded to threats against Czechoslovakia by backing down,
rather than face Nazi Germany they falsely believed was ready for war.
In 1968 they had no choice.''

It was one more chapter in
\href{https://www.nytimes.com/interactive/2018/01/15/us/1968-history.html}{a
remarkable year} around the world, marked by tragedy, turmoil and
triumph. Students on college campuses from
\href{https://www.nytimes.com/2018/05/05/world/europe/france-may-1968-revolution.html}{Paris}
to Berkeley, Calif., were in revolt. The assassinations of Robert F.
Kennedy and the Rev. Dr. Martin Luther King Jr. left America shattered
and bitterly divided. The Vietnam War raged on even as
\href{https://timesmachine.nytimes.com/timesmachine/1968/12/25/76924273.html?action=click\&contentCollection=Archives\&module=ArticleEndCTA\&region=ArchiveBody\&pgtype=article}{Apollo
8} soared into the heavens, becoming the first manned spacecraft to
orbit the moon.

For the millions living under the heel of invading forces in
Czechoslovakia, the shift from hope to despair was as swift as it was
shocking.

By 7:45 a.m., according to press reports at the time, Soviet-led forces
had shot unarmed civilians gathered in protest.

Confusion quickly turned to anger and desperation, as tens of thousands
of civilians --- young and old alike --- gathered in the grand plazas of
Prague, Bratislava and other major cities.

Image

Lining up for food in Prague in August 1968.Credit...Josef
Koudelka/Magnum Photo

They had no weapons, only defiance.

As the chaos spread, some pleaded with the soldiers --- many of whom
were as bewildered as the people on the streets, as they had been told
that they were to stop an insidious counterrevolution, only to be
greeted with scorn.

The most violent episode took place outside the Prague radio station,
the city's only major fountainhead of defiance. In an attempt to keep
broadcasting, protesters moved city buses around the building and set
them ablaze. When Soviet tanks rammed the fortifications, several set
themselves on fire.

It is still unclear how many people died during the invasion, with
estimates ranging from 80 to several hundred. But in the months that
followed, as scores of people were arrested and thousands sent for
``re-education'' under a program of ``normalization,'' hope was replaced
by fear and defiance with dejected resignation.

That despair was captured most drastically on Jan. 16, 1969, when Jan
Palach, a student at Charles University in Prague, went to Wenceslas
Square and
\href{http://www.praguemorning.cz/czech-student-jan-palach-burns-himself-to-death-in-anti-soviet-protest-Es2oY7kyqA}{set
himself on fire} in protest --- a moment captured on film. He died
several days later, and thousands attended his funeral.

Others would imitate his self-immolation, but the Prague Spring was
over. The Stalinist winter would last two decades.

Image

The funeral service for Jan Palach at Wenceslas Square in Prague in
1969. The Czech student set himself on fire to protest the Soviet
occupation.Credit...Ullstein Bild, via Getty Images

Image

Buildings in ruins after the uprising.Credit...Thomas Hoepker/Magnum
Photos

Image

Soviet soldiers in Prague. Moscow viewed events in Czechoslovakia as
something like a virus, fearing they would spread and infect other
Warsaw Pact nations.Credit...Josef Koudelka/Magnum Photos

Jirina Siklova, a sociologist in Prague who was a member of the
Communist Party before the invasion, said that in the 1960s, she would
frequently travel abroad and talk to curious students who viewed the
socialist system as a possible cure to what ailed their own societies.

That notion died when the tanks rolled into Prague and more than 100
civilians were killed.

``After the invasion, I never met anybody who would advocate it, not
even among the Soviets,'' she said. ``Fifty years later, we still have
not found any alternative to fighting problems of the Western world, and
that is why many people turn to extremists.''

Indeed, Europe finds itself more divided than at any point since the end
of the Cold War. Bedrock institutions of the postwar order, such as
NATO, have come under questioning from
\href{https://www.nytimes.com/2018/07/18/world/europe/trump-nato-self-defense-montenegro.html}{an
American administration inherently suspicious of alliances}.

The events that played out 50 years ago in Prague serve as a reminder of
the fragility of the systems created to guard against war and tyranny.

Advertisement

\protect\hyperlink{after-bottom}{Continue reading the main story}

\hypertarget{site-index}{%
\subsection{Site Index}\label{site-index}}

\hypertarget{site-information-navigation}{%
\subsection{Site Information
Navigation}\label{site-information-navigation}}

\begin{itemize}
\tightlist
\item
  \href{https://help.nytimes.com/hc/en-us/articles/115014792127-Copyright-notice}{©~2020~The
  New York Times Company}
\end{itemize}

\begin{itemize}
\tightlist
\item
  \href{https://www.nytco.com/}{NYTCo}
\item
  \href{https://help.nytimes.com/hc/en-us/articles/115015385887-Contact-Us}{Contact
  Us}
\item
  \href{https://www.nytco.com/careers/}{Work with us}
\item
  \href{https://nytmediakit.com/}{Advertise}
\item
  \href{http://www.tbrandstudio.com/}{T Brand Studio}
\item
  \href{https://www.nytimes.com/privacy/cookie-policy\#how-do-i-manage-trackers}{Your
  Ad Choices}
\item
  \href{https://www.nytimes.com/privacy}{Privacy}
\item
  \href{https://help.nytimes.com/hc/en-us/articles/115014893428-Terms-of-service}{Terms
  of Service}
\item
  \href{https://help.nytimes.com/hc/en-us/articles/115014893968-Terms-of-sale}{Terms
  of Sale}
\item
  \href{https://spiderbites.nytimes.com}{Site Map}
\item
  \href{https://help.nytimes.com/hc/en-us}{Help}
\item
  \href{https://www.nytimes.com/subscription?campaignId=37WXW}{Subscriptions}
\end{itemize}
