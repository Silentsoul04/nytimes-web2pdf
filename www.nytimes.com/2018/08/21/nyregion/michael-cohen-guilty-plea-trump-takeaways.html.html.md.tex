Sections

SEARCH

\protect\hyperlink{site-content}{Skip to
content}\protect\hyperlink{site-index}{Skip to site index}

\href{https://www.nytimes.com/section/nyregion}{New York}

\href{https://myaccount.nytimes.com/auth/login?response_type=cookie\&client_id=vi}{}

\href{https://www.nytimes.com/section/todayspaper}{Today's Paper}

\href{/section/nyregion}{New York}\textbar{}6 Takeaways From Michael
Cohen's Guilty Plea

\url{https://nyti.ms/2MKkzRA}

\begin{itemize}
\item
\item
\item
\item
\item
\item
\end{itemize}

Advertisement

\protect\hyperlink{after-top}{Continue reading the main story}

Supported by

\protect\hyperlink{after-sponsor}{Continue reading the main story}

\hypertarget{6-takeaways-from-michael-cohens-guilty-plea}{%
\section{6 Takeaways From Michael Cohen's Guilty
Plea}\label{6-takeaways-from-michael-cohens-guilty-plea}}

\includegraphics{https://static01.nyt.com/images/2018/08/22/nyregion/22cohentakeaway_xp/22cohentakeaway_xp-articleLarge.jpg?quality=75\&auto=webp\&disable=upscale}

By \href{http://www.nytimes.com/by/alan-feuer}{Alan Feuer}

\begin{itemize}
\item
  Aug. 21, 2018
\item
  \begin{itemize}
  \item
  \item
  \item
  \item
  \item
  \item
  \end{itemize}
\end{itemize}

You could easily be confused by the sheer number and variety of the
criminal charges that Michael D. Cohen, President Trump's onetime fixer
and personal lawyer,
\href{https://www.nytimes.com/2018/08/21/nyregion/michael-cohen-plea-deal-trump.html}{pleaded
guilty to on Tuesday}in Federal District Court in Manhattan.

After all, the combative Mr. Cohen, a former vice president at the Trump
Organization, was accused of violating laws that involved
\href{https://www.nytimes.com/2018/05/05/business/michael-cohen-lawyer-trump.html}{his
taxi business},
\href{https://www.nytimes.com/2018/08/19/nyregion/michael-cohen-loans-donald-trump.html}{his
financial dealings with at least three banks} and --- it was the
headline allegation --- his secretive efforts to influence the 2016
presidential election. He admitted
\href{https://www.nytimes.com/2018/04/11/us/politics/trump-national-enquirer-american-media.html}{joining
forces with the nation's best-known supermarket tabloid} to buy the
silence of at least two women who claimed they had affairs with Mr.
Trump.

Making matters more arcane, some of these purported crimes overlapped,
the government said.

Mr. Cohen, for instance, was said to have used a fraudulently obtained
home-equity loan to pay off one of the women, a pornographic film star,
Stephanie Clifford, better known as Stormy Daniels.

The charging documents describe a universe of shady dealings and
unsavory characters. None of the revelations seem helpful to Mr. Trump.

Here are six takeaways from what happened in court --- and what was
disclosed in court papers.

\includegraphics{https://static01.nyt.com/images/2018/12/13/nyregion/13cohen-promo2/13cohen-promo2-videoSixteenByNine3000-v4.jpg}

\hypertarget{the-stormy-daniels-cover-up-almost-fell-apart}{%
\subsection{The Stormy Daniels cover-up almost fell
apart.}\label{the-stormy-daniels-cover-up-almost-fell-apart}}

According to the government, in October 2016 --- one month before the
presidential election --- Ms. Clifford, who has a second career as an
exotic dancer, reached out through her agent to the National Enquirer,
the gossip magazine owned by
\href{https://www.nytimes.com/2018/03/29/us/politics/trump-national-enquirer-david-pecker.html}{David
J. Pecker}, a longtime friend and supporter of Mr. Trump.

She had what she believed was a hot story, the government said: the tale
of her alleged affair with Mr. Trump.

Court papers say Mr. Pecker and an editor at the National Enquirer then
reached out to Mr. Cohen, putting him in touch with Keith Davidson, a
Los Angeles lawyer who was representing Ms. Clifford. Over the next few
days, the papers claim, Mr. Cohen negotiated a deal to pay Ms. Clifford
\$130,000 to keep her silent about the affair.

By Oct. 25, however, just two weeks before voters would go to the polls,
the deal had not been signed yet, the government said. And even worse,
prosecutors claim, Mr. Davidson was threatening to take his client and
her scoop to another publication.

It was at that point, court papers say, that the unnamed editor from the
National Enquirer sent Mr. Cohen a text message, saying, ``We have to
coordinate something on the matter'' or ``it could look awfully bad for
everyone.''

Not long after, prosecutors said, the editor and Mr. Pecker called Mr.
Cohen on an encrypted phone application, and Mr. Cohen agreed to make
the payment.

The very next day, the government said, Mr. Cohen withdrew \$131,000
from the fraudulent home-equity loan he had gotten that year and placed
it into the account of a shell company he had created, Essential
Consultants LLC.

Then, on Oct. 27, in an effort to influence the 2016 presidential
election, prosecutors say, he wired \$130,000 to Ms. Clifford's lawyer,
Mr. Davidson, apparently keeping the extra \$1,000 for himself.

\includegraphics{https://static01.nyt.com/images/2018/03/09/us/politics/PRE_COHEN_COVER-IMAGE_v3_BW/PRE_COHEN_COVER-IMAGE_v3_BW-videoSixteenByNineJumbo1600.jpg}

\hypertarget{cohens-plea-deal-and-charges}{%
\subsection{Cohen's Plea Deal and
Charges}\label{cohens-plea-deal-and-charges}}

The plea agreement between Michael D. Cohen and prosecutors along with
the charges to which Mr. Cohen pleaded guilty.

\includegraphics{https://int.nyt.com/data/documenthelper/182-cohen-plea-deal/9bc6cd47e7c48e9f9469/optimized/thumbnail.png}

\hypertarget{he-seemed-to-like-to-hold-on-to-evidence}{%
\subsection{He seemed to like to hold on to
evidence.}\label{he-seemed-to-like-to-hold-on-to-evidence}}

A few months earlier, in June 2016,
\href{https://www.nytimes.com/2018/03/22/us/politics/karen-mcdougal-interview.html}{Karen
McDougal}, a former Playboy model, started searching for a publication
to which she could sell her own tale of an affair with Mr. Trump, the
government said. She, too, was represented by Mr. Davidson.

In August that year, the National Enquirer struck a deal with Ms.
McDougal and Mr. Davidson to purchase what court papers called the
``limited life rights'' to her story of infidelity.

In exchange, the government said, the National Enquirer agreed to pay
Ms. McDougal \$150,000 and promised to feature her on two of its covers
and to publish more than 100 articles she wrote.

Mr. Cohen was also part of this deal, prosecutors claim, and to
facilitate it, he created another shell company called Resolution
Consultants LLC.

But before the agreement was consummated, court papers say, the National
Enquirer's owner, Mr. Pecker, told Mr. Cohen to tear it up.

Mr. Cohen, however, did not tear it up, the government said.

The paperwork was later found by federal agents when they performed ``a
judicially authorized search'' of Mr. Cohen's office, prosecutors said.

\hypertarget{he-lied-to-banks-prosecutors-say-often}{%
\subsection{He lied to banks, prosecutors say.
Often.}\label{he-lied-to-banks-prosecutors-say-often}}

Much of the 22-page criminal information detailing Mr. Cohen's alleged
legal violations involved false statements he is said to have made to
banks.

Beginning in 2010, Mr. Cohen began to rack up debts with one bank that
ultimately totaled about \$20 million, according to court papers.

But his problems started in earnest in 2013, the government said, when
he successfully applied, through a different bank, for a mortgage for an
apartment on Park Avenue and claimed in his paperwork that he only owed
the first bank \$6.4 million in outstanding loans.

He neglected to mention he was also on the hook for another \$14 million
in lines of credit, according to court documents.

Compounding his troubles, the government said, Mr. Cohen also tried to
buy an \$8.5 million summer home in 2015 and, once again, never
disclosed his line of credit. When the second bank questioned him about
the \$14 million he owed, he ``misled'' it, prosecutors said, saying he
had closed the line of credit in 2014.

In December 2015, Mr. Cohen asked the bank for more money --- this time
for a \$500,000 home-equity loan, the government said. (The same one he
is accused of having used to pay Ms. Clifford.)

\href{https://www.nytimes.com/interactive/2018/05/03/us/politics/giuliani-stormy-trump-statements.html}{}

\includegraphics{https://static01.nyt.com/images/2018/08/22/us/giuliani-stormy-trump-statements-promo-1534971404035/giuliani-stormy-trump-statements-promo-1534971404035-articleLarge.png}

\hypertarget{heres-everything-trumps-team-has-said-about-the-payment-to-stormy-daniels}{%
\subsection{Here's Everything Trump's Team Has Said About the Payment to
Stormy
Daniels}\label{heres-everything-trumps-team-has-said-about-the-payment-to-stormy-daniels}}

From complete denial to acknowledging involvement, what President Trump
and his lawyers said about the \$130,000 paid to the pornographic film
actress.

In getting the loan, court papers say, Mr. Cohen ``significantly
understated'' his debt and falsely represented that he was worth more
than \$40 million at the time.

\hypertarget{a-profitable-taxi-business--but-maybe-not-on-his-taxes}{%
\subsection{A profitable taxi business --- but maybe not on his
taxes.}\label{a-profitable-taxi-business--but-maybe-not-on-his-taxes}}

Mr. Cohen had been involved in the taxi business for years, the
government said, earning millions of dollars by leasing taxi medallions
to operators in Chicago and New York who paid him a portion of their
income.

He also made money, prosecutors said, by offering what amounted to a
total of \$6 million in personal loans to one taxi operator and
collecting interest.

The problem was, the government said, Mr. Cohen did not pay taxes on
much of the money he made from the medallions and the loans.

Instead, the government said, he hid millions of dollars in profits in
his and his wife's bank accounts and failed to tell his personal
accountant.

\hypertarget{wait-theres-more-including-a-french-handbag}{%
\subsection{Wait, there's more (including a French
handbag).}\label{wait-theres-more-including-a-french-handbag}}

Mr. Cohen not only disguised the income he earned from his taxi
business, prosecutors said, but he also failed to disclose \$100,000 he
made in 2014 from brokering the sale of a piece of property in ``a
private aviation community'' in Ocala, Fla., and another \$30,000 he
made from brokering the sale of a Birkin bag, ``a highly coveted French
handbag,'' the government explained.

Then there was the \$200,000 in consulting fees that he took in and did
not disclose as income from working with ``an assisted living company,''
prosecutors said, which he gave advice to about real-estate deals.

\hypertarget{enough-about-mr-cohen-what-does-this-mean-for-the-president}{%
\subsection{Enough about Mr. Cohen. What does this mean for the
president?}\label{enough-about-mr-cohen-what-does-this-mean-for-the-president}}

At the moment, it's hard to say.

Mr. Cohen's plea agreement with the prosecutors in Manhattan does not
require him to cooperate with other pending investigations. But it also
does not preclude him in telling what he knows about Mr. Trump to
investigators working with the special counsel, Robert S. Mueller III,
who is looking into potential ties between the Trump campaign and
Russia.

Mr. Cohen's agreement with the government contains a provision that
could allow him to receive a significantly reduced sentence. If Mr.
Cohen were to substantially assist the special counsel's investigation,
Mr. Mueller could recommend a reduction.

Looming over negotiations between prosecutors and Mr. Cohen has been the
possibility of a presidential pardon.

Mr. Cohen's lawyer at one point raised the
\href{https://www.nytimes.com/2018/05/31/us/politics/pardons-trump.html}{issue
of a pardon} with Mr. Trump's several months ago, The Times reported.

By striking a deal with Mr. Cohen that includes prison time, federal
authorities were aware of the risk that the president might pardon him.

But the president has given no indication that he was leaning toward
one.

Advertisement

\protect\hyperlink{after-bottom}{Continue reading the main story}

\hypertarget{site-index}{%
\subsection{Site Index}\label{site-index}}

\hypertarget{site-information-navigation}{%
\subsection{Site Information
Navigation}\label{site-information-navigation}}

\begin{itemize}
\tightlist
\item
  \href{https://help.nytimes.com/hc/en-us/articles/115014792127-Copyright-notice}{©~2020~The
  New York Times Company}
\end{itemize}

\begin{itemize}
\tightlist
\item
  \href{https://www.nytco.com/}{NYTCo}
\item
  \href{https://help.nytimes.com/hc/en-us/articles/115015385887-Contact-Us}{Contact
  Us}
\item
  \href{https://www.nytco.com/careers/}{Work with us}
\item
  \href{https://nytmediakit.com/}{Advertise}
\item
  \href{http://www.tbrandstudio.com/}{T Brand Studio}
\item
  \href{https://www.nytimes.com/privacy/cookie-policy\#how-do-i-manage-trackers}{Your
  Ad Choices}
\item
  \href{https://www.nytimes.com/privacy}{Privacy}
\item
  \href{https://help.nytimes.com/hc/en-us/articles/115014893428-Terms-of-service}{Terms
  of Service}
\item
  \href{https://help.nytimes.com/hc/en-us/articles/115014893968-Terms-of-sale}{Terms
  of Sale}
\item
  \href{https://spiderbites.nytimes.com}{Site Map}
\item
  \href{https://help.nytimes.com/hc/en-us}{Help}
\item
  \href{https://www.nytimes.com/subscription?campaignId=37WXW}{Subscriptions}
\end{itemize}
