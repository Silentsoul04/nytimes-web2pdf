Sections

SEARCH

\protect\hyperlink{site-content}{Skip to
content}\protect\hyperlink{site-index}{Skip to site index}

\href{https://www.nytimes.com/section/technology}{Technology}

\href{https://myaccount.nytimes.com/auth/login?response_type=cookie\&client_id=vi}{}

\href{https://www.nytimes.com/section/todayspaper}{Today's Paper}

\href{/section/technology}{Technology}\textbar{}Too Many Chinese
Children Need Glasses. Beijing Blames Video Games.

\url{https://nyti.ms/2LJnDsr}

\begin{itemize}
\item
\item
\item
\item
\item
\end{itemize}

Advertisement

\protect\hyperlink{after-top}{Continue reading the main story}

Supported by

\protect\hyperlink{after-sponsor}{Continue reading the main story}

\hypertarget{too-many-chinese-children-need-glasses-beijing-blames-video-games}{%
\section{Too Many Chinese Children Need Glasses. Beijing Blames Video
Games.}\label{too-many-chinese-children-need-glasses-beijing-blames-video-games}}

\includegraphics{https://static01.nyt.com/images/2018/09/01/world/01chinagames-span/merlin_142714389_dd941c45-c042-4e37-b9f5-332bbb8e5508-articleLarge.jpg?quality=75\&auto=webp\&disable=upscale}

By \href{https://www.nytimes.com/by/raymond-zhong}{Raymond Zhong}

\begin{itemize}
\item
  Aug. 31, 2018
\item
  \begin{itemize}
  \item
  \item
  \item
  \item
  \item
  \end{itemize}
\end{itemize}

\href{https://cn.nytimes.com/china/20180903/china-videogames-myopia-tencent/}{阅读简体中文版}\href{https://cn.nytimes.com/china/20180903/china-videogames-myopia-tencent/zh-hant/}{閱讀繁體中文版}

BEIJING --- It started this week with a
\href{http://www.xinhuanet.com/politics/2018-08/28/c_1123341203.htm}{call
to action} from China's leader, Xi Jinping. Too many of the country's
children need glasses, he said, and the government was going to do
something about it.

It ended on Friday with billions of dollars being wiped from the market
value of the world's largest video game company.

New controls on online games were among Chinese authorities'
recommendations for reducing adolescent nearsightedness on Thursday,
sending shares in the country's leading game publisher, Tencent,
tumbling the next day. Shares of Japanese game makers like Capcom,
Konami and Bandai Namco also fell on Friday, a sign of the size and
importance of the Chinese market.

The sell-off is the latest in a series of government-related stumbles
for Tencent, one of the world's largest technology companies. Chinese
state media has blamed video games for causing young people to become
addicted, lowering their grades and worse. An episode last year, in
which a 17-year-old in the southern city of Guangzhou died after playing
a smartphone game for 40 hours straight, received wide attention.

As the biggest game distributor in the world's biggest game market,
Tencent has grown fantastically rich in recent years. It has bought up
game developers around the world, including the makers of influential
titles such as League of Legends and Clash of Clans. It owns a stake in
Epic Games, creator of
\href{https://www.nytimes.com/2018/05/02/style/fortnite.html}{the
international blockbuster Fortnite}.

Back at home, Tencent also operates China's most popular messaging app,
WeChat, and processes a big chunk of the smartphone payments that are
now used to make transactions of all kinds in the country.

But over the last year, Tencent's hugely profitable game business has
come under fire as Beijing takes a more forceful approach to guiding
Chinese culture --- a reminder of the state's
\href{https://www.nytimes.com/2018/05/02/technology/china-xi-jinping-technology-innovation.html}{growing
role} in deciding the fortunes of the country's largest and most
innovative private companies.

Last year, the Communist Party's official mouthpiece, the People's
Daily,
\href{https://www.nytimes.com/2017/08/16/business/china-honor-of-kings.html}{called
the Tencent-developed battle game Honor of Kings} a ``poison'' on young
minds. In response, the company imposed limits on the amount of time
young people could spend playing it each day.

More recently, Chinese regulators blocked sales of another Tencent title
--- Monster Hunter: World --- because it was deemed too gory. The
company's stock also took a slide after executives said that a
bureaucratic reshuffle had slowed the process for getting licenses to
make money on new games such as the mobile version of PlayerUnknown's
Battlegrounds.

Tencent's shares fell 5 percent in Friday trading in Hong Kong. A
company spokesman declined to comment.

In China, bad eyesight has become an increasingly common childhood
scourge. The state news agency Xinhua
\href{http://www.xinhuanet.com/politics/2018-08/28/c_1123341203.htm}{reported
this week} that Mr. Xi was moved to act after reading a press report on
the subject. Nearly half of all Chinese are nearsighted, according to
Xinhua.

``The vision health of our country's young people has always been of
great concern to General Secretary Xi Jinping,''
\href{http://www.xinhuanet.com/politics/xxjxs/2018-08/29/c_1123347514.htm}{the
news agency wrote}, using one of Mr. Xi's official titles.

The
\href{http://www.moe.edu.cn/srcsite/A17/moe_943/s3285/201808/t20180830_346672.html}{resulting
plan}, issued on Thursday by the Ministry of Education, directs China's
media regulator to limit the number of new games approved for
distribution, although it does not suggest a specific limit. It also
encourages the regulator to explore measures to limit the amount of time
minors can spend playing games, and to consider a system for rating
games for age appropriateness.

Other recommendations in the notice include improving physical education
in schools, limiting the amount of written homework given to elementary
school students and improving lighting conditions in classrooms.

The link between screen time and myopia, while popularly held, is a
matter of continuing study among scientists.
\href{http://www.who.int/blindness/causes/MyopiaReportforWeb.pdf}{A 2015
report} by the World Health Organization cited research indicating that
nearsightedness was related to less time spent outdoors and more time
doing activities such as reading, studying and focusing on screens.

Advertisement

\protect\hyperlink{after-bottom}{Continue reading the main story}

\hypertarget{site-index}{%
\subsection{Site Index}\label{site-index}}

\hypertarget{site-information-navigation}{%
\subsection{Site Information
Navigation}\label{site-information-navigation}}

\begin{itemize}
\tightlist
\item
  \href{https://help.nytimes.com/hc/en-us/articles/115014792127-Copyright-notice}{©~2020~The
  New York Times Company}
\end{itemize}

\begin{itemize}
\tightlist
\item
  \href{https://www.nytco.com/}{NYTCo}
\item
  \href{https://help.nytimes.com/hc/en-us/articles/115015385887-Contact-Us}{Contact
  Us}
\item
  \href{https://www.nytco.com/careers/}{Work with us}
\item
  \href{https://nytmediakit.com/}{Advertise}
\item
  \href{http://www.tbrandstudio.com/}{T Brand Studio}
\item
  \href{https://www.nytimes.com/privacy/cookie-policy\#how-do-i-manage-trackers}{Your
  Ad Choices}
\item
  \href{https://www.nytimes.com/privacy}{Privacy}
\item
  \href{https://help.nytimes.com/hc/en-us/articles/115014893428-Terms-of-service}{Terms
  of Service}
\item
  \href{https://help.nytimes.com/hc/en-us/articles/115014893968-Terms-of-sale}{Terms
  of Sale}
\item
  \href{https://spiderbites.nytimes.com}{Site Map}
\item
  \href{https://help.nytimes.com/hc/en-us}{Help}
\item
  \href{https://www.nytimes.com/subscription?campaignId=37WXW}{Subscriptions}
\end{itemize}
