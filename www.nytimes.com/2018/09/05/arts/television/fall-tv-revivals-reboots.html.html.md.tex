Sections

SEARCH

\protect\hyperlink{site-content}{Skip to
content}\protect\hyperlink{site-index}{Skip to site index}

\href{https://www.nytimes.com/section/arts/television}{Television}

\href{https://myaccount.nytimes.com/auth/login?response_type=cookie\&client_id=vi}{}

\href{https://www.nytimes.com/section/todayspaper}{Today's Paper}

\href{/section/arts/television}{Television}\textbar{}This Fall,
Everything Old Is TV Again

\href{https://nyti.ms/2Nlp70Q}{https://nyti.ms/2Nlp70Q}

\begin{itemize}
\item
\item
\item
\item
\item
\end{itemize}

Advertisement

\protect\hyperlink{after-top}{Continue reading the main story}

Supported by

\protect\hyperlink{after-sponsor}{Continue reading the main story}

\hypertarget{this-fall-everything-old-is-tv-again}{%
\section{This Fall, Everything Old Is TV
Again}\label{this-fall-everything-old-is-tv-again}}

\includegraphics{https://static01.nyt.com/images/2018/08/16/arts/16tv-revival-grid/16tv-revival-grid-articleLarge.jpg?quality=75\&auto=webp\&disable=upscale}

By \href{https://www.nytimes.com/by/james-poniewozik}{James Poniewozik}

\begin{itemize}
\item
  Sept. 5, 2018
\item
  \begin{itemize}
  \item
  \item
  \item
  \item
  \item
  \end{itemize}
\end{itemize}

Looking at the 2018-19 TV schedule is like flipping the dials on a time
machine. The season includes ``Magnum P.I.'' (whose predecessor debuted
in 1980),``Murphy Brown'' (1988), ``Charmed'' (1998), ``Roswell, New
Mexico'' (from ``Roswell,'' 1999) and ``The Conners,'' spun off after
Roseanne Barr's self-immolation by Twitter, from the revival of
``Roseanne'' (1988).

And this is just the \emph{new} old stuff. These remakes join, to name a
scant few, Netflix's ``Lost in Space'' (1965); CW's ``Dynasty'' (1981);
and CBS's ``Hawaii Five-0'' (1968), ``S.W.A.T.'' (1975) and ``MacGyver''
(1985). We've gotten more of ``The X-Files'' (1993) and ``Prison Break''
(2005). Still to come, new versions of ``Rugrats'' (1991) and ``Daria''
(1997) and a wrap-up movie for ``Deadwood'' (2004).

It's enough temporal hopscotching to fill an entire new series of
``Quantum Leap.'' (Which somehow has not been greenlighted yet, but give
it a month or two.)

\includegraphics{https://static01.nyt.com/images/2018/09/16/arts/16tv-revival-oneday/16tv-revival-oneday-articleLarge.jpg?quality=75\&auto=webp\&disable=upscale}

I don't generally get excited about the comeback of a show that last
aired back when I ate out of a lunchbox (possibly a lunchbox for that
same show). And yet my own record betrays me: For all my bias toward the
new,
\href{https://www.nytimes.com/2017/12/04/arts/television/best-tv-shows.html}{my
top 10 TV list for 2017} included two blasts from the past, Netflix's
``One Day at a Time'' and ``Twin Peaks: The Return'' on Showtime.

{[}\href{https://www.nytimes.com/interactive/2018/arts/nyc-arts-calendar.html}{\emph{Never
miss a show again: Add this fall's most anticipated cultural events
directly to your calendar}}\emph{.}{]}

So if we're destined to be borne back ceaselessly into TV's past, we may
as well look at what makes a good remake and what doesn't.

First, some definitions are in order. This broad genre --- Re TV, let's
call it --- falls mainly into two categories.

First you've got your reboots: old titles being remade for another era,
with new casts, and possibly new settings and characters. This includes
the new ``Magnum,'' ``Roswell'' and ``Charmed.''

Then you have revivals: series, exhumed as if from the grave, with the
same characters played by the same actors, picking up years or decades
later. These include the new ``Murphy Brown,'' ``Roseanne''/''The
Conners'' and the reconstituted ``Will \& Grace.''

It is true that TV has been ransacking its past almost as long as it's
had one; there is a history of ``Brady Bunch'' and ``Gilligan's Island''
reunion movies to attest to that. Critics like me have been griping
about TV's recycling of ideas for just as long. So yes: even the
complaints are remakes.

But the sheer tonnage of Re TV today reflects a culture that's
increasingly like a Netflix home page, full of ``If You Liked \ldots''
and ``Watch It Again.'' And in many ways the glut of these shows, even
when they appear on the old legacy networks, is a product of the
streaming-TV era.

Image

From left, Lauren Graham and Alexis Bledel in ``Gilmore Girls: A Year in
the Life'' and in the original ``Gilmore Girls.''Credit...Saeed
Adyani/Netflix; WB

New media have made TV's past ever-present and ever more accessible.
``Gilmore Girls,'' for instance, developed a brand-new audience in
reruns and on Netflix, which was primed for its 2016 revival. (Netflix
in particular --- which also gave us ``Arrested Development'' and
``Fuller House'' --- is able to both manufacture nostalgic cravings and
satisfy them.)

What's more, between streaming and cable, there's simply so much TV now
(there will probably be more than 500 original scripted series in 2018)
that it's a battle for anything new to get attention. A familiar title,
like ``S.W.A.T.,'' offers a boost that ``Another CBS Cop Show'' doesn't.

But these business considerations intersect with a broader hunger to
relive the past and sometimes to relitigate it.

Take the movies, where the new, more diverse ``Star Wars'' films and the
2016 female-led remake of ``Ghostbusters'' became culture-war battles
over representation and who owns the past and the future. It's not just
coincidence that ``erasure'' has been a theme in both the arguments over
these movies and the removal of Confederate monuments, or that the
``Ghostbusters'' fight drew in alt-right figures like Milo Yiannopoulos.

TV remakes are a form of comfort food, but they're also spreading at
precisely the time that TV's past has been returning, sometimes
discomfitingly: the conviction of Bill Cosby, for instance, and
\href{https://www.nytimes.com/2018/04/29/opinion/simpsons-apu-brownface.html}{the
controversy over Apu}, the Indian-immigrant character on ``The
Simpsons'' since 1990.

Discomfort was, for better and worse,
\href{https://www.nytimes.com/2018/05/22/arts/television/roseanne-season-finale-roseanne-barr.html}{a
feature of last season's ``Roseanne,''} which found its extended family
still struggling financially and its title character having made a
bitter right-wing turn that echoed its star's.

Image

Candice Bergen, as Murphy Brown, with her son Avery, on the original
series; right, Candice Bergen as Murphy Brown, and Jake McDorman as
Avery on the revival.Credit... Bettmann/Getty Images; CBS

The revived ``Murphy Brown'' hasn't been screened for critics (CBS
picked it up without shooting a pilot), but it stands to be a
counterpoint to ``Roseanne'' just as it was when it began three decades
ago.

Where ``Roseanne'' modeled a populist blue-collar feminism, the
hard-edge TV-anchor protagonist of ``Murphy Brown'' (Candice Bergen)
spoke to power from a position of power. The original series was
relentlessly of-the-moment, both within its scripts and without. (Vice
President Dan Quayle famously went to war with the show when Murphy had
a baby while single.) The revival, from its description, aims to be just
as topical, with Murphy hosting a cable-news morning show, while her
grown son (Jake McDorman) hosts a conservative competitor.

In her recent book, ``Stealing the Show,'' Joy Press notes that while
Diane English, the creator of ``Murphy Brown,'' admired ``Roseanne,''
Ms. Barr ``felt alienated by the kind of middle-class liberal feminism
Murphy Brown represented.'' This may have foreshadowed the two shows'
divergence in 2018, with Roseanne having gone MAGA into the sunset,
while Murphy hangs in with the post-Hillary left.

But revivals are also conservative by nature --- artistically, if not
necessarily politically. By definition, they're trying to re-create the
past in the present, simulating the appeal of the original even as they
show how the world has changed around the characters. For the typical
revival, the best-case scenario is getting the viewer to say, ``This
feels like the same show I used to watch back then.''

Image

From left, Kyle MacLachlan in the finale of ``Twin Peaks: The Return''
and Frank Silva and Mr. MacLachlan in ``Twin Peaks.''Credit...Suzanne
Tenner/Showtime; ABC

There are exceptions; one reason ``Twin Peaks: The Return'' worked so
well was that, visually and in tone, it felt like a different work from
the series that aired in the early `90s. When it overtly called back to
the past --- as when Audrey reprised her dance from the original series
--- it was with the nightmarish sense that something had died and come
back changed.

But more often, the point of a revival is to deny change, at least
creatively. ``Will \& Grace'' acknowledged the new era with a
heavy-handed premiere episode about the Trump administration. But the
new season was so eerily like the original in its rhythms and tone that
you could convince yourself it had never gone off the air (as long as
you forgot about the original series ending that the revival
Etch-a-Sketched).

At its heart, the appeal of a revival is the appeal of a high school
reunion, or a visit to Facebook: What are they doing now? But it can be
as unsettling to find your old prime-time favorites unnaturally
preserved as to find them changed.

Most sitcoms could theoretically go on for decades. But there's usually
an unspoken expiration date: the point beyond which the characters'
lives would change --- making it a different show --- or, in remaining
the same, they would be less funny than sad. (This is one reason I pray
no one ever loads up a big enough Brink's truck to reunite the cast of
``Friends.'')

A reboot, on the other hand, may be successful or disastrous, but it at
least offers the possibility, and the requirement, for rethinking and
transformation. ``Battlestar Galactica,'' after 9/11, turned a breezy
1970s space opera into an ambitious story about politics, religion and
the ethics of survival in the face of an existential threat.

More recently, when Netflix imagined ``One Day at a Time'' with a
Cuban-American family, it was able to speak to modern questions about
immigration and representation, about who defines America and the
working class.

All this points to another distinction of reboots: Revivals, which
reproduce TV's past down to the original casting, have tended to be very
white, as TV's history is. But the planned reboot of ``Party of Five''
will focus on Mexican-American siblings after their parents are
deported, and the ``Roswell'' reboot (also not yet screened for review)
will reportedly have an immigration twist alongside its space-alien
plot.

Image

From left, Jason Behr, Katherine Heigl, Brendan Fehr and Emilie de Ravin
in ``Roswell.''Credit...Scott Humbert/WB

CW's reboot of the witchcraft drama ``Charmed'' aims for another kind of
update. The original was an artifact of the late-`90s ``Buffy'' era of
empowered female heroines. The new version, produced by Jennie Snyder
Urman (``Jane the Virgin''), recalibrates it for the \#MeToo era with a
story of sexual harassment and vengeance.

The pilot hammers hard for timeliness from its opening lines, ``This is
not a witch hunt; it's a reckoning.'' But there's potential in using its
metaphor-heavy subject matter to look at what has and hasn't changed in
20 years.

Not all reboots are so ambitious. CBS's new ``Magnum,'' judging from the
rough pilot sent to critics, is the sort of
charming-rogue-solving-crimes-in-paradise show that could as easily have
aired in 1988 as 2018. Its chief innovation, seemingly, is that Tom
Selleck's successor, Jay Hernandez, is clean-shaven. (The original
`stache can still be found on display with its owner on CBS's ``Blue
Bloods.'')

The popularity of Re TV may partly be a sign of TV's maturing. After
some 70 years as a commercial medium, it has a body of texts that can be
revisited, reattempted, remixed or responded to.

But history can be both a spur and a burden. CBS All Access, for
instance, is rebooting ``The Twilight Zone,'' which has already had two
unmemorable relaunches.

Image

Burgess Meredith in ``The Twilight Zone.''Credit...Silver Screen
Collection/Getty Images

And you could argue that we have another ``Twilight Zone'' already; it's
just called ``Black Mirror.'' Would Charlie Brooker's travelogue of
social-media hell be just as powerful had it launched with the title
``The Twilight Zone: Cyber''? Or would it have been constrained by the
obligation to call back and pay homage to the original?

Then again, I'm not going to bet sight unseen against Jordan Peele, one
of the new ``Twilight Zone'' producers, whose ``Get Out'' had its own
Serlingesque allegorical creepiness. If Mr. Peele and company can use
the ``Twilight Zone'' mantle as a cover to smuggle onto TV something
just as stunning and full of voice, I won't complain about being fed
leftovers.

In the end, nostalgia is like any other genre. You can only judge so
much from a title and a premise, because it comes down to ideas and
execution. Not all Re TV is (re)created equal.

Advertisement

\protect\hyperlink{after-bottom}{Continue reading the main story}

\hypertarget{site-index}{%
\subsection{Site Index}\label{site-index}}

\hypertarget{site-information-navigation}{%
\subsection{Site Information
Navigation}\label{site-information-navigation}}

\begin{itemize}
\tightlist
\item
  \href{https://help.nytimes.com/hc/en-us/articles/115014792127-Copyright-notice}{©~2020~The
  New York Times Company}
\end{itemize}

\begin{itemize}
\tightlist
\item
  \href{https://www.nytco.com/}{NYTCo}
\item
  \href{https://help.nytimes.com/hc/en-us/articles/115015385887-Contact-Us}{Contact
  Us}
\item
  \href{https://www.nytco.com/careers/}{Work with us}
\item
  \href{https://nytmediakit.com/}{Advertise}
\item
  \href{http://www.tbrandstudio.com/}{T Brand Studio}
\item
  \href{https://www.nytimes.com/privacy/cookie-policy\#how-do-i-manage-trackers}{Your
  Ad Choices}
\item
  \href{https://www.nytimes.com/privacy}{Privacy}
\item
  \href{https://help.nytimes.com/hc/en-us/articles/115014893428-Terms-of-service}{Terms
  of Service}
\item
  \href{https://help.nytimes.com/hc/en-us/articles/115014893968-Terms-of-sale}{Terms
  of Sale}
\item
  \href{https://spiderbites.nytimes.com}{Site Map}
\item
  \href{https://help.nytimes.com/hc/en-us}{Help}
\item
  \href{https://www.nytimes.com/subscription?campaignId=37WXW}{Subscriptions}
\end{itemize}
