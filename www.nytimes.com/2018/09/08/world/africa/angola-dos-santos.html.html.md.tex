Sections

SEARCH

\protect\hyperlink{site-content}{Skip to
content}\protect\hyperlink{site-index}{Skip to site index}

\href{https://www.nytimes.com/section/world/africa}{Africa}

\href{https://myaccount.nytimes.com/auth/login?response_type=cookie\&client_id=vi}{}

\href{https://www.nytimes.com/section/todayspaper}{Today's Paper}

\href{/section/world/africa}{Africa}\textbar{}Admitting Errors,
Ex-President of Angola Exits Politics

\url{https://nyti.ms/2O4wYgI}

\begin{itemize}
\item
\item
\item
\item
\item
\end{itemize}

Advertisement

\protect\hyperlink{after-top}{Continue reading the main story}

Supported by

\protect\hyperlink{after-sponsor}{Continue reading the main story}

\hypertarget{admitting-errors-ex-president-of-angola-exits-politics}{%
\section{Admitting Errors, Ex-President of Angola Exits
Politics}\label{admitting-errors-ex-president-of-angola-exits-politics}}

\includegraphics{https://static01.nyt.com/images/2018/09/09/world/09angola/merlin_105621083_5e4c5cc8-2fa9-4389-b8a9-fb99b781716d-articleLarge.jpg?quality=75\&auto=webp\&disable=upscale}

By The Associated Press

\begin{itemize}
\item
  Sept. 8, 2018
\item
  \begin{itemize}
  \item
  \item
  \item
  \item
  \item
  \end{itemize}
\end{itemize}

José Eduardo dos Santos, the former president of Angola who was one of
Africa's longest-serving heads of state, stepped down on Saturday from
the leadership of the country's governing party, acknowledging that he
had made mistakes in almost four decades in power but saying that he
held his ``head high.''

The governing party, the Popular Movement for the Liberation of Angola,
known as M.P.L.A., has led the oil-rich country since its independence
from Portugal in 1975. After the party won the most parliamentary seats
\href{https://www.nytimes.com/2017/08/25/world/africa/angola-election-dos-santos-president-lourenco.html?action=click\&module=RelatedCoverage\&pgtype=Article\&region=Footer}{in
elections a year ago}, it designated João Lourenço, the defense minister
and a former governor, to replace Mr. dos Santos, who had announced that
he would step down.

Since taking office in September 2017, Mr. Lourenço has pledged to
dismantle
\href{https://www.nytimes.com/2017/06/24/world/africa/angola-luanda-jose-eduardo-dos-santos.html?action=click\&module=RelatedCoverage\&pgtype=Article\&region=Footer}{the
corruption that flourished under his predecessor.}

Mr. dos Santos assumed office in 1979, becoming Angola's second
president. In remarks on Saturday, he said that he had not expected to
remain in power so long, and he acknowledged that everyone makes
mistakes.

``I accept that I also committed them,'' said Mr. dos Santos, 76. But he
also defended his record and described the M.P.L.A. as a party of
transformation.

Mr. dos Santos was president during most of Angola's devastating civil
war of 1975 to 2002, but he also oversaw the building boom that
followed. When the conflict ended, the country had a rare opportunity to
rebuild: Its oil production could swell just as prices were high,
allowing it to finance rebuilding and delivering an economic boom that
was rare on the continent.

But the oil-for-infrastructure model came with serious drawbacks. The
deals between Angola and some of its foreign partners in the rebuilding
effort lacked transparency and often resulted in projects of poor
quality, either because of a lack of oversight or outright corruption.

It also presented the politically connected with an opportunity for
self-enrichment, and Mr. dos Santos's inner circle of family and allies
amassed extraordinary wealth.

The former president, who has had health problems, was not a candidate
in elections last year and instead campaigned for Mr. Lourenço as the
party's candidate. Despite that show of unity, divisions between the two
emerged soon after the election.

While few observers expected Mr. Lourenço to move boldly and quickly
against his predecessor, one of his first acts as president was
\href{https://www.nytimes.com/2017/11/15/business/energy-environment/angola-oil.html}{to
dismiss Isabel dos Santos,} the former president's daughter, as
chairwoman of the state-owned oil company Sonangol. She had become
Africa's first female billionaire, according to Forbes.

Mr. Lourenço then dismissed José Filomeno dos Santos, the former
president's son, from the leadership of Angola's sovereign wealth fund.
Both Isabel and José Filomeno dos Santos face corruption investigations
but deny any wrongdoing.

Angola's main opposition party, Unita, has questioned how far Mr.
Lourenço will go to dismantle state corruption. The party said the
country had faced enormous election fraud last year, and filed court
challenges that ultimately failed.

For many in the country, an Angola without Mr. dos Santos at the top is
hard to imagine. But they will continue to see him regularly: His face
appears on the country's currency and on every citizens' national
identity card.

Advertisement

\protect\hyperlink{after-bottom}{Continue reading the main story}

\hypertarget{site-index}{%
\subsection{Site Index}\label{site-index}}

\hypertarget{site-information-navigation}{%
\subsection{Site Information
Navigation}\label{site-information-navigation}}

\begin{itemize}
\tightlist
\item
  \href{https://help.nytimes.com/hc/en-us/articles/115014792127-Copyright-notice}{©~2020~The
  New York Times Company}
\end{itemize}

\begin{itemize}
\tightlist
\item
  \href{https://www.nytco.com/}{NYTCo}
\item
  \href{https://help.nytimes.com/hc/en-us/articles/115015385887-Contact-Us}{Contact
  Us}
\item
  \href{https://www.nytco.com/careers/}{Work with us}
\item
  \href{https://nytmediakit.com/}{Advertise}
\item
  \href{http://www.tbrandstudio.com/}{T Brand Studio}
\item
  \href{https://www.nytimes.com/privacy/cookie-policy\#how-do-i-manage-trackers}{Your
  Ad Choices}
\item
  \href{https://www.nytimes.com/privacy}{Privacy}
\item
  \href{https://help.nytimes.com/hc/en-us/articles/115014893428-Terms-of-service}{Terms
  of Service}
\item
  \href{https://help.nytimes.com/hc/en-us/articles/115014893968-Terms-of-sale}{Terms
  of Sale}
\item
  \href{https://spiderbites.nytimes.com}{Site Map}
\item
  \href{https://help.nytimes.com/hc/en-us}{Help}
\item
  \href{https://www.nytimes.com/subscription?campaignId=37WXW}{Subscriptions}
\end{itemize}
