Sections

SEARCH

\protect\hyperlink{site-content}{Skip to
content}\protect\hyperlink{site-index}{Skip to site index}

\href{https://www.nytimes.com/section/climate}{Climate}

\href{https://myaccount.nytimes.com/auth/login?response_type=cookie\&client_id=vi}{}

\href{https://www.nytimes.com/section/todayspaper}{Today's Paper}

\href{/section/climate}{Climate}\textbar{}These 3 Hurricane
Misconceptions Can Be Dangerous. Scientists Want to Clear Them Up.

\url{https://nyti.ms/2x5DIUe}

\begin{itemize}
\item
\item
\item
\item
\item
\end{itemize}

\href{https://www.nytimes.com/section/climate?action=click\&pgtype=Article\&state=default\&region=TOP_BANNER\&context=storylines_menu}{Climate
and Environment}

\begin{itemize}
\tightlist
\item
  \href{https://www.nytimes.com/2020/07/30/climate/sea-level-inland-floods.html?action=click\&pgtype=Article\&state=default\&region=TOP_BANNER\&context=storylines_menu}{Rising
  Seas}
\item
  \href{https://www.nytimes.com/interactive/2020/climate/trump-environment-rollbacks.html?action=click\&pgtype=Article\&state=default\&region=TOP_BANNER\&context=storylines_menu}{Trump's
  Changes}
\item
  \href{https://www.nytimes.com/interactive/2020/04/19/climate/climate-crash-course-1.html?action=click\&pgtype=Article\&state=default\&region=TOP_BANNER\&context=storylines_menu}{Climate
  101}
\item
  \href{https://www.nytimes.com/interactive/2018/08/30/climate/how-much-hotter-is-your-hometown.html?action=click\&pgtype=Article\&state=default\&region=TOP_BANNER\&context=storylines_menu}{Is
  Your Hometown Hotter?}
\item
  \href{https://www.nytimes.com/newsletters/climate-change?action=click\&pgtype=Article\&state=default\&region=TOP_BANNER\&context=storylines_menu}{Newsletter}
\end{itemize}

Advertisement

\protect\hyperlink{after-top}{Continue reading the main story}

Supported by

\protect\hyperlink{after-sponsor}{Continue reading the main story}

\hypertarget{these-3-hurricane-misconceptions-can-be-dangerous-scientists-want-to-clear-them-up}{%
\section{These 3 Hurricane Misconceptions Can Be Dangerous. Scientists
Want to Clear Them
Up.}\label{these-3-hurricane-misconceptions-can-be-dangerous-scientists-want-to-clear-them-up}}

\includegraphics{https://static01.nyt.com/images/2018/09/12/science/12cli-risks-print/merlin_143593116_af9e19e2-6414-4369-8aa0-f9530b04a829-articleLarge.jpg?quality=75\&auto=webp\&disable=upscale}

\href{https://www.nytimes.com/by/kendra-pierre-louis}{\includegraphics{https://static01.nyt.com/images/2018/07/16/multimedia/author-kendra-pierre-louis/author-kendra-pierre-louis-thumbLarge.png}}

By \href{https://www.nytimes.com/by/kendra-pierre-louis}{Kendra
Pierre-Louis}

\begin{itemize}
\item
  Sept. 11, 2018
\item
  \begin{itemize}
  \item
  \item
  \item
  \item
  \item
  \end{itemize}
\end{itemize}

\emph{Want climate news in your inbox?}
\href{https://www.nytimes.com/newsletters/climate-change}{\emph{Sign up
here
for}}\textbf{\href{https://www.nytimes.com/newsletters/climate-change}{\emph{Climate
Fwd:}}}\emph{, our email newsletter.}

They warn. They plead. They scold and cajole. Forecasters and public
officials will try just about anything to get residents to flee
coastlines ahead of a hurricane. Last year as Hurricane Harvey barreled
toward the Gulf Coast, the mayor pro tem of Rockport, Tex.,
\href{https://www.kiiitv.com/article/weather/rockport-mayor-pro-tem-those-who-dont-evacuate-should-mark-social-security-number-on-their-arm/467431326}{said
people who insisted on staying} should ``mark their arm with a Sharpie
pen --- put their Social Security number on it and their name.''

Fearing that Hurricane Florence could also be deadly, the governors of
North and South Carolina
\href{https://www.nytimes.com/2018/09/10/us/hurricane-florence.html}{ordered
evacuations} this week in many coastal counties. But experts know that
not all residents will heed the warnings, and some say part of the
reason is that storm forecasts and risks are inadequately communicated
to the public.

``There's a big gap between the forecasts that are available within the
weather community and in some cases the information that people receive
and are able to use,'' said Rebecca Morss, a senior scientist at the
National Center for Atmospheric Research in Boulder, Colo.

\hypertarget{the-cone-of-uncertainty-is-confusing}{%
\subsection{The `cone of uncertainty' is
confusing}\label{the-cone-of-uncertainty-is-confusing}}

A prime example of that perception gap is the familiar ``cone of
uncertainty'' seen in hurricane tracking maps, which can be easily
misread.

*``*The cone is misunderstood,'' said Jeff Masters, a meteorologist with
the forecasting service Weather Underground. ``A lot of people look at
the cone and think, `Oh, that's the width of the storm, or that's the
area that we expect to get the impacts.' But no, that's where we expect
the center of the storm to track.''

Even if the eye of the hurricane stays within the cone, which it does
\href{https://www.nhc.noaa.gov/aboutcone.shtml}{about two-thirds of the
time}, people outside the cone can still experience catastrophic winds,
floods and storm surges.

\href{https://www.nytimes.com/interactive/2018/09/10/us/hurricane-florence-tracking-map.html}{}

\includegraphics{https://static01.nyt.com/images/2018/09/11/us/hurricane-florence-tracking-map-promo-1536673849664/hurricane-florence-tracking-map-promo-1536673849664-articleLarge-v31.png}

\hypertarget{maps-hurricane-florences-approach-toward-the-carolinas}{%
\subsection{Maps: Hurricane Florence's Approach Toward the
Carolinas}\label{maps-hurricane-florences-approach-toward-the-carolinas}}

The Category 2 storm approached the North Carolina coast on Thursday,
with winds of up to 110 miles an hour.

\hypertarget{whats-deadlier-wind-or-water}{%
\subsection{What's deadlier, wind or
water?}\label{whats-deadlier-wind-or-water}}

A hurricane's category, which refers to the storm's powerful wind
speeds, also captures attention. (Hurricane Florence is currently a
Category 4 storm.) But the storm surge, the rising water pushed ashore
by those winds, is
\href{https://journals.ametsoc.org/doi/full/10.1175/BAMS-D-12-00074.1}{far
deadlier} than the wind itself, mostly because of drownings. The storm
surge
\href{https://www.nhc.noaa.gov/surge/StormSurgeCanBeDeadly10tips-single.pdf}{does
not correlate} with the hurricane category.

*``*The reason you evacuate is for the storm surge,'' Dr. Masters said.
``You don't need to evacuate for winds --- it's better to shelter in
place.''

Hurricane Florence is expected to create significant storm surge in
North Carolina, in part because human-caused climate change has raised
sea levels in the region by several inches since 1954, the last time a
Category 4 storm hit the state.

\href{https://www.nytimes.com/section/climate?action=click\&pgtype=Article\&state=default\&region=MAIN_CONTENT_1\&context=storylines_keepup}{}

\hypertarget{climate-and-environment-}{%
\subsubsection{Climate and Environment
›}\label{climate-and-environment-}}

\hypertarget{keep-up-on-the-latest-climate-news}{%
\paragraph{Keep Up on the Latest Climate
News}\label{keep-up-on-the-latest-climate-news}}

Updated July 30, 2020

Here's what you need to know about the latest climate change news this
week:

\begin{itemize}
\item
  \begin{itemize}
  \tightlist
  \item
    \href{https://www.nytimes.com/2020/07/30/climate/bangladesh-floods.html?action=click\&pgtype=Article\&state=default\&region=MAIN_CONTENT_1\&context=storylines_keepup}{Floods
    in}\href{https://www.nytimes.com/2020/07/30/climate/bangladesh-floods.html?action=click\&pgtype=Article\&state=default\&region=MAIN_CONTENT_1\&context=storylines_keepup}{Bangladesh}
    are punishing the people least responsible for climate change.
  \item
    As climate change raises sea levels,
    \href{https://www.nytimes.com/2020/07/30/climate/sea-level-inland-floods.html?action=click\&pgtype=Article\&state=default\&region=MAIN_CONTENT_1\&context=storylines_keepup}{storm
    surges and high tides} are likely to push farther inland.
  \item
    The E.P.A. inspector general plans to investigate whether a rollback
    of fuel efficiency standards
    \href{https://www.nytimes.com/2020/07/27/climate/trump-fuel-efficiency-rule.html?action=click\&pgtype=Article\&state=default\&region=MAIN_CONTENT_1\&context=storylines_keepup}{violated
    government rules}.
  \end{itemize}
\end{itemize}

\emph{{[}For the latest updates,}
\href{https://www.nytimes.com/2018/09/11/us/hurricane-florence-updates.html}{\emph{read
our Hurricane Florence live briefing here}}\emph{.{]}}

Hurricane winds push water the way a snowplow pushes and piles up snow,
said Arthur DeGaetano, the director of the Northeast Regional Climate
Center at Cornell University. ``Those persistent strong winds blowing in
the same direction literally pile up the water,'' he said.

The speed of the storm surge can catch people off guard, said Julie
Demuth, a research scientist who works with Dr. Morss. ``If they think
they have three hours to get out of the way, or a day to get out of the
way --- when in fact storm surge in some cases can cause inundation,
deep inundation, in a matter of minutes --- then that shapes how they
think about what they're able to do and how they can respond.''

\hypertarget{the-threat-isnt-limited-to-the-coasts}{%
\subsection{The threat isn't limited to the
coasts}\label{the-threat-isnt-limited-to-the-coasts}}

Even the height of the storm surge may not reflect the true danger, Dr.
DeGaetano said. ``The impact of the surge is not necessarily how high it
is but how far inland --- how far horizontally --- that that amount of
surge will eventually flood when it reaches the coast,'' he said.

Hurricane Florence may create additional complications after making
landfall. The storm is expected to stall over the region for days,
dumping as much as two feet of rain, including over inland regions.

``If you live next to a river that's been subject to repeat flooding
over the last few decades, you might also want to consider leaving if
you're in eastern North Carolina, because we're going to see a lot of
freshwater flooding from heavy rains,'' Dr. Masters said.

\emph{{[}Here are}
\href{https://www.nytimes.com/2018/09/11/us/hurricane-preparedness-evacuation.html}{\emph{some
tips for how you can prepare to evacuate}}\emph{.{]}}

\hypertarget{heres-what-scientists-want-to-change}{%
\subsection{Here's what scientists want to
change}\label{heres-what-scientists-want-to-change}}

Dr. Morss and Dr. Demuth, the scientists who work at the National Center
for Atmospheric Research, are part of a growing social research effort
to understand how people respond to weather messages.

It is undeniable that improved weather forecasting has helped drive down
deaths linked to extreme weather. Still, there is room for improvement:
As Hurricane Sandy approached New Jersey in 2012, only 49 percent of
coastal residents under mandatory evacuation orders left before the
storm,
\href{https://www.monmouth.edu/polling-institute/documents/monmouthpoll_njsandycoast_050713.pdf/}{according
to the Monmouth University Polling Institute}.

Research by Dr. Morss, Dr. Demuth and their colleagues found that many
people have difficulty grasping storm surge information. So they tried
using different visual and written messages to see if they better
communicated what storm surge could look like at different levels: one
foot, three to six feet and six to nine feet.

\includegraphics{https://static01.nyt.com/images/2018/09/11/climate/11cli-risk/11cli-risk-articleLarge.gif?quality=75\&auto=webp\&disable=upscale}

``We found that for some people it really helped them visualize the risk
and understand what it was going to be or what it could be,'' said Dr.
Morss.

One limitation of surveys and interviews is that they happen after the
fact, which means participants can't provide detailed recollections of
what they were doing at specific times or in response to specific
information. So the researchers are turning to Twitter for a real-time
record of what people are thinking, doing and saying as a weather event
approaches.

Twitter ``gives us a sense of when they do start talking about weather
information, how that fits into their broader lives, and what are the
kinds of pieces of information that really attract the attention,'' Dr.
Demuth said.

The National Hurricane Center, which produces the hurricane forecast
maps that include the cone of uncertainty, said it would be using social
science to study improvements. The center's storm surge graphics were
already updated last year based on social studies, said Dennis Feltgen,
a spokesman.

Dr. Morss and Dr. Demuth cautioned that better messaging was only part
of the battle. No matter how good the information is, many people cannot
act on it for health, financial or other reasons. During Hurricane
Katrina, for example, many people did not evacuate because they lacked
access to a car or had nowhere to go.

Dr. Demuth recalled interviewing a woman in her 70s or 80s who had used
an underground shelter to survive a tornado in northern Arkansas. ``She
said that had her son-in-law not come home, she would not have been able
to go to that shelter, because she's too weak to be able to get the door
open,'' Dr. Demuth said.

``The people who live next door to her, a family of three, was killed
because they were not able to get underground in time.''

For more news on climate and the environment,
\href{https://twitter.com/nytclimate}{follow @NYTClimate on Twitter}.

Advertisement

\protect\hyperlink{after-bottom}{Continue reading the main story}

\hypertarget{site-index}{%
\subsection{Site Index}\label{site-index}}

\hypertarget{site-information-navigation}{%
\subsection{Site Information
Navigation}\label{site-information-navigation}}

\begin{itemize}
\tightlist
\item
  \href{https://help.nytimes.com/hc/en-us/articles/115014792127-Copyright-notice}{©~2020~The
  New York Times Company}
\end{itemize}

\begin{itemize}
\tightlist
\item
  \href{https://www.nytco.com/}{NYTCo}
\item
  \href{https://help.nytimes.com/hc/en-us/articles/115015385887-Contact-Us}{Contact
  Us}
\item
  \href{https://www.nytco.com/careers/}{Work with us}
\item
  \href{https://nytmediakit.com/}{Advertise}
\item
  \href{http://www.tbrandstudio.com/}{T Brand Studio}
\item
  \href{https://www.nytimes.com/privacy/cookie-policy\#how-do-i-manage-trackers}{Your
  Ad Choices}
\item
  \href{https://www.nytimes.com/privacy}{Privacy}
\item
  \href{https://help.nytimes.com/hc/en-us/articles/115014893428-Terms-of-service}{Terms
  of Service}
\item
  \href{https://help.nytimes.com/hc/en-us/articles/115014893968-Terms-of-sale}{Terms
  of Sale}
\item
  \href{https://spiderbites.nytimes.com}{Site Map}
\item
  \href{https://help.nytimes.com/hc/en-us}{Help}
\item
  \href{https://www.nytimes.com/subscription?campaignId=37WXW}{Subscriptions}
\end{itemize}
