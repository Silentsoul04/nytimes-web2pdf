Sections

SEARCH

\protect\hyperlink{site-content}{Skip to
content}\protect\hyperlink{site-index}{Skip to site index}

\href{https://myaccount.nytimes.com/auth/login?response_type=cookie\&client_id=vi}{}

\href{https://www.nytimes.com/section/todayspaper}{Today's Paper}

\href{/section/opinion}{Opinion}\textbar{}Why Does the Tennis Season End
Before It's Over?

\href{https://nyti.ms/2PVJQWM}{https://nyti.ms/2PVJQWM}

\begin{itemize}
\item
\item
\item
\item
\item
\end{itemize}

Advertisement

\protect\hyperlink{after-top}{Continue reading the main story}

\href{/section/opinion}{Opinion}

Supported by

\protect\hyperlink{after-sponsor}{Continue reading the main story}

Sporting

\hypertarget{why-does-the-tennis-season-end-before-its-over}{%
\section{Why Does the Tennis Season End Before It's
Over?}\label{why-does-the-tennis-season-end-before-its-over}}

In October and November, the sport exists (Laver Cup!) and doesn't exist
simultaneously. I'm both fascinated and irritated by this.

By Rowan Ricardo Phillips

Mr. Phillips is the writer of the forthcoming ``The Circuit: A Tennis
Odyssey.''

\begin{itemize}
\item
  Sept. 22, 2018
\item
  \begin{itemize}
  \item
  \item
  \item
  \item
  \item
  \end{itemize}
\end{itemize}

\includegraphics{https://static01.nyt.com/images/2018/09/22/opinion/22sportingWeb/22sportingWeb-articleLarge.jpg?quality=75\&auto=webp\&disable=upscale}

I love tennis. I play it (too much, maybe), and I'm a big fan of the
game. That said, considering myself a fan of the game increasingly feels
strange to me. I can't quite put my finger on why.

It's not because I'm falling out of love with the sport. It's that the
big picture is completely out of focus. I know what a tennis match is.
But what is it that we're talking about when we talk about the game of
tennis? This weekend's Laver Cup --- a men's tournament being held in
Chicago that features top players from around the world, including Roger
Federer and Novak Djokovic ---~has got me wondering.

I'm not referring to \emph{a} game of tennis (15-love, 15-all, etc.),
I'm referring to \emph{the} game of tennis. When we say ``the game of''
--- the game of basketball or baseball, for example --- we're talking
about the idea of the game. We know its structures: its rules, its
contexts and, quite crucially, what a season is. In other words, we know
why the players are playing.

In tennis, the United States Open is a type of unofficial end of the
season for casual viewers. It takes place at the end of summer, the
final of the four annual Grand Slam tournaments. It's also the final
time in the year that tennis forces its way into the public sphere
---~this year it was
\href{https://www.nytimes.com/2018/09/09/sports/serena-williams-us-open-naomi-osaka.html}{Serena
Williams and the women's final}.

But after the Open, there are still two months of tennis on the
calendar. Do you care who will end the season with the No. 1 ranking?
Will you keep up with October and November matches in Beijing, Tokyo and
Shanghai? Is the cold autumn brutalism of the European indoor season
what you endured the warm and golden days of summer for?

These aren't intended to be rhetorical questions. No doubt some of you
will say yes to all of that (or, as in my case: no, yes, yes). However,
I also suspect that some of you will hardly have any idea what I'm even
talking about. You bailed after the Open, and hey, I can't blame you.

The game of tennis promotes itself around the four majors. They are the
four big events in the story of the year: Who will win the Grand Slam
tournaments and what obstacles will be overcome on that journey? Stay
tuned. At the Open, Naomi Osaka defeated her idol, Williams, playing
off-the-charts tennis, won \$3.8 million, and is now known worldwide.
For 2018 that's the end of the story.

How will Williams respond on the court to the controversy surrounding
her defeat? Tune in next year to know. Novak Djokovic is back and
suddenly dominating again. How will Roger Federer and Rafael Nadal, the
only two players in front of him in the rankings and career major
titles, handle this? I've been asked about this ad nauseam, but as it
relates not to the Shanghai Masters tournament or the Rolex Paris
Masters next month but to the 2019 Australian Open and beyond.

As it stands, tennis in October and November exists and doesn't exist
simultaneously. I'm both fascinated and annoyed by all of this.

The Laver Cup is a three-day team-tennis tournament dreamed up by
Federer (and his management team) and named after his tennis idol, the
Australian great Rod Laver. Last year's inaugural edition dangled the
rare carrot before the public of seeing Federer and his great rival
Nadal share the court as teammates. By all measures it was a rousing
success. But will that success be permanent or, as Sade sang it, are
things never as good as the first time?

\includegraphics{https://static01.nyt.com/images/2018/09/22/opinion/22sporting2/22sporting2-articleLarge.jpg?quality=75\&auto=webp\&disable=upscale}

We'll know a bit after this year's version: with Nadal absent from Team
Europe and Team Rest of the World once again lacking in genuine star
power, the Laver Cup faces an existential question: What exactly is it,
and what will it become? A genuine, competitive fixture on the annual
schedule, as Federer insists it already is? Or, in a sport with a
schedule already bursting at the seams and its players overextended, is
it best served as an exhibition at the back end of a grueling calendar
to celebrate the game?

What's clear is that the Laver Cup has pounced on an opportunity in a
fallow part of the tennis schedule. Those days just after the United
States Open are an exit door for many casual viewers. The Laver Cup is
has offered itself as a stay against the exodus, banking on Federer's
presence and the novelty of rivals becoming teammates for a few days.

Just a year ago, the tournament had the post-Open stage to itself. But
other investors have gotten wise to the act. The
\href{https://www.daviscup.com/en/organisation/davis-cup-history.aspx}{Davis
Cup}, an international team-tennis tournament founded in 1900, plans to
change its format in 2019 amid controversy. The Cup used to take place
throughout the year. In 2019, the 18-nation final of the competition
will be played in one week in November. And now, announced just days
ago, you can add to this weekend the proposed Majesty Cup, a 64-player
winner-take-all roughly \$10 million exhibition tournament. Both the
revamped Davis Cup and the hilariously tone-deaf ``Hunger
Games''-meets-Belle- Époque Majesty Cup are the brainstorm of the soccer
player Gerard Piqué's investment group Kosmos.

Yes, I'm as lost as you.

And so, come September 2019, the three-year old Laver Cup will be the
venerable old institution among a gaggle of tennis tournaments that have
nothing to do with either one another or the tennis season itself.
Another clot on the calendar to try to make sense of or ignore. I guess
the idea is that we'll still be excited to see Roger Federer play
doubles with another European player --- maybe Andy Murray will be
healthy again by then. Or, if not, we'll switch over to whatever
consultancy-speak format Kosmos rolls out for us --- I just hope it
comes with an instruction manual \ldots{} and a return slip.

Rowan Ricardo Phillips
(\href{https://twitter.com/rowanricardo?lang=en}{@RowanRicardo}) is the
writer of the forthcoming
``\href{https://us.macmillan.com/books/9780374123772}{The Circuit: A
Tennis Odyssey}.''

\emph{Follow The New York Times Opinion section on}
\href{https://www.facebook.com/nytopinion}{\emph{Facebook}} \emph{and}
\href{http://twitter.com/NYTOpinion}{\emph{Twitter
(@NYTopinion)}}\emph{, and sign up for the}
\href{http://www.nytimes.com/newsletters/opiniontoday/}{**}
\emph{Opinion Today newsletter.}

Advertisement

\protect\hyperlink{after-bottom}{Continue reading the main story}

\hypertarget{site-index}{%
\subsection{Site Index}\label{site-index}}

\hypertarget{site-information-navigation}{%
\subsection{Site Information
Navigation}\label{site-information-navigation}}

\begin{itemize}
\tightlist
\item
  \href{https://help.nytimes.com/hc/en-us/articles/115014792127-Copyright-notice}{©~2020~The
  New York Times Company}
\end{itemize}

\begin{itemize}
\tightlist
\item
  \href{https://www.nytco.com/}{NYTCo}
\item
  \href{https://help.nytimes.com/hc/en-us/articles/115015385887-Contact-Us}{Contact
  Us}
\item
  \href{https://www.nytco.com/careers/}{Work with us}
\item
  \href{https://nytmediakit.com/}{Advertise}
\item
  \href{http://www.tbrandstudio.com/}{T Brand Studio}
\item
  \href{https://www.nytimes.com/privacy/cookie-policy\#how-do-i-manage-trackers}{Your
  Ad Choices}
\item
  \href{https://www.nytimes.com/privacy}{Privacy}
\item
  \href{https://help.nytimes.com/hc/en-us/articles/115014893428-Terms-of-service}{Terms
  of Service}
\item
  \href{https://help.nytimes.com/hc/en-us/articles/115014893968-Terms-of-sale}{Terms
  of Sale}
\item
  \href{https://spiderbites.nytimes.com}{Site Map}
\item
  \href{https://help.nytimes.com/hc/en-us}{Help}
\item
  \href{https://www.nytimes.com/subscription?campaignId=37WXW}{Subscriptions}
\end{itemize}
