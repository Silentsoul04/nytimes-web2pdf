Sections

SEARCH

\protect\hyperlink{site-content}{Skip to
content}\protect\hyperlink{site-index}{Skip to site index}

\href{https://www.nytimes.com/section/politics}{Politics}

\href{https://myaccount.nytimes.com/auth/login?response_type=cookie\&client_id=vi}{}

\href{https://www.nytimes.com/section/todayspaper}{Today's Paper}

\href{/section/politics}{Politics}\textbar{}11 Takeaways From The
Times's Investigation Into Trump's Wealth

\url{https://nyti.ms/2P3FbC1}

\begin{itemize}
\item
\item
\item
\item
\item
\item
\end{itemize}

Advertisement

\protect\hyperlink{after-top}{Continue reading the main story}

Supported by

\protect\hyperlink{after-sponsor}{Continue reading the main story}

\hypertarget{11-takeaways-from-the-timess-investigation-into-trumps-wealth}{%
\section{11 Takeaways From The Times's Investigation Into Trump's
Wealth}\label{11-takeaways-from-the-timess-investigation-into-trumps-wealth}}

\includegraphics{https://static01.nyt.com/images/2018/09/02/us/02inheritance-13/02inheritance-13-articleLarge.jpg?quality=75\&auto=webp\&disable=upscale}

By \href{https://www.nytimes.com/by/russ-buettner}{Russ Buettner},
\href{https://www.nytimes.com/by/susanne-craig}{Susanne Craig} and
\href{https://www.nytimes.com/by/david-barstow}{David Barstow}

\begin{itemize}
\item
  Oct. 2, 2018
\item
  \begin{itemize}
  \item
  \item
  \item
  \item
  \item
  \item
  \end{itemize}
\end{itemize}

\href{https://www.nytimes.com/es/2018/10/02/investigacion-fortuna-donald-trump/?}{Leer
en español}

\href{https://www.nytimes.com/2020/07/09/us/politics/trump-taxes.html}{Donald
J. Trump} built a business empire and won the presidency proclaiming
himself a self-made billionaire, and he has long insisted that his
father, the legendary New York City builder
\href{https://www.nytimes.com/2020/07/28/us/politics/donald-fred-trump.html}{Fred
C. Trump}, provided almost no financial help. ``I built what I built
myself,'' the president has repeatedly said.

But
\href{https://www.nytimes.com/interactive/2018/10/02/us/politics/donald-trump-tax-schemes-fred-trump.html}{an
investigation by The New York Times} has revealed that Donald
\href{https://www.nytimes.com/interactive/2018/10/02/us/politics/donald-trump-tax-schemes-fred-trump.html?action=click\&module=Top\%20Stories\&pgtype=Homepage}{Trump}
received the equivalent today of at least \$413 million from his
father's real estate empire. What's more, much of this money came to
\href{https://www.nytimes.com/interactive/2019/05/07/us/politics/donald-trump-taxes.html}{Mr.
Trump through dubious tax schemes} he participated in during the 1990s,
including instances of outright fraud, The Times found.

In all, the president's parents transferred well over \$1 billion in
wealth to their children, which could have produced a
\href{https://www.nytimes.com/2020/07/09/us/politics/trump-taxes.html}{tax}
bill of at least \$550 million under the 55 percent tax rate on gifts
and inheritances that was in place at the time. Helped by a variety of
tax dodges, the Trumps paid \$52.2 million, or about 5 percent, tax
returns show.

The president declined requests over several weeks to comment for this
article.

A lawyer for Mr. Trump, Charles J. Harder, provided a written statement.
``There was no fraud or tax evasion by anyone. The facts upon which The
Times bases its false allegations are extremely inaccurate,'' he said.
``President Trump had virtually no involvement whatsoever with these
matters,'' he continued, saying the president had delegated those tasks
to relatives and tax professionals. ``The affairs were handled by other
Trump family members who were not experts themselves and therefore
relied entirely upon the aforementioned licensed professionals to ensure
full compliance with the law.''

\emph{{[}Read}
\href{https://int.nyt.com/data/documenthelper/353-trump-inheritance-taxes-statement/2986a100d3d19a917cdc/optimized/full.pdf\#page=1?action=click\&module=Intentional\&pgtype=Article}{\emph{the
full statement}}\emph{{]}}

In a statement on behalf of the
\href{https://www.nytimes.com/interactive/2019/05/07/us/politics/donald-trump-taxes.html}{Trump}
family, the president's brother, Robert Trump, said, ``All appropriate
gift and estate tax returns were filed, and the required taxes were
paid.''

Since
\href{https://www.nytimes.com/interactive/2018/10/02/us/politics/donald-trump-tax-schemes-fred-trump.html}{Donald
Trump} first refused to release his income tax returns, his campaign and
then his presidency have been suffused with questions about the extent
and sources of his wealth, questions that have only intensified with the
Russia investigation. The Times's new reporting reveals little about his
recent business dealings. But the investigation --- based on a vast
trove of confidential tax returns and financial records, and at more
than 13,000 words one of the longest investigative articles ever
published in The Times --- offers the first comprehensive examination of
the inherited fortune and tax dodges that guaranteed Mr. Trump a gilded
life.

Here are some key takeaways.

\hypertarget{the-trumps-tax-maneuvers-show-a-pattern-of-deception-tax-experts-say}{%
\subsection{The Trumps' tax maneuvers show a pattern of deception, tax
experts
say}\label{the-trumps-tax-maneuvers-show-a-pattern-of-deception-tax-experts-say}}

The line between legal tax avoidance and illegal tax evasion is often
murky, and there is no shortage of clever tax-avoidance tricks that have
been blessed by either the courts or the Internal Revenue Service
itself; the wealthiest Americans rarely pay anything close to full
freight. The Trumps' tax maneuvers met with little resistance from the
I.R.S., The Times found.

But tax experts briefed on The Times's findings said the Trumps appeared
to have done more than exploit legal loopholes. They said the conduct
described here represented a pattern of deception and obfuscation that
repeatedly prevented the I.R.S. from taxing large transfers of wealth to
Fred Trump's children.

\hypertarget{donald-trump-began-reaping-wealth-from-his-fathers-real-estate-empire-as-a-toddler}{%
\subsection{Donald Trump began reaping wealth from his father's real
estate empire as a
toddler}\label{donald-trump-began-reaping-wealth-from-his-fathers-real-estate-empire-as-a-toddler}}

In Donald Trump's version of how he got rich, he was the master
dealmaker who broke free from his father's ``tiny'' Brooklyn and Queens
real estate operation and built a \$10 billion empire that would slap
the Trump name on hotels, high-rises, casinos and golf courses the world
over.

\includegraphics{https://static01.nyt.com/images/2018/09/02/nyregion/02inheritance15/xxinheritance15-articleLarge.jpg?quality=75\&auto=webp\&disable=upscale}

But The Times's investigation makes clear that in every era of Mr.
Trump's life, his finances were deeply entwined with, and dependent on,
his father's wealth. By age 3, he was earning \$200,000 a year in
today's dollars from his father's empire. He was a millionaire by age 8.
In his 40s and 50s, he was receiving more than \$5 million a year.

There was a clear pattern to this largess: When his son began expensive
new projects, Fred Trump increased his help. In the late 1970s, when
Donald Trump crossed the river into the glittering precincts of
Manhattan --- converting the old Commodore Hotel near Grand Central
Terminal into a Grand Hyatt --- his father opened a spigot of loans.
When he made his first forays into Atlantic City casinos a few years
later, his father devised a plan to sharply increase the flow of aid.

\hypertarget{that-small-loan-of-1-million-was-actually-at-least-607-million--much-of-it-never-repaid}{%
\subsection{That `small loan' of \$1 million was actually at least
\$60.7 million --- much of it never
repaid}\label{that-small-loan-of-1-million-was-actually-at-least-607-million--much-of-it-never-repaid}}

In Mr. Trump's books and TV shows and on the campaign trail, a central
trope of his self-mythology has been that, as he began building his own
empire, the only financial help he got from his father was a \$1 million
loan. Not only that: ``I had to pay him back with interest.''

In fact, The Times found, Fred Trump lent his son at least \$60.7
million, or \$140 million in today's dollars. Much of it was never
repaid, records show.

\hypertarget{fred-trump-wove-a-safety-net-that-rescued-his-son-from-one-bad-bet-after-another}{%
\subsection{Fred Trump wove a safety net that rescued his son from one
bad bet after
another}\label{fred-trump-wove-a-safety-net-that-rescued-his-son-from-one-bad-bet-after-another}}

As the 1980s ended, Donald Trump's big bets began to go bust --- Trump
Shuttle, the Plaza Hotel, the Atlantic City casinos. But as he careened
from one financial disaster to another, family partnerships and
companies dramatically increased their payouts.

Between 1989 and 1992, four of the entities that Fred Trump created paid
his son today's equivalent of \$8.3 million. And when Donald Trump
pleaded with bankers for an emergency line of credit, he used as
collateral the stake his father had given him in a group of apartment
buildings.

Tax records also reveal that at the peak of Mr. Trump's financial
distress, in 1990, his father extracted an extraordinary sum --- nearly
\$50 million --- from his empire. While The Times could find no evidence
that Fred Trump made any significant debt payments, charitable donations
or personal expenditures, there are indications that he wanted plenty of
cash on hand to bail out his son if need be.

That was what happened at Trump's Castle casino, where an \$18.4 million
bond payment was due in December 1990. Fred Trump dispatched a trusted
bookkeeper to Atlantic City with checks to buy \$3.5 million in casino
chips without placing a bet. With this ruse --- an illegal loan under
New Jersey gaming laws, resulting in a \$65,000 civil penalty --- Donald
Trump narrowly avoided defaulting on his bonds.

\hypertarget{the-trumps-turned-an-11-million-loan-debt-into-a-legally-questionable-tax-write-off}{%
\subsection{The Trumps turned an \$11 million loan debt into a legally
questionable tax
write-off}\label{the-trumps-turned-an-11-million-loan-debt-into-a-legally-questionable-tax-write-off}}

By 1987, Donald Trump's loan debt to his father had grown to at least
\$11 million. Had Fred Trump simply forgiven the debt, his son would
have owed millions in income taxes. They found another solution --- one
that appears to constitute both an unreported multimillion-dollar gift
and an illegal tax write-off.

That December, records show, Fred Trump spent \$15.5 million to buy a
7.5 percent stake in Trump Palace, his son's condo tower rising on the
Upper East Side of Manhattan. Four years later, tax returns and
financial statements show, Fred Trump sold that stake for just \$10,000.
The buyer, other documents indicate, was his son.

According to tax experts, with Trump Palace condos selling briskly,
selling shares worth \$15.5 million to your son for a mere sliver of
that would constitute a multimillion-dollar gift under I.R.S. rules. But
Fred Trump's tax returns show no such gift to Donald Trump. What they do
reveal is that he used the transaction to declare an enormous tax
write-off. That appears to violate federal tax law that prohibits
deducting any loss from the sale or exchange of property between family
members.

In all, Fred Trump dodged roughly \$8 million in gift taxes and \$5
million in income taxes on the transaction.

\hypertarget{father-and-son-set-out-to-create-the-myth-of-a-self-made-billionaire}{%
\subsection{Father and son set out to create the myth of a self-made
billionaire}\label{father-and-son-set-out-to-create-the-myth-of-a-self-made-billionaire}}

All told, The Times documented 295 distinct streams of revenue Fred
Trump created over five decades to channel wealth to his son.

But the partnership between Donald Trump and his father was about more
than the pursuit, and the preservation, of riches. They were also
confederates in a more ambitious project: creating the myth of Donald J.
Trump, Self-Made Billionaire. If Fred Trump was the silent partner,
helping finance the accouterments of wealth, it was Donald Trump who
spun them into a seductive narrative.

Emblematic of this dynamic is Trump Tower, the talisman of privilege
that established Donald Trump as a player in New York. Fred Trump's
money helped build it. His son recognized and exploited its iconic power
as the primary stage for both ``The Apprentice'' and his presidential
campaign.

\hypertarget{donald-trump-tried-to-change-his-ailing-fathers-will-setting-off-a-family-reckoning}{%
\subsection{Donald Trump tried to change his ailing father's will,
setting off a family
reckoning}\label{donald-trump-tried-to-change-his-ailing-fathers-will-setting-off-a-family-reckoning}}

In December 1990, Donald Trump sent his father a document that left him
both angered and alarmed. It was a codicil seeking to make a variety of
changes to Fred Trump's will. Among them: strengthening provisions that
made Donald Trump sole executor of his estate. But amid Mr. Trump's
financial shambles --- it was the month of the \$3.5 million Trump's
Castle rescue --- Fred Trump feared that the document potentially put
his life's work at risk, that his son might use the empire as collateral
to save his own failing businesses, according to depositions given years
later during a family dispute.

Fred Trump rebuffed the maneuver, refusing to sign the codicil. But the
episode prompted a family reckoning: Fred Trump was aging and ailing.
Without speedy intervention, he could die leaving a vast estate --- not
just his real estate empire, but also tens of millions of dollars in
cash --- vulnerable to the 55 percent inheritance tax.

So with Donald Trump playing a central role, the family formulated a
plan that included unorthodox tax strategies that experts told The Times
were legally dubious and, in some cases, appeared to be fraudulent.

\hypertarget{the-trumps-created-a-company-that-siphoned-cash-from-the-empire}{%
\subsection{The Trumps created a company that siphoned cash from the
empire}\label{the-trumps-created-a-company-that-siphoned-cash-from-the-empire}}

The first major component was creating a company called All County
Building Supply \& Maintenance. On paper, All County was Fred Trump's
purchasing agent, buying everything from boilers to cleaning supplies.
But All County was, in fact, a company only on paper, records and
interviews show --- a vehicle to siphon cash from Fred Trump's empire by
simply marking up purchases already made by his employees. Those
millions in markups, effectively untaxed gifts, then flowed to All
County's owners --- Donald Trump, his siblings and a cousin.

Lee-Ford Tritt, a leading expert in gift and estate tax law at the
University of Florida, said the Trumps' use of All County was ``highly
suspicious'' and could constitute criminal tax fraud. ``It certainly
looks like a disguised gift,'' he said.

Image

In President Trump's version of how he got rich, he was the master
dealmaker who parlayed a \$1 million loan from his father into a \$10
billion empire.Credit...Marilynn K. Yee/The New York Times

All County also had an insidious downside for Fred Trump's tenants. He
used the padded invoices to justify higher rent increases in
rent-regulated buildings, records show.

Mr. Harder, the president's lawyer, disputed The Times's reporting:
``Should The Times state or imply that President Trump participated in
fraud, tax evasion or any other crime, it will be exposing itself to
substantial liability and damages for defamation.''

\hypertarget{the-trump-parents-dodged-hundreds-of-millions-in-gift-taxes-by-grossly-undervaluing-the-assets-they-would-pass-on}{%
\subsection{The Trump parents dodged hundreds of millions in gift taxes
by grossly undervaluing the assets they would pass
on}\label{the-trump-parents-dodged-hundreds-of-millions-in-gift-taxes-by-grossly-undervaluing-the-assets-they-would-pass-on}}

With the cash flowing out of Fred Trump's empire, the Trumps began
transferring ownership of the lion's share of the empire itself to
Donald Trump and his siblings. The vehicle they created to do that was a
special kind of trust called a grantor-retained annuity trust, or GRAT.

The purpose of a GRAT is to pass wealth across generations without
paying the 55 percent estate tax. The Trump parents did have to pay gift
taxes based on one crucial number: the market value of Fred Trump's
empire. But The Times found evidence that they dodged hundreds of
millions of dollars in gift taxes by submitting tax returns that grossly
undervalued the assets placed in two GRATs, one for each parent.

Fred Trump's 1995 gift tax return claimed that the 25 apartment
complexes and other properties in the trusts were worth just \$41.4
million. The implausibility of this claim would be made plain in 2004,
when banks valued that same real estate at nearly \$900 million.

\includegraphics{https://static01.nyt.com/images/2017/01/29/podcasts/the-daily-album-art/the-daily-album-art-articleInline-v2.jpg?quality=75\&auto=webp\&disable=upscale}

\hypertarget{listen-to-the-daily-how-trump-really-got-rich}{%
\subsubsection{Listen to `The Daily': How Trump Really Got
Rich}\label{listen-to-the-daily-how-trump-really-got-rich}}

We don't have President Trump's tax returns. But we have his father's.

``They play around with valuations in extreme ways,'' said Mr. Tritt,
the tax law expert, who was briefed on The Times's findings. ``There are
dramatic fluctuations depending on their purpose.''

Mr. Harder, the president's lawyer, said: ``All estate matters were
handled by licensed attorneys, licensed C.P.A.'s and licensed real
estate appraisers who followed all laws and rules strictly.''

\hypertarget{after-fred-trumps-death-his-empires-most-valuable-asset-was-an-iou-from-donald-trump}{%
\subsection{After Fred Trump's death, his empire's most valuable asset
was an I.O.U. from Donald
Trump}\label{after-fred-trumps-death-his-empires-most-valuable-asset-was-an-iou-from-donald-trump}}

When Fred Trump died in June 1999 at the age of 93, the vast bulk of his
empire was nowhere to be found in his estate --- testament to the
success of the tax strategies devised by the Trumps in the early 1990s.
The single largest item included in his estate tax return was a \$10.3
million I.O.U. from Donald Trump, money his son appears to have borrowed
the year before he died. As for the remnants of empire left in Fred
Trump's estate, the tax return cited appraisals that once again grossly
understated their market values.

As their father's executors, Donald, Maryanne and Robert Trump were
legally responsible for the accuracy of his estate tax return. They were
obligated not only to give the I.R.S. a complete accounting of the value
of his estate's assets, but also to disclose all the taxable gifts he
had made during his lifetime. If they knew anything was wrong and failed
to reveal it, tax experts said, they could be in violation of tax law.

Mr. Harder, the president's lawyer, defended the tax returns filed by
the Trumps. ``The returns and tax positions that The Times now attacks
were examined in real time by the relevant taxing authorities,'' he
said. ``These matters have now been closed for more than a decade.''

\hypertarget{donald-trump-got-a-windfall-when-the-empire-was-sold-but-he-may-have-left-money-on-the-table}{%
\subsection{Donald Trump got a windfall when the empire was sold. But he
may have left money on the
table.}\label{donald-trump-got-a-windfall-when-the-empire-was-sold-but-he-may-have-left-money-on-the-table}}

In 2003, once again in financial trouble, Donald Trump began engineering
the sale of the empire Fred Trump had hoped would never leave the
family. The sale, completed in 2004, brought him his biggest payday ever
from his father: His cut was \$177.3 million, or \$236.2 million in
today's dollars. But as it turned out, banks at the time valued the
empire at hundreds of millions more than the sale price. Donald Trump,
master dealmaker, had sold low.

Advertisement

\protect\hyperlink{after-bottom}{Continue reading the main story}

\hypertarget{site-index}{%
\subsection{Site Index}\label{site-index}}

\hypertarget{site-information-navigation}{%
\subsection{Site Information
Navigation}\label{site-information-navigation}}

\begin{itemize}
\tightlist
\item
  \href{https://help.nytimes.com/hc/en-us/articles/115014792127-Copyright-notice}{©~2020~The
  New York Times Company}
\end{itemize}

\begin{itemize}
\tightlist
\item
  \href{https://www.nytco.com/}{NYTCo}
\item
  \href{https://help.nytimes.com/hc/en-us/articles/115015385887-Contact-Us}{Contact
  Us}
\item
  \href{https://www.nytco.com/careers/}{Work with us}
\item
  \href{https://nytmediakit.com/}{Advertise}
\item
  \href{http://www.tbrandstudio.com/}{T Brand Studio}
\item
  \href{https://www.nytimes.com/privacy/cookie-policy\#how-do-i-manage-trackers}{Your
  Ad Choices}
\item
  \href{https://www.nytimes.com/privacy}{Privacy}
\item
  \href{https://help.nytimes.com/hc/en-us/articles/115014893428-Terms-of-service}{Terms
  of Service}
\item
  \href{https://help.nytimes.com/hc/en-us/articles/115014893968-Terms-of-sale}{Terms
  of Sale}
\item
  \href{https://spiderbites.nytimes.com}{Site Map}
\item
  \href{https://help.nytimes.com/hc/en-us}{Help}
\item
  \href{https://www.nytimes.com/subscription?campaignId=37WXW}{Subscriptions}
\end{itemize}
