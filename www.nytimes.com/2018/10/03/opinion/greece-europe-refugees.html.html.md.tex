Sections

SEARCH

\protect\hyperlink{site-content}{Skip to
content}\protect\hyperlink{site-index}{Skip to site index}

\href{https://myaccount.nytimes.com/auth/login?response_type=cookie\&client_id=vi}{}

\href{https://www.nytimes.com/section/todayspaper}{Today's Paper}

\href{/section/opinion}{Opinion}\textbar{}No Exit: Refugees Trapped in a
Squalid Greek Camp

\url{https://nyti.ms/2P9C2AX}

\begin{itemize}
\item
\item
\item
\item
\item
\end{itemize}

Advertisement

\protect\hyperlink{after-top}{Continue reading the main story}

\href{/section/opinion}{Opinion}

Supported by

\protect\hyperlink{after-sponsor}{Continue reading the main story}

\hypertarget{no-exit-refugees-trapped-in-a-squalid-greek-camp}{%
\section{No Exit: Refugees Trapped in a Squalid Greek
Camp}\label{no-exit-refugees-trapped-in-a-squalid-greek-camp}}

Camp Moria, on the Greek island of Lesbos, is a visible reminder of
Europe's hardening stance toward migrants.

By
\href{https://www.nytimes.com/interactive/opinion/editorialboard.html}{The
Editorial Board}

The editorial board represents the opinions of the board, its editor and
the publisher. It is separate from the newsroom and the Op-Ed section.

\begin{itemize}
\item
  Oct. 3, 2018
\item
  \begin{itemize}
  \item
  \item
  \item
  \item
  \item
  \end{itemize}
\end{itemize}

\includegraphics{https://static01.nyt.com/images/2018/10/03/opinion/03greece-EDT/merlin_142198440_72a45029-28f2-4aa3-8c6c-41220736eb58-articleLarge.jpg?quality=75\&auto=webp\&disable=upscale}

It may seem paradoxical that while the number of migrants arriving in
Europe has fallen by 90 percent from its 2015 peak, the refugee camp on
the Greek island of Lesbos has grown into an unspeakable hell, where
asylum seekers are driven to madness and suicide. Sadly,
\href{https://www.nytimes.com/2018/10/02/world/europe/greece-lesbos-moria-refugees.html}{the
horror of Camp Moria}described by Patrick Kingsley of The Times this
week is the price of the actions that have stemmed the flow of refugees.

When the flood of refugees was at its high point, Camp Moria was
basically a way station, one of the first stops for
\href{https://www.nytimes.com/2015/10/08/world/europe/refugee-migrant-crisis-asylum-seekers-germany.html}{asylum
seekers}, many fleeing war in Syria, Iraq and Afghanistan, on their way
to the European mainland. But as the European Union has
\href{https://www.nytimes.com/interactive/2018/06/27/world/europe/europe-migrant-crisis-change.html}{responded
to the crisis} by closing internal borders and cutting deals with Turkey
and African governments and warlords to slow the exodus, many of the
migrants have become stranded where they first make landfall.

Though only about 23,000 refugees have reached the Greek islands this
year, down from 850,000 in 2015, they must now wait at camps like Moria
for as long as two years before they are either sent back or sent on.

The squalor and dangers of the camp are unlikely to draw criticism from
President Trump, whose cynical efforts to curb legal and illegal
immigration were highlighted by the decision announced last month to
reduce next year's
\href{https://www.nytimes.com/2018/09/17/us/politics/trump-refugees-historic-cuts.html}{refugee
resettlement quota} to 30,000, the lowest ever. As recently as 2016, the
United States admitted 85,000 refugees.

Mr. Trump has also shown no scruples about his inhumane policy of
separating families. Struggling to find room for 13,000 detained migrant
children, whose numbers have increased more than fivefold since last
year, the administration has roused hundreds of children in the middle
of the night and bused them to a
\href{https://www.nytimes.com/2018/09/30/us/migrant-children-tent-city-texas.html}{sprawling
tent city} in the West Texas desert.

At Camp Moria in Greece, conditions are grim. Mr. Kingsley describes a
camp built for 3,100 now overflowing with three times that many. There's
one cold shower for each 80 people, one foul toilet per 70; people stand
in line all day for food and wait months for an interview; gangs prey on
the weak; sexual assaults are common; and suicide attempts are constant.
With no room for more refugees, the camp has overflowed into rough
encampments beyond its fences. Greece has now begun moving some of the
most vulnerable refugees to Athens, but that is not likely to make a big
difference soon.

All that raises a lot of questions.
\href{https://ec.europa.eu/home-affairs/sites/homeaffairs/files/what-we-do/policies/european-agenda-migration/20180404-managing-migration-eu-financial-support-to-greece_en.pdf}{The
European Union} has allocated nearly 1.62 billion euros for the Greek
asylum effort, most of which has been paid out piecemeal to 20
governmental and nongovernmental organizations. Mr. Kingsley reported
suspicions in some quarters that the failure to improve the camp was
being used as a way to deter future migration to Greece, somewhat like
the Trump administration's family separation policy.

That is not official policy. Yet what is clear is that Europe should be
taking advantage of the relative lull in migration at least to improve
the conditions at the camps and accelerate the processing of asylum
seekers.

Despite the calm, the perception of a continuing migrant ``crisis''
continues to roil European politics and provide fuel for demagogues. A
\href{https://www.nytimes.com/2018/08/02/opinion/editorials/trump-conte-italy-immigration.html}{new
populist government in Italy} famously turned away a shipload of
\href{https://www.nytimes.com/2018/06/11/world/europe/italy-migrant-boat-aquarius.html}{refugees
in June}; hard-line Central and East European countries flatly oppose
taking in any refugees; and in Germany, which took in more than a
million migrants in 2015, a far-right party has made gains.

At a marathon summit meeting in Brussels in June, European leaders
managed to paper over some of their most gaping differences with an
agreement to set up screening centers outside Europe for asylum seekers
and to distribute refugees picked up at sea around the bloc for
processing. That might help politicians convince voters they're doing
something about the ``crisis,'' but so far it has done nothing for Camp
Moria. It stands, as Mr. Kingsley wrote, as the most visible symbol of
the steep moral and humanitarian cost at which a hardened Europe has
reduced the flow of migrants.

\emph{Follow The New York Times Opinion section on}
\href{https://www.facebook.com/nytopinion}{\emph{Facebook}} \emph{and}
\href{http://twitter.com/NYTOpinion}{\emph{Twitter
(@NYTOpinion)}}\emph{.}

Advertisement

\protect\hyperlink{after-bottom}{Continue reading the main story}

\hypertarget{site-index}{%
\subsection{Site Index}\label{site-index}}

\hypertarget{site-information-navigation}{%
\subsection{Site Information
Navigation}\label{site-information-navigation}}

\begin{itemize}
\tightlist
\item
  \href{https://help.nytimes.com/hc/en-us/articles/115014792127-Copyright-notice}{©~2020~The
  New York Times Company}
\end{itemize}

\begin{itemize}
\tightlist
\item
  \href{https://www.nytco.com/}{NYTCo}
\item
  \href{https://help.nytimes.com/hc/en-us/articles/115015385887-Contact-Us}{Contact
  Us}
\item
  \href{https://www.nytco.com/careers/}{Work with us}
\item
  \href{https://nytmediakit.com/}{Advertise}
\item
  \href{http://www.tbrandstudio.com/}{T Brand Studio}
\item
  \href{https://www.nytimes.com/privacy/cookie-policy\#how-do-i-manage-trackers}{Your
  Ad Choices}
\item
  \href{https://www.nytimes.com/privacy}{Privacy}
\item
  \href{https://help.nytimes.com/hc/en-us/articles/115014893428-Terms-of-service}{Terms
  of Service}
\item
  \href{https://help.nytimes.com/hc/en-us/articles/115014893968-Terms-of-sale}{Terms
  of Sale}
\item
  \href{https://spiderbites.nytimes.com}{Site Map}
\item
  \href{https://help.nytimes.com/hc/en-us}{Help}
\item
  \href{https://www.nytimes.com/subscription?campaignId=37WXW}{Subscriptions}
\end{itemize}
