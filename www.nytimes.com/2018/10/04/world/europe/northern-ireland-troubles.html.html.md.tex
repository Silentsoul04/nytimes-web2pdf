Sections

SEARCH

\protect\hyperlink{site-content}{Skip to
content}\protect\hyperlink{site-index}{Skip to site index}

\href{https://www.nytimes.com/section/world/europe}{Europe}

\href{https://myaccount.nytimes.com/auth/login?response_type=cookie\&client_id=vi}{}

\href{https://www.nytimes.com/section/todayspaper}{Today's Paper}

\href{/section/world/europe}{Europe}\textbar{}50 Years Later, Troubles
Still Cast `Huge Shadow' Over Northern Ireland

\url{https://nyti.ms/2Cskw88}

\begin{itemize}
\item
\item
\item
\item
\item
\item
\end{itemize}

Advertisement

\protect\hyperlink{after-top}{Continue reading the main story}

Supported by

\protect\hyperlink{after-sponsor}{Continue reading the main story}

\hypertarget{50-years-later-troubles-still-cast-huge-shadow-over-northern-ireland}{%
\section{50 Years Later, Troubles Still Cast `Huge Shadow' Over Northern
Ireland}\label{50-years-later-troubles-still-cast-huge-shadow-over-northern-ireland}}

By \href{https://www.nytimes.com/by/alan-cowell}{Alan Cowell}

\begin{itemize}
\item
  Oct. 4, 2018
\item
  \begin{itemize}
  \item
  \item
  \item
  \item
  \item
  \item
  \end{itemize}
\end{itemize}

\includegraphics{https://static01.nyt.com/images/2018/10/05/world/05nireland-1-print/merlin_144751068_6518dcf1-6a6b-4424-b7cd-12353a6718c3-articleLarge.jpg?quality=75\&auto=webp\&disable=upscale}

LONDON --- It had been billed as a civil rights march to redress
long-festering hurts, one among many that freckled Europe in the heady
days a half-century ago when the streets from Paris to
\href{https://www.nytimes.com/2018/08/20/world/europe/prague-spring-communism.html}{Prague}
became arenas of revolt.

But that particular protest in Northern Ireland on Oct. 5, 1968,
signaled the beginning of something that endured for three decades,
seeding an insurgency that became known with weary understatement as the
Troubles.

From then until a settlement known as
\href{https://www.nytimes.com/1998/04/11/world/irish-accord-overview-irish-talks-produce-accord-stop-decades-bloodshed-with.html}{the
Good Friday Agreement} was signed in 1998, some 3,600 people died in
conflict that had all the appearances of civil war, with roadblocks and
bomb blasts, sniper fire and the suspension of civil rights.

The British authorities deployed the army against their own citizens in
a province that had been carved out as a Protestant enclave at the
partition of Ireland in 1921. Protest drawing on centuries of
disaffection turned to armed revolt spearheaded by the underground Irish
Republican Army and its political wing, Sinn Fein, which cast themselves
as the most radical champions of an aggrieved Roman Catholic minority.

At the same time, loyalist paramilitary groups challenged the I.R.A.,
supposedly to protect a Protestant majority, fearful that any dilution
of the bond with Britain might destroy its power and identity. Their
activities injected one more element of violence into a war of many
dimensions.

Image

The scene of an I.R.A. bombing in Belfast in 1972.Credit...Abbas/Magnum
Photos

Image

Street fighting in Londonderry in 1971.Credit...Bruno Barbey/Magnum
Photos

Image

Masked escorts firing a final salute to Patsy O'Hara, who died in prison
in 1981 after 61 days on hunger strike.Credit...Gilles Peress/Magnum
Photos

The conflict was not confined to the six counties that make up Northern
Ireland. The I.R.A. drew significant support from groups as disparate as
Irish-Americans in the United States and the Libyan dictator Muammar
el-Qaddafi, who supplied significant amounts of arms and powerful
explosives.

The bombings spread to the rest of Britain, targeting senior figures
including Prime Minister Margaret Thatcher. Mortars were fired at 10
Downing Street, the prime minister's official residence and office, and
at Heathrow Airport outside London. British troops hunted down I.R.A.
members as far afield as Gibraltar.

Even today, 20 years after the Good Friday Agreement brought a form of
peace, low-level violence persists. Quasi-tribal
\href{https://www.nytimes.com/2000/08/12/world/belfast-journal-murals-of-troubles-draw-passions-and-tourists.html}{divisions
are preserved in huge murals} on the gable ends of rowhouses, depicting
each side's heroes. A shared executive authority, set up as part of the
1998 accord,
\href{https://www.nytimes.com/2017/11/20/world/europe/northern-ireland-stormont-adams.html}{has
been suspended} since January 2017, because of intractable disputes
between the main players --- largely Protestant unionists seeking
continued ties to Britain and mainly Catholic nationalists pressing for
a united Ireland free of British control.

The Good Friday pact ``cooled things down a bit,'' said Paul Bew, a
leading historian and emeritus professor at Queen's University in
Belfast. ``But if you are talking about a shared view of history, in
therapy terms it's like an agreement between a husband and wife who
still can't stand each other but have to find a way to live together.''

Image

Irish youths confronting British soldiers in Londonderry in
1971.Credit...Gilles Peress/Magnum Photos

Image

A British soldier in Londonderry in 1969.Credit...Gilles Caron/Fondation
Gilles Caron

Image

A British soldier and a young boy in Belfast in 1971.Credit...Bruno
Barbey/Magnum Photos

Most ominously, the Northern Irish issue that preoccupied six British
prime ministers from Harold Wilson to Tony Blair has interposed itself
anew into the halting negotiations on
\href{https://www.nytimes.com/2017/08/05/world/europe/brexit-northern-ireland-ireland.html}{Britain's
withdrawal from the European Union}, expected to happen in just six
months.

At the time of the Good Friday pact, Britain and Ireland were both
members of the European Union, meaning that they could largely dismantle
\href{https://www.nytimes.com/2016/08/07/world/europe/a-question-lingers-on-the-irish-british-border-whats-next.html}{the
border between Ireland and Northern Ireland} in line with the bloc's
commitment to the free passage of goods, services and people among
member nations. But a chaotic British withdrawal could scuttle that
arrangement.

A so-called hard border would ``require infrastructure that will damage
economic and social ties along the border,'' said Edward Burke, an
international politics professor at Nottingham University in England who
has written a book on the British Army's campaign in Northern Ireland.
``All the artfully created foundations of the agreement will be
damaged.''

Such weighty considerations might have seemed remote on Oct. 5, 1968,
though the harbingers of deepening division and rival narratives were
already plain enough. Even the geography of the protest reflected the
schism: Unionists called the town where the march took place
Londonderry; nationalists called it Derry. Merely using the wrong term
in the wrong place would invite hostility in the battle of emblems and
perceptions that suffused and sustained the Troubles.

Image

Michael Stone, an Ulster loyalist, attacked an I.R.A. funeral service at
Milltown Cemetery in Belfast in 1988. Filmed by TV news crews, the
attack caused shock around the world.Credit...Chris
Steele-Perkins/Magnum Photos

\href{https://www.nytimes.com/2017/10/11/world/europe/belfast-catholics-protestants-cantrell-close.html}{Some
neighborhoods still remain segregated} by so-called peace walls as high
as 45 feet that keep mutually inimical communities apart.

``I think the Troubles cast a huge shadow today,'' said Susan McKay, an
author, journalist and documentary filmmaker from Londonderry. ``The
reality is that the areas from which a lot of the Troubles emanated ---
the poorest and most deprived parts of Northern Ireland --- are still
the poorest and most deprived parts of Northern Ireland. The children
and grandchildren of those who participated in the Troubles the most are
still scarred by them today.''

Fifty years ago, hundreds of nationalist protesters gathered on Duke
Street in Londonderry. Their demonstration, organized by the Northern
Ireland Civil Rights Association --- inspired in part by the civil
rights movement in the United States --- had been outlawed when unionist
opponents announced plans for a rival march. The organizers resolved to
protest anyhow, fired by a long-simmering discontent with what was
perceived as widespread discrimination.

Suddenly, the terms of battle shifted. Officers from the
Protestant-dominated police force --- the Royal Ulster Constabulary ---
surrounded the demonstrators with batons drawn, cutting off lines of
retreat. A water cannon sprayed the crowd.

One protester, Deirdre O'Doherty, told the BBC that she fled into a cafe
as ``police battered people left, right and center.'' One officer burst
in ``with a baton in his hand with the blood dripping off it,'' she
said. ``He was young. He looked vicious. I never saw a face with so much
hatred in my life.''

As the strife deepened, the British Army was deployed.

In time, as the Troubles burgeoned, so, too, did the competing versions
of what lay behind them. For many in Britain, who became stoically
inured to the threat of I.R.A. bombings, it was about suppressing
terrorism. For nationalists, it was a broader fight to throw off the
yoke of colonialism and foreign oppression.

Northern Ireland's heroes were often its martyrs. On Jan. 30, 1972,
thousands of marchers, most of them Catholics, took to the streets of
the Bogside district of Londonderry to display opposition to the new
policy of internment without trial. British soldiers opened fire,
killing 14 protesters, all of them Catholic.

Image

Rioters throwing stones at British troops in Londonderry in
1972.Credit...Gilles Peress/Magnum Photos

Image

Barney McGuigan was one of 14 unarmed demonstrators killed by British
soldiers on Jan. 30, 1972. The events became known as Bloody
Sunday.Credit...Gilles Peress/Magnum Photos

Image

Mourners at burials in the aftermath of Bloody Sunday.Credit...Gilles
Peress/Magnum Photos

The events became known as Bloody Sunday. An official British apology
did not come until 2010, when Prime Minister David Cameron described the
killings as
``\href{https://www.nytimes.com/2010/06/16/world/europe/16nireland.html}{both
unjustified and unjustifiable}.''

Like other turning points in the Troubles, and in the propaganda war
that was one of the era's most striking features, ``Bloody Sunday''
became woven into the republican narrative, offsetting accusations that
the I.R.A. was far more brutal in its tactics than the British Army.

The chronology of the Troubles offers a tally of bloody episodes leading
to yet more carnage in a murky underground war of spies, hit men,
summary executions and still unexplained disappearances.

In less than two weeks in March 1988, for instance, British Special
Forces operatives killed three I.R.A. members in Gibraltar. When their
funerals were held in Belfast's Milltown Cemetery, a lone extremist from
the loyalist side, Michael Stone, attacked the ceremony with pistols and
grenades, killing three mourners --- one of them an I.R.A. supporter ---
in front of camera crews, photographers and journalists covering the
burial. Three days later, I.R.A. operatives seized two nonuniformed
British Army corporals mistaken for loyalist gunmen at the funeral of
one of those killed in Milltown Cemetery. The soldiers were beaten and
shot to death.

Image

The I.R.A. detonated a powerful truck bomb on Bishopsgate, a major
thoroughfare in London's financial district, in 1993.Credit...Adam
Butler/Press Association, via Getty Images

Sometimes, the I.R.A. offered warnings of its intention to detonate
explosives in Britain. In 1993, the group told the police that it
planned to detonate
\href{https://www.nytimes.com/1993/04/25/world/1-dead-40-hurt-as-a-blast-rips-central-london.html}{a
bomb in London's financial district}, but the explosion killed a news
photographer and injured some 40 people.

But the campaign was not fought exclusively with bombs and bullets. In
1981, Bobby Sands, a jailed I.R.A. commander sentenced on firearms
charges, drew global attention to a hunger strike by inmates in response
to the withdrawal of their special status within the prison system.
Already, by virtue of a since-repealed law that permitted prisoners to
stand as electoral candidates, Mr. Sands had been voted into the British
Parliament.

Image

The funeral of Bobby Sands in Belfast in 1981.Credit...Yan Morvan/Hans
Lucas

Image

A demonstration in support of the hunger strikers in Belfast in
1981.Credit...Ian Berry/Magnum Photos

Image

Belfast in 1985.Credit...Stuart Franklin/Magnum Photos

After 66 days without food, he died at the age of 27.
\href{https://www.nytimes.com/1981/05/05/world/sands-dies-in-northern-ireland-jail-on-the-66th-day-of-hunger-strike.html}{His
death} drew broad international criticism of the British government for
its handling of the hunger strike.

But Mrs. Thatcher, the prime minister at the time, remained resolute.
``Mr. Sands was a convicted criminal,'' she told Parliament in London.
``He chose to take his own life. It was a choice that his organization
did not allow to many of its victims.'' Her remark was oddly prophetic.

In 1984, a long-delay time bomb in
\href{https://www.nytimes.com/1984/10/13/world/ira-says-it-set-bomb-that-ripped-thatcher-s-hotel.html}{a
hotel in Brighton, England}, exploded as Mrs. Thatcher, its principal
target, and many members of her Conservative Party elite were there for
an annual conference. Mrs. Thatcher escaped unhurt, but five people were
killed.

``Today we were unlucky,'' the I.R.A. said in a statement, ``but
remember we only have to be lucky once. You will have to be lucky
always. Give Ireland peace and there will be no more war.''

It was a reminder of the essentially asymmetric nature of a conflict
that pitted a NATO army against insurgents and irregulars fueled by
competing visions of destiny that have endured far beyond the formal
silencing of their weapons.

Decades later, the Troubles ``are so burned into our lives that they are
part of our DNA,'' said Monica McWilliams, a former civil rights
marcher, peace activist and feminist leader. ``They are with us every
day --- especially those of us who were bereaved. It's a festering sore,
because it's never been dealt with.''

Image

A peace march in Londonderry in 1976.Credit...Peter Marlow/Magnum Photos

Image

A summer evening in Woodvale, a Protestant neighborhood in Belfast, in
1989.Credit...Gilles Peress/Magnum Photos

Image

A 45-foot ``peace wall,'' erected by the British authorities, separating
Catholic neighborhoods, left, from Protestants in
Belfast.Credit...Abbas/Magnum Photos

Advertisement

\protect\hyperlink{after-bottom}{Continue reading the main story}

\hypertarget{site-index}{%
\subsection{Site Index}\label{site-index}}

\hypertarget{site-information-navigation}{%
\subsection{Site Information
Navigation}\label{site-information-navigation}}

\begin{itemize}
\tightlist
\item
  \href{https://help.nytimes.com/hc/en-us/articles/115014792127-Copyright-notice}{©~2020~The
  New York Times Company}
\end{itemize}

\begin{itemize}
\tightlist
\item
  \href{https://www.nytco.com/}{NYTCo}
\item
  \href{https://help.nytimes.com/hc/en-us/articles/115015385887-Contact-Us}{Contact
  Us}
\item
  \href{https://www.nytco.com/careers/}{Work with us}
\item
  \href{https://nytmediakit.com/}{Advertise}
\item
  \href{http://www.tbrandstudio.com/}{T Brand Studio}
\item
  \href{https://www.nytimes.com/privacy/cookie-policy\#how-do-i-manage-trackers}{Your
  Ad Choices}
\item
  \href{https://www.nytimes.com/privacy}{Privacy}
\item
  \href{https://help.nytimes.com/hc/en-us/articles/115014893428-Terms-of-service}{Terms
  of Service}
\item
  \href{https://help.nytimes.com/hc/en-us/articles/115014893968-Terms-of-sale}{Terms
  of Sale}
\item
  \href{https://spiderbites.nytimes.com}{Site Map}
\item
  \href{https://help.nytimes.com/hc/en-us}{Help}
\item
  \href{https://www.nytimes.com/subscription?campaignId=37WXW}{Subscriptions}
\end{itemize}
