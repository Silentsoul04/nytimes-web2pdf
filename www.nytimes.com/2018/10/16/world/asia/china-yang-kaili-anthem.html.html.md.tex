Sections

SEARCH

\protect\hyperlink{site-content}{Skip to
content}\protect\hyperlink{site-index}{Skip to site index}

\href{https://www.nytimes.com/section/world/asia}{Asia Pacific}

\href{https://myaccount.nytimes.com/auth/login?response_type=cookie\&client_id=vi}{}

\href{https://www.nytimes.com/section/todayspaper}{Today's Paper}

\href{/section/world/asia}{Asia Pacific}\textbar{}Chinese Internet Star
Detained for `Disrespectful' Version of National Anthem

\url{https://nyti.ms/2AcvrAH}

\begin{itemize}
\item
\item
\item
\item
\item
\end{itemize}

Advertisement

\protect\hyperlink{after-top}{Continue reading the main story}

Supported by

\protect\hyperlink{after-sponsor}{Continue reading the main story}

\hypertarget{chinese-internet-star-detained-for-disrespectful-version-of-national-anthem}{%
\section{Chinese Internet Star Detained for `Disrespectful' Version of
National
Anthem}\label{chinese-internet-star-detained-for-disrespectful-version-of-national-anthem}}

By \href{https://www.nytimes.com/by/amy-qin}{Amy Qin}

\begin{itemize}
\item
  Oct. 16, 2018
\item
  \begin{itemize}
  \item
  \item
  \item
  \item
  \item
  \end{itemize}
\end{itemize}

\href{https://cn.nytimes.com/china/20181017/china-yang-kaili-anthem/}{阅读简体中文版}\href{https://cn.nytimes.com/china/20181017/china-yang-kaili-anthem/zh-hant/}{閱讀繁體中文版}

HONG KONG --- Clad in a furry, antler-shaped headband and denim
overalls, the young woman smiled at the camera, waved her arms in the
air as if conducting an orchestra and belted out a garbled snippet of
``March of the Volunteers,'' the national anthem of China.

It seemed to be a lighthearted moment. But the police in China saw it
differently. The woman, Yang Kaili, 20, a Chinese live-streaming star
with tens of millions of followers, was detained for five days for
singing the national anthem in a ``disrespectful'' manner while
broadcasting live.

In a
\href{https://www.weibo.com/1589614077/GDMdawDuo?from=page_1001061589614077_profile\&wvr=6\&mod=weibotime}{statement}
posted over the weekend on the microblogging platform Weibo, the police
in the Jing'an district of Shanghai described Ms. Yang's behavior as
``an insult to the dignity of the national anthem which repelled
internet users.''

Yang Kaili, a Chinese live-streaming star, was detained for five days
for singing the national anthem in a ``disrespectful''
manner.Credit...CreditVideo by xia qiu

According to the police, Ms. Yang was detained under China's National
Anthem Law,
\href{https://www.nytimes.com/2017/11/04/world/asia/china-hong-kong-national-anthem.html}{implemented
last year}, which threatens up to three years of detention for people
who disrespect the anthem.

In previous live broadcasts, Ms. Yang could often be seen dancing and
singing along to saccharine pop songs. In a sign of the booming
popularity of China's live-streaming industry, last month Ms. Yang was
even invited to perform on a televised variety show on CCTV, the
state-run broadcaster.

Following the incident, the Chinese live-streaming platform Huya deleted
Ms. Yang's account and took her videos offline. And in what has become
something of a rite of passage for Chinese celebrities who fall afoul of
the law, Ms. Yang issued two self-critical statements on
\href{https://www.weibo.com/5195037706/GDaRg9k1k?from=page_1005055195037706_profile\&wvr=6\&mod=weibotime}{her
official Weibo account}, apologizing for her ``stupid, low-level
mistakes.''

``I sincerely apologize for the fact that I did not sing the national
anthem in the live broadcast in a serious manner,'' she wrote. ``My
behavior deeply hurt everyone's feelings. Sorry. Sorry to the
motherland, sorry to my fans, sorry to everyone online, sorry to the
platform.''

Ms. Yang said she would ``stop all live broadcasting work, carry out
self-reformation, painstakingly deepen reflection, fully accept
ideological politics and patriotic education and diligently study the
relevant laws and regulations of the National Anthem Law of the People's
Republic of China.'' Ms. Yang said she would also watch patriotic
propaganda films and participate in volunteer activities.

Earlier this month, Fan Bingbing, China's most famous actress, also
apologized on Weibo after
\href{https://www.nytimes.com/2018/10/02/world/asia/fan-bingbing-tax-evasion-china.html}{being
fined nearly \$70 million} in unpaid taxes and penalties. In her
statement, she said she owed her success to the policies of the ruling
Communist Party.

The Yang incident comes at a time when the Chinese government, under
President Xi Jinping, has stepped up demands for patriotic devotion even
in areas of life that are typically seen as apolitical. Last year, for
example, a large number of Chinese celebrity gossip blogs were
\href{https://www.nytimes.com/2017/06/09/world/asia/china-celebrity-news-wechat.html}{taken
down} as part of the Communist Party's push to spread ``positive
energy'' and ``socialist core values.''

In mainland China, where nationalistic sentiment runs high and space for
public debate is shrinking, there has been little discussion of the
National Anthem Law since its implementation.

By contrast, conduct surrounding the national anthem has been a subject
of heated debate in Hong Kong, a semiautonomous Chinese city with broad
protections for freedom of speech. Amid discontent with what many
residents see as Beijing's growing influence in the former British
colony, some spectators have taken to
\href{https://www.nytimes.com/2017/10/11/world/asia/hong-kong-china-national-anthem-protest.html}{booing
and catcalling the anthem} when it is played before soccer matches in
Hong Kong.

Last year, the National People's Congress in Beijing, China's
rubber-stamp legislature, demanded that Hong Kong adopt the National
Anthem Law. But the law has yet to be enacted in Hong Kong, which
retains a degree of legal autonomy from the mainland under a framework
known as ``one country, two systems.'' As a result, spectators have
continued to boo the anthem without being punished, most recently at a
\href{https://www.scmp.com/sport/hong-kong/article/2168186/hong-kong-beaten-thailand-bumpy-start-gary-whites-tenure-national}{soccer
match between Hong Kong and Thailand} last week.

Advertisement

\protect\hyperlink{after-bottom}{Continue reading the main story}

\hypertarget{site-index}{%
\subsection{Site Index}\label{site-index}}

\hypertarget{site-information-navigation}{%
\subsection{Site Information
Navigation}\label{site-information-navigation}}

\begin{itemize}
\tightlist
\item
  \href{https://help.nytimes.com/hc/en-us/articles/115014792127-Copyright-notice}{©~2020~The
  New York Times Company}
\end{itemize}

\begin{itemize}
\tightlist
\item
  \href{https://www.nytco.com/}{NYTCo}
\item
  \href{https://help.nytimes.com/hc/en-us/articles/115015385887-Contact-Us}{Contact
  Us}
\item
  \href{https://www.nytco.com/careers/}{Work with us}
\item
  \href{https://nytmediakit.com/}{Advertise}
\item
  \href{http://www.tbrandstudio.com/}{T Brand Studio}
\item
  \href{https://www.nytimes.com/privacy/cookie-policy\#how-do-i-manage-trackers}{Your
  Ad Choices}
\item
  \href{https://www.nytimes.com/privacy}{Privacy}
\item
  \href{https://help.nytimes.com/hc/en-us/articles/115014893428-Terms-of-service}{Terms
  of Service}
\item
  \href{https://help.nytimes.com/hc/en-us/articles/115014893968-Terms-of-sale}{Terms
  of Sale}
\item
  \href{https://spiderbites.nytimes.com}{Site Map}
\item
  \href{https://help.nytimes.com/hc/en-us}{Help}
\item
  \href{https://www.nytimes.com/subscription?campaignId=37WXW}{Subscriptions}
\end{itemize}
