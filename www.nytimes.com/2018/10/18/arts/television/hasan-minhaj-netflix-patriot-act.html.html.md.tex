Sections

SEARCH

\protect\hyperlink{site-content}{Skip to
content}\protect\hyperlink{site-index}{Skip to site index}

\href{/section/arts/television}{Television}\textbar{}Can Hasan Minhaj
Make Topical Comedy Work on Netflix?

\href{https://nyti.ms/2Ag69By}{https://nyti.ms/2Ag69By}

\begin{itemize}
\item
\item
\item
\item
\item
\item
\end{itemize}

\includegraphics{https://static01.nyt.com/images/2018/10/21/arts/21late-night-streaming2/merlin_145337208_f17746b9-c6c9-4e19-a793-907655ef8edb-articleLarge.jpg?quality=75\&auto=webp\&disable=upscale}

\hypertarget{can-hasan-minhaj-make-topical-comedy-work-on-netflix}{%
\section{Can Hasan Minhaj Make Topical Comedy Work on
Netflix?}\label{can-hasan-minhaj-make-topical-comedy-work-on-netflix}}

With ``Patriot Act,'' the former ``Daily Show'' correspondent plans to
cover news most late-night hosts ignore, and looks to defy Netflix's
spotty talk show track record.

Hasan Minhaj on the set of ``Patriot Act,'' his new weekly satirical
news show debuting Oct. 28 on Netflix.Credit...Bryan Derballa for The
New York Times

Supported by

\protect\hyperlink{after-sponsor}{Continue reading the main story}

By \href{https://www.nytimes.com/by/dave-itzkoff}{Dave Itzkoff}

\begin{itemize}
\item
  Oct. 18, 2018
\item
  \begin{itemize}
  \item
  \item
  \item
  \item
  \item
  \item
  \end{itemize}
\end{itemize}

Standing on the set of his new Netflix series,
\href{https://www.youtube.com/watch?v=g_6d77TABBE}{``Patriot Act,''} one
evening earlier this month,
\href{https://www.nytimes.com/2017/06/21/magazine/hasan-minhaj-thinks-comedy-is-for-weirdos.html}{Hasan
Minhaj} asked his studio audience if they had any questions about what
they were about to see. He knew that his stage, an immense digital
screen encircling the diamond-shape platform he was standing on, was a
bit of a technological monstrosity --- ``it's like if Michael Bay
directed a PowerPoint presentation,'' he joked to the crowd --- and so
some clarification might be required.

Sure enough, someone asked: ``What \emph{is} this?''

Minhaj, 33, a lean, energetic stand-up and
\href{https://www.youtube.com/watch?v=toUjuVlHi1Q}{a recent alumnus of
``The Daily Show,''} explained that ``Patriot Act'' (whose first two
episodes will be released on Oct. 28) was a project he had been
developing for more than two years.

Before the success of
\href{https://tv.avclub.com/the-daily-show-s-hasan-minhaj-crafts-a-hilarious-spell-1798191421}{his
stand-up special ``Homecoming King''} and his incisive turn as
\href{https://www.youtube.com/watch?v=Z7oG74nHSTQ}{host of the 2017
White House Correspondents Dinner}, he said he'd already been thinking
about applying his comedic style to news stories that weren't
necessarily at the center of everyone's attention, in a format that
didn't look like another cookie-cutter late-night comedy.

The test show that Minhaj was about to perform --- a 24-minute monologue
about the role of Asian-Americans in reshaping affirmative action, and a
10-minute piece about digital security in Estonia --- could very well
end up looking like a ``woke TED Talk,'' he said.

Good or bad, it was the show he always wanted to make and ``I'm going to
give you everything I have,'' Minhaj said.

He added, ``Culturally, for us, I think we need something like this.''

``We'' here could mean the racially diverse group that had come to see
Minhaj, who often talks in his act about his identity as a Muslim and a
child of Indian immigrant parents. It could refer to the demographic of
viewers in their 20s and 30s that Netflix would love to see him bring to
the streaming service. Or it might be anyone who has tired of ``Daily
Show'' clones and is eager for anything even slightly different.

But even if Minhaj and his colleagues have cracked the code and created
a genuinely new kind of topical comedy, is there an audience for it? And
is Netflix the place where it belongs?

If the post-Jon Stewart era of television once looked like a potential
paradise for any host with a political perspective and a few zingers
about the Trump administration, it is now a battlefield littered with
casualties.

While hosts with established identities --- sharp wits like Stephen
Colbert and Samantha Bee, or John Oliver and Seth Meyers, known for
their long, researched takedowns --- have become increasingly
entrenched, newer entrants have stumbled. In two years, Comedy Central
has canceled two 11:30 p.m. programs intended as companions for ``The
Daily Show'':
\href{https://www.nytimes.com/2016/08/16/business/media/comedy-central-cancels-larry-wilmores-late-night-show.html}{``The
Nightly Show With Larry Wilmore''} and
\href{https://www.nytimes.com/2018/06/15/arts/television/comedy-central-cancels-the-opposition-with-jordan-klepper.html}{``The
Opposition With Jordan Klepper.''} BET gave only one season to its
late-night series
\href{https://www.nytimes.com/2017/10/08/arts/television/on-the-rundown-robin-thede-is-filling-a-void-in-late-night-talk.html}{``The
Rundown With Robin Thede.''}

Netflix, despite its rapid expansion in other traditional TV categories,
has struggled to create this kind of appointment viewing. Last year
\href{https://www.nytimes.com/2017/10/19/arts/television/chelsea-handler-netflix-talk-show.html}{it
canceled} its first high-profile attempt at a topical talk show,
``Chelsea,'' hosted by Chelsea Handler, and this past August,
\href{https://variety.com/2018/tv/news/michelle-wolf-joel-mchale-tnetflix-1202909399/}{it
lowered the boom} on two weekly programs, ``The Joel McHale Show With
Joel McHale,'' which debuted in February, and ``The Break With Michelle
Wolf,'' which started in May.

\includegraphics{https://static01.nyt.com/images/2018/10/21/arts/21late-night-streaming3/merlin_145337028_9e98ea23-5685-48ac-9c5d-74c7b53a84da-articleLarge.jpg?quality=75\&auto=webp\&disable=upscale}

Maybe there's no way for topical comedy to succeed on streaming TV and
maybe Minhaj is doomed before he starts. But Minhaj doesn't see it this
way.

``If we get this wrong, well, it's what everyone predicted,'' he told me
a few days before the test show. ``But if we get this right?'' His eyes
widened and a grin spread across his face. ``It feels like `Rocky' 1 all
over again.''

On an earlier morning in September, Minhaj and about 20 of his ``Patriot
Act'' colleagues were in the bowels of their midtown Manhattan office,
gathered around an enormous TV (and a basket of croissants) for what
they called a ``pre-viz'' meeting.

The monitor showed a computer rendering of the ``Patriot Act'' stage, on
which stood a small digital silhouette of Minhaj. The flesh-and-blood
comedian was sitting on a couch, dressed in athleisure clothes and a
pair of Air Jordans as he read from the script for his affirmative
action monologue.

At a breakneck clip, he narrated the story of Edward Blum, the
conservative activist and president of Students for Fair Admissions,
\href{https://www.nytimes.com/2018/08/30/us/politics/asian-students-affirmative-action-harvard.html}{which
is suing} Harvard University for allegedly discriminating against
Asian-American applicants.

On screen, bar graphs rose and fell like roller coasters and pie charts
exploded into existence while Minhaj recited admissions figures for
elite colleges. (Noting that Caltech had nine black students in 2012, he
quipped, ``There are more black people in the Wu-Tang Clan.'')

Amid the deluge of data and punch lines, Minhaj was also weaving a
personal story: one of growing up a proud first-generation American in
Davis, Calif., while navigating a murky ecosystem of race and class.

In his college-prep classes, Minhaj said, he was told not to declare
himself an Asian on his application forms or he'd risk the penalty of a
possible racial quota.

``I thought I wasn't going to get into Stanford because some black kid
was going to take my spot,'' he said in the monologue. ``But I didn't
get into Stanford because I was dumb.'' (This is his modest way of
saying he cracked 1300 on his SAT exam.)

On paper, Minhaj is very much a grown-up: a husband of three years to
his wife, Beena, a management consultant, and father to their daughter,
who was born in March. But in person, he has a childlike buoyancy, kept
aloft by his lifelong loves of hip-hop and professional basketball and
his occasional tendency to talk like an internet meme come to life.
He'll say aloud a phrase like ``tools→clear history'' when he means he's
trying to put something out of his mind.

In late 2014, he was hired as a ``Daily Show'' correspondent. ``He was
just undeniable,'' Jon Stewart said of him. ``I can teach the false-news
correspondent mechanics, but not the singularity of someone's talent.
When you get somebody like that, who's a great storyteller,
introspective and humble, you just go: `O.K. We're done. He's good.'''

Image

Recent talk show casualties include, clockwise from top right: Larry
Wilmore, Jordan Klepper, Robin Thede, Joel McHale, Michelle Wolf and
Chelsea Handler.Credit...Clockwise from top right: Comedy Central (2),
BET, Netflix (3)

Only a few months later, Stewart announced his departure from the
program. Minhaj said he couldn't forget his admired boss's explanation
for why he was leaving: ``Jon was like, `I've manipulated this chess
piece in every single way I could. There's no further place that I can
take it.''' The message to Minhaj was clear, even then, that he had to
start thinking about his own next moves.

When Trevor Noah took over at ``The Daily Show,'' Minhaj was seen as
something of a curiosity there. ``Hasan was intrinsically different from
all of the caricatures and archetypes of what `Daily Show'
correspondents had been,'' Noah told me. ``He wasn't snide, he wasn't
sarcastic --- he was just a different person.''

Minhaj appeared in recurring features like
\href{http://www.cc.com/video-clips/rtzt0s/the-daily-show-with-trevor-noah-brown-in-town---coal-country}{``Brown
in Town,''} where he reported on stories outside New York, and ``Hasan
the Record,'' in which he answered burning questions --- say,
\href{http://www.cc.com/video-clips/w7rowx/the-daily-show-with-trevor-noah-hasan-the-record---what-is-impeachment-}{``What
is impeachment?''} --- in pretaped segments edited at a pace apparently
intended to induce seizures.

His breakout opportunities arrived elsewhere. First was his one-man
show, ``Homecoming King,'' which he started performing live in 2014 and
released as a Netflix special last year. It is his exuberant recitation
of his origin story, of learning the ropes from his father (while his
mother studied at medical school in India) and confronting bias and
bigotry in America.

``Homecoming King'' also established a signature visual style for
Minhaj's stage show, full of vivid digital graphics and screencaps from
social media. It led to his invitation to host the White House
Correspondents Dinner at perhaps the worst possible moment, when
President Trump had already announced he would not attend, media
organizations were questioning whether the event should even go forward
and Minhaj said he knew the gig was ``radioactive.''

``How far down the totem pole do you have to go to be like, `Let's get
the second- or third-most popular correspondent on `The Daily Show'?'''
he said.

Minhaj knew that he'd been underestimated and he used this to his
advantage. He delivered a stirring routine that was less a taunting of
Trump officials than a reminder to the journalists watching of the
weighty responsibility facing them.

``In the age of Trump,'' he said in his speech, ``I know that you guys
have to be more perfect now more than ever. Because you are how the
president gets his news.''

He added: ``You can't make any mistakes. Because when one of you messes
up, he blames your entire group. And now you know what it feels like to
be a minority.''

Looking back on the experience over lunch at a Greek restaurant near his
office, Minhaj told me this was a pivotal career moment, not because he
got good reviews or because he resisted the advice of fellow comedians
who told him he had to (metaphorically) burn the room down.

What he learned that night --- though he'd already suspected as much ---
was that he wasn't a slinger of one-liners so much as a designer of
narratives. ``I want to be surgical,'' Minhaj said. ``I want to build to
a moment.''

Image

In ``pre-viz'' meetings, Minhaj, center, and his staff plan out the show
segments.Credit...Bryan Derballa for The New York Times

When Netflix pursued him, in the afterglow of the White House
Correspondents Dinner, to create a series for them, Minhaj had a very
clear sense of what he didn't want to do.

He took out his phone and showed me a series of photos from other
late-night shows --- all the hosts you'd expect --- seated at their
desks in identical poses with graphics placed in identical locations
above their shoulders.

His voice was uncharacteristically ferocious as he swiped through the
photos. ``It. Doesn't. Matter. The. Network. It all. Stays. The same. I
swear. To. God,'' Minhaj said.

If he didn't assert himself and find his own approach, he said, ``I was
going to be in a suit, behind a desk, in front of a fake city skyline,
and people would be, like, `Oh, it's Indian John Oliver.'''

In their international travels, Minhaj and his collaborators were seeing
``Homecoming King'' and his correspondents dinner performance connect
with a global viewership.

Prashanth Venkataramanujam, a fellow comedian and longtime friend who
worked with Minhaj on these routines, said they began to devise a plan
for tapping into this underrepresented audience.

``How do we talk about subjects that just don't normally make it into
the mainstream conversation?'' he said. ``We have the ability to find a
thesis and then work our way backwards from it.''

The news stories he should focus on, Minhaj said, were the ones in which
he felt some sense of personal investment. ``Not, do I have a take?'' he
said. ``But do I have the best take? I have no desire to be the 19th
hyena jumping on the carcass. Do I have something of value to add? Then
let's do it.''

Over a period of months, Minhaj workshopped some routines (like the one
about affirmative action) at the Fat Black Pussycat in Manhattan. He
spent his own money to produce a proof-of-concept video for ``Patriot
Act'' --- essentially, a rough pilot episode that he could show to
Netflix and other broadcasters.

It closely approximates how Minhaj intends to present ``Patriot Act''
now: with him always standing and in perpetual motion --- no desk to sit
behind, no chair to sit on --- surrounded at all times by graphics, data
and video.

Bela Bajaria, who is Netflix's vice president of content, said that when
she first spoke with Minhaj, he told her he wanted to wait until he had
a very clear sense of his vision. ``I didn't actually know what that
meant,'' Bajaria said. ``I figured he'd come back with this really great
pitch.'' When he showed her the proof-of-concept video, Bajaria said her
response was, ``Let's do \emph{that} show.''

Image

There are no desks or chairs on ``Patriot Act'' --- Minhaj will be in
perpetual motion, surrounded at all times by graphics, data and
video.Credit...Bryan Derballa for The New York Times

Netflix ordered 32 episodes of the series, which will be released in
cycles of six to eight weekly episodes. Even now, in the days before its
first episodes are released, Minhaj was still deciding whether one long
monologue was enough content for an entire show or if viewers would also
want another shorter, less substantive segment (a ``wine pairing'' to go
with their ``steak,'' as he put it).

Minhaj's comedy peers believe that he has as good a shot as anyone at
finding a new approach to this well-worn genre. But no one is in denial
about the challenges he faces, either.

``You never know what's going to hit and what's not going to,'' Jon
Stewart said, but when it came to Minhaj, ``I'd buy that raffle ticket
any day of the week.''

Though other recent shows with promising hosts had been short-lived,
Stewart said, ``I don't think it says anything about the talent of the
individuals. If you told me, `I'm going to let Jordan Klepper or Robin
Thede or Michelle Wolf do what they do,' I'd be like, `Yeah, that's a
smart choice.'''

He added, ``I hate to see things get pulled before they've fully
developed their voice and figured out who they are, and I thought they
all showed a worthiness to continue. But that's why I'm not a network
executive.''

Larry Wilmore, the longtime ``Daily Show'' correspondent and former
``Nightly Show'' host, said he wondered if the audiences for these
comedy shows had already made their choices and their viewing habits
were fixed in place.

``There is just a lot of that type of content out there,'' Wilmore said.
``There are already trusted outlets for it. They're home cooking for a
lot of people.''

In an oversaturated market, Wilmore said, audiences would be won not by
pioneering formats but memorable personalities, and he certainly
considered Minhaj to be such a performer.

``The people who are coming to his shows are doing it in exhilaration
--- `Finally, somebody like us is doing this,''' he said of Minhaj. ``I
think that's very meaningful.''

Robin Thede, the former ``Nightly Show'' head writer and host of ``The
Rundown,'' said that there was still ample room for a show that did not
focus on the day-in, day-out spectacle of the Trump presidency. ``I
think people are tired of hearing about Trump,'' she said. ``Whether
you're for Trump or against Trump, everyone's exhausted.''

Achieving diversity in the field is still a crucial goal, Thede said:
``At the end of the day, there's still at least three men who were born
with the name James who host late-night shows.''

Image

``Hasan was intrinsically different from all of the caricatures and
archetypes of what `Daily Show' correspondents had been,'' said Trevor
Noah, left.Credit...Comedy Central

But broadcasters also have to recognize the long-term commitments
required to make these programs viable.

\href{https://www.nytimes.com/2017/10/04/arts/television/sarah-silverman-wants-to-pop-your-bubble.html}{Sarah
Silverman}, whose Hulu series,
\href{https://www.youtube.com/watch?v=WDhhss6nuSY}{``I Love You,
America,''} is a rare topical comedy that is thriving on a streaming
platform, said that TV shows are getting ``less of a chance to develop
--- especially in a genre like this one that absolutely must have
at-bats to find itself and succeed.''

She added, ``Any talk show needs a couple years to find their voice but
a lot of places can't afford that luxury. It's all niche, really.''

To many viewers, the very idea of a comedy show drawn from news and
events happening right this moment seems to contradict the fundamental
proposition that streaming platforms are offering.

``The whole point is that they liberated us from appointments,'' Noah
said. ``It's like McDonald's saying, `Hey, would you like to wait 30
minutes for your food?'''

Netflix still aspires to have its own breakthrough topical comedy
series. ``We'd like to aspire to be best in class in programming, in
every category,'' Bajaria said. With a sardonic chuckle, she added,
``And we do like a challenge.''

She pointed to shows like Jerry Seinfeld's
\href{https://www.netflix.com/title/80171362}{``Comedians in Cars
Getting Coffee''} and David Letterman's
\href{https://www.netflix.com/title/80209096}{``My Next Guest Needs No
Introduction''} --- both hosted by monolithic entertainers --- as
examples where Netflix had innovated in the category.

But she acknowledged there were still challenges in selling Netflix
viewers on these shows. ``We've spent all this time saying, `Come at any
point to the platform, watch whenever you want,''' Bajaria said. ``In
this category, we're saying there's a timeliness and trying to drive
viewers on a weekly basis, which is different.''

The abrupt, 10-episode run that Netflix gave Michelle Wolf --- another
``Daily Show'' veteran with a recent,
\href{https://www.nytimes.com/2018/04/29/business/media/michelle-wolfs-routine-sets-off-a-furor-at-an-annual-washington-dinner.html}{much-discussed
turn at the dais} of the White House Correspondents Dinner --- would
seem like a bad omen for Minhaj.

But Bajaria said each series is its own proposition. In the case of
``The Break With Michelle Wolf,'' she said, ``There's a lot of factors
that we take into consideration, obviously, when we're not renewing a
series. It didn't find as large of an audience as we would have hoped.
She's very talented and we hope she felt really supported, making the
show she wanted to make.'' (Wolf declined to comment for this article.)

Netflix is hoping that Minhaj's ethnicity and his personal compass for
news stories will help him reach viewers in the other 190 or so
countries outside the United States where it is offered, and it plans to
promote the show heavily on YouTube and social media.

Image

Minhaj said he was determined that ``Patriot Act'' not look like ``this
open mic that Netflix is paying for as we figure it out.''Credit...Bryan
Derballa for The New York Times

The success of ``Patriot Act,'' Bajaria said, would be measured by its
viewership figures (which Netflix doesn't make public) as well as its
ability to insinuate itself into the zeitgeist (which no one has yet
figured out how to quantify).

``Do you have something to say?'' she said. ``Do you have a fresh way to
say this? We really think Hasan's P.O.V. and his take on things can cut
through.''

A short time after the ``pre-viz'' meeting, Minhaj was in his office at
``Patriot Act,'' his Air Jordans propped up on his desk near a Mike
Bibby bobblehead. On a dry-erase board in front of him was a rough
outline for an episode he hoped to perform about Saudi Arabia and its
crown prince, Mohammed bin Salman. Attached to a corkboard behind him
were notecards with other topics for future exploration: ``Amazon and
antitrust''; ``Stand Your Ground''; ``Myanmar/Rohingya.''

Minhaj said he was especially excited for a possible episode exploring
the Carlyle Group, the private equity firm, and
\href{https://wwd.com/business-news/financial/carlyle-rumored-to-buy-supreme-stake-louis-vuitton-james-jebbia-11022057/}{its
\$500 million investment} in the Supreme streetwear brand; he called it
a story of ``hypebeasts and xenophobia.''

He had to make his choices shrewdly: every segment he commits to means
months of work for his news research, writing and graphics departments,
a staff of about 74 people total.

Minhaj said he was determined that ``Patriot Act'' not come out of the
gate looking like ``this open mic that Netflix is paying for as we
figure it out.'' He could still remember the day he learned that Wolf's
and McHale's shows were both canceled and how it ratcheted up the
pressure on his project.

As he recalled it, ``Prashanth walked into my office and he could see
that I was definitely stressed. He goes, `Remember, this is how you felt
during the correspondents dinner.' It's this massive question mark ---
what is going to happen?''

At his test show a few weeks later, in a rare moment when he wasn't
feverishly delivering his affirmative action monologue or guzzling down
gulps of water when the camera wasn't on him, Minhaj paused to reflect
on a joke he had told earlier.

He'd played a series of clips from interviews with Edward Blum, the
conservative activist, in which he variously claimed that Harvard had a
precise quota of 17, 19 or 15 percent for Asians in its incoming
freshman class.

After the videos, Minhaj mocked Blum, telling him he should have studied
harder at Kumon, the boot-camp education franchise that is especially
popular with Asian immigrant families.

The line had got some laughter, and Minhaj was proud of that. He told
the audience, ``All the writers were like, `I don't know if the Kumon
joke is going to hit. It's, like, so nerdy.' I'm like, `Trust me ---
it's going to hit.'''

And he was right. It was just one choice, but it was his choice and he
was happy with it.

Advertisement

\protect\hyperlink{after-bottom}{Continue reading the main story}

\hypertarget{site-index}{%
\subsection{Site Index}\label{site-index}}

\hypertarget{site-information-navigation}{%
\subsection{Site Information
Navigation}\label{site-information-navigation}}

\begin{itemize}
\tightlist
\item
  \href{https://help.nytimes.com/hc/en-us/articles/115014792127-Copyright-notice}{©~2020~The
  New York Times Company}
\end{itemize}

\begin{itemize}
\tightlist
\item
  \href{https://www.nytco.com/}{NYTCo}
\item
  \href{https://help.nytimes.com/hc/en-us/articles/115015385887-Contact-Us}{Contact
  Us}
\item
  \href{https://www.nytco.com/careers/}{Work with us}
\item
  \href{https://nytmediakit.com/}{Advertise}
\item
  \href{http://www.tbrandstudio.com/}{T Brand Studio}
\item
  \href{https://www.nytimes.com/privacy/cookie-policy\#how-do-i-manage-trackers}{Your
  Ad Choices}
\item
  \href{https://www.nytimes.com/privacy}{Privacy}
\item
  \href{https://help.nytimes.com/hc/en-us/articles/115014893428-Terms-of-service}{Terms
  of Service}
\item
  \href{https://help.nytimes.com/hc/en-us/articles/115014893968-Terms-of-sale}{Terms
  of Sale}
\item
  \href{https://spiderbites.nytimes.com}{Site Map}
\item
  \href{https://help.nytimes.com/hc/en-us}{Help}
\item
  \href{https://www.nytimes.com/subscription?campaignId=37WXW}{Subscriptions}
\end{itemize}
