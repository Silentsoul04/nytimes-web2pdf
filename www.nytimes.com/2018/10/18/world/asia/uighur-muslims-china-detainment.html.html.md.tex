Sections

SEARCH

\protect\hyperlink{site-content}{Skip to
content}\protect\hyperlink{site-index}{Skip to site index}

\href{https://www.nytimes.com/section/world/asia}{Asia Pacific}

\href{https://myaccount.nytimes.com/auth/login?response_type=cookie\&client_id=vi}{}

\href{https://www.nytimes.com/section/todayspaper}{Today's Paper}

\href{/section/world/asia}{Asia Pacific}\textbar{}Uighur Americans Speak
Against China's Internment Camps. Their Relatives Disappear.

\url{https://nyti.ms/2QY9f2E}

\begin{itemize}
\item
\item
\item
\item
\item
\end{itemize}

Advertisement

\protect\hyperlink{after-top}{Continue reading the main story}

Supported by

\protect\hyperlink{after-sponsor}{Continue reading the main story}

\hypertarget{uighur-americans-speak-against-chinas-internment-camps-their-relatives-disappear}{%
\section{Uighur Americans Speak Against China's Internment Camps. Their
Relatives
Disappear.}\label{uighur-americans-speak-against-chinas-internment-camps-their-relatives-disappear}}

\includegraphics{https://static01.nyt.com/images/2018/10/19/us/politics/19dc-china1-print/19dc-china1-articleLarge.jpg?quality=75\&auto=webp\&disable=upscale}

By \href{https://www.nytimes.com/by/edward-wong}{Edward Wong}

\begin{itemize}
\item
  Oct. 18, 2018
\item
  \begin{itemize}
  \item
  \item
  \item
  \item
  \item
  \end{itemize}
\end{itemize}

\href{https://cn.nytimes.com/usa/20181019/uighur-muslims-china-detainment/}{阅读简体中文版}\href{https://cn.nytimes.com/usa/20181019/uighur-muslims-china-detainment/zh-hant/}{閱讀繁體中文版}

ROSSLYN, Va. --- Speaking last month at a Washington think tank, Rushan
Abbas relayed tales of suffering she had heard about China's repression
of
\href{https://www.nytimes.com/2009/07/12/weekinreview/12wong.html}{ethnic
Uighur Muslims} --- including the detention of members of her husband's
family in a
\href{https://jamestown.org/program/evidence-for-chinas-political-re-education-campaign-in-xinjiang/}{widespread
system of mass internment camps}.

Within six days, Ms. Abbas's ailing sister and 64-year-old aunt
disappeared from their homes in northwest China. No family members or
neighbors have heard from them in more than a month.

Ms. Abbas is an American citizen and Virginia resident; her sister has
two daughters, and both live in the United States. They all assume the
women are
\href{https://www.nytimes.com/2018/05/15/opinion/china-re-education-camps.html}{being
detained in the camps}, which Western analysts estimate hold up to one
million people.

Ms. Abbas said they had fallen victim to
\href{https://www.nytimes.com/2017/07/19/world/asia/dodging-chinese-police-in-kashgar-a-silk-road-oasis-town.html}{the
persecution} against which she had been campaigning --- and because of
her.

``I'm exercising my rights under the U.S. Constitution as an American
citizen,'' Ms. Abbas, a business consultant, said from her 12th-floor
office in Rosslyn, Va., overlooking the Key Bridge and Potomac River.
``They shouldn't punish my family members for this.''

``I hope the Chinese ambassador here reads this,'' she added, wiping
away tears. ``I will not stop. I will be everywhere and speak on this at
every event from now on.''

Ms. Abbas, 50, is among a growing number of Uighur Americans who have
had family members detained by the Chinese police and placed in the
\href{https://www.nytimes.com/2018/10/16/world/asia/china-muslim-camps-xinjiang-uighurs.html}{anti-Islam
camp system} that is spread
\href{https://supchina.com/2018/08/22/xinjiang-explainer-chinas-reeducation-camps-for-a-million-muslims}{across
the northwest region of Xinjiang}. Chinese officials describe the
internment as ``transformation through education'' and
\href{https://www.nytimes.com/2018/10/16/world/asia/china-muslim-camps-xinjiang-uighurs.html}{``vocational
education.''}

The Washington area has the largest population of Uighurs in the United
States, so stories like that of Ms. Abbas are now common here. Chinese
officers aim to silence Uighurs abroad by detaining their family
members.

\includegraphics{https://static01.nyt.com/images/2018/10/19/us/politics/19dc-china3-print/merlin_145476297_e71d5de4-078f-4f1f-b891-97231b201be2-articleLarge.jpg?quality=75\&auto=webp\&disable=upscale}

But that tactic is backfiring. Although some Uighurs abroad are afraid
to speak out for fear that relatives in Xinjiang will be detained, Ms.
Abbas said, there are ones like her who are more willing to voice their
outrage.

Those in Washington could sway United States policy toward China, at a
time when officials are debating a much tougher stand
\href{http://www.chinafile.com/conversation/how-should-world-respond-intensifying-repression-xinjiang}{on
defending Uighurs}. Some like Ms. Abbas have acquaintances at think
tanks, including at the conservative Hudson Institute, where she spoke
on Sept. 5, and in Congress and the White House. Ms. Abbas has also
spoken to staff members at the Congressional-Executive Commission on
China, which is led by Senator Marco Rubio, Republican of Florida, and
Representative Christopher H. Smith, Republican of New Jersey.

``Harassing the relatives of U.S. citizens is what Chairman Mao used to
call dropping a rock on your own feet,'' said
\href{https://www.hudson.org/experts/724-michael-pillsbury}{Michael
Pillsbury}, director for Chinese strategy at the Hudson Institute,
noting that repression of Uighurs would also erode relations between
China and Muslim nations.

This month, a daughter of Ms. Abbas's detained sister wrote to Mr. Rubio
about her mother's plight. The daughter, an American citizen, lives in
Florida, Mr. Rubio's home state. The other daughter, a legal permanent
resident, lives in Maryland. Their mother, Gulshan Abbas, 56, has severe
health problems.

Asked for comment about issues facing Uighur Americans, Mr. Rubio said,
``The long arm of the Chinese government's domestic repression directly
impacts the broader Uighur diaspora community, including in the United
States.''

``This is unacceptable, and it takes tremendous courage for these
individuals to even come forward given the growing number of reports of
Chinese government harassment, intimidation and threats aimed at the
Chinese, Uighur and Tibetan diaspora communities living in the United
States,'' Mr. Rubio added.

Mr. Rubio is pushing legislation to compel the United States to take
action on behalf of Uighurs. It says the F.B.I. and other government
agencies ``should track and take steps to hold accountable'' Chinese
officials who harass or threaten people from China who are American
citizens or living or studying here, including Uighurs.

Separately, officials at the White House and the State and Treasury
Departments are discussing
\href{https://www.nytimes.com/2018/09/10/world/asia/us-china-sanctions-muslim-camps.html}{imposing
economic sanctions on Chinese officials}, under the Global Magnitsky
Act, who are involved in repression of Uighurs.

Secretary of State Mike Pompeo has spoken about the plight of the
Uighurs and the harassment of Uighur Americans. In April, the State
Department's chief spokeswoman met with Gulchehra Hoja, a Uighur
American journalist for Radio Free Asia who said two dozen of her family
members had been detained in Xinjiang. Ms. Hoja
\href{https://www.youtube.com/watch?v=8awQJ2Xnd1U}{testified in July} at
the congressional commission.

Image

Ms. Abbas showed a photo of her family members, including her sister,
second from right, who recently went missing.Credit...Tom Brenner for
The New York Times

In a China policy speech this month, Vice President Mike Pence
\href{https://www.whitehouse.gov/briefings-statements/remarks-vice-president-pence-administrations-policy-toward-china/}{denounced
China's attempts to shape public opinion in the United States} through
coercion and other means.

\href{https://www.hrw.org/about/people/sophie-richardson}{Sophie
Richardson}, China director at Human Rights Watch, said, ``Beijing's
harassment now factors into whether citizens of countries like Australia
and the United States feel safe enough to attend public discussions
about Xinjiang at events ranging from congressional hearings in
Washington or think tank talks in Sydney.''

``Ending abuses in Xinjiang now depends in part on ensuring that these
communities are safe to exercise their rights around the world, and on
governments following Germany's and Sweden's lead and committing to not
sending Uighur asylum seekers back to China,'' she said.

Ferkat Jawdat, a Uighur and American citizen who lives in Chantilly,
Va., last spoke to his mother in February. She was forced to stay in
Xinjiang when he and his siblings came to the United States in 2011
because the Chinese authorities would not give her a passport. She told
him in February that she feared she was going to be put in a camp; Mr.
Jawdat has not been able to reach her since.

Representative Barbara Comstock, Republican of Virginia, pressed Mr.
Jawdat's case in an Oct. 3 letter to China's ambassador to the United
States, Cui Tiankai. It asked why Mr. Jawdat's missing mother,
Minaiwaier Tuersun, ``has been imprisoned, why the Chinese government
refused to issue her a passport in 2011, and when she will be
released.''

There has been no response from the Chinese embassy, Mr. Jawdat said.

The youngest of four children of a prominent biologist and a doctor, Ms.
Abbas grew up in Urumqi, the capital of Xinjiang, and attended a
university there. She has lived in the United States since May 1989,
when she came as a visiting scholar to Washington State University. She
got a master's degree in plant pathology there and became an American
citizen in 1995.

Ms. Abbas has been active in Uighur issues for decades. She joined Radio
Free Asia in Washington in 1998 as its first Uighur reporter before
moving to California. She worked as an interpreter for the Defense
Department when it
\href{https://www.nytimes.com/2014/01/01/us/us-frees-last-of-uighur-detainees-from-guantanamo.html}{detained
22 Uighurs in Guantánamo Bay}, then
\href{https://www.nytimes.com/2009/06/15/world/americas/15uighur.html}{helped
with their relocations} to other countries. She moved back to Washington
in 2009 to be an advocate for Uighurs.

She said she waited one month before speaking to a journalist about the
simultaneous disappearances of her sister and her aunt, Mayinur Abliz,
in the hopes that officials would release them. Now she sees a dark
future for them unless she speaks out.

She plans to mention them at a talk she is scheduled to give on Friday
at Indiana University.

``China needs to respect international laws,'' Ms. Abbas said. ``This is
so childish, what they're doing --- taking hostage the family members of
someone who left when she was 21.''

Advertisement

\protect\hyperlink{after-bottom}{Continue reading the main story}

\hypertarget{site-index}{%
\subsection{Site Index}\label{site-index}}

\hypertarget{site-information-navigation}{%
\subsection{Site Information
Navigation}\label{site-information-navigation}}

\begin{itemize}
\tightlist
\item
  \href{https://help.nytimes.com/hc/en-us/articles/115014792127-Copyright-notice}{©~2020~The
  New York Times Company}
\end{itemize}

\begin{itemize}
\tightlist
\item
  \href{https://www.nytco.com/}{NYTCo}
\item
  \href{https://help.nytimes.com/hc/en-us/articles/115015385887-Contact-Us}{Contact
  Us}
\item
  \href{https://www.nytco.com/careers/}{Work with us}
\item
  \href{https://nytmediakit.com/}{Advertise}
\item
  \href{http://www.tbrandstudio.com/}{T Brand Studio}
\item
  \href{https://www.nytimes.com/privacy/cookie-policy\#how-do-i-manage-trackers}{Your
  Ad Choices}
\item
  \href{https://www.nytimes.com/privacy}{Privacy}
\item
  \href{https://help.nytimes.com/hc/en-us/articles/115014893428-Terms-of-service}{Terms
  of Service}
\item
  \href{https://help.nytimes.com/hc/en-us/articles/115014893968-Terms-of-sale}{Terms
  of Sale}
\item
  \href{https://spiderbites.nytimes.com}{Site Map}
\item
  \href{https://help.nytimes.com/hc/en-us}{Help}
\item
  \href{https://www.nytimes.com/subscription?campaignId=37WXW}{Subscriptions}
\end{itemize}
