\href{/section/opinion}{Opinion}\textbar{}Dale Earnhardt Jr.: Bringing
Concussions Out of the Darkness

\href{https://nyti.ms/2ApKo2j}{https://nyti.ms/2ApKo2j}

\begin{itemize}
\item
\item
\item
\item
\item
\end{itemize}

\includegraphics{https://static01.nyt.com/images/2018/10/23/opinion/23earnhardt/merlin_144102582_8adbab97-d108-4fab-ae7d-a385a8fff340-articleLarge.jpg?quality=75\&auto=webp\&disable=upscale}

Sections

\protect\hyperlink{site-content}{Skip to
content}\protect\hyperlink{site-index}{Skip to site index}

\href{/section/opinion}{Opinion}

\hypertarget{dale-earnhardt-jr-bringing-concussions-out-of-the-darkness}{%
\section{Dale Earnhardt Jr.: Bringing Concussions Out of the
Darkness}\label{dale-earnhardt-jr-bringing-concussions-out-of-the-darkness}}

I used to hide my suffering. But I've learned that brain injuries don't
have to be permanent.

Dale Earnhardt Jr. at the Richmond Raceway in Virginia in
September.Credit...Brian Lawdermilk/Getty Images

Supported by

\protect\hyperlink{after-sponsor}{Continue reading the main story}

By Dale Earnhardt Jr.

Mr. Earnhardt is a former Nascar driver.

\begin{itemize}
\item
  Oct. 23, 2018
\item
  \begin{itemize}
  \item
  \item
  \item
  \item
  \item
  \end{itemize}
\end{itemize}

I never wanted to be a concussion expert. I know some of the world's
leading authorities on head injuries and I'm certainly not one of them,
but ``expert'' is a relative term. My expertise comes from personal
experience.

During my two decades behind the wheel as a full-time Nascar driver, I
suffered more than a dozen concussions. For a long time, I managed to
keep most of them a secret, but then my symptoms got too severe to keep
up the charade and I was forced to get help. My battle with head
injuries has given me a wealth of firsthand knowledge of the causes,
symptoms, and types of concussions, and their treatments.

Racers get every injury you can think of, from broken legs to cracked
collarbones. But it was concussions, not fractures, that forced me to
retire as a full-time Nascar driver in 2017. Twice I was pushed out of
the driver's seat because of concussion-related symptoms, missing two
major races in 2012 and an entire half-season in 2016.

During the four years in between I had other injuries too, but I kept
them hidden until doctors intervened and told me to get out of my car.
In the days following a race, I would often feel disoriented and
confused, detached from my body. Some people experience sharp headaches
or ringing in the ears when concussed. For me, my balance was off and my
mind felt swishy, lagging behind whatever my body was trying to do.

In 1998, at the Daytona 300, my Chevy was tossed into the air and
slammed down so hard on its nose that my helmet dented the steel roll
cage. Later that week when I was working inside a car at the shop, I
suddenly felt the car rolling. I sat up and realized it hadn't moved an
inch. I'd eventually find out my vestibular system --- the communication
lines between the brain, inner ear and body --- had been damaged.

But at the time, driven by a will to win and a hardheaded racing
tradition of never showing vulnerability, I concealed my suffering. I
would usually rally by the time the next race weekend came around.
Still, the stress chemicals produced by the anxiety of keeping my secret
worsened my condition. And as I got older, I needed longer and longer to
recover.

\includegraphics{https://static01.nyt.com/images/2018/10/23/opinion/23earnhardt-02/merlin_145686930_c88768ce-3a67-44dc-abed-483a1ff6fb21-articleLarge.jpg?quality=75\&auto=webp\&disable=upscale}

I persisted because it's what racecar drivers are supposed to do. You
tough it out. I also believed then what so many still do now: that a
concussion is permanent. I worried if I revealed how I really felt, my
peers on the racetrack would see me as damaged goods.

Those same myths and fears affect football players and construction
workers, kids playing youth sports and even people who get in the odd
car accident on their way to the office. But these myths lead us to make
uninformed decisions that harm our lives and livelihoods. A recent
Harris Poll commissioned by the doctors who treated me at the University
of Pittsburgh Medical Center's Sports Medicine Concussion Program found
that
\href{http://rethinkconcussions.upmc.com/wp-content/uploads/2015/09/harris-poll-report.pdf}{25
percent of parents} prevent their children from playing contact sports
because of concussion concerns.

I'm a parent now and I would never tell someone how to raise his child.
And I don't deny
\href{https://www.upmc.com/services/sports-medicine/services/concussion/facts-statistics}{the
estimate that there are 1.7 million to three million} sports-related
concussions a year.

However, I am sure that there is a middle ground, that we can encourage
our kids both to be active and competitive --- and to be safe. Research
shows that concussion risks can be reduced by playing smarter and using
the proper equipment.

When concussions do occur, it's important to remember that brain
injuries can be treated and healed like any other athletic injury ---
but only if the proper steps are taken, the right doctors are reached
and the prescribed treatment is followed through to the end.

That treatment is not easy. I'd never been a gym guy, but I learned how
to become one. My rehabilitation in 2016 was the hardest I have ever
worked. I wasn't told to sit in a dark room, the stereotypical treatment
for concussion. That's not how it works anymore. Instead, I was pushed
mentally and physically through fine motor skill tuning, exhausting
computer-based eye tests, and a lot of old-fashioned cardio. After
months of work I could feel my brain, eyes, ears and body communicating
properly again.

I also felt my life returning. The constant, dull feeling of fear
lifted. I was smiling again.

Now, I tell my story to let people know they don't have to silently walk
it off. I tell it to my racing friends who confess they've also been
suffering in secret and to many others who've never raced a lap. I've
given out the phone number to my doctor, Micky Collins at the University
of Pittsburgh, more times than I can count in the past few years. And
when those people reconnect later to tell me that Micky and his team
have given them their lives back, it feels like winning a race.

The advancements in brain science since my first major injury in 2012
are incredible. But all that science won't mean much if those of us who
are hurting don't come out of hiding and allow it to be put to use.

I don't blame my sport for my suffering. Neither do the other
professional athletes I know who love their sport every bit as much as I
love mine. And people hurt on the job performing other tasks are likely
just as passionate about what they do.

I will always wonder how many more races I could have won or how much
longer I could have raced if not for my stubbornness.

Don't make the mistakes I made. Help is out there. You just have to ask.

Dale Earnhardt Jr. is a television analyst for NBC Sports and former
Nascar driver.

\emph{Follow The New York Times Opinion section on}
\href{https://www.facebook.com/nytopinion}{\emph{Facebook}}\emph{,}
\href{http://twitter.com/NYTOpinion}{\emph{Twitter (@NYTopinion)}}
\emph{and}
\href{https://www.instagram.com/nytopinion/}{\emph{Instagram}}\emph{.}

Advertisement

\protect\hyperlink{after-bottom}{Continue reading the main story}

\hypertarget{site-index}{%
\subsection{Site Index}\label{site-index}}

\hypertarget{site-information-navigation}{%
\subsection{Site Information
Navigation}\label{site-information-navigation}}

\begin{itemize}
\tightlist
\item
  \href{https://help.nytimes.com/hc/en-us/articles/115014792127-Copyright-notice}{©~2020~The
  New York Times Company}
\end{itemize}

\begin{itemize}
\tightlist
\item
  \href{https://www.nytco.com/}{NYTCo}
\item
  \href{https://help.nytimes.com/hc/en-us/articles/115015385887-Contact-Us}{Contact
  Us}
\item
  \href{https://www.nytco.com/careers/}{Work with us}
\item
  \href{https://nytmediakit.com/}{Advertise}
\item
  \href{http://www.tbrandstudio.com/}{T Brand Studio}
\item
  \href{https://www.nytimes.com/privacy/cookie-policy\#how-do-i-manage-trackers}{Your
  Ad Choices}
\item
  \href{https://www.nytimes.com/privacy}{Privacy}
\item
  \href{https://help.nytimes.com/hc/en-us/articles/115014893428-Terms-of-service}{Terms
  of Service}
\item
  \href{https://help.nytimes.com/hc/en-us/articles/115014893968-Terms-of-sale}{Terms
  of Sale}
\item
  \href{https://spiderbites.nytimes.com}{Site Map}
\item
  \href{https://help.nytimes.com/hc/en-us}{Help}
\item
  \href{https://www.nytimes.com/subscription?campaignId=37WXW}{Subscriptions}
\end{itemize}
