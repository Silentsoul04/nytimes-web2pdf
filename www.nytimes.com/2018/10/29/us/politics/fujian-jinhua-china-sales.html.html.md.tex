Sections

SEARCH

\protect\hyperlink{site-content}{Skip to
content}\protect\hyperlink{site-index}{Skip to site index}

\href{https://www.nytimes.com/section/politics}{Politics}

\href{https://myaccount.nytimes.com/auth/login?response_type=cookie\&client_id=vi}{}

\href{https://www.nytimes.com/section/todayspaper}{Today's Paper}

\href{/section/politics}{Politics}\textbar{}U.S. to Block Sales to
Chinese Tech Company Over Security Concerns

\url{https://nyti.ms/2CO3bWy}

\begin{itemize}
\item
\item
\item
\item
\item
\end{itemize}

Advertisement

\protect\hyperlink{after-top}{Continue reading the main story}

Supported by

\protect\hyperlink{after-sponsor}{Continue reading the main story}

\hypertarget{us-to-block-sales-to-chinese-tech-company-over-security-concerns}{%
\section{U.S. to Block Sales to Chinese Tech Company Over Security
Concerns}\label{us-to-block-sales-to-chinese-tech-company-over-security-concerns}}

\includegraphics{https://static01.nyt.com/images/2018/10/30/business/30dc-jinhua/merlin_144059883_a5990dd2-78de-4500-a80b-bc72a2201971-articleLarge.jpg?quality=75\&auto=webp\&disable=upscale}

By \href{https://www.nytimes.com/by/alan-rappeport}{Alan Rappeport}

\begin{itemize}
\item
  Oct. 29, 2018
\item
  \begin{itemize}
  \item
  \item
  \item
  \item
  \item
  \end{itemize}
\end{itemize}

\href{https://cn.nytimes.com/usa/20181030/fujian-jinhua-china-sales/}{阅读简体中文版}\href{https://cn.nytimes.com/usa/20181030/fujian-jinhua-china-sales/zh-hant/}{閱讀繁體中文版}

WASHINGTON --- The United States said on Monday that it would block a
Chinese state-owned technology company from buying American components
because it posed a national security threat, the latest volley in an
\href{https://www.nytimes.com/2018/09/19/us/politics/trump-china-trade-war.html}{escalating
dispute} between the world's two largest economies.

The company, Fujian Jinhua Integrated Circuit, a manufacturer of
semiconductors, ``poses a significant risk'' of becoming involved in
activities that might infringe on national security, the
\href{https://www.commerce.gov/news/press-releases/2018/10/addition-fujian-jinhua-integrated-circuit-company-ltd-jinhua-entity-list}{Commerce
Department said}.

{[}\emph{Behind accusations that Fujian Jinhua}
\href{https://www.nytimes.com/2018/06/22/technology/china-micron-chips-theft.html}{\emph{was
stealing American technology}} \emph{to power China's future.}{]}

The move could cripple Jinhua, which relies on American components for
its semiconductors, and followed similar action taken by the Commerce
Department this year to
\href{https://www.nytimes.com/2018/04/16/technology/chinese-tech-company-blocked-from-buying-american-components.html?module=inline}{block
sales of components to ZTE}, a Chinese telecom company. The ZTE ban was
rescinded after President Trump ---
\href{https://www.nytimes.com/2018/05/13/business/trump-vows-to-save-jobs-at-chinas-zte-lost-after-us-sanctions.html}{responding
to a request} from President Xi Jinping of China in May --- asked the
department to lighten the penalty. ZTE agreed to pay a large fine,
reshuffle its leadership and undergo compliance monitoring by the United
States.

But relations between the United States and China have worsened since
then, and the Trump administration is taking an increasingly hard line
on transactions involving Chinese entities. It is eager to prevent
China's ascendance as an economic and technological powerhouse and has
begun aggressively scrutinizing foreign deals to prevent Beijing from
gaining access to valuable American intellectual property.

This month, the Treasury Department
\href{https://www.nytimes.com/2018/10/10/business/us-china-investment-cfius.html}{outlined
how it would use} new powers that allow the United States to review a
wider range of foreign transactions, including those in sensitive
industries like technology and telecommunications.

``When a foreign company engages in activity contrary to our national
security interests, we will take strong action to protect our national
security,'' said Wilbur Ross, the commerce secretary. ``Placing Jinhua
on the entity list will limit its ability to threaten the supply chain
for essential components in our military systems.''

Jinhua has been on the
\href{https://www.nytimes.com/2018/06/22/technology/china-micron-chips-theft.html}{Trump
administration's radar} for several months. Micron Technology, a
computer memory company in Idaho, accused Jinhua last year of stealing
intellectual property. In July, Micron was barred from selling some of
its products in China after Jinhua and its Taiwanese partner, United
Microelectronics, accused Micron of violating their patents.

Jinhua is opening \$5.7 billion factory in China's Fujian Province and
has become increasingly ambitious in its desire to become a global
player in the memory chip business.

The United States and China have been engaged in a trade war, with Mr.
Trump imposing tariffs on \$250 billion worth of Chinese goods and
threatening to hit all imports from China with levies. China has
responded with its own tariffs, and the two countries have exchanged
\href{https://www.nytimes.com/2018/09/27/world/asia/china-trump-election-meddling.html}{increasingly
heated words} in recent weeks.

The United States wants China to open its market to American businesses
and end its longstanding practice of pressuring American companies to
hand over valuable technology as a condition of doing business there.
Mr. Trump and Mr. Xi are expected to meet in Argentina next month at the
Group of 20 summit meeting, where they plan to discuss trade, North
Korea and other issues.

While Mr. Trump's tariffs have proved to be unpopular with both
Republican and Democratic lawmakers, his efforts to stop the theft of
intellectual property have drawn praise even from his skeptics on trade.

``China's state-owned \& directed companies lie, cheat \& steal at
government's behest,'' Senator Marco Rubio, a Republican from Florida,
said on Twitter on Monday. ``Fujian Jinhua must be held accountable for
being part of that illegality. This was the right move today to protect
our tech knowledge.''

Advertisement

\protect\hyperlink{after-bottom}{Continue reading the main story}

\hypertarget{site-index}{%
\subsection{Site Index}\label{site-index}}

\hypertarget{site-information-navigation}{%
\subsection{Site Information
Navigation}\label{site-information-navigation}}

\begin{itemize}
\tightlist
\item
  \href{https://help.nytimes.com/hc/en-us/articles/115014792127-Copyright-notice}{©~2020~The
  New York Times Company}
\end{itemize}

\begin{itemize}
\tightlist
\item
  \href{https://www.nytco.com/}{NYTCo}
\item
  \href{https://help.nytimes.com/hc/en-us/articles/115015385887-Contact-Us}{Contact
  Us}
\item
  \href{https://www.nytco.com/careers/}{Work with us}
\item
  \href{https://nytmediakit.com/}{Advertise}
\item
  \href{http://www.tbrandstudio.com/}{T Brand Studio}
\item
  \href{https://www.nytimes.com/privacy/cookie-policy\#how-do-i-manage-trackers}{Your
  Ad Choices}
\item
  \href{https://www.nytimes.com/privacy}{Privacy}
\item
  \href{https://help.nytimes.com/hc/en-us/articles/115014893428-Terms-of-service}{Terms
  of Service}
\item
  \href{https://help.nytimes.com/hc/en-us/articles/115014893968-Terms-of-sale}{Terms
  of Sale}
\item
  \href{https://spiderbites.nytimes.com}{Site Map}
\item
  \href{https://help.nytimes.com/hc/en-us}{Help}
\item
  \href{https://www.nytimes.com/subscription?campaignId=37WXW}{Subscriptions}
\end{itemize}
