Sections

SEARCH

\protect\hyperlink{site-content}{Skip to
content}\protect\hyperlink{site-index}{Skip to site index}

\href{https://www.nytimes.com/section/politics}{Politics}

\href{https://myaccount.nytimes.com/auth/login?response_type=cookie\&client_id=vi}{}

\href{https://www.nytimes.com/section/todayspaper}{Today's Paper}

\href{/section/politics}{Politics}\textbar{}Women Lead Parade of
Victories to Help Democrats Win House

\url{https://nyti.ms/2D6Mt4W}

\begin{itemize}
\item
\item
\item
\item
\item
\item
\end{itemize}

Advertisement

\protect\hyperlink{after-top}{Continue reading the main story}

Supported by

\protect\hyperlink{after-sponsor}{Continue reading the main story}

\hypertarget{women-lead-parade-of-victories-to-help-democrats-win-house}{%
\section{Women Lead Parade of Victories to Help Democrats Win
House}\label{women-lead-parade-of-victories-to-help-democrats-win-house}}

\includegraphics{https://static01.nyt.com/images/2018/11/07/us/politics/07women-1/merlin_146458899_a01830f1-9dbc-4679-8956-32b9474a1ce9-articleLarge.jpg?quality=75\&auto=webp\&disable=upscale}

By \href{https://www.nytimes.com/by/susan-chira}{Susan Chira} and
\href{https://www.nytimes.com/by/kate-zernike}{Kate Zernike}

\begin{itemize}
\item
  Nov. 6, 2018
\item
  \begin{itemize}
  \item
  \item
  \item
  \item
  \item
  \item
  \end{itemize}
\end{itemize}

They marched, they ran, and on Election Day, they won.

Women led a parade of victories and unexpected upsets Tuesday
to\href{https://www.nytimes.com/2018/11/06/us/politics/midterm-elections-results.html}{win
control of the House for the Democrats}.

It was the culmination of two years of anger, frustration and activism
driven by women appalled by Donald J. Trump's election and presidency.
\href{https://www.nytimes.com/2018/11/03/us/politics/women-activism-midterms.html?fbclid=IwAR0HEkwChpr-ouNizi4LoqrMMQC5n5AA1dzZI5uG0Lw8zyTN2DEAMW14wYw}{Women
poured into grass-roots groups determined to regain} Democratic control
of Congress and flooded organizations that trained them to run for
office. As candidates, women broke the rules and upended conventional
political wisdom. As activists, they expanded the definition of women's
issues beyond education and reproductive rights to include health care,
immigration, gun violence and the environment.

It was a litany of
\href{https://www.nytimes.com/2018/11/07/us/politics/election-history-firsts-blackburn-pressley.html}{historic
firsts}, most of them by Democrats: In Massachusetts, Ayanna Pressley
became the first woman of color in her state's congressional delegation.
Rashida Tlaib in Michigan and Ilhan Omar in Minnesota will be the first
Muslim women in Congress. Sharice Davids toppled a Republican man in
Kansas and Deb Haaland prevailed in New Mexico, becoming the first
Native American women elected to Congress. In Tennessee, Marsha
Blackburn, a Republican, became the state's first woman elected to the
Senate.

But several prominent women were also defeated --- Senator Claire
McCaskill lost to Josh Hawley in Missouri, Amy McGrath lost a closely
watched House race in Kentucky, and Senator Heidi Heitkamp lost her
re-election race in North Dakota. Stacey Abrams of Georgia, who had
hoped to become the first black woman in the country to be elected
governor, was trailing her Republican opponent, Brian Kemp.

\emph{{[}See the results for}
\emph{\href{https://www.nytimes.com/interactive/2018/11/06/us/elections/results-governor-elections.html?action=click\&module=Intentional\&pgtype=Article}{governors'
races}, the}
\emph{\href{https://www.nytimes.com/interactive/2018/11/06/us/elections/results-house-elections.html?action=click\&module=Spotlight\&pgtype=Homepage}{House
of Representatives}} \emph{and the}
\emph{\href{https://www.nytimes.com/interactive/2018/11/06/us/elections/results-senate-elections.html}{Senate}.{]}}

Pennsylvania, which had no women in its 21-member congressional
delegation, will now have four. Democratic women flipped three
Republican-held seats: Mary Gay Scanlon, Chrissy Houlahan and Susan
Wild; and Madeleine Dean won an open seat.

\includegraphics{https://static01.nyt.com/images/2018/11/07/us/politics/07women-2/merlin_146450637_4e89091a-47b4-4861-92f9-9f6f117b1d89-articleLarge.jpg?quality=75\&auto=webp\&disable=upscale}

Two women helped Democrats pick up seats in Florida: Debbie
Mucarsel-Powell and Donna Shalala, a member of former President Bill
Clinton's cabinet. Ms. Houlahan was one of four female military veterans
and political newcomers to win seats for Democrats; the others were
Mikie Sherrill in New Jersey, and Elaine Luria and Abigail Spanberger in
Virginia. Lauren Underwood in Illinois helped Democrats produce another
unexpected victory.

``I urge you to work for a better future long after tonight,'' Ms.
Sherrill said before a thunderous crowd that included dozens of women
who had spent months canvassing and phone banking. ``The thousands of
women who are ready to join me to make sure we have a better future for
our kids, for New Jersey and for the United States of America.''

She told how she had asked her daughter Maggie, the oldest of her four
young children, if she was ``O.K. with this.'' Her daughter, she said,
``asked, `If you don't run, who will?'''

It was striking to consider just how far women had come since the
women's marches across the country the day after President Trump's
inauguration. Women like Ms. Sherrill and Ms. Davids had started as long
shots, but their victories seemed assured by Election Day.

With a Democratic majority in the House, women will wield more
institutional power --- Representative
\href{https://www.nytimes.com/2018/11/07/us/politics/house-democrats-nancy-pelosi.html}{Nancy
Pelosi} is expected to beat back a leadership challenge to again become
speaker, the only woman to ever hold that post. Representative Nita
Lowey, Democrat of New York, would chair the Appropriations Committee,
and Representative Maxine Waters, Democrat of California, would chair
the Financial Service Committee.

Image

Amy McGrath, a Democrat and former Marine combat aviator, lost a closely
watched House race in Kentucky.Credit...Maddie McGarvey for The New York
Times

\emph{{[}Make sense of the country's political landscape}
\emph{\href{https://www.nytimes.com/newsletters/politics?smid=rd\%3Faction\%3Dclick\&module=inline\&pgtype=Article}{with
our newsletter}.{]}}

The energy among Democratic women made it harder for Republican women to
emerge as candidates.

And in the first big defeat of the evening for Republicans,
Representative Barbara Comstock lost by wide margin to a Democrat,
Jennifer Wexton, in the Virginia suburbs. Ms. Comstock, a prolific
fund-raiser, had survived previous challenges in the blue district.

A challenge from a Democratic woman threatened the highest-ranking
Republican woman in the House, Representative Cathy McMorris Rodgers of
Washington.

According to
\href{http://cawp.rutgers.edu/potential-candidate-summary-2018}{figures
tallied by the Center for American Women and Politics at Rutgers}, 428
women ran for Congress or governor as Democrats, compared with 162
Republicans. Of these, 210 Democratic women and 63 Republican women
remained nominees by Election Day.

Republican women were animated by their own issues, including fears of
borders being overrun and a backlash to the \#MeToo movement.

\includegraphics{https://static01.nyt.com/images/2018/11/07/us/politics/07women-4/07women-4-videoSixteenByNine3000.jpg}

Kelly Dittmar, a political scientist at the Rutgers center, the surge of
women had changed American politics.

``For some women, that meant not waiting their turn,'' she said. ``For
other women, it also meant running in ways that embraced gender and race
as an asset they bring to candidacy and office-holding, instead of a
hurdle they have to overcome to be successful in what has been a man's
world of electoral politics.''

This cycle, the first since the defeat of the first female major party
presidential candidate, many women ran without being asked.
\href{https://www.nytimes.com/2018/07/14/us/politics/women-candidates-midterms.html?fbclid=IwAR22W1gSst0XpyPVG9i_iuyqX6HZYBhWPVk6UOGo6_635Yv9_kNSwQS0Hg0}{And
they ran differently, ignoring the timeworn advice to female candidates
to talk about your résumé} and pretend you don't have a personal life.
Instead, they featured their children in ads, offered personal testimony
about sexual harassment and abuse, and opened up about family struggles
with drug addiction and debt, to connect to many Americans with the same
struggles.

Women shattered records and precedents. One-third of the female nominees
for the House were women of
\href{https://www.nytimes.com/interactive/2018/10/31/us/politics/midterm-election-candidates-diversity.html}{color,
the highest ever.} A record number of women faced off against other
women, from Arizona to New York.
Ms.\href{https://www.nytimes.com/2018/06/29/us/politics/is-this-the-year-women-break-the-rules-and-win.html?fbclid=IwAR3snD-ieWNMXl2mKlTKc7S_78NE3pbQrqIQMhBEnnz5ACZmDXJN9MneQmI}{Pressley
in Massachusetts and Alexandria Ocasio-Cortez in New York were among
women who} defeated long-serving white male incumbents in party
primaries and won tonight.

Candidates like Ms. Sherrill, Ms. McGrath and Katie Hill, who was
running for a House seat in California,
\href{https://www.nytimes.com/2018/10/30/us/politics/women-campaign-fundraising.html?fbclid=IwAR2Cnvn7eUTjYOx9Yi-l6FvFDAyjNZuzFH-VAq6CtMtdhWHkmL8_QCnCOE8}{raised
staggering amounts of money,}though women still raised less, on average,
than men. And women played bigger roles as donors, giving 36 percent
more money to congressional campaigns than in 2016.

\href{https://www.nytimes.com/interactive/2018/11/06/us/politics/questions-about-women-midterms.html}{}

\includegraphics{https://static01.nyt.com/images/2018/11/05/us/politics/05womensqs/05womensqs-articleLarge.gif}

\hypertarget{13-questions-about-women-and-the-midterms}{%
\subsection{13 Questions About Women and the
Midterms}\label{13-questions-about-women-and-the-midterms}}

The 2018 election season has raised new questions about gender and power
that could affect the outcome Tuesday for the record numbers of women
seeking office. Here are a few of those questions that might reshape
conventional wisdom about women and politics.

But as many more women ran, it was perhaps inevitable that many more
would lose, as well. Heightened political activism in the Trump era
brought out many more men running for office, too, and many of the
female candidates were Democrats running in districts that are
gerrymandered or all but assured to vote Republican. In Florida, two
challengers, Lauren Baer and Mary Barzee Flores, lost to Republican
incumbents.

Despite being more than half the population and the voters, women were
still less than a third of all candidates for Congress, the governors'
offices and other statewide executive seats.

Women running for governor, from Idaho to Texas to Maine,
\href{https://www.nytimes.com/interactive/2018/08/06/us/politics/women-governors-primaries.html?smid=fb-share\&fbclid=IwAR2lTEirDusEZmi8R2-zcuYy7h4VuHKLGL06q0HKv63QnhbDSVw8GxVr7oY}{faced
the steepest hurdles of all}. Twenty-two states have never elected a
woman as governor --- six states have female governors now --- and
research has shown voters are more reluctant to choose women as chief
executives than as legislators. Janet Mills became the first woman
elected governor of Maine. Gretchen Whitmer was elected governor in
Michigan, Laura Kelly in Kansas and Michelle Lujan Grisham in New
Mexico.

In a political season in which disgust with Washington runs high, many
women hope their lack of traditional political credentials will enhance
their outsider appeal: Jahana Hayes, a former teacher of the year, was a
surprise winner of a Democratic House primary in Connecticut and on
Tuesday became the state's first black woman elected to Congress.

The elections also could bring a younger generation to Washington: Ms.
Ocasio-Cortez as well as Abby Finkenauer, a Democrat running for a House
seat in Iowa, are both in their late 20s.

President Trump was elected by the largest gender gap on record, and
women have moved even more leftward throughout the first two years of
his presidency, even as men have gravitated toward the Republican Party.

In a Gallup survey of registered voters in September, while men favored
Republicans over Democrats, 50 percent to 44 percent, women preferred
Democrats by 58 percent to 34 percent. That 24-point split had widened
from eight points in June. The gap between the genders is even more
striking among millennials. Earlier this year, a Pew poll found that 70
percent of millennial women affiliated with or leaned toward the
Democrats, up from 56 percent four years ago. Just under half of
millennial men did.

Advertisement

\protect\hyperlink{after-bottom}{Continue reading the main story}

\hypertarget{site-index}{%
\subsection{Site Index}\label{site-index}}

\hypertarget{site-information-navigation}{%
\subsection{Site Information
Navigation}\label{site-information-navigation}}

\begin{itemize}
\tightlist
\item
  \href{https://help.nytimes.com/hc/en-us/articles/115014792127-Copyright-notice}{©~2020~The
  New York Times Company}
\end{itemize}

\begin{itemize}
\tightlist
\item
  \href{https://www.nytco.com/}{NYTCo}
\item
  \href{https://help.nytimes.com/hc/en-us/articles/115015385887-Contact-Us}{Contact
  Us}
\item
  \href{https://www.nytco.com/careers/}{Work with us}
\item
  \href{https://nytmediakit.com/}{Advertise}
\item
  \href{http://www.tbrandstudio.com/}{T Brand Studio}
\item
  \href{https://www.nytimes.com/privacy/cookie-policy\#how-do-i-manage-trackers}{Your
  Ad Choices}
\item
  \href{https://www.nytimes.com/privacy}{Privacy}
\item
  \href{https://help.nytimes.com/hc/en-us/articles/115014893428-Terms-of-service}{Terms
  of Service}
\item
  \href{https://help.nytimes.com/hc/en-us/articles/115014893968-Terms-of-sale}{Terms
  of Sale}
\item
  \href{https://spiderbites.nytimes.com}{Site Map}
\item
  \href{https://help.nytimes.com/hc/en-us}{Help}
\item
  \href{https://www.nytimes.com/subscription?campaignId=37WXW}{Subscriptions}
\end{itemize}
