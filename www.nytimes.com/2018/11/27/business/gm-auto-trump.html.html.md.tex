Sections

SEARCH

\protect\hyperlink{site-content}{Skip to
content}\protect\hyperlink{site-index}{Skip to site index}

\href{https://www.nytimes.com/section/business}{Business}

\href{https://myaccount.nytimes.com/auth/login?response_type=cookie\&client_id=vi}{}

\href{https://www.nytimes.com/section/todayspaper}{Today's Paper}

\href{/section/business}{Business}\textbar{}What's Behind the G.M.
Cutbacks, and Why Trump Is Angry

\url{https://nyti.ms/2DKzcPX}

\begin{itemize}
\item
\item
\item
\item
\item
\item
\end{itemize}

Advertisement

\protect\hyperlink{after-top}{Continue reading the main story}

Supported by

\protect\hyperlink{after-sponsor}{Continue reading the main story}

\hypertarget{whats-behind-the-gm-cutbacks-and-why-trump-is-angry}{%
\section{What's Behind the G.M. Cutbacks, and Why Trump Is
Angry}\label{whats-behind-the-gm-cutbacks-and-why-trump-is-angry}}

The White House signaled that the automaker could suffer for planning
thousands of layoffs, but the company says the course is essential to
its future.

\includegraphics{https://static01.nyt.com/images/2018/11/28/business/28autos04/merlin_50058373_9a31f881-76f0-4af7-ae68-c24fdbca5279-articleLarge.jpg?quality=75\&auto=webp\&disable=upscale}

By \href{https://www.nytimes.com/by/neal-e-boudette}{Neal E. Boudette}

\begin{itemize}
\item
  Nov. 27, 2018
\item
  \begin{itemize}
  \item
  \item
  \item
  \item
  \item
  \item
  \end{itemize}
\end{itemize}

When General Motors announced that it would idle five North American
plants and eliminate thousands of jobs, it said the move would ease the
burden of spending billions of dollars to develop the battery-powered
vehicles of the future.

But the White House put a question mark over those plans on Tuesday when
President Trump --- irate over the cutbacks --- threatened to punish
G.M. by ending federal tax credits that have helped underwrite that
automaker's electric-vehicle fleet.

``Very disappointed with General Motors and their CEO, Mary Barra, for
closing plants in Ohio, Michigan and Maryland,'' Mr. Trump said in
\href{https://twitter.com/realDonaldTrump/status/1067494680416407552}{a
Twitter post}. ``Nothing being closed in Mexico \& China.''

Apparently referring to G.M.'s federal rescue from bankruptcy in 2009,
the president added: ``The U.S. saved General Motors, and this is the
THANKS we get! We are now looking at cutting all @GM subsidies,
including for electric cars.''

So a day after announcing a plan meant to put it on firmer financial
ground by shedding money-losing operations and refocusing its resources,
the biggest of Detroit's automakers found itself on the defensive.

At a White House news briefing, Larry Kudlow, the director of the
National Economic Council, said ``there's a lot of disappointment, even
anger'' in the administration over G.M.'s decision --- a sentiment he
said he had shared in a ``lengthy conversation'' with Ms. Barra. He
cited the recent renegotiation of a trade agreement with Canada and
Mexico on terms that he said were ``a great help to the automobile
industry, and to autoworkers.''

As the president's pique became increasingly evident, the automaker put
out
\href{https://media.gm.com/media/us/en/gm/home.detail.html/content/Pages/news/us/en/2018/nov/1127-gm-statement.html}{a
statement} on its ``commitment to U.S. manufacturing'' that said in
part: ``We appreciate the actions this administration has taken on
behalf of industry to improve the overall competitiveness of U.S.
manufacturing.''

The company
\href{https://www.nytimes.com/2018/11/26/business/general-motors-cutbacks.html}{said
on Monday} that it would halt operations at four plants in the United
States and one in Canada, at the cost of roughly 6,000 factory jobs,
while also cutting its salaried work force in North America by 8,000.

G.M.'s shares rose almost 5 percent on Monday, but they gave back about
half of those gains on Tuesday in the face of the administration's
unhappiness and its threat over the tax credits.

A bright future for electric vehicles is one of G.M.'s goals. Here are
some of the calculations, and the uncertainties, behind the course it
has chosen.

\hypertarget{what-difference-do-the-electric-vehicle-credits-make}{%
\subsection{What difference do the electric-vehicle credits
make?}\label{what-difference-do-the-electric-vehicle-credits-make}}

The federal government offers a \$7,500 tax credit to buyers of
battery-powered and plug-in hybrid vehicles, an incentive that many
consumers have found attractive. The full credit, however, is available
only on the first 200,000 electric vehicles an automaker sells. Once
that threshold is reached, the credit falls to \$3,750 for six months,
then to \$1,875 for an additional six months. Beyond that, there is no
tax credit.

Tesla is the only carmaker to have sold more than 200,000 electric cars
in the United States. At the end of this year, the tax credit on Tesla
vehicles falls to \$3,750.

G.M. says it has sold about 190,000 --- mostly hybrid Volts, a model it
said Monday it was discontinuing, and fully electric Bolts --- and will
pass 200,000 early next year.

There are proposals in Congress that would extend or expand the credit,
but the Trump administration's opposition could dim or kill those
prospects.

``We're going to be looking at certain subsidies regarding electric cars
and others and whether they should apply or not,'' Mr. Kudlow said
Tuesday. ``Can't say anything final about that, but we are looking into
it.''

\hypertarget{why-did-gm-choose-the-plants-it-is-shuttering}{%
\subsection{Why did G.M. choose the plants it is
shuttering?}\label{why-did-gm-choose-the-plants-it-is-shuttering}}

Of the five factories that G.M. is mothballing, three are assembly
plants: Lordstown, Ohio; Oshawa, Ontario; and the Detroit-Hamtramck
plant in Michigan. All are almost certainly losing money, and make cars
whose sales have plunged. As a result, the plants are operating well
below capacity.

Production in Lordstown fell by more than half last year from 2016. This
year it has operated just a single, eight-hour shift each day.
Typically, car plants must operate two shifts to generate profits.

Moreover, the outlook for each plant is grim. With gas prices low,
American consumers have flocked to S.U.V.s and all but abandoned small
cars, like the Cruze, which is made in Lordstown, and the Impala, the
Buick LaCrosse, and the Cadillac CT6, made at Detroit-Hamtramck. The
plant in Oshawa also produces the Impala.

\hypertarget{are-these-plants-closed-for-good}{%
\subsection{Are these plants closed for
good?}\label{are-these-plants-closed-for-good}}

Not necessarily. G.M. specifically announced that the plants were now
``unassigned'' --- that is, they have not yet been assigned new vehicles
to make after they stop production in 2019.

Next year the company and the United Auto Workers union will negotiate a
new labor contract. In past negotiations, the union has given
concessions on wages and other cost-saving measures in exchange for the
company's keeping plants open. Idling the plants now gives G.M. a
powerful bargaining chip in the contract talks.

``Many of the U.S. workers impacted by these actions will have the
opportunity to shift to other G.M. plants where we will need more
employees to support growth in trucks, crossovers and S.U.V.s,'' the
automaker said Tuesday --- in other words, not the models it has been
making in those plants. Ford Motor Company has also
\href{https://www.nytimes.com/2018/10/05/business/ford-motor-cars.html}{cut
jobs} and
\href{https://www.nytimes.com/2018/04/25/business/ford-earnings.html}{dropped
sedans} from its North American lineup this year.

In addition, G.M. is working on a dozen or more electric vehicles that
it plans to roll out in two to four years. Those models will have to be
made somewhere, and some could end up in one or more of the three idled
plants.

\includegraphics{https://static01.nyt.com/images/2018/11/28/business/28autos01/merlin_145756608_c7844482-24cd-443e-8f5a-f2e649c806e3-articleLarge.jpg?quality=75\&auto=webp\&disable=upscale}

\hypertarget{will-the-billions-invested-in-new-technologies-pay-off}{%
\subsection{Will the billions invested in new technologies pay
off?}\label{will-the-billions-invested-in-new-technologies-pay-off}}

It's not clear. Tesla has proved that tens of thousands of people are
willing to buy upscale electric cars. It has not yet proved definitively
that it can make money doing so. That's also a big hurdle for G.M., as
well as Ford and most other major automakers.

G.M. and others have produced electric vehicles, but sales for each
model have usually amounted to a few thousand cars a month --- far too
few to turn a profit. G.M. produces the Chevrolet Bolt, an electric car
that is supposed to go up to 238 miles before needing a recharge. But so
far this year, it has sold only about 13,000 Bolts.

As for the huge investments in self-driving technologies, their profit
potential is even less certain. G.M. has poured billions of dollars into
its autonomous-driving unit, G.M. Cruise, and has recruited Honda and
SoftBank, the Japanese technology giant, as partners.

G.M. Cruise is working on a self-driving car with no steering wheel and
no pedals, and it intends to use the car in driverless and delivery
services that the company hopes will become a lucrative line of
business.

G.M. may be one of the first to rush into this field, but it faces an
unfamiliar path and plenty of competition. Ford, Uber, Lyft and Waymo,
the self-driving company started by Alphabet, the parent of Google, are
all working on driverless services. Even if these services generate
profits, it could be years before the companies earn back their
investments.

\hypertarget{what-do-gms-cutbacks-say-about-the-economy}{%
\subsection{What do G.M.'s cutbacks say about the
economy?}\label{what-do-gms-cutbacks-say-about-the-economy}}

Not a lot. The halting of production at these three sites doesn't mean
G.M. is struggling. On the contrary, the company continues to report
billions in profits, thanks to buoyant pickup and S.U.V. sales.

G.M. took this action because it is being squeezed. It is losing money
on the small cars and sedans these plants make, and it needs to keep
investing in new technologies. By idling the three assembly plants ---
the other two make transmissions --- it hopes to use the savings to help
fund its electric-vehicle and autonomous-driving ambitions.

That particular need to shift capital investment is not an issue in
other sectors. And while the industry is the largest manufacturing
sector, it makes up only about 4 percent of the nation's gross domestic
product.

A broader economic slowdown could certainly cause grief for automakers.
While Ms. Barra said one factor in G.M.'s move was to ``improve our
downturn protection,'' she also made a point of saying the automaker was
acting ``while the company and the economy are strong.'' So a few idle
auto plants are not necessarily a leading indicator for the broader
American economy.

Advertisement

\protect\hyperlink{after-bottom}{Continue reading the main story}

\hypertarget{site-index}{%
\subsection{Site Index}\label{site-index}}

\hypertarget{site-information-navigation}{%
\subsection{Site Information
Navigation}\label{site-information-navigation}}

\begin{itemize}
\tightlist
\item
  \href{https://help.nytimes.com/hc/en-us/articles/115014792127-Copyright-notice}{©~2020~The
  New York Times Company}
\end{itemize}

\begin{itemize}
\tightlist
\item
  \href{https://www.nytco.com/}{NYTCo}
\item
  \href{https://help.nytimes.com/hc/en-us/articles/115015385887-Contact-Us}{Contact
  Us}
\item
  \href{https://www.nytco.com/careers/}{Work with us}
\item
  \href{https://nytmediakit.com/}{Advertise}
\item
  \href{http://www.tbrandstudio.com/}{T Brand Studio}
\item
  \href{https://www.nytimes.com/privacy/cookie-policy\#how-do-i-manage-trackers}{Your
  Ad Choices}
\item
  \href{https://www.nytimes.com/privacy}{Privacy}
\item
  \href{https://help.nytimes.com/hc/en-us/articles/115014893428-Terms-of-service}{Terms
  of Service}
\item
  \href{https://help.nytimes.com/hc/en-us/articles/115014893968-Terms-of-sale}{Terms
  of Sale}
\item
  \href{https://spiderbites.nytimes.com}{Site Map}
\item
  \href{https://help.nytimes.com/hc/en-us}{Help}
\item
  \href{https://www.nytimes.com/subscription?campaignId=37WXW}{Subscriptions}
\end{itemize}
