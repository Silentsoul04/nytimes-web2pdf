Sections

SEARCH

\protect\hyperlink{site-content}{Skip to
content}\protect\hyperlink{site-index}{Skip to site index}

\href{https://www.nytimes.com/section/politics}{Politics}

\href{https://myaccount.nytimes.com/auth/login?response_type=cookie\&client_id=vi}{}

\href{https://www.nytimes.com/section/todayspaper}{Today's Paper}

\href{/section/politics}{Politics}\textbar{}After a Hiatus, China
Accelerates Cyberspying Efforts to Obtain U.S. Technology

\url{https://nyti.ms/2DOnqnE}

\begin{itemize}
\item
\item
\item
\item
\item
\item
\end{itemize}

Advertisement

\protect\hyperlink{after-top}{Continue reading the main story}

Supported by

\protect\hyperlink{after-sponsor}{Continue reading the main story}

\hypertarget{after-a-hiatus-china-accelerates-cyberspying-efforts-to-obtain-us-technology}{%
\section{After a Hiatus, China Accelerates Cyberspying Efforts to Obtain
U.S.
Technology}\label{after-a-hiatus-china-accelerates-cyberspying-efforts-to-obtain-us-technology}}

\includegraphics{https://static01.nyt.com/images/2018/11/30/us/politics/30dc-cyber-1-print/merlin_147443805_1a40ea59-ae5a-477e-9f01-8605eae3d322-articleLarge.jpg?quality=75\&auto=webp\&disable=upscale}

By \href{https://www.nytimes.com/by/david-e-sanger}{David E. Sanger} and
\href{https://www.nytimes.com/by/steven-lee-myers}{Steven Lee Myers}

\begin{itemize}
\item
  Nov. 29, 2018
\item
  \begin{itemize}
  \item
  \item
  \item
  \item
  \item
  \item
  \end{itemize}
\end{itemize}

\href{https://cn.nytimes.com/world/20181130/china-trump-cyberespionage/}{阅读简体中文版}\href{https://cn.nytimes.com/world/20181130/china-trump-cyberespionage/zh-hant/}{閱讀繁體中文版}

WASHINGTON --- Three years ago, President Barack Obama struck a deal
with China that few thought was possible: President Xi Jinping agreed to
end his nation's yearslong practice of breaking into the computer
systems of American companies, military contractors and government
agencies to obtain designs, technology and corporate secrets, usually on
behalf of China's state-owned firms.

The pact was celebrated by the Obama administration as one of the first
arms-control agreements for cyberspace --- and for 18 months or so, the
number of Chinese attacks plummeted. But the victory was fleeting.

Soon after President Trump took office, China's cyberespionage picked up
again and, according to intelligence officials and analysts, accelerated
in the last year as trade conflicts and other tensions began to poison
relations between the world's two largest economies.

The nature of China's espionage has also changed. The hackers of the
People's Liberation Army --- whose famed Unit 61398 tore through
American companies until its operations from a base in Shanghai
\href{https://www.nytimes.com/2013/02/19/technology/chinas-army-is-seen-as-tied-to-hacking-against-us.html}{were
exposed in 2013} --- were forced to stand down, some of them indicted by
the United States. But now, the officials and analysts say, they have
begun to be replaced by stealthier operatives in the country's
intelligence agencies.

The new operatives have intensified their focus on America's commercial
and industrial prowess, and on technologies that the Chinese believe can
give them a military advantage.

That, in turn, has prompted a flurry of criminal cases, including
\href{https://www.nytimes.com/2018/10/10/us/politics/china-spy-espionage-arrest.html}{the
extraordinary arrest and extradition} from Belgium of a Chinese
intelligence official in October. Trump administration officials said
the arrest reflected a more determined counterattack against a threat
that has infuriated some of the country's most powerful corporations.

``We have certainly seen the behavior change over the past year,'' said
Rob Joyce, Mr. Trump's former White House cybercoordinator, speaking at
the Aspen Cyber Summit in San Francisco this month.

Mr. Trump and administration officials often suggest that all
technology-acquisition efforts by China amount to theft. In doing so,
they are blurring the line between stealing technology and negotiated
deals in which corporations agree to transfer technology to Chinese
manufacturing or marketing partners in return for access to China's
market --- a practice American companies often view as a form of
corporate blackmail but one distinct from outright theft.

The stealing of industrial designs and intellectual property --- from
blueprints for power plants or high-efficiency solar panels, or the F-35
fighter jet --- is a long-running problem. The United States trade
representative published a report this month detailing old and new
examples. But the administration has never said whether cracking down on
theft and cyberattacks is part of the negotiations or simply a demand
that China cease activity that Beijing has already acknowledged, in the
Obama years, was illegitimate.

But as Mr. Trump and Mr. Xi
\href{https://www.nytimes.com/2018/11/27/us/politics/trump-xi-trade-g-20.html}{prepare
to meet at the Group of 20 gathering} in Argentina this weekend, China's
corporate espionage has once again emerged as a core American grievance.

Whatever the reason for the renewed hacking, it is a cautionary tale as
Mr. Trump tries to use tariffs and threats of more restrictions to
strike a new trade deal with Mr. Xi, one that presumably would address,
once again, the Chinese practices that Mr. Obama thought he had halted.

American trade and intelligence officials, as well as experts from
private cybersecurity firms, all acknowledged that the previous
agreement had completely fallen apart.

And that, they agreed, has made it still more difficult to imagine how
any new agreement struck between Mr. Trump and Mr. Xi would become a
permanent solution to a problem that reaches back years, and seems
rooted in completely different views of what constitutes reasonable
competition.

``Our two systems are so dissimilar that I think there was never real
hope that crafting an agreement like this would last that long anyway,''
said Matthew Brazil, a former government official who now runs Madeira
Security Consulting, a firm in San Jose, Calif.

Why the espionage has spiked again is a matter of debate. Some officials
and analysts call it a cause of the worsening trade relationships,
others a symptom. Still others argued that the tightening of American
export controls in critical industries like aerospace and rules on
Chinese investment in Silicon Valley --- which China sees as part of a
``containment'' strategy to blunt its industrial and geopolitical rise
--- has led the Chinese once again to try to steal what they cannot buy.

The impetus for the 2015 accord was one of the most blatant espionage
operations ever conducted by the Chinese government: the removal, over a
period of more than a year, of 22 million security-clearance files on
American officials, military personnel, contractors and American
intelligence officers.

The Obama administration, partly out of embarrassment, said little about
the breach, never naming the Chinese publicly --- except by mistake when
the director of national intelligence blurted out the truth.

Privately, American intelligence officials concluded that the Chinese
were assembling a giant database of who worked with whom, and on what,
in the American national security sphere, and were applying ``big data''
techniques to analyze the information. The C.I.A. could not move some
officers to China, for fear their cover had been blown. Publicly, Obama
administration officials offered millions of Americans credit protection
for a few years in the wake of the data breach --- as if Mr. Xi's agents
were looking for credit card numbers.

\includegraphics{https://static01.nyt.com/images/2018/11/30/us/politics/30dc-cyber-2-print/merlin_99622690_c0d336c0-2f36-462b-9169-0567b11ddfc8-articleLarge.jpg?quality=75\&auto=webp\&disable=upscale}

But Mr. Obama used the episode, and the threat of sanctions, to force
Mr. Xi into what he called a ``common understanding'' that neither the
United States nor China should engage in state-sponsored cyberintrusions
to poach intellectual property, and that they would together seek
``\href{https://www.nytimes.com/2015/09/26/world/asia/xi-jinping-white-house.html}{international
rules of the road for appropriate conduct in cyberspace}.''

All that was forgotten after Mr. Obama left office. Mr. Trump has never
referred publicly to the 2015 agreement.

Michael Kovrig, a former Canadian diplomat who is now a China analyst
for the International Crisis Group, said that China had a fundamentally
different understanding of what was acceptable in espionage. While the
Central Intelligence Agency, say, would not act to help a private
company gain a competitive advantage over a foreign competitor, he said,
China's Communist Party, which has control over practically all aspects
of policy there, would make no such distinction.

``If you view economic growth as an existential pillar of your party's
political legitimacy and in fact your national security, it follows that
you would do anything possible to maintain that competitive edge,'' he
said.

Indeed, the latest spike in corporate espionage cases --- including some
not yet made public --- has focused on industries critical to Mr. Xi's
Made in China 2025 program.

That is a plan to jump ahead of the United States and others in
cutting-edge industries like aerospace, automation, artificial
intelligence and quantum computing.

``We are seeing it in high tech, in law firms, in insurance companies,''
said Dmitri Alperovitch, one of the founders of CrowdStrike, who early
in his career was one of the first to identify the teams of state-run
Chinese hackers aiming at the United States, and who tracked their
retreat after the 2015 pledge.

With the arrest of the intelligence officer in Belgium in October, the
Trump administration claimed it had exposed what the assistant F.B.I.
director, Bill Priestap, called ``the Chinese government's direct
oversight of economic espionage against the United States.''

That case involves Xu Yanjun, a deputy division director in the Jiangsu
branch of the Ministry of State Security, China's main intelligence
agency.

According to a secret criminal complaint filed in Ohio in March but not
unsealed until October, Mr. Xu tried to recruit an employee of General
Electric Aviation and entice him to provide proprietary information
about jet fan blade designs.

Instead the employee alerted the company, which went to the F.B.I. and
organized a sting. Mr. Xu flew from China to Belgium in April on the
hope he would be able to copy the employee's computer hard drive. He was
arrested on April 1 when he arrived in Brussels and was extradited to
the United States on Oct. 9, the day before the Justice Department made
the case public.

China's Foreign Ministry denounced the criminal case as ``pure
fabrication,'' but it has neither confirmed nor denied that Mr. Xu was
an intelligence officer. China's relatively muted reaction could be an
effort to minimize attention on an embarrassing intelligence failure and
leave room for quiet negotiations for an exchange.

Mr. Xu's was the most high profile of several recent cases, including
two others that had links to the Ministry of State Security's branch in
Jiangsu Province, which extends north from Shanghai.

In September, the Justice Department
\href{https://www.nytimes.com/2018/09/25/us/politics/ji-chaoqun-china-spy.html}{announced
the arrest} of Ji Chaoqun, a 27-year-old graduate student who had joined
the Army Reserves under a special waiver for foreigners.

The F.B.I. affidavit in the case said that Mr. Ji's handler ---
presumably Mr. Xu --- had been arrested, allowing the bureau to send an
undercover officer to meet the student in April. Mr. Ji, the affidavit
said, had been recruited to gather background information about eight
potential recruits for the Jiangsu branch.

Mr. Xu, who went by at least two aliases, often claimed to represent the
Jiangsu Association for International Science and Technology Cooperation
and Nanjing University of Aeronautics and Astronautics, both based in
the provincial capital, Nanjing.

The reasons Jiangsu has become a hotbed of China's cyberespionage are
not entirely clear, though it is an important manufacturing center, with
many foreign investments, and is thus one of China's richest provinces.

In 2016, the director of the Jiangsu intelligence branch, Liu Yang,
declared that ``the national security departments should actively
cooperate and promote enterprises'' in their efforts to expand and
compete globally, according to
\href{http://www.szcc.org.cn/news-micro/jiang-su-sheng-guo-jia-an-quan-ting-ting-chang-2978.html}{a
report} from the Suzhou General Chamber of Commerce. In January, Mr. Liu
was promoted and is now the vice governor of the province.

Another American criminal case of espionage in the same region of China
was announced Oct. 30. The Justice Department accused two other
intelligence officers from that branch, as well as five hackers and two
employees of a French aerospace company in Suzhou. The target was
Safran, which operates a joint venture, CFM International, that builds
jet engines with General Electric.

The hackers were accused of using a variety of sophisticated techniques
and tools against the Suzhou plant, and against other companies. But as
in the cases the Obama administration brought, the suspects are believed
to still be in China and thus beyond the reach of American law
enforcement.

Advertisement

\protect\hyperlink{after-bottom}{Continue reading the main story}

\hypertarget{site-index}{%
\subsection{Site Index}\label{site-index}}

\hypertarget{site-information-navigation}{%
\subsection{Site Information
Navigation}\label{site-information-navigation}}

\begin{itemize}
\tightlist
\item
  \href{https://help.nytimes.com/hc/en-us/articles/115014792127-Copyright-notice}{©~2020~The
  New York Times Company}
\end{itemize}

\begin{itemize}
\tightlist
\item
  \href{https://www.nytco.com/}{NYTCo}
\item
  \href{https://help.nytimes.com/hc/en-us/articles/115015385887-Contact-Us}{Contact
  Us}
\item
  \href{https://www.nytco.com/careers/}{Work with us}
\item
  \href{https://nytmediakit.com/}{Advertise}
\item
  \href{http://www.tbrandstudio.com/}{T Brand Studio}
\item
  \href{https://www.nytimes.com/privacy/cookie-policy\#how-do-i-manage-trackers}{Your
  Ad Choices}
\item
  \href{https://www.nytimes.com/privacy}{Privacy}
\item
  \href{https://help.nytimes.com/hc/en-us/articles/115014893428-Terms-of-service}{Terms
  of Service}
\item
  \href{https://help.nytimes.com/hc/en-us/articles/115014893968-Terms-of-sale}{Terms
  of Sale}
\item
  \href{https://spiderbites.nytimes.com}{Site Map}
\item
  \href{https://help.nytimes.com/hc/en-us}{Help}
\item
  \href{https://www.nytimes.com/subscription?campaignId=37WXW}{Subscriptions}
\end{itemize}
