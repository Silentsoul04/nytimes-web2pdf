Sections

SEARCH

\protect\hyperlink{site-content}{Skip to
content}\protect\hyperlink{site-index}{Skip to site index}

\href{https://www.nytimes.com/section/politics}{Politics}

\href{https://myaccount.nytimes.com/auth/login?response_type=cookie\&client_id=vi}{}

\href{https://www.nytimes.com/section/todayspaper}{Today's Paper}

\href{/section/politics}{Politics}\textbar{}Trump Cancels Meeting With
Putin, Citing Naval Clash Between Russia and Ukraine

\url{https://nyti.ms/2zwxDSw}

\begin{itemize}
\item
\item
\item
\item
\item
\item
\end{itemize}

Advertisement

\protect\hyperlink{after-top}{Continue reading the main story}

Supported by

\protect\hyperlink{after-sponsor}{Continue reading the main story}

\hypertarget{trump-cancels-meeting-with-putin-citing-naval-clash-between-russia-and-ukraine}{%
\section{Trump Cancels Meeting With Putin, Citing Naval Clash Between
Russia and
Ukraine}\label{trump-cancels-meeting-with-putin-citing-naval-clash-between-russia-and-ukraine}}

\includegraphics{https://static01.nyt.com/images/2018/11/30/us/politics/30dc-prexy-print/merlin_147479475_27eadb36-8d3b-49ee-baf2-9c39a883fdcd-articleLarge.jpg?quality=75\&auto=webp\&disable=upscale}

By \href{https://www.nytimes.com/by/peter-baker}{Peter Baker}

\begin{itemize}
\item
  Nov. 29, 2018
\item
  \begin{itemize}
  \item
  \item
  \item
  \item
  \item
  \item
  \end{itemize}
\end{itemize}

BUENOS AIRES --- President Trump on Thursday abruptly canceled his
planned meeting with President Vladimir V. Putin of Russia, citing the
unresolved naval standoff
\href{https://www.nytimes.com/2018/11/26/world/europe/russia-ukraine-kerch-strait.html}{between
Russia and Ukraine} and upending his hopes of further cementing the
relationship between the two leaders.

The president's decision, announced on Twitter barely an hour after he
told reporters he still expected to go through with the meeting, came
shortly after new revelations that Mr. Trump's personal lawyer had
negotiated to build a tower in Moscow much later during the 2016
presidential election than previously acknowledged.

The last-minute cancellation underscored just how fraught the
Russian-American relationship has grown despite the president's
concerted efforts to make friends, as the Kremlin increasingly asserts
itself overseas while Washington is absorbed by the investigation into
ties between Mr. Trump's circle and Moscow.

When Mr. Trump met with Mr. Putin in Helsinki last summer, it came just
days after the special counsel, Robert S. Mueller III, had
\href{https://www.nytimes.com/2018/07/13/us/politics/mueller-indictment-russian-intelligence-hacking.html}{indicted
12 Russian intelligence officers} in the hacking of Democratic emails
during the 2016 campaign. Undaunted, Mr. Trump went ahead with that
meeting and, with Mr. Putin at his side,
\href{https://www.nytimes.com/2018/07/16/world/europe/trump-putin-election-intelligence.html}{challenged
the conclusions} of American intelligence agencies about Russian
election interference.

The Buenos Aires session, which had been scheduled for Saturday on the
sidelines of the Group of 20 economic summit meeting, was only the
second to be canceled between top American and Russian or Soviet leaders
since an American U-2 spy plane piloted by Francis Gary Powers was shot
down over Russian territory in 1960. The other time came in 2013 when
President Barack Obama
\href{https://www.nytimes.com/2013/08/08/world/europe/obama-cancels-visit-to-putin-as-snowden-adds-to-tensions.html}{called
off a trip to Moscow} to protest Mr. Putin's decision to shelter Edward
J. Snowden, the National Security Agency leaker.

Mr. Trump has adamantly denied any collusion with Russia during the
campaign and dismissed questions about business ventures or economic
interests in Russia. But Michael D. Cohen, his former personal lawyer
and fixer, admitted in court on Thursday that he had engaged in
negotiations for a Moscow tower well into the campaign and had
personally briefed Mr. Trump and members of his family.

The president said that Mr. Cohen was ``weak'' and lying in order to
reduce his sentence for various criminal charges, adding that while a
Moscow tower had been considered, he had opted against it because he was
running for president. But Mr. Trump insisted that there would have been
nothing wrong with pursuing such a project as a candidate if he had.

Even as he denounced his former lawyer on Thursday morning, Mr. Trump
told reporters that he still planned to go ahead with his meeting with
Mr. Putin.

``I probably will be meeting with President Putin,'' he told reporters
on the South Lawn of the White House just after 10:30 a.m. as he left on
the trip to Buenos Aires. ``I think it's a very good time to have the
meeting.'' He added that he would be getting a report on Air Force One
about the Russia-Ukraine confrontation ``and that will determine what
I'm going to be doing.''

At 11:34 a.m., he reversed himself, announcing on Twitter that he would
scrap the meeting after all, attributing the move to the Ukraine
conflict.
\href{https://www.nytimes.com/2018/11/26/world/europe/russia-ukraine-kerch-strait.html}{Russian
forces seized three small Ukrainian naval vessels} and more than 20
sailors on Sunday, including at least three wounded in a shooting by the
Russian side.

``Based on the fact that the ships and sailors have not been returned to
Ukraine from Russia, I have decided it would be best for all parties
concerned to cancel my previously scheduled meeting in Argentina with
President Vladimir Putin,''
\href{https://twitter.com/realDonaldTrump/status/1068181367857397760}{Mr.
Trump wrote.}

``I look forward to a meaningful Summit again as soon as this situation
is resolved!'' he added.

Sarah Huckabee Sanders, the White House press secretary, told reporters
on Air Force One that Mr. Trump had scrubbed the meeting after reviewing
the report on Russia's actions against Ukraine. Mr. Trump conferred with
Secretary of State Mike Pompeo and John F. Kelly, the White House chief
of staff, who were on the plane, and by telephone with John R. Bolton,
his national security adviser, who was in Brazil.

But a meeting with Mr. Putin also could have raised questions about Mr.
Trump's ties to Russia after Mr. Cohen's revelations, producing an
politically uncomfortable moment.

Russia analysts noted that nothing had changed in the Russian-Ukrainian
standoff in days and that Mr. Trump had not seen it necessary to scratch
the meeting before.

``This is a no-brainer,'' said Michael Carpenter, senior director of the
Penn Biden Center for Diplomacy and Global Engagement and a former
Pentagon official under Mr. Obama.

``It's all about the political optics in light of the Trump Tower Moscow
news and the fact that Trump simply can't bring himself to ever confront
Putin in public,'' Mr. Carpenter said. ``This would have been a P.R.
disaster of epic proportions if he had agreed to meet and not confronted
Putin.''

Once again, however, Mr. Trump's ad hoc decision-making caught a foreign
government unawares. Dmitri S. Peskov, the Kremlin spokesman, told
Russian reporters that they had seen Mr. Trump's Twitter posting but had
no other word from the American government.

``We don't have official information,'' he said, according to the Tass
news agency. He added, ``If this is so,'' then Mr. Putin ``will have a
few additional hours in the schedule for useful meetings on the
sidelines of the summit.''

Other sub-dramas were percolating in Buenos Aires as Mr. Trump flew on
Thursday. The president downgraded scheduled meetings with two allies,
Presidents Moon Jae-in of South Korea and Recep Tayyip Erdogan of
Turkey, without explanation. Instead of full-fledged sessions, the
president will have ``pull-asides'' with both, according to the White
House, meaning casual chats on the sideline of the main sessions.

At the same time, speculation focused on whether Prime Minister Justin
Trudeau of Canada would attend the ceremonial signing on Friday of the
revised version of the North American Free Trade Agreement, which Mr.
Trump has rebranded the United States-Mexico-Canada Agreement. Mr.
Trudeau's schedule did not include the ceremony, but a Canadian official
late Thursday confirmed that he would attend.

And then there was Chancellor Angela Merkel of Germany, who also has a
tense relationship with Mr. Trump and was scheduled to meet with him on
Friday. Her government plane was forced to land en route to Buenos Aires
by technical difficulties and she was reported to be turning to a
commercial flight on Friday, an embarrassment for the leader of a major
power.

Assuming they do meet, Mr. Trump and Ms. Merkel will presumably discuss
the Ukraine crisis. While Mr. Trump had said earlier in the week that he
was not happy about Russia's latest aggression, he had left any stronger
denunciation to his United Nations ambassador. Members of Congress on
both sides of the aisle had called on Mr. Trump to take a tougher stance
and even cancel the meeting with Mr. Putin.

Mr. Trump had been seeking another meeting with Mr. Putin for months,
first suggesting the Russian leader visit the White House and later
trying to arrange to sit down together in Paris earlier this month, but
neither idea went ahead. Instead, the two leaders settled on Buenos
Aires for their next meeting.

The session was already freighted by multiple tension points between the
two countries in addition to the lingering issues of the election
meddling and the clash with Ukraine.

Mr. Trump recently declared that
\href{https://www.nytimes.com/2018/10/19/us/politics/russia-nuclear-arms-treaty-trump-administration.html}{he
would withdraw the United States from the Intermediate-Range Nuclear
Forces Treaty} signed by Ronald Reagan and Mikhail Gorbachev in 1987,
\href{https://www.nytimes.com/2018/10/23/world/europe/inf-treaty-russia-united-states-trump-nuclear.html}{citing
Russian violations}, an issue that was sure to come up. Syria and Iran
were other flash points expected to be discussed.

Some veteran diplomats said the scrubbed meeting was a missed
opportunity to pressure Mr. Putin into easing the conflict with Ukraine
and to find a possible compromise that would enable them to salvage the
I.N.F. Treaty.

``I would have preferred that he had stuck with the meeting and
delivered a strong rebuke to Putin for Sunday's aggression, with a
threat of punitive steps if Russia doesn't quickly release the ships and
the crews,'' said Alexander Vershbow, who served as ambassador to Russia
under President George W. Bush.

``But given that Trump may be incapable of doing that, and might have
endorsed Putin's `vigorous denials' that Russia was responsible,'' he
added, ``cancellation may have been the best course.''

Molly McKew, a consultant who has advised the leaders of governments
threatened by Moscow, said Mr. Trump clearly wanted to avoid having to
confront Mr. Putin directly over Ukraine. ``Since he doesn't have the
constitution to say it to Putin's face, he sent it as a tweet,'' she
said. ``In many respects --- and to avoid Helsinki 2.0 --- this is
probably the best outcome. He can say he did it and not end up looking
the fool.''

Advertisement

\protect\hyperlink{after-bottom}{Continue reading the main story}

\hypertarget{site-index}{%
\subsection{Site Index}\label{site-index}}

\hypertarget{site-information-navigation}{%
\subsection{Site Information
Navigation}\label{site-information-navigation}}

\begin{itemize}
\tightlist
\item
  \href{https://help.nytimes.com/hc/en-us/articles/115014792127-Copyright-notice}{©~2020~The
  New York Times Company}
\end{itemize}

\begin{itemize}
\tightlist
\item
  \href{https://www.nytco.com/}{NYTCo}
\item
  \href{https://help.nytimes.com/hc/en-us/articles/115015385887-Contact-Us}{Contact
  Us}
\item
  \href{https://www.nytco.com/careers/}{Work with us}
\item
  \href{https://nytmediakit.com/}{Advertise}
\item
  \href{http://www.tbrandstudio.com/}{T Brand Studio}
\item
  \href{https://www.nytimes.com/privacy/cookie-policy\#how-do-i-manage-trackers}{Your
  Ad Choices}
\item
  \href{https://www.nytimes.com/privacy}{Privacy}
\item
  \href{https://help.nytimes.com/hc/en-us/articles/115014893428-Terms-of-service}{Terms
  of Service}
\item
  \href{https://help.nytimes.com/hc/en-us/articles/115014893968-Terms-of-sale}{Terms
  of Sale}
\item
  \href{https://spiderbites.nytimes.com}{Site Map}
\item
  \href{https://help.nytimes.com/hc/en-us}{Help}
\item
  \href{https://www.nytimes.com/subscription?campaignId=37WXW}{Subscriptions}
\end{itemize}
