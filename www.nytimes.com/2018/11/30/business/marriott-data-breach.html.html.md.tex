Sections

SEARCH

\protect\hyperlink{site-content}{Skip to
content}\protect\hyperlink{site-index}{Skip to site index}

\href{https://www.nytimes.com/section/business}{Business}

\href{https://myaccount.nytimes.com/auth/login?response_type=cookie\&client_id=vi}{}

\href{https://www.nytimes.com/section/todayspaper}{Today's Paper}

\href{/section/business}{Business}\textbar{}Marriott Hacking Exposes
Data of Up to 500 Million Guests

\url{https://nyti.ms/2DPC2D4}

\begin{itemize}
\item
\item
\item
\item
\item
\item
\end{itemize}

Advertisement

\protect\hyperlink{after-top}{Continue reading the main story}

Supported by

\protect\hyperlink{after-sponsor}{Continue reading the main story}

\hypertarget{marriott-hacking-exposes-data-of-up-to-500-million-guests}{%
\section{Marriott Hacking Exposes Data of Up to 500 Million
Guests}\label{marriott-hacking-exposes-data-of-up-to-500-million-guests}}

\includegraphics{https://static01.nyt.com/images/2018/12/01/business/01marriot/merlin_147517380_3026d611-16f6-434c-bcf9-86ce56c8b3da-articleLarge.jpg?quality=75\&auto=webp\&disable=upscale}

By \href{https://www.nytimes.com/by/nicole-perlroth}{Nicole Perlroth},
\href{https://www.nytimes.com/by/amie-tsang}{Amie Tsang} and
\href{https://www.nytimes.com/by/adam-satariano}{Adam Satariano}

\begin{itemize}
\item
  Nov. 30, 2018
\item
  \begin{itemize}
  \item
  \item
  \item
  \item
  \item
  \item
  \end{itemize}
\end{itemize}

The hotel chain asked guests checking in for a treasure trove of
personal information: credit cards, addresses and sometimes passport
numbers. On Friday, consumers learned the risk. Marriott International
revealed that hackers had breached its Starwood reservation system and
had stolen the personal data of up to 500 million guests.

The assault started as far back as 2014, and was one of the largest
known thefts of personal records, second only to a
\href{https://www.nytimes.com/2017/10/03/technology/yahoo-hack-3-billion-users.html}{2013
breach of Yahoo} that affected three billion user accounts and larger
than a 2017 episode
\href{https://www.nytimes.com/2017/09/14/business/equifax-hack-what-we-know.html}{involving
the credit bureau Equifax}.

The intrusion was a reminder that after years of headline-grabbing
attacks, the computer networks of big companies are still vulnerable.

The Starwood attack happened roughly the same time as a number of other
breaches at American health insurers and government agencies, including
the United States Office of Personnel Management, in what security
research firms and government officials described as an effort to
compile a vast database of personal information on potential espionage
targets.

Experts don't know if the Starwood attack was connected to those other
episodes. But Starwood's data has not popped up on the so-called dark
web, according to Recorded Future, a cybersecurity firm, and Coalition,
a cyber insurance provider, which suggested that the hotel attackers
weren't looking to sell what they took.

``Usually when stolen data doesn't appear, it's a state actor collecting
it for intelligence purposes,'' said James A. Lewis, a cybersecurity
expert at the Center for Strategic Studies in Washington.

The breach hit customers who made reservations for the Marriott-owned
Starwood hotel brands from 2014 to September 2018. The properties
include Sheraton, Westin, W Hotels, St. Regis, Four Points, Aloft, Le
Méridien, Tribute, Design Hotels, Element and the Luxury Collection.

Marriott hotels, including Residence Inn and the Ritz-Carlton, operate
on a separate reservation system. The company has plans to merge that
system with Starwood's.

The names, addresses, phone numbers, birth dates, email addresses and
encrypted credit card details of hotel customers were stolen. The travel
histories and passport numbers of a smaller group of guests were also
taken.

Marriott said it had set up a
\href{https://info.starwoodhotels.com/}{dedicated website} and call
center to deal with guests and said it would try to reach affected
customers on Friday to inform them of the breach. The site was having
problems staying online shortly after the attack was announced.

The company is offering one year of free enrollment in a service called
Web Watcher to people who live in the United States, Canada and Britain.
Marriott described it as a service that keeps an eye on websites where
thieves swap and sell personal information and then alerts people if
anyone is selling their information.

``We deeply regret this incident,'' Arne Sorenson, Marriott's president
and chief executive, said
\href{http://news.marriott.com/2018/11/marriott-announces-starwood-guest-reservation-database-security-incident/}{in
a statement}. ``We fell short of what our guests deserve and what we
expect of ourselves.''

The intrusion went unnoticed for four years by Starwood, which was
acquired by Marriott in 2016 for \$13.6 billion. It was uncovered in
early September, when a security tool alerted Marriott officials to an
unauthorized attempt to access Starwood's guest reservation database.
The alert prompted Marriott to work with outside security experts, who
discovered that the hackers had grabbed a foothold in Starwood's systems
starting in 2014.

\href{https://www.nytimes.com/interactive/2017/technology/how-to-protect-data-online.html}{}

\includegraphics{https://static01.nyt.com/images/2017/09/07/technology/how-to-protect-data-online-1504825386336/how-to-protect-data-online-1504825386336-thumbLarge.jpg}

\hypertarget{after-marriott-breach-how-to-protect-your-information-online}{%
\subsection{After Marriott Breach, How to Protect Your Information
Online}\label{after-marriott-breach-how-to-protect-your-information-online}}

There are more reasons than ever to protect your personal information,
as major website breaches become ever more frequent.

On Nov. 19, digital forensics experts uncovered the full scope of the
attack. It was the second major security breach Starwood has reported.
Its cash register systems were penetrated in 2015.

The Federal Bureau of Investigation said in a statement that it was
aware of the breach and was tracking the situation. It added that any
suspected instances of identity theft should be reported to the F.B.I.'s
Internet Crime Complaint Center at
\href{https://www.google.com/url?q=http://www.ic3.gov\&sa=D\&source=hangouts\&ust=1543708313847000\&usg=AFQjCNEFY6mpufQ23K8VkSByfEN0fX59Jg}{www.ic3.gov}.

In recent years, cybersecurity experts said, the hospitality industry
has become a rich target for nation-state hackers looking to track the
travel movements and preferences of heads of states, diplomats, chief
executives and other people of interest to espionage agencies.

Going after hotel customer lists has been part of a broader effort to
obtain giant databases of information. So big, in fact, that they would
be of little use to run-of-the-mill hackers. But to a government, they
would be very useful.

That information could be fed, for example, into an
\href{https://www.nytimes.com/2018/11/29/us/politics/china-trump-cyberespionage.html}{analysis
program run by}a country's state security apparatus, Mr. Lewis said.
Using ``big data'' technology similar to what marketers use in targeted
advertising, the country could try to pinpoint the comings and going of
intelligence agents from other nations. Did they stay, for example, in
the same hotel as a potential source for that country?

The breach could get expensive for Marriott. Verizon cut what it paid to
acquire Yahoo by \$350 million after the internet company reported its
breach in 2016. And Equifax reported recovery costs of \$400 million
from its 2017 incident, which affected 148 million people.

Despite months of due diligence, finding out there was a major network
attack long after a deal closes is ``everybody's worst-case scenario,''
said Jake Olcott, vice president at BitSight, a computer security
ratings company in Boston.

Several lawsuits were filed against Marriott on Friday, and
investigations were announced by
\href{https://twitter.com/NewYorkStateAG/status/1068510072396029952}{New
York's attorney general}, Barbara D. Underwood, and European regulators.

In Europe, where companies can be fined up to 4 percent of global
revenue under data protection laws, companies must alert government
authorities within 72 hours of a known breach.

Given the volume and sensitivity of personal data taken, and the length
of the breach, Marriott ``has the potential to trigger the first hefty
G.D.P.R. fine,'' said Enza Iannopollo, a security analyst with Forrester
Research, referring to the European data protection law enacted this
year.

Marriott told shareholders that it did not expect the breach would
affect the company's long-term financial prospects. The company's share
price was down more than 5 percent on Friday.

Marriott has also been dealing with
\href{https://www.nytimes.com/2018/11/29/travel/marriott-strikes-hawaii-settle-not-an-francisco-.html}{strikes
by thousands of workers} in nine cities, as well as customer complaints
about problems
\href{https://www.nytimes.com/2018/10/08/business/marriott-hotel-reward-program-members.html}{with
rewards programs} after efforts to merge data from Starwood's rewards
program into Marriott's left the records of millions of customers in
limbo.

Lawmakers said the episode was yet another example of why the United
States needs data privacy laws that punish companies for failing to keep
customers' information private.

``It is past time we enact data security laws that ensure companies
account for security costs rather than making their consumers shoulder
the burden and harms resulting from these lapses,'' Senator Mark R.
Warner, a Democrat from Virginia, said in a statement.

Privacy advocates said there was no excuse for a breach to go unnoticed
for four years.

``They can say all they want that they take security seriously, but they
don't if you can be hacked over a four-year period without noticing,''
said Gus Hosein, executive director of Privacy International, a group
that supports strong data protection laws.

Advertisement

\protect\hyperlink{after-bottom}{Continue reading the main story}

\hypertarget{site-index}{%
\subsection{Site Index}\label{site-index}}

\hypertarget{site-information-navigation}{%
\subsection{Site Information
Navigation}\label{site-information-navigation}}

\begin{itemize}
\tightlist
\item
  \href{https://help.nytimes.com/hc/en-us/articles/115014792127-Copyright-notice}{©~2020~The
  New York Times Company}
\end{itemize}

\begin{itemize}
\tightlist
\item
  \href{https://www.nytco.com/}{NYTCo}
\item
  \href{https://help.nytimes.com/hc/en-us/articles/115015385887-Contact-Us}{Contact
  Us}
\item
  \href{https://www.nytco.com/careers/}{Work with us}
\item
  \href{https://nytmediakit.com/}{Advertise}
\item
  \href{http://www.tbrandstudio.com/}{T Brand Studio}
\item
  \href{https://www.nytimes.com/privacy/cookie-policy\#how-do-i-manage-trackers}{Your
  Ad Choices}
\item
  \href{https://www.nytimes.com/privacy}{Privacy}
\item
  \href{https://help.nytimes.com/hc/en-us/articles/115014893428-Terms-of-service}{Terms
  of Service}
\item
  \href{https://help.nytimes.com/hc/en-us/articles/115014893968-Terms-of-sale}{Terms
  of Sale}
\item
  \href{https://spiderbites.nytimes.com}{Site Map}
\item
  \href{https://help.nytimes.com/hc/en-us}{Help}
\item
  \href{https://www.nytimes.com/subscription?campaignId=37WXW}{Subscriptions}
\end{itemize}
