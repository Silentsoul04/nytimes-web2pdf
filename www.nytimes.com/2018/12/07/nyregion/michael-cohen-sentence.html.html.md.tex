Sections

SEARCH

\protect\hyperlink{site-content}{Skip to
content}\protect\hyperlink{site-index}{Skip to site index}

\href{https://www.nytimes.com/section/nyregion}{New York}

\href{https://myaccount.nytimes.com/auth/login?response_type=cookie\&client_id=vi}{}

\href{https://www.nytimes.com/section/todayspaper}{Today's Paper}

\href{/section/nyregion}{New York}\textbar{}Prosecutors Say Trump
Directed Illegal Payments During Campaign

\url{https://nyti.ms/2G5VWwD}

\begin{itemize}
\item
\item
\item
\item
\item
\item
\end{itemize}

Advertisement

\protect\hyperlink{after-top}{Continue reading the main story}

Supported by

\protect\hyperlink{after-sponsor}{Continue reading the main story}

\hypertarget{prosecutors-say-trump-directed-illegal-payments-during-campaign}{%
\section{Prosecutors Say Trump Directed Illegal Payments During
Campaign}\label{prosecutors-say-trump-directed-illegal-payments-during-campaign}}

\includegraphics{https://static01.nyt.com/images/2018/12/08/nyregion/08cohen/merlin_147476190_27005d3e-718c-4c8c-b993-1872d2e7f144-articleLarge.jpg?quality=75\&auto=webp\&disable=upscale}

By \href{https://www.nytimes.com/by/sharon-lafraniere}{Sharon
LaFraniere}, \href{https://www.nytimes.com/by/benjamin-weiser}{Benjamin
Weiser} and \href{https://www.nytimes.com/by/maggie-haberman}{Maggie
Haberman}

\begin{itemize}
\item
  Dec. 7, 2018
\item
  \begin{itemize}
  \item
  \item
  \item
  \item
  \item
  \item
  \end{itemize}
\end{itemize}

Federal prosecutors said on Friday that President Trump directed illegal
payments to ward off a potential sex scandal that threatened his chances
of winning the White House in 2016, putting the weight of the Justice
Department behind accusations previously made by his former lawyer.

The lawyer, Michael D. Cohen, had said that as the election neared, Mr.
Trump directed payments to two women who claimed they had affairs with
Mr. Trump. But in a new memo arguing for a prison term for Mr. Cohen,
prosecutors in Manhattan said he ``acted in coordination and at the
direction of'' an unnamed individual, clearly referring to Mr. Trump.

In another filing, prosecutors for the special counsel investigating
Russia's 2016 election interference said an unnamed Russian offered Mr.
Cohen ``government level'' synergy between Russia and Mr. Trump's
campaign in November 2015. That was months earlier than other approaches
detailed in indictments secured by prosecutors.

And in a separate case on Friday, the special counsel accused Paul
Manafort, Mr. Trump's campaign chairman, of lying about his contacts
with an individual they accuse of having ties to Russian intelligence,
and about his interactions with Trump administration officials after he
was indicted on criminal charges.

Together, the filings laid bare the most direct evidence to date linking
Mr. Trump to potentially criminal conduct, and added to an already
substantial case that Russia was seeking to sway the 2016 election in
his favor.

Mr. Trump sought on Friday to dismiss the news,
\href{https://twitter.com/realDonaldTrump/status/1071177621445230596}{claiming}
it ``Totally clears the President. Thank you!''

The White House press secretary, Sarah Huckabee Sanders, was less
unequivocal. ``The government's filings in Mr. Cohen's case tell us
nothing of value that wasn't already known,'' she said in a statement.
``Mr. Cohen has repeatedly lied and as the prosecution has pointed out
to the court, Mr. Cohen is no hero.''

She tried to distance Mr. Trump from the accusations against Mr.
Manafort, who was convicted on financial fraud and conspiracy charges
unrelated to his work for the Trump campaign. President Trump has
repeatedly defended Mr. Manafort as a ``brave man'' and dangled the
possibility of a pardon for his 10 felonies, likely to result in a
prison term of at least 10 years.

The revelations came in multiple filings by federal prosecutors for the
Southern District of New York and by the special counsel, Robert S.
Mueller III. Their work has intersected because both teams have charged
Mr. Cohen with crimes, and he had sought to cooperate with both.

{[}\href{https://int.nyt.com/data/documenthelper/516-michael-cohen-manhattan/d85a4cc24e25b7ecf4ab/optimized/full.pdf\#page=1}{\emph{Read
the Southern District of New York's memo.}}{]}

{[}\href{https://int.nyt.com/data/documenthelper/517-michael-cohen-congress/d85a4cc24e25b7ecf4ab/optimized/full.pdf\#page=1}{\emph{Read
the special counsel office's memo}}\emph{.}{]}

The prosecutors in New York mounted a scathing attack on Mr. Cohen's
character. They rejected his plea to avoid a prison term, saying that he
had ``repeatedly used his power and influence for deceptive ends.''

They argued that he deserved a ``substantial'' prison term that, giving
him some credit for his cooperation, could amount to just under four
years. ``His offenses strike at several pillars of our society and
system of government: the payment of taxes; transparent and fair
elections; and truthfulness before government and in business,'' they
wrote.

Mr. Cohen, 52, is to be sentenced next week for campaign finance
violations, financial crimes and lying to Congress about the extent of
Mr. Trump's business dealings in Russia.

Mr. Cohen's crimes marked ``a pattern of deception that permeated his
professional life,'' the Manhattan prosecutors wrote, saying that he did
not deserve much leniency in exchange for his cooperation.

``The sentence imposed should reflect the seriousness of Cohen's brazen
violations of the election laws and attempt to counter the public
cynicism that may arise when individuals like Cohen act as if the
political process belongs to the rich and powerful,'' they said, adding
that he ``sought to influence the election from the shadows.''

They emphasized that Mr. Cohen had implicated the president in payments
to two women during the campaign to conceal affairs that they said they
had with Mr. Trump. ``Cohen himself has now admitted, with respect to
both payments, he acted in coordination with and at the direction of
Individual-1,'' the prosecutors wrote. ``Individual-1'' is how Mr. Trump
is referred to in the document.

\includegraphics{https://static01.nyt.com/images/2019/01/11/us/politics/11dc-cohen-print/11dc-cohen-videoSixteenByNineJumbo1600.jpg}

The prosecutors have said that a \$130,000 payment to Stormy Daniels, a
pornographic film actress, violated campaign finance law prohibitions
against donations of more than \$2,700 in a general election. A
\$150,000 payment by American Media Inc. to silence Karen McDougal, a
former Playboy model, constituted an illegal corporate donation to Mr.
Trump's campaign, the prosecutors said.

The special counsel's prosecutors offered a somewhat more positive view
of Mr. Cohen than the New York team, saying he went ``to significant
lengths to assist'' their inquiry, including by providing relevant
information he had learned from Trump Organization executives during the
campaign.

They cited a series of disclosures during their seven meetings with Mr.
Cohen. He revealed to them that he talked with Mr. Trump about meeting
President Vladimir V. Putin of Russia during Mr. Putin's trip to New
York for a United Nations session in September 2015. That was three
months after Mr. Trump had declared his candidacy for president, at a
time when Mr. Cohen was aggressively pursuing the building of a Trump
hotel in Moscow that could generate hundreds of millions of dollars for
the Trump Organization.

After conferring with Mr. Trump, Mr. Cohen said, he reached out ``to
gauge Russia's interest in such a meeting.'' It ultimately did not take
place.

Two months later, Mr. Cohen said, he was approached by a Russian
claiming to be a ```trusted person' in the Russian Federation.'' The
individual, who was not named, offered ``synergy on a government level''
with the Trump campaign. He pushed for a meeting between Mr. Trump and
Mr. Putin to discuss politics and the proposed hotel, saying Mr. Putin's
consent was the biggest ``warranty'' for any project.

That appears to be the earliest known contact between an aide to Mr.
Trump and a Russian offering to help Mr. Trump's campaign. The timing of
the interaction matched one disclosed earlier this year
\href{https://www.buzzfeednews.com/article/anthonycormier/ivanka-trump-putin-moscow-meeting-michael-cohen-tower}{by
BuzzFeed}.

Mr. Cohen said he never followed up on the Russian's invitation, in part
because he was working with someone else who he believed had Kremlin
connections. He also ``provided information about attempts by other
Russian nationals to reach the campaign'' and about his own interactions
with Russian officials who might have tried to use the business proposal
as leverage over Mr. Trump.

He also revealed that
\href{https://www.nytimes.com/2018/11/29/nyregion/michael-cohen-trump-russia-mueller.html?module=inline}{Mr.
Trump had been more involved in discussions} about the venture than was
previously known. Although Mr. Cohen testified to Congress that
negotiations ended in January 2016, before the first Republican
presidential primary, he told prosecutors that those discussions had
continued until June of that year, just before Mr. Trump won the
Republican nomination and only five months before the election.

Before he testified falsely to Congress about the Moscow hotel project,
Mr. Cohen has admitted, he consulted with White House staff members and
Mr. Trump's legal team. The prosecutors characterized his information
about his contacts with people tied to the White House over the past two
years as ``relevant and useful information.''

Mr. Cohen has said he lied out of loyalty to Mr. Trump and to be
consistent with the president's ``political messaging.'' His lawyers,
Guy Petrillo and Amy Lester, have asked Judge William H. Pauley III
\href{https://www.nytimes.com/2018/12/01/nyregion/michael-cohen-leniency.html}{to
allow Mr. Cohen to avoid a prison sentence}, saying he cooperated with
prosecutors even though he never signed a formal agreement.

They portrayed him as a remorseful man whose
\href{https://www.nytimes.com/2018/12/03/nyregion/michael-cohen-trump-strategy.html}{life
had been shattered by his relationship with Mr. Trump}, who has lost
friends and associates and who wanted to come clean so he could begin
his life anew.

Under federal guidelines, Mr. Cohen faces about four to five years in
the Manhattan case, and up to six months in Mr. Mueller's case. The
special counsel suggested that any prison terms be served concurrently.
But the guidelines are not binding, and Judge Pauley will decide his
punishment.

Mr. Trump has accused Mr. Cohen of lying to prosecutors in hope of a
lighter sentence. On Friday, he kept up his continuing effort to
undermine public trust in the special counsel's office and in the
Justice Department. In a series of Twitter messages, he derided Mr.
Mueller as a friend of James B. Comey, the former F.B.I. director who
said the president fired him in May 2017 after demanding ``loyalty.''

Asked why the president was so upset about the special counsel, Roger J.
Stone Jr., a fellow critic of Mr. Mueller, said it had dawned on Mr.
Trump that the inquiry was not going away, his lawyers' promises
notwithstanding. ``He has finally figured out that this is about him,''
he said.

Spurred on by the White House, House Republican lawmakers used their
last days in control of the chamber's majority to press Mr. Comey Friday
on what they claim is a pattern of abuse of power by the F.B.I. Mr.
Comey reluctantly agreed to testify behind closed doors before two House
committees --- Judiciary and Oversight and Government Reform --- after
they agreed to release a public transcript of his testimony.

Lawmakers emerged frustrated that Mr. Comey had not been allowed to
answer questions about classified matters involving the Mueller
investigation, in which he is a central witness. Mr. Trump
\href{https://twitter.com/realDonaldTrump/status/1071159669949911044}{took
up their complaints}, tweeting that Justice Department lawyers had
demonstrated ``total bias and corruption at the highest levels of
previous Administration.''

Emerging after six hours of questioning, Mr. Comey defended the F.B.I.'s
actions in the Russia case, and accused Republicans of a misguided
preoccupation with the bureau's handling its 2016 inquiry into Hillary
Clinton's use of a private email server.

Advertisement

\protect\hyperlink{after-bottom}{Continue reading the main story}

\hypertarget{site-index}{%
\subsection{Site Index}\label{site-index}}

\hypertarget{site-information-navigation}{%
\subsection{Site Information
Navigation}\label{site-information-navigation}}

\begin{itemize}
\tightlist
\item
  \href{https://help.nytimes.com/hc/en-us/articles/115014792127-Copyright-notice}{©~2020~The
  New York Times Company}
\end{itemize}

\begin{itemize}
\tightlist
\item
  \href{https://www.nytco.com/}{NYTCo}
\item
  \href{https://help.nytimes.com/hc/en-us/articles/115015385887-Contact-Us}{Contact
  Us}
\item
  \href{https://www.nytco.com/careers/}{Work with us}
\item
  \href{https://nytmediakit.com/}{Advertise}
\item
  \href{http://www.tbrandstudio.com/}{T Brand Studio}
\item
  \href{https://www.nytimes.com/privacy/cookie-policy\#how-do-i-manage-trackers}{Your
  Ad Choices}
\item
  \href{https://www.nytimes.com/privacy}{Privacy}
\item
  \href{https://help.nytimes.com/hc/en-us/articles/115014893428-Terms-of-service}{Terms
  of Service}
\item
  \href{https://help.nytimes.com/hc/en-us/articles/115014893968-Terms-of-sale}{Terms
  of Sale}
\item
  \href{https://spiderbites.nytimes.com}{Site Map}
\item
  \href{https://help.nytimes.com/hc/en-us}{Help}
\item
  \href{https://www.nytimes.com/subscription?campaignId=37WXW}{Subscriptions}
\end{itemize}
