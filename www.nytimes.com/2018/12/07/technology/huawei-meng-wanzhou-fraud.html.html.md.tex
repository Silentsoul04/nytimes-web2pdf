Sections

SEARCH

\protect\hyperlink{site-content}{Skip to
content}\protect\hyperlink{site-index}{Skip to site index}

\href{https://www.nytimes.com/section/technology}{Technology}

\href{https://myaccount.nytimes.com/auth/login?response_type=cookie\&client_id=vi}{}

\href{https://www.nytimes.com/section/todayspaper}{Today's Paper}

\href{/section/technology}{Technology}\textbar{}Huawei Executive Took
Part in Sanctions Fraud, Prosecutors Say

\url{https://nyti.ms/2zLmoGg}

\begin{itemize}
\item
\item
\item
\item
\item
\end{itemize}

Advertisement

\protect\hyperlink{after-top}{Continue reading the main story}

Supported by

\protect\hyperlink{after-sponsor}{Continue reading the main story}

\hypertarget{huawei-executive-took-part-in-sanctions-fraud-prosecutors-say}{%
\section{Huawei Executive Took Part in Sanctions Fraud, Prosecutors
Say}\label{huawei-executive-took-part-in-sanctions-fraud-prosecutors-say}}

\includegraphics{https://static01.nyt.com/images/2018/12/08/business/08huaweihearings03/merlin_147853938_3fe9765e-46a8-464c-a036-cc607f3e2fac-articleLarge.jpg?quality=75\&auto=webp\&disable=upscale}

By \href{https://www.nytimes.com/by/kate-conger}{Kate Conger}

\begin{itemize}
\item
  Dec. 7, 2018
\item
  \begin{itemize}
  \item
  \item
  \item
  \item
  \item
  \end{itemize}
\end{itemize}

\href{https://cn.nytimes.com/technology/20181210/huawei-meng-wanzhou-fraud/}{阅读简体中文版}\href{https://cn.nytimes.com/technology/20181210/huawei-meng-wanzhou-fraud/zh-hant/}{閱讀繁體中文版}

VANCOUVER, British Columbia --- The reasons that the United States asked
the Canadian authorities to
\href{https://www.nytimes.com/2018/12/05/business/huawei-cfo-arrest-canada-extradition.html?module=inline}{arrest
a top executive} of the Chinese technology company Huawei last week had
been shrouded in mystery.

On Friday, the details of the arrest and what led up to it came out in a
Canadian courtroom.

At a bail hearing in Vancouver for Meng Wanzhou, the chief financial
officer of Huawei and a daughter of the company's founder, Canadian
prosecutors said she was accused of fraud. The heart of the charges
related to how Ms. Meng may have participated in a scheme to trick
financial institutions into making transactions that violated United
States sanctions against Iran, they said.

Ms. Meng had ``direct involvement'' with Huawei's representations to
banks, said John Gibb-Carsley, an attorney with Canada's Justice
Department.

The hearing shed light on an incident that has
\href{https://www.nytimes.com/2018/12/07/world/asia/huawei-arrest-china.html?action=click\&module=Top\%20Stories\&pgtype=Homepage}{rattled
the relationship} between the United States and China as they prepare to
enter negotiations to cease a brutal trade war. While changing planes in
Vancouver on Dec. 1, Ms. Meng was arrested at the behest of the United
States, which has for years looked into potential ties between
\href{https://www.nytimes.com/2018/12/06/technology/huawei-arrest-meng-wanzhou.html}{Huawei}
and the Chinese government or Communist Party.

Because of Ms. Meng's stature in China as a top executive and part of
its elite, news of her arrest has rippled through the country.

A Huawei spokesman said late Friday, ``We have every confidence that the
Canadian and U.S. legal systems will reach the right conclusion.''

With Ms. Meng, 46, seated inside a glass box at British Columbia's
Supreme Court, Mr. Gibb-Carsley laid out what had led to her arrest. He
said that between 2009 and 2014, Huawei used a Hong Kong company, Skycom
Tech, to make transactions in Iran and do business with telecom
companies there, in violation of American sanctions. Banks in the United
States cleared financial transactions for Huawei, inadvertently doing
business with Skycom, he said.

The banks were ``victim institutions'' of fraud by Ms. Meng, Mr.
Gibb-Carsley said.

In 2013,
\href{https://www.reuters.com/article/us-huawei-skycom/exclusive-huawei-cfo-linked-to-firm-that-offered-hp-gear-to-iran-idUSBRE90U0CC20130131}{articles
by Reuters} alleged that Huawei used Skycom to do business in Iran, and
had tried to import American-made computer equipment into the country in
violation of sanctions. Several financial institutions asked Huawei if
the allegations were true, Mr. Gibb-Carsley said.

At the time, Ms. Meng arranged a meeting with an executive from one of
the financial institutions, he said. During the meeting, she spoke
through an English interpreter and presented PowerPoint slides in
Chinese, saying that Huawei operated in Iran in strict compliance with
United States sanctions. Ms. Meng explained that Huawei's engagement
with Skycom was part of normal business operations and that Huawei had
sold the shares it once held in Skycom.

But there was no distinction between Skycom and Huawei, Mr. Gibb-Carsley
said. Huawei operated Skycom as an unofficial subsidiary, making efforts
to keep the connection between the companies secret.

Skycom employees used Huawei email addresses and had badges and a
letterhead featuring the Huawei logo, he said. Skycom documents showed
that an entity to which the company was sold in 2009 was also controlled
by Huawei until at least 2014, according to an affidavit read in court.

Ms. Meng's presentation to the financial institution constituted fraud,
Mr. Gibb-Carsley said. Her attorney, David Martin, said the bank was
HSBC.

A global bank based in London with operations in the United States, HSBC
has repeatedly landed in trouble with the American authorities for
violating anti-money-laundering rules. As a result, HSBC had officials
from a consulting firm, Exiger, stationed inside the bank to monitor its
compliance. Exiger officials noticed suspicious Iranian-linked
transactions involving Huawei and flagged them to the United States
Justice Department, according to two people familiar with the matter who
weren't authorized to speak publicly.

Mr. Martin said Ms. Meng's PowerPoint presentation to HSBC had been
prepared by Huawei's legal team and disclosed the sale of Skycom. He
added that the American government had provided only a ``skeletal
description'' of the accusations.

Stuart Levey, the chief legal officer at HSBC, said, ``The U.S.
Department of Justice has confirmed that HSBC is not under investigation
in this case.''

Marc Raimondi, a spokesman for the Justice Department, declined to
comment on the charges revealed in court on Friday.

The accusations against Ms. Meng and Huawei are similar to ones that the
United States government made in 2016 against ZTE, another large Chinese
technology company. In that case, American
\href{https://www.nytimes.com/2016/03/19/technology/zte-document-raises-questions-about-huawei-and-sanctions.html?module=inline}{officials
released internal ZTE documents} in which executives had described
creating ``cutoff companies'' that would do business with Iran, North
Korea and other nations placed under sanctions by the American
government.

A warrant for Ms. Meng's arrest was issued in the Eastern District of
New York on Aug. 22, Mr. Gibb-Carsley said. A Canadian justice then
issued a warrant for Ms. Meng on Nov. 30 after it became known that she
would change planes in Vancouver on her way from Hong Kong to Mexico.

Ms. Meng had traveled to the United States regularly in 2014, 2015,
2016, Mr. Gibb-Carsley said. Her last trip was in February and March
2017. In April 2017, Huawei found out about the United States
investigation into the company when its subsidiaries were served with a
grand jury subpoena, he said.

Ms. Meng and other executives then stopped visiting the United States,
even though she has a 16-year-old son --- one of three sons from a
previous marriage --- at a school in Boston, Mr. Gibb-Carsley said. Ms.
Meng, who remarried, also has a daughter, according to an affidavit.

Mr. Martin pushed back on the idea that Ms. Meng had avoided the United
States because she feared prosecution. After
\href{https://www.nytimes.com/2018/01/09/business/att-huawei-mate-smartphone.html}{AT\&T
canceled a deal to distribute Huawei smartphones} in the United States
in January and the federal government banned the use of Huawei products
in government contracts, Huawei essentially abandoned the market, Mr.
Martin said.

``An entity would have to be tone deaf to not understand that the United
States had become a hostile place for Huawei to do business,'' he said.

Mr. Gibb-Carsley argued against bail for Ms. Meng. He said that she had
vast financial resources and no strong ties in Canada, and that China
had no extradition treaty with the United States or Canada.

Mr. Martin offered two properties in Vancouver and a cash deposit to
secure Ms. Meng's bail. Ms. Meng would not breach a court order, he
said, adding that doing so would ``humiliate and embarrass her father,
who she loves,'' and embarrass Huawei's thousands of employees. Ms.
Meng's father is Ren Zhengfei, Huawei's founder.

``She would not embarrass China itself,'' Mr. Martin said. When he later
repeated that point, Ms. Meng touched her eyes and the sheriff gave her
tissues.

By the end of the day, no bail had been set, and the judge in the case
said the hearing would continue on Monday morning.

Any extradition process can take weeks or months, depending on the rules
of the country that arrests the suspect and whether the suspect chooses
to fight the extradition request. The United States Justice Department
must now present evidence to the Canadian court that supports its
request and has 60 days from the arrest to make a full request for
extradition.

Advertisement

\protect\hyperlink{after-bottom}{Continue reading the main story}

\hypertarget{site-index}{%
\subsection{Site Index}\label{site-index}}

\hypertarget{site-information-navigation}{%
\subsection{Site Information
Navigation}\label{site-information-navigation}}

\begin{itemize}
\tightlist
\item
  \href{https://help.nytimes.com/hc/en-us/articles/115014792127-Copyright-notice}{©~2020~The
  New York Times Company}
\end{itemize}

\begin{itemize}
\tightlist
\item
  \href{https://www.nytco.com/}{NYTCo}
\item
  \href{https://help.nytimes.com/hc/en-us/articles/115015385887-Contact-Us}{Contact
  Us}
\item
  \href{https://www.nytco.com/careers/}{Work with us}
\item
  \href{https://nytmediakit.com/}{Advertise}
\item
  \href{http://www.tbrandstudio.com/}{T Brand Studio}
\item
  \href{https://www.nytimes.com/privacy/cookie-policy\#how-do-i-manage-trackers}{Your
  Ad Choices}
\item
  \href{https://www.nytimes.com/privacy}{Privacy}
\item
  \href{https://help.nytimes.com/hc/en-us/articles/115014893428-Terms-of-service}{Terms
  of Service}
\item
  \href{https://help.nytimes.com/hc/en-us/articles/115014893968-Terms-of-sale}{Terms
  of Sale}
\item
  \href{https://spiderbites.nytimes.com}{Site Map}
\item
  \href{https://help.nytimes.com/hc/en-us}{Help}
\item
  \href{https://www.nytimes.com/subscription?campaignId=37WXW}{Subscriptions}
\end{itemize}
