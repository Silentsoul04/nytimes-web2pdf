Sections

SEARCH

\protect\hyperlink{site-content}{Skip to
content}\protect\hyperlink{site-index}{Skip to site index}

\href{https://www.nytimes.com/section/technology}{Technology}

\href{https://myaccount.nytimes.com/auth/login?response_type=cookie\&client_id=vi}{}

\href{https://www.nytimes.com/section/todayspaper}{Today's Paper}

\href{/section/technology}{Technology}\textbar{}Meng Wanzhou Was
Huawei's Professional Face, Until Her Arrest

\url{https://nyti.ms/2zKyfnF}

\begin{itemize}
\item
\item
\item
\item
\item
\end{itemize}

Advertisement

\protect\hyperlink{after-top}{Continue reading the main story}

Supported by

\protect\hyperlink{after-sponsor}{Continue reading the main story}

\hypertarget{meng-wanzhou-was-huaweis-professional-face-until-her-arrest}{%
\section{Meng Wanzhou Was Huawei's Professional Face, Until Her
Arrest}\label{meng-wanzhou-was-huaweis-professional-face-until-her-arrest}}

\includegraphics{https://static01.nyt.com/images/2018/12/07/business/00meng-01/00meng-01-articleLarge.jpg?quality=75\&auto=webp\&disable=upscale}

By \href{https://www.nytimes.com/by/raymond-zhong}{Raymond Zhong}

\begin{itemize}
\item
  Dec. 7, 2018
\item
  \begin{itemize}
  \item
  \item
  \item
  \item
  \item
  \end{itemize}
\end{itemize}

\href{https://cn.nytimes.com/technology/20181210/meng-wanzhou-huawei-arrest/}{阅读简体中文版}\href{https://cn.nytimes.com/technology/20181210/meng-wanzhou-huawei-arrest/zh-hant/}{閱讀繁體中文版}

BEIJING --- As Huawei's finance chief and daughter of its founder, Meng
Wanzhou has been a polished, professional face for a huge technology
firm that long was opaque to the outside world.

Ms. Meng, the eldest daughter of Ren Zhengfei, the leader of the telecom
equipment maker, has appeared before reporters to announce the company's
financial results. She has spoken at company events in New York; Cancún,
Mexico; and beyond. She helped inaugurate centers in Britain, a key
market for the Chinese giant's expansion into the Western world.

She also sat on the board of a Huawei partner company in Hong Kong
called Skycom Tech that Canadian authorities now say did business in
Iran. And through that position and her job at Huawei, Ms. Meng may have
personally been involved in tricking financial institutions into making
transactions that violated United States sanctions against Iran, they
said.

That has thrust Ms. Meng, 46, into the center of what promises to be a
complex diplomatic tussle between the United States and China. She was
\href{https://www.nytimes.com/2018/12/05/business/huawei-cfo-arrest-canada-extradition.html}{arrested
Dec. 1} in Vancouver, Canada, while changing flights, at the request of
the American government, which is seeking to extradite her. The action
escalated what had already been a roller-coaster year of economic
conflict between the two powers, ahead of tricky negotiations to end a
brutal trade war.

Huawei has said that it is not aware of any wrongdoing by Ms. Meng, and
China's Foreign Ministry has called for her immediate release.

But on Friday, in a bail hearing in British Columbia's Supreme Court,
Canadian authorities said
\href{https://www.nytimes.com/2018/12/07/technology/huawei-meng-wanzhou-fraud.html}{Ms.
Meng was accused of fraud}. They said she had ``direct involvement''
with Huawei's representations to banks, telling at least one financial
executive that Huawei and Skycom were operating in Iran in strict
compliance with United States sanctions when that was not the case.

Larry Kudlow, director of the White House's National Economic Council,
said on CNBC on Friday that the
\href{https://www.nytimes.com/2018/12/07/us/politics/white-house-kudlow-huawei-arrest.html?action=click\&module=Top\%20Stories\&pgtype=Homepage}{United
States had repeatedly warned Huawei} about violating sanctions on Iran.

``We have these sanctions on Iran, it runs against our policy, why
shouldn't we enforce that?'' he said.

For years, Ms. Meng's name has appeared in connection with Huawei's
business in Iran, the subject of a yearslong
\href{https://www.nytimes.com/2017/04/26/business/huawei-investigation-sanctions-subpoena.html}{United
States investigation}.

\includegraphics{https://static01.nyt.com/images/2018/12/07/business/00meng-02/00meng-02-articleLarge.jpg?quality=75\&auto=webp\&disable=upscale}

Reuters reported several years ago that Skycom, one of Huawei's partners
in that country, had tried to sell Hewlett-Packard equipment to an
Iranian telecom carrier in 2010. The sale, which Huawei said was never
completed, would have violated Washington's ban on exporting computer
products to Iran.

Huawei said at the time that its Iranian business was entirely lawful,
and that it required its local partners to heed the same laws and
regulations.

According to Hong Kong corporate filings, Ms. Meng was a member of
Skycom's board from February 2008 to April 2009.

In a May 2007 filing, Skycom reported that all of the company's shares
had been transferred that year to a Hong Kong company called Hua Ying
Management. In August 2007, Hua Ying reported to the Hong Kong
authorities that its company secretary was Ms. Meng.

Ms. Meng, who started at Huawei as a secretary 25 years ago, is not its
most prominent executive. But as chief financial officer, she has played
a part in the company's efforts over the past five years to become more
transparent about its operations. After United States lawmakers labeled
Huawei and another Chinese manufacturer, ZTE, as security threats,
Huawei saw openness as a way to help dispel the swirl of suspicions
surrounding it.

Some of the distrust has had to do with Ms.
Meng's\href{https://www.nytimes.com/2013/04/18/technology/succession-at-huawei-offers-glimpse-into-secretive-firm.html}{powerful
and secretive father}.

Mr. Ren, 74, was a member of the Chinese military's engineering corps
for nearly a decade before starting Huawei in 1987. His military service
has informed American officials' concerns that Huawei has links to the
Chinese government or the Communist Party --- something the company has
strenuously denied.

``She's very presentable,'' Duncan Clark, the chairman of the advisory
firm BDA China, who once did consulting work for Huawei, said of Ms.
Meng.

That is a stark contrast with her father, Mr. Clark added. ``He is, for
me at least, refreshingly unpolished and direct.''

For many people in China, Huawei represents how far their nation has
come since it began climbing out of the economic ravages left by
Chairman Mao --- and how far it can continue to go.

Over the past three decades, Huawei has transformed from a small maker
of telephone switches into the world's largest supplier of
telecommunications equipment, as well as the No. 2 smartphone maker,
behind Samsung. The company has worked to
\href{https://www.nytimes.com/2017/11/19/technology/huawei-mate-10-smartphone.html}{build
a consumer brand} associated with quality and innovation. The name
``Huawei'' means ``China's Achievement.''

But after winning over cellular providers across the developing world
with its cost-effective networking gear, the company faced a tougher
task convincing large carriers in the wealthier nations of Europe and
North America.

Image

Huawei headquarters in Shenzhen, China. After United States lawmakers
labeled Huawei as a security threat, the company saw openness as a way
to help dispel the swirl of suspicions surrounding it.Credit...Dake
Kang/Associated Press

For many years, Mr. Ren's reluctance to appear in public, combined with
the company's aversion to the news media, even after it had become a
globe-straddling giant, fed the impression that he and Huawei had
something to hide. How much of the company did he own? How did key
decisions get made? Could there really be a military link?

Ms. Meng became part of an attempt to address such issues in January
2013, when she was
\href{https://blogs.wsj.com/chinarealtime/2013/01/21/a-confusing-debut-for-daughter-of-mysterious-huawei-founder/}{brought
before reporters} in Beijing to discuss Huawei's business outlook. The
company, which is privately held, had published some financial details
before. But it had never held a news conference of this kind.

``We will honor our commitment to transparency and openness,'' Ms. Meng
said then.

This commitment has failed to persuade the United States government that
Huawei's products are safe to use, but the company has become a supplier
to
\href{https://www.nytimes.com/2012/10/12/business/global/huawei-chinese-telecom-company-finds-warmer-welcome-in-europe.html}{many
of Europe's telecom providers}.

This year, Ms. Meng was made Huawei's deputy chairwoman in addition to
finance chief, leading some to wonder whether she might succeed her
father at the top someday. But hers was not an heiress's upbringing.

Ms. Meng, who also uses the names Sabrina and Cathy, was born in 1972 in
the western city of Chengdu, to Mr. Ren's first wife, Meng Jun. The
family moved to Shenzhen, in China's south, during the turbulent
economic reforms of the 1980s.

Shenzhen eventually became a hub of China's mighty manufacturing base
and home to Huawei's global headquarters. Back then, it was a backwater.

As Ms. Meng later recalled in a Huawei employee newspaper, the walls of
the family's house let in all the neighbors' chatter. The roof leaked.
When it rained --- which it did constantly in southern China ---
everything got wet.

After college, Ms. Meng hoped to attend graduate school in the United
States. A university gave her an offer, she recalled in a 2016 speech.
But her visa was rejected because an American consular interviewer
decided that her English was too poor.

Ms. Meng found a job at a bank instead. She was laid off after a year.
In 1993, she joined fledgling Huawei as one of its three secretaries.

She answered the phone, printed out documents and put together product
catalogs. A few years later, after completing a master's degree in
management, she returned to Huawei, this time in the finance department.
And she began climbing the ladder.

As Huawei's business spread across the world in the 2000s, Ms. Meng
helped expand its accounting operations with it. Her brother, Ren Ping,
works for a Huawei-owned company. Annabel Yao, a daughter of the elder
Mr. Ren by his second wife, is an undergraduate at Harvard.

Advertisement

\protect\hyperlink{after-bottom}{Continue reading the main story}

\hypertarget{site-index}{%
\subsection{Site Index}\label{site-index}}

\hypertarget{site-information-navigation}{%
\subsection{Site Information
Navigation}\label{site-information-navigation}}

\begin{itemize}
\tightlist
\item
  \href{https://help.nytimes.com/hc/en-us/articles/115014792127-Copyright-notice}{©~2020~The
  New York Times Company}
\end{itemize}

\begin{itemize}
\tightlist
\item
  \href{https://www.nytco.com/}{NYTCo}
\item
  \href{https://help.nytimes.com/hc/en-us/articles/115015385887-Contact-Us}{Contact
  Us}
\item
  \href{https://www.nytco.com/careers/}{Work with us}
\item
  \href{https://nytmediakit.com/}{Advertise}
\item
  \href{http://www.tbrandstudio.com/}{T Brand Studio}
\item
  \href{https://www.nytimes.com/privacy/cookie-policy\#how-do-i-manage-trackers}{Your
  Ad Choices}
\item
  \href{https://www.nytimes.com/privacy}{Privacy}
\item
  \href{https://help.nytimes.com/hc/en-us/articles/115014893428-Terms-of-service}{Terms
  of Service}
\item
  \href{https://help.nytimes.com/hc/en-us/articles/115014893968-Terms-of-sale}{Terms
  of Sale}
\item
  \href{https://spiderbites.nytimes.com}{Site Map}
\item
  \href{https://help.nytimes.com/hc/en-us}{Help}
\item
  \href{https://www.nytimes.com/subscription?campaignId=37WXW}{Subscriptions}
\end{itemize}
