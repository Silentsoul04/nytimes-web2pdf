Sections

SEARCH

\protect\hyperlink{site-content}{Skip to
content}\protect\hyperlink{site-index}{Skip to site index}

\href{https://www.nytimes.com/section/politics}{Politics}

\href{https://myaccount.nytimes.com/auth/login?response_type=cookie\&client_id=vi}{}

\href{https://www.nytimes.com/section/todayspaper}{Today's Paper}

\href{/section/politics}{Politics}\textbar{}Nick Ayers, Aide to Pence,
Declines Offer to Be Trump's Chief of Staff

\url{https://nyti.ms/2G7apbn}

\begin{itemize}
\item
\item
\item
\item
\item
\item
\end{itemize}

Advertisement

\protect\hyperlink{after-top}{Continue reading the main story}

Supported by

\protect\hyperlink{after-sponsor}{Continue reading the main story}

\hypertarget{nick-ayers-aide-to-pence-declines-offer-to-be-trumps-chief-of-staff}{%
\section{Nick Ayers, Aide to Pence, Declines Offer to Be Trump's Chief
of
Staff}\label{nick-ayers-aide-to-pence-declines-offer-to-be-trumps-chief-of-staff}}

\includegraphics{https://static01.nyt.com/images/2018/12/09/us/10dc-ayers/10dc-ayers-articleLarge-v4.jpg?quality=75\&auto=webp\&disable=upscale}

By \href{https://www.nytimes.com/by/maggie-haberman}{Maggie Haberman}

\begin{itemize}
\item
  Dec. 9, 2018
\item
  \begin{itemize}
  \item
  \item
  \item
  \item
  \item
  \item
  \end{itemize}
\end{itemize}

As President Trump heads into the fight of his political life, the man
he had hoped would help guide him through it has now turned him down,
and he finds himself in the unaccustomed position of having no obvious
second option.

\href{https://www.nytimes.com/2018/11/21/us/politics/nick-ayers-white-house.html}{Nick
Ayers}, the main focus of President Trump's search to replace John F.
Kelly as chief of staff in recent weeks, said on Sunday that he was
leaving the administration at the end of the year. Mr. Ayers, 36, the
chief of staff to Vice President Mike Pence, is returning to Georgia
with his wife and three young children, according to people familiar
with his plans.

The decision leaves Mr. Trump to contend with fresh uncertainty as he
enters the 2020 campaign amid growing danger from the Russia
investigation and from Democrats who have vowed tougher oversight, and
could even pursue impeachment, after they take over the House next
month.

As the president hastily restarted the search process, speculation
focused on a group that was led by Representative Mark Meadows, a North
Carolina Republican who is the hard-edge chairman of the conservative
House Freedom Caucus, but also included the Treasury secretary, Steven
Mnuchin; Mr. Trump's budget director, Mick Mulvaney; and the United
States trade representative, Robert Lighthizer.

Former Gov. Chris Christie of New Jersey, who as a onetime United States
attorney could help Mr. Trump in an impeachment fight, was also being
mentioned. And some Trump allies were pushing for David N. Bossie, the
deputy campaign manager in 2016.

Mr. Trump's ultimate choice will be faced with a president who the two
previous chiefs of staff found nearly impossible to manage. But Mr.
Meadows, for instance, could still aid Mr. Trump in the coming political
battle with congressional leaders, despite his own frayed relationships
on Capitol Hill. Weeks ago, Mr. Trump started asking people what they
would think of Mr. Meadows, a fierce supporter of the president, as a
chief of staff, before moving on to Mr. Ayers.

The president on Sunday disputed news reports that he had settled on Mr.
Ayers as his pick. ``I am in the process of interviewing some really
great people for the position of White House Chief of Staff,''
\href{https://twitter.com/realDonaldTrump/status/1071939400517500929}{he
said on Twitter}. ``Fake News has been saying with certainty it was Nick
Ayers, a spectacular person who will always be with our \#MAGA agenda. I
will be making a decision soon!''

But two people close to Mr. Trump said that a news release announcing
Mr. Ayers's appointment had been drafted, and that the president had
wanted to announce it as soon as possible.

Kellyanne Conway, an adviser to the president, said Mr. Ayers's ``unique
qualification was that he had been doing the same job for the vice
president.'' But ``those of us with young kids very well understand the
personal decision he made,'' she said.

Other advisers to Mr. Trump were stunned by the turn of events. One
former senior administration official called it a humiliation for Mr.
Trump and his adult children, an emotion that the president tries to
avoid at all costs.

For more than six months, Mr. Ayers had been viewed as the favored
candidate of the president's daughter and son-in-law, Ivanka Trump and
Jared Kushner, who have been seen as maneuvering for greater control and
influence around the president. They had clashed repeatedly with Mr.
Kelly as he tried to establish more regulated channels to the president.
Matt Drudge, an ally of Mr. Kushner, weeks ago posted a photo of Mr.
Ayers on The Drudge Report as the next chief of staff.

But some West Wing officials said Mr. Ayers had been measured and
cautious in recent days as he negotiated with Mr. Trump and his family.
Before turning down the job, Mr. Ayers told the president that he would
be willing to do it only on an interim basis, through the spring.

\href{https://www.nytimes.com/interactive/2018/03/16/us/politics/all-the-major-firings-and-resignations-in-trump-administration.html}{}

\includegraphics{https://static01.nyt.com/images/2018/07/05/us/all-the-major-firings-and-resignations-in-trump-administration-promo-1530825933054/all-the-major-firings-and-resignations-in-trump-administration-promo-1530825933054-articleLarge-v2.jpg}

\hypertarget{the-turnover-at-the-top-of-the-trump-administration}{%
\subsection{The Turnover at the Top of the Trump
Administration}\label{the-turnover-at-the-top-of-the-trump-administration}}

Since President Trump's inauguration, White House staffers and cabinet
officials have left in firings and resignations, one after the other.

Mr. Trump wants a long-term chief of staff, given the difficult period
approaching, and he and Mr. Ayers were unable to agree on certain other
terms, including whom he could dispose of from the current staff, three
people familiar with the events said.

Other factors may also have weighed on Mr. Ayers. His ascension to the
top West Wing job would have meant newfound scrutiny of his personal
finances --- last year he reported a net worth of \$12.2 million to
\$54.8 million, a sizable sum for a political operative in his 30s who
has amassed his own fortune. He accumulated his wealth partly through a
web of political and consulting companies in which he has held ownership
stakes.

And Mr. Ayers, who has been seen as a potential candidate for statewide
office in Georgia, could have potentially faced a fate shared by many
who have left the administration: a diminished public standing after an
ugly parting with a mercurial president who often insults his former
aides on Twitter.

Those who remain in the White House past the end of the year will have
to face a fraught and uncertain dynamic. Several potential outcomes of
the battles Mr. Trump confronts --- on impeachment, in the special
counsel inquiry and over allegations that he directed illegal hush
payments in 2016 --- may not have been advantageous for Mr. Ayers if he
makes a run for office.

On Sunday, Mr. Ayers took to Twitter to say that it had been an ``honor
to serve our Nation at The White House.''

``I will be departing at the end of the year but will work with the
\#MAGA team to advance the cause,''
\href{https://twitter.com/nick_ayers/status/1071879332283453440}{he
wrote}.

The monthslong process to replace Mr. Kelly, who Mr. Trump announced on
Saturday is
\href{https://www.nytimes.com/2018/12/08/us/politics/john-kelly-chief-staff-trump.html}{leaving
at the end of the year}, is a rare instance in which the president has
not been courting candidates simultaneously. Historically, he has
signaled to competing prospects that each one is his choice, and then
picks one even as he tells both that they are still in the running.

But this time, Mr. Ayers was the only person Mr. Trump had focused on
since he made up his mind to part ways with Mr. Kelly. With a head of
blond hair, Mr. Ayers somewhat resembles Mr. Trump in his younger days,
a fact that the president often looks for as a positive signal. The
president had an unusual affinity for Mr. Ayers, telling aides who
expressed concern about Mr. Ayers that he liked him.

And after barreling from a chief of staff recommended by Republican
congressional leaders (Reince Priebus) to a military general who shared
some of Mr. Trump's personality traits (Mr. Kelly), the president seemed
intent this time on simply picking someone he personally liked.

Mr. Kelly is expected to stay on only another three weeks, at least one
of which the president is scheduled to spend at his private club in
Florida. Hiring to fill several open jobs in the West Wing has been on
hold for weeks, as people waited to see whether Mr. Kelly would depart
and Mr. Ayers would replace him and bring in his own team.

Mr. Ayers
\href{https://www.nytimes.com/2017/06/29/us/politics/mike-pence-josh-pitcock-chief-of-staff.html}{replaced
Mr. Pence's initial chief}
\href{https://www.nytimes.com/2017/06/29/us/politics/mike-pence-josh-pitcock-chief-of-staff.html}{of
staff}, Josh Pitcock, in 2017. A former executive director of the
Republican Governors Association, he is the type of raw political
operative who Mr. Trump had felt he needed as he heads into what will
almost certainly be a brutal re-election campaign.

While Mr. Kelly and Mr. Trump were barely talking in recent weeks, the
retired four-star Marine general was a figure the president had
difficulty firing. Mr. Kelly fought loudly with the president over some
of Mr. Trump's most incendiary ideas.

One such shouting match came earlier this year, when Mr. Trump wanted to
pull security clearances from up to a dozen former national security
officials or cabinet secretaries who had criticized him. Mr. Kelly
argued vociferously against it, according to people familiar with what
took place.

Advertisement

\protect\hyperlink{after-bottom}{Continue reading the main story}

\hypertarget{site-index}{%
\subsection{Site Index}\label{site-index}}

\hypertarget{site-information-navigation}{%
\subsection{Site Information
Navigation}\label{site-information-navigation}}

\begin{itemize}
\tightlist
\item
  \href{https://help.nytimes.com/hc/en-us/articles/115014792127-Copyright-notice}{©~2020~The
  New York Times Company}
\end{itemize}

\begin{itemize}
\tightlist
\item
  \href{https://www.nytco.com/}{NYTCo}
\item
  \href{https://help.nytimes.com/hc/en-us/articles/115015385887-Contact-Us}{Contact
  Us}
\item
  \href{https://www.nytco.com/careers/}{Work with us}
\item
  \href{https://nytmediakit.com/}{Advertise}
\item
  \href{http://www.tbrandstudio.com/}{T Brand Studio}
\item
  \href{https://www.nytimes.com/privacy/cookie-policy\#how-do-i-manage-trackers}{Your
  Ad Choices}
\item
  \href{https://www.nytimes.com/privacy}{Privacy}
\item
  \href{https://help.nytimes.com/hc/en-us/articles/115014893428-Terms-of-service}{Terms
  of Service}
\item
  \href{https://help.nytimes.com/hc/en-us/articles/115014893968-Terms-of-sale}{Terms
  of Sale}
\item
  \href{https://spiderbites.nytimes.com}{Site Map}
\item
  \href{https://help.nytimes.com/hc/en-us}{Help}
\item
  \href{https://www.nytimes.com/subscription?campaignId=37WXW}{Subscriptions}
\end{itemize}
