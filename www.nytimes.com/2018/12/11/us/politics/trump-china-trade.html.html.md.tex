Sections

SEARCH

\protect\hyperlink{site-content}{Skip to
content}\protect\hyperlink{site-index}{Skip to site index}

\href{https://www.nytimes.com/section/politics}{Politics}

\href{https://myaccount.nytimes.com/auth/login?response_type=cookie\&client_id=vi}{}

\href{https://www.nytimes.com/section/todayspaper}{Today's Paper}

\href{/section/politics}{Politics}\textbar{}Marriott Data Breach Is
Traced to Chinese Hackers as U.S. Readies Crackdown on Beijing

\url{https://nyti.ms/2zQpXuv}

\begin{itemize}
\item
\item
\item
\item
\item
\item
\end{itemize}

Advertisement

\protect\hyperlink{after-top}{Continue reading the main story}

Supported by

\protect\hyperlink{after-sponsor}{Continue reading the main story}

\hypertarget{marriott-data-breach-is-traced-to-chinese-hackers-as-us-readies-crackdown-on-beijing}{%
\section{Marriott Data Breach Is Traced to Chinese Hackers as U.S.
Readies Crackdown on
Beijing}\label{marriott-data-breach-is-traced-to-chinese-hackers-as-us-readies-crackdown-on-beijing}}

\includegraphics{https://static01.nyt.com/images/2018/12/12/business/DC-USCHINA-2/merlin_147534942_cca02071-988f-4b5b-97b1-dd41f3060f5c-articleLarge.jpg?quality=75\&auto=webp\&disable=upscale}

By \href{https://www.nytimes.com/by/david-e-sanger}{David E. Sanger},
\href{https://www.nytimes.com/by/nicole-perlroth}{Nicole Perlroth},
\href{https://www.nytimes.com/by/glenn-thrush}{Glenn Thrush} and
\href{https://www.nytimes.com/by/alan-rappeport}{Alan Rappeport}

\begin{itemize}
\item
  Dec. 11, 2018
\item
  \begin{itemize}
  \item
  \item
  \item
  \item
  \item
  \item
  \end{itemize}
\end{itemize}

\href{https://cn.nytimes.com/usa/20181212/trump-china-trade/}{阅读简体中文版}\href{https://cn.nytimes.com/usa/20181212/trump-china-trade/zh-hant/}{閱讀繁體中文版}

WASHINGTON --- The cyberattack on the Marriott hotel chain that
collected personal details of roughly 500 million guests was part of a
Chinese intelligence-gathering effort that also hacked health insurers
and the security clearance files of millions more Americans, according
to two people briefed on the investigation.

The hackers, they said, are suspected of working on behalf of the
Ministry of State Security, the country's Communist-controlled civilian
spy agency. The discovery comes as the Trump administration is planning
actions targeting China's trade, cyber and economic policies, perhaps
within days.

Those moves include indictments against Chinese hackers working for the
intelligence services and the military, according to four government
officials who spoke on the condition of anonymity. The Trump
administration also plans to declassify intelligence reports to reveal
Chinese efforts dating to at least 2014 to build a database containing
names of executives and American government officials with security
clearances.

Other options include an executive order intended to make it harder for
Chinese companies to obtain critical components for telecommunications
equipment, a senior American official with knowledge of the plans said.

The moves stem from a growing concern within the administration that the
90-day trade truce negotiated two weeks ago by President Trump and
President Xi Jinping in Buenos Aires might do little to change China's
behavior --- including the coercion of American companies to hand over
valuable technology if they seek to enter the Chinese market, as well as
the theft of industrial secrets on behalf of state-owned companies.

The hacking of Marriott's Starwood chain, which was discovered only in
September and
\href{https://www.nytimes.com/2018/11/30/business/marriott-data-breach.html}{revealed
late last month}, is not expected to be part of the coming indictments.
But two of the government officials said that it has added urgency to
the administration's crackdown, given that Marriott is the top hotel
provider for American government and military personnel.

It also is a prime example of what has vexed the Trump administration as
\href{https://www.nytimes.com/2018/11/29/us/politics/china-trump-cyberespionage.html}{China
has reverted over the past 18 months} to the kind of intrusions into
American companies and government agencies that President Barack Obama
thought he had ended in 2015 in an agreement with Mr. Xi.

Geng Shuang, a spokesman for China's Ministry of Foreign Affairs, denied
any knowledge of the Marriott hacking. ``China firmly opposes all forms
of cyberattack and cracks down on it in accordance with the law,'' he
said. ``If offered evidence, the relevant Chinese departments will carry
out investigations according to the law.''

Trade negotiators on both sides of the Pacific have been working on an
agreement under which China would commit to purchasing \$1.2 trillion
more of American goods and services over the next several years, and
would address intellectual property concerns.

Mr. Trump said Tuesday that the United States and China were
\href{https://twitter.com/realDonaldTrump/status/1072480983683870720}{having
``very productive conversations''} as top American and Chinese officials
held their first talks via telephone since the two countries agreed to a
truce on Dec. 1.

But while top administration officials insist that the trade talks are
proceeding on a separate track, the broader crackdown on China could
undermine Mr. Trump's ability to reach an agreement with Mr. Xi.

\includegraphics{https://static01.nyt.com/images/2018/12/11/business/12DC-USCHINA/12DC-USCHINA-articleLarge.jpg?quality=75\&auto=webp\&disable=upscale}

American charges against senior members of China's intelligence services
risk hardening opposition in Beijing to negotiations with Mr. Trump.
Another obstacle is the targeting of high-profile technology executives,
like Meng Wanzhou, the chief financial officer of the communications
giant Huawei and daughter of its founder.

The
\href{https://www.nytimes.com/2018/12/07/technology/meng-wanzhou-huawei-arrest.html}{arrest
of Ms. Meng}, who has been detained in Canada on suspicion of fraud
involving violations of United States sanctions against Iran, has
angered China. She was granted bail of 10 million Canadian dollars, or
\$7.5 million, while awaiting extradition to the United States, a
Canadian judge ruled on Tuesday.

Mr. Trump, in an
\href{https://www.reuters.com/article/us-usa-trump-huawei-tech-exclusive/exclusive-trump-says-he-could-intervene-in-u-s-case-against-huawei-cfo-idUSKBN1OA2PQ}{interview
on Tuesday with Reuters}, said that he would consider intervening in the
Huawei case if it would help serve national security and help get a
trade deal done with China. Such a move would essentially pit Mr. Trump
against his own Justice Department, which coordinated with Canada to
arrest Ms. Meng as she changed planes in Vancouver, British Columbia.

``If I think it's good for what will be certainly the largest trade deal
ever made --- which is a very important thing --- what's good for
national security --- I would certainly intervene if I thought it was
necessary,'' Mr. Trump said.

American business leaders have been bracing for retaliation from China,
which has demanded the immediate release of Ms. Meng and accused both
the United States and Canada of violating her rights.

On Tuesday, the International Crisis Group said that one of its
employees, a former Canadian diplomat, had been detained in China. The
\href{https://www.nytimes.com/2018/12/11/world/asia/michael-kovrig-china-canada.html}{disappearance
of the former diplomat}, Michael Kovrig, could further inflame tensions
between China and Canada.

``We are doing everything possible to secure additional information on
Michael's whereabouts, as well as his prompt and safe release,'' the
group said in
\href{https://www.crisisgroup.org/who-we-are/crisis-group-updates/detention-crisis-group-senior-advisor}{a
statement on its website}.

From the
\href{https://www.nytimes.com/2018/11/30/business/marriott-data-breach.html}{first
revelation} that the Marriott chain's computer systems had been
breached, there was widespread suspicion in both Washington and among
cybersecurity firms that the hacking was not a matter of commercial
espionage, but part of a much broader spy campaign to amass Americans'
personal data.

While American intelligence agencies have not reached a final assessment
of who performed the hacking, a range of firms brought in to assess the
damage quickly saw computer code and patterns familiar to operations by
Chinese actors.

The Marriott database contains not only credit card information but
passport data. Lisa Monaco, a former homeland security adviser under Mr.
Obama, noted last week at a conference that passport information would
be particularly valuable in tracking who is crossing borders and what
they look like, among other key data.

But officials on Tuesday said it was only part of an aggressive
operation whose centerpiece was the
\href{https://www.nytimes.com/2015/06/05/us/breach-in-a-federal-computer-system-exposes-personnel-data.html}{2014
hacking into the Office of Personnel Management}. At the time, the
government bureau loosely guarded the detailed forms that Americans fill
out to get security clearances --- forms that contain financial data;
information about spouses, children and past romantic relationships; and
any meetings with foreigners.

Such information is exactly what the Chinese use to root out spies,
recruit intelligence agents and build a rich repository of Americans'
personal data for future targeting. With those details and more that
were stolen from insurers like Anthem, the Marriott data adds another
critical element to the intelligence profile: travel habits.

James A. Lewis, a cybersecurity expert at the Center for Strategic
Studies in Washington, said the Chinese have collected ``huge pots of
data'' to feed a Ministry of State Security database seeking to identify
American spies --- and the Chinese people talking to them.

``Big data is the new wave for counterintelligence,'' Mr. Lewis said.

``It's big-data hoovering,'' said Dmitri Alperovitch, the chief
technology officer at CrowdStrike, who first highlighted Chinese hacking
as a threat researcher in 2011. ``This data is all going back to a data
lake that can be used for counterintelligence, recruiting new assets,
anticorruption campaigns or future targeting of individuals or
organizations.''

In the Marriott case, Chinese spies stole passport numbers for up to 327
million people --- many of whom stayed at Sheraton, Westin and W hotels
and at other Starwood-branded properties. But Marriott has not said if
it would pay to replace those passports, an undertaking that would cost
tens of billions of dollars.

Instead, Connie Kim, a Marriott spokeswoman, said the hotel chain would
cover the cost of replacement if ``fraud has taken place.'' That means
the company would not cover the cost of having exposed private data to
the Chinese intelligence agencies if they did not use it to conduct
commercial transactions --- even though that is a breach of privacy and,
perhaps, security.

And even for those guests who did not have passport information on file
with the hotels, their phone numbers, birth dates and itineraries remain
vulnerable.

That data, Mr. Lewis and others said, can be used to track which Chinese
citizens visited the same city, or hotel, as an American intelligence
agent who was identified in data taken from the Office of Personnel
Management or from American health insurers that document patients'
medical histories and Social Security numbers.

The effort to amass Americans' personal information so alarmed
government officials that in 2016, the Obama administration threatened
to block a \$14 billion bid by China's Anbang Insurance Group Co. to
acquire Starwood Hotel \& Resorts Worldwide, according to one former
official familiar with the work of the Committee on Foreign Investments
in the United States, a secretive government body that reviews foreign
acquisitions.

Ultimately, the failed bid cleared the way later that year for Marriott
Hotels to acquire Starwood for \$13.6 billion, becoming the world's
largest hotel chain.

As it turned out, it was too late: Starwood's data had already been
stolen by Chinese state hackers, though the breach was not discovered
until this past summer, and was disclosed by Marriott on Nov. 30.

It is unclear that any kind of trade agreement reached with China by the
Trump administration can address this kind of theft.

The Chinese regard intrusions into hotel chain databases as a standard
kind of espionage. So does the United States, which has often seized
guest data from foreign hotels.

Even the Office of Personnel Management hacking was viewed by American
intelligence officials with some admiration. ``If we had the opportunity
to do the same thing, we'd probably do it,'' James R. Clapper Jr., the
former director of national intelligence, told Congress afterward.

``One thing is very clear to me, and it is that they are not going to
stop this,'' Mr. Alperovitch said. ``This is what any nation-state
intelligence agency would do. No nation-state is going to handcuff
themselves and say, `You can't do this,' because they all engage in
similar detection.''

Since 2012, analysts at the National Security Agency and its British
counterpart, the GCHQ, have watched with growing alarm as sophisticated
Chinese hackers, based in Tianjin, began switching targets from
companies and government agencies in the defense, energy and aerospace
sectors to organizations that housed troves of Americans' personal
information.

At the time, one classified National Security Agency report noted that
the hackers' ``exact affiliation with Chinese government entities is not
known, but their activities indicate a probable intelligence requirement
feed'' from China's Ministry of State Security.

Advertisement

\protect\hyperlink{after-bottom}{Continue reading the main story}

\hypertarget{site-index}{%
\subsection{Site Index}\label{site-index}}

\hypertarget{site-information-navigation}{%
\subsection{Site Information
Navigation}\label{site-information-navigation}}

\begin{itemize}
\tightlist
\item
  \href{https://help.nytimes.com/hc/en-us/articles/115014792127-Copyright-notice}{©~2020~The
  New York Times Company}
\end{itemize}

\begin{itemize}
\tightlist
\item
  \href{https://www.nytco.com/}{NYTCo}
\item
  \href{https://help.nytimes.com/hc/en-us/articles/115015385887-Contact-Us}{Contact
  Us}
\item
  \href{https://www.nytco.com/careers/}{Work with us}
\item
  \href{https://nytmediakit.com/}{Advertise}
\item
  \href{http://www.tbrandstudio.com/}{T Brand Studio}
\item
  \href{https://www.nytimes.com/privacy/cookie-policy\#how-do-i-manage-trackers}{Your
  Ad Choices}
\item
  \href{https://www.nytimes.com/privacy}{Privacy}
\item
  \href{https://help.nytimes.com/hc/en-us/articles/115014893428-Terms-of-service}{Terms
  of Service}
\item
  \href{https://help.nytimes.com/hc/en-us/articles/115014893968-Terms-of-sale}{Terms
  of Sale}
\item
  \href{https://spiderbites.nytimes.com}{Site Map}
\item
  \href{https://help.nytimes.com/hc/en-us}{Help}
\item
  \href{https://www.nytimes.com/subscription?campaignId=37WXW}{Subscriptions}
\end{itemize}
