Sections

SEARCH

\protect\hyperlink{site-content}{Skip to
content}\protect\hyperlink{site-index}{Skip to site index}

\href{https://www.nytimes.com/section/politics}{Politics}

\href{https://myaccount.nytimes.com/auth/login?response_type=cookie\&client_id=vi}{}

\href{https://www.nytimes.com/section/todayspaper}{Today's Paper}

\href{/section/politics}{Politics}\textbar{}Deciphering the Patterns in
Trump's Falsehoods

\url{https://nyti.ms/2RlJddn}

\begin{itemize}
\item
\item
\item
\item
\item
\end{itemize}

Advertisement

\protect\hyperlink{after-top}{Continue reading the main story}

Supported by

\protect\hyperlink{after-sponsor}{Continue reading the main story}

\hypertarget{deciphering-the-patterns-in-trumps-falsehoods}{%
\section{Deciphering the Patterns in Trump's
Falsehoods}\label{deciphering-the-patterns-in-trumps-falsehoods}}

We review how President Trump bent the truth this year by repeating and
inflating falsehoods, shifting his statements, embellishing or omitting
details, and offering misleading attacks.

\includegraphics{https://static01.nyt.com/images/2018/12/26/us/politics/00DC-FACTCHECK/merlin_147967236_a585df47-ed82-4b18-8bf6-9432089bed7a-articleLarge.jpg?quality=75\&auto=webp\&disable=upscale}

\href{https://www.nytimes.com/by/linda-qiu}{\includegraphics{https://static01.nyt.com/images/2018/06/12/multimedia/author-linda-qiu/author-linda-qiu-thumbLarge.png}}

By \href{https://www.nytimes.com/by/linda-qiu}{Linda Qiu}

\begin{itemize}
\item
  Dec. 29, 2018
\item
  \begin{itemize}
  \item
  \item
  \item
  \item
  \item
  \end{itemize}
\end{itemize}

President Trump has a well-documented problem telling the truth.

Fact checkers have compiled lists of all of Mr. Trump's falsehoods since
he took office (The Washington Post
\href{https://www.washingtonpost.com/graphics/politics/trump-claims-database/?utm_term=.f0fb3ee77790}{counts
over 7,500}, and The Toronto Star
\href{http://projects.thestar.com/donald-trump-fact-check/}{over
3,900}), rounded up his
\href{https://www.politifact.com/truth-o-meter/article/2018/dec/11/trump-file-10-top-falsehoods-2018/}{most
egregious whoppers} in
\href{https://www.factcheck.org/2018/12/the-whoppers-of-2018/}{year-end
lists} and scrutinized his claims in real time with
\href{https://www.washingtonpost.com/graphics/2018/lifestyle/style/how-cable-news-chyrons-have-adapted-to-the-trump-era/?utm_term=.5d2df35f9e33}{television
chyrons}.

Here at The New York Times,
\href{https://www.nytimes.com/spotlight/fact-checks}{we have also
fact-checked} countless campaign rallies, news conferences, interviews
and Twitter posts. After nearly two years of assessing the accuracy of
Mr. Trump's statements, we can draw some conclusions not just about the
scale of the president's mendacity, but also about how he uses
inaccurate claims to advance his agenda, criticize the news media and
celebrate his achievements.

\hypertarget{repetition-and-inflation}{%
\subsection{Repetition and Inflation}\label{repetition-and-inflation}}

Mr. Trump refuses to correct most of his inaccurate claims, instead
asserting them over and over again. They become, by sheer force of
repetition, ``alternative facts'' and staples of his campaign rallies
and speeches.

Examples abound. He has falsely characterized the December 2017 tax cuts
as the ``largest'' or the ``biggest'' in American history over 100 times
(\href{https://www.nytimes.com/2017/11/29/us/politics/fact-check-trump-tax-cuts.html}{several
others were larger}). He has misleadingly said over 90 times that his
promised wall along the southern border is being built
(\href{https://www.nytimes.com/2018/12/11/us/politics/fact-check-trump-border-wall.html}{construction
has not begun}
\href{https://www.nytimes.com/2018/05/11/us/politics/trump-misleadingly-says-construction-has-already-begun-on-his-border-wall.html}{on
any new section}). He has falsely accused Democrats of supporting ``open
borders'' over 60 times (Democratic lawmakers
\href{https://www.nytimes.com/2018/06/27/us/politics/fast-check-donald-trump-democrats-open-borders.html}{support
border security}, but not his border wall). And he has lobbed
\href{https://www.nytimes.com/2018/08/18/us/politics/fact-check-trump-russia-election-interference-.html}{over
250 inaccurate attacks} on the investigation into Russian election
interference.

Yet Mr. Trump does not rely on repetition alone. He also embellishes
talking points to amplify his achievements.

Take his repeated fabrication about the construction of new steel mills.
After his administration
\href{https://www.nytimes.com/2018/03/01/business/trump-tariffs.html}{announced
tariffs on steel and aluminum imports} in March, the president claimed
in June that United States Steel was
``\href{https://www.nytimes.com/2018/06/28/us/politics/fact-check-trump-north-dakota-rally.html}{opening
six new plants}.'' A month later, the number rose to
\href{https://www.whitehouse.gov/briefings-statements/remarks-president-trump-economy/}{seven}.
He has also occasionally cited
\href{https://www.whitehouse.gov/briefings-statements/remarks-president-trump-signing-executive-order-strengthening-retirement-security-america/}{eight},
possibly
\href{https://www.whitehouse.gov/briefings-statements/remarks-president-trump-united-states-mexico-canada-agreement/}{nine}
or a vague ``many plants,'' and he claimed once that plants were
``\href{https://www.whitehouse.gov/briefings-statements/remarks-president-trump-dinner-business-leaders/}{opening
up literally on a daily basis}.'' To date, United States Steel has yet
to open or build one new plant, though the company has restarted idled
components of some plants.

Mr. Trump often pairs his misleading claim that his border wall is being
built with the statement that he received \$1.6 billion from Congress to
fund it --- though a spending bill from March stipulated that the money,
for border security, could not be used for the wall. In August, he
\href{https://www.youtube.com/watch?v=izMFU8g6uBo}{added another \$1.6
billion} to the count, with no evidence. And despite
\href{https://twitter.com/realDonaldTrump/status/1042740913968164864}{criticizing
a spending bill} passed in September for not appropriating any money for
the wall, Mr. Trump told campaign rally audiences in
\href{https://www.c-span.org/video/?453628-1/president-trump-campaigns-republicans-illinois\&start=1920}{Illinois},
\href{https://www.c-span.org/video/?453256-1/president-trump-campaigns-senator-ted-cruz-houston\&start=1526}{Texas}
and
\href{https://www.c-span.org/video/?452371-1/president-trump-holds-rally-mississippi\&start=2935}{Mississippi}
in October that his administration had secured a ``third \$1.6
billion.''

In some cases, true claims morph into false ones in Mr. Trump's telling.
In December 2017, he was largely accurate in
\href{https://www.whitehouse.gov/briefings-statements/remarks-president-trump-marine-helicopter-squadron-one/}{saying
a military spending law} would give troops the largest raise
\href{https://fas.org/sgp/crs/natsec/IF10260.pdf\#page=2}{in eight
years}. In March, he then exaggerated that time frame to
``\href{https://www.nytimes.com/2018/03/23/us/politics/fact-check-military-pay-immigrants.html}{over
a decade}.'' And in May, as he addressed Naval Academy graduates at
Annapolis, Mr. Trump wrongly characterized the wage increase as not just
the largest, but the
``\href{https://www.nytimes.com/2018/05/25/us/politics/fact-check-trump-naval-academy-speech.html}{first
in 10 years}.''

\hypertarget{shifting-and-deflecting}{%
\subsection{Shifting and Deflecting}\label{shifting-and-deflecting}}

In the face of controversy or criticism, Mr. Trump has defended initial
falsehoods with additional dubious claims.

This approach is evident in his shifting statements about the payment
that Michael D. Cohen, his former lawyer, made to a pornographic film
actress to keep her from speaking about their alleged affair. In April,
Mr. Trump
\href{https://www.whitehouse.gov/briefings-statements/remarks-president-trump-press-gaggle-en-route-washington-d-c/}{falsely
denied knowing about the payment}.

After the F.B.I. raided Mr. Cohen's office, Mr. Trump
\href{https://twitter.com/realDonaldTrump/status/991992302267785216}{acknowledged
on Twitter} in May that Mr. Cohen received reimbursement for the payment
and asserted that it had nothing to do with his presidential campaign.
Mr. Cohen would later tell prosecutors that he acted at Mr. Trump's
direction and to influence the election.

After the release of
\href{https://www.nytimes.com/2018/07/25/us/politics/trump-michael-cohen-recording.html?module=inline}{an
audio recording} of the two men discussing a hush-money payment to
another woman, Mr. Trump
\href{https://www.nytimes.com/2018/08/23/us/politics/fact-check-trump-fox-interview.html}{claimed
in an August interview on ``Fox \& Friends''} that he did not know about
the payments until ``later on'' and that Mr. Cohen ``made the deals.''
He then misleadingly compared Mr. Cohen's actions --- a willful
violation of campaign finance law --- to a civil infraction incurred by
former President Barack Obama's 2008 campaign.

By December, Mr. Trump's defense had shifted further:
``\href{https://twitter.com/realDonaldTrump/status/1073205176872435713}{I
never directed Michael Cohen to break the law}.''

\hypertarget{misleading-vagueness-and-fanciful-details}{%
\subsection{Misleading Vagueness and Fanciful
Details}\label{misleading-vagueness-and-fanciful-details}}

The president is known for being unscripted and loose with language, but
he sometimes shows tactical restraint.

After Justice Brett M. Kavanaugh was confirmed to the Supreme Court and
in the days before the midterm elections, Mr. Trump told rallygoers
\href{https://www.c-span.org/video/?453853-1/president-trump-holds-missouri-rally-rush-limbaugh-sean-hannity\&start=3234}{in
Missouri} that ``the accuser admitted she never met him, she never saw
him, he never touched her, talked to her, he had nothing to do with her,
she made up the story, it was false accusations.''

The omission of a name and the use of the words ``the accuser'' may give
the misleading impression that Christine Blasey Ford, who testified to
Congress that Justice Kavanaugh had sexually assaulted her when they
were teenagers, had recanted her account. But in fact, Mr. Trump was
referring to another little-known accuser named Judy Munro-Leighton, who
recanted her claim of sexual assault.

Mr. Trump also regales his audience with elaborate stories. Some ---
like his tales of unnamed ``strong'' or ``tough'' men, miners or
steelworkers crying and thanking him --- may have occurred but are
impossible to verify.

Others, like his frequent attacks on Senator Richard Blumenthal,
Democrat of Connecticut, contain invented details. Not content with just
accurately pointing out that Mr. Blumenthal falsely claimed to have
served in Vietnam, Mr. Trump adds --- with no evidence --- that Mr.
Blumenthal said he had
``\href{https://www.c-span.org/video/?452582-1/president-trump-rally-topeka-kansas\&start=3080}{charged
up Da Nang},''
\href{https://www.c-span.org/video/?453851-1/president-trump-campaigns-republicans-indiana\&start=698}{dodged
bullets} and
\href{https://www.c-span.org/video/?c4754001/president-trump-criticizes-senator-blumenthal}{saved
the lives of fellow soldiers}.

\hypertarget{inventing-straw-men}{%
\subsection{Inventing Straw Men}\label{inventing-straw-men}}

The usual target of this particular strain of falsehoods is the news
media, which Mr. Trump suggests purposely underestimates or
misinterprets him.

Mr. Trump often lauds
\href{https://data.bls.gov/timeseries/ces0000000001?output_view=net_1mth}{strong
job growth} under his watch and says that the ``fake news'' would have
deemed such numbers ``impossible'' or ``ridiculous'' during the 2016
campaign. Yet he neglects to mention that the number of jobs added in
the 22 months after his inauguration --- 4.2 million --- is lower than
the 4.8 million jobs added in the 22 months before he took office,
undermining the premise of his retrodiction.

In another example, Mr. Trump turned a hypothetical talking point first
into a purported reality and then --- after headlines debunking his
claim appeared --- into a joke that he implied reporters had failed to
grasp.

\href{https://www.c-span.org/video/?453287-1/president-trump-campaigns-mesa-arizona}{At
an October rally in Arizona}, Mr. Trump criticized Democrats for
allowing undocumented immigrants to apply for driver's licenses. ``Next
thing you know, they'll want to buy 'em a car,'' he speculated. ``Then
they'll say the car's not good enough, we want --- how about a
Rolls-Royce?''

A day later,
\href{https://www.c-span.org/video/?453249-1/president-trump-campaigns-nevada-republican-senator-dean-heller}{at
a campaign rally in Nevada}, Mr. Trump presented this theory as reality,
telling supporters that Democrats wanted to give cars and licenses to
undocumented immigrants in addition to free health care and education.

After the claim was
\href{https://www.snopes.com/fact-check/democrats-free-cars-immigrants/}{debunked}
\href{https://www.politifact.com/truth-o-meter/statements/2018/oct/22/donald-trump/donald-trump-says-democrats-want-give-cars-undocum/}{by
news outlets,} Mr. Trump responded by ridiculing the news media for
mentioning the Rolls-Royce, which was not the actual subject of the fact
checks.

``They said he gets a Pinocchio for that,'' Mr. Trump
\href{https://www.youtube.com/watch?time_continue=1318\&v=0ri5QQVhJhg}{said
to laughter at a campaign rally in Mississippi}, referring to The
Washington Post's rating system for false claims. ``They got me!''

\emph{To suggest claims to check, email}
\href{mailto:factcheck@nytimes.com}{\emph{factcheck@nytimes.com}}\emph{.}

Advertisement

\protect\hyperlink{after-bottom}{Continue reading the main story}

\hypertarget{site-index}{%
\subsection{Site Index}\label{site-index}}

\hypertarget{site-information-navigation}{%
\subsection{Site Information
Navigation}\label{site-information-navigation}}

\begin{itemize}
\tightlist
\item
  \href{https://help.nytimes.com/hc/en-us/articles/115014792127-Copyright-notice}{©~2020~The
  New York Times Company}
\end{itemize}

\begin{itemize}
\tightlist
\item
  \href{https://www.nytco.com/}{NYTCo}
\item
  \href{https://help.nytimes.com/hc/en-us/articles/115015385887-Contact-Us}{Contact
  Us}
\item
  \href{https://www.nytco.com/careers/}{Work with us}
\item
  \href{https://nytmediakit.com/}{Advertise}
\item
  \href{http://www.tbrandstudio.com/}{T Brand Studio}
\item
  \href{https://www.nytimes.com/privacy/cookie-policy\#how-do-i-manage-trackers}{Your
  Ad Choices}
\item
  \href{https://www.nytimes.com/privacy}{Privacy}
\item
  \href{https://help.nytimes.com/hc/en-us/articles/115014893428-Terms-of-service}{Terms
  of Service}
\item
  \href{https://help.nytimes.com/hc/en-us/articles/115014893968-Terms-of-sale}{Terms
  of Sale}
\item
  \href{https://spiderbites.nytimes.com}{Site Map}
\item
  \href{https://help.nytimes.com/hc/en-us}{Help}
\item
  \href{https://www.nytimes.com/subscription?campaignId=37WXW}{Subscriptions}
\end{itemize}
