\href{/section/business}{Business}\textbar{}The Rise and Fall of Carlos
Ghosn

\url{https://nyti.ms/2Rq7kaX}

\begin{itemize}
\item
\item
\item
\item
\item
\item
\end{itemize}

\includegraphics{https://static01.nyt.com/images/2018/12/31/business/31ghosn-print-illo/merlin_148535898_66890dc1-8036-4526-886c-349e06968891-articleLarge.jpg?quality=75\&auto=webp\&disable=upscale}

Sections

\protect\hyperlink{site-content}{Skip to
content}\protect\hyperlink{site-index}{Skip to site index}

\hypertarget{the-rise-and-fall-of-carlos-ghosn}{%
\section{The Rise and Fall of Carlos
Ghosn}\label{the-rise-and-fall-of-carlos-ghosn}}

Mr. Ghosn, the ousted Nissan executive, wasn't supposed to succeed in
Japan, but he never expected to fail like this. He faces charges of
financial wrongdoing at the company he helped save.

Credit...Kelsey Dake

Supported by

\protect\hyperlink{after-sponsor}{Continue reading the main story}

By \href{https://www.nytimes.com/by/amy-chozick}{Amy Chozick} and
\href{https://www.nytimes.com/by/motoko-rich}{Motoko Rich}

\begin{itemize}
\item
  Dec. 30, 2018
\item
  \begin{itemize}
  \item
  \item
  \item
  \item
  \item
  \item
  \end{itemize}
\end{itemize}

\href{https://www.nytimes.com/2020/01/02/business/carlos-ghosn-france-extradite.html}{Carlos
Ghosn} was tired. At 64 years old, the chairman of an auto empire that
spanned several continents and included Nissan, Renault and Mitsubishi
wasn't bouncing back from jet lag the way he used to. Melatonin wasn't
working anymore, and he had bouts of insomnia, phoning his children in
the middle of the night or going on long walks around his Tokyo or Paris
neighborhood. He planned to retire soon, stepping back from spending his
life on an airplane, albeit a luxurious one paid for by
\href{https://www.nytimes.com/2020/01/02/business/carlos-ghosn-france-extradite.html}{Nissan}.

Last month, just before Thanksgiving weekend,
\href{https://www.nytimes.com/2020/01/02/business/carlos-ghosn-france-extradite.html}{Mr.
Ghosn} headed to Tokyo to meet his youngest daughter and her boyfriend
and attend a board meeting. He was scheduled to land at Haneda Airport
at 4 p.m.

The daughter, Maya Ghosn, 26, had spent most of her childhood in Japan
and wanted to introduce her boyfriend, Patrick, to her favorite places.
Bringing a boyfriend home is a common rite of passage, but a
particularly intimidating prospect when growing up Ghosn --- a child of
one of the most romanticized and ruthless chief executives the global
business community has ever seen.

Ms. Ghosn had made a 7:30 dinner reservation at Jiro, the
Michelin-starred sushi counter hidden in a basement in the city's Ginza
district.

On the tarmac in Beirut, Lebanon, Mr. Ghosn opened WhatsApp and texted
his four children on a group chain labeled ``Game of Ghosns,'' for his
favorite TV show, ``Game of Thrones,'' the bloody HBO drama about
dynasties under siege. ``On my way to Tokyo! Love you guys!'' Mr. Ghosn
texted as his jet lifted off.

\includegraphics{https://static01.nyt.com/images/2018/12/31/business/31ghosn-01/merlin_148370886_69c244a2-efa1-494a-8b89-94c09ba57217-articleLarge.jpg?quality=75\&auto=webp\&disable=upscale}

He never made it to dinner.

On Nov. 19, Japanese prosecutors surrounded Mr. Ghosn's Gulfstream after
its arrival and arrested him on allegations that for years he had
withheld millions of dollars in income from Nissan's financial filings.

Ms. Ghosn was staying at her father's corporate apartment, and when he
didn't show up she checked with his longtime driver at Nissan, who
assured her his flight had probably been delayed. She texted: ``Hey,
just heard your flight got delayed. Please let me know when you land,
worried about you.''

\emph{{[}Breaking a silence,}
\emph{\href{https://www.nytimes.com/2018/12/29/business/carlos-ghosn-nissan-children.html}{Mr.
Ghosn's daughters said}} \emph{they suspected that the charges against
him were part of a revolt within Nissan.{]}}

Exhausted from jet lag, she took a nap. Patrick woke her when he saw a
tweet about Mr. Ghosn's arrest. ``I was in shock,'' she said in an
interview.

Minutes later, the doorbell rang. Two Japanese men in black suits
slipped off their shoes to enter the two-bedroom apartment and showed
Ms. Ghosn a brief note in English.

``There is a case against your father,'' it read, according to Ms.
Ghosn's account. ``The Tokyo judge has warranted us access to search the
house. We need a witness. Thank you for cooperating.''

Fifteen men, also in suits, followed. They locked the front door, told
Ms. Ghosn that they were prosecutors, warned the couple not to use their
phones and suggested that they might tap the apartment. They rummaged
through Mr. Ghosn's drawers, studying family photos, Maya's 10th-grade
report card, personal letters, her parents' divorce papers.

Image

Mr. Ghosn with two of his four children: his daughters Maya, center, and
Nadine. Maya Ghosn said that ``inside, I was shaking'' when
investigators searched her father's corporate apartment.Credit...Ghosn
Family

``I wanted my dad to know that in this situation I was polite and
handled it maturely, and I didn't want to give them any reason to feel
satisfied by an ounce of despair in my eyes,'' Ms. Ghosn said. ``But
inside, I was shaking. I couldn't stand up. I had to hold the wall.''

Six and a half hours later, at 11:30 p.m., the men left.

Worried that anything they said was being recorded, Ms. Ghosn and her
boyfriend went into the bathroom, climbed into the shower fully clothed,
turned on the water and whispered about what to do next. She called her
siblings to figure out how to tackle Japan's labyrinthine legal system.

Told by the authorities that she was forbidden to contact her father,
Ms. Ghosn waited at the apartment for nearly two days until an American
lawyer working for her family called.

``We got very clear instructions to leave as soon as possible for fear
of being detained or interrogated to extort my dad,'' she said. ``So we
got on the first flight out.''

\hypertarget{a-person-who-was-above-the-clouds}{%
\subsection{`A Person Who Was Above the
Clouds'}\label{a-person-who-was-above-the-clouds}}

Carlos Ghosn wasn't supposed to succeed in Japan, but he wasn't supposed
to fail like this. He first made headlines in 1999 when, in a nation
known for its distrust of outsiders, Mr. Ghosn, a brash Brazilian-born
and Lebanese- and French-educated engineer, showed up in sunglasses and
a pinstripe suit with plans to carry out an American-style restructuring
of a failing Nissan. The Japanese carmaker had \$35 billion in debt,
provided lifetime employment to a bloated work force and produced a
fleet of the kind of cars you'd dread getting at the rental counter.

Mr. Ghosn, then 45 and a vice president at Renault, had helped oversee a
turnaround at the middling French automaker, which had agreed to spend
\$5.4 billion to buy a 36.8 percent stake in Nissan Motors.

John Casesa, then a top auto analyst at Merrill Lynch, advised Mr. Ghosn
to rent a house in Tokyo rather than buy one.

``The widely held consensus was that he would fail, that Nissan wasn't
worth saving and it couldn't be done,'' Mr. Casesa said.

At the time, Bob Lutz, the loquacious vice chairman of General Motors,
\href{https://www.cnbc.com/2018/11/20/ghosn-scandal-could-trigger-crises-for-nissan-renault-mitsubishi.html}{assessed
the deal this way}: Renault would be better off ``taking \$5 billion,
putting it on a barge and sinking it in the middle of the ocean.''

But Mr. Ghosn, with his severe black eyebrows and puffed chest, was
undeterred. He closed factories, slashed suppliers, laid off 14 percent
of the work force and invested in design. Six years later, Nissan had
surpassed Honda to become Japan's No. 2 automaker, its market
capitalization had quintupled and its operating margin had risen
tenfold. Altima sedans, Titan pickup trucks and Murano S.U.V.s made
Nissan a major player in the United States market --- an achievement
that Wall Street once deemed impossible.

By the early 2000s, Mr. Ghosn was head of the Renault-Nissan alliance
and the first person to simultaneously serve as chief executive of two
Fortune Global 500 companies, the type of chief executive who even if
you didn't know how to pronounce his name (rhymes with phone), you'd
know his products.

Image

Emperor Akihito greeting Mr. Ghosn at a garden party at Akasaka Palace
in 2004. That year, Mr. Ghosn became the first foreign business leader
to receive a Blue Ribbon Medal from the emperor.Credit...Toru
Yamanaka/Agence France-Presse --- Getty Images

The enigmatic ``gaijin'' (as foreigners are called in Japan) had
achieved a status bestowed on only a handful of chief executives, akin,
at least in Japan, to Steve Jobs, Warren E. Buffett or Elon Musk.
Paparazzi swarmed. Fans asked for autographs. Japanese businessmen,
eager to emulate the Nissan chief, inquired where Mr. Ghosn had bought
his rectangular sunglasses and custom suits.

In 2004, Emperor Akihito awarded Mr. Ghosn a Blue Ribbon Medal for his
extraordinary contributions, making him the first foreign business
leader to receive the honor. A manga comic book,
``\href{https://www.archynety.com/business/when-carlos-ghosn-was-a-manga-hero/}{The
True Story of Carlos Ghosn},'' heralded a shadowy hero from a faraway
land. Lebanon put Mr. Ghosn's face
\href{https://www.cnn.com/2018/11/23/business/carlos-ghosn-lebanon-icon/index.html}{on
a postage stamp}.

But even as many in Nissan celebrated the comeback, others scoffed at
Mr. Ghosn's celebrity.

From the start, he faced distrust from the Japanese policymaking and
business establishment. The very idea of an outsider's bringing
free-market capitalism to Japan's quasi-socialist corporate culture
jabbed at historical wounds.

``When MacArthur came after World War II, the Japanese just surrendered
to his leadership,'' a retired Nissan executive
\href{https://www.newsweek.com/can-company-be-saved-167730}{told
Newsweek}.

Mr. Ghosn pulled on a white jumpsuit to tour factory floors, but beyond
the photo ops, there were signs that his splashy --- some would say
autocratic --- presence was out of sync with modest Japanese culture. In
2004, Mr. Ghosn grazed a motorbike
\href{https://www.autonews.com/article/20040209/REG/402090904/nissan-ceo-ghosn-in-accident-driving-porsche}{while
driving a Porsche} in the Roppongi area of Tokyo, a haven for moneyed
foreigners. (The couple on the bike had minor injuries.) The Japanese
media groused that Mr. Ghosn wasn't driving a Nissan.

Then the man whose militant approach to cutting jobs (21,000, if you're
counting) earned him the nickname ``Le Cost Killer'' spent more than
\$200 million for Nissan to be a sponsor of the Rio Olympics in 2016,
\href{https://www.pressreader.com/canada/national-post-latest-edition/20160809/282054801421333}{casting
himself in the Olympic torch relay}. He hopped between homes paid for by
Nissan. In 2017, he paid a Lebanese artist and friend \$888,000 to
create a statue,
``\href{https://www.youtube.com/watch?v=mCLqEI8VURo}{Wheels of
Innovation},'' for the entrance of Nissan's Yokohama headquarters.
(Having a lavish second wedding reception in Versailles the same year,
with Marie Antoinette-themed costumes and,
\href{https://www.townandcountrymag.com/the-scene/weddings/a9634/versailles-wedding/}{yes,
lots of cake}, did not help.)

``He was a person who was above the clouds,'' said Yuichi Ishino, who
worked in Nissan's finance department from 2002 to 2005. ``No one dared
to say anything that would confront his opinions.''

The stickiest issue was always Mr. Ghosn's pay.

In Japan, salarymen slave away at the kaisha (or company) with a sense
of communal pride almost as important as the salary. Last year, Mr.
Ghosn
\href{https://www.reuters.com/article/us-nissan-ghosn-compensation-factbox/factbox-how-carlos-ghosns-pay-compares-with-other-top-auto-executives-idUSKCN1NS0P5}{made
\$16.9 million} (\$8.4 million from Renault, \$6.5 million from Nissan
and \$2 million from Mitsubishi). That's nearly 11 times what the
chairman of Toyota, the world's largest carmaker, earns but well below
the \$21.96 million paid to Mary Barra, the chief executive of General
Motors.

In 2008, the same year that Japanese law began requiring companies to
disclose directors' pay in their annual reports, Nissan's shareholders
voted to set an annual cap of about \$27 million on compensation for all
board directors combined.

After that, Mr. Ghosn made the case to the public that he was underpaid
--- instructing Nissan to hand out background materials reminding
investors and the news media that he made significantly less than his
counterparts at other global automakers.

At the company's most
\href{https://www.youtube.com/watch?v=kTL7yOit6U0\&feature=youtu.be}{recent
annual meeting, in June}, Mr. Ghosn stressed to shareholders that the
company's compensation policy was ``designed to reward performance and
to attract, promote and retain the best management talent in the auto
industry.'' He added that while Nissan tried to reward senior management
``competitively,'' the company remained ``financially very
disciplined.''

Asked by the Financial Times that same month if he was overpaid, Mr.
Ghosn laughed. ``You won't have any C.E.O. say, `I'm overly
compensated,'''
\href{https://www.ft.com/content/e3acccf2-6e20-11e8-92d3-6c13e5c92914}{he
said}.

Such brazenness rankled employees and the public in Japan.

``Even when a company is a global multinational company, it's still
stamped by its country of origin and the place where it has its
headquarters,'' said Sanford M. Jacoby, a professor of management at the
University of California, Los Angeles, who has studied Japanese
corporate culture. The Japanese, he said, put more weight ``on
egalitarian policies of government and pay and other things.''

In France, where the government owns a 15 percent stake in Renault,
shareholders have also taken issue with Mr. Ghosn's pay. ``We believe
that anyone making 240 times more than the minimum pay of his employees
is out of control,'' said Pierre-Henri Leroy, the head of Proxinvest, a
French shareholder advisory group.

In October, a whistle-blower inside Nissan said he had evidence that Mr.
Ghosn had been instructing Greg Kelly, a top aide and a board member,
and a small group of confidants at Nissan to effectively create two
salary pots for Mr. Ghosn's compensation.

Image

Hiroto Saikawa, whom Mr. Ghosn mentored and chose to succeed him as
Nissan's chief executive, at a news conference after the
arrest.Credit...Christopher Jue/EPA, via Shutterstock

One pot would be paid in the current year and reported in the company's
annual report and securities filings. Another amount would be designated
to be paid out after Mr. Ghosn left Nissan, according to a person
familiar with Nissan's internal investigation. The whistle-blower's
findings were sent to Hiroto Saikawa, the company's chief executive, and
an internal auditor.

Nissan went to prosecutors with allegations that Mr. Ghosn, working
directly with Mr. Kelly, who was once the head of human resources at
Nissan, had underreported his income from 2009 to 2017, according to a
person with knowledge of the internal investigation. Nissan's
investigation found that the underreporting had occurred when some of
the compensation, though committed, was deferred and not reported in
securities filings.

Nissan also told prosecutors that it had evidence Mr. Ghosn and Mr.
Kelly developed plans to pay Mr. Ghosn a further \$124 million in cash
and other financial instruments, some as compensation for a future
advisory position for Mr. Ghosn.

Hari Nada, a Nissan executive and confidant of Mr. Kelly's, sent a
private jet to fly him from Nashville to Tokyo for the same board
meeting that Mr. Ghosn planned to attend. The two men were arrested
hours apart. Mr. Kelly's family said Mr. Nada had assured him that he
would be back in Nashville by Thanksgiving, in time for scheduled neck
surgery.

Nissan would not comment about the Kelly family's statements about Mr.
Nada. Mr. Nada did not answer phone calls seeking comment.

Mr. Kelly was released on Christmas after his family cited his ill
health and posted bail of 70 million yen (about \$640,000). His lawyer
in Nashville, Aubrey Harwell Jr., said his client denied wrongdoing. Mr.
Kelly and Mr. Ghosn ``had conversations regarding legal ways they could
defer compensation,'' Mr. Harwell said.

Mr. Ghosn, Mr. Kelly and Nissan itself all face charges they violated
financial reporting laws. The company's board removed Mr. Ghosn and Mr.
Kelly as representative directors, positions with power to sign company
documents.

Image

Greg Kelly, a top Nissan aide who was also charged in the investigation,
was released on bail last week because of ill health.Credit...Kim
Kyung-Hoon/Reuters

Thirty-two days after Mr. Ghosn's initial arrest, when his
\href{https://www.nytimes.com/2019/03/04/world/asia/carlos-ghosn-bail-japan.html?module=inline}{release
on bail} appeared likely, the Japanese authorities rearrested him on new
charges that he shifted personal losses during the 2008 financial crisis
temporarily onto Nissan's books. On Monday a court extended his
detention until Jan. 11.

That Mr. Ghosn may have deceived regulators while enriching himself runs
afoul of cultural norms in Japan, where the public is more likely to
forgive corporate cover-ups when executives appear to be protecting the
company.

``Although you don't see it written down, there is almost a social
consensus that `OK, you did your crime, but you did it for the
company,''' said Seijiro Takeshita, dean and professor at the School of
Management and Information at the University of Shizuoka.

Or as Jesper Koll, who has worked in Japan for decades as an economist
and is head of Japan for WisdomTree investments in Tokyo, said: ``The
one thing that Japan does not want and would never tolerate is personal
greed.''

\hypertarget{as-the-world-ghosns}{%
\subsection{`As the World Ghosns'}\label{as-the-world-ghosns}}

Mr. Ghosn's longtime driver has been out of touch since shortly after
the arrest. The driver told the Ghosn children the day after their
father was detained that the Japanese authorities had found his car in
Tokyo. They tore up the leather seats and found only cat food.

Mr. Ghosn's chief of staff, Frédérique Le Greves, who arrived in Tokyo
the same day as Mr. Ghosn, has not made any statement and returned to
France after she learned of the arrest, a person close to the Ghosn
family said.

Their silence is one of many plot twists in the corporate saga. A person
close to the family has started to call it ``As the World Ghosns.''

Under Japanese law, only Mr. Ghosn's Japanese lawyer and representatives
from the French, Brazilian and Lebanese Embassies have been allowed to
visit or talk to him.

Mr. Ghosn's allies view his incarceration, with no foreseeable chance
for bail, as revenge by Nissan (and, by extension, Japan) on a foreign
adversary. He lives in a 16-by-10-foot cell with a tatami mat, a toilet
in a corner and the lights always on, in the same facility that once
housed the
\href{https://www.autonews.com/article/20181122/COPY01/311229975/ghosn-said-to-be-held-in-cold-detention-cell-as-misconduct-allegations-are-investigated}{death-row
inmates} who committed a deadly sarin attack on the Tokyo subway in
1995.

The frustration has led a few of Mr. Ghosn's longtime friends in France
to some extreme and possibly culturally insensitive metaphors. Two of
them compared the meticulously planned surprise arrest to the 1941
attack on Pearl Harbor, which killed 2,400 Americans.

Mr. Ghosn's children have learned from his visitors that he has lost
weight, at least 20 pounds. Prosecutors question him daily. Several
requests to the jail authorities for a mattress were denied, but a
Lebanese diplomat succeeded in getting him a thin cot and vitamin C
pills.

The books that Mr. Ghosn is reading in jail --- including ``When Things
Fall Apart,'' by Pema Chodron; ``Teachings of the Buddha,'' by Jack
Kornfield; and ``A Little Life,'' a dark novel by Hanya Yanagihara ---
speak to his state of mind. He has been denied other items, including
family photos, a pen and paper, dental floss (``He is a big flosser,''
his daughter Maya said) and an iPod Nano loaded with music by his
favorite, Phil Collins.

Mr. Ghosn's defenders, largely in the business community, contend that
he is being treated harshly because he is a foreigner. They claim that
the latest charge, rooted in dealings from 2008, was beyond the statute
of limitations for Japanese citizens. According to Japanese law, the
statute is tied not to citizenship but to how much time the accused has
spent outside Japan.

His defenders also said Japanese executives at Takata and Toshiba, who
were embroiled in serious accounting scandals in 2014 and 2015, didn't
receive the same harsh treatment or any jail time. (Three executives
from
\href{https://www.nytimes.com/2011/11/10/business/global/corporate-japan-rocked-by-scandal-at-olympus.html?module=inline}{Olympus}
were detained for nearly six weeks in 2012 and convicted of accounting
fraud but served no prison time.)

``It seems really strategic. It's a political fight,'' said Ralph
Jazzar, a banker in Paris and Mr. Ghosn's first cousin. He recited an
expression in French and then translated it: ``He who wants to get rid
of a great dog pretends the dog has the plague.''

Mr. Jazzar and Mr. Ghosn grew up together in a middle-class neighborhood
in Beirut. Mr. Ghosn, who was born in Rio de Janeiro, was 6 when his
Lebanese father moved the family to Beirut.

His sister, Claudine Bichara de Oliveira, said he was fascinated by cars
at an early age. She remembers him lying in the back seat of the family
car, ``closing his eyes and guessing the kind of car just by hearing its
horn.''

From Lebanon, Mr. Ghosn went to Paris to attend France's most
prestigious schools, Lycee Saint-Louis and the Ecole Polytechnique. And
then he worked his way up Michelin.

Whether in Lebanon or France, Mr. Ghosn always assumed the role of the
ambitious outsider. In 1989, he perfected his English and added America
to his résumé, soon becoming the chief executive of Michelin's North
American operations. Mr. Ghosn moved his young family to Greenville,
S.C. He took a road trip to see the Grand Canyon, Las Vegas and Los
Angeles, studying the world's largest tire market along the way.

If there was any community in which Mr. Ghosn seemed to finally fit, it
was the global elite, a coterie of chief executives and billionaire
philanthropists who have yachts in the south of France and standing
invitations to the World Economic Forum in Davos, Switzerland. (``If
Davos Were a Person, It Would Be Carlos Ghosn'' was the headline of a
Bloomberg profile last year.)
\href{https://www.nytimes.com/2020/01/07/business/carlos-ghosn-escape.html}{Mr.
Ghosn's new wife, Carole} Nahas, persuaded him to take ski lessons at
age 60.

Image

President Emmanuel Macron of France, left, visiting a Renault factory
with Mr. Ghosn in early November. In 2015, Mr. Macron called Mr. Ghosn's
\$8 million salary at Renault ``excessive.''Credit...Pool photo by
Etienne Laurent

Mr. Ghosn's unabashed globalism clashed with the current era of
inequality and off-with-their-heads isolationism. In 2015, Emmanuel
Macron, then the French finance minister, criticized Mr. Ghosn, calling
his \$8 million salary at Renault ``excessive.'' Early this year, an
auditor at Nissan began investigating the homes that a company
subsidiary had bought for Mr. Ghosn's personal use, according to a
person with knowledge of the investigation.

In an internal investigation, Nissan learned that a subsidiary set up in
the Netherlands ostensibly to fund venture capital investments had been
used to buy or rent corporate properties that Mr. Ghosn lived in when he
traveled, according to a person familiar with the investigation. Nissan
had invested 73 million euros (currently equivalent to about \$83
million) in the venture, known as Zi-A, and Mr. Kelly was put in charge
of it.

In addition to a 5,400-square-foot flat in Paris's elegant 16th
arrondissement, Zi-A bought an apartment in Rio in 2011 for \$6 million.
(The Ghosn family planned to spend Christmas there this year with his
ailing mother.) In Beirut, there is a salmon-hued mansion on a
tree-lined street that Zi-A paid \$8.75 million for in 2012, followed by
\$6 million in renovations and furnishings, according to a person
briefed on Nissan's investigation.

Mr. Ghosn's family said Nissan had known about the homes. ``Over 19
years, the company put these things in place to maximize his
productivity,'' his eldest child, Caroline Ghosn, 31, said in an
interview.

Mr. Ghosn hasn't been charged with any illegal activity related to his
corporate residences. Caroline Ghosn said media accounts about the homes
were part of Nissan's and Japanese prosecutors' efforts to ``muddy the
waters'' in a public-relations campaign against her father.

Nissan declined to comment, but a person familiar with its investigation
said the fact that the Dutch subsidiary was buying homes rather than
paying for car-related start-ups was among the red flags for internal
auditors. The person also pointed out that Nissan did not have
substantial operations in Beirut, the location of one of the disputed
homes.

\hypertarget{what-have-you-done-for-us-lately}{%
\subsection{`What Have You Done for Us
Lately?'}\label{what-have-you-done-for-us-lately}}

``Do not take this as a coup d'état,'' Mr. Saikawa, the current chief
executive of Nissan, whom Mr. Ghosn had mentored, told reporters hours
after the arrest.

Mr. Saikawa said he felt ``strong anger and despair'' over Nissan's
findings, but analysts and investors closely watching the company
believed that complicated interpersonal dynamics were at play.

Critics inside and outside Nissan had started to question whether Mr.
Ghosn's star had faded. In recent years, sales had slowed. The
miraculous turnaround he orchestrated started to stall. One former
executive, who spoke on the condition of anonymity, summed up the new
sentiment spreading inside the company as ``What have you done for us
lately?''

Image

Mr. Ghosn at a Nissan plant in 2011. That year, he unveiled his Power 88
plan, calling for Nissan to reach 8 percent profit margins and 8 percent
market share in the countries where it operated.Credit...Reuters

Midway through a plan known as Power 88, which Mr. Ghosn unveiled to
much fanfare in 2011, it became clear that Nissan would fall short of
the ambitious targets he had set. He wanted Nissan to reach 8 percent
profit margins and 8 percent market share in the countries where it
operated. Dealers complained that they were losing money and that Mr.
Ghosn's big incentives to buyers to meet his targets were eating into
their margins. They also grumbled that Nissan was selling too many
vehicles to rental companies that then would flood the secondary buyers'
market.

``They would sell cars in any manner and in any way without any regard
for what the long-term implications were,'' said Steve Kalafer, chief
executive of a chain of auto dealerships in New Jersey. After 36 years
of owning a Nissan dealership, Mr. Kalafer said, he sold it two years
ago because he objected to Mr. Ghosn's policies.

Mr. Ghosn's daughters said that in the past few years he had started on
a succession plan to help cement his legacy and plan for his retirement.
Mr. Ghosn explored what he called a ``reimagining of the alliance'' that
would permanently bind Nissan and Renault. And he picked Mr. Saikawa,
his close confidant, to succeed him as chief executive.

``He is like Carlos Ghosn in many ways,'' Patrick le Quément, a former
head of design at Renault, said of Mr. Saikawa. ``Not much feeling.''

But as Mr. Ghosn sought to integrate Nissan's operations more closely
with Renault, maybe connecting them permanently, the relationship was
getting shaky. Some Nissan executives, engineers and marketing staff
began to resent what they saw as Renault's unfairly piggybacking on
Nissan's technology, research and brand strength, according to three
former managerial employees.

The French saw things another way, accusing Mr. Ghosn of favoring the
Japanese and Nissan and blocking Renault's expansion into China, the
world's largest car market, to clear the field for Nissan.

``We felt he was escaping us,'' Mr. le Quément said. ``A lot of
decisions were being taken that were to the detriment of Renault.''

Asked about merger discussions, a Nissan spokesman, Nicholas Maxfield,
said, ``It is true that the `Alliance 2022' six-year plan
\href{https://www.alliance-2022.com/news/alliance-2022-announcement/}{announced
last year} calls for additional synergies and further convergence among
member companies in specific operational areas.''

Image

As tensions at Nissan grew, Mr. Ghosn mused about retirement, his
children said, telling them that he hoped to teach, write history books
and even learn to play an ancient flute.Credit...Ghosn Family

As tensions grew, Mr. Ghosn mused about getting out. On his long walks
around Tokyo when he couldn't sleep, he would pass an old man playing
the shakuhachi, an end-blown bamboo flute that dates to seventh-century
Japan. Mr. Ghosn told his children that in retirement he hoped to learn
how to play it. A Byzantine Empire buff, he said he also might write
history books or lecture M.B.A. students.

Then late in 2017, speculation spread that Mr. Ghosn and Mr. Saikawa's
relationship had become strained after Nissan faced accusations that it
had been using uncertified technicians for vehicle inspections, leading
to a recall and
\href{https://www.nytimes.com/2017/10/19/business/nissan-japan-suspends-production.html}{halts
in production}. Mr. Ghosn left Mr. Saikawa to take the blame. As the new
chief executive offered a deeply apologetic bow, as is customary in
Japan, and told a voracious news media that the carmaker had ``done
something inexcusable,'' Mr. Ghosn was nowhere to be seen.

Caroline and Maya Ghosn used to joke that Nissan was the ``very
demanding fifth child'' in their family. To them, Mr. Saikawa's
statements (without an apologetic bow) the night of their father's
arrest were proof that his fall was akin to a palace coup.

What doesn't make sense to Mr. Ghosn's friends and family is how the man
with a preternatural talent for seeing around every corner --- whether
maneuvering through Japanese bureaucracy, managing French ministers or
designing a midsize S.U.V. --- didn't see this coming. Maybe, they
theorized, it was the jet lag and the 100 days a year he spent on an
airplane, and that old man with the flute whom he saw himself becoming.

Mr. Jazzar, his cousin, said Mr. Ghosn had failed, in the end, at the
``P.Y.A.'' approach to management: Protect Your Ass.

``God only knows what is going on inside his head,'' Mr. Jazzar said.

Advertisement

\protect\hyperlink{after-bottom}{Continue reading the main story}

\hypertarget{site-index}{%
\subsection{Site Index}\label{site-index}}

\hypertarget{site-information-navigation}{%
\subsection{Site Information
Navigation}\label{site-information-navigation}}

\begin{itemize}
\tightlist
\item
  \href{https://help.nytimes.com/hc/en-us/articles/115014792127-Copyright-notice}{©~2020~The
  New York Times Company}
\end{itemize}

\begin{itemize}
\tightlist
\item
  \href{https://www.nytco.com/}{NYTCo}
\item
  \href{https://help.nytimes.com/hc/en-us/articles/115015385887-Contact-Us}{Contact
  Us}
\item
  \href{https://www.nytco.com/careers/}{Work with us}
\item
  \href{https://nytmediakit.com/}{Advertise}
\item
  \href{http://www.tbrandstudio.com/}{T Brand Studio}
\item
  \href{https://www.nytimes.com/privacy/cookie-policy\#how-do-i-manage-trackers}{Your
  Ad Choices}
\item
  \href{https://www.nytimes.com/privacy}{Privacy}
\item
  \href{https://help.nytimes.com/hc/en-us/articles/115014893428-Terms-of-service}{Terms
  of Service}
\item
  \href{https://help.nytimes.com/hc/en-us/articles/115014893968-Terms-of-sale}{Terms
  of Sale}
\item
  \href{https://spiderbites.nytimes.com}{Site Map}
\item
  \href{https://help.nytimes.com/hc/en-us}{Help}
\item
  \href{https://www.nytimes.com/subscription?campaignId=37WXW}{Subscriptions}
\end{itemize}
