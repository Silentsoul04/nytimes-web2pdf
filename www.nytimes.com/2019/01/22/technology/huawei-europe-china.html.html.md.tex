Sections

SEARCH

\protect\hyperlink{site-content}{Skip to
content}\protect\hyperlink{site-index}{Skip to site index}

\href{https://www.nytimes.com/section/technology}{Technology}

\href{https://myaccount.nytimes.com/auth/login?response_type=cookie\&client_id=vi}{}

\href{https://www.nytimes.com/section/todayspaper}{Today's Paper}

\href{/section/technology}{Technology}\textbar{}How Huawei Wooed Europe
With Sponsorships, Investments and Promises

\url{https://nyti.ms/2RZava0}

\begin{itemize}
\item
\item
\item
\item
\item
\item
\end{itemize}

Advertisement

\protect\hyperlink{after-top}{Continue reading the main story}

Supported by

\protect\hyperlink{after-sponsor}{Continue reading the main story}

\hypertarget{how-huawei-wooed-europe-with-sponsorships-investments-and-promises}{%
\section{How Huawei Wooed Europe With Sponsorships, Investments and
Promises}\label{how-huawei-wooed-europe-with-sponsorships-investments-and-promises}}

\includegraphics{https://static01.nyt.com/images/2019/01/18/business/00europehuawei01/00europehuawei01-articleLarge.jpg?quality=75\&auto=webp\&disable=upscale}

By \href{https://www.nytimes.com/by/adam-satariano}{Adam Satariano} and
\href{https://www.nytimes.com/by/raymond-zhong}{Raymond Zhong}

\begin{itemize}
\item
  Jan. 22, 2019
\item
  \begin{itemize}
  \item
  \item
  \item
  \item
  \item
  \item
  \end{itemize}
\end{itemize}

\href{https://cn.nytimes.com/technology/20190123/huawei-europe-china/}{阅读简体中文版}\href{https://cn.nytimes.com/technology/20190123/huawei-europe-china/zh-hant/}{閱讀繁體中文版}

LONDON --- When new employees join Huawei, they are given books about
the Chinese telecommunications company's achievements. One featured
accomplishment is how the company has thrived in Europe.

A chapter in one of the books describes the hard work of Huawei
employees to win over European telecom providers and sell them equipment
that forms the backbone of mobile wireless networks. It says little
about how the company has also subtly lobbied, promised jobs and made
research investments to ingratiate itself with European governments over
the last 15 years.

In Britain, for example, Huawei set up a special corporate board, led by
John Browne, a former chief executive of the oil giant BP. It donated to
schools, including Cambridge University; held parties for political
leaders; and sponsored prominent charities like the Prince's Trust,
founded by Prince Charles. In Germany, Huawei opened facilities to
conduct research on new innovations and sponsored events, including the
recent convention of Germany's governing Christian Democratic Union
party.

European governments have been welcoming. Last year, after Huawei
announced a five-year commitment to spend
\href{https://www.huawei.com/en/press-events/news/2018/2/Huawei-new3bn-fiveyear-commitment-UK}{3
billion pounds, or about \$3.8 billion, to expand} in Britain, where the
company employs 1,500 workers, Prime Minister Theresa May met in Beijing
with Sun Yafang, Huawei's chairwoman. Chancellor Angela Merkel of
Germany posed for
\href{https://www.huawei.eu/media-centre/press-releases/chancellor-merkel-visits-huawei-dlt-2018}{pictures}
at the company's booth at a trade show last year.

Yet despite the careful cultivation, Huawei's position in Europe is now
at risk of unraveling. The United States has moved to restrict the use
of Chinese technology because of concerns that it is being used for
espionage. Last month, the American authorities asked Canada to detain a
Huawei executive, who is the daughter of the company's founder, on
charges of committing bank fraud to help the company's business in Iran.
And federal prosecutors in Seattle are also investigating Huawei for
\href{https://www.nytimes.com/2019/01/16/technology/huawei-investigation-trade-secrets.html?rref=collection\%2Fbyline\%2Fraymond-zhong\&action=click\&contentCollection=undefined\&region=stream\&module=stream_unit\&version=latest\&contentPlacement=1\&pgtype=collection}{intellectual
property theft}.

\emph{{[}Read More:}
\href{https://www.nytimes.com/2019/01/22/us/politics/meng-wanzhou-extradition.html?action=click\&module=Intentional\&pgtype=Article}{\emph{The
United States plans to formally make}} \emph{its extradition request
within a week.{]}}

The fallout is growing across Europe, which has become Huawei's biggest
market outside China, foreshadowing what the company faces in the rest
of the world. This month, one of its
\href{https://www.nytimes.com/2019/01/11/world/europe/poland-china-huawei-spy.html}{employees
was arrested} in Poland and charged with espionage.

Officials in Germany, France and the Czech Republic are now among those
considering restricting Huawei from the next-generation wireless
networks,
\href{https://www.nytimes.com/2018/12/31/technology/personaltech/5g-what-you-need-to-know.html?module=inline}{known
as 5G}. The head of Britain's intelligence service, MI6, has raised
alarms about using Chinese networking technology. European carriers,
including Deutsche Telekom, are reassessing their use of Huawei. And on
Thursday, Oxford University announced that it would
\href{https://www.bbc.co.uk/news/business-46911265}{suspend donations}
and scholarships from Huawei.

``Until there were red flags on security risks, it was smooth sailing
for them in Europe,'' Thorsten Benner, a founder and director of the
Global Public Policy Institute, a policy think tank in Berlin, said of
Huawei. ``The movement that you've seen over the past three months is
all in one direction: to find regulatory measures to curtail the use of
Chinese equipment in Europe.''

Heli Tiirmaa-Klaar, an Estonian diplomat involved in cybersecurity
discussions with American and European officials about Huawei, said
Europe was shifting on Huawei because of suspicions about China rather
than specific actions by the company. She highlighted China's history of
hacking and stealing trade secrets, its poor record on human rights and
internet censorship, and Chinese
\href{https://www.nytimes.com/2017/05/31/business/china-cybersecurity-law.html}{cybersecurity
rules} that could require network operators to defend national security
interests.

``The Chinese are increasingly departing from the common vision that
we've had at the U.N. level about what are cybernorms and how we follow
international law in cyberspace,'' she said. ``They have no willingness
to play along in this space.''

Huawei (pronounced ``HWA-way'') has
\href{https://www.nytimes.com/2019/01/15/technology/huawei-ren-zhengfei.html}{steadfastly
denied wrongdoing}. In a statement, the company said it had
``established strong relations with customers, suppliers and governments
across Europe, where cybersecurity has been our top priority.'' It added
that its growth in Europe was the result of its strong track record, not
lobbying. Interacting with government officials is common among large
corporations, Huawei said.

For European countries, disentangling from Huawei won't be easy. Its
equipment is a crucial part of wireless infrastructure in Europe, and
the company has spent hundreds of millions of dollars on 5G research,
opening testing hubs in Britain, Germany and Poland. The company has
said countries that ban Huawei, like the United States, risk delaying
construction of the new hyperfast network.

\includegraphics{https://static01.nyt.com/images/2019/01/18/business/00europehuawei02/merlin_148010739_fe22def2-8e7c-4454-884b-32fd35988671-articleLarge.jpg?quality=75\&auto=webp\&disable=upscale}

Huawei, based in Shenzhen, China, entered Europe in 2001 but struggled
to gain a foothold. It kept pushing because it wanted to expand
internationally. The European market became even more important after
the company was mostly shut out of the United States in 2012 over
security concerns.

At one point in 2004, Huawei was so eager to woo European customers that
it outfitted a shipping container with its wireless equipment and parked
it in front of the headquarters of phone carriers in Germany. The
marketing stunt was intended to get employees of those carriers to stop
and look.

``A lot of them said, `Who are you?''' said Stefan Scheuerle, a former
Huawei sales manager in Europe who is now chief revenue officer at
Sensorberg, a German tech company.

In a breakthrough, Huawei struck a deal to provide equipment to BT
Group, the British telecom giant, in 2005. To prepare its bid for the BT
project, Huawei moved about 100 Chinese workers to hastily rented
apartments outside London, according to ``Growing Up in a Hail of
Gunfire,'' a collection of essays used as training material for new
hires. The employees often worked until 3 or 4 a.m., and because they
didn't have credit cards, they had to use wads of cash to buy groceries.

In 2005, Huawei also secured a deal with the European carrier Vodafone
to sell about \$1 million of gear. Even so, Mr. Scheuerle said, he was
quickly given a new goal: Sell Vodafone \$1 billion of equipment within
three years. Huawei soon opened an office within steps of Vodafone's
complex in Düsseldorf, Germany.

``When I left less than three years later, we were at \$850 million ---
we were almost there,'' Mr. Scheuerle said.

Huawei moved other employees from China to help in Europe, and hired
many non-Chinese employees. It was a jarring cultural change for
newcomers unfamiliar with
\href{https://www.nytimes.com/2018/12/18/technology/huawei-workers-iran-sanctions.html}{Huawei's
blunt management style}; Mr. Scheuerle said staff emails criticized
people by name for unsatisfactory work.

Mr. Scheuerle held workshops for new employees to help them adapt to the
company. He advised them to befriend Chinese colleagues with better
knowledge of what was happening at Huawei headquarters in Shenzhen.

Today, 12,000 of Huawei's 180,000 employees are in Europe, up from 7,300
in 2013. The company has vowed to hire nearly 3,000 more by next year.
In 2017, it earned more than \$20 billion in revenue in Europe, the
Middle East and Africa, roughly a quarter of its total business.

Even when American officials warned allies of Huawei's risks, the
company largely deflected the concerns. It opened a testing center where
British officials inspected its products. It opened 23 research and
development facilities in 14 European countries, and supported work at
more than 150 universities. The company also hosts government officials
and business leaders from Europe at its Shenzhen headquarters.

In 2011, Huawei created the board to oversee operations in Britain. In
reality, the group has a limited view of Huawei's inner workings and few
responsibilities beyond quarterly meetings and attending some corporate
events such as annual summer and winter parties, two people familiar
with its activities said.

Huawei said the board had legal responsibilities of directors overseeing
compliance with laws, health and safety, and potential abuses of power.

Now, as criticism mounts, Huawei is meeting with customers and
government officials to assuage concerns. This includes allowing
\href{https://www.huawei.eu/media-centre/press-releases/huawei-opens-security-innovation-lab-bonn}{German
officials} to inspect its engineering and code.

Next month at the mobile tech conference MWC Barcelona, formerly the
Mobile World Congress, Huawei plans to announce new handsets and provide
an update of its 5G plans. A new advertising campaign is also in the
works, the company said.

Advertisement

\protect\hyperlink{after-bottom}{Continue reading the main story}

\hypertarget{site-index}{%
\subsection{Site Index}\label{site-index}}

\hypertarget{site-information-navigation}{%
\subsection{Site Information
Navigation}\label{site-information-navigation}}

\begin{itemize}
\tightlist
\item
  \href{https://help.nytimes.com/hc/en-us/articles/115014792127-Copyright-notice}{©~2020~The
  New York Times Company}
\end{itemize}

\begin{itemize}
\tightlist
\item
  \href{https://www.nytco.com/}{NYTCo}
\item
  \href{https://help.nytimes.com/hc/en-us/articles/115015385887-Contact-Us}{Contact
  Us}
\item
  \href{https://www.nytco.com/careers/}{Work with us}
\item
  \href{https://nytmediakit.com/}{Advertise}
\item
  \href{http://www.tbrandstudio.com/}{T Brand Studio}
\item
  \href{https://www.nytimes.com/privacy/cookie-policy\#how-do-i-manage-trackers}{Your
  Ad Choices}
\item
  \href{https://www.nytimes.com/privacy}{Privacy}
\item
  \href{https://help.nytimes.com/hc/en-us/articles/115014893428-Terms-of-service}{Terms
  of Service}
\item
  \href{https://help.nytimes.com/hc/en-us/articles/115014893968-Terms-of-sale}{Terms
  of Sale}
\item
  \href{https://spiderbites.nytimes.com}{Site Map}
\item
  \href{https://help.nytimes.com/hc/en-us}{Help}
\item
  \href{https://www.nytimes.com/subscription?campaignId=37WXW}{Subscriptions}
\end{itemize}
