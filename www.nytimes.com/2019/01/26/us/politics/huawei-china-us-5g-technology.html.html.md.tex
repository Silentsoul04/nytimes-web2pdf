Sections

SEARCH

\protect\hyperlink{site-content}{Skip to
content}\protect\hyperlink{site-index}{Skip to site index}

\href{https://www.nytimes.com/section/politics}{Politics}

\href{https://myaccount.nytimes.com/auth/login?response_type=cookie\&client_id=vi}{}

\href{https://www.nytimes.com/section/todayspaper}{Today's Paper}

\href{/section/politics}{Politics}\textbar{}In 5G Race With China, U.S.
Pushes Allies to Fight Huawei

\url{https://nyti.ms/2S6LObM}

\begin{itemize}
\item
\item
\item
\item
\item
\item
\end{itemize}

Advertisement

\protect\hyperlink{after-top}{Continue reading the main story}

Supported by

\protect\hyperlink{after-sponsor}{Continue reading the main story}

\hypertarget{in-5g-race-with-china-us-pushes-allies-to-fight-huawei}{%
\section{In 5G Race With China, U.S. Pushes Allies to Fight
Huawei}\label{in-5g-race-with-china-us-pushes-allies-to-fight-huawei}}

\includegraphics{https://static01.nyt.com/images/2019/01/27/us/politics/27dc-huawei1/merlin_149755908_ae9a6f84-554a-48ee-b836-a615b70d23e6-articleLarge.jpg?quality=75\&auto=webp\&disable=upscale}

By \href{https://www.nytimes.com/by/david-e-sanger}{David E. Sanger},
\href{https://www.nytimes.com/by/julian-e-barnes}{Julian E. Barnes},
\href{https://www.nytimes.com/by/raymond-zhong}{Raymond Zhong} and
\href{https://www.nytimes.com/by/marc-santora}{Marc Santora}

\begin{itemize}
\item
  Jan. 26, 2019
\item
  \begin{itemize}
  \item
  \item
  \item
  \item
  \item
  \item
  \end{itemize}
\end{itemize}

\href{https://cn.nytimes.com/usa/20190128/huawei-china-us-5g-technology/}{阅读简体中文版}\href{https://cn.nytimes.com/usa/20190128/huawei-china-us-5g-technology/zh-hant/}{閱讀繁體中文版}

Jeremy Hunt, the British foreign minister, arrived in Washington last
week for a whirlwind of meetings facing a critical question: Should
Britain risk its relationship with Beijing and agree to the Trump
administration's request to ban Huawei, China's leading
telecommunications producer, from building its next-generation computer
and phone networks?

Britain is not the only American ally feeling the heat. In Poland,
officials are also under pressure from the United States to bar Huawei
from building its
\href{https://www.nytimes.com/2018/12/31/technology/personaltech/5g-what-you-need-to-know.html?module=inline}{fifth
generation, or 5G, network}. Trump officials suggested that future
deployments of American troops --- including the prospect of a permanent
base labeled ``Fort Trump'' --- could hinge on Poland's decision.

And a delegation of American officials showed up last spring in Germany,
where most of Europe's giant fiber-optic lines connect and Huawei wants
to build the switches that make the system hum. Their message: Any
economic benefit of using cheaper Chinese telecom equipment is
outweighed by the security threat to the NATO alliance.

Over the past year, the United States has embarked on a stealthy,
occasionally threatening, global campaign to prevent Huawei and other
Chinese firms from participating in the most dramatic remaking of the
plumbing that controls the internet since it sputtered into being, in
pieces, 35 years ago.

The administration contends that the world is engaged in a new arms race
--- one that involves technology, rather than conventional weaponry, but
poses just as much danger to America's national security. In an age when
the most powerful weapons, short of nuclear arms, are cyber-controlled,
whichever country dominates 5G will gain an economic, intelligence and
military edge for much of this century.

The transition to 5G --- already beginning in prototype systems in
cities from Dallas to Atlanta --- is likely to be more revolutionary
than evolutionary. What consumers will notice first is that the network
is faster --- data should download almost instantly, even over cellphone
networks.

It is the first network built to serve the sensors, robots, autonomous
vehicles and other devices that will continuously feed each other vast
amounts of data, allowing factories, construction sites and even whole
cities to be run with less moment-to-moment human intervention. It will
also enable greater use of virtual reality and artificial intelligence
tools.

But what is good for consumers is also good for intelligence services
and cyberattackers. The 5G system is a physical network of switches and
routers. But it is more reliant on layers of complex software that are
far more adaptable, and constantly updating, in ways invisible to users
--- much as an iPhone automatically updates while charging overnight.
That means whoever controls the networks controls the information flow
--- and may be able to change, reroute or copy data without users'
knowledge.

In interviews with current and former senior American government
officials, intelligence officers and top telecommunications executives,
it is clear that the potential of 5G has created a zero-sum calculus in
the Trump White House --- a conviction that there must be a single
winner in this arms race, and the loser must be banished. For months,
the White House has been
\href{https://www.nytimes.com/2018/12/11/us/politics/trump-china-trade.html}{drafting
an executive order}, expected in the coming weeks, that would
effectively ban United States companies from using Chinese-origin
equipment in critical telecommunications networks. That goes far beyond
the existing rules, which ban such equipment only from government
networks.

Nervousness about Chinese technology has long existed in the United
States, fueled by the fear that the Chinese could insert a ``back door''
into telecom and computing networks that would allow Chinese security
services to intercept military, government and corporate communications.
And Chinese cyberintrusions of American companies and government
entities have occurred repeatedly,
\href{https://www.nytimes.com/2018/12/20/us/politics/us-and-other-nations-to-announce-china-crackdown.html}{including
by hackers suspected} of working on behalf of China's Ministry of State
Security.

But the concern has taken on more urgency as countries around the world
begin deciding which equipment providers will build their 5G networks.

American officials say the old process of looking for ``back doors'' in
equipment and software made by Chinese companies is the wrong approach,
as is searching for ties between specific executives and the Chinese
government. The bigger issue, they argue, is the increasingly
authoritarian nature of the Chinese government, the fading line between
independent business and the state and new laws that will give Beijing
the power to look into, or maybe even take over, networks that companies
like Huawei have helped build and maintain.

``It's important to remember that Chinese company relationships with the
Chinese government aren't like private sector company relationships with
governments in the West,'' said William R. Evanina, the director of
America's National Counterintelligence and Security Center. ``China's
2017 National Intelligence Law requires Chinese companies to support,
provide assistance and cooperate in China's national intelligence work,
wherever they operate.''

The White House's focus on Huawei coincides with the Trump
administration's broader crackdown on China, which has involved sweeping
tariffs on Chinese goods, investment restrictions and the
\href{https://www.nytimes.com/2018/10/30/us/politics/justice-department-china-espionage.html}{indictments
of several Chinese nationals} accused of hacking and cyberespionage.
President Trump has accused China of ``ripping off our country'' and
plotting to grow stronger at America's expense.

Mr. Trump's views, combined with a lack of hard evidence implicating
Huawei in any espionage, have prompted some countries to question
whether America's campaign is really about national security or if it is
aimed at preventing China from gaining a competitive edge.

Administration officials see little distinction in those goals.

``President Trump has identified overcoming this economic problem as
critical, not simply to right the balance economically, to make China
play by the rules everybody else plays by, but to prevent an imbalance
in political/military power in the future as well,'' John R. Bolton, Mr.
Trump's national security adviser,
\href{https://www.washingtontimes.com/news/2019/jan/25/john-bolton-explains-trumps-strategy-on-north-kore/}{told
The Washington Times} on Friday. ``The two aspects are very closely tied
together in his mind.''

The administration is warning allies that the next six months are
critical. Countries are beginning to auction off radio spectrum for new,
5G cellphone networks and decide on multibillion-dollar contracts to
build the underlying switching systems. This past week, the Federal
Communications Commission announced that it had concluded its first
high-band 5G spectrum auction.

The Chinese government sees this moment as its chance to wire the world
--- especially European, Asian and African nations that find themselves
increasingly beholden to Chinese economic power.

``This will be almost more important than electricity,'' said Chris
Lane, a telecom analyst in Hong Kong for Sanford C. Bernstein.
``Everything will be connected, and the central nervous system of these
smart cities will be your 5G network.''

\includegraphics{https://static01.nyt.com/images/2019/01/27/us/politics/27dc-huawei2/merlin_149673720_81bb518b-8b5f-4df9-96ba-4fdfab297b01-articleLarge.jpg?quality=75\&auto=webp\&disable=upscale}

\hypertarget{a-new-red-scare}{%
\subsection{A New Red Scare?}\label{a-new-red-scare}}

So far, the fear swirling around Huawei is almost entirely theoretical.
Current and former American officials whisper that classified reports
implicate the company in possible Chinese espionage but have produced
none publicly. Others familiar with the secret case against the company
say there is no smoking gun --- just a heightened concern about the
firm's rising technological dominance and the new Chinese laws that
require Huawei to submit to requests from Beijing.

Ren Zhengfei, Huawei's founder,
\href{https://www.nytimes.com/2019/01/15/technology/huawei-ren-zhengfei.html}{has
denied} that his company spied for China. ``I still love my country. I
support the Communist Party of China. But I will never do anything to
harm any other nation,''
\href{https://www.scmp.com/tech/big-tech/article/2182367/transcript-huawei-founder-ren-zhengfeis-responses-media-questions}{he
said} earlier this month.

Australia last year banned Huawei and another Chinese manufacturer, ZTE,
from supplying 5G equipment. Other nations are wrestling with whether to
follow suit and risk inflaming China, which could hamper their access to
the growing Chinese market and deprive them of cheaper Huawei products.

Government officials in places like Britain note that Huawei has already
invested heavily in older-style networks --- and has employed Britons to
build and run them. And they argue that Huawei isn't going away --- it
will run the networks of half the world, or more, and will have to be
connected, in some way, to the networks of the United States and its
allies.

Yet BT Group, the British telecom giant, has plans to rip out part of
Huawei's existing network. The company says that was part of its plans
after acquiring a firm that used existing Huawei equipment; American
officials say it came after Britain's intelligence services warned of
growing risks. And Vodafone Group, which is based in London, said on
Friday that it would
\href{https://www.nytimes.com/2019/01/25/technology/vodafone-huawei.html}{temporarily
stop buying Huawei equipment} for parts of its 5G network.

Nations have watched warily as
\href{https://www.nytimes.com/2019/01/18/world/canada/canada-china-rift.html}{China
has retaliated} against countries that cross it. In December, Canada
arrested a top Huawei executive,
\href{https://www.nytimes.com/2018/12/07/technology/meng-wanzhou-huawei-arrest.html}{Meng
Wanzhou}, at the request of the United States. Ms. Meng, who is Mr.
Ren's daughter, has been
\href{https://www.nytimes.com/2018/12/07/technology/huawei-meng-wanzhou-fraud.html}{accused
of defrauding banks} to help Huawei's business evade sanctions against
Iran. Since her arrest, China has detained two Canadian citizens and
sentenced to death a third Canadian, who had previously been given 15
years in prison for drug smuggling.

``Europe is fascinating because they have to take sides,'' said Philippe
Le Corre, nonresident senior fellow at the Carnegie Endowment for
International Peace. ``They are in the middle. All these governments,
they need to make decisions. Huawei is everywhere.''

Image

A Huawei store in Warsaw. This month, the Polish government made two
high-profile espionage arrests, including an employee of
Huawei.Credit...Maciek Nabrdalik for The New York Times

\hypertarget{growing-suspicions}{%
\subsection{Growing Suspicions}\label{growing-suspicions}}

This month, the Polish government made
\href{https://www.nytimes.com/2019/01/11/world/europe/poland-china-huawei-spy.html}{two
high-profile espionage arrests}: a former intelligence official, Piotr
Durbajlo, and Wang Weijing, an employee of Huawei. The arrests are the
strongest evidence so far that links Huawei with spying activities.

Mr. Wang, who was
\href{https://www.nytimes.com/2019/01/12/world/asia/huawei-wang-weijing-poland.html}{quickly
fired} by Huawei, has been accused of working for Chinese intelligence
agencies, said a top former Polish intelligence official. Mr. Wang,
according to American diplomats, was the handler of Mr. Durbajlo, who
appears to have helped the Chinese penetrate the Polish government's
most secure communications network.

A senior American official said the case was a prime example of how the
Chinese government plants intelligence operatives inside Huawei's vast
global network. Those operatives potentially have access to overseas
communications networks and can conduct espionage that the affected
companies are not aware of, the official said.

Huawei said Mr. Wang had brought ``disrepute'' on the company and his
actions had nothing to do with its operations.

Mr. Wang's lawyer, Bartlomiej Jankowski, says his client has been caught
up in a geopolitical tug of war between the United States and China.

American and British officials had already grown concerned about
Huawei's abilities after cybersecurity experts, combing through the
company's source code to look for back doors, determined that Huawei
could remotely access and control some networks from the company's
Shenzhen headquarters.

On careful examination, the code that Huawei had installed in its
network-control software did not appear to be malicious. Nor was it
hidden. It appeared to be part of a system to update remote networks and
diagnose trouble. But in some circumstances, it could also route traffic
around corporate data centers --- where firms monitor and control their
networks --- and its mere existence is now cited as evidence that
hackers or Chinese intelligence could use Huawei equipment to penetrate
millions of networks.

American officials and academics say Chinese telecommunications
companies have also temporarily hijacked parts of the internet,
rerouting basic traffic from the United States and Canada to China.

\href{https://scholarcommons.usf.edu/cgi/viewcontent.cgi?article=1050\&context=mca}{One
academic paper}, co-written by Chris C. Demchak, a Naval War College
professor, outlined how traffic from Canada meant for South Korea was
redirected to China for six months. That 2016 attack has been repeated,
according to American officials, and provides opportunity for espionage.

Last year, AT\&T and Verizon stopped selling Huawei phones in their
stores after Huawei begin equipping the devices with its own sets of
computer chips --- rather than relying on American or European
manufacturers. The National Security Agency quietly raised alarms that
with Huawei supplying its own parts, the Chinese company would control
every major element of its networks. The N.S.A. feared it would no
longer be able to rely on American and European providers to warn of any
evidence of malware, spying or other covert action.

Image

An assembly line at Huawei's cellphone plant in Dongguan, China. The
company has already surpassed Apple as the world's second biggest
cellphone provider.Credit...Qilai Shen/Bloomberg

\hypertarget{the-rise-of-huawei}{%
\subsection{The Rise of Huawei}\label{the-rise-of-huawei}}

In three decades, Huawei has transformed itself from a small reseller of
low-end phone equipment into a global giant with a dominant position in
one of the crucial technologies of the new century.

Last year, Huawei edged out Apple as the second-biggest provider of
cellphones around the world. Richard Yu, who heads the company's
consumer business, said in Beijing several days ago that ``even without
the U.S. market we will be No. 1 in the world,'' by the end of this year
or sometime in 2020.

The company was founded in 1987 by Mr. Ren, a former People's Liberation
Army engineer who has become one of China's most successful
entrepreneurs.

American officials say the company started through imitation, and even
theft, of American technology. Cisco Systems sued Huawei in 2003,
\href{https://www.nytimes.com/2003/10/02/business/technology-cisco-agrees-to-suspend-patent-suit-for-6-months.html}{saying
it had illegally copied} the American company's source code. The two
companies settled out of court.

But Huawei did not just imitate. It opened research centers (including
one in California) and built alliances with leading universities around
the world. Last year, it generated \$100 billion in revenue, twice as
much as Cisco and significantly more than IBM. Its ability to deliver
well-made equipment at a lower cost than Western firms drove
once-dominant players like Motorola and Lucent out of the
telecom-equipment industry.

While American officials refuse to discuss it, the government snooping
was a two-way street. As early as 2010, the N.S.A.
\href{https://www.nytimes.com/2014/03/23/world/asia/nsa-breached-chinese-servers-seen-as-spy-peril.html}{secretly
broke into}
\href{https://www.nytimes.com/2014/03/23/world/asia/nsa-breached-chinese-servers-seen-as-spy-peril.html}{Huawei's
headquarters, in an operation,}
\href{https://www.nytimes.com/2014/03/23/world/asia/nsa-breached-chinese-servers-seen-as-spy-peril.html}{code-named
``Shotgiant,''} a discovery revealed by Edward J. Snowden, the former
N.S.A. contractor now living in exile in Moscow.

Documents show that the N.S.A. was looking to prove suspicions that
Huawei was secretly controlled by the People's Liberation Army --- and
that Mr. Ren never really left the powerful army unit. It never found
the evidence, according to former officials. But the Snowden documents
also show that the N.S.A. had another goal: to better understand
Huawei's technology and look for potential back doors. This way, when
the company sold equipment to American adversaries, the N.S.A. would be
able to target those nations' computer and telephone networks to conduct
surveillance and, if necessary, offensive cyberoperations.

In other words, the Americans were trying to do to Huawei the exact
thing they are now worried Huawei will do to the United States.

Image

President Trump met with Andrzej Duda, his Polish counterpart, last
year. Mr. Duda has suggested that the United States build a \$2 billion
base and training area, which Mr. Duda only half-jokingly called ``Fort
Trump.''Credit...Doug Mills/The New York Times

\hypertarget{a-global-campaign}{%
\subsection{A Global Campaign}\label{a-global-campaign}}

After an uproar in 2013 about Huawei's
\href{https://www.nytimes.com/2019/01/22/technology/huawei-europe-china.html}{growing
dominance in Britain}, the country's powerful Intelligence and Security
Committee, a parliamentary body, argued for banning Huawei, partly
because of Chinese cyberattacks aimed at the British government. It was
overruled, but Britain created a system to require that Huawei make its
hardware and source code available to GCHQ, the country's famous
code-breaking agency.

In July, Britain's National Cyber Security Center for the first time
\href{https://assets.publishing.service.gov.uk/government/uploads/system/uploads/attachment_data/file/727415/20180717_HCSEC_Oversight_Board_Report_2018_-_FINAL.pdf}{said
publicly} that questions about Huawei's current practices and the
complexity and dynamism of the new 5G networks meant it would be
difficult to find vulnerabilities.

At roughly the same time, the N.S.A., at a series of classified meetings
with telecommunications executives, had to decide whether to let Huawei
bid for parts of the American 5G networks. AT\&T and Verizon argued
there was value in letting Huawei set up a ``test bed'' in the United
States since it would have to reveal the source code for its networking
software. Allowing Huawei to bid would also drive the price of building
the networks down, they argued.

The director of the N.S.A. at the time, Adm. Michael S. Rogers, never
approved the move and Huawei was blocked.

In July 2018, with these decisions swirling, Britain, the United States
and other members of the ``Five Eyes'' intelligence-sharing alliance met
for their annual meeting in Halifax, Nova Scotia, where Chinese
telecommunications companies, Huawei and 5G networks were at the top of
the agenda. They decided on joint action to try to block the company
from building new networks in the West.

American officials are trying to make clear with allies around the world
that the war with China is not just about trade but a battle to protect
the national security of the world's leading democracies and key NATO
members.

On Tuesday, the heads of American intelligence agencies will appear
before the Senate to deliver their annual threat assessment, and they
are expected to cite 5G investments by Chinese telecom companies,
including Huawei, as a threat.

In Poland, the message has quietly been delivered that countries that
use Chinese telecommunications networks would be unsafe for American
troops, according to people familiar with the internal discussions.

That has gotten Poland's attention, given that its president, Andrzej
Duda, visited the White House in September and presented a plan to build
a \$2 billion base and training area, which Mr. Duda only half-jokingly
called ``Fort Trump.''

Col. Grzegorz Malecki, now retired, who was the head of the Foreign
Intelligence Agency in Poland, said it was understandable that the
United States would want to avoid potentially compromising its troops.

``And control over the 5G network is such a potentially dangerous
tool,'' said Mr. Malecki, now board president of the Institute of
Security and Strategy. ``From Poland's perspective, securing this troop
presence outweighs all other concerns.''

Advertisement

\protect\hyperlink{after-bottom}{Continue reading the main story}

\hypertarget{site-index}{%
\subsection{Site Index}\label{site-index}}

\hypertarget{site-information-navigation}{%
\subsection{Site Information
Navigation}\label{site-information-navigation}}

\begin{itemize}
\tightlist
\item
  \href{https://help.nytimes.com/hc/en-us/articles/115014792127-Copyright-notice}{©~2020~The
  New York Times Company}
\end{itemize}

\begin{itemize}
\tightlist
\item
  \href{https://www.nytco.com/}{NYTCo}
\item
  \href{https://help.nytimes.com/hc/en-us/articles/115015385887-Contact-Us}{Contact
  Us}
\item
  \href{https://www.nytco.com/careers/}{Work with us}
\item
  \href{https://nytmediakit.com/}{Advertise}
\item
  \href{http://www.tbrandstudio.com/}{T Brand Studio}
\item
  \href{https://www.nytimes.com/privacy/cookie-policy\#how-do-i-manage-trackers}{Your
  Ad Choices}
\item
  \href{https://www.nytimes.com/privacy}{Privacy}
\item
  \href{https://help.nytimes.com/hc/en-us/articles/115014893428-Terms-of-service}{Terms
  of Service}
\item
  \href{https://help.nytimes.com/hc/en-us/articles/115014893968-Terms-of-sale}{Terms
  of Sale}
\item
  \href{https://spiderbites.nytimes.com}{Site Map}
\item
  \href{https://help.nytimes.com/hc/en-us}{Help}
\item
  \href{https://www.nytimes.com/subscription?campaignId=37WXW}{Subscriptions}
\end{itemize}
