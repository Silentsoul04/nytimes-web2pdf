Sections

SEARCH

\protect\hyperlink{site-content}{Skip to
content}\protect\hyperlink{site-index}{Skip to site index}

\href{https://myaccount.nytimes.com/auth/login?response_type=cookie\&client_id=vi}{}

\href{https://www.nytimes.com/section/todayspaper}{Today's Paper}

\href{/section/opinion}{Opinion}\textbar{}India Can Hide Unemployment
Data, but Not the Truth

\url{https://nyti.ms/2DLf2ok}

\begin{itemize}
\item
\item
\item
\item
\item
\end{itemize}

Advertisement

\protect\hyperlink{after-top}{Continue reading the main story}

\href{/section/opinion}{Opinion}

Supported by

\protect\hyperlink{after-sponsor}{Continue reading the main story}

\hypertarget{india-can-hide-unemployment-data-but-not-the-truth}{%
\section{India Can Hide Unemployment Data, but Not the
Truth}\label{india-can-hide-unemployment-data-but-not-the-truth}}

The jobs situation may be even worse than was suspected.

By Kaushik Basu

Mr. Basu is an economist.

\begin{itemize}
\item
  Feb. 1, 2019
\item
  \begin{itemize}
  \item
  \item
  \item
  \item
  \item
  \end{itemize}
\end{itemize}

\includegraphics{https://static01.nyt.com/images/2019/02/01/opinion/01basu/merlin_137549379_419abebc-ada0-4054-8d43-bae0273fa2f0-articleLarge.jpg?quality=75\&auto=webp\&disable=upscale}

India has a job crisis, and the government would rather you didn't
notice. Last month, it hastily amended the Constitution
\href{https://www.thehindubusinessline.com/news/national/the-124th-amendment-a-look-at-the-facts/article25990756.ece}{to
set aside 10 percent of all government posts} for the ``economically
weak.'' But it defined the ``economically weak'' as anybody from a
household earning less than 800,000 rupees, roughly \$11,200, a year or
owning a very tiny bit of land. And as the sociologist Sonalde Desai has
argued, that covers
\href{https://www.thehindu.com/opinion/lead/a-solution-in-search-of-a-problem/article25962037.ece}{about
95 percent of India's population}.

A quota that includes virtually everybody means little. But the new 10
percent quota is even worse than that: It contains a caveat that
explicitly leaves out individuals belonging to India's disadvantaged
castes, who benefit from other affirmative action measures. That's a
little bit as if the United States government announced that it was
reserving 10 percent of government jobs for all but the richest 5
percent of Americans, and African Americans need not apply.

How did India get to this point, especially on the watch of Prime
Minister Narendra Modi, who came to power in 2014 partly on the back of
promises to create more jobs? Back then,
\href{https://www.bjp.org/images/pdf_2014/full_manifesto_english_07.04.2014.pdf}{the
manifesto of his Bharatiya Janata Party} had called India's labor force
``the pillar of our growth.''

According to data
\href{https://economictimes.indiatimes.com/news/economy/indicators/unemployment-rate-up-at-4-9-in-2013-14-labour-bureau/articleshow/45796927.cms}{released
by the Labour Bureau}, a wing of the Ministry of Labour and Employment,
unemployment in 2013-14 was 4.9 percent. But an undisclosed study by the
National Sample Survey Office (NSSO), a government agency that conducts
large-scale research, has
\href{https://www.business-standard.com/article/economy-policy/unemployment-rate-at-five-decade-high-of-6-1-in-2017-18-nsso-survey-119013100053_1.html}{reportedly
placed the figure for 2017-18 at 6.1 percent --- a 45-year high}.

Measuring employment is inherently difficult in India. One reason is
that the standard definition of what constitutes work --- being in
regular employment for a certain number of hours and a regular salary
--- comes from industrialized nations. Yet according to
\href{http://www.ihdindia.org/sarnet/books/IEG_2016_ES.pdf}{various}
\href{https://www.ilo.org/global/publications/books/WCMS_626831/lang--en/index.htm}{reports,}more
than 80 percent of Indians who are working or seeking work are in the
informal sector, many of them doing odd jobs for multiple employers.
Their activity is far more complicated for economists to measure
accurately.

Making matters worse --- and fueling speculation that the unemployment
situation in India is even more dire than suspected --- the government
has withheld official data about jobs. The two members of the NSSO who
were not government officials resigned this week, in protest over the
decision to not release their office's results even though those had
been cleared for publication.

The other official source economists have traditionally turned to are
the employment statistics of the Labour Bureau. The office had been
releasing this data regularly for nearly a decade --- until 2016, when
the Labour Ministry suddenly decided
to\href{https://www.dnaindia.com/india/report-survey-discontinued-centre-clueless-about-unemployment-2591121}{discontinue
the series}.

This information blackout is uncharacteristic for India, which has been
praised, including by the Nobel Prize-winning economist
\href{https://siteresources.worldbank.org/INTPA/Resources/deaton_kozel_2004.pdf}{Angus
Deaton}, for playing a pioneering role, globally, in statistical data
collection.

Now, analysts have to rely on other sources, indirect evidence and
private studies.

The findings from those are alarming.

The
\href{https://www.cmie.com/kommon/bin/sr.php?kall=warticle\&dt=2018-07-17\%2009:45:21\&msec=123}{Center
for Monitoring the Indian Economy}, a well-respected business
information company that collects primary data on various aspects of the
Indian economy, estimates that
\href{https://unemploymentinindia.cmie.com/}{the country's unemployment
rate in December 2018 reached 7.38 percent}\emph{.}

According to the
``\href{https://cse.azimpremjiuniversity.edu.in/wp-content/uploads/2018/12/State_of_Working_India_2018.pdf}{State
of Working India 2018},'' a large study conducted by the Center for
Sustainable Employment at Azim Premji University, India's youth
unemployment now stands at 16 percent. Women hold just 16 percent of
jobs in the service sector. In 2011, only 13 percent of senior officers,
legislators and managers were women; by 2015, the figure had dropped to
7 percent.

Anecdotal evidence suggests that people are hurting. Early last year,
\href{http://www.indianrailways.gov.in/railwayboard/view_section.jsp?lang=0\&id=0,4,1244}{Indian
Railways} advertised about 89,400 new jobs. Government posts
\href{http://www.bbc.com/capital/story/20180601-the-jobs-in-india-that-attract-millions-of-applicants}{always
are coveted} in India --- they mean job security and a decent salary ---
but
\href{https://timesofindia.indiatimes.com/business/india-business/railways-receives-2-3-crore-applications-for-90000-advertised-jobs/articleshow/63898755.cms}{more
than 23 million adults applied} for these positions, defying all
expectations. Earlier this year, the secretariat of the government of
Maharashtra, a state in west-central India, advertised 13 waiting jobs
in its canteen. There were
\href{https://www.ndtv.com/india-news/graduates-among-7-000-who-apply-for-13-waiter-jobs-at-maharashtra-secretariat-1981639}{7,000
applicants, many of them university graduates}.

\href{https://economictimes.indiatimes.com/news/economy/indicators/world-bank-forecasts-7-3-per-cent-growth-for-india-making-it-fastest-growing-economy/articleshow/64478238.cms}{India's
growth rate remains robust}, but the benefits of the country's growth
have been
\href{https://www.oxfamindia.org/sites/default/files/himanshu_inequality_Inequality_report_2018.pdf}{concentrated
almost entirely at the top}, with grim implications for the working
classes and the lower-middle classes, women and the young.

These effects aren't just the accidental results of the government's
decision, say,
\href{https://www.nytimes.com/2016/11/27/opinion/in-india-black-money-makes-for-bad-policy.html?_r=0\&module=inline}{to
ban certain currency bills in late 2016} (which proved to be terribly
\href{https://www.nytimes.com/2017/06/29/opinion/india-and-the-visible-hand-of-the-market.html}{misguided})
or to transform the indirect tax system into the new Goods and Services
Tax (a move in the right direction but that was poorly executed and
\href{https://www.hindustantimes.com/business-news/a-year-after-gst-small-businesses-report-huge-drop-in-sales-struggle-with-high-costs-of-compliance/story-2NezKZ0MJ0EOxsocnavDtJ.html}{hurt
small businesses}).
\href{https://www.oxfamindia.org/sites/default/files/himanshu_inequality_Inequality_report_2018.pdf}{Inequality
has grown}, in numerous ways, a recent report indicates.

The Modi government's economic policy has been disproportionately
focused on a few big corporations, neglecting small firms and traders,
the agricultural sector and most workers. The results are now showing.

Kaushik Basu, the C. Marks Professor of International Studies and
professor of economics at Cornell University, was chief economic adviser
to the Indian government in 2009-12 and chief economist of the World
Bank in 2012-16.

\emph{Follow The New York Times Opinion section on}
\href{https://www.facebook.com/nytopinion}{\emph{Facebook}}\emph{,}
\href{http://twitter.com/NYTOpinion}{\emph{Twitter (@NYTopinion)}}
\emph{and}
\href{https://www.instagram.com/nytopinion/}{\emph{Instagram}}\emph{.}

Advertisement

\protect\hyperlink{after-bottom}{Continue reading the main story}

\hypertarget{site-index}{%
\subsection{Site Index}\label{site-index}}

\hypertarget{site-information-navigation}{%
\subsection{Site Information
Navigation}\label{site-information-navigation}}

\begin{itemize}
\tightlist
\item
  \href{https://help.nytimes.com/hc/en-us/articles/115014792127-Copyright-notice}{©~2020~The
  New York Times Company}
\end{itemize}

\begin{itemize}
\tightlist
\item
  \href{https://www.nytco.com/}{NYTCo}
\item
  \href{https://help.nytimes.com/hc/en-us/articles/115015385887-Contact-Us}{Contact
  Us}
\item
  \href{https://www.nytco.com/careers/}{Work with us}
\item
  \href{https://nytmediakit.com/}{Advertise}
\item
  \href{http://www.tbrandstudio.com/}{T Brand Studio}
\item
  \href{https://www.nytimes.com/privacy/cookie-policy\#how-do-i-manage-trackers}{Your
  Ad Choices}
\item
  \href{https://www.nytimes.com/privacy}{Privacy}
\item
  \href{https://help.nytimes.com/hc/en-us/articles/115014893428-Terms-of-service}{Terms
  of Service}
\item
  \href{https://help.nytimes.com/hc/en-us/articles/115014893968-Terms-of-sale}{Terms
  of Sale}
\item
  \href{https://spiderbites.nytimes.com}{Site Map}
\item
  \href{https://help.nytimes.com/hc/en-us}{Help}
\item
  \href{https://www.nytimes.com/subscription?campaignId=37WXW}{Subscriptions}
\end{itemize}
