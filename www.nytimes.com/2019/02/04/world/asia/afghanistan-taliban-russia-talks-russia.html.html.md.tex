Sections

SEARCH

\protect\hyperlink{site-content}{Skip to
content}\protect\hyperlink{site-index}{Skip to site index}

\href{https://www.nytimes.com/section/world/asia}{Asia Pacific}

\href{https://myaccount.nytimes.com/auth/login?response_type=cookie\&client_id=vi}{}

\href{https://www.nytimes.com/section/todayspaper}{Today's Paper}

\href{/section/world/asia}{Asia Pacific}\textbar{}In Moscow, Afghan
Peace Talks Without the Afghan Government

\url{https://nyti.ms/2DTesoN}

\begin{itemize}
\item
\item
\item
\item
\item
\end{itemize}

Advertisement

\protect\hyperlink{after-top}{Continue reading the main story}

Supported by

\protect\hyperlink{after-sponsor}{Continue reading the main story}

\hypertarget{in-moscow-afghan-peace-talks-without-the-afghan-government}{%
\section{In Moscow, Afghan Peace Talks Without the Afghan
Government}\label{in-moscow-afghan-peace-talks-without-the-afghan-government}}

\includegraphics{https://static01.nyt.com/images/2019/02/06/world/06taliban-1-print/merlin_150218136_b3573890-bd0a-4a60-b3d8-f212f22fc94c-articleLarge.jpg?quality=75\&auto=webp\&disable=upscale}

By \href{https://www.nytimes.com/by/andrew-higgins}{Andrew Higgins} and
\href{https://www.nytimes.com/by/mujib-mashal}{Mujib Mashal}

\begin{itemize}
\item
  Feb. 4, 2019
\item
  \begin{itemize}
  \item
  \item
  \item
  \item
  \item
  \end{itemize}
\end{itemize}

MOSCOW --- Thirty years after the last Soviet troops retreated from
Afghanistan, Russia on Tuesday reasserted itself as a player in the
region, hosting talks between the Taliban and senior Afghan politicians
aimed at speeding the exit of another superpower --- this time the
United States.

The talks, held in Moscow's President Hotel, which is owned by the
Kremlin, offered a clearer view of how the Taliban see an end to the
18-year war. In a room dripping with chandeliers, more than 50 delegates
--- many in flowing robes, some in Western suits and ties, and nearly
all old and sometimes violent rivals --- faced each other across a
large, circular conference table.

While the Afghan politicians, part of a delegation led by former
President Hamid Karzai, spoke of protecting the hard gains of the past
two decades, the Taliban denounced a new Afghan Constitution that lays
out a system of governance built at enormous cost.

The Taliban representatives also offered a rare look at how they now see
the role of women. While they barred women from public life during their
time in power, they said they now believed in women's rights, including
to education and work --- a claim met with skepticism by some women in
Afghanistan.

The Moscow gathering, which included a Taliban delegation led by their
chief negotiator, Sher Mohammad Abbas Stanekzai, represented the most
significant contact between senior Afghan politicians and the Taliban
since the United States toppled the hard-line Islamist group from power
at the end of 2001.

Absent from the talks, however, was the American-backed Afghan
government of President Ashraf Ghani, which has strongly criticized the
meeting as an affront designed to undermine his office's authority and
the Afghan state.

Mr. Ghani is in an uncomfortable position, at odds not only with his
American backers, whom he sees as moving too quickly to reach a deal,
but also with others in the country's political elite who are rallying
around the American-led effort.

``What are they agreeing to, with whom? Where is their implementing
power?'' Mr. Ghani told the Afghan channel ToloNews on Tuesday,
dismissing the talks. ``They could hold a hundred such meetings, but
until the Afghan government, the Afghan Parliament, the legal
institutions of Afghanistan approve it, it is just agreements on
paper.''

The delegation headed by Mr. Karzai consisted entirely of former
officials, representatives of political parties --- many of them
involved in the country's bloody civil war --- and current members of
Parliament. There were only two women in the group.

Afghans on social media were critical of the delegation, questioning
whether they represented Afghanistan.

``Those who are in the meeting in Moscow, they have been pushed aside,''
said Khaled Abedy, 31, who works at a private company in Kabul, the
Afghan capital. ``They just want to build their own business. The
country isn't important to them. I think this sort of meeting can't help
the peace process at all.''

But Atta Muhammad Noor, one of the Afghan politicians in the delegation
in Moscow, said the participants there considered themselves to be more
representative of Afghanistan than Mr. Ghani's government.

``We have been fighting for 40 years, and we are the people with
influence, not Ghani,'' said Mr. Noor, who was the longtime governor of
Balkh Province before Mr. Ghani dismissed him last year.

Speaking on the sidelines of the event, he said that all foreign forces,
including around 14,000 American troops, must leave Afghanistan. But he
cautioned that they should be withdrawn gradually, to avoid a repeat of
the chaos that engulfed Afghanistan after the abrupt Soviet pullout in
1989.

\includegraphics{https://static01.nyt.com/images/2019/02/06/world/06taliban-2-print/merlin_150214083_cc14b824-ba95-4089-b717-67232f92a0da-articleLarge.jpg?quality=75\&auto=webp\&disable=upscale}

The talks, scheduled to last two days, opened just a week after American
diplomats and Taliban representatives
\href{https://www.nytimes.com/2019/01/28/world/asia/taliban-peace-deal-afghanistan.html}{ended
six days of negotiations in Doha}, the capital of Qatar. Each side said
those negotiations had made progress toward ending a conflict that began
when the United States
\href{https://www.nytimes.com/2001/11/08/world/nation-challenged-month-1-month-difficult-battlefield-assessing-us-war-strategy.html}{invaded
Afghanistan} in 2001, not long after the terror attacks of Sept. 11.

Both sides said they had agreed, in principle, to a framework on two
issues: a Taliban guarantee that Afghan soil would never again be used
by terrorist groups like Al Qaeda, and a pledge from the United States
to withdraw its troops. But many Afghans are concerned that the
Americans might be too eager to strike a deal.

The organizer of the Moscow talks is ostensibly the Afghan diaspora in
Russia, not the Russian government. But Afghan officials and Taliban
members have said that the Kremlin is
\href{https://www.nytimes.com/2019/02/02/world/asia/taliban-moscow-ghani-talks.html}{playing
a major role} orchestrating the meeting behind the scenes.

Russia, chastened by the damage done to the Soviet Union during its
occupation of Afghanistan, has shown no interest in getting involved
militarily again, at least not directly. But it has positioned itself as
a force to be reckoned with, relishing Washington's agonies at the hands
of Taliban insurgents.

Russia designated the Taliban a terrorist organization in 2003, and at
first strongly supported American efforts to purge Afghanistan of
extremist Islamist groups, which President Vladimir V. Putin described
as a threat to Russia's security.

But amid a rising Cold War-style rivalry between Moscow and Washington,
Russia has hedged its bets by opening channels with the Taliban. Moscow
allowed a 10-member delegation from the Taliban, still officially barred
as terrorists, to enter Russia for the Moscow talks.

On Tuesday in Afghanistan, the violence continued unabated. The Taliban
attacked police and army outposts around the northern city of Kunduz
before dawn, killing at least 23 members of the Afghan security forces.
In Takhar Province, gunmen attacked a women's radio station, killing two
staff members.

Mr. Karzai made an appeal to end the bloodshed, saying Afghan soldiers
and Taliban fighters were buried next to each other.

``All around them in these graveyards are the regular Afghans --- their
graves are plenty,'' Mr. Karzai said. ``The dream of every mother, the
hope of every father is buried there.''

Mr. Karzai was first installed as Afghanistan's leader by the United
States in late 2001, but the relationship soured. He has visited Russia
often since leaving office in 2014, and in meetings with Mr. Putin and
other officials he has aligned himself with Moscow's view that the
United States must leave Afghanistan, as the Soviet Union did.

Mr. Stanekzai, the Taliban's chief negotiator, said in a speech lasting
more than half an hour that the group did not seek to monopolize power
inside Afghanistan. He said that they were pursuing an Islamist
government, ``in consultation with all Afghans,'' and that the group did
not recognize the country's current Constitution, calling it copied from
the West.

Perhaps the most revealing part of his speech came when he described the
Taliban's view of a future role for Afghan women. When in power, the
group sent its religious police to patrol the streets, giving out lashes
to women for, among other things, showing their ankles.

``We are committed to all rights given to women by Islam,'' Mr.
Stanekzai said. ``Islam has given women all fundamental rights --- such
as trade, ownership, inheritance, education, work and the choice of
partner, security and education, and a good life.''

Considering the group's history, some Afghan women immediately
questioned the statement's sincerity.

But Fawzia Koofi, a female member of the Afghan Parliament and one of
the two women in attendance, said she was happy to have heard the
Taliban promise that women would not be stripped of their rights and
would be allowed to serve as prime minister --- though not as president.

However, she cautioned, ``We have gained so much in the last 18 years,
whatever the problems, that we do not want to go back the Taliban
period.''

Advertisement

\protect\hyperlink{after-bottom}{Continue reading the main story}

\hypertarget{site-index}{%
\subsection{Site Index}\label{site-index}}

\hypertarget{site-information-navigation}{%
\subsection{Site Information
Navigation}\label{site-information-navigation}}

\begin{itemize}
\tightlist
\item
  \href{https://help.nytimes.com/hc/en-us/articles/115014792127-Copyright-notice}{©~2020~The
  New York Times Company}
\end{itemize}

\begin{itemize}
\tightlist
\item
  \href{https://www.nytco.com/}{NYTCo}
\item
  \href{https://help.nytimes.com/hc/en-us/articles/115015385887-Contact-Us}{Contact
  Us}
\item
  \href{https://www.nytco.com/careers/}{Work with us}
\item
  \href{https://nytmediakit.com/}{Advertise}
\item
  \href{http://www.tbrandstudio.com/}{T Brand Studio}
\item
  \href{https://www.nytimes.com/privacy/cookie-policy\#how-do-i-manage-trackers}{Your
  Ad Choices}
\item
  \href{https://www.nytimes.com/privacy}{Privacy}
\item
  \href{https://help.nytimes.com/hc/en-us/articles/115014893428-Terms-of-service}{Terms
  of Service}
\item
  \href{https://help.nytimes.com/hc/en-us/articles/115014893968-Terms-of-sale}{Terms
  of Sale}
\item
  \href{https://spiderbites.nytimes.com}{Site Map}
\item
  \href{https://help.nytimes.com/hc/en-us}{Help}
\item
  \href{https://www.nytimes.com/subscription?campaignId=37WXW}{Subscriptions}
\end{itemize}
