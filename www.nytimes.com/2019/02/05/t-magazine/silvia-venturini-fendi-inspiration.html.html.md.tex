Sections

SEARCH

\protect\hyperlink{site-content}{Skip to
content}\protect\hyperlink{site-index}{Skip to site index}

\href{https://myaccount.nytimes.com/auth/login?response_type=cookie\&client_id=vi}{}

\href{https://www.nytimes.com/section/todayspaper}{Today's Paper}

Silvia Venturini Fendi's Playful Sort of Luxury

\href{https://nyti.ms/2DUgNji}{https://nyti.ms/2DUgNji}

\begin{itemize}
\item
\item
\item
\item
\item
\item
\end{itemize}

Advertisement

\protect\hyperlink{after-top}{Continue reading the main story}

Supported by

\protect\hyperlink{after-sponsor}{Continue reading the main story}

Profile in Style

\hypertarget{silvia-venturini-fendis-playful-sort-of-luxury}{%
\section{Silvia Venturini Fendi's Playful Sort of
Luxury}\label{silvia-venturini-fendis-playful-sort-of-luxury}}

By \href{https://www.nytimes.com/by/lindsay-talbot}{Lindsay Talbot}

\begin{itemize}
\item
  Feb. 5, 2019
\item
  \begin{itemize}
  \item
  \item
  \item
  \item
  \item
  \item
  \end{itemize}
\end{itemize}

``I always say that I was born into a very special yet very peculiar
family,'' says Silvia Venturini Fendi, whose official involvement with
the storied Italian fashion house her grandparents founded in 1925 began
at age 6, when she appeared in a Fendi ad wearing a beaver bomber jacket
and matching hat. ``I didn't quite understand what my family did for a
living, but I knew I just had to be a part of it.'' It was Venturini
Fendi's mother and four aunts who hired
\href{https://www.nytimes.com/2015/10/12/t-magazine/karl-lagerfeld-interview.html}{Karl
Lagerfeld} as creative director in 1965 and turned the company, which
got its start selling umbrellas and leather trunks out of a small
\emph{pellicceria} (or fur shop) on Rome's Via del Plebiscito, into an
international force. Lagerfeld scribbled a sketch of the iconic (and
once again ubiquitous) logo of double F's upon his arrival, and added
women's ready-to-wear just over a decade later. To Venturini Fendi,
Lagerfeld has been a lifelong mentor.

When she was in her early 20s --- after stints answering phones,
gift-wrapping and picking up stray sewing pins off the atelier floors
with a magnet --- Venturini Fendi decided to try her own hand at
designing. Her biggest coup came in 1997, when she created the Baguette,
an oblong, short-strapped pochette meant to be tucked just under the
shoulder. The ``it'' bag of the '90s, with more than 1,000 iterations
made with everything from embroidered silk to mosaiclike beadwork, it
reappeared on the runway last fall. ``I had been seeing all the cool
kids posting their mothers' Baguettes on Instagram,'' explains Venturini
Fendi, who since 2000 has overseen the brand's men's wear, an eccentric
mix of graphic street styles and more traditional Italian tailoring that
exuberantly plays with expectations. That distressed denim shirt? It's
actually bonded leather. Those crocodile panels on a biker jacket? Turns
out they're neoprene. This element of surprise extends to the Rome-based
designer's larger ideas about masculinity, too: ``I like to accentuate a
man's more vulnerable sides,'' she says. ``And to create elusive things
that always leave you wanting to know more.''

\includegraphics{https://static01.nyt.com/images/2019/02/05/t-magazine/05tmag-fendi-slide-8FY7/05tmag-fendi-slide-8FY7-articleLarge.jpg?quality=75\&auto=webp\&disable=upscale}

``That's me. The portrait was taken in October near our Fendi
headquarters, which is in the Palazzo della Civiltà Italiana in Rome, a
de Chirico-like marble building commissioned by Mussolini in 1943. I
hate seeing pictures of myself, but I like what I'm wearing: a caramel
coat with Fendi's Zucca logo on the collar and a super-geometric silk
shirt from my recent men's collection.''

\begin{center}\rule{0.5\linewidth}{\linethickness}\end{center}

Image

Credit...From left: Nico Vascellari, ``Dream Merda,'' courtesy of Fendi;
Courtesy of Formafantasma, photo by Luisa Zanzani

\emph{Left:} ``A neon installation by the artist
\href{http://www.nicovascellari.com/}{Nico Vascellari} in the entrance
of my home, a very old house in the center of Rome that once belonged to
my grandmother Adele. Nico is my daughter
\href{https://www.nytimes.com/2016/02/08/t-magazine/fashion/delfina-delettrez-fendis-whimsical-jewelry-and-many-inspirations.html}{Delfina
Delettrez}'s partner and shares my strange sense of humor, which you see
in his cheeky anagrams like this one, called ``Dream Merda.'' When I
come home at night, the hall is always illuminated bright red, which
makes me feel as though I've entered another dimension.''

\emph{Right:} ``This cork vase is from a 2012 collaboration we did with
the Amsterdam-based studio
\href{https://www.nytimes.com/2016/10/20/t-magazine/design/studio-formafantasma-andrea-trimarchi-simone-farresin.html}{Formafantasma}.
It incorporates pig and cow bladders --- the designers make use of waste
from the food industry. It's poetic but also primitive and strong, which
is very Fendi.''

\begin{center}\rule{0.5\linewidth}{\linethickness}\end{center}

Image

Credit...From left: Insectmania brooch, courtesy of Delfina Delettrez;
Photo by Karl Lagerfeld

\emph{Left:} ``I wear this oversize prasiolite and pearl brooch by my
daughter Delfina, who's a jewelry designer, all the time. Though I'm
partial, I think she's so imaginative. The brilliance of the piece is
that it's totally flexible, so it moves whenever you move, almost like
an actual beetle. Even its little legs come alive. I like to dress like
a nun, but when I pin it on a simple black shirt, it becomes the
conversation of the evening.''

\emph{Right:} ``Here is the entire Fendi family in 1989 at a traditional
Roman trattoria. Obviously, there are a lot of us --- you can see my
mother, my aunts, my sisters. I'm in the back, on the right, with
Delfina on my lap. We have big lunches and dinners like this all the
time, so we can talk about work ideas in an informal way. There's
nothing more Italian than sitting around a table eating pasta.''

\begin{center}\rule{0.5\linewidth}{\linethickness}\end{center}

Image

Credit...Courtesy of Fendi (2)

\emph{Left:} ``My
\href{https://www.nytimes.com/slideshow/2018/09/20/fashion/runway-womens/fendi-spring-2019.html}{spring
2019 show} drew on anagrams --- `Fendi' became `Fiend' and `Roma' became
`Amor' --- alchemical symbols and the dual realms of hell and paradise.
It was also about elevating street wear. This white guayabera shirt
looks like a simple cotton, but it's actually a fantastic suede, while
the intarsia of colorful stripes is made of leather. I like things that
appear a certain way but are entirely different when you touch them.
Like little magic tricks.''

\emph{Right:} ``Karl Lagerfeld took this Polaroid of me in the late '70s
at his house in Monte Carlo. Everything was Memphis-style (he was an
early adopter), including the boxing ring where we are having tea before
going to dinner at the Hôtel de Paris. Though I was still a wild party
girl at the time, I'd just started working for the brand and was in
Monaco for the weekend to pick up Karl's sketches for a new collection.
Things took so much longer before email, but they were much more
personal.''

\begin{center}\rule{0.5\linewidth}{\linethickness}\end{center}

Image

Credit...From left: © Martin Parr/Magnum Photos; Joel Sternfeld,
``Summer Interns Having Lunch, Wall Street, New York, New York, August
1987''

\emph{Left:} ``Photography can be a fascinating witness to time, so I'm
constantly pinning snapshots and old pictures on my mood boards. Martin
Parr's 1997 image of a simple cup of tea helped inspire my spring 2018
men's show, for which the artist Sue Tilley painted ordinary objects
onto totes, knits and jackets.''

\emph{Right:} ``I also looked to
\href{https://www.nytimes.com/2017/01/24/t-magazine/art/joel-sternfeld-photographs.html}{Joel
Sternfeld}'s image of Wall Street summer interns eating lunch in the
late '80s. The collection was about pairing street wear and tailoring,
as in wearing a business shirt and tie on top with Bermuda shorts below.
I call that `the Skype look.'''

\begin{center}\rule{0.5\linewidth}{\linethickness}\end{center}

Image

Credit...Slim Aarons/Hulton Archive/Getty Images

``Slim Aarons's portraits of jet-set society are a nostalgic reminder of
a period that will never come back. I love this 1975 photo of La Concha
Beach Club in Acapulco in particular, because the pink stripes recall
Fendi's own."

\begin{center}\rule{0.5\linewidth}{\linethickness}\end{center}

Image

Credit...Clockwise from top left: Richard Isaac/Shutterstock; Fendi \&
Cristina Celestino, ``The Happy Room,'' 2016, courtesy of Fendi;
Courtesy of Fendi

\emph{Top left:} ``Whenever I'm in Cuba, I visit Coppelia, which is
where everyone in Havana goes for ice cream --- the '60s-era structure
was built around nature, with lush jungle greens juxtaposed against the
clean modern lines. Being a Roman, I need to have my gelato.''

\emph{Center:} ``This 2000 edition of the Baguette was made with the
\href{https://www.fondazionelisio.org/en/}{Lisio Foundation}, which
manufactures fabrics on looms from the 19th century. The chestnut and
grape brocade was done by hand --- the artisans could only do about 15
centimeters' worth a day. The buckle is real gold, but this bag isn't
bling-bling --- it's more of a museum piece.''

\emph{Right:} ``So many artists are part of the Fendi gang. We've shown
at
\href{https://www.nytimes.com/2018/12/04/arts/design-miami-2018-art-basel-.html}{Design
Miami} for 10 years now and, in 2016, teamed up with the architect and
furniture designer \href{http://cristinacelestino.com/}{Cristina
Celestino}, who made cocktail tables topped with graphic Roman
travertine, red Lepanto marble and onyx. The best thing about them is
their satin brass bases, shaped to resemble the backs of giant studded
earrings.''

Advertisement

\protect\hyperlink{after-bottom}{Continue reading the main story}

\hypertarget{site-index}{%
\subsection{Site Index}\label{site-index}}

\hypertarget{site-information-navigation}{%
\subsection{Site Information
Navigation}\label{site-information-navigation}}

\begin{itemize}
\tightlist
\item
  \href{https://help.nytimes.com/hc/en-us/articles/115014792127-Copyright-notice}{©~2020~The
  New York Times Company}
\end{itemize}

\begin{itemize}
\tightlist
\item
  \href{https://www.nytco.com/}{NYTCo}
\item
  \href{https://help.nytimes.com/hc/en-us/articles/115015385887-Contact-Us}{Contact
  Us}
\item
  \href{https://www.nytco.com/careers/}{Work with us}
\item
  \href{https://nytmediakit.com/}{Advertise}
\item
  \href{http://www.tbrandstudio.com/}{T Brand Studio}
\item
  \href{https://www.nytimes.com/privacy/cookie-policy\#how-do-i-manage-trackers}{Your
  Ad Choices}
\item
  \href{https://www.nytimes.com/privacy}{Privacy}
\item
  \href{https://help.nytimes.com/hc/en-us/articles/115014893428-Terms-of-service}{Terms
  of Service}
\item
  \href{https://help.nytimes.com/hc/en-us/articles/115014893968-Terms-of-sale}{Terms
  of Sale}
\item
  \href{https://spiderbites.nytimes.com}{Site Map}
\item
  \href{https://help.nytimes.com/hc/en-us}{Help}
\item
  \href{https://www.nytimes.com/subscription?campaignId=37WXW}{Subscriptions}
\end{itemize}
