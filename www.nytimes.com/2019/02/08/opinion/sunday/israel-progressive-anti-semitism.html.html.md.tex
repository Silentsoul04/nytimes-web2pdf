Sections

SEARCH

\protect\hyperlink{site-content}{Skip to
content}\protect\hyperlink{site-index}{Skip to site index}

\href{https://www.nytimes.com/section/opinion/sunday}{Sunday Review}

\href{https://myaccount.nytimes.com/auth/login?response_type=cookie\&client_id=vi}{}

\href{https://www.nytimes.com/section/todayspaper}{Today's Paper}

\href{/section/opinion/sunday}{Sunday Review}\textbar{}The Progressive
Assault on Israel

\url{https://nyti.ms/2E1f0Jh}

\begin{itemize}
\item
\item
\item
\item
\item
\item
\end{itemize}

Advertisement

\protect\hyperlink{after-top}{Continue reading the main story}

\href{/section/opinion}{Opinion}

Supported by

\protect\hyperlink{after-sponsor}{Continue reading the main story}

\hypertarget{the-progressive-assault-on-israel}{%
\section{The Progressive Assault on
Israel}\label{the-progressive-assault-on-israel}}

A movement that can detect a racist dog-whistle from miles away is
strangely deaf when it comes to some of the barking on its own side of
the fence.

\href{https://www.nytimes.com/by/bret-stephens}{\includegraphics{https://static01.nyt.com/images/2017/08/27/insider/bretstephens/bretstephens-thumbLarge-v6.png}}

By \href{https://www.nytimes.com/by/bret-stephens}{Bret Stephens}

Opinion Columnist

\begin{itemize}
\item
  Feb. 8, 2019
\item
  \begin{itemize}
  \item
  \item
  \item
  \item
  \item
  \item
  \end{itemize}
\end{itemize}

\includegraphics{https://static01.nyt.com/images/2019/02/10/sunday-review/10Stephens1/merlin_150331608_674f736c-0ca0-4557-a89b-439fd0188e0d-articleLarge.jpg?quality=75\&auto=webp\&disable=upscale}

It happened again last month in Detroit. Pro-Palestinian demonstrators
seized the stage of the National L.G.B.T.Q. Task Force's marquee
conference, ``Creating Change'' and demanded a boycott of Israel. ``From
the river to the sea, Palestine will be free,'' they chanted --- the
tediously malign, thinly veiled call to end Israel as a Jewish state.

They were met with sustained applause by the audience at what is the
largest annual conference of L.G.B.T.Q. activists in the United States.
Conference organizers did nothing to stop the disruption or disavow the
demonstrators.

For Tyler Gregory, neither the behavior of the protesters nor the
passivity of the organizers came as a surprise. Gregory is executive
director of \href{https://awiderbridge.org/}{A Wider Bridge}, a North
American L.G.B.T.Q. organization that works to support Israel and its
gay community. In 2016, his group hosted a reception at the Task Force's
conference in Chicago. The event was mobbed by some
\href{https://www.youtube.com/watch?v=Rz4KkvvjBB8\&feature=youtu.be}{200
aggressive demonstrators}, and Gregory and his audience had to barricade
themselves in their room while those outside were harassed.

``Whether you believe in the concept of intersectionality is beside the
point,'' Gregory told me recently, referring to the idea that the
oppression of one group is the oppression of all others. ``If this is
your value system, you are not following it. As Jews we were denied our
safe space. We were denied our place in a movement that fights
bigotry.''

Scenes of the kind that played out at the L.G.B.T.Q. conferences --- not
to mention college campuses across the United States --- are familiar to
anyone involved in the politics of the American Jewish community. They
have burst into wider consciousness in recent months, thanks to
revelations that Jewish organizers of the 2017 Women's March were
\href{https://www.tabletmag.com/jewish-news-and-politics/276694/is-the-womens-march-melting-down}{deliberately
sidelined, excluded and attacked} by some of its founders, at least one
of whom, activist Tamika Mallory, is an
\href{https://www.nationalreview.com/news/tamika-mallory-defends-decision-to-praise-farrakhan-as-the-greatest-of-all-time/}{unapologetic
admirer} of Louis Farrakhan, the Nation of Islam's unapologetically
anti-Semitic leader.

\includegraphics{https://static01.nyt.com/images/2019/02/10/opinion/sunday/10Stephens3/merlin_149441079_71c6cab6-9a50-46f8-99ab-1f0ac5ac99f4-articleLarge.jpg?quality=75\&auto=webp\&disable=upscale}

They have also burst into Congress, largely as a result of the election
of Democratic Representatives Rashida Tlaib of Michigan and Ilhan Omar
of Minnesota. Both women support boycotts of Israel. Both have also
\href{https://twitter.com/rashidatlaib/status/1082095303325609984?lang=en}{written
tweets} with distinctly
\href{https://twitter.com/ilhanmn/status/269488770066313216?lang=en}{anti-Semitic
undertones}. Far from being reproached or condemned by their party, as
Iowa's Steve King was by Republicans, they have become Democratic rock
stars. (Omar, to her credit, recanted her tweet; Tlaib did not.)

Progressives --- including presidential hopefuls Cory Booker, Kamala
Harris and Elizabeth Warren --- also united behind Vermont's Bernie
Sanders in a failed bid to block a Senate bill, passed on Tuesday, that
includes an anti-B.D.S. measure prohibiting federal contracts with
businesses that boycott Israel, ostensibly on free-speech grounds. One
wonders how these same Democrats feel about, say, championing First
Amendment protections for bakers who refuse to make cakes for gay
couples.

All of this is profoundly unsettling to a Jewish community that has
generally seen the Democratic Party as its political home. That's not
because American Jews are unfamiliar with the radical left's militant
hostility toward the Jewish state. That's been true for decades. Nor is
it because American Jews are suddenly tilting right: Some 76 percent
voted for Democrats in the midterms.

What's unsettling is that the far-left's hostility is now being
mainstreamed by the not-so-far left. Anti-Zionism --- that is, rejection
not just of this or that Israeli policy, but also of the idea of a
Jewish state itself --- is becoming a respectable position among people
who would never support the elimination of any other country in any
other circumstance. And it is churning up a new wave of nakedly
anti-Jewish bigotry in its wake, as when three women holding rainbow
flags embossed with a Star of David at the 2017 Chicago Dyke March were
ejected on grounds that the star was ``a trigger.''

How did this happen?

The progressive answer is straightforward: Israel and its supporters,
they say, did this to themselves. More than a half-century of occupation
of Palestinian territories is a massive injustice that fair-minded
people can no longer ignore, especially given America's financial
support for Israel. Continued settlement expansion in the West Bank
proves Israel has no interest in making peace on equitable terms. And
endless occupation makes Israel's vaunted democracy less about Jewish
self-determination than it is about ethnic subjugation.

There's more to the indictment, but that's the nub of it. It would be
damning if it were true, or even half-true. It's not.

A few facts ought at least to stir the thinking of those who subscribe
to the progressive narrative. Israel's enemies were committed to its
destruction long before it occupied a single inch of Gaza or the West
Bank. **** In proportion to its size, Israel has voluntarily
relinquished more territory taken in war than any state in the world.
Israeli prime ministers offered a Palestinian state in
\href{https://www.theguardian.com/world/2002/may/23/israel3}{2000} and
\href{https://www.haaretz.com/1.5014018}{2008}; they were refused both
times. The government of Ariel Sharon
\href{https://www.nytimes.com/2005/08/18/world/middleeast/tearfully-but-forcefully-israel-removes-gaza-settlers.html}{removed
every Israeli settlement} and soldier from the Gaza Strip in 2005. The
result of Israel's withdrawal allowed Hamas to seize power two years
later and spark three wars, causing ordinary Israelis to think twice
about the wisdom of duplicating the experience in the West Bank. Nearly
1,300 Israeli civilians have been killed in Palestinian terrorist
attacks in this century: That's the proportional equivalent of about 16
Sept. 11's in the United States.

Also: If the Jewish state is really so villainous, why doesn't it behave
more like Syria's Bashar al-Assad or Russia's Vladimir Putin --- both of
whom, curiously, continue to have prominent
\href{https://www.haaretz.com/opinion/as-trump-shores-up-assad-regime-u-s-hard-left-cheers-him-on-1.5431962}{sympathizers
and apologists} on the anti-Israel left?

None of this is to embrace the ``Likud narrative'' of the conflict, or
support the policies of Benjamin Netanyahu, or reject the idea of
Palestinian statehood, or suggest that Israel is above criticism and
reproach. For the record, I support a two-state solution, just as I
supported Israel's withdrawal from the Gaza Strip when I was the editor
of The Jerusalem Post.

What it \emph{is} to say is that the Israel-Palestinian conflict is far
more complicated than the black-and-white picture drawn by Israel's
progressive critics. But the deeper flaw in progressive thinking on
Israel --- the flaw that has resulted in this efflorescence of bigotry
--- isn't that it rests on a faulty factual foundation. It's that its
core intellectual assumptions are wrong and rotten.

The first assumption is that Israel's choices toward the Palestinians
aren't agonizingly hard (as they are for some of the reasons mentioned
above), but actually are quite easy --- just a matter of stopping
settlement construction, reaching a reasonable settlement with the
Palestinians, making peace, and living relatively happily ever after.
But this is a caricature, and it's one that quickly descends to calumny:
That is, the idea that Israel's failure to make the ``right'' choice is
proof of its boundless greed for Palestinian land and wicked
indifference to their plight.

Next is the belief that anti-Zionism is a legitimate political position,
and not another form of prejudice.

It is one thing to argue, in the moot court of historical what-ifs, that
Israel should not have come into being, at least not where it is now. It
is also fair to say that there is much to dislike about Israel's current
leadership, just as there's much not to like about America's. But nobody
claims the election of Donald Trump makes America an illegitimate state.

Israel is now the home of nearly nine million citizens, with an identity
that is as distinctively and proudly Israeli as the Dutch are Dutch or
the Danes Danish. Anti-Zionism proposes nothing less than the
elimination of that identity and the political dispossession of those
who cherish it, with no real thought of what would likely happen to the
dispossessed. Do progressives expect the rights of Jews to be protected
should Hamas someday assume the leadership of a reconstituted
``Palestine''?

Then there's the astounding view that anti-Zionism bears only a
tangential relationship to anti-Semitism. Hatred of Jews is a
shape-shifting phenomenon that historically has melded with the
prejudices of the time in order to gain greater political currency. Jews
have been hated for reasons of religion, race, lack of national
attachments, and now an excess of national attachment. The arguments for
hating Jews vary; the target of the hatred tragically remains the same.

Of course it's theoretically possible to distinguish anti-Zionism from
anti-Semitism, just as it's theoretically possible to distinguish
segregationism from racism. But the striking feature of anti-Zionist
rhetoric is how broadly it overlaps with traditionally anti-Semitic
tropes.

To say, as progressives sometimes do, that Jews are
\href{https://people.socsci.tau.ac.il/mu/noah/files/2018/07/Ethnic-origin-and-identity-in-Israel-JEMS-2018.pdf}{``colonizers''}
in Israel is anti-Semitic because it advances the lie that there is no
ancestral or historic Jewish tie to the land. To claim that Israel is
\href{https://www.timesofisrael.com/uk-labour-mp-under-fire-for-accusing-israel-of-genocide-in-gaza-in-2012/}{committing
genocide} in Gaza, when manifestly it is not, is anti-Semitic because
it's an attempt to Nazify the Jewish state. To insist that the
\emph{only} state in the world that has
\href{https://www.counterpunch.org/2015/05/19/why-israel-should-not-exist/}{forfeited
the moral right} to exist just happens to be the Jewish state is
anti-Semitic, too: Are Israel's purported crimes really worse than those
of, say, Zimbabwe or China, whose rights to exist are never called into
question?

But the most toxic assumption is that Jews, whether in Israel or the
U.S., can never really be thought of as victims or even as a minority
because they are white, wealthy, powerful and ``privileged.'' This
relies on a simplistic concept of power that collapses on a moment's
inspection.

Jews in Germany were economically and even politically powerful in the
1920s. And then they were in Buchenwald. Israel appears powerful
vis-à-vis the Palestinians, but considerably less so in the context of a
broader Middle East saturated with genocidal anti-Semitism. American
Jews are comparatively wealthy. But wealth without political power, as
Hannah Arendt understood, is a recipe for hatred. The Jews of the
Squirrel Hill neighborhood of Pittsburgh are almost surely
``privileged'' according to various socio-economic measures. But
privilege didn't save the congregants of the Tree of Life synagogue last
year.

Nor can the racial politics of the United States or any other country be
projected onto the Israeli-Palestinian conflict, as some have
desperately sought to do. Nearly half of all Jewish Israelis have Middle
Eastern roots; some, in fact, are black. Martin Luther King Jr. preached
nonviolent resistance; Yasir Arafat practiced terrorism. The civil
rights movement was about getting America to live up its founding
ideals; anti-Zionism is about destroying Israel's founding ideals.

As for the oft-cited apartheid analogy, black South Africans did not
have a place in the old regime's Parliament, as Israeli Arabs have in
the Knesset; nor were they admitted to white universities, as Israeli
Arabs are to Israeli universities. Israel can do more to advance the
rights of its Arab citizens (just as the United States, France, Britain
and other countries can for their own minorities). And Israel can also
do more to ease the lives of Palestinians who are not citizens. But the
comparison of Israel to apartheid South Africa is unfair to the former
and an insult to the victims of the latter.

None of this should be hard for most progressives to understand. Indeed,
progressives have no trouble spotting anti-Semitism when it emanates
from the political right --- the effigies of George Soros, the attacks
on ``globalists'' with names like Blankfein and Yellen, the social media
memes borrowed from neo-Nazis. Yet it seems that a movement that can
detect a racist dog-whistle from miles away is strangely deaf when it
comes to some of the barking on its own side of the fence. And even when
it does hear it, it doesn't have the sense to banish it.

This is dangerous, and not just to Israeli and American Jews. In
Britain, the Labour Party is now led by a militant anti-Zionist whose
deep-seated anti-Semitism
\href{https://www.nytimes.com/2018/08/27/opinion/jeremy-corbyn-anti-semitism-labour-britain.html}{occasionally
slips out.} And yet Jeremy Corbyn remains in firm control of his party,
is reshaping it in his image and may yet become Britain's next prime
minister.

Image

The Labour Party in Britain is led by Jeremy Corbyn, who has expressed
anti-Zionist views.Credit...Andy Rain/European Pressphoto Agency, via
Shutterstock

The prospect of Corbynism coming to America may still seem remote. But
that can't be counted on in an era of sharp and rapid polarization. When
New York Representative Alexandria Ocasio-Cortez
\href{https://twitter.com/AOC/status/1092210825228636161}{tweeted}recently
about the ``honor'' of her ``lovely and wide-reaching conversation''
with Corbyn, it was a sign either of indifference or purposeful alliance
that ought to profoundly alarm every sensible Democrat worried about the
ideological direction and moral health of the party. Now is the time for
party leaders to make sure that doesn't happen by insisting that
anti-Zionism has no more a place in the Democratic fold than any form of
prejudice.

American democracy is already in jeopardy for having one party that has
surrendered to the politics of ethnic bigotry disguised as social
concern. To have two such parties would be fatal.

\emph{The Times is committed to publishing}
\href{https://www.nytimes.com/2019/01/31/opinion/letters/letters-to-editor-new-york-times-women.html}{\emph{a
diversity of letters}} \emph{to the editor. We'd like to hear what you
think about this or any of our articles. Here are some}
\href{https://help.nytimes.com/hc/en-us/articles/115014925288-How-to-submit-a-letter-to-the-editor}{\emph{tips}}\emph{.
And here's our email:}
\href{mailto:letters@nytimes.com}{\emph{letters@nytimes.com}}\emph{.}

\emph{Follow The New York Times Opinion section on}
\href{https://www.facebook.com/nytopinion}{\emph{Facebook}}\emph{,}
\href{http://twitter.com/NYTOpinion}{\emph{Twitter (@NYTopinion)}}
\emph{and}
\href{https://www.instagram.com/nytopinion/}{\emph{Instagram}}\emph{.}

Advertisement

\protect\hyperlink{after-bottom}{Continue reading the main story}

\hypertarget{site-index}{%
\subsection{Site Index}\label{site-index}}

\hypertarget{site-information-navigation}{%
\subsection{Site Information
Navigation}\label{site-information-navigation}}

\begin{itemize}
\tightlist
\item
  \href{https://help.nytimes.com/hc/en-us/articles/115014792127-Copyright-notice}{©~2020~The
  New York Times Company}
\end{itemize}

\begin{itemize}
\tightlist
\item
  \href{https://www.nytco.com/}{NYTCo}
\item
  \href{https://help.nytimes.com/hc/en-us/articles/115015385887-Contact-Us}{Contact
  Us}
\item
  \href{https://www.nytco.com/careers/}{Work with us}
\item
  \href{https://nytmediakit.com/}{Advertise}
\item
  \href{http://www.tbrandstudio.com/}{T Brand Studio}
\item
  \href{https://www.nytimes.com/privacy/cookie-policy\#how-do-i-manage-trackers}{Your
  Ad Choices}
\item
  \href{https://www.nytimes.com/privacy}{Privacy}
\item
  \href{https://help.nytimes.com/hc/en-us/articles/115014893428-Terms-of-service}{Terms
  of Service}
\item
  \href{https://help.nytimes.com/hc/en-us/articles/115014893968-Terms-of-sale}{Terms
  of Sale}
\item
  \href{https://spiderbites.nytimes.com}{Site Map}
\item
  \href{https://help.nytimes.com/hc/en-us}{Help}
\item
  \href{https://www.nytimes.com/subscription?campaignId=37WXW}{Subscriptions}
\end{itemize}
