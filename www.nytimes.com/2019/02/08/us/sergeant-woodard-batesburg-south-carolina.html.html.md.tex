Sections

SEARCH

\protect\hyperlink{site-content}{Skip to
content}\protect\hyperlink{site-index}{Skip to site index}

\href{https://www.nytimes.com/section/us}{U.S.}

\href{https://myaccount.nytimes.com/auth/login?response_type=cookie\&client_id=vi}{}

\href{https://www.nytimes.com/section/todayspaper}{Today's Paper}

\href{/section/us}{U.S.}\textbar{}Why a Town Is Finally Honoring a Black
Veteran Attacked by Its White Police Chief

\url{https://nyti.ms/2E1Brhr}

\begin{itemize}
\item
\item
\item
\item
\item
\end{itemize}

Advertisement

\protect\hyperlink{after-top}{Continue reading the main story}

Supported by

\protect\hyperlink{after-sponsor}{Continue reading the main story}

\hypertarget{why-a-town-is-finally-honoring-a-black-veteran-attacked-by-its-white-police-chief}{%
\section{Why a Town Is Finally Honoring a Black Veteran Attacked by Its
White Police
Chief}\label{why-a-town-is-finally-honoring-a-black-veteran-attacked-by-its-white-police-chief}}

\includegraphics{https://static01.nyt.com/images/2019/02/09/us/09woodard-1-print/merlin_149392098_2232d607-cab5-4d7d-b828-e94e59068f89-articleLarge.jpg?quality=75\&auto=webp\&disable=upscale}

By \href{https://www.nytimes.com/by/audra-d-s-burch}{Audra D. S. Burch}

\begin{itemize}
\item
  Feb. 8, 2019
\item
  \begin{itemize}
  \item
  \item
  \item
  \item
  \item
  \end{itemize}
\end{itemize}

It was a racial assault so vicious that it became one of the early
chords of the civil rights movement, and led to the desegregation of the
military.

Sgt. Isaac Woodard Jr., 26, was a decorated African-American veteran. He
had just been honorably discharged from the United States Army in 1946
and was headed home to Winnsboro, S.C. Still in uniform, Mr. Woodard,
was forcibly removed from the bus, brutally beaten and jailed by the
white police chief in the town of Batesburg.

But in the small town where Mr. Woodard was beaten so severely that he
lost his sight, the crime went unpunished and largely faded from memory.
Almost three generations later, a black Army veteran in Georgia and a
white federal judge in South Carolina separately stumbled upon Mr.
Woodard's story and vowed to honor his memory.

On Saturday afternoon, town and civic leaders and groups of veterans
will walk the two blocks from the bus stop to the jail where Mr. Woodard
was taken to honor his memory and acknowledge the cruelty that was done
to him. Part of the trickle of small towns throughout the country
confronting their violent, racist histories, the town of 5,000 --- now
called Batesburg-Leesville --- will
\href{https://www.postandcourier.com/columnists/column-a-cop-gouged-out-a-black-vet-s-eyes/article_b112cf02-2a7d-11e9-ad8c-07e0bc45c3aa.html}{unveil
a historic marker} downtown as a permanent reminder of the racial
injustice that happened there.

\emph{{[}For more coverage of race,}
\emph{\href{https://www.nytimes.com/2018/10/01/us/subscribe-race-related-newsletter.html?action=click\&module=inline\&pgtype=Article}{sign
up here}} \emph{to have our Race/Related newsletter delivered weekly to
your inbox.{]}}

``Here is this hero that so many people have forgotten or didn't know
about,'' said Don North, a former Army major from Carrollton, Ga., who
spent three years researching and raising money for the marker. ``This
is about remembering him, what he endured and the legacy he left
behind.''

Mr. Woodard enlisted in the Army in 1942, serving as a longshoreman in
the Pacific Theater of World War II. After he was discharged, he left
from Camp Gordon in Augusta, Ga.

What happened on the bus is still unclear, but a dispute over a restroom
break led the driver to call the police when they stopped in Batesburg,
35 miles southwest of Columbia.

Mr. Woodard and the bus driver argued after Mr. Woodard asked to take a
bathroom break. The bus company's policy required drivers to accommodate
such requests. The driver later said Mr. Woodard had been drunk and
unruly.

The police chief, Lynwood Shull, and another officer ordered Mr. Woodard
off the bus. He was beaten at various points while in police custody,
despite protesting that he had done nothing to warrant the assault.
Chief Shull jammed the ends of his blackjack into Mr. Woodard's eyes, at
one point striking him so violently that the stick broke.

The next morning, Mr. Woodard appeared before the local judge where he
was convicted of drunken and disorderly conduct and fined \$50, the coda
to a story that intimately portrayed the disrespect and horrific
treatment of black veterans returning from service.

``There were multiple episodes of black veterans abused across the
South,'' said Judge Richard Gergel, the federal judge who began
researching Mr. Woodard's history in 2011. ``They were serving their
country, fighting for American liberty and freedom and not given liberty
and freedom when they came back home.''

News of the blinding of a World World II veteran traveled beyond the
South, much of it carried by the black press. President Harry Truman was
sickened by the assault and ordered a federal investigation. It was a
highly unusual move at the time: investigating a white law enforcement
officer for violating a black person's civil rights.

At the trial, Chief Shull said he acted in self-defense and had only
struck Mr. Woodard once. Medical records never shown in court disproved
Chief Shull's claim. It took 28 minutes for an all-white jury to acquit
the police chief.

The judge who presided over the trial, J. Waties Waring, was deeply
angered by the verdict, later issuing several decisions that helped
upend Jim Crow laws.

``Two of the people profoundly affected by this story were Truman and J.
Waties Waring,'' Judge Gergel said. ``Truman formed the first
presidential civil rights committee and Judge Waring, who was influenced
by the travesty of justice, began writing landmark civil rights
decisions.''

\emph{{[}Judge Gergel has written a book about the case.}
\href{https://www.nytimes.com/2019/02/07/books/review/richard-gergel-unexampled-courage.html}{\emph{Read
our review here.}}\emph{{]}}

Mr. Woodard recovered in a veteran's hospital and eventually moved to
New York without his wife, who walked out on the marriage after the
incident. His family in New York helped to care for him until his death
in 1992. He was 73.

Robert Young, 81, Mr. Woodard's nephew and main caretaker, said that for
some time his uncle was understandably bitter about what had happened.
``But at some point, he just tried to live his life,'' he said. ``I
still have this wonderful memory of him coming home for a break before
he went overseas. He was standing in the kitchen in his Army uniform and
my mom and sister were so excited to see him. He had made us proud.''

Mr. Woodard's story might never have been formally noted by
Batesburg-Leesville if not for Mr. North and Judge Gergel.

Mr. North was stationed at Fort McPherson in Atlanta in the early 1990s
when he first heard about Mr. Woodard. A student of black military
history, Mr. North researched more and decided a historic marker at the
site of the bus stop was the best way to honor the veteran.

He launched the Sgt. Isaac Woodard Jr. Historical Marker Association and
submitted plans to state officials in 2017. The Disabled American
Veterans organization funded most of the cost of the marker.

Judge Gergel, who presided in the Dylann Roof murder trial in 2017,
began poring over police, court and medical records looking for the
tiniest details of the Woodard attack around 2011. He talked to the
town's city attorney, Chris Spradley. He had never heard of the case.
Neither had Mr. Spradley's father, who was a former mayor.

The papers and archives were not enough to fully unravel an attack that
was six decades old. So on a cool winter morning in December 2017, Judge
Gergel traveled from Charleston to Batesburg-Leesville to see where it
all happened, and walk the 250 or so paces from the bus stop to the
jail.

Town leaders met him there. They, too, wanted to find a meaningful way
to honor Mr. Woodard.

``I had not heard about this,'' said Mayor Lance Shull, who is not
related to the late police chief. ``This event had been mostly swept
under the rug. I know it can't be corrected, we can't erase what
happened but we can acknowledge this horrible incident.''

``We want people to know it is not 1946 here anymore,'' he added.

The town jumped in to support Mr. North's plans for the marker. Leaders
also decided to do something else, equally permanent: they reopened the
case in June against Mr. Woodard and a municipal court judge dismissed
the drunken and disorderly charge.

On Saturday, a group of town leaders, African-American and disabled
veterans as well as Mr. Young and Mr. Gergel, who are meeting in person
for the first time, will walk the two blocks through downtown and past
the remnants of what Mr. Woodard likely last saw before being robbed of
his eyesight: the railroad station, a drugstore, a hardware store.

They will gather at the corner of a vacant lot where the old jail once
stood. The state historic marker will rise from a tiny patch of earth,
filled with freshly planted yellow nandina plants.

Mr. Woodard's story is etched in the marker. In print. And Braille.

Advertisement

\protect\hyperlink{after-bottom}{Continue reading the main story}

\hypertarget{site-index}{%
\subsection{Site Index}\label{site-index}}

\hypertarget{site-information-navigation}{%
\subsection{Site Information
Navigation}\label{site-information-navigation}}

\begin{itemize}
\tightlist
\item
  \href{https://help.nytimes.com/hc/en-us/articles/115014792127-Copyright-notice}{©~2020~The
  New York Times Company}
\end{itemize}

\begin{itemize}
\tightlist
\item
  \href{https://www.nytco.com/}{NYTCo}
\item
  \href{https://help.nytimes.com/hc/en-us/articles/115015385887-Contact-Us}{Contact
  Us}
\item
  \href{https://www.nytco.com/careers/}{Work with us}
\item
  \href{https://nytmediakit.com/}{Advertise}
\item
  \href{http://www.tbrandstudio.com/}{T Brand Studio}
\item
  \href{https://www.nytimes.com/privacy/cookie-policy\#how-do-i-manage-trackers}{Your
  Ad Choices}
\item
  \href{https://www.nytimes.com/privacy}{Privacy}
\item
  \href{https://help.nytimes.com/hc/en-us/articles/115014893428-Terms-of-service}{Terms
  of Service}
\item
  \href{https://help.nytimes.com/hc/en-us/articles/115014893968-Terms-of-sale}{Terms
  of Sale}
\item
  \href{https://spiderbites.nytimes.com}{Site Map}
\item
  \href{https://help.nytimes.com/hc/en-us}{Help}
\item
  \href{https://www.nytimes.com/subscription?campaignId=37WXW}{Subscriptions}
\end{itemize}
