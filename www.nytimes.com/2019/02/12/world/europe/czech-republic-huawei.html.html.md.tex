Sections

SEARCH

\protect\hyperlink{site-content}{Skip to
content}\protect\hyperlink{site-index}{Skip to site index}

\href{https://www.nytimes.com/section/world/europe}{Europe}

\href{https://myaccount.nytimes.com/auth/login?response_type=cookie\&client_id=vi}{}

\href{https://www.nytimes.com/section/todayspaper}{Today's Paper}

\href{/section/world/europe}{Europe}\textbar{}Huawei Was a Czech
Favorite. Now? It's a National Security Threat.

\url{https://nyti.ms/2UVcM3H}

\begin{itemize}
\item
\item
\item
\item
\item
\end{itemize}

Advertisement

\protect\hyperlink{after-top}{Continue reading the main story}

Supported by

\protect\hyperlink{after-sponsor}{Continue reading the main story}

\hypertarget{huawei-was-a-czech-favorite-now-its-a-national-security-threat}{%
\section{Huawei Was a Czech Favorite. Now? It's a National Security
Threat.}\label{huawei-was-a-czech-favorite-now-its-a-national-security-threat}}

\includegraphics{https://static01.nyt.com/images/2019/02/12/world/12Huawei-Czech1/merlin_144343896_f776ac7d-f8ea-47c7-babd-42c78af761dd-articleLarge.jpg?quality=75\&auto=webp\&disable=upscale}

By \href{https://www.nytimes.com/by/marc-santora}{Marc Santora} and Hana
de Goeij

\begin{itemize}
\item
  Feb. 12, 2019
\item
  \begin{itemize}
  \item
  \item
  \item
  \item
  \item
  \end{itemize}
\end{itemize}

\href{https://cn.nytimes.com/world/20190213/czech-republic-huawei/}{阅读简体中文版}\href{https://cn.nytimes.com/world/20190213/czech-republic-huawei/zh-hant/}{閱讀繁體中文版}

PRAGUE --- For more than 1,000 years, the sprawling castle complex
perched high above Prague has been the seat of power for Holy Roman
emperors, the kings of Bohemia and, now, the Czech president, Milos
Zeman. And for the last four years, the Chinese technology giant Huawei
has had a contract to fulfill the communication needs of the president
and his staff.

The presidential contract is the most visible symbol of how deeply
Huawei has established itself in the Czech Republic, long viewed by
China as a springboard country for its interests across the European
Union.

So when the Czech government's cybersecurity agency issued a directive
in December warning that Huawei represented a potential national
security threat, company officials were shocked --- as was Mr. Zeman,
known for his closeness to China. Huawei has threatened legal and
financial retaliation. Mr. Zeman has accused his own intelligence
services, including the cybersecurity agency, known as Nukib, of ``dirty
tricks.''

The unexpected confrontation in the Czech Republic comes as Huawei,
already entangled in the trade war between China and the United States,
is running into deepening problems in European Union countries, where it
has worked for years to build inroads. Only weeks after Nukib issued its
directive against Huawei, Polish authorities in January arrested a
Huawei employee in Warsaw on charges of spying.

Looming above everything is the question of which companies will build
the infrastructure of the fifth generation of wireless technology, known
as 5G. American officials regard the technology as a national security
issue and have moved aggressively to limit the role of Chinese
companies, especially Huawei.

Speaking in Budapest on Monday, Secretary of State Mike Pompeo, who is
touring Central Europe, warned about ``risks that Huawei's presence in
their networks present --- actual risks to their people, to the loss of
privacy protections.''

\includegraphics{https://static01.nyt.com/images/2019/02/14/world/14jpHUAWEI-CZECH/merlin_138809460_d58947bd-f2e6-4be6-a91e-56739c4382e0-articleLarge.jpg?quality=75\&auto=webp\&disable=upscale}

Huawei is already hitting problems in Germany and Britain, traditional
American allies. Yet few people would have predicted that the most
direct action against the company would come from the Czech Republic,
where Mr. Zeman has long courted Chinese investment and is planning to
make his fifth visit to China in the spring. The Nukib directive has
infuriated the president and driven a wedge into the Czech government.

In an interview, Dusan Navratil, the head of the agency, said the stakes
are high because 5G ``will change the whole way societies function.''
While the public might think of 5G as simply a matter of faster download
times, the system is designed to be far more, linking everything from
the cars we drive to the hospitals we visit in a way unimaginable only a
decade ago.

Mr. Navratil said one reason his agency penalized Huawei was China's
\href{https://www.nytimes.com/2017/05/31/business/china-cybersecurity-law.html}{National
Intelligence Law}, which was passed in 2017 and requires Chinese
companies to support, provide assistance and cooperate in the
authoritarian nation's national intelligence work, wherever they
operate. While declining to discuss classified intelligence, he offered
an analogy.

``Imagine there is a restaurant where the hygiene is filthy rotten,''
Mr. Navratil said. ``The agency overseeing the restaurant has no
evidence someone has gotten sick or died. But does that mean you should
eat there?''

As a result of the agency's move, Huawei's business in the Czech
Republic could be devastated. The directive doesn't affect consumer
products, such as mobile phones, but it aims to severely restrict the
role Huawei can play in 5G, and to block it from supplying equipment to
public and private entities deemed critical to national security.

``This is the first time we have ever issued such a warning,'' Mr.
Navratil said.

Since then, the Ministries of Health and Justice announced that they
would not honor existing contracts to buy servers from Huawei. The
country's largest carmaker, Skoda, has temporarily stopped Huawei from
bidding on projects as it undergoes a security review demanded by the
new cyberdirective.

Even Huawei's contract with the presidential offices is under review.

In turn, Huawei has been hitting back. Last week, the company threatened
litigation against the cyberagency, as well as economic retaliation. It
also has intensified lobbying campaigns with some lawmakers and others
in the political elite, including a Czech delegation that the company
hosted in China last month.

Image

``We cannot just rush headlong into this,'' Dusan Navratil, the head of
the Czech cybersecurity agency, said of the country's relationship with
Huawei.Credit...Vaclav Salek/CTK, via Associated Press

``The Chinese believe they are one to two years ahead of everyone and
they see this as an effort to slow them down,'' said Jaroslav Roman, a
Czech Communist Party official on the visit. ``They wanted us to know
that if the discrimination continued, there would be repercussions.''

Huawei officials have argued that the Chinese law regarding intelligence
is being deliberately misinterpreted, a point disputed by many Western
experts.

But it is the threat of economic retaliation that has most alarmed many
business leaders and lawmakers, given how ardently Mr. Zeman has built
economic ties between the two countries. One example is PPF Group, the
largest Czech company, which has focused on China in recent years.

Its home lending business counts some 15 million Chinese customers,
which is not much in Chinese terms but hugely important for the
Czech-owned company. PPF also controls the country's largest telecom
provider, O2, and had signed a memorandum of understanding with Huawei
regarding future projects before the security agency issued its warning.
That will now have to be reviewed.

Mel Carvill, a board member for Home Credit, a consumer loan company
that is a part of the PPF Group, said the Chinese regulatory agency
overseeing the home lending business had not taken any action against
the company.

``Of course, if relations between countries deteriorate, things might
become difficult,'' he said.

Huawei officials, for their part, were so confident in their standing in
the Czech Republic that in June, they decided to push for Czech security
clearance to work on critical infrastructure --- a move that may have
contributed to the cyberagency's decision to more closely scrutinize the
Chinese company.

``It was overreach on their part,'' said Jakub Janda, the director at
the Prague-based think tank
\href{https://www.europeanvalues.net/o-nas/nas-tym/}{European Values}.

Image

Huawei's offices in Warsaw. The company has threatened litigation
against the cyberagency and economic retaliation against the
country.Credit...Maciek Nabrdalik for The New York Times

Mr. Janda said there was particular concern about the dozens of
technology assistants working for Huawei who had unlimited access to all
the data gathered by their mobile partners.

Unlike cases involving Russians accused of spying in the Czech Republic,
Mr. Janda noted, concerns about Chinese espionage are kept quiet for
diplomatic reasons. No one has been charged with spying in the country.

Even as Huawei is trying to regain footing in the Czech Republic, the
company is also mounting a public relations counteroffensive in Poland,
after the arrest in January of the company executive. Last week, Huawei
issued a news release in Poland, warning that communication services
would increase by 300 percent if the company were excluded from the
market.

Meanwhile, Huawei has sought to ease Polish security concerns by
promising to spend hundreds of millions on a new cybersecurity center.
It has also offered the Polish government access to Huawei's source code
in order to prove that its equipment does not have a ``back door'' for
the Chinese intelligence services.

However, some experts have said such gestures are hollow, given how
often codes can be updated or changed.

For Mr. Navratil, the head of the Czech cyberagency, the issue facing
security officials was as evident as it was urgent.

``We are a sovereign country and we have a right to say what is in our
security interest,'' he said. It would be helpful, he said, if NATO or
the E.U. would offer some specific guidance since it would give small
nations greater ability to act without fear of retaliation.

That does not appear to be happening anytime soon.

``Given the stakes,'' Mr. Navratil said, ``we cannot just rush headlong
into this.''

Advertisement

\protect\hyperlink{after-bottom}{Continue reading the main story}

\hypertarget{site-index}{%
\subsection{Site Index}\label{site-index}}

\hypertarget{site-information-navigation}{%
\subsection{Site Information
Navigation}\label{site-information-navigation}}

\begin{itemize}
\tightlist
\item
  \href{https://help.nytimes.com/hc/en-us/articles/115014792127-Copyright-notice}{©~2020~The
  New York Times Company}
\end{itemize}

\begin{itemize}
\tightlist
\item
  \href{https://www.nytco.com/}{NYTCo}
\item
  \href{https://help.nytimes.com/hc/en-us/articles/115015385887-Contact-Us}{Contact
  Us}
\item
  \href{https://www.nytco.com/careers/}{Work with us}
\item
  \href{https://nytmediakit.com/}{Advertise}
\item
  \href{http://www.tbrandstudio.com/}{T Brand Studio}
\item
  \href{https://www.nytimes.com/privacy/cookie-policy\#how-do-i-manage-trackers}{Your
  Ad Choices}
\item
  \href{https://www.nytimes.com/privacy}{Privacy}
\item
  \href{https://help.nytimes.com/hc/en-us/articles/115014893428-Terms-of-service}{Terms
  of Service}
\item
  \href{https://help.nytimes.com/hc/en-us/articles/115014893968-Terms-of-sale}{Terms
  of Sale}
\item
  \href{https://spiderbites.nytimes.com}{Site Map}
\item
  \href{https://help.nytimes.com/hc/en-us}{Help}
\item
  \href{https://www.nytimes.com/subscription?campaignId=37WXW}{Subscriptions}
\end{itemize}
