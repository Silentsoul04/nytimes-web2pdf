Sections

SEARCH

\protect\hyperlink{site-content}{Skip to
content}\protect\hyperlink{site-index}{Skip to site index}

\href{https://www.nytimes.com/section/technology}{Technology}

\href{https://myaccount.nytimes.com/auth/login?response_type=cookie\&client_id=vi}{}

\href{https://www.nytimes.com/section/todayspaper}{Today's Paper}

\href{/section/technology}{Technology}\textbar{}India Proposes
Chinese-Style Internet Censorship

\href{https://nyti.ms/2UXhaPS}{https://nyti.ms/2UXhaPS}

\begin{itemize}
\item
\item
\item
\item
\item
\end{itemize}

Advertisement

\protect\hyperlink{after-top}{Continue reading the main story}

Supported by

\protect\hyperlink{after-sponsor}{Continue reading the main story}

\hypertarget{india-proposes-chinese-style-internet-censorship}{%
\section{India Proposes Chinese-Style Internet
Censorship}\label{india-proposes-chinese-style-internet-censorship}}

\includegraphics{https://static01.nyt.com/images/2019/02/15/business/15indiacensor1/merlin_121556441_f33419db-157b-49bc-b54a-27d8cfdf6620-articleLarge.jpg?quality=75\&auto=webp\&disable=upscale}

By \href{https://www.nytimes.com/by/vindu-goel}{Vindu Goel}

\begin{itemize}
\item
  Feb. 14, 2019
\item
  \begin{itemize}
  \item
  \item
  \item
  \item
  \item
  \end{itemize}
\end{itemize}

NEW DELHI --- India's government has proposed giving itself vast new
powers to suppress internet content, igniting a heated battle with
global technology giants and prompting comparisons to censorship in
China.

Under the
\href{https://meity.gov.in/content/comments-suggestions-invited-draft-\%E2\%80\%9C-information-technology-intermediary-guidelines}{proposed
rules}, Indian officials could demand that Facebook, Google, Twitter,
\href{https://www.nytimes.com/2018/09/28/technology/bytedance-fundraising-toutiao-tiktok.html}{TikTok}
and others remove posts or videos that they deem libelous, invasive of
privacy, hateful or deceptive. Internet companies would also have to
build automated screening tools to block Indians from seeing ``unlawful
information or content.'' Another provision would weaken the privacy
protections of messaging services like WhatsApp so that the authorities
could trace messages back to their original senders.

The new rules could be imposed by Prime Minister Narendra Modi's
government anytime after the public comment period ends on Thursday
night. The administration has been eager to get them in place before the
date is set for this spring's national elections, which will prompt
special pre-election rules limiting new policies.

Civil liberties groups and other critics said the changes would violate
constitutional protections for free speech and privacy and put India in
the same league as autocratic countries like China and Russia. Some of
them suggested that the Modi administration was rushing to adopt the
regulations so it could more easily pressure the tech platforms to
remove social media posts by political opponents in the coming election.

``The proposed changes have an authoritarian bent,'' said Apar Gupta,
executive director of the Internet Freedom Foundation, a digital rights
group, which plans to challenge the rules in court if they are enacted.
``This is very similar to what China does to its citizens, where it
polices their every move and tracks their every post on social media.''

India's proposals add to the growing resistance worldwide against
internet behemoths like Google and Facebook, which once flourished
largely unimpeded. In Europe, officials last year
\href{https://www.nytimes.com/2018/05/24/technology/europe-gdpr-privacy.html}{enacted
tough new rules} to protect people's online data, forcing the companies
to change some practices. China has long used
\href{https://www.nytimes.com/interactive/2018/11/18/world/asia/china-internet.html}{a
system of internet filters}, known as the Great Firewall, to block
content and shut out global tech companies. And
\href{https://www.nytimes.com/2017/09/17/technology/facebook-government-regulations.html}{in
a 2017 review}, The New York Times tallied more than 50 countries that
had passed laws in recent years to gain greater control over how their
people use the web.

The result may be a splintering internet, where a onetime unified
information superhighway has become increasingly restricted in certain
areas. In India, the
\href{https://www.nytimes.com/2018/08/31/technology/india-technology-american-giants.html}{government
has used laws to nudge people away} from the American tech giants and
toward
\href{https://www.nytimes.com/2019/01/30/technology/amazon-walmart-flipkart-india.html?rref=collection\%2Fbyline\%2Fvindu-goel\&action=click\&contentCollection=undefined\&region=stream\&module=stream_unit\&version=latest\&contentPlacement=1\&pgtype=collection}{local
competitors}, such as ShareChat, a social network that operates only in
Indian languages, and Reliance Jio, a cellphone giant bankrolled by
India's richest man.

Mishi Choudhary, founder of SFLC.in, a legal advocacy group in New
Delhi, said Indian governments had tried for nearly a decade to exercise
more control over internet content but had been restrained by the
courts. Now the government is trying again, she said.

``Ministers have said these companies will have to comply with Indian
rules, and Indian rules are pretty regressive,'' said Ms. Choudhary,
whose group filed comments opposing the new rules. ``We have Indian
morality. We have to keep law and order. And we cannot hurt religious
sentiments.''

India's Ministry of Electronics and Information Technology, which
proposed
\href{https://meity.gov.in/content/comments-suggestions-invited-draft-\%E2\%80\%9C-information-technology-intermediary-guidelines}{the
online content changes}, has said the new rules simply build on existing
laws and are necessary to combat false and illegal information on social
media. The regulations also seek to hold foreign tech companies ---
including Chinese app makers that have little presence in India now ---
more accountable. Companies with more than five million Indian users
would be required to set up a local subsidiary and appoint leaders based
in the country.

Ajay Sawhney, the information technology secretary, did not respond to
requests for an interview. Officials have privately suggested that they
will consider the public responses to their draft rules and make
adjustments before issuing a final version.

\includegraphics{https://static01.nyt.com/images/2019/02/15/business/15indiacensor3/merlin_84188417_64efdb4a-c02d-4096-b891-44f3c9782b26-articleLarge.jpg?quality=75\&auto=webp\&disable=upscale}

Working independently as well as through trade groups, Microsoft,
Facebook and dozens of other tech companies are fighting back against
the proposals. They criticized the rules as technically impractical and
said they were a sharp departure from how the rest of the world
regulates ``data intermediaries,'' a term for companies that host data
provided by their customers and users.

In most countries, including under India's existing laws, such
intermediaries are given a ``safe harbor.'' That means they are exempted
from responsibility for illegal or inappropriate content posted on their
services, as long as they remove it once notified by a court or another
designated authority.

In a
\href{https://meity.gov.in/content/addendum1-comments-received-publicstakeholders-draft-\%E2\%80\%9C-information-technology-intermediary}{filing
with the ministry} last week, Microsoft said that complying with India's
new standards would be ``impossible from the process, legal and
technology point of view.''

The company --- whose Hyderabad-born chief executive, Satya Nadella, is
a business icon in India --- said the proposal lumped together all
intermediaries, as varied as social networks and Wi-Fi hot spots, even
though each has a different level of control over content that flows
through it. In the filing, Microsoft said it would be difficult to
screen out gambling content, as the rules would require. Filtering the
full range of content demanded by the government would not only violate
privacy and freedom of expression, the company wrote, but would also be
so challenging that ``the cost of even attempting compliance will be
prohibitive.''

Trade groups representing the largest tech companies have made similar
arguments in their filings.

Last week, WhatsApp, which is owned by Facebook and has some 250 million
active users in India, said it could not meet the proposed requirement
that it trace viral messages to their origin without destroying the
privacy protections that are core to the service.

``WhatsApp cares deeply about creating a space for private conversations
online,'' Carl Woog, a company spokesman, said at a news conference in
New Delhi. He said the proposed rules ``would require us to re-architect
WhatsApp, leading to a different product, one that would not be
fundamentally private.''

Image

Indian authorities are struggling to deal with inflammatory messages
that spread on social platforms. They shut down the entire internet in
2017 in the state of Punjab just before a popular spiritual guru was
convicted on rape charges. That did not stop his followers from flooding
the streets, and at least 30 people died in the
violence.Credit...Narinder Nanu/Agence France-Presse --- Getty Images

Google and Facebook declined to comment beyond filings made by the
industry groups to which they belong. Twitter, which is
\href{https://economictimes.indiatimes.com/tech/internet/house-panel-gives-twitter-15-days-to-present-ceo-jack-dorsey/articleshow/67951655.cms}{jousting
with India's Parliament} over claims that it suppresses right-wing
content, said in a statement that it hoped that any changes to the rules
would ``strike a careful balance that protects important values such as
freedom of expression.''

India began signaling last year that it
\href{https://www.nytimes.com/2018/08/31/technology/india-technology-american-giants.html}{planned
to impose tough rules} on the tech industry, ending the free rein that
American tech giants have long enjoyed in this country of 1.3 billion
people, which has been the world's fastest-growing market for new
internet users. Among other things, officials discussed European-style
limits on what big internet companies can do with users' personal data.

The newest proposals on internet content were introduced at a private
meeting with tech companies
\href{https://meity.gov.in/content/comments-suggestions-invited-draft-\%E2\%80\%9C-information-technology-intermediary-guidelines}{in
December}. They were on track for quick passage until the details
\href{https://indianexpress.com/article/india/it-act-amendments-data-privacy-freedom-of-speech-fb-twitter-5506572/}{leaked
to The Indian Express}, a local newspaper, which prompted the government
to invite broader feedback.

Officials have offered little public explanation for the proposals,
beyond a desire to curb the kind of
\href{https://www.nytimes.com/interactive/2018/11/23/technology/whatsapp-india-killings-ES.html}{false
rumors about child kidnappers that spread on WhatsApp} a year ago and
that incited angry mobs to kill two dozen innocent people. That wave of
violence has since subsided.

The coming national election has added urgency to the proposals. India's
Election Commission, which administers national and state elections, is
considering a ban on all social media content and ads aimed at
influencing voters for the 48 hours before voting begins, according to
an internal report obtained by the news media. To buttress its legal
authority to order such a ban, the commission wrote to the I.T. ministry
last week asking it to amend the new rules to specifically prohibit
online content that violates election laws or commission orders.

One of the biggest cheerleaders for the new rules was Reliance Jio, a
fast-growing mobile phone company controlled by Mukesh Ambani, India's
richest industrialist. Mr. Ambani, an ally of Mr. Modi, has made no
secret of his plans to turn Reliance Jio into an all-purpose information
service that offers streaming video and music, messaging, money
transfer, online shopping, and home broadband services.

In a filing last week, Reliance Jio said the new rules were necessary to
combat ``miscreants'' and urged the government to ignore free-speech
protests. The company also said that encrypted messaging services like
WhatsApp, ``although perceivably beneficial to users, are detrimental to
national interest and hence should not be allowed.''

Advertisement

\protect\hyperlink{after-bottom}{Continue reading the main story}

\hypertarget{site-index}{%
\subsection{Site Index}\label{site-index}}

\hypertarget{site-information-navigation}{%
\subsection{Site Information
Navigation}\label{site-information-navigation}}

\begin{itemize}
\tightlist
\item
  \href{https://help.nytimes.com/hc/en-us/articles/115014792127-Copyright-notice}{©~2020~The
  New York Times Company}
\end{itemize}

\begin{itemize}
\tightlist
\item
  \href{https://www.nytco.com/}{NYTCo}
\item
  \href{https://help.nytimes.com/hc/en-us/articles/115015385887-Contact-Us}{Contact
  Us}
\item
  \href{https://www.nytco.com/careers/}{Work with us}
\item
  \href{https://nytmediakit.com/}{Advertise}
\item
  \href{http://www.tbrandstudio.com/}{T Brand Studio}
\item
  \href{https://www.nytimes.com/privacy/cookie-policy\#how-do-i-manage-trackers}{Your
  Ad Choices}
\item
  \href{https://www.nytimes.com/privacy}{Privacy}
\item
  \href{https://help.nytimes.com/hc/en-us/articles/115014893428-Terms-of-service}{Terms
  of Service}
\item
  \href{https://help.nytimes.com/hc/en-us/articles/115014893968-Terms-of-sale}{Terms
  of Sale}
\item
  \href{https://spiderbites.nytimes.com}{Site Map}
\item
  \href{https://help.nytimes.com/hc/en-us}{Help}
\item
  \href{https://www.nytimes.com/subscription?campaignId=37WXW}{Subscriptions}
\end{itemize}
