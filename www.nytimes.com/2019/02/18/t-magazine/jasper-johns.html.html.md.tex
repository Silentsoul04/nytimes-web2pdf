Jasper Johns, American Legend

\begin{itemize}
\item
\item
\item
\item
\item
\end{itemize}

\includegraphics{https://static01.nyt.com/images/2019/02/18/t-magazine/18tmag-johns-slide-8VTF/18tmag-johns-slide-8VTF-articleLarge.jpg?quality=75\&auto=webp\&disable=upscale}

Sections

\protect\hyperlink{site-content}{Skip to
content}\protect\hyperlink{site-index}{Skip to site index}

\hypertarget{jasper-johns-american-legend}{%
\section{Jasper Johns, American
Legend}\label{jasper-johns-american-legend}}

The artist's work has managed to speak both to and for the country's
consciousness for the last 60 years --- and he's not done yet.

Jasper Johns, photographed in Sharon, Conn., on Dec. 19,
2018.Credit...Joel Sternfeld

Supported by

\protect\hyperlink{after-sponsor}{Continue reading the main story}

By \href{https://www.nytimes.com/by/m-h-miller}{M.H. Miller}

\begin{itemize}
\item
  Feb. 18, 2019
\item
  \begin{itemize}
  \item
  \item
  \item
  \item
  \item
  \end{itemize}
\end{itemize}

JASPER JOHNS LIVES on a sprawling estate in Sharon, Conn., a rural town
in the Berkshires with a population of about 2,700 people. The property
is stark and hilly, made up of a series of small barnlike structures,
one of which houses Johns's studio. Every detail inside the studio seems
intentional, as if each object were a clue about the man himself.
There's the unframed poster, pinned to a wall, of Leonardo da Vinci's
portrait of
``\href{https://www.nga.gov/collection/art-object-page.50724.html}{Ginevra
de' Benci}'' (circa 1478), with whom Johns occasionally shares an
intense, unsmiling gaze; there are the silver cans that hold his
brushes; there is the model of the human skull on a table.

In a 1977 interview with the author Edmund White, Johns described his
experience of meeting
\href{https://www.nytimes.com/topic/person/marcel-duchamp}{Marcel
Duchamp}, one of his artistic idols, who, with the Cubist paintings and
ready-made sculptures he began making in the years leading up to World
War I, helped drag art into the 20th century in much the same way that
Johns would recalibrate the priorities of painting and sculpture at the
end of the 1950s. ``Just his physical presence was impressive,'' Johns
said of Duchamp. ``I suppose all the mythology sensitizes you, prepares
you to be impressed, to feel awe.'' This is an apt description of Johns
himself, who has, for much of his adult life, cultivated the aura of an
enigma.

\emph{{[}}\href{https://www.nytimes.com/newsletters/t-list?module=inline}{\emph{Sign
up here}} \emph{for the T List newsletter, a weekly roundup of what T
Magazine editors are noticing and coveting now.{]}}

At 88, Johns remains physically imposing: He is barrel-chested, and his
once boyish face has weathered into a craggy atlas. When I visited his
home, he showed me the mostly bare walls of his studio before leading me
upstairs, into an informal gallery room, with windows that looked out on
the surrounding landscape, which felt, with the naked trees on the
horizon, vast and lonely in the early December chill. Even his gait as
he climbed the stairs had a meaningful vigor, as if he was trying to
prove to the steps that he could still conquer them. Johns is a solitary
figure, among the final survivors of an era, and for the better part of
60 years, he has declined to offer any easy explanations about his work,
or to be a spokesperson for postwar American art, though people would
like him to be. He has been one of the primary architects of the
contemporary art world, and has also opted out of its social trappings
entirely. For decades, he has divided his time between quiet towns along
the East Coast and a remote retreat designed by
\href{https://www.nytimes.com/2005/01/27/arts/design/philip-johnson-architectures-restless-intellect-dies-at-98.html}{Philip
Johnson} in St. Martin. Now, he rarely leaves Connecticut. The curator
John Elderfield has called him ``the hermit of Sharon.''

The next years will be busy ones for Johns. His first show of new works
in five years is currently running at
\href{https://www.matthewmarks.com/new-york/exhibitions/2019-02-09_jasper-johns/}{Matthew
Marks Gallery} in New York. He has been the subject of numerous surveys,
and in the fall of 2020, he'll have the largest one to date, split
between two institutions --- the
\href{https://www.nytimes.com/topic/organization/whitney-museum-of-american-art}{Whitney
Museum of American Art}, where he had a retrospective in 1977, and the
\href{https://www.nytimes.com/topic/organization/philadelphia-museum-of-art}{Philadelphia
Museum of Art}, which he started visiting in 1958 to view the museum's
collection of sculptures by Duchamp. The show will run simultaneously at
both museums, which the Whitney's Scott Rothkopf, the co-curator of the
show along with Carlos Basualdo in Philadelphia, described as
``unprecedented.''

Johns has assiduously avoided his public throughout his career, and yet
he has also managed to consistently speak to what it is to be alive in
America at any given moment. From his
\href{https://www.moma.org/collection/works/78805}{iconic paintings} in
the mid-50s of the American flag, which seemed to embody the fallout of
Red Scare nationalism, to the modish apathy of his bronze sculptures of
banal objects like flashlights and light bulbs, to his almost compulsive
return in his later paintings to a holistic system of ambiguous symbols
like galaxy spirals and cartoonish stick figures holding exaggeratedly
large paintbrushes, he has been in a constant state of reinvention. He
is the rare artist whose work has never become stale, who in his 80s is
still creating strange and mysterious images that could be looked at
endlessly and never fully reveal themselves. Whether Johns is actually
\emph{about} anything (or nothing) in particular has been the central
question of his work, and yet it is ultimately less important than his
endless search for meaning itself --- the mere act of the lone artist
entering the studio every day and deciding to continue. His constant
presence is defined mostly by self-erasure, which has made him an artist
who has disappeared almost entirely into his work. There is a sense that
he's been here forever, and that no one will replace him once he's gone.

\href{https://www.nytimes.com/slideshow/2019/02/18/t-magazine/jasper-johns-in-images.html}{}

\hypertarget{jasper-johns-in-images}{%
\subsection{Jasper Johns, in Images}\label{jasper-johns-in-images}}

13 Photos

View Slide Show ›

\includegraphics{https://static01.nyt.com/images/2019/02/18/t-magazine/18tmag-johns-slide-5TPA/18tmag-johns-slide-5TPA-articleLarge-v6.jpg?quality=75\&auto=webp\&disable=upscale}

Jasper Johns, ``Flag,'' 1954-55, encaustic, oil and collage on fabric
mounted on wood (3 panels), The Museum of Modern Art, New York, NY, USA.
Digital image © The Museum of Modern Art/licensed by Scala, NY/Art
Resource, NY © 2019 Jasper Johns/licensed by VAGA at ARS, NY

I CAN'T SAY that my encounter with Johns did much to upend his
reputation as an impenetrable figure. He had a remarkable ability to cut
off a conversational thread with a single look. When asked if there were
any younger artists he admired, he said, ``Mmhmm.'' Asked for specific
names, he responded with an unsmiling, ``No.'' His speech was punctuated
by long, powerful silences during which he stared out into the distance,
looking at nothing in particular but doing so with such a sense of
purpose it was as if he were searching the hills for the words he wanted
to say before emerging with a full-paragraph answer. When we sat
upstairs --- a book of paintings by
\href{https://www.nytimes.com/topic/person/edvard-munch}{Edvard Munch},
with whom Johns shares a morbid sense of symbolism, between us --- there
was a certain amount of negotiation regarding my recording our
interview. I told him it was the only way I could know that I've quoted
him accurately. ``That's what I'm worried about,'' he said grimly before
relenting.

Yet there was also a great warmth to him. He would frequently laugh at
the things he said, his eyes brightening. He spoke with a mid-Atlantic
accent that recalled
\href{https://www.nytimes.com/1986/11/30/obituaries/cary-grant-dies-in-iowa-at-82-hollywood-epitome-of-style.html}{Cary
Grant}, but when the conversation turned to his childhood in South
Carolina, a southern lilt announced itself. In these moments, he could
be surprisingly forthcoming, almost avuncular. I asked him why, unlike
most artists who achieve the level of acclaim he has, he has never
taught at a university. ``I wouldn't know how,'' he said. Then he told
me a story. When he was in basic training at Fort Jackson in South
Carolina in 1951, he became friendly with a woman named Augusta Burch,
who was in charge of the base's service club, where soldiers would go to
listen to records and write letters. Learning that he aspired to be an
artist, she asked him to make a mural based on the work of
\href{https://www.nytimes.com/topic/person/charles-dickens}{Charles
Dickens} for the service club's stage. Later, she started a culture
center at Fort Jackson devoted to the study of music and art, and when
Johns finished his basic training, she helped get him transferred there.
He was supposed to give art lessons to other soldiers. ``But I certainly
didn't teach,'' he said. ``I just watched what people did.''

Part of the job also involved arranging small exhibitions, and he
organized one around artworks by children, which he had borrowed from a
school at the recently opened
\href{https://www.columbiamuseum.org/}{Columbia Museum of Art} nearby.
One afternoon, a general from another base visited the culture center
and mistook the works on the walls for those made by the soldiers
instead of children. ``He looked around,'' Johns said, ``and he
announced, `What these men need is a teacher!'''

Johns was born in 1930 in Augusta, Ga., and grew up in South Carolina.
According to family lore, he was named for William Jasper, a sergeant in
the Second South Carolina Regiment during the Revolutionary War who was
best known for hoisting the regimental flag after the mast was broken in
the battle of Fort Moultrie, near Charleston, in 1776 and holding it
under fire from a British warship until it could be repaired. (He died
during the 1779 Siege of Savannah, purportedly attempting to plant the
regiment's colors again.) Johns's parents divorced when he was 2, and
his mother started a new family --- without him. Asked once by Deborah
Solomon, his biographer, why he didn't simply live with his father,
Johns responded, ``He didn't invite me.'' Raised by his paternal
grandfather, a farmer, until his death, when Johns was 8, he spent most
of the remainder of his childhood living with an aunt, the only teacher
at a two-room schoolhouse that Johns attended in the town of Allendale.
There were about 12 people at the school, and his aunt taught all of the
grades. The students would rotate around his aunt's desk for their
lessons, which was situated near a wood stove: In first grade, you sat
on a bench. By fifth grade, you stood behind the teacher's desk. It was
a childhood of disappointment and rejection, of suffocating loneliness.
``No one in my family, as far as I know, had ever traveled anywhere,''
he said. ``I don't know that my mother had ever been out of the state
until I was an adult.'' He wanted to be ``anywhere but there, at that
time.'' The only art he was exposed to growing up were paintings by his
dead grandmother in his grandfather's house, but, without encouragement
or any real reason, he found himself compelled to make art from a young
age. It became both a way of attracting his indifferent family's
attention to himself and, later, his primary means of escape.

\includegraphics{https://static01.nyt.com/images/2019/02/18/t-magazine/18tmag-johns-slide-HUSY/18tmag-johns-slide-HUSY-articleLarge.jpg?quality=75\&auto=webp\&disable=upscale}

After leaving the army in 1953, he arrived in New York to start his
career as an artist. The city had only recently usurped Paris as the
center of contemporary art with the rise of Abstract Expressionism, a
distinctly American painting style defined by spontaneous marks and
obsessive explorations of color, which by the beginning of the '50s had
become a world-famous aesthetic movement thanks to artists like
\href{https://www.nytimes.com/topic/person/jackson-pollock}{Jackson
Pollock}, \href{https://www.nytimes.com/topic/person/mark-rothko}{Mark
Rothko} and
\href{https://www.nytimes.com/1997/03/20/arts/willem-de-kooning-dies-at-92-reshaped-us-art.html}{Willem
de Kooning}. Their popularity was so great that, in helping to
mainstream a radically new style of painting, they inspired a rigid set
of parameters and rules for what painting should look like. In his essay
```American-Type' Painting'' (1955), the critic
\href{https://www.nytimes.com/1994/05/08/obituaries/clement-greenberg-dies-at-85-art-critic-championed-pollock.html}{Clement
Greenberg} praised Abstract Expressionism's ``new kind of flatness, one
that breathes and pulsates,'' one that represented a break with the
Cubist and Surrealist painting traditions that flourished in Europe
between the world wars.

By the mid-50s, when a 23-year-old Johns entered this insular milieu,
the Abstract Expressionist artists and their supporters were already
taking their artistic superiority for granted. He met
\href{https://www.nytimes.com/2008/05/14/arts/design/14rauschenberg.html}{Robert
Rauschenberg} through mutual friends around early 1954, while Johns was
working at Marboro Books, a discount chain that sold overstock titles.
The two would soon reorient the direction of contemporary art, but they
had humble beginnings. Shortly after meeting him, Rauschenberg enlisted
Johns's help in his own day job, designing window displays for the
\href{https://www.nytimes.com/2014/10/05/realestate/fifth-avenue-bonwit-teller-opulence-lost.html}{Bonwit
Teller} department store. The following year, the two moved their
studios into the same building together on Pearl Street, and through
Rauschenberg, Johns met
\href{https://www.nytimes.com/topic/person/merce-cunningham}{Merce
Cunningham} and his collaborator and romantic partner, the composer
\href{https://www.nytimes.com/topic/person/john-cage}{John Cage}.

Johns was the youngest of this group --- and the only one without an
extensive arts education. He'd studied art at the
\href{https://www.nytimes.com/search?query=University+of+South+Carolina}{University
of South Carolina} for three semesters and then spent less than a year
at the Parsons School of Design in New York before dropping out and
joining the army because he couldn't afford the tuition. Cage and
Cunningham both taught at
\href{https://www.nytimes.com/2015/12/18/arts/design/the-short-life-and-long-legacy-of-black-mountain-college.html}{Black
Mountain College}, the experimental art school in North Carolina where
Cunningham founded his dance company, and where Rauschenberg had also
studied, along with his soon-to-be-wife Susan Weil, under the tutelage
of the German painter
\href{https://www.nytimes.com/1976/03/26/archives/josef-albers-artist-and-teacher-dies.html}{Josef
Albers}. (Albers, a renowned instructor of the
\href{https://www.nytimes.com/2019/02/04/t-magazine/bauhaus-school-architecture-history.html}{Bauhaus}
in Germany who left the country after the rise of the Nazis in 1933,
taught ``learning by doing.'') These men's ideas ``were better formed
than mine,'' Johns said, ``and they were more experienced and strongly
motivated to do what they were doing. Cage and Rauschenberg and Merce,
those three people were the people most important to me at that time.
They had been to various kinds of schools, had traveled, had worked at
Black Mountain, which I think was important to them. And I benefited
from that. That reinforced a kind of forward movement.''

In March 1957, he and Rauschenberg were living in the building on Pearl
Street --- Rauschenberg upstairs, Johns downstairs --- when the dealer
\href{https://www.nytimes.com/1999/08/23/arts/leo-castelli-influential-art-dealer-dies-at-91.html}{Leo
Castelli} paid a visit to Rauschenberg. Castelli, an early champion of
Abstract Expressionism, had opened his own gallery on the Upper East
Side a month earlier and was courting Rauschenberg. During the visit,
Rauschenberg told the ambitious dealer that he had to go downstairs to
get some ice from Johns's studio (the two shared a refrigerator).
Castelli said he was curious about Johns --- he'd just seen a painting
he'd done of a green target that was included in a group show at the
\href{https://www.nytimes.com/topic/organization/jewish-museum-nyc}{Jewish
Museum}. ``So Bob came down to my studio,'' Johns told me, ``and said,
`Leo Castelli is upstairs and would like to meet you.''' Johns told
Castelli to call sometime and he'd show him his work. ``He said, `Can't
I see it now?' I thought it was inappropriate, since he'd come to see
Bob, but at any rate, we went downstairs.'' Castelli offered him a show
on the spot. ``That was the beginning,'' Johns said. He would show with
Castelli for the next 40 years, until Castelli's death in 1999.

Image

Johns's ``Untitled'' (2018).Credit...Jasper Johns, ``Untitled,'' 2018.
Oil on canvas © 2019 Jasper Johns/licensed by VAGA at ARS, NY. Courtesy
of Matthew Marks Gallery

Johns's debut solo exhibition at Castelli's gallery on the Upper East
Side 10 months later is now legendary. His work was a series of familiar
symbols --- American flags, targets, numbers --- painted on newsprint
with encaustic, in which pigment was mixed with heated beeswax. (This
was, at the time, a truly esoteric method, best known from the Fayum
mummy portraits from the first couple centuries A.D.) The show now has
the reputation of doing nothing less than announcing the death of
Abstract Expressionism. This is reductive, of course --- de Kooning and
others would continue to have long and successful careers after the
arrival of Johns --- but his work did dramatically reimagine the
possibilities of what could happen on a canvas. If Abstract
Expressionism was a melodramatically psychological exercise, with each
splash of paint communicating some anguished search for American
identity in the midst of the Cold War's atomic glow, here was something
cool and detached, familiar and yet forever unknowable. It was as if
\href{https://www.nytimes.com/2004/07/02/movies/marlon-brando-oscarwinning-actor-is-dead-at-80.html}{Marlon
Brando} had driven his motorcycle onto the set of a
\href{https://www.nytimes.com/1960/11/17/archives/clark-gable-dies-in-hollywood-of-heart-ailment-at-age-of-59-king-of.html}{Clark
Gable} movie.

At 27, Johns became an overnight success and an indecipherable oracle of
modern America. He was clearly saying something, but what? Was his
American flag a canny critique of Eisenhower-era imperialism? Was it an
ironic tribute to his namesake, someone who quite literally died for the
flag? Was it a tautological exercise, a heroic attempt to separate the
flag from its context? Johns has only ever given the same story: One
night, in 1954, he dreamed of painting a flag, and the next morning he
got up and started doing it. He spent much of the next five years
painting the flag in various forms, and then, for a while, mostly
stopped. When asked about this in a 1963 interview with a German
magazine, he said: ``They added two stars,'' a reference to Alaska and
Hawaii becoming states in 1959. ``Since then, the design does not
interest me anymore.''

IT'S TEMPTING TO think of Johns's career in two parts: with Rauschenberg
and post-Rauschenberg. Johns may not be forthcoming with details of his
personal life, but he is an expert self-mythologizer, and there is
likely no romance between two visual artists in postwar America that
occupies the same level of importance in the public's imagination as
Johns's with Rauschenberg. Between 1955 and 1961, the two lived and
worked in proximity to one another, first on Pearl Street and then, in
1958, when the Pearl Street building was condemned, over a hero sandwich
shop at 128 Front Street, this time with Johns upstairs and Rauschenberg
downstairs. They shared ideas, motifs and materials and ended up carving
a path for much of the art that has emerged in the 60 years since. Most
of the subsequent artistic milestones one can think of --- from
\href{https://www.nytimes.com/2018/11/01/arts/design/andy-warhol-inc-how-he-made-business-his-art.html}{Andy
Warhol}'s first Campbell's soup cans in 1962 to Tracey Emin putting
\href{https://www.tate.org.uk/art/artworks/emin-my-bed-l03662}{her own
bed} on view inside the Tate Gallery in 1999 to
\href{https://www.nytimes.com/2016/10/17/t-magazine/kerry-james-marshall-artist.html}{Kerry
James Marshall}'s use of collage, printing and other variables on the
canvases of his early paintings --- originated with the work Johns and
Rauschenberg produced during these years.

Rauschenberg's
``\href{https://www.rauschenbergfoundation.org/art/series/combine}{combines},''
a series of works the artist began making in 1954 that combined elements
of painting and sculpture (Johns described them as ``painting playing
the game of sculpture''), took contemporary art down from the wall and
into new territory entirely. Johns was often present in these works,
sometimes spiritually, often literally. One,
``\href{https://www.artic.edu/artworks/209926/short-circuit}{Short
Circuit}'' (1955), included a flag painting by Johns inside a
compartment with a small door. Another combine, ``Untitled'' (circa
1954), sometimes referred to as ``Man With White Shoes,'' was a wooden
structure that resembled a front porch, partially painted and mounted
with objects ranging from a taxidermied hen to personal ephemera: a
letter from Rauschenberg's son, photographs of his parents, a picture of
Johns. In 1958, Johns expanded on the strange familiarity found in his
paintings of flags and numbers by making sculptures of everyday objects
(light bulbs, beer cans) that he cast in bronze, prefiguring Warhol's
fascination with mass production and pop culture symbolism by several
years. His paintings, like Rauschenberg's, began to incorporate
sculptural elements, as with
``\href{https://www.moma.org/collection/works/78393}{Target With Four
Faces}'' (1955), which included the familiar target but with four
plaster casts of an eerily expressionless face, cut off just below the
eyes, peering out from beneath a wooden slat affixed to the top of the
painting.

Image

Another photograph of Johns's studio. Every detail inside the studio
seems intentional, as if each object were a clue about the man
himself.Credit...Joel Sternfeld

As they worked on redefining the avant-garde, the two made money by
creating window displays for department stores under the name Matson
Jones (Matson was Rauschenberg's mother's maiden name; Jones was a near
homonym for Johns). There is a certain art historical narrative that
claims Johns and Rauschenberg were the founders of Pop Art, engaged in
some ideological battle of wits with Warhol. Johns downplayed this ---
``I don't think I saw myself in relation to Andy,'' he said, though he
added that Warhol did give him a silk-screen with which to practice ---
but he and Rauschenberg were at least competitive with him in terms of
their commercial work. For Johns and Rauschenberg, the window displays
were merely a financial tool, but Warhol, who was best-known at the time
for his commercial illustration, did so under his own name and had
become quite successful. He'd bought a drawing of Johns's from a group
show, and when the two met, around 1958, Johns told Warhol that he knew
his work. ``He seemed very pleased for a moment,'' Johns said, ``until
it became clear that I was referring to his commercial work. He did
drawings for I. Miller,'' the shoe manufacturer with a shop in Times
Square, ``and I explained that Bob and I had been commissioned to do a
window display for I. Miller based on Andy's drawings. And he said'' ---
here Johns pitched his voice into a kind of sheepish whine --- ```Why
didn't they ask \emph{me} to do it?'''

Then, in 1961, Johns and Rauschenberg's relationship ended. They crossed
paths from time to time after that, but they never had any planned
encounters. (Rauschenberg died in 2008, at 82.) Little is known about
what caused this breakup --- Johns told me that they simply drifted
apart. The fact that so few details of this partnership have emerged, in
no small part because Johns and Rauschenberg usually declined to talk
about it, means that it has taken on a kind of mythic status.
Rauschenberg frequently conscripted his romantic partners in his work,
including his wife, Susan Weil, and
\href{https://www.nytimes.com/2011/07/06/arts/cy-twombly-american-artist-is-dead-at-83.html}{Cy
Twombly}, with whom Rauschenberg
\href{https://www.nytimes.com/2018/10/11/t-magazine/cy-twombly-robert-rauschenberg-art-travel.html}{had
an affair} that contributed to the end of his short-lived marriage with
Weil. For his part, Johns's relationship with Rauschenberg has not
necessarily been confined to the closet. Writers of the '60s and '70s
called them ``close friends,'' though this has shifted in more recent
years to ``sometime lovers.'' The details that do exist mostly come from
the work they made during this period and directly following their
estrangement. Critics have long speculated, for instance, that Johns's
dour gray painting ``Liar,'' from 1961, which features the titular word
in prominent capital letters at the top of the paper, was about
Rauschenberg. Regardless of any details, sexual or otherwise, about
their union, what emerges most powerfully from the work they produced
while together is evidence of an intertwined existence, an intimacy that
was in a way closer than marriage. ``Jasper and I literally traded
ideas,'' Rauschenberg told
\href{https://www.nytimes.com/2011/10/05/arts/calvin-tomkins-continues-to-chronicle-artists.html}{Calvin
Tomkins} in his 1980 biography
``\href{https://us.macmillan.com/books/9780312425852}{Off the Wall.}''

After they parted ways, each remained a celebrated artist, but their
respective roles shifted. Rauschenberg became more overtly political and
was embraced by an American counterculture that used his work to
critique a variety of causes, from the
\href{https://www.nytimes.com/topic/subject/vietnam-war}{Vietnam War} to
climate change. Later, in the '80s, he would start a project called the
Rauschenberg Overseas Culture Interchange, in which he traveled the
world, attempting to use his art as the impetus for cross-cultural
dialogues that he hoped would lead to world peace. He'd fund this
program in part by selling works from his collection, including at least
one Johns painting from when the two were living together.

Johns retreated increasingly inward. He'd never discuss topical
concerns, or be so bold as to think his work might solve any of the
world's problems. He largely shunned public adoration. (One exception
came in 1999, when Johns played a kleptomaniac version of himself in an
episode of ``The Simpsons,'' stealing light bulbs and appetizers from
the kind of art opening he has generally avoided in his actual life.)
But in speaking about Rauschenberg, Johns was, if not quite animated, at
least open. Asked if he ever grows tired of being asked about
Rauschenberg, he replied, ``Not today.'' When I suggested that
Rauschenberg and Johns were each other's main audience, Johns replied
without hesitation, ``That's true.'' He said the two would show each
other their work constantly, and once their friendship ended, Johns was
never able to find another kindred spirit on this level. ``The
relationship with Bob was extremely important to me,'' he said, ``as an
artist and as a human being. So to end contact with an honest opinion
that you are willing to accept --- to have it or not to have it is a
huge difference.''

Image

The sculpture ``Painted Bronze'' (1960), made after Willem de Kooning
remarked that an artist could give the gallerist Leo Castelli two beer
cans and Castelli could sell them.Credit...Jasper Johns, ``Painted
Bronze,'' 1960 (cast and painted 1964), bronze and oil paint (3 parts),
edition of 2 (2/2), collection of the artist © 2019 Jasper
Johns/licensed by VAGA at ARS, NY, photo by Jamie Stukenberg,
Professional Graphics, Rockford, Ill. © 2019 the Wildenstein Plattner
Institute, New York

NOT LONG AFTER his split from Rauschenberg, Johns found himself once
again in South Carolina, for a show at the Columbia Museum of Art, which
turned into a kind of reunion with the people from his childhood. As he
recounted in a 1966 television documentary, someone at that gathering
told him about a house for sale in remote Edisto Beach, south of
\href{https://www.nytimes.com/interactive/2017/01/12/travel/what-to-do-36-hours-in-charleston-south-carolina.html}{Charleston}.
He went to see the place and didn't particularly like it --- the
interior was painted pink --- but, perusing the cabinets, he found a
half-empty bottle of bourbon beneath the kitchen sink, which he took as
a good omen. Johns began taking breaks from New York --- first in
Edisto, then at his houses in the Caribbean and in Stony Point, N.Y.,
where John Cage was his neighbor --- until he finally moved to Sharon in
1998.

Since the mid-60s, Johns has been working with the same strange codex of
symbols that seem to comprise a language that only Johns himself is
fully fluent in. ``There are not that many artists in the 20th and 21st
century who are quite so consistently returning to and reworking what is
essentially a small lexicon of images,'' the Whitney's Scott Rothkopf
said. His work following his first decade as an artist has been defined
by a continual resuscitation of these images, like ghosts that he can't
quite elude. There is the empty coffee can, crammed with paintbrushes,
something like Johns's primal scene, which he has sculpted and cast in
bronze, made into a lithograph and painted at various points in his
career. There are his almost absurdly simple crosshatched marks, which
began appearing in his work in 1972, and which he claimed offered ``the
possibility of a complete lack of meaning.'' There are stick figures
holding paintbrushes. There are skulls. There is his remarkable use of
the color gray to blanket a painting in sorrow. Though he returns to the
same motifs again and again, he complicates their meaning continuously
with the addition of some new, often acutely personal element --- a
portrait of Castelli (as in ``Racing Thoughts,'' from 1983); the
blueprints for his grandfather's house (as in ``Mirror's Edge,'' from
1992); the artist's own signature, sourced from a rubber stamp he had
made that said ``Regrets, Jasper Johns,'' an object that made it all the
easier to turn down whatever someone wanted him to do. Indeed, there are
two Jasper Johns --- one who is unable to keep the past at bay, whose
memories seem to trickle into his paintings like water from a broken
faucet; and one who does things by rote, with a kind of cool
indifference --- and both are present in his work.

These personas have been at odds in his life as well, of course. There
is the Jasper Johns who wrote a letter in 1959 responding to a review by
the critic
\href{https://www.nytimes.com/2012/03/28/arts/design/hilton-kramer-critic-who-championed-modernism-dies-at-84.html}{Hilton
Kramer} that he didn't like (``thank God, art tends to be less what
critics write than what artists make,'' he wrote) and the Jasper Johns
who claims to be surprised anyone has an opinion about his work at all,
as he said to me and nearly every other interviewer he's spoken with.
There is the Jasper Johns who spoke in the 1970s about wanting to sell a
painting for \$1 million, a figure then unheard-of for a living artist
(in 2014, a flag painting
\href{https://www.nytimes.com/2014/11/12/arts/design/rothko-and-johns-paintings-are-stars-of-a-sluggish-auction-for-sothebys.html}{sold
for \$36 million} at a Sotheby's auction) and the Jasper Johns who
avoids any knowledge of the contemporary art world like a conscientious
objector. For someone who claims to have no opinion about art fairs or
auctions or any of the other global machinations of the market, he also
recounted, admiringly, a story he'd heard about Jackson Pollock walking
by a Mercedes-Benz showroom, pointing at a car and saying, ``One
painting.'' Johns has only attended two art fairs, the Foire
Internationale d'Art Contemporain, in Paris, and Art Basel in
Switzerland, in the '70s. A bookseller at Basel was selling a book he
made with
\href{https://www.nytimes.com/topic/person/samuel-beckett}{Samuel
Beckett} in 1976, called
``\href{https://www.manhattanrarebooks.com/pages/books/195/jasper-johns-samuel-beckett/foirades-fizzles/?soldItem=true}{Foirades/Fizzles}.''
``I looked at that,'' he said, ``and then I just sort of walked through
the place. It was all foreign and unpleasant to me.''

Though I had expected Johns to be guarded about his past or his personal
life, it was a surprise to find him cagiest about his artistic process.
Trying to ask him about his personal approach to art making felt like
pushing a boulder up a hill and watching it tumble back down once I'd
reached the top.

Does he have a routine when he goes into the studio?

``Just going into it is the routine, I guess,'' he said.

How many works does he produce in a given year?

``I have absolutely no idea how to answer that.''

Has the way he works changed over time?

``Probably,'' he said, though after a momentary pause, he continued:
``It's hard to say how. I think I produce fewer works now. I assume that
has to do with age, but I don't know. And I don't know what other people
think of my production. I've occasionally worked on something for a very
long time, so it doesn't seem like I'm producing anything, but
eventually something will get done.''

He went on, telling me that he just finished a painting, which would be
included in his exhibition at Matthew Marks, that he spent three or four
years working on. He wouldn't say which work.

How does he know when something he's spent that amount of time on is
finally done?

``I think you just give up,'' he said.

Image

\textbf{On the Cover:} Jasper Johns is featured in T's March 3 Men's
Fashion issue.Credit...Joel Sternfeld

WHEN WE WERE finished talking, I asked if Johns would show me some of
his new work. ``I don't think we should,'' he said with a sigh. ``But we
can. We'll have to make a trip.'' We went outside and got into a Polaris
Ranger UTV --- a kind of off-road golf cart. (``It's necessary, I'm
afraid,'' Johns said; his property is large, and he moves slowly these
days.) He drove us to another structure on the property called the Blue
Barn, where he'd hung his recently completed works. A
\href{https://www.nytimes.com/2018/10/15/t-magazine/bruce-nauman-art-interview.html}{Bruce
Nauman} sculpture, ``Jasper's Cat'' (1990) --- a 60th birthday gift from
Castelli --- which looked like an enormous scale balancing two
animalistic globs of flesh, was just inside the barn's entrance. Johns
had seen \href{https://www.moma.org/calendar/exhibitions/3852}{Nauman's
show} at the Museum of Modern Art that had opened a few months before,
during a rare trip into the city. It wasn't an enjoyable experience. The
show was loud, the museum was loud and there were people everywhere.

Deeper into the barn, there were about a dozen works hanging on the
walls, made between 2012 and 2018. The three most recent works, a
charcoal drawing and two paintings, all had the same figure at the
center, a shadowy silhouette of a man wearing a tiny fedora, his
skeleton visible like an X-ray, the body slightly angled and holding
what appeared to be a thin cane, as if it were about to break out into a
song and dance. There were familiar symbols competing for space on one
of the paintings: crosshatches, stick figures holding paintbrushes, a
presumably random sheet of newspaper. I kept returning to the skull with
its goofy hat, the closed mouth stretched into a sinister smile. I
assumed this was a kind of self-portrait: the artist as he nears 90,
trying in vain to protect himself as his work is once more dragged in
front of his public under threat of the same cyclical routine he's been
performing for 60 years --- being asked what it all means, keeping his
mouth shut, surviving as everything around him both changes and stays
exactly the same.

But I said none of this, and we both stood there for a while in silence.
The painting, once again, spoke for itself.

Advertisement

\protect\hyperlink{after-bottom}{Continue reading the main story}

\hypertarget{site-index}{%
\subsection{Site Index}\label{site-index}}

\hypertarget{site-information-navigation}{%
\subsection{Site Information
Navigation}\label{site-information-navigation}}

\begin{itemize}
\tightlist
\item
  \href{https://help.nytimes.com/hc/en-us/articles/115014792127-Copyright-notice}{©~2020~The
  New York Times Company}
\end{itemize}

\begin{itemize}
\tightlist
\item
  \href{https://www.nytco.com/}{NYTCo}
\item
  \href{https://help.nytimes.com/hc/en-us/articles/115015385887-Contact-Us}{Contact
  Us}
\item
  \href{https://www.nytco.com/careers/}{Work with us}
\item
  \href{https://nytmediakit.com/}{Advertise}
\item
  \href{http://www.tbrandstudio.com/}{T Brand Studio}
\item
  \href{https://www.nytimes.com/privacy/cookie-policy\#how-do-i-manage-trackers}{Your
  Ad Choices}
\item
  \href{https://www.nytimes.com/privacy}{Privacy}
\item
  \href{https://help.nytimes.com/hc/en-us/articles/115014893428-Terms-of-service}{Terms
  of Service}
\item
  \href{https://help.nytimes.com/hc/en-us/articles/115014893968-Terms-of-sale}{Terms
  of Sale}
\item
  \href{https://spiderbites.nytimes.com}{Site Map}
\item
  \href{https://help.nytimes.com/hc/en-us}{Help}
\item
  \href{https://www.nytimes.com/subscription?campaignId=37WXW}{Subscriptions}
\end{itemize}
