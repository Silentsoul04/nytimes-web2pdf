Sections

SEARCH

\protect\hyperlink{site-content}{Skip to
content}\protect\hyperlink{site-index}{Skip to site index}

\href{https://www.nytimes.com/section/technology/personaltech}{Personal
Tech}

\href{https://myaccount.nytimes.com/auth/login?response_type=cookie\&client_id=vi}{}

\href{https://www.nytimes.com/section/todayspaper}{Today's Paper}

\href{/section/technology/personaltech}{Personal Tech}\textbar{}Facebook
Did Not Securely Store Passwords. Here's What You Need to Know.

\url{https://nyti.ms/2YdyAKj}

\begin{itemize}
\item
\item
\item
\item
\item
\end{itemize}

Advertisement

\protect\hyperlink{after-top}{Continue reading the main story}

Supported by

\protect\hyperlink{after-sponsor}{Continue reading the main story}

\href{/column/tech-fix}{Tech Fix}

\hypertarget{facebook-did-not-securely-store-passwords-heres-what-you-need-to-know}{%
\section{Facebook Did Not Securely Store Passwords. Here's What You Need
to
Know.}\label{facebook-did-not-securely-store-passwords-heres-what-you-need-to-know}}

\includegraphics{https://static01.nyt.com/images/2019/03/21/business/21fbpasswords/merlin_138480210_19941a79-fdbb-47aa-96ac-5e69847165e1-articleLarge.jpg?quality=75\&auto=webp\&disable=upscale}

By \href{https://www.nytimes.com/by/brian-x-chen}{Brian X. Chen}

\begin{itemize}
\item
  March 21, 2019
\item
  \begin{itemize}
  \item
  \item
  \item
  \item
  \item
  \end{itemize}
\end{itemize}

SAN FRANCISCO --- Facebook said on Thursday that millions of user
account passwords had been stored insecurely, potentially allowing
employees to gain access to people's accounts without their knowledge.

The Silicon Valley company
\href{https://newsroom.fb.com/news/2019/03/keeping-passwords-secure/}{publicized
the security failure} around the same time that
\href{https://krebsonsecurity.com/2019/03/facebook-stored-hundreds-of-millions-of-user-passwords-in-plain-text-for-years/}{Brian
Krebs,} a cybersecurity writer, reported the password vulnerability. Mr.
Krebs said an audit by Facebook had found that hundreds of millions of
user passwords dating to 2012 were stored in a format known as plain
text, which makes the passwords readable to more than 20,000 of the
company's employees.

Facebook said that it had found no evidence of abuse and that it would
begin alerting millions of its users and thousands of Instagram users
about the issue. The company said it would not require people to reset
their passwords.

The security failure is another embarrassment for Facebook, a \$470
billion colossus that employs some of the most sought-after
cybersecurity experts in the industry. It adds to a growing list of data
scandals that have tarnished Facebook's reputation over the last few
years. Last year, amid revelations that a political consulting firm
\href{https://www.nytimes.com/2018/03/17/us/politics/cambridge-analytica-trump-campaign.html}{improperly
gained access to the data of millions}, Facebook also revealed that an
attack on its network had exposed the personal information of
\href{https://www.nytimes.com/2018/09/28/technology/facebook-hack-data-breach.html?action=click\&module=RelatedCoverage\&pgtype=Article\&region=Footer}{tens
of millions of users}.

In response, the company has repeatedly said it plans to improve how it
safeguards people's data.

``There is nothing more important to us than protecting people's
information, and we will continue making improvements as part of our
ongoing security efforts at Facebook,'' Pedro Canahuati, Facebook's vice
president of engineering in security and privacy, said in a blog post on
Thursday.

Here's a rundown of what you need to know about the password
vulnerability and what you can do.

\hypertarget{whats-the-problem}{%
\subsection{What's the problem?}\label{whats-the-problem}}

Storing passwords in plain text is a poor security practice. It leaves
passwords wide open to cyberattacks or potential employee abuse. A
better security practice would have been to keep the passwords in a
scrambled format that is indecipherable.

Facebook said it had not found evidence of abuse, but that does not mean
it did not occur. Citing a Facebook insider, Mr. Krebs said access
records revealed that 2,000 engineers or developers had made nine
million queries for data that included plain-text user passwords.

A Facebook employee could have shared your password with someone else
who would then have improper access to your account, for instance. Or an
employee could have read your password and used it to log on to a
different site where you used the same password. There are plenty of
possibilities.

Ultimately, a company as large, rich and well staffed as Facebook should
have known better.

\hypertarget{how-do-i-know-whether-someone-had-access-my-account}{%
\subsection{How do I know whether someone had access my
account?}\label{how-do-i-know-whether-someone-had-access-my-account}}

There's no easy way to know. Facebook is still investigating, and will
begin alerting people who might have had their passwords stored in the
plain text format.

\hypertarget{what-should-i-do}{%
\subsection{What should I do?}\label{what-should-i-do}}

Facebook is not requiring users to change their passwords, but you
should do it anyway.

There are many methods for setting
\href{https://www.nytimes.com/interactive/2017/technology/how-to-protect-data-online.html}{strong
passwords} --- for example, do not use the same password across multiple
sites, and do not use your Social Security number as a username or a
password. You can set up security features such as two-step verification
as well.

There are a few other steps to take. I recommend also setting up your
Facebook account to
\href{https://www.facebook.com/about/basics/stay-safe-and-secure/login-alerts\#1}{receive
alerts} in the event that an unrecognized device logs in to the account.
To do so, go to your Facebook app settings, tap Security and Login, and
then tap Get alerts about unrecognized logins. From here, you can choose
to receive the alerts via messages, email or notifications.

An audit of devices that are logged in to your account may also be in
order, so that you know what laptops, phones and other gadgets are
already accessing your account. On Facebook's
\href{https://www.facebook.com/settings?tab=security}{Security and Login
page}, under the tab labeled ``Where You're Logged In,'' you can see a
list of devices that are signed in to your account, as well as their
locations.

If you see an unfamiliar gadget or a device signed in from an odd
location, you can click the ``Remove'' button to boot the device out of
your account.

Advertisement

\protect\hyperlink{after-bottom}{Continue reading the main story}

\hypertarget{site-index}{%
\subsection{Site Index}\label{site-index}}

\hypertarget{site-information-navigation}{%
\subsection{Site Information
Navigation}\label{site-information-navigation}}

\begin{itemize}
\tightlist
\item
  \href{https://help.nytimes.com/hc/en-us/articles/115014792127-Copyright-notice}{©~2020~The
  New York Times Company}
\end{itemize}

\begin{itemize}
\tightlist
\item
  \href{https://www.nytco.com/}{NYTCo}
\item
  \href{https://help.nytimes.com/hc/en-us/articles/115015385887-Contact-Us}{Contact
  Us}
\item
  \href{https://www.nytco.com/careers/}{Work with us}
\item
  \href{https://nytmediakit.com/}{Advertise}
\item
  \href{http://www.tbrandstudio.com/}{T Brand Studio}
\item
  \href{https://www.nytimes.com/privacy/cookie-policy\#how-do-i-manage-trackers}{Your
  Ad Choices}
\item
  \href{https://www.nytimes.com/privacy}{Privacy}
\item
  \href{https://help.nytimes.com/hc/en-us/articles/115014893428-Terms-of-service}{Terms
  of Service}
\item
  \href{https://help.nytimes.com/hc/en-us/articles/115014893968-Terms-of-sale}{Terms
  of Sale}
\item
  \href{https://spiderbites.nytimes.com}{Site Map}
\item
  \href{https://help.nytimes.com/hc/en-us}{Help}
\item
  \href{https://www.nytimes.com/subscription?campaignId=37WXW}{Subscriptions}
\end{itemize}
