Sections

SEARCH

\protect\hyperlink{site-content}{Skip to
content}\protect\hyperlink{site-index}{Skip to site index}

\href{https://www.nytimes.com/section/politics}{Politics}

\href{https://myaccount.nytimes.com/auth/login?response_type=cookie\&client_id=vi}{}

\href{https://www.nytimes.com/section/todayspaper}{Today's Paper}

\href{/section/politics}{Politics}\textbar{}Mueller Finds No
Trump-Russia Conspiracy, but Stops Short of Exonerating President on
Obstruction

\url{https://nyti.ms/2HReyQg}

\begin{itemize}
\item
\item
\item
\item
\item
\item
\end{itemize}

Advertisement

\protect\hyperlink{after-top}{Continue reading the main story}

Supported by

\protect\hyperlink{after-sponsor}{Continue reading the main story}

\hypertarget{mueller-finds-no-trump-russia-conspiracy-but-stops-short-of-exonerating-president-on-obstruction}{%
\section{Mueller Finds No Trump-Russia Conspiracy, but Stops Short of
Exonerating President on
Obstruction}\label{mueller-finds-no-trump-russia-conspiracy-but-stops-short-of-exonerating-president-on-obstruction}}

\includegraphics{https://static01.nyt.com/images/2019/03/26/us/politics/26dc-cong-promo/24dc-cong-promo-articleLarge-v2.jpg?quality=75\&auto=webp\&disable=upscale}

By \href{https://www.nytimes.com/by/mark-mazzetti}{Mark Mazzetti} and
\href{https://www.nytimes.com/by/katie-benner}{Katie Benner}

\begin{itemize}
\item
  March 24, 2019
\item
  \begin{itemize}
  \item
  \item
  \item
  \item
  \item
  \item
  \end{itemize}
\end{itemize}

WASHINGTON --- The investigation led by Robert S. Mueller III found no
evidence that President Trump or any of his aides coordinated with the
Russian government's 2016 election interference, according to a summary
of
\href{https://www.nytimes.com/interactive/2019/03/24/us/politics/barr-letter-mueller-report.html}{the
special counsel's key findings} made public on Sunday by Attorney
General William P. Barr.

Mr. Mueller, who spent nearly two years investigating Moscow's
determined effort to sabotage the last presidential election, found no
conspiracy ``despite multiple offers from Russian-affiliated individuals
to assist the Trump campaign,'' Mr. Barr wrote in
\href{https://www.nytimes.com/interactive/2019/03/24/us/politics/barr-letter-mueller-report.html}{a
letter to lawmakers}.

Mr. Mueller's team drew no conclusions about whether Mr. Trump illegally
obstructed justice, Mr. Barr said, so he made his own decision. The
attorney general and his deputy, Rod J. Rosenstein, determined that the
special counsel's investigators had insufficient evidence to establish
that the president committed that offense.

He cautioned, however, that Mr. Mueller's report states that ``while
this report does not conclude that the president committed a crime, it
also does not exonerate him'' on the obstruction of justice issue.

Still, the release of the findings was a significant political victory
for Mr. Trump and lifted a cloud that has hung over his presidency since
before he took the oath of office. It is also likely to alter discussion
in Congress about the fate of the Trump presidency; some Democrats had
pledged to wait until the special counsel finished his work before
deciding whether to initiate impeachment proceedings.

\href{https://www.nytimes.com/interactive/2019/03/24/us/politics/barr-letter-mueller-report.html}{}

\includegraphics{https://static01.nyt.com/images/2019/03/24/us/politics/barr-letter-mueller-report-1553456906394/barr-letter-mueller-report-1553456906394-articleLarge-v3.jpg}

\hypertarget{read-attorney-general-william-barrs-summary-of-the-mueller-report}{%
\subsection{Read Attorney General William Barr's Summary of the Mueller
Report}\label{read-attorney-general-william-barrs-summary-of-the-mueller-report}}

The letter, by Attorney General William P. Barr, details the main
findings of the special counsel's two-year investigation into Russian
interference in the 2016 presidential election.

\emph{{[}Read}
\href{https://www.nytimes.com/interactive/2019/03/24/us/politics/barr-letter-mueller-report.html}{\emph{the
key Mueller findings}}\emph{.{]}}

The president trumpeted the news almost immediately, even as he
mischaracterized the special counsel's findings. ``It was a complete and
total exoneration,'' Mr. Trump told reporters in Florida before boarding
Air Force One. ``It's a shame that our country had to go through this.
To be honest, it's a shame that your president has had to go through
this.''

He added, ``This was an illegal takedown that failed.''

Mr. Barr's letter was the culmination of a tense two days since Mr.
Mueller delivered his report to the Justice Department. Mr. Barr spent
the weekend poring over the special counsel's work, as Mr. Trump
strategized with lawyers and political aides at his Mar-a-Lago estate in
Florida.

Mr. Mueller, who has been a spectral presence in the capital for nearly
two years --- so often discussed, but so rarely seen --- was
photographed leaving a church on Sunday morning just across Lafayette
Square from the White House.

\includegraphics{https://static01.nyt.com/images/2017/01/29/podcasts/the-daily-album-art/the-daily-album-art-articleInline-v2.jpg?quality=75\&auto=webp\&disable=upscale}

\hypertarget{listen-to-the-daily-coordination-not-established-obstruction-more-complicated}{%
\subsubsection{Listen to `The Daily': Coordination: Not Established.
Obstruction: More
Complicated.}\label{listen-to-the-daily-coordination-not-established-obstruction-more-complicated}}

What does the Mueller report say? The attorney general offered an early
glimpse.

transcript

Back to The Daily

bars

0:00/28:55

-28:55

transcript

\hypertarget{listen-to-the-daily-coordination-not-established-obstruction-more-complicated-1}{%
\subsection{Listen to `The Daily': Coordination: Not Established.
Obstruction: More
Complicated.}\label{listen-to-the-daily-coordination-not-established-obstruction-more-complicated-1}}

\hypertarget{hosted-by-michael-barbaro-produced-by-theo-balcomb-and-jessica-cheung-and-edited-by-lisa-tobin}{%
\subsubsection{Hosted by Michael Barbaro, produced by Theo Balcomb and
Jessica Cheung, and edited by Lisa
Tobin}\label{hosted-by-michael-barbaro-produced-by-theo-balcomb-and-jessica-cheung-and-edited-by-lisa-tobin}}

\hypertarget{what-does-the-mueller-report-say-the-attorney-general-offered-an-early-glimpse}{%
\paragraph{What does the Mueller report say? The attorney general
offered an early
glimpse.}\label{what-does-the-mueller-report-say-the-attorney-general-offered-an-early-glimpse}}

\begin{itemize}
\item
  michael schmidt\\
  Hello? Hello? Hello?
\item
  maggie haberman\\
  Hello.
\item
  michael schmidt\\
  Maggie?
\item
  maggie haberman\\
  Hi.
\item
  michael schmidt\\
  Can you hear me?
\item
  maggie haberman\\
  Yes, can you hear me?
\item
  michael schmidt\\
  Yeah, hold on, hold on, hold on. Wearing two headphones right now.
\item
  maggie haberman\\
  Mm. Sounds about right.
\item
  michael schmidt\\
  Can you hear me? Oh, that's better.
\item
  maggie haberman\\
  We gotcha.

  We're just waiting on the star.
\item
  michael barbaro\\
  Hi.
\item
  maggie haberman\\
  It's Michael Barbaro, host of ``The Daily.''
\item
  michael schmidt\\
  Doo doo doo doo doo doo doo doo doo doo doo doo doo doo doo.
\item
  michael barbaro\\
  From The New York Times, I'm Michael Barbaro. This is ``The Daily.''
  Today --- Attorney General William Barr has sent a letter to Congress
  summarizing the findings of the special counsel investigation: No
  coordination with Russia. More complicated on obstruction of justice.
  My colleagues Maggie Haberman and Mike Schmidt explain.

  It's Monday, March 25.

  O.K. Maggie, like, it is 6:00 p.m. on Sunday evening. Two hours ago,
  the attorney general, Bill Barr, sent a letter to congressional
  leaders outlining the major conclusions of the special counsel's
  report.
\item
  michael schmidt\\
  The report is broken into two parts. The first one is on the biggest,
  most central question that has surrounded the president since he was
  elected: What are the ties between his campaign and Russia? And on
  that issue, Barr is unequivocal.
\item
  michael barbaro\\
  What does he say?
\item
  michael schmidt\\
  He says there are no ties between what Russia did in the election and
  Trump's campaign. That despite how aggressively the Russians tried to
  interfere in our election, and even outreach that they made to the
  campaign, they were able to find no evidence that they actually worked
  together.
\item
  michael barbaro\\
  But Mike, help me understand this. In the lead-up to the report, the
  Mueller investigation issued a number of subpoenas and charges that
  felt like they revealed forms of collusion. I'm thinking about
  Papadopoulos, Manafort, a meeting inside Trump Tower that involved the
  president's son and his son-in-law. Those all felt like forms of
  collusion.
\item
  michael schmidt\\
  I think that what went on was that the Trump campaign was sort of
  collusion-curious in the sense that they were open to sort of talking
  to anyone about anything. And the Russians were reaching out right.
  And that meant that there were a lot of odd contacts that occurred.
  What I think Mueller and Barr are saying is that even though a lot of
  that stuff looks funky, we have no evidence that they linked hands and
  tried to work together to hurt Clinton.
\item
  michael barbaro\\
  So collusion-curious does not equal coordination.
\item
  maggie haberman\\
  No. Being interested in getting information is something that you have
  heard either the president say or people around the president say. You
  know, of course, we took these meetings. Of course when people reached
  out to us, we listened to them. Who wouldn't? Their argument has been
  that they were just doing what anyone would do. And the other argument
  they've made repeatedly is that they were too discombobulated and too
  green and too new at this to even know how to collude. And as someone
  who covered that campaign, I can tell you there is some logic to that
  argument.
\item
  michael barbaro\\
  As somebody who's slightly more removed from the day to day of this
  investigation than either of you, it's starting to feel from this
  summary from Bill Barr of the Mueller report that what the
  investigation found was what we all kind of knew the investigation
  found as it was finding. That there was nothing all that big being
  held back. And that helps us understand this conclusion.
\item
  maggie haberman\\
  What we saw in indictments either of Paul Manafort, or of his deputy
  Rick Gates, who pleaded guilty, or of Mike Flynn or George
  Papadopoulos, this might have been, it seems, all there was. This was
  all the information that they were able to prove existed.
\item
  michael barbaro\\
  Mike, what do you think of that?
\item
  michael schmidt\\
  Let's go back and look at the indictments and guilty pleas. George
  Papadopoulos, a campaign official who had contacts with individuals
  who said the Russians were going to be releasing information about
  Hillary Clinton's emails before it came out. When George Papadopoulos
  pleads guilty, he does not plead guilty to conspiring with the
  Russians. He pleads guilty to making false statements to investigators
  about that contact. The president's former national security adviser,
  Michael Flynn, has these odd phone calls with the Russian ambassador
  during the transition in which they discuss lifting sanctions that
  have just been imposed by the Obama administration for election
  meddling. When Mike Flynn goes into court to plead guilty, he doesn't
  plead guilty to conspiring with the Russians. He pleads guilty to
  making false statements to the F.B.I. about it. So as we went along in
  the investigation over the past 22 months, we never saw charges that
  said that the campaign conspired with Russia. There was some question
  --- was Mueller holding out something like that until the end? Was he
  going to wait and then make a move on it? Well, today, we know that
  they never found anything.
\item
  michael barbaro\\
  Right. He wasn't making those kinds of indictments because he didn't
  have any evidence of it.
\item
  maggie haberman\\
  Or at least he didn't have enough to bring an indictment. I mean, he
  might have found strains of information, he might have found pieces
  along the way, but it clearly was not enough to bring a charge.
\item
  michael barbaro\\
  So in the end, this was a bunch of people around the president lying,
  perhaps to avoid what might look like coordination if they were going
  to tell the truth, but not actually coordinating, according to the
  legal definition.
\item
  michael schmidt\\
  Correct.
\item
  michael barbaro\\
  And what about the second question in this letter, the question of
  obstruction of justice? What does Barr's summary of the Mueller report
  say about that?
\item
  michael schmidt\\
  So the obstruction part is not nearly as clean-cut as the collusion
  section. On obstruction, Barr essentially says that Mueller did not
  come to his own determination on whether the president obstructed
  justice. That Mueller could not indict or exonerate the president for
  that charge. But that left the door open for Barr to say, look,
  Mueller has not made a determination on this. I, as the attorney
  general, along with the deputy attorney general, Rod Rosenstein, will
  make that call. And we do not believe there is a case to be made the
  president obstructed justice.
\item
  michael barbaro\\
  I don't quite understand what's going on here. Why do we think, Mike,
  that Mueller declined to weigh in on this question of obstruction of
  justice? Why leave it kind of up in the air?
\item
  maggie haberman\\
  There's not really a full explanation for that. And I'm not sure. For
  the past 22 months, Democrats and a lot of folks in the media have
  built Mueller up as sort of this paragon of justice, and someone who
  was willing and able to go out and make determinations and calls on
  really tough issues in non-partisan, follow-the-facts ways. And here,
  we have Mueller essentially saying, I don't really have a
  determination on it. For each of the relevant actions, the letter
  says, that were investigated, the report sets out evidence on both
  sides of the question and leaves unresolved what the special counsel
  views as, quote, ``difficult issues of law and fact concerning whether
  the president's actions and intent could be viewed as obstruction.''
  The special counsel states that, quote, ``while this report does not
  conclude the president committed a crime, it also does not exonerate
  him.''
\item
  michael barbaro\\
  Let's pick apart what I think we can all agree is the most salient
  example of this, where maybe disentangling motive and law and
  obstruction of justice is really tangled up. And that would be the
  firing of James Comey. The president at first says it's not about
  Russia, then goes on television, says, it really is about Russia. And
  yet we all know that the president can pretty much fire anyone at that
  level if he wants to. Is that what Bill Barr, in summarizing Mueller,
  is saying? That this stuff is just really difficult, and so we're not
  making a call?
\item
  maggie haberman\\
  Mueller is saying this is really difficult, and it's hard to get in
  the president's head and know what his intent was and his frame of
  mind was. And he decides that he is not going to be the one to make
  that call. Bill Barr, not wanting, I think, to leave this as an
  open-ended thing puts a pin in it and says, this is done.
\item
  michael barbaro\\
  And says, I will make that call.
\item
  maggie haberman\\
  I will make that call. And we do not believe that this is sufficient
  to say that the president obstructed justice. And remember,
  obstruction of justice was key to Richard Nixon. This was a piece that
  the president's folks were worried about. They had felt really
  confident the whole time that there would not be an evidence-based
  case of conspiracy related to Russia. But they were concerned about
  what it would find on the obstruction piece.
\item
  michael schmidt\\
  Interestingly --- and I still don't think this issue has gotten enough
  attention --- the central question here was the president's intent
  when he took actions like firing Comey. But at the end of the day, the
  president of the United States never sat for an interview with
  investigators to answer those questions. So here you have the attorney
  general saying, this thing is done, but we've never heard from Donald
  Trump on it. And the reason we never heard from Donald Trump on it was
  that his lawyers were so afraid that if he sat down to answer
  questions, he would make a factually inaccurate statement and would
  increase his criminal exposure.
\item
  michael barbaro\\
  And it's within the president's rights not to sit down for such an
  interview.
\item
  michael schmidt\\
  That's not true. The Justice Department could have subpoenaed him for
  an interview.
\item
  michael barbaro\\
  But why didn't they?
\item
  michael schmidt\\
  One of the unanswered questions in this. But in an investigation where
  the central question was what was the president's intent, Bob Mueller
  was never able to ask him that question.
\item
  michael barbaro\\
  But he wasn't able to ask him the question because he never issued a
  subpoena. He never pursued all legal avenues?
\item
  maggie haberman\\
  We don't know whether he pursued a legal avenue and was told that
  there wasn't sufficient anything to go get a subpoena, or whether he
  just didn't seek one. And that is, to me right now as we sit here, an
  open question. I don't know whether Bob Mueller actually sought to go
  after this subpoena, which a lot of people thought that he would do.
  We're not going to know for some time.
\item
  michael schmidt\\
  But what we did learn on Friday when Barr first announced the end of
  the Mueller investigation was there was no instance in the
  investigation where Mueller wanted to take a major step, like indict
  someone, or subpoena them, or get a search warrant, in which he was
  told by his superiors at the Justice Department that he could not do
  that. So that would sort of bolster the argument that Mueller was not
  stopped by the Justice Department from seeking a subpoena to interview
  the president. He's saying there were no instances in which Mueller
  was stopped from moving forward with something he wanted to do.
\item
  michael barbaro\\
  No matter what, in this moment where so much of this second question
  of obstruction of justice comes down to the president's motivation,
  which is essentially a matter of getting into someone's head, never
  hearing from President Trump feels very significant.
\item
  maggie haberman\\
  It's a huge deal. Barr lays it out in this letter to Congress about
  what Mueller's thinking was, where he describes that the report sets
  out evidence on both sides of the question. And I think what you have
  there is the president's lawyers have repeatedly offered up
  explanations for why the president could have been doing something
  that were not necessarily nefarious or that were not undermining. And
  the main thing that they have said over and over is, this isn't
  obstruction because it's all playing out in public. It's not hidden.
  He has a right to express his views as a citizen of the country. And I
  anticipate that should we ever see what is in this report, that is
  going to show up a lot.
\item
  michael schmidt\\
  When Maggie and I would talk to Rudy Giuliani during the investigation
  ---
\item
  michael barbaro\\
  The president's lawyer.
\item
  michael schmidt\\
  He would say things like, obstruction of justice and trying to
  interfere in an investigation is going into a dark alley and
  threatening to break someone's leg if they cooperate with
  investigators. It's bribing someone to not testify. But what Giuliani
  said is that all this stuff is happening out in the open. He's not
  trying to twist anyone's arm. He's just blowing off steam about the
  investigation. And what conspiracy is there in that?
\item
  michael barbaro\\
  Were either of you surprised by either of these actions outlined in
  this letter when it comes to obstruction of justice --- by Mueller
  declining to make a call, or by Barr being so quick to make a call?
\item
  maggie haberman\\
  I was not surprised that Mueller didn't make a call. I was surprised
  that Barr moved so quickly. I did not think that the collusion aspect
  of this letter was going to be revelatory, but I did think that the
  obstruction one possibly would be. And it shuts it down definitively.
  There is a piece of the letter that we haven't talked about, which is
  that Bill Barr, the attorney general, tells Congress that Robert
  Mueller, the special counsel, recognized that, quote, ``the evidence
  does not establish that the president was involved in an underlying
  crime related to Russian election interference. And that while not
  determinative, the absence of such evidence bears upon the president's
  intent with respect to obstruction.'' So a big piece of the thinking
  for Barr seems to have been that since there is not an underlying
  crime that was found related to the president, that obstruction is
  harder to prove.
\item
  michael barbaro\\
  In other words, there is a connection between these two questions.
  Between coordination and obstruction of justice. That if there was no
  coordination, if there is no original sin, then it's hard to establish
  that there was obstruction of justice.
\item
  maggie haberman\\
  Right. If there is no crime that was committed, it is hard to suggest
  that the president was trying to obstruct justice in the course of
  pursuing evidence about that crime. And that is an argument that Mike
  and I also heard the president's lawyers say repeatedly, which was
  that there couldn't be obstruction of justice because there's nothing
  to obstruct.
\item
  michael barbaro\\
  Does that seem like universally sound legal logic? That you can't
  commit a crime in obstructing investigations into something that
  wasn't a crime? Because it certainly feels like if someone's
  investigating me, and I decide to stop the investigation in violation
  of the law, that that could be itself a violation of a law, separate
  and apart from whether or not what I was being investigated for in the
  first place was a crime? I mean, are those two things necessarily
  connected?
\item
  michael schmidt\\
  I think there's two reasons why the Justice Department would not want
  to bring a case like this. The first is that if there's not an
  underlying crime, then do you really want to bring an obstruction
  case? Do you really want to go to court to try and convince a jury
  that someone took actions where there was nothing to cover up? I think
  that's a tough case to make. The second thing is that while the
  president huffed and puffed a lot about this investigation, I'm not
  sure how much real damage it did to it. Bob Mueller was able to pursue
  his inquiry. He was not fired. They finished what they had. They
  weren't impeded by the Justice Department. So what was the real
  damage? If the president was obstructing, how did he really hurt the
  investigation? Give me the damage report. And I don't think at the end
  of the day there was that much that actually hurt Mueller's team's
  ability to do their job. It may have been loud, it may have been
  annoying, it may have been dispiriting. But I think they were able to
  pursue what they needed to.
\item
  maggie haberman\\
  I think, also, there's a third point, which is that I think that Bob
  Mueller was very mindful that he was dealing with the presidency and
  the damage that an ongoing case like this, if it was iffy, and not
  locked solid, not completely nailed down, could cause not just the
  office of the presidency, but the kind of trauma that brings to a
  country.
\item
  michael barbaro\\
  I'm very struck as we're talking, Maggie and Mike, that, at least
  according to this letter, this is exactly what we have had in front of
  us all along.
\item
  maggie haberman\\
  Unless the report has a lot of information that we're not seeing, what
  we saw play out publicly is what was there. The attorney general made
  clear that this investigation was ending without a recommendation of
  more indictments. That includes sealed indictments. This is what we're
  getting. It will now move to Capitol Hill, and Democrats will seek to
  get as much material as they can. But in terms of the part the
  president was really concerned about in terms of Mueller, that is
  over.
\item
  michael barbaro\\
  So let's talk about what Congress is going to do now. Congressional
  leaders have just received this letter. Mike, you have told us in the
  past that Democrats were going to be very cautious in how they
  proceeded under any circumstances. And that if Mueller didn't find the
  sort of thing that would force even Republicans in Congress to
  acknowledge that the president had violated the law, impeachment
  proceedings would be very unlikely. And it seems like this letter is
  not at all what Democrats would have needed to move forward in that
  sort of way.
\item
  michael schmidt\\
  I think it takes a lot of wind out of their sails. They basically have
  full-blown investigations into obstruction and collusion going on. And
  here, you have Mueller, someone who has far better tools than they
  will ever have to investigate, coming out and clearing the president
  on the Russia issue and giving a mixed message on obstruction that
  allows Barr to clear him there. It gives the Republicans a very good
  argument to say, why are you guys continuing to look at these issues?
  Why are you continuing to rummage around in the president's life when
  we've received clarity on this from the Justice Department? So I think
  it hurts them.
\item
  maggie haberman\\
  I think that you're going to see Nancy Pelosi, the Democratic House
  speaker, proceed carefully. As of late evening Sunday, she still had
  not said anything about how she was going to handle this. And I think
  that you are going to see a lot of Republicans reminding Democrats
  that they were holding Robert Mueller up as this avatar of credibility
  when the investigation was going on, so that it is hard to now say, we
  need to know more, because we need to make a determination that he
  couldn't. That will almost inevitably look very political.
\item
  michael schmidt\\
  So the Democrats are going to ask for everything that Mueller had in
  the report. And it's clear from the letter that a lot of obstruction
  of justice issues were looked at and detailed and analyzed. And the
  question is that if the Democrats get their hands on that, how
  damaging is the obstruction stuff? And is it strong enough to give
  Democrats more of an oomph on their obstruction investigation?
\item
  michael barbaro\\
  Is it both of your understandings that Congress will get a full copy
  of the Mueller report that they can then examine for these answers?
\item
  maggie haberman\\
  I don't think that's been resolved yet.
\item
  michael schmidt\\
  That's the big question.
\item
  michael barbaro\\
  Maggie, after months of the president calling the Mueller
  investigation a witch hunt or worse, would it be better now for him to
  lean into that description or turn around and say, you know, this
  investigation is the gold star of investigations, and it says I'm in
  the clear?
\item
  maggie haberman\\
  There are a lot of people in the White House who are encouraging the
  president to turn and say, this was a highly respected prosecutor who
  just said there was no collusion. And I hope we can all accept this
  and move on, and try to strike a healing note. It is the president's
  political instinct to burrow into his base and say, this was a witch
  hunt, this was a hoax. And that is what I have heard a number of his
  advisers say publicly in the last few hours. It's what his son said in
  a statement that he gave me. I think they are going to go full-on by
  telling people this was a waste of two years. This was an effort to
  undo an electoral result in 2016, and don't let this happen again.
\item
  michael barbaro\\
  And I wonder if you can explain for listeners why it wasn't what you
  just said the president might call it --- a waste of two years. A
  waste of everyone's time and an effort to politically undermine him.
\item
  maggie haberman\\
  It might have had the effect of undermining him. That might have been
  one of the impacts. But we do know, based on numerous assessments from
  the intelligence community, that Russians did seek to interfere in the
  2016 election. We do know that there were many contacts between
  Russians and people involved in the Trump campaign. So trying to find
  out what was at the heart of that, one would think, was not a waste of
  time. And now there is an answer. There is a world in which the
  president could say, this gave me a clean bill of health, and now you
  can trust me going forward.
\item
  michael barbaro\\
  Mike, I want to ask you the same question. What is your explanation
  for why this investigation was not a waste of time?
\item
  michael schmidt\\
  Well, there were legitimate issues and questions that had to be
  answered. And along the way, the president took action after action
  that only made the perception that he did something wrong worse. He
  fired the F.B.I. director. He asked the F.B.I. director to end an
  investigation into whether his former national security adviser was
  talking improperly to the Russian ambassador. He said things publicly
  about the investigation time and time again to demonize the
  investigators. He criticized his own attorney general, who recused
  himself from the Russia investigation. He called Mueller's team a
  bunch of angry Democrats. He made the perception worse. And there were
  things that he did that walked up to the line of obstruction and had
  to be looked at, that had to be sorted through. Whether it was to
  clear the president and say, hey, look, he did nothing wrong, for his
  sake, or to give the public some confidence that the president of the
  United States was not a Russian agent.
\item
  michael barbaro\\
  So among the questions that had to be answered in an investigation
  into Russian interference in our election, regardless of the answer,
  was what was the president's role in this? And it happens to be that
  the answer was seemingly nothing illegal. But that answer had to be
  unearthed.
\item
  michael schmidt\\
  A significant portion of the country thought that the president of the
  United States either wittingly or unwittingly had worked with a
  foreign adversary to influence the election. That was something that
  had to be looked into.
\item
  michael barbaro\\
  And because he didn't do that, Bill Barr also believes he did not
  obstruct this investigation.
\item
  michael schmidt\\
  There was nothing to obstruct.
\item
  maggie haberman\\
  At least that's what they found.
\item
  michael barbaro\\
  No one asked me, but ---
\item
  maggie haberman\\
  Michael, what do you think?
\item
  michael barbaro\\
  I think that Russia could never have fathomed just how cosmically
  disruptive their efforts to influence the election would be.
\item
  maggie haberman\\
  I think they succeeded beyond their wildest dreams, yes.
\item
  michael schmidt\\
  Look at how distracted we've been as a country because of this for the
  past two years. Think about all the other issues that we could have
  been talking about or looking at. This was an enormous, enormous
  distraction for the country.
\item
  maggie haberman\\
  And for Donald Trump.
\item
  michael barbaro\\
  Maggie, Mike, thank you very much.
\item
  maggie haberman\\
  Thank you.
\item
  michael schmidt\\
  Thanks for having us.
\item
  archived recording (donald trump)\\
  So after a long look, after a long investigation,

  after so many people have been so badly hurt, after not looking at the
  other side, where a lot of bad things happened, a lot of horrible
  things happened, a lot of very bad things happened for our country, it
  was just announced there was no collusion with Russia. The most
  ridiculous thing I've ever heard ---
\end{itemize}

michael barbaro

On Sunday afternoon, on his way back to Washington from his home in
Florida, President Trump addressed the major findings of the Mueller
report.

\begin{itemize}
\tightlist
\item
  archived recording (donald trump)\\
  There was no obstruction, and none whatsoever. And it was a complete
  and total exoneration. It's a shame that our country had to go through
  this. To be honest, it's a shame that your president has had to go
  through this for --- before I even got elected, it began. And it began
  illegally. And hopefully, somebody is going to look at the other side.
  This was an illegal takedown that failed. And hopefully, somebody is
  going to be looking at the other side. So it's complete ---
\end{itemize}

michael barbaro

Not long after, the Democratic chairman of the House Judiciary
Committee, Representative Jerry Nadler of New York, disputed the
president's characterization of the Mueller report during a news
conference in New York.

\begin{itemize}
\tightlist
\item
  archived recording (jerry nadler)\\
  Earlier today, I received a four-page letter from Attorney General
  Barr outlining his summary of Special Counsel Robert Mueller's report
  while making a few questionable legal arguments of his own. I take
  from this letter three points. First, President Trump is wrong. This
  report does not amount to a so-called ``total exoneration.''
\end{itemize}

michael barbaro

Nadler seemed to accept Mueller's conclusion that the president did not
coordinate with Russia, but seized on less definitive statements made by
Mueller and Barr about whether the president obstructed justice.

\begin{itemize}
\tightlist
\item
  archived recording (jerry nadler)\\
  The attorney general's comments make it clear that Congress must step
  in to get the truth and provide full transparency to the American
  people. The president has not been exonerated by the special counsel.
  Yet the attorney general has decided not to go further or, apparently,
  to share those findings with the public. We cannot simply rely on what
  may be a hasty partisan interpretation of the facts.
\end{itemize}

michael barbaro

That's it for ``The Daily.'' I'm Michael Barbaro. See you tomorrow.

Hours later, Mr. Barr delivered his letter describing the special
counsel's findings to Congress. But congressional Democrats have
demanded more, and the letter could be just the beginning of
\href{https://www.nytimes.com/2019/03/22/us/politics/executive-privilege-mueller.html}{a
lengthy constitutional battle} between Congress and the Justice
Department about whether Mr. Mueller's full report will be made public.
Democrats have also called for the attorney general to turn over all of
the special counsel's investigative files.

Mr. Barr's letter said that his ``goal and intent'' was to release as
much of the Mueller report as possible, but warned that some of the
report was based on grand jury material that ``by law cannot be made
public.'' Mr. Barr planned at a later date to send lawmakers the
detailed summary of Mr. Mueller's full report that the attorney general
is required under law to deliver to Capitol Hill.

\includegraphics{https://static01.nyt.com/images/2019/03/26/us/26trump-video/25trump-video-videoSixteenByNine3000.jpg}

Lawmakers on Sunday also criticized Mr. Barr's conclusion that the
president had not obstructed justice --- which requires making a
determination about whether Mr. Trump had ``corrupt intent'' when he
took steps to impede the investigation at different turns --- when the
special counsel's team never questioned the president in person. After
months of debate over a potential interview, Mr. Mueller's investigators
agreed to accept written answers from the president.

Mr. Barr's letter said that the Mueller report identified no actions
that, in his and Mr. Rosenstein's minds, ``constitute obstructive
conduct, had a nexus to a pending or contemplated proceeding, and were
done with corrupt intent.'' Mr. Barr did not consult Mr. Mueller in
writing his letter to leaders of the congressional judiciary committees,
a Justice Department official said on Sunday.

Shortly after the release of the Mueller findings, Representative
Jerrold Nadler of New York, the chairman of the House Judiciary
Committee,
\href{https://twitter.com/RepJerryNadler/status/1109913142933573632}{said
on Twitter} that he planned to call Mr. Barr to testify about what he
said were ``very concerning discrepancies and final decision making at
the Justice Department.''

The Russia investigation has buffeted the White House from the earliest
days of the Trump administration, with many current and former aides to
Mr. Trump brought for questioning to the special counsel's warren of
offices in a plain office building in downtown Washington. F.B.I. agents
fanned out across the nation and traveled to numerous foreign countries.
Members of Mr. Mueller's team questioned some witnesses at airports
after they landed in the United States.

Ultimately, a half-dozen former Trump aides
\href{https://www.nytimes.com/interactive/2018/08/21/us/mueller-trump-charges.html}{were
indicted or convicted of crimes}, most for conspiracy or lying to
investigators. Twenty-five Russian
\href{https://www.nytimes.com/2018/07/13/us/politics/mueller-indictment-russian-intelligence-hacking.html}{intelligence
operatives} and
\href{https://www.nytimes.com/2018/02/16/us/politics/russians-indicted-mueller-election-interference.html}{experts
in social media manipulation} were charged last year in two
extraordinarily detailed indictments released by the special counsel.
The inquiry concluded without charging any Americans for conspiring with
the Russian campaign.

\href{https://www.nytimes.com/interactive/2019/03/20/us/politics/mueller-investigation-people-events.html}{}

\includegraphics{https://static01.nyt.com/images/2019/03/15/us/mueller-report-people-events-promo-1552676143429/mueller-report-people-events-promo-1552676143429-articleLarge-v3.jpg}

\hypertarget{mueller-report-who-and-what-the-special-counsel-investigated}{%
\subsection{Mueller Report: Who and What the Special Counsel
Investigated}\label{mueller-report-who-and-what-the-special-counsel-investigated}}

More than two years of criminal indictments and steady revelations about
Trump campaign contacts with Russians reveal the scope of the special
counsel investigation.

The findings could bring closure for some who have obsessed over the
myriad threads of a byzantine investigation. A cottage industry of
Mueller watchers has spent months on social media and cable news
debating thorny constitutional issues, spinning conspiracy theories and
amassing encyclopedic details about once obscure figures --- Carter
Page, Konstantin V. Kilimnik, George Papadopoulos and others.

How many minds it changes is another matter. Opinions have hardened over
time, with many Americans already convinced they knew the answers before
Mr. Mueller submitted his conclusions. Some believe that the special
counsel's previous indictments, twinned with voluminous news reporting,
have already shown a conspiracy between the Trump campaign and the
Kremlin. Some believe that the investigation is, as Mr. Trump has long
described it, a ``witch hunt.''

To prove a conspiracy, former prosecutors said, Mr. Mueller's team would
have had to show that Mr. Trump or one or more of his associates agreed
that Russia should interfere in the election through computer espionage,
illegal use of social media or other criminal means.

Campaign officials at times were eager to accept benefits from Russia's
covert operation. ``I love it,'' Donald Trump Jr., the president's
eldest son, responded when an intermediary said a Russian emissary
wanted to give the campaign damaging information on Hillary Clinton at a
Trump Tower meeting in June 2016.

Mr. Trump himself urged Russia to try to unearth deleted emails from a
private server Mrs. Clinton had used when she was secretary of state.

\includegraphics{https://static01.nyt.com/images/2019/03/25/us/25dc-mueller-2/25dc-mueller-2-articleLarge.jpg?quality=75\&auto=webp\&disable=upscale}

And Roger J. Stone, Jr., the president's longtime friend, tried to
enlist intermediaries to connect with WikiLeaks, Russia's chosen
depository for Democratic emails stolen by Russian hackers.

But absent an agreement with the Russian government to break the law,
former Justice Department officials said, none of that made Mr. Trump or
his associates into co-conspirators with the Kremlin.

``There is a big difference between saying, `Gosh, I think WikiLeaks has
the ability to hack into the Democratic National Committee computers'
and saying `We would like them to dump those out in the public, so let's
call them up and ask them to do that,''' said Mary McCord, a former
top-ranking national security official at the Justice Department.

The release of Mr. Mueller's findings could force a decision by
Democrats on a simmering issue they have said would wait until the
investigation's end: whether to begin impeachment proceedings against
the president. Speaker Nancy Pelosi of California has said it would not
be ``worth it'' to try to impeach Mr. Trump, but suggested she could
change her mind if an overwhelming bipartisan consensus emerged.

For months, the president and his lawyers have waged as much of a public
relations campaign as a legal one ---
\href{https://www.nytimes.com/2019/02/19/us/politics/trump-investigations.html}{trying
to discredit the Mueller investigation} to keep public opinion from
swaying lawmakers to move against Mr. Trump.

\href{https://www.nytimes.com/interactive/2019/03/21/us/the-mueller-report-photos.html}{}

\includegraphics{https://static01.nyt.com/images/2019/03/21/us/00dc-muellervisuals-promo/00dc-muellervisuals-promo-articleLarge.jpg}

\hypertarget{glimpses-of-the-mystery-that-is-the-mueller-investigation}{%
\subsection{Glimpses of the Mystery That Is the Mueller
Investigation}\label{glimpses-of-the-mystery-that-is-the-mueller-investigation}}

Here are some pieces of the jigsaw puzzle. The full picture is missing.

Mr. Mueller's work has proceeded in the face of blistering attacks by
Mr. Trump and his allies, who painted the investigation as part of a
relentless campaign by the ``deep state'' to reverse the results of the
2016 election.

He was
\href{https://www.justice.gov/opa/press-release/file/967231/download}{given
a wide mandate} --- to investigate not only Russian election
interference but also ``any matters that may arise directly from that
investigation.'' Mr. Mueller has farmed out multiple aspects of his
inquiry to several United States attorneys' offices, and those
investigations continue.

Mr. Barr's letter said that the special counsel's office employed 19
lawyers and was assisted by about 40 F.B.I. agents, intelligence
analysts, forensic accountants and other staff. About 500 witnesses were
interviewed, and 13 foreign governments were asked to turn over
evidence.

Over all, the special counsel's office issued more than 2,800 subpoenas,
executed nearly 500 search warrants and obtained more than 230 orders
for communications records.

The Justice Department regulations governing the Mueller inquiry only
required the special counsel to give a succinct, confidential report to
the attorney general explaining his decisions to either seek --- or
decline to seek --- further criminal charges. Mr. Mueller operated under
tighter restrictions than similar past inquiries, notably the
investigation of President Bill Clinton by Ken Starr, who ended up
delivering a 445-page report in 1998 that contained lascivious details
about an affair the president had with a White House intern.

Mr. Mueller will not recommend new indictments, ending speculation that
he might charge some of Mr. Trump's aides in the future. The Justice
Department's general practice is not to identify the targets of its
investigations if prosecutors decide not to bring charges, so as not to
tarnish their reputations. Mr. Rosenstein emphasized this point in a
speech last month.

``It's important,'' Mr. Rosenstein said, ``for government officials to
refrain from making allegations of wrongdoing when they're not backed by
charges that we are prepared to prove in court.''

Advertisement

\protect\hyperlink{after-bottom}{Continue reading the main story}

\hypertarget{site-index}{%
\subsection{Site Index}\label{site-index}}

\hypertarget{site-information-navigation}{%
\subsection{Site Information
Navigation}\label{site-information-navigation}}

\begin{itemize}
\tightlist
\item
  \href{https://help.nytimes.com/hc/en-us/articles/115014792127-Copyright-notice}{©~2020~The
  New York Times Company}
\end{itemize}

\begin{itemize}
\tightlist
\item
  \href{https://www.nytco.com/}{NYTCo}
\item
  \href{https://help.nytimes.com/hc/en-us/articles/115015385887-Contact-Us}{Contact
  Us}
\item
  \href{https://www.nytco.com/careers/}{Work with us}
\item
  \href{https://nytmediakit.com/}{Advertise}
\item
  \href{http://www.tbrandstudio.com/}{T Brand Studio}
\item
  \href{https://www.nytimes.com/privacy/cookie-policy\#how-do-i-manage-trackers}{Your
  Ad Choices}
\item
  \href{https://www.nytimes.com/privacy}{Privacy}
\item
  \href{https://help.nytimes.com/hc/en-us/articles/115014893428-Terms-of-service}{Terms
  of Service}
\item
  \href{https://help.nytimes.com/hc/en-us/articles/115014893968-Terms-of-sale}{Terms
  of Sale}
\item
  \href{https://spiderbites.nytimes.com}{Site Map}
\item
  \href{https://help.nytimes.com/hc/en-us}{Help}
\item
  \href{https://www.nytimes.com/subscription?campaignId=37WXW}{Subscriptions}
\end{itemize}
