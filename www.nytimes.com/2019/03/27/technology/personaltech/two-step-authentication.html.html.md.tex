Sections

SEARCH

\protect\hyperlink{site-content}{Skip to
content}\protect\hyperlink{site-index}{Skip to site index}

\href{https://www.nytimes.com/section/technology/personaltech}{Personal
Tech}

\href{https://myaccount.nytimes.com/auth/login?response_type=cookie\&client_id=vi}{}

\href{https://www.nytimes.com/section/todayspaper}{Today's Paper}

\href{/section/technology/personaltech}{Personal
Tech}\textbar{}Protecting Your Internet Accounts Keeps Getting Easier.
Here's How to Do It.

\url{https://nyti.ms/2CAIqwv}

\begin{itemize}
\item
\item
\item
\item
\item
\item
\end{itemize}

Advertisement

\protect\hyperlink{after-top}{Continue reading the main story}

Supported by

\protect\hyperlink{after-sponsor}{Continue reading the main story}

Tech Fix

\hypertarget{protecting-your-internet-accounts-keeps-getting-easier-heres-how-to-do-it}{%
\section{Protecting Your Internet Accounts Keeps Getting Easier. Here's
How to Do
It.}\label{protecting-your-internet-accounts-keeps-getting-easier-heres-how-to-do-it}}

There are many tools for setting up two-factor authentication, a
security mechanism that prevents improper access. These four methods are
the most compelling.

\includegraphics{https://static01.nyt.com/images/2019/03/27/business/27Techfix-illo/27Techfix-illo-articleLarge.gif?quality=75\&auto=webp\&disable=upscale}

\href{https://www.nytimes.com/by/brian-x-chen}{\includegraphics{https://static01.nyt.com/images/2018/02/16/multimedia/author-brian-x-chen/author-brian-x-chen-thumbLarge.jpg}}

By \href{https://www.nytimes.com/by/brian-x-chen}{Brian X. Chen}

\begin{itemize}
\item
  March 27, 2019
\item
  \begin{itemize}
  \item
  \item
  \item
  \item
  \item
  \item
  \end{itemize}
\end{itemize}

When Facebook revealed last week that it had
\href{https://www.nytimes.com/2019/03/21/technology/personaltech/facebook-passwords.html}{stored
millions of people's account passwords in an insecure format}, it
underlined the importance of a security setting that many of us neglect
to use:
\href{https://www.nytimes.com/2017/08/08/technology/personaltech/protecting-your-accounts-by-text-or-app.html?module=inline}{two-factor
authentication}.

That might sound like a mouthful, but it has become essential for our
digital protection. What it stands for is basically two steps to verify
that you are who you say you are, so that even if a password falls into
the hands of the wrong people, they cannot pretend to be you.

Here's how two-factor authentication has generally worked: Say, for
instance, you enter your user name and password to get into your online
bank account. That's step one. The bank then sends a text message to
your phone with a temporary code that must be punched in before the site
lets you log in. That's step two. In this way, you prove your identity
by having access to your phone and that code.

Sounds simple and safer, right? Yet barely anyone uses it. According to
Google, fewer than 10 percent of its users have signed up for two-factor
authentication to protect their Google accounts for services including
email, photos and calendars.

``It's really, really hard to get a user to sign up,'' said Guemmy Kim,
Google's head of account security. ``It sounds cumbersome.''

In reality, it isn't that complicated. And in recent years, the
technique has evolved to become more secure and, in some cases, even
easier to use.

That's because in addition to receiving text messages, you can now log
in by using codes shown in an app, by plugging in a physical security
key or by setting up your phone to receive a notification and hitting a
button. More on that below.

Using just one or two of these methods will go a long way toward
preventing an inappropriate person, like a jealous ex or a hacker, from
getting access to your account. So here's a guide to four ways of
setting up two-factor authentication on some of the most popular sites
--- and the pros and cons of each method.

\hypertarget{securing-your-instagram-account-with-text-messaged-codes}{%
\subsection{Securing your Instagram account with text-messaged
codes}\label{securing-your-instagram-account-with-text-messaged-codes}}

Let's start by setting up your Instagram account with traditional
two-factor authentication using text messages. This is the most common
verification technique across apps and websites, though it has some of
the biggest vulnerabilities.

\includegraphics{https://static01.nyt.com/images/2019/03/27/business/27techfix-instagram/fileupload-1553626116142-articleLarge.png?quality=75\&auto=webp\&disable=upscale}

Here's what to do:

\begin{itemize}
\tightlist
\item
  Inside your Instagram app, open settings, then tap privacy and
  security and select two-factor authentication.
\end{itemize}

\begin{itemize}
\item
  Enter your phone number. You will receive a text message containing a
  six-digit code. Enter the code.
\item
  From now on, whenever you log in to your Instagram account, you will
  receive a text message containing a temporary code. This must be
  entered before you log in.
\end{itemize}

\textbf{Pros:} This method is super easy: You do not need to install any
additional apps on your phone to receive texts. And if you lose your
device or switch to a new phone, you can still receive your login codes
as long as you have the same phone number.

\textbf{Cons:} Phone numbers and text messages are susceptible to
phishing or hijacking by hackers (though this is unlikely to happen
unless you are a high-profile target such as a well-known activist). If
you travel abroad, receiving text messages on a foreign carrier can be
pricey. And there are security risks in receiving texts on foreign
networks in countries with heavy surveillance such as China and Russia.

\hypertarget{setting-up-an-app-to-authenticate-your-facebook-account}{%
\subsection{Setting up an app to authenticate your Facebook
account}\label{setting-up-an-app-to-authenticate-your-facebook-account}}

Another way to start two-factor authentication is to receive a temporary
code via a so-called authenticator app. For this example, let's protect
your Facebook account with such an app.

Here's how it works:

\begin{itemize}
\item
  On your phone, open your app store and download a free authenticator
  app, like
  \href{https://itunes.apple.com/us/app/google-authenticator/id388497605?mt=8}{Google
  Authenticator} or
  \href{https://itunes.apple.com/us/app/authy/id494168017?mt=8}{Authy}.
\item
  Then on Facebook's website, go to your
  \href{https://www.facebook.com/settings?tab=security}{security and
  login settings}. Click ``use two-factor authentication,'' then ``get
  started.'' After re-entering your password, choose authentication app
  as your security method. From here, follow the onscreen instructions.
\item
  From now on, whenever you log in to Facebook, you can open the
  authenticator app and look at the temporary six-digit code generated
  for your Facebook account. You must enter this code before being able
  to log in.
\end{itemize}

\textbf{Pros:} You do not need an internet or a cellphone connection to
receive a code via an authentication app. Most important, a hijacker
can't easily steal your codes from an authenticator app.

\textbf{Cons:} If you lose your phone or switch to a new one, you have
to regain access to your account through a recovery method such as
entering a backup code or asking the app provider to reset your account.
That can be time consuming.

\hypertarget{setting-up-google-prompt-on-google-mail}{%
\subsection{Setting up Google Prompt on Google
Mail}\label{setting-up-google-prompt-on-google-mail}}

Google Prompt is a relatively new authentication feature for securing
Google accounts. Instead of receiving a text message with a code, you
receive a notification through a Google app asking whether the person
trying to sign in is you. Hitting ``Yes'' logs you in.

Image

Credit...BrianXChen

Here are the steps:

\begin{itemize}
\tightlist
\item
  On Gmail.com, go to your account settings and click ``security.''
  Click 2-Step Verification, and then click Add Google Prompt.
\end{itemize}

\begin{itemize}
\item
  Click Get Started and select your smartphone.
\item
  On your phone, open the Google or Gmail app. Google will show a device
  trying to log in to your account. Tap Yes on the prompt.
\item
  From now on, whenever you log in to your Gmail account, the Gmail or
  Google app will ask whether the person seeking access is you. Hitting
  Yes will log you in.
\end{itemize}

\textbf{Pros:} It's easy. Receiving a notification requires only an
internet connection. Selecting Yes is faster than typing in a code.

\textbf{Cons:} Not all apps and sites have a prompt-based verification
method, meaning your banking site, for example, may still text you a
temporary code. If your internet connection is spotty, you may also have
a difficult time receiving the prompt.

\hypertarget{securing-your-twitter-account-with-a-physical-key}{%
\subsection{Securing your Twitter account with a physical
key}\label{securing-your-twitter-account-with-a-physical-key}}

Last, let's go over the most physical two-factor authentication method,
which involves plugging in a key. Google was one of the first to
introduce a security key program in 2017, and many websites, including
Twitter and Facebook, have since adopted the method.

Here's how to secure a Twitter account with a security key:

\begin{itemize}
\item
  Buy a security key, such as Google's \$50
  \href{https://store.google.com/product/titan_security_key_kit}{Titan
  security key bundle}.
\item
  On Twitter's website, go to your account settings and click ``Set up
  login verification.'' Enter your phone number, and then punch in the
  code you receive via text message.
\item
  In ``Security key,'' click set up. Insert the security key into a USB
  port, and press the button on the key. Press the button again to
  verify the key.
\item
  The next time you log in to Twitter, click ``Choose different
  verification method'' and select ``Use your security key.'' After
  plugging the key into your computer, you will be able to log in.
\end{itemize}

\textbf{Pros:} For people who are extra paranoid about being phished or
hacked, this is one of the most secure authentication methods because
physical access to your key is required for logging in.

\textbf{Cons:} The keys cost money. What's more, some sites require you
to insert the key every time, so if you forget to carry your key,
logging in with a backup method can be complicated. And not all web
browsers support logging in with security keys.

Advertisement

\protect\hyperlink{after-bottom}{Continue reading the main story}

\hypertarget{site-index}{%
\subsection{Site Index}\label{site-index}}

\hypertarget{site-information-navigation}{%
\subsection{Site Information
Navigation}\label{site-information-navigation}}

\begin{itemize}
\tightlist
\item
  \href{https://help.nytimes.com/hc/en-us/articles/115014792127-Copyright-notice}{©~2020~The
  New York Times Company}
\end{itemize}

\begin{itemize}
\tightlist
\item
  \href{https://www.nytco.com/}{NYTCo}
\item
  \href{https://help.nytimes.com/hc/en-us/articles/115015385887-Contact-Us}{Contact
  Us}
\item
  \href{https://www.nytco.com/careers/}{Work with us}
\item
  \href{https://nytmediakit.com/}{Advertise}
\item
  \href{http://www.tbrandstudio.com/}{T Brand Studio}
\item
  \href{https://www.nytimes.com/privacy/cookie-policy\#how-do-i-manage-trackers}{Your
  Ad Choices}
\item
  \href{https://www.nytimes.com/privacy}{Privacy}
\item
  \href{https://help.nytimes.com/hc/en-us/articles/115014893428-Terms-of-service}{Terms
  of Service}
\item
  \href{https://help.nytimes.com/hc/en-us/articles/115014893968-Terms-of-sale}{Terms
  of Sale}
\item
  \href{https://spiderbites.nytimes.com}{Site Map}
\item
  \href{https://help.nytimes.com/hc/en-us}{Help}
\item
  \href{https://www.nytimes.com/subscription?campaignId=37WXW}{Subscriptions}
\end{itemize}
