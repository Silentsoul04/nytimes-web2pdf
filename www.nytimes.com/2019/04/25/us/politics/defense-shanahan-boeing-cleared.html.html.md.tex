Sections

SEARCH

\protect\hyperlink{site-content}{Skip to
content}\protect\hyperlink{site-index}{Skip to site index}

\href{https://www.nytimes.com/section/politics}{Politics}

\href{https://myaccount.nytimes.com/auth/login?response_type=cookie\&client_id=vi}{}

\href{https://www.nytimes.com/section/todayspaper}{Today's Paper}

\href{/section/politics}{Politics}\textbar{}Acting Defense Secretary
Patrick Shanahan Is Cleared in Ethics Inquiry

\url{https://nyti.ms/2XI1jWC}

\begin{itemize}
\item
\item
\item
\item
\item
\end{itemize}

Advertisement

\protect\hyperlink{after-top}{Continue reading the main story}

Supported by

\protect\hyperlink{after-sponsor}{Continue reading the main story}

\hypertarget{acting-defense-secretary-patrick-shanahan-is-cleared-in-ethics-inquiry}{%
\section{Acting Defense Secretary Patrick Shanahan Is Cleared in Ethics
Inquiry}\label{acting-defense-secretary-patrick-shanahan-is-cleared-in-ethics-inquiry}}

\includegraphics{https://static01.nyt.com/images/2019/04/25/us/25dc-pentagon/merlin_153371454_16089607-fe5f-45e0-b579-961ad582ec43-articleLarge.jpg?quality=75\&auto=webp\&disable=upscale}

By \href{https://www.nytimes.com/by/helene-cooper}{Helene Cooper}

\begin{itemize}
\item
  April 25, 2019
\item
  \begin{itemize}
  \item
  \item
  \item
  \item
  \item
  \end{itemize}
\end{itemize}

WASHINGTON --- Acting Defense Secretary Patrick Shanahan overcame a
major hurdle to obtaining the top Pentagon job on Thursday when an
internal ethics investigation cleared him of allegations that he
promoted his former employer, Boeing, and disparaged its competitors in
official discussions about military contractors.

But while the Defense Department's inspector general found no evidence
to support complaints that Mr. Shanahan ``repeatedly dumped'' on
Boeing's competitors, it did cite a number of times that he praised the
Boeing 787 Dreamliner that he
\href{https://www.nytimes.com/2017/04/01/business/patrick-shanahan-pentagon-nominee.html}{famously
rescued} as a senior executive at the aircraft maker before joining the
government.

Mr. Shanahan
\href{https://www.nytimes.com/2018/12/23/us/politics/trump-mattis.html}{took
over the top job} at the Pentagon in an acting capacity in December, but
President Trump has not nominated him to the Senate for approval.

``The evidence showed that Acting Secretary Shanahan fully complied with
his ethical obligations and ethical agreements with regard to Boeing and
its competitors,'' Glenn A. Fine, the acting inspector general, said in
a statement on Thursday.

Two months ago, Defense Department officials were preparing talking
points for Mr. Shanahan's expected nomination. But those plans were
shelved after the ethics investigation was announced, and after the
crash of a Boeing 737 Max 8 in Ethiopia in March,
\href{https://www.nytimes.com/2019/04/04/world/asia/ethiopia-crash-boeing.html?module=inline}{killing
all 157 people} aboard.

Mr. Shanahan had managed to dodge being directly tied to the fallout
after that crash and that of
\href{https://www.nytimes.com/2019/02/03/world/asia/lion-air-plane-crash-pilots.html}{another
737 Max off the Indonesian coast} in October, and no Dreamliners have
failed like those two jets. But the inspector general's report cited
numerous meetings at the Pentagon during which Defense Department
officials said Mr. Shanahan had promoted his experiences solving
production problems on the Dreamliner as techniques that should be
copied by the government.

The praise could come back to bite him: Boeing's plant in North
Charleston, S.C., where the Dreamliner is built, has been
\href{https://www.nytimes.com/2019/04/20/business/boeing-dreamliner-production-problems.html}{plagued
by employee complaints} about safety issues on the plane stemming from
the rapid pace that was pushed by management to meet quotas after the
jet faced production delays.

Mr. Shanahan is well known for playing a key role in rescuing the
Dreamliner during his three-decade career at Boeing when
\href{https://www.nytimes.com/2017/04/01/business/patrick-shanahan-pentagon-nominee.html}{delays
and cost overruns} threatened the future of that project.

Mr. Trump, who had previously expressed a willingness to consider Mr.
Shanahan for the top Pentagon job, has not said much on his prospects
since. Kellyanne Conway, a counselor to the president, told reporters on
Wednesday that Mr. Trump was ``very pleased with his defense team'' and
that the lack of a Senate-approved defense secretary had not impeded the
Pentagon's work.

Mr. Shanahan has skeptics in the Senate, where his predecessor, the
retired Marine Gen. Jim Mattis, was uniformly respected.

``The senior leadership of D.O.D. oversees hundreds of billions of
taxpayer dollars, and to do that job well they must be above reproach
and focused only on the country's interests,'' Senator Jack Reed of
Rhode Island, the top Democrat on the Armed Services Committee, said in
a statement after the inspector general's report was released on
Thursday.

He said the investigation ``shows the wide swath of national security
matters that Acting Secretary Shanahan is barred from, which strikes me
as something the Senate needs to consider.''

Senator James M. Inhofe, Republican of Oklahoma and the chairman of the
committee, and Senator Lindsey Graham, Republican of South Carolina,
have also appeared skeptical of Mr. Shanahan.

The inspector general's office interviewed more than 30 witnesses,
including senior officials and Mr. Shanahan himself, according to Dwrena
Allen, a spokeswoman for the office. Mr. Mattis was also interviewed.

The investigation examined formal complaints filed by the watchdog group
Citizens for Responsibility and Ethics in Washington. Some Pentagon
staff members had also complained about Mr. Shanahan, a Defense
Department official said on Thursday, describing an unusual level of
hostility within the military headquarters that could make it difficult
for him to run the department if he is nominated and confirmed.

The investigation did not find that Mr. Shanahan had acted
inappropriately on behalf of Boeing while in a senior leadership role at
the Pentagon, going back to the start of the Trump administration, when
he was confirmed as the deputy defense secretary. Nor did it find that
he had ``boosted'' Boeing in Pentagon meetings, or ``repeatedly dumped''
on competitor Lockheed Martin's F-35 aircraft.

The report did include one unusual passage in which Mr. Shanahan sought
to explain that his disparaging remarks about the F-35, an expensive
stealth jet that has been plagued by cost overruns and stolen
technology, were actually about the program --- not the warplane itself.

``Mr. Shanahan told us that he did not say that the F-35 aircraft was
`f---ed up,''' the report said. ``He told us that the F-35 aircraft is
`awesome.' Mr. Shanahan told us that he said the F-35 \emph{program} was
`f---ed up.'''

The investigation also did not find that Mr. Shanahan pressured Gen.
Robert B. Neller, the Marine Corps commandant, to buy Boeing F/A-18
fighter jets. Nor did he threaten to cut other Air Force programs unless
the Air Force chief of staff, Gen. David L. Goldfein, supported buying
Boeing F-15Xs. All were allegations that were investigated by the
inspector general's office.

The
\href{https://media.defense.gov/2019/Apr/25/2002120979/-1/-1/1/DODIG-\%202019-082.PDF}{47-page
report} cited numerous senior Defense Department officials who said Mr.
Shanahan had, in their eyes, acted properly. At one point, the report
quoted an Air Force general who said he had approached Mr. Shanahan in
September 2017 to brief him on a Boeing program, and was stopped.

``Stop. That's a Boeing program. I can't talk about it,'' Mr. Shanahan
said, as relayed by Gen. John E. Hyten, the commander of United States
Strategic Command, to investigators.

``Not even conceptually about future capabilities?'' General Hyten said
he asked. He said Mr. Shanahan replied, ``No, I can't talk about that at
all.''

But what Mr. Shanahan did talk about --- repeatedly --- was the
Dreamliner.

Heather A. Wilson, the
\href{https://www.nytimes.com/2019/03/08/us/politics/heather-wilson-air-force-.html}{retiring
Air Force secretary}, told investigators that ``in almost every meeting,
there were references to the Dreamliner.'' William Roper, an assistant
Air Force secretary, told investigators ``that Mr. Shanahan would often
discuss his experiences in solving issues on the Boeing commercial
Dreamliner as program techniques that should be used.''

None of the officials said they felt pressured on the Dreamliner, a
commercial aircraft.

But in one instance, Ms. Wilson told investigators, Mr. Shanahan or his
staff may have ``created the appearance of favoritism'' by ordering that
a meeting on delivery of a Boeing refueling tanker, the KC-46, be led by
a Pentagon official who was believed to favor the aerospace contractor's
terms.

As a former Boeing executive, Mr. Shanahan was not supposed to have
anything to do with decisions on the KC-46. Ms. Wilson told
investigators that she felt she had to make sure that the KC-46
procurement process ``was protected from undue influence.''

Mr. Shanahan denied to investigators that he pressured the Air Force to
accept the KC-46 delivery. The inspector general report concluded that
Mr. Shanahan had no involvement in the Air Force's acceptance of the
tanker delivery, and that he was appropriately ``screened'' from
decision-making on the issue by his staff.

While Mr. Mattis resigned in protest of Mr. Trump's policies ---
including the surprise announcement that American troops would withdraw
from Syria --- Mr. Shanahan has held the line for the president.

As the deputy defense secretary, Mr. Shanahan made clear that
\href{https://www.nytimes.com/2018/12/24/us/politics/patrick-shanahan-defense-secretary.html}{``we
are not the Department of No,''} as he told officials after the
administration announced plans to create a stand-alone Space Force at
the Pentagon. (It has since been moved to the oversight of the Air
Force.)

Advertisement

\protect\hyperlink{after-bottom}{Continue reading the main story}

\hypertarget{site-index}{%
\subsection{Site Index}\label{site-index}}

\hypertarget{site-information-navigation}{%
\subsection{Site Information
Navigation}\label{site-information-navigation}}

\begin{itemize}
\tightlist
\item
  \href{https://help.nytimes.com/hc/en-us/articles/115014792127-Copyright-notice}{©~2020~The
  New York Times Company}
\end{itemize}

\begin{itemize}
\tightlist
\item
  \href{https://www.nytco.com/}{NYTCo}
\item
  \href{https://help.nytimes.com/hc/en-us/articles/115015385887-Contact-Us}{Contact
  Us}
\item
  \href{https://www.nytco.com/careers/}{Work with us}
\item
  \href{https://nytmediakit.com/}{Advertise}
\item
  \href{http://www.tbrandstudio.com/}{T Brand Studio}
\item
  \href{https://www.nytimes.com/privacy/cookie-policy\#how-do-i-manage-trackers}{Your
  Ad Choices}
\item
  \href{https://www.nytimes.com/privacy}{Privacy}
\item
  \href{https://help.nytimes.com/hc/en-us/articles/115014893428-Terms-of-service}{Terms
  of Service}
\item
  \href{https://help.nytimes.com/hc/en-us/articles/115014893968-Terms-of-sale}{Terms
  of Sale}
\item
  \href{https://spiderbites.nytimes.com}{Site Map}
\item
  \href{https://help.nytimes.com/hc/en-us}{Help}
\item
  \href{https://www.nytimes.com/subscription?campaignId=37WXW}{Subscriptions}
\end{itemize}
