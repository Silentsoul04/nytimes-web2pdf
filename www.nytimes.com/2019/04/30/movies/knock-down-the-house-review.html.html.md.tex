Sections

SEARCH

\protect\hyperlink{site-content}{Skip to
content}\protect\hyperlink{site-index}{Skip to site index}

\href{https://www.nytimes.com/section/movies}{Movies}

\href{https://myaccount.nytimes.com/auth/login?response_type=cookie\&client_id=vi}{}

\href{https://www.nytimes.com/section/todayspaper}{Today's Paper}

\href{/section/movies}{Movies}\textbar{}`Knock Down the House' Review:
Running to Win Hearts and Minds and Votes

\url{https://nyti.ms/2DFNmkx}

\begin{itemize}
\item
\item
\item
\item
\item
\item
\end{itemize}

Advertisement

\protect\hyperlink{after-top}{Continue reading the main story}

Supported by

\protect\hyperlink{after-sponsor}{Continue reading the main story}

Critic's Pick

\hypertarget{knock-down-the-house-review-running-to-win-hearts-and-minds-and-votes}{%
\section{`Knock Down the House' Review: Running to Win Hearts and Minds
and
Votes}\label{knock-down-the-house-review-running-to-win-hearts-and-minds-and-votes}}

\includegraphics{https://static01.nyt.com/images/2019/04/30/arts/30knock3/30knock3-articleLarge-v2.jpg?quality=75\&auto=webp\&disable=upscale}

\begin{itemize}
\tightlist
\item
  Knock Down the House\\
  **NYT Critic's Pick Directed by Rachel Lears Documentary PG 1h 26m
\end{itemize}

\href{https://www.imdb.com/showtimes/title/tt9358052?ref_=ref_ext_NYT}{Find
Tickets}

When you purchase a ticket for an independently reviewed film through
our site, we earn an affiliate commission.

By \href{https://www.nytimes.com/by/manohla-dargis}{Manohla Dargis}

\begin{itemize}
\item
  April 30, 2019
\item
  \begin{itemize}
  \item
  \item
  \item
  \item
  \item
  \item
  \end{itemize}
\end{itemize}

The big draw of the exuberant documentary ``Knock Down the House'' ---
about four women who ran for Congress in 2018 --- is Representative
\href{https://ocasio-cortez.house.gov/}{Alexandria Ocasio-Cortez,
Democrat of New York.} The former bartender turned first-time
congresswoman is the most high-profile member of the freshman class of
2019 and scarcely needs an introduction. Her social media presence alone
shows why she has crossed over into pop celebrity, whether she's
tweet-storming or live-streaming on Instagram while eating popcorn,
talking about staying grounded and assembling Ikea furniture: ``Boom! I
did it.''

The director Rachel Lears
\href{https://www.missourireview.com/true-false-film-festival-an-interview-with-rachel-lears/}{has
said in interviews} that after the 2016 presidential election she
reached out to the progressive groups Brand New Congress and Justice
Democrats, which ran unknowns (or what one organizer in the documentary
describes as ``non-career politicians'') in the 2018 midterms. From this
pool, Lears chose four female Democratic candidates --- Ocasio-Cortez,
Cori Bush, Paula Jean Swearengin and Amy Vilela --- each with an
appealing back story, a seemingly unbeatable opponent and a progressive
platform. (Lears and her husband, Robin Blotnick, wrote and produced
``Knock Down the House''; she served as its cinematographer while he
took on the role of editor.)

The women's stories emerge piecemeal in ``Knock Down the House,'' which
follows them on the campaign trail up to election night and, in one
case, beyond. Some are more sharply delineated --- what makes them run
and why --- than others. Vilela, who ran in the Democratic primary in
Nevada, seems the most personally invested, having been driven into
politics by an agonizing family tragedy that she speaks about with raw,
heart-heavy candor. Both Bush (from Missouri) and Swearengin (West
Virginia) largely appear fed up with the establishment Democrats
representing their districts. ``We're coming out of the belly of the
beast kicking and screaming,'' Swearengin says.

Ocasio-Cortez quickly, and unsurprisingly, emerges as the focus. She's a
ready-made camera presence: sharp, young, emphatic and a tremendous,
blazingly confident public speaker. Nothing seems to throw her off her
game, whether she's smiling at passers-by oblivious to her campaign
leafleting or vigorously cleaning the clock of her very
surprised-looking opponent, Joe Crowley, who had been in office since
1999. By the 2018 midterms, Crowley --- who had originally been put on
the ballot by
\href{https://www.nytimes.com/1998/07/22/nyregion/manton-plans-to-retire-from-congress-at-end-of-year.html}{his
predecessor, Thomas J. Manton} --- was
\href{Democratic\%20Caucus\%20Chairman\%20of\%20the\%20United\%20States\%20House\%20of\%20Representatives,}{the
fourth ranking House Democrat}.

He never saw Ocasio-Cortez coming; few did, other than these filmmakers.
Because of Lears's early access to the candidates, she and her camera
seem to have been in the war rooms right from the start. Quickly and
efficiently, and with the aid of some concise onscreen text and
talking-head interviews, she sketches in how Brand New Congress and
Justice Democrats operated. A lot of this is fascinating but could be
clearer; it's hazy how the groups coordinated their efforts in 2018, how
they actually run candidates and what running someone fully entails.
There are atmospheric back-room scenes with the candidates and their
teams, but who hired whom?

The timeline jumps around a bit, not always helpfully. Ocasio-Cortez is
already running when the movie opens on her delivering an amusing gender
analysis while putting on makeup. Soon, the movie cuts to Kentucky nine
months earlier where a group of young people with laptops (one
ornamented with a ``Bernie for U.S. President'' bumper sticker) are
discussing an unnamed potential candidate. There's talk and a call for a
vote: ``All those in favor of moving her to the next round, raise your
hand.'' Cut to Corbin Trent, of the Justice Democrats, who says that the
``biggest shared goal'' of his group and Brand New Congress is
``removing the corrupting influence of money in politics.''

These two organizations, Trent continues, are offering an alternative
path to Congress, one free of lobbyists and special-interest groups.
``Right now our Congress is 81 percent men,'' Trent says. ``It's mostly
white men, it's mostly millionaires, it's mostly lawyers.'' This peek at
outsider organizations is fascinating and could easily be spun into a
separate documentary, as could the story of Jo-Ann Floyd-Whitehead, a
community organizer who
\href{https://www.nytimes.com/2009/09/04/nyregion/04race.html}{in 2008,
with her husband, helped Barack Obama win the Democratic primary}. ``The
Whiteheads are legendary,'' says Ocasio-Cortez as people pore over her
petition to get on the 2018 primary ballot. ``This is the war room for
every insurgent campaign in Queens.''

This particular insurgency, of course, was
\href{https://www.nytimes.com/2017/11/08/us/politics/democrats-women-minorities.html}{part
of a larger national shift}that ``Knock Down the House'' (and the Bernie
Sanders stickers and T-shirts) makes clear was only partly in reaction
to the Trump presidency. Lears doesn't dig into that shift as deeply as
she could (this is the rare time that you yearn for a movie to be
longer), but it's there as the candidates push forward and their races
and the documentary both develop great urgency. These women are running
for office --- while driving, walking, talking and walking some more ---
to win hearts and minds and votes. Often, it feels as if they're running
for their lives. They are also, as Ocasio-Cortez says forcefully,
``running to win.''

Advertisement

\protect\hyperlink{after-bottom}{Continue reading the main story}

\hypertarget{site-index}{%
\subsection{Site Index}\label{site-index}}

\hypertarget{site-information-navigation}{%
\subsection{Site Information
Navigation}\label{site-information-navigation}}

\begin{itemize}
\tightlist
\item
  \href{https://help.nytimes.com/hc/en-us/articles/115014792127-Copyright-notice}{©~2020~The
  New York Times Company}
\end{itemize}

\begin{itemize}
\tightlist
\item
  \href{https://www.nytco.com/}{NYTCo}
\item
  \href{https://help.nytimes.com/hc/en-us/articles/115015385887-Contact-Us}{Contact
  Us}
\item
  \href{https://www.nytco.com/careers/}{Work with us}
\item
  \href{https://nytmediakit.com/}{Advertise}
\item
  \href{http://www.tbrandstudio.com/}{T Brand Studio}
\item
  \href{https://www.nytimes.com/privacy/cookie-policy\#how-do-i-manage-trackers}{Your
  Ad Choices}
\item
  \href{https://www.nytimes.com/privacy}{Privacy}
\item
  \href{https://help.nytimes.com/hc/en-us/articles/115014893428-Terms-of-service}{Terms
  of Service}
\item
  \href{https://help.nytimes.com/hc/en-us/articles/115014893968-Terms-of-sale}{Terms
  of Sale}
\item
  \href{https://spiderbites.nytimes.com}{Site Map}
\item
  \href{https://help.nytimes.com/hc/en-us}{Help}
\item
  \href{https://www.nytimes.com/subscription?campaignId=37WXW}{Subscriptions}
\end{itemize}
