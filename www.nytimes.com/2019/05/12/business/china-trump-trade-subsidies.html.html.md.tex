Sections

SEARCH

\protect\hyperlink{site-content}{Skip to
content}\protect\hyperlink{site-index}{Skip to site index}

\href{https://www.nytimes.com/section/business}{Business}

\href{https://myaccount.nytimes.com/auth/login?response_type=cookie\&client_id=vi}{}

\href{https://www.nytimes.com/section/todayspaper}{Today's Paper}

\href{/section/business}{Business}\textbar{}U.S.-China Trade Talks
Stumble on Beijing's Spending at Home

\url{https://nyti.ms/2YtKkrD}

\begin{itemize}
\item
\item
\item
\item
\item
\item
\end{itemize}

Advertisement

\protect\hyperlink{after-top}{Continue reading the main story}

Supported by

\protect\hyperlink{after-sponsor}{Continue reading the main story}

\hypertarget{us-china-trade-talks-stumble-on-beijings-spending-at-home}{%
\section{U.S.-China Trade Talks Stumble on Beijing's Spending at
Home}\label{us-china-trade-talks-stumble-on-beijings-spending-at-home}}

\includegraphics{https://static01.nyt.com/images/2019/05/03/business/00CHINASUBSIDIES-1/00CHINASUBSIDIES-1-articleLarge-v2.jpg?quality=75\&auto=webp\&disable=upscale}

By \href{https://www.nytimes.com/by/keith-bradsher}{Keith Bradsher} and
\href{https://www.nytimes.com/by/ana-swanson}{Ana Swanson}

\begin{itemize}
\item
  May 12, 2019
\item
  \begin{itemize}
  \item
  \item
  \item
  \item
  \item
  \item
  \end{itemize}
\end{itemize}

\href{https://cn.nytimes.com/business/20190513/china-trump-trade-subsidies/}{阅读简体中文版}\href{https://cn.nytimes.com/business/20190513/china-trump-trade-subsidies/zh-hant/}{閱讀繁體中文版}

BEIJING --- One year ago, when he began a multibillion-dollar trade war
with China that shook the global economy, President Trump
\href{https://www.nytimes.com/2018/09/19/us/politics/trump-china-trade-war.html}{demanded
that Beijing end} lavish government spending aimed at making the country
a world power in computer chips, robotics, commercial aircraft and other
industries of the future.

Today, as the two sides struggle to reach a truce, the Trump
administration is finding just how difficult that will be.

Trade talks between the United States and China nearly ground to a halt
this past week, and a seemingly intractable dispute over subsidies is a
big part of it. Robert E. Lighthizer, the United States trade
representative,
\href{https://www.nytimes.com/2019/05/06/us/politics/trump-tariffs-china.html}{accused
China last Monday of reneging} on what he described as ``good, firm
commitments on eliminating market-distorting subsidies.'' Vice Premier
Liu He, the leader of China's negotiating team, said that it was normal
for negotiations to have ups and downs, but has also nodded to the
subsidies issue in vowing repeatedly over the last several days not to
bend on China's principles.

President Trump on Friday raised tariffs on \$200 billion a year worth
of Chinese goods, hitting goods leaving China's shores as of that day.
He has directed Mr. Lighthizer to start on Monday the long process for
raising tariffs on all Chinese goods.

In talks and in an exchange of documents, Chinese negotiators surprised
their American counterparts by calling at the start of this month for
numerous changes, people familiar with the negotiations said. While the
requests covered everything from intellectual property to currency
manipulation, the hardened Chinese stance against limiting government
subsidies poses a particular challenge.

The United States wants China to enshrine limits on subsidies in its
national laws. China says it will not let a foreign country tell it how
to change its laws. A schedule of planned legislation released by
Chinese officials on Saturday did not include any of the subsidy-related
measures that Washington has sought.

Beijing has long helped its homegrown industries in strategically
important areas like jetliners and parts for nuclear reactors. It also
supports efforts to build up China's high-tech industries like
microchips and self-driving cars to make sure the economy will stay
competitive.

Stopping, or even tracking, China's subsidies is a difficult task. Many
subsidies take the form of cheap loans from government-controlled banks
or through other opaque arrangements. Foreign companies also complain
that they are often shut out of local government contracts through
written and unwritten rules, giving Chinese competitors a strong base at
home while they pursue global expansion plans.

\includegraphics{https://static01.nyt.com/images/2019/05/03/business/00CHINASUBSIDIES-2/merlin_146208684_2d8ceae7-ce5f-4881-924b-77eef9a0c2a2-articleLarge.jpg?quality=75\&auto=webp\&disable=upscale}

China has agreed to disclose more information about its subsidies and
stop those that violate rules under the World Trade Organization, the
global trade referee. But the two sides are also at loggerheads over how
to interpret those W.T.O. rules, said people familiar with the talks,
who asked for anonymity because they were not authorized to speak
publicly.

In his news briefing last Monday, Mr. Lighthizer said China's trade
negotiators had made significant, enforceable commitments to the United
States, but added that ``some people'' in China had objected to them,
without saying who. China's trade negotiators are heavily drawn from the
ranks of the country's market-oriented economic reformers and have long
been at odds with officials who want greater reliance on heavily
subsidized state-owned enterprises.

The Trump administration insists on leaving in place tariffs on imports
from heavily subsidized Chinese industries, at least for this year. That
would protect the American market in industries that trade hawks within
the administration see as strategically crucial.

Chinese officials oppose those tariffs. Mr. Liu told Chinese
state-controlled media on Saturday that the Chinese government
``believes that tariffs are the starting point for trade disputes
between the two sides --- if an agreement is to be reached, the tariffs
must all be canceled.''

Chad Bown, a senior fellow at the Peterson Institute for International
Economics, said that tariffs imposed bilaterally were a poor tool to
address a global problem like overcapacity. Even if the United States
successfully kept part of the tariffs in place, they would protect only
American business at home. Subsidized Chinese business could still
compete at home, in Europe and almost everywhere else around the globe,
hurting prospects for American exporters.

In the United States, Democrats have been increasingly critical of the
Trump administration for not obtaining more trade policy concessions.
Yet even some Democrats said they saw limited prospects that China will
agree to reduce subsidies.

``To expect the end of essentially a planned or a centralized economy
would be awfully ambitious,'' Senator Chris Coons, Democrat of Delaware,
said in a recent interview in Beijing.

``To be fair the Obama administration got nowhere, the Bush
administration got nowhere,'' Derek Scissors, a resident scholar at the
American Enterprise Institute, said about convincing China to roll back
its subsidies. ``This is a crucial way the Chinese run their economy.''

If a trade deal does not fully cover subsidies, the United States could
resort to unconventional responses. For example, the United States has
pushed for an extensive revision of its laws
\href{https://www.nytimes.com/2018/10/10/business/us-china-investment-cfius.html}{surrounding
foreign investments and exports of high-tech products}, primarily aimed
at China, to try to preserve its commercial and military edge.

Image

President Trump with Liu He, the lead Chinese trade negotiator, last
month. Trade talks between the United States and China nearly ground to
a halt this past week.Credit...Sarah Silbiger/The New York Times

The Trump administration has made some progress in the emerging trade
deal on other ways the Chinese government props up its industries.
Beijing has promised to tell its state-controlled banks to show less
favoritism in lending to state-owned enterprises instead of private
sector businesses. Beijing has also pledged to open up the bidding for
government contracts to foreign companies, instead of reserving them
almost completely for Chinese companies.

If China opens up the bidding, ``that would actually, genuinely move the
market needle on opportunities for foreign companies in China,'' said
Scott Kennedy, a China economic policy specialist at the Center for
Strategic and International Studies.

On the issue of subsidies, China has grown more quiet. Its ``Made in
China 2025'' plan two years ago
\href{https://www.nytimes.com/2017/03/07/business/china-trade-manufacturing-europe.html}{called
for \$300 billion in special financing and other assistance for 10
advanced manufacturing industries}. China
\href{https://www.nytimes.com/2018/12/12/business/china-trade-war.html}{shelved
the catchy name for the program} in recent months, while expressing
determination to continue investing in ``high-quality manufacturing.''

China is willing to publicly list and disclose subsidies from its
central government, people familiar with the trade talks said. But
instead of disclosing these subsidies to the United States, which might
be seen by the Chinese public as humiliating, the Chinese government
wants to disclose them through the W.T.O., which would then pass on the
list to its members.

W.T.O. rules ban governments from helping exporting companies with cash,
free land and other easily measured gifts. The rules are somewhat looser
on measures like cheap loans from state-controlled banks or efforts to
replace imports by fostering domestic production of the same goods.

Beijing has told American negotiators that it will end subsidies if they
are breaking W.T.O. rules. But the Chinese national government's
assistance to industries tends to fall into the categories that are
hardest to prove as violating W.T.O. rules.

In China, the subsidies more likely to break W.T.O. rules
\href{https://www.nytimes.com/2010/09/09/business/global/09trade.html}{tend
to be given to exporters by provincial and local government agencies in
China}. In the trade talks with the United States, Beijing has agreed to
look for provincial and local subsidies that may violate W.T.O. rules,
but has been resistant to passing legislation that would abolish them,
people familiar with the talks said.

At least a few market-oriented Chinese government officials have worried
that broad subsidies might be squandered by companies more interested in
taking the government's money than in creating competitive products. But
these critics appear to be a shrinking minority.

Lou Jiwei, a prominent advocate of economic reform and the chairman of
China's social security fund,
\href{https://www.scmp.com/news/china/diplomacy/article/2189046/chinas-tech-strategy-all-talk-no-action-and-waste-taxpayers}{told
The South China Morning Post in early March} that the Made in China 2025
plan ``wasted taxpayers' money.''

Mentions of Mr. Lou immediately disappeared from state-controlled media.
There followed a cursory statement by the official Xinhua news agency on
April 4 that
\href{http://www.xinhuanet.com/politics/2019-04/04/c_1124326653.htm}{he
had been removed from his post} at the social security fund. No reason
was given.

Advertisement

\protect\hyperlink{after-bottom}{Continue reading the main story}

\hypertarget{site-index}{%
\subsection{Site Index}\label{site-index}}

\hypertarget{site-information-navigation}{%
\subsection{Site Information
Navigation}\label{site-information-navigation}}

\begin{itemize}
\tightlist
\item
  \href{https://help.nytimes.com/hc/en-us/articles/115014792127-Copyright-notice}{©~2020~The
  New York Times Company}
\end{itemize}

\begin{itemize}
\tightlist
\item
  \href{https://www.nytco.com/}{NYTCo}
\item
  \href{https://help.nytimes.com/hc/en-us/articles/115015385887-Contact-Us}{Contact
  Us}
\item
  \href{https://www.nytco.com/careers/}{Work with us}
\item
  \href{https://nytmediakit.com/}{Advertise}
\item
  \href{http://www.tbrandstudio.com/}{T Brand Studio}
\item
  \href{https://www.nytimes.com/privacy/cookie-policy\#how-do-i-manage-trackers}{Your
  Ad Choices}
\item
  \href{https://www.nytimes.com/privacy}{Privacy}
\item
  \href{https://help.nytimes.com/hc/en-us/articles/115014893428-Terms-of-service}{Terms
  of Service}
\item
  \href{https://help.nytimes.com/hc/en-us/articles/115014893968-Terms-of-sale}{Terms
  of Sale}
\item
  \href{https://spiderbites.nytimes.com}{Site Map}
\item
  \href{https://help.nytimes.com/hc/en-us}{Help}
\item
  \href{https://www.nytimes.com/subscription?campaignId=37WXW}{Subscriptions}
\end{itemize}
