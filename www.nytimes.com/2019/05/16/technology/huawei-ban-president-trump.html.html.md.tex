Sections

SEARCH

\protect\hyperlink{site-content}{Skip to
content}\protect\hyperlink{site-index}{Skip to site index}

\href{https://www.nytimes.com/section/technology}{Technology}

\href{https://myaccount.nytimes.com/auth/login?response_type=cookie\&client_id=vi}{}

\href{https://www.nytimes.com/section/todayspaper}{Today's Paper}

\href{/section/technology}{Technology}\textbar{}Trump's Latest Move
Takes Straight Shot at Huawei's Business

\url{https://nyti.ms/2Yuc8vV}

\begin{itemize}
\item
\item
\item
\item
\item
\item
\end{itemize}

Advertisement

\protect\hyperlink{after-top}{Continue reading the main story}

Supported by

\protect\hyperlink{after-sponsor}{Continue reading the main story}

\hypertarget{trumps-latest-move-takes-straight-shot-at-huaweis-business}{%
\section{Trump's Latest Move Takes Straight Shot at Huawei's
Business}\label{trumps-latest-move-takes-straight-shot-at-huaweis-business}}

\includegraphics{https://static01.nyt.com/images/2019/05/17/business/17-HUAWEI-PRINT/16HUAWEI-2-articleLarge.jpg?quality=75\&auto=webp\&disable=upscale}

By \href{https://www.nytimes.com/by/raymond-zhong}{Raymond Zhong}

\begin{itemize}
\item
  May 16, 2019
\item
  \begin{itemize}
  \item
  \item
  \item
  \item
  \item
  \item
  \end{itemize}
\end{itemize}

\href{https://cn.nytimes.com/technology/20190517/huawei-export-controls/}{阅读简体中文版}\href{https://cn.nytimes.com/technology/20190517/huawei-export-controls/zh-hant/}{閱讀繁體中文版}

BEIJING --- The Trump administration has filed criminal charges against
Huawei for stealing technology. It has all but snuffed out the Chinese
tech giant's sales in the United States, calling the firm an espionage
threat. And it has tried to persuade other governments to do similarly.

But Washington had not taken a straight shot at Huawei's ability to do
business anywhere in the world until late Wednesday, when the Commerce
Department
\href{https://www.nytimes.com/2019/05/15/business/huawei-ban-trump.html}{announced
restrictions} on the company's access to American technology.

American companies including Qualcomm, Intel and Broadcom sell Huawei
microchips and other specialized parts that go into its smartphones and
telecom equipment. Google's Android software powers its phones. Of the
\$70 billion that Huawei spent on components and other supplies last
year, \$11 billion went to American companies, a Huawei spokesman, Joe
Kelly, said.

If Huawei is cut off from these suppliers, the effect could be
catastrophic for the millions of people who use Huawei smartphones ---
and for the mobile networks, across a wide swath of the planet, that run
on Huawei gear.

It would be ``the trade equivalent of a nuclear bomb,'' said Kevin J.
Wolf, a partner at the law firm Akin Gump Strauss Hauer \& Feld and an
assistant secretary of commerce under President Barack Obama.

Much remains unclear, however, about the scope and potential impact of
the Commerce Department's move. The department says it is putting Huawei
on its ``entity list'' of firms that need special permission to buy
American components and technology. How it decides to grant such
permissions, and how broad a range of products the policy covers, will
determine how badly Huawei's business is disrupted.

According to a
\href{https://www.federalregister.gov/documents/2019/05/21/2019-10616/addition-of-entities-to-the-entity-list}{notice
posted to the Federal Register} on Thursday, licenses for selling to
Huawei and 68 affiliated companies around the world will be reviewed
with a ``presumption of denial,'' indicating they will likely be hard to
obtain. The notice is scheduled to be officially published in the
Federal Register on Tuesday.

Given the spiraling tensions between China and the United States on
tariffs, the move against Huawei may also be short-lived.
\href{https://www.nytimes.com/2019/05/15/business/us-china-trade-war-economy.html}{Talks
to resolve the trade fight} have stalled, and both sides are digging in
their heels. The pressure is on to find common ground ahead of a
potential meeting next month between President Trump and China's top
leader, Xi Jinping, in Japan. Washington's campaign against Huawei could
become a bargaining chip.

``In every other administration, the entity listing was purely a tool of
law enforcement and national security,'' Mr. Wolf said. ``The thing to
watch is whether this will become a tool of trade policy and used as
leverage in the negotiations.''

In a statement on Thursday, Huawei said the Commerce Department's move
was ``in no one's interest.''

``It will do significant economic harm to the American companies with
which Huawei does business,'' the company said, and ``affect tens of
thousands of American jobs.''

China's Ministry of Foreign Affairs and Ministry of Commerce condemned
Washington's decision in regularly scheduled news briefings on Thursday.

``We urge the United States to stop these wrongful practices and to
create favorable conditions for normal cooperation between the two
nations' companies,'' said Gao Feng, a spokesman for China's Commerce
Ministry.

\emph{{[}Read more about the}
\href{https://www.nytimes.com/2019/05/15/business/huawei-ban-trump.html}{\emph{executive
order}} \emph{on foreign-made equipment.{]}}

Tensions between the Trump administration and Huawei escalated after
American officials arranged the arrest of Meng Wanzhou, the company's
chief financial officer and a daughter of its founder, in Canada late
last year. The company and Ms. Meng face criminal charges in the United
States in connection with alleged theft of industrial secrets and
violations of sanctions against Iran. Ms. Meng remains in Canada while
officials there decide whether she will be extradited.

Washington's action this week against Huawei puts the company in the
same position that ZTE, a much smaller Chinese rival in telecom
equipment, found itself in a few years ago.

The Commerce Department
\href{https://www.nytimes.com/2016/03/08/technology/us-restricts-sales-to-zte-saying-it-breached-sanctions.html}{added
ZTE to the entity list in 2016} after determining that it had violated
United States sanctions by selling American-made goods to Iran.
Eventually, the department relented, and ZTE
\href{https://www.nytimes.com/2017/03/07/technology/zte-china-fine.html}{agreed
to a hefty fine}. But a year later, the Commerce Department said ZTE had
failed to comply with the terms of the agreement, and American
technology companies were
\href{https://www.nytimes.com/2018/04/16/technology/chinese-tech-company-blocked-from-buying-american-components.html}{barred
outright} from selling to the company.

Cut off from American microchips and other parts, ZTE halted production
and
\href{https://www.nytimes.com/2018/05/09/technology/zte-china-us-trade-war.html}{was
near collapse} until President Trump intervened and softened the
punishment to appease the Chinese leadership.

The episode
\href{https://www.nytimes.com/2018/06/10/technology/china-technology-zte-sputnik-moment.html}{galvanized
China's government} and business community. It revealed the extent to
which the country's growing technological prowess had been built on
American know-how, and how important it was for China to innovate on its
own if its economy was to thrive.

Huawei also got a stark demonstration of the power Washington wielded
over it.

The company has since stockpiled components ``for uncertain times,'' Guo
Ping, a Huawei deputy chairman, told reporters in March. The firm has
also worked to build up a geographically diverse network of suppliers,
Mr. Guo said.

``Huawei has made sustained and deep investments over the past 30 years,
and I believe that has been of great help to Huawei's global supply,''
he said.

In particular, the company has invested for many years in producing its
own microchips, a key area in which most Chinese firms are laggards.
Sravan Kundojjala, an analyst based in Hyderabad, India, with the
technology research firm Strategy Analytics, estimates that
three-quarters of the smartphones that Huawei ships today contain chips
developed in-house.

Mr. Kundojjala acknowledges that he was skeptical when Huawei's
semiconductor unit, HiSilicon, began building its own high-end
smartphone chips.

``Initially, I thought this was not going to work out,'' he said. ``It's
maybe a pet project. Maybe they just want to play games with their
suppliers.''

Instead, HiSilicon has become a formidable asset for Huawei, with chip
technology that analysts say rivals that of market leaders such as
Qualcomm.

Yet Huawei still depends on American suppliers for enough critical
components that an all-out export ban from Washington would create a
sizable headache, even if it does not lead to near-ruin as it did for
ZTE.

``When you've got something as complicated as a router or a cellphone,
even if there's one part you're not able to get, you can't deliver,
because you don't have that widget to make the cellphone or router
function,'' Mr. Wolf, the lawyer, said.

Advertisement

\protect\hyperlink{after-bottom}{Continue reading the main story}

\hypertarget{site-index}{%
\subsection{Site Index}\label{site-index}}

\hypertarget{site-information-navigation}{%
\subsection{Site Information
Navigation}\label{site-information-navigation}}

\begin{itemize}
\tightlist
\item
  \href{https://help.nytimes.com/hc/en-us/articles/115014792127-Copyright-notice}{©~2020~The
  New York Times Company}
\end{itemize}

\begin{itemize}
\tightlist
\item
  \href{https://www.nytco.com/}{NYTCo}
\item
  \href{https://help.nytimes.com/hc/en-us/articles/115015385887-Contact-Us}{Contact
  Us}
\item
  \href{https://www.nytco.com/careers/}{Work with us}
\item
  \href{https://nytmediakit.com/}{Advertise}
\item
  \href{http://www.tbrandstudio.com/}{T Brand Studio}
\item
  \href{https://www.nytimes.com/privacy/cookie-policy\#how-do-i-manage-trackers}{Your
  Ad Choices}
\item
  \href{https://www.nytimes.com/privacy}{Privacy}
\item
  \href{https://help.nytimes.com/hc/en-us/articles/115014893428-Terms-of-service}{Terms
  of Service}
\item
  \href{https://help.nytimes.com/hc/en-us/articles/115014893968-Terms-of-sale}{Terms
  of Sale}
\item
  \href{https://spiderbites.nytimes.com}{Site Map}
\item
  \href{https://help.nytimes.com/hc/en-us}{Help}
\item
  \href{https://www.nytimes.com/subscription?campaignId=37WXW}{Subscriptions}
\end{itemize}
