Sections

SEARCH

\protect\hyperlink{site-content}{Skip to
content}\protect\hyperlink{site-index}{Skip to site index}

\href{https://www.nytimes.com/section/travel}{Travel}

\href{https://myaccount.nytimes.com/auth/login?response_type=cookie\&client_id=vi}{}

\href{https://www.nytimes.com/section/todayspaper}{Today's Paper}

\href{/section/travel}{Travel}\textbar{}Be Among the First to Visit
North Macedonia

\href{https://nyti.ms/2Jq47Vo}{https://nyti.ms/2Jq47Vo}

\begin{itemize}
\item
\item
\item
\item
\item
\end{itemize}

Advertisement

\protect\hyperlink{after-top}{Continue reading the main story}

Supported by

\protect\hyperlink{after-sponsor}{Continue reading the main story}

heads up

\hypertarget{be-among-the-first-to-visit-north-macedonia}{%
\section{Be Among the First to Visit North
Macedonia}\label{be-among-the-first-to-visit-north-macedonia}}

It is an ancient land, but as of February it has a new name, giving
travelers seeking a cool passport stamp a novel reason to visit.

\includegraphics{https://static01.nyt.com/images/2019/05/19/travel/17North-Macedonia1/17North-Macedonia1-articleLarge.jpg?quality=75\&auto=webp\&disable=upscale}

By Alex Crevar

\begin{itemize}
\item
  May 17, 2019
\item
  \begin{itemize}
  \item
  \item
  \item
  \item
  \item
  \end{itemize}
\end{itemize}

The world has a newly named country,
\href{https://www.nytimes.com/2020/07/14/world/europe/north-macedonia-election-zoran-zaev.html}{North
Macedonia}. And that is good news for regional relations and travelers,
who are visiting the southeastern corner of Europe in growing numbers.

In February, ``the former Yugoslav Republic of Macedonia'' --- as the
United Nations referred to the Balkan country during its admittance in
1993 --- officially became the Republic of North Macedonia.

For many who know this nation in the heart of the Balkan Peninsula
simply as ``Macedonia,'' this may seem like semantics. It is not.
Macedonia agreed to change its name to resolve a decades-old dispute
with neighboring Greece, and, in return, Greece said it would drop its
objection to the neighboring country's entry into the European Union and
NATO. Greece had long opposed the name ``Macedonia,'' saying it implied
territorial aspirations over the northern Greek region of the same name.

For travelers, the end of the dispute means a new passport stamp and a
novel reason to discover this nascent, yet ancient land, which is about
the size of New Hampshire and borders Greece, Bulgaria, Serbia, Kosovo
and Albania.

Dense with old-world culture, rustic gourmet cuisine, mountain chains,
remote villages, and some of the oldest and deepest lakes in Europe, the
country is a synapse connecting traditions on the crossroads between
empires --- Greek, Macedonian, Roman, Byzantine, Ottoman ---where the
Occident and Orient have long found middle ground.

\includegraphics{https://static01.nyt.com/images/2019/05/17/travel/17North-Macedonia2/17North-Macedonia2-articleLarge.jpg?quality=75\&auto=webp\&disable=upscale}

The hope is that the name change will, in part, inspire a publicity
makeover.

``What the agreement does, in my opinion, is take away our philosophical
boundaries,'' said Aleksandar Donev, Macedonia's former director of the
Agency for Promotion and Support of Tourism. ``It takes away the word
`former' from our name, and stops defining us as something we were in
the past. It sets us free to be present with a much clearer and positive
vision for our future.''

Mr. Donev is now the owner of
\href{http://www.mustseedonia.com/}{Mustseedonia}, a sustainability
consultancy and travel operation that leads eco-adventure tours. He sat
at a cafe on a bistro-lined street in the Debar Maalo neighborhood of
North Macedonia's capital, Skopje, a city with a millennia-old heart
where business meetings often turn into multicourse, three-hour lunches.

``Our physical strengths and the cultural experiences that we've been
perfecting for centuries --- like our food, wine and traditions --- have
never been in question,'' he said.

The crux of the issue between Greece and North Macedonia --- once a
republic within Yugoslavia from the end of the Second World War until
1991, when it declared independence --- stems from the fact that Greece
has its own province named Macedonia, which borders the country of North
Macedonia. The Greeks have long argued that an independent nation of the
same name on its northern frontier represented a territorial threat.

Image

The Dushan Bridge over the Vardar River in North Macedonia's capital,
Skopje.Credit...Marko Risovic for The New York Times

The accord, which quelled those territorial tensions by adding the
geographical determinant, North, was the culmination of many years of
United Nations-mediated negotiations that had intensified in recent
months amid hopes by Western governments that a breakthrough would allow
newly named North Macedonia to join the international alliances and
would stabilize the western Balkans.

Kocho Angjushev, North Macedonia's deputy prime minister, said the
country is already seeing an uptick in favorable publicity, which he
believes increases its economic potential.

Arguably, this positive surge is coming at the right time for the nation
--- just before tourism high season, which traditionally extends from
late spring into the autumn. The timing also dovetails well for
travelers to discover the Balkan region, one of the continent's
burgeoning cultural and adventure destinations.

A microcosm of the region, North Macedonia stuffs the entire gamut of
Balkan experiences into its small size. Travelers sleep in nomadic
shepherd settlements and huts during multiday hikes. They dance in
nightclubs until the early morning hours. And they stumble upon
traditional festivals in villages, where horn-and-drum rhythms pulsate
and tables overflow with grilled meats, vegetables, cheeses and breads
--- and excellent local wine.

Image

The Debar Maalo neighborhood in Skopje is known as the bohemian
quarters. Lined with restaurants and cafes, it attracts both locals and
tourists.Credit...Marko Risovic for The New York Times

A dream for do-it-yourself adventurers, North Macedonia's mountains run
down its western edge and form the border with Kosovo and then Albania
before reaching Greece. Along the string of peaks, three national parks
and the enormous Lake Ohrid dominate the landscape. In 1980, Unesco
inscribed the Ohrid region, where centuries-old Orthodox Christian
monasteries perch atop hills overlooking the water, as ``one of the
oldest human settlements in Europe.''

Further north, just off Skopje's main square, travelers can embrace the
city's historic diversity in the Ottoman-era bazaar, known as Carsija
(CHAR-she-yah). Here, cobblers, jewelers, cafes, wine bars, markets,
souvenir shops and restaurants share street space along the tangle of
flagstone pedestrian avenues winding through the district.

``My main hope and feeling is that North Macedonia's name change
represents a positive sense of freedom for locals and travelers alike,''
said Mr. Donev, the tour operator. ``By working to ease tensions with a
neighbor, we are setting a clear message and example to visitors,
citizens, the region, and the world about the power of compromise and
understanding. Love of one's country is important, but why should love
stop at a border?''

\begin{center}\rule{0.5\linewidth}{\linethickness}\end{center}

\emph{Follow} \href{https://twitter.com/nytimestravel}{\emph{NY Times
Travel on Twitter}}\emph{,}
\href{https://www.instagram.com/nytimestravel/}{\emph{Instagram}}
\emph{and}
\href{https://www.facebook.com/nytimestravel/}{\emph{Facebook}}\emph{.}
******
\href{https://www.nytimes.com/newsletters/traveldispatch?module=inline}{\emph{Get
weekly updates from our Travel Dispatch newsletter, with tips on
traveling smarter, destination coverage and photos from all over the
world.}}

Advertisement

\protect\hyperlink{after-bottom}{Continue reading the main story}

\hypertarget{site-index}{%
\subsection{Site Index}\label{site-index}}

\hypertarget{site-information-navigation}{%
\subsection{Site Information
Navigation}\label{site-information-navigation}}

\begin{itemize}
\tightlist
\item
  \href{https://help.nytimes.com/hc/en-us/articles/115014792127-Copyright-notice}{©~2020~The
  New York Times Company}
\end{itemize}

\begin{itemize}
\tightlist
\item
  \href{https://www.nytco.com/}{NYTCo}
\item
  \href{https://help.nytimes.com/hc/en-us/articles/115015385887-Contact-Us}{Contact
  Us}
\item
  \href{https://www.nytco.com/careers/}{Work with us}
\item
  \href{https://nytmediakit.com/}{Advertise}
\item
  \href{http://www.tbrandstudio.com/}{T Brand Studio}
\item
  \href{https://www.nytimes.com/privacy/cookie-policy\#how-do-i-manage-trackers}{Your
  Ad Choices}
\item
  \href{https://www.nytimes.com/privacy}{Privacy}
\item
  \href{https://help.nytimes.com/hc/en-us/articles/115014893428-Terms-of-service}{Terms
  of Service}
\item
  \href{https://help.nytimes.com/hc/en-us/articles/115014893968-Terms-of-sale}{Terms
  of Sale}
\item
  \href{https://spiderbites.nytimes.com}{Site Map}
\item
  \href{https://help.nytimes.com/hc/en-us}{Help}
\item
  \href{https://www.nytimes.com/subscription?campaignId=37WXW}{Subscriptions}
\end{itemize}
