Sections

SEARCH

\protect\hyperlink{site-content}{Skip to
content}\protect\hyperlink{site-index}{Skip to site index}

\href{https://www.nytimes.com/section/politics}{Politics}

\href{https://myaccount.nytimes.com/auth/login?response_type=cookie\&client_id=vi}{}

\href{https://www.nytimes.com/section/todayspaper}{Today's Paper}

\href{/section/politics}{Politics}\textbar{}He's One of the Biggest
Backers of Trump's Push to Protect American Steel. And He's Canadian.

\url{https://nyti.ms/2HIuhza}

\begin{itemize}
\item
\item
\item
\item
\item
\item
\end{itemize}

Advertisement

\protect\hyperlink{after-top}{Continue reading the main story}

Supported by

\protect\hyperlink{after-sponsor}{Continue reading the main story}

\hypertarget{hes-one-of-the-biggest-backers-of-trumps-push-to-protect-american-steel-and-hes-canadian}{%
\section{He's One of the Biggest Backers of Trump's Push to Protect
American Steel. And He's
Canadian.}\label{hes-one-of-the-biggest-backers-of-trumps-push-to-protect-american-steel-and-hes-canadian}}

\includegraphics{https://static01.nyt.com/images/2019/05/21/us/politics/20dc-steel-print/merlin_154242810_9514ea10-90f5-4f25-9686-8e913b95f5e9-articleLarge.jpg?quality=75\&auto=webp\&disable=upscale}

By \href{https://www.nytimes.com/by/eric-lipton}{Eric Lipton}

\begin{itemize}
\item
  May 20, 2019
\item
  \begin{itemize}
  \item
  \item
  \item
  \item
  \item
  \item
  \end{itemize}
\end{itemize}

HARROW, Ontario --- Barry Zekelman, a Canadian billionaire whose
business is mostly in the United States, is not a guy who likes to lose
--- or go slow.

On days off, he likes to race his Ferrari 488 sports cars. Or he might
climb aboard his Gulfstream IV jet to fly to the Bahamas to visit his
121-foot superyacht, which he named ``Man of Steel'' in a nod to his
role as chief executive of Zekelman Industries, North America's largest
steel-tube manufacturer.

So when Mr. Zekelman saw a chance to address his greatest frustration
--- a flood of cheap steel tube imports into the United States that was
undermining sales at his family-owned, Chicago-based company --- he went
all out to win in another intensely competitive arena: influencing
policy in Washington.

He called on well-placed connections, including a lawyer who had done
work for him and had gone on to a senior position helping oversee trade
policy in the Trump administration. He put his Washington-based lobbyist
into action, and his company took a high-profile role with a trade group
that was backing his cause. He funded his own
\href{https://www.youtube.com/watch?v=DXuG8tvi2W4}{advertising campaign}
to build public support for his efforts to protect makers of steel tube
in the United States.

And Zekelman Industries made political donations in the United States
--- skirting or possibly violating a ban on contributions by foreigners
--- including \$1.75 million last year to a group supporting President
Trump.

That lobbying effort was how he and his wife found themselves being
ushered into a private dining room at the Trump International Hotel in
Washington last spring for a small dinner with the president and his son
Donald Trump Jr. Mr. Zekelman said they discussed quotas the United
States was about to impose on imports of steel from competitors in South
Korea.

``He's attacking the problems that should have been attacked for many
years,'' Mr. Zekelman, 52, said of Mr. Trump in an interview.

\includegraphics{https://static01.nyt.com/images/2019/05/10/us/politics/00dc-steel2/merlin_154241940_87aef507-a586-41c6-9b16-8bdc342861de-articleLarge.jpg?quality=75\&auto=webp\&disable=upscale}

His status as a foreigner seeking to promote protectionist policies in
the United States makes him unusual. But Mr. Zekelman's effort amounts
to a case study in how to gain and employ access in Mr. Trump's
Washington, where an ideological commitment to aiding business meets an
open door to lobbyists, interest groups and donors --- especially those
from industries, like oil and gas, chemicals, casinos and steel, that
are strong supporters of Mr. Trump.

``The United States government has put the industry in charge of trade
policy on steel,'' said Julie C. Mendoza, a lawyer whose clients include
Borusan Mannesmann, a manufacturer of steel tube whose imports to the
United States from Turkey have prompted Zekelman Industries to lodge
protests with the administration. ``It's just not right.''

The lobbying campaign has, at times, come close to the edge of the
federal rules, including the law that prohibits foreigners from donating
to election campaigns, an examination by The New York Times found. But
it has also proved highly successful in encouraging actions that have
benefited Mr. Zekelman's company's bottom line and his American
employees.

The administration has ruled in favor of Zekelman Industries on a series
of claims the company has made against foreign competitors. Sales and
profits have surged at the privately held company, which has annual
revenues of nearly \$3 billion. Employment at the company's 14 plants in
the United States --- in Illinois, Pennsylvania, California, Ohio and
other states, which operate under names including Wheatland Tube, Sharon
Tube and Atlas Tube --- has increased by 600, and he hopes to add
another 500 jobs this year.

``In the 33 years I have been in business, I have never been more
encouraged,'' said Mr. Zekelman, who took over the business at 19, when
his father died.

Mr. Trump has won backing ---
\href{https://www.opensecrets.org/2020-presidential-race/contributors?id=N00023864}{and
political donations} --- from a number of steel companies, including
executives at Nucor, the nation's largest steel producer, and AK Steel.

But Zekelman Industries now stands out as the biggest steel industry
donor to Mr. Trump's
\href{https://www.opensecrets.org/pacs/pacgave2.php?cycle=2018\&cmte=C00637512}{affiliated
political committees}, records show.

Image

Mr. Trump speaking about trade at the Granite City Works Steel Coil
warehouse in Missouri last year. He campaigned in 2016 on helping
industries like steel.Credit...Tom Brenner for The New York Times

The \$1.75 million in contributions were delivered in three chunks last
year to \href{https://www.a1apac.org/}{America First Action SuperPAC},
which was created in January 2017 by former Trump campaign aides to push
Mr. Trump's agenda.

Federal Election Commission rules prohibit any foreigner from
``directing, dictating, controlling, or directly or indirectly
participating in the decision-making process'' related to any campaign
contribution,
\href{https://www.fec.gov/help-candidates-and-committees/taking-receipts-pac/contributions-to-super-pacs-and-hybrid-pacs/}{including
super PACs}.

Mr. Zekelman, who does not have United States citizenship, said in an
interview that he did not play a role in the decision to donate. But he
added that he did discuss the matter with other company executives,
after a representative from America First Action approached one of
Zekelman Industries' lawyers and asked for a contribution.

``They contacted our people, our people brought it to me,'' Mr. Zekelman
said. ``I said, great, I would love to find a way to support him.''

Mr. Zekelman said the donation was legal because the final decision was
made by members of his board who are American citizens or legal
residents of the United States, and the money was donated through
Wheatland Tube, a United States-based subsidiary of Zekelman Industries,
which he owns with his two brothers.

After The New York Times raised questions about the donation, Mickey
McNamara, general counsel at Zekelman Industries and president of
Wheatland Tube, said he did not recall discussing the matter with Mr.
Zekelman. Mr. McNamara said he decided to make the donation without
consulting with Mr. Zekelman.

In a statement, Brian O. Walsh, the president of America First Action,
said the organization did not accept foreign contributions. ``All
contributors are expressly asked to affirm they are a U.S. citizen or
permanent resident,'' he said.

Image

Machinery at the Atlas plant. Mr. Zekelman said he discussed quotas the
United States was about to impose on steel imports from South Korea with
Mr. Trump.Credit...Mark Felix for The New York Times

Adav Noti, a former associate general counsel at the Federal Election
Commission, said that if Mr. Zekelman had discussed the matter with
colleagues at work, he had most likely violated federal law, even if the
formal decision to donate was made by others.

``This sounds pretty clearly unlawful to me,'' said Mr. Noti, now chief
of staff at Campaign Legal Center, which monitors election law
compliance.

Mr. Zekelman's drive to get help from the Trump administration started
just 10 days after Mr. Trump took office in 2017, email records show,
and more than a year before his company made its first donation to
America First Action.

He reached out to Stephen P. Vaughn, then the acting United States trade
representative, who had been a lawyer at King \& Spalding, where his
clients included Zekelman Industries.

``I would like to bring Barry Zekelman by to meet you at 3:30 or 4 p.m.
on Wednesday, February 1,'' Bonnie Byers, a lobbyist at King \& Spalding
who still represents Zekelman Industries,
\href{https://www.documentcloud.org/documents/6015721-Zekelman-Visit-With-Acting-United-States-Trade.html\#document/p1/a501810}{wrote
to Mr. Vaughn}, her former colleague. ``Hope that will be doable. I know
how busy you are and we will not take much time.''

Three minutes later Mr. Vaughn wrote back: ``Let's plan to meet
Wednesday at 4 p.m.''

\href{https://www.documentcloud.org/documents/6015721-Zekelman-Visit-With-Acting-United-States-Trade.html\#document/p4/a501811}{Mr.
Vaughn's calendar} is blacked out for the time of the meeting. But
\href{https://www.documentcloud.org/documents/6015721-Zekelman-Visit-With-Acting-United-States-Trade.html\#document/p5/a501812}{a
log} from the agency's headquarters shows that Mr. Zekelman and Ms.
Byers signed in shortly before the scheduled meeting.

\href{https://www.documentcloud.org/documents/6015721-Zekelman-Visit-With-Acting-United-States-Trade.html\#document/p12/a501815}{Federal
ethics rules} prohibit senior administration officials from having
meetings or communications with a former employer or former client for
one year unless it is a public event ``open to all interested parties.''
White House officials said Mr. Vaughn did not receive a waiver for the
meeting. Asked about the visit, the trade representative's office said
it was ``a brief personal meeting, not a business meeting.''

Image

Mr. Zekelman with an employee at the Atlas plant. His efforts have
proved successful in encouraging the administration to take actions that
have benefited his company's bottom line.Credit...Mark Felix for The New
York Times

But Mr. Zekelman said his business message to Mr. Vaughn during that
visit was clear.

``Imports are a real issue, hurting the industry, and they have got to
be dealt with,'' he said, recalling the pitch he made to his former
lawyer. ``Find a way to dig in and get after these things and start to
hold these countries and companies accountable.''

Mr. Vaughn, who until this month served as general counsel in the trade
representative's office, is far from the only contact that Mr. Zekelman
has inside the administration.

\href{https://www.documentcloud.org/documents/4387599-Gilbert-Kaplan-Financial-Disclosure.html}{Gilbert
B. Kaplan}, a lawyer who has filed more than a dozen lawsuits or
complaints on behalf of Zekelman Industries, was named under secretary
of commerce for international trade, a role in which he helps oversee
trade negotiations related to the steel industry.

Agency records show the extensive access that Mr. Zekelman and his
executives have had to top trade officials, including Commerce Secretary
Wilbur Ross; the White House trade adviser, Peter Navarro; Robert
Lighthizer, the United States trade representative; and officials at the
Customs and Border Protection agency, which enforces trade laws.

Some of those meetings, including one with Mr. Ross last week, were
organized by a trade group called the Committee on Pipe and Tube
Imports,
\href{https://www.steel.org/-/media/doc/steel/policy/testimony/steel-caucus-march-2018/032118-steel-caucus-cpti-frabotta-testimony.ashx}{led
for the past year by Tony Frabotta}, a vice president at Zekelman
Industries. The group has also reached out to Capitol Hill for support
--- including visits to dozens of congressional offices this month ---
requests sometimes followed up with campaign contributions by the
committee,
\href{https://www.opensecrets.org/pacs/pacgot.php?cycle=2018\&cmte=C00436485}{records
show}.

Zekelman Industries also began
\href{https://www.chooseamericanmetal.com/}{its own influence campaign}
--- television and radio spots with patriotic music and shots of Mr.
Zekelman --- about the need to protect American steel makers. ``It's
time for US to show our support,'' said one of the ads, a message that
dovetails with Mr. Trump's.

The most important step the administration has taken to help the
industry is a 25 percent tariff on imports, which resulted in a surge in
sales from Zekelman's United States plants. This
\href{https://www.whitehouse.gov/presidential-actions/presidential-proclamation-adjusting-imports-steel-united-states/}{came
in March 2018}, weeks before Zekelman Industries wrote its first check
to America First Action, for \$1 million.

Image

The most important step the administration has taken to help the steel
industry is a 25 percent tariff on imports, which resulted in a surge in
sales at Zekelman Industries.Credit...Mark Felix for The New York Times

The tariff was followed by a
\href{https://www.cbp.gov/trade/quota/bulletins/qb-18-137-absolute-quota-steel-mill-articles-argentina-brazil-and-south-korea}{cap
on steel imports} from three countries, including South Korea, which had
been a major competitor for Mr. Zekelman.

The administration has also been moving aggressively on complaints that
companies are trying to import goods from nations like Thailand, Turkey,
Vietnam and the United Arab Emirates where manufacturing is government
subsidized or gets other unfair benefits, said Roger B. Schagrin, a
lawyer
\href{https://soprweb.senate.gov/index.cfm?event=getFilingDetails\&filingID=50FA9571-88E2-4AA1-8EFB-8464A0AD7534\&filingTypeID=51}{and
lobbyist} who represents Mr. Zekelman and other American-based steel
tube makers.

``Whack-a-mole'' is how Mr. Zekelman described his effort to head off
subsidized imports, given that as soon as he managed to get punitive
action against one competitor, another emerged.

Steel tube imports from Turkey have been a particular target.

One case involved a request to the administration by Borusan Mannesmann
to import steel tube tariff free from Turkey while it completed the
expansion of a steel pipe mill in Texas. Zekelman Industries objected,
telling the Commerce Department that a Zekelman plant in Arkansas could
make the steel Borusan needed.

The International Trade Administration, a part of the Commerce
Department overseen by Mr. Kaplan, the former Zekelman lawyer
\href{https://soprweb.senate.gov/index.cfm?event=getFilingDetails\&filingID=6279C6F2-E904-47FA-8530-AC4C66610EFD\&filingTypeID=73}{and
lobbyist}, sided with Zekelman, recommending that Borusan's request be
denied,
\href{https://www.documentcloud.org/documents/6003752-BIS-June-19-2018-Decision-Memo-BIS-2018-0006-0034.html}{documents
show}. A spokeswoman said Mr. Kaplan had honored all ethics
requirements, which would prohibit him from helping a recent client.

Mr. Trump
\href{https://www.nytimes.com/2018/08/10/us/politics/trump-turkey-tariffs-currency.html}{announced
last summer} that he would double tariffs on steel from Turkey to 50
percent. Those imports --- including steel tube ---
\href{https://www.whitehouse.gov/presidential-actions/proclamation-adjusting-imports-steel-united-states/}{dropped
48 percent last year} as a result, a drop so big that Mr. Trump moved
last week to return the tariff to 25 percent.

Mr. Zekelman said he played no role in asking for the temporary doubling
of the tariff on Turkey, but each of these moves helped his business.
The company is also making structural steel pipe for use in the parts of
the border wall Mr. Trump intends to build with emergency government
funds.

On Friday, the administration
\href{https://www.nytimes.com/2019/05/17/business/tariffs-metals-canada-mexico.html}{reached
a deal} to eliminate the 25 percent tariffs on steel imports from Canada
and Mexico, while leaving them in place for most of the rest of the
world. The decision will benefit Mr. Zekelman, whose biggest
manufacturing plant is in Harrow, Ontario, meaning the company will now
be able to ship Canadian-made pipe into the United States tariff free,
even as Mr. Zekelman advocates protection for his United States plants.

The success of his tactics has not gone unnoticed by competitors. The
American-Turkish Council, whose sponsors include Borusan Mannesmann,
moved its annual meeting from the Ritz-Carlton hotel to the Trump
International Hotel in Washington. Among the guests at
\href{http://atctaikconference.com/}{the event last month} were Mr.
Ross, the commerce secretary.

``What else are you going to do?'' said Ms. Mendoza, the lawyer who has
represented Borusan. ``Everybody is trying to play the game.''

Advertisement

\protect\hyperlink{after-bottom}{Continue reading the main story}

\hypertarget{site-index}{%
\subsection{Site Index}\label{site-index}}

\hypertarget{site-information-navigation}{%
\subsection{Site Information
Navigation}\label{site-information-navigation}}

\begin{itemize}
\tightlist
\item
  \href{https://help.nytimes.com/hc/en-us/articles/115014792127-Copyright-notice}{©~2020~The
  New York Times Company}
\end{itemize}

\begin{itemize}
\tightlist
\item
  \href{https://www.nytco.com/}{NYTCo}
\item
  \href{https://help.nytimes.com/hc/en-us/articles/115015385887-Contact-Us}{Contact
  Us}
\item
  \href{https://www.nytco.com/careers/}{Work with us}
\item
  \href{https://nytmediakit.com/}{Advertise}
\item
  \href{http://www.tbrandstudio.com/}{T Brand Studio}
\item
  \href{https://www.nytimes.com/privacy/cookie-policy\#how-do-i-manage-trackers}{Your
  Ad Choices}
\item
  \href{https://www.nytimes.com/privacy}{Privacy}
\item
  \href{https://help.nytimes.com/hc/en-us/articles/115014893428-Terms-of-service}{Terms
  of Service}
\item
  \href{https://help.nytimes.com/hc/en-us/articles/115014893968-Terms-of-sale}{Terms
  of Sale}
\item
  \href{https://spiderbites.nytimes.com}{Site Map}
\item
  \href{https://help.nytimes.com/hc/en-us}{Help}
\item
  \href{https://www.nytimes.com/subscription?campaignId=37WXW}{Subscriptions}
\end{itemize}
