Sections

SEARCH

\protect\hyperlink{site-content}{Skip to
content}\protect\hyperlink{site-index}{Skip to site index}

\href{https://www.nytimes.com/section/politics}{Politics}

\href{https://myaccount.nytimes.com/auth/login?response_type=cookie\&client_id=vi}{}

\href{https://www.nytimes.com/section/todayspaper}{Today's Paper}

\href{/section/politics}{Politics}\textbar{}Trump Officials Prepare to
Bypass Congress to Sell Weapons to Gulf Nations

\url{https://nyti.ms/2HOfkMe}

\begin{itemize}
\item
\item
\item
\item
\item
\end{itemize}

Advertisement

\protect\hyperlink{after-top}{Continue reading the main story}

Supported by

\protect\hyperlink{after-sponsor}{Continue reading the main story}

\hypertarget{trump-officials-prepare-to-bypass-congress-to-sell-weapons-to-gulf-nations}{%
\section{Trump Officials Prepare to Bypass Congress to Sell Weapons to
Gulf
Nations}\label{trump-officials-prepare-to-bypass-congress-to-sell-weapons-to-gulf-nations}}

\includegraphics{https://static01.nyt.com/images/2019/05/22/world/middleeast/160816-crater-2-Abduljabbar-Zeyad-Reuters/160816-crater-2-Abduljabbar-Zeyad-Reuters-videoSixteenByNine3000.jpg}

By \href{https://www.nytimes.com/by/edward-wong}{Edward Wong},
\href{https://www.nytimes.com/by/catie-edmondson}{Catie Edmondson} and
\href{https://www.nytimes.com/by/eric-schmitt}{Eric Schmitt}

\begin{itemize}
\item
  May 23, 2019
\item
  \begin{itemize}
  \item
  \item
  \item
  \item
  \item
  \end{itemize}
\end{itemize}

WASHINGTON --- The Trump administration is preparing to circumvent
Congress to allow the export to Saudi Arabia and the United Arab
Emirates of billions of dollars of munitions that are now on hold,
according to current and former American officials and legislators
familiar with the plan.

\href{https://www.nytimes.com/2019/02/24/us/politics/secretary-of-state-mike-pompeo.html}{Secretary
of State Mike Pompeo} and some political appointees in the State
Department are pushing for the administration to invoke an emergency
provision that would allow President Trump to prevent Congress from
halting the sales, worth about \$7 billion. The transactions, which
include precision-guided munitions and combat aircraft, would infuriate
lawmakers in both parties.

They would also further inflame tensions between the United States and
Iran, which views Saudi Arabia as its main rival and has been supporting
the
\href{https://www.nytimes.com/interactive/2018/10/31/magazine/yemen-war-saudi-arabia.html}{Houthi
rebels in Yemen} in their campaign against a Saudi-led military
coalition that includes the United Arab Emirates.

American legislators from both parties remain incensed by the Trump
administration's equivocal response to the grisly killing last October
by Saudi agents of
\href{https://www.nytimes.com/2018/10/16/world/middleeast/khashoggi-saudi-prince.html}{Jamal
Khashoggi}, a Washington Post columnist and Virginia resident. They are
also frustrated by the administration's role in supporting the Saudi-led
coalition in the
\href{https://www.nytimes.com/interactive/2018/10/26/world/middleeast/saudi-arabia-war-yemen.html}{Yemen
war}, a four-year conflict that the United Nations has deemed the
world's worst humanitarian crisis, with
\href{https://www.nytimes.com/interactive/2018/10/20/world/middleeast/saudi-arabia-invisible-war-yemen.html?module=inline}{thousands
of civilians killed and millions suffering from famine}.

This spring,
\href{https://www.nytimes.com/2019/04/04/us/politics/yemen-war-end-vote.html}{both
the House and Senate approved} bipartisan legislation to cut off
military assistance to Saudi Arabia's war in Yemen using the 1973 War
Powers Act, only to see
\href{https://www.nytimes.com/2019/04/16/us/politics/trump-veto-yemen.html}{it
vetoed in April}.

Senator Marco Rubio, a Florida Republican who sits on the Foreign
Relations Committee, said that circumventing Congress on a Middle East
arms sale would be ``a big mistake,'' though he added that he would need
to see the specifics of such a deal.

``We have a gold standard for that sort of arrangement, and to violate
it for Saudi Arabia is going to open the door for it to happen in
multiple other places,'' he said.

\includegraphics{https://static01.nyt.com/images/2019/05/23/us/politics/23dc-weapons1/merlin_153324303_88b31cae-39db-4a15-bd61-a79a6ff880f6-articleLarge.jpg?quality=75\&auto=webp\&disable=upscale}

Senator Lindsey Graham, Republican of South Carolina and an outspoken
ally of the president, told reporters Thursday that he would ``not do
business as usual with the Saudis until we have a better reckoning''
with the crown prince, Mohammed bin Salman, whom American intelligence
agencies consider to be responsible for the killing of Mr. Khashoggi and
the Saudi role in the Yemen war.

No other foreign policy issue has created as large a rift between Mr.
Trump and Congress, and the move on the arms sales, which could take
place within days, would deepen the divide.
\href{https://www.nytimes.com/2019/03/30/us/politics/pompeo-christian-policy.html}{Mr.
Pompeo} would oversee the action, and the State Department is bracing
for lawmakers to stall confirmations on all State Department nominees if
it is implemented. Within the department, veteran Foreign Service
officers have strongly opposed Mr. Pompeo's position.

The proposal emerged publicly on Wednesday when Senator Christopher S.
Murphy, Democrat of Connecticut,
\href{https://www.cnn.com/2019/05/22/politics/trump-murphy-saudi-arms-deal/index.html}{criticized
it} on Twitter.

Members of Congress ordinarily are given a review period during which
they can pass legislation modifying or prohibiting a prospective arms
sale. But a provision in the
\href{https://legcounsel.house.gov/Comps/Arms\%20Export\%20Control\%20Act.pdf}{Arms
Export Control Act} allows the president to bypass congressional review
if he deems ``an emergency exists which requires the proposed sale in
the national security interest of the United States.''

``It sets an incredibly dangerous precedent that future presidents can
use to sell weapons without a check from Congress,'' Mr. Murphy said in
an interview on Thursday. ``We have the constitutional duty to declare
war and the responsibility to oversee arm sales that contravene our
national security interests. If we don't stand up to this abuse of
authority, we will permanently box ourselves out of deciding who we
should sell weapons to.''

Senator Robert Menendez of New Jersey, the top Democrat on the Foreign
Relations Committee, warned that he would ``pursue all appropriate
legislative and other means to nullify these and any planned ongoing
sales should the administration move forward in this manner.''

Mr. Menendez withheld his support last summer for a Trump administration
plan to sell precision-guided munitions to Saudi Arabia and the United
Arab Emirates, effectively blocking it.

Image

Mike Pompeo, the secretary of state, has been pushing for a declaration
of emergency based on what he says is a heightened threat against
American interests in the region from Iran.Credit...Erin Schaff/The New
York Times

Mr. Pompeo's emergency declaration would be based on what he says is a
heightened threat against American interests in the region from Iran.
Mr. Pompeo took the extraordinary step this month of
\href{https://www.nytimes.com/2019/05/15/us/politics/us-iraq-embassy-evacuation.html}{ordering
a withdrawal} of almost all American diplomats from the Baghdad embassy
and Erbil consulate in Iraq. European allies and Iraqi leaders have
\href{https://www.nytimes.com/2019/05/14/world/middleeast/trump-iran-threats.html}{expressed
skepticism} about American alarm over Iran.

Asked about the proposal, Morgan Ortagus, the main State Department
spokeswoman, said, ``We do not comment to confirm or deny potential arms
sales or transfers until Congress is formally notified.''

Tensions between the United States and Iran have soared since May 5,
when John R. Bolton, the national security adviser and an Iran hawk,
\href{https://www.nytimes.com/2019/05/05/world/middleeast/us-iran-military-threat-.html}{announced
that the White House} was ordering an aircraft carrier strike group and
bombers to speed up their movement to the Persian Gulf. In the days
afterward, American officials told reporters that they had gotten
\href{https://www.nytimes.com/2019/05/15/world/middleeast/iran-war-usa.html}{several
strands of intelligence} about potential attacks on American troops or
diplomats by Iranian forces or Arab militias with Iranian ties.

Mr. Bolton issued an expansive warning against Iran, saying that ``any
attack on United States interests or on those of our allies will be met
with unrelenting force.'' Critics of the escalation, which was supported
by Mr. Pompeo, said the Trump administration had provoked Iran by
withdrawing from a 2015 nuclear containment deal,
\href{https://www.nytimes.com/2019/04/22/world/middleeast/us-iran-oil-sanctions-.html}{reimposing
harsh sanctions} and
\href{https://www.nytimes.com/2019/04/08/world/middleeast/trump-iran-revolutionary-guard-corps.html}{designating
an arm of the Iranian military} as a terrorist organization.

Tensions in the Persian Gulf also rose this month after four oil tankers
were attacked with explosives. Two of the tankers are from Saudi Arabia,
one is from the United Arab Emirates and the fourth is from Norway; the
countries have not revealed the results of investigations, but Mr.
Pompeo said this week, without presenting evidence, that it was ``quite
possible that Iran was behind these.''

On Thursday, the acting defense secretary, Patrick Shanahan, said Mr.
Trump might send more troops to the Middle East because of the tensions
with Iran.

``The U.S. and Iran are entering into an escalatory dynamic from which
it will become increasingly difficult to escape,'' said
\href{https://www.crisisgroup.org/who-we-are/people/robert-malley-0}{Robert
Malley}, the president of the International Crisis Group, a nonprofit
that tries to defuse conflict.

Image

A billboard showing King Salman, left, and the crown prince, Mohammed
bin Salman.Credit...Aamir Qureshi/Agence France-Presse --- Getty Images

\href{https://www.rand.org/about/people/k/kaye_dalia_dassa.html}{Dalia
Dassa Kaye}, a Middle East analyst at RAND Corporation, a research
group, said, ``Pushing through arms sales at this moment would not just
escalate tensions with Congress but also with Iran, and likely undermine
peace efforts in Yemen.''

The end run around Congress would come just weeks before the White House
is
\href{https://www.nytimes.com/2019/05/19/us/politics/trump-middle-east-peace-plan.html}{expected
to unveil} a plan to end the Israeli-Palestinian conflict. Mr. Trump's
son-in-law and main Middle East adviser, Jared Kushner, is seeking
support from Saudi Arabia and other Arab nations for the plan, which
will probably include economic aid for the Palestinians but will not
address their aspirations to nationhood. King Salman of Saudi Arabia
voiced his disapproval of any White House plan after Mr. Trump
\href{https://www.nytimes.com/2017/12/06/world/middleeast/trump-jerusalem-israel-capital.html}{recognized
the contested city of Jerusalem} as the capital of Israel in December
2017.

``It was only a matter of time before the administration might try to
push back against congressional upset with Saudi over Yemen and the
Khashoggi murder to resume arms sales,'' said
\href{https://www.wilsoncenter.org/person/aaron-david-miller}{Aaron
David Miller}, a former State Department Middle East adviser and
negotiator in Democratic and Republican administrations. ``What better
justification than a semi-manufactured national security war scare with
Iran?''

Lawmakers have never successfully blocked an arms sale through the
passage of a joint resolution when the president has invoked a national
security justification, according to
\href{https://www.csis.org/people/melissa-dalton}{Melissa Dalton}, the
director of the Cooperative Defense Project at the Center for Strategic
and International Studies, and instead have typically registered their
disapproval privately.

But the invocation of the emergency provision is ``pretty unusual,'' Ms.
Dalton said, because ``normally Congress has a good sense of what is
coming. Typically there isn't any surprise or need to invoke this unless
there has been some controversy in the broader bilateral relationship.''

In Washington, citing Mr. Khashoggi's killing, a bipartisan group of
lawmakers has called for a prohibition of certain weapons sales to
Riyadh and a blanket prohibition on the refueling of Saudi-led coalition
aircraft engaged in the civil war in Yemen.

In a winding and remarkable
\href{https://www.whitehouse.gov/briefings-statements/statement-president-donald-j-trump-standing-saudi-arabia/}{statement
released} last year after the killing of Mr. Khashoggi, Mr. Trump argued
that punishing Saudi Arabia would jeopardize \$110 billion in military
sales to Boeing, Lockheed Martin, Raytheon and other military
contractors. ``If we foolishly cancel these contracts, Russia and China
would be the enormous beneficiaries --- and very happy to acquire all of
this newfound business,'' he said.

While the Pentagon and State Department
\href{https://www.vox.com/2018/8/9/17671386/yemen-airstrikes-saudi-arabia-coalition-pentagon}{have
denied} knowing whether American bombs were used in Saudi Arabia's
airstrikes in Yemen --- which have struck weddings, mosques and funerals
--- a former senior State Department official said last year that the
United States had access to records of every airstrike since the early
days of the war.

Advertisement

\protect\hyperlink{after-bottom}{Continue reading the main story}

\hypertarget{site-index}{%
\subsection{Site Index}\label{site-index}}

\hypertarget{site-information-navigation}{%
\subsection{Site Information
Navigation}\label{site-information-navigation}}

\begin{itemize}
\tightlist
\item
  \href{https://help.nytimes.com/hc/en-us/articles/115014792127-Copyright-notice}{©~2020~The
  New York Times Company}
\end{itemize}

\begin{itemize}
\tightlist
\item
  \href{https://www.nytco.com/}{NYTCo}
\item
  \href{https://help.nytimes.com/hc/en-us/articles/115015385887-Contact-Us}{Contact
  Us}
\item
  \href{https://www.nytco.com/careers/}{Work with us}
\item
  \href{https://nytmediakit.com/}{Advertise}
\item
  \href{http://www.tbrandstudio.com/}{T Brand Studio}
\item
  \href{https://www.nytimes.com/privacy/cookie-policy\#how-do-i-manage-trackers}{Your
  Ad Choices}
\item
  \href{https://www.nytimes.com/privacy}{Privacy}
\item
  \href{https://help.nytimes.com/hc/en-us/articles/115014893428-Terms-of-service}{Terms
  of Service}
\item
  \href{https://help.nytimes.com/hc/en-us/articles/115014893968-Terms-of-sale}{Terms
  of Sale}
\item
  \href{https://spiderbites.nytimes.com}{Site Map}
\item
  \href{https://help.nytimes.com/hc/en-us}{Help}
\item
  \href{https://www.nytimes.com/subscription?campaignId=37WXW}{Subscriptions}
\end{itemize}
