Sections

SEARCH

\protect\hyperlink{site-content}{Skip to
content}\protect\hyperlink{site-index}{Skip to site index}

\href{https://myaccount.nytimes.com/auth/login?response_type=cookie\&client_id=vi}{}

\href{https://www.nytimes.com/section/todayspaper}{Today's Paper}

\href{/section/opinion}{Opinion}\textbar{}The Negroni Is 100 Years Old
--- and the Perfect Cocktail for 2019

\url{https://nyti.ms/2X50D0Y}

\begin{itemize}
\item
\item
\item
\item
\item
\item
\end{itemize}

Advertisement

\protect\hyperlink{after-top}{Continue reading the main story}

\href{/section/opinion}{Opinion}

Supported by

\protect\hyperlink{after-sponsor}{Continue reading the main story}

\hypertarget{the-negroni-is-100-years-old--and-the-perfect-cocktail-for-2019}{%
\section{The Negroni Is 100 Years Old --- and the Perfect Cocktail for
2019}\label{the-negroni-is-100-years-old--and-the-perfect-cocktail-for-2019}}

It is refreshing, yet bitter.

\href{https://topics.nytimes.com/top/reference/timestopics/people/b/jennifer_finney_boylan/index.html}{\includegraphics{https://static01.nyt.com/images/2011/08/04/opinion/BOYLAN_NEW/BOYLAN_NEW-thumbLarge-v6.png}}

By
\href{https://topics.nytimes.com/top/reference/timestopics/people/b/jennifer_finney_boylan/index.html}{Jennifer
Finney Boylan}

Contributing Opinion Writer

\begin{itemize}
\item
  June 12, 2019
\item
  \begin{itemize}
  \item
  \item
  \item
  \item
  \item
  \item
  \end{itemize}
\end{itemize}

\includegraphics{https://static01.nyt.com/images/2019/06/14/opinion/12boylan-illo/dcb35dc043a54f98beab08f1b46b01f7-articleLarge.jpg?quality=75\&auto=webp\&disable=upscale}

There, there. Have a Negroni. This most delicious of bitter cocktails,
the perfect elixir for a summer afternoon, turns 100 years old this
year. You can raise your glass for a good cause, too: the end of June is
\href{https://negroniweek.com/}{Negroni Week}, when bars and restaurants
mix Negronis and Negroni variations for charity. By drinking a Negroni
at a participating venue, you can raise money for organizations from
Lambda Legal to the Colorado Water Trust, from the San Francisco-Marin
Food Bank to United Cerebral Palsy. Has there ever been a more soothing
way to raise money for a cause? I don't think so, not unless I somehow
missed Martinis for Puppy Dogs week.

Invented in 1919 by Count Camillo Negroni in Florence, Italy, the
Negroni is actually a variation on another classic cocktail, the
\href{https://www.thespruceeats.com/americano-cocktail-recipe-759279}{Americano}.
A mixture of Campari, sweet vermouth and soda water, served with a lemon
slice, the Americano was originally known as the Milano-Torino, because
of the origins of its two primary ingredients --- Campari from Milan and
Vermouth di Torino, from, well, you know. The name changed during
Prohibition, when it became a favorite of Americans on vacation in
Italy.

It was also a favorite of James Bond. The Americano, in fact, is the
first cocktail ordered in Ian Fleming's first 007 novel. The gentleman
spy disdains the thought of drinking whiskey or vodka in a French cafe.
On a sunny sidewalk, Fleming writes, ``Bond always had the same thing
--- an Americano.''

It was 100 years ago that Count Negroni asked his bartender at the
Cassoni Cafe on the Via de' Tornabuoni to stiffen his Americano by
replacing the soda with gin. History records that the bartender, one
\href{https://www.diffordsguide.com/people/51630/bartender/fosco-scarselli}{Fosco
Scarselli}, also replaced the lemon with an orange slice. Did he add the
bitters as well? The legends do not tell.

\href{https://www.nytimes.com/newsletters/cooking}{\emph{{[}Sign up for
Sam Sifton's email newsletter, which provides recipe suggestions and
food tips.{]}}}

Orson Welles
\href{http://www.ahistoryofdrinking.com/wordpress/orson-welles-and-the-negroni/}{sang
the cocktail's praises} while he was making the film ``Black Magic'' in
Rome in 1947. ``The bitters are excellent for your liver,'' he said,
``the gin is bad for you. They balance each other.''

I like to mix them with a single jigger each of gin, Campari and sweet
vermouth, a few shakes of bitters and an orange peel. A writer friend of
mine \href{https://www.youtube.com/watch?v=w-vJpeIwVwk}{flames the
citrus peel,} which is dramatic and makes the drink more aromatic,
although I have never been able to do this without setting off the fire
alarm.

Another key element, I believe, is to serve the cocktail with a single
gigantic ice cube from an oversize ice cube tray. I got mine online. I
regret nothing.

\includegraphics{https://static01.nyt.com/images/2019/06/12/opinion/12Boylan/12Boylan-articleLarge.jpg?quality=75\&auto=webp\&disable=upscale}

Each era creates its own signature cocktail, and sometimes the drinks
that become popular during any particular decade can tell us something
about the culture that produced them. In the 1930s, for instance, some
drinks covered up the sketchy quality of the spirits during Prohibition.
My favorite Depression-era drink is probably the
\href{https://food52.com/recipes/13206-voila-l-ete-the-french-75}{French
75}, so named because consuming some early versions of it allegedly felt
like being shelled by a field gun called the Canon de 75 model 1897. In
``Casablanca,'' Yvonne and her Nazi date order French 75s at Rick's
Café. I'll let someone else parse the significance of a Nazi ordering a
drink named after a French artillery weapon, although Claude Rains, as
Captain Renault, does note that Yvonne ``may constitute an entire second
front.''

In the 1950s and '60s, drinks turned sweeter; the iconic cocktail of
that era might be the brandy Alexander, a potent potable that we've
bizarrely come to think of as feminine. Mary Richards drinks one in the
pilot of ``The Mary Tyler Moore Show''; years later, we see Peggy Olson
drink one in Season 1 of ``Mad Men.'' (The writer Sarah Baird observes
that by the penultimate episode of the series, Peggy is drinking Scotch
just like her male counterparts, and it's hard not to view this as
progress.)

In the 1980s, I drank
\href{https://www.bonappetit.com/recipe/bas-best-pina-colada}{piña
colada} --- because I could, I suppose; and in the 2000s I found my way
to \href{http://www.drinksmixer.com/drink234.html}{Cosmopolitans}(thank
you, ``Sex and the City'') and
\href{https://www.foodnetwork.com/recipes/mojito-recipe0-1939252}{mojitos}.
Those are sweet drinks, too, although they were a significant step up
from the first drink I ever ordered at a bar: the, ahem,
\href{https://www.thespruceeats.com/sombrero-drink-recipe-759459}{Kahlua
Sombrero}.

In the end, though, all cocktails aspire to the martini. It is hard to
argue with Ogden Nash's observation: ``There is something about a
Martini, / A tingle remarkably pleasant / A yellow, a mellow Martini / I
wish I had one at present.''

Martinis make me crazy, though; I always want a second after I've had
the first. But the second one is rarely a good idea. As James Thurber
once said, ``Two are too many, and three are not enough.''

And so I turn to the Negroni for solace on a summer afternoon. If you
feel that the Aperol spritz is
\href{https://www.nytimes.com/2019/05/09/dining/drinks/aperol-spritz.html}{not
a good drink} --- and the internet has been abuzz with this very
question this spring --- why not try a Negroni, especially now during
Negroni Week? You will raise money for a cause and, who knows, perhaps
discover something new.

Is it too bitter, you ask?

Given the age we now live in, with its fury and its noise, I feel it is
exactly bitter enough.

Cheers.

\textbf{Update, June 17}: The generous and impassioned response to this
column reminded me a little bit of the line from ``The Man Who Shot
Liberty Valance,'' in which a newspaper editor tells Jimmy Stewart,
``When the legend becomes fact, print the legend.'' One of the delights
of writing a piece like this is in hearing from Negroni drinkers with
stories of their own.

There is, first and foremost, no unanimity on the perfect ratio of its
ingredients. While the standard recipe calls for a ratio of 1:1:1
between the gin, Campari and sweet vermouth, many Negroniati insist that
the gin should outweigh the other ingredients, although by how much is a
matter of some debate. Other readers question whether additional bitters
are necessary at all, given that the Campari already provides this
flavor.

Given the age in which we live in, my own opinion is that more
bitterness is not inappropriate.

I filed this column while I was on assignment in Hobart, Tasmania, where
an Australian bartender at a restaurant in the harbor (at which the
Americano was the house special) counseled me passionately against
flaming an orange peel in a Negroni --- an unquestionably theatrical
presentation, he admitted, but adding an unnecessary smokiness to the
taste. Personally, I love the theater of the flamed peel, but never
bother with it at home, not least because I dislike setting myself on
fire.

The actual genesis of the Negroni, too, is a subject that threatens
fisticuffs among some enthusiasts. While Count Negroni, in consort with
his bartender, Fosco Scarselli, is traditionally credited with the
invention, there is a counter-story that the drink was in fact created
in Senegal in 1870 by one General Pascal Olivier Comte de Negroni; an
article in the publication Drinking Cup is headlined, ``New Evidence the
Negroni Was Invented in Africa --- Sorry Italy.'' As James Thurber liked
to say, You could look it up.

Then there's the question of the Americano. I wrote that the Americano
derived from an older drink, called the Milano-Torino. But the cocktail
writer
\href{https://www.thedailybeast.com/the-history-of-how-the-negroni-conquered-america}{David
Wondrich holds} that their relationship is not chronological; rather, he
wrote in an email, ``the Americano is the generic, while the
Milano-Torino a specific variation.'' Mr. Wondrich also argues that the
name ``Americano'' came not out of Americans vacationing in Italy during
Prohibition --- an origin story widely recounted at
\href{https://www.saveur.com/article/Recipes/Americano}{Saveur},
\href{https://punchdrink.com/recipes/americano/}{Punch},
\href{http://imbibemagazine.com/recipe-americano/}{Imbibe} magazine and
\href{https://drinkstraightup.com/2013/01/20/americano/}{elsewhere} ---
but rather, according to his research, from the ``American practice of
adding bitters to vermouth to make a Vermouth Cocktail (attested as
early as 1868).'' He notes that author
\href{https://www.amazon.com/Vermouth-Torino-Monografia-Incisioni-Fototipiche/dp/1359865985}{Arnaldo
Strucchi} described this in a work entitled ``Il vermouth di Torino,''
published in 1907.

Tradition holds that Count Negroni asked his bartender to replace the
soda with gin, a story repeated in articles in
\href{https://www.esquire.com/food-drink/drinks/recipes/a3683/negroni-drink-recipe/}{Esquire,}
the
\href{https://guide.michelin.com/us/en/washington/washington-dc/article/features/negroni-week-guide-recipe}{Michelin
guide} and Food and Wine, among others. Mr. Wondrich, however, notes
that the default Negroni recipe in Italy ``does not replace soda with
gin, but adds it.'' Go figure.

What struck me above all in the wake of this column was how many
variations there are on the Negroni, and how devoted some readers are to
their particular twists. A Pink Negroni adds Lillet Blanc, lemon juice
and a sprig of tarragon; a Boulevardier replaces the gin with bourbon.
There are many others.

On the day that this column ran, I returned to my New York apartment
late, absolutely deranged from the 20-hour plane ride home from
Australia. To celebrate, I set about making myself a Negroni --- only to
find that I was out of sweet vermouth. Somewhat resentfully I made the
drink instead with the dry vermouth I keep chilled in the refrigerator
for martinis. The result was crisp and meditative --- and is, of course,
the recipe for another version of the cocktail---the Cardinale.

In no time at all I found myself reading all about the legend concerning
that drink; there is disagreement about whether it sprang to life at
Harry's Bar in Venice, or at the Orum Bar in a Roman hotel then called
The Excelsior. Difford's Guide has the whole story, which --- you will
be shocked to learn --- is complicated.

All of these contradictory legends are part of the delight of cocktails:
their flavors, their history, and the solace they can bring. There,
there. Have a Negroni, and enjoy!

\emph{The Times is committed to publishing}
\href{https://www.nytimes.com/2019/01/31/opinion/letters/letters-to-editor-new-york-times-women.html}{\emph{a
diversity of letters}} \emph{to the editor. We'd like to hear what you
think about this or any of our articles. Here are some}
\href{https://help.nytimes.com/hc/en-us/articles/115014925288-How-to-submit-a-letter-to-the-editor}{\emph{tips}}\emph{.
And here's our email:}
\href{mailto:letters@nytimes.com}{\emph{letters@nytimes.com}}\emph{.}

\emph{Follow The New York Times Opinion section on}
\href{https://www.facebook.com/nytopinion}{\emph{Facebook}}\emph{,}
\href{http://twitter.com/NYTOpinion}{\emph{Twitter (@NYTopinion)}}
\emph{and}
\href{https://www.instagram.com/nytopinion/}{\emph{Instagram}}\emph{.}

Advertisement

\protect\hyperlink{after-bottom}{Continue reading the main story}

\hypertarget{site-index}{%
\subsection{Site Index}\label{site-index}}

\hypertarget{site-information-navigation}{%
\subsection{Site Information
Navigation}\label{site-information-navigation}}

\begin{itemize}
\tightlist
\item
  \href{https://help.nytimes.com/hc/en-us/articles/115014792127-Copyright-notice}{©~2020~The
  New York Times Company}
\end{itemize}

\begin{itemize}
\tightlist
\item
  \href{https://www.nytco.com/}{NYTCo}
\item
  \href{https://help.nytimes.com/hc/en-us/articles/115015385887-Contact-Us}{Contact
  Us}
\item
  \href{https://www.nytco.com/careers/}{Work with us}
\item
  \href{https://nytmediakit.com/}{Advertise}
\item
  \href{http://www.tbrandstudio.com/}{T Brand Studio}
\item
  \href{https://www.nytimes.com/privacy/cookie-policy\#how-do-i-manage-trackers}{Your
  Ad Choices}
\item
  \href{https://www.nytimes.com/privacy}{Privacy}
\item
  \href{https://help.nytimes.com/hc/en-us/articles/115014893428-Terms-of-service}{Terms
  of Service}
\item
  \href{https://help.nytimes.com/hc/en-us/articles/115014893968-Terms-of-sale}{Terms
  of Sale}
\item
  \href{https://spiderbites.nytimes.com}{Site Map}
\item
  \href{https://help.nytimes.com/hc/en-us}{Help}
\item
  \href{https://www.nytimes.com/subscription?campaignId=37WXW}{Subscriptions}
\end{itemize}
