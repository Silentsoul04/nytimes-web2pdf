Sections

SEARCH

\protect\hyperlink{site-content}{Skip to
content}\protect\hyperlink{site-index}{Skip to site index}

\href{https://myaccount.nytimes.com/auth/login?response_type=cookie\&client_id=vi}{}

\href{https://www.nytimes.com/section/todayspaper}{Today's Paper}

\href{/section/opinion}{Opinion}\textbar{}I'm a Judge. Here's How
Surveillance Is Challenging Our Legal System.

\url{https://nyti.ms/2XeHFoq}

\begin{itemize}
\item
\item
\item
\item
\item
\item
\end{itemize}

Advertisement

\protect\hyperlink{after-top}{Continue reading the main story}

\href{/section/opinion}{Opinion}

Supported by

\protect\hyperlink{after-sponsor}{Continue reading the main story}

\hypertarget{im-a-judge-heres-how-surveillance-is-challenging-our-legal-system}{%
\section{I'm a Judge. Here's How Surveillance Is Challenging Our Legal
System.}\label{im-a-judge-heres-how-surveillance-is-challenging-our-legal-system}}

Prosecutors have stepped into the void left by Congress's failure to say
how far the police can go in using investigative technology.

By James Orenstein

Mr. Orenstein is a federal magistrate judge.

\begin{itemize}
\item
  June 13, 2019
\item
  \begin{itemize}
  \item
  \item
  \item
  \item
  \item
  \item
  \end{itemize}
\end{itemize}

\includegraphics{https://static01.nyt.com/images/2019/06/13/opinion/13orenstein/e13bb5907ae042d49fbf657ae600c5f8-articleLarge.jpg?quality=75\&auto=webp\&disable=upscale}

On most weekdays in the federal courthouse in Brooklyn, prosecutors will
ask the magistrate judge on duty to issue lots of sealed orders
authorizing them to use all sorts of investigative technologies or
requiring technology companies to keep tech-based searches secret.

But that typically won't happen when I'm the judge on duty. When it's my
turn, the docket gets awfully quiet as prosecutors wait for another
judge. That's not because the prosecutors or other judges are doing
something they shouldn't. It's because prosecutors think they'll stand a
better chance of getting what they want from another judge. This waiting
game is a symptom of how new surveillance technologies are challenging a
legal system that hasn't figured out how to handle them. (The views here
are my own, not those of the federal courts.)

Congress is way behind in determining how far the police can go in using
technology to invade people's privacy, and many of the legal disputes
arising from this collision have not reached the Supreme Court. For the
public, as a practical matter, the rules of the road are being decided
by prosecutors. Your privacy is not their highest priority.

Here's an example. Last summer, the Supreme Court decided in
\href{https://supreme.justia.com/cases/federal/us/585/16-402/}{Carpenter
v. United States} that when the police want to get records of a mobile
phone's prior locations over extended periods --- basically, to track a
suspect's whereabouts at every moment over weeks or months in the past
--- the Constitution requires them to persuade a judge to issue a
warrant based on probable cause. But for several years before that,
savvy prosecutors were able to proceed without a warrant by seeking out
judges they thought would rule their way. This tactic helped delay the
legal question from reaching the Supreme Court for years.

The precise issue in Carpenter is now settled law nationwide, but there
are plenty of other questions that remain up in the air. In what
circumstances can the police use
\href{https://www.nytimes.com/2019/06/11/opinion/police-dna-warrant.html?action=click\&module=Opinion\&pgtype=Homepage}{genetic
genealogy} to identify suspects? Can authorities compel companies to
decrypt devices or use malware to overcome data anonymization used to
protect privacy? The judiciary has yet to resolve these and other
questions that we can anticipate now; more exotic controversies
undoubtedly await.

Unlike traditional court cases, in which there are two parties and the
loser can always seek a higher court's review, requests to use
investigative technologies are one-sided. The prosecutor presents her
arguments and evidence with no one to respond. That's already an edge,
but that advantage increases for a couple of reasons.

First, the person whose privacy is at stake doesn't even know about the
surveillance order until much later, if ever, because such orders are
usually secret. If that person is charged and later learns about it,
there are reasons no effective challenge to the order's legality will be
possible, including the ``good faith'' doctrine --- which allows
prosecutors to use evidence if it was obtained in good-faith reliance on
a magistrate's order --- and the many pressures imposed on defendants
that result in well over 95 percent of them pleading guilty without
testing the evidence against them. Second, federal prosecutors go to
judges they expect to rule their way. If a judge denies the request,
it's up to the prosecutor to appeal to an appellate court, whose ruling
would become law in the several states in its judicial circuit.

So, for example, when a district judge upheld my 2010 decision that
historical location tracking requires a warrant, prosecutors decided not
to appeal. They just took their surveillance requests to other judges.
In contrast, when a colleague in Texas
\href{https://www.eff.org/cases/fifth-circuit-cell-phone-tracking-case}{ruled
similarly} on location tracking, prosecutors there apparently liked
their chances of winning a reversal and mounted a successful appeal. The
result: The law was set in Texas as well as in Mississippi and
Louisiana, the other states in that judicial circuit, allowing
prosecutors to get the information without a warrant, while in New York,
the law remained unsettled.

Although I grant requests from prosecutors far more often than I deny
them, I've written several opinions disagreeing with them on the use of
investigative technology. As a former prosecutor, I understand their
preference that other judges consider requests on surveillance
technologies. My goal in deciding these issues (and the goal of some
like-minded colleagues) is not to say no to the government but to ask
the right question. It's not enough to search laws written before modern
technology and find one that comes closest to fitting today's facts. The
question is whether any law allows a judge to issue the order the
government wants. If no such law exists, society needs to make
deliberate choices about how best to balance the promise of more
efficient investigative technologies against the risks to personal
liberty.

Those decisions are best made in Congress, but if Congress fails to do
so, judges should at least hear opposing views and give a public account
of the reasoning behind their decision. These choices should not be left
to the secret deliberation of a judge, handpicked by prosecutors, who
sits on the lowest tier of the judiciary.

And ultimately, that's the problem: A Congress that has failed to keep
pace with the times, not prosecutors aggressively using new
technological tools. Congress has not enacted any thorough updates of
the
\href{https://www.brookings.edu/blog/techtank/2019/01/07/will-this-new-congress-be-the-one-to-pass-data-privacy-legislation/}{digital
privacy laws} that govern law enforcement investigations since the early
days of the internet --- long before we entrusted virtually every bit of
information about our lives to our electronic devices and to the cloud.
(That's why the Justice Department relied on a law written in 1789 when
it tried to force Apple to help search an iPhone by disabling the
device's password protection. I ruled against the government in that
case.)

It is impossible to measure the cost in privacy losses and years spent
behind bars suffered by people who could have successfully raised the
issue decided in the Carpenter case if it hadn't taken so many years to
reach the Supreme Court. It is likewise impossible to know how many
investigations and prosecutions have been stymied because prosecutors
lacked clear rules and could not take the risk that a particular
investigative technology could undermine an important case.

If Congress won't write laws for this century's technology, courts must
craft rules that ensure a fair and orderly review of new investigative
methods. For example, the Foreign Intelligence Surveillance Court (which
also confronts the tension between effective investigations and privacy)
has a system for bringing in
\href{https://www.lawfareblog.com/amici-curiae-fisc-announced}{independent}\href{https://www.lawfareblog.com/amici-curiae-fisc-announced}{lawyers}
called ``amici curiae'' to argue novel or significant legal issues that
occasionally arise when the government asks for technology-based
surveillance orders. Those amici can argue in favor of the target's
presumed privacy interests but don't represent him and can't give him
information about the investigation.

Magistrate judges occasionally do the same on an ad hoc basis, but in
those cases the amici don't have the same access to information as is
allowed in the surveillance court, and, like the amici there, they can't
appeal a lower-court ruling. Giving these independent lawyers the
information they need to argue about the legality of novel law
enforcement requests, as well as the right to appeal, would at least
provide for a more balanced assessment of new surveillance technologies
and a quicker way for questions about them to be decided on a national
basis.

I don't presume that any of my rulings have struck the right policy
balance between law enforcement and personal privacy. That's not even a
question a judge like me should try to answer. But as the pace of
technological advancement increases, the need becomes more urgent for
society to balance those interests in a coherent, fair and democratic
way.

James Orenstein is a United States magistrate judge in the Eastern
District of New York.

\emph{Like other media companies, The Times collects data on its
visitors when they read stories like this one. For more detail please
see}
\href{https://help.nytimes.com/hc/en-us/articles/115014892108-Privacy-policy?module=inline}{\emph{our
privacy policy}} \emph{and}
\href{https://www.nytimes.com/2019/04/10/opinion/sulzberger-new-york-times-privacy.html?rref=collection\%2Fspotlightcollection\%2Fprivacy-project-does-privacy-matter\&action=click\&contentCollection=opinion\&region=stream\&module=stream_unit\&version=latest\&contentPlacement=8\&pgtype=collection}{\emph{our
publisher's description}} \emph{of The Times's practices and continued
steps to increase transparency and protections.}

\emph{Follow}
\href{https://twitter.com/privacyproject}{\emph{@privacyproject}}
\emph{on Twitter and The New York Times Opinion Section on}
\href{https://www.facebook.com/nytopinion}{\emph{Facebook}}
\emph{and}\href{https://www.instagram.com/nytopinion/}{\emph{Instagram}}\emph{.}

\hypertarget{glossary-replacer}{%
\subsection{glossary replacer}\label{glossary-replacer}}

Advertisement

\protect\hyperlink{after-bottom}{Continue reading the main story}

\hypertarget{site-index}{%
\subsection{Site Index}\label{site-index}}

\hypertarget{site-information-navigation}{%
\subsection{Site Information
Navigation}\label{site-information-navigation}}

\begin{itemize}
\tightlist
\item
  \href{https://help.nytimes.com/hc/en-us/articles/115014792127-Copyright-notice}{©~2020~The
  New York Times Company}
\end{itemize}

\begin{itemize}
\tightlist
\item
  \href{https://www.nytco.com/}{NYTCo}
\item
  \href{https://help.nytimes.com/hc/en-us/articles/115015385887-Contact-Us}{Contact
  Us}
\item
  \href{https://www.nytco.com/careers/}{Work with us}
\item
  \href{https://nytmediakit.com/}{Advertise}
\item
  \href{http://www.tbrandstudio.com/}{T Brand Studio}
\item
  \href{https://www.nytimes.com/privacy/cookie-policy\#how-do-i-manage-trackers}{Your
  Ad Choices}
\item
  \href{https://www.nytimes.com/privacy}{Privacy}
\item
  \href{https://help.nytimes.com/hc/en-us/articles/115014893428-Terms-of-service}{Terms
  of Service}
\item
  \href{https://help.nytimes.com/hc/en-us/articles/115014893968-Terms-of-sale}{Terms
  of Sale}
\item
  \href{https://spiderbites.nytimes.com}{Site Map}
\item
  \href{https://help.nytimes.com/hc/en-us}{Help}
\item
  \href{https://www.nytimes.com/subscription?campaignId=37WXW}{Subscriptions}
\end{itemize}
