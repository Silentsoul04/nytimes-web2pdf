Sections

SEARCH

\protect\hyperlink{site-content}{Skip to
content}\protect\hyperlink{site-index}{Skip to site index}

\href{https://myaccount.nytimes.com/auth/login?response_type=cookie\&client_id=vi}{}

\href{https://www.nytimes.com/section/todayspaper}{Today's Paper}

\href{/section/opinion}{Opinion}\textbar{}Queer Dating Apps Are Unsafe
by Design

\url{https://nyti.ms/2J1rO3s}

\begin{itemize}
\item
\item
\item
\item
\item
\item
\end{itemize}

Advertisement

\protect\hyperlink{after-top}{Continue reading the main story}

\href{/section/opinion}{Opinion}

Supported by

\protect\hyperlink{after-sponsor}{Continue reading the main story}

\hypertarget{queer-dating-apps-are-unsafe-by-design}{%
\section{Queer Dating Apps Are Unsafe by
Design}\label{queer-dating-apps-are-unsafe-by-design}}

Privacy is particularly important for L.G.B.T.Q. people.

By Ari Ezra Waldman

Mr. Waldman is a professor of law at New York Law School.

\begin{itemize}
\item
  June 20, 2019
\item
  \begin{itemize}
  \item
  \item
  \item
  \item
  \item
  \item
  \end{itemize}
\end{itemize}

\includegraphics{https://static01.nyt.com/images/2019/06/20/autossell/20WaldmanCover/20WaldmanCover-square640.jpg}

Pete Buttigieg
\href{https://www.nytimes.com/2018/06/18/fashion/weddings/mayor-peter-buttigieg-wedding-democratic-party.html}{met
his husband} on a dating app called Hinge. And although that's unique
among presidential candidates, it's not unique for Mr. Buttigieg's
generation --- he's 37 --- or other members of the L.G.B.T.Q. community.

In 2016, the Pew Research Center found that use of online dating apps
among young adults had
\href{https://www.pewresearch.org/fact-tank/2016/02/29/5-facts-about-online-dating/}{tripled
in three years}, and nearly six in 10 adults of all ages thought apps
were a good way to meet someone. The rates are higher among queer
people, many of whom turn to digital spaces when stigma, discrimination
and long distances make face-to-face interaction difficult. One study
\href{https://www.ncbi.nlm.nih.gov/pubmed/24754360}{reported} that in
2013 more than one million gay and bisexual men logged in to a dating
app every day and sent more than seven million messages and two million
photos over all.

Privacy over our
\href{https://www.yalelawjournal.org/article/sexual-privacy}{sexual
selves} protects our dignity and autonomy. It allows us to speak our
minds and maintain social relationships. But for queer people, privacy
is uniquely important. Because employers in
\href{https://www.hrc.org/state-maps/employment}{29 states} can fire
workers simply for being gay or transgender, privacy with respect to our
sexual orientations and gender identities protects our livelihoods.
Privacy can also make us safer, especially with
\href{https://www.nbcnews.com/feature/nbc-out/anti-lgbtq-hate-crimes-rose-3-percent-17-fbi-finds-n936166}{anti-queer
hate crimes} increasing. Privacy lets us both ``come out'' in our own
time and, once we do, live our best lives out and proud, and modest
changes in design and in the law of platform liability can help us
achieve and maintain the privacy we need to survive and thrive.

The frequency with which queer people using social media, generally, and
mobile dating apps, in particular, amplifies the privacy concerns we
face compared with the general population. All digital dating platforms
require significant disclosure. Selfies and other personal information
are the currencies on which someone decides whether to swipe right or
left, or click a heart, or send a message. But the demand for disclosure
is powerful among gay people. In one peer-reviewed
\href{https://www.cambridge.org/core/journals/law-and-social-inquiry/article/law-privacy-and-online-dating-revenge-porn-in-gay-online-communities/BCCE05CF25AA4C2E05CCF8D64980E839/core-reader}{study},
87.4 percent of gay male app users reported sharing ``graphic, explicit
or nude photos or videos'' of themselves, higher than among those
looking for opposite-sex relationships.

\emph{{[}If you use technology, someone is using your information. We'll
tell you how --- and what you can do about it.}
\href{https://www.nytimes.com/newsletters/privacy-project?action=click\&module=Intentional\&pgtype=Article}{\emph{Sign
up for our limited-run newsletter}}\emph{.{]}}

Sometimes, the disclosure can cause real pain.
\href{https://www.cagoldberglaw.com/matthew-herrick-v-grindr-llc/}{Matthew
Herrick}, a gay man from New York, was stalked and harassed by his ex on
the geosocial app Grindr. His intimate images were disseminated without
his consent, and over 1,000 men were sent to his home and place of
business looking for sex. In 2017, two North Carolina high school
students
\href{http://www.towleroad.com/2017/05/catfish-teacher/}{created a fake
profile} and solicited a nude photo from their teacher, and then
distributed the picture throughout the school. The teacher was at first
suspended and then transferred. And 14.5 percent of gay and bisexual men
who use geosocial dating apps
\href{https://www.cambridge.org/core/journals/law-and-social-inquiry/article/law-privacy-and-online-dating-revenge-porn-in-gay-online-communities/BCCE05CF25AA4C2E05CCF8D64980E839/core-reader}{report}
that someone has shared their intimate images without their consent.
These stories are extreme, but not isolated: striking stories of
extortion, race-based sexual harassment, catfishing and revenge porn are
common on queer dating platforms.

Maintaining privacy in this environment seems difficult. Many people
think we can't. They blame victims for sharing intimate images, as if
victims are responsible for the bad behavior of their abusers. I
disagree. The problem isn't online dating or the hard-earned
\href{https://supreme.justia.com/cases/federal/us/539/558/}{freedom}
queer people have to live our lives out and proud. It's the law, or lack
thereof, that contributes to app designs that put our privacy at risk.

Over the past three years, I have studied the designs of different
queer-oriented dating platforms and surveyed and interviewed hundreds of
users. These individuals were diverse on multiple metrics: race, gender,
age, geographic location and apps used. They used dating apps for
different reasons, too, from long-term companionship or friendship to
sex or idle chat. And they had varying degrees of success. Some had
since deleted their accounts; many had not.

Other than their queerness, many shared similar thoughts and strategies
about sharing personal information in an environment with strong
disclosure norms. A plurality felt that sharing intimate images was
impliedly necessary, with the pressure to disclose particularly strong
among gay men. Stephen P., a gay app user from Boston, noted that ``if
you don't share photos, you can't really participate.'' Jason R.
admitted that ``it's the culture; {[}it's{]} hard to avoid.'' Others
shared photos to verify their identity to others, while some shared
photos in the name of sex positivity.

Despite this, significant majorities share with the expectation that
their images will not be disseminated further. And many take steps to
determine the trustworthiness of the people they meet online. Some
anonymize their photos, sending intimate images without faces or other
identifying characteristics. Many only share photos, graphic or
otherwise, after ``chatting with the other person'' for some time ---
ranging from a few hours to a few weeks --- sufficient to ``develop a
rapport'' or, as Jared S. responded, ``feel somewhat comfortable with
the other person.'' Often, users share intimate photos only after
another user has shared with them, maintaining power in a social
exchange for as long as possible and relying on reciprocity and mutual
vulnerability to reduce the likelihood of bad behavior. And many rely on
the comfort and familiarity of an app's exclusive queerness. John H.
noted that ``someone who is also gay, also about the same age, also
single, also lonely, also looking for the same thing you're looking for,
just seems less likely to hurt you than someone else who doesn't share
the same personal narrative.''

These strategies help develop trust among users, which facilitates
disclosure.
\href{https://papers.ssrn.com/sol3/papers.cfm?abstract_id=3278719}{But
trust cannot operate alone}. The design of the platforms --- the
socially constructed processes and code that make them function --- and
the laws governing behavior of users on the platforms have to work
together to buttress trust norms and ensure our safety.

Right now, the law isn't helping. Tort law, the regime we use to seek
damages from harassers, has been ineffectual because many courts look at
\href{https://www.law.uw.edu/wlr/print-edition/print-edition/vol-93/4/skinner-thompson}{gay
people} sharing selfies and conclude that they gave up their privacy the
moment they clicked ``send.'' Despite the tireless work of
\href{https://www.cybercivilrights.org/}{advocates}, we've only just
introduced a federal revenge porn bill. And the federal law we do have,
Communications Decency Act Section 230, immunizes digital platforms from
most legal liability associated with the bad behavior of their users.
That means that dating apps can ignore hundreds of complaints from their
users about harassment, racism and invasions of privacy. They know no
one is going to punish them for their negligence.

That makes us entirely dependent on the design choices of the platforms
themselves.
\href{https://slate.com/human-interest/2019/04/pete-chasten-buttigieg-hinge-marriage-dating-app-president.html}{Hinge
made a commitment} to privacy by designing in automatic deletion of all
communications the moment users delete their accounts. Scruff, another
gay-oriented app, makes it easy to flag offending accounts within the
app and claims to respond to all complaints within 24 hours. Grindr, on
the other hand, ignored 100 complaints from Mr. Herrick about his
harassment. \href{https://ir.lawnet.fordham.edu/flr/vol86/iss2/3/}{If,
as scholars have argued, Section 230 had a good-faith threshold}, broad
immunity would be granted only to those digital platforms that deserve
it.

Privacy isn't anathematic to online dating. Users want it, and they try
hard to maintain it. The problem isn't sharing intimate selfies, no
matter what victim-blamers would have us believe. The problem is the law
permits the development of apps that are unsafe by design.

Ari Ezra Waldman is a professor of law and the founding director of the
Innovation Center for Law and Technology at New York Law School.

\emph{Like other media companies,} \emph{The Times collects data on its
visitors when they read stories like this one. For more detail please
see}
\href{https://help.nytimes.com/hc/en-us/articles/115014892108-Privacy-policy?module=inline}{\emph{our
privacy policy}} \emph{and}
\href{https://www.nytimes.com/2019/04/10/opinion/sulzberger-new-york-times-privacy.html?rref=collection\%2Fspotlightcollection\%2Fprivacy-project-does-privacy-matter\&action=click\&contentCollection=opinion\&region=stream\&module=stream_unit\&version=latest\&contentPlacement=8\&pgtype=collection}{\emph{our
publisher's description}} \emph{of The Times's practices and continued
steps to increase transparency and protections.}

\emph{Follow}
\href{https://twitter.com/privacyproject}{\emph{@privacyproject}}
\emph{on Twitter and The New York Times Opinion Section on}
\href{https://www.facebook.com/nytopinion}{\emph{Facebook}}
\emph{and}\href{https://www.instagram.com/nytopinion/}{\emph{Instagram}}\emph{.}

\hypertarget{glossary-replacer}{%
\subsection{glossary replacer}\label{glossary-replacer}}

Advertisement

\protect\hyperlink{after-bottom}{Continue reading the main story}

\hypertarget{site-index}{%
\subsection{Site Index}\label{site-index}}

\hypertarget{site-information-navigation}{%
\subsection{Site Information
Navigation}\label{site-information-navigation}}

\begin{itemize}
\tightlist
\item
  \href{https://help.nytimes.com/hc/en-us/articles/115014792127-Copyright-notice}{©~2020~The
  New York Times Company}
\end{itemize}

\begin{itemize}
\tightlist
\item
  \href{https://www.nytco.com/}{NYTCo}
\item
  \href{https://help.nytimes.com/hc/en-us/articles/115015385887-Contact-Us}{Contact
  Us}
\item
  \href{https://www.nytco.com/careers/}{Work with us}
\item
  \href{https://nytmediakit.com/}{Advertise}
\item
  \href{http://www.tbrandstudio.com/}{T Brand Studio}
\item
  \href{https://www.nytimes.com/privacy/cookie-policy\#how-do-i-manage-trackers}{Your
  Ad Choices}
\item
  \href{https://www.nytimes.com/privacy}{Privacy}
\item
  \href{https://help.nytimes.com/hc/en-us/articles/115014893428-Terms-of-service}{Terms
  of Service}
\item
  \href{https://help.nytimes.com/hc/en-us/articles/115014893968-Terms-of-sale}{Terms
  of Sale}
\item
  \href{https://spiderbites.nytimes.com}{Site Map}
\item
  \href{https://help.nytimes.com/hc/en-us}{Help}
\item
  \href{https://www.nytimes.com/subscription?campaignId=37WXW}{Subscriptions}
\end{itemize}
