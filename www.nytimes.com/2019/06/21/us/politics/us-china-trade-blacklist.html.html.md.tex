Sections

SEARCH

\protect\hyperlink{site-content}{Skip to
content}\protect\hyperlink{site-index}{Skip to site index}

\href{https://www.nytimes.com/section/politics}{Politics}

\href{https://myaccount.nytimes.com/auth/login?response_type=cookie\&client_id=vi}{}

\href{https://www.nytimes.com/section/todayspaper}{Today's Paper}

\href{/section/politics}{Politics}\textbar{}U.S. Blacklists More Chinese
Tech Companies Over National Security Concerns

\url{https://nyti.ms/2YcQVqk}

\begin{itemize}
\item
\item
\item
\item
\item
\end{itemize}

Advertisement

\protect\hyperlink{after-top}{Continue reading the main story}

Supported by

\protect\hyperlink{after-sponsor}{Continue reading the main story}

\hypertarget{us-blacklists-more-chinese-tech-companies-over-national-security-concerns}{%
\section{U.S. Blacklists More Chinese Tech Companies Over National
Security
Concerns}\label{us-blacklists-more-chinese-tech-companies-over-national-security-concerns}}

By \href{https://www.nytimes.com/by/ana-swanson}{Ana Swanson},
\href{https://www.nytimes.com/by/paul-mozur}{Paul Mozur} and
\href{https://www.nytimes.com/by/steve-lohr}{Steve Lohr}

\begin{itemize}
\item
  June 21, 2019
\item
  \begin{itemize}
  \item
  \item
  \item
  \item
  \item
  \end{itemize}
\end{itemize}

\href{https://cn.nytimes.com/usa/20190624/us-china-trade-blacklist/}{阅读简体中文版}\href{https://cn.nytimes.com/usa/20190624/us-china-trade-blacklist/zh-hant/}{閱讀繁體中文版}

WASHINGTON --- The Trump administration added five Chinese entities to a
United States blacklist on Friday, further restricting China's access to
American technology and stoking already high tensions before a planned
meeting between President Trump and President Xi Jinping of China in
Japan next week.

The Commerce Department
\href{https://s3.amazonaws.com/public-inspection.federalregister.gov/2019-13245.pdf}{announced}
that it would add four Chinese companies and one Chinese institute to an
``entity list,'' saying they posed risks to American national security
or foreign policy interests. The move essentially bars them from buying
American technology and components without a waiver from the United
States government, which could all but cripple them because of their
reliance on American chips and other technology to make advanced
electronics.

The entities are one of China's leading supercomputer makers, Sugon;
three subsidiaries set up to design microchips, Higon, Chengdu Haiguang
Integrated Circuit and Chengdu Haiguang Microelectronics Technology; and
the Wuxi Jiangnan Institute of Computing Technology. They lead China's
development of high-performance computing, some of which is used in
military applications like simulating nuclear explosions, the Commerce
Department said.

Other Chinese companies that have been barred from access to American
technology include the telecom equipment giant Huawei, which was added
to the entity list in May. The Trump administration is also considering
adding Hikvision, a surveillance-technology company,
\href{https://www.nytimes.com/2019/05/21/us/politics/hikvision-trump.html}{The
New York Times has reported}.

The restriction of additional companies could further complicate efforts
to reach a trade deal. American and Chinese officials recently restarted
talks after negotiations collapsed in May, with Mr. Trump accusing China
of breaking a previous deal and the two countries intensifying their
tariff fight.

This week, Mr. Trump said he would have an ``extended meeting'' with Mr.
Xi at the Group of 20 summit next week in Osaka, Japan --- a
conversation that could determine whether the trade war ends or persists
indefinitely.

``It's ill timed,'' said William Reinsch, a former United States trade
official and now a senior adviser at the Center for Strategic and
International Studies. ``It clearly will be received negatively by the
Chinese.''

A fast resolution to the trade war was already looking unlikely. Both
sides
\href{https://www.nytimes.com/2019/06/02/business/china-trump-trade-fedex.html}{have
escalated their language} on remaining tough in trade talks, and China
has said it is putting together its own
\href{https://www.nytimes.com/2019/05/31/business/china-list-us-huawei-retaliate.html}{``unreliable
entities list''} of foreign companies and people, an apparent first step
toward retaliating against the denial of vital American technology to
Chinese companies.

The United States has begun targeting Chinese tech companies that rely
on American products as it seeks to thwart China's efforts to dominate
advanced technologies that could have military applications.

Supercomputing is one of the key technologies of the future that China
is seeking to dominate, in addition to artificial intelligence and
quantum computing. America is home to the
\href{https://www.nytimes.com/2018/06/08/technology/supercomputer-china-us.html}{world's
fastest computer,} at the Oak Ridge National Laboratory in Tennessee,
but China is building more of the ultrafast machines than any other
country.

Sugon is one of China's most important makers of high-performance
computers and servers. The machines serve China's government and its
largest technology companies, powering everything from military
simulations to weather prediction.

In most cases, the computers rely on a mix of microchips from the
American manufacturers Intel and Nvidia. By placing Sugon on the list,
the Trump administration is effectively cutting it off from the tiny
brains it needs to make the billions of calculations required to model
weather patterns and support video apps and online shopping.

One of the blacklisted subsidiaries had formed a partnership with the
American chip maker Advanced Micro Devices to create microchips that
could satisfy security demands for Chinese government customers. Some
Chinese officials said the chips could be used in
\href{https://www.nytimes.com/2017/11/07/business/made-in-china-technology-trade.html}{a
new generation of faster supercomputers}. By relying on chips made by
AMD and Intel, Sugon's supercomputers can run a wider array of software
than some of the country's faster computers built around domestically
produced chips.

With just over \$1 billion in revenue last year, Sugon is tiny compared
with the previously blacklisted Huawei. Still, its exclusion from
American technology is an especially bitter pill for Beijing to swallow,
as its supercomputers form the core of some of the Chinese government's
most sensitive and important systems.

Sugon supercomputers support State Grid, the monopoly that runs China's
electric grid; China Mobile, the country's largest telecom services
provider; and the China Meteorological Administration. It also makes
data centers for companies like the e-commerce giant JD.com and
Bytedance, the owner of the social media app TikTok.

The companies could petition the United States government for a license
to buy American technology, but their addition to the list suggests that
they would receive intense scrutiny and that approval might be unlikely.

For years, chip companies like Intel have sold widely available
microchips to supercomputer makers in China, even some with close ties
to the military. In 2015, the Commerce Department moved to add China's
National University of Defense and Technology to the entity list, to cut
it off from using Intel chips in supercomputers that the United States
government said were being used to model nuclear detonations.

On Friday, the list was updated with a number of new addresses and names
through which the National University of Defense Technology was
procuring chips. According to China's official list of its fastest
supercomputers, the institute still runs two of China's three fastest
supercomputers on Intel processors.

In its current order, the Commerce Department highlights the next
frontier in supercomputing: ``exascale'' machines, which China, the
United States and other nations are racing to build. They will be five
times as fast as the fastest machine today --- the Summit, which is
housed at Oak Ridge and was built by IBM in a partnership with Nvidia.

The Commerce Department order cites the three groups ``leading China's
development of exascale high-performance computing'': Sugon, the Wuxi
Jiangnan Institute of Computing Technology and the National University
of Defense Technology.

These three companies are all developing prototype exascale machines in
China that are powered by Chinese-made microprocessors, said Jack
Dongarra, a supercomputer expert at the University of Tennessee and a
co-creator of the Top 500 list of the swiftest machines.

But other key technologies in the fastest supercomputers will be
affected. Supercomputers are made by lashing together thousands of
processors, linked by a specialized fabric of digital circuitry known as
interconnect technology. The leading producer of high-performance
interconnect technology is Mellanox, an Israeli company that Nvidia
agreed this year to buy for \$6.9 billion.

``The interconnect technology is as important if not more important than
the processors,'' Mr. Dongarra said. ``The impact of this government
order is going to be far-reaching.''

It is likely to hamper the Chinese in the short run, he said, but also
encourage China to redouble its efforts to replace American technology.

Advertisement

\protect\hyperlink{after-bottom}{Continue reading the main story}

\hypertarget{site-index}{%
\subsection{Site Index}\label{site-index}}

\hypertarget{site-information-navigation}{%
\subsection{Site Information
Navigation}\label{site-information-navigation}}

\begin{itemize}
\tightlist
\item
  \href{https://help.nytimes.com/hc/en-us/articles/115014792127-Copyright-notice}{©~2020~The
  New York Times Company}
\end{itemize}

\begin{itemize}
\tightlist
\item
  \href{https://www.nytco.com/}{NYTCo}
\item
  \href{https://help.nytimes.com/hc/en-us/articles/115015385887-Contact-Us}{Contact
  Us}
\item
  \href{https://www.nytco.com/careers/}{Work with us}
\item
  \href{https://nytmediakit.com/}{Advertise}
\item
  \href{http://www.tbrandstudio.com/}{T Brand Studio}
\item
  \href{https://www.nytimes.com/privacy/cookie-policy\#how-do-i-manage-trackers}{Your
  Ad Choices}
\item
  \href{https://www.nytimes.com/privacy}{Privacy}
\item
  \href{https://help.nytimes.com/hc/en-us/articles/115014893428-Terms-of-service}{Terms
  of Service}
\item
  \href{https://help.nytimes.com/hc/en-us/articles/115014893968-Terms-of-sale}{Terms
  of Sale}
\item
  \href{https://spiderbites.nytimes.com}{Site Map}
\item
  \href{https://help.nytimes.com/hc/en-us}{Help}
\item
  \href{https://www.nytimes.com/subscription?campaignId=37WXW}{Subscriptions}
\end{itemize}
