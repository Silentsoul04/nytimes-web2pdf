Sections

SEARCH

\protect\hyperlink{site-content}{Skip to
content}\protect\hyperlink{site-index}{Skip to site index}

\href{https://www.nytimes.com/section/world/australia}{Australia}

\href{https://myaccount.nytimes.com/auth/login?response_type=cookie\&client_id=vi}{}

\href{https://www.nytimes.com/section/todayspaper}{Today's Paper}

\href{/section/world/australia}{Australia}\textbar{}A Timeline of
Despair in Australia's Offshore Detention Centers

\url{https://nyti.ms/2IQ24Iy}

\begin{itemize}
\item
\item
\item
\item
\item
\item
\end{itemize}

Advertisement

\protect\hyperlink{after-top}{Continue reading the main story}

Supported by

\protect\hyperlink{after-sponsor}{Continue reading the main story}

\hypertarget{a-timeline-of-despair-in-australias-offshore-detention-centers}{%
\section{A Timeline of Despair in Australia's Offshore Detention
Centers}\label{a-timeline-of-despair-in-australias-offshore-detention-centers}}

The New York Times worked with human rights groups and asylum seekers on
Manus Island to examine a rash of suicide attempts and acts of self-harm
since Australia's May 18 election.

\includegraphics{https://static01.nyt.com/images/2019/06/26/world/26manus-1/26manus-1-articleLarge.jpg?quality=75\&auto=webp\&disable=upscale}

\href{https://www.nytimes.com/by/damien-cave}{\includegraphics{https://static01.nyt.com/images/2018/10/08/multimedia/author-damien-cave/author-damien-cave-thumbLarge.png}}

By \href{https://www.nytimes.com/by/damien-cave}{Damien Cave}

\begin{itemize}
\item
  June 26, 2019
\item
  \begin{itemize}
  \item
  \item
  \item
  \item
  \item
  \item
  \end{itemize}
\end{itemize}

\href{https://www.nytimes.com/es/2019/06/30/detencion-extraterritorial-australia/}{Leer
en español}

\emph{For thoughtful coverage of Australia delivered to your inbox,}
\href{https://www.nytimes.com/newsletters/australia-letter?module=inline}{\emph{sign
up for the Australia Letter}}\emph{.}

SYDNEY, Australia --- Human rights groups call them a violation of
international law. The Australian government says they are crucial to
regulating the flow of immigration.

But one thing is indisputable about Australia's offshore detention
centers on the islands of Manus and Nauru: Despair is soaring among
asylum seekers being held there.

Since Australia's national election on May 18, which
\href{https://www.nytimes.com/2019/05/18/world/australia/election-results-scott-morrison.html}{returned
to power a conservative government that has maintained hard-line
policies intended to deter asylum seekers,}there have been dozens of
suicide attempts and acts of self-harm at the refugee centers.

``It's hard to know how many cases are serious cases of people trying to
end their lives or a cry for help, but in any case it's a big
escalation,'' said Elaine Pearson, the Australia director for Human
Rights Watch, who has made several visits to Manus. ``People are very
worried they are going to be completely forgotten about.''

The Australian government argues that its strict border protection
policy, which bars settlement for migrants who try to reach the country
by sea, has worked: Fewer boats with asylum seekers on board are trying
to reach Australia compared to a decade ago.

Australia's immigration policies received a forceful endorsement on
Thursday morning from President Trump, who was scheduled to have dinner
that evening with Prime Minister Scott Morrison in Osaka, Japan.

But many of the detainees on Manus and Nauru, cognizant of Australian
polls that showed the opposition Labor Party leading before the May
election, had hoped a change in immigration policy was on the horizon.
When Labor lost, desperation intensified.

Adding to the despair, the United States has rejected resettlement
applications for roughly 300 refugees on the two islands, despite a deal
reached by the Obama administration to take in more than 1,000 of them.

The situation has grown especially dark on Manus, a remote island in
Papua New Guinea where several hundred men from Afghanistan, Iran and
other countries are being held.

It was grim
\href{https://www.nytimes.com/interactive/2017/11/18/world/australia/manus-island-australia-detainees.html?_r=0}{when
I visited nearly two years ago to write about the situation}. Now, it's
worse. Every few days, it seems, a new instance emerges of people
cutting themselves, setting themselves on fire or trying to harm
themselves in some other way.

To better understand the problem, The New York Times worked with human
rights groups and asylum seekers on Manus to compile a rough timeline of
events since the May 18 election.

This is not a comprehensive list, but rather a selection of significant
moments --- including official responses from Papua New Guinea and
Australia --- that are adding pressure to an already intense debate.

\hypertarget{may-20}{%
\subsection{May 20}\label{may-20}}

Within 48 hours of the election that returned Mr. Scott Morrison to
power, the details of six suicide attempts
\href{https://twitter.com/BehrouzBoochani/status/1130424427139674112?s=20}{start
to emerge}.

Four of the men, including a Sudanese man who left a suicide note, ended
up in the hospital. The two others were held by the police after they
tried to set fire to themselves in their rooms.

``We are really devastated with the election results,'' Shamindan
Kanapathi, 28, a Sri Lankan refugee on Manus, wrote in a text message to
refugee advocates at the time. ``We are really disappointed.''

\hypertarget{may-29}{%
\subsection{May 29}\label{may-29}}

The main hospital on Manus, an under-equipped one-story clinic, begins
to turn away those arriving with injuries from self-harm.

Suicide attempts also start to emerge in Port Moresby, the capital of
Papua New Guinea, where some asylum seekers have been moved.

The Refugee Action Coalition, an advocacy group, reports that one Iraqi
asylum seeker tried to hang himself and was saved when he was cut down
by a guard.

In several other cases, the coalition said, refugees swallowed razor
blades and needles.

There are 490 asylum seekers in Papua New Guinea. To try to keep them
safe, the police step up patrols near their housing.

\hypertarget{june-3}{%
\subsection{June 3}\label{june-3}}

Mr. Kanapathi, whose resettlement application was rejected by the United
States last year, sends a series of text messages just before midnight:

\begin{quote}
``Just short while ago an Iranian cut his hand bit deeper and lost too
much blood.''

``Also one Sudanese man harmed himself tonight.''
\end{quote}

\hypertarget{june-8}{%
\subsection{June 8}\label{june-8}}

Abdul Aziz Muhamat, 25, a refugee from Sudan who has spent six years on
Manus, announces on Twitter that Switzerland has granted him asylum.

\hypertarget{june-10}{%
\subsection{June 10}\label{june-10}}

Another text from Mr. Kanapathi: ``Bloody hell just few minutes ago a
highly depressed man set himself in fire. I think it's very serious
situation.''

\hypertarget{june-13}{%
\subsection{June 13}\label{june-13}}

As the suicide attempts continue, the governor of Manus Province,
Charlie Benjamin,
\href{https://www.abc.net.au/news/2019-06-13/manus--self-harm-crisis-escalates-as-governor-calls-for-help/11199258}{calls
for Australia to finally accept the men} they have sent to Manus.

``They don't want to be here, and Australia, you have to take
responsibility,'' he says. ``You have to move them.''

\includegraphics{https://static01.nyt.com/images/2019/06/26/world/26manus-2/merlin_156720906_5ec798cf-94a5-428a-bdab-19f4b0a0ba70-articleLarge.jpg?quality=75\&auto=webp\&disable=upscale}

\hypertarget{june-16}{%
\subsection{June 16}\label{june-16}}

Peter Dutton, Australia's home affairs minister, who oversees border
protection and detention,
\href{https://www.abc.net.au/news/2019-06-16/peter-dutton-unsure-medevac-arrivals-refugees-serious-offences/11214584}{says
the government is looking at repealing the so-called Medevac law},
passed this year, which makes it easier for refugees and asylum seekers
on Manus and Nauru to seek medical treatment in Australia.

He says the law could allow criminals to reach Australia. He adds that
531 people from Manus and Nauru have already been settled in the United
States, and that a few hundred more are ``still in the pipeline.''

``I want to reduce the number down to zero on both islands,'' he says.
``But I don't want to bring people here who pose a risk.''

\hypertarget{june-20}{%
\subsection{June 20}\label{june-20}}

Mr. Dutton
\href{https://www.abc.net.au/news/2019-06-20/warnings-of-boat-arrivals/11226254}{warns}
that a court decision upholding the Medevac law could lead to a surge of
boat arrivals. He pledges to keeping pushing for repeal.

Later that afternoon, a man near the main transit center in Manus climbs
to the top of an internet tower and threatens to jump. A few hours
later, he climbs back down.

\hypertarget{june-21}{%
\subsection{June 21}\label{june-21}}

A 31-year-old asylum seeker from India
\href{https://www.smh.com.au/national/man-sets-himself-on-fire-on-manus-island-after-being-denied-medical-treatment-20190621-p5204i.html}{lights
himself and his room on fire}. Refugees on Manus report that he had
sought care at a hospital but did not receive it, leading him to harm
himself.

``While the home affairs minister spends his time arguing to repeal the
Medevac bill, the situation on Manus spins out of control,'' says Ian
Rintoul, spokesman for the Refugee Action Coalition.

\hypertarget{june-24}{%
\subsection{June 24}\label{june-24}}

An asylum seeker on Manus posts images of an X-ray showing a man who
swallowed a nail clipper.

\hypertarget{june-25}{%
\subsection{June 25}\label{june-25}}

Papua New Guinea's new prime minister, James Marape,
\href{https://www.abc.net.au/news/2019-06-25/png-james-marape-wants-paladin-manus-contract-terminated/11245330}{demands
local control} of the lucrative contract to run the refugee centers on
Manus.

The Australian Financial Review, a business newspaper, had revealed that
a
\href{https://www.afr.com/news/policy/foreign-affairs/the-secretive-firm-earning-20-million-from-refugees-on-manus-island-20190210-h1b2sm}{little-known
security firm}, Paladin, was given a contract worth 20.9 million
Australian dollars (\$14.6 million) a month without a competitive
tender. In January, one of the firm's local directors was arrested on
charges of fraud and money laundering.

\hypertarget{june-26}{%
\subsection{June 26}\label{june-26}}

The governments of Australia and Papua New Guinea
\href{https://www.abc.net.au/news/2019-06-26/australia-and-png-agree-to-limited-extension-manus-contracts/11249702}{agree
to extend the Paladin contract}.

When a handful of the asylum seekers on Manus count up all the incidents
of self-harm and attempted suicide since the election, they tell me the
figure approaches 100 --- with many examples kept hidden because people
don't want to shame their families.

``No matter what company takes the contract --- a local company or
foreign company --- our situation is the same,'' Mr. Kanapathi said.
``Both the Australian and P.N.G. governments are playing with our lives.
To them, we are nothing other than commodities that are used to benefit
their political careers whilst we are on the ground fighting for our
lives.''

\begin{center}\rule{0.5\linewidth}{\linethickness}\end{center}

\emph{In Australia, the crisis support service Lifeline can be reached
at 13 11 14. In the United States, the National Suicide Prevention
Lifeline is 1-800-273-8255. Other international suicide help lines can
be found at}
\href{https://www.befrienders.org/}{\emph{befrienders.org}}\emph{.}

\emph{Want more Australia coverage and discussion? Sign up for the
weekly}\href{https://www.nytimes.com/newsletters/australia-letter?utm_source=ausend}{\emph{Australia
Letter}}\emph{, start your day with your local}
\href{https://www.nytimes.com/interactive/2018/briefing/global-morning-briefing-newsletter-signup.html?utm_source=ausend}{\emph{Morning
Briefing}} \emph{and join us in
our}\href{https://www.facebook.com/groups/nytaustralia/}{\emph{Facebook
group}}\emph{.}

Advertisement

\protect\hyperlink{after-bottom}{Continue reading the main story}

\hypertarget{site-index}{%
\subsection{Site Index}\label{site-index}}

\hypertarget{site-information-navigation}{%
\subsection{Site Information
Navigation}\label{site-information-navigation}}

\begin{itemize}
\tightlist
\item
  \href{https://help.nytimes.com/hc/en-us/articles/115014792127-Copyright-notice}{©~2020~The
  New York Times Company}
\end{itemize}

\begin{itemize}
\tightlist
\item
  \href{https://www.nytco.com/}{NYTCo}
\item
  \href{https://help.nytimes.com/hc/en-us/articles/115015385887-Contact-Us}{Contact
  Us}
\item
  \href{https://www.nytco.com/careers/}{Work with us}
\item
  \href{https://nytmediakit.com/}{Advertise}
\item
  \href{http://www.tbrandstudio.com/}{T Brand Studio}
\item
  \href{https://www.nytimes.com/privacy/cookie-policy\#how-do-i-manage-trackers}{Your
  Ad Choices}
\item
  \href{https://www.nytimes.com/privacy}{Privacy}
\item
  \href{https://help.nytimes.com/hc/en-us/articles/115014893428-Terms-of-service}{Terms
  of Service}
\item
  \href{https://help.nytimes.com/hc/en-us/articles/115014893968-Terms-of-sale}{Terms
  of Sale}
\item
  \href{https://spiderbites.nytimes.com}{Site Map}
\item
  \href{https://help.nytimes.com/hc/en-us}{Help}
\item
  \href{https://www.nytimes.com/subscription?campaignId=37WXW}{Subscriptions}
\end{itemize}
