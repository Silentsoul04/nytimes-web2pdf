Sections

SEARCH

\protect\hyperlink{site-content}{Skip to
content}\protect\hyperlink{site-index}{Skip to site index}

\href{/section/us}{U.S.}\textbar{}As Cameras Track Detroit's Residents,
a Debate Ensues Over Racial Bias

\url{https://nyti.ms/2JtnB94}

\begin{itemize}
\item
\item
\item
\item
\item
\end{itemize}

\includegraphics{https://static01.nyt.com/images/2019/07/02/us/00greenlight-01/merlin_156466074_e9dc031b-f704-4eda-bee4-a7d86adb1106-articleLarge.jpg?quality=75\&auto=webp\&disable=upscale}

\hypertarget{as-cameras-track-detroits-residents-a-debate-ensues-over-racial-bias}{%
\section{As Cameras Track Detroit's Residents, a Debate Ensues Over
Racial
Bias}\label{as-cameras-track-detroits-residents-a-debate-ensues-over-racial-bias}}

Studies have shown that facial recognition software can return more
false matches for African-Americans than for white people, a sign of
what experts call ``algorithmic bias.''

Surveillance cameras have been deployed across Detroit as part of
Project Green Light, which is meant to deter crime.Credit...Brittany
Greeson for The New York Times

Supported by

\protect\hyperlink{after-sponsor}{Continue reading the main story}

By \href{https://www.nytimes.com/by/amy-harmon}{Amy Harmon}

\begin{itemize}
\item
  July 8, 2019
\item
  \begin{itemize}
  \item
  \item
  \item
  \item
  \item
  \end{itemize}
\end{itemize}

\emph{{[}For more coverage of race,}
\emph{\href{https://www.nytimes.com/2018/10/01/us/subscribe-race-related-newsletter.html?module=inline}{sign
up here}} \emph{to have our Race/Related newsletter delivered weekly to
your inbox.{]}}

DETROIT --- Twenty-four hours a day, video from
\href{https://detroitmi.gov/webapp/project-green-light-map}{thousands of
cameras} stationed around Detroit, at gas stations, restaurants,
mini-marts, apartment buildings, churches and schools, streams into the
Police Department's downtown headquarters.

The surveillance program, which began in 2016, is the opposite of
covert. A flashing green light marks each participating location, and
the point of the popular initiative, known as Project Green Light, has
been for the cameras to be noticed and help deter crime. Detroit's
mayor, Mike Duggan, received applause when he promised at his
\href{https://www.youtube.com/watch?v=dXAZoMCl3qs}{State of the City
address} earlier this year that expanding the network to include several
hundred traffic light cameras would allow the police to ``track any
shooter or carjacker across the city.''

But in recent weeks, a public outcry has erupted over a less-touted tool
employed in conjunction with the cameras: software that can, in a matter
of seconds, suggest the identities of the anonymous people captured on
video.

The facial recognition program matches the faces picked up across the
city against 50 million driver's license photographs and mug shots
contained in a Michigan police database. The practice has attracted
public attention recently as the department seeks approval for a
\href{https://detroitmi.gov/document/facial-recognition}{formal policy}
governing its use from a
\href{https://detroitmi.gov/government/boards/board-police-commissioners}{civilian
oversight board}.

\includegraphics{https://static01.nyt.com/images/2019/07/09/us/09Greenlight-Print/merlin_156466131_cdc242e2-0883-4605-9436-acc01b7faa9c-articleLarge.jpg?quality=75\&auto=webp\&disable=upscale}

``Please, facial recognition software --- that's too far,'' pleaded one
resident at a recent meeting of the board.

The debate in Detroit is one of
\href{https://www.nytimes.com/2019/07/01/us/facial-recognition-san-francisco.html}{several
unfolding around the country} as
\href{https://www.nytimes.com/2017/11/28/technology/artificial-intelligence-research-toronto.html}{rapid
advances in facial recognition} offer potentially disquieting new powers
to a
\href{https://www.nytimes.com/2018/05/26/us/chicago-police-surveillance.html}{surveillance
infrastructure}that Americans have largely accepted as a fact of urban
life.
\href{https://www.nytimes.com/2019/07/07/us/politics/ice-drivers-licenses-facial-recognition.html}{Immigration
officials have mined} driver's license databases in at least three
states, according to newly released records. The F.B.I. also routinely
uses facial recognition technology to scan state driver's license
databases without the approval or knowledge of the license-holders,
which a bipartisan group of lawmakers said last month raises privacy
concerns.

In Detroit, whose share of black residents is larger than in any other
sizable American city, it is a racial disparity in the performance of
facial recognition technology that is a primary source of consternation.

``Facial recognition software proves to be less accurate at identifying
people with darker pigmentation,'' George Byers II, a black software
engineer, told the police board last month. ``We live in a major black
city. That's a problem.''

Researchers at the Massachusetts Institute of Technology
\href{https://www.nytimes.com/2019/01/24/technology/amazon-facial-technology-study.html}{reported
in January} that facial recognition software marketed by Amazon
misidentified darker-skinned women as men 31 percent of the time. Others
have shown that algorithms used in facial recognition
\href{https://arxiv.org/pdf/1904.07325.pdf}{return false matches at a
higher rate for African-Americans} than white people unless explicitly
recalibrated for a black population --- in which case their failure rate
at finding positive matches for white people climbs.
\href{https://arxiv.org/pdf/1904.07325.pdf}{That study}, posted in May
by computer scientists at the Florida Institute of Technology and the
University of Notre Dame, suggests that a single algorithm cannot be
applied to both groups with equal accuracy.

Mr. Byers and other critics spoke at
\href{http://video.detroitmi.gov/CablecastPublicSite/show/7361?channel=3}{a
public hearing} called by the Detroit Board of Police Commissioners
after what the board called unprecedented public interest in two facial
recognition items on its agenda. One item, specific to the new traffic
light cameras, was approved last week. The other, a comprehensive
``acceptable use'' policy for facial recognition, has yet to be put to a
vote.

Image

Residents spoke in opposition to the expansion of Project Green Light,
which uses facial recognition technology, during a police board meeting
in Detroit in June.Credit...Brittany Greeson for The New York Times

Gathered in a packed church in the Second Precinct on the city's west
side, those who expressed concerns about what is called ``algorithmic
bias'' included Denzel McCampbell, press secretary to Representative
Rashida Tlaib, the Michigan Democrat whose district includes Detroit,
and Blair Anderson, a
\href{https://www.nytimes.com/1970/05/09/archives/7-panthers-freed-in-chicago-clash-states-attorney-cites-lack-of.html}{former
member of the Black Panther Party} who invoked the
\href{https://www.intelligence.senate.gov/sites/default/files/94755_II.pdf}{law
enforcement surveillance} that helped destroy the political group as
\href{https://openjurist.org/600/f2d/600/hampton-v-hanrahan}{a
cautionary tale}.

Tawana Petty, an activist with the
\href{https://detroitcommunitytech.org/?q=content/critical-summary-detroit\%E2\%80\%99s-project-green-light-and-its-greater-context}{Detroit
Community Technology Project}, urged fellow Detroiters to consider the
city's place in the national conversation on facial recognition. ``If we
allow racially biased technologies to succeed here,'' she said in an
interview, ``there really isn't any hope for black residents anywhere
else in the United States.''

Not everyone who spoke was against the use of facial recognition.

``I'm the pastor getting the call from mothers whose son was shot or
their baby got snatched up,'' said Maurice Hardwick, a black pastor at a
nondenominational ministry who founded a group that works with high
school gang members. ``People want to know two things: What happened to
my child, my loved one? And who did this?''

Another Detroit resident, a white woman who walked with a cane, added:
``If you're afraid of the cameras, either you're paranoid or you've got
something to hide.''

Others were more concerned with a provision that would allow the police
to go beyond identifying violent crime suspects with facial recognition
and allow officers to try to identify anyone for whom a ``reasonable
suspicion'' exists that they could provide information relevant to an
active criminal investigation. There was also concern that the
photograph of anyone who gets a Michigan state ID or driver's license is
searchable by state and local law enforcement agencies, and the F.B.I.,
\href{https://www.freep.com/story/news/local/michigan/2019/03/11/michigan-state-police-facial-recognition-database/3102139002/}{likely
without their knowledge}.

Facial recognition, the Detroit police stress, has indeed helped lead to
arrests. In late May, for instance, officers ran a video image through
facial recognition after survivors of a shooting directed police
officers to a gas station equipped with Green Light cameras where they
had met with a man now charged with three counts of first-degree murder
and two counts of assault. The lead generated by the software matched
the description provided by the witnesses.

Image

A map at police headquarters shows the various camera locations
throughout the city.Credit...Brittany Greeson for The New York Times

In the absence of federal legislation regulating the technology, experts
say cities and states are destined to be the first to weigh the societal
risks of technology that
\href{https://www.nytimes.com/2019/05/18/us/facial-recognition-police.html}{many
law enforcement officials} say is
\href{https://www.nytimes.com/2019/06/09/opinion/facial-recognition-police-new-york-city.html}{critical
for ensuring public safety}.

As in San Francisco, which this spring became the first major city to
\href{https://www.nytimes.com/2019/07/01/us/facial-recognition-san-francisco.html}{block
the police from using facial recognition}, critics here have argued that
facial recognition threatens civil liberties and that the
\href{https://www.bjs.gov/content/pub/pdf/cpp15.pdf}{pervasive}
\href{https://www.detroitnews.com/story/news/local/detroit-city/2019/04/24/detroit-police-chief-cites-racially-tone-deaf-culture-6th-precinct/3554141002/}{racial
bias}
\href{https://www.washingtonpost.com/news/opinions/wp/2018/09/18/theres-overwhelming-evidence-that-the-criminal-justice-system-is-racist-heres-the-proof/?utm_term=.01ee3c3c5895}{in
policing} will inevitably extend to how it is wielded, not least because
African-Americans are disproportionately represented in mug-shot
databases.

When James White, an assistant police chief in charge of the Detroit
Police Department's technology, rose to respond to critics at the public
hearing, he provided unexpected backup to the charge that the software
comes with baked-in bias. He himself, the assistant chief said, had been
misidentified as other African-American men by the facial recognition
algorithm that Facebook uses to tag photos.

``On the question of false positives --- that is absolutely factual, and
it's well-documented,'' he said. ``So that concerns me as an
African-American male.''

The solution, Chief White said, is to exercise extra care. The
department's policy specifies that facial recognition will be used only
to investigate violent crimes. Although the department has the ability
to implement real-time screening of anyone who passes by a camera --- as
detailed in \href{https://www.americaunderwatch.com/}{a recent report}by
the Georgetown Law Center on Privacy and Technology --- there is no plan
to use it, he said, except in extraordinary circumstances.

No one in Detroit, Chief White emphasized, would be arrested solely on
the basis of a facial recognition match.

Image

A business participating in Project Green Light.Credit...Brittany
Greeson for The New York Times

``Facial recognition technology isn't where the work stops,'' he said.
``It's where the work starts.''

Civil liberties advocates say that protection isn't enough, especially
because defendants are not typically informed that facial recognition
has been used in their identification. In one of the few cases to have
\href{https://www.jacksonville.com/public-safety/2016-11-11/how-accused-drug-dealer-revealed-jso-s-facial-recognition-network}{argued
that such information should be disclosed because it is potentially
exonerating}, a Florida appeals court ruled that a black man, Willie
Allen Lynch, had
\href{https://www.jacksonville.com/news/20190123/florida-court-prosecutors-had-no-obligation-to-turn-over-facial-recognition-evidence}{no
legal right to see the other matches} returned by the facial recognition
program that helped lead to his drug-offense conviction. Mr. Lynch had
argued that he was misidentified.

A January 2018 study by two M.I.T. researchers first
\href{https://www.nytimes.com/2018/02/09/technology/facial-recognition-race-artificial-intelligence.html}{focused
public attention} on the higher misidentification rates for dark-skinned
women by three leading purveyors of facial recognition algorithms. One
of the co-authors, Joy Buolamwini, posted YouTube videos showing the
technology misclassifying famous African-American women, like Michelle
Obama, as men. The phenomenon, Ms. Buolamwini
\href{https://www.nytimes.com/2018/06/21/opinion/facial-analysis-technology-bias.html}{wrote
in a New York Times Op-Ed}, is ``a reminder that artificial
intelligence, often heralded for its potential to change the world, can
actually reinforce bias and exclusion, even when it's used in the most
well-intended ways.''

The companies examined in the paper subsequently improved their
algorithms for that particular test. But a second paper this year found
that Amazon's software had more trouble identifying the gender of female
and darker-skinned faces, prompting prominent artificial-intelligence
researchers to call on the company to
\href{https://www.nytimes.com/2019/04/03/technology/amazon-facial-recognition-technology.html}{stop
selling its software} to law enforcement agencies. Amazon executives
have disputed the study.

It is not clear why facial recognition algorithms perform differently on
different racial groups, researchers say. One reason may be that the
algorithms, which learn to recognize patterns in faces by looking at
large numbers of them, are not being trained on a diverse enough array
of photographs.

But Kevin Bowyer, a Notre Dame computer scientist, said that was not the
case for a study he recently published. Nor is it certain that skin tone
is the culprit: Facial structure, hairstyles and other factors may
contribute.

In Dr. Bowyer's experiments, the recognition algorithms could achieve
the same degree of accuracy for white and black Americans, but only when
the algorithm was tuned to a cutoff, say, of no more than one in 10,000
false matches for the two separate groups. Given that the norm is to use
the same threshold for everybody, ``those programs are seeing a higher
false match rate for the population of African-Americans,'' Dr. Bowyer
said.

A dual-threshold system would not necessarily solve the problem, he
added. That would require law enforcement authorities to make a judgment
about each individual's race and apply the appropriately tweaked facial
recognition software --- which would in turn introduce human bias.

``Technically, it's a very reasonable thing to say to do,'' Dr. Bowyer
said. ``But how do you defend it, and once you put that knob out there
for police to use, how do you make sure it's not misused?''

Advertisement

\protect\hyperlink{after-bottom}{Continue reading the main story}

\hypertarget{site-index}{%
\subsection{Site Index}\label{site-index}}

\hypertarget{site-information-navigation}{%
\subsection{Site Information
Navigation}\label{site-information-navigation}}

\begin{itemize}
\tightlist
\item
  \href{https://help.nytimes.com/hc/en-us/articles/115014792127-Copyright-notice}{©~2020~The
  New York Times Company}
\end{itemize}

\begin{itemize}
\tightlist
\item
  \href{https://www.nytco.com/}{NYTCo}
\item
  \href{https://help.nytimes.com/hc/en-us/articles/115015385887-Contact-Us}{Contact
  Us}
\item
  \href{https://www.nytco.com/careers/}{Work with us}
\item
  \href{https://nytmediakit.com/}{Advertise}
\item
  \href{http://www.tbrandstudio.com/}{T Brand Studio}
\item
  \href{https://www.nytimes.com/privacy/cookie-policy\#how-do-i-manage-trackers}{Your
  Ad Choices}
\item
  \href{https://www.nytimes.com/privacy}{Privacy}
\item
  \href{https://help.nytimes.com/hc/en-us/articles/115014893428-Terms-of-service}{Terms
  of Service}
\item
  \href{https://help.nytimes.com/hc/en-us/articles/115014893968-Terms-of-sale}{Terms
  of Sale}
\item
  \href{https://spiderbites.nytimes.com}{Site Map}
\item
  \href{https://help.nytimes.com/hc/en-us}{Help}
\item
  \href{https://www.nytimes.com/subscription?campaignId=37WXW}{Subscriptions}
\end{itemize}
