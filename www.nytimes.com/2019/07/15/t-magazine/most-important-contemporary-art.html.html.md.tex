Sections

SEARCH

\protect\hyperlink{site-content}{Skip to
content}\protect\hyperlink{site-index}{Skip to site index}

\href{https://myaccount.nytimes.com/auth/login?response_type=cookie\&client_id=vi}{}

\href{https://www.nytimes.com/section/todayspaper}{Today's Paper}

The 25 Works of Art That Define the Contemporary Age

\href{https://nyti.ms/2XOV5Uz}{https://nyti.ms/2XOV5Uz}

\begin{itemize}
\item
\item
\item
\item
\item
\item
\end{itemize}

Advertisement

\protect\hyperlink{after-top}{Continue reading the main story}

Supported by

\protect\hyperlink{after-sponsor}{Continue reading the main story}

\includegraphics{https://static01.nyt.com/images/2019/07/15/t-magazine/15tmag-art-list-topper-image/15tmag-art-list-topper-image-videoSixteenByNineJumbo1600.jpg}

\hypertarget{the-25-works-of-art-that-define-the-contemporary-age}{%
\section{The 25 Works of Art That Define the Contemporary
Age}\label{the-25-works-of-art-that-define-the-contemporary-age}}

Three artists and a pair of curators came together at The New York Times
to attempt to make a list of the era's essential artworks. Here's their
conversation.

On a recent afternoon in June, T Magazine assembled two curators and
three artists --- \textbf{David Breslin}, the director of the collection
at the
\href{https://www.nytimes.com/topic/organization/whitney-museum-of-american-art}{Whitney
Museum of American Art}; the American conceptual artist \textbf{Martha
Rosler}; \textbf{Kelly Taxter}, a curator of contemporary art at the
\href{https://www.nytimes.com/topic/organization/jewish-museum-nyc}{Jewish
Museum}; the Thai conceptual artist \textbf{Rirkrit Tiravanija}; and the
American artist \textbf{Torey Thornton} --- at the New York Times
building to discuss what they considered to be the 25 works of art made
after 1970 that define the contemporary age, by anyone, anywhere. The
assignment was intentionally wide in its range: What qualifies as
``contemporary''? Was this an artwork that had a personal significance,
or was its meaning widely understood? Was its influence broadly
recognized by critics? Or museums? Or other artists? Originally, each of
the participants was asked to nominate 10 artworks --- the idea being
that everyone would then rank each list to generate a master list that
would be debated upon meeting.

\emph{{[}}\href{https://www.nytimes.com/newsletters/t-list?module=inline}{\emph{Sign
up here}} \emph{for the T List newsletter, a weekly roundup of what T
Magazine editors are noticing and coveting now.{]}}

Unsurprisingly, the system fell apart. It was impossible, some argued,
to rank art. It was also impossible to select just 10. (Rosler, in fact,
objected to the whole premise, though she brought her own list to the
discussion in the end.) And yet, to everyone's surprise, there was a
significant amount of overlap: works by
\href{https://www.nytimes.com/2016/03/25/arts/design/david-hammons-is-still-messing-with-what-art-means.html}{David
Hammons},
\href{https://www.mariangoodman.com/artists/dara-birnbaum}{Dara
Birnbaum},
\href{https://www.nytimes.com/2017/05/11/t-magazine/art/felix-gonzalez-torres-zwirner-new-york-show.html}{Felix
Gonzalez-Torres},
\href{https://www.nytimes.com/2018/06/12/t-magazine/danh-vo-pho-recipe.html}{Danh
Vo}, \href{https://gagosian.com/artists/cady-noland/}{Cady Noland},
\href{https://www.nytimes.com/2016/06/09/t-magazine/art/kara-walker-father-larry-art.html}{Kara
Walker},
\href{https://www.nytimes.com/2017/03/08/t-magazine/art/mike-kelley-mobile-homestead.html}{Mike
Kelley},
\href{https://www.nytimes.com/2017/09/15/t-magazine/art/barbara-kruger-berlin.html}{Barbara
Kruger} and
\href{https://gavinbrown.biz/artists/arthur_jafa/exhibitions/2019}{Arthur
Jafa} were cited multiple times. Had the group, perhaps, stumbled upon
some form of agreement? Did their selections reflect our values,
priorities and a unified idea of what matters today? Did focusing on
artworks, rather than artists, allow for a different framework?

\includegraphics{https://static01.nyt.com/images/2019/07/15/t-magazine/15tmag-artlist/15tmag-artlist-articleLarge.jpg?quality=75\&auto=webp\&disable=upscale}

Naturally, when re-evaluating the canon of the last five decades, there
were notable omissions. The group failed to name many artists who most
certainly had an impact on how we view art today: Bigger names of recent
Museum of Modern Art retrospectives, internationally acclaimed artists
and high earners on the secondary market were largely excluded. Few
paintings were singled out;
\href{https://www.nytimes.com/2018/11/21/t-magazine/female-land-artists.html}{land
art} was almost entirely absent, as were, to name just a few more
categories, works on paper, sculpture, photography, fiber arts and
\href{https://www.nytimes.com/2015/06/01/t-magazine/outsider-art-essay-christine-smallwood.html}{outsider
art}.

It's important to emphasize that no consensus emerged from the meeting.
Rather, this list of works is merely what has been culled from the
conversation, each chosen because it appeared on a panelist's original
submission of 10 (in two instances, two different works by the same
artist were nominated, which were considered jointly). The below is not
definitive, nor is it comprehensive. Had this meeting happened on a
different day, with a different group, the results would have been
different. Some pieces were debated heavily; others were fleetingly
passed over, as if the group intuitively understood why they had been
brought up; a few were spoken of with appreciation and wonder. What came
out of the conversation was more of a sensibility than a declaration.
This list --- which is ordered chronologically, from oldest work to most
recent --- is who we circled around, who we defended, who we questioned,
and who we, perhaps most of all, wish might be remembered. \emph{---}
\href{https://www.nytimes.com/by/thessaly-la-force}{\emph{Thessaly La
Force}}

\emph{This conversation has been edited and condensed. The artwork
summaries are by Zoë Lescaze.}

\hypertarget{1-sturtevant-warhol-flowers-1964-71}{%
\subsection{1. Sturtevant, ``Warhol Flowers,''
1964-71}\label{1-sturtevant-warhol-flowers-1964-71}}

Image

Sturtevant's ``Warhol Flowers'' (1969-70).Credit...© Estate of
Sturtevant, courtesy of Galerie Thaddaeus Ropac, London, Paris, Salzburg

Image

Sturtevant's ``Warhol Flowers'' (1969-70).Credit...© Estate of
Sturtevant, courtesy of Galerie Thaddaeus Ropac, London, Paris, Salzburg

Known professionally by her surname,
\href{https://www.nytimes.com/2014/05/17/arts/design/elaine-sturtevant-appropriation-artist-is-dead-at-89.html}{Elaine
Sturtevant} (b. Lakewood, Ohio, 1924; d. 2014) began ``repeating'' the
works of other artists in 1964, more than a decade before
\href{https://www.nytimes.com/topic/person/richard-prince}{Richard
Prince} photographed his first Marlboro ad and
\href{https://www.nytimes.com/1987/09/12/arts/original-slant-on-originality.html}{Sherrie
Levine} appropriated the images of
\href{https://www.nytimes.com/topic/person/edward-weston}{Edward
Weston}. Her targets tended to be famous male painters (largely because
the work of women was less broadly recognized). Over the course of her
career, she imitated canvases by
\href{https://www.nytimes.com/2019/02/17/arts/design/frank-stella-black-paintings.html}{Frank
Stella},
\href{https://www.nytimes.com/2017/04/01/arts/james-rosenquist-dead-pop-art.html}{James
Rosenquist} and
\href{https://www.nytimes.com/topic/person/roy-lichtenstein}{Roy
Lichtenstein}, among others. Perhaps unsurprisingly, given his own
puckish understanding of authorship and originality,
\href{https://www.nytimes.com/2018/11/01/arts/design/andy-warhol-inc-how-he-made-business-his-art.html}{Andy
Warhol} approved of Sturtevant's project and even lent her one of his
``Flowers'' screens. Other artists, including
\href{https://www.nytimes.com/2017/10/16/t-magazine/claes-oldenburg.html}{Claes
Oldenburg}, were unamused, and collectors largely shied away from
purchasing the works. Gradually, however, the art world came around to
understanding her conceptual reasons for copying canonical works: to
skewer the grand modernist myths of creativity and the artist as lone
genius. By focusing on Pop Art, itself a comment on mass production and
the suspect nature of authenticity, Sturtevant was taking the genre to
its full logical extension. Playful and subversive, somewhere between
parody and homage, her efforts also echo the centuries-old tradition of
young artists copying old masters.

\hypertarget{2-marcel-broodthaers-musuxe9e-dart-moderne-duxe9partement-des-aigles-1968-72}{%
\subsection{2. Marcel Broodthaers, ``Musée d'Art Moderne, Département
des Aigles,''
1968-72}\label{2-marcel-broodthaers-musuxe9e-dart-moderne-duxe9partement-des-aigles-1968-72}}

Image

Left: Marcel Broodthaers's ``Musée d'Art Moderne, Département des
Aigles, Section Financière, à Vendre Pour Cause de Faillite''
(1970-71).~ Right: ``Musée d'Art Moderne, Département des Aigles,
Section des Figures'' (1972) at Städtische Kunsthalle,
Düsseldorf.Credit...© 2019 Estate of Marcel Broodthaers/Artists Rights
Society (ARS), N.Y./SABAM, Brussels. Photo (right)~by Maria Gilissen.

In 1968, \href{https://www.moma.org/collection/works/146915}{Marcel
Broodthaers} (b. Brussels, 1924; d. 1976) opened his nomadic museum, the
``Musée d'Art Moderne, Département des Aigles,'' complete with a staff,
wall labels, period rooms and slide carousels. His **``**Museum of
Modern Art'' existed in various locations, beginning with Broodthaers's
Brussels home, where the artist filled the space with storage crates for
people to use as seats and postcard reproductions of 19th-century
paintings. He painted the words ``musée'' and ``museum'' on two windows
facing the street. The museum, which gently mocked various curatorial
and financial aspects of traditional institutions, grew from there, with
sections identified as 17th century, folklore and cinema, among others.
At one point, Broodthaers had a gold bar stamped with an eagle, which he
intended to sell at twice its market value in order to raise money for
the museum. Failing to find a buyer, he declared the museum bankrupt and
put it up for sale. Nobody was interested enough to make a purchase, and
in 1972, he erected a new section of his museum in an actual
institution, the Kunsthalle Düsseldorf. There, he installed hundreds of
works and everyday objects --- from flags to beer bottles --- depicting
eagles, the symbol of his museum.

\hypertarget{3-hans-haacke-moma-poll-1970}{%
\subsection{3. Hans Haacke, ``MoMA Poll,''
1970}\label{3-hans-haacke-moma-poll-1970}}

Image

Hans Haacke's ``MoMA Poll'' (1970), recreated at the Central Pavilion at
the 2015 Venice Biennale.Credit...Courtesy of Contemporary Art Daily. ©
Hans Haacke/Artists Rights Society (ARS), N.Y./VG Bild-Kunst, Bonn

Image

A yellow ballot used for Haacke's ``MoMA Poll.''Credit...Courtesy of the
Museum of Modern Art Archives, New York. © Hans Haacke/Artists Rights
Society (ARS), N.Y./VG Bild-Kunst, Bonn

In 1969, the Guerrilla Art Action Group, an art workers' coalition,
called for the resignation of the Rockefellers from the board of the
Museum of Modern Art, believing the family was involved in the
manufacture of weapons (chemical gas and napalm) destined for Vietnam. A
year later,
\href{https://www.nytimes.com/2014/10/24/arts/design/hans-haacke-gets-establishment-nod-of-approval.html}{Hans
Haacke} (b. Cologne, Germany, 1936) took the fight inside the museum.
His seminal installation, ``MoMA Poll,'' presented visitors with two
transparent ballot boxes, a ballot and a sign that posed a question
about the upcoming gubernatorial race: ``Would the fact that Governor
Rockefeller has not denounced President Nixon's Indochina Policy be a
reason for you not to vote for him in November?'' (By the time the
exhibition closed, roughly twice as many participants had answered
``yes'' as ``no.'') MoMA did not censor the work, but not all
institutions were as tolerant. In 1971, just three weeks before it was
set to open, the Guggenheim Museum
\href{https://www.nytimes.com/1971/04/07/archives/the-guggenheim-cancels-haackes-show.html}{canceled}
what would have been the German artist's first major international solo
show when he wouldn't remove three provocative works. The same year,
Cologne's Wallraf-Richartz Museum refused to exhibit ``Manet-Projekt
'74,'' which examined the provenance of an
\href{https://www.nytimes.com/topic/person/edouard-manet}{Édouard Manet}
painting donated to that museum by a Nazi sympathizer.

\textbf{Thessaly La Force:} There's one work here that really looks at
the institution of the museum. Rirkrit, you listed Marcel Broodthaers's
piece.

\textbf{Rirkrit Tiravanija:} That's the beginning of breaking --- at
least for me --- the institution. The beginning, for me, in Western art,
to question that kind of accumulation of knowledge. I like the Hans
Haacke that's also on this list. Definitely on my list, but I didn't put
it down.

\textbf{Martha Rosler:} I put it down. Hans showed the audience that it
was part of a system. By collecting their opinions and information about
who they were, he was able to construct a picture. I thought that it was
transformative and riveting for anyone who was interested in thinking
about \emph{who} the art world was. Also because it was totally data
driven and it wasn't aesthetic. It was the revolutionary idea that the
art world itself was not outside the question of: Who are we? It gave a
lot of space for people to think systematically about things the art
world had relentlessly refused to recognize were systematic issues.

\hypertarget{4-philip-guston-untitled-poor-richard-1971}{%
\subsection{4. Philip Guston, ``Untitled (Poor Richard),''
1971}\label{4-philip-guston-untitled-poor-richard-1971}}

Image

Philip Guston's ``Untitled (Poor Richard)'' (1971).Credit...© Estate of
Philip Guston, courtesy of Hauser \& Wirth

Richard Nixon was up for re-election in 1971 when
\href{https://www.nytimes.com/2017/05/16/t-magazine/art/philip-guston-venice.html}{Philip
Guston} (b. Montreal, 1913; d. 1980) created an astounding, little-known
series of nearly 80 cartoons depicting the president's rise to office
and destructive tenure. In Guston's spindly line drawings, we see Nixon,
portrayed with a phallic nose and testicular cheeks, swimming on Key
Biscayne and drafting foreign policy in China with caricatured
politicians, including
\href{https://www.nytimes.com/topic/person/henry-a-kissinger}{Henry
Kissinger} as a pair of glasses; the president's pet dog, Checkers, also
makes cameos. Guston captures Nixon's bitterness and insincerity while
crafting a poignant meditation on the abuse of power. Despite its
enduring relevance, **** the series languished in Guston's studio for
more than 20 years following the artist's death in 1980; it was finally
exhibited and published in 2001. The drawings were shown most recently
in 2017 at Hauser \& Wirth in London.

\textbf{TLF:} Back to my larger question: What do we mean by
``contemporary''? Does anyone want to take a stab at that?

\textbf{RT:} I think Philip Guston's
\href{https://www.nytimes.com/2016/10/31/arts/design/philip-guston-and-his-barbed-pen-nixon-years.html}{series
of Nixon drawings} became completely contemporary because it's ---

\textbf{Torey Thornton:} A mirror of sorts.

\textbf{RT:} It's like talking about what we're looking at today.

\textbf{TLF:} Well, that's a question I had, too. Do some works of art
have the capacity to change over time? Do some get stuck in amber and
remain a mirror of that particular moment? What you're describing is a
current event changing the meaning of Guston's paintings and drawings.

\textbf{Kelly Taxter:} I think that absolutely happens.

\textbf{MR:} It's all about the institution. When you mentioned the
Guston piece, which is great, I was thinking, ``Yeah, but there's at
least two videotapes that were about the same exact thing.'' What about
``\href{https://www.moma.org/collection/works/118185}{Television
Delivers People}'' {[}a 1973 short film by
\href{https://www.nytimes.com/topic/person/richard-serra}{Richard Serra}
and \href{https://www.moma.org/artists/34941}{Carlota Schoolman}{]}? I'm
also thinking of
``\href{https://www.eai.org/titles/four-more-years}{Four More Years}''
{[}a documentary about the 1972 Republican National Convention{]} by
\href{http://www.vdb.org/artists/tvtv}{TVTV}, which was about Nixon, and
``\href{https://www.eai.org/titles/the-eternal-frame}{The Eternal
Frame}'' {[}a 1975 satirical re-creation of the John F. Kennedy
assassination by
\href{https://www.nytimes.com/2003/06/21/arts/doug-michels-radical-artist-and-architect-dies-at-59.html}{Ant
Farm} and \href{https://www.eai.org/artists/t-r-uthco/titles}{T.R.
Uthco}{]}, about the Kennedys.

Image

Left: Jacqueline Humphries's ``:)green'' (2016). Right: Charline von
Heyl's ``Poetry Machine \#3'' (2018).Credit...Left: courtesy of the
artist and Greene Naftali, N.Y. Right: courtesy of the artist and
Petzel, N.Y.

\textbf{TLF:} There aren't that many paintings on the lists.

\textbf{KT:} No. Wow. I didn't realize that until two days later. I love
painting, it's just not here.

\textbf{TLF:} Is painting not --- Torey, you're a painter ---
contemporary?

\textbf{TT:} It's old. I don't know. I tried to look at what types of
painting happened and then see who started it.

\textbf{RT:} I put Guston on my list.

\textbf{David Breslin:} On my longer list, I had
\href{https://www.nytimes.com/2002/01/27/magazine/an-artist-beyond-isms.html}{Gerhard
Richter}'s
\href{https://www.gerhard-richter.com/en/art/paintings/photo-paintings/baader-meinhof-56}{Baader-Meinhof
cycle} {[}a series of paintings titled ``October 18, 1977,'' made by
Richter in 1988, based on photographs of members of the Red Army
Faction, a German left-wing militant group that carried out bombings,
kidnappings and assassinations throughout the 1970s{]}. It speaks to the
history of countercultural formation. How, if one decides not to
peaceably demonstrate, what the alternatives are. How, in many ways,
some of those things could only be recorded or thought about a
decade-plus later. So, how can certain moments of participatory action
be thought about in their time, and then also in a deferred moment?

\textbf{KT:} I thought of all the women painters. I thought of
\href{https://www.greenenaftaligallery.com/artists/jacqueline-humphries}{Jacqueline
Humphries},
\href{http://www.petzel.com/artists/charline-von-heyl}{Charline von
Heyl},
\href{https://www.nytimes.com/2013/09/29/arts/design/amy-sillman-brings-together-abstraction-and-figuration.html}{Amy
Sillman},
\href{https://www.nytimes.com/2017/09/04/t-magazine/art/ugly-painting-laura-owens-karen-kilimnik-sam-mckinniss.html}{Laura
Owens}. Women taking up the very difficult task of abstraction and
bringing some meaning to it. That, to me, feels like important terrain
women have staked out in a really serious way. Maybe one or two of those
people deserve to be on this list, but somehow I didn't put them on.

\textbf{DB:} It's that problem of a body of work versus the individual.

\textbf{KT:} But am I going to pick one painting of Charline's? I can't.
I just saw that show at the
\href{https://www.nytimes.com/2018/11/08/t-magazine/tino-sehgal-hirshhorn-museum-art.html}{Hirshhorn
Museum} in Washington, D.C., and every painting in the last 10 years is
\emph{good}. Is one better than the other? It's this kind of practice
and this discourse around abstraction --- and what women are doing with
it --- that I think is the key.

\begin{center}\rule{0.5\linewidth}{\linethickness}\end{center}

\hypertarget{5-judy-chicago-miriam-schapiro-and-the-calarts-feminist-art-program-womanhouse-1972}{%
\subsection{5. Judy Chicago, Miriam Schapiro and the CalArts Feminist
Art Program, ``Womanhouse,''
1972}\label{5-judy-chicago-miriam-schapiro-and-the-calarts-feminist-art-program-womanhouse-1972}}

\includegraphics{https://static01.nyt.com/images/2019/07/15/t-magazine/15tmag-chicago/15tmag-chicago-videoSixteenByNineJumbo1600.png}

``Womanhouse'' existed for just one month, and few material traces of
the groundbreaking art project --- room-size installations in a derelict
Hollywood mansion --- survive. The collaborative project, conceived by
the art historian
\href{https://www.nytimes.com/2012/06/26/arts/design/paula-hays-harper-feminist-art-historian-dies-at-81.html}{Paula
Harper} and led by
\href{https://www.nytimes.com/2018/02/07/t-magazine/judy-chicago-dinner-party.html}{Judy
Chicago} (b. Chicago, 1939) and
\href{https://www.nytimes.com/2015/06/25/arts/design/miriam-schapiro-91-a-feminist-artist-who-harnessed-craft-and-pattern-dies.html}{Miriam
Schapiro} (b. Toronto, 1923; d. 2015), brought together students and
artists who put on some of the earliest feminist performances and
produced painting, craft and sculpture in one radical context. Working
brutally long hours without running water or heat, the artists and
students renovated the dilapidated building to house numerous
installations and showcase six performances. Chicago's ``Menstruation
Bathroom'' confronted visitors with a wastebasket overflowing with
tampons painted to look as if soaked with blood. Faith Wilding
\href{https://www.nytimes.com/2018/03/14/t-magazine/art/fiber-knitting-weaving-politics.html}{crocheted
a large weblike shelter} for ``Womb Room'' --- somewhere between a
cocoon and a yurt --- out of grasses, branches and weeds. Taken as a
whole, the works created a new paradigm for female artists interested in
women's collective history and their relationships to domesticity, sex
and gender.

\textbf{TLF:} I think what's interesting is that everything here is
strictly art. No one threw a curveball.

\textbf{KT:} Is ``Womanhouse'' strictly art? I don't know.

\textbf{MR:} What is it, if not art?

\textbf{KT:} Well, in the way that it existed. It came out of an art
school. It was ephemeral. It was a location that came and went.

\textbf{MR:} It was an exhibition space. It became a collective
installation.

\textbf{KT:} But then it went away, and, until recently, there was very
little documentation available \ldots{} I think it's art. I put it
there. It's certainly institutionalized.

\hypertarget{6-lynda-benglis-artforum-advertisement-1974}{%
\subsection{6. Lynda Benglis, Artforum advertisement,
1974}\label{6-lynda-benglis-artforum-advertisement-1974}}

Image

Lynda Benglis at her New York studio in 2010. Due to its graphic nature,
the artwork is not published here.Credit...Photo by Billie Scheepers

\href{https://www.nytimes.com/2011/02/13/arts/design/13benglis.html}{Lynda
Benglis} (b. Lake Charles, La., 1941) wanted the 1974 profile Artforum
was writing about her to be accompanied by a nude self-portrait.
\href{https://www.nytimes.com/2003/08/22/arts/john-coplans-83-an-artist-and-a-founder-of-artforum.html}{John
Coplans}, the editor in chief at the time, refused. Undaunted, Benglis
persuaded her New York dealer,
\href{https://www.nytimes.com/2016/10/11/t-magazine/art/paula-cooper-art-gallery-life-pictures.html}{Paula
Cooper}, to take out a two-page ad in the magazine (Benglis paid for
it). Readers opened the November issue of Artforum and saw a sun-tanned
Benglis striking a pose, hip cocked, staring down at the viewer through
pointy, white-framed sunglasses. She wears nothing else and holds an
enormous dildo between her legs. The image caused bedlam. Five editors
--- Rosalind Krauss, Max Kozloff,
\href{https://www.nytimes.com/1990/01/03/obituaries/lawrence-alloway-is-dead-at-63-art-historian-curator-and-critic.html}{Lawrence
Alloway}, Joseph Masheck and
\href{https://www.nytimes.com/2018/09/18/obituaries/annette-michelson-dead.html}{Annette
Michelson} --- wrote a scathing letter to the magazine condemning the ad
as a ``shabby mockery of the aims of {[}women's liberation{]}.'' The
critic
\href{https://www.nytimes.com/2006/12/09/arts/design/09rose.html}{Robert
Rosenblum} wrote a letter to the magazine congratulating Benglis for
exposing the prudishness of people who considered themselves arbiters of
avant-garde taste: ``Let's give three dildos and a Pandora's Box to Ms.
Benglis, who finally brought out of the closet the Sons and Daughters of
the Founding Fathers of the \emph{Artforum} Committee of Public Decency
and Ladies Etiquette.'' The ad became an iconic image of resistance to
the sexism and double standards that continue to pervade the art world.

\textbf{DB:} I'm surprised no one included
\href{https://www.nytimes.com/topic/person/cindy-sherman}{Cindy
Sherman}. {[}Between 1977 and 1980, Sherman made a series of
black-and-white photographs of herself posing in various stereotypical
female roles, titled
``\href{https://www.nytimes.com/2014/10/17/arts/design/cindy-shermans-untitled-film-stills-go-to-auction-.html}{Untitled
Film Stills}.''{]}

\textbf{KT:} I had such a hard time with that. It was one of those
things that I was like, ``This is going to be on other peoples' lists.
It's so obvious, I'm not going to put it down.''

\textbf{TLF:} No one did.

\textbf{RT:} Well, I have Lynda Benglis's Artforum ad, which has a
relation to photography later on.

\textbf{MR:} I thought that was really good.

\textbf{KT:} I wanted to put Sherrie Levine's
``\href{http://www.afterwalkerevans.com/}{After Walker Evans}'' {[}in
1981, Levine exhibited reproductions of
\href{https://lens.blogs.nytimes.com/2013/08/08/a-new-look-at-walk-evanss-american-photographs/}{Depression-era
photographs} by Walker Evans that she rephotographed, questioning the
value of authenticity{]}, but didn't because \ldots{} I don't why. I ran
out of room in the '80s.

\hypertarget{7-gordon-matta-clark-splitting-1974}{%
\subsection{7. Gordon Matta-Clark, ``Splitting,''
1974}\label{7-gordon-matta-clark-splitting-1974}}

Image

Gordon Matta-Clark's ``Splitting'' (1974).Credit...© 2019 Estate of
Gordon Matta-Clark/Artists Rights Society (ARS), N.Y.

\href{https://www.davidzwirner.com/artists/gordon-matta-clark}{Gordon
Matta-Clark} (b. New York City, 1943; d. 1978) trained as an architect
at Cornell University. By the 1970s, he was working as an artist,
cutting chunks out of vacant properties, documenting the voids and
exhibiting the amputated bits of architecture. Abandoned buildings were
easy to find at the time --- New York City was economically depressed
and crime-ridden. Matta-Clark was looking for a new site when the art
dealer
\href{https://www.nytimes.com/2002/06/10/arts/holly-solomon-adventurous-art-dealer-is-dead-at-68.html}{Holly
Solomon} offered him a house she owned in suburban New Jersey that was
slated for demolition. ``Splitting'' (1974) was one of Matta-Clark's
first monumental works. With the help of the craftsman Manfred Hecht,
among other assistants, Matta-Clark sliced the whole thing in two with a
power saw, then jacked up one side of the structure while they beveled
the cinder blocks beneath it before slowly lowering it back down. The
house cleaved perfectly, leaving a slender central gap through which the
sunlight could enter the rooms. The piece was demolished three months
later to make way for new apartments. ``It was always exciting working
with Gordon,'' Hecht once said. ``There was always a good chance of
getting killed.''

\textbf{TLF:} Why is there no land art?

\textbf{RT:} I have Gordon Matta-Clark.

\textbf{MR:} Is that land art?
``\href{https://www.nytimes.com/2017/03/13/arts/design/spiral-jetty-is-named-an-official-state-work-of-art-by-utah.html}{Spiral
Jetty}'' {[}the giant coil of mud, salt and basalt constructed in 1970
at Rozel Point, Utah, by the American sculptor
\href{https://www.nytimes.com/topic/person/robert-smithson}{Robert
Smithson}{]} is land art.

\textbf{TT:} That's crazy! The jetty 100 percent has to be on my list.

\textbf{KT:}
``\href{https://www.nytimes.com/2012/06/08/arts/design/lightning-field-restoration-campaign-is-set.html}{The
Lightning Field}'' {[}a 1977 work by the American sculptor
\href{https://www.nytimes.com/2013/07/27/arts/design/walter-de-maria-artist-on-grand-scale-dies-at-77.html}{Walter
De Maria} comprising 400 stainless steel poles staked in the New Mexico
desert{]},
``\href{https://www.nytimes.com/2007/11/25/arts/design/25fink.html}{Roden
Crater}'' {[}the American light artist
\href{https://www.nytimes.com/topic/person/james-turrell}{James
Turrell}'s still-in-progress naked-eye observatory in Northern
Arizona{]}.

\textbf{TT:} I thought, ``Who can see it? What does `influence' mean,
what does it mean to be influenced through seeing something on a
screen?'' I was thinking, ``Do I list what I've seen versus what I've
obsessed over?'' At that point, it's all a reproduction or a sort of
theatrical representation.

\textbf{MR:} Totally.

Image

A collaged photograph of Gordon Matta-Clark's ``Splitting''
(1974).Credit...Courtesy of the estate of Gordon Matta-Clark and David
Zwirner

\textbf{TT:} I put
\href{https://tmagazine.blogs.nytimes.com/2010/05/28/just-looking-michael-ashers-all-nighter/}{Michael
Asher}'s show in the Santa Monica Museum {[}No. 19, see below{]} but
with something like that --- once it's gone, it's reproduction
\emph{only}. You can't visit it, it doesn't move somewhere else.

\textbf{TLF:} Are the questions that the land artists were asking ---
are they no longer questions we're asking today?

\textbf{TT:} There's no more land.

\textbf{MR:} It's a really interesting question. It's mainly that,
because of the move to the cities, we've become urban-obsessed. The
pastoral question --- which also applies to the cities, though we're not
that aware of it --- has receded. But am I wrong that the land-art stuff
was also in Europe? There were Dutch artists and English artists.

\textbf{RT:} Yeah, there were. Still are.

\textbf{MR:} Land art was international in an interesting way, which
coincided with the
\href{https://www.nytimes.com/interactive/projects/cp/summer-of-science-2015/latest/blue-marble-earth-photos-comparison}{Blue
Marble} {[}an image taken of Earth in 1972 by the crew of Apollo 17{]}.

\textbf{TLF:} The
\href{https://www.nytimes.com/1974/11/08/archives/-whole-earth-catalog-recycled-as-epilog-new-group-to-serve.html}{Whole
Earth Catalog}.

\textbf{MR:} Sure. The idea of the whole earth as an entity made up of
actual stuff rather than a social space.

\textbf{RT:} Maybe it also has to do with this idea of property and
wealth, too. The value of land and what it's used for has changed. It
used to be you could just go out in Montana and probably ---

\textbf{MR:} Bury some Cadillacs.

\textbf{RT:} --- dig a big hole. I mean,
\href{https://www.nytimes.com/2015/05/17/arts/design/michael-heizers-big-work-and-long-view.html}{Michael
Heizer} still does stuff, but it's only interior now. He's just doing
big rocks inside a space. Then again, that's why Smithson is
interesting, because it's almost like the non-site now {[}Smithson used
the term ``non-site'' to describe works that were presented outside
their original context, such as rocks from a New Jersey quarry exhibited
in a gallery alongside photos or maps of the site where they came
from{]}.

\textbf{TLF:} Then why did you include Gordon Matta-Clark?

\textbf{RT:} There are many references for me, but I feel like
``Splitting'' hits all the other things that I'm thinking about. With
``Splitting,'' it's like a comic ending. Also, the idea of the house
divided and what's happening with domesticity --- people aren't able to
sit together at Thanksgiving anymore.

\hypertarget{8-jenny-holzer-truisms-1977-79}{%
\subsection{8. Jenny Holzer, ``Truisms,''
1977-79}\label{8-jenny-holzer-truisms-1977-79}}

Image

One of Jenny Holzer's ``Truisms'' featured on a Spectacolor light board
in Times Square as part of the Public Art Fund's ``Messages to the
Public'' exhibition (1982).Credit...Artwork © 1982 Jenny Holzer, member
Artists Rights Society (ARS), N.Y, courtesy of Jane Dickson, project
initiator and animator, and Public Art Fund, N.Y. Photo by Lisa Kahane ©
1982 Lisa Kahane N.Y.C. Art Resource, N.Y.

\href{https://www.nytimes.com/2016/06/21/t-magazine/art/jenny-holzer-ibiza.html}{Jenny
Holzer} (b. Gallipolis, Ohio, 1950) was 25 years old when she began
compiling her ``Truisms,'' more than 250 cryptic maxims, terse commands
and shrewd observations. Culled from world literature and philosophy,
some of the one-liners are judgmental (``Any surplus is immoral''),
others bleak (``Ideals are replaced by conventional goals at a certain
age''), and a few echo the half-baked platitudes found in fortune
cookies (``You must have one grand passion''). The most resonant are the
political ones, none more so than ``Abuse of power comes as no
surprise.'' After printing them as posters, which she pasted among real
advertisements throughout downtown Manhattan, Holzer reproduced them on
objects, including baseball caps, T-shirts and condoms. She projected
them on the enormous Spectacolor LED board in Times Square in 1982, with
smaller scrolling signs to evoke the digital clocks and screens through
which we are continuously fed information (and told what to think) in
urban environments. Holzer continues to use the ``Truisms'' today,
incorporating them into electronic signs, benches, footstools and
T-shirts.

\textbf{DB:} Thessaly, when you asked earlier if Trump was in the room,
that's why I went to Jenny Holzer. In their original iterations,
``Truisms'' were these kind of street posters that people were marking
up ---

\textbf{MR:} But they were never not art-world things.

\textbf{DB:} I agree. They came out of the Whitney Independent Study
Program. But I think this is where the work takes on such a different
resonance now. The original intention behind them was that these codes
are free-floating and, of course, unconscious. But I think now the idea
that one is constantly assembling these truths, that it isn't a list of
unconsciousness, is really alive in that work.

\textbf{MR:} It's an interesting hypothesis. The reason I chose
\href{https://www.nytimes.com/2017/10/29/arts/barbara-kruger-designed-metrocards-are-coming-to-new-york-city.html}{Barbara
Kruger} {[}No. 11, see below{]} instead was that I thought she did an
interesting collision of fashion-world typography with this kind of punk
street-postering. She actually enunciates things people might cleverly
say but would never say in the art world: ``Your gaze hits the side of
my face.'' Or all kinds of feminist stuff: ``You construct intricate
rituals which allow you to touch the skin of other men.'' Who says stuff
like that? Who expects to be rewarded by capitalism for saying things
they don't want to hear? When Barbara joined a high-profile gallery, it
was a change in strategy just when the market recaptured all that
dissonant stuff that they had no idea what to do with. Finally the
market figured it out. Just let the artist do it, and we'll say it's art
and it's O.K.

\hypertarget{9-dara-birnbaum-technologytransformation-wonder-woman-1978-79}{%
\subsection{9. Dara Birnbaum, ``Technology/Transformation: Wonder
Woman,''
1978-79}\label{9-dara-birnbaum-technologytransformation-wonder-woman-1978-79}}

\includegraphics{https://static01.nyt.com/images/2019/07/15/t-magazine/15tmag-artlist-slide-C3P6/15tmag-artlist-slide-C3P6-superJumbo.jpg}

In an age of online piracy, supercuts, remixes, mash-ups and memes that
flare up and fizzle in minutes, it is difficult to appreciate how
radical it was to assemble art out of stolen TV clips 40 years ago. To
create her early masterpiece ``Technology/Transformation: Wonder
Woman,'' Dara Birnbaum (b. New York City, 1946) had to lay her hands on
the reels of the 1970s show ``Wonder Woman'' and re-edit them to tell a
different story. The piece opens with a looped explosion before we see
the actor
\href{https://www.nytimes.com/2018/03/31/style/where-is-lynda-carter-now-wonder-woman.html}{Lynda
Carter} twirl and transform from a meek secretary into a superhero.
There is violence, Birnbaum suggests, in requiring women to be either
demure office girls or scantily clad Amazons. Although Wonder Woman had
been heralded as a feminist role model, Birnbaum didn't buy it. ``I
wouldn't call that liberation,'' she
\href{http://www.artnews.com/2018/03/27/icons-dara-birnbaum/}{told
ARTnews} last year. ``How dare you confront me with this supposedly
super-powered image of a woman who is stronger than I am and can also
save mankind? I can't do that, and I won't.''

\textbf{MR:} Dara figured out how to get her work into the art world, as
opposed to the video people I named earlier, who weren't interested in
that. In the '70s, the dealer world couldn't figure out what to do with
the heterogeneity of works.

\hypertarget{10-david-hammons-bliz-aard-ball-sale-1983-how-ya-like-me-now-1988}{%
\subsection{10. David Hammons, ``Bliz-aard Ball Sale,'' 1983; ``How Ya
Like Me Now?,''
1988}\label{10-david-hammons-bliz-aard-ball-sale-1983-how-ya-like-me-now-1988}}

Image

David Hammons's ``Bliz-aard Ball Sale I'' (1983); ``How Ya Like Me
Now?'' (1988).Credit...Photo by Dawoud Bey, courtesy of Tilton Gallery,
New York; photo by Tim Nighswander/Imaging4art.com, courtesy of the
Glenstone Museum

\href{https://www.nytimes.com/2018/02/09/t-magazine/art/steve-cannon-david-hammons.html}{David
Hammons} (b. Springfield, Ill., 1943) studied art in Los Angeles at Otis
Art Institute (now Otis College of Art and Design) under
\href{https://www.nytimes.com/1979/10/06/archives/charles-w-white-is-dead-at-61-artist-with-work-in-49-museums.html}{Charles
White}, the painter acclaimed for his depictions of African-American
life. Hammons absorbed White's sense of social justice but gravitated
toward radical, unorthodox materials. Early on, he sought to challenge
the institutionalization of art, often creating ephemeral installations,
such as ``Bliz-aard Ball Sale,'' in which he sold snowballs of varying
sizes alongside New York street vendors and the homeless to critique
conspicuous consumption and hollow notions of value. (The ethos of the
piece continues to inform his engagement with the art world; he works
without exclusive gallery representation and rarely gives interviews.)
In 1988, he painted the Rev. Jesse Jackson, the African-American civil
rights activist who twice ran for the Democratic presidential
nomination, as a blond-haired, blue-eyed white man, a comment on how
skin color unfairly and arbitrarily determines opportunities. **** A
group of young African-American men who happened to walk by as the work
was being installed the following year in downtown Washington, D.C.,
perceived the painting as racist and smashed it with a sledgehammer.
(Jackson understood the artist's intentions.) The destruction --- and
the collective pain it represented --- became part of the piece. Now,
when Hammons exhibits the painting, he installs a semicircle of
sledgehammers around it.

\textbf{KT:} The ``Bliz-aard Ball Sale'' was a performance documented
with photographs. It falls into the legacy of performative ephemeral
works that begins with
\href{https://www.nytimes.com/2019/03/20/t-magazine/postmodern-dance.html}{Judson
Dance Theater} {[}a 1960s dance collective that included
\href{https://www.nytimes.com/1996/07/15/arts/robert-ellis-dunn-67-a-pioneer-in-postmodern-dance-movement.html}{Robert
Dunn},
\href{https://www.nytimes.com/2017/06/16/arts/dance/seven-seconds-of-yvonne-rainer-trio-a.html}{Yvonne
Rainer} and
\href{https://www.nytimes.com/2017/03/20/arts/dance/trisha-brown-dead-modern-dance-choreographer.html}{Trisha
Brown}, among many others{]} and the Happenings {[}a term coined by the
artist
\href{https://www.nytimes.com/2006/04/10/arts/design/allan-kaprow-creator-of-artistic-happenings-dies-at-78.html}{Allan
Kaprow} to describe loosely defined performance art pieces or events
that often involved the audience{]} of the 1960s. Why he stays relevant,
to some extent, is because so much of his work happens somehow in secret
--- his studio is the street. You can talk around what he's doing for a
very long time without coming up with a finite answer. He does not
follow a straight line and can be contradictory --- he defies
expectations.

\textbf{DB:} So much of the work begins from a place of opposition,
whether materially or at the site in which it's made or performed. I
chose ``How Ya Like Me Now?'' mostly for the ability to misread so much
about the work. In some ways, it's a point of danger. The fact that a
group of people took sledgehammers to it --- why weren't certain people
taking Jackson seriously as a candidate? Confusing the boundaries
between what's expected and what isn't makes Hammons always relevant.

\textbf{MR:} I think that work is really problematic, though. It defines
why we're talking about the art world. That work was offensive, and yet
we understand how to read something against its apparent presentation.
It speaks to us as educated people, and that's one of the reasons we
defend it. I love Hammons's work. But I always felt really strange about
that piece, because it didn't take into consideration that the community
might be offended. Or, he didn't give a damn. Which, you know, he's an
artist. So it's the art world speaking to the art world about this work.
But I also wonder about its problematic appearance just at that moment
when the public was turning against public art in general, and in
particular mysterious public art, which usually meant abstract. But this
was worse --- it was not only laughing at the public, it was laughing at
a \emph{specific} public, even if that wasn't his intention.

\hypertarget{11-barbara-kruger-untitled-when-i-hear-the-word-culture-i-take-out-my-checkbook-1985-untitled-i-shop-therefore-i-am-1987}{%
\subsection{11. Barbara Kruger, ``Untitled (When I Hear the Word
Culture, I Take Out My Checkbook),'' 1985; ``Untitled (I Shop Therefore
I Am),''
1987}\label{11-barbara-kruger-untitled-when-i-hear-the-word-culture-i-take-out-my-checkbook-1985-untitled-i-shop-therefore-i-am-1987}}

Image

Barbara Kruger's ``Untitled (When I hear the word culture, I take out my
check book)'' (1985); ``Untitled (I Shop Therefore I Am)''
(1987).Credit...Courtesy of Barbara Kruger

\href{http://www.barbarakruger.com/}{Barbara Kruger} (b. Newark, 1945)
briefly studied at the Parsons School of Design in 1965, but her real
education was in the world of magazines. She dropped out early on to
work at Mademoiselle as an assistant to the art director, rapidly became
head designer, and then switched to freelance, conceiving layouts for
House \& Garden*,* Vogue ** and ** Aperture, among other publications.
Through these projects, Kruger learned how to command the viewer's
attention and manipulate desire. A close reader of Roland Barthes and
other theorists focused on media, culture and the power of images,
Kruger brought her professional life and philosophical leanings together
in the early 1980s with her iconic works: agitprop images of terse,
satirical slogans in white or black Futura Bold Oblique type on
close-cropped images primarily from old magazines. They confront gender
roles and sexuality, corporate greed and religion. Several of the most
well-known indict consumerism, including 1985's ``Untitled (When I Hear
the Word Culture, I Take Out My Checkbook),'' in which the words slash
across the face of a ventriloquist's dummy, and ``Untitled (I Shop
Therefore I Am),'' from 1987.

\hypertarget{12-nan-goldin-the-ballad-of-sexual-dependency-1985-86}{%
\subsection{12. Nan Goldin, ``The Ballad of Sexual Dependency,''
1985-86}\label{12-nan-goldin-the-ballad-of-sexual-dependency-1985-86}}

Image

Left: Nan Goldin's ``C.Z. and Max on the Beach, Truro, Massachusetts''
(1976), from ``The Ballad of Sexual Dependency''. Right: a poster for
``The Ballad of Sexual Dependency'' in New York (1983).Credit...© Nan
Goldin, courtesy of the artist and Marian Goodman Gallery

When
\href{https://www.nytimes.com/2018/06/11/t-magazine/a-heroin-chic-photographers-new-project-tackling-the-opioid-epidemic.html}{Nan
Goldin} (b. Washington, D.C., 1953) moved to New York City in 1979, she
rented a loft on the Bowery and embarked on what would prove to be one
of the most influential photographic series of the century. Her subjects
were herself, her lovers and her friends --- drag queens, fellow drug
addicts, runaways and artists. We see them fight, make up, have sex,
apply makeup, shoot up and nod off in the several hundred candid images
comprising ``The Ballad of Sexual Dependency.'' Goldin first shared the
pictures as slide shows in downtown clubs and bars, partly out of
necessity (she lacked a darkroom to print but could get slides processed
at a drugstore), partly because these haunts were part of the world of
the photographs. Cult heroes and neighborhood stars, including
\href{https://www.nytimes.com/topic/person/keith-haring}{Keith Haring},
\href{https://www.nytimes.com/2018/05/02/t-magazine/andy-warhol-photo-portraits.html}{Andy
Warhol} and \href{https://www.nytimes.com/topic/person/john-waters}{John
Waters}, appear in some frames, but the focus is on Goldin's intimates,
including her glowering boyfriend Brian, who beat her nearly blind one
night: ``Nan One Month After Being Battered'' (1984) is one of the most
haunting portraits in the series. Goldin edited and reconfigured the
series repeatedly, eventually titling it after a song in Bertolt
Brecht's ``Threepenny Opera'' and setting it to a playlist that has
included \href{https://www.nytimes.com/topic/person/james-brown}{James
Brown}, the
\href{https://www.nytimes.com/topic/organization/the-velvet-underground}{Velvet
Underground},
\href{https://www.nytimes.com/2010/09/21/nyregion/21computer.html}{Dionne
Warwick}, opera, rock and blues. A version appeared in the 1985 Whitney
Biennial and the Aperture Foundation published a selection of 127 images
as a book in 1986, which includes some of Goldin's fiercely honest
writing. A decade later, most of the people pictured in the book had
died of AIDS or drug overdoses. In a recent exhibition at New York's
Museum of Modern Art, Goldin concluded the sequence of nearly 700
photographs with a nod to these losses --- a snapshot of two graffiti
skeletons having sex.

\textbf{KT:}
\href{https://www.nytimes.com/2018/01/22/arts/design/nan-goldin-oxycontin-addiction-opioid.html}{Nan
Goldin} continues to have a very prominent role in the discourse,
whether that's about the art itself, like what she's making, or the
problems that we're dealing with in the culture of the art world and
beyond. That body of work made visible a whole realm, a whole social
structure, a whole group of people who were invisible in a lot of ways.
It talked about the AIDS crisis. It talked about queer culture. It
talked about her abuse. It was like a confessional, laying bare things
that are still really relevant issues.

\textbf{MR:} It has the word ``sexual'' in it. Do you want to talk about
that a little bit?

\textbf{KT:} It has a lot to do with her relationship to sex and love,
and her friends' relationships to sex and love, and the unraveling of
it. There's a lot of dirt and degradation in it, and yet there is a lot
of celebration in it, too, I think: being able to see what one might see
as dirty or wrong as right. I saw it when I was a kid. Her prints are
super gorgeous, but sometimes they are just snapshots in the freedom of
the work itself, the freedom that she took with it.

\hypertarget{13-cady-noland-oozewald-1989-the-big-slide-1989}{%
\subsection{13. Cady Noland, ``Oozewald,'' 1989; ``The Big Slide,''
1989}\label{13-cady-noland-oozewald-1989-the-big-slide-1989}}

Image

Cady Noland's ``Oozewald'' (1989).Credit...Photo courtesy of Collection
M HKA/Clinckx, Antwerp

The work of \href{http://ensembles.mhka.be/items/2245/assets/2517}{Cady
Noland} (b. Washington, D.C., 1956) probes the dark corners of American
culture. Many of her installations, including
``\href{https://www.artic.edu/artworks/186274/the-big-slide}{The Big
Slide}'' (1989), involve rails or barriers --- allusions to the limits
on access, opportunity and freedom in this country. (To enter Noland's
debut exhibition at New York's White Columns Gallery in 1988, visitors
had to duck under a metal pole blocking the door.) ``Oozewald'' features
a silk-screened version of the famed photograph of President John F.
Kennedy's assassin, Lee Harvey Oswald, as he's being shot and killed by
the nightclub owner Jack Ruby. Eight oversize bullet holes perforate the
surface --- an American flag is wadded up inside one, where his mouth
would be. Noland disappeared from the art world around 2000, a move that
has become as much a part of her oeuvre as her work. While she can't
stop galleries and museums from displaying old pieces, disclaimers
noting the artist's lack of consent often appear on the exhibition
walls. In recent years, Noland has disowned some works entirely, roiling
the market. She has become known as the art world's boogeyman, but she
might be its conscience.

\textbf{TT:} I started having this thing happen where years later, after
thinking about an artist a lot, I started seeing how they've influenced
other artists. I realized Cady Noland is so \emph{everywhere} in a weird
way. Particularly within installation art and sculpture. I've seen a lot
of work recently that feels like it's really leaning on something she's
made. Sometimes, something is made in a certain time and then it loops
back, and it's relevant again. There's this overarching criticism or
analysis of Americana in her work. Her name came back in, and it's
around and around and around.

\textbf{MR:} Isn't that the way the art world always works? Everyone
hated Warhol. Even \emph{after} he was famous, the art world said,
``No.'' It's why we got minimalism.

\textbf{KT:} I think Cady occupies a place of resistance, too. I think
Cady's character --- both her resistant character and approach to her
work --- is part of the mythmaking of her practice. She's an elusive
Hammons-type figure. She's not speaking on the work. \emph{Everybody}
else is.

\textbf{DB:} So much of the work has to do with conspiracy and paranoia,
which feels way too ``right now.'' These things that have this immediate
conjuring, like the Oswald figure being shot, or with Clinton and the
Whitewater stuff that she does, with just the quick image of the figure
and a line from a newspaper article. It's her ability to distill the
information, to get to that paranoid tendency in American culture. To
your point, Kelly, when she's come up, it's been through lawsuits.

\textbf{MR:} Really?

\textbf{DB:} Yeah, she's suing people for how her work is treated. This
is a total guess on my part, but even if you think about \emph{that} as
being a mode of communication --- that if she's going to function
publicly, it's going to be through the legal system --- you see, even
now, I'm making a conspiracy out of it!

\textbf{KT:} You're paranoid!

\textbf{DB:} I think we all are.

\hypertarget{14-jeff-koons-ilona-on-top-rosa-background-1990}{%
\subsection{14. Jeff Koons, ``Ilona on Top (Rosa Background),''
1990}\label{14-jeff-koons-ilona-on-top-rosa-background-1990}}

Image

Jeff Koons with his then wife, Ilona Staller, in June 1992.~The artist
did not grant permission for the named work to be
published.Credit...Photo by Patrick Piel/Gamma-Rapho via Getty Images

\href{https://www.nytimes.com/topic/person/jeff-koons}{Jeff Koons} (b.
York, Pa., 1955) rose to prominence in the mid-1980s making conceptual
sculpture from vacuum cleaners and basketballs. When the Whitney Museum
of American Art invited him to create a billboard-size work for an
exhibition called ``Image World,'' the postmodern provocateur submitted
a blown-up, grainy photograph, printed on canvas, of himself and Ilona
Staller --- the Hungarian-Italian porn star he would later marry --- in
campy coital ecstasy, advertising an unmade film. The series that
followed, ``\href{http://www.jeffkoons.com/artwork/made-in-heaven}{Made
in Heaven},'' shocked viewers when it debuted at the Venice Biennale in
1990. With descriptive titles such as ``Ilona's Asshole'' and ``Dirty
Ejaculation,'' the photo-realistic paintings portrayed the couple in
every conceivable position. They appeared at a moment when the country
was divided over propriety in art, with religious and conservative
forces rallying against sexually explicit work. Koons has claimed it is
an exploration of freedom, an examination of the origins of shame, a
celebration of the act of procreation, even a vision of transcendence.
``I'm not interested in pornography,'' he said in 1990. ``I'm interested
in the spiritual.'' Koons destroyed portions of the series during a
protracted custody battle with Staller for their son, Ludwig.

\textbf{TLF:} Money defines the art world, too. There are certain
artists who reflect that but who no one named.

\textbf{KT:} I thought it was super interesting that we all didn't go to
that. There are many different art worlds. The one you're referring to
is one of them.

\textbf{MR:} What's your argument for keeping more commercial artists
off the list?

\textbf{KT:} In my opinion, because art is so much more than that. The
artists who are at that level are such a small percentage of the art
being made. I didn't grow up revering that work.

\textbf{TT:} I think there are a lot of younger artists now who are
subliminally or quietly trying to find a way in between, of being like,
``Oh, I'm really interested in the production of this type of studio,
but I also want to be more rigorous and hands-on with my practice.'' Or
maybe they're secretly obsessed with
\href{https://www.nytimes.com/2010/02/28/arts/design/28koons.html}{Jeff
Koons}, but it's not something they would ever say for a New York Times
interview. I'm not going to name any names, but I've heard it enough to
where I'm like, ``This is for real.''

\textbf{MR:} Could you name one or two artists you're talking about?

\textbf{TLF:} Name names.

\textbf{TT:} Is
\href{https://www.nytimes.com/topic/person/damien-hirst}{Damien Hirst}
an example?

\textbf{TLF:} Damien Hirst,
\href{https://www.nytimes.com/2016/02/14/t-magazine/art/takashi-murakami-art-collection-yokohama-museum.html}{Takashi
Murakami} \ldots{}

\textbf{KT:} Yeah, we left off Jeff Koons. We left off Damien Hirst.

\textbf{MR:} We did.

\textbf{TT:} I brought Jeff.

\textbf{KT:} I think they're present. I would like the conversation to
be about some other artists. I could have put in Damien.

\textbf{MR:} A more legitimate artist, in my opinion, than Jeff Koons.
But that's just me, sorry.

\textbf{TLF:} Well, who would you want to talk about, then, if we could?

\textbf{KT:} I would have picked
``\href{http://www.jeffkoons.com/artwork/equilibrium}{Equilibrium}''
{[}a series of works in the mid-1980s that included basketballs
suspended in tanks of distilled water{]}, if it were Jeff Koons. If it
were Damien Hirst, I would have put
``\href{http://www.damienhirst.com/the-physical-impossibility-of}{The
Physical Impossibility of Death in the Mind of Someone Living}'' {[}a
1991 piece consisting of a tiger shark preserved in formaldehyde in a
vitrine{]}. I think it's a really good piece that influenced artists on
this list, as did ``Equilibrium.'' Maybe they should be on the list.
Maybe we're being disingenuous. I'm totally fine with that. They're on
my long list. I just took them off. I wanted to talk about some other
people for a change, and some more women, frankly.

\textbf{TT:} I hear you. I agree with that.

\hypertarget{15-mike-kelley-the-arenas-1990}{%
\subsection{15. Mike Kelley, ``The Arenas,''
1990}\label{15-mike-kelley-the-arenas-1990}}

Image

Mike Kelley's ``Arena \#7 (Bears)'' (1990).Credit...Photo by Douglas
Parker © Mike Kelley Foundation for the Arts, all rights reserved/VAGA
at ARS, N.Y.

After dabbling in Detroit's music scene as a teenager,
\href{http://www.mikekelleyfoundation.org/}{Mike Kelley} (b. Wayne,
Mich., 1954; d. 2012) moved to Los Angeles to attend CalArts. In each of
the 11 works of ``The Arenas,'' originally exhibited at Metro Pictures
gallery in 1990, stuffed animals and other toys sit alone or in eerie
groups on dingy blankets. In
\href{https://www.christies.com/lotfinder/Lot/mike-kelley-b-1954-arena-11-5495764-details.aspx}{one},
a handcrafted bunny with a scraggly pompom tail is positioned on a
crocheted afghan before an open thesaurus, appearing to be studying the
entry on ``volition,'' as two cans of Raid threaten from a distance. In
\href{https://www.christies.com/lotfinder/Lot/mike-kelley-1954-2012-arena-8-leopard-5994792-details.aspx}{another},
a stuffed leopard is splayed atop an ominous lump beneath a
black-and-orange coverlet. The works summon up themes of perversion,
shame, dread, vulnerability and pathos. Kelley used toys because he felt
they revealed far more about how adults see children --- or want to see
them --- than they do about kids. ``The stuffed animal is a
pseudo-child,'' a ``cutified, sexless being that represents the adult's
perfect model of a child --- a neutered pet,'' he once wrote. But the
toys in Kelley's arrangements are faded, soiled, grubby and worn in
sordid ways.

\textbf{KT:} I think that a lot of Mike Kelley's work is about class but
also about abuse and other things that kids, at least when they're
teenagers, begin articulating and thinking about. That series of work
was so abject. There are layers of revelation in it that were pivotal
for me personally, and then as I got older, I realized it had a bigger
impact. And I see it in the work of some of the younger artists today.

\hypertarget{16-felix-gonzalez-torres-untitled-portrait-of-ross-in-la-1991}{%
\subsection{16. Felix Gonzalez-Torres, ``Untitled" (Portrait of Ross in
L.A.),
1991}\label{16-felix-gonzalez-torres-untitled-portrait-of-ross-in-la-1991}}

Image

Felix Gonzalez-Torres's ``Untitled'' (Portrait of Ross in L.A.)
(1991).Credit...Photo by Serge Hasenboehler, courtesy of the Felix
Gonzalez-Torres Foundation,~© Felix Gonzalez-Torres

\href{https://www.felixgonzalez-torresfoundation.org/}{Felix
Gonzalez-Torres} (b. Cuba, 1957; d. 1996) came to New York City in 1979.
When he created ``Untitled'' (Portrait of Ross in L.A.) in 1991, he was
mourning the loss of his lover, Ross Laycock, who had died of
AIDS-related illness that year. The installation ideally comprises 175
pounds of candies, wrapped in bright cellophane, an approximation of the
body weight of a healthy adult male. Viewers are free to take pieces
from the pile, and over the course of the exhibition, the work
deteriorates, just as Laycock's body did. The candies, however, may or
may not be routinely replenished by the staff, evoking eternity and
rebirth at the same time as they conjure mortality.

\textbf{DB:} The work engages where we are today, this idea about the
participatory and the experiential. Gonzalez-Torres also makes the point
about responsibility, that an onus comes with this kind of taking. The
idea, too, that it's referencing one person as the ideal body weight,
that the participatory element is not just this generalized mass thing,
that the referent is just one other person, I think is very profound.

\textbf{RT:} I was thinking about AIDS. I almost put the Act Up logo as
an artifact. We should talk about works of art that are more than just
art, addressing all those other conditions. I find it very beautiful in
that way.

\textbf{KT:} That work, in a metaphorical sense, is a virus. It
dissipates and goes into other people's bodies.

\textbf{RT:} I don't even know if the audience really understands.
That's the thing. They are just taking candies.

\textbf{TLF:} I certainly just thought I was taking candies.

\textbf{DB:} There's also the idea of replenishment. He comes back the
next day. The obligation to restore is so much different than the
obligation to take. The person is surviving. The institution is
refilling. You could go away one day and not know that this returns to
its own form. This idea of who knows and who doesn't, I think, is
important to it.

\hypertarget{17-catherine-opie-self-portraitcutting-1993}{%
\subsection{17. Catherine Opie, ``Self-Portrait/Cutting,''
1993}\label{17-catherine-opie-self-portraitcutting-1993}}

Image

Catherine Opie's ``Self-Portrait/Cutting'' (1993).Credit...© Catherine
Opie, courtesy of Regen Projects, Los Angeles, and Lehmann Maupin, New
York, Hong Kong and Seoul

In her photograph ``Self-Portrait/Cutting,''
\href{https://www.nytimes.com/search?query=Opie\%252C+Catherine}{Catherine
Opie} (b. Sandusky, Ohio, 1961) faces away from the viewer, confronting
us with her bare back, on which a house --- the kind a child might draw
--- and two stick figures in skirts have been carved. The figures hold
hands, completing the idyllic domestic dream, which, at the time was
just that --- a dream --- for lesbian couples. This work and others
responded to the national firestorm surrounding ``obscenity'' in art. In
1989, Senators Alfonse D'Amato and Jesse Helms had denounced
``\href{https://www.christies.com/lotfinder/Lot/andres-serrano-b-1950-piss-christ-5070403-details.aspx}{Piss
Christ},'' a photograph depicting a crucifix submerged in urine by
\href{https://www.nytimes.com/2019/04/10/arts/andres-serrano-lets-objects-do-the-talking.html}{Andres
Serrano}, which was part of a traveling exhibition that had received
funding from the National Endowment for the Arts. A few weeks later, the
Corcoran Gallery of Art in Washington, D.C. opted to cancel a show
featuring homoerotic and sadomasochistic photographs by
\href{https://www.nytimes.com/2018/11/23/t-magazine/robert-mappelthorpe-michael-cunningham-elif-batuman-hilton-als.html}{Robert
Mapplethorpe}, whose exhibition at the Institute of Contemporary Art at
the University of Pennsylvania had also received federal funding. In
1990, the N.E.A. denied funding to four artists because of their
explicit themes of frank sexuality, trauma or subjugation. (In 1998, the
Supreme Court ruled that the N.E.A.'s statute was valid and did not
result in discrimination against the artists, nor did it suppress their
expression.) By creating and exhibiting these works when she did, Opie
openly defied those looking to shame queer communities and censor their
visibility in art. ``She is an insider and an outsider,''
\href{https://www.nytimes.com/2008/09/26/arts/design/26opie.html}{wrote}
the Times art critic Holland Cotter on the occasion of Opie's 2008
Guggenheim midcareer retrospective. ``{[}Opie is{]} a documentarian and
a provocateur; a classicist and a maverick; a trekker and a
stay-at-home; a lesbian feminist mother who resists the gay mainstream;
an American --- birthplace: Sandusky, Ohio --- who has serious arguments
with her country and culture.''

\textbf{DB:} This question of intimacy --- who's trying to police what I
do with my body and how I choose to constitute what a family is --- all
these issues are one, if we're thinking about how some of these works
resound now. These are still things that we are urgently dealing with.
The presence of motherhood and parenting are profound in the work. The
vulnerability of presenting oneself to one's own camera like that, which
I think is also incredible in Goldin's work --- the question of who is
my world, and who do I want to be a part of it?

\textbf{MR:} In both their cases, it's about \emph{me and them}, which
is a huge thing that women brought. With the AIDS crisis, there were a
lot of works about ``me'' in the same way, but it was really a huge
change for Cathy and Nan to be the subject.

\textbf{KT:} Also, with Nan, this idea of a community in some sense of
collaboration. As opposed to a photographer taking a picture of you,
you're taking a picture \emph{with} you.

\hypertarget{18-lutz-bacher-closed-circuit-1997-2000}{%
\subsection{18. Lutz Bacher, ``Closed Circuit,''
1997-2000}\label{18-lutz-bacher-closed-circuit-1997-2000}}

\includegraphics{https://static01.nyt.com/images/2019/07/15/t-magazine/15tmag-artlist-slide-5T3B/15tmag-artlist-slide-5T3B-superJumbo.jpg}

\href{https://www.nytimes.com/2019/05/26/obituaries/lutz-bacher-dies-at-75.html}{Lutz
Bacher} (b. United States, 1943; d. 2019) is an anomaly in an age of
easily searchable biographies and online profiles. The artist used a
pseudonym, one that has obscured her original name. Few photos of her
face exist. Perhaps it is not surprising, then, that so many of Bacher's
works focus on questions of exposure, visibility and privacy. After Pat
Hearn, the famed downtown art dealer who represented her, was diagnosed
with liver cancer on January 22, 1997, Bacher installed a camera above
Hearn's desk, filming continuously for 10 months. We see Hearn sit, make
phone calls, meet with artists; Hearn is featured in the frame less and
less as her illness worsens. Bacher edited 1,200 hours of footage into
40 minutes of video stills upon the dealer's death in 2000, forming an
unusual window into the inner workings of a gallery, as well as an
intimate record of an influential woman as she stares down death.

\textbf{TLF:} Here's something that I'm wondering: Cady Noland, Lutz
Bacher and Sturtevant are --- elusive is one word, anonymous could be
another --- people. It's interesting that they resonate in a time when
there is so much celebrity.

\textbf{KT:} I don't think Lutz was ever elusive.

\textbf{MR:} I don't think so either.

\textbf{TLF:} Well, never really named.

\textbf{MR:} Pseudonymous.

\textbf{KT:} She had a name. It was Lutz.

\textbf{TT:} But there's only two images of her online versus a hundred
of someone else. The pressure to be so present in order for the work to
live properly is something I hear a lot.

\textbf{MR:} Look what happened when
\href{https://www.nytimes.com/topic/person/jackson-pollock}{Jackson
Pollock} wound up in Life magazine. The Abstract Expressionists
definitely didn't want to be turned into brands. More recently, curators
started asking crazy things, like, ``Put your picture up with your
label.'' No thank you. The Times reporters now even have little pictures
in their bios --- everybody's been personalized because we don't
remember that the work is supposed to stand for itself.

\hypertarget{19-michael-asher-michael-asher-santa-monica-museum-of-art-2008}{%
\subsection{19. Michael Asher, ``Michael Asher,'' Santa Monica Museum of
Art,
2008}\label{19-michael-asher-michael-asher-santa-monica-museum-of-art-2008}}

Image

Michael Asher's ``Michael Asher'' at the Santa Monica Museum of Art
(2008).Credit...Photo by Grant Mudford, courtesy of the Institute of
Contemporary Art, Los Angeles (I.C.A. L.A.)

\href{https://www.nytimes.com/2012/10/18/arts/design/michael-asher-artist-dies-at-69.html}{Michael
Asher} (b. Los Angeles, 1943; d. 2012) spent his career responding to
each gallery or museum space with site-specific works that illuminated
the architectural or abstract qualities of the venue. When the Santa
Monica Museum of Art (now the Institute of Contemporary Art, Los
Angeles) approached the conceptualist in 2001 to mount an exhibition, he
tapped into the history of the institution, recreating the wood or metal
skeletons of all of the temporary walls that had been built for the 38
previous exhibitions. The result was a labyrinth of studs that
effectively collapsed time and space, bringing multiple chapters of the
museum's history into the present. That work characterized his unique
practice over more than 40 years: In 1970, Asher removed all the doors
of an exhibition space at Pomona College in Claremont, Calif., to allow
light, air and sound into the galleries, calling viewers' attention to
the ways such places are usually closed off --- both literally and
metaphorically --- from the outside world; for a 1991 show at Paris's
Centre Pompidou, he searched all the books filed under
``psychoanalysis'' in the museum's library for abandoned paper
fragments, including bookmarks; in 1999, he created a
\href{http://moma.org/d/c/exhibition_catalogues/W1siZiIsIjMwMDA5OTYwMiJdLFsicCIsImVuY292ZXIiLCJ3d3cubW9tYS5vcmcvY2FsZW5kYXIvZXhoaWJpdGlvbnMvMTg1IiwiaHR0cDovL21vbWEub3JnL2NhbGVuZGFyL2V4aGliaXRpb25zLzE4NT9sb2NhbGU9cHQiLCJpIl1d.pdf?sha=9748391a05a89941}{volume
listing} nearly all of the artworks that the Museum of Modern Art in New
York had deaccessioned since its founding --- privileged information
rarely made public.

\hypertarget{20-ak-burns-and-al-steiner-community-action-center-2010}{%
\subsection{20. A.K. Burns and A.L. Steiner, ``Community Action
Center,''
2010}\label{20-ak-burns-and-al-steiner-community-action-center-2010}}

\includegraphics{https://static01.nyt.com/images/2019/07/15/t-magazine/15tmag-ak-burns/15tmag-ak-burns-superJumbo.png}

*``*Community Action Center,'' a 69-minute erotic romp through the
imaginations of artists \href{http://www.akburns.net/}{A.K. Burns} (b.
Capitola, Calif., 1975) and
\href{https://www.hellomynameissteiner.com/}{A.L. Steiner} (b. Miami,
1967) and their community of friends, is a celebration of queer
sexuality as playful as it is political. We watch as a diverse,
multigenerational cast engage in joyfully hedonistic acts of private and
shared pleasure involving paint, egg yolks, carwashes and corn on the
cob. Although the video opens with the cabaret star
\href{https://www.nytimes.com/2016/05/06/t-magazine/entertainment/my-10-favorite-books-justin-vivian-bond.html}{Justin
Vivian Bond} reading lines from Jack Smith's experimental film
``\href{https://www.moma.org/calendar/exhibitions/3754?locale=en}{Normal
Love},'' there is otherwise little dialogue. Instead, the focus is on
the dreamlike visuals --- captured with an offhand intimacy on rented
and borrowed cameras --- and the visceral sensations they evoke.
``Community Action Center'' is the rare ribald work that doesn't refer
to male desire or gratification, which is partly why Steiner and Burns,
who are activists as well as artists, describe it as ``socio-sexual.''
Radical politics needn't come at the cost of sensuality, however. The
piece is meant to titillate.

\textbf{KT:} It's a really important work, too.

\textbf{TLF:} I haven't seen it.

\textbf{KT:} They spearheaded this project to essentially make porn, but
it's much more than that, with all kinds of people from their queer
community. It includes so many artists that we know and that are making
work now, and very visible, but it was all about figuring out how to
show their body, show their sexuality, share their body, share their
sexuality, make light of it, make it serious, collaborate with
musicians. It's a crazy document of a moment that opened up a
conversation.

\hypertarget{21-danh-vo-we-the-people-2010-14}{%
\subsection{21. Danh Vo, ``We the People,''
2010-14}\label{21-danh-vo-we-the-people-2010-14}}

Image

Danh Vo's ``We the People'' (detail) (2011-16).Credit...Photo by Nils
Klinger, taken at Kunsthalle Fridericianum in Kassel, Germany

\href{https://www.publicartfund.org/exhibitions/view/danh-vo-we-the-people/}{Danh
Vo} (b. Vietnam, 1975) immigrated to Denmark with his family after the
fall of Saigon in 1979. ``We the People,'' a full-size copper replica of
the Statue of Liberty, may be his most ambitious work. Fabricated in
Shanghai, the colossal figure exists in roughly 250 pieces, dispersed
throughout public and private collections around the world. It will
never be assembled or exhibited as a whole. In its fragmented state,
Vo's statue alludes to the hypocrisy and contradictions of Western
foreign policy. A gift from France to the United States, dedicated in
1886, the original monument was billed as a celebration of freedom and
democracy --- values both nations proved willing to overlook when
dealing with other countries. At the time of the dedication, France
possessed colonies in Africa and Asia, including Vietnam, where a
miniature version of the statue was installed on the roof of the Tháp
Rùa temple (or Turtle Tower) in Hanoi. Later, the United States
financially supported the French military in Vo's home country, waging
war in the name of protecting democracy from Communism. By then, of
course, the Statue of Liberty had welcomed millions of immigrants to the
United States and had become a symbol of the American dream. In the wake
of current violent crackdowns on immigration at the U.S.-Mexico border,
Vo's fragmented icon has never felt more darkly apropos.

\textbf{DB:} I chose this because it totally takes away the masterpiece
idea. It's the one statue, with many meanings embedded within it, but
totally distributed. The sections are made in China, right?

\textbf{RT:} Yes.

\textbf{DB:} So it's also the idea that this object, which is synonymous
with the United States, is now made in what will be the superpower of
the future. It's signaling what other futures will be, and it gets back
to this idea that ``contemporary'' is a total unknowingness. We don't
know what the hell the ``contemporary'' is, and I think in some ways,
these works affirm that that unknowingness is where we begin.

\textbf{KT:} That work had so much violence and anger in it. Anger is a
big part of the work that's being made by artists now --- everyone's
feeling it --- specifically the anger of a displaced person. This idea
of what we've done as a country, all over the world.

\hypertarget{22-kara-walker-a-subtlety-or-the-marvelous-sugar-baby-2014}{%
\subsection{22. Kara Walker, ``A Subtlety, or the Marvelous Sugar
Baby,''
2014}\label{22-kara-walker-a-subtlety-or-the-marvelous-sugar-baby-2014}}

Image

Kara Walker's ``A Subtlety, or the Marvelous Sugar Baby, an Homage to
the Unpaid and Overworked Artisans Who Have Refined Our Sweet Tastes
From the Cane Fields to the Kitchens of the New World on the Occasion of
the Demolition of the Domino Sugar Refining Plant''
(2014).Credit...Photo by Jason Wyche © Kara Walker, courtesy of Sikkema
Jenkins \& Co., N.Y.

Ever since 1994, when the 24-year-old
\href{https://www.nytimes.com/topic/person/kara-walker}{Kara Walker} (b.
Stockton, Calif., 1969) first astounded audiences with cut-paper
installations depicting plantation barbarism, she has plumbed this
country's long history of racial violence. In 2014, Walker created ``A
Subtlety,'' a monumental polystyrene sphinx coated in white sugar. The
piece dominated an enormous hall of the Domino Sugar refinery in
Brooklyn, shortly before much of the factory was demolished for
condominiums. In a reversal of her black-paper silhouettes of white
slave owners, Walker gave the colossal white sculpture the features of a
stereotypical black ``mammy'' in a kerchief, the sort of imagery used by
molasses brands to market their product. Walker's sphinx also conjures
up forced labor in ancient Egypt. ``In my own life, in my own way of
moving through the world, I have a hard time making a distinction
between the past and the present,'' she
\href{https://observer.com/2014/05/kara-walker-on-domino-demolition-it-makes-me-very-sad/}{has
said}. ``Everything is kind of hitting me all at once.''

\textbf{MR:} ``A Subtlety'' made lots of people furious because it was
about the history of labor and sugar in a place that was already about
to be gentrified. It was this gigantic, mammy-like, sphinxlike, female
object, and then it had all these little melting children. ``A
Subtlety'' is part of a very longstanding tradition that began in the
Arab world that had to do with creating objects out of clay but also out
of sugar. So it's the impacted value of extractive mining, but it's also
the impacted value of the labor of slaves. And it's \emph{also} on the
site where wage slavery had occurred --- sugar work was the worst. The
Domino Sugar factory was once owned by the Havemeyers, and Henry
Havemeyer was one of the main donors to the Metropolitan Museum of Art.
The sugar king was the art king. So it had all of these things --- and
then there's the idea of all these people taking selfies in front of it.
It was extremely brilliant without having to say a thing.

\begin{center}\rule{0.5\linewidth}{\linethickness}\end{center}

Image

Left: Cindy Sherman's ``Untitled Film Still \#14'' (1978). Right: Robert
Mapplethorpe's ``Embrace'' (1982).Credit...Left: courtesy of the artist
and Metro Pictures, N.Y. Right: © Robert Mapplethorpe Foundation. Used
by permission.

\textbf{TLF:} Martha, you wrote to me in an email that you are against
the idea of the game-changing masterpiece. I thought we should put that
on the record.

\textbf{MR:} I'm happy to say that it makes no sense in a contemporary
era to talk about a work in isolation, because as soon as a work is
noticed, everybody then notices what the person did before or who was
around them. Art is not made in isolation. This brings me to the
``genius'': The masterwork and the genius go together. That was one of
the first things women artists attacked. As much as we revere the work
of Mike Kelley, he always said that everything he did depended on what
the feminists in L.A. had done before. What he meant by that, I believe,
was that abjection and pain and abuse are things that are worth paying
attention to in art. And that was something no man would have done at
that point, except
\href{https://www.nytimes.com/topic/person/paul-mccarthy}{Paul
McCarthy}, maybe. The masterpiece idea is highly reductive.

\textbf{KT:} This brings up a good point about how there's a
responsibility to question this. Is that how it's going to be?

\textbf{TT:} No, but listing a work that ``defines the contemporary
age'' doesn't necessarily mean it has to be a masterwork.

\textbf{MR:} Well, it could be a \emph{bad} masterwork. You could say
\href{https://www.nytimes.com/2019/01/09/arts/design/dana-schutz-painting-emmett-till-petzel-gallery.html}{Dana
Schutz} {[}the painter of a
\href{https://www.nytimes.com/2017/07/27/arts/design/dana-schutz-emmett-till-painting-protests.html}{controversial}
2016 work based on a photograph of the mutilated body of Emmett Till,
lynched, in his coffin{]}. But the questions of ownership go back to
Sherrie Levine and the Walker Evans work. What's ownership of an image?
What's reproduction of a photo? The culture wars of the '80s all
depended on photographs, whether it was ``Piss Christ'' or Robert
Mapplethorpe's work --- and we're still fighting these things. We don't
want to talk about them. Nobody here named Mapplethorpe --- interesting.

\textbf{KT:} Thought about it.

\textbf{MR:} Nobody mentioned
\href{https://www.nytimes.com/2016/10/17/t-magazine/william-eggleston-photographer-interview-augusten-burroughs.html}{William
Eggleston} because we really hate photography in the art world. Nobody
named
\href{https://www.nytimes.com/2018/07/03/lens/susan-meiselas-mediations.html}{Susan
Meiselas}. We always want photography to be something else, which is
art, which is actually what you said about Cindy Sherman's
``\href{https://www.moma.org/learn/moma_learning/cindy-sherman-untitled-film-stills-1977-80/}{Untitled
Film Stills}.'' We know it's not really photography. I'm always
interested in the way art is always ready to kick photography out of the
room unless called upon to say, ``Yeah but this was really important for
identity, formation or recognition.'' It's always thematic. It's never
formal.

\hypertarget{23-heji-shin-baby-series-2016}{%
\subsection{23. Heji Shin, ``Baby'' (series),
2016}\label{23-heji-shin-baby-series-2016}}

Image

Left: Heji Shin's ``Baby 12'' (2016). Right: Shin's ``Baby 16''
(2016).Credit...Courtesy of Galerie Buchholz, Berlin/Cologne

Birth is the subject of ``Baby,'' seven photographs by
\href{http://www.hejishin.com/}{Heji Shin} (b. Seoul, South Korea, 1976)
that capture the moments after crowning. Shin illuminates some of the
undeniably gory scenes with a scorching red light. Other pictures are
barely lit at all, and the puckered faces of the almost-born emerge from
menacing black shadows. While these photographs might remind us of our
common humanity, they are hardly sentimental or celebratory --- several
are downright scary. This complexity is at the core of Shin's practice,
from pornographic photographs of chiseled men dressed as beefcake cops
to
\href{https://www.galeriebuchholz.de/exhibitions/heji-shin-berlin-2019/}{colossal
portraits of Kanye West} that debuted shortly after the rapper's
inflammatory conversation with Donald Trump. (Two Kanye portraits and
five of the ``Babies'' were in the 2019 Whitney Biennial.) At a time
when political art is everywhere, with young artists telling predictably
left-leaning audiences exactly what they want to hear, Shin is an
outlier. Her photographs do not answer any questions. Instead, they ask
a lot of their audiences.

\textbf{TT:} I was obsessed with the ``Baby'' photos. I mean, I wanted
one myself. But then my partner was like, ``Well what's the \ldots{} ''
Like, ``I've seen pregnancy, what's the difference?''

\textbf{KT:} ``A kid could do that?''

\textbf{TT:} Or not quite that, but: I understand it aesthetically and
I'm interested in the photo, but what's it saying and what's it doing?

\textbf{KT:} No one wants to look at that work. No one wants to look at
that act. No one wants to talk about motherhood. No one wants to look at
women like that. No one wants to see a vagina like that. No one wants to
see a human being that looks like that. I think there's something gross
and revolting and very brave about that work.

\hypertarget{24-cameron-rowland-new-york-state-unified-court-system-2016}{%
\subsection{24. Cameron Rowland, ``New York State Unified Court
System,''
2016}\label{24-cameron-rowland-new-york-state-unified-court-system-2016}}

Image

Cameron Rowland's ``New York State Unified Court System''
(2016).~Credit...Cameron Rowland, ``New York State Unified Court
System,'' 2016, oak wood, distributed by Corcraft, 165 x 57.5 x 36
inches, rental at cost. ``Courtrooms throughout New York State use
benches built by prisoners in Green Haven Correctional Facility. The
court reproduces itself materially through the labor of those it
sentences. Rental at cost: Artworks indicated as `Rental at cost' are
not sold. Each of these artworks may be rented for 5 years for the total
price realized at police auction.'' Courtesy of the artist and Essex
Street, N.Y.

In a much-discussed 2016 exhibition titled ``91020000'' at the New York
nonprofit Artists Space,
\href{http://www.essexstreet.biz/artist/rowland}{Cameron Rowland} (b.
Philadelphia, 1988) exhibited furniture and other objects fabricated by
inmates often working for less than a dollar an hour, as well as heavily
footnoted research on the mechanics of mass incarceration. The New York
State Department of Corrections sells these commodities under the brand
name Corcraft to government agencies and nonprofit organizations.
Artists Space was eligible to acquire the benches, manhole cover rings,
firefighter uniforms, metal bars and other objects comprising the
exhibition, which Rowland rents to collectors and museums instead of
selling them. The spare installation recalled those of the Minimalist
sculptor
\href{https://www.nytimes.com/1994/02/13/obituaries/donald-judd-leading-minimalist-sculptor-dies-at-65.html}{Donald
Judd}, while Rowland's politically driven approach to Conceptualism and
focus on racial injustice garnered comparisons to Kara Walker and the
American light and text artist
\href{https://www.nytimes.com/2018/06/18/t-magazine/glenn-ligon-adrian-piper-art.html}{Glenn
Ligon}.
\href{https://www.newyorker.com/goings-on-about-town/art/cameron-rowland}{The
New Yorker} traced Rowland's artistic ancestry back to ``Duchamp, by way
of Angela Davis.''

\textbf{TT:} Cameron Rowland's work is further out on the edges of
what's considered art. You apply to get a catalog in order to purchase
prison goods. A lot of the work he makes, I don't even understand how. I
still have a lot of questions, and we're friends. There's this
unraveling of a new sort of sideways information that I find really
interesting and confusing at the same time.

\hypertarget{25-arthur-jafa-love-is-the-message-the-message-is-death-2016}{%
\subsection{25. Arthur Jafa, ``Love Is the Message, the Message Is
Death,''
2016}\label{25-arthur-jafa-love-is-the-message-the-message-is-death-2016}}

\includegraphics{https://static01.nyt.com/images/2019/07/15/t-magazine/15tmag-artlist-slide-F5QE/15tmag-artlist-slide-F5QE-videoSixteenByNine3000.jpg}

At a moment when the volume of images --- from pictures of suffering to
bathroom selfies --- threatens to preclude empathy, Arthur Jafa's
seven-and-a-half-minute video, ``Love Is the Message, the Message Is
Death,'' is a profoundly moving antidote to indifference. Through film
clips, TV broadcasts, music videos and personal footage, Jafa (b.
Tupelo, Miss., 1960) portrays the triumphs and terrors of black life in
America. We see the Rev. Martin Luther King Jr. and Miles Davis; Cam
Newton racing to score a touchdown; a Texas police officer slamming a
teenage girl onto the ground; Barack Obama
\href{https://www.nytimes.com/2015/07/04/arts/obamas-eulogy-which-found-its-place-in-history.html}{singing
``Amazing Grace''} at the Charleston church where nine people were
murdered by a white supremacist; and Jafa's daughter on her wedding day.
The film made its official art-world debut at Gavin Brown's Enterprise
in Harlem just days after Donald Trump won the presidential election in
November 2016. Jafa set the images to Kanye West's gospel-inflected
anthem ``Ultralight Beam.''

\textbf{TLF:} Jafa strikes me as more popular, in a sense, if I can use
that word. He crosses over into other worlds.

\textbf{TT:} This goes back to David Hammons because --- I threw away my
{[}Adidas Yeezy{]} sneakers Kanye made. {[}West alienated many of his
fans when he made
\href{https://www.nytimes.com/2018/10/11/us/politics/kanye-trump-white-house-monologue.html}{a
visit to the White House} in October 2018, offering his verbal support
of President Trump and wearing a Make America Great Again baseball
cap.{]}

\textbf{KT:} How do you justify that work, then? You still put Arthur
Jafa on the list, which is what I'm really curious about.

\textbf{TT:} Because it's not \emph{my} list. In my head, I thought,
``This is contemporary.'' And I think that a good artwork can be
problematic. Art is one of the few things that can transcend or
complicate a problem. ``Love Is the Message'' can still be a very good
artwork and I can disagree with Arthur Jafa's approach to it. No one
else has done that. No one else in history has produced a video like
that. It's still moving things forward, even if they're moving back a
little bit.

\textbf{DB:} I think Arthur Jafa is coming out of a lineage of collage
and photomontage artists --- from Martha Rosler, sitting right here, to
early artists coming out of the Russian avant-garde --- this idea that
you don't have to agree or adhere to a singular point of view. Each
image or piece of music doesn't mean something on its own; it's in the
juxtaposition where meaning comes together. What's so interesting about
the piece is how seductive it can be, and also, in some ways, it begs
for us to resist that seductive quality because of the violence of some
of the imagery.

\begin{center}\rule{0.5\linewidth}{\linethickness}\end{center}

Image

Left: LaToya Ruby Frazier's ``Momme'' (2018). Right: ``Students and
Residents Outside Northwestern High School (Est. 1964) Awaiting the
Arrival of President Barack Obama, May 4th 2016, III.''
(2016-17).Credit...Courtesy of the artist and Gavin Brown's Enterprise,
New York/Rome

\href{https://www.nytimes.com/2018/10/17/t-magazine/carrie-mae-weems-influence-latoya-ruby-frazier-laurie-simmons.html}{LaToya
Ruby Frazier} (b. Braddock, Pa., 1982) was raised in an economically
ravaged suburb of Pittsburgh, where she began photographing her family
at the age of 16. In arresting pictures of her terminally ill
grandmother, dilapidated homes, shuttered businesses and air thick with
pollution, Frazier exposed the effects of poverty and political
indifference on working-class African-Americans. Using her camera as a
weapon of social justice, Frazier highlights the effects of trickle-down
economics, union busting and other policies that have widened the wealth
gap across the nation. Frazier's series was published as a book,
``\href{https://lens.blogs.nytimes.com/2014/10/14/latoya-ruby-fraziers-notion-of-family/}{The
Notion of Family,}'' in 2014. Since then, she has pursued her blend of
art and activism, embedding herself in Flint, Mich., and other
marginalized communities.

\textbf{MR:} I am surprised not to see LaToya on this list. Maybe she's
too young?

\textbf{TLF:} Why don't you state the case for why you'd like to see
her?

\textbf{MR:} Because she's not only a sharp, clear and intelligent
observer of black life but specifically of female-centered,
working-class, black life in a small city in the Rust Belt. Most of the
African-American artists we think about deal with urban-centered
questions and relationships. But she knows how to put together activism
with social critique in a way that many other people have been afraid to
deal with --- not just with black identity but also class identity. She
documented the closure of the hospital in Braddock, Pa., and called
attention to the fact that the residents' physical conditions resulted
from living in a town polluted by industry and waste dumping. I think
she's pushed the boundaries of photography in the art world.

\begin{center}\rule{0.5\linewidth}{\linethickness}\end{center}

\emph{Source photographs and videos at top, in order of appearance:
copyright Estate of Sturtevant, courtesy of Galerie Thaddaeus Ropac,
London, Paris, Salzburg; courtesy of Dara Birnbaum, Electronic Arts
Intermix, New York and Marian Goodman Gallery; courtesy of Gavin Brown's
Enterprise, New York/Rome; Studio Danh Vo; courtesy of Barbara Kruger; ©
Judy Chicago/Artists Rights Society (ARS), New York, and courtesy of
Through the Flower Archives; courtesy of Arthur Jafa and Gavin's Brown
Enterprise, New York/Rome; courtesy of collection M HKA/clinckx,
Antwerp; David Seidner; copyright Lutz Bacher, courtesy of Greene
Naftali, New York and Galerie Buchholz, Berlin/Cologne; courtesy of the
Estate of Gordon Matta-Clark and David Zwirner; the Museum of Modern Art
Archives, New York; Juergen Frank/Contour RA by Getty Images; © Dawoud
Bey, Stephen Daiter Gallery, Chicago and Rena Bransten Gallery, San
Francisco; Patrick Piel/Gamma-Rapho via Getty Images}

Advertisement

\protect\hyperlink{after-bottom}{Continue reading the main story}

\hypertarget{site-index}{%
\subsection{Site Index}\label{site-index}}

\hypertarget{site-information-navigation}{%
\subsection{Site Information
Navigation}\label{site-information-navigation}}

\begin{itemize}
\tightlist
\item
  \href{https://help.nytimes.com/hc/en-us/articles/115014792127-Copyright-notice}{©~2020~The
  New York Times Company}
\end{itemize}

\begin{itemize}
\tightlist
\item
  \href{https://www.nytco.com/}{NYTCo}
\item
  \href{https://help.nytimes.com/hc/en-us/articles/115015385887-Contact-Us}{Contact
  Us}
\item
  \href{https://www.nytco.com/careers/}{Work with us}
\item
  \href{https://nytmediakit.com/}{Advertise}
\item
  \href{http://www.tbrandstudio.com/}{T Brand Studio}
\item
  \href{https://www.nytimes.com/privacy/cookie-policy\#how-do-i-manage-trackers}{Your
  Ad Choices}
\item
  \href{https://www.nytimes.com/privacy}{Privacy}
\item
  \href{https://help.nytimes.com/hc/en-us/articles/115014893428-Terms-of-service}{Terms
  of Service}
\item
  \href{https://help.nytimes.com/hc/en-us/articles/115014893968-Terms-of-sale}{Terms
  of Sale}
\item
  \href{https://spiderbites.nytimes.com}{Site Map}
\item
  \href{https://help.nytimes.com/hc/en-us}{Help}
\item
  \href{https://www.nytimes.com/subscription?campaignId=37WXW}{Subscriptions}
\end{itemize}
