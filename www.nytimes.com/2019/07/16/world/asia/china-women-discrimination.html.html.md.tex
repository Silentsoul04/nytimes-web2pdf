Sections

SEARCH

\protect\hyperlink{site-content}{Skip to
content}\protect\hyperlink{site-index}{Skip to site index}

\href{https://www.nytimes.com/section/world/asia}{Asia Pacific}

\href{https://myaccount.nytimes.com/auth/login?response_type=cookie\&client_id=vi}{}

\href{https://www.nytimes.com/section/todayspaper}{Today's Paper}

\href{/section/world/asia}{Asia Pacific}\textbar{}A Prosperous China
Says `Men Preferred,' and Women Lose

\url{https://nyti.ms/2lgkBV6}

\begin{itemize}
\item
\item
\item
\item
\item
\item
\end{itemize}

Advertisement

\protect\hyperlink{after-top}{Continue reading the main story}

Supported by

\protect\hyperlink{after-sponsor}{Continue reading the main story}

\hypertarget{a-prosperous-china-says-men-preferred-and-women-lose}{%
\section{A Prosperous China Says `Men Preferred,' and Women
Lose}\label{a-prosperous-china-says-men-preferred-and-women-lose}}

\includegraphics{https://static01.nyt.com/images/2019/05/24/world/00china-women-top/merlin_139349787_ac284300-6802-48df-be86-75a3b19d7815-articleLarge.jpg?quality=75\&auto=webp\&disable=upscale}

By \href{https://www.nytimes.com/by/amy-qin}{Amy Qin}

\begin{itemize}
\item
  July 16, 2019
\item
  \begin{itemize}
  \item
  \item
  \item
  \item
  \item
  \item
  \end{itemize}
\end{itemize}

\href{https://cn.nytimes.com/china/20190717/china-women-discrimination/}{阅读简体中文版}\href{https://cn.nytimes.com/china/20190717/china-women-discrimination/zh-hant/}{閱讀繁體中文版}\href{https://www.nytimes.com/es/2019/07/21/china-discriminacion-genero-mujeres}{Leer
en español}

TIANJIN, China --- Bella Wang barely noticed the section on the
application inquiring whether she was married or had children. Employers
in China routinely ask women such questions, and she had encountered
them before in job interviews.

It was a surprise, though, after she accepted a position as a manager at
the company, a big language-training business in the northern city of
Tianjin, when she was told the job came with a condition.

As a married woman without children, she would have to sign a ``special
agreement'' promising not to get pregnant for two years. If she broke
that promise, the company said, she could be fired, without
compensation.

Ms. Wang, 32, fluent in English with a degree in international trade,
was outraged --- but she signed.

Such agreements
\href{https://www.nytimes.com/2019/02/21/world/china-gender-discrimination-workplace.html?searchResultPosition=5}{are
illegal but increasingly common} in China, where discrimination against
women is on the rise. From the womb to the workplace, from the political
arena to the home, women in China are losing ground at every turn.

Driving this regression in women's status is a looming aging crisis, and
the relaxing of the draconian ``one-child'' birth restrictions that
contributed to the graying population. The Communist Party now wants to
try to
\href{https://www.nytimes.com/2018/08/11/world/asia/china-one-child-policy-birthrate.html}{stimulate
a baby boom}.

But instead of making it easier for women to both work and have
children, China's leader, Xi Jinping, has led a resurgence in
traditional gender roles that has increasingly pushed women back into
the home.

``When the state policymakers needed women's hands, they sent them to do
labor,'' said \href{https://irwg.umich.edu/people/wang-zheng}{Wang
Zheng}, professor of women's studies and history at the University of
Michigan. ``Now they want to push women into marriage and have a bunch
of babies.''

In a stark turnaround from the early decades of Communist rule,
officials now look the other way when employers, reluctant to cover
costs related to maternity leave, openly pick men over women for hiring
and promotions. At home, women are increasingly disadvantaged in divorce
and losing out on gains in the country's property boom.

As a result, Chinese women are being squeezed out of the workplace by
employers who penalize them if they have children, and by party
officials urging them to focus on domestic life. At the same time, those
who have managed to keep working are increasingly earning less relative
to men.

\includegraphics{https://static01.nyt.com/images/2019/05/24/world/00china-women-1/merlin_152723676_bfff651c-60d2-4eab-b053-431e860f1e68-articleLarge.jpg?quality=75\&auto=webp\&disable=upscale}

Mao famously told women they held up ``half the sky'' and outlawed
arranged marriage and the practice of taking concubines. Despite
political turmoil and persistent bias, Chinese women entered the work
force in record numbers, began to enjoy greater rights and were
celebrated for their economic contributions.

Thirty years ago, when the country first began implementing market
reforms, Chinese women earned just under
\href{http://www.stats.gov.cn/tjsj/tjgb/qttjgb/qgqttjgb/200203/t20020331_30606.html}{80
percent} of what men made. By 2010, according to the
\href{http://www.china.com.cn/zhibo/zhuanti/ch-xinwen/2011-10/21/content_23687810.htm}{latest
official data}, the average income of women in Chinese cities had fallen
to 67 percent that of men, and in the countryside 56 percent.

In a break with the Marxist ambition of liberating women from
patriarchal oppression, President Xi has called on women to embrace
their ``unique role'' in the family and ``shoulder the responsibilities
of taking care of the old and young, as well as educating children.''

``No Communist leader before Xi has dared to openly say that women
should shoulder the domestic burden,'' Professor Wang said.

Eager to preserve the stability of the family unit, the party has also
done little to help women following a
\href{https://www.nytimes.com/2011/09/08/world/asia/08iht-letter08.html}{recent
court ruling} that weakened their claim to property in divorce
proceedings. And with divorce numbers on the rise, millions of Chinese
women have been cut out of the nation's real-estate boom, experts say.

To be sure, with China's rapid economic transformation, women are
\href{http://www.chinadaily.com.cn/china/2017-07/26/content_30256796.htm}{living
longer}, earning
\href{https://core.ac.uk/download/pdf/41447073.pdf}{more money} and
\href{https://www.1xuezhe.exuezhe.com/Qk/art/465154?dbcode=1\&flag=2\&logohome=1}{graduating
from university} in greater numbers than ever before.

But the country's gains have disproportionately benefited men. Gender is
now one of the most important factors behind income inequality in China,
perhaps more so than even the longstanding divide separating Chinese
cities and the countryside, according to a
\href{https://onlinelibrary.wiley.com/doi/10.1111/cwe.12266}{recent
study}.

Over the past decade, China's ranking in the World Economic Forum's
\href{http://www3.weforum.org/docs/WEF_GGGR_2018.pdf}{global gender gap
index} has declined significantly --- from 57th out of 139 countries in
2008 to 103rd in 2018.

China once enjoyed one of the highest rates of female labor force
participation in the world, with nearly three in four women working as
recently as 1990. Now the figure is down to 61 percent, according to the
\href{https://data.worldbank.org/indicator/sl.tlf.cact.fe.zs}{International
Labor Organization}.

``When it came to promoting women's rights, China used to be in the
lead,'' said Feng Yuan, a feminist scholar in Beijing. ``But now we are
falling behind.''

Image

``The approach to raising children has totally changed,'' said Wang Yan,
a stay-at-home mother in the eastern city of Yantai. Many women are
leaving the work force as a result.Credit...Yan Cong for The New York
Times

\hypertarget{either-way-we-will-lose}{%
\subsection{`Either way, we will lose'}\label{either-way-we-will-lose}}

Since signing the special agreement two years ago, Ms. Wang has been
terrified of getting pregnant, and for good reason: In her first months
on the job, a pregnant co-worker was fired.

Ms. Wang wanted to have a baby, too, she recalled, but signed the
contract because she was excited about the job. Reporting her employer
to the authorities also seemed unlikely to do much good.

``I'm still a Chinese woman,'' she said recently in a coffee shop in
Tianjin. ``Even though we have some complaints, we cannot risk bringing
them up. Because either way, we will lose.''

Forced to choose between career and family, Ms. Wang chose career. Many
other Chinese women are dropping out of the work force.

The return of Chinese women to the home began in the 1980s, when mass
layoffs at state factories meant women were often the first to be let
go. It accelerated with rising expectations around child rearing.

Wang Yan, 35, a stay-at-home mother in the eastern city of Yantai, said
that her parents ``only needed to make sure their kids weren't hungry.''

Image

Office workers taking a lunch break in Beijing last year. Women in
Chinese cities earn 67 percent of what men make on average, and that gap
is growing.Credit...Gilles Sabrié for The New York Times

Now, facing a more competitive economy, parents, usually mothers, are
expected to supervise homework, after-school tutoring and
extracurricular activities --- all while navigating safety scandals
involving
\href{https://www.nytimes.com/2013/07/26/world/asia/chinas-search-for-infant-formula-goes-global.html}{baby
formula},
\href{https://www.nytimes.com/2017/11/24/world/asia/beijing-kindergarten-abuse.html}{day
care} and
\href{https://www.nytimes.com/2018/07/23/world/asia/china-vaccines-scandal-investigation.html}{vaccinations}.

At work, managers are eager to rid their payrolls of women who might
need maternity leave.

Since 2012, China has required companies to offer at least 14 weeks of
paid leave to women having children. Fathers typically get two weeks.
The disparity means help-wanted ads often openly specify ``men only'' or
``men preferred.''

This is illegal, but even government agencies do it. One ministry in
Beijing specified ``men only'' for more than half the jobs it advertised
over the course of a year, an
\href{https://www.hrw.org/report/2018/04/23/only-men-need-apply/gender-discrimination-job-advertisements-china}{investigation}
by Human Rights Watch found.

Employers often see women like Ms. Wang who are married without children
as the biggest gamble for hiring or promotions. And reports abound of
pregnant women being reassigned to less important positions, or
returning from leave to find their jobs have been filled.

As a result, opportunities for women to advance to company leadership
roles have stagnated in recent years. Only 21 percent of Chinese
companies had women in top manager roles last year, according to the
World Economic Forum's gender gap report.

The problem has become more apparent since 2015, when party leaders,
worried about the impact of slowing population growth on the economy,
\href{https://www.nytimes.com/2015/10/30/world/asia/china-end-one-child-policy.html}{ended
the one-child policy} and began allowing all couples to have two
children.

In an \href{http://www.wsic.ac.cn/academicnews/90644.htm}{official
survey} in 2017, about 54 percent of women said they had been asked
about their marriage and childbearing status in job interviews.

Beijing issued a
\href{https://www.nytimes.com/2019/02/21/world/china-gender-discrimination-workplace.html}{directive}
in February urging stronger enforcement of laws against gender
discrimination. But it has not been a priority, and the party-controlled
courts have not sided with women on other issues.

Image

China's highest court has made it harder for many women to win the
family home in divorce proceedings. That has cut millions of women out
of the real estate boom.Credit...Lam Yik Fei for The New York Times

\hypertarget{a-mans-law}{%
\subsection{`A man's law'}\label{a-mans-law}}

When Sharon Shao approached several divorce lawyers in the spring of
2013, they all had the same advice: Don't bother taking your husband to
court. You have no hope of getting the apartment.

It did not matter that she had been the primary breadwinner for most of
their marriage and had made all the mortgage payments.

It did not matter that he hit her. It did not matter that he had cheated
on her.

None of it mattered because her husband's parents had put up the down
payment and because her name was not on the property title.

Under a ruling issued by China's highest court in 2011, the lawyers
said, that meant the apartment was his.

For Ms. Shao, 36, who had no other home because her parents died when
she was young, it was devastating. ``After the divorce, I wandered
around with no sense of belonging,'' she said. ``I was just floating.''

Growing numbers of women in China have been through a similar
experience. In a country where real estate accounts for over
\href{https://www.ncbi.nlm.nih.gov/pmc/articles/PMC4589866/}{70 percent}
of personal wealth, the high court's ruling has been a significant
setback for women.

Chinese law had previously recognized a family's home as joint property
in divorce proceedings. But
\href{https://www.nytimes.com/2011/09/08/world/asia/08iht-letter08.html}{the
2011 ruling} held that real estate purchased before marriage, either
outright or on mortgage, should revert to the buyer in a divorce --- and
that is usually the husband.

Driven by the popular belief that a woman will only marry a man if he
owns a home, families often save for years to help their sons buy an
apartment. Experts say the high court was responding to fears that women
were using marriage to swindle their in-laws out of their savings.

Image

A mother and a child in Tiananmen Square last year. The end of the
``one-child'' policy has led to hiring discrimination by employers
worried about rising maternity leave costs.Credit...Gilles Sabrié for
The New York Times

Though the ruling makes no distinction between men and women, it is a
``man's law,'' said Lü Xiaoquan, a lawyer at Beijing Qianqian Law Firm.

There are about 31 million more men in China than women, an imbalance
caused by a traditional preference for sons, the one-child policy and
sex-selective abortions.

But Chinese women often accept marriage on unfavorable terms.

One 2012 survey by Horizon China, a research firm in Beijing, found that
70 percent of married women contributed financially to the family's
purchases of real estate but that less than a third of home deeds
included the woman's name. Researchers at Nankai University in Tianjin
in 2017 examined 4,253 property deeds and found the wife's name listed
on only
\href{https://www.academia.edu/35082071/_and_ldquo_\%E6\%88\%BF\%E4\%BA\%A7\%E8\%AF\%81\%E4\%B8\%8A\%E7\%9A\%84\%E7\%88\%B1\%E6\%83\%85_and_quot_and_mdash_and_mdash_and_mdash_\%E5\%A9\%9A\%E5\%A7\%BB\%E5\%B8\%82\%E5\%9C\%BA\%E4\%B8\%8E\%E5\%AE\%B6\%E5\%BA\%AD\%E6\%88\%BF\%E4\%BA\%A7\%E4\%BA\%A7\%E6\%9D\%83\%E5\%88\%86\%E5\%B8\%83}{about
one in five}.

These missing names have been disastrous for women in divorce
proceedings since the 2011 ruling, said
\href{https://www.letahongfincher.com/about}{Leta Hong Fincher}, author
of a book about the subject.

``The entire deck is stacked against women in so many ways,'' she said.

Image

A couple having a drink in Chongqing. Women are under pressure to marry
early to avoid becoming ``leftover women,'' a derogatory term for those
who remain single into and past their late 20s.Credit...Lam Yik Fei for
The New York Times

Ms. Shao, who graduated with a degree in computer science from one of
China's top universities, said her ex-husband suggested investing in an
apartment together even before they were married. At the time, he was
finishing a doctorate and she was making about \$600 a month as a
computer programmer.

His parents made the \$29,000 down payment, as a gift and investment,
and she agreed to cover the \$450 monthly payments.

``I was just very foolish, very innocent,'' she recalled.

Ms. Shao asked her ex-husband to add her name to the deed several times,
but he always talked her out of it, arguing that she could enjoy
benefits as a new buyer later if they invested in another property, she
recalled.

Years later, after they married and moved to Shanghai, Ms. Shao
discovered he was having an affair. Because she had proof that she made
the mortgage payments, her relatives managed to negotiate a cash
settlement for her.

Most women in China, though, have fewer options, and many end up with
nothing in a divorce. Others choose to remain in even abusive marriages.

Image

Some women in China have remained in unhappy or even abusive marriages
because they might get little or nothing in a divorce.Credit...Lam Yik
Fei for The New York Times

\href{https://www.nytimes.com/2018/07/26/world/asia/china-metoo.html}{Taking
cues} from \#MeToo activism overseas and China's
\href{https://www.nytimes.com/interactive/2018/obituaries/overlooked-qiu-jin.html}{own
history of feminism}, some Chinese women have staged street protests and
campaigns on social media for greater rights.

There are also broader signs of dissatisfaction among Chinese women: The
marriage rate
\href{http://www.xinhuanet.com/politics/2019-04/02/c_1124313756.htm}{fell
last year to its lowest point} since Mr. Xi took power, and the
birthrate dropped to a level unseen in the 70-year history of the
People's Republic of China.

The divorce rate is climbing, too, with women initiating most cases. In
\href{http://mzj.beijing.gov.cn/news/root/tjnb/2019-04/129817.shtml}{Beijing},
the authorities reported one divorce for every two marriages in 2017.

``They aren't having kids and getting married,'' said Lü Pin, a
prominent Chinese feminist activist. ``That's their way of pushing
back.''

Advertisement

\protect\hyperlink{after-bottom}{Continue reading the main story}

\hypertarget{site-index}{%
\subsection{Site Index}\label{site-index}}

\hypertarget{site-information-navigation}{%
\subsection{Site Information
Navigation}\label{site-information-navigation}}

\begin{itemize}
\tightlist
\item
  \href{https://help.nytimes.com/hc/en-us/articles/115014792127-Copyright-notice}{©~2020~The
  New York Times Company}
\end{itemize}

\begin{itemize}
\tightlist
\item
  \href{https://www.nytco.com/}{NYTCo}
\item
  \href{https://help.nytimes.com/hc/en-us/articles/115015385887-Contact-Us}{Contact
  Us}
\item
  \href{https://www.nytco.com/careers/}{Work with us}
\item
  \href{https://nytmediakit.com/}{Advertise}
\item
  \href{http://www.tbrandstudio.com/}{T Brand Studio}
\item
  \href{https://www.nytimes.com/privacy/cookie-policy\#how-do-i-manage-trackers}{Your
  Ad Choices}
\item
  \href{https://www.nytimes.com/privacy}{Privacy}
\item
  \href{https://help.nytimes.com/hc/en-us/articles/115014893428-Terms-of-service}{Terms
  of Service}
\item
  \href{https://help.nytimes.com/hc/en-us/articles/115014893968-Terms-of-sale}{Terms
  of Sale}
\item
  \href{https://spiderbites.nytimes.com}{Site Map}
\item
  \href{https://help.nytimes.com/hc/en-us}{Help}
\item
  \href{https://www.nytimes.com/subscription?campaignId=37WXW}{Subscriptions}
\end{itemize}
