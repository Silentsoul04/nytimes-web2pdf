Sections

SEARCH

\protect\hyperlink{site-content}{Skip to
content}\protect\hyperlink{site-index}{Skip to site index}

\href{https://www.nytimes.com/section/arts/music}{Music}

\href{https://myaccount.nytimes.com/auth/login?response_type=cookie\&client_id=vi}{}

\href{https://www.nytimes.com/section/todayspaper}{Today's Paper}

\href{/section/arts/music}{Music}\textbar{}Beyoncé Reimagines `The Lion
King' as Global 21st-Century Pop

\url{https://nyti.ms/30RVebB}

\begin{itemize}
\item
\item
\item
\item
\item
\item
\end{itemize}

Advertisement

\protect\hyperlink{after-top}{Continue reading the main story}

Supported by

\protect\hyperlink{after-sponsor}{Continue reading the main story}

Critic's Pick

\hypertarget{beyoncuxe9-reimagines-the-lion-king-as-global-21st-century-pop}{%
\section{Beyoncé Reimagines `The Lion King' as Global 21st-Century
Pop}\label{beyoncuxe9-reimagines-the-lion-king-as-global-21st-century-pop}}

On a companion to the remade film's official soundtrack, the pop
superstar seeks full-fledged fusions with African musicians.

\includegraphics{https://static01.nyt.com/images/2019/07/25/arts/23album-beyonce/merlin_157950375_176502dd-d833-42e6-8e21-5660ec625f46-articleLarge.jpg?quality=75\&auto=webp\&disable=upscale}

\href{https://www.nytimes.com/by/jon-pareles}{\includegraphics{https://static01.nyt.com/images/2018/06/14/multimedia/author-jon-pareles/author-jon-pareles-thumbLarge.png}}

By \href{https://www.nytimes.com/by/jon-pareles}{Jon Pareles}

\begin{itemize}
\item
  July 24, 2019
\item
  \begin{itemize}
  \item
  \item
  \item
  \item
  \item
  \item
  \end{itemize}
\end{itemize}

\begin{itemize}
\tightlist
\item
  The Lion King: The Gift\\
  **NYT Critic's Pick
\end{itemize}

Beyoncé flexes both her musicianship and her cultural leverage with
``The Lion King: The Gift,'' her companion album to the state-of-the-art
\href{https://www.nytimes.com/2019/07/11/movies/the-lion-king-review.html}{remake
of ``The Lion King.''} It's her latest lesson in commandeering
mass-market expectations, as she bends ``The Lion King'' to her own
agenda of African-diaspora unity, self-worth, parental responsibility
and righteous ambition.

Beyoncé was an obvious choice to be cast in an anointed blockbuster: the
25th-anniversary update of ``The Lion King,'' the 1994 animated Disney
parable set in Africa. Its story of a young lion fleeing and then
reclaiming his birthright had already generated a
\href{https://www.nytimes.com/1997/11/14/movies/theater-review-cub-comes-of-age-a-twice-told-cosmic-tale.html}{1997
Broadway adaptation} --- still running --- and movie sequels. Beyoncé
has a voice role in the new version as the brave, conscientious lioness
Nala; she also, of course, sings on the soundtrack.

On the official soundtrack album, Beyoncé joins in a remake of
\href{https://www.youtube.com/watch?v=0_USvdbYS1g}{``Can You Feel the
Love Tonight,''} the Oscar-winning song that ended the original ``Lion
King,'' and caps the existing soundtrack songs with her new one,
\href{https://www.youtube.com/watch?v=civgUOommC8}{``Spirit,''} a
dynamic secular-gospel exhortation to ``Rise up!'' Beyoncé wrote and
produced ``Spirit'' with the British producer Labrinth and with Ilya
Salmanzadeh, a member of Max Martin's Swedish songwriting stable; it's
also on ``The Gift.''

Image

Each song on ``The Lion King: The Gift'' is a coalition, almost always a
trans-Atlantic one.

But ``The Gift'' goes much further. With Beyoncé as executive producer
and a songwriter and performer on most of its tracks, it's essentially
an alternative soundtrack album, tied to the plot of ``The Lion King''
(and interspersed with dialogue snippets) but decidedly more Afrocentric
and more attuned to women's strengths and experiences.

On ``The Gift,'' the movie's plot points are springboards for songs like
``Keys to the Kingdom,'' ``Scar'' and ``Already.'' The album's first
full song, ``Bigger,'' is at once maternally protective and acutely
aware of generational cycles and, as the video clip emphasizes,
ecological interdependence: ``You're part of something way bigger,''
Beyoncé sings, adding, ``I'll be the roots/You be the tree,'' as a
somber beat gathers under churchy keyboard chords. She follows
``Bigger'' with a paternal counterpart: ``Find Your Way Back (Circle of
Life),'' with Beyoncé recalling a father's lessons on a track that
samples the \href{https://www.youtube.com/watch?v=nCbjK4259RM}{Nigerian
singer Niniola}.

Like many other Disney projects set outside the United States, in 1994
\href{https://www.nytimes.com/2019/07/11/movies/the-lion-king-review.html}{``The
Lion King''} fudged the specifics of a distant (from Hollywood) place
with a well-intentioned but hazy first-world perspective; Africa is just
Africa, without particular cultures, countries or regions. (It's also
unquestioningly celebrated as a patrilineal monarchy.) The wildlife and
landscape of ``The Lion King'' suggest the Serengeti plains of Tanzania
and Kenya, and its African names and words are in the Swahili language
--- all East African.

Meanwhile, the movie's music is largely non-African, steeped in
Hollywood and Broadway idioms, with an orchestral score by the German
composer Hans Zimmer
(\href{https://variety.com/2019/artisans/production/hans-zimmer-recreates-score-lion-king-1203271637/}{reworked
for the 2019 version}) and wordplay-loving, musical-theater-style songs
by two Englishmen, Elton John and the lyricist Tim Rice. At key moments
in the 1994 soundtrack, the South African musician Lebo M. (Lebohang
Morake) provided South African-style choir arrangements and his own
vocals, including the indelible opening incantation in ``Circle of
Life.'' He gets far more prominent billing in the remake.

Untethered to previous productions, Beyoncé has rethought ``The Lion
King'' as 21st-century global pop, frequently drawing on Africa. Her
throngs of collaborators include musicians, singers and producers from
the U.S., England, Sweden, Nigeria, South Africa, Ghana and Cameroon
(though not East Africa). It's a canny, forward-looking move, both
musically and with an eye to an international market that's increasingly
receptive to African innovations and non-English lyrics. Beyoncé even
sings in Swahili at the end of
\href{https://www.youtube.com/watch?v=PaMbTX-yDT0}{``Otherside,''} a
ballad invoking life after death.

American and British songwriters --- Paul Simon, David Byrne, Peter
Gabriel,
\href{https://www.nytimes.com/2019/06/25/arts/music/santana-buika-africa-speaks.html}{Carlos
Santana} --- have all found renewal in African music, as jazz musicians
did before them. With ``The Lion King: The Gift,'' Beyoncé joins their
ranks soulfully and attentively, seeking full-fledged fusions. She mixes
(apparently) personal thoughts and archetypal ones; she savors musical
hybrids and rhythmic challenges; and she digs in to every line she
sings.

Internationalism reigns.
\href{https://www.youtube.com/watch?v=V_ZsbqSg4aE}{``My Power''} ---
with Beyoncé alongside
\href{https://www.nytimes.com/2018/06/06/arts/music/tierra-whack-whack-world-interview.html}{Tierra
Whack} from Philadelphia, Yemi Alade from Nigeria and Nija, Busiswa,
Moonchild Sanelly and DJ Lag from South Africa --- is built on the deep
bass thuds and jittery double time percussion of the South African dance
music called gqom. In
\href{https://www.youtube.com/watch?v=essike_sirI}{``Water,''} Beyoncé
and Pharrell Williams are joined by Salatiel, a songwriter from
Cameroon, in a bouncy, sinuous track with leaping vocal inflections that
also includes a credit for a Ghanaian songwriter, Afriye. The track for
\href{https://www.youtube.com/watch?v=i978rjXw3P0}{``Mood 4 Eva,''}
Beyoncé's and Jay-Z's latest celebration of their luxurious life,
transforms a sample from the Malian singer and songwriter Oumou Sangaré.

Some of the album's guest performers have racked up tens of millions of
streams worldwide without extensive recognition --- yet --- in the U.S.
Prominent among them is a Nigerian contingent that draws on the crisp,
computerized rhythms that are known internationally as
\href{https://www.nytimes.com/2017/06/03/world/africa/nigeria-lagos-afrobeats-music-piracy-seyi-shay.html}{Afrobeats}
(and are clearly related to reggaeton's ubiquitous dembow rhythm via
West African-Caribbean roots and internet cross-pollination).

The album includes the Nigerian stars Burna Boy (who gets a song of his
own, \href{https://www.youtube.com/watch?v=8hlE_Fc2jk4}{``Ja Ara E,''}
that suavely warns, ``Watch out for them hyenas'') and Mr Eazi (who
shares \href{https://www.youtube.com/watch?v=P7xLNmYM8yk}{``Don't
Jealous Me''} with Tekno, Lord Afrixana and Yemi Alade and ``Keys to the
Kingdom'' with Tiwa Savage, all fellow Nigerians). Wizkid, the Nigerian
songwriter who collaborated with Drake on the worldwide hit
\href{https://www.youtube.com/watch?v=vcer12OFU2g}{``One Dance,''} duets
with Beyoncé to praise the beauty of a ``Brown Skin Girl''; the track
also has the voice of Beyoncé's and Jay-Z's daughter Blue Ivy Carter.

Each song on ``The Gift'' is a coalition, almost always a trans-Atlantic
one. And the African elements are at the core of the music; they're not
souvenirs or accessories. Unlike the movie that occasioned it, ``The
Lion King: The Gift'' is no remake or reiteration, no faraway fable. It
tells a story of its own.

\textbf{Various Artists}\\
``The Lion King: The Gift''\\
(Parkwood/Columbia)

Advertisement

\protect\hyperlink{after-bottom}{Continue reading the main story}

\hypertarget{site-index}{%
\subsection{Site Index}\label{site-index}}

\hypertarget{site-information-navigation}{%
\subsection{Site Information
Navigation}\label{site-information-navigation}}

\begin{itemize}
\tightlist
\item
  \href{https://help.nytimes.com/hc/en-us/articles/115014792127-Copyright-notice}{©~2020~The
  New York Times Company}
\end{itemize}

\begin{itemize}
\tightlist
\item
  \href{https://www.nytco.com/}{NYTCo}
\item
  \href{https://help.nytimes.com/hc/en-us/articles/115015385887-Contact-Us}{Contact
  Us}
\item
  \href{https://www.nytco.com/careers/}{Work with us}
\item
  \href{https://nytmediakit.com/}{Advertise}
\item
  \href{http://www.tbrandstudio.com/}{T Brand Studio}
\item
  \href{https://www.nytimes.com/privacy/cookie-policy\#how-do-i-manage-trackers}{Your
  Ad Choices}
\item
  \href{https://www.nytimes.com/privacy}{Privacy}
\item
  \href{https://help.nytimes.com/hc/en-us/articles/115014893428-Terms-of-service}{Terms
  of Service}
\item
  \href{https://help.nytimes.com/hc/en-us/articles/115014893968-Terms-of-sale}{Terms
  of Sale}
\item
  \href{https://spiderbites.nytimes.com}{Site Map}
\item
  \href{https://help.nytimes.com/hc/en-us}{Help}
\item
  \href{https://www.nytimes.com/subscription?campaignId=37WXW}{Subscriptions}
\end{itemize}
