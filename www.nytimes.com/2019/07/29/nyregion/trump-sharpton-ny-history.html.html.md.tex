Sections

SEARCH

\protect\hyperlink{site-content}{Skip to
content}\protect\hyperlink{site-index}{Skip to site index}

\href{https://www.nytimes.com/section/nyregion}{New York}

\href{https://myaccount.nytimes.com/auth/login?response_type=cookie\&client_id=vi}{}

\href{https://www.nytimes.com/section/todayspaper}{Today's Paper}

\href{/section/nyregion}{New York}\textbar{}How Trump and Sharpton
Became the Ultimate New York Frenemies

\url{https://nyti.ms/30YFT9a}

\begin{itemize}
\item
\item
\item
\item
\item
\item
\end{itemize}

Advertisement

\protect\hyperlink{after-top}{Continue reading the main story}

Supported by

\protect\hyperlink{after-sponsor}{Continue reading the main story}

\hypertarget{how-trump-and-sharpton-became-the-ultimate-new-york-frenemies}{%
\section{How Trump and Sharpton Became the Ultimate New York
Frenemies}\label{how-trump-and-sharpton-became-the-ultimate-new-york-frenemies}}

President Trump and the Rev. Al Sharpton have a history of using each
other for their own purposes, whether as foes or friends.

\includegraphics{https://static01.nyt.com/images/2019/07/29/nyregion/29trumpsharpton/29trumpsharpton-articleLarge.jpg?quality=75\&auto=webp\&disable=upscale}

\href{https://www.nytimes.com/by/james-barron}{\includegraphics{https://static01.nyt.com/images/2018/02/16/multimedia/author-james-barron/author-james-barron-thumbLarge.jpg}}\href{https://www.nytimes.com/by/jeffery-c-mays}{\includegraphics{https://static01.nyt.com/images/2018/07/18/multimedia/author-jeffery-c-mays/author-jeffery-c-mays-thumbLarge.png}}

By \href{https://www.nytimes.com/by/james-barron}{James Barron} and
\href{https://www.nytimes.com/by/jeffery-c-mays}{Jeffery C. Mays}

\begin{itemize}
\item
  July 29, 2019
\item
  \begin{itemize}
  \item
  \item
  \item
  \item
  \item
  \item
  \end{itemize}
\end{itemize}

\emph{{[}What you need to know to start the day:}
\href{https://www.nytimes.com/newsletters/newyorktoday?module=inline}{\emph{Get
New York Today in your inbox}}\emph{.{]}}

Both were enduring characters in New York, their names in headlines,
their faces on television. Both began their lives outside Manhattan ---
one was born in Brownsville, Brooklyn, the other in Jamaica, Queens ---
and both worked their way toward its irresistibly sizzling spotlight.
Both were flamboyant attention-getters who found fame on television,
\href{https://www.nytimes.com/2019/07/29/nyregion/trump-sharpton-ny-history.html}{Donald
J. Trump} as a reality-show host, the Rev. Al Sharpton as a cable news
commentator.

{[}\emph{Related:}
\href{https://www.nytimes.com/2019/07/29/us/politics/trump-al-sharpton.html}{\emph{Trump
lashes out at Al Sharpton, saying he ``hates whites.''}}\emph{{]}}

From time to time, they crossed paths, drawing energy from each other
even as foes, as
\href{https://www.nytimes.com/2002/10/23/nyregion/trump-draws-criticism-for-ad-he-ran-after-jogger-attack.html}{Mr.
Trump made claims} that Mr. Sharpton challenged --- about five black and
Latino teenagers who were charged with raping a white jogger in Central
Park in the 1980s, and about whether Barack Obama was born in the United
States.

But sometimes they were friends in the way that public figures in New
York can be. Mr. Trump cut the ribbon at Mr. Sharpton's National Action
Network annual convention in 2002, returning four years later to pose
with Mr. Sharpton, the Rev. Jesse Jackson and the singer James Brown.

``Different tune now,''
\href{https://twitter.com/thereval/status/1155793807797043201?s=21}{Mr.
Sharpton observed on Monday}.

The two have once again found themselves convenient foils, after
President Trump on Saturday denounced Representative Elijah E. Cummings,
calling the African-American congressman, a Democrat who represents much
of Baltimore, ``racist,'' and his district a ``disgusting, rat- and
rodent-infested mess.''

That was enough to draw Mr. Sharpton
\href{https://www.nytimes.com/2019/07/28/us/trump-baltimore.html}{to
Baltimore}, only to have Mr. Trump fire a pre-emptive strike, using
Twitter on Monday to assail Mr. Sharpton as ``a con man'' who ``Hates
Whites \& Cops!''

Mr. Trump has been irritated by Mr. Sharpton's increasing criticism of
him, believing that Mr. Sharpton has long since broken an informal
truce, according to aides and people who have spoken to him. The
president also believes that his attacks on Mr. Sharpton will appeal to
his base.

If so, this would hardly be the first time that one has used the other
for professional or political gain.

In 1989, Mr. Sharpton led demonstrations at Mr. Trump's Plaza Hotel
``because of what he did to the Central Park Five,'' he said in an
interview on Monday, referring to the defendants in the rape case.

It was one of the most widely publicized crimes of the 1980s, when crime
in New York was far more prevalent than it is now. The case pushed the
rawness of racial animosity into the public conversation, which was
nothing new in New York, but the jogger attack horrified the city.

Mr. Trump, who had
\href{https://www.nytimes.com/1988/03/27/nyregion/plaza-hotel-is-sold-to-donald-trump-for-390-million.html}{owned
the Plaza for about a year} when the jogger was assaulted, took out
full-page newspaper advertisements demanding the reinstatement of the
death penalty. The five defendants were convicted, but their sentences
were vacated 14 years later, based on DNA evidence and a confession from
another man. Mr. Trump has
\href{https://www.nytimes.com/2019/06/18/nyregion/central-park-five-trump.html}{refused
to apologize} for his actions or comments at the time.

By then, the two were already ``classic New York characters,'' recalled
George Arzt, the press secretary to Mayor Edward I. Koch. ``There was a
clash of who is the loudest voice in New York,'' he said, adding that
the two made Mr. Koch's tenure --- which was marred by racial tensions
--- more difficult.

``Koch's problem as mayor is that he had Trump on one side and Al
Sharpton on the other side,'' Mr. Arzt said. ``Both of them were
inflaming things.''

A year or so before the jogger in Central Park was attacked, Mr.
Sharpton played a large role in publicizing another case that stunned
New York: Tawana Brawley, a teenager from Wappingers Falls, N.Y., near
Poughkeepsie, said she had been kidnapped, tortured and raped by a group
of six white men who left her smeared with feces, and wrapped in a
plastic bag.

But the incident never happened; Mr. Sharpton was found
\href{https://www.nytimes.com/2001/06/15/nyregion/sharpton-s-debt-in-brawley-defamation-is-paid-by-supporters.html}{guilty
of defamation} for claiming that Steven A. Pagones, a former Dutchess
County assistant district attorney, had been involved in the assault.

More scrutiny followed, and Mr. Sharpton was indicted on a charge of
stealing at least \$250,000 from the National Youth Movement, the
precursor to his National Action Network. He was acquitted of all
charges.

Mr. Trump and Mr. Sharpton, who seemed to regain his footing quickly,
became more prominent over time, and there were overlaps.

In 2009, both made the Museum of the City of New York's list of the 400
New Yorkers who had made a difference in the 400 years since Henry
Hudson's voyage along the river that was later named for him.

``When you are in New York you know people that are high profile and
have a certain amount of power,'' Mr. Sharpton said in the interview. He
said he had ``no relationship with Donald Trump --- I've never been to
Mar-a-Lago. I never hung out with him.''

Mr. Trump, on Twitter, suggested otherwise. ``Went to fights with him \&
Don King, always got along well,'' Mr. Trump wrote. ``He `loved Trump!'
He would ask me for favors often. Al is a con man, a troublemaker,
always looking for a score. Just doing his thing. Must have intimidated
Comcast/NBC. Hates Whites \& Cops!''

Mr. Sharpton said he believed that Mr. Trump, a casino owner in the
1980s and 1990s, was more interested in using him to find favor with
local officials on the Atlantic City Council.

Mr. Sharpton said on Monday that the trip was ``totally transactional,
and I understood that.'' He added, ``Don and him were doing business.
Yeah, I would sit at the fights.''

Over the years, Mr. Trump and Mr. Sharpton have expressed grudging
admiration for each other. In a 2016 interview with Politico, Mr.
Sharpton remarked that the president has ``called me names on Fox and
all of that, but Donald Trump knows deep down in his heart that I
believe in what I'm doing, and I know that he believes in what he's
doing.''

Even in the president's calling Mr. Sharpton a con man and a
troublemaker on Twitter, he said that he had ``known Al for 25 years,''
adding that they ``always got along well.''

The president, in a follow-up Twitter post, said that Mr. Sharpton
``would always ask me to go to his events. He would say, `It's a
personal favor to me.'''

Mr. Trump added that Mr. Sharpton had visited him at Trump Tower
``during the presidential campaign to apologize for the way he was
talking about me. Just a con man at work!''

Mr. Sharpton said in the interview on Monday that the episode never
happened.

``Him saying that I met with him during the campaign in '16 is a lie,''
he said. The last time he saw Mr. Trump in person was in 2015, at the
40th anniversary broadcast of ``Saturday Night Live,'' he said ---
before Mr. Trump announced his candidacy for president.

``He came with a thumb handshake,'' Mr. Sharpton recalled, ``and he
said, `You gotta do what you gotta do, I gotta do what I gotta do.'''

As for being called a con man, Mr. Sharpton said that Mr. Trump must not
have meant it.

``If he really thought I was a con man, he'd be nominating me for his
cabinet,'' Mr. Sharpton said, to laughter at a news conference in
Baltimore on Monday. He then added that Mr. Trump had called him ``right
after he was elected.''

Maggie Haberman contributed reporting.

Advertisement

\protect\hyperlink{after-bottom}{Continue reading the main story}

\hypertarget{site-index}{%
\subsection{Site Index}\label{site-index}}

\hypertarget{site-information-navigation}{%
\subsection{Site Information
Navigation}\label{site-information-navigation}}

\begin{itemize}
\tightlist
\item
  \href{https://help.nytimes.com/hc/en-us/articles/115014792127-Copyright-notice}{©~2020~The
  New York Times Company}
\end{itemize}

\begin{itemize}
\tightlist
\item
  \href{https://www.nytco.com/}{NYTCo}
\item
  \href{https://help.nytimes.com/hc/en-us/articles/115015385887-Contact-Us}{Contact
  Us}
\item
  \href{https://www.nytco.com/careers/}{Work with us}
\item
  \href{https://nytmediakit.com/}{Advertise}
\item
  \href{http://www.tbrandstudio.com/}{T Brand Studio}
\item
  \href{https://www.nytimes.com/privacy/cookie-policy\#how-do-i-manage-trackers}{Your
  Ad Choices}
\item
  \href{https://www.nytimes.com/privacy}{Privacy}
\item
  \href{https://help.nytimes.com/hc/en-us/articles/115014893428-Terms-of-service}{Terms
  of Service}
\item
  \href{https://help.nytimes.com/hc/en-us/articles/115014893968-Terms-of-sale}{Terms
  of Sale}
\item
  \href{https://spiderbites.nytimes.com}{Site Map}
\item
  \href{https://help.nytimes.com/hc/en-us}{Help}
\item
  \href{https://www.nytimes.com/subscription?campaignId=37WXW}{Subscriptions}
\end{itemize}
