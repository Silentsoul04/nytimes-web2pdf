Sections

SEARCH

\protect\hyperlink{site-content}{Skip to
content}\protect\hyperlink{site-index}{Skip to site index}

\href{https://www.nytimes.com/section/world/asia}{Asia Pacific}

\href{https://myaccount.nytimes.com/auth/login?response_type=cookie\&client_id=vi}{}

\href{https://www.nytimes.com/section/todayspaper}{Today's Paper}

\href{/section/world/asia}{Asia Pacific}\textbar{}Pakistan Hits Back at
India Over Kashmir Move, Targeting Bilateral Trade

\url{https://nyti.ms/2TaJtdG}

\begin{itemize}
\item
\item
\item
\item
\item
\item
\end{itemize}

Advertisement

\protect\hyperlink{after-top}{Continue reading the main story}

Supported by

\protect\hyperlink{after-sponsor}{Continue reading the main story}

\hypertarget{pakistan-hits-back-at-india-over-kashmir-move-targeting-bilateral-trade}{%
\section{Pakistan Hits Back at India Over Kashmir Move, Targeting
Bilateral
Trade}\label{pakistan-hits-back-at-india-over-kashmir-move-targeting-bilateral-trade}}

\includegraphics{https://static01.nyt.com/images/2019/08/07/world/07kashmir1/merlin_158962737_23c73409-78e8-47b2-894a-19f2cee72642-articleLarge.jpg?quality=75\&auto=webp\&disable=upscale}

By \href{https://www.nytimes.com/by/kai-schultz}{Kai Schultz},
\href{https://www.nytimes.com/by/suhasini-raj}{Suhasini Raj} and
\href{https://www.nytimes.com/by/salman-masood}{Salman Masood}

\begin{itemize}
\item
  Aug. 7, 2019
\item
  \begin{itemize}
  \item
  \item
  \item
  \item
  \item
  \item
  \end{itemize}
\end{itemize}

NEW DELHI --- Pakistan announced on Wednesday that it would halt trade
with India and expel the country's top diplomat in Islamabad in
retaliation for India's decision to unilaterally eliminate the autonomy
of Kashmir.

The Pakistani government, which also claims the restive region of
Kashmir, said it would recall its own chief diplomat based in New Delhi.

A statement from a national security committee headed by the Pakistani
prime minister, Imran Khan, said the changes would be put in place
because of ``illegal actions'' by the Indian government regarding
Kashmir, which has a Muslim majority.

Mr. Khan denounced Prime Minister Narendra Modi of India, accusing his
government of promoting ``an ideology that puts Hindus above all other
religions and seeks to establish a state that represses all other
religious groups.''

The statement on Wednesday from the committee headed by Mr. Khan said
that India's stripping of Kashmiri autonomy would also be raised by
Pakistan with the United Nations Security Council, which recognizes the
region as disputed.

In addition to ending bilateral trade, which has been
\href{https://www.business-standard.com/article/pti-stories/indo-pak-bilateral-trade-posted-growth-despite-tensions-report-119022400294_1.html}{valued
at several billion dollars} annually, and downgrading diplomatic ties,
Pakistani officials threatened to close the country's airspace to Indian
aircraft. The statement said all bilateral agreements would also be
reviewed.

It remained unclear Wednesday when Pakistan would begin enforcing the
promised retaliatory measures.

Alyssa Ayres, a senior fellow for South Asia at the Council on Foreign
Relations in Washington, said that suspending trade was an unusual,
perhaps unprecedented, move by Pakistan, even though its effect would
most likely be muted.

``This will not frankly have any economic impact on either country,''
she said, noting that trade volume was still relatively low between
India and Pakistan. ``But under any circumstances, I'd rather see
diplomatic and symbolic steps like these than terrorism.''

In a speech in the Pakistani Parliament before the measures were
announced, Fawad Chaudhry, the science and technology minister, called
India a ``fascist regime'' and said another war over Kashmir, where
decades of fighting has killed tens of thousands of people, was not off
the table.

``Pakistan should not let Kashmir become another Palestine,'' Mr.
Chaudhry said. ``We have to choose between dishonor and war.''

The call for action comes after Amit Shah, the Indian home minister,
announced on Monday that the Indian government was
\href{https://www.nytimes.com/2019/08/05/world/asia/india-pakistan-kashmir-jammu.html?rref=collection\%2Fbyline\%2Fkai-schultz}{revoking
Kashmir's special status}, which served as a foundation for most of the
contested region's joining India as an autonomous area more than 70
years ago.

The Indian Parliament overwhelmingly approved a bill this week that
split the Indian state of Jammu and Kashmir into two federal
territories. The move puts Kashmir under tighter control of the central
government.

India and Pakistan, both of which have nuclear arms, have fought several
bitter wars over Kashmir, a mountainous, predominantly Muslim territory
claimed by both countries.

Low-intensity conflict has become a fact of life in the region, stunting
development, leaving its people deeply alienated and providing the
backdrop to a stubborn battle for independence by a few hundred
militants against tens of thousands of Indian troops.

\includegraphics{https://static01.nyt.com/images/2019/08/07/world/07kashmir2/merlin_158933688_09197582-8957-4929-9814-1bdbc9f1d5e1-articleLarge.jpg?quality=75\&auto=webp\&disable=upscale}

Mr. Shah said removing the region's semiautonomous status, which
included a provision barring non-Kashmiris from owning land, would spur
investment and encourage peace building. The government framed its plans
as ``purely administrative.''

But for decades, Hindu nationalists from the Bharatiya Janata Party, now
led by Prime Minister Narendra Modi, have
\href{https://www.nytimes.com/2019/08/06/world/asia/jammu-kashmir-india.html?rref=collection\%2Fbyline\%2Fkai-schultz}{vowed
to curtail special freedoms} enjoyed by Kashmir under Article 370 of the
Indian Constitution. Defanging the provision was central to their
broader agenda of moving India closer to a Hindu nation.

Many analysts said Pakistan cannot afford to go to war and has limited
latitude on Kashmir. The country has a history of providing support to
militant groups in the region, despite repeated calls from allies to
stop such assistance.

But some Pakistani opposition leaders called for an even broader review
of the country's foreign policy dealings, including with the United
States, which has increasingly cozied up to India, seeing it as a check
on China.

Though President Trump has recently offered to mediate the Kashmir
dispute, Raza Rabbani, a former chairman of the Senate and a senior
opposition politician, urged Mr. Khan's government to move away from
dependency on Washington, saying the United States had formed a
``nexus'' with India and Israel.

``Have we forgotten that when Trump mediated he gave Golan Heights to
Israel?'' Mr. Rabbani said, warning that India could make settlements
along the Line of Control and push Kashmiris into Pakistan.

``Pakistan would be under pressure and a constant threat of war,'' he
said.

Indian lawyers were split about the constitutionality of diluting
Article 370, and many said the government's plan was likely to face
court challenges. On Tuesday, a veteran public interest lawyer filed the
first legal challenge to the government's actions in the Supreme Court.

Over the past few days, stunned opposition members argued in India's
Parliament that change was needed in Kashmir, but that the government's
move was undemocratic and a disturbing attempt to undermine India's
secular identity. Some likened it to a coup.

Before the government announced the end of the special status, Kashmiri
voices were almost completely silenced. Internet connections, mobile
service and landlines were cut. Thousands of additional Indian troops
were deployed in the region, and tourists were evacuated.

Asrar Sultanpuri, a Kashmiri writer who lives in New Delhi, said he
could not reach his wife and son, who recently went to Kashmir to meet
relatives. He sobbed in a telephone interview.

``I am angry and sad and worried,'' he said. ``We should at least have
been allowed to communicate with our families.''

An official in the Ministry of Home Affairs, who was not authorized to
speak publicly, said on Wednesday that Section 144, a part of India's
criminal code that allows bans on gatherings of more than four people,
was in place. She said that people in Jammu and Kashmir were free to
move around.

Many top Indian politicians, including Mr. Modi and Mr. Shah, were
focused on attending funeral ceremonies on Wednesday for a former
foreign minister.

But in Srinagar, Kashmir's biggest city, the few people who were able to
transmit messages said that they were still terrified and that stores
were closed and streets empty. Indian soldiers were patrolling
barricaded intersections, and curfew passes were required. One Kashmiri
resident said that there had been sporadic incidents of stone pelting in
south Kashmir, though there were no confirmed reports of serious
injuries.

Iltija Javed, the daughter of a prominent Kashmiri politician and one of
the few people who has managed to send updates, said the city was under
a ``complete information blackout'' and expressed concern that the
situation would only get worse.

``The way we are being treated is absolutely appalling,'' she said in a
voice mail message on Wednesday. ``We don't even know if everyone has
enough food supplies, enough medicinal supplies to last them for this
indefinite curfew.''

Advertisement

\protect\hyperlink{after-bottom}{Continue reading the main story}

\hypertarget{site-index}{%
\subsection{Site Index}\label{site-index}}

\hypertarget{site-information-navigation}{%
\subsection{Site Information
Navigation}\label{site-information-navigation}}

\begin{itemize}
\tightlist
\item
  \href{https://help.nytimes.com/hc/en-us/articles/115014792127-Copyright-notice}{©~2020~The
  New York Times Company}
\end{itemize}

\begin{itemize}
\tightlist
\item
  \href{https://www.nytco.com/}{NYTCo}
\item
  \href{https://help.nytimes.com/hc/en-us/articles/115015385887-Contact-Us}{Contact
  Us}
\item
  \href{https://www.nytco.com/careers/}{Work with us}
\item
  \href{https://nytmediakit.com/}{Advertise}
\item
  \href{http://www.tbrandstudio.com/}{T Brand Studio}
\item
  \href{https://www.nytimes.com/privacy/cookie-policy\#how-do-i-manage-trackers}{Your
  Ad Choices}
\item
  \href{https://www.nytimes.com/privacy}{Privacy}
\item
  \href{https://help.nytimes.com/hc/en-us/articles/115014893428-Terms-of-service}{Terms
  of Service}
\item
  \href{https://help.nytimes.com/hc/en-us/articles/115014893968-Terms-of-sale}{Terms
  of Sale}
\item
  \href{https://spiderbites.nytimes.com}{Site Map}
\item
  \href{https://help.nytimes.com/hc/en-us}{Help}
\item
  \href{https://www.nytimes.com/subscription?campaignId=37WXW}{Subscriptions}
\end{itemize}
