Sections

SEARCH

\protect\hyperlink{site-content}{Skip to
content}\protect\hyperlink{site-index}{Skip to site index}

\href{https://www.nytimes.com/section/world/asia}{Asia Pacific}

\href{https://myaccount.nytimes.com/auth/login?response_type=cookie\&client_id=vi}{}

\href{https://www.nytimes.com/section/todayspaper}{Today's Paper}

\href{/section/world/asia}{Asia Pacific}\textbar{}Photos Emerge From
Kashmir, a Land on Lockdown

\url{https://nyti.ms/2ZSMP80}

\begin{itemize}
\item
\item
\item
\item
\item
\end{itemize}

Advertisement

\protect\hyperlink{after-top}{Continue reading the main story}

Supported by

\protect\hyperlink{after-sponsor}{Continue reading the main story}

\hypertarget{photos-emerge-from-kashmir-a-land-on-lockdown}{%
\section{Photos Emerge From Kashmir, a Land on
Lockdown}\label{photos-emerge-from-kashmir-a-land-on-lockdown}}

Indian photographers managed to work around a communication blockade to
publish their images.

\href{https://www.nytimes.com/by/jeffrey-gettleman}{\includegraphics{https://static01.nyt.com/images/2018/10/10/multimedia/author-jeffrey-gettleman/author-jeffrey-gettleman-thumbLarge.png}}

By \href{https://www.nytimes.com/by/jeffrey-gettleman}{Jeffrey
Gettleman}

\begin{itemize}
\item
  Aug. 9, 2019
\item
  \begin{itemize}
  \item
  \item
  \item
  \item
  \item
  \end{itemize}
\end{itemize}

\includegraphics{https://static01.nyt.com/images/2019/08/09/world/09Kashmir-Photo/09Kaashmir-Photo-articleLarge.jpg?quality=75\&auto=webp\&disable=upscale}

NEW DELHI --- For most of the past week, the entire Kashmir Valley, home
to about eight million people, has been put on virtual house arrest.

Indian soldiers rolled in by the tens of thousands. They barricaded
roads, closed schools, took positions on rooftops and cut off the
internet, mobile phone service and even landlines, rendering the valley
mostly incommunicado. At gunpoint, residents were ordered to stay inside
their homes.

The Indian government says these measures, in place since Sunday night,
are necessary to keep law and order. Human rights activists have likened
them to mass incarceration.

This week, India's Hindu nationalist government jolted the region by
\href{https://www.nytimes.com/2019/08/05/world/asia/india-pakistan-kashmir-jammu.html}{erasing
the autonomy} of the one Muslim-majority state in India, Jammu and
Kashmir, which includes the Kashmir Valley. India knew this move would
be deeply unpopular in the valley so they chose to lock it down.

Despite the crackdown, protests have erupted. On Friday, the unrest
continued, gunshots rang out and foreign journalists continued to be
barred from entering Kashmir without permission. These pictures are some
of the first images to emerge, taken by Indian photographers who managed
to work around the communication blockade and the miles of razor wire to
take and publish their images.

Image

The police detaining an activist during a protest against the Indian
government in Jammu on Saturday.Credit...Rakesh Bakshi/Agence
France-Presse --- Getty Images

Image

Afshana Farooq, 14, was treated in a Srinagar hospital on Friday after
being nearly trampled in a stampede when Indian forces opened fire on
demonstrators.Credit...Atul Loke for The New York Times

Image

A resident looked out a window as soldiers patrolled the street
below.Credit...Atul Loke for The New York Times

Kashmir is an exquisitely beautiful, staggeringly fertile land. Its
meadows are carpeted with wildflowers, its white-toothed mountains push
up against a flawless blue sky.

Many Indians consider Kashmir an integral part of India and have wanted
to bring it more intimately into the fold. Indian officials say that
removing its autonomy and exerting more central government control will
cut down on corruption, improve security and lift the local economy.

But many Kashmiris don't want to be part of India at all. They see India
as foreign and oppressive and, for decades, have chafed under the
country's rule, expressing their frustration in countless ways, from
school sit-ins to armed insurgency.

Image

Some Indian news outlets have reported that families in Srinagar are
beginning to run out of food.Credit...Atul Loke for The New York Times

Image

A woman with her nephew in Srinagar, in Jammu and Kashmir. Indian
soldiers have forced most Kashmiris to stay at or near
home.Credit...Atul Loke for The New York Times

Image

Shops remained shuttered in Srinagar on Saturday.Credit...Atul Loke for
The New York Times

The one important thing the opposing sides agree on --- perhaps the only
thing --- is that for too long Kashmir has suffered stagnation,
hopelessness, squandered potential and endless killing.

Clamping down on millions of people is an extraordinary step for the
world's largest democracy. Even on Friday, five days after a curfew was
imposed, many people were still marooned in their homes. Some have said
they are running out of food. A few cautiously emerged to pray at their
neighborhood mosques. Soldiers stared at them from behind metal face
masks.

Anybody who can get out is doing exactly that. Soon after the lockdown
was imposed, scores of migrant workers from other parts of India rushed
bus and train terminals. Getting in is even harder.

Image

Men in Srinagar buying bus tickets to leave Kashmir.Credit...Sajjad
Hussain/Agence France-Presse --- Getty Images

Image

Indian migrant laborers fleeing Kashmir by boarding a train through an
emergency window.Credit...Jaipal Singh/EPA, via Shutterstock

Image

Srinagar residents on Thursday.Credit...Atul Loke for The New York Times

Many Kashmiris were shocked and demoralized by the news that their
autonomy had been instantly erased. The Indian government, led by
Narendra Modi, made its move on Monday, dismantling an
\href{https://www.nytimes.com/interactive/2019/world/asia/india-pakistan-crisis.html}{article
in the Indian Constitution} that guaranteed Kashmiris special land
ownership rights and allowed Kashmir to frame its own laws.

That article protected the special status Kashmir enjoyed since it
joined India in 1947. Mr. Modi's government also stripped away the
statehood of Kashmir and turned it into a federal territory. Mr. Modi's
political party has deep roots in a Hindu nationalist ideology, and
critics saw this move as another example of his sowing divisions between
India's Hindu majority and its Muslim minority.

As tensions have risen in recent days, groups of young men, full of
years of pent-up frustration, have squared off with soldiers, hurling
rocks and ducking buckshot. Security forces arrested more than 500
people and put them in makeshift detention centers.

Image

Indian security personnel on a street in Srinagar on
Friday.Credit...Agence France-Presse --- Getty Images

Image

A line for gas in Srinagar.Credit...Tauseef Mustafa/Agence France-Presse
--- Getty Images

Image

Indian security personnel patrolling an empty street in
Srinagar.Credit...Agence France-Presse --- Getty Images

Across India, most people supported the move to take away Kashmir's
autonomy. Even progressive politicians who usually clash with Mr. Modi
backed him on this. But in Pakistan, many people feel Kashmir should be
part of their country and are infuriated by India's actions.

Both Pakistan and India have nuclear arms, and they have fought wars
over Kashmir. Most seasoned observers think
\href{https://www.nytimes.com/2019/08/09/world/asia/kashmir-india-pakistan.html}{another
war is unlikely}, but anything that increases tension between Pakistan
and India is taken very seriously.

Image

People in New Delhi celebrating the decision to revoke Kashmir's
autonomy.Credit...Danish Siddiqui/Reuters

Image

Watching a live address by Prime Minister Narendra
Modi.Credit...Divyakant Solanki/EPA, via Shutterstock

Image

People protesting India's move in Islamabad, Pakistan.Credit...Aamir
Qureshi/Agence France-Presse --- Getty Images

Advertisement

\protect\hyperlink{after-bottom}{Continue reading the main story}

\hypertarget{site-index}{%
\subsection{Site Index}\label{site-index}}

\hypertarget{site-information-navigation}{%
\subsection{Site Information
Navigation}\label{site-information-navigation}}

\begin{itemize}
\tightlist
\item
  \href{https://help.nytimes.com/hc/en-us/articles/115014792127-Copyright-notice}{©~2020~The
  New York Times Company}
\end{itemize}

\begin{itemize}
\tightlist
\item
  \href{https://www.nytco.com/}{NYTCo}
\item
  \href{https://help.nytimes.com/hc/en-us/articles/115015385887-Contact-Us}{Contact
  Us}
\item
  \href{https://www.nytco.com/careers/}{Work with us}
\item
  \href{https://nytmediakit.com/}{Advertise}
\item
  \href{http://www.tbrandstudio.com/}{T Brand Studio}
\item
  \href{https://www.nytimes.com/privacy/cookie-policy\#how-do-i-manage-trackers}{Your
  Ad Choices}
\item
  \href{https://www.nytimes.com/privacy}{Privacy}
\item
  \href{https://help.nytimes.com/hc/en-us/articles/115014893428-Terms-of-service}{Terms
  of Service}
\item
  \href{https://help.nytimes.com/hc/en-us/articles/115014893968-Terms-of-sale}{Terms
  of Sale}
\item
  \href{https://spiderbites.nytimes.com}{Site Map}
\item
  \href{https://help.nytimes.com/hc/en-us}{Help}
\item
  \href{https://www.nytimes.com/subscription?campaignId=37WXW}{Subscriptions}
\end{itemize}
