Sections

SEARCH

\protect\hyperlink{site-content}{Skip to
content}\protect\hyperlink{site-index}{Skip to site index}

\href{https://www.nytimes.com/section/world/asia}{Asia Pacific}

\href{https://myaccount.nytimes.com/auth/login?response_type=cookie\&client_id=vi}{}

\href{https://www.nytimes.com/section/todayspaper}{Today's Paper}

\href{/section/world/asia}{Asia Pacific}\textbar{}Photos Show the
Devastation of Monsoons Ravaging India and Pakistan

\url{https://nyti.ms/31BpmIS}

\begin{itemize}
\item
\item
\item
\item
\item
\item
\end{itemize}

Advertisement

\protect\hyperlink{after-top}{Continue reading the main story}

Supported by

\protect\hyperlink{after-sponsor}{Continue reading the main story}

\hypertarget{photos-show-the-devastation-of-monsoons-ravaging-india-and-pakistan}{%
\section{Photos Show the Devastation of Monsoons Ravaging India and
Pakistan}\label{photos-show-the-devastation-of-monsoons-ravaging-india-and-pakistan}}

The torrential rains led to landslides in the countryside and inundated
cities, electrocuting some residents and burying others under mud.

\includegraphics{https://static01.nyt.com/images/2019/08/12/world/12india-pakistan-3/12india-pakistan-3-articleLarge.jpg?quality=75\&auto=webp\&disable=upscale}

By \href{https://www.nytimes.com/by/maria-abi-habib}{Maria Abi-Habib},
Saba Imtiaz and Ayesha Venkataraman

\begin{itemize}
\item
  Aug. 12, 2019
\item
  \begin{itemize}
  \item
  \item
  \item
  \item
  \item
  \item
  \end{itemize}
\end{itemize}

NEW DELHI --- Furious
\href{https://www.nytimes.com/2020/07/15/world/asia/monsoon-asia-bangladesh-india.html}{monsoon
rains} pounded India and Pakistan over the weekend, wiping away entire
villages, submerging cities and leaving civilians desperately crouching
on rooftops and, in one instance, frantically clutching onto a
construction crane for rescue.

In the south Indian state of Kerala, nearly 290,000 people were
displaced from their homes from Thursday to Sunday morning, with 76
people killed, 32 injured and 58 missing, according to the local
government, which expects the toll to increase. At least 97 people died
in flooding in three other states in India.

Indian military forces fanned out across the south and west of the
country to perform rescue missions. In one case, they stripped down tree
limbs to haul a wheelchair-bound man to safety through a thick forest
and, in another, provided dinghies to ferry civilians to one of the
hundreds of relief camps set up across the nation.

\includegraphics{https://static01.nyt.com/images/2019/08/12/world/12india-pakistan-floods-4/12india-pakistan-floods-4-articleLarge.jpg?quality=75\&auto=webp\&disable=upscale}

Image

A soldier carries an infant to safety in the western state of
Maharashtra, India, on Sunday.Credit...Reuters

Image

Indian military personnel preparing for a rescue and relief operation on
Saturday.Credit...Indian Defense Ministry

Image

A woman being evacuated by helicopter in Maharashtra, India, on
Saturday.Credit...EPA, via Shutterstock

In Kavalappara, in Kerala state, the heavy rains caused a mudslide on
Thursday night that wiped away an entire village, except for one house.
When rescue workers finally reached the area 12 hours later, only three
people remained alive. The bodies of other villagers were pinned down by
felled trees or buried deep in the mud.

``We are finding bodies carried away by the mudslide about a kilometer
from where their homes once stood. We think there are 30 children buried
in the mud,'' said Pratheesh K.P., a rescue worker at the site, who
added that this was the worst flooding he had seen in his seven years on
the job.

Nearly 120 miles away from the devastated village in Kavalappara, about
500 people from the remote village of Sreekandapuram considered
themselves lucky to be alive: Their homes had been rocked by two
landslides, about four hours apart.

One of those villagers, Adarsh K.K., was taking shelter in one of the
1,654 government-run displacement camps set up across Kerala State.

``Most people were rescued from the water using fishermen and army
boats,'' he said. ``Most of the houses are underwater.''

Image

A search for survivors after a landslide at Puthumala at Meppadi in the
Indian state of Kerala on Saturday.Credit...Agence France-Presse ---
Getty Images

Image

Residents being rescued by boat in Jamkhandi Taluk in Karnataka State on
Sunday.Credit...Agence France-Presse --- Getty Images

Image

Military personnel carrying a man in a wheelchair through a thick forest
in Karnataka on Sunday.Credit...Indian Defense Ministry

In Pakistan, the southern city of Karachi, the country's most populated,
was inundated by the floods. The city's residents, long inured to
government apathy, posted videos online showing submerged roads and
neighborhoods flooded with rainwater. At least 11 people died in the
city over the weekend, according to the local government.

On Sunday, as thunder resounded across the city, neighborhood streets
turned into muddy rivers with garbage and sewage flowing past people's
houses. People waded through knee-deep water, and three young men were
killed in an upscale neighborhood, electrocuted by the current from an
electricity pole submerged in the floodwater.

The floods left a cluster of people on a causeway fearing for their
lives, the raging waters quickly encroaching on the small patch of land
that had not yet been submerged. The government was ill-prepared in its
relief efforts, but posted footage of the hazardous rescue they did
offer to those trapped on the causeway: a construction crane hoisting
civilians high into the air, their bodies, stiff with fear, swaying
above the waves.

Karachi's flooding quickly became a point of political contention, with
various political parties trading volleys of blame for the haphazard
urban planning in the megalopolis and its creaking infrastructure.
Governance and corruption problems have plagued the municipality of
about 15 million residents, and citizens demanded to know where their
taxes had gone.

By Monday, the pools of stagnant water had mixed with blood as people
began sacrificing animals as part of the rituals for the start of the
Eid al Adha festival, a major Muslim religious holiday.

Image

Children in a flooded street on the outskirts of Karachi on
Saturday.Credit...Rizwan Tabassum/Agence France-Presse --- Getty Images

Image

A flooded house in Karachi on Sunday.Credit...Asif Hassan/Agence
France-Presse --- Getty Images

Image

Residents wade through a flooded street after heavy monsoon rains in
Karachi on Sunday.Credit...Asif Hassan/Agence France-Presse --- Getty
Images

\emph{Shalini Venugopal contributed reporting from New Delhi, Saba
Imtiaz from Karachi and Ayesha Venkataraman from Mumbai.}

Advertisement

\protect\hyperlink{after-bottom}{Continue reading the main story}

\hypertarget{site-index}{%
\subsection{Site Index}\label{site-index}}

\hypertarget{site-information-navigation}{%
\subsection{Site Information
Navigation}\label{site-information-navigation}}

\begin{itemize}
\tightlist
\item
  \href{https://help.nytimes.com/hc/en-us/articles/115014792127-Copyright-notice}{©~2020~The
  New York Times Company}
\end{itemize}

\begin{itemize}
\tightlist
\item
  \href{https://www.nytco.com/}{NYTCo}
\item
  \href{https://help.nytimes.com/hc/en-us/articles/115015385887-Contact-Us}{Contact
  Us}
\item
  \href{https://www.nytco.com/careers/}{Work with us}
\item
  \href{https://nytmediakit.com/}{Advertise}
\item
  \href{http://www.tbrandstudio.com/}{T Brand Studio}
\item
  \href{https://www.nytimes.com/privacy/cookie-policy\#how-do-i-manage-trackers}{Your
  Ad Choices}
\item
  \href{https://www.nytimes.com/privacy}{Privacy}
\item
  \href{https://help.nytimes.com/hc/en-us/articles/115014893428-Terms-of-service}{Terms
  of Service}
\item
  \href{https://help.nytimes.com/hc/en-us/articles/115014893968-Terms-of-sale}{Terms
  of Sale}
\item
  \href{https://spiderbites.nytimes.com}{Site Map}
\item
  \href{https://help.nytimes.com/hc/en-us}{Help}
\item
  \href{https://www.nytimes.com/subscription?campaignId=37WXW}{Subscriptions}
\end{itemize}
