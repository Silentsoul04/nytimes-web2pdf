Sections

SEARCH

\protect\hyperlink{site-content}{Skip to
content}\protect\hyperlink{site-index}{Skip to site index}

\href{https://www.nytimes.com/section/politics}{Politics}

\href{https://myaccount.nytimes.com/auth/login?response_type=cookie\&client_id=vi}{}

\href{https://www.nytimes.com/section/todayspaper}{Today's Paper}

\href{/section/politics}{Politics}\textbar{}The House v. Trump: Stymied
Lawmakers Increasingly Battle in the Courts

\url{https://nyti.ms/2ORRENz}

\begin{itemize}
\item
\item
\item
\item
\item
\item
\end{itemize}

Advertisement

\protect\hyperlink{after-top}{Continue reading the main story}

Supported by

\protect\hyperlink{after-sponsor}{Continue reading the main story}

\hypertarget{the-house-v-trump-stymied-lawmakers-increasingly-battle-in-the-courts}{%
\section{The House v. Trump: Stymied Lawmakers Increasingly Battle in
the
Courts}\label{the-house-v-trump-stymied-lawmakers-increasingly-battle-in-the-courts}}

Lawmakers say President Trump's disregard for their authority leaves
them no choice but to use the courts with greater frequency, but that
could have constitutional consequences.

\includegraphics{https://static01.nyt.com/images/2019/08/11/us/11dc-houselawsuits/merlin_158074485_6ef15439-58f0-45f7-a52a-b383ebcef16a-articleLarge.jpg?quality=75\&auto=webp\&disable=upscale}

\href{https://www.nytimes.com/by/charlie-savage}{\includegraphics{https://static01.nyt.com/images/2018/06/12/multimedia/author-charlie-savage/author-charlie-savage-thumbLarge-v2.png}}\href{https://www.nytimes.com/by/nicholas-fandos}{\includegraphics{https://static01.nyt.com/images/2018/11/06/multimedia/author-nicholas-fandos/author-nicholas-fandos-thumbLarge-v2.png}}

By \href{https://www.nytimes.com/by/charlie-savage}{Charlie Savage} and
\href{https://www.nytimes.com/by/nicholas-fandos}{Nicholas Fandos}

\begin{itemize}
\item
  Aug. 13, 2019
\item
  \begin{itemize}
  \item
  \item
  \item
  \item
  \item
  \item
  \end{itemize}
\end{itemize}

WASHINGTON --- Democrats took control of the House this year promising
to use legislation and investigations to check President Trump. But
facing substantial roadblocks to each, they are increasingly opposing
him in a different way: Eight months into their majority, the House is
going to court at a tempo never seen before.

Fighting in courtrooms as much as in hearing rooms, the House has
already become a party to nine separate lawsuits this year, while also
filing briefs for judges in four others. More lawsuits are being
drafted, according to a senior aide to Speaker Nancy Pelosi.

The fights include efforts to reveal Mr. Trump's hidden financial
dealings, force his aides to testify about his attempts to obstruct the
Russia investigation, challenge his invocation of emergency powers to
spend more taxpayer money on a border wall than Congress approved and
defend laws like the Affordable Care Act that his Justice Department
abandoned.

While it is routine for the executive branch to be in court, it was once
vanishingly rare for Congress, which has typically used its authority to
pass laws, appropriate funds and investigate the executive branch, to
balance out the president's power.

Now, as Senate Republicans refuse to take up the bills they pass, Trump
administration witnesses refuse to show up for their hearings and the
president levels his own highly unusual lawsuits against the House's
oversight requests, two of the three branches of government are
regularly facing off before the third, creating a new stress with
uncertain consequences for the political system.

``It is unprecedented,'' said Charles Tiefer, a former longtime House
lawyer who is now a University of Baltimore law professor. ``The
challenges for the House counsel ebb and flow over time, but this is
like nothing else in history.''

The consequences of the specific disputes could be significant. In the
short term, they could determine whether House Democrats are able to
drag information to light about Mr. Trump that could lead to his
impeachment or damage his re-election prospects. And potential decisions
by the higher courts could clarify the long-ambiguous line between a
president's secrecy power and Congress's oversight authority ---
determining whether future presidents can systematically stonewall
congressional subpoenas.

But the broader phenomenon is also significant.

As an immediate matter, the surge in litigation is a consequence of Mr.
Trump's norm-busting presidency. House Democrats are looking for
additional venues through which to take him on --- or, in some cases,
fighting lawsuits that the president filed against Congress himself to
try to block lawmakers from obtaining information about him from
entities outside the federal government. But it is also bringing into
clearer view how, over the past generation, Congress was already
starting to go to court more often than had been the historical norm, as
political compromise gave way to deadlock amid growing partisan
polarization.

That trend line suggests that even if the number of congressional
lawsuits declines when the next president takes office, the
constitutional order could change in a way that Mr. Tiefer and other
legal scholars view as dangerous. He said it is better if the two
parties were able to resolve high-level policy disputes through
compromise rather than through the ``rigid and formalized system'' of
litigation, and suggested that routinely pushing those disputes into
court could heighten politicization of the judiciary.

To handle the mounting workload, the House's general counsel, Douglas
Letter, a former Justice Department litigator, and his staff of seven
lawyers have increasingly relied on volunteer lawyers at white-shoe law
firms and at public interest groups --- including several prominent
veterans of the Obama legal team --- to help research and draft court
filings.

Representative Adam B. Schiff, Democrat of California and the chairman
of the Intelligence Committee, which was a party in one of the subpoena
lawsuits for Mr. Trump's banking records, portrayed the increasing
litigation as an unfortunate necessity.

``The blanket refusal to comply with any legitimate process has forced
us to go to court to validate Congress's power of oversight,'' he said.
``If we don't, we are at risk of losing that power, and that would be a
tragedy for the country because it would take any limit off the
executive.''

Republicans say it is the Democrats who are out of control and violating
traditional norms of governance. The Justice Department has accused the
House of asking the judiciary ``to take its side in political disputes''
and of trying to ``use federal courts to accomplish through litigation
what it cannot achieve using the tools the Constitution gives to
Congress.''

For most of American history, Mr. Tiefer said, Congress never went to
court over disputes with the executive branch. The seeds of change began
during the Watergate scandal, when Congress enacted a special law
enabling a Senate committee to sue President Richard M. Nixon to try to
gain access to his Oval Office tapes.

In the post-Watergate reform era, the House created a small general
counsel office, raising the possibility of filing civil lawsuits seeking
enforcement of its subpoenas to executive branch officials if a
president tried to block them. But that threat remained essentially
theoretical for more than a generation, as the two branches resolved
such disputes through negotiations.

But in 2008, House Democrats went to court to compel disclosure of
information from the Bush administration about its firing of a group of
United States attorneys. In 2012, House Republicans sued for internal
Obama Justice Department documents related to the botched
gun-trafficking case known as Operation Fast and Furious.

Those rare subpoena-related cases have now grown routine. Already this
year, Democrats have filed lawsuits seeking to
enforce\href{https://www.nytimes.com/2019/07/02/us/politics/trump-taxes-lawsuit.html}{subpoenas
for Mr. Trump's federal tax returns} and
\href{https://www.nytimes.com/2019/08/07/us/politics/don-mcgahn-subpoena.html}{for
testimony from Donald F. McGahn II}, his former White House counsel and
a key witness to several obstruction episodes in the special counsel's
Russia investigation report.

Separately, Mr. Trump's personal lawyers have filed lawsuits against the
House seeking to block financial firms
---\href{https://www.nytimes.com/2019/04/22/us/politics/trump-sues-congress.htmlhttps://www.nytimes.com/2019/04/22/us/politics/trump-sues-congress.html}{}\href{https://www.nytimes.com/2019/04/22/us/politics/trump-sues-congress.html}{Mazars
USA} and
\href{https://www.nytimes.com/2019/04/29/us/politics/trump-lawsuit-deutsche-bank.html}{Deutsche
Bank and Capital One} --- from complying with its subpoenas for his
business records. Mr. Trump has also filed one against the House
\href{https://www.nytimes.com/2019/07/23/us/politics/trump-tax-returns-new-york.html}{to
block it from requesting his state tax returns} from New York.

Several more subpoena-related lawsuits are under development, including
most likely a suit for executive branch information that could reveal
the administration's motivation for trying to add a citizenship question
to the census.

The House has also filed litigation
\href{https://www.nytimes.com/2019/07/26/us/politics/donald-trump-impeachment.html}{asking
a judge to grant the House Judiciary Committee access to secret
evidence} Robert S. Mueller III, the former special counsel, gathered
using a grand jury. That effort draws on a Watergate-era precedent, but
the House was not itself a party to the 1974 request.

Also on an upward trend are cases in which the House is intervening in
court to defend statutes because the executive branch refuses to do so,
contrary to the Justice Department's longstanding role of defending acts
of Congress that come under constitutional challenge.

Although there have been several lawsuits in the past where Congress
stepped in to defend a law that the executive branch said permitted
lawmakers to unconstitutionally encroach on presidential power, it is
now starting to become more common for the Justice Department to refuse
to defend a law --- and for the House to step in --- without a
separation-of-powers rationale.

In 2011,
\href{https://www.nytimes.com/2011/03/05/us/politics/05marriage.html}{House
Republicans intervened} to defend a law that barred federal recognition
of same-sex marriages that were lawful at the state level after the
Obama administration deemed it unconstitutional and stopped defending it
in court. At the time, such an intervention by Congress was highly
unusual.

But this year, it has already happened twice. The Justice Department
under Mr. Trump has refused to defend the Affordable Care Act and a law
against female genital mutilation. The House has sought to intervene and
provide lawyers to argue that both are constitutional.

It is also becoming more common for the House to sue the executive
branch over spending and policy disputes. In 2014, the
Republican-controlled House
\href{https://www.nytimes.com/2014/11/22/us/politics/obamacare-lawsuit-filed-by-republicans.html}{filed
an unusual lawsuit} over how the Obama administration was putting the
Affordable Care Act into effect, including its provision of insurer
subsidies the lawmakers said were unauthorized.

Echoing that move, House Democrats this year filed a lawsuit challenging
Mr. Trump's plan to use emergency powers to spend more on a border wall
than Congress appropriated for that purpose. A district court judge
ruled that the House lacked standing, and it has appealed.

Across this array of litigation, Mr. Letter --- a former Justice
Department civil litigation specialist --- has handled courtroom
appearances. But his office has been working with outside lawyers for
research and brief drafting.

The aide to Ms. Pelosi said that all of the outside legal services have
been provided for free, in contrast to the several million dollars that
House Republicans spent on outside lawyers for the marriage law case.

The lawyers volunteering to help the House Democrats include Donald B.
Verrilli Jr., a former solicitor general, and a team at his firm,
Munger, Tolles \& Olson; Virginia A. Seitz, a former head of the Justice
Department's Office of Legal Counsel, and others at her firm, Sidley
Austin; Neal K. Katyal, a former acting solicitor general and a partner
at Hogan Lovells; and Georgetown Law Center's Institute for
Constitutional Advocacy and Protection, run by two former national
security officials, Joshua A. Geltzer and Mary McCord.

Beyond immediate political consequences, legal specialists said that the
fate of the litigation wave might carry broader implications for
Congress's ability to counterbalance the presidency in the future,
regardless of which party was in power.

Kerry W. Kircher, who served as House counsel under Republican speakers
between 2011 and 2016, said many of the House's lawsuits were justified,
regardless of politics, to ensure that Mr. Trump's
\href{https://www.nytimes.com/2019/04/24/us/politics/donald-trump-subpoenas.html}{vow
to defy ``all'' of its subpoenas} did not set a precedent.

House Democrats ``are trying to do what they think they were elected to
do and make sure that the House in the future will be able to conduct
oversight,'' he said. ``What is going on with this administration is no
less than an all-out declaration of war on oversight.''

Advertisement

\protect\hyperlink{after-bottom}{Continue reading the main story}

\hypertarget{site-index}{%
\subsection{Site Index}\label{site-index}}

\hypertarget{site-information-navigation}{%
\subsection{Site Information
Navigation}\label{site-information-navigation}}

\begin{itemize}
\tightlist
\item
  \href{https://help.nytimes.com/hc/en-us/articles/115014792127-Copyright-notice}{©~2020~The
  New York Times Company}
\end{itemize}

\begin{itemize}
\tightlist
\item
  \href{https://www.nytco.com/}{NYTCo}
\item
  \href{https://help.nytimes.com/hc/en-us/articles/115015385887-Contact-Us}{Contact
  Us}
\item
  \href{https://www.nytco.com/careers/}{Work with us}
\item
  \href{https://nytmediakit.com/}{Advertise}
\item
  \href{http://www.tbrandstudio.com/}{T Brand Studio}
\item
  \href{https://www.nytimes.com/privacy/cookie-policy\#how-do-i-manage-trackers}{Your
  Ad Choices}
\item
  \href{https://www.nytimes.com/privacy}{Privacy}
\item
  \href{https://help.nytimes.com/hc/en-us/articles/115014893428-Terms-of-service}{Terms
  of Service}
\item
  \href{https://help.nytimes.com/hc/en-us/articles/115014893968-Terms-of-sale}{Terms
  of Sale}
\item
  \href{https://spiderbites.nytimes.com}{Site Map}
\item
  \href{https://help.nytimes.com/hc/en-us}{Help}
\item
  \href{https://www.nytimes.com/subscription?campaignId=37WXW}{Subscriptions}
\end{itemize}
