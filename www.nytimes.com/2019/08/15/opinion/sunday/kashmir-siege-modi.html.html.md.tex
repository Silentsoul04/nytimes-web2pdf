Sections

SEARCH

\protect\hyperlink{site-content}{Skip to
content}\protect\hyperlink{site-index}{Skip to site index}

\href{/section/opinion/sunday}{Sunday Review}\textbar{}The Silence Is
the Loudest Sound

\url{https://nyti.ms/304xrFp}

\begin{itemize}
\item
\item
\item
\item
\item
\end{itemize}

\includegraphics{https://static01.nyt.com/images/2019/08/18/opinion/sunday/18Roy3/merlin_159132348_0be71bd7-84b9-4290-b01e-c0db2c3e7d06-articleLarge.jpg?quality=75\&auto=webp\&disable=upscale}

\href{/section/opinion}{Opinion}

\hypertarget{the-silence-is-the-loudest-sound}{%
\section{The Silence Is the Loudest
Sound}\label{the-silence-is-the-loudest-sound}}

The Indian government has confined about seven million Kashmiris to
their homes and imposed a complete communications blackout.

Indian security personnel on the streets of Srinagar, Kashmir, last
week.Credit...Atul Loke for The New York Times

Supported by

\protect\hyperlink{after-sponsor}{Continue reading the main story}

By Arundhati Roy

Ms. Roy is a writer and lives in Delhi.

\begin{itemize}
\item
  Aug. 15, 2019
\item
  \begin{itemize}
  \item
  \item
  \item
  \item
  \item
  \end{itemize}
\end{itemize}

NEW DELHI --- As India celebrates her 73rd year of independence from
British rule, ragged children thread their way through traffic in Delhi,
selling outsized national flags and souvenirs that say, ``Mera Bharat
Mahan.'' My India is Great. Quite honestly, it's hard to feel that way
right now, because it looks very much as though our government has gone
rogue.

Last week it
\href{https://www.nytimes.com/2019/08/08/opinion/modis-majoritarian-march-to-kashmir.html}{unilaterally
breached}the fundamental conditions of the Instrument of Accession, by
which the former Princely State of Jammu and Kashmir acceded to India in
1947. In preparation for this, at midnight on Aug. 4, it turned all of
\href{https://www.nytimes.com/2019/08/10/world/asia/kashmir-india-pakistan.html}{Kashmir
into a giant prison camp}. Seven million Kashmiris were barricaded in
their homes, internet connections were cut and their phones went dead.

On Aug. 5, India's home minister proposed in Parliament that
\href{https://www.nytimes.com/interactive/2019/world/asia/india-pakistan-crisis.html}{Article
370 of the Indian Constitution} (the article that
\href{https://thewire.in/law/murder-of-insaniyat-and-of-indias-solemn-commitment-to-kashmir}{outlines
the legal obligations}that arise from the Instrument of Accession) be
overturned. The opposition parties rolled over. By the next evening the
Jammu and Kashmir Reorganization Act, 2019 had been passed by the upper
as well as the lower house.

The act strips the State of Jammu and Kashmir of its special status ---
which includes its right to have its own constitution and its own flag.
It also strips it of statehood and partitions it into two Union
territories. The first, Jammu and Kashmir, will be administered directly
by the central government in New Delhi, although it will continue to
have a locally elected legislative assembly but one with drastically
reduced powers. The second, Ladakh, will be
\href{https://www.nytimes.com/2019/08/08/opinion/modis-majoritarian-march-to-kashmir.html}{administered
directly from New Delhi} and will not have a legislative assembly.

\includegraphics{https://static01.nyt.com/images/2019/08/18/opinion/sunday/18Roy4/merlin_159179976_b0c3a61a-6f7b-4337-ad17-dec96b8ec427-articleLarge.jpg?quality=75\&auto=webp\&disable=upscale}

The passing of the act was welcomed in Parliament by the very British
tradition of desk-thumping. There was a distinct whiff of colonialism in
the air. The masters were pleased that a recalcitrant colony had
finally, formally, been brought under the crown. For its own good. Of
course.

Indian citizens can now buy land and settle in their new domain. The new
territories are open for business. Already India's richest
industrialist, Mukesh Ambani, of Reliance Industries, has promised
several ``announcements.'' What this might mean to
\href{https://www.nytimes.com/2018/12/01/opinion/himalayas-mountains-dams.html}{the
fragile Himalayan ecology} of Ladakh and Kashmir, the land of vast
glaciers, high-altitude lakes and five major rivers, barely bears
consideration.

The dissolution of the legal entity of the state also means the
dissolution of Article 35A, which granted residents
\href{https://www.aljazeera.com/news/2019/08/kashmir-special-status-explained-articles-370-35a-190805054643431.html}{rights
and privileges} that made them stewards of their own territory. So,
``being open for business,'' it must be clarified, can also include
Israeli-style settlements and Tibet-style population transfers.

For Kashmiris, in particular, this has been an old, primal fear. Their
recurring nightmare (an inversion of the one being peddled by Donald
Trump) of being swept away by a tidal wave of triumphant Indians wanting
a little home in their sylvan valley could easily come true.

Image

A supporter of India's revocation of Kashmir's special status displaying
an artist's rendering of a map of India decorated with a shawl of
saffron, the dominant color in the ruling Bharatiya Janata Party's
symbol.Credit...Rajesh Kumar Singh/Associated Press

As news of the new act spread, Indian nationalists of all stripes
cheered. The mainstream media, for the most part, made a low, sweeping
bow. There was dancing in the streets and horrifying misogyny on the
internet. Manohar Lal Khattar, chief minister of the state of Haryana,
bordering Delhi, while speaking about the improvement he had brought
about in the skewed gender ratio in his
state\href{https://twitter.com/manakgupta/status/1160083098253455360}{,
joked}: ``Our Dhakarji used to say we will bring in girls from Bihar.
Now they say Kashmir is open, we can bring girls from there.''

Amid these vulgar celebrations the loudest sound, however, is
\href{https://www.nytimes.com/2019/08/10/world/asia/kashmir-india-pakistan.html}{the
deathly silence from Kashmir's patrolled, barricaded streets} and its
approximately seven million caged, humiliated people, stitched down by
razor wire, spied on by drones, living under a complete communications
blackout. That in this age of information, a government can so easily
cut off a whole population from the rest of the world for days at a
time, says something serious about the times we are heading toward.

Kashmir, they often say, is the unfinished business of the
``Partition.'' That word suggests that in 1947, when the
\href{https://www.nytimes.com/2017/08/18/opinion/india-pakistan-partition-imperial-britain.html}{British
drew their famously careless border}through the subcontinent, there was
a ``whole'' that was then partitioned. In truth, there was no ``whole.''
Apart from the territory of British India, there were hundreds of
sovereign principalities, each of which individually negotiated the
terms on which it would merge with either India or Pakistan. Many
that\href{https://www.bbc.co.uk/news/magazine-24159594}{did not wish to
merge} were forced to.

While Partition and the horrifying violence that it caused is a deep,
unhealed wound in the memory of the subcontinent, the violence of those
times, as well as in the years since, in India and Pakistan, has as much
to do with assimilation as it does with partition. In India the project
of assimilation, which goes under the banner of nation-building, has
meant that there has not been a single year since 1947 when the Indian
Army has not been deployed within India's borders against its ``own
people.'' The list is long --- Kashmir, Mizoram, Nagaland, Manipur,
Hyderabad, Assam.

The business of assimilation has been complicated and painful and has
cost tens of thousands of lives. What is unfolding today on both sides
of the border of the erstwhile state of Jammu and Kashmir is the
unfinished business of assimilation.

What happened in the Indian Parliament last week was tantamount to
cremating the
\href{https://thewire.in/history/public-first-time-jammu-kashmirs-instrument-accession-india}{Instrument
of Accession}. It was a document with a complicated provenance that had
been signed by a discredited king, the Dogra Hindu King, Maharaja Hari
Singh. His unstable, tattered kingdom of Jammu and Kashmir lay on the
fault lines of the new border between India and Pakistan.

The rebellions that had broken out against him in 1945 had been
aggravated and subsumed by the spreading bush fires of Partition. In the
western mountain district of Poonch, Muslims, who were the majority,
turned on the Maharaja's forces and on Hindu civilians. In Jammu, to the
south, the Maharaja's forces assisted by troops borrowed from other
princely states, massacred Muslims. Historians and news reports of the
time estimated that somewhere between
\href{https://kashmirlife.net/circa-1947-a-long-story-67652/}{70,000 and
200,000 were murdered} in the streets of the city, and in its
neighboring districts.

Inflamed by the news of the Jammu massacre, Pakistani ``irregulars''
swooped down from the mountains of the North Western Frontier Province,
burning and pillaging their way across the Kashmir Valley. Hari Singh
fled from Kashmir to Jammu from where he appealed to Jawaharlal Nehru,
the Indian prime minister, for help. The document that provided legal
cover for the Indian Army to enter Kashmir was the
\href{https://thewire.in/history/public-first-time-jammu-kashmirs-instrument-accession-india}{Instrument
of Accession}.

Image

Indian soldiers, supplied by the British, arriving in Srinagar in 1947,
to fight Pakistani troops for ownership of the Kashmir
region.Credit...Bettmann Archive/Getty Images

The Indian Army, with some help from local people, pushed back the
Pakistani ``irregulars,'' but only as far as the ring of mountains on
the edge of the valley. The former Dogra kingdom now lay divided between
India and Pakistan. The Instrument of Accession was meant to be
\href{https://undocs.org/S/RES/47(1948)}{ratified by a referendum} to
ascertain the will of the people of Jammu and Kashmir. That promised
referendum never took place. So was born the subcontinent's most
intractable and dangerous political problem.

In the 72 years since then, successive Indian governments have
undermined terms of the Instrument of Accession until all that was left
of it was the skeletal structure. Now even that has been shot to hell.

It would be foolhardy to try to summarize the twists and turns of how
things have come to this. Let's just say that it's as complicated and as
dangerous as the games the United States played with its puppet regimes
in South Vietnam all through the 50s and 60s.

After a long history of electoral manipulation, the watershed moment
came in 1987 when New Delhi flagrantly rigged the state elections. By
1989, the thus far mostly nonviolent demand for self-determination grew
into a
\href{https://www.nybooks.com/articles/2000/09/21/death-in-kashmir/}{full-throated
freedom struggle}. Hundreds of thousands of people poured onto the
streets only to be cut down in massacre after massacre.

The Kashmir valley soon thronged with militants, Kashmiri men from both
sides of the border, as well as foreign fighters, trained and armed by
Pakistan and embraced, for the most part, by the Kashmiri people. Once
again, Kashmir was caught up in the political winds that were blowing
across the subcontinent --- an increasingly radicalized Islam from
Pakistan and Afghanistan, quite foreign to Kashmiri culture, and the
fanatical Hindu nationalism that was on the rise in India.

Image

A heated debate among moderate and militant Kashmiri separatists at a
mosque in Anantnag, Jammu and Kashmir, on Oct. 8, 1989, two years after
the Jammu and Kashmir state legislative elections, which were said to
have been rigged.~Credit...Robert Nickelsberg/Getty Images

The first casualty of the uprising was the age-old bond between
Kashmir's Muslims and its tiny minority of Hindus, known locally as
Pandits. When the violence began, according to the Kashmiri Pandit
Sangharsh Samiti, or the K.P.S.S., an organization run by Kashmiri
Pandits, about 400 Pandits were targeted and murdered by militants. By
the
\href{https://indianexpress.com/article/explained/why-kashmiris-want-a-fair-probe-into-the-killings-of-pandits-prosecution-of-guilty-4786855/}{end
of 1990}, according to a government estimate, 25,000 Pandit families had
left the valley.

They lost their homes, their homeland and everything they had. Over the
years thousands more left ---
\href{https://www.aljazeera.com/indepth/spotlight/kashmirtheforgottenconflict/2011/07/2011724204546645823.html}{almost
the entire population}. As the conflict continued, in addition to tens
of thousands of Muslims, the K.P.S.S. says
650\href{https://www.aljazeera.com/indepth/spotlight/kashmirtheforgottenconflict/2011/07/201176134818984961.html}{Pandits
have been killed}in the conflict.

Since then, great numbers of Pandits have lived in miserable refugee
camps in Jammu city. Thirty years have gone by, yet successive
governments in New Delhi have not tried to help them return home. They
have preferred instead to keep them in limbo, and stir their anger and
understandable bitterness into a mephitic brew with which to fuel
India's dangerous and extremely effective nationalistic narrative about
Kashmir. In this version, a single aspect of an epic tragedy is cannily
and noisily used to draw a curtain across the rest of the horror.

Image

Security personnel on the streets of Srinagar last week.Credit...Atul
Loke for The New York Times

Today Kashmir is one of the most or perhaps \emph{the} most densely
militarized zone in the world. More than a half-million soldiers have
been deployed to counter what the army itself admits is now just a
handful of ``terrorists.'' If there were any doubt earlier it should be
abundantly clear by now that their real enemy is the Kashmiri people.
What India has done in Kashmir over the last 30 years is unforgivable.
An estimated 70,000 people, civilians, militants and security forces
have been killed in the conflict.
\href{https://www.csmonitor.com/World/Asia-South-Central/2008/0201/p07s03-wosc.html}{Thousands
have been ``disappeared,''} and tens of thousands have passed through
torture chambers that dot the valley like
\href{http://jkccs.net/wp-content/uploads/2019/05/TORTURE-Indian-State\%E2\%80\%99s-Instrument-of-Control-in-Indian-administered-Jammu-and-Kashmir.pdf}{a
network of small-scale Abu Ghraibs}.

Over the last few years, hundreds of
\href{https://www.nytimes.com/2016/08/29/world/asia/pellet-guns-used-in-kashmir-protests-cause-dead-eyes-epidemic.html}{teenagers
have been blinded} by the use of pellet-firing shotguns, the security
establishment's new weapon of choice for crowd control. Most militants
operating in the valley today are young Kashmiris, armed and trained
locally. They do what they do knowing full well that the minute they
pick up a gun, their ``shelf life'' is unlikely to be more than six
months. Each time a ``terrorist'' is killed, Kashmiris turn up in their
tens of thousands to bury a young man whom they revere as a
\emph{shaheed}, a martyr.

Image

Sameer Ahamed, a page designer for local newspapers, whose eyes and arms
were injured from pellets, in Srinagar on Aug. 10.~Credit...Atul Loke
for The New York Times

These are only the rough coordinates of a 30-year-old military
occupation. The most cruel effects of an occupation that has lasted
decades are impossible to describe in an account as short as this.

In Narendra Modi's first term as India's prime minister, his hard-line
approach exacerbated the violence in Kashmir. In February, after a
\href{https://www.nytimes.com/2019/03/02/opinion/sunday/kashmir-india-pakistan.html}{Kashmiri
suicide bomber} killed 40 Indian security personnel, India launched an
airstrike against Pakistan. Pakistan retaliated. They became the first
two nuclear powers in history to actually launch airstrikes against each
other. Now two months into Narendra Modi's second term, his government
has played its most dangerous card of all. It has tossed a lit match
into a powder keg.

If that were not bad enough, the cheap, deceitful way in which it did it
is disgraceful. In the last week of July, 45,000 extra troops were
\href{https://economictimes.indiatimes.com/news/defence/before-abolishing-article-370-indian-army-identified-possible-trouble-spots-in-kashmir/articleshow/70583869.cms}{rushed
into Kashmir}on various pretexts. The one that got the most traction was
that there was a Pakistani ``terror'' threat to the
\href{https://uk.reuters.com/article/uk-india-kashmir-pilgrimage/india-boosts-hindu-pilgrimage-to-holy-cave-in-conflict-torn-kashmir-idUKKCN1UN04Q}{Amarnath
Yatra} --- the annual pilgrimage in which hundreds of thousands of Hindu
devotees trek (or are carried by Kashmiri porters) through high
mountains to visit the Amarnath cave and pay their respects to a natural
ice formation that they believe is an avatar of Shiva.

On Aug. 1, some Indian
\href{https://www.youtube.com/watch?v=8Ta1Dj9LHgM}{television networks
announced} that a land mine with Pakistani Army markings on it had been
found on the pilgrimage route. On Aug. 2, the government published a
notice asking all pilgrims (and even tourists who were miles from the
pilgrimage route) to
\href{https://www.indiatoday.in/india/story/leave-kashmir-j-k-administration-issues-security-advisory-for-amarnath-pilgrims-1576494-2019-08-02}{leave
the valley immediately}. That set off a panicky exodus. The
approximately 200,000 Indian migrant day laborers in Kashmir were
clearly not a concern to those supervising the evacuation. Too poor to
matter, I'm guessing. By Saturday, Aug. 3, tourists and pilgrims had
left and the security forces had taken up position across the valley.

Image

An Indian paramilitary trooper standing guard on an empty street in
Srinagar on Aug. 4.Credit...Tauseef Mustafa/Agence France-Presse ---
Getty Images

By midnight Sunday, Kashmiris were barricaded in their homes, and all
communication networks went down. The next morning, we learned that,
along with several hundred others, three former chief ministers, Farooq
Abdullah, his son, Omar Abdullah of the National Conference and
\href{https://www.livemint.com/politics/news/mehbooba-mufti-omar-abdullah-arrested-after-scrapping-of-article-370-1565015217174.html}{Mehbooba
Mufti of the People's Democratic Party}, had been arrested. Those are
the mainstream pro-India politicians who have carried India's water
through the years of insurrection.

Newspapers report that the Jammu \& Kashmir police force
\href{https://www.telegraphindia.com/india/disarmed-fall-guys-of-article-370/cid/1696748}{has
been disarmed}. More than anybody else, these local police men have put
their bodies on the front line, have done the groundwork, provided the
apparatus of the occupation with the intelligence that it needs, done
the brutal bidding of their masters and, for their pains, earned the
contempt of their own people. All to keep the Indian flag flying in
Kashmir. And now, when the situation is nothing short of explosive, they
are going to be fed to the furious mob like so much cannon fodder.

The betrayal and public humiliation of India's allies by Narendra Modi's
government comes from a kind of hubris and ignorance that has gutted the
sly, elaborate structures painstakingly cultivated over decades by
cunning, but consummate, Indian statecraft. Now that that's done --- it
is down to the Street vs. the Soldier. Apart from what it does to the
young Kashmiris on the street, it is also a preposterous thing to do to
soldiers.

The more militant sections of the Kashmiri population, who have been
demanding the right to self-determination or merger with Pakistan, have
little regard for India's laws or constitution. They will no doubt be
pleased that those they see as collaborators have been sold down the
river and that the game of smoke and mirrors is finally over. It might
be too soon for them to rejoice. Because as sure as eggs are eggs and
fish are fish, there will be new smoke and new mirrors. And new
political parties. And a new game in town.

On Aug. 8, four days into the lockdown,
\href{https://www.youtube.com/watch?v=n0bNYhPJnxk}{Narendra Modi
appeared on television} to address an ostensibly celebrating India and
an incarcerated Kashmir. He sounded like a changed man. Gone was his
customary aggression and his jarring, accusatory tone. Instead he spoke
with the tenderness of a young mother. It's his most chilling avatar to
date.

Image

Prime Minister Narendra Modi's televised address to the nation is
watched by a crowd in Ahmedabad, India.Credit...Amit Dave/Reuters

His voice quivered and his eyes shone with unspilled tears as he listed
the slew of benefits that would rain down on the people of the former
State of Jammu and Kashmir, now that it was rid of its old, corrupt
leaders, and was going to be ruled directly from New Delhi. He evoked
the marvels of Indian modernity as though he were educating a bunch of
feudal peasants who had emerged from a time capsule. He spoke of how
Bollywood films would once again be shot in their verdant valley.

He didn't explain why Kashmiris needed to be locked down and put under a
communications blockade while he delivered his stirring speech. He
didn't explain why the decision that supposedly benefited them so hugely
was taken without consulting them. He didn't say how the great gifts of
Indian democracy could be enjoyed by a people who live under a military
occupation. He remembered to greet them in advance for Eid, a few days
away. But he didn't promise that the lockdown would be lifted for the
festival. It wasn't.

The next morning, the Indian newspapers and several liberal
commentators, including some of Narendra Modi's most trenchant critics
gushed over his moving speech. Like true colonials, many in India who
are so alert to infringements of their own rights and liberties, have a
completely different standard for Kashmiris.

On Thursday, Aug. 15, in his Independence Day speech, Narendra Modi
boasted from the ramparts of Delhi's Red Fort that his government
finally had achieved India's dream of
``\href{https://uk.reuters.com/article/uk-india-independenceday-modi/indias-modi-trumpets-kashmir-muslim-marriage-moves-in-independence-day-speech-idUKKCN1V50K4}{One
Nation, One Constitution,}'' with his Kashmir move. But just the
previous evening,
\href{https://www.indiatoday.in/india/story/kashmir-effect-rebel-groups-ban-independence-day-celebrations-in-northeast-1580947-2019-08-14}{rebel
groups in several troubled states in the north east of India}, many of
which have Special Status like the erstwhile State of Jammu and Kashmir,
announced a boycott of Independence Day. While Narendra Modi's Red Fort
audience cheered, about seven million Kashmiris remained locked down.
The communication shutdown, we now hear, could be extended for some time
to come.

When it ends, as it must, the violence that will spiral out of Kashmir
will inevitably spill into India. It will be used to further inflame the
\href{https://www.nytimes.com/2017/08/17/opinion/india-muslims-hindus-partition.html}{hostility
against Indian Muslims} who are already being demonized, ghettoized,
pushed down the economic ladder, and, with terrifying regularity,
\href{https://www.youtube.com/watch?v=UFRuKs7ZfEk}{lynched}. The state
will use it as an opportunity to close in on others, too --- the
activists, lawyers, artists, students, intellectuals, journalists ---
who have protested courageously and openly.

The danger will come from many directions. The most powerful
organization in India, the far-right Hindu nationalist
\href{https://www.nytimes.com/2003/02/02/magazine/the-other-face-of-fanaticism.html}{Rashtriya
Swayamsevak Sangh, or the R.S.S.}, with more than 600,000 members
including Narendra Modi and many of his ministers, has a trained
``volunteer'' militia, inspired by Mussolini's Black Shirts. With each
passing day, the R.S.S. tightens its grip on every institution of the
Indian state. In truth, it has reached a point when it more or less
\emph{is} the state.

In the benevolent shadow of such a state, numerous smaller
\href{https://caravanmagazine.in/vantage/the-rss-bhonsala-military-school-dhirendra-k-jha}{Hindu
vigilante organizations}, the storm troopers of the Hindu Nation, have
mushroomed across the country, and are conscientiously going about their
deadly business.

\href{https://www.nytimes.com/2019/03/14/magazine/gauri-lankesh-murder-journalist.html}{Intellectuals
and academics} are a major preoccupation. In May, the morning after the
Bharatiya Janata Party won the general elections, Ram Madhav, a general
secretary of the party and a former spokesman for the R.S.S., wrote that
the ``remnants'' of the ``pseudo-secular/liberal cartels that held a
disproportionate sway and stranglehold over the intellectual and policy
establishment of the country \ldots{}
\href{https://indianexpress.com/article/opinion/columns/lok-sabha-elections-result-narendra-modi-bjp-government-congress-5745313/}{need
to be discarded}from the country's academic, cultural and intellectual
landscape.''

On Aug. 1, in preparation for that ``discarding,'' the
\href{https://www.hrw.org/report/2010/07/27/back-future/indias-2008-counterterrorism-laws}{already
draconian} Unlawful Activities Prevention Act was amended to expand the
definition of ``terrorist'' to include individuals, not just
organizations. The amendment allows the government
\href{https://www.thehindubusinessline.com/news/uapa-amendment-bill-gets-rajya-sabha-approval/article28796520.ece}{to
designate any individual as a terrorist}without following the due
process of a First Information Report, charge sheet, trial and
conviction. Just who --- just what kind of individuals it means --- was
clear when in Parliament, Amit Shah, our chilling
\href{https://www.youtube.com/watch?v=fnd_ELCFhCM}{home minister,}said:
``Sir, guns do not give rise to terrorism, the
\href{https://thewire.in/rights/uapa-bjp-terrorist-amit-shah-nia}{root
of terrorism is the propaganda} that is done to spread it \ldots{} And
if all such individuals are designated terrorists, I don't think any
member of Parliament should have any objection.''

Image

Kashmiri journalists at the press club in Srinagar
recently.Credit...Atul Loke for The New York Times

Several of us felt his cold eyes staring straight at us. It didn't help
to know that he has done time as the main accused in a series of murders
in his home state, Gujarat. His trial judge, Justice
\href{https://caravanmagazine.in/tag/loya}{Brijgopal Harkishen Loya},
died mysteriously during the trial and was replaced by another who
acquitted him speedily. Emboldened by all this, far-right television
anchors on hundreds of India's news networks, now openly denounce
dissidents, make wild allegations about them and call for their arrest,
or worse. ``Lynched by TV,'' is likely to be the new political
phenomenon in India.

As the world looks on, the architecture of Indian fascism is quickly
being put into place.

I was booked to fly to Kashmir to see some friends on July 28. The
whispers about trouble, and troops being flown in, had already begun. I
was of two minds about going. A friend of mine and I were chatting about
it at my home. He is a senior doctor at a government hospital who has
dedicated his life to public service, and happens to be Muslim. We
started talking about the new phenomenon of mobs surrounding people,
Muslims in particular, and forcing them to chant
``\href{https://www.bbc.co.uk/news/world-asia-india-48882053}{Jai Shri
Ram}!'' (``Victory to Lord Ram!'')

If Kashmir is occupied by security forces, India is occupied by the mob.

He said he had been thinking about that, too, because he often drove on
the highways out of Delhi to visit his family who live some hours away.

``I could easily be stopped,'' he said.

``You must say it then,'' I said. ``You must survive.''

``I won't,'' he said, ``because they'll kill me either way. That's what
they did to
\href{https://indianexpress.com/article/india/tabrez-ansari-18th-mob-violence-victim-in-jharkhand-in-three-years-5808122/}{Tabrez
Ansari}.''

These are the conversations we are having in India while we wait for
Kashmir to speak. And speak it surely will.

Arundhati Roy is the author of the novel ``The Ministry of Utmost
Happiness.'' Her most recent book is a collection of essays, ``My
Seditious Heart.''

\emph{The Times is committed to publishing}
\href{https://www.nytimes.com/2019/01/31/opinion/letters/letters-to-editor-new-york-times-women.html}{\emph{a
diversity of letters}} \emph{to the editor. We'd like to hear what you
think about this or any of our articles. Here are some}
\href{https://help.nytimes.com/hc/en-us/articles/115014925288-How-to-submit-a-letter-to-the-editor}{\emph{tips}}\emph{.
And here's our email:}
\href{mailto:letters@nytimes.com}{\emph{letters@nytimes.com}}\emph{.}

\emph{Follow The New York Times Opinion section on}
\href{https://www.facebook.com/nytopinion}{\emph{Facebook}}\emph{,}
\href{http://twitter.com/NYTOpinion}{\emph{Twitter (@NYTopinion)}}
\emph{and}
\href{https://www.instagram.com/nytopinion/}{\emph{Instagram}}\emph{.}

Advertisement

\protect\hyperlink{after-bottom}{Continue reading the main story}

\hypertarget{site-index}{%
\subsection{Site Index}\label{site-index}}

\hypertarget{site-information-navigation}{%
\subsection{Site Information
Navigation}\label{site-information-navigation}}

\begin{itemize}
\tightlist
\item
  \href{https://help.nytimes.com/hc/en-us/articles/115014792127-Copyright-notice}{©~2020~The
  New York Times Company}
\end{itemize}

\begin{itemize}
\tightlist
\item
  \href{https://www.nytco.com/}{NYTCo}
\item
  \href{https://help.nytimes.com/hc/en-us/articles/115015385887-Contact-Us}{Contact
  Us}
\item
  \href{https://www.nytco.com/careers/}{Work with us}
\item
  \href{https://nytmediakit.com/}{Advertise}
\item
  \href{http://www.tbrandstudio.com/}{T Brand Studio}
\item
  \href{https://www.nytimes.com/privacy/cookie-policy\#how-do-i-manage-trackers}{Your
  Ad Choices}
\item
  \href{https://www.nytimes.com/privacy}{Privacy}
\item
  \href{https://help.nytimes.com/hc/en-us/articles/115014893428-Terms-of-service}{Terms
  of Service}
\item
  \href{https://help.nytimes.com/hc/en-us/articles/115014893968-Terms-of-sale}{Terms
  of Sale}
\item
  \href{https://spiderbites.nytimes.com}{Site Map}
\item
  \href{https://help.nytimes.com/hc/en-us}{Help}
\item
  \href{https://www.nytimes.com/subscription?campaignId=37WXW}{Subscriptions}
\end{itemize}
