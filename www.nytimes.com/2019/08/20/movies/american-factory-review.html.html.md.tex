Sections

SEARCH

\protect\hyperlink{site-content}{Skip to
content}\protect\hyperlink{site-index}{Skip to site index}

\href{https://www.nytimes.com/section/movies}{Movies}

\href{https://myaccount.nytimes.com/auth/login?response_type=cookie\&client_id=vi}{}

\href{https://www.nytimes.com/section/todayspaper}{Today's Paper}

\href{/section/movies}{Movies}\textbar{}`American Factory' Review: The
New Global Haves and Have-Nots

\href{https://nyti.ms/2Hh2ZR3}{https://nyti.ms/2Hh2ZR3}

\begin{itemize}
\item
\item
\item
\item
\item
\item
\end{itemize}

Advertisement

\protect\hyperlink{after-top}{Continue reading the main story}

Supported by

\protect\hyperlink{after-sponsor}{Continue reading the main story}

Critic's Pick

\hypertarget{american-factory-review-the-new-global-haves-and-have-nots}{%
\section{`American Factory' Review: The New Global Haves and
Have-Nots}\label{american-factory-review-the-new-global-haves-and-have-nots}}

A documentary looks at what happened when a Chinese company took over a
closed General Motors factory in Ohio.

\includegraphics{https://static01.nyt.com/images/2019/08/21/arts/20factory-1/merlin_159346203_caa79359-b839-4481-8005-c6d16d207dd8-articleLarge.jpg?quality=75\&auto=webp\&disable=upscale}

By \href{https://www.nytimes.com/by/manohla-dargis}{Manohla Dargis}

\begin{itemize}
\item
  Aug. 20, 2019
\item
  \begin{itemize}
  \item
  \item
  \item
  \item
  \item
  \item
  \end{itemize}
\end{itemize}

\href{https://cn.nytimes.com/culture/20190827/american-factory-review/}{阅读简体中文版}\href{https://cn.nytimes.com/culture/20190827/american-factory-review/zh-hant/}{閱讀繁體中文版}

\begin{itemize}
\tightlist
\item
  American Factory\\
  **NYT Critic's Pick Directed by Steven Bognar, Julia Reichert
  Documentary 1h 55m
\end{itemize}

\href{https://www.imdb.com/showtimes/title/tt9351980?ref_=ref_ext_NYT}{Find
Tickets}

When you purchase a ticket for an independently reviewed film through
our site, we earn an affiliate commission.

``The most important thing is not how much money we earn,'' the Chinese
billionaire Cao Dewang says in ``American Factory'' soon before we see
him on a private jet. What's important, he says, are Americans' views
toward China and its people.

In 2016, Cao opened a division of Fuyao, his global auto-glass
manufacturing company, in a shuttered General Motors factory near
Dayton, Ohio. Blaming slumping S.U.V. sales, G.M. had closed the plant
--- known as the General Motors Moraine Assembly Plant --- in December
2008, throwing thousands out of work the same month the American
government began a multibillion
\href{https://dealbook.nytimes.com/2013/12/09/u-s-sells-remaining-stake-in-gm/?rref=collection\%2Ftimestopic\%2FAuto\%20Bailout\&action=click\&contentCollection=timestopics\&region=stream\&module=stream_unit\&version=latest\&contentPlacement=9\&pgtype=collection}{dollar
bailout} of the auto industry. The Dayton factory remained idle until
Fuyao announced it was taking it over, investing millions and hiring
hundreds of local workers,
\href{https://www.daytondailynews.com/news/breaking-news/fuyao-set-double-hiring-moraine-plant-500-jobs/XIW2hI9dJ3w0Rw5Uw8KVnL/}{numbers
it soon} increased.

\href{https://www.documentary.org/online-feature/meet-academy-award-nominees-steven-bognar-and-julia-reichert-last-truck-closing-gm}{The
veteran filmmakers} Steven Bognar and Julia Reichert, who are a couple
and live outside of Dayton, documented the G.M. plant when it closed.
They included the image of the last truck rolling off the line in their
2009 short,
``\href{https://www.hbo.com/content/hboweb/en/documentaries/the-last-truck-closing-of-a-gm-plant/synopsis.html}{The
Last Truck: Closing of a GM Plant}.'' That crystallizing image also
appears in ``American Factory,'' which revisits the plant six years
later. The feature-length story they tell here is complex, stirring,
\href{https://www.nytimes.com/2019/07/26/us/politics/trump-wto-china.html}{timely}
and beautifully shaped, spanning continents as it surveys the past,
present and possible future of American labor. (This is the first movie
that Barack and Michelle Obama's company
\href{https://www.nytimes.com/2019/04/30/business/media/obama-netflix-shows.html}{Higher
Ground Productions} is releasing with Netflix.)

``American Factory'' opens with a brief, teary look back at the plant's
closing that sketches in the past and foreshadows the difficult times
ahead. The story proper begins in 2015 amid the optimistic bustle of new
beginnings, including a rah-rah Fuyao presentation for American job
seekers. Bognar and Reichert, who shot the movie with several others ---
the editor is Lindsay Utz --- have a great eye for faces and they
quickly narrow in on the range of expressions in the room. Some
applicants sit and listen stoically; one woman, her hand over her mouth,
gently rocks in her seat, tapping out a nervous rhythm as the Fuyao
representative delivers his pitch.

\includegraphics{https://static01.nyt.com/images/2019/08/20/arts/20factory-2/merlin_159346173_dc29a19d-06e7-4f26-933a-be254303b5d2-articleLarge.jpg?quality=75\&auto=webp\&disable=upscale}

With detail and sweep, interviews and you-are-there visuals, the
filmmakers quickly establish a clear, strong narrative line as the new
enterprise --- Fuyao Glass America --- gets off the ground. The optimism
of the workers is palpable; the access the filmmakers secured
remarkable. Bognar and Reichert spent a number of years making
``American Factory,'' a commitment that's evident in its layered
storytelling and the trust they earned. American and visiting Chinese
workers alike open their homes and hearts, including Wong He, an
engaging, quietly melancholic furnace engineer who speaks movingly of
his wife and children back in China.

His is just one story in an emotionally and politically trenchant
chronicle of capitalism, propaganda, conflicting values and labor
rights. As the factory ramps up, optimism gives way to unease, dissent
and fear. Some workers are hurt, others are at risk; glass breaks,
tempers fray. Both the Chinese and American management complain about
production and especially about the American workers who, in turn, seem
mainly grateful for a new shot. A forklift operator named Jill Lamantia
is living in her sister's basement when we first meet her. A job at
Fuyao allows her to move into her own apartment, but like everyone else
she struggles with the company's demands.

By the time the documentary shifts to China, for a visit by American
managers to the Fuyao mother ship, it has become clear that something
will have to give. The American subsidiary is losing money and Chairman
Cao, as he's called, is not happy. His frustration can seem amusing, but
as his dissatisfaction mounts, the temperature grows colder and
management becomes openly hostile. For viewers who have never peered
inside a Chinese factory, these scenes --- with their singalongs,
team-building exercises and extravagant pageants --- may seem strange or
perhaps a gung-ho variation on contemporary corporate management
practice (cue the next Apple confab).

``American Factory'' is political without being self-servingly didactic
or strident, connecting the sociopolitical dots intelligently, sometimes
with the help of a stirring score from Chad Cannon that evokes Aaron
Copland. The filmmakers don't villainize anyone, though a few
participants come awfully close to twirling waxed mustaches, like an
American manager who jokes to a Chinese colleague that it would be a
good idea to duct-tape the mouths of talky American workers. It's a
shocking exchange --- only the Chinese manager appears concerned that
they're on camera --- simply because of the openness of the antagonism
toward the company's own labor force.

It's these men and women --- Timi Jernigan, John Crane, Shawnea Rosser,
Robert Allen and so many others --- whose optimism and disappointment
give the movie its emotional through-line and whose stories stand in
contrast to Cao's own self-made tale. He recalls that the China of his
youth was poor; now he is, according to
\href{https://www.forbes.com/profile/cho-tak-wong/\#5f04e25416cc}{Forbes,
one of ``China's richest''} and his hobbies include golfing and
collecting art. You see the fruits of his endeavors in ``American
Factory,'' in scenes of him relaxing and pontificating. And working too,
of course, always working, including in a luxurious office where a
couple of
\href{https://www.nytimes.com/2008/09/05/arts/design/05revo.html}{socialist
realist paintings} show him against the sky like a sleekly updated Mao
--- an image that the filmmakers linger on, letting its meaning bloom
like a hundred flowers.

\textbf{American Factory}

Not rated. Running time: 1 hour 55 minutes.

Advertisement

\protect\hyperlink{after-bottom}{Continue reading the main story}

\hypertarget{site-index}{%
\subsection{Site Index}\label{site-index}}

\hypertarget{site-information-navigation}{%
\subsection{Site Information
Navigation}\label{site-information-navigation}}

\begin{itemize}
\tightlist
\item
  \href{https://help.nytimes.com/hc/en-us/articles/115014792127-Copyright-notice}{©~2020~The
  New York Times Company}
\end{itemize}

\begin{itemize}
\tightlist
\item
  \href{https://www.nytco.com/}{NYTCo}
\item
  \href{https://help.nytimes.com/hc/en-us/articles/115015385887-Contact-Us}{Contact
  Us}
\item
  \href{https://www.nytco.com/careers/}{Work with us}
\item
  \href{https://nytmediakit.com/}{Advertise}
\item
  \href{http://www.tbrandstudio.com/}{T Brand Studio}
\item
  \href{https://www.nytimes.com/privacy/cookie-policy\#how-do-i-manage-trackers}{Your
  Ad Choices}
\item
  \href{https://www.nytimes.com/privacy}{Privacy}
\item
  \href{https://help.nytimes.com/hc/en-us/articles/115014893428-Terms-of-service}{Terms
  of Service}
\item
  \href{https://help.nytimes.com/hc/en-us/articles/115014893968-Terms-of-sale}{Terms
  of Sale}
\item
  \href{https://spiderbites.nytimes.com}{Site Map}
\item
  \href{https://help.nytimes.com/hc/en-us}{Help}
\item
  \href{https://www.nytimes.com/subscription?campaignId=37WXW}{Subscriptions}
\end{itemize}
