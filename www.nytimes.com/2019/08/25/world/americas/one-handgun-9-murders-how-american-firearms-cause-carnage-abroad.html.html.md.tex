Sections

SEARCH

\protect\hyperlink{site-content}{Skip to
content}\protect\hyperlink{site-index}{Skip to site index}

\href{/section/world/americas}{Americas}\textbar{}How American Gun Laws
Are Fueling Jamaica's Homicide Crisis

\url{https://nyti.ms/3216RxC}

\begin{itemize}
\item
\item
\item
\item
\item
\item
\end{itemize}

\includegraphics{https://static01.nyt.com/images/2019/08/22/world/jamaica/jamaica-articleLarge.jpg?quality=75\&auto=webp\&disable=upscale}

\hypertarget{how-american-gun-laws-are-fueling-jamaicas-homicide-crisis}{%
\section{How American Gun Laws Are Fueling Jamaica's Homicide
Crisis}\label{how-american-gun-laws-are-fueling-jamaicas-homicide-crisis}}

Hundreds of thousands of guns sold in the United States vanish because
of loose American gun laws. Many reappear on the Caribbean island,
turning its streets into battlefields.

The Jamaican police conducting raids in downtown Kingston to disrupt
criminal activity and violence that have left the country with one of
the highest homicide rates in the world.Credit...Tyler Hicks/The New
York Times

Supported by

\protect\hyperlink{after-sponsor}{Continue reading the main story}

By \href{https://www.nytimes.com/by/azam-ahmed}{Azam Ahmed}

Photographs by \href{https://www.nytimes.com/by/tyler-hicks}{Tyler
Hicks}

\begin{itemize}
\item
  Published Aug. 25, 2019Updated Aug. 29, 2019
\item
  \begin{itemize}
  \item
  \item
  \item
  \item
  \item
  \item
  \end{itemize}
\end{itemize}

\href{https://www.nytimes.com/es/2019/08/26/espanol/america-latina/jamaica-violencia-armas.html}{Leer
en español}

CLARENDON, Jamaica --- She came to Jamaica from the United States about
four years ago, sneaking in illegally, stowed away to avoid detection.
Within a few short years, she became one of the nation's most-wanted
assassins.

She preyed on the parish of Clarendon, carrying out nine confirmed
kills, including a double homicide outside a bar, the killing of a
father at a wake and the murder of a single mother of three. Her
violence was indiscriminate: She shot and nearly killed a 14-year-old
girl getting ready for church.

With few clues to identify her, the police named her Briana. They knew
only her country of origin --- the United States --- where she had been
virtually untraceable since 1991. She was a phantom, the
eighth-most-wanted killer on an island with no shortage of murder,
suffering one of the highest homicide rates in the world. And she was
only one of thousands.

Briana, serial number 245PN70462, was a 9-millimeter Browning handgun.

An outbreak of violence is afflicting Jamaica, born of small-time gangs,
warring criminals and neighborhood feuds that go back generations ---
hand-me-down hatred fueled by pride. This year, the government called a
state of emergency to stop the bloodshed in national hot spots, sending
the military into the streets.

Guns like Briana reside at the epicenter of the crisis. Worldwide, 32
percent of homicides are committed with firearms, according to
\href{https://igarape.org.br/en/apps/homicide-monitor/}{the Igarapé
Institute}, a research group. In Jamaica, the figure is higher than 80
percent. And most of those guns come from the United States, amassed by
exploiting loose American gun laws that facilitate the carnage.

\includegraphics{https://static01.nyt.com/images/2019/08/22/world/Jamaica-91/Jamaica-91-articleLarge.jpg?quality=75\&auto=webp\&disable=upscale}

Image

The burial of Ms. Burke-Frazer, who was killed with one of the many
illegal guns on the street.Credit...Tyler Hicks/The New York Times

While the gun control debate has flared in the United States for decades
--- most recently after the mass shootings this month in
\href{https://www.nytimes.com/2019/08/03/us/el-paso-shooting.html?module=inline}{El
Paso}and
\href{https://www.nytimes.com/2019/08/04/us/dayton-ohio-shooting.html}{Dayton}
--- American firearms are pouring into neighboring countries and
igniting record violence, in part because of federal and state
restrictions that make it difficult, or sometimes nearly impossible, to
track the weapons and interrupt smuggling networks.

In the United States, the dispute over guns focuses almost exclusively
on the policies, consequences and constitutional rights of American
citizens, often framed by the assertion ``guns don't kill people, people
kill people'' --- that the reckless acts of a few should not dictate
access for all.

But here in Jamaica, there is no such debate. Law enforcement officials,
politicians and even gangsters on the street agree: It's the abundance
of guns, typically from the United States, that makes the country so
deadly. And while the argument over gun control plays on a continual
loop in the United States, Jamaicans say they are dying because of it
--- at a rate that is nine times the global average.

``Many people in the U.S. see gun control as a purely domestic issue,''
said Anthony Clayton, the lead author of Jamaica's 2014 National
Security Policy. But America's ``long-suffering neighbors, whose
citizens are being murdered by U.S. weapons, have a very different
perspective.''

Firearms play such a central role in Jamaican murders that the
authorities keep a list of the nation's 30 deadliest guns, based on
ballistic matches. To keep track of them, they are given names, like
Ghost or Ambrogio.

Some, like Briana, are so poorly documented that the United States
Bureau of Alcohol, Tobacco, Firearms and Explosives has nothing more
than a piece of paper with the name and details of the original buyer,
according to confidential documents reviewed by The New York Times.

Purchased in 1991 by a farmer in Greenville, N.C., the Browning vanished
from the public record for nearly 24 years --- until it suddenly started
wreaking havoc in Jamaica. For three years, its ballistic fingerprint
connected it to shootings, mystifying law enforcement. Finally, after a
firefight with the police, it was recovered last year and its bloody run
came to an end.

The authorities traced the serial number back to the handgun's original
owner. But that did not explain how the weapon wound up in Jamaica
decades later. Or how the authorities could prevent the next Briana from
arriving.

The mystery is no accident. By law, licensed gun merchants in the United
States are not required to do much more than record retail sales, and
usually don't have to report them to the authorities. After that, if a
gun is stolen, lost or handed to someone else, paperwork is only
sometimes required.

Only a few American states
\href{https://lawcenter.giffords.org/gun-laws/policy-areas/gun-owner-responsibilities/registration/}{mandate
the registration} of some or all firearms. Several other states
explicitly prohibit it. And there is no national, comprehensive registry
of gun ownership. The federal government
\href{https://www.nraila.org/get-the-facts/registration-licensing/\#_edn4}{is
forbidden}to create one.

Drawing on court documents, case files, dozens of interviews and
confidential data from law enforcement officials in both countries, The
Times traced a single gun --- Briana --- to nine different homicides in
Clarendon, a largely rural area of Jamaica where violence has spiked in
recent years.

It is just one of the hundreds of thousands of guns that leak out of the
United States and overwhelm countries in Latin America and the
Caribbean. More than 100,000 people
\href{https://www.nytimes.com/2019/08/18/world/americas/guatemala-violence-women-asylum.html}{are
killed every year}
\href{https://www.nytimes.com/interactive/2019/05/04/world/americas/honduras-gang-violence.html}{across
the region} --- most of them by firearms.

``I still love him and miss him all the time,'' said Clovis Cooke Sr.,
weeping over the murder of his son, Clovis Jr., who was gunned down in
2017 with the Browning the authorities call Briana.

``He took care of me,'' Mr. Cooke said of his son. ``Every week he would
come by and bring food and groceries and pay the bills.''

Jamaica brims with losses like his. American weapons are routinely
funneled into the country aboard ships, flooding cities like Kingston,
the capital, where high-grade assault rifles are wielded by warring
gangs.

Image

A gang member holding a Kalashnikov rifle at a house in Kingston,
Jamaica.Credit...Tyler Hicks/The New York Times

Image

The police raiding a house in Kingston while looking for a gang
member.Credit...Tyler Hicks/The New York Times

Jamaica's own gun laws are relatively strict, with fewer than 45,000
legal firearms in a country of almost three million.

But it is awash in illegal weapons. The Jamaican authorities, who
estimate that 200 guns are smuggled into the country from the United
States every month, routinely ask American officials to examine some of
the weapons they seize in raids, during traffic stops or at the ports.

Of the nearly 1,500 weapons the A.T.F. checked from 2016 through 2018,
71 percent came from the United States.

The figures are similar in Mexico, which has been lobbying the United
States for more than a decade to stop
\href{https://www.nytimes.com/2009/04/15/us/15guns.html}{the illegal
guns flowing south}. By some estimates, more than 200,000 guns are
trafficked into Mexico each year, many to feed the vast criminal
networks fighting over the multibillion-dollar drug trade to the United
States.

But here in Jamaica, the killings are rarely driven by such enormous
profits. The drug trade has fallen from its heyday, organized crime has
been fractured and most of the historic kingpins have been killed or
imprisoned.

Instead, the guns in Jamaica are often used in petty feuds, neighborhood
beefs and turf wars that go back decades, to when political parties
authored the majority of the country's violence.

Because guns are so plentiful, small insults and old vendettas that
might otherwise leave few casualties grow much more dangerous --- not
just for the combatants, but also for anyone who happens to be in the
way.

``A lot of violence is the result of people settling their disputes, and
with all the guns in the country, it is easy to settle things that
way,'' said Orlando Patterson, a Jamaican-born sociology professor at
Harvard University. ``That is where it's at right now. The early
factors, the politics, international drugs, they are gone.''

Even some of the gang members agree they are often fighting over small
stakes --- and sometimes no financial stakes at all.

``I mean, with or without the guns, we will still fight,'' said one gang
leader in Kingston, speaking on condition of anonymity for fear of
arrest. ``But the guns make it deadlier. There would be a big difference
without as many guns.''

\hypertarget{from-north-carolina-into-thin-air}{%
\subsection{From North Carolina Into Thin
Air}\label{from-north-carolina-into-thin-air}}

Johnnie Ray Dunn walked into a North Carolina gun store in the fall of
1991 and purchased an American icon: a 9-millimeter Browning.

With its all-steel frame, the gun was built to weather abuse, with a
reputation for accuracy and functionality.

Mr. Dunn, a farmer, handed over his details and went home with a gun
that, if maintained, would last a lifetime.

That's where Briana's paper trail began --- and ended.

President Ronald Reagan
had\href{https://www.congress.gov/bill/99th-congress/senate-bill/49}{signed
a bill}that prohibited the creation of any sweeping national gun
registry five years earlier, a pivotal piece of legislation in the
history of American gun law.

The National Rifle Association lobbied heavily for the bill, which many
saw as a way of expanding gun sales by ensuring easy access to firearms.
Underpinning the effort was a warning that still resonates with many of
the law's supporters today: that a national registry would enable the
United States government to keep track of gun owners and crack down on
their right to bear arms.

``It will be used to take away our guns,'' said John Donohue III, a
professor at Stanford Law School, explaining one of the main talking
points against a national registry.

The law effectively ruled out a federal system of tracking all firearms.
So when Mr. Dunn's gun suddenly showed up in Jamaica, linked to a series
of homicides from 2015 through early 2018, no one could figure out how
it got there.

The A.T.F. was unable to trace the gun beyond its initial purchase, and
Mr. Dunn would not have been required to report if it had been sold,
swapped, lost or stolen. The weapon disappeared into what some experts
call the black hole of American gun laws.

Mr. Dunn died in 2011, according to a local newspaper obituary, and is
not considered a suspect in the gun's path to Jamaica. The Times
attempted to reach his family, without success.

Guns like his regularly torment Jamaican officials. Most firearms used
in crimes are orphans of a system that seems geared to forget them.
Purchased legally, they eventually fall into the vast ocean of what the
A.T.F. estimates to be more than 300 million guns circulating in the
United States, their chain of ownership often irrevocably broken.

``This is the stereotypical crime gun,'' said Joseph Blocher, a
professor at Duke University School of Law. ``They almost all originate
with a legal sale and are then passed on, stolen or otherwise vanish
before reappearing in a crime.''

Because Jamaican officials cannot tell how handguns like the
9-millimeter Browning entered their country --- even with the assistance
of American law enforcement --- they struggle to shut down the smuggling
rings that fuel the nation's violence.

Image

American weapons are routinely funneled into the country, often by
container, flooding cities like Kingston.Credit...Tyler Hicks/The New
York Times

All they know is that, more than 20 years after being sold in North
Carolina, the handgun became one of the most lethal in Jamaica, the tool
of a one-eyed gangster named Hawk Eye.

Samuda Daley got the nickname as a boy. He saw poorly out of one eye,
and after an unsuccessful surgery left it covered in a milky film, his
alias was born.

Mr. Daley was a product of violence, shaped by its near constant
presence in his life. As a child, a relative said, his mother was
stabbed to death by his uncle.

By ninth grade, he had dropped out of school to start working at a sugar
factory, telling his family he didn't want to rely on anyone. He joined
the Gaza gang, a clique of young men who had grown up together in a
knotted cluster of streets in Clarendon.

They began by hanging out, not fighting, his family said. But in the
crucible of poverty and desperation, where small conflicts can turn
deadly, they ran afoul of a similar group, the King Street gang. The
rivalry grew quickly.

On Sept. 19, 2015, almost exactly 24 years after Mr. Dunn purchased the
gun, the first sign that it had made its way to Jamaica appeared: A man
named Okeeve Martin was killed with an unknown 9-millimeter Browning.

There was no money or territory at stake, residents say. The motive
seemed to be revenge --- the girlfriend of the Gaza gang's leader had
been shot by mistake in an earlier episode.

She survived, but the rumor mill led to Mr. Martin, and retribution came
swiftly.

The gun lay dormant for a year before claiming the life of a
17-year-old, Shane Sewell, on Sept. 6, 2016. He was walking home, having
left a bar after a night with friends. He ended up in a ditch, riddled
with bullets, some from the mysterious Browning.

Officials believe he was killed in a dispute over a different firearm.
In Jamaica, guns are often rented out by their owners, as a hardware
store might rent out valuable tools. The borrower, looking to commit a
robbery or even kill someone, pays a fee to use the weapon. Afterward,
the gun is returned. Given a gun's income potential, when one is lost or
stolen, the consequences can be deadly.

In the summer of 2017, the Browning struck again. Kurt Mitchell, a
fisherman believed to be a member of the King Street gang, was gunned
down at a party --- a reprisal for an earlier homicide against the Gaza
gang, the authorities believe.

His death, in turn, generated still more deaths, in the tragic rhythm
that violence often takes in Jamaica.

Much of the fighting today stems from political conflicts that stretch
back long before the shooters were born. In past decades, armed groups
loyal to one of the two major parties --- the Jamaica Labour Party and
the People's National Party --- battled one another for dominance.

The patronage networks eventually transitioned to crime, stripped of
their political focus. Local leaders, known as Dons, grew incredibly
powerful, as deep connections to the United States, Canada and Britain
enabled their criminal enterprises to become transnational.

But that, too, changed as the government cracked down on the Dons and
targeted the drug trade in Jamaica. By 2010, the Dons were all but a
thing of the past, with the last major player, Christopher Coke, known
locally as Dudus,
\href{https://www.nytimes.com/2010/06/26/world/americas/26coke.html}{arrested
and extradited}to the United States after battles that resulted in the
deaths of at least 73 people.

``2010 was a watershed moment,'' said Damian Hutchinson, the executive
director of the Peace Management Initiative, which works to stop
violence in Jamaica's most dangerous neighborhoods. ``The Don culture
started to change. The political enforcers were now undermined by
younger, less conscientious individuals with less purpose to the
violence.''

Image

The police conducting raids in high-crime areas of downtown
Kingston.Credit...Tyler Hicks/The New York Times

Image

Residents watched during a raid as the police looked for a gang member
wanted in connection with weapons charges.Credit...Tyler Hicks/The New
York Times

The splintered factions began fighting one another, leading to more ---
and more random --- violence. Wars broke out between once-aligned blocks
and the gangs multiplied, to more than 250 nationwide today.

Those armed factions, fighting a small-scale war, have lifted homicides
to new peaks.

The 9-millimeter Browning became a terrifying facet of this landscape,
with evidence tying it to more than eight homicide scenes.

As officials tried to stitch together the clues, the gun was repeatedly
being used as an enforcement tool of the Gaza gang, often by Mr. Daley,
the killer known as Hawk Eye.

He was quiet, never bragging about his exploits, residents and family
members said. He didn't need to. His ruthlessness was well known, and
neighbors afforded him a grudging respect.

Mr. Daley had become embroiled in a personal feud with another gangster,
Christopher Lynch, and some of the shootings that plagued Clarendon in
2017 came from their hatred for each other, officials say.

They had once been close friends, almost like family, relatives said,
but that former intimacy now burned with an equally intense hostility.
Mr. Daley tried to kill him on a Sunday in 2017, when he spotted him
walking home from a soccer game.

He fired at Mr. Lynch, who took off running through the woods and
escaped, officials say. But a stray bullet struck a 14-year-old girl in
the stomach as she prepared for church. Luckily, the girl survived.

Months later, Mr. Lynch's father was at a wake, a late-night affair with
drinks and music, a celebration of life common in parts of Jamaica. At
around 10:30 p.m., investigators now believe, Mr. Daley stormed the wake
and began shooting. The elder Lynch died. Three others were injured.

Once again, the bullet fragments connected the shootings to the 9
millimeter.

\hypertarget{from-idaho-to-montego-bay}{%
\subsection{From Idaho to Montego Bay}\label{from-idaho-to-montego-bay}}

Not all guns vanish without a trace and suddenly reappear, decades
later. Some are bought openly and sent overseas right away.

From late 2016 through early 2017, a 74-year-old man from Idaho
purchased three military-style rifles and a Glock .45 pistol in
Meridian, Idaho, a town of about 100,000 people surrounded by more than
two dozen gun stores.

Six months later, all four guns were recovered by the Jamaican
authorities in a raid in the Montego Bay area, where criminal violence
has overwhelmed the parish of St. James.

The area is a notable exception to Jamaica's vendetta violence. A
multimillion-dollar scamming industry has flourished there, inciting so
many homicides that the government sent in the military.

Image

A cemetery in Kingston where numerous gang members have been
buried.Credit...Tyler Hicks/The New York Times

Image

The police patrolling in downtown Kingston. The Jamaican authorities
have struggled to trace weapons, including many bought legally in the
United States.Credit...Tyler Hicks/The New York Times

The scammers --- who swindle American citizens into sending money or
divulging their bank account information --- are well financed and
capable of building armories to battle their competitors.

The weapons, like other illicit arms in Jamaica, arrive in containers
aboard the hundreds of ships that come to the island each month. Often,
they slip through in small batches, broken down into parts and hidden in
freezers or car engines to evade inspectors.

Of course, not all illegal guns in Latin America and the Caribbean come
from the United States. In some countries, including those with weapons
left over from civil wars, fewer than half of the illicit weapons trace
back to American soil.

But firearms trafficking from the United States is such a big problem
that the A.T.F. says it is dedicated to fighting it. Commercial traffic
between the United States and Jamaica has become more closely surveilled
in recent years, so smugglers have started bringing in the guns through
Haiti, too, often in exchange for marijuana or even meat.

Criminal networks, like those in the scamming industry, also turn to
straw buyers in the United States --- people who purchase the guns
legally and send them to Jamaica, either complicit, misled or
uninterested in how they are used.

The Idaho man may have been a victim of the scammers himself. Officials
say the swindlers appear to have pressured him into buying the weapons,
promising to return his pilfered savings.

It was the Glock .45 that caught the attention of American and Jamaican
authorities. Only three months after the Idaho man purchased it, the gun
was already in Jamaica --- and had killed Jeffrey Cato, a 39-year-old
mentally ill man, on March 17, 2017.

Mr. Cato, a beloved figure in the community of Flankers, had no obvious
enemies. He seemed to float in his own space, neighbors said, harmless
and uninvolved.

On the day of his death, Mr. Cato was getting food for one of his
children. The police never identified a motive, but believe he may have
witnessed a murder.

``He had no gang connections whatsoever,'' said one detective, speaking
anonymously because the investigation was still open. ``In my eight
years working, there's only a few cases that still stick with me. This
is one of them.''

Last July, the gun was used again, to kill Nicholas Kerr, a quiet
41-year-old who lived in the basement of his mother's home. He was shot
at a corner store, buying a soda.

``We've always had enemies here,'' said Mr. Kerr's mother, withholding
her name for fear of retribution. ``But Nicholas?'' she added. ``He was
peaceful.''

\hypertarget{every-day-they-kill-people}{%
\subsection{`Every Day They Kill
People'}\label{every-day-they-kill-people}}

Joviane Hall was D.J.-ing at a local bar near Clarendon at 11:30 p.m. on
Oct. 6, 2017, when gunmen burst in without warning.

After robbing the bar and its patrons, they opened fire, hitting Mr.
Hall, who died on the way to the hospital. Officials recognized the
culprit, a weapon they had come to loathe: the Browning.

The murder was the beginning of a spree. Two days later, another
shooting occurred at the Three Sisters Bar. At around 10:50 p.m., two
friends, Clovis Cooke and Otis Gordon, were standing outside, drinking,
when a car pulled up.

The shooter fired 21 shots and sped off. Investigators found yet another
set of 9-millimeter fragments.

Every murder committed, every life taken, left a wound that never
healed. Ten minutes from the Three Sisters Bar, which is now dormant and
overgrown with dense foliage, Mr. Cooke's parents live in their simple,
vinyl-sided home off the side of the highway.

His father, recovering from cataract surgery, plodded around in the
dark, searching for overdue bills in the drift of papers on the small
dining table.

Image

``I still love him and miss him all the time,'' said Clovis Cooke Sr.,
as he recounted the murder of his son, Clovis Jr., who was gunned down
in 2017 with the 9-millimeter Browning the authorities call
Briana.Credit...Tyler Hicks/The New York Times

He wept at the mention of his son, 33, who used to pay the bills and
help out around the house. Married at 15, his parents grew up raising
him. But time had inverted their roles, and now, without him, they were
nearly destitute.

``I think about him everyday,'' he said. ``Every day they kill people,''
he said, ``and every day we grieve about it.''

The same void haunted the home where Jody Ann Harvey was killed less
than two months after Mr. Cooke, in what some believe was a case of
mistaken identity.

Gunmen charged into her one-room shack, kicking open the door and firing
on Ms. Harvey and her daughter as they slept in the small bed they
shared. Ms. Harvey covered the girl with her body, taking six
9-millimeter rounds in the hail of gunfire. Her daughter survived.

Last spring, the home still sat abandoned in a thicket of trees, its
wooden stairwell rotting, its blue and green paint blistered. Ashley
Wilson, Ms. Harvey's sister, had come by --- to visit, to fill the
single room with memories. To mourn.

``I just miss her, I guess,'' she said, swinging the rickety door open.
``I go inside, into her room, where it happened. It brings back a lot of
memories. I'll look at pictures of her, listen to music we liked, talk
to her daughter. This is how it goes.''

The deadly run of the Browning ended, in some ways, the way it began.

Joy Commock, the girlfriend of the Gaza gang's leader --- the person who
had been shot by mistake and survived, starting the cycle of revenge
that first set the handgun loose on Jamaica --- was killed on Jan. 21,
2018.

The casings matched the earlier crimes: The gun killed Ms. Commock as
well, officials say.

She was home alone with her daughter when she heard a noise, the police
say. It was just after midnight and the smell of smoke filled the air.

She raced outside and found a fire burning in her front yard. She knelt
to extinguish the flames, and was shot multiple times by an assailant
hiding in the shadows.

Her daughter, one of three, hid inside. When the girl emerged, her
mother was dead, lying face down in the yard.

``She was the sole breadwinner,'' said Ms. Commock's sister, Lotoya
Evans. ``They were her life.''

``They expect you to forget about it, but when you lose somebody, you
don't just get up and act normal,'' she added, holding her own daughter
tight.

By early 2018, the authorities were still no closer to finding the gun.
They knew its caliber, and even the conflict the gun was caught in. But
while Mr. Daley, the enforcer known as Hawk Eye, was still alive, no
witnesses dared to testify.

At around 11 p.m. on April 28, an off-duty policeman was having a drink
at a local bar in Clarendon when two men showed up to rob it. One of
them was Mr. Daley, who flashed the Browning at patrons and demanded
money. The officer drew on the two men and announced himself, officials
say.

Mr. Daley turned and fired, but the policeman had the drop on both men,
killing Mr. Daley on the spot.

And like that, the gun was off the streets.

Witnesses came forward to link Mr. Daley to other shootings, officials
say, and the police later asked the A.T.F. to run a trace on his weapon.

It led all the way to North Carolina, to a time before Mr. Daley was
even born.

Image

A street in Kingston. Much of the current violence stems from political
conflicts that stretch back decades, before some of today's gang members
were born.Credit...Tyler Hicks/The New York Times

Advertisement

\protect\hyperlink{after-bottom}{Continue reading the main story}

\hypertarget{site-index}{%
\subsection{Site Index}\label{site-index}}

\hypertarget{site-information-navigation}{%
\subsection{Site Information
Navigation}\label{site-information-navigation}}

\begin{itemize}
\tightlist
\item
  \href{https://help.nytimes.com/hc/en-us/articles/115014792127-Copyright-notice}{©~2020~The
  New York Times Company}
\end{itemize}

\begin{itemize}
\tightlist
\item
  \href{https://www.nytco.com/}{NYTCo}
\item
  \href{https://help.nytimes.com/hc/en-us/articles/115015385887-Contact-Us}{Contact
  Us}
\item
  \href{https://www.nytco.com/careers/}{Work with us}
\item
  \href{https://nytmediakit.com/}{Advertise}
\item
  \href{http://www.tbrandstudio.com/}{T Brand Studio}
\item
  \href{https://www.nytimes.com/privacy/cookie-policy\#how-do-i-manage-trackers}{Your
  Ad Choices}
\item
  \href{https://www.nytimes.com/privacy}{Privacy}
\item
  \href{https://help.nytimes.com/hc/en-us/articles/115014893428-Terms-of-service}{Terms
  of Service}
\item
  \href{https://help.nytimes.com/hc/en-us/articles/115014893968-Terms-of-sale}{Terms
  of Sale}
\item
  \href{https://spiderbites.nytimes.com}{Site Map}
\item
  \href{https://help.nytimes.com/hc/en-us}{Help}
\item
  \href{https://www.nytimes.com/subscription?campaignId=37WXW}{Subscriptions}
\end{itemize}
