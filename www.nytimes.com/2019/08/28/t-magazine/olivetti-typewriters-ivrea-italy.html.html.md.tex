Sections

SEARCH

\protect\hyperlink{site-content}{Skip to
content}\protect\hyperlink{site-index}{Skip to site index}

\href{https://myaccount.nytimes.com/auth/login?response_type=cookie\&client_id=vi}{}

\href{https://www.nytimes.com/section/todayspaper}{Today's Paper}

Utopia, Abandoned

\href{https://nyti.ms/2PhfQcw}{https://nyti.ms/2PhfQcw}

\begin{itemize}
\item
\item
\item
\item
\item
\item
\end{itemize}

Advertisement

\protect\hyperlink{after-top}{Continue reading the main story}

Supported by

\protect\hyperlink{after-sponsor}{Continue reading the main story}

\hypertarget{utopia-abandoned}{%
\section{Utopia, Abandoned}\label{utopia-abandoned}}

The Italian town Ivrea was once a model for workers' rights and
progressive design. Now, it's both a cautionary tale and evidence of a
grand experiment in making labor humane.

\includegraphics{https://static01.nyt.com/images/2019/08/28/t-magazine/28tmag-companytowns-slide-MJLG/28tmag-companytowns-slide-MJLG-articleLarge.jpg?quality=75\&auto=webp\&disable=upscale}

By \href{https://www.nytimes.com/by/nikil-saval}{Nikil Saval}

\begin{itemize}
\item
  Published Aug. 28, 2019Updated Sept. 19, 2019
\item
  \begin{itemize}
  \item
  \item
  \item
  \item
  \item
  \item
  \end{itemize}
\end{itemize}

IN THE 1950S, the small town of
\href{https://www.nytimes.com/2007/02/19/world/europe/19orange.html}{Ivrea},
which is about an hour's train ride north of Turin, became the site of
an unheralded experiment in living and working. Olivetti, a renowned
designer and manufacturer of typewriters and accounting machines,
decided to provide for its employees through retirement. They were given
the opportunity to take classes at an on-site sale and trade school;
their lunchtime hours would be filled with speeches or performances from
visiting dignitaries (actors, musicians, poets); and they would receive
a substantial pension upon retirement. They would be housed, if they
liked, in Olivetti-constructed modern homes and apartments. Their
children would receive free day care, and expecting mothers would be
granted 10 months maternity leave. July would be a time of holiday, so
that workers with homes in the surrounding countryside could tend to
small farms --- it was important to the company that workers not feel a
division between city and country. Italy's best
\href{https://www.nytimes.com/2019/08/08/t-magazine/osvaldo-borsani-home-varedo.html}{Modernist}
architects would be hired to design in the Modernist style: Factories,
canteens, offices and study areas would be airy palaces of glass curtain
walls, flat concrete roofs and glazed brick tile. It would be a model
for the nation, and for the world.

All of this was the initiative of Adriano Olivetti, who had inherited
the company from his father, Camillo, who founded it in the early 20th
century. Adriano, born in 1901, was a businessman of unusually wide
learning, with strong inclinations toward humanism. He was a self-taught
student of city planning, and he read extensively the architectural and
urbanist literature of the day. He hired famous designers to work on his
products, making some of them, such as the 1949 Lettera 22 typewriter
and the 1958 Elea 9003 mainframe computer, into icons of design.
Olivetti was a devout Christian and a socialist, but he was distant from
the two main political parties, the Christian Democrats and the
Communists, that occupied these poles in midcentury Italy. Instead, in
1946, he formed his own political party, Il Movimento Comunità, which
was intended to shift power to the diverse social bases and competences
of a broadly conceived community, away from the patronage and
bureaucracy encouraged by Italy's political parties, thereby charting a
new course for not only the country but for the entire modern age.
Though it was a failure, his ideas of increased welfare provision became
more common and acceptable in Italian politics.

\emph{{[}}\href{https://www.nytimes.com/newsletters/t-list?module=inline}{\emph{Sign
up here}} \emph{for the T List newsletter, a weekly roundup of what T
Magazine editors are noticing and coveting now.{]}}

Today, the infrastructure the company built might sound like the
standard ``company town,'' such as 19th-century
\href{https://www.nytimes.com/1990/08/12/travel/the-town-that-pullman-built.html}{Pullman,
Ill.}, built by the Pullman railway company, but Olivetti was in fact
different. In America, company towns first arose as a result of low-wage
workers lacking both rights and basic amenities like transportation. The
more dependent an employee was on the company he worked for, the more
control the company had: Complacent workers whose boss is also their
landlord don't strike or ask for sick leave or better health care --- or
so the logic went. This era of company towns in America was effectively
ushered out by modernity, as labor rights increased thanks to New Deal
domestic policies --- and also because, in some instances, workers began
striking when employers attempted to evict them from company housing.
The rise of mass transport also made proximity to the workplace less of
an essential need.

In Europe, however, the company town had its roots in the model estates
of the Victorian era, where wealthy landowners housed workers and
caretakers in paltry accommodations. At the dawn of the 20th century,
and in a rapidly industrializing Italy especially, the fortunes of
various small towns were, and for the most part remain, inextricably
linked to private companies. The main draw of Rosignano Solvay,
established in 1912 in southern Tuscany, for instance, is its beautiful
white sand beaches, the blanching of the sand a result of toxic chemical
waste from the still-operational Solvay plant, which gave the town its
name. Colleferro, a dreary town just outside of Rome built around a
munitions factory that closed in 1968, has been plagued for the last 70
years by occasional explosions. (There are still company towns in Italy
--- the designer
\href{https://www.nytimes.com/2017/06/15/t-magazine/food/brunello-cucinelli-italian-dinner-party-tips.html}{Brunello
Cucinelli} has spent the last 30 years restoring the Umbrian hamlet of
\href{https://www.nytimes.com/2018/09/21/fashion/brunello-cucinelli-italy.html}{Solomeo}
to serve as his eponymous company's headquarters, and
\href{https://www.nytimes.com/2016/09/23/t-magazine/fashion/diego-della-valle-tods-interview-milan-fashion-week.html}{Diego
Della Valle}, C.E.O. of the fashion brand
\href{https://www.nytimes.com/2015/09/25/t-magazine/tods-shoes-milan-fashion-week.html}{Tod's
Group}, relies on local craftsmen from Casette d'Ete, a region on the
country's east coast, where his company's main factory is located.) But
many of the best-known towns that orbit around a single industry or
company can seem decidedly un-Italian: There is no ancient architecture
or grand cultural tradition because much of what remains of their
history is contained almost exclusively within the 20th century. The
people who still live in these towns are often descendants of the
original company workers that inhabited them, even though the company
has long since packed up and left. But Olivetti is unique among these
places; for a time, it was likely the most progressive and successful
company town anywhere in the world, existing not for the sake of control
or convenience but rather representing a new and short-lived kind of
corporate idealism, in which business, politics, architecture and the
daily life of the company's employees all informed one another.

Image

The interior of the former La Serra Complex.Credit...Nick Ballón

Image

Olivetti's 1932 MPI typewriter.Credit...Nick Ballón

In 1960, Adriano died, and the company --- already saddled with its
ill-advised
\href{https://www.nytimes.com/1963/10/26/archives/olivetti-underwood-corp-completes-merger-terms.html}{acquisition}
of the American typewriter company Underwood --- went into a crisis.
Adriano's brother, Roberto, took over but lacked Adriano's sense of
vision. Twenty-eight years later, Carlo De Benedetti, a figure imbued
with the ethos of a corporate raider, began to streamline Olivetti,
shedding its socialist impulses in a bid to compete in the computer age.
His efforts
\href{https://www.nytimes.com/1996/09/04/business/de-benedetti-steps-down-as-the-chairman-of-olivetti.html}{failed}.
By the 1980s, Olivetti had become subject to the same global headwinds
as many manufacturers, and the company foundered. In the early 2000s, it
was merged with a telecom giant. At its peak in the 1970s, the company
had 73,283 workers worldwide; today, it has around 400. But it's the
surrounding town that has been affected most deeply. Ivrea today has a
population of 24,000, having lost a quarter of its residents since the
1980s. The average age is 48.

In 2018, UNESCO declared Ivrea a World Heritage site; the effect has so
far, for better or for worse, been unnoticeable. (``UNESCO's `World
Heritage' listing is the kiss of death,'' the acerbic Italian critic
Marco D'Eramo
\href{https://newleftreview.org/issues/II88/articles/marco-d-eramo-unescocide}{wrote
in a 2014 article} for New Left Review. ``Once the label is affixed, the
city's life is snuffed out; it is ready for taxidermy.'') Arriving by
commuter rail from Turin, one would have no idea that one was in a
former capital of industrial design. An eerie spellbound nothingness
prevails. Except for a set of fading explanatory placards along the
town's main road, there are few signs pointing to the landmark buildings
--- a housing project designed by Marcello Nizzoli, the lead designer of
the Lettera 22; the Olivetti Research Center, designed by the architect
Eduardo Vittoria, where the Elea computer was conceptualized --- once
renowned as much for their design as for the part they played in a
munificent private welfare state. Only one of the office buildings is
still in use. A former factory has been converted into a gym. Many of
the remaining dozen or so structures are empty, speechless monuments to
an aborted utopia.

\includegraphics{https://static01.nyt.com/images/2019/08/28/t-magazine/28tmag-companytowns-slide-VUUO/28tmag-companytowns-slide-VUUO-articleLarge.jpg?quality=75\&auto=webp\&disable=upscale}

FROM ABOVE, IVREA is an hourglass, cinched in the middle where it is
crossed by the Dora Baltea river. The northern side is the historic
center, with the usual array of squeezed cobblestoned streets issuing
into breathable piazzas. The southern side, with the buildings located
at distances best traversed by automobile, sometimes set back from the
Via Jervis and fronted by useless ceremonial greenery, is where the
city's decrepit industrial and managerial heritage lies. Via Jervis is
the chief artery. It is a name that feels strange to say in Italian,
though it is dedicated to the partisan Guglielmo ``Willy'' Jervis, who
was captured by fascists in 1944 and executed by firing squad. To walk
it, as I did from my guest home in an adjoining town, is to experience
the desolation of an idea that has gone to seed. An office building from
the 1980s looks faded and unremarkable without the hum of activity that
must have once surrounded it. Tennis courts are covered with weeds.

Ivrea had been a settlement since the fifth century B.C., and under the
Roman Republic it went by the name of Eporedia. But it came into greater
prominence during the Renaissance, when it fell under the sway of the
\href{https://www.nytimes.com/2017/05/10/t-magazine/travel/turin-italy-art-carol-rama-carlo-mollino-castello-di-rivoli.html}{Turin}-based
House of Savoy. A sterling example of this past lingers in the convent
of San Bernardino, with its excellent frescoes of the life of Christ
completed around 1490 by the minor Italian artist Giovanni Martino
Spanzotti. It was to this convent that Camillo Olivetti, born and raised
in the surrounding Alpine foothills that are visible from nearly
anywhere in the town, moved his family when he established his
typewriter company in a still-standing brick building. If you stand in
front of the tan stucco of San Bernardino, you stare directly at the
once-modern exteriors of Olivetti, whose glass exteriors were meant to
exude the future and reflect the past.

In contemporary Ivrea, however, it is hard to imagine the bustle of the
recent past. A former employee, Enrico Capellaro, who had started in
manufacturing in the 1950s before working his way up to management,
described his daily routine as fairly relaxed: Renowned Italian actors
like
\href{https://www.nytimes.com/2000/06/30/movies/vittorio-gassman-77-veteran-italian-star-comfortable-in-classics-and-comedy-dies.html}{Vittorio
Gassman} and comedians came through at lunchtime. New books and
magazines could be consulted at the 30,000-volume library (which was
open to all Ivreans). A Pullman bus would drive through town at midday,
carrying workers home for lunch, if they wanted. The Social Services
Building, built of sandy concrete and organized entirely around
repeating hexagonal shapes, from spindly columns to large rooms, was
across from the main factory buildings and was where the company offered
health care to its workers.

\hypertarget{for-a-time-ivrea-was-likely-the-most-progressive-and-successful-company-town-in-the-world-representing-a-new-and-short-lived-kind-of-corporate-idealism}{%
\subsection{For a time, Ivrea was likely the most progressive and
successful company town in the world, representing a new and short-lived
kind of corporate
idealism.}\label{for-a-time-ivrea-was-likely-the-most-progressive-and-successful-company-town-in-the-world-representing-a-new-and-short-lived-kind-of-corporate-idealism}}

Two major additions to the red brick building, built between 1939 and
1949, look like perfect representations of a moment in architectural
thought: The first is a long, low-slung block threaded with ribbon
windows; the second, known as Ico Centrale, is a fully glazed,
curtain-walled facade, shielded from the light by
\href{https://www.nytimes.com/2018/08/08/t-magazine/le-corbusier-japan-modernism.html}{Corbusier}-style
brises-soleil. A third building, also covered with a slick glass-skinned
facade, now houses a nursing school. The others are empty, filled with
the detritus of companies past, having only recently been acquired by a
developer who is attempting to secure contracts for new firms while
preserving the buildings. These are glorious, light-filled spaces,
unsung monuments to the rationalist, functionalist architecture that
dominated progressive thinking in the midcentury. On the southern side
of Ico Centrale, a perpendicular bend causes two portions of the
building to face and reflect each other --- the implicit idea being that
employees on either side would have the opportunity to see each other in
their daily work, and, even more implicitly, that the company was open
and transparent to itself and the world.

Olivetti also built housing and hotels, two of which are the most
strange and wonderful buildings in any city. The West Residential
Center, popularly known as the Talponia, is a crescent-shaped block
built into a hillside. Its roof is paved and walkable, its facade
entirely glass, articulated into rectangles by dark gray metal framing.
Originally intended for short business stays, it projects a spirit of
efficiency, with modular furniture and bedrooms separated only by
curtains. The Hotel La Serra, outside the main center, was built in the
1970s and betrays the influence of postmodernism. Composed of an
irregular series of stacked, graduated floors, it is meant to look like
a typewriter, but from the inside, the rooms feel like the tightly
constructed cabin of a ship, with oval porthole-like windows and a
secret armoire holding a vanity mirror, whose curved doors open
perfectly into the concave surrounding space.

It makes sense that Olivetti would be a symbol of historic pride. As a
principal player in the 20th-century ``miracle,'' when Italy climbed out
of the depths of fascism and the catastrophe of World War II to become
the eighth largest economy in the world, it is essential to Italian
identity. The nostalgia for this time in Ivrea can be intense.
\href{https://www.comune.ivrea.to.it/entra-in-comune/amministrazione-trasparente/organizzazione/articolazione-degli-uffici/item/stefano-sertoli.html}{Stefano
Sertoli}, the recently elected mayor, mentioned how often he came across
people with an incredibly precise recall for eras and moments in company
history. Some 1,900 residents of the city are recipients of the spille
d'oro, or ``gold pins,'' which represent 25 years of continuous service
to the company. The legacy of Olivetti is, he said, ``un patrimonio
pazzesco'' --- an insanely rich heritage.

Image

The exterior of the La Serra complex.Credit...Nick Ballón

IF AMERICAN COMPANY TOWNS were tied to private industry's desire to
quell progressive movements, in Italy, the company town was just as
influenced by the rise of fascism.
\href{https://intransit.blogs.nytimes.com/2008/06/13/sun-sand-and-mussolini/}{Sabaudia},
a coastal town near Rome, was created in 1933 as a result of orders from
\href{https://www.nytimes.com/topic/person/benito-mussolini}{Benito
Mussolini}, who transported the urban poor from Rome to the coast in
order to drain the surrounding malaria-infested marshlands. Monfalcone,
near the border of Slovenia, became a part of Italy only after World War
I but was soon converted into an important shipbuilding outpost by the
fascist regime. Many of these towns began a slow decline in business and
population following World War II, though it's no accident that the best
of Ivrea as imagined by Olivetti emerged as a postwar phenomenon, a
place in direct ideological opposition to Mussolini's government. If
most company towns, both in Europe and America, were paternalistic in
the extreme --- some of them going so far as to pay workers with
``company scrip,'' which could only be used at company-owned stores ---
this was not the intention in Ivrea, thanks in no small part to Adriano,
who combined in his person the grandiose impulses of a humanitarian, the
self-obsession of an entrepreneur and the sententiousness of a rich
autodidact. Inducted into an already successful company, he also,
crucially, had experience working in a factory. Unlike the mechanical
engineer
\href{https://archive.nytimes.com/www.nytimes.com/learning/general/onthisday/bday/0320.html}{Frederick
Winslow Taylor} from a generation earlier, who also came into factory
life from an elite background but drew the conclusion that work needed
to be rationalized within an inch of its life --- leading to his concept
of ``scientific management'' --- Adriano arrived at a factory and
experienced the full spectrum of alienation. He would later testify to
knowing ``the awful monotony and the weight of repeating actions ad
infinitum, on a drill or a press.'' His experience led him to the
realization that ``it was necessary to set man free from this degrading
slavery.'' Gastone Garziera, an engineer who had worked on computing and
electronics in the 1960s and '70s, recalled Adriano Olivetti's ``desire
to lighten in any way possible'' the burden of work.

Adriano returned to Italy to take up the mantle of the family firm; he
became president in 1938. He was committed to Modernism --- not just as
an architectural and aesthetic phenomenon but as a political program.
Though he joined the Fascist Party during the years of Mussolini, he
eventually sought to make contacts with Americans and supported the
resistance, for which he was arrested. Adriano makes an indelible
character in the Italian novelist
\href{https://www.nytimes.com/1991/10/09/books/natalia-ginzburg-75-novelist-essayist-and-translator-is-dead.html}{Natalia
Ginzburg}'s marvelous
``\href{https://www.nyrb.com/products/a_family_lexicon?variant=1094928849}{Family
Lexicon}'' (1963), a memoiristic novel of life in Turin during the two
World Wars:

\begin{quote}
He was fat and pale and his uniform fit badly over his round, fat
shoulders. I've never seen anyone wear that gray-green outfit with a
pistol at the waist more awkwardly and less martially than him. He had a
pronounced melancholic air about him, which was perhaps because he
didn't like being a soldier in the least. He was shy and quiet, but when
he did speak he talked for a long time in a low voice and said confusing
and enigmatic things while staring off into space with his small blue
eyes, at once cold and dreamy.
\end{quote}

This implicitly self-regarding personality expressed itself in the
spirit of the company he led and in the products they created. As with
the Bauhaus, the short-lived but highly influential German school of
design, there was an attempt to unify aesthetically the entire
production, from the products themselves to the advertisements for them,
but with a markedly stronger emphasis on rendering the work environment
itself a humane one. (The influence of the
\href{https://www.nytimes.com/2019/02/04/t-magazine/bauhaus-school-architecture-history.html}{Bauhaus}
was in some cases direct:
\href{https://www.nytimes.com/2015/03/19/arts/artsspecial/archive-of-xanti-schawinsky-bauhaus-artist-is-exhibited-in-zurich.html}{Alexander
``Xanti'' Schawinsky}, an alumnus of the Bauhaus, designed a new
typewriter for Olivetti, the Studio 42, and consulted on the
construction of the new headquarters.
\href{https://www.nytimes.com/1985/10/01/arts/herbert-bayer-85-a-designer-and-artist-of-bauhaus-school.html}{Herbert
Bayer}, one of his instructors, designed company advertisements.)
Olivetti wanted to build, as the modernist architecture critic Mario
Labó wrote, ``a place of work ruled by progress, guided by justice, and
fired by the light of beauty.'' Workers became part of the management of
the company through a system of co-determination, and thus helped build
the welfare institutions that catered to them.

Image

Housing from 1956 in the neighborhood called Canton Vesco, in central
Ivrea.Credit...Nick Ballón

Image

The entrance to another housing complex, Edificio 18, built in 1954 by
Marcello Nizzoli.Credit...Nick Ballón

In these years, Olivetti produced several of the products that brought
it world renown. The lightweight and (relatively) portable Lettera 22,
one of the most beautiful and functional machines ever made, became a
popular typewriter for business as well as private use. Its baby blue
coloration and the light, springy action of its rounded keys were part
of the transformation from a typewriter as a loud, mechanical object for
processing business to one that lent itself to contemplative, private
writing. (It was the favorite of many American writers, including
\href{https://www.nytimes.com/topic/person/thomas-pynchon}{Thomas
Pynchon},
\href{https://www.nytimes.com/topic/person/sylvia-plath}{Sylvia Plath},
\href{https://www.nytimes.com/topic/person/gore-vidal}{Gore Vidal}.) A
couple of decades later, in 1968, and with the help of the designer
\href{https://www.nytimes.com/2013/12/12/garden/rare-gems-from-ettore-sottsass.html}{Ettore
Sottsass Jr.}, Olivetti would produce the apotheosis of the
typewriter-for-pleasure, the
\href{https://www.metmuseum.org/art/collection/search/739409}{Valentine},
a lollipop of a machine, the high point of Pop Art in design.
Advertisements for the Valentine showed its users taking the typewriter
to the beach.

But by the '70s, people were moving from typewriters to electronic
devices, and though the company had created what is considered the first
personal computer, the P101, the company's success on this front had
stalled. Some observers attribute Olivetti's downfall less to company
failings than to nefarious plotting by foreign powers. Mario Tchou,
Olivetti's brilliant chief computer programmer, died in a car accident,
and Olivetti's last independent president,
\href{https://www.nytimes.com/1992/04/17/business/olivetti-s-chief-is-convicted-in-collapse-of-bank-in-1982.html}{Carlo
De Benedetti}, suggested that it was widely believed among Olivettians
that ``he had been killed by forces connected to American secret
agents.'' Garziera also vouched that the Americans were suspicious of
computing advances falling into the hands of a country that was
perpetually on the verge of Communism. And in 2019's
``\href{https://www.penguinrandomhouse.ca/books/544552/the-mysterious-affair-at-olivetti-by-meryle-secrest/9780451493651}{The
Mysterious Affair at Olivetti},'' the journalist
\href{https://www.penguinrandomhouse.com/authors/27553/meryle-secrest}{Meryle
Secrest} advances a circumstantial version of the same theory (without,
it must be admitted, confirming it). Whatever the reasons, the failure
to achieve results in computing doomed the company, and --- at least for
the near term --- Ivrea with it.

Image

The former Sertec building, designed by Ezio Sgrelli and built in 1968,
which held Olivetti's engineering offices.Credit...Nick Ballón

HOW DOES A company town reinvent itself once the company leaves town? In
some respects, Ivrea reflects broader trends in Italy, rather than
circumstances unique to itself. Changes in technology may have made
Olivetti obsolete, but the Italian economic miracle of the 1950s and
'60s peaked around 1970 anyway, when Fiat's production headquarters, in
nearby Turin, became one of the largest car factories in Europe (and
during which time Olivetti was still one of the most successful
manufacturers of typewriters and other business machines in the world).
This growth was helped by a mass migration of workers from the country's
impoverished south to the heavily industrialized northwest. But as the
'70s turned into the '80s, Turin, Ivrea and other cities and towns that
had grown rapidly after World War II fell victim to the same economic
trends that would stunt the growth of American manufacturing towns
across the Rust Belt: Recurring recessions meant that costs were cut
across all industries, labor was outsourced to cheaper countries and
companies like Fiat and Olivetti began
\href{https://www.nytimes.com/1982/04/02/business/fiat-layoffs-set.html}{laying
off} thousands of workers, plunging the very concept of the company town
into an existential crisis. There are, according to a 2016 Italian
environmental association report, some 2,500 rural Italian towns that
are
\href{https://www.nytimes.com/2017/09/07/t-magazine/abandoned-italian-towns.html}{nearly
abandoned and depopulated}, half-empty monuments to departed industry.
Others, like Ivrea, are more of a nostalgic time capsule, less a ruin
than a shell of the past trying to find ways to bring back their old
glory.

As major companies shrank in size or merged with larger corporations
(Fiat now
\href{https://www.nytimes.com/topic/company/fiat-chrysler-automobiles-nv}{owns}
Chrysler), their corporate paternalism faded from view, replaced by more
immediate economic concerns. This would end Olivetti's well-intentioned
experiment in humane labor. Now, as is the case in so many small
municipalities in Italy and elsewhere in the world, Ivrea has
experienced an alarming turn in its politics. After decades of
center-left rule --- including a stint by Adriano himself as mayor ---
last year the leadership shifted to the right, with a
\href{https://www.nytimes.com/2018/03/04/world/europe/italy-election.html}{new
government affiliated with the anti-immigrant party La Lega}. I spoke
with the new mayor, Sertoli, who had been part of the effort to secure
UNESCO recognition for the city. He talked vaguely of the need to
``bring back excellence'' to the city, but also noted it was problematic
that so many of Olivetti's structures were in various private hands. The
current holder of the Brick Factory building is
\href{https://www.icona.srl/?lang=en}{Icona}, a coalition attempting to
redevelop the original Olivetti buildings in the hope of returning
industry and innovation to the area. Icona's slogan is ``The Future Is
Back Home.'' The atrium connecting the Brick Factory to the others still
has a mosaic tile statue of Camillo Olivetti.

Image

The West Residential Center, popularly known as the
Talponia.Credit...Nick Ballón

Image

The interior of the Hotel La Serra, in Ivrea, an Italian company town
run in its prime by the Olivetti typewriter manufacturer.Credit...Nick
Ballón

Other efforts at reviving Ivrea don't take their cues from Olivetti at
all. Gianmario Pilo, a book marketer in Turin whose father worked at the
company for 35 years, has started a literary festival,
\href{http://www.lagrandeinvasione.it/}{La Grande Invasione}, with the
aim of jump-starting the cultural life of the town and encouraging
younger residents to stay. He spoke about how his parents were always
passionate readers, partly because of the company's efforts to inculcate
culture in its workers' lives.

The extraordinary achievement of Olivetti is also part of what
overwhelms and partly vitiates the lives that have come after it. The
afterglow that still hovers over Ivrea is that of young Adriano in
``Family Lexicon'': dreamy, speaking at once to everyone and no one,
quietly saying ``enigmatic things.'' It may be the best example in
history of a city organized around a single company and its vision, in
which some profits were reinvested into the life of the company's
workers and the surrounding community. At its best, the spirit of
reinvestment can be given back, if perhaps never again at the level it
once had. When I asked Pilo why he pursued his desires in Ivrea, he said
simply that it was to ``give back to the city that had gifted me a happy
childhood and adolescence.'' The children of Olivetti may yet restore
the ideas that still whistle down the quiet streets of the town it once
dominated.

Advertisement

\protect\hyperlink{after-bottom}{Continue reading the main story}

\hypertarget{site-index}{%
\subsection{Site Index}\label{site-index}}

\hypertarget{site-information-navigation}{%
\subsection{Site Information
Navigation}\label{site-information-navigation}}

\begin{itemize}
\tightlist
\item
  \href{https://help.nytimes.com/hc/en-us/articles/115014792127-Copyright-notice}{©~2020~The
  New York Times Company}
\end{itemize}

\begin{itemize}
\tightlist
\item
  \href{https://www.nytco.com/}{NYTCo}
\item
  \href{https://help.nytimes.com/hc/en-us/articles/115015385887-Contact-Us}{Contact
  Us}
\item
  \href{https://www.nytco.com/careers/}{Work with us}
\item
  \href{https://nytmediakit.com/}{Advertise}
\item
  \href{http://www.tbrandstudio.com/}{T Brand Studio}
\item
  \href{https://www.nytimes.com/privacy/cookie-policy\#how-do-i-manage-trackers}{Your
  Ad Choices}
\item
  \href{https://www.nytimes.com/privacy}{Privacy}
\item
  \href{https://help.nytimes.com/hc/en-us/articles/115014893428-Terms-of-service}{Terms
  of Service}
\item
  \href{https://help.nytimes.com/hc/en-us/articles/115014893968-Terms-of-sale}{Terms
  of Sale}
\item
  \href{https://spiderbites.nytimes.com}{Site Map}
\item
  \href{https://help.nytimes.com/hc/en-us}{Help}
\item
  \href{https://www.nytimes.com/subscription?campaignId=37WXW}{Subscriptions}
\end{itemize}
