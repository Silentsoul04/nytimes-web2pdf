\href{/section/arts/design}{Art \& Design}\textbar{}`It's About Time!'
Betye Saar's Long Climb to the Summit

\href{https://nyti.ms/30UDBID}{https://nyti.ms/30UDBID}

\begin{itemize}
\item
\item
\item
\item
\item
\item
\end{itemize}

\includegraphics{https://static01.nyt.com/images/2019/09/15/arts/15new-saar2/merlin_159749076_6fc75e92-f046-4f4b-817d-d6240e23b337-articleLarge.jpg?quality=75\&auto=webp\&disable=upscale}

Sections

\protect\hyperlink{site-content}{Skip to
content}\protect\hyperlink{site-index}{Skip to site index}

Fall Preview

\hypertarget{its-about-time-betye-saars-long-climb-to-the-summit}{%
\section{`It's About Time!' Betye Saar's Long Climb to the
Summit}\label{its-about-time-betye-saars-long-climb-to-the-summit}}

At 93, with major attention finally coming her way, an artist central to
the black women's revolution says she's waited long enough.

Saar in her Los Angeles garden.Credit...Erik Carter for The New York
Times

Supported by

\protect\hyperlink{after-sponsor}{Continue reading the main story}

\href{https://www.nytimes.com/by/holland-cotter}{\includegraphics{https://static01.nyt.com/images/2018/02/16/multimedia/author-holland-cotter/author-holland-cotter-thumbLarge.jpg}}

By \href{https://www.nytimes.com/by/holland-cotter}{Holland Cotter}

\begin{itemize}
\item
  Published Sept. 4, 2019Updated Sept. 13, 2019
\item
  \begin{itemize}
  \item
  \item
  \item
  \item
  \item
  \item
  \end{itemize}
\end{itemize}

LOS ANGELES --- I ask the artist \href{http://www.betyesaar.net/}{Betye
Saar}, who is 93 and set to open concurrent solo shows this fall at two
major museums --- the
\href{https://www.moma.org/artists/5102?gclid=EAIaIQobChMIvY62loKT5AIVUQOGCh06tQ5iEAAYASAAEgJRffD_BwE\&gclsrc=aw.ds}{Museum
of Modern Art} in New York and the
\href{https://www.lacma.org/art/exhibition/betye-saar-call-and-response}{Los
Angeles County Museum of Art} --- if she has any theories as to why
big-ticket attention is finally coming her way. She skips mentioning the
obvious factors: She's a woman; she's black; she's lived her whole life
on what she calls ``the other side of the planet'' (Southern
California). ``Because it's about time!'' she says. ``I've had to wait
till I'm practically 100.''

We're standing in her home, which is also her studio, in the Laurel
Canyon section of Los Angeles. She has lived and worked here since 1962,
when the neighborhood was becoming a New Agey arts enclave. The house,
stacked vertically up the side of a ravine, is all stairs and
platformlike rooms with a small garden nestled within. The division
between domestic and work space feels indeterminate. Order prevails but
clear surfaces are hard to find.

For half a century, Ms. Saar has been one of the country's most
inventive and influential makers of intimately scaled assemblage. And
she has brought a distinctive range of content to the medium,
encompassing global culture, popular mysticism, personal history and
American racism, which she coolly refers to as ``national racism,'' as
if it were a scientific category, or a consumer brand.

\includegraphics{https://static01.nyt.com/images/2019/09/15/arts/15new-saar11/merlin_159300561_0559a200-cd38-4a5b-87b7-f0057e70ad91-articleLarge.jpg?quality=75\&auto=webp\&disable=upscale}

Over a long career she has, against serious odds, maintained visibility.
After a latish start as an artist --- she was in her 30s --- she made
steady initial headway in a male-dominated Black Power movement and a
white-dominated feminist movement. And she has held her own in a
mainstream art market that has been, until very recently, unwelcoming to
African-American art. During her time on the scene, political grounds
have shifted and art tastes have changed. But her work has retained
pertinence and complexity.

A glance around her studio reveals some of the complexity. The place is
packed chockablock with clusters of objects grouped by type: alarm
clocks (maybe two dozen), antique books, model clipper ships, African
masks, birdcages, globes, painted wood watermelon slices, the Mexican
healing charms known as milagros and so-called mammy dolls piled on a
chair.

Image

Ms. Saar in her studio in Laurel Canyon.Credit...Erik Carter for The New
York Times

There are also arrangements by color. All-red or mostly red items
include an angel's head, an hourglass, a Tuxedo Club pomade can, a small
plastic menorah and a nosegay of crimson bird feathers. (``You can't
beat Nature for color,'' Ms. Saar says. ``She's got it down.'') And
there are black-and-brown things: a plastic skull, an Afro pick, a dried
starfish and a paper crow.

Almost all was gathered on salvage campaigns to antiques shops, swap
meets, flea markets and rubbish bins, local and international. ``I
consider myself a recycler,'' Ms. Saar says. (She's also called herself
a ``junkie.'') ``I've been that way since I was a kid, going through
trash to see what people left behind. Good stuff.''

Scavenging is how she gets the raw material for her art, notably for the
three-dimensional assemblages for which she's best known. Thematically,
they are of three basic kinds. Some are curated time capsules preserving
materials that were once owned by, or are reminders of, generations of
women in Ms. Saar's family.

Image

A Saar assemblage-in-progress on the subject of slavery includes an
archival picture of an African-American man seen from behind, his bare
back scarred from whippings.Credit...Erik Carter for The New York Times

Image

Always foraging for objects and ideas, the artist groups artifacts by
color. Slices of watermelon are wood.Credit...Erik Carter for The New
York Times

Image

Ms. Saar displays like items together, including this large grouping of
clocks, in her home.Credit...Erik Carter for The New York Times

Others are symbolic self-portraits coded with personal and cosmological
references. In the 1969
\href{https://www.moma.org/audio/playlist/26/539}{``Black Girl's
Window''} which will anchor the MoMA exhibition, she surrounds a
silhouette of her head with floating moons and stars; an etching (her
own) of a lion, her birth sign; a tintype of a woman who could be her
Irish grandmother; and, at the center, a novelty shop Halloween skeleton
alluding to her father's death when she was a child, a loss she says she
still lives with.

And there's work that's forthrightly political. A famous, indeed
career-defining example,
``\href{http://revolution.berkeley.edu/liberation-aunt-jemima/}{The
Liberation of Aunt Jemima''} from 1972, was her response to the killing
of the Rev. Dr. Martin Luther King Jr. and the
\href{https://www.nytimes.com/2015/08/11/us/50-years-after-watts-riots-a-recovery-is-in-progress.html}{Watts
riots}. The main image here is a store-bought relief of a Jim Crow-era
plastic mammy meant to hold a kitchen notepad and pencil. Ms. Saar
transformed it, replacing the pad with a Black Power fist and putting a
rifle in the figure's hand.

The piece was an instant sensation in black and feminist art circles.
(\href{https://www.history.com/topics/black-history/angela-davis}{Angela
Davis} has since described it as the spark that fired the black women's
revolution.) ``I was lucky to have an icon,'' Ms. Saar says about the
work that put her on the national map. Yet over the years, the celebrity
of the piece has created an unbalanced view of a long career that has
taken many forms and directions.

Image

``The Liberation of Aunt Jemima''(1972), Ms. Saar's response to the
assassination of Martin Luther King Jr., and the Watts riots,
transformed a plastic mammy into a symbol of black power.Credit...Betye
Saar, via Roberts Projects, Los Angeles, Berkeley Art Museum and Pacific
Film Archive, Berkeley, California

Ms. Saar was born, the oldest of three children, in the Watts
neighborhood of Los Angeles in 1926. Her middle-class family was of
mixed African-American, Irish and Native American descent. In a poem Ms.
Saar has written:

\emph{My roots are tangled....}

\emph{A blend of black, white and red,}

\emph{I am labeled Creole, mulatto, mixed, colored in every sense.}

\emph{Enslaved by the `one-drop-rule'}

\emph{But liberated by the truth}

\emph{That all blood is red.}

A lifelong attraction to spirituality started early. Religion was woven
into her upbringing. ``My mother was Episcopalian. My dad was a
Methodist Sunday school teacher. When he died, my mom was pretty
distraught, and we joined the Christian Science church. And being in
California, we knew, of course, about Buddhism.'' She also counts her
childhood love of fairy tales as part of the mystical mix. ``It was all
magic to me.''

Image

The artist in her studio in 1970, with ``Black Girl's
Window.''Credit...Bob Nakamura, via Roberts Projects, Los Angeles

When she was 4, her family moved to Pasadena, Ms. Saar says. ``In
Pasadena my mother always had a garden. You need nature somehow in your
life to make you feel real. The bottom line in politics is: one planet,
one people. And we are so far from that now.''

After her father's death in 1931, when she was 5, Ms. Saar was nurtured
by a great-aunt, Hattie Parsons Haynes Keyes, a long-lived mentor and
model of self-shaping in a matriarchal line. As a child, Ms. Saar also
returned to Watts for visits. There, watching the Italian immigrant
\href{https://tclf.org/pioneer/sabato-simon-rodia}{}
\href{https://tclf.org/pioneer/sabato-simon-rodia}{Simon Rodia} build
his fantastical towers from scrap materials --- mirrors, seashells,
broken titles --- embedded in cement, she learned a lasting lesson:
``You can make art out of \emph{anything}.''

In college she studied art, an unpromising option for a black woman at
the time. She ended up doing social work, then moved into the design
field. In 1952, she met and married (and later divorced) the ceramist
Richard Saar. They had three daughters,
\href{https://www.amazon.com/Family-Legacies-Betye-Lezley-Alison/dp/029598564X}{Alison
and Lezley,} both now artists, and Trayce, a writer. Toward the end of
the decade, Ms. Saar went back to school for a degree to teach design.
Destiny, in which she believes, had other plans. ``One day I wandered
into a printmaking workshop,'' she says, ``and forgot about teaching.''
She joined and became an artist.

Her MoMA solo,
\href{https://www.moma.org/calendar/exhibitions/5060}{``Betye Saar: The
Legends of `Black Girl's Window,'''} which will debut with the reopening
of the newly expanded museum on Oct. 21, is a survey of her rare, early
works on paper --- 42 of which MoMA recently acquired --- supplemented
by a selection of her assemblages.

When asked why MoMA had come so late to collecting her work,
\href{https://www.moma.org/about/senior-staff/ann-temkin}{Ann Temkin},
the museum's chief curator of painting and sculpture, said, ``For the
most part (and with notable exceptions) until this past decade we were
not looking in the directions where we would have found Saar's work. And
speaking personally,'' she added, ``for that reason now is such an
inspiring and rewarding time to happen to be a curator.''

Image

The installation of Ms. Saar's solo show, which will open the renovated
Museum of Modern Art, includes early works on paper --- 42 of which MoMA
recently acquired. ``Until this past decade we were not looking in the
directions where we would have found Saar's work,'' said Ann Temkin,
chief curator of painting and sculpture.Credit...Brad Ogbonna for The
New York Times

The variety and virtuosity of Ms. Saar's prints are impressive. Working
at home to be with her young children, she was clearly doing some
adventurous self-teaching. ``I was never a pure printmaker,'' she says.
``I fooled around with all kinds of techniques.''

By the mid-60s, pushing hard against the conventional boundaries of the
medium, she came up with new display formats. She began placing
different prints, sometimes with drawings and photographs, in thick wood
frames made from repurposed window sashes. ``Black Girl's Window'' is an
example. It's as much about sculpture as printmaking. It's a
self-portrait as an altarpiece.

In 1967, she saw an exhibition of Joseph Cornell's boxed assemblages and
with that experience she turned the corner. She started making
assemblages of her own.

Her learning continued to expand. A 1970 visit to the Field Museum of
Natural History in Chicago, in the company of a fellow Los Angeles
artist, \href{https://www.moma.org/artists/2486}{David Hammons},
introduced her to the charisma of African and Oceanic art ---
ritual-intensive, spiritually empowered. It also delivered a lesson in
cultural politics: Most of this ``primitive'' art was installed, as if
in storage, in the museum's basement. Nobody was looking at it.

``Back then,'' she says, ``even black Americans were sort of ashamed of
African art.'' But what others rejected, she embraced: the art's use of
organic matter --- feathers, skins, dirt, hair --- and its empowering
function. Her enthusiasm, which infused her art, had an impact.

``One of the things that gave her work importance for African-American
artists, especially in the mid-70s, was the way it embraced the mystical
and ritualistic aspects of African art and culture,'' says the painter
\href{https://www.jackshainman.com/artists/kerryjames-marshall/}{Kerry
James Marshall}, who took a collage course with Ms. Saar at
\href{https://www.otis.edu/}{Otis College of Art and Design} in the late
1970s. ``Her art really embodied the longing for a connection to
ancestral legacies and alternative belief systems --- specifically
African belief systems --- fueling the Black Arts Movement.''

She started traveling --- to Bali, Brazil, Haiti, Mexico, Morocco,
Nigeria, Senegal --- always foraging for objects and images, and
particularly attracted to those with devotional associations. ``Wherever
I went, I'd go to religious stores to see what they had,'' she says.

Everywhere she went, she carried small sketchbooks that did double duty
as memory banks and portable studios. Many are repositories for quick,
preliminary ballpoint pen studies for future assemblages. Others are
filled with watercolor paintings that are polished creations in
themselves. The show at the Los Angeles County Museum of Art,
\href{https://www.lacma.org/art/exhibition/betye-saar-call-and-response}{``Betye
Saar: Call and Response,''} which opens on Sept. 22, will reunite
several sketchbooks with related finished works.

Image

``Sketchbook 1998,'' at the Los Angeles County Museum of
Art.Credit...Betye Saar, via Roberts Projects, Los Angeles

Image

And ``Supreme Quality'' (1998), the assemblage based on the
sketch.Credit...Betye Saar, via The Rose Art Museum, Brandeis
University; Tim Lanterman/Scottsdale Museum of Contemporary Art

Ms. Saar's responses to American racial and gender politics have grown
increasingly complicated over time. In 1974, she and another artist,
\href{https://hammer.ucla.edu/now-dig-this/artists/samella-lewis/}{Samella
Lewis}, organized a group show of black women artists for Womanspace, a
pioneering cultural center in Los Angeles. (Ms. Saar was on its founding
board.) She was shocked to find the audience split along racial lines.
Blacks came; whites didn't. ``The white women did not support it,'' she
told the art historian Ruth Askey in 1981. ``It was as if we were
invisible.''

She was later embroiled in a divisive art world conflict around the use
of derogatory racial images. In 1997, she spearheaded a letter-writing
campaign directed at a young African-American colleague, Kara Walker,
who, at 28, had won a MacArthur ``genius'' grant.
\href{https://editions.lib.umn.edu/panorama/article/kara-walkers-about-the-title-the-ghostly-presence-of-transgenerational-trauma-as-a-connective-tissue-between-the-past-and-present/}{Ms.
Walker's silhouette tableaus of antebellum slavery followed} Ms. Saar's
lead in mining racial caricatures, but within them created morally
ambiguous narratives in which everyone, black and white, slave and
master, was implicated as corrupt. On the subjects of slavery and white
supremacy, Ms. Saar had little patience with ambiguity. Ms. Walker's
art, she concluded in a 1999 television interview, had been made ``for
the amusement and the investment of the white art establishment.''

Soon afterward, she herself returned to the subject of racism. Much of
the work in her 2017 solo show, ``Betye Saar: Keepin' it Clean,'' at the
Craft and Folk Art Museum in Los Angeles (and later at
\href{https://www.nyhistory.org/exhibitions/betye-saar-keepin\%E2\%80\%99-it-clean}{the
New-York Historical Society}) featured gun-toting mammies. ``I keep
thinking of giving up political subjects,'' she says, ``But you can't.
Because racism is still here. Worse than ever.''

On my visit to her studio, there is an assemblage-in-progress on this
subject. In August, its components included two striking photographs,
one of a slender young African woman, almost nude, playing a musical
instrument; the other, a well-known 1863 archival picture of an
African-American man seen from behind, his bare back scarred from
whippings. Ms. Saar positioned the pictures on either side of a
scuffed-up child-size piano with missing keys, added an 18th-century
diagram of a ship filled with slaves and capped the ensemble with an
antique clock.

``It's about slavery, before and after,'' she says. ``I call it `Skin
Song.''' She later removed the photograph of the African woman.
Improvisation has always been her modus operandi.

Mr. Marshall remembers that ``in her class, we made a collage for the
first critique. We were then told to bring the same collage back the
next week, but with changes, and we kept changing the collage over and
over and over, throughout the semester. From that I got the very useful
idea that you should never let your work become so precious that you
couldn't change it.''

A second assemblage in her studio is very much in the planning stages.
Its encasing frame is set: a small, light rectangular box painted gray.
But its components still lie unfixed on a worktable: a small metal
heart; a tin healing charm in the shape of eyes; a skeletal antique fan;
an abstract print by Ms. Saar suggesting a cloudy sky (``Of all natural
things, the sky is my favorite,'' the artist says). She had just added a
tiny computer board glinting like a filigreed brooch.

Image

Ms. Saar at the Museum of Modern Art with ``Anticipation'' (1961), a
screenprint in her solo show. It is a self-portrait of the artist, who
was pregnant with her youngest daughter Tracye at the
time.Credit...Betye Saar, via Roberts Projects, Los Angeles; Brad
Ogbonna for The New York Times

The format and contents suggest a reliquary. It's dedicated to Ms.
Saar's great-aunt Hattie, her early role model. When Hattie died in the
early 1970s, the artist inherited a trunk filled with personal effects:
letters, family photographs, handkerchiefs, trinkets. Over the years,
she's been preserving them in her work, inspired by the spiritual
belief, shared by many cultures, that the dead live on in what they've
touched and treasured.

``I think the chanciest thing is to put spirituality in art,'' Ms. Saar
says as she gently shifts elements of the assemblage around, trying this
combination and that. ``Because people don't understand it. Writers
don't know what to do with it. They're scared of it, so they ignore it.
But if there's going to be any universal consciousness-raising, you have
to deal with it, even though people will ridicule you.''

``And you have to deal with personal emotions, because they're there,''
she added. ``I think people are afraid of those too. My younger sister's
husband died this year. I said to her, you've got to start making
something beautiful. Beauty is a form of spirituality. Once you start
making something with your hands, the healing starts. I call this
creative grieving.''

Not that grieving, creative or otherwise, is necessarily uppermost in
Ms. Saar's mind. She has a scheduling crunch to deal with. There's work
to finish and send out. Coming up fast is a date to fly to New York to
oversee the installation of the MoMA show. (``For the first time ever''
--- at her request --- ``the museum will have purple walls,'' she
confides.) And there are longer-range plans.

As we move from the workroom into the exuberantly planted garden for a
sit-down, she says: ``I'm 93, and going for 94. Aunt Harriet lived to
95. I want to beat her record.''

Advertisement

\protect\hyperlink{after-bottom}{Continue reading the main story}

\hypertarget{site-index}{%
\subsection{Site Index}\label{site-index}}

\hypertarget{site-information-navigation}{%
\subsection{Site Information
Navigation}\label{site-information-navigation}}

\begin{itemize}
\tightlist
\item
  \href{https://help.nytimes.com/hc/en-us/articles/115014792127-Copyright-notice}{©~2020~The
  New York Times Company}
\end{itemize}

\begin{itemize}
\tightlist
\item
  \href{https://www.nytco.com/}{NYTCo}
\item
  \href{https://help.nytimes.com/hc/en-us/articles/115015385887-Contact-Us}{Contact
  Us}
\item
  \href{https://www.nytco.com/careers/}{Work with us}
\item
  \href{https://nytmediakit.com/}{Advertise}
\item
  \href{http://www.tbrandstudio.com/}{T Brand Studio}
\item
  \href{https://www.nytimes.com/privacy/cookie-policy\#how-do-i-manage-trackers}{Your
  Ad Choices}
\item
  \href{https://www.nytimes.com/privacy}{Privacy}
\item
  \href{https://help.nytimes.com/hc/en-us/articles/115014893428-Terms-of-service}{Terms
  of Service}
\item
  \href{https://help.nytimes.com/hc/en-us/articles/115014893968-Terms-of-sale}{Terms
  of Sale}
\item
  \href{https://spiderbites.nytimes.com}{Site Map}
\item
  \href{https://help.nytimes.com/hc/en-us}{Help}
\item
  \href{https://www.nytimes.com/subscription?campaignId=37WXW}{Subscriptions}
\end{itemize}
