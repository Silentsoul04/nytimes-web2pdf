Sections

SEARCH

\protect\hyperlink{site-content}{Skip to
content}\protect\hyperlink{site-index}{Skip to site index}

\href{https://www.nytimes.com/section/world/africa}{Africa}

\href{https://myaccount.nytimes.com/auth/login?response_type=cookie\&client_id=vi}{}

\href{https://www.nytimes.com/section/todayspaper}{Today's Paper}

\href{/section/world/africa}{Africa}\textbar{}Mugabe's Reign Began With
Bob Marley and Good Schools. Despotism Soon Followed.

\url{https://nyti.ms/2ZIWMrH}

\begin{itemize}
\item
\item
\item
\item
\item
\end{itemize}

Advertisement

\protect\hyperlink{after-top}{Continue reading the main story}

Supported by

\protect\hyperlink{after-sponsor}{Continue reading the main story}

Recollections

\hypertarget{mugabes-reign-began-with-bob-marley-and-good-schools-despotism-soon-followed}{%
\section{Mugabe's Reign Began With Bob Marley and Good Schools.
Despotism Soon
Followed.}\label{mugabes-reign-began-with-bob-marley-and-good-schools-despotism-soon-followed}}

\includegraphics{https://static01.nyt.com/images/2019/09/08/world/08mugabe-essay-1/merlin_160266945_90576b35-6f01-408b-b805-b9c1f8f9dc11-articleLarge.jpg?quality=75\&auto=webp\&disable=upscale}

By \href{https://www.nytimes.com/by/alan-cowell}{Alan Cowell}

\begin{itemize}
\item
  Sept. 6, 2019
\item
  \begin{itemize}
  \item
  \item
  \item
  \item
  \item
  \end{itemize}
\end{itemize}

It was April 1980, and I was among the crowd serenaded by Bob Marley and
the Wailers when Britain's Union flag was lowered to be replaced by a
new banner. A whiff of tear gas floated over a wall at the Rufaro soccer
stadium in Harare, the capital of the nascent state of Zimbabwe.

Prince Charles was on hand to formally give up control of his country's
last African colony. A slender, bespectacled man who would become one of
Africa's most notorious despots was about to secure a prize he had long
coveted.

At midnight, Zimbabwe --- the former Rhodesia --- became Africa's newest
independent state, and Robert Mugabe, the clear winner of
internationally supervised elections just weeks earlier, was its first
prime minister.

It was a crowning moment. A seven-year war had ended in victory for the
nationalist guerrillas, and Mr. Mugabe was about to start on a
trajectory that led from democratic roots to an inexorable gathering of
power unto himself.

And until his
\href{https://www.nytimes.com/2017/11/21/world/africa/zimbabwe-mugabe-mnangagwa.html}{ouster
in 2017} and
\href{https://www.nytimes.com/2019/09/06/obituaries/robert-mugabe-dead.html}{his
death,} announced on Friday, the tantalizing riddle of the Mugabe regime
was the question of what had turned a hero of Africa's liberation ---
and self-proclaimed champion of universal suffrage --- into a despot.

\emph{{[}Robert Mugabe, a liberation hero,}
\href{https://www.nytimes.com/2019/09/06/obituaries/robert-mugabe-dead.html}{\emph{became
a strongman}} \emph{who once proclaimed ``Zimbabwe is mine.''{]}}

As a young Reuters correspondent, I covered some of the final years of
the war from neighboring Zambia and from Zimbabwe itself. Later I was a
frequent visitor to Zimbabwe on behalf of The New York Times.

This was the pre-internet era. During a cease-fire that preceded the
festivities in Rufaro stadium, I had sent my reports from remote
bushlands
\href{https://www.nytimes.com/2019/09/06/reader-center/robert-mugabe-zimbabwe-independence.html}{by
carrier pigeon}. At the independence ceremony, I telephoned confirmation
of the flag-raising by hand-crank telephone to a Reuters bureau, where
the news was passed on by a telex machine.

From today's vantage, it could be argued the war that brought Mr. Mugabe
to power in Zimbabwe was the product of an equally bygone era, framed
variously by the struggle against colonialism, the rivalries of the Cold
War and the unbending obduracy of white minority rulers.

\includegraphics{https://static01.nyt.com/images/2019/09/06/world/06mugabe-essay-2/06mugabe-essay-2-articleLarge.jpg?quality=75\&auto=webp\&disable=upscale}

In the late 1970s, the rickety passenger airplanes that landed in
Salisbury, as Harare was then known, did so in a steep and gut-wrenching
spiral to avoid antiaircraft missiles. The gallows humorists of the day,
mimicking flight attendants, told travelers to turn back their watches
to the 1950s.

That decade had been a golden age for white settlers drawn from the
deprivations of postwar Britain to the sunlit uplands of a distant
colonial outpost. For whites, even in war, Rhodesia was caught in a time
warp of country clubs and drinks at sunset on shaded terraces scented by
bougainvillea and jacaranda.

By contrast, the black majority endured segregation in urban townships
and in rural reserves, denied access to the most fertile land and often
serving whites in menial roles as housekeepers, gardeners, laborers and
farmhands.

Those two bitterly divided worlds spawned a conflict from 1972 to 1979
marked by brutal tactics on both sides, with the fighting spilling into
the neighboring states that harbored the nationalist guerrillas.

Mr. Mugabe, who sought to overturn the majority's racially defined
status as third-class citizens in the country of their birth, drew
inspiration from Mao's doctrine of liberation through the barrel of the
gun. The prevailing political orthodoxy he embraced favored one-party
states, not democracy. Once secured, power was rarely given up
voluntarily.

\emph{{[}\href{https://www.nytimes.com/2019/09/06/world/africa/mugabe-death-zimbabwe.html?module=inline}{As
Zimbabweans learned of the death of their former leader},}
\href{https://www.nytimes.com/2019/09/06/world/africa/mugabe-death-zimbabwe.html?module=inline}{\emph{who
held the country in his grip for decades, the reaction was
muted.}}\emph{{]}}

With the announcement of Mr. Mugabe's death at the age of 95, it struck
me that he, too, had been caught in an era --- the liberation era ---
that has been overtaken by newer times. He was unable to shake off the
recourse to violent ways, a path embraced by many of his contemporaries,
as the legitimate counter to violent oppression. Maybe he, too, was
trapped in a time warp of his own making.

That is probably too charitable an interpretation of the increasing
ferocity of Mr. Mugabe's intolerance of dissent as he tightened his grip
on the reins of power in independent Zimbabwe, promoting himself from
prime minister to executive president, sidelining political rivals and
unleashing military force on civilians.

Yet for all of the trappings of power he had, it was sometimes tempting
to think that he was not at ease --- and always suspicious of those
around him --- for good reason.

Image

Emmerson Mnangagwa, left, with Robert Mugabe in 2014. Mr. Mnangagwa
succeeded his mentor as president in 2017.Credit...Tsvangirayi
Mukwazhi/Associated Press

It was no coincidence that he was replaced by one of his closest aides,
\href{https://www.nytimes.com/2019/08/10/world/africa/zimbabwe-president-emmerson-mnangagwa-mugabe.html}{Emmerson
Mnangagwa}.

At the start of his rule, there were signals of a possibly different
path. While Mr. Mugabe had reason enough for bitterness toward the white
authorities, who had imprisoned and reviled him, he offered
reconciliation with the minority and a new start for the majority in his
first broadcast to the nation as prime minister.

In the 1980s, he devoted much energy to an expansion of secondary
education that made Zimbabweans some of the best-schooled in southern
Africa.

In the early 1980s, in Harare, a resident might have been forgiven for
thinking the peace settlement ultimately reached was one of Africa's
greatest ever diplomatic transformations.

But there was always a duality to him. I first met Mr. Mugabe in the
mid-1970s as I traveled between guerrilla headquarters in Zambia and
Mozambique and on to peace conferences in Geneva, Malta and London,
where he came under immense pressure from his African supporters to
agree to the negotiated settlement.

In private moments, unnoticed by him, I had seen him rail against his
aides as he sought to cement his authority among the increasingly
powerful commanders of the guerrillas who fought white rule in his name.
(One such commander, Josiah Tongogara, died in a car crash in Mozambique
in late December 1979, provoking never-resolved suspicions of foul
play.)

But in that same era, former guerrilla forces fought murderous pitched
battles with government troops, clashes that offered a prelude to the
massacre of civilians by Mr. Mugabe's North Korean-trained Fifth Brigade
in 1982.

A decade after independence, the black majority's hunger for land drove
a mass takeover of white-owned farms, encouraged by Mr. Mugabe.

Mr. Mugabe's rule offered conflicting narratives of violence and a
fragile inclusiveness, a powerful impulse to dictatorship barely cloaked
by lip service to democracy.

In the end, muscle always won out over moderation. Even at the best of
times, Mr. Mugabe was a reluctant peacemaker, as his enemies --- real or
imagined --- discovered to their cost.

Advertisement

\protect\hyperlink{after-bottom}{Continue reading the main story}

\hypertarget{site-index}{%
\subsection{Site Index}\label{site-index}}

\hypertarget{site-information-navigation}{%
\subsection{Site Information
Navigation}\label{site-information-navigation}}

\begin{itemize}
\tightlist
\item
  \href{https://help.nytimes.com/hc/en-us/articles/115014792127-Copyright-notice}{©~2020~The
  New York Times Company}
\end{itemize}

\begin{itemize}
\tightlist
\item
  \href{https://www.nytco.com/}{NYTCo}
\item
  \href{https://help.nytimes.com/hc/en-us/articles/115015385887-Contact-Us}{Contact
  Us}
\item
  \href{https://www.nytco.com/careers/}{Work with us}
\item
  \href{https://nytmediakit.com/}{Advertise}
\item
  \href{http://www.tbrandstudio.com/}{T Brand Studio}
\item
  \href{https://www.nytimes.com/privacy/cookie-policy\#how-do-i-manage-trackers}{Your
  Ad Choices}
\item
  \href{https://www.nytimes.com/privacy}{Privacy}
\item
  \href{https://help.nytimes.com/hc/en-us/articles/115014893428-Terms-of-service}{Terms
  of Service}
\item
  \href{https://help.nytimes.com/hc/en-us/articles/115014893968-Terms-of-sale}{Terms
  of Sale}
\item
  \href{https://spiderbites.nytimes.com}{Site Map}
\item
  \href{https://help.nytimes.com/hc/en-us}{Help}
\item
  \href{https://www.nytimes.com/subscription?campaignId=37WXW}{Subscriptions}
\end{itemize}
