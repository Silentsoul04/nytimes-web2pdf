Sections

SEARCH

\protect\hyperlink{site-content}{Skip to
content}\protect\hyperlink{site-index}{Skip to site index}

\href{https://www.nytimes.com/section/world/asia}{Asia Pacific}

\href{https://myaccount.nytimes.com/auth/login?response_type=cookie\&client_id=vi}{}

\href{https://www.nytimes.com/section/todayspaper}{Today's Paper}

\href{/section/world/asia}{Asia Pacific}\textbar{}In Blow to Taiwan,
Solomon Islands Is Said to Switch Relations to China

\url{https://nyti.ms/34GpJnE}

\begin{itemize}
\item
\item
\item
\item
\item
\end{itemize}

Advertisement

\protect\hyperlink{after-top}{Continue reading the main story}

Supported by

\protect\hyperlink{after-sponsor}{Continue reading the main story}

\hypertarget{in-blow-to-taiwan-solomon-islands-is-said-to-switch-relations-to-china}{%
\section{In Blow to Taiwan, Solomon Islands Is Said to Switch Relations
to
China}\label{in-blow-to-taiwan-solomon-islands-is-said-to-switch-relations-to-china}}

\includegraphics{https://static01.nyt.com/images/2019/09/16/world/16taiwan/merlin_153900630_4b389913-e138-488d-9dac-509f01e40269-articleLarge.jpg?quality=75\&auto=webp\&disable=upscale}

By Chris Horton

\begin{itemize}
\item
  Sept. 16, 2019
\item
  \begin{itemize}
  \item
  \item
  \item
  \item
  \item
  \end{itemize}
\end{itemize}

\href{https://cn.nytimes.com/asia-pacific/20190917/solomon-islands-taiwan-china/}{阅读简体中文版}\href{https://cn.nytimes.com/asia-pacific/20190917/solomon-islands-taiwan-china/zh-hant/}{閱讀繁體中文版}

DULAN, Taiwan --- The Solomon Islands has reportedly decided to break
diplomatic relations with the government of Taiwan in order to establish
official ties with China, dealing a blow to both Taipei's global
standing and Washington's regional diplomacy in the Pacific.

Joseph Wu, the Taiwanese foreign minister, said at a news conference on
Monday that Taiwan had learned that the Solomons, an archipelago east of
Australia, had chosen to end 36 years of recognition of Taiwan's
government, leaving only 16 countries that maintain official relations
with Taipei. These countries are the most likely to speak up for Taiwan
in international bodies such as the United Nations General Assembly,
where Taipei is not a member.

For the United States, the
\href{https://www.nytimes.com/2019/09/05/world/asia/taiwan-solomon-islands.html}{Solomons'
decision} is a setback in its effort to prevent China from continuing to
make diplomatic inroads among island nations in the Pacific, a region of
increasing geostrategic competition between Washington and Beijing. Five
of the nations that still have diplomatic ties with Taiwan are in the
region.

China's Communist Party claims self-governing Taiwan as its territory,
but has never ruled it. Beijing has intensified its efforts to peel off
Taiwan's remaining official allies, and some have found China's economic
might too much to resist.

In a statement, Taiwan's Ministry of Foreign Affairs accused China of
bribing Solomons politicians
\href{https://www.nytimes.com/2019/03/20/world/asia/taiwan-south-pacific-tsai-ing-wen-china.html}{to
abandon Taipei} in the run-up to the 70th anniversary on Oct. 1 of the
founding of the People's Republic of China under the Communist Party.

``The government of China has once again resorted to dollar diplomacy
and false promises of large amounts of foreign assistance to buy off a
small number of politicians, so as to ensure that the government of
Solomon Islands adopted a resolution to terminate relations with Taiwan
before China's National Day,'' the statement said. ``Beijing's purpose
is to diminish Taiwan's international presence, hurt the Taiwanese
people, and gradually suppress and eliminate Taiwan's sovereignty.''

Washington broke official ties with Taipei in 1979 in order to establish
diplomatic relations with Beijing as a Cold War counterweight against
the Soviet Union. But Taiwan has remained an important, if unofficial,
American ally in East Asia.

Relations with Taiwan have grown significantly stronger since President
Tsai Ing-wen took office in 2016, followed by President Trump's
inauguration in early 2017. The United States has authorized two major
potential arms sales to Taiwan in recent months.

Both Ms. Tsai and Mr. Trump are facing re-election challenges next year,
with Taiwan's presidential and legislative elections scheduled for
January.

``It is absolutely evident that China, through this case, deliberately
seeks to influence Taiwan's upcoming presidential and legislative
elections,'' the Ministry of Foreign Affairs statement said.

Ms. Tsai's challenger, Han Kuo-yu, the China-friendly mayor of the city
of Kaohsiung in Taiwan's south, has attacked her over deteriorating
cross-strait relations, even though Beijing has ignored her calls for
dialogue since she became president.

Shortly after the announcement about the Solomons, Mr. Han's campaign
office released a statement criticizing both China and Ms. Tsai.

``We strongly suggest that the president find concrete steps to stop the
domino effects of allies' diplomatic de-recognition, thus safeguarding
the R.O.C.'s sovereignty as she has promised,'' the statement said,
referring to the Republic of China, the official name for Taiwan --- and
the government that lost the Chinese civil war in the 1940s.

On Monday, Ms. Tsai acknowledged that the Solomons' decision was a
letdown.

``Changes in the diplomatic arena are indeed challenging,'' she said.
``But Taiwan still has many friends around the world willing to stand
with us, and we are not alone.''

Ms. Tsai also pushed back against a standing offer by the Chinese
leader, Xi Jinping, in which Beijing would administer Taiwan under the
``one country, two systems'' model that China uses to preside over Hong
Kong.

For months, Hong Kong has been roiled by protests against its eroding
semi-autonomy under the system, a model that Mr. Han appeared to tacitly
endorse earlier this year when he met with Hong Kong's chief executive,
Carrie Lam, and Beijing's top official in Hong Kong, Wang Zhimin. He
later met with the director of China's Taiwan Affairs Office, Liu Jieyi.

``China has sought to damage the morale of the Taiwanese people and
force Taiwan to accept `one country, two systems,' '' Ms. Tsai said. ``I
am confident that the 23 million people of Taiwan have this to say in
response: `Not a chance.' ''

Advertisement

\protect\hyperlink{after-bottom}{Continue reading the main story}

\hypertarget{site-index}{%
\subsection{Site Index}\label{site-index}}

\hypertarget{site-information-navigation}{%
\subsection{Site Information
Navigation}\label{site-information-navigation}}

\begin{itemize}
\tightlist
\item
  \href{https://help.nytimes.com/hc/en-us/articles/115014792127-Copyright-notice}{©~2020~The
  New York Times Company}
\end{itemize}

\begin{itemize}
\tightlist
\item
  \href{https://www.nytco.com/}{NYTCo}
\item
  \href{https://help.nytimes.com/hc/en-us/articles/115015385887-Contact-Us}{Contact
  Us}
\item
  \href{https://www.nytco.com/careers/}{Work with us}
\item
  \href{https://nytmediakit.com/}{Advertise}
\item
  \href{http://www.tbrandstudio.com/}{T Brand Studio}
\item
  \href{https://www.nytimes.com/privacy/cookie-policy\#how-do-i-manage-trackers}{Your
  Ad Choices}
\item
  \href{https://www.nytimes.com/privacy}{Privacy}
\item
  \href{https://help.nytimes.com/hc/en-us/articles/115014893428-Terms-of-service}{Terms
  of Service}
\item
  \href{https://help.nytimes.com/hc/en-us/articles/115014893968-Terms-of-sale}{Terms
  of Sale}
\item
  \href{https://spiderbites.nytimes.com}{Site Map}
\item
  \href{https://help.nytimes.com/hc/en-us}{Help}
\item
  \href{https://www.nytimes.com/subscription?campaignId=37WXW}{Subscriptions}
\end{itemize}
