Sections

SEARCH

\protect\hyperlink{site-content}{Skip to
content}\protect\hyperlink{site-index}{Skip to site index}

\href{/section/style}{Style}\textbar{}Young People Are Going to Save Us
All From Office Life

\url{https://nyti.ms/32QLq2P}

\begin{itemize}
\item
\item
\item
\item
\item
\item
\end{itemize}

\includegraphics{https://static01.nyt.com/images/2019/09/17/fashion/17office-generations-1/17office-generations-1-articleLarge.jpg?quality=75\&auto=webp\&disable=upscale}

The office: AN ANALYSIS

\hypertarget{young-people-are-going-to-save-us-all-from-office-life}{%
\section{Young People Are Going to Save Us All From Office
Life}\label{young-people-are-going-to-save-us-all-from-office-life}}

Gen Z-ers and millennials have been called lazy and entitled. Could
they, instead, be among the first to understand the proper role of work
in life?

~Credit...Illustration by Shannon Lin/The New York Times

Supported by

\protect\hyperlink{after-sponsor}{Continue reading the main story}

By \href{https://www.nytimes.com/by/claire-cain-miller}{Claire Cain
Miller} and \href{https://www.nytimes.com/by/sanam-yar}{Sanam Yar}

\begin{itemize}
\item
  Published Sept. 17, 2019Updated Sept. 20, 2019
\item
  \begin{itemize}
  \item
  \item
  \item
  \item
  \item
  \item
  \end{itemize}
\end{itemize}

When Ariel Coleman, 28, quit her last job, as a project manager in the
corporate office of a bank, it wasn't because her new employer offered
her a raise, a different role or more seniority. ``The work-life balance
is just much better,'' she said.

At her new company, Omfgco, a branding and design firm in Portland,
Ore., everyone works from home on Tuesdays and Thursdays at whichever
hours they choose. Ms. Coleman can go for a run or walk her dog.

At the bank, she said, people judged her for taking all her paid time
off. At Omfgco, it's encouraged, which is why she didn't mind answering
work emails while sitting by the fire on a recent camping trip.

\emph{\emph{\emph{{[}}\href{https://www.nytimes.com/interactive/2019/09/17/style/the-office.html}{\emph{Read
our full package, ``The Office: An In-Depth Analysis of Workplace User
Behavior.''}}}{]}}**

``It's: Get your work done, but don't worry about when those hours
are,'' Ms. Coleman said. ``A client calls me at 8 o'clock at night and
I'm happy to talk to them, because that means the next day at 10 a.m., I
can take my dog to the vet. It enables me to make my career more
seamless with my life. It makes it feel more like people are human.''

Many of her friends have chosen their jobs for similar reasons, she
said. ``That's how millennials and Gen Z-ers are playing the game ---
it's not about jumping up titles, but moving into better work
environments,'' she said. ``They're like silent fighters, rewriting
policy under the nose of the boomers.''

For many Americans, work has become
\href{https://www.nytimes.com/2019/01/26/business/against-hustle-culture-rise-and-grind-tgim.html}{an
obsession}, and long hours and
\href{https://www.theatlantic.com/ideas/archive/2019/02/religion-workism-making-americans-miserable/583441/}{endless
striving} something to aspire to. It has caused
\href{https://www.buzzfeednews.com/article/annehelenpetersen/millennials-burnout-generation-debt-work}{burnout},
\href{https://worldhappiness.report/ed/2019/}{unhappiness} and
\href{https://www.nytimes.com/2019/04/26/upshot/women-long-hours-greedy-professions.html}{gender
inequity}, as people struggle to find time for children or passions or
pets or any sort of life besides what they do for a paycheck.

But increasingly, younger workers are pushing back. More of them expect
and demand flexibility --- paid leave for a new baby, say, and generous
vacation time, along with daily things, like the ability to work
remotely, come in late or leave early, or make time for exercise or
meditation. The rest of their lives happens on their phones, not tied to
a certain place or time --- why should work be any different?

Today's young workers have been called lazy and entitled. Could they,
instead, be among the first to understand the proper role of work in
life --- and end up remaking work for everyone else?

It's still rare for companies to operate this way, and the obstacles are
bigger than any one company's H.R. policies. Some older employees may
think new hires should suffer the way they did, and
\href{https://www.nytimes.com/2015/05/31/upshot/the-24-7-work-cultures-toll-on-families-and-gender-equality.html}{employers
benefit} from having always-on workers. Even those that are offering
more flexibility might be
\href{https://www.nytimes.com/2019/05/15/upshot/employers-flexible-work-america.html}{doing
it because unemployment is so low} and they're competing for workers,
which could change if there is an economic downturn.

Also, it's a luxury to be able to demand flexibility in the first place.
Those who can tend to have college degrees and white-collar careers, and
can afford to take a pay cut in exchange, or be highly selective about
their jobs.

That's a kind of freedom that
\href{https://www.nytimes.com/2019/09/01/opinion/working-two-jobs.html}{people
in vast sectors} of the economy
\href{https://www.nytimes.com/interactive/2014/08/13/us/starbucks-workers-scheduling-hours.html}{don't
have} --- and often, it's given to highly regarded employees on a
one-off basis, but not to everyone at a firm.

Still, there are signs that things could change for more workers. Some
large and influential companies, including Walmart and Apple, have
recently
\href{https://www.nytimes.com/2019/08/20/business/dealbook/business-roundtable-corporate-responsibility.html}{begun
talking about the need} to shift from prioritizing shareholders above
all else to taking care of their employees too. And as more millennials
become bosses and more job seekers demand a saner way to work, companies
will have no choice.

``They have proven the model that you don't need to be in the office 9
to 5 to be effective,'' said Ana Recio, the executive vice president of
global recruiting at Salesforce, the tech company. ``This generation is
single-handedly paving the way for the entire work force to do their
jobs remotely and flexibly.''

\hypertarget{when-your-office-is-on-a-mountain-trail}{%
\subsubsection{When Your Office Is on a Mountain
Trail}\label{when-your-office-is-on-a-mountain-trail}}

\href{https://www.pwc.com/gx/en/hr-management-services/publications/assets/pwc-nextgen.pdf}{A
survey by PwC}, an accounting and consulting firm, found that for
millennials, work is a thing, not a place.

Flexibility no longer means what it did to older generations --- the
ability to work from home when a plumber is coming or a child is sick.
But it's also not about 21st-century perks like free meals, on-site dry
cleaning and Wi-Fi-equipped shuttles that help keep people at work
longer.

Instead, it's about employees
\href{https://www.sciencedirect.com/science/article/pii/S000187911930079X?via\%3Dihub\&_ga=2.20924735.68249205.1567371007-1130687845.1567371007}{shaping
their jobs} in ways that fit with their daily lives. That could mean
working remotely or shifting hours when needed. More companies are
offering sabbaticals; free plane tickets for vacations; meditation
rooms; exercise or therapy breaks; paid time off to volunteer; and
extended paid family leave.

One firm has an employee who works mostly from places like Hawaii and
Costa Rica. At another, someone worked remotely while living out of a
van for three months, skiing in the mornings and working in the
afternoons. One person goes to the office at midnight so he can surf in
the morning, and another takes Fridays off to backpack.

``They're maybe not on the partner track, but they're not being
penalized,'' said Abby Engers, a strategist at Boly:Welch, an employment
search firm in Portland, Ore. ``People are burnt out. They're making a
commitment to themselves to take time off. If they see you're doing the
work and doing it well, it doesn't matter if you're doing it at 10 p.m.
or 10 a.m.''

And it's no longer just mothers of young children who are
\href{https://onlinelibrary.wiley.com/doi/10.1111/soc4.12700}{using
flexible schedules}. Women get
\href{https://www.nytimes.com/2014/09/07/upshot/a-child-helps-your-career-if-youre-a-man.html}{penalized}
when that happens --- social scientists call it
\href{https://spssi.onlinelibrary.wiley.com/toc/15404560/69/2}{the
flexibility stigma} --- and their careers
\href{https://www.nytimes.com/2018/04/09/upshot/the-10-year-baby-window-that-is-the-key-to-the-womens-pay-gap.html}{often
never recover} in terms of pay or promotions. But if more fathers and
people who aren't parents ask for flexibility, the
\href{https://journals.sagepub.com/doi/full/10.1177/0731121418768235}{stigma
could lessen}.

Jonathan Wong, 36, worked 80-hour weeks in management consulting when he
became a father. His son would cry every time he saw his roller bag
packed for another work trip, he said, and it was hard to take a break
even to FaceTime his son before bedtime. So he moved to a job at RAND
Corporation, the nonprofit policy research group --- and took a 30
percent pay cut.

``I can bring my kid to preschool every morning,'' he said. ``If the
overwork problem will ever be solved, guys need to be part of the
solution.''

Some employers aren't comfortable giving people autonomy over where and
when they work.

``When younger workers talk about balance, what they are saying is, `I
will work hard for you, but I also need a life,''' said Cali Williams
Yost, the chief executive and founder of Flex Strategy Group, which
helps organizations build flexible work cultures. ``Unfortunately, what
leaders hear is, `I want to work less.'''

But employees say that when they're not forced to cleave life from work,
they work more, and more efficiently. Melanie Neiman, 28, is a project
manager at Breather, a work space rental company. Unlike at her former,
more traditional job, she comes in later in the morning because she is
more productive that way, and visits her family more often because she
can work from where they live.

``When I'm on vacation, if my Slack pings on my phone, I'll probably
answer it, so maybe I work more,'' she said. Yet she is happy to answer
messages when traveling, she said, because it's on her terms. ``I would
never answer emails at my old job on vacation.''

\hypertarget{taking-care-of-employees-too}{%
\subsubsection{Taking Care of Employees,
Too}\label{taking-care-of-employees-too}}

Social scientists have found that not all young people are asking for
these benefits, even if they want them, because they fear they will be
perceived as lazy or disloyal. Even when they aspire to more balanced
lives,
\href{https://www.nytimes.com/2015/07/31/upshot/millennial-men-find-work-and-family-hard-to-balance.html}{they
often find} that traditional workplaces won't enable it.

But dozens of consulting and research
\href{https://www.accenture.com/us-en/insight-gen-z-rising}{firms} that
have
\href{https://www2.deloitte.com/global/en/pages/about-deloitte/articles/millennialsurvey.html}{surveyed}
young people
\href{https://workforceinstitute.org/wp-content/uploads/2019/05/Meet-Gen-Z-Hopeful-Anxious-Hardworking-and-Searching-for-Inspiration.pdf}{have
found} that for them, flexibility is a job requirement.

When
\href{https://www.pewsocialtrends.org/2017/03/23/views-of-paid-leave-relative-to-other-workplace-benefits/}{Pew
Research Center asked} which work arrangement would be most helpful to
people, young people were more likely than older people to say the
flexibility to choose when they worked. Of people 18 to 29, men were
more likely than women to say it, and people without children at home
were as likely as parents to say it.

In a survey of 11,000 workers and 6,500 business leaders by Harvard
Business School and Boston Consulting Group,
\href{https://www.hbs.edu/managing-the-future-of-work/research/Pages/future-positive.aspx}{the
vast majority said} that among the new developments most urgently
affecting their businesses were employees' expectations for flexible,
autonomous work; better work-life balance; and remote working. (Just 30
percent, though, said their businesses were prepared.)

Technology is a big reason for the change. The youngest people entering
the work force don't remember a time when people weren't always
reachable, so they don't see why they would need to sit in an office to
work. (They also say they are more practiced than older colleagues at
setting boundaries on how much they use their phones, so it doesn't
become overbearing.)

Another reason young people are asking for more flexibility is that
they're marrying and having children later, so they're more invested in
their careers by the time they do, and have more leverage to ask for
what they need. Many are caring for aging parents too.

Ali Levitan, 39, worked at a large media firm when she had her first
child and decided to look for a new job. She wanted flexibility, but
also to stay on her ``extremely ambitious'' career path.

It was unwise to mention children in job interviews, she had been told.
But once she had a job offer at General Assembly, an education company,
she asked if she would be able to work from home most Fridays and pick
up her child from school. They immediately agreed.

``I almost fell over at the response because that was not what I had
experienced or expected,'' Ms. Levitan said.

Demanding that employers treat employees well is part of the value
system of the youngest generation of workers, which is the
\href{https://www.pewsocialtrends.org/2019/01/17/generation-z-looks-a-lot-like-millennials-on-key-social-and-political-issues/}{most
diverse ever}, researchers and recruiters say.

``Gen Z is so socially aware and so progressive, they're asking for
things that older generations have been scared to ask for,'' Ms. Recio
at Salesforce said.

Many have also seen their parents struggle with inflexible employers or
unstable jobs. Millennials were the
\href{https://www.nytimes.com/2015/07/23/upshot/more-than-their-mothers-young-women-plan-career-pauses.html}{first
generation raised by women} who entered professions in big numbers. Many
young adults saw their parents lose jobs and savings during the Great
Recession. They no longer expect a lifetime of loyalty from an employer,
so some say they don't want to give their whole life to work.

``They've watched what's happened to the generations before them and
they see the problems that might come ahead,'' said Kathleen Gerson, a
sociologist at New York University whose recent research on the topic
will be published this month by the
\href{https://contemporaryfamilies.org/wp-content/uploads/2019/09/Parents-Cant-Go-It-Alone-Symposium-2019-Full.pdf}{Council
on Contemporary Families}. ``As the work force becomes more diverse, men
as well as women are saying there's more to life than work, and we want
a satisfying life as well.''

\hypertarget{change-the-system-so-we-can-all-succeed}{%
\subsubsection{`Change the System So We Can All
Succeed'}\label{change-the-system-so-we-can-all-succeed}}

\href{https://www.rand.org/pubs/research_briefs/RB9973.html}{Few people
want} to work long, inflexible hours, yet many either work them anyway
or
\href{https://www.nytimes.com/2015/05/05/upshot/how-some-men-fake-an-80-hour-workweek-and-why-it-matters.html}{sneak
out} without asking for permission,
\href{https://pubsonline.informs.org/doi/abs/10.1287/orsc.2015.0975}{research
shows}.

But more young people, recruiters say, are asking for flexibility
upfront, and some prioritize it over pay or seniority. Recruiters who
visit college campuses say new graduates no longer see it as something
to negotiate for, said Marcee Harris Schwartz, the national director of
diversity and inclusion at BDO, the accounting firm: ``It's just assumed
it's part of the deal.''

``Years ago, the interview was, for lack of a better word, a test,''
said Kamaj Bailey, who works in recruiting at Con Edison, the power
company. ``Now it's a conversation. Yes, I want to show that I'm a good
candidate, but I'm also seeing if I'm going to get what I expect.''

John Paul Graff, 34, is a pathologist, as was his father, who worked in
private practice at least 12 hours a day. Dr. Graff decided to work in
academic medicine, and the No. 1 reason was for work-life balance. He
estimated that he gave up about \$100,000 a year but said it's worth it
to work 40 hours a week.

``What we settled on was that the most important thing was time,'' Dr.
Graff said. ``Money will come, it will go, but you're only given so much
time.''

A \href{https://werk.co/research}{survey by Werk}, which helps companies
add flexibility strategies, found that older employees are just as
likely as younger people to want flexibility. They're less likely to
have it, though, because they're less likely to ask for it. Sometimes,
tensions flare between young people who demand a life outside work and
deskbound older workers.

``As boomers age, they too are looking for more workplace flexibility,
but they seem to begrudge giving the same to younger workers when they
didn't have it themselves at their ages and life stage,'' said Pamela
Stone, a sociologist at Hunter College.

Ms. Coleman, who works at the design firm in Portland, said it comes
down to this: The members of her generation are unwilling to settle for
the way things have always been done. It's especially true of the women,
she said, and she is hopeful that men will continue to join them.

``We are just fed up and fired up about asking for what we need,'' she
said. ``We're changing the rules. We're the ones tasked with: Let's
change the system so we can all succeed.''

Advertisement

\protect\hyperlink{after-bottom}{Continue reading the main story}

\hypertarget{site-index}{%
\subsection{Site Index}\label{site-index}}

\hypertarget{site-information-navigation}{%
\subsection{Site Information
Navigation}\label{site-information-navigation}}

\begin{itemize}
\tightlist
\item
  \href{https://help.nytimes.com/hc/en-us/articles/115014792127-Copyright-notice}{©~2020~The
  New York Times Company}
\end{itemize}

\begin{itemize}
\tightlist
\item
  \href{https://www.nytco.com/}{NYTCo}
\item
  \href{https://help.nytimes.com/hc/en-us/articles/115015385887-Contact-Us}{Contact
  Us}
\item
  \href{https://www.nytco.com/careers/}{Work with us}
\item
  \href{https://nytmediakit.com/}{Advertise}
\item
  \href{http://www.tbrandstudio.com/}{T Brand Studio}
\item
  \href{https://www.nytimes.com/privacy/cookie-policy\#how-do-i-manage-trackers}{Your
  Ad Choices}
\item
  \href{https://www.nytimes.com/privacy}{Privacy}
\item
  \href{https://help.nytimes.com/hc/en-us/articles/115014893428-Terms-of-service}{Terms
  of Service}
\item
  \href{https://help.nytimes.com/hc/en-us/articles/115014893968-Terms-of-sale}{Terms
  of Sale}
\item
  \href{https://spiderbites.nytimes.com}{Site Map}
\item
  \href{https://help.nytimes.com/hc/en-us}{Help}
\item
  \href{https://www.nytimes.com/subscription?campaignId=37WXW}{Subscriptions}
\end{itemize}
