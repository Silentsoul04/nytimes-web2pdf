Sections

SEARCH

\protect\hyperlink{site-content}{Skip to
content}\protect\hyperlink{site-index}{Skip to site index}

\href{/section/style}{Style}\textbar{}They Are What You Eat

\url{https://nyti.ms/2LB2Pqi}

\begin{itemize}
\item
\item
\item
\item
\item
\end{itemize}

\includegraphics{https://static01.nyt.com/images/2019/09/13/fashion/OFFICEHQ/merlin_160727946_1faa8a22-43ab-4057-80a3-f9a15083bd95-articleLarge.jpg?quality=75\&auto=webp\&disable=upscale}

the office: AN ANaLYSis

\hypertarget{they-are-what-you-eat}{%
\section{They Are What You Eat}\label{they-are-what-you-eat}}

A visit to the headquarters of your favorite mall foods.

The three-story Atlanta offices of Focus Brands were redesigned in 2015
``with Generation Y, Z as a primary interest in mind, because I'm
building for the future,'' said Steve DeSutter, the
C.E.O.Credit...Johnathon Kelso for The New York Times

Supported by

\protect\hyperlink{after-sponsor}{Continue reading the main story}

By \href{https://www.nytimes.com/by/caity-weaver}{Caity Weaver}

\begin{itemize}
\item
  Sept. 17, 2019
\item
  \begin{itemize}
  \item
  \item
  \item
  \item
  \item
  \end{itemize}
\end{itemize}

In a 1780 letter to his wife, Abigail, John Adams proposed a chronology
of generational obligations for learning. It was his duty,
\href{https://www.masshist.org/digitaladams/archive/doc?id=L17800512jasecond}{the
future president wrote} during a sojourn in France, to ``study politicks
and war,'' that the next generation ``may have the liberty to study
mathematicks and philosophy,'' that the next should have ``the right to
study Painting, Poetry, Musick, Architecture, Statuary, Tapestry and
Porcelaine.''

Mr. Adams's epistle ending there, modern readers cannot know whether,
with more paper and time, he would have eventually previsioned the
21st-century corporate campus where word clouds are studied in both
digital and material states so that the current youngest generation of
workers may attain a perfect knowledge of Cinnabon brand identity.

We can only assume that when, in a separate letter written three years
earlier, in the midst of the Revolutionary War, he expressed his wish
that posterity would
``\href{https://www.masshist.org/digitaladams/archive/doc?id=L17770426ja}{make
a good use of}'' the liberty he sought, he had in mind something like
the 62,000-square-foot headquarters of Focus Brands, the Atlanta-based
operator of 6,536 food franchise locations.

\includegraphics{https://static01.nyt.com/images/2019/09/17/fashion/17OFFICE-HQh/merlin_160556670_c6e2fffa-f0e2-49f3-a5b9-df090f97340d-articleLarge.jpg?quality=75\&auto=webp\&disable=upscale}

Behind every snack food franchise is a story --- a young mother raised
in the Amish church takes over a pretzel stand in the wake of
devastating personal tragedy; a flat tire on a hot day forces a Greek
immigrant ice cream man to serve a rapidly softening frozen product ---
that is eventually obscured by the unrelenting popularity of its
low-cost, high-indulgence, addictively delicious treats.

Behind every Cinnabon Classic Roll is an office of people dedicated to
preserving its legal status, ingredient consistency, financial
viability, informative yet friendly online presence, uniformity of
storefront appearance, marketing power and cultural legacy.

Not to mention designing, photographing, printing, filming and
distributing the materials that make it possible for anyone of legal
working age in the United States to acquire the knowledge of Cinnabon
Classic Roll fabrication in accordance with company standards. Ditto a
Fudgie the Whale ice cream cake or an Aloha Pineapple smoothie.

Roughly 375 such individuals work in the main office of Focus Brands.

Image

Focus is the brand behind many of America's favorite mall food brands,
including Auntie Anne's and Jamba.Credit...Johnathon Kelso for The New
York Times

Focus owns and governs the Auntie Anne's, Carvel, Cinnabon, Jamba
(formerly Jamba Juice), McAlister's Deli, Moe's Southwest Grill and
Schlotzsky's brands. In its offices, these have been boiled down to
their very essences and those essences in turn splashed upon the walls,
suspended from the ceiling and drilled into the exposed concrete
pillars.

The most arresting manifestation of these brands is the floating text
that hovers in altocumulus formation above a brightly lit walkway off
the building's main entrance. ``Creamy frosting buttery AWESOME
irresistible HOMEWRECKER,'' declare (some of) the words, which Steven
DeSutter, the C.E.O., said, were chosen by brand leaders to ``give them
a chance to say what words define your brand.'' (``Homewrecker'' is a
type of burrito at Moe's.)

Closer inspection reveals the text's hues are not the standard ROYGBIV
rainbow array, but reflective of the idiosyncratic palette of Focus
Brands: Cinnabon turquoise, Auntie Anne lapis, the oxblood standard of
Carvel.

\emph{\emph{\emph{{[}}\href{https://www.nytimes.com/interactive/2019/09/17/style/the-office.html}{\emph{Read
our full package, ``The Office: An In-Depth Analysis of Workplace User
Behavior.''}}}{]}}**

The colors signify that here are employed the diplomats who bring
Cinnabon to Pizza Hut and to Egypt; the researchers who have determined
that the sticky bun recipe should be less sweet in Asia to appeal to
consumers; the men and women who photograph ``milkshakes'' made of
mashed potatoes and crushed up Oreo cookies that exactly resemble Carvel
milkshakes, but do not melt.

Image

Focus Brands acquired Jamba Juice, along with a cache of Jamba
Juice-branded novelty squishy fruits, in 2018.Credit...Johnathon Kelso
for The New York Times

Mr. DeSutter's aims in creating a new headquarters for Focus ---
designed in 2015 by Gensler, one of the world's largest architectural
firms (it also worked on a recent redesign of the New York Times
offices) --- were twofold: to fortify connections between the siloed
brands, and to create an environment attractive for millennial workers.
To that end, he directed his head of human resources to seek input from
young Focus employees.

``I said, `They will be your primary adviser group, because I want to
make sure that whatever we build, we build with Generation Y, Z as a
primary interest in mind, because I'm building for the future. I'm
building for the next generation,''' Mr. DeSutter, 65, recalled.
``Because believe me, baby boomers were not going to get excited about
having no walls in an office. So there was no reason to ask them what
they wanted, because they're going to tell you they wanted high cubicles
or walls in an office.''

Erin Greer, 38, a director of Gensler Atlanta's Workplace Studio who
contributed to the project, said that a challenge of multigenerational
office design is to create a ``space that doesn't alienate anyone''
while bearing in mind that corporate clients typically acquire their
spaces under leases of at least 10 years, a time span in which employee
demographics can shift ``drastically.''

Image

The most arresting manifestation of the brands is the floating rainbow
text that hovers in altocumulus formation above a brightly lit walkway
off the building's main entrance.~Credit...Johnathon Kelso for The New
York Times

``Another underlying layer is how people are educated,'' Mrs. Greer
said. ``Traditionally education happened in a very rigid kind of form.
And now when you go into schools at whatever level, it's much more
collaborative and solution oriented. And that, I think, has led to some
of the evolution you see in the workplace and how people are taught to
interact with each other and work and problem solve across industries.''
A Montessori office.

What did Focus's millennial advisers call for in the workplace of their
dreams? A pool table, which never materialized because of lack of space,
according to Mr. DeSutter. The desired swing sets in conference rooms
\emph{were} installed, suspended from sturdy metal chains, giving said
rooms a louche corporate pleasure dungeon vibe.

Most have now been removed (``they're easy to fall out of,'' one
employee said), but two remain --- black accents in a lime green meeting
room (lime carpet, lime upholstered benches, lime walls) where varieties
of potato chips offered at McAlister's are listed on the walls in
concentric circles of jumbo text.

Image

While Focus is the institutional home of Cinnabon, the nearest location
where employees can buy Cinnabon products is a mall roughly 15 minutes
from the office.Credit...Johnathon Kelso for The New York Times

Besides bright colors, silver stools that resemble tree stumps, and
unexpected graphic presentations of food-related adjectives, Gensler and
Focus's aesthetic feast for millennial tastes manifests primarily in the
layout: open and collaborative.

Research on the merits of open offices is mixed. Gensler company
literature suggests a majority of ``knowledge workers'' prefer ``some
sort of'' open plan, while some studies have found that open offices
imperil employees'
\href{https://www.ncbi.nlm.nih.gov/pubmed/29334117}{job satisfaction},
\href{https://www.ncbi.nlm.nih.gov/pubmed/11055149}{musculoskeletal
health} and
``\href{https://journals.aom.org/doi/full/10.5465/255498}{psychological
privacy}.'' Whether for good or for ill, they are prevalent.

Proponents cite their shared, common windows as a major advantage over
cubicle warrens. These airy, symmetrical, light-and-glass-filled spaces
that stretch to the near horizon are indeed beautiful, in much the same
way the ninth floor of the Asch building in Manhattan --- its large
windows casting luxuriant light onto its open work area from their
generous southern exposure --- was beautiful, until the afternoon of the
Triangle Shirtwaist Factory fire.

This is not to imply that the Focus offices, which conduct regular fire
and shelter-in-place drills, are in any way unsafe, merely that they
provide the kind of dreamy access to natural light that designers have
pursued for centuries. Nor is it intended to portend ill for, or in any
way curse, the work force of Focus, a presumably representative sample
of whom were, on a recent visit, welcoming, good-humored, eager to share
knowledge and friendly. Nor is that intended to imply Triangle
seamstresses lacked such qualities.

Image

Credit...Johnathon Kelso for The New York Times

Image

Credit...Johnathon Kelso for The New York Times

At Focus, most work spaces are structured around so-called bench
seating, where employees are seated on chairs at long communal tables,
with individual spaces demarcated by low partitions.

Singular features abound on the premises, like the thousands of little
glass balls that hang from what looks like threads of the finest
spider's silk, arranged into the word ``FOCUS'' on the executive floor.
Then there are the test kitchens.

Here, on a recent Wednesday morning, Cinnabon's lab-coat-clad executive
chef Jennifer Holwill, 41, scrutinized smears of caramel spread before
her on parchment paper for purposes related to ``risk management.'' (The
risk: a DEFCON 2 scenario wherein Cinnabon's caramel manufacturer would
be rendered unable to produce the vast quantities of caramel required to
satiate a nation.)

It is in these miniature commercial kitchens that new products are
researched and designed (with the notable exception of menu items from
Auntie Anne's, whose test kitchen and creative facilities are based out
of a historic former post office in Lancaster, Pa., for reasons that
seem largely to hinge on the undeniably accurate tautology that Auntie
Anne's is ``a Pennsylvania-based company'').

Image

The chefs Brenda McGranahan, left, and Jennifer Holwill work in the
research and development test kitchen at the Focus
headquarters.~Credit...Johnathon Kelso for The New York Times

It is here where a limited-time offering like a Cinnabon Churro Frosting
Sandwich can be tweaked to such perfection that the Dr. Frankensteins
who created it are left with no choice but to prolong its life
indefinitely, incorporating it into menus as a permanent feature.

Ms. Holwill said that colleagues have offered to buy fresh-baked samples
from her --- verboten, as test kitchen chefs are only allowed to offer
samples (should they care to offer samples) first come first served.
(The nearest location where Cinnabon MiniBons can be legally and
ethically purchased by employees is a mall roughly 15 minutes away.)

But because it is 2019, some of the most urgent work in the building is
done by the social teams. A major force of Focus public relations is
Marissa Sharpless, a virtuosically engaging senior manager of social
media and P.R. for Cinnabon who, speaking off the cuff, paints a brand
picture of such depth and volume that it rivals any by Caravaggio.

Mrs. Sharpless, 34, can keep watch over Focus's byzantine, constantly
updating social channels on the building's ground floor, where a
wall-mounted TV displays social interactions. It is possible here to
observe in real time as people direct their ire, daydreams and profane,
improbable questions @JambaJuice, @Schlotzskys, @AuntieAnnes.

Image

This is not a restaurant.Credit...Johnathon Kelso for The New York Times

Through a deft series of clicks, Mrs. Sharpless summoned a page of
graphics onto her laptop screen. Here was the world's collected
knowledge (or at least articulated Twitter observations) about Cinnabon
over the past month, broken down into hashtags, emoji clouds, choropleth
maps of the United States, pie charts, bar graphs, line graphs and days
of the week (Cinnabon typically ``pops'' on Tuesdays).

There was also some kind of something where purple was supposed to stay
high because purple was sentiment and, said Mrs. Sharpless, a rise in
conversation coupled with a drop in sentiment is ``the last thing you
want to see.''

The top emojis people had felt about Cinnabon were crying laughing,
sobbing, heart eyes and anguished. Words and phrases were assigned a
color: green for positive emotions (``crave''), red for negative ones
(``not that great'').

Image

The unbreakable cipher.Credit...Johnathon Kelso for The New York Times

Upstairs, Robby Ayala, 28, a social media manager for Moe's Southwest
Grill, maintains the brand's Instagram, Twitter and Facebook accounts.
He said the most difficult aspect of his job was coming up with unique
content about Moe's every day.

``I try to tweet once a day. That's a lot more challenging than it
sounds,'' he said, and laughed.

The difficulty isn't in coming up with an appealingly irreverent tweet
to share with fans of Moe's (a task many could likely accomplish) but in
thinking of another one the next day, and another the day after that,
and another the day after that, hundreds --- potentially thousands ---
of times in a row, without repeating or embarrassing yourself.

``If you jump in inappropriately, you will get roasted,'' he said,
reflecting on the recent Twitter storm over Popeyes' Spicy Chicken
Sandwich. ``Like Zaxby's
\href{https://twitter.com/Zaxbys/status/1163608537056063490}{jumped in,
they got roasted}. Boston Market
\href{https://twitter.com/bostonmarket/status/1163520399562498054}{jumped
in}. Chick-fil-A
\href{https://twitter.com/PopeyesChicken/status/1163510538959069184?ref_src=twsrc\%5Etfw\%7Ctwcamp\%5Etweetembed\%7Ctwterm\%5E1163510538959069184\&ref_url=https\%3A\%2F\%2Fwww.vox.com\%2Fthe-goods\%2F2019\%2F8\%2F28\%2F20836936\%2Fpopeyes-chick-fil-a-fried-chicken-sandwich-twitter}{took
a lot of L's}.''

Mr. Ayala's job is, essentially, to talk about Moe's in a brief,
hilarious and charming way, without stopping, forever. He found
delirious success one day this summer when
\href{https://twitter.com/Moes_HQ/status/1150800074835972096}{his tweet}
combining a meme about aliens in Area 51 with the very notion of Moe's
burritos received roughly 2,100 retweets. But then he had to tweet
again.

Maintaining the high corporate standards of some of America's iconic
snack foods in 2019 demands stamina, adrenaline, a commitment to
unironically observing National Lemonade Day, high levels of
organization and good cheer.

It also requires a willingness to work in a space bursting with digital
screens, where, for the general population, all but the most essential
walls have been eliminated, and the resulting din camouflaged with pink
noise pumped in from ceiling speakers --- which is to say, it's not for
everyone.

In his own office, Mr. DeSutter recalled one worker who, years ago,
faced with the impending renovation, ``said, `I'm not going to be able
to work in this environment,' and they left.'' That wasn't the only
staff member with reservations. Another employee, who Mr. DeSutter
described as long tenured, asked him, ```Aren't you worried that you're
going to ruin the company?'''

``I remember, on the first day we moved into this building,'' Mr.
DeSutter said, ``when that particular individual sat down at their open
desk, I sat beside him for a few minutes and said, `This is going to be
O.K.'''

He stayed.

Advertisement

\protect\hyperlink{after-bottom}{Continue reading the main story}

\hypertarget{site-index}{%
\subsection{Site Index}\label{site-index}}

\hypertarget{site-information-navigation}{%
\subsection{Site Information
Navigation}\label{site-information-navigation}}

\begin{itemize}
\tightlist
\item
  \href{https://help.nytimes.com/hc/en-us/articles/115014792127-Copyright-notice}{©~2020~The
  New York Times Company}
\end{itemize}

\begin{itemize}
\tightlist
\item
  \href{https://www.nytco.com/}{NYTCo}
\item
  \href{https://help.nytimes.com/hc/en-us/articles/115015385887-Contact-Us}{Contact
  Us}
\item
  \href{https://www.nytco.com/careers/}{Work with us}
\item
  \href{https://nytmediakit.com/}{Advertise}
\item
  \href{http://www.tbrandstudio.com/}{T Brand Studio}
\item
  \href{https://www.nytimes.com/privacy/cookie-policy\#how-do-i-manage-trackers}{Your
  Ad Choices}
\item
  \href{https://www.nytimes.com/privacy}{Privacy}
\item
  \href{https://help.nytimes.com/hc/en-us/articles/115014893428-Terms-of-service}{Terms
  of Service}
\item
  \href{https://help.nytimes.com/hc/en-us/articles/115014893968-Terms-of-sale}{Terms
  of Sale}
\item
  \href{https://spiderbites.nytimes.com}{Site Map}
\item
  \href{https://help.nytimes.com/hc/en-us}{Help}
\item
  \href{https://www.nytimes.com/subscription?campaignId=37WXW}{Subscriptions}
\end{itemize}
