Sections

SEARCH

\protect\hyperlink{site-content}{Skip to
content}\protect\hyperlink{site-index}{Skip to site index}

\href{/section/style}{Style}\textbar{}Women Poop. Sometimes At Work. Get
Over It.

\url{https://nyti.ms/2LB2vrA}

\begin{itemize}
\item
\item
\item
\item
\item
\item
\end{itemize}

\includegraphics{https://static01.nyt.com/images/2019/09/17/fashion/17office-bathroom-1/17office-bathroom-1-articleLarge.png?quality=75\&auto=webp\&disable=upscale}

The office: AN ANALYSIS

\hypertarget{women-poop-sometimes-at-work-get-over-it}{%
\section{Women Poop. Sometimes At Work. Get Over
It.}\label{women-poop-sometimes-at-work-get-over-it}}

Why must the bathroom continue to be fraught?

Because everyone does it.Credit...Illustration by Shannon Lin/The New
York Times

Supported by

\protect\hyperlink{after-sponsor}{Continue reading the main story}

By \href{https://www.nytimes.com/by/jessica-bennett}{Jessica Bennett}
and Amanda McCall

\begin{itemize}
\item
  Published Sept. 17, 2019Updated Sept. 20, 2019
\item
  \begin{itemize}
  \item
  \item
  \item
  \item
  \item
  \item
  \end{itemize}
\end{itemize}

There once was a woman who walked regularly from her office in Midtown
Manhattan to a hotel across the street in order to use the restroom, and
that woman may have been one of us.

That woman had a friend, at another office job, who carried a book of
matches and a can of air freshener in her purse --- more willing to set
off the office fire alarm than leave any hint of odor in a public
lavatory.

That friend had another friend, at another office job, who repeatedly
forced her body to do the deed so quickly --- racing from cubicle to
bathroom and back, in an effort to deflect attention from what she might
be doing in there --- that it led to a semi-serious hemorrhoid problem.

As her former colleague put it: ``She was pooping at the speed of pee.''

Remember the children's book,
``\href{https://www.amazon.com/Everyone-Turtleback-School-Library-Binding/dp/0613685725/ref=sr_1_1?keywords=everybody+poops\&qid=1568487804\&s=books\&sr=1-1}{Everyone
Poops}''? It is meant to teach kids that defecating is a natural,
healthy part of digestion, and it does so by illustrating a wide variety
of creatures --- dogs, cats, snakes, whales, hippos, little boys ---
happily defecating. But you know who you won't see defecating in that
book, happily or unhappily? Women.

\emph{\textbf{{[}Read our full package,}}
\textbf{\href{https://www.nytimes.com/interactive/2019/09/17/style/the-office.html}{\emph{``The
Office: An In-Depth Analysis of Workplace User Behavior.''}}\emph{{]}}}

We may be living in an age where certain pockets of the corporate world
are breathlessly adapting to women's needs --- company-subsidized
tampons, salary workshops, lactation rooms. But even in the world's most
progressive workplace, it's not a stretch to think that you might have
an empowered female executive leading a meeting at one moment and then
sneaking off to another floor to relieve herself, the next.

Poop shame is real --- and it disproportionately affects women, who
suffer \href{https://gi.org/topics/common-gi-problems-in-women/}{from
higher rates of irritable bowel syndrome and inflammatory bowel
disease}. In other words, the patriarchy has seeped into women's
intestinal tracts. Let's call it the pootriarchy.

Girls aren't born with poo shame --- it's something they're taught.

In
``\href{https://www.amazon.com/Psychology-Bathroom-Nick-Haslam/dp/0230368255}{Psychology
in the Bathroom},'' the psychologist Nicholas Haslam writes that girls
tend to be toilet trained earlier than boys, learning at a young age to
neatly keep their bodily functions contained (our words, not his).

When those girls get a bit older, they learn to pass gas silently ---
while boys do it loudly, and think it's hilarious. (Yes, there is a kind
of Kinsey scale to gas-passing and it goes like this: According to a
study called
``\href{https://www.jstor.org/stable/10.1525/sp.2005.52.3.315?seq=1\#page_scan_tab_contents}{Fecal
Matters''} that was published in a journal called ``Social Problems,''
adult heterosexual men are far more likely to engage in scatological
humor than heterosexual women and are more likely to report
intentionally passing gas. Gay men are less likely to intentionally pass
gas than heterosexual women, and lesbian women are somewhere in
between.)

``If a boy farts, everyone laughs, including the boy,'' said Sarah
Albee, the author of
``\href{https://www.amazon.com/Poop-Happened-History-World-Bottom/dp/0802720773}{Poop
Happened!: A History of the World from the Bottom Up}.'' ``If a girl
farts, she is mortified.''

Which is not to say that anxious poopers or audible flatulators of all
genders don't exist: Indeed, a male friend of ours, a U.S. Marine,
recently explained that he often changes out of his military uniform and
into another while on base in order to enter an entirely different
facility to use the restroom. (He was one of three individuals who
responded to a survey we sent out to 100 people, mostly women, about
fecal habits at work. Even with the cloak of anonymity, apparently
nobody wanted to talk about it.)

But while boys and men are more likely to develop
``\href{https://www.ncbi.nlm.nih.gov/pubmed/24056834}{paruresis},'' the
D.S.M.-recognized medical term for pee-shyness --- theorized by some to
stem, in part, from the pressure of standing next to each other at open
urinals --- it is women who are more likely to have ``parcopresis,'' the
corresponding bowel movement anxiety, which is not in the D.S.M.,
according to a variety of fecal scholars.

``The bathroom is saturated with gender in fascinating ways,'' said Mr.
Haslam, a professor of psychology at the University of Melbourne, who
noted that women's aversion, particularly at work, is not entirely
unfounded: One unpublished study he mentions in his book found that a
woman who excused herself to go to the bathroom was evaluated more
negatively than one who excused herself to tend to ``paperwork'' ---
while there was no difference in the way participants viewed the men.

``At one level it's an association of women with purity,'' said Mr.
Haslam, referring to the double standard. ``At another it's a double
standard applied to hygiene and civility, where the weight falls
disproportionately on women to be clean, odorless and groomed.''

Or, as one of the woman interviewed in that ``Fecal Matters'' study put
it: ``Women are supposed to be non-poopers.''

For most of history, it would seem, they have fallen in line ---
adopting all sorts of creative ways to avoid mention, inference,
acknowledgment, or God forbid, smell, even when inside the bathroom.

According to Ms. Albee, in the Gold Rush days, while the men on the open
range would simply find a shrub or pop a squat, prairie women would form
elaborate protective circles to shield one another. ``They'd all stand
in a circle, facing out, holding their skirts out to the side to form a
`wall,''' she said. ``Then one at a time, they'd take turns going to the
bathroom in the middle of the circle, away from prying eyes.''

These days, bathroom camouflage antics look far less sisterly.

There are those who engage in the Flush Hush, which involves flushing
the toilet over and over again to drown out any sound.

There is the Scatological Standoff, in which two or more women sit
silently in stalls next to one another, waiting for one to break the
silence and have a bowel movement first --- or simply give up and
retreat back to their cubicle, only to begin the same standoff an hour
later.

And then there's the Poop Dupe --- when you walk into the bathroom, see
a co-worker you know, and immediately beeline to the mirror to check
your hair. (Because you'd rather be known as superficial than
defecating, obviously.)

Or maybe you just hold it. According to
a\href{http://news.nationalpost.com/2013/04/30/is-pooping-the-last-taboo-for-women-at-work/}{recent
survey of 1,000 Canadian women}, 71 percent said they go ``to great
lengths to avoid defecating --- especially in a public washroom.'' (Is
it any surprise to hear that women are more constipated than men?)

Historians have long noted that public facilities were created for ---
and built by --- men, and bathrooms are no exception. Most architects
are men, most plumbers are men, and early public facilities were
tailored to the white men --- and then later, white women --- who were
engaging in public life enough to use them.

Which might help explain why nobody stopped to think that just because
the square footage of a bathroom facility may be equal, that doesn't
mean you can necessarily fit an equal number of stalls. Men's room users
have the luxury of urinals and speed, while women --- who must contend
with things like periods, changing-tables, one-piece rompers and wiping
---
\href{https://journals.sagepub.com/doi/abs/10.1177/0885412206295846}{take
longer to use the restroom}, while doing so with less real estate.

In Congress, women didn't have their own bathrooms on the House floor
until 2011 (2011! When there were 76 of them serving!) while those in
the Senate got theirs off the Senate floor in 1993.

``I didn't have the five minutes to get'' to the restroom ``and then the
five minutes to get back,'' Rep. Donna F. Edwards
(D-Md.),\href{https://www.washingtonpost.com/lifestyle/style/women-in-the-house-get-a-restroom/2011/07/28/gIQAFgdwfI_story.html}{told
the Washington Post} in 2011. ``I would have missed a vote.''

Previously, those female House members had to trek out of chambers and
fight off tourists in another hall --- a scene that reminded us of
``Hidden Figures,'' the film about the early black women scientists of
NASA, who
\href{https://www.thecut.com/2017/01/hidden-figures-shows-how-a-bathroom-break-can-change-history.html}{had
to hike half a mile to the closest segregated women's restroom to
relieve themselves}. The scene may have been fictional, but suffice to
say, women of color have had to endure much worse.

And then there are the biological factors at play.

According to the work of Dr. Robynne Chutkan, an integrative
gastroenterologist and the author of ``Gutbliss,'' women's poop anxiety
might not simply be cultural or even psychological. It could be
physical, as there are actually some profound differences between the
female and male digestive tracts, beginning with the length of the
colon, which is longer in women (Dr. Chutkan calls it the ``voluptuous
Venus'').

``What that extra length in the colon does is create this redundancy,
these sort of extra twists and
turns,''\href{https://www.theatlantic.com/health/archive/2013/10/what-we-eat-affects-everything/279922/}{she
has said}. ``Think of the male colon as kind of a gentle horseshoe, and
the female colon as being a tangled-up Slinky.''

As it turns out, the ideal position for a person to comfortably relieve
their bowels --- at least according to gastroenterologists --- is a lot
like a squat, with the knees at a 90-degree angle to the waist, and not
a seated position. Which means that perhaps all of us should be
investing in
a\href{https://www.theguardian.com/news/2018/nov/30/bowel-movement-change-the-way-you-poo-squatty-potty-toilet?CMP=Share_iOSApp_Other}{squatty
potty} to prop up our feet, but particularly those of us with a tangled
up Slinky for a colon, sitting on a toilet in an office building that
was built for the height of men.

Or, a better idea: We could invest in educating girls to accept their
bodies as they are, along with all the smells and sounds that come with
it. Because, quite frankly, women have enough crap to deal with.

\href{http://jessicabennett.com/}{Jessica Bennett} writes on gender and
culture, and is pretty sure this is not the type of writing her parents
expected when she told them she'd taken a job at The New York Times.

\href{https://amandacmccall.com/}{Amanda McCall} is a writer, producer
and co-author of the book ``Grandma's Dead: Breaking Bad News With Baby
Animals.''

Advertisement

\protect\hyperlink{after-bottom}{Continue reading the main story}

\hypertarget{site-index}{%
\subsection{Site Index}\label{site-index}}

\hypertarget{site-information-navigation}{%
\subsection{Site Information
Navigation}\label{site-information-navigation}}

\begin{itemize}
\tightlist
\item
  \href{https://help.nytimes.com/hc/en-us/articles/115014792127-Copyright-notice}{©~2020~The
  New York Times Company}
\end{itemize}

\begin{itemize}
\tightlist
\item
  \href{https://www.nytco.com/}{NYTCo}
\item
  \href{https://help.nytimes.com/hc/en-us/articles/115015385887-Contact-Us}{Contact
  Us}
\item
  \href{https://www.nytco.com/careers/}{Work with us}
\item
  \href{https://nytmediakit.com/}{Advertise}
\item
  \href{http://www.tbrandstudio.com/}{T Brand Studio}
\item
  \href{https://www.nytimes.com/privacy/cookie-policy\#how-do-i-manage-trackers}{Your
  Ad Choices}
\item
  \href{https://www.nytimes.com/privacy}{Privacy}
\item
  \href{https://help.nytimes.com/hc/en-us/articles/115014893428-Terms-of-service}{Terms
  of Service}
\item
  \href{https://help.nytimes.com/hc/en-us/articles/115014893968-Terms-of-sale}{Terms
  of Sale}
\item
  \href{https://spiderbites.nytimes.com}{Site Map}
\item
  \href{https://help.nytimes.com/hc/en-us}{Help}
\item
  \href{https://www.nytimes.com/subscription?campaignId=37WXW}{Subscriptions}
\end{itemize}
