Sections

SEARCH

\protect\hyperlink{site-content}{Skip to
content}\protect\hyperlink{site-index}{Skip to site index}

\href{https://www.nytimes.com/section/books/review}{Book Review}

\href{https://myaccount.nytimes.com/auth/login?response_type=cookie\&client_id=vi}{}

\href{https://www.nytimes.com/section/todayspaper}{Today's Paper}

\href{/section/books/review}{Book Review}\textbar{}Maaza Mengiste Sings
a Modern Song of War

\url{https://nyti.ms/2mWhwKI}

\begin{itemize}
\item
\item
\item
\item
\item
\end{itemize}

Advertisement

\protect\hyperlink{after-top}{Continue reading the main story}

Supported by

\protect\hyperlink{after-sponsor}{Continue reading the main story}

Fiction

\hypertarget{maaza-mengiste-sings-a-modern-song-of-war}{%
\section{Maaza Mengiste Sings a Modern Song of
War}\label{maaza-mengiste-sings-a-modern-song-of-war}}

\includegraphics{https://static01.nyt.com/images/2019/10/13/books/review/13Serpell-COVER/13Serpell-COVER-articleLarge.jpg?quality=75\&auto=webp\&disable=upscale}

Buy Book ▾

\begin{itemize}
\tightlist
\item
  \href{https://www.amazon.com/gp/search?index=books\&tag=NYTBSREV-20\&field-keywords=The+Shadow+King+Maaza+Mengiste}{Amazon}
\item
  \href{https://du-gae-books-dot-nyt-du-prd.appspot.com/buy?title=The+Shadow+King\&author=Maaza+Mengiste}{Apple
  Books}
\item
  \href{https://www.anrdoezrs.net/click-7990613-11819508?url=https\%3A\%2F\%2Fwww.barnesandnoble.com\%2Fw\%2F\%3Fean\%3D9780393083569}{Barnes
  and Noble}
\item
  \href{https://www.anrdoezrs.net/click-7990613-35140?url=https\%3A\%2F\%2Fwww.booksamillion.com\%2Fp\%2FThe\%2BShadow\%2BKing\%2FMaaza\%2BMengiste\%2F9780393083569}{Books-A-Million}
\item
  \href{https://bookshop.org/a/3546/9780393083569}{Bookshop}
\item
  \href{https://www.indiebound.org/book/9780393083569?aff=NYT}{Indiebound}
\end{itemize}

When you purchase an independently reviewed book through our site, we
earn an affiliate commission.

By Namwali Serpell

\begin{itemize}
\item
  Published Sept. 26, 2019Updated Sept. 27, 2019
\item
  \begin{itemize}
  \item
  \item
  \item
  \item
  \item
  \end{itemize}
\end{itemize}

\textbf{THE SHADOW KING}\\
By Maaza Mengiste

A conundrum: How to sing a song of war? For those who send humans to die
in battle for notions like property and borders, it must seem simple:
Sing war broadly, with your whole chest, with lyrics like patriotism,
courage and loyalty. For those who mourn the dead --- scribes like the
Greek poet Simonides, who wrote epitaphs for fallen warriors --- your
voice must tremble with poignancy, with a certain piteousness. But what
about for denizens of the 21st century, those who wish to address the
weight of a past crammed with war, while sounding a clarion warning to
the future?

There's much to consider, especially for writers. There's the problem of
scale; both the minutiae and the panoramas of warfare must be balanced
and accurate. There's the problem of beauty. The long history of war
writing --- from Tennyson to Pat Barker to Adichie --- tends toward the
grandeur and grace we call ``the sublime.'' Yet it feels like a mockery
or an offense to depict battle too beautifully. You must not be maudlin,
either. You must not bray or propagandize. You must be neither too
abstract nor too clinical. You must not succumb to the comic-book
temptation to glorify gore or indulge in the macabre bloodsloshing of
murder. And what on earth do you do with the women?

As Cara Hoffman noted in The New York Times in 2014, ``stories about
female veterans are nearly absent from our culture. It's not that their
stories are poorly told. It's that their stories are simply not told in
our literature, film and popular culture.'' In the ancient tales, women
do crop up as both the causes and trophies of conflict (Helen, Briseis).
In recent narratives, they mostly appear as refugees and casualties. In
war stories both old and new, women are nurses to the wounded, and the
victims of rape --- war's unbanishable shadow. Rarely are they depicted
as warriors. Is that a profound truth or a blind spot?

Image

Maaza Mengiste's lyrical, remarkable new novel, ``The Shadow King,'' set
during the Second Italo-Ethiopian War, somehow manages to solve the
riddle of how to sing war now.Credit...Nina Subin

Maaza Mengiste's lyrical, remarkable new novel, ``The Shadow King,'' set
during the Second Italo-Ethiopian War, somehow manages to solve the
riddle of how to sing war now. She doesn't seek a narrow path between
the straits of these artistic and ethical questions. Instead, she
encompasses them in all their contradiction, laying them out in
breathtakingly skillful juxtaposition. Indeed, the problem of the woman
at war --- the fundamental opposition of being the ultimate cause versus
the ultimate victim, of power versus precarity --- is the structure of
conflict at the core of the novel, which ends with the phrase ``Settle
into a heart and split it forever.''

``The Shadow King'' tells the story of Hirut, a young Ethiopian woman
who goes from lowly servant to proud warrior. She begins the novel as an
orphan who works alongside an unnamed cook in the household of a man
named Kidane and his wife, Aster. The relationships between the
characters --- a tangle of lust, loyalty, jealousy, resentment,
tenderness --- emerge, fittingly, around a battle over Hirut's gun. This
Wujigra, ``designed to deliver a single lethal shot with consistent
accuracy,'' was given to Hirut by her late father, who used it as a
soldier in the First Italo-Ethiopian War. Once the family's most prized
possession, the weapon becomes even more valuable when Italy invades
Ethiopia again in 1935. Hirut hides the gun under her bed; Aster,
suspicious that the girl has stolen a necklace, finds it during one of
her raids; Kidane takes it from both of them. In response, Hirut --- in
a surprising but convincing turn --- starts stealing things from the
household and burying them in a hole by the stable.

Upon discovering the trove, Aster beats and whips Hirut in a harrowing
scene, the beauty and psychological precision of which distinguish the
novel as a whole: ``The blow comes as a relief to Hirut. It is something
to do: to be hit. It is somewhere to go: to be in pain. She welcomes the
distraction from the tremor she feels seeping out of Aster and sinking
into her own skin. \ldots{} Slowly, she feels the cuts and gashes, the
burn of open wounds. She is splitting into pieces.'' When the violence
pauses, ``the world spins in an unnatural quiet. There is just Aster
pressing her face on the ground, sliding toward her. Hirut notes the
frantic sorrow in her eyes, the way her mouth is chewing words to spit
them out. Dust blooms as Aster drags herself over dirt as if she has
forgotten the use of her legs, as if her body cannot contain the full
weight of her fury.''

As the novel proceeds, the relationship between Aster and Hirut ebbs and
flows, takes odd twists, and bubbles into fury and love. Both women are
at different times raped by Kidane: Aster as a child bride, in a
wedding-night scene that teeters delicately between desire and terror;
and Hirut decades later, after he has become her commander. The second
time he violates her, Hirut's response is startling: ``She yawns.'' And
yawns and yawns again --- ``it is both absurd and luxurious. A shock and
a relief. It is a fist uncoiling and expanding inside her body, a long,
extended breath singed and shaped by hate'' --- until, in utter
humiliation and bewilderment, Kidane stops. This is not a glib safety
tip, a narrative pepper spray, but rather a strange, complex negotiation
of power between this man and this woman.

Image

For once, all this grandeur, all this grace, is in the service of a tale
of a woman, Hirut, as indelible and compelling a hero as any I've read
in years.Credit...

When the war begins in earnest, these personal conflicts do not fall
away; they accrue even greater stakes. Like Hirut, Aster too needs
``something to do,'' to be, beyond her life at home, so she takes on the
role of soldier, with wild and fierce pride. When Kidane assumes command
of a rebel force in the north to fight off the Italians, his wife
insists on gathering and training local women to support the effort.
Hirut joins the troop, gets her gun back and fights alongside these
women, who are celebrated in rhetoric usually reserved for male
warriors. When Hirut trains, the foe she pictures is Kidane. She later
takes on a key role in the war when she notices that a peasant musician
named Minim (``Nothing'') bears an uncanny resemblance to Haile
Selassie, the emperor of Ethiopia who has vanished from the country.
Kidane orders Hirut to disguise Minim as the eponymous ``Shadow King''
in order to rally the nation's disheartened rebels. She appears beside
him as his royal guard.

Hirut and Aster are eventually captured and kept in a war camp run by a
Colonel Fucelli, who commands a Jewish-Italian soldier, Ettore Navarra
(or ``Foto''; he always keeps a camera with him), to capture the
Ethiopian prisoners in another, perhaps more horrific way: by taking
pictures of them as they're thrown off a cliff. Mengiste ventures into
the minds of these two men, with their confusions of cruelty and
self-pity, their musings about masculinity and representation. She also
narrates the inner worlds of two other women, Kidane's cook and a
prostitute named Fifi, both of whom end up working in their own ways as
rebel spies. Mengiste even enters the mind of the historical Selassie,
whose ``interludes'' turn this second novel into a kind of prelude to
her first, ``Beneath the Lion's Gaze,'' about his last days of rule in
1974. Twice, we're treated to a digressive, informational ``Brief
History'' of minor characters. Interspersed throughout are brief
descriptions of Ettore's photographs, which become a material plot point
that draws him and Hirut together decades later, while also working as a
canny literalization of the Homeric metaphor. In addition to these many
perspectives, a ``chorus'' interjects lyrical commentary on the action.

This isn't as complicated as it sounds. All these chapter forms are
short, if unpredictable, and the reader feels not at the whim of an
experimental dictator, but in the steady hands of a master. As in the
passage where Aster beats Hirut, certain words recur (split, spin,
bloom, awkward, frantic), and hearts tend to pound and thud a lot.
There's no humor in this novel; laughter is bitter, sarcastic, mad and,
just once, happy. But we come to realize that these are deliberate
poetic choices --- for simplicity and sublimity --- as even more
references to the Greeks emerge. There are allusions to the ``Iliad,''
and echoes of Icarus and Daedalus as Ethiopians take flight from the
cliff. Selassie has a vision of Simonides, along with a character from
``Aida'' and a ghost, in a surreal penultimate scene that seemed the
only misstep in this majestic novel.

I forgave it because, for once, all this grandeur, all this grace, is in
the service of a tale of a woman, Hirut, as indelible and compelling a
hero as any I've read in years. This novel made me feel pity and fear,
and more times than is reasonable, gave me goose bumps. Reading it was
like this: In the middle of battle, tortured by the thought of Kidane's
endless power over her, Hirut suddenly loses her fear of death. She runs
toward the Italian Army, taps at her own chest, and says: \emph{Boom}.

Advertisement

\protect\hyperlink{after-bottom}{Continue reading the main story}

\hypertarget{site-index}{%
\subsection{Site Index}\label{site-index}}

\hypertarget{site-information-navigation}{%
\subsection{Site Information
Navigation}\label{site-information-navigation}}

\begin{itemize}
\tightlist
\item
  \href{https://help.nytimes.com/hc/en-us/articles/115014792127-Copyright-notice}{©~2020~The
  New York Times Company}
\end{itemize}

\begin{itemize}
\tightlist
\item
  \href{https://www.nytco.com/}{NYTCo}
\item
  \href{https://help.nytimes.com/hc/en-us/articles/115015385887-Contact-Us}{Contact
  Us}
\item
  \href{https://www.nytco.com/careers/}{Work with us}
\item
  \href{https://nytmediakit.com/}{Advertise}
\item
  \href{http://www.tbrandstudio.com/}{T Brand Studio}
\item
  \href{https://www.nytimes.com/privacy/cookie-policy\#how-do-i-manage-trackers}{Your
  Ad Choices}
\item
  \href{https://www.nytimes.com/privacy}{Privacy}
\item
  \href{https://help.nytimes.com/hc/en-us/articles/115014893428-Terms-of-service}{Terms
  of Service}
\item
  \href{https://help.nytimes.com/hc/en-us/articles/115014893968-Terms-of-sale}{Terms
  of Sale}
\item
  \href{https://spiderbites.nytimes.com}{Site Map}
\item
  \href{https://help.nytimes.com/hc/en-us}{Help}
\item
  \href{https://www.nytimes.com/subscription?campaignId=37WXW}{Subscriptions}
\end{itemize}
