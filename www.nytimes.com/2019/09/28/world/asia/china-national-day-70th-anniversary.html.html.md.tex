Sections

SEARCH

\protect\hyperlink{site-content}{Skip to
content}\protect\hyperlink{site-index}{Skip to site index}

\href{https://www.nytimes.com/section/world/asia}{Asia Pacific}

\href{https://myaccount.nytimes.com/auth/login?response_type=cookie\&client_id=vi}{}

\href{https://www.nytimes.com/section/todayspaper}{Today's Paper}

\href{/section/world/asia}{Asia Pacific}\textbar{}Tanks, Missiles and No
Pigeons: China to Celebrate 70th Birthday of the People's Republic

\url{https://nyti.ms/2nBpwkv}

\begin{itemize}
\item
\item
\item
\item
\item
\item
\end{itemize}

Advertisement

\protect\hyperlink{after-top}{Continue reading the main story}

Supported by

\protect\hyperlink{after-sponsor}{Continue reading the main story}

\hypertarget{tanks-missiles-and-no-pigeons-china-to-celebrate-70th-birthday-of-the-peoples-republic}{%
\section{Tanks, Missiles and No Pigeons: China to Celebrate 70th
Birthday of the People's
Republic}\label{tanks-missiles-and-no-pigeons-china-to-celebrate-70th-birthday-of-the-peoples-republic}}

\includegraphics{https://static01.nyt.com/images/2019/09/29/world/29china-xi/merlin_161570895_2dd55827-0ddb-437e-a56d-3d7fa1775d08-articleLarge.jpg?quality=75\&auto=webp\&disable=upscale}

By \href{https://www.nytimes.com/by/steven-lee-myers}{Steven Lee Myers}

\begin{itemize}
\item
  Published Sept. 28, 2019Updated Oct. 1, 2019
\item
  \begin{itemize}
  \item
  \item
  \item
  \item
  \item
  \item
  \end{itemize}
\end{itemize}

\href{https://cn.nytimes.com/china/20190930/china-national-day-70th-anniversary/}{阅读简体中文版}\href{https://cn.nytimes.com/china/20190930/china-national-day-70th-anniversary/zh-hant/}{閱讀繁體中文版}

BEIJING --- Enormous flower arrangements in Beijing extol the signature
promise of China's leader to realize the Chinese Dream. Red banners urge
people to ``rally closer'' to the Communist Party with ``Comrade Xi
Jinping at its core.'' The authorities have restricted live
entertainment venues,
\href{https://www.nytimes.com/2019/09/23/world/asia/china-xi-jinping-communist-party-70th-anniversary.html}{ordered
people to vacate apartments} and banned flying kites, sky lanterns and
even homing pigeons, a charming feature of many neighborhoods.

As the Communist Party of China prepares to celebrate the 70th
anniversary of its rule,
\href{https://www.nytimes.com/2019/09/30/world/asia/china-national-day-hong-kong-protests.html}{the
state is choreographing} the pomp and pageantry to exalt President Xi as
the unassailable leader of a rising nation and the indispensable bulwark
against an array of challenges that threaten to erode its iron grip on
power.

On Tuesday, the country's National Day, he will preside over a military
parade through Tiananmen Square whose preparations appear as ambitious
and, arguably, as grandiose as the leader himself. It will involve
15,000 soldiers and sailors, 160 fighter jets, bombers and other
aircraft, and 580 tanks and other weapons --- some of them, military
commanders hinted coyly, never before seen in public.

\includegraphics{https://static01.nyt.com/images/2019/09/27/world/00china-xi-2/merlin_161455902_938fbdab-361d-417c-aeac-0d3264158c58-articleLarge.jpg?quality=75\&auto=webp\&disable=upscale}

\emph{{[}Read\href{https://www.nytimes.com/2019/09/30/world/asia/china-national-day-hong-kong-protests.html?module=inline}{live
updates to China's National Day parade and Hong Kong protests}.{]}}

For Mr. Xi, the anniversary has come at an opportune, and much-needed,
moment. It has given him a chance to bask in the party's achievements at
a time when it is coming under increasing strain, especially from the
economic drag of the trade war with the United States. It is also
\href{https://www.nytimes.com/2019/09/21/world/asia/china-islam-crackdown.html}{fending
off} condemnation of the government's mass detentions of Muslims in
Xinjiang; battling an epidemic of African swine fever that has driven up
prices for
\href{https://www.nytimes.com/2019/09/10/business/china-pork-prices.html}{that
Chinese staple, pork}; and trying to contain months of
\href{https://www.nytimes.com/2019/09/28/world/asia/hong-kong-protests-china.html}{protests
in Hong Kong} that have surged into
\href{https://www.nytimes.com/2019/09/27/world/asia/hong-kong-protests-identity.html}{an
open defiance of Beijing's rule.}

``The republic is built by each brick and tile like this,'' Mr. Xi said
on Wednesday after riding an express train 25 miles south from the city
center to inaugurate
\href{https://www.nytimes.com/2018/11/24/world/asia/china-beijing-daxing-airport.html}{Beijing's
new international airport}, one of the mega-projects that China has
built as an affirmation of its political and economic greatness. The
airport, with seven runways and
\href{https://www.bbc.co.uk/news/world-asia-china-49750182}{a
star-shaped terminal}by the famed Iraqi architect Zaha Hadid, is set to
be the world's largest.

On the occasion of a simple ribbon cutting, he sought to channel the
spirit of the founding of the People's Republic of China by quoting from
one of Mao Zedong's revolutionary war poems,
\href{https://allpoetry.com/Loushan-Pass}{``Loushan Pass,''} about a
battle in the mountains of Guizhou in 1935.

\begin{quote}
Idle boast the strong pass is a wall of iron
\end{quote}

\begin{quote}
With firm strides we are crossing its summit
\end{quote}

In seven years in office, Mr. Xi has successfully consolidated his power
by purging rivals, crushing dissent and
\href{https://www.nytimes.com/2018/03/11/world/asia/china-xi-constitution-term-limits.html?searchResultPosition=1}{removing
the constitutional limits on his power}. He has made the party the
arbiter of all aspects of Chinese life and urged greater ideological
purity to gird the nation for what he repeatedly describes as
\href{https://www.nytimes.com/2019/09/07/world/asia/china-hong-kong-xi-jinping.html}{a
great and continuing struggle}.

Image

Soldiers awaiting the announcement of the founding of the People's
Republic of China at Tiananmen Square on Oct. 1, 1949.Credit...Visual
China Group, via Getty Images

He has sought to amass power as great as any leader since Chairman Mao,
and even, to the dismay of some critics, presumed to place himself
beside Mao in the pantheon of China's Communist leaders.

``Xi gets it that the story matters,'' Timothy Cheek, a professor of
Chinese history at the University of British Columbia, wrote in an
email. ``He is facing what the Soviet bloc faced in the 1980s: lack of
belief in the professed values of the party in the face of obvious
contractions and corruption.''

That failure is a lesson that Mr. Xi and other leaders have studied
well. China has thrived in ways the Soviet Union did not, largely
because it managed a transition to a kind of free-market capitalism that
offered millions of Chinese greater material prosperity.

Sustaining that is now the greatest challenge facing Mr. Xi and the
party. China's growth has slowed to its
\href{https://www.nytimes.com/2019/07/14/business/china-economy-growth-gdp-trade-war.html}{weakest
point} in decades, as the trade war grinds on the economy and consumers
pull back. While trade talks continue, President Trump has continued to
lash out at China, calling it a
``\href{https://www.nytimes.com/2019/09/20/us/politics/trump-china-theat-to-world.html}{threat
to the world}.''

\includegraphics{https://static01.nyt.com/images/2019/09/27/video/00vid-nationalday-china-still2/00vid-nationalday-china-still2-videoSixteenByNineJumbo1600.jpg}

``The weak link is not the lack of democracy,'' Mr. Cheek said of the
foundations of the party's power. ``It is the prospect of an economic
downturn, of a failure to deliver to the ordinary citizen.''

The
\href{https://www.nytimes.com/2019/09/28/world/asia/hong-kong-protest.html}{expectation
of protests in Hong Kong}, the semiautonomous territory returned to
Chinese sovereignty 22 years ago, has already
\href{https://www.nytimes.com/2019/09/29/business/hong-kong-china-power.html}{created
a counternarrative} to the Communist Party's birthday celebration.
Around the world, the televised cuts from Tuesday's military parade in
Beijing --- and fireworks in the evening --- to another round of tear
gas and gas bombs in Hong Kong seem inevitable.

\emph{{[}Read more on how the protests in Hong Kong}
\emph{\href{https://www.nytimes.com/2019/09/28/world/asia/hong-kong-protests-china.html?action=click\&module=inline\&pgtype=Article}{expose
a generational rift}.{]}}

National Day commemorates Oct. 1, 1949, when Mao appeared on the same
balcony on the Gate of Heavenly Peace that Mr. Xi will on Tuesday and
proclaimed the formation of the People's Republic of China.

At the time, the civil war against the nationalists still raged, and the
Communist Party had few resources to govern so large an impoverished and
war-battered country. In one famous anecdote repeated by officials and
state news media this past week, some of the 17 aircraft that took part
in the first parade flew over Tiananmen Square a second time to make the
air force seem bigger than it really was.

Image

Deng Xiaoping during the celebration of the 35th anniversary, in
1984.Credit...Sovfoto, via Getty Images

A military parade was held each National Day from 1949 to 1959, but the
tradition lapsed during the darkest years of Mao's reign, which included
the Great Famine and the Cultural Revolution. It resumed in 1984,
overseen by Deng Xiaoping, the leader who began the reforms that opened
up China's economy.

``National Day parades and celebrations are carefully designed to
communicate the self-image the regime wants Chinese people and the world
to see,'' said Susan L. Shirk, an expert on China at the University of
California, San Diego, who was in the bleachers on Tiananmen Square
during the 35th anniversary.

The party then wanted to promote capitalist policies to a generation
raised on communist ones, so the parade featured banners with slogans
like ``Time Is Money'' and floats promising material comforts. One float
carried a 20-foot-tall chicken, another a refrigerator stocked with cold
beer.

``Today,'' Ms. Shirk added, ``China has become more of a national
security state.''

The 40th anniversary fell only months after the massacre that put down
the student-led democracy movement on Tiananmen Square on June 4, 1989.
The parade went on, but without the military hardware, ``apparently
because Beijing residents had seen enough weaponry in recent months,''
The Times's correspondent at the time,
\href{https://www.nytimes.com/1989/10/02/world/people-s-china-celebrates-but-without-the-people.html}{Nicholas
D. Kristof, wrote}.

Image

A special parade was held in September 2015 to commemorate the 70th
anniversary of the end of World War II, or as it is known in China, the
War of Chinese People's Resistance Against Japanese
Aggression.Credit...Damir Sagolj/Reuters

Tuesday's parade will not be the biggest by the number of participants,
military officials acknowledged at a briefing in Beijing this past week,
but they emphasized it would include new weapons, all of them made in
China, tested and ready to use.

Researchers scouring
\href{https://www.frstrategie.org/sites/default/files/documents/publications/images-strategiques/2019/strategic-imagery-military-parade-70th-anniversary-PRC-2019.pdf}{satellite
photos of staging areas} near Beijing have spotted a hypersonic drone
and a new mobile intercontinental ballistic missile, the DF-41, which is
capable of hurling multiple nuclear warheads toward any city in the
United States.

The parade will showcase a military that is, according to foreign
military analysts, more capable, more integrated and better equipped
than at any point in Chinese history.

``In the past 70 years, the development and growth of the Chinese
military is there for all to see,'' the Defense Ministry's spokesman,
Senior Col. Wu Qian, said in the briefing, when asked if the display of
weaponry was intended to send a message to potential adversaries. ``We
have neither the intention nor the need to flex muscle through military
parades.''

Image

An employee of a neighborhood committee distributing national flags to
residents' homes in Beijing last week.Credit...Gilles Sabrié for The New
York Times

He and the other officers emphasized the military's obedience to the
Communist Party leadership --- as if it were in question --- and its
embrace of the
\href{https://www.nytimes.com/2017/10/11/world/asia/xi-jinping-military-china-purge.html}{military
reorganization} overseen in recent years by Mr. Xi, whose many titles
include chairman of the Central Military Commission, despite some
resistance in the ranks.

The preparations have transformed the city, turning it into a stage to
glorify the country, the party and Mr. Xi himself.
\href{https://www.nytimes.com/2019/09/23/world/asia/china-xi-jinping-communist-party-70th-anniversary.html}{Nothing
has been left to chance}.

Even the troops involved are chosen not only according to physical
standards --- the male soldiers marching in formation should be 5 feet 9
to 6 feet 1 --- but also for political ones, said Maj. Gen. Tan Min,
executive deputy director of the Military Parade Joint Command and
deputy chief of staff of the Central Theater Command.

They, too, face inconveniences. According to
\href{http://www.globaltimes.cn/content/1163738.shtml}{an article} in
The Global Times, troops were given disposable diapers to wear during
rehearsals because there was no time allotted for bathroom breaks.

A salesman in Beijing, Zhang Sheng, said he very much looked forward to
the parade. ``I want to come on that day, but probably can't because
there are too many people.''

In fact, access to the parade is restricted to only those with passes.
Mr. Zhang, 35, brought roller skates to ride along the parade route
ahead of the festivities, but it was forbidden.

Advertisement

\protect\hyperlink{after-bottom}{Continue reading the main story}

\hypertarget{site-index}{%
\subsection{Site Index}\label{site-index}}

\hypertarget{site-information-navigation}{%
\subsection{Site Information
Navigation}\label{site-information-navigation}}

\begin{itemize}
\tightlist
\item
  \href{https://help.nytimes.com/hc/en-us/articles/115014792127-Copyright-notice}{©~2020~The
  New York Times Company}
\end{itemize}

\begin{itemize}
\tightlist
\item
  \href{https://www.nytco.com/}{NYTCo}
\item
  \href{https://help.nytimes.com/hc/en-us/articles/115015385887-Contact-Us}{Contact
  Us}
\item
  \href{https://www.nytco.com/careers/}{Work with us}
\item
  \href{https://nytmediakit.com/}{Advertise}
\item
  \href{http://www.tbrandstudio.com/}{T Brand Studio}
\item
  \href{https://www.nytimes.com/privacy/cookie-policy\#how-do-i-manage-trackers}{Your
  Ad Choices}
\item
  \href{https://www.nytimes.com/privacy}{Privacy}
\item
  \href{https://help.nytimes.com/hc/en-us/articles/115014893428-Terms-of-service}{Terms
  of Service}
\item
  \href{https://help.nytimes.com/hc/en-us/articles/115014893968-Terms-of-sale}{Terms
  of Sale}
\item
  \href{https://spiderbites.nytimes.com}{Site Map}
\item
  \href{https://help.nytimes.com/hc/en-us}{Help}
\item
  \href{https://www.nytimes.com/subscription?campaignId=37WXW}{Subscriptions}
\end{itemize}
