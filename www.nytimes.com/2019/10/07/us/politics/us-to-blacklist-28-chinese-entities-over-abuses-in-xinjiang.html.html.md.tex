Sections

SEARCH

\protect\hyperlink{site-content}{Skip to
content}\protect\hyperlink{site-index}{Skip to site index}

\href{https://www.nytimes.com/section/politics}{Politics}

\href{https://myaccount.nytimes.com/auth/login?response_type=cookie\&client_id=vi}{}

\href{https://www.nytimes.com/section/todayspaper}{Today's Paper}

\href{/section/politics}{Politics}\textbar{}U.S. Blacklists 28 Chinese
Entities Over Abuses in Xinjiang

\url{https://nyti.ms/2IvAEHo}

\begin{itemize}
\item
\item
\item
\item
\item
\end{itemize}

Advertisement

\protect\hyperlink{after-top}{Continue reading the main story}

Supported by

\protect\hyperlink{after-sponsor}{Continue reading the main story}

\hypertarget{us-blacklists-28-chinese-entities-over-abuses-in-xinjiang}{%
\section{U.S. Blacklists 28 Chinese Entities Over Abuses in
Xinjiang}\label{us-blacklists-28-chinese-entities-over-abuses-in-xinjiang}}

\includegraphics{https://static01.nyt.com/images/2019/11/07/business/07dc-entity-SWAP2/merlin_145763160_7125128d-c003-45ec-9be5-2a2163e9558b-articleLarge.jpg?quality=75\&auto=webp\&disable=upscale}

By \href{https://www.nytimes.com/by/ana-swanson}{Ana Swanson} and
\href{https://www.nytimes.com/by/paul-mozur}{Paul Mozur}

\begin{itemize}
\item
  Oct. 7, 2019
\item
  \begin{itemize}
  \item
  \item
  \item
  \item
  \item
  \end{itemize}
\end{itemize}

WASHINGTON --- The Trump administration said Monday that it had
\href{https://s3.amazonaws.com/public-inspection.federalregister.gov/2019-22210.pdf}{added
28 Chinese organizations} to a United States blacklist over concerns
about their role in human rights violations, effectively blocking those
entities from buying American products.

The organizations have been implicated in China's campaign targeting
Uighurs and other predominantly Muslim minorities in the autonomous
region of Xinjiang, according to a Commerce Department filing.

Among the entities being placed on the list are Hikvision and Dahua
Technology, two of the world's largest manufacturers of video
surveillance products. It also hits China's well-funded, newly emerging
class of artificial-intelligence start-ups. Together, the companies'
products are central to China's ambitions to be the top global exporter
of surveillance technology.

The list also includes companies that specialize in artificial
intelligence, voice recognition and data as well as provincial and local
security bureaus that have helped construct what amounts to a police
state in Xinjiang. These entities have been involved ``in the
implementation of China's campaign of repression, mass arbitrary
detention and high-technology surveillance,'' the filing said.

The move was announced just days before high-level Chinese and American
officials meet in Washington to try to resolve a trade war that has
begun inflicting pain on both sides of the Pacific.

The blacklist's impact on the companies is likely to be mixed.

In many cases, they could find ways to replace American components and
have likely already stockpiled key parts, limiting the short-term
impact.

Over the longer term, it could hamper their access to United States and
European markets, as well as damage recruitment efforts. American
customers, universities and others will likely look askance at striking
up relations with Chinese companies on the blacklist.

A Commerce Department spokesman said Monday that the action was not
related to those talks. But the decision is likely to rankle the Chinese
government, which has helped support some of these companies as they
have developed into cutting-edge technology firms.

``The U.S. government and Department of Commerce cannot and will not
tolerate the brutal suppression of ethnic minorities within China,''
Commerce Secretary Wilbur Ross said in a statement.

Hikvision said in a statement that it strongly opposed the decision and
had been trying to address the administration's concerns for the past
year. The punishment will ``hurt Hikvision's U.S. business partners and
negatively impact the U.S. economy,'' the company said.

China has faced
\href{https://www.hrw.org/video-photos/interactive/2019/05/02/china-how-mass-surveillance-works-xinjiang}{growing
condemnation} from human rights groups in recent months for its
detention of up to one million ethnic Uighurs and other minority Muslims
in
\href{https://www.nytimes.com/2018/09/08/world/asia/china-uighur-muslim-detention-camp.html?module=inline}{large
internment camps} in Xinjiang.

Beijing has constructed an advanced surveillance system, in what it
describes as an effort to fight Islamic extremism among the Uighurs, the
largest ethnic group in Xinjiang. But many Uighurs and others around the
world say Chinese officials are trying to suppress their culture and
religion.

Human Rights Watch
\href{https://www.hrw.org/report/2018/09/09/eradicating-ideological-viruses/chinas-campaign-repression-against-xinjiangs}{has
said} the violations are of a ``scope and scale not seen in China since
the 1966-1976 Cultural Revolution,'' and Secretary of State Mike Pompeo
has called China's treatment of the Uighurs the ``stain of the
century.''

Yet administration officials have wavered on how much to keep human
rights and economic concerns separate in their negotiations with China.
Many officials emphasize that the topics are separate, but the
administration has shelved several proposals that would have shined
light on China's abuses over concerns that a tough stance could upset
trade talks. And President Trump himself has often linked national
security and other concerns to trade talks.

On Monday, Mr. Trump said ``bad'' action in Hong Kong, the site of
violent protests, would hurt progress on trade and urged China to find a
``humane solution.''

``I think they're coming to make a deal,'' he said of the Chinese.
``It's got to be a fair deal.''

The Trump administration has steadily ratcheted up pressure on China
through tariffs on more than \$360 billion of Chinese products and other
restrictions on Chinese investment in the United States. The
administration has also begun looking to restrict exports to China.

This year, the administration placed Huawei, the Chinese telecom
equipment giant, on the blacklist, saying it posed national security
concerns. It
\href{https://www.nytimes.com/2019/06/21/us/politics/us-china-trade-blacklist.html}{added
five Chinese entities} to the list in June, also citing national
security.

American companies can still apply for licenses to supply products to
organizations that have been placed on the Commerce Department entity
list, but the government may deny the applications.

The companies on the list help illustrate the breadth and development of
China's surveillance industry, which increasingly uses predictive
technology to track its own citizens, or spot potential protests or
crimes as they occur.

The new additions include several artificial intelligence start-ups:
Megvii, SenseTime and Yitu Technologies. They also include iFlytek,
which makes voice recognition software; Xiamen Meiya Pico Information
Company, a data forensics company; and Yixin Science and Technology
Company, which makes nanotechnology.

The listed government entities include Xinjiang's public security bureau
and 19 subordinate bureaus and institutes.

Several of the firms have grown into global operations while servicing
an extensive market in China. Hikvision said it had more than 34,000
employees and dozens of divisions worldwide, and it has supplied
products to the Beijing Olympics, the World Cup in Brazil and Linate
Airport in Milan. Dahua Technologies has more than 16,000 employees,
according to its website, with divisions in North America, Europe and
Latin America.

The companies run the gamut in terms of capabilities and focus. Some are
specialists in facial-recognition systems, voice-recognition software,
surveillance cameras and phone tracking systems that are sold largely to
China's security forces. Others have ambitions of building systems that
can filter social media content and products that could help doctors
reach cancer diagnoses.

As a result, the impact of the bans will be varied. For the surveillance
camera maker Hikvision, which makes a significant portion of the world's
security cameras, the impact could be significant. A block could have a
roughly 10 percent impact on revenue, technology research firm Sanford
Bernstein said in a Monday note. While the company would have to replace
some American-made chips placed in its high-end cameras, most of the
impact would come on the back-end servers that help analyze footage as
part of its security systems.

The impact could be larger in terms of broader sales ambitions.
Hikvision has worked to court the American market, including a number of
government-connected customers.

For those focused on artificial-intelligence software, like Yitu and
Megvii, the direct hit could be small. Nonetheless, the blacklist could
impact them in other ways. The artificial intelligence start-ups on the
list have partnerships with American software firms, connections to
American universities and have been active in trying to hire foreign
talent. In all those cases, the blocks will likely hamper their efforts.

Advertisement

\protect\hyperlink{after-bottom}{Continue reading the main story}

\hypertarget{site-index}{%
\subsection{Site Index}\label{site-index}}

\hypertarget{site-information-navigation}{%
\subsection{Site Information
Navigation}\label{site-information-navigation}}

\begin{itemize}
\tightlist
\item
  \href{https://help.nytimes.com/hc/en-us/articles/115014792127-Copyright-notice}{©~2020~The
  New York Times Company}
\end{itemize}

\begin{itemize}
\tightlist
\item
  \href{https://www.nytco.com/}{NYTCo}
\item
  \href{https://help.nytimes.com/hc/en-us/articles/115015385887-Contact-Us}{Contact
  Us}
\item
  \href{https://www.nytco.com/careers/}{Work with us}
\item
  \href{https://nytmediakit.com/}{Advertise}
\item
  \href{http://www.tbrandstudio.com/}{T Brand Studio}
\item
  \href{https://www.nytimes.com/privacy/cookie-policy\#how-do-i-manage-trackers}{Your
  Ad Choices}
\item
  \href{https://www.nytimes.com/privacy}{Privacy}
\item
  \href{https://help.nytimes.com/hc/en-us/articles/115014893428-Terms-of-service}{Terms
  of Service}
\item
  \href{https://help.nytimes.com/hc/en-us/articles/115014893968-Terms-of-sale}{Terms
  of Sale}
\item
  \href{https://spiderbites.nytimes.com}{Site Map}
\item
  \href{https://help.nytimes.com/hc/en-us}{Help}
\item
  \href{https://www.nytimes.com/subscription?campaignId=37WXW}{Subscriptions}
\end{itemize}
