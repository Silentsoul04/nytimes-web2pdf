Sections

SEARCH

\protect\hyperlink{site-content}{Skip to
content}\protect\hyperlink{site-index}{Skip to site index}

\href{https://www.nytimes.com/section/politics}{Politics}

\href{https://myaccount.nytimes.com/auth/login?response_type=cookie\&client_id=vi}{}

\href{https://www.nytimes.com/section/todayspaper}{Today's Paper}

\href{/section/politics}{Politics}\textbar{}Trump's Trade War Could Put
Swiss-Size Dent in Global Economy, I.M.F. Warns

\url{https://nyti.ms/35gME9x}

\begin{itemize}
\item
\item
\item
\item
\item
\end{itemize}

Advertisement

\protect\hyperlink{after-top}{Continue reading the main story}

Supported by

\protect\hyperlink{after-sponsor}{Continue reading the main story}

\hypertarget{trumps-trade-war-could-put-swiss-size-dent-in-global-economy-imf-warns}{%
\section{Trump's Trade War Could Put Swiss-Size Dent in Global Economy,
I.M.F.
Warns}\label{trumps-trade-war-could-put-swiss-size-dent-in-global-economy-imf-warns}}

\includegraphics{https://static01.nyt.com/images/2019/10/08/business/08DC-TRADEECON-01/merlin_161906535_b04ddc6e-a716-4f87-9bef-9f83b68056c1-articleLarge.jpg?quality=75\&auto=webp\&disable=upscale}

By \href{https://www.nytimes.com/by/ana-swanson}{Ana Swanson}

\begin{itemize}
\item
  Oct. 8, 2019
\item
  \begin{itemize}
  \item
  \item
  \item
  \item
  \item
  \end{itemize}
\end{itemize}

WASHINGTON --- The new head of the International Monetary Fund warned on
Tuesday that America's trade war with China could cost the global
economy around \$700 billion by 2020 --- a loss equivalent to the size
of Switzerland's entire economy.

In her first speech as managing director, Kristalina Georgieva said the
global economy had shifted from a synchronized upswing two years ago to
a synchronized slowdown, weighed down in part by the pain of President
Trump's trade war. The fund will be downgrading its projections for
global growth in 2019 and 2020 next week, when it releases new
projections of the economic losses related to the trade war between the
United States and China.

``We have spoken in the past about the dangers of trade disputes,'' Ms.
Georgieva said. ``Now, we see that they are actually taking a toll.''

The warning comes at a potential turning point in the trade conflict
between the world's two largest economies.
\href{https://www.nytimes.com/2019/09/21/business/united-states-china-trade.html}{American
and Chinese negotiators are meeting in Washington this week} to try to
resolve a trade war that has begun to inflict economic pain in both
countries. Negotiators are hoping to reach an agreement that would
improve some of the Chinese economic practices the Trump administration
has complained about and potentially roll back some of the tariffs Mr.
Trump has placed on more than \$360 billion of Chinese goods.

The stakes for some type of resolution are high. The United States is
poised to raise tariffs on \$250 billion of Chinese goods to 30 percent
from 25 percent next Tuesday and plans to tax more consumer goods,
including laptops, smartphones and apparel, in December. Barring an
agreement, the Trump administration will tax nearly every product that
the country imports from China by the year's end.

In a note to investors, Joshua Shapiro, the chief United States
economist at the research firm MFR, said that the path of the American
economy was partly dependent on ``political decisions regarding tariffs
and other measures that are designed to force China's hand.''

``In this very dangerous game of `chicken,' the global economy may end
up as roadkill,'' Mr. Shapiro said.

Economists have debated how much of the global economic slowdown is a
result of trade tensions and how much stems from other factors,
including sluggish growth in Europe and a credit slowdown in China. The
Trump administration has blamed weak growth overseas, as well as a
relatively tight monetary policy from the Federal Reserve.

But as the trade conflict has dragged on and the United States and China
have steadily expanded the lists of goods that are subject to tariffs,
the costs of the trade war are becoming more evident.

Major measures of consumer sentiment have dipped, with many consumers
citing tariffs as a reason for their pessimism. Last week, a closely
watched gauge of American manufacturing revealed that factories had
slowed for the second straight month in September, while new export
orders plummeted.

Other countries have also suffered. German factory orders have plunged,
in part because Chinese companies that have been hurt by the trade war
have less money to spend on German machinery. That
\href{https://www.nytimes.com/2019/10/01/business/wto-global-trade.html}{threatens
to clamp down on German spending}, which would spill over to affect
other parts of the eurozone.

Major international institutions have also slashed their expectations
for global growth. In a report last week, the World Trade Organization
halved its forecast for growth in the global trade of goods to only 1.2
percent during 2019, in what would be the weakest year since 2009.

\href{https://www.worldbank.org/en/news/speech/2019/10/07/driving-growth-from-the-ground-up}{In
a speech on Monday} in Montreal, David Malpass, the president of the
World Bank, said the institution would also be lowering its most recent
forecast for global economic growth.

In June, the World Bank said that it expected the global economy to grow
by 2.6 percent in 2019, the slowest pace in three years. But growth this
year has slowed further as a result of Britain's potential exit from the
European Union, a recession in Europe and global trade uncertainty, Mr.
Malpass said.

In her address, Ms. Georgieva warned that global trade growth has come
to a near standstill, weighing on global manufacturing and investment.
The slowdown threatens to spill over to the service sector and consumer
behavior, as well, she said.

She added that, even if growth picked up in 2020, trade tensions were
leading to changes that might last a generation. Supply chains have been
broken as companies that once manufactured products in China have tried
to find alternatives in order to avoid the tariffs. And a broader fight
between the United States and China over which country will dominate
technologies like 5G is creating ``a `digital Berlin Wall' that forces
countries to choose between technological systems,'' she said.

That \$700 billion loss due to the trade war with China translates to
roughly 0.8 percent of global gross domestic product, she said, and
includes more than \$200 billion of direct losses to businesses and
consumers. Much larger secondary effects from a loss of confidence and
disruption to markets will also occur.

``In this scenario, the whole economy of Switzerland disappears,'' Ms.
Georgieva said.

Until recently, much of the American economy had been insulated from the
pain of the trade war. That is because the bulk of the economy is
powered by consumption and services, unlike some smaller countries that
are more exposed to shifting global winds of trade. Trade accounts for
just 27 percent of the American economy, according to World Bank
figures, less than China at 38 percent and Germany at 87 percent.

The Trump administration has argued that the tariffs it began imposing
more than a year ago are having only a limited impact on the American
economy. Officials have blamed any signs of slowdown on weakness
overseas, a strong United States dollar dragging on exports and the
actions of the Federal Reserve, which has maintained tighter monetary
conditions than many global central banks.

``We are still suffering the aftermath of severe monetary restraint in
2018,'' Larry Kudlow, one of the White House's top economic advisers,
said in an interview on Friday on Fox Business. ``And the eurozone in
general is in a virtual recession.''

But several recent studies --- from academics, Wall Street researchers
and the Federal Reserve --- have found varying degrees of economic
damage from Mr. Trump's trade policies, primarily through reduced
business investment.

Steven J. Davis, an economist at the University of Chicago who has built
several indexes to measure policy uncertainty,
\href{http://www.policyuncertainty.com/media/Rising\%20Policy\%20Uncertainty.pdf}{wrote
in a research paper released in August} that Mr. Trump's trade
uncertainty had unsettled trading partners and the global economy,
contributed to depressed investment and become a major source of
volatility in stock markets. Mr. Davis called those developments ``an
extraordinary departure from recent history'' for American trade policy.

In
\href{https://www2.bc.edu/matteo-iacoviello/research_files/TPU_PAPER.pdf}{another
recent study}, Fed economists estimated that trade policy uncertainty
reduced the level of investment in the United States by at least 1
percent in 2018, which equates to several hundred billions of dollars.

Goldman Sachs researchers ran a similar analysis this month, but they
came to a much different conclusion. The trade war, they said in a
research note, was having only ``modest'' effects on investment in the
United States.

In \href{https://www.nber.org/papers/w26353}{a paper published Monday},
Michael Waugh, a professor of economics at the New York University
Leonard N. Stern School of Business, argued that China's retaliatory
tariffs were also having an effect on spending by Americans. The paper
showed that parts of the United States that were most exposed to the
tariffs China had put on American products in response to the trade war
--- including the upper Midwest, the West Coast, and parts of Texas,
Kansas, Oklahoma and Nebraska --- had also seen a drop in consumer
spending on automobiles, as well as job creation.

``The retaliation is having real impacts on the people that are hit by
it,'' Mr. Waugh said.

Trade uncertainty has also factored prominently in the Fed's recent
interest rate cuts. Fed officials reduced borrowing costs in July for
the first time since the Great Recession, then followed that up with a
second move in September.

In both cases, Chair Jerome H. Powell cited slowing global growth and
trade tensions as factors that had put central bank officials on guard
by intensifying risks to the economic outlook. The question now is
whether the central bank will lower rates again, or whether it views its
moves so far as sufficient protection for the economy. Officials have
left the door open to a cut at their late-October meeting without
clearly signaling that one is coming.

``Growth around much of the world has weakened over the past year and a
half, and uncertainties around trade, Brexit and other issues pose risks
to the outlook,'' Mr. Powell said during a speech Tuesday in Denver.

Advertisement

\protect\hyperlink{after-bottom}{Continue reading the main story}

\hypertarget{site-index}{%
\subsection{Site Index}\label{site-index}}

\hypertarget{site-information-navigation}{%
\subsection{Site Information
Navigation}\label{site-information-navigation}}

\begin{itemize}
\tightlist
\item
  \href{https://help.nytimes.com/hc/en-us/articles/115014792127-Copyright-notice}{©~2020~The
  New York Times Company}
\end{itemize}

\begin{itemize}
\tightlist
\item
  \href{https://www.nytco.com/}{NYTCo}
\item
  \href{https://help.nytimes.com/hc/en-us/articles/115015385887-Contact-Us}{Contact
  Us}
\item
  \href{https://www.nytco.com/careers/}{Work with us}
\item
  \href{https://nytmediakit.com/}{Advertise}
\item
  \href{http://www.tbrandstudio.com/}{T Brand Studio}
\item
  \href{https://www.nytimes.com/privacy/cookie-policy\#how-do-i-manage-trackers}{Your
  Ad Choices}
\item
  \href{https://www.nytimes.com/privacy}{Privacy}
\item
  \href{https://help.nytimes.com/hc/en-us/articles/115014893428-Terms-of-service}{Terms
  of Service}
\item
  \href{https://help.nytimes.com/hc/en-us/articles/115014893968-Terms-of-sale}{Terms
  of Sale}
\item
  \href{https://spiderbites.nytimes.com}{Site Map}
\item
  \href{https://help.nytimes.com/hc/en-us}{Help}
\item
  \href{https://www.nytimes.com/subscription?campaignId=37WXW}{Subscriptions}
\end{itemize}
