Sections

SEARCH

\protect\hyperlink{site-content}{Skip to
content}\protect\hyperlink{site-index}{Skip to site index}

The Jean-Georges Recipe for Restaurants

\url{https://nyti.ms/33GYsAn}

\begin{itemize}
\item
\item
\item
\item
\item
\item
\end{itemize}

\includegraphics{https://static01.nyt.com/images/2019/10/20/magazine/20mag-Jean-Georges-image1/20mag-Jean-Georges-image1-articleLarge.jpg?quality=75\&auto=webp\&disable=upscale}

Feature

\hypertarget{the-jean-georges-recipe-for-restaurants}{%
\section{The Jean-Georges Recipe for
Restaurants}\label{the-jean-georges-recipe-for-restaurants}}

In the era of the auteur chef, Jean-Georges Vongerichten has figured out
how to create high-end restaurants by the dozen. How --- and why ---
does he do it?

Vongerichten eating breakfast at his namesake restaurant.Credit...Philip
Montgomery for The New York Times

Supported by

\protect\hyperlink{after-sponsor}{Continue reading the main story}

By Christopher Cox

\begin{itemize}
\item
  Oct. 17, 2019
\item
  \begin{itemize}
  \item
  \item
  \item
  \item
  \item
  \item
  \end{itemize}
\end{itemize}

**O**n Monday, May 13, Jean-Georges Vongerichten got into a car outside
his apartment in the West Village and asked to be taken to the airport.
It would have been an odd time to leave town: The next day he was
opening a new restaurant, the Fulton, in Lower Manhattan, on the
waterfront facing Brooklyn. But Vongerichten wasn't flying anywhere. He
was going to check in on another restaurant, this one opening on
Wednesday, inside the new TWA Hotel at J.F.K.

Opening two restaurants back to back, on consecutive days, would be
impressive for a Chipotle or an In-N-Out Burger. It's absolutely
unheard-of for a fine-dining chef like Vongerichten. It also wasn't part
of the plan. The two openings had been years in the making, both tied up
in larger redevelopment projects that the chef had no control over, so
he could do little but watch as the deadlines slowly converged on each
other: The opening date for the Fulton kept getting pushed back, while
the other, for the Paris Café, didn't budge. As late as mid-April,
Vongerichten still thought he would have a few days' buffer between
them, but then that, too, disappeared.

The 62-year-old Vongerichten looked grumpy, or whatever grumpy turns
into when it's deployed on the face of a man whose default mode is glee.
He squirmed in his seat and kept glancing out the window distractedly.
The developers of the
\href{https://www.nytimes.com/2019/07/01/travel/twa-hotel-jfk.html}{TWA
Hotel} had only turned the Paris Café kitchen over to Vongerichten the
day before, which was ridiculously late. At the Fulton, the kitchen was
ready six weeks before opening, and Vongerichten and his team had been
training nonstop since then. The goal for both restaurants was to stage
an opening night that felt like nothing of the sort, as if the
restaurant had been up and running perfectly for months. At this point
it looked as if only the Fulton would make it. ``It's a massive
pressure,'' Vongerichten said.

\href{https://www.nytimes.com/2019/07/30/dining/the-fulton-review-pete-wells-jean-georges.html}{The
Fulton} and
\href{https://www.nytimes.com/2019/05/15/dining/twa-terminal-restaurants-jean-georges-vongerichten.html}{the
Paris Café} would give Vongerichten 14 restaurants in New York and 38
around the world. In the time it took to report and write this article,
he added two more, both in the new Four Seasons Hotel in Philadelphia.
Four restaurants in three months --- it's a lot, but 2019 will probably
still be slower than 2017, when he opened seven, in New York, Los
Angeles, Singapore, São Paulo and London. This pace is intentional. ``My
dream,'' he told me, ``would be to open a restaurant a month and then
get rid of it.''

His friend Eric Ripert, who has spent a career focused on only one
restaurant,
\href{https://www.nytimes.com/2012/05/23/dining/reviews/le-bernardin-in-midtown-manhattan.html}{Le
Bernardin}, says Vongerichten would be ``bored to death'' if he was
stuck with only his New York flagship on Central Park,
\href{https://www.nytimes.com/2014/04/09/dining/restaurant-review-jean-georges-on-the-upper-west-side.html}{Jean-Georges},
to run. Ripert was equally insistent, however, that Jean-Georges
remained one of the best restaurants in the world, despite his
colleague's ever-expanding portfolio of commitments. Even his
detractors, those who think the individual restaurants suffer for the
good of the whole, have trouble hiding their wonder at the juggernaut he
has assembled. One critic, in a review of a relatively early addition to
the Vongerichten culinary universe, asked if the chef had perhaps been
cloned. Vongerichten himself credits it all to ``the formula,'' a set of
procedures that he and his team put in place to make all these openings
run as smoothly as possible.

In the car's back seat, one member of that team, Daniel Del Vecchio,
executive vice president of Jean-Georges Management, was taking calls
and typing on a laptop. In addition to Del Vecchio, who hardly leaves
Vongerichten's side, the two most important players for openings are
Gregory Brainin, who leads a sort of commando unit that trains cooks at
Jean-Georges restaurants all over the world, and Lois Freedman, the
president of the company and the only person I saw (regularly) overrule
Vongerichten himself. ``We're a very tight-knit group,'' Del Vecchio
told me. ``Greg Brainin has been with Jean-Georges, what, almost 20
years. I'm 27 years. Lois is more than 30 years.'' All of them started
as cooks and grew into executives as the business grew. They now oversee
5,000 employees in 12 countries. (Facebook, by comparison, had only
3,200 employees when it went public.) Last year, the Jean-Georges group
did \$350 million in total sales.

In the car, Vongerichten took a call from his fish supplier, running
through a list of sea creatures that grew increasingly obscure as he
went down it. He and Del Vecchio then talked about the new menus they
were having printed for the Jean-Georges flagship. They had decided to
jettison the à la carte menu and offer only a six- or 10-course tasting,
each in omnivore and vegetarian versions. Vongerichten called it a
``major change,'' the biggest move he has made since the restaurant
opened in 1997.

\includegraphics{https://static01.nyt.com/images/2019/10/20/magazine/20mag-Jean-Georges-image-02/20mag-Jean-Georges-image-02-articleLarge.jpg?quality=75\&auto=webp\&disable=upscale}

The mood in the car changed slightly during this conversation, from
nervous energy to something more pensive. Despite Vongerichten's
insistence that he values all 38 of his restaurants equally ---
``they're all my babies'' --- Jean-Georges was still the one that
dominated his imagination, not the firstborn but the doted-upon middle
child, the one who had achieved the greatest success but also required
the most care and attention.

For the first time since we left the West Village, Vongerichten grew
silent. But then he saw the sign for the TWA Hotel, and he yelped with
happiness. ``Look,'' he said, ``there's our staff!'' Pressed up against
the second-floor window of the restaurant was a group of about 40
servers and line cooks. They had just turned on the gas in the kitchen.
The first customers would be arriving in 48 hours.

\textbf{To get a sense} of what Vongerichten has built, without quite
yet understanding how he built it, it might help to learn his breakfast
schedule when he's in New York. He doesn't cook in his (huge,
immaculate) kitchen at home but rather tours his restaurants. On Monday
he eats at the Mercer, in SoHo; on Tuesday he's at the Mark, on the
Upper East Side; on Wednesday he's at
\href{https://www.nytimes.com/2017/07/03/dining/abcv-review-vegetables-restaurant-jean-georges-vongerichten.html}{ABCV},
in the Flatiron district; Thursday is the wild card; and Friday it's
breakfast at Jean-Georges.

What are his 38 restaurants? They don't feel as if they are part of a
chain --- though in a manner of speaking, they are. They aren't hotel
restaurants, though a small number of them are in hotels. And, with the
exception of Jean-Georges, they aren't formal dining rooms, though the
service at each exudes some of the stateliness of the highest-end,
black-tie-and-silver-cloche places. They resemble instead a species of
restaurant that has proliferated with the rise of the middle-class
foodie. Precise but not fussy. Lush but not luxe. Expensive but not
meant for expense accounts. A place you might go on a date night, but
before you leave the house, you have to stare at your closet and ask,
``Can I wear jeans?'' (The answer is yes.)

Most of the restaurants in this class are one-offs, neighborhood joints
created by culinary-school grads and sous chefs who have reached escape
velocity from whatever kitchens they trained in. These are passion
projects --- the realization of a single chef's vision, now that she
finally gets to run her own shop. The bewildering trick that
Vongerichten and his team have pulled off is to replicate these labors
of love, but at scale.

The result is a group of restaurants that feels more like a commonwealth
of independent states than an evil empire. They are inflected by a
single sensibility --- French technique; Asian spices; light, acidic
sauces --- but the joy the Jean-Georges team takes in making each place
new is apparent. ``That's the best part: creating a menu, a concept,''
Vongerichten said. ``The hardest part is to keep it running for the next
20 years.''

The highlight reel is impressive: potato-and-goat-cheese terrine with
arugula juice at Jojo (Vongerichten, Freedman and Del Vecchio go there
for it every Tuesday); scallops with cauliflower and caper-raisin
emulsion at Jean-Georges (a version of which Brainin and Vongerichten
use to test new chefs during the hiring process); tuna and tapioca
pearls with Thai chiles, Sichuan peppercorns, cinnamon, chipotle and
makrut lime at Spice Market (``We've never made food that complicated
again,'' Brainin said); wild-mushroom burdock noodles, tempeh and
pickles at ABCV (reflecting Vongerichten's recent preoccupation with
health and environmental sustainability). The molten-chocolate cake that
took over dessert menus all over the country in the aughts? That was
cribbed from the menu at Lafayette, the first New York restaurant run by
Vongerichten, which he left in 1991.

Image

Vongerichten in the kitchen of The Fulton on Oct. 4.Credit...Philip
Montgomery for The New York Times

The astounding thing about this system is how consistently it works.
It's one thing to build something that looks like a neighborhood gem.
It's another to make it a place that people really want to go, producing
dishes that sway even critics who might otherwise grumble about the
whole towering Jean-Georges edifice. (Pete Wells recently
\href{https://www.nytimes.com/2019/04/23/dining/wayan-restaurant-review.html}{coined
the term} ``Vongerichtenstein'' in a review.) Each new restaurant is
instantly a Best New Restaurant. The achievement might be compared with
James Patterson's regularly being shortlisted for the National Book
Award, except Patterson in this analogy would also have to write seven
books a year (maybe he already does this), while constantly touring the
country to promote every book he'd ever published.

This prolificacy has led to some suspicion about his methods. The
metaphors shift from the realm of art to those of the business world:
Vongerichten has built a factory, a franchise, an assembly line. You
might imagine an enterprise of cut-and-paste, from the lighting in the
dining room to the items on the menu. The reality, however, is weirder,
a space where rigidity and a more freewheeling spirit can mix. At the
Fulton, I saw how granular the formula Vongerichten mentioned could get,
but I also saw improvisation right up to the last minute --- all to
breathe life into that rare, counterintuitive thing: a neighborhood
restaurant, created by a cast of thousands.

\href{https://www.nytimes.com/2019/07/30/dining/the-fulton-review-pete-wells-jean-georges.html}{\emph{{[}Read
Pete Wells's review of the Fulton.{]}}}

\textbf{The Fulton was} born three years ago, in a board room
overlooking New York Harbor. Its parents were Jean-Georges Management
and the Howard Hughes Corporation, the century-old oil, real estate and
aircraft company that has been redeveloping Manhattan's South Street
Seaport. Howard Hughes asked Vongerichten to install a restaurant inside
Pier 17, a boxy mall on stilts that they were building over the East
River. Vongerichten had always wanted to open a seafood restaurant, and
here was a space that couldn't be any closer to the water, steps from
the former Fulton Fish Market. The location determined the concept and
the name.

And, for a while, that's all he had. Construction dragged on, and
Vongerichten refuses to begin planning a menu until a restaurant's
design is locked in. Freedman takes the lead during this phase, choosing
everything from the color of the banquettes (sea-foam green) to the
price point of the water glasses (Pure by Pascale Naessens for Serax ---
a name only Douglas Adams could love, and just over \$7 each wholesale).

Freedman started working for Vongerichten at Lafayette, right out of
cooking school. Then, in 1991, they opened a bistro together called
Jojo. She was all too happy to take over the business side of things and
soon found out she had a knack for it. ``I wanted to be able to grow my
fingernails and dress up,'' she said. ``In the kitchen, both of my arms
all the way up had burn marks.'' I asked if she would have guessed she
would eventually be running 38 restaurants. ``I didn't think past Jojo
at the time,'' she said. ``No one had multiple restaurants. That just
wasn't what people did back then. Chefs were not expanding.''

Expansion was made possible by a shift in the way that Vongerichten did
business. The early restaurants were owned and operated by the
Jean-Georges group. Most of the new restaurants are management deals.
For a percentage of gross revenue and a percentage of net profit,
Jean-Georges Management designs the restaurant and runs the kitchen, but
a partner owns or leases the space, does payroll, pays vendors and,
ultimately, takes home any profit after the licensing fees are paid.
Today these agreements provide three-quarters of Vongerichten's total
revenue. (The group's most profitable restaurant is Prime at the
Bellagio in Las Vegas. It and the Jean-Georges flagship each have
revenues of \$25 million a year, though the cost to run the flagship is
much higher.)

At the Fulton, menu planning began in January, once the construction was
far enough along that Vongerichten and Brainin felt comfortable hiring
an executive chef, who would run things day to day. Normally the
Jean-Georges team would promote a sous chef from the flagship to lead
the new venture, like a plant that is propagated through cuttings. But
this time they plucked a young chef named Noah Poses from the Watergate
Hotel, after a trial tasting that impressed Brainin enough that he
didn't even make Poses audition for Vongerichten himself.

Image

Everything in a Jean-Georges restaurant is measured to the gram, and
deviations are not allowed.Credit...Philip Montgomery for The New York
Times

Poses, Brainin and Vongerichten spent about three months experimenting
in the Jean-Georges kitchen until they had a rough draft of a menu they
were happy with. In March, they moved to the kitchen at the Fulton.
There they continued to refine the dishes, cutting some and adding
others. Anchovies were on the menu (environmentally friendly), and then
they were off (not enough people like them). Snow crab was added to the
risotto. (``Once Jean-Georges tries a better version of something,''
Brainin said, ``there's no selling him on going back.'') Some dishes
were judged too hard to make in a reasonable amount of time, but a
labor-intensive Manhattan clam chowder was included at the last minute
because it's just so popular.

Brainin and Vongerichten also agonized over one particular decision:
whether to put fluke crudo on the menu. In December, Grub Street
published an article with the headline
\href{http://www.grubstreet.com/2018/12/fluke-crudo-scourge-nyc.html}{``Fluke
Crudo Is a Scourge That Must Be Stopped,''} but the chefs decided they
couldn't give up on seafood that was local, sustainable and versatile
just because some critic wanted to plant a flag in the ground. Plus,
Brainin said, ``we love that fish.'' To fend off any complaints of
unoriginality, they had added a fermented habanero vinaigrette to it,
along with Sichuan buds.

Newly hired servers were taught everything from how to properly clear a
plate from a table to the pronunciation of menu items like \emph{cremant
de Bourgogne}. ``I don't mind the knife to be a little crooked on the
table, but the person must have a personality, and they must be able to
sell,'' Vongerichten said. Freedman told me she likes hiring actors as
servers because they can memorize long blocks of text.

Once Poses brought his four sous chefs on board in April, they could
begin the most important part of the Jean-Georges formula: simulating a
real dinner service as early and often as possible. The first of these
daily mock services was offered to just 20 employees, but eventually the
team opening the Fulton would pull staff members from the corporate
office, from the Howard Hughes Corporation and from their vendors in
order to fill the restaurant. The diners were given menus, but their
choices were already highlighted for them. Otherwise, Vongerichten said,
everyone would order the lobster and the kitchen wouldn't be properly
tested.

After each of these meals, Brainin, Vongerichten and Poses would take
the menu they had planned and tweak it, dish by dish. Or, more
precisely, gram by gram: Everything in a Jean-Georges restaurant is
measured to the gram, and deviations are not allowed. ``We make sure
that we test, we test, we test and test again,'' Vongerichten said.

At the final mock service, one week before opening, I watched as a line
cook prepared a kale salad. Brainin quizzed him on the number of grams
of olive oil, of kale leaves, of Parmesan, and the cook knew each one by
heart and without hesitation. The partly assembled salad was placed on a
scale, and the cook shaved Parmesan onto it until he hit the desired
number. Later, after eating an entire bowl of tagliatelle with clams,
Brainin announced that the recipe needed six more grams of olive oil.
``No other kitchen runs this way,'' he said. ``Even if they say they
do.''

I asked one of the culinary trainers working under Brainin if the cooks
ever objected to the rigidness of all this gram-counting. ``It sounds
tedious,'' he said, ``but you learn to respect the ingredients and the
dish.'' Obeying the scales was like obeying the rules of a sonnet --- a
limitation that allowed for almost unlimited artistry. Vongerichten said
it was also a clear-cut way to ensure that, even if he wasn't cooking in
all 38 of his kitchens, the dishes would still be true to his vision,
without any unhelpful improvisations by local cooks. The only other way
to achieve the same end would be to radically downsize: ``I would have a
counter with seven seats. I cook, I serve you and I clean. That would be
J.G. 100 percent.''

Image

Jean-Georges Vongerichten attributes the success of his culinary empire
to ``the formula,'' a set of procedures that he and his team put in
place to make all the openings run as smoothly as
possible.Credit...Philip Montgomery for The New York Times

\textbf{If there is} a fault line in the Jean-Georges system as it's
currently constructed, it resides within the namesake restaurant itself.
The 2018 edition of the Michelin Guide downgraded the restaurant from
three stars to two --- the first time Jean-Georges hadn't earned the top
ranking since Michelin started covering New York. ``That was a sad day
for us,'' Freedman said. ``I was sad for him, because he is a chef who's
always in his restaurants. Even though he's really busy, he's always in
his restaurants working.''

Hidden in that defense is a problem that has been haunting Vongerichten
and his team. Is it even possible to run a three-star restaurant like
Jean-Georges and a globe-spanning corporation at the same time? The
first is meant to offer a once-in-a-lifetime experience, while the
second depends on being able to take that experience and repackage it
for different audiences, cuisines and budgets. To find someone able to
do both is incredibly rare, as if Leonardo da Vinci were able to produce
both ``The Last Supper'' and ``Last Supper'' tote bags. Most of
Vongerichten's peers don't even try: The median number of restaurants
for a three-Michelin-star chef in the United States is two.

If Vongerichten didn't love both equally --- the empire and its namesake
--- his choice would be easy. Only the spinoffs earn him any money. But
he started his career as a teenage apprentice in a three-star kitchen,
and that rarefied world maintains an unshakable grip on his imagination.
Thus, in a summer dominated by the demands of the Fulton, the Paris Café
and the new restaurants in Philadelphia, Vongerichten was forging ahead
with the new menu for Jean-Georges. His team had already contacted
Michelin and asked it to hold off making its determination for the next
edition until reviewers had tried it. It was time, he said, to ``claim
our status again.''

At the same moment, clone-world Vongerichten was at the Fulton,
preparing for the final test before opening: two friends-and-family
meals. Guests could now pick whatever they wanted off the menu. They
could even, like living, breathing people, make special requests and
send things back and otherwise be pains in the neck. At 6 o'clock, I
spotted Vongerichten wandering around, on and off his phone. I asked him
if he felt ready, and he said: ``Yes, it's time. We had plenty of
training.'' He seemed a little nervous. A few famous people, called PXes
(\emph{personnes extraordinaires}) by the staff, showed up. One table
near the front window was wobbling.

I asked Poses what he could learn tonight that he didn't already know.
With 105 meals, he said, the most they've done yet, they would learn
which parts of the menu create bottlenecks: ``The dish might be great,
but is it feasible for a cook to turn out a hundred of them?'' he said.
During the mock services, the runners kept getting backed up near the
pass-through to the kitchen. Brainin had shown me a long list of orders
for a single table, some hot, some cold, some quick to prepare and some
requiring a nonnegotiably long cook time. The best kitchens will figure
out how to prepare everything to come together at the right moment. The
Fulton hadn't gotten there yet, so runners waited with half-full trays.

Vongerichten was right: Everyone ordered lobster. The Fulton served 65
of them before the first friends-and-family meal was over. But if there
was one dish they were most excited about, something that was meant to
be both a showstopper and deeply familiar, it was the sea bass en
croûte. This was a whole sea bass for two, head on, served underneath a
flaky crust. ``It's the one,'' Vongerichten said. ``It's a classic that
nobody else is doing in town.'' He called it a fourth-generation dish:
Fernand Point, author of ``Ma Gastronomie,'' developed it at La Pyramide
in France, passed it on to Louis Outhier, who taught it to Vongerichten
at L'Oasis near Cannes, who taught it to Poses.

The sous chefs put one up on the pass-through and Vongerichten, Freedman
and I followed it upstairs to a table of four. The pastry had been
etched with the tip of a paring knife and painted in egg wash to give it
fishy definition: scales, eyes, spines in the fins. The crust was cut
tableside with scissors and delicately placed on the edge of the
platter. The skin was peeled back, and the fillets were moved to a
plate, deboned, reunited with the pastry and served with some tomatoes
and hollandaise. The whole ceremony felt both formal and whimsical and
was done with unbelievable delicacy. ``It's beyond,'' Freedman said.

Image

Vongerichten at his restaurant ABC Cocina in Manhattan. ``I went into
this business because I love to pamper and tend to people,'' he
said.Credit...Philip Montgomery for The New York Times

By 9 o'clock, the downstairs had mostly emptied, but the upstairs had
become almost rowdy. Freedman said that they didn't like to invite
restaurant people to friends and family, that it was more intimate than
that. She wasn't eating yet: Del Vecchio and Vongerichten would probably
sit down with her around 11. In the meantime, Del Vecchio claimed a
place at the bar, where he was trying to solve a seating problem at one
of the restaurants uptown via a WhatsApp chat with the staff there.
While he was typing, he got a text from Singapore congratulating
everyone back in New York on the impending opening of the Fulton.

\textbf{The first paying} customers in the history of the Fulton arrived
at 5:30 on May 14. They were greeted by hostesses, had their coats taken
and were shown to their seats. They weren't friends or family or PXes.
They had made their reservation using an app. Poses gave a
start-of-service speech to his staff, stressing the importance of moving
tickets along quickly to the runners, and then the first orders began to
come in. I asked if he was nervous. ``I go into every service with a
certain level of anxiety,'' he said, ``but I don't think it's
necessarily an unhealthy amount of anxiety. It's a normal chef feeling,
the anxiety.''

Poses had opened a restaurant before, a place called the Mildred in
Philadelphia. He started as a cook there, in 2012, and was eventually
promoted to \emph{chef de cuisine}. The restaurant didn't last: Business
was inconsistent, and the employees didn't always work together the way
they should. ``Maybe part of that was due to not setting up those
systems and training staff properly,'' Poses said. The experience made
him a big believer in the Vongerichten way of opening a restaurant.
``Look at this room,'' he said. ``Look at the support. If I was doing
this by myself, I would probably have no hair and would be shaking.''

At that moment, Vongerichten arrived. ``It feels good in here,'' he
said. He noticed a young woman eating alone at the bar and wondered if
it was perhaps Hannah Goldfield, the food critic from The New Yorker. It
was not, and I told him so, but he didn't believe me. He ran over to the
pass-through and grabbed a printout with headshots of prominent
restaurant critics, including Goldfield, Pete Wells of The Times and
Adam Platt of New York. He showed it to me and pointed to Goldfield,
then looked again at the woman at the bar. O.K., he admitted, false
alarm.

We watched as Brainin demonstrated proper plating of the kampachi, with
mounds of radish sitting atop the fish, to one cook. Freedman was at the
hostess stand, busily solving a seating problem. Five men who lived in
the neighborhood wanted a table but didn't have a reservation. They
handed Freedman \$100 (for the staff, they said), and she said she would
see what she could do. After a minute, during which she added the money
to the staff tip pool, she told the men she could seat them. They were
ecstatic.

There was no chance that the first dinner at the Paris Café the next day
would be nearly this seamless, but it was easy to imagine how much worse
it would be if the Jean-Georges team weren't running it. Shut out of the
TWA Hotel, the cooks had been able to practice at ABC Kitchen and the
Mercer and Jean-Georges itself. On opening day, the culinary trainers
would help the greener members of the line. And they would all be
presided over by the gram and the scale. Tomorrow, the Paris Café would
probably be the worst of Vongerichten's 38 restaurants, but it would
still be one of the best restaurants in the city. ``We have it down to a
science with our team, with Lois and Greg and Danny and everybody,''
Vongerichten said. ``We know how to put it all together.''

In certain moods, Vongerichten will talk wistfully about the simpler
days of having just one restaurant to run, when all he had to worry
about was Lafayette or Jojo. Downsizing, if he could ever do it, could
also provide the quickest path back to that third Michelin star: Critics
want ceaseless innovation from a chef, but they also reward something
closer to asceticism. The solitary genius, presiding over the counter
with seven seats. It's a more appealing story than the chef who can open
seven restaurants in a year.

Vongerichten's dilemma is that the drive that made Lafayette and
Jean-Georges great is the same one that made it impossible to stop with
just one or two restaurants. It's the desire to say yes to everything,
to solve every problem, to make everybody happy. ``I went into this
business,'' he said, ``because I love to pamper and tend to people.'' If
you had the ability to do that in 18 cities on four continents, rather
than in one restaurant on Central Park, wouldn't you?

It was dark now, and the lights shining on the Brooklyn Bridge reflected
off the water. Brainin stepped back and admired the line at work. It
never ceased to amaze him, he said, watching a kitchen come together.
Three weeks ago, they could barely serve 20 meals without panicking. Now
they were doing 140. ``Opening a restaurant is like having a baby,'' he
said. ``It's a strenuous, arduous, complicated process. You've got to be
sure that the baby can breathe on its own and eat on its own and walk on
its own and grow on its own.'' He would keep coming to the Fulton every
night for a month. ``After that I will be here at least once a week, you
know, forever.''

With that, he turned back to the kitchen, where Poses and Vongerichten
were conferring over a dish. If it wasn't already perfect, it was a gram
or two away at most.

Advertisement

\protect\hyperlink{after-bottom}{Continue reading the main story}

\hypertarget{site-index}{%
\subsection{Site Index}\label{site-index}}

\hypertarget{site-information-navigation}{%
\subsection{Site Information
Navigation}\label{site-information-navigation}}

\begin{itemize}
\tightlist
\item
  \href{https://help.nytimes.com/hc/en-us/articles/115014792127-Copyright-notice}{©~2020~The
  New York Times Company}
\end{itemize}

\begin{itemize}
\tightlist
\item
  \href{https://www.nytco.com/}{NYTCo}
\item
  \href{https://help.nytimes.com/hc/en-us/articles/115015385887-Contact-Us}{Contact
  Us}
\item
  \href{https://www.nytco.com/careers/}{Work with us}
\item
  \href{https://nytmediakit.com/}{Advertise}
\item
  \href{http://www.tbrandstudio.com/}{T Brand Studio}
\item
  \href{https://www.nytimes.com/privacy/cookie-policy\#how-do-i-manage-trackers}{Your
  Ad Choices}
\item
  \href{https://www.nytimes.com/privacy}{Privacy}
\item
  \href{https://help.nytimes.com/hc/en-us/articles/115014893428-Terms-of-service}{Terms
  of Service}
\item
  \href{https://help.nytimes.com/hc/en-us/articles/115014893968-Terms-of-sale}{Terms
  of Sale}
\item
  \href{https://spiderbites.nytimes.com}{Site Map}
\item
  \href{https://help.nytimes.com/hc/en-us}{Help}
\item
  \href{https://www.nytimes.com/subscription?campaignId=37WXW}{Subscriptions}
\end{itemize}
