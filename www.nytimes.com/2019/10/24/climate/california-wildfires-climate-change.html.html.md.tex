Sections

SEARCH

\protect\hyperlink{site-content}{Skip to
content}\protect\hyperlink{site-index}{Skip to site index}

\href{/section/climate}{Climate}\textbar{}A Forecast for a Warming
World: Learn to Live With Fire

\url{https://nyti.ms/2BGMPNH}

\begin{itemize}
\item
\item
\item
\item
\item
\item
\end{itemize}

\href{https://www.nytimes.com/section/climate?action=click\&pgtype=Article\&state=default\&region=TOP_BANNER\&context=storylines_menu}{Climate
and Environment}

\begin{itemize}
\tightlist
\item
  \href{https://www.nytimes.com/2020/07/30/climate/sea-level-inland-floods.html?action=click\&pgtype=Article\&state=default\&region=TOP_BANNER\&context=storylines_menu}{Rising
  Seas}
\item
  \href{https://www.nytimes.com/interactive/2020/climate/trump-environment-rollbacks.html?action=click\&pgtype=Article\&state=default\&region=TOP_BANNER\&context=storylines_menu}{Trump's
  Changes}
\item
  \href{https://www.nytimes.com/interactive/2020/04/19/climate/climate-crash-course-1.html?action=click\&pgtype=Article\&state=default\&region=TOP_BANNER\&context=storylines_menu}{Climate
  101}
\item
  \href{https://www.nytimes.com/interactive/2018/08/30/climate/how-much-hotter-is-your-hometown.html?action=click\&pgtype=Article\&state=default\&region=TOP_BANNER\&context=storylines_menu}{Is
  Your Hometown Hotter?}
\item
  \href{https://www.nytimes.com/newsletters/climate-change?action=click\&pgtype=Article\&state=default\&region=TOP_BANNER\&context=storylines_menu}{Newsletter}
\end{itemize}

\includegraphics{https://static01.nyt.com/images/2019/10/24/climate/24CLI-FIRES1/24CLI-FIRES1-articleLarge.jpg?quality=75\&auto=webp\&disable=upscale}

\hypertarget{a-forecast-for-a-warming-world-learn-to-live-with-fire}{%
\section{A Forecast for a Warming World: Learn to Live With
Fire}\label{a-forecast-for-a-warming-world-learn-to-live-with-fire}}

The Kincade fire, burning in Sonoma County near Geyserville, Calif.,
which burned through 10,000 acres within hours of igniting on
Wednesday.Credit...Josh Edelson/Agence France-Presse --- Getty Images

Supported by

\protect\hyperlink{after-sponsor}{Continue reading the main story}

\href{https://www.nytimes.com/by/thomas-fuller}{\includegraphics{https://static01.nyt.com/images/2018/06/12/multimedia/author-thomas-fuller/author-thomas-fuller-thumbLarge.png}}\href{https://www.nytimes.com/by/kendra-pierre-louis}{\includegraphics{https://static01.nyt.com/images/2018/07/16/multimedia/author-kendra-pierre-louis/author-kendra-pierre-louis-thumbLarge.png}}

By \href{https://www.nytimes.com/by/thomas-fuller}{Thomas Fuller} and
\href{https://www.nytimes.com/by/kendra-pierre-louis}{Kendra
Pierre-Louis}

\begin{itemize}
\item
  Published Oct. 24, 2019Updated Oct. 30, 2019
\item
  \begin{itemize}
  \item
  \item
  \item
  \item
  \item
  \item
  \end{itemize}
\end{itemize}

SAN FRANCISCO --- Facing down 600
\href{https://www.nytimes.com/2019/10/28/us/california-fires-getty-kincade-tick-sonoma-county.html}{wildfires}
in the past three days alone, emergency workers rushed to evacuate tens
of thousands of people in Southern
\href{https://www.nytimes.com/2019/10/24/us/california-fires-today.html}{California}
on Thursday as a state utility said one of its major transmission lines
broke near the source of the out-of-control
\href{https://www.nytimes.com/2019/10/24/us/california-fires-today.html}{Kincade}
blaze in Northern California.

The
\href{https://www.nytimes.com/2019/10/24/us/california-fires-today.html}{Kincade
fire}, the largest this week, tore through steep canyons in the wine
country of northern Sonoma County, racing across 16,000 acres within
hours of igniting. Wind gusts pushed the fire through forests like blow
torches, leaving firefighters with little opportunity to stop or slow
down the walls of flames tromping across wild lands and across highways
overnight.

And north of Los Angeles, 50,000 people were evacuated as strong winds
swept
\href{https://www.nytimes.com/2019/10/24/us/california-fires-today.html}{fires}
into the canyons of Santa Clarita, threatening many homes.

Aerial footage of the Kincade fire showed homes engulfed in flames
propelled by high winds that could become even stronger in the coming
days. But beyond the destruction, which appeared limited on Thursday to
several dozen buildings, hundreds of thousands of people were affected,
both by the fires and a deliberate blackout meant to prevent them.
Schools and businesses closed and thousands of people evacuated their
homes.

All this is happening after three straight years of record-breaking
fires that researchers say are likely to continue in a warming world and
which raise an important question: How to live in an ecosystem that is
primed to burn?

``I think the perception is that we're supposed to control them. But in
a lot of cases we cannot,'' said John Abatzoglou, an associate professor
at the University of Idaho. ``And that may allow us to think a little
bit differently about how we live with fire. We call it wildfire for
reason --- it's not domesticated fire.''

According to the
\href{https://www.nytimes.com/2018/11/23/climate/us-climate-report.html?module=inline}{National
Climate Assessmen}t, the government report that summarizes present and
future effects of a warming climate on the United States, fire is a
growing problem. Climate change will lead to more wildfires nationwide
as hotter temperatures dry out plants, making them easier to ignite.

The total area burned in a single year by wildfires in the United States
has only exceeded 13,900 square miles --- an area larger than the
country of Belgium --- four times since the middle of last century. All
four times have happened this decade,
\href{https://www.esrl.noaa.gov/csd/projects/firex-aq/whitepaper.pdf}{according
to the National Oceanic and Atmospheric Administration and NASA}.

\href{https://www.nytimes.com/section/climate?action=click\&pgtype=Article\&state=default\&region=MAIN_CONTENT_1\&context=storylines_keepup}{}

\hypertarget{climate-and-environment-}{%
\subsubsection{Climate and Environment
›}\label{climate-and-environment-}}

\hypertarget{keep-up-on-the-latest-climate-news}{%
\paragraph{Keep Up on the Latest Climate
News}\label{keep-up-on-the-latest-climate-news}}

Updated July 30, 2020

Here's what you need to know about the latest climate change news this
week:

\begin{itemize}
\item
  \begin{itemize}
  \tightlist
  \item
    \href{https://www.nytimes.com/2020/07/30/climate/bangladesh-floods.html?action=click\&pgtype=Article\&state=default\&region=MAIN_CONTENT_1\&context=storylines_keepup}{Floods
    in}\href{https://www.nytimes.com/2020/07/30/climate/bangladesh-floods.html?action=click\&pgtype=Article\&state=default\&region=MAIN_CONTENT_1\&context=storylines_keepup}{Bangladesh}
    are punishing the people least responsible for climate change.
  \item
    As climate change raises sea levels,
    \href{https://www.nytimes.com/2020/07/30/climate/sea-level-inland-floods.html?action=click\&pgtype=Article\&state=default\&region=MAIN_CONTENT_1\&context=storylines_keepup}{storm
    surges and high tides} are likely to push farther inland.
  \item
    The E.P.A. inspector general plans to investigate whether a rollback
    of fuel efficiency standards
    \href{https://www.nytimes.com/2020/07/27/climate/trump-fuel-efficiency-rule.html?action=click\&pgtype=Article\&state=default\&region=MAIN_CONTENT_1\&context=storylines_keepup}{violated
    government rules}.
  \end{itemize}
\end{itemize}

``There is anger in the community,'' said Michael Gossman, the deputy
county administrator of Sonoma County's office of recovery and
resilience, in an interview this year. In 2017 his California county was
devastated by the Sonoma Complex fires, which killed 24 and burned more
than 170 square miles. Gov. Gavin Newsom said the conditions this week
were analogous to those of 2017.

Many residents in Northern California faced a twin threat on Thursday:
fires, but also the deliberate power outages meant to mitigate the
blazes. Both the Kincade fire and a small fire that ignited Thursday
morning, the Spring fire, occurred in or near areas where the state
utility, Pacific Gas and Electric, had turned off the power.

The fires ``brought out some longer standing institutional issues around
equity,'' Mr. Gossman said. Critics say electricity cutoffs
disproportionately harm low-income people who cannot afford solar and
battery backup systems or gas-based generators, as well as sick and
disabled people who rely on electricity to run life-saving medical
equipment.

\includegraphics{https://static01.nyt.com/images/2019/10/24/climate/24CLI-FIRES2/merlin_163246050_ee8f9d3f-7476-42a4-adca-337f20bf6e09-articleLarge.jpg?quality=75\&auto=webp\&disable=upscale}

Image

Setting the perimeter of a prescribed burn area on Brawley Mountain in
northern Georgia earlier this year.Credit...Dustin Chambers for The New
York Times

Although winds in California were forecast to subside later on Thursday,
officials warned that the extreme winds and dry conditions that create
high risk for fires could return on Sunday. This is why government
agencies are preparing themselves to deal with fires that are
increasingly seen as inevitable.

Prescribed burns, or planned fires, like one set last spring on Brawley
Mountain in Georgia in Southern Appalachia roughly 100 miles north of
Atlanta, are often seen as part of the solution.

The idea that fire could itself be used to help fight fire and restore
ecosystems first gained institutional acceptance in the South. In 1958 a
policy change was made to allow for the first prescribed burn in a
national park, at Everglades National Park in Florida.

For some time, the practice remained anomalous outside of the South. But
within the south, according to Nathan Klaus, a senior wildlife biologist
with the Georgia Department of Natural Resources, even private
landowners would occasionally set smaller, controlled fires on their
property.

Before the era of fire suppression, north Georgia around Brawley
Mountain used to burn roughly every three to five years, according to
Dr. Klaus. Those blazes allowed species that could withstand some fire,
like the longleaf pine, to proliferate and flourish, shaping local
ecosystems.

Some of those fires were caused by natural events like lightning; others
were caused by human activity. The Forest Service notes that Native
Americans used prescribed burns to help with food production. These
smaller fires act as a kind of incendiary rake, clearing out grasses,
shrubs and other plant matter before they can overgrow to become fuel
for bigger, more extreme fires.

Dave Martin, who oversees fire and aviation management in the Forest
Service's southern region, said that a prescribed burn costs about \$30
to \$35 an acre --- versus spending about \$1,000 dollars an acre for
putting out a fire. ``The cost of suppressing a fire is more than a
prescribed burn,'' he said.

It was a combination of forest overgrowth and drought conditions that
helped fuel Tennessee's
\href{https://www.nytimes.com/2016/11/29/us/gatlinburg-tennessee-wildfire.html?searchResultPosition=4}{Great
Smoky Mountains Fires} in 2016, which killed at least 14 people. Several
fires burned across eight southeastern states that year, the same year
Kansas experienced the largest wildfire in its history to date. That
blaze, the Anderson Creek prairie fire, which also affected Oklahoma,
blackened some 625 square miles.

The 2016 wildfires also allowed researchers to compare fire intensity
between areas that had undergone a prescribed burn and those that had
not. The fires in areas that had undergone prescribed were less intense.
``It went from a 20- to 30-foot breaking front,'' said Dr. Klaus in
reference to the height of the leading edge of the blaze on wild lands
that had not burned, ``to two to three feet.''

Reintroducing fire to the land is more complex than lighting a match.
You cannot burn where people live, for example. But nationwide, housing
near wild lands
\href{https://www.nytimes.com/2018/11/15/climate/california-fires-wildland-urban-interface.html}{is
the fastest growing land-use type in the United States}. More people are
moving into areas that are more likely to burn, and in some cases they
may oppose prescribed burning.

``Part of doing this work means educating local communities,'' said Mike
Brod, the fire and natural resources staff officer of the
Chattahoochee-Oconee National Forests.

Image

Monitoring the prescribed burn on Brawley Mountain earlier this
year.~Credit...Dustin Chambers for The New York Times

Image

The Kincade Fire ravaging a vineyard in Geyserville,
Calif.Credit...Justin Sullivan/Getty Images

And there are limits to prescribed burning. If conditions are too wet, a
fire won't ignite, but if it's too dry, the fire is hard to contain.
Like Goldilocks, for wild land managers the conditions have to be just
right. This includes not just the wind's speed, which can affect the
spread of a fire, but also its direction.

And once the burn starts, its smoke can travel great distances. Smoke
from last year's California's wildfires not only threw a haze over much
of the state, but transformed sunsets as far away as Washington, D.C. On
Thursday, NOAA warned residents of the Bay Area that ``shifting winds
tomorrow will likely cause the smoke to be directly over much of the
region,'' as a result of the Kincade fire.

So during planned burns great pains have to be taken to make sure that
the smoke is directed away from population centers. ``If the smoke isn't
doing what we want it to do, we'll shut it down,'' said Nick Peters, the
acting district fire management officer for the Chattooga River ranger
district in the Chattahoochee-Oconee National Forests.

The particulates in wildfire smoke are similar to the kind of pollution
that gets released from burning gasoline or coal. Called PM 2.5, the
tiny particles are associated with negative health effects. Out west,
the rise of giant wildfires has worsened air pollution enough to erode
some of the \href{https://www.pnas.org/content/115/31/7901}{air-quality
gains from the Clean Air Act}.

Image

A map of the Chattahoochee National Forest in Georgia. Rangers take
pains to ensure smoke from prescribed burns is directed away from
population centers.Credit...Dustin Chambers for The New York Times

Image

Firefighters lighting a fire ahead of the Kincade fire as a containment
measure on Thursday.Credit...Justin Sullivan/Getty Images

Earlier this year NOAA and NASA launched a
\href{https://esrl.noaa.gov/csd/projects/firex-aq/}{mission} to learn
more about wildfire smoke. The program flew planes into western
wildfires and Midwestern agricultural fires throughout the summer and
into the fall.

A lot of wildfire and climate research is divided into two camps:
observational modelers (who run large computer simulations) and
researchers (who gather observational data using sophisticated monitors)
said Rajan Chakrabarty, an assistant professor at the Washington
University in St. Louis. The goal of the mission was to bridge that gap.

But flying into a fire is not for the weak bellied. As the plane flies
through a blaze, the cabin fills with the smell of smoke evocative of a
barbecue or a campfire. And sampling a fire plume often involves the
kind of rollicking, stomach churning turbulence that commercial flights
go out of their way to avoid.

By taking samples during an active fire, scientists hope to understand
what's in the smoke, and how the chemical makeup changes over time.

``This air is getting blown downwind, so it's going to impact areas
outside of just where the fire was burning,'' said Hannah Halliday, a
researcher at NASA Langley, who also participated in the mission. ``And
we have models for how emissions change, but we want to make sure that
we have that chemistry right, and the physics right.''

The hope is that, over the long term, the smoke models will be as
sophisticated as weather models, and can let people know well in advance
when they'll need to prepare for smoke, even if they are relatively far
from the site of a fire.

Image

New growth sprouting three months after a prescribed burn in Tallulah
Gorge State Park in northeastern Georgia.Credit...Dustin Chambers for
The New York Times

For more news on climate and the environment,
\href{https://twitter.com/nytclimate}{follow @NYTClimate on Twitter}.

Thomas Fuller reported from San Francisco. Kendra Pierre-Louis reported
from Brawley Mountain, Ga., and Idaho.

Advertisement

\protect\hyperlink{after-bottom}{Continue reading the main story}

\hypertarget{site-index}{%
\subsection{Site Index}\label{site-index}}

\hypertarget{site-information-navigation}{%
\subsection{Site Information
Navigation}\label{site-information-navigation}}

\begin{itemize}
\tightlist
\item
  \href{https://help.nytimes.com/hc/en-us/articles/115014792127-Copyright-notice}{©~2020~The
  New York Times Company}
\end{itemize}

\begin{itemize}
\tightlist
\item
  \href{https://www.nytco.com/}{NYTCo}
\item
  \href{https://help.nytimes.com/hc/en-us/articles/115015385887-Contact-Us}{Contact
  Us}
\item
  \href{https://www.nytco.com/careers/}{Work with us}
\item
  \href{https://nytmediakit.com/}{Advertise}
\item
  \href{http://www.tbrandstudio.com/}{T Brand Studio}
\item
  \href{https://www.nytimes.com/privacy/cookie-policy\#how-do-i-manage-trackers}{Your
  Ad Choices}
\item
  \href{https://www.nytimes.com/privacy}{Privacy}
\item
  \href{https://help.nytimes.com/hc/en-us/articles/115014893428-Terms-of-service}{Terms
  of Service}
\item
  \href{https://help.nytimes.com/hc/en-us/articles/115014893968-Terms-of-sale}{Terms
  of Sale}
\item
  \href{https://spiderbites.nytimes.com}{Site Map}
\item
  \href{https://help.nytimes.com/hc/en-us}{Help}
\item
  \href{https://www.nytimes.com/subscription?campaignId=37WXW}{Subscriptions}
\end{itemize}
