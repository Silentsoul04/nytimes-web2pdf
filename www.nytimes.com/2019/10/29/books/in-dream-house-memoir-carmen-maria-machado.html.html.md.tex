Sections

SEARCH

\protect\hyperlink{site-content}{Skip to
content}\protect\hyperlink{site-index}{Skip to site index}

\href{https://www.nytimes.com/section/books}{Books}

\href{https://myaccount.nytimes.com/auth/login?response_type=cookie\&client_id=vi}{}

\href{https://www.nytimes.com/section/todayspaper}{Today's Paper}

\href{/section/books}{Books}\textbar{}`In the Dream House' Recounts an
Abusive Relationship Using Dozens of Genres

\url{https://nyti.ms/2WlXSWc}

\begin{itemize}
\item
\item
\item
\item
\item
\item
\end{itemize}

Advertisement

\protect\hyperlink{after-top}{Continue reading the main story}

Supported by

\protect\hyperlink{after-sponsor}{Continue reading the main story}

\href{/column/books-of-the-times}{Books of The Times}

\hypertarget{in-the-dream-house-recounts-an-abusive-relationship-using-dozens-of-genres}{%
\section{`In the Dream House' Recounts an Abusive Relationship Using
Dozens of
Genres}\label{in-the-dream-house-recounts-an-abusive-relationship-using-dozens-of-genres}}

By \href{https://www.nytimes.com/by/parul-sehgal}{Parul Sehgal}

\begin{itemize}
\item
  Published Oct. 29, 2019Updated Nov. 4, 2019
\item
  \begin{itemize}
  \item
  \item
  \item
  \item
  \item
  \item
  \end{itemize}
\end{itemize}

\includegraphics{https://static01.nyt.com/images/2019/10/30/books/30bookmachado1/30bookmachado1-articleLarge.jpg?quality=75\&auto=webp\&disable=upscale}

Buy Book ▾

\begin{itemize}
\tightlist
\item
  \href{https://www.amazon.com/gp/search?index=books\&tag=NYTBSREV-20\&field-keywords=In+the+Dream+House+Carmen+Maria+Machado}{Amazon}
\item
  \href{https://du-gae-books-dot-nyt-du-prd.appspot.com/buy?title=In+the+Dream+House\&author=Carmen+Maria+Machado}{Apple
  Books}
\item
  \href{https://www.anrdoezrs.net/click-7990613-11819508?url=https\%3A\%2F\%2Fwww.barnesandnoble.com\%2Fw\%2F\%3Fean\%3D9781644450031}{Barnes
  and Noble}
\item
  \href{https://www.anrdoezrs.net/click-7990613-35140?url=https\%3A\%2F\%2Fwww.booksamillion.com\%2Fp\%2FIn\%2Bthe\%2BDream\%2BHouse\%2FCarmen\%2BMaria\%2BMachado\%2F9781644450031}{Books-A-Million}
\item
  \href{https://bookshop.org/a/3546/9781644450031}{Bookshop}
\item
  \href{https://www.indiebound.org/book/9781644450031?aff=NYT}{Indiebound}
\end{itemize}

When you purchase an independently reviewed book through our site, we
earn an affiliate commission.

All ornate appearances to the contrary, the gothic is a thrifty form.
Its first, and essential, ingredients, according to the scholar Mary Ann
Doane, are simply ``woman plus habitation.'' No special effects needed;
the horror of a woman finding fear where she expected safety is enough
to power an entire genre.

Variations abound. Add a deranged, possibly homicidal house and you have
Shirley Jackson's ``The Haunting of Hill House.'' Throw in some fey,
freaky children and you get ``The Turn of the Screw.'' Merge the house
and the woman --- watch the woman experience \emph{her own body} as a
haunted house, a place of sudden, inexplicable terrors --- and you are
reading the
\href{https://www.nytimes.com/2018/03/05/books/vanguard-books-by-women-in-21st-century.html}{blazingly
talented} Carmen Maria Machado.

Machado's previous book, the short story collection
\href{https://www.nytimes.com/2017/10/04/books/review-her-body-and-other-parties-carmen-maria-machado.html}{``Her
Body and Other Parties''} (2017), a finalist for the National Book
Award, is one of the most original and exuberantly celebrated debuts of
recent years. The stories are louche, mischievous and very queer ---
fairy tales scrambled with fan fiction and body horror. In ``Difficult
at Parties,'' a woman attends a housewarming. She pokes around the
house, exploring, until her host stops her. ``That room is being
renovated,'' she is warned. ``There's no floor. You could go in there,
but you'd go straight down to the cellar.''

Welcome to the House of Machado. Proceed directly into the forbidden
room; enjoy the view as the floor gives way.

Her new book, ``In the Dream House,'' is a memoir of her frightening and
abusive relationship with another woman while in graduate school. It is
a book in shards. Each chapter hews to the conventions of a different
genre: road trip, romance novel, creature feature, lesbian pulp novel,
stoner comedy. The technique borrows a bit from Raymond Queneau's 1947
``Exercises in Style,'' 99 retellings of the same story in different
genres. What could seem gimmicky --- I confess I braced myself at first
--- quickly feels like the only natural way to tell the story of a
couple. What relationship exists in purely one genre? What life?

\emph{{[} This book was one of our most anticipated titles of November.}
\href{https://www.nytimes.com/2019/10/31/books/new-november-books.html}{\emph{See
the full list}}\emph{. {]}}

Machado has described herself as a ``form vampire,'' obsessed with
structures, real and narrative, and with mingling techniques from
fiction and nonfiction. This book is a hive of frenetic experimentation,
tactics and tricks; nested essays on film scholarship, queer villains in
Disney films, ``Star Trek,'' the campaigns addressing domestic violence
in lesbian relationships in the early '80s. It's narrated in the second
person, with Machado addressing her younger self, tenderly and severely.
Each section comes heavily footnoted, indicating the appearance of
traditional folk tale motifs --- taboos, odd coincidences.

Image

Carmen Maria Machado's new memoir, ``In the Dream House,'' details an
abusive relationshipCredit...Art Streiber

There is something anxious, and very intriguing, in the degree of
experimentation in this memoir, in its elaborately titivated sentences,
its thicket of citations. The flurry --- the excess --- feels
deliberate, and summons up the image of the writer holding a ring of
keys, trying each of them in turn to unlock a resistant story, to open a
door she might be hesitant to enter. (\emph{``There's no floor. You
could go in there, but you'd go straight down to the cellar.''})

In the beginning, of course, Machado cannot believe her good fortune.
The woman she falls for (who remains unnamed) has white-blonde hair and
is ``that mix of butch and femme that drives you crazy.'' She is not
Machado's first female lover, but it's the first time Machado has been
loved in \emph{that} way, with ravenous, mutual obsession. ``Sometimes
when you look at your phone, she has sent you something stunningly
filthy, and there is a kick of want between your legs. Sometimes when
you catch her looking at you, you feel like the luckiest person in the
whole world.''

The look of love turns quickly to scrutiny, and then to blame.
``Sometimes when you look at your phone, she has sent you something
stunningly cruel, and there is a kick of fear between your shoulder
blades. Sometimes when you catch her looking at you, you feel like she's
determining the best way to take you apart.''

The woman reveals herself in all her paranoia, possessiveness and fury.
She rages, calls Machado names, throws things at her --- shoes, a
suitcase --- forces her to barricade herself in a bathroom. The woman
retreats and returns later, undressed, warm and seductive, to ask, ``Why
are you crying?\emph{''} Her voice is so concerned, so sweet that ``your
heart splits open like a peach.''

There will be an end to all this --- even a happy ending. Machado lets
her younger self glimpse into the future, at love again, marriage to a
beautiful woman, ``a sun-soaked apartment.'' But she remains bedeviled
by her questions. ``Was she trained to find you, or were you trained to
be found? Was it the fact that you'd already been tenderized like a pork
chop by: never having been properly in love, being told you should be
grateful for anything you get as a fat woman, getting weird messages
that relationships are about fighting and being at odds with each
other?''

This is to say nothing of the questions she imagines others might ask:
``Maybe it was rough, but was it really abusive? What does that mean,
anyway?''

``In the Dream House'' is written into the silence surrounding violence
in queer relationships, the silences around emotional and psychological
abuse. Did lesbians of the past ``hurl inkwells and figurines?'' Machado
asks. ``Did any of them wonder if what had happened to them had any name
at all?'' She begins to cobble together a language for what she has
experienced. What queer abuse feels most like, she realizes, is
homophobia, the same way abuse in heterosexual relationships can feel
like sexism: ``I am doing this because I can get away with it; I can get
away with it because you exist on some cultural margin, some societal
periphery.''

Machado repeatedly returns to the idea of the archive --- the places
where stories are enshrined, entered into an official record. The word
itself refers to a kind of structure; ``archive'' derives from the
ancient Greek \emph{arkheion}, ``house of the ruler.'' There was no such
record for Machado to draw on. ``I toss the stone of my story into a
vast crevice; measure the emptiness by its small sound,'' she writes.
That is, of course, how she begins the book. At its conclusion, what
does she leave us but a library in miniature --- those long-invisible,
long-suppressed stories now culled from every quarter of history, and
explored in every conceivable genre --- a living archive of her own
loving, idiosyncratic design.

Advertisement

\protect\hyperlink{after-bottom}{Continue reading the main story}

\hypertarget{site-index}{%
\subsection{Site Index}\label{site-index}}

\hypertarget{site-information-navigation}{%
\subsection{Site Information
Navigation}\label{site-information-navigation}}

\begin{itemize}
\tightlist
\item
  \href{https://help.nytimes.com/hc/en-us/articles/115014792127-Copyright-notice}{©~2020~The
  New York Times Company}
\end{itemize}

\begin{itemize}
\tightlist
\item
  \href{https://www.nytco.com/}{NYTCo}
\item
  \href{https://help.nytimes.com/hc/en-us/articles/115015385887-Contact-Us}{Contact
  Us}
\item
  \href{https://www.nytco.com/careers/}{Work with us}
\item
  \href{https://nytmediakit.com/}{Advertise}
\item
  \href{http://www.tbrandstudio.com/}{T Brand Studio}
\item
  \href{https://www.nytimes.com/privacy/cookie-policy\#how-do-i-manage-trackers}{Your
  Ad Choices}
\item
  \href{https://www.nytimes.com/privacy}{Privacy}
\item
  \href{https://help.nytimes.com/hc/en-us/articles/115014893428-Terms-of-service}{Terms
  of Service}
\item
  \href{https://help.nytimes.com/hc/en-us/articles/115014893968-Terms-of-sale}{Terms
  of Sale}
\item
  \href{https://spiderbites.nytimes.com}{Site Map}
\item
  \href{https://help.nytimes.com/hc/en-us}{Help}
\item
  \href{https://www.nytimes.com/subscription?campaignId=37WXW}{Subscriptions}
\end{itemize}
