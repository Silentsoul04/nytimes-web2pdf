Sections

SEARCH

\protect\hyperlink{site-content}{Skip to
content}\protect\hyperlink{site-index}{Skip to site index}

\href{https://www.nytimes.com/section/business}{Business}

\href{https://myaccount.nytimes.com/auth/login?response_type=cookie\&client_id=vi}{}

\href{https://www.nytimes.com/section/todayspaper}{Today's Paper}

\href{/section/business}{Business}\textbar{}China's TikTok Blazes New
Ground. That Could Doom It.

\url{https://nyti.ms/2WIQZyg}

\begin{itemize}
\item
\item
\item
\item
\item
\item
\end{itemize}

Advertisement

\protect\hyperlink{after-top}{Continue reading the main story}

Supported by

\protect\hyperlink{after-sponsor}{Continue reading the main story}

The New New World

\hypertarget{chinas-tiktok-blazes-new-ground-that-could-doom-it}{%
\section{China's TikTok Blazes New Ground. That Could Doom
It.}\label{chinas-tiktok-blazes-new-ground-that-could-doom-it}}

The rise of the seemingly innocuous app is forcing Americans to consider
a world influenced by a Chinese-backed social media network.

\includegraphics{https://static01.nyt.com/images/2019/11/05/business/05newworld/05newworld-articleLarge.jpg?quality=75\&auto=webp\&disable=upscale}

\href{https://www.nytimes.com/by/li-yuan}{\includegraphics{https://static01.nyt.com/images/2019/12/04/reader-center/author-li-yuan/author-li-yuan-thumbLarge.png}}

By \href{https://www.nytimes.com/by/li-yuan}{Li Yuan}

\begin{itemize}
\item
  Nov. 5, 2019
\item
  \begin{itemize}
  \item
  \item
  \item
  \item
  \item
  \item
  \end{itemize}
\end{itemize}

\href{https://cn.nytimes.com/business/20191105/tiktok-china-bytedance/}{阅读简体中文版}\href{https://cn.nytimes.com/business/20191105/tiktok-china-bytedance/zh-hant/}{閱讀繁體中文版}

American leaders have effectively
\href{https://www.nytimes.com/2019/08/07/business/huawei-us-ban.html}{thrown
Huawei} and
\href{https://www.nytimes.com/2019/10/08/business/china-human-rights-technology-xinjiang.html}{a
handful of Chinese surveillance technology companies} out of the
country, warning darkly of the national security and privacy threats of
installing Made-in-China products into sensitive parts of the nation's
electronic infrastructure.

Now they have cast their fearful gaze on a new Chinese target: the
dancing and singing teens and tweens of TikTok.

A secretive federal panel with a national security focus is reviewing
\href{https://www.nytimes.com/2017/11/10/business/dealbook/musically-sold-app-video.html}{the
purchase of TikTok two years ago} by a Chinese company called Bytedance,
\href{https://www.nytimes.com/2019/11/01/technology/tiktok-national-security-review.html}{The
New York Times and others} reported last week. Three senators
\href{https://www.cotton.senate.gov/?p=press_release\&id=1239}{have
asked the Trump administration} to review potential national security
and privacy threats posed by the app, warning that Bytedance
\href{https://www.rubio.senate.gov/public/index.cfm/2019/10/rubio-requests-cfius-review-of-tiktok-following-reports-of-chinese-censorship}{could
strip out content that displeases the Communist Party}, such as videos
of pro-democracy protests in Hong Kong.

TikTok denies that it censors political content. Videos supporting the
protesters and declaring ``Reclaim Hong Kong!'' can be found on the app.

Still, TikTok's popularity --- if you're the parent of a teenager, that
teenager probably has it on her phone --- blazes uncharted territory.
Its unexpected rise is forcing Americans for the first time to consider
living in a world influenced by a Chinese-backed social media network.

That invites the public to take a closer look at Bytedance. It is not an
arm of the Communist Party. However, it owes much of its success to its
ability to navigate Beijing's political currents, as well as its skills
in delivering harmless fluff that could pass muster with censors.

In fact, its founder, Zhang Yiming,
\href{https://www.nytimes.com/2018/04/11/business/zuckerberg-facebook-congress.html}{has
echoed the comments of Facebook's Mark Zuckerberg} that he runs a
technology company, not a media company. And yet that hasn't stopped
Facebook from influencing events around the world,
\href{https://www.nytimes.com/2019/10/21/technology/facebook-disinformation-russia-iran.html}{with}
\href{https://www.nytimes.com/2017/11/01/us/politics/russia-2016-election-facebook.html}{profound}
and troubling
\href{https://www.nytimes.com/2018/04/21/world/asia/facebook-sri-lanka-riots.html}{consequences}.

No other Chinese tech company has succeeded on the global internet like
Bytedance. The country's vibrant social media platforms and viral video
apps have had little success elsewhere beyond the Chinese diaspora. In a
time when some in Washington are trying to pry apart the close economic
ties between the United States in China, the two countries already live
in different worlds when it comes to cyberspace.

TikTok, however, is ascending on the global internet scene just as the
\href{https://www.nytimes.com/2019/05/20/business/huawei-trump-china-trade.html?rref=collection\%2Fbyline\%2Fli-yuan}{technological
Iron Curtain} is drawing. It has had nearly 1.5 billion downloads
globally and 122 million in the United States, according to the data
firm Sensor Tower.

That alarms some in Washington. In a letter to American intelligence
officials, Senators Chuck Schumer and Tom Cotton said the TikTok
platform was ``a potential target of foreign influence campaigns like
those carried out during the 2016 election on U.S.-based social media
platforms.''

They and others point to
\href{https://www.theguardian.com/technology/2019/sep/25/revealed-how-tiktok-censors-videos-that-do-not-please-beijing}{recent
articles} claiming or
\href{https://www.washingtonpost.com/technology/2019/09/15/tiktoks-beijing-roots-fuel-censorship-suspicion-it-builds-huge-us-audience/}{suggesting}
that TikTok strips out videos of Hong Kong's protests.

TikTok
\href{https://newsroom.tiktok.com/en-us/statement-on-tiktoks-content-moderation-and-data-security-practices}{disputes
the allegations}, saying that it stores its American user data locally
and doesn't ``remove content based on sensitivities related to China.''

``We have never been asked by the Chinese government to remove any
content, and we would not do so if asked,'' it said. ``Period.''

That statement doesn't cover the breadth of the pressure Beijing puts on
media companies in China. Rather than outright censoring content,
Chinese officials often set broad guidelines for media companies and
then let them censor themselves.

TikTok at times has avoided uncomfortable topics entirely. A look at the
guidelines on the app when downloaded in Hong Kong shows that they
forbid politics-related content and comments.

TikTok updated the guidelines after it was asked about them by The
Times, saying that they were supposed to be taken down in May and
resurfaced by error. Its Hong Kong content moderation team is partly
based in mainland China but is moving to Hong Kong, and it will follow
the same rules that the company sets everywhere else, TikTok said.

All that said, TikTok's critics have yet to offer evidence that Beijing
is using TikTok to spread propaganda to young minds, or that it is
misusing user data. That may not help its fortunes in the United States:
The American government has never offered proof that Huawei's equipment
is a security threat, yet its equipment has been essentially banned from
telecommunications networks anyway.

Mr. Zuckerberg,
\href{https://www.nytimes.com/2019/11/03/technology/tiktok-facebook-youtube.html}{whose
company is busily trying to emulate TikTok}, argues that the
Chinese-owned app represents a competition of values.

``Until recently, the internet in almost every country outside China has
been defined by American platforms with strong free expression values.
There's no guarantee these values will win out,'' he said in
\href{https://www.washingtonpost.com/technology/2019/10/17/zuckerberg-standing-voice-free-expression/}{a
recent speech} at Georgetown University, where he specifically cited
TikTok.

Of course, his company once tried to develop software that
\href{https://www.nytimes.com/2016/11/22/technology/facebook-censorship-tool-china.html}{would
allow potential Chinese partners to block content they didn't like}. But
the comments strike at a broader truth: To do business on the Chinese
internet, any company there has to abide by the government's wishes,
whether it wants to or not.

Bytedance is no different. It was founded in 2012 by Mr. Zhang when he
was 29. A geeky programmer, Mr. Zhang says the day in second grade when
he received a Tetris hand-held game device was one of the happiest of
his life. In college, he was known for his skills in repairing personal
computers.

Some people in China compare him to Mr. Zuckerberg. Like the Facebook
boss, Mr. Zhang had said machines do a better job than people at
distributing content. When Bytedance introduced Toutiao, a news
aggregation app that became hugely popular in China, it was run by
software instead of an editor in chief. Mostly, it delivered headlines
to users based on what they had clicked before.

``I can't accurately decide whether something is good or bad, highbrow
or lowbrow,'' Mr. Zhang told a Chinese business magazine in 2016. ``I
may have my judgment, but I don't want to impose it on Toutiao.''

For many users, that system delivered a steady stream of pablum,
depending on what they had clicked before: bikini photos, silly videos,
pet memes. Soon a backlash developed.

``Zhang Yiming and his engineers train the machine to understand your
heart,'' a blogger, He Jiayan, wrote. ``At the same time they're
training you so you'll be addicted to the machine.''

The TikTok spokesman said the app and other Bytedance apps had a time
management feature that allowed users and parents to set limits of 40,
60 or 120 minutes.

The algorithm-driven approach has worked well for Bytedance. By the end
of 2016, Toutiao became China's No. 2 app after the messaging app
WeChat, based on the time users spent on it per day, according to
QuestMobile, a Chinese data firm. Last year, it raised a new round of
funding from investors including SoftBank that valued it at \$75
billion, making it one of the most valuable start-ups in the world.

Toutiao also attracted the attention of
\href{https://www.nytimes.com/2014/12/02/world/asia/gregarious-and-direct-chinas-web-doorkeeper.html}{Lu
Wei}, China's heavy-handed internet czar at the time, according to
industry executives. When other internet companies complained that
Toutiao was stealing their content, one of Mr. Lu's top lieutenants told
them that he was a fan and that they should stop complaining and work
with the company, said two of them. They asked for anonymity because
discussing the work of China's censors in public is tantamount to
corporate suicide.

Bytedance denies that it had a close relationship with Mr. Lu or
benefited from any connection with him. Mr. Lu stepped down from his
post in mid-2016 and was sentenced this year to 14 years in prison for
corruption.

Then the company had its own run-ins with the censors. In April 2018,
Bytedance and a few other star start-ups were punished for running
``unhealthy'' content. The company was ordered to shut down an app
called
\href{https://www.nytimes.com/2018/04/12/business/china-bytedance-duanzi-censor.html}{Neihan
Duanzi} for hosting vulgar jokes and videos.

Mr. Zhang apologized, saying in a letter that he took responsibility for
content that was incompatible with ``core socialist values.'' He vowed
to strengthen party building at Bytedance and announced that he would
enhance the role of the editor in chief and expand the content
moderation team to 10,000 from 6,000. Like other online outlets, Toutiao
also began featuring stories about Xi Jinping, China's leader, at the
top of its feed.

His apology seemed to be well received. Within two weeks, he delivered a
keynote speech at a technology conference hosted by internet regulators.
His topic: Bytedance's global expansion strategy.

Advertisement

\protect\hyperlink{after-bottom}{Continue reading the main story}

\hypertarget{site-index}{%
\subsection{Site Index}\label{site-index}}

\hypertarget{site-information-navigation}{%
\subsection{Site Information
Navigation}\label{site-information-navigation}}

\begin{itemize}
\tightlist
\item
  \href{https://help.nytimes.com/hc/en-us/articles/115014792127-Copyright-notice}{©~2020~The
  New York Times Company}
\end{itemize}

\begin{itemize}
\tightlist
\item
  \href{https://www.nytco.com/}{NYTCo}
\item
  \href{https://help.nytimes.com/hc/en-us/articles/115015385887-Contact-Us}{Contact
  Us}
\item
  \href{https://www.nytco.com/careers/}{Work with us}
\item
  \href{https://nytmediakit.com/}{Advertise}
\item
  \href{http://www.tbrandstudio.com/}{T Brand Studio}
\item
  \href{https://www.nytimes.com/privacy/cookie-policy\#how-do-i-manage-trackers}{Your
  Ad Choices}
\item
  \href{https://www.nytimes.com/privacy}{Privacy}
\item
  \href{https://help.nytimes.com/hc/en-us/articles/115014893428-Terms-of-service}{Terms
  of Service}
\item
  \href{https://help.nytimes.com/hc/en-us/articles/115014893968-Terms-of-sale}{Terms
  of Sale}
\item
  \href{https://spiderbites.nytimes.com}{Site Map}
\item
  \href{https://help.nytimes.com/hc/en-us}{Help}
\item
  \href{https://www.nytimes.com/subscription?campaignId=37WXW}{Subscriptions}
\end{itemize}
