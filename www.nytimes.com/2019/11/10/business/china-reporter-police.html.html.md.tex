Sections

SEARCH

\protect\hyperlink{site-content}{Skip to
content}\protect\hyperlink{site-index}{Skip to site index}

\href{https://www.nytimes.com/section/business}{Business}

\href{https://myaccount.nytimes.com/auth/login?response_type=cookie\&client_id=vi}{}

\href{https://www.nytimes.com/section/todayspaper}{Today's Paper}

\href{/section/business}{Business}\textbar{}Hunting for White Elephants
in China's Coal Country

\url{https://nyti.ms/2Q6kbhE}

\begin{itemize}
\item
\item
\item
\item
\item
\end{itemize}

Advertisement

\protect\hyperlink{after-top}{Continue reading the main story}

Supported by

\protect\hyperlink{after-sponsor}{Continue reading the main story}

TIMES INSIDER

\hypertarget{hunting-for-white-elephants-in-chinas-coal-country}{%
\section{Hunting for White Elephants in China's Coal
Country}\label{hunting-for-white-elephants-in-chinas-coal-country}}

We came looking for signs of government overspending. We left with a
police escort. These days in China, the economy is a sensitive subject.

\includegraphics{https://static01.nyt.com/images/2019/11/11/business/00Insider-Stevenson-1/merlin_163827192_9bada054-0674-46a9-888e-cd11c42cece6-articleLarge.jpg?quality=75\&auto=webp\&disable=upscale}

\href{https://www.nytimes.com/by/alexandra-stevenson}{\includegraphics{https://static01.nyt.com/images/2018/02/20/multimedia/author-alexandra-stevenson/author-alexandra-stevenson-thumbLarge.jpg}}

By \href{https://www.nytimes.com/by/alexandra-stevenson}{Alexandra
Stevenson}

\begin{itemize}
\item
  Nov. 10, 2019
\item
  \begin{itemize}
  \item
  \item
  \item
  \item
  \item
  \end{itemize}
\end{itemize}

\href{https://cn.nytimes.com/business/20191111/china-reporter-police/}{阅读简体中文版}{[}閱讀繁體中文版{]}(
"Read in Traditional Chinese")

\href{https://www.nytimes.com/series/times-insider}{\emph{Times
Insider}} \emph{explains who we are and what we do, and delivers
behind-the-scenes insights into how our journalism comes together.}

RUZHOU, China --- A police officer threatened us with a criminal
investigation. A local official pleaded with us to write a positive
article.

My colleague Cao Li and I came to Ruzhou to hunt for white elephants.
During China's boom times, the local authorities borrowed money to build
these vanity projects --- stadiums, theme parks, highways to nowhere ---
to create jobs. Now China's economy is slowing, and economists warn that
the borrowing has set ticking time bombs that threaten the economy.

Ruzhou, a city of one million in China's coal country, looked as if it
might be one of those places. A source of local government financing had
defaulted. Later, we would learn that the local hospital system was
trying to borrow money from its modestly paid doctors and nurses.

We did not expect local officials to greet us happily. At the same time,
we did not anticipate that our reporting would touch off a strong
response from the Chinese government.

Beijing tries to suppress reporting that it thinks could undermine the
authority of the Communist Party or that deals with sensitive
humanitarian or political subjects. But local economic issues have long
been fair game.

Not anymore. As the economy slows, it has become a sensitive subject,
too.

Six hours after we arrived, on a hot and sticky summer day buzzing with
cicadas, more than a dozen Ruzhou officials appeared out of nowhere to
confront us. They trapped us in a large parking lot where our driver was
waiting. Some were police. Others were plainclothes officials who never
fully identified themselves.

\includegraphics{https://static01.nyt.com/images/2019/11/08/business/-00Insider-Stevenson-videostill/-00Insider-Stevenson-videostill-superJumbo.jpg}

They did not seem to have a game plan. Yet they would not let us leave.
They hovered around us and fussed, sometimes breaking away to confer.
They threatened our driver. They scanned our IDs. They questioned our
motives. They asked to see our phones and to delete our photos.

They had no legal right to hold us there. Then they called in two empty
police vans. Who were we to bicker about the finer details of the law?

When you are a reporter in China, it can be hard to know when you're
crossing a line with the authorities. Foreign news is censored, but
reporters are still given accreditation. The local police sometimes
throw obstacles and threats in the way.

These moments can be simultaneously terrifying and humorous.

Having exhausted all attempts at reason, we called a more senior
official in Beijing. This man's job is to help journalists navigate
these situations. We placed him on speaker phone, and he told the police
that we were not doing anything illegal. Things, it seemed, were looking
up.

And then the police officer shouted back into the phone: ``How am I
supposed to believe you are who you say you are?''

We were right that Ruzhou would be home to some white elephants. A
sprawling athletic compound had been converted to a ``big data'' center.
When we visited, it was largely empty. Inside one sports arena, a
caretaker told us the building was worth more than \$14 million. When he
saw the confused look on our faces, he quickly explained that millions
of dollars of invisible technology was coursing through the walls. It
was hard to tell if he was letting us in on a state secret or a joke.

\includegraphics{https://static01.nyt.com/images/2019/11/08/business/00Insider-Stevenson-2/00Insider-Stevenson-2-articleLarge.jpg?quality=75\&auto=webp\&disable=upscale}

We would never find out its real worth, at least not from any officials.
In the parking lot, as the badgering continued into a second hour, we
locked ourselves in our car.

At 6:31 p.m., a female police officer walked up to the car. It was
dinner time and the group suddenly had somewhere else to be. She reached
into Cao Li's window to shake her hand.

``You're welcome here,'' the officer said, and smiled. ``Thanks for your
cooperation.''

The drama didn't end there. One police car followed us as we retreated
to our hotel in Zhengzhou, a city two hours away. The authorities then
camped out in the lobby of our hotel, rang our rooms and knocked on our
doors.

It was only when they learned that we had booked tickets on the first
train out of town the next morning that they finally gave up.

Follow the \href{https://twitter.com/readercenter}{@ReaderCenter} on
Twitter for more coverage highlighting your perspectives and experiences
and for insight into how we work.

Advertisement

\protect\hyperlink{after-bottom}{Continue reading the main story}

\hypertarget{site-index}{%
\subsection{Site Index}\label{site-index}}

\hypertarget{site-information-navigation}{%
\subsection{Site Information
Navigation}\label{site-information-navigation}}

\begin{itemize}
\tightlist
\item
  \href{https://help.nytimes.com/hc/en-us/articles/115014792127-Copyright-notice}{©~2020~The
  New York Times Company}
\end{itemize}

\begin{itemize}
\tightlist
\item
  \href{https://www.nytco.com/}{NYTCo}
\item
  \href{https://help.nytimes.com/hc/en-us/articles/115015385887-Contact-Us}{Contact
  Us}
\item
  \href{https://www.nytco.com/careers/}{Work with us}
\item
  \href{https://nytmediakit.com/}{Advertise}
\item
  \href{http://www.tbrandstudio.com/}{T Brand Studio}
\item
  \href{https://www.nytimes.com/privacy/cookie-policy\#how-do-i-manage-trackers}{Your
  Ad Choices}
\item
  \href{https://www.nytimes.com/privacy}{Privacy}
\item
  \href{https://help.nytimes.com/hc/en-us/articles/115014893428-Terms-of-service}{Terms
  of Service}
\item
  \href{https://help.nytimes.com/hc/en-us/articles/115014893968-Terms-of-sale}{Terms
  of Sale}
\item
  \href{https://spiderbites.nytimes.com}{Site Map}
\item
  \href{https://help.nytimes.com/hc/en-us}{Help}
\item
  \href{https://www.nytimes.com/subscription?campaignId=37WXW}{Subscriptions}
\end{itemize}
