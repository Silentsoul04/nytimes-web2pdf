Sections

SEARCH

\protect\hyperlink{site-content}{Skip to
content}\protect\hyperlink{site-index}{Skip to site index}

\href{https://www.nytimes.com/section/world/australia}{Australia}

\href{https://myaccount.nytimes.com/auth/login?response_type=cookie\&client_id=vi}{}

\href{https://www.nytimes.com/section/todayspaper}{Today's Paper}

\href{/section/world/australia}{Australia}\textbar{}Refugee and Author
Detained by Australia Is Given Visa to Travel

\url{https://nyti.ms/2KjV9ri}

\begin{itemize}
\item
\item
\item
\item
\item
\end{itemize}

Advertisement

\protect\hyperlink{after-top}{Continue reading the main story}

Supported by

\protect\hyperlink{after-sponsor}{Continue reading the main story}

\hypertarget{refugee-and-author-detained-by-australia-is-given-visa-to-travel}{%
\section{Refugee and Author Detained by Australia Is Given Visa to
Travel}\label{refugee-and-author-detained-by-australia-is-given-visa-to-travel}}

Behrouz Boochani, who was held for years as part of Australia's offshore
immigration detention program, arrived in New Zealand for a literary
festival.

\includegraphics{https://static01.nyt.com/images/2019/11/14/world/14nz-refugees/merlin_114968525_00f0f09b-cfcd-4500-afa1-22291e1cf420-articleLarge.jpg?quality=75\&auto=webp\&disable=upscale}

\href{https://www.nytimes.com/by/megan-specia}{\includegraphics{https://static01.nyt.com/images/2018/06/13/multimedia/megan-specia/megan-specia-thumbLarge.jpg}}

By \href{https://www.nytimes.com/by/megan-specia}{Megan Specia}

\begin{itemize}
\item
  Nov. 14, 2019
\item
  \begin{itemize}
  \item
  \item
  \item
  \item
  \item
  \end{itemize}
\end{itemize}

A Kurdish-Iranian refugee and award-winning writer who was held by
Australia for years at a detention center on a remote Pacific island
arrived in New Zealand late Thursday, after becoming a cause célèbre for
rights activists.

The refugee, Behrouz Boochani, 35, who became a prominent voice for the
hundreds of people exiled on Manus Island, Papua New Guinea, as part of
\href{https://www.nytimes.com/2018/07/21/world/australia/australia-refugee-policy-protest.html}{Australia's
controversial detention program}, received permission to leave for the
first time in years.

The Australian government has paid neighboring countries to detain
hundreds of asylum seekers and other refugees who arrive by sea. This is
the first time Mr. Boochani has been granted permission to leave Papua
New Guinea since his detention began in 2013, after he was given a
visitor visa to enter New Zealand to speak at the WORD Christchurch
literary festival
\href{https://wordchristchurch.co.nz/programme/behrouz-boochani/}{on
Nov. 29}.

He was held in a detention center on Manus Island until it closed in
2017, and he was then moved to another part of Papua New Guinea. It is
not clear what his status will be after the festival, although he has
\href{https://www.theguardian.com/australia-news/2019/nov/14/behrouz-boochani-free-voice-manus-island-refugees-new-zealand-australia}{vowed
never to return to Papua New Guinea.}

``So exciting to get freedom after more than six years,'' Mr. Boochani
said on Twitter on his arrival at Auckland Airport late on Thursday
night. ``Thank you,'' he added, ``to all the friends who made this
happen.''

Mr. Boochani, who worked as a journalist with the Kurdish-language
magazine Werya,
\href{https://www.nytimes.com/2017/02/13/insider/manus-island-refugee-australia.html?module=inline}{fled
Iran in 2013} after officials raided his office. He traveled to
Indonesia and tried to reach Australia by boat, where he planned to seek
asylum. But he was intercepted by the authorities.

After being moved to a remote detention camp on Manus Island, he
documented rights abuses against the hundreds of asylum seekers held
alongside him on Manus Island, where he has spent the last six years,
and those on the tiny island nation of Nauru. Many of the detainees were
from the Middle East and Africa and were seeking refugee status.

Mr. Boochani's writings raised awareness of the situation on the island,
which was largely closed off to international journalists and rights
groups. He wrote of daily life on the island, posting updates on social
media about squalid conditions at the camp and the suicides and
self-harm that became commonplace among the detainees. He revealed
inadequate access to heath care, deteriorating facilities and
desperation among the men housed there.

He shared details about Manus in 2017 with The New York Times after
President Trump walked back promises made by President Barack Obama to
relocate all detainees who had received refugee status to the United
States. His work was also published by numerous
\href{https://www.thesaturdaypaper.com.au/contributor/behrouz-boochani}{local}
and
\href{https://www.theguardian.com/profile/behrouz-boochani}{international
outlets}.

``We are under such pressure and a hard situation,'' he said at the
time. ``We cannot go back, and we are still here.''

Working with a translator for five years via messages in Farsi over
WhatsApp, he wrote a book, ``No Friend but the Mountains,'' for which he
won the
\href{https://www.wheelercentre.com/news/behrouz-boochani-wins-the-2019-victorian-prize-for-literature}{2019
Victorian Prize for Literature}. The prestigious Australian award,
selected from a short list of winners in other categories,
\href{https://www.nytimes.com/2019/01/31/world/australia/behrouz-boochani-victorian-prize-manus-island.html}{comes
with an award} of 125,000 Australian dollars, or about \$90,000.

He has also won the New South Wales Premier's Award, the Australian Book
Industry Award for nonfiction and the National Biography Award.

Human rights groups have
\href{https://www.nytimes.com/2019/06/26/world/australia/australia-manus-suicide.html}{long
denounced Australia's policy} of detaining migrants and asylum seekers
in offshore detention centers like the one on Manus, but the government
has maintained that it is necessary to deter those arriving in the
country by illegal means.

Australia
\href{https://www.nytimes.com/2017/11/02/world/australia/manus-island-refugees.html?module=inline}{closed
the center on Manus Island} in 2017 and moved some of the asylum
seekers. Others who received official refugee status have been
\href{https://www.nytimes.com/2018/01/23/world/australia/manus-refugees-trump.html?module=inline}{resettled
in the United States}, but hundreds remain in limbo on the island.

The Australian government said last month
that\href{https://www.refugeecouncil.org.au/operation-sovereign-borders-offshore-detention-statistics/}{562
refugees and asylum seekers}remain in Papua New Guinea and on Nauru.

\includegraphics{https://static01.nyt.com/images/2017/02/03/world/03manus-diary/03manus-diary-videoSixteenByNine3000-v2.jpg}

Meg de Ronde, the executive director of Amnesty New Zealand, which was
involved in obtaining a visitor visa for Mr. Boochani to attend the
literary festival,
\href{https://twitter.com/MegdeRonde/status/1194932484758110208}{said on
Twitter} that it was a ``life \& career highlight to be a small part''
of his arrival.

In an earlier statement, Ms. de Ronde said, ``This is a spark of hope
after he has fled violence and persecution, first in Iran and then from
Australian authorities.''

On Thursday, the organizers of the festival released a statement on Mr.
Boochani's behalf. In it, he said he would continue to speak out against
the Australian offshore detention system, ``which is designed to deter
refugees from seeking asylum and, ultimately, has caused grave harm and
torture.''

He also said he would ask the government of New Zealand to aid those who
remained in Papua New Guinea and Nauru.

Michelle Bachelet,
\href{https://www.abc.net.au/news/2019-10-09/un-bachelet-criticises-australia-asylum-seeker-policies/11588084?pfmredir=sm}{the
United Nations rights chief}, recently spoke out about Australia's
migration and asylum policies in
\href{https://www.whitlam.org/publications/2019/10/9/2019-whitlam-oration-un-human-rights-commissioner-michelle-bachelet}{a
speech at the Whitlam Institute in Sydney.}

``I have a number of serious concerns about migration and asylum
policies in this country, including the so-called offshore processing
regime and prolonged mandatory detention of refugees and asylum
seekers,'' she said.

Ms. Bachelet added that she feared the public narrative in Australia
about migration and asylum ``has become weaponized by misinformation and
discriminatory and even racist attitudes.''

Advertisement

\protect\hyperlink{after-bottom}{Continue reading the main story}

\hypertarget{site-index}{%
\subsection{Site Index}\label{site-index}}

\hypertarget{site-information-navigation}{%
\subsection{Site Information
Navigation}\label{site-information-navigation}}

\begin{itemize}
\tightlist
\item
  \href{https://help.nytimes.com/hc/en-us/articles/115014792127-Copyright-notice}{©~2020~The
  New York Times Company}
\end{itemize}

\begin{itemize}
\tightlist
\item
  \href{https://www.nytco.com/}{NYTCo}
\item
  \href{https://help.nytimes.com/hc/en-us/articles/115015385887-Contact-Us}{Contact
  Us}
\item
  \href{https://www.nytco.com/careers/}{Work with us}
\item
  \href{https://nytmediakit.com/}{Advertise}
\item
  \href{http://www.tbrandstudio.com/}{T Brand Studio}
\item
  \href{https://www.nytimes.com/privacy/cookie-policy\#how-do-i-manage-trackers}{Your
  Ad Choices}
\item
  \href{https://www.nytimes.com/privacy}{Privacy}
\item
  \href{https://help.nytimes.com/hc/en-us/articles/115014893428-Terms-of-service}{Terms
  of Service}
\item
  \href{https://help.nytimes.com/hc/en-us/articles/115014893968-Terms-of-sale}{Terms
  of Sale}
\item
  \href{https://spiderbites.nytimes.com}{Site Map}
\item
  \href{https://help.nytimes.com/hc/en-us}{Help}
\item
  \href{https://www.nytimes.com/subscription?campaignId=37WXW}{Subscriptions}
\end{itemize}
