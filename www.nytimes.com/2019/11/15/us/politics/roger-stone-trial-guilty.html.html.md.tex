Sections

SEARCH

\protect\hyperlink{site-content}{Skip to
content}\protect\hyperlink{site-index}{Skip to site index}

\href{https://www.nytimes.com/section/politics}{Politics}

\href{https://myaccount.nytimes.com/auth/login?response_type=cookie\&client_id=vi}{}

\href{https://www.nytimes.com/section/todayspaper}{Today's Paper}

\href{/section/politics}{Politics}\textbar{}Roger Stone Is Convicted of
Impeding Investigators in a Bid to Protect Trump

\url{https://nyti.ms/2qS99Sz}

\begin{itemize}
\item
\item
\item
\item
\item
\item
\end{itemize}

Advertisement

\protect\hyperlink{after-top}{Continue reading the main story}

Supported by

\protect\hyperlink{after-sponsor}{Continue reading the main story}

\hypertarget{roger-stone-is-convicted-of-impeding-investigators-in-a-bid-to-protect-trump}{%
\section{Roger Stone Is Convicted of Impeding Investigators in a Bid to
Protect
Trump}\label{roger-stone-is-convicted-of-impeding-investigators-in-a-bid-to-protect-trump}}

Mr. Stone, a longtime informal adviser to President Trump, obstructed
one of Congress's Russia investigations and lied to lawmakers.

\includegraphics{https://static01.nyt.com/images/2019/11/15/nyregion/15dc-stone11/15dc-stone11-articleLarge-v3.jpg?quality=75\&auto=webp\&disable=upscale}

\href{https://www.nytimes.com/by/sharon-lafraniere}{\includegraphics{https://static01.nyt.com/images/2018/07/12/multimedia/author-sharon-lafraniere/author-sharon-lafraniere-thumbLarge.png}}\href{https://www.nytimes.com/by/zach-montague}{\includegraphics{https://static01.nyt.com/images/2019/12/13/reader-center/author-zach-montague/author-zach-montague-thumbLarge.png}}

By \href{https://www.nytimes.com/by/sharon-lafraniere}{Sharon
LaFraniere} and \href{https://www.nytimes.com/by/zach-montague}{Zach
Montague}

\begin{itemize}
\item
  Published Nov. 15, 2019Updated June 16, 2020
\item
  \begin{itemize}
  \item
  \item
  \item
  \item
  \item
  \item
  \end{itemize}
\end{itemize}

WASHINGTON --- For decades,
\href{https://www.nytimes.com/2020/06/16/us/politics/roger-stone-prosecutor-hearing.html}{Roger
J. Stone Jr.} played politics as a kind of performance art, starring
himself as a professional lord of mischief, as a friend once called him.
He tossed bombs and spun tales from the political periphery with no real
reckoning, burnishing a reputation as a dirty trickster.

On Friday morning, a reckoning arrived, the consequence of his efforts
to sabotage a congressional investigation that threatened his longtime
friend
\href{https://www.nytimes.com/2020/02/12/us/politics/trump-stone.html}{President
Trump}.

Mr. Stone, 67, was convicted in federal court of seven felonies for
obstructing the congressional inquiry, lying to investigators under oath
and trying to block the testimony of a witness whose account would have
exposed his lies. Jurors deliberated for a little over seven hours
before convicting him on all counts. Together, the charges carry a
maximum prison term of 50 years.

In a last-minute bid for salvation, prosecutors said, Mr. Stone appealed
to Mr. Trump for a pardon on Thursday, using a right-wing conspiracy
theorist who runs the website Infowars as his proxy. Mr. Trump attacked
the guilty verdict against Mr. Stone
\href{https://twitter.com/realDonaldTrump/status/1195389483664990208}{in
a tweet} on Friday but made no mention of a pardon.

To some friends, Mr. Stone's fatal flaw was that he did not know when
the time for gamesmanship was over. ``The only thing worse than being
talked about is not being talked about,'' he liked to say. But that
mantra seemed to ring hollow as Mr. Stone, forced to stand in silence,
heard a courtroom deputy read the word ``guilty'' seven times.

The impeachment inquiry underway nearby on Capitol Hill overshadowed
news of the verdict, but it was nonetheless another setback for the
president. Mr. Stone is the sixth former Trump aide to be convicted in
cases stemming from the investigation by the special counsel, Robert S.
Mueller III, into Russia's interference in the 2016 election.

And the trial revived the saga of Russia's efforts to bolster Mr.
Trump's chances of winning the White House just as House impeachment
investigators are scrutinizing how Mr. Trump pressured another
government, Ukraine, to help with his 2020 re-election chances.

Prosecutors said Mr. Stone tried to thwart the work of the House
Intelligence Committee because the truth would have ``looked terrible''
for both Mr. Trump and his campaign. They built their case over the past
week with testimony from a friend of Mr. Stone and two former Trump
campaign officials: Rick Gates, the deputy campaign chairman, and
Stephen K. Bannon, who led the campaign through its final three months
and served as a White House strategist early in the administration.

Hundreds of exhibits that exposed Mr. Stone's disdain for congressional
and criminal investigators buttressed the testimony.

The evidence showed that in the months before the 2016 election, Mr.
Stone
\href{https://www.nytimes.com/2018/11/27/us/politics/jerome-corsi-roger-stone-wikileaks.html}{strove
to obtain emails} that Russia had stolen from Democratic computers and
\href{https://www.nytimes.com/2019/01/28/podcasts/the-daily/roger-stone-trump-mueller-wikileaks.html}{funneled
to WikiLeaks}, which released them at strategic moments timed to damage
Hillary Clinton, Mr. Trump's Democratic opponent. ``Every chance he
got,'' prosecutors said, Mr. Stone briefed the Trump campaign about
whatever he had picked up about WikiLeaks' plans.

But he told the House committee in September 2017 that he never
described to anyone involved in the Trump campaign his conversations
with an intermediary to WikiLeaks.

The trial called into question Mr. Trump's own answers to queries from
Mr. Mueller. The president, who refused to be interviewed and agreed to
respond to questions only in writing, said he could not recall the
specifics of any of 21 conversations he had with Mr. Stone in the six
months before the election.

In one of the trial's most revealing moments, Mr. Gates recounted a July
31, 2016, phone call between Mr. Stone and Mr. Trump, just days after
WikiLeaks had released a trove of emails embarrassing the Clinton
campaign. As soon as he hung up with Mr. Stone, Mr. Gates testified, Mr.
Trump declared that ``more information'' was coming, an apparent
reference to future releases from WikiLeaks that would rattle his
political rival.

Within minutes of the verdict, Mr. Trump
\href{https://twitter.com/realDonaldTrump/status/1195389483664990208}{protested
on Twitter} that it was unfair. ``So they now convict Roger Stone of
lying and want to jail him for many years to come,'' Mr. Trump wrote,
though his own administration's Justice Department prosecuted Mr. Stone.

He then listed the names of nearly a dozen favorite targets of his ire,
including Mrs. Clinton, Mr. Mueller, the former F.B.I. director James B.
Comey and Representative Adam B. Schiff, who heads the House
Intelligence Committee. ``Didn't they lie?'' he tweeted, and then added:
``A double standard like never seen before in the history of our
Country?''

Mr. Stone joins a notable list of former Trump aides who either pleaded
guilty or were convicted of federal crimes in cases stemming from Mr.
Mueller's work. It includes Mr. Gates; Michael T. Flynn, the former
national security adviser; Michael D. Cohen, the president's longtime
fixer; George Papadopoulos, a former Trump campaign aide; and Paul
Manafort, Mr. Trump's former campaign chairman and Mr. Stone's onetime
partner in a political consulting firm.

Although the counts against him add up to 50 years, the punishment for
Mr. Stone, who had no previous criminal record, will almost certainly be
far lighter. On the other hand, his multiple run-ins earlier this year
with Judge Amy Berman Jackson, who will sentence him on Feb. 6, could
work against him. After a series of infractions, including posting a
photo of the judge with an image of cross-hairs next to her head on
Instagram in February, she
\href{https://www.nytimes.com/2019/07/16/us/politics/roger-stone-gag-order.html}{banned
him from social media}.

After the verdict, prosecutors asked Judge Jackson to put Mr. Stone
behind bars, arguing that he had defied her once again by passing the
message to Infowars' Alex Jones, saying, ``I appeal to the president to
pardon me.'' But the judge said it was not entirely clear whether Mr.
Stone had disobeyed her and noted that he had complied with her orders
in recent months.

Mr. Stone's lawyers argued that the prosecution's case was based on
speculation and false assumptions about Mr. Stone's motives. Mr. Stone
had no reason to lie in order to protect the president nearly a year
after Mr. Trump had won the election, Bruce S. Rogow, the lead defense
lawyer, told jurors.

Mr. Stone had simply confined his answers to the strict parameters of
the committee's inquiry, he argued. He also said that even though Mr.
Stone had
\href{https://www.nytimes.com/2018/11/01/us/politics/roger-stone-trump-campaign-mueller-wikileaks.html}{portrayed
himself to the campaign as Mr. Trump's link to WikiLeaks}, that was just
more of Mr. Stone's typical braggadocio.

Much of the trial revolved around relationship between Mr. Stone and
Randy Credico, a New York radio host and comedian. The charge that Mr.
Stone had tried to block Mr. Credico from testifying to the House
committee was the most serious one he faced, carrying a maximum penalty
of 20 years. Mr. Credico ultimately asserted his Fifth Amendment rights
against self-incrimination and refused to testify to the House
committee.

He testified at the trial that despite the fact he repeatedly urged Mr.
Stone to tell the truth, he falsely identified him to congressional
investigators as his intermediary with Julian Assange, the founder of
WikiLeaks. Previously in their tortured 17-year friendship, he said, Mr.
Stone had treated him as his ``patsy.''

\includegraphics{https://static01.nyt.com/images/2019/11/14/nyregion/14dc-stone2/merlin_164062608_c68e3007-7a6b-4191-a4b6-71790cd52955-articleLarge.jpg?quality=75\&auto=webp\&disable=upscale}

In an effort to ward off Mr. Credico's congressional testimony, the
evidence showed, Mr. Stone alternately flattered, bullied and threatened
the radio host. At one point, Mr. Stone pretended that he had written a
letter to the House committee characterizing Mr. Credico as highly
talented and successful.

At other points, he urged Mr. Credico to ``Do a Frank Pentangeli,''
referring
\href{https://www.nytimes.com/2019/11/06/movies/roger-stone-trial-the-godfather.html?module=inline}{to
a character} in the movie ``The Godfather: Part II'' who gave false
testimony during a Senate hearing on organized crime. Borrowing
\href{https://www.nytimes.com/1974/10/22/archives/jury-hears-tape-of-nixon-urgingaides-to-stonewall-balking-of.html}{a
quote from Richard M. Nixon to a top aide} during the Watergate
cover-up, Mr. Stone texted Mr. Credico in late 2017: ``Stonewall it.
Plead the fifth. Anything to save the plan.''

If he refused to go along, Mr. Credico testified, Mr. Stone promised to
retaliate against him and Margaret Ratner Kunstler, a lawyer for Mr.
Assange and one of Mr. Credico's dearest friends. Prosecutors described
Ms. Kunstler as a particularly effective ``pressure point'' with Mr.
Credico, an unmarried man with no children and a 34-year history of
alcohol abuse.

One of Mr. Stone's most blatant deceptions, prosecutors said, was hiding
records of his communications. He told congressional investigators that
he and Mr. Credico only spoke on the phone, because Mr. Credico ``was
not an email guy.''

In fact, in the year and a half before Mr. Stone testified, he and Mr.
Credico exchanged more than 1,500 emails and text messages, including 72
texts alone on the day of Mr. Stone's congressional testimony. Because
Mr. Stone misled the committee, prosecutors said, investigators failed
to pursue promising leads and arrived at inaccurate conclusions in its
final report on Russia's election interference.

Mr. Credico tried for months to warn Mr. Stone that his lies would catch
up to him. Among Mr. Stone's responses: ``Nice try.'' ``Meaningless.''
``So what.''

And, ``No one cares.''

Advertisement

\protect\hyperlink{after-bottom}{Continue reading the main story}

\hypertarget{site-index}{%
\subsection{Site Index}\label{site-index}}

\hypertarget{site-information-navigation}{%
\subsection{Site Information
Navigation}\label{site-information-navigation}}

\begin{itemize}
\tightlist
\item
  \href{https://help.nytimes.com/hc/en-us/articles/115014792127-Copyright-notice}{©~2020~The
  New York Times Company}
\end{itemize}

\begin{itemize}
\tightlist
\item
  \href{https://www.nytco.com/}{NYTCo}
\item
  \href{https://help.nytimes.com/hc/en-us/articles/115015385887-Contact-Us}{Contact
  Us}
\item
  \href{https://www.nytco.com/careers/}{Work with us}
\item
  \href{https://nytmediakit.com/}{Advertise}
\item
  \href{http://www.tbrandstudio.com/}{T Brand Studio}
\item
  \href{https://www.nytimes.com/privacy/cookie-policy\#how-do-i-manage-trackers}{Your
  Ad Choices}
\item
  \href{https://www.nytimes.com/privacy}{Privacy}
\item
  \href{https://help.nytimes.com/hc/en-us/articles/115014893428-Terms-of-service}{Terms
  of Service}
\item
  \href{https://help.nytimes.com/hc/en-us/articles/115014893968-Terms-of-sale}{Terms
  of Sale}
\item
  \href{https://spiderbites.nytimes.com}{Site Map}
\item
  \href{https://help.nytimes.com/hc/en-us}{Help}
\item
  \href{https://www.nytimes.com/subscription?campaignId=37WXW}{Subscriptions}
\end{itemize}
