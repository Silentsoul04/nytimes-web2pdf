Sections

SEARCH

\protect\hyperlink{site-content}{Skip to
content}\protect\hyperlink{site-index}{Skip to site index}

\href{https://www.nytimes.com/section/politics}{Politics}

\href{https://myaccount.nytimes.com/auth/login?response_type=cookie\&client_id=vi}{}

\href{https://www.nytimes.com/section/todayspaper}{Today's Paper}

\href{/section/politics}{Politics}\textbar{}Key Moments From Sondland,
Cooper and Hale Testimony

\url{https://nyti.ms/2Ou8Y7X}

\begin{itemize}
\item
\item
\item
\item
\item
\item
\end{itemize}

Advertisement

\protect\hyperlink{after-top}{Continue reading the main story}

Supported by

\protect\hyperlink{after-sponsor}{Continue reading the main story}

\hypertarget{key-moments-from-sondland-cooper-and-hale-testimony}{%
\section{Key Moments From Sondland, Cooper and Hale
Testimony}\label{key-moments-from-sondland-cooper-and-hale-testimony}}

A Pentagon official provided new details about when the Ukrainians may
have learned about the hold on security aid. Gordon D. Sondland also
testified that multiple top officials, including Mike Pompeo and Mike
Pence, knew about the campaign to pressure Ukraine to announce
investigations of Democrats.

\href{https://www.nytimes.com/by/michael-d-shear}{\includegraphics{https://static01.nyt.com/images/2018/06/13/multimedia/author-michael-d-shear/author-michael-d-shear-thumbLarge-v2.png}}\href{https://www.nytimes.com/by/peter-baker}{\includegraphics{https://static01.nyt.com/images/2018/06/13/multimedia/peter-baker/peter-baker-thumbLarge-v2.png}}

By \href{https://www.nytimes.com/by/michael-d-shear}{Michael D. Shear}
and \href{https://www.nytimes.com/by/peter-baker}{Peter Baker}

\begin{itemize}
\item
  Published Nov. 20, 2019Updated Dec. 30, 2019
\item
  \begin{itemize}
  \item
  \item
  \item
  \item
  \item
  \item
  \end{itemize}
\end{itemize}

\includegraphics{https://static01.nyt.com/images/2019/11/20/us/politics/c20dc-impeach1vid-copy/20dc-impeach1-videoSixteenByNine3000-v3.jpg}

\hypertarget{heres-what-you-need-to-know}{%
\subsubsection{Here's what you need to
know:}\label{heres-what-you-need-to-know}}

\begin{itemize}
\tightlist
\item
  \protect\hyperlink{link-137a0fab}{The Ukrainians may have learned
  about the hold on aid much earlier than previously known.}
\item
  \protect\hyperlink{link-4911ce00}{Sondland testified he worked with
  Giuliani to pressure Ukraine `at the express direction of the
  president.'}
\item
  \protect\hyperlink{link-437d93f8}{`Everyone was in the loop,' Sondland
  said, including Pence, Pompeo, Mulvaney and others.}
\item
  \protect\hyperlink{link-5226be42}{Trump dismissed Sondland, saying `I
  don't know him very well.'}
\item
  \protect\hyperlink{link-555ff837}{Democrats argued that Sondland's
  testimony bolstered their case for impeaching Trump.}
\item
  \protect\hyperlink{link-7937b34e}{For Trump, what mattered most was
  that investigations of Democrats be publicly announced.}
\item
  \protect\hyperlink{link-4eb48fee}{Sondland confirmed an indiscreet
  conversation with Trump but disputed descriptions of July 10 meeting.}
\end{itemize}

\href{https://www.nytimes.com/2019/11/21/us/politics/impeachment-hearing.html}{\emph{Follow
our live coverage of David Holmes and Fiona Hill testifying in the
impeachment hearings}}.

\hypertarget{the-ukrainians-may-have-learned-about-the-hold-on-aid-much-earlier-than-previously-known}{%
\subsection{The Ukrainians may have learned about the hold on aid much
earlier than previously
known.}\label{the-ukrainians-may-have-learned-about-the-hold-on-aid-much-earlier-than-previously-known}}

Ukraine officials may have been aware that security aid was cut off by
July 25 --- much earlier than previously known and the same day that
\href{https://www.nytimes.com/2019/11/21/podcasts/the-daily/impeachment-sondland.html}{President
Trump} talked on the phone with the president of Ukraine, a top Pentagon
official said Wednesday.

Laura K. Cooper, the deputy assistant secretary of defense for Russia,
Ukraine and Eurasia, said that she was aware of multiple communications
between Ukrainian Embassy officials and members of her staff in which
the embassy officials asked questions about delivery of the security aid
to their country.

Ms. Cooper said that a member of her staff received a question about the
aid on July 25 from the Ukrainian Embassy, which asked ``what was going
on with Ukraine assistance.'' She said that during the week of Aug. 6,
other members of her staff saw officials from the embassy who raised the
issue of the aid.

The timing of when Ukraine knew that the aid had been frozen is a
critical question as Democrats build a case that Mr. Trump tried to
leverage the aid for a public announcement of investigations into his
political rivals. The security aid was frozen in early July, and
Republicans have insisted that Ukraine did not know about the hold until
it was reported by a news outlet on Aug. 28.

Mr. Trump's allies have also said that Mr. Trump could not have coerced
Volodymyr Zelensky, the president of Ukraine, during the July 25 call
because Mr. Zelensky did not know at the time that the aid was held up.
The new information from Ms. Cooper could undercut the Republican
efforts to defend the president.

\includegraphics{https://static01.nyt.com/images/2019/11/20/us/politics/20dc-impeachbriefing-hale/merlin_164753199_f9915c49-5a4e-46e4-8ebc-1b0d70dc9ee8-articleLarge.jpg?quality=75\&auto=webp\&disable=upscale}

Ms. Cooper said her staff recalled the issue of concerns about Ukraine's
security aid coming up with members of the Ukrainian Embassy in other
meetings during the month of August, though they could not recall
precisely when those meetings took place.

``They believe the question of the hold came up at some point,'' Ms.
Cooper said.

She also cited several emails dated July 25 between members of her staff
and State Department officials in which the diplomats wrote that the
Ukrainian Embassy knew about the hold on the security assistance. Ms.
Cooper said she did not believe she was shown the emails at the time.

Ms. Cooper said she learned of the new information about the inquiries
from Ukrainian officials after members of her staff saw the transcript
of her earlier, closed-door testimony when it was released to the public
on Nov. 11, and brought new details of the timeline to her attention.

\href{https://www.nytimes.com/2019/11/20/us/politics/david-hale-impeachment.html}{David
Hale}, the State Department's No. 3 official, also fielded questions
about the hold on security aid to Ukraine and the attacks by Rudolph W.
Giuliani, President Trump's personal lawyer, on the reputation of Marie
L. Yovanovitch, the United States ambassador to Ukraine. She was
eventually recalled from her post.

Mr. Hale told lawmakers that what happened to Ms. Yovanovitch was
``wrong'' and that ``I believe that she should have been able to stay at
post and continue to do the outstanding work.''

\hypertarget{sondland-testified-he-worked-with-giuliani-to-pressure-ukraine-at-the-express-direction-of-the-president}{%
\subsection{Sondland testified he worked with Giuliani to pressure
Ukraine `at the express direction of the
president.'}\label{sondland-testified-he-worked-with-giuliani-to-pressure-ukraine-at-the-express-direction-of-the-president}}

Image

Mr. Sondland testifying on Wednesday.Credit...Doug Mills/The New York
Times

\includegraphics{https://static01.nyt.com/images/2017/01/29/podcasts/the-daily-album-art/the-daily-album-art-articleInline-v2.jpg?quality=75\&auto=webp\&disable=upscale}

\hypertarget{listen-to-the-daily-we-followed-the-presidents-orders}{%
\subsubsection{Listen to `The Daily': `We Followed the President's
Orders'}\label{listen-to-the-daily-we-followed-the-presidents-orders}}

In highly anticipated testimony, a top envoy said the operation to
extract a political favor from Ukraine was done at the direction of the
president, vice president and secretary of state.

transcript

Back to The Daily

bars

0:00/27:39

-27:39

transcript

\hypertarget{listen-to-the-daily-we-followed-the-presidents-orders-1}{%
\subsection{Listen to `The Daily': `We Followed the President's
Orders'}\label{listen-to-the-daily-we-followed-the-presidents-orders-1}}

\hypertarget{hosted-by-michael-barbaro-produced-by-alexandra-leigh-young-theo-balcomb-and-kelly-prime-with-help-from-neena-pathak-edited-by-mj-davis-lin-and-lisa-tobin}{%
\subsubsection{Hosted by Michael Barbaro; produced by Alexandra Leigh
Young, Theo Balcomb and Kelly Prime; with help from Neena Pathak; edited
by M.J. Davis Lin and Lisa
Tobin}\label{hosted-by-michael-barbaro-produced-by-alexandra-leigh-young-theo-balcomb-and-kelly-prime-with-help-from-neena-pathak-edited-by-mj-davis-lin-and-lisa-tobin}}

\hypertarget{in-highly-anticipated-testimony-a-top-envoy-said-the-operation-to-extract-a-political-favor-from-ukraine-was-done-at-the-direction-of-the-president-vice-president-and-secretary-of-state}{%
\paragraph{In highly anticipated testimony, a top envoy said the
operation to extract a political favor from Ukraine was done at the
direction of the president, vice president and secretary of
state.}\label{in-highly-anticipated-testimony-a-top-envoy-said-the-operation-to-extract-a-political-favor-from-ukraine-was-done-at-the-direction-of-the-president-vice-president-and-secretary-of-state}}

\begin{itemize}
\item
  michael barbaro\\
  From The New York Times, I'm Michael Barbaro. This is ``The Daily.''

  Gordon Sondland is a Trump donor turned E.U. ambassador turned witness
  in the impeachment inquiry, whose testimony has been contradicted on
  multiple occasions. Today: How both Democrats and Republicans on the
  House Intelligence Committee choose to handle their most complicated
  witness to date. It's Thursday, November 21.

  So Nick, give us a sense of what it felt like to be in that room this
  morning as the committee prepared to hear from Gordon Sondland, this
  very complicated witness.
\item
  nicholas fandos\\
  So we knew that in the middle of this jam-packed week of impeachment
  testimony, Gordon Sondland's appearance was going to be something
  different.
\item
  michael barbaro\\
  Nick Fandos was in the Capitol for Sondland's testimony.
\item
  nicholas fandos\\
  Sondland has really been described by almost everybody as having been
  in the middle of this pressure campaign on Ukraine this summer. Now,
  there was particular anticipation, because Sondland's story, unlike
  some other witnesses, has evolved as time has gone on. So when he
  first came in and spoke in private with investigators in October, he
  was pretty unclear about whether or not the president was trying to
  leverage a White House meeting or security aid for Ukraine to get the
  investigations that he wanted. He told them repeatedly, I can't
  remember this, I can't remember that. Well, after that happened,
  Sondland submitted a supplement to his testimony in writing that
  updated that and said, actually, I do think there was a quid pro quo.
  And so there was a lot of anticipation heading into his testimony as
  to which Gordon Sondland would show up. What story was he going to
  tell? Is he going to cop to all these different things? How is he
  going to treat the president? How's he going to treat his own evolving
  story? And it really felt like it could go either way.
\item
  michael barbaro\\
  So what happens? How does this much-anticipated hearing begin?
\item
  nicholas fandos\\
  So a little after 9:00, Gordon Sondland walks into the same grand
  hearing room where all the impeachment hearings have been taking
  place. Chairman Schiff dropped his gavel.
\item
  archived recording (adam schiff)\\
  Meeting will come to order. {[}GAVEL BANG{]}
\end{itemize}

nicholas fandos

And then, after taking an oath to tell the truth and nothing but the
truth ---

\begin{itemize}
\tightlist
\item
  archived recording (adam schiff)\\
  And nothing but the truth, so help you God.
\end{itemize}

nicholas fandos

--- Sondland sits down and begins reading from a pretty thick stack of
papers, an opening statement that he's brought with him. And he starts
by saying that ---

\begin{itemize}
\tightlist
\item
  archived recording (gordon sondland)\\
  I have not had access to all of my phone records, State Department
  emails and many, many other State Department documents.
\end{itemize}

nicholas fandos

--- his job is made more difficult by the fact that he has not been
allowed to go back to the State Department and look at notes or records
that might be able to fill in his recollections.

\begin{itemize}
\tightlist
\item
  archived recording (gordon sondland)\\
  Having access would have been very helpful to me in trying to
  reconstruct with whom I spoke and met and when and what was said.
\end{itemize}

nicholas fandos

Seems to be a nod at the fact that his story is, in the most generous
sense, evolving. And he begins to tell the story about his involvement
with Ukraine policy over the last, say, five or six months.

\begin{itemize}
\tightlist
\item
  archived recording (gordon sondland)\\
  The U.S. delegation developed a very positive view of the Ukraine
  government.
\end{itemize}

nicholas fandos

He begins talking about a meeting with President Trump in May in the
Oval Office after he attended the inauguration of Ukraine's new
president.

\begin{itemize}
\tightlist
\item
  archived recording (gordon sondland)\\
  Unfortunately, President Trump was skeptical. In response to our
  persistent efforts in that meeting to change his views, President
  Trump directed us to, quote, ``talk with Rudy.'' We weren't happy with
  the president's directive to talk with Rudy. We did not want to
  involve Mr. Giuliani.
\end{itemize}

nicholas fandos

What he then starts to unspool is a set of facts and meetings that we've
now heard a lot about in the last couple of months.

\begin{itemize}
\tightlist
\item
  archived recording (gordon sondland)\\
  Simply put, we were playing the hand we were dealt. We all understood
  that if we refused to work with Mr. Giuliani, we would lose a very
  important opportunity to cement relations between the United States
  and Ukraine. So we followed the president's orders.
\end{itemize}

nicholas fandos

How, you know, May became June, became July and then August, and it
became clear to him over time that President Trump, through his lawyer
Mr. Giuliani and through actions that he took, wanted to extract from
the Ukrainians certain politically beneficial investigations. And at one
point, pretty early in his statement, he addresses the Latin phrase that
has been confusing everyone as this has gone along.

\begin{itemize}
\tightlist
\item
  archived recording (gordon sondland)\\
  Was there a quid pro quo?
\end{itemize}

nicholas fandos

Was it a quid pro quo, a this for that, or was it not? And he says ---

\begin{itemize}
\tightlist
\item
  archived recording (gordon sondland)\\
  With regard to the requested White House call and the White House
  meeting, the answer is yes.
\end{itemize}

nicholas fandos

Unequivocally, that as far as I'm concerned, there was a quid pro quo
around the White House meeting.

michael barbaro

Right, and I was struck by how casually he says that, this thing that is
very much at the center of the entire inquiry.

nicholas fandos

And the thing that has been contested by other witnesses. He states, you
know, very authoritatively, without reservation. Now, he's a little bit
more cautious about whether or not there was a quid pro quo around the
suspended military assistance.

\begin{itemize}
\tightlist
\item
  archived recording (gordon sondland)\\
  In July and August of 2019, we learned that the White House had also
  suspended security aid to Ukraine. I tried diligently to ask why the
  aid was suspended, but I never received a clear answer. Still haven't
  to this day.
\end{itemize}

nicholas fandos

So he came to conclude by August that that, too, was dependent on
Ukraine announcing these investigations that the president wanted.

\begin{itemize}
\tightlist
\item
  archived recording (gordon sondland)\\
  Committing to the investigations of the 2016 elections and Burisma, as
  Mr. Giuliani had demanded.
\end{itemize}

nicholas fandos

Now, Sondland has another clear objective as he's laying out his story,
and that is to defend himself against the testimony from other witnesses
who have tried to describe him as either a rogue actor or somebody that
was working through an improper diplomatic channel.

\begin{itemize}
\tightlist
\item
  archived recording (gordon sondland)\\
  I'm not sure how someone could characterize something as an irregular
  channel when you're talking to the president of the United States, the
  secretary of state, the national security adviser, the chief of staff
  at the White House, the secretary of energy. I don't know how that's
  irregular, if a bunch of ---
\end{itemize}

nicholas fandos

He says, no. This was the real channel. I was working with the president
of the United States. I was working with top American diplomats and
officials. And not only was it proper, but all of those people knew what
I was up to and what we were trying to accomplish as this went along.

\begin{itemize}
\tightlist
\item
  archived recording (gordon sondland)\\
  Within my State Department emails, there is a July 19 email. This
  email was sent, this email was sent to Secretary Pompeo, Secretary
  Perry, Brian McCormack, who was Secretary Perry's chief of staff at
  the time ---
\end{itemize}

nicholas fandos

Basically, you know, if for weeks we've had witness after witness, and
certainly the White House and the Republicans, willing to throw Gordon
Sondland under the bus to try and push all of this onto him as a kind of
lone wolf, I mean, he was taking everybody under the bus with him.

\begin{itemize}
\tightlist
\item
  archived recording (gordon sondland)\\
  --- Chief of Staff Mulvaney and Mr. Mulvaney's senior adviser, Rob
  Blair, a lot of senior officials. These emails show that the
  leadership of the State Department, the National Security Council and
  the White House were all informed about the Ukraine efforts.
\end{itemize}

michael barbaro

Right, he's saying, everybody who now wants to distance themselves from
me, they were actually all in the loop.

nicholas fandos

That's right. It's a phrase he uses several times.

\begin{itemize}
\tightlist
\item
  archived recording (gordon sondland)\\
  It was no secret. Everyone's in the loop. Everyone was in the loop.
\end{itemize}

nicholas fandos

In the loop.

\begin{itemize}
\tightlist
\item
  archived recording (gordon sondland)\\
  Again ---
\end{itemize}

nicholas fandos

In the loop.

\begin{itemize}
\tightlist
\item
  archived recording (gordon sondland)\\
  --- everyone was in the loop.
\end{itemize}

nicholas fandos

They were in the loop.

\begin{itemize}
\tightlist
\item
  archived recording (gordon sondland)\\
  There was a September 1 meeting with President Zelensky in Warsaw.
  During the actual meeting, President Zelensky raised the issue of
  security assistance directly with Vice President Pence. And the vice
  president said that he would speak to President Trump about it.
\end{itemize}

michael barbaro

And, Nick, what's the significance of what he's saying here? I mean, in
theory, this is very explosive. The vice president, the secretary of
state are being, in Sondland's testimony, directly drawn into this.

nicholas fandos

Right, he's trying to make the point that this wasn't just me. Like,
everyone understood this to be Trump's objective out of this
relationship, and we were all working toward that. And nobody found it
unusual that I was doing what I was doing.

\begin{itemize}
\tightlist
\item
  archived recording (gordon sondland)\\
  I sent Secretary Pompeo an email to express my appreciation for his
  joining a series of meetings in Brussels following the Warsaw trip. I
  wrote, ``Mike, thanks for schlepping to Europe. I think it was really
  important and the chemistry seems promising. Really appreciate it.''
  Secretary Pompeo replied the next day, on Wednesday, September 4,
  quote, ``All good. You're doing great work; keep banging away.''
\end{itemize}

nicholas fandos

And with that, Sondland completes his opening statement.

\begin{itemize}
\tightlist
\item
  archived recording (gordon sondland)\\
  It remains an honor to serve the people of the United States as their
  United States ambassador to the European Union. I look forward to
  answering the committee's questions. Thank you.
\end{itemize}

nicholas fandos

And it's the Democrat's turn for 45 minutes to begin asking questions.

\begin{itemize}
\tightlist
\item
  archived recording (adam schiff)\\
  We will now proceed the first round of questions, as detailed ---
\end{itemize}

nicholas fandos

Of course, Chairman Schiff has a few of his own. But mostly, he passes
the mic over to Dan Goldman, who is his chief investigator, who's been
playing a role in these hearings, questioning witnesses directly.

\begin{itemize}
\item
  archived recording (adam schiff)\\
  Mr. Goldman.
\item
  archived recording (daniel goldman)\\
  Thank you, Mr. Chairman. In your opening statement, Ambassador
  Sondland, you detailed the benefits that you have gained from
  obtaining some additional documents over the past few weeks. Is that
  right?
\item
  archived recording (gordon sondland)\\
  In terms of refreshing my recollection, that's ---
\item
  archived recording (daniel goldman)\\
  Right, because ---
\end{itemize}

nicholas fandos

He doesn't shy away from what becomes clear over the course of the day
is a pretty shoddy memory on the part of Sondland.

\begin{itemize}
\item
  archived recording (daniel goldman)\\
  You have remembered a lot more than you did when you were deposed. Is
  that right?
\item
  archived recording (gordon sondland)\\
  That's correct.
\item
  archived recording (daniel goldman)\\
  And one of the things that ---
\end{itemize}

nicholas fandos

And in doing that, Goldman does something interesting, which is that ---

\begin{itemize}
\item
  archived recording (daniel goldman)\\
  --- conversation, Ambassador Taylor also testified, under oath, that
  you said that President Trump wanted Zelensky in a public box. Do you
  recall using that expression?
\item
  archived recording (gordon sondland)\\
  Yeah, it goes back ---
\end{itemize}

nicholas fandos

He repeatedly leans on those witnesses with better memories, or people
who took contemporaneous notes.

\begin{itemize}
\item
  archived recording (daniel goldman)\\
  Do you have any reason to question Ambassador Taylor's testimony based
  on his meticulous and careful contemporaneous notes?
\item
  archived recording (gordon sondland)\\
  I'm not going to question or not question. I'm just telling you what I
  believe I was referring to.
\end{itemize}

nicholas fandos

To describe things that Sondland told them or describe things that
Sondland did, and basically asked Sondland to confirm that these things
happened ---

\begin{itemize}
\tightlist
\item
  archived recording (daniel goldman)\\
  Let me fast forward a week and show you another text exchange, which
  may help refresh your recollection. On September 8, you sent a text to
  Ambassador Taylor and Ambassador Volker. Can you read what you wrote
  there?
\end{itemize}

nicholas fandos

--- to basically bring Sondland up into line with testimony that they've
gotten from other people about him.

\begin{itemize}
\item
  archived recording (daniel goldman)\\
  So you do acknowledge you spoke to President Trump, as you indicated
  in that text, right?
\item
  archived recording (gordon sondland)\\
  If I said I did, I did.
\end{itemize}

nicholas fandos

And that's important as they move towards writing a report and
presenting a kind of full case about what happened to the American
people. It helps eliminate some of the discrepancies in the testimony
they've gotten.

michael barbaro

Which of these exchanges stood out to you?

nicholas fandos

Well, one in particular is fresh in my mind, because we just learned
about it for the first time last week.

\begin{itemize}
\tightlist
\item
  archived recording (daniel goldman)\\
  And one of the things that you now remember is the discussion that you
  had with President Trump on July 26 in that restaurant in Kyiv, right?
\end{itemize}

nicholas fandos

Now remember, that's the day after Trump himself spoke to Zelensky and
told him that I want these investigations into the Bidens in 2016. Other
witnesses have described Sondland basically speaking on the phone with
the president, who was speaking so loud that he had to hold the phone
away from his ear.

\begin{itemize}
\tightlist
\item
  archived recording (gordon sondland)\\
  He claims to have overheard part of the conversation, and I'm not
  going to dispute what he did or didn't hear.
\end{itemize}

nicholas fandos

Others could overhear President Trump asking about, quote, unquote,
``the investigation,'' and Sondland assured him, you know, don't worry,
with some expletive laid in, President Zelensky will do whatever we
want.

michael barbaro

Right, because Ukraine loves your tush.

nicholas fandos

Something like that.

\begin{itemize}
\item
  archived recording (daniel goldman)\\
  Well, he also testified that President Zelensky, quote, ``loves your
  ass,'' unquote. Do you recall saying that?
\item
  archived recording (gordon sondland)\\
  Yeah, it sounds like something I would say. {[}LAUGHTER{]}
\end{itemize}

nicholas fandos

And what he basically does over the course of this back-and-forth is
says ---

\begin{itemize}
\tightlist
\item
  archived recording (gordon sondland)\\
  Putting it in Trump-speak, by saying he loves your ass, he'll do
  whatever you want, meant that he would really work with us on a whole
  host of issues.
\end{itemize}

nicholas fandos

--- you've heard the president talk.

\begin{itemize}
\tightlist
\item
  archived recording (gordon sondland)\\
  That's how President Trump and I communicate, a lot of four-letter
  words. In this case, three-letter.
\end{itemize}

nicholas fandos

He also contests a couple of small points.

\begin{itemize}
\tightlist
\item
  archived recording (gordon sondland)\\
  I don't think I would have said that. I would have --- I would have
  ---
\end{itemize}

nicholas fandos

But in the end, he confirms the essence of the story as it's been
related by other witnesses, and for the first time, kind of established
that that happened.

\begin{itemize}
\tightlist
\item
  archived recording (gordon sondland)\\
  Again, trying to reconstruct a very busy day without the benefit, but
  if someone said I had a meeting, and I went to the meeting, then I'm
  not going to dispute that.
\end{itemize}

michael barbaro

So it's kind of that familiar mix of Sondland's self-protection, lack of
recall, but overall, damning testimony.

nicholas fandos

I think that's right.

\begin{itemize}
\tightlist
\item
  archived recording (daniel goldman)\\
  Mr. Chairman, I yield back.
\end{itemize}

nicholas fandos

So the Democrats wrap up their questioning.

\begin{itemize}
\tightlist
\item
  archived recording (adam schiff)\\
  That concludes our 45 minutes ---
\end{itemize}

nicholas fandos

And after weeks of investigation, literally hundreds of hours of
testimony, in private and in public, they've finally gotten, from a lead
witness who had direct access to the president, a statement, an
admission, of one of the things that they've been looking for and trying
to prove out all along, that this White House meeting, an official act,
was conditioned upon Ukraine publicly announcing investigations. And
that Sondland, a top official, also believed that the withheld aid money
--- that that would only flow, too, if the investigations were
announced. And so it's with that in front of them ---

\begin{itemize}
\item
  archived recording (adam schiff)\\
  Why don't we take a 5- or 10-minute break?
\item
  archived recording (gordon sondland)\\
  Thank you.
\end{itemize}

nicholas fandos

That Schiff abruptly gavels the hearing into a pause, somewhat
inexplicably at first. And then we quickly see why as he walks out
behind the chamber ---

\begin{itemize}
\tightlist
\item
  archived recording (adam schiff)\\
  Just want to make a couple of quick observations while we're on a
  break here.
\end{itemize}

nicholas fandos

--- to a bank of cameras, where he begins to talk about what just took
place.

\begin{itemize}
\tightlist
\item
  archived recording (adam schiff)\\
  And what we have just heard from Ambassador Sondland is that the
  knowledge of this scheme was a basic quid pro quo. It was the
  conditioning ---
\end{itemize}

nicholas fandos

Which all, of course, carried it live in the middle of the proceedings.
And when he was done ---

\begin{itemize}
\tightlist
\item
  archived recording (adam schiff)\\
  So I think a very important moment in the history of this inquiry.
\end{itemize}

nicholas fandos

--- he wrapped up and walked right back into the hearing room and
gaveled the thing back into session.

\begin{itemize}
\tightlist
\item
  archived recording (adam schiff)\\
  Come to order.
\end{itemize}

nicholas fandos

Now, it wasn't lost on Devin Nunes, the top Republican on the committee,
who quickly called Schiff out.

\begin{itemize}
\tightlist
\item
  archived recording (devin nunes)\\
  Thank the gentleman. For those of you watching at home, that was not a
  bathroom break. That was actually a chance for the Democrats to go out
  and hold a press conference, Ambassador, for all the supposed
  bombshells that were in your opening testimony.
\end{itemize}

nicholas fandos

So there's quite a bit of bad blood, obviously, between the two sides.
But I don't think that this helped.

michael barbaro

We'll be right back.

So once we're back, given that the Democrats have already sort of
embraced the messiness and unreliability of Gordon Sondland and
addressed it pretty directly, what is the Republican strategy when it's
their turn to question?

nicholas fandos

Well, Republicans are looking at the same guy and the same tendencies,
and they see an amazing opportunity to weaponize it in defense of the
president.

\begin{itemize}
\item
  archived recording (steve castor)\\
  Hello again, Ambassador.
\item
  archived recording (gordon sondland)\\
  Hi.
\end{itemize}

nicholas fandos

So it's left to Steve Castor, who's the Republicans' lawyer in all of
this, to start trying to do that work.

\begin{itemize}
\tightlist
\item
  archived recording (steve castor)\\
  I just want to go through some distinctions between your opener and
  your deposition.
\end{itemize}

nicholas fandos

There's a phone call that Sondland has where he says that he was getting
such mixed signals and couldn't get answers about what Trump wanted from
Ukraine and why he was withholding the aid money.

\begin{itemize}
\tightlist
\item
  archived recording (gordon sondland)\\
  And I was getting tired of going around in circles, frankly. So I made
  the call ---
\end{itemize}

nicholas fandos

And he asked President Trump, just kind of point blank ---

\begin{itemize}
\item
  archived recording (gordon sondland)\\
  --- what do you want from Ukraine? And that's when I got the answer.
\item
  archived recording (steve castor)\\
  And he was unequivocal, nothing.
\end{itemize}

nicholas fandos

The president tells him, I don't want a quid pro quo. I don't want
anything from Ukraine. They should just do what's right. They should do
what Zelensky said he was going to do. And Republicans kept going back
to this call again and again and again.

\begin{itemize}
\item
  archived recording (john ratcliffe)\\
  Tell me if there's anything sinister or nefarious in any of this, a
  vanilla request about corruption, a call to say, I'm on my way to
  Ukraine, a five-minute call you didn't remember as significant, a call
  that you made where the president said, I want nothing, I want no quid
  pro quo, I want Zelensky to do the right thing, I want him to do what
  he ran on, and him telling you to go tell Congress the truth. Anything
  sinister and nefarious about any of that?
\item
  archived recording (gordon sondland)\\
  Not the way you present it.
\item
  archived recording (john ratcliffe)\\
  O.K., and that is the truth.
\end{itemize}

nicholas fandos

Because in this case, as they pointed out, you know, when the president
is speaking directly to Sondland, he's saying, no quid pro quo. I don't
want anything in particular from the Ukrainians.

\begin{itemize}
\tightlist
\item
  archived recording (jim jordan)\\
  And you told Mr. Castor that the president never told you that the
  announcement had to happen to get anything. In fact, he didn't just
  not tell you that, he explicitly said the opposite.
\end{itemize}

michael barbaro

So the most senior figure, and the most involved figure in all this from
the Trump administration, who declares that this was a quid pro quo, is
simultaneously testifying that the president gets on the phone with him
and says, this is no quid pro quo. That is complicated.

nicholas fandos

It is. It's very complicated. And if you're listening at home and
listening to Republican questioning, they're able, I think, to raise
some doubts about this account that he's giving.

\begin{itemize}
\tightlist
\item
  archived recording (michael turner)\\
  And if you pull up CNN today, right now, their banner says, ``Sondland
  ties Trump to withholding aid.'' Is that your testimony today,
  Ambassador Sondland, that you have evidence that Donald Trump tied the
  investigation to the aide? Because I don't think you're saying that.
\end{itemize}

nicholas fandos

And in this questioning, you know, Sondland says, I didn't hear anything
directly.

\begin{itemize}
\item
  archived recording (gordon sondland)\\
  I've said repeatedly, Congressman, I was presuming. I also said that
  President Trump ---
\item
  archived recording (michael turner)\\
  So no one told you, not just the president. Giuliani didn't tell you.
  Mulvaney didn't tell you. Nobody --- Pompeo didn't tell you --- nobody
  else on this planet told you that Donald Trump was tying aid to these
  investigations. Is that correct?
\item
  archived recording (gordon sondland)\\
  I think I already testified to that.
\item
  archived recording (michael turner)\\
  No, answer the question, yes or no.
\item
  archived recording (gordon sondland)\\
  Yes.
\end{itemize}

michael barbaro

Nick, why do you think that they are so focused on that, what the
president told him versus what he has concluded from all the information
and conversations around him?

nicholas fandos

Republicans' argument is basically that, if you're going to impeach the
president, you know, we need to know directly what his intentions were.
What he told you, that's evidence. That's primary evidence. What you
concluded, I mean, that's no better than what we've heard from other
witnesses. It's hearsay.

\begin{itemize}
\item
  archived recording (michael turner)\\
  I mean, that's what I don't understand. So, you know what hearsay
  evidence is, Ambassador? Hearsay is when I testify what someone else
  told me. You know what made-up testimony is? Made-up testimony is when
  I just presume it. I mean, you're just assuming all of these things,
  and then you're giving them the evidence --- that they're running out
  and doing press conferences, and CNN's headline is saying that you're
  saying the president of the United States should be impeached because
  he tied aid to investigations. And you don't know that. Correct?
\item
  archived recording (gordon sondland)\\
  I never said the president of the United States should be impeached.
\item
  archived recording (michael turner)\\
  Nope, but you did ---
\end{itemize}

nicholas fandos

So as this hearing starts to wind down after five or six hours of this
back-and-forth, after a pretty explosive and direct opening statement
and then cross-examination and examination and more cross-examination,
spirits are starting to flag a little bit in the hearing room.

\begin{itemize}
\item
  archived recording (adam schiff)\\
  Mr. Maloney.
\item
  archived recording (sean patrick maloney)\\
  Mr. Ambassador, let's pick up right there.
\end{itemize}

nicholas fandos

There's an exchange near the end with Sean Patrick Maloney, a Democrat
of New York.

\begin{itemize}
\item
  archived recording (sean patrick maloney)\\
  Let me ask you something. Who would have benefited from an
  investigation of the president's political opponents?
\item
  archived recording (gordon sondland)\\
  I don't want to characterize who would have and who would not have.
\item
  archived recording (sean patrick maloney)\\
  I know you don't want to, sir. That's my question. Would you answer it
  for me?
\end{itemize}

nicholas fandos

And Maloney, you know, seeming to kind of burst forth with Democratic
frustration that's been boiling beneath the surface all day, basically
says ---

\begin{itemize}
\item
  archived recording (sean patrick maloney)\\
  I guess I'm have trouble why you can't just say ---
\item
  archived recording (gordon sondland)\\
  When he asked about investigations, I assumed he meant ---
\item
  archived recording (sean patrick maloney)\\
  I know what you assumed. But who would benefit from an investigation
  of the Bidens?
\item
  archived recording (gordon sondland)\\
  I assume President Trump would benefit.
\item
  archived recording (sean patrick maloney)\\
  There we have it. See? {[}APPLAUSE{]} Didn't hurt a bit, did it? But
  let me ask you something ---
\item
  archived recording (gordon sondland)\\
  Mr. Maloney?
\item
  archived recording (sean patrick maloney)\\
  Hold on, sir.
\item
  archived recording (gordon sondland)\\
  Excuse me. I've been very forthright, and I really resent what you're
  trying to do.
\item
  archived recording (sean patrick maloney)\\
  Fair enough. You've been very forthright. This your third try to do
  so, sir. Didn't work so well the first time, did it? And now we're
  here a third time, and we got a doozy of a statement from you this
  morning. There's a whole bunch of stuff you don't recall. So all due
  respect, sir, we appreciate your candor. But let's be really clear on
  what it took to get it out of you.
\end{itemize}

nicholas fandos

And that's kind of the tone as this hearing begins to come to an end.
Democrats are frustrated about some things. Republicans are frustrated
about some things. Sondland, who's trying to catch a flight back to
Brussels, is certainly frustrated by some things. And we've had a pretty
complex and conflicting day of testimony, where some people seem to be
coming away quite happy, but not perfectly so.

michael barbaro

So is this unreliable witness turning out to be the most important
witness in this inquiry? Or is he just an unreliable witness?

nicholas fandos

Well, the thing about Sondland is, the Democrats will tell you,
prosecutors successfully bring cases all the time on highly flawed
witnesses who maybe don't even have a great history with the truth. But
that doesn't mean that what they're saying isn't true in a given
scenario. And in this case, it's important to remember, what he's saying
is incredibly, politically inconvenient for him. He still works for the
president of the United States, who he donated a bunch of money to and
whose policies he believes in. That itself lends it some power and
credibility.

Democrats seem to be emerging from today more comfortable and more
certain that they need to bring forward this case of the president
abusing his office, of committing high crimes and misdemeanors worthy of
impeachment and putting it before the American people.

michael barbaro

Thank you, Nick.

nicholas fandos

Thanks for having me, Michael.

michael barbaro

For the next few weeks, we'll be covering the latest developments in the
impeachment inquiry in our new podcast. It's called ``The Latest.'' You
can hear these episodes at the end of the day, right here on ``The
Daily.'' Or subscribe to ``The Latest'' wherever you listen. We'll be
right back.

Here's what else you need to know today.

\begin{itemize}
\tightlist
\item
  archived recording (elizabeth warren)\\
  How did Ambassador Sondland get there? You know, this is not a man who
  had any qualifications except one. He wrote a check for a million
  dollars.
\end{itemize}

michael barbaro

In the fifth Democratic presidential debate, candidates expressed
outrage over Wednesday's testimony in the impeachment inquiry, with
Senator Elizabeth Warren accusing President Trump of selling off key
ambassadorships to wealthy donors like Gordon Sondland.

\begin{itemize}
\tightlist
\item
  archived recording (elizabeth warren)\\
  And that tells us about what's happening in Washington.
\end{itemize}

michael barbaro

But with Mayor Pete Buttigieg now leading the polls in Iowa, the first
state to pick a Democratic nominee, several of his rivals, including
Senator Amy Klobuchar, sought to challenge his credentials and
experience.

\begin{itemize}
\tightlist
\item
  archived recording (amy klobuchar)\\
  Just like I have won statewide, and Mayor, I have all appreciation for
  your good work as a local official, and you did not when you tried, I
  also have actually done this work. I think experience should matter.
\end{itemize}

michael barbaro

Buttigieg fired back, suggesting that the federal experience of his
opponent was its own liability.

\begin{itemize}
\tightlist
\item
  archived recording (pete buttigieg)\\
  So first of all, Washington experience is not the only experience that
  matters. There's more than 100 years of Washington experience on this
  stage. And where are we right now as a country? {[}CHEERING{]}
\end{itemize}

michael barbaro

That's it for ``The Daily.'' I'm Michael Barbaro. See you tomorrow.

\href{https://www.nytimes.com/2019/11/21/podcasts/the-daily/impeachment-sondland.html}{Mr.
Sondland} told the committee that he and other advisers to Mr. Trump
pressured Ukraine to investigate Democrats ``because the president
directed us to do so.''

Mr. Sondland said that he, Energy Secretary Rick Perry and Kurt D.
Volker, the special envoy for Ukraine, were reluctant to work with Mr.
Giuliani on the pressure campaign and agreed only at Mr. Trump's
insistence.

``Secretary Perry, Ambassador Volker and I worked with Mr. Rudy Giuliani
on Ukraine matters at the express direction of the president of the
United States,'' Mr. Sondland told the committee. ``We did not want to
work with Mr. Giuliani. Simply put, we were playing the hand we were
dealt.'' With no alternative, he said, ``we followed the president's
orders.''

Mr. Sondland confirmed what has already been known, that there was a
clear ``quid pro quo'' linking a coveted White House meeting for
Ukraine's president to the investigations Mr. Trump wanted. And he said
he was concerned about ``a potential quid pro quo'' linking \$391
million in security aid that Mr. Trump suspended to the investigations
he desired.

But under questioning, Mr. Sondland acknowledged that Mr. Trump never
told him that. ``I never heard from President Trump that aid was
conditioned on an announcement of investigations,'' he testified.

And he was asked by Republicans to repeat a conversation he had with Mr.
Trump that he has previously described in which he asked the president
what he wanted from Ukraine. ``It was a very short, abrupt
conversation,'' Mr. Sondland said. ``He was not in a good mood. And he
just said, `I want nothing. I want nothing. I want no quid pro quo. Tell
Zelensky to do the right thing.'''

The conversation took place after the White House had already learned a
whistle-blower had come forward with a complaint alleging that the
president was abusing his power to try to enlist Ukraine to interfere on
his behalf in the 2020 election.

Mr. Giuliani challenged Mr. Sondland in a tweet, saying the ambassador
was ``speculating based on VERY little contact. I never met him and had
very few calls with him, mostly with Volker. Volker testified I answered
their questions and described them as my opinions, NOT demands. I.E. no
quid pro quo.''

He later deleted the tweet.

Mr. Perry also took issue with Mr. Sondland, issuing a statement through
his department saying that the testimony ``misrepresented both Secretary
Perry's interaction with Rudy Giuliani and direction the secretary
received from President Trump.''

The statement said Mr. Perry spoke with Mr. Giuliani only once. ``At no
point before, during or after that phone call did the words `Biden' or
`Burisma' ever come up in the presence of Secretary Perry,'' the
statement said.

\href{https://www.nytimes.com/interactive/2019/11/20/us/politics/gordon-sondland-opening-statement-ukraine.html}{}

\includegraphics{https://static01.nyt.com/images/2019/11/20/us/politics/GORDON-SONDLAND-OPENING-STATEMENT-UKRAINE/GORDON-SONDLAND-OPENING-STATEMENT-UKRAINE-articleLarge.jpg}

\hypertarget{read-gordon-sondlands-opening-statement}{%
\subsection{Read Gordon Sondland's Opening
Statement}\label{read-gordon-sondlands-opening-statement}}

The United States ambassador to the European Union testified that he
pressured Ukraine for investigations at President Trump's ``express
direction.''

\hypertarget{everyone-was-in-the-loop-sondland-said-including-pence-pompeo-mulvaney-and-others}{%
\subsection{`Everyone was in the loop,' Sondland said, including Pence,
Pompeo, Mulvaney and
others.}\label{everyone-was-in-the-loop-sondland-said-including-pence-pompeo-mulvaney-and-others}}

Image

Secretary of State Mike Pompeo at the White House last
week.Credit...T.J. Kirkpatrick for The New York Times

Mr. Sondland testified that he told Vice President Mike Pence in late
August that he feared the military aid withheld from Ukraine was tied to
the investigations Mr. Trump sought and that he kept Secretary of State
Mike Pompeo apprised of his efforts to pressure Ukraine.

The revelations suggested that Mr. Sondland has decided to publicly
implicate the senior-most members of Mr. Trump's administration in the
matter, including Mick Mulvaney, the acting White House chief of staff,
and he provided a series of text messages and emails to buttress his
account.

``Everyone was in the loop,'' he said told the committee. ``It was no
secret.''

If other officials were concerned that he was doing something wrong, as
testimony now indicates, Mr. Sondland said they did not tell him at the
time. ``Everyone's hair was on fire,'' he said, ``but no one decided to
talk to us.''

The striking account --- a major departure from Mr. Sondland's initial
closed-door testimony in the impeachment inquiry last month --- also
indicated that the ambassador who played a central role in the pressure
campaign was eager to demonstrate that he did so only reluctantly with
the knowledge and approval of the president and top members of his team.

Mr. Sondland rejected the notion that he was part of an illicit
\href{https://www.nytimes.com/interactive/2019/11/18/us/politics/trump-ukraine-impeachment-testimony.html}{shadow
foreign policy} that worked around the normal national security process.
``The suggestion that we were engaged in some irregular or rogue
diplomacy is absolutely false,'' he said, pointing to messages and phone
calls in which he
\href{https://www.nytimes.com/2019/11/20/us/politics/sondland-pompeo-ukraine.html}{kept
the White House and State Department informed} of his actions. He added:
``Any claim that I somehow muscled my way into the Ukraine relationship
is simply false.''

The ambassador said that he ``mentioned to Vice President Pence before
the meetings with the Ukrainians that I had concerns that the delay in
aid had become tied to the issue of investigations.'' He testified that
the conversation occurred shortly before Mr. Pence met with President
Volodymyr Zelensky of Ukraine while they were in Warsaw.

At that meeting, Mr. Zelensky brought up the issue of the withheld aid
and Mr. Pence said he would discuss the matter with Mr. Trump.
Afterward, Mr. Sondland said he informed Andriy Yermak, a top Ukrainian
official, that the money would probably not flow without Mr. Zelensky
making a public commitment to the investigations.

Marc Short, Mr. Pence's chief of staff, issued a statement after his
testimony denying Mr. Sondland's account.

``The vice president never had a conversation with Gordon Sondland about
investigating the Bidens, Burisma, or the conditional release of
financial aid to Ukraine based upon potential investigations,'' Mr.
Short said. ``This alleged discussion recalled by Ambassador Sondland
never happened.''

Mr. Sondland also said that ``even as late as September,'' after the
pressure campaign emerged in the news media, ``Secretary Pompeo was
directing Kurt Volker to speak with Mr. Giuliani.''

In a statement issued from Mr. Pompeo's plane as he returned to
Washington from Brussels, his spokeswoman denied something that Mr.
Sondland never testified to.

``Gordon Sondland never told Secretary Pompeo that he believed the
president was linking aid to investigations of political opponents,''
Morgan Ortagus, the State Department spokeswoman, said in the statement.
``Any suggestion to the contrary is flat out false.''

\hypertarget{trump-dismissed-sondland-saying-i-dont-know-him-very-well}{%
\subsection{Trump dismissed Sondland, saying `I don't know him very
well.'}\label{trump-dismissed-sondland-saying-i-dont-know-him-very-well}}

Image

President Trump speaking to reporters at the White House on
Wednesday.Credit...T.J. Kirkpatrick for The New York Times

President Trump distanced himself from Gordon D. Sondland, a top donor
he appointed as ambassador to the European Union, after the diplomat
told lawmakers that he and other advisers pressured Ukraine to
investigate Democrats at the president's ``express direction.''

As he headed to Marine One to depart on a trip to Texas, Mr. Trump
stopped to talk with reporters briefly and pointed out that Mr. Sondland
had testified that the president had told him at one point that he
wanted nothing from Ukraine and there was no quid pro quo.

``That means it's all over,'' Mr. Trump said, shouting over the roar of
helicopter rotors and reading from handwritten notes scrawled out in
large block letters. ``This is the final word from the president of the
United States: `I want nothing.'''

Image

Mr. Trump holding handwritten notes.Credit...Erin Scott/Reuters

In a tweet later Wednesday afternoon, Mr. Trump declared the
``impeachment witch hunt'' to be over, quoting Mr. Sondland's testimony
in all caps.

The president's press secretary, Stephanie Grisham, later issued a
statement emphasizing those points. ``Ambassador Sondland's testimony
made clear that in one of the few brief phone calls he had with
President Trump, the president clearly stated that he `wanted nothing'
from Ukraine and repeated `no quid pro quo over and over again,''' she
said.

Despite that, Mr. Sondland told the House Intelligence Committee on the
fourth day of public impeachment hearings that it was clear to him that
the president was intently interested in having the Ukrainians publicly
commit to investigating Democrats, including former Vice President
Joseph R. Biden Jr., whose son served on the board of the Ukrainian
energy company Burisma.

Mr. Trump often disavows knowing advisers once they become problematic
for him. Just last month, Mr. Trump called Mr. Sondland, who gave the
president's inaugural fund \$1 million,
\href{https://twitter.com/realDonaldTrump/status/1181560708808486914?s=20}{``a
really good man and great American.''}

But on Wednesday he said: ``I don't know him very well. I have not
spoken to him much. This is not a man I know well. He seems like a nice
guy though.'' Ms. Grisham's statement amplified that by referring to
``the few brief phone calls'' she said the two men have had.

Mr. Sondland portrayed their relationship differently, describing it as
a chummy one that ranged even beyond the issues at hand. ``I've had a
lot of conversations with the president about completely unrelated
matters that have nothing to do with Ukraine,'' he said. Their
conversations, he testified, featured, ``a lot of four-letter words.''

\hypertarget{democrats-argued-that-sondlands-testimony-bolstered-their-case-for-impeaching-trump}{%
\subsection{Democrats argued that Sondland's testimony bolstered their
case for impeaching
Trump.}\label{democrats-argued-that-sondlands-testimony-bolstered-their-case-for-impeaching-trump}}

Image

Representative Adam B. Schiff, chairman of the Intelligence Committee,
listening to Mr. Sondland during the hearing.Credit...Anna
Moneymaker/The New York Times

After Mr. Sondland testified that everyone from Mr. Trump on down was
aware of the pressure campaign on Ukraine, House Democrats quickly
declared that he had bolstered their case for impeachment.

Representative Adam B. Schiff, Democrat of California and chairman of
the House Intelligence Committee, called Mr. Sondland's testimony
``among the most significant evidence to date,'' saying he described ``a
basic quid pro quo'' that conditioned American security aid on Ukraine
agreeing to investigate Mr. Trump's political rivals.

Mr. Schiff mocked Republican attempts to undermine Mr. Sondland's
testimony, saying that his colleagues on the Intelligence Committee
``seem to be under impression that unless the president spoke the words,
`Ambassador Sondland, I am bribing the Ukrainian president,' that
there's no evidence of bribery. If he didn't say, `Ambassador Sondland,
I'm telling you I'm not going to give the aid unless they do this,' that
there's no evidence of a quid pro quo.''

``Nonetheless,'' Mr. Schiff said, ``you have given us a lot of evidence
of precisely that conditionality.''

Republicans scoffed. Representative Mike Turner, Republican of Ohio,
pressed Mr. Sondland to acknowledge that he was never explicitly told
that Ukraine's military aid was tied to the investigations that Mr.
Trump wanted.

``No one told you? Not just the president --- Giuliani didn't tell you,
Mulvaney didn't tell you, nobody,'' Mr. Turner said. ``Pompeo didn't
tell you?

``No one on this planet told you that President Trump was tying aid to
investigations,'' Mr. Turner added. ``Yes or no?''

``Yes,'' Mr. Sondland answered.

\hypertarget{for-trump-what-mattered-most-was-that-investigations-of-democrats-be-publicly-announced}{%
\subsection{For Trump, what mattered most was that investigations of
Democrats be publicly
announced.}\label{for-trump-what-mattered-most-was-that-investigations-of-democrats-be-publicly-announced}}

Image

Reporters watching the hearing in a hallway on Capitol
Hill.Credit...Erin Schaff/The New York Times

Under questioning, Mr. Sondland put his finger on a distinction that
often gets overlooked in the discussion of Mr. Trump's interest in
Ukraine: For the president, it seemed more important that Ukrainian
officials announce that they were investigating Democrats than for them
to actually follow through.

``I never heard, Mr. Goldman, anyone say that the investigations had to
start or had to be completed,'' Mr. Sondland
\href{https://www.nytimes.com/2019/11/19/us/politics/house-impeachment-lawyers-goldman-castor.html}{told
Daniel S. Goldman}, the top Democratic counsel who questioned him. ``The
only thing I heard from Mr. Giuliani or otherwise was that they had to
be announced in some form and that form kept changing.''

The distinction is important because Democrats are arguing that Mr.
Trump was not trying to fight corruption, but instead trying to enlist a
foreign power to discredit his rivals in a way that would benefit him in
the 2020 election. In pressing Mr. Sondland on the matter, Mr. Goldman
noted that, ``there would be political benefits to a public
announcement.''

Mr. Sondland responded, ``The way it was expressed to me was that the
Ukrainians had a long history of committing to things privately and then
never following through, so President Trump, presumably, again
communicated through Mr. Giuliani, wanted the Ukrainians on record
publicly that they were going to do these investigations.''

``But you never heard anyone say that they really wanted them to do the
investigations, just that they wanted to announce'' them, Mr. Goldman
said.

``I didn't hear either way,'' Mr. Sondland said. ``I didn't hear either
way.''

\hypertarget{sondland-confirmed-an-indiscreet-conversation-with-trump-but-disputed-descriptions-of-july-10-meeting}{%
\subsection{Sondland confirmed an indiscreet conversation with Trump but
disputed descriptions of July 10
meeting.}\label{sondland-confirmed-an-indiscreet-conversation-with-trump-but-disputed-descriptions-of-july-10-meeting}}

Image

Text messages between Mr. Sondland and other American diplomats in
Ukraine were displayed during the hearing.Credit...Erin Schaff/The New
York Times

Mr. Sondland in his prepared testimony confirmed a conversation with Mr.
Trump at a key moment in the timeline that he did not volunteer during
his original testimony. But he disputed descriptions by other witnesses
of another key meeting.

Mr. Sondland did not challenge the account of a lunch meeting on the
outdoor patio of a Kyiv restaurant on July 26, the day after Mr. Trump's
phone call with Mr. Zelensky. David Holmes, the political counselor at
the American Embassy in Ukraine, told investigators that he overheard
Mr. Trump and Mr. Sondland talking on the phone.

``So, he's going to do the investigation?'' Mr. Trump asked, according
to Mr. Holmes. Mr. Sondland told him yes. Mr. Zelensky ``loves your
ass'' and would do ``anything you ask him to,'' Mr. Sondland said,
according to Mr. Holmes's statement.

But in his testimony Wednesday, Mr. Sondland also denied that a July 10
meeting at the White House with Ukrainian officials turned sharply
tense, as others have testified in recent days.

Fiona Hill, then the senior director for Europe and Russia at the
National Security Council, and her deputy for Ukraine policy, Lt. Col.
Alexander S. Vindman, previously told lawmakers that the meeting led to
a confrontation over Mr. Sondland's unconventional role in Ukraine
policy.

Mr. Sondland said he did not remember that.

``Their recollections of those events simply don't square with my own or
with those of Ambassador Volker or Secretary Perry,'' he said in his
prepared testimony.

Emily Cochrane contributed reporting.

Advertisement

\protect\hyperlink{after-bottom}{Continue reading the main story}

\hypertarget{site-index}{%
\subsection{Site Index}\label{site-index}}

\hypertarget{site-information-navigation}{%
\subsection{Site Information
Navigation}\label{site-information-navigation}}

\begin{itemize}
\tightlist
\item
  \href{https://help.nytimes.com/hc/en-us/articles/115014792127-Copyright-notice}{©~2020~The
  New York Times Company}
\end{itemize}

\begin{itemize}
\tightlist
\item
  \href{https://www.nytco.com/}{NYTCo}
\item
  \href{https://help.nytimes.com/hc/en-us/articles/115015385887-Contact-Us}{Contact
  Us}
\item
  \href{https://www.nytco.com/careers/}{Work with us}
\item
  \href{https://nytmediakit.com/}{Advertise}
\item
  \href{http://www.tbrandstudio.com/}{T Brand Studio}
\item
  \href{https://www.nytimes.com/privacy/cookie-policy\#how-do-i-manage-trackers}{Your
  Ad Choices}
\item
  \href{https://www.nytimes.com/privacy}{Privacy}
\item
  \href{https://help.nytimes.com/hc/en-us/articles/115014893428-Terms-of-service}{Terms
  of Service}
\item
  \href{https://help.nytimes.com/hc/en-us/articles/115014893968-Terms-of-sale}{Terms
  of Sale}
\item
  \href{https://spiderbites.nytimes.com}{Site Map}
\item
  \href{https://help.nytimes.com/hc/en-us}{Help}
\item
  \href{https://www.nytimes.com/subscription?campaignId=37WXW}{Subscriptions}
\end{itemize}
