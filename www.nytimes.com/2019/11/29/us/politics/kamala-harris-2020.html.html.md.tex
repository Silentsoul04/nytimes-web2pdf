Sections

SEARCH

\protect\hyperlink{site-content}{Skip to
content}\protect\hyperlink{site-index}{Skip to site index}

\href{https://www.nytimes.com/section/politics}{Politics}

\href{https://myaccount.nytimes.com/auth/login?response_type=cookie\&client_id=vi}{}

\href{https://www.nytimes.com/section/todayspaper}{Today's Paper}

\href{/section/politics}{Politics}\textbar{}How Kamala Harris's Campaign
Unraveled

\url{https://nyti.ms/2L2GQYw}

\begin{itemize}
\item
\item
\item
\item
\item
\item
\end{itemize}

\begin{itemize}
\item
  \href{https://www.nytimes.com/2020/07/31/us/elections/biden-vs-trump.html?action=click\&pgtype=Article\&state=default\&region=TOP_BANNER\&context=storylines_menu}{Election
  Updates}
\item
  \href{https://www.nytimes.com/article/biden-vice-president-2020.html?action=click\&pgtype=Article\&state=default\&region=TOP_BANNER\&context=storylines_menu}{Biden's
  V.P. Search}
\item
  \href{https://www.nytimes.com/interactive/2020/07/24/us/politics/trump-biden-campaign-donors.html?action=click\&pgtype=Article\&state=default\&region=TOP_BANNER\&context=storylines_menu}{Map
  of Donations}
\item
  \href{https://www.nytimes.com/interactive/2020/us/elections/delegate-count-primary-results.html?action=click\&pgtype=Article\&state=default\&region=TOP_BANNER\&context=storylines_menu}{Delegate
  Count}
\item
  \href{https://www.nytimes.com/interactive/2019/us/politics/2020-presidential-candidates.html?action=click\&pgtype=Article\&state=default\&region=TOP_BANNER\&context=storylines_menu}{The
  Candidates}
\item
  \href{https://www.nytimes.com/newsletters/politics?action=click\&pgtype=Article\&state=default\&region=TOP_BANNER\&context=storylines_menu}{Politics
  Newsletter}
\end{itemize}

Advertisement

\protect\hyperlink{after-top}{Continue reading the main story}

Supported by

\protect\hyperlink{after-sponsor}{Continue reading the main story}

\hypertarget{how-kamala-harriss-campaign-unraveled}{%
\section{How Kamala Harris's Campaign
Unraveled}\label{how-kamala-harriss-campaign-unraveled}}

Ms. Harris is the only 2020 Democrat who has fallen hard out of the top
tier of candidates. She has proved to be an uneven campaigner who
changes her message and tactics to little effect and has a staff torn
into factions.

\includegraphics{https://static01.nyt.com/images/2019/11/30/us/politics/30harris-jump1-alt/merlin_162352476_24d90952-c16b-4099-8860-d097259f3c74-articleLarge.jpg?quality=75\&auto=webp\&disable=upscale}

\href{https://www.nytimes.com/by/jonathan-martin}{\includegraphics{https://static01.nyt.com/images/2018/11/06/multimedia/author-jonathan-martin/author-jonathan-martin-thumbLarge.png}}\href{https://www.nytimes.com/by/astead-w-herndon}{\includegraphics{https://static01.nyt.com/images/2018/09/14/us/author-head-astead/author-head-astead-thumbLarge-v2.png}}\href{https://www.nytimes.com/by/alexander-burns}{\includegraphics{https://static01.nyt.com/images/2018/09/25/multimedia/author-alexander-burns/author-alexander-burns-thumbLarge-v2.png}}

By \href{https://www.nytimes.com/by/jonathan-martin}{Jonathan Martin},
\href{https://www.nytimes.com/by/astead-w-herndon}{Astead W. Herndon}
and \href{https://www.nytimes.com/by/alexander-burns}{Alexander Burns}

\begin{itemize}
\item
  Published Nov. 29, 2019Updated May 11, 2020
\item
  \begin{itemize}
  \item
  \item
  \item
  \item
  \item
  \item
  \end{itemize}
\end{itemize}

WASHINGTON --- In early November, a few days after Senator
\href{https://www.nytimes.com/2020/05/11/us/politics/kamala-harris-biden-vp.html}{Kamala
Harris}'s presidential campaign announced widespread layoffs and an
intensified focus on Iowa, her senior aides gathered for a staff meeting
at their Baltimore headquarters and pelted the campaign manager, Juan
Rodriguez, with questions.

What exactly was Ms. Harris's new strategy? How much money and manpower
could they put into Iowa? What would their presence be like in other
early voting states?

Mr. Rodriguez offered general, tentative answers that didn't satisfy the
room, according to two campaign officials directly familiar with the
conversation. Some Harris aides sitting at the table could barely
suppress their fury about what they saw as the undoing of a
once-promising campaign. Their feelings were reflected days later by
Kelly Mehlenbacher, the state operations director, in a blistering
resignation letter obtained by The Times.

``This is my third presidential campaign and I have never seen an
organization treat its staff so poorly,'' Ms. Mehlenbacher wrote,
assailing Mr. Rodriguez and Ms. Harris's sister, Maya, the campaign
chairwoman, for
\href{https://www.nytimes.com/2019/10/30/us/politics/kamala-harris-campaign.html}{laying
off aides} with no notice. ``With less than 90 days until Iowa we still
do not have a real plan to win.''

The 2020 Democratic field has been defined by its turbulence, with some
contenders rising, others dropping out and two more jumping in just this
month. Yet there is only one candidate who rocketed to the top tier and
then plummeted in early state polls to the low single digits: Ms.
Harris.

\hypertarget{kelly-mehlenbachers-resignation-letter}{%
\subsection{Kelly Mehlenbacher's Resignation
Letter}\label{kelly-mehlenbachers-resignation-letter}}

The state operations director for Kamala Harris's presidential campaign
wrote this resignation letter earlier this month. (PDF, 1 page, 0.31 MB)

\includegraphics{https://int.nyt.com/data/documenthelper/6541-mehlenbacher-harris-resignation-letter/f0475e62906e5d2056f8/optimized/thumbnail.png}

From those polling results to Ms. Harris's campaign operation,
fund-raising and debate performances, it has been a remarkable comedown
for a senator from the country's largest state, a politician with star
power who was compared to President Obama even before Californians
elected her to the Senate in 2016.

Yet, even to some Harris allies, her decline is more predictable than
surprising. In one instance after another, Ms. Harris and her closest
advisers made flawed decisions about which states to focus on, issues to
emphasize and opponents to target, all the while refusing to make
difficult personnel choices to impose order on an unwieldy campaign,
according to more than 50 current and former campaign staff members and
allies, most of whom spoke on condition of anonymity to disclose private
conversations and assessments involving the candidate.

Many of her own advisers are now pointing a finger directly at Ms.
Harris. In interviews several of them criticized her for going on the
offensive against rivals, only to retreat, and for not firmly choosing a
side in the party's ideological feud between liberals and moderates. She
also created an organization with a campaign chairwoman, Maya Harris,
who goes unchallenged in part because she is Ms. Harris's sister, and a
manager, Mr. Rodriguez, who could not be replaced without likely
triggering the resignations of the candidate's consulting team. Even at
this late date, aides said it's unclear who's in charge of the campaign.

With just over two months until the Iowa caucuses, her staff is now
riven between competing factions eager to belittle one another, and the
candidate's relationship with Mr. Rodriguez has turned frosty, according
to multiple Democrats close to Ms. Harris. Several aides, including
Jalisa Washington-Price, the state director in crucial South Carolina,
have already had conversations about post-campaign jobs.

Representative Marcia Fudge, who has endorsed Ms. Harris and is a former
chairwoman of the Congressional Black Caucus, said in an interview that
the senator was an exceptional candidate who had been poorly served by
some top staff and who must fire Mr. Rodriguez. But she also
acknowledged that Ms. Harris bore a measure of responsibility for her
problems --- ``it's her campaign'' --- and that the structure she
created has not served her well.

\hypertarget{latest-updates-2020-election}{%
\section{\texorpdfstring{\href{https://www.nytimes.com/2020/07/31/us/elections/biden-vs-trump.html?action=click\&pgtype=Article\&state=default\&region=MAIN_CONTENT_1\&context=storylines_live_updates}{Latest
Updates: 2020
Election}}{Latest Updates: 2020 Election}}\label{latest-updates-2020-election}}

Updated 2020-08-01T01:26:45.732Z

\begin{itemize}
\tightlist
\item
  \href{https://www.nytimes.com/2020/07/31/us/elections/biden-vs-trump.html?action=click\&pgtype=Article\&state=default\&region=MAIN_CONTENT_1\&context=storylines_live_updates\#link-29fdff45}{Kamala
  Harris, a top vice-presidential contender, confronts double
  standards.}
\item
  \href{https://www.nytimes.com/2020/07/31/us/elections/biden-vs-trump.html?action=click\&pgtype=Article\&state=default\&region=MAIN_CONTENT_1\&context=storylines_live_updates\#link-13ec3d9c}{Karen
  Bass and Susan Rice are rising on Biden's vice-presidential
  shortlist.}
\item
  \href{https://www.nytimes.com/2020/07/31/us/elections/biden-vs-trump.html?action=click\&pgtype=Article\&state=default\&region=MAIN_CONTENT_1\&context=storylines_live_updates\#link-49e9a016}{Trump
  says Russian bounties to kill U.S. troops `never took place.'}
\end{itemize}

\href{https://www.nytimes.com/2020/07/31/us/elections/biden-vs-trump.html?action=click\&pgtype=Article\&state=default\&region=MAIN_CONTENT_1\&context=storylines_live_updates}{See
more updates}

``I have told her there needs to be a change,'' said Ms. Fudge, one of
several women of color who have been delivering hard-to-hear advice to
Ms. Harris in recent weeks. ``The weakness is at the top. And it's
clearly Juan. He needs to take responsibility --- that's where the buck
stops.''

Ms. Harris declined an interview request for this article.

Mr. Rodriguez, in a statement, said: ``Our team, from the candidate to
organizers across the country, are working day in and out to make sure
Kamala is the nominee to take on Donald Trump and end the national
nightmare that is his presidency. Just like every campaign, we have made
tough decisions to have the resources we need to place in Iowa and
springboard into the rest of the primary calendar.''

Ms. Harris is reluctant to make a leadership change within her campaign
so late in the race, some aides say, but they describe her as cleareyed
about the mistakes she has made and the difficulty of her task ahead.
They say she has bought into
\href{https://www.nytimes.com/2019/10/06/us/politics/kamala-harris-campaign-iowa.html}{focusing
on Iowa}, where her campaign has structured more one-on-one settings for
her to woo supporters or at least enjoy herself in otherwise difficult
days.

But her troubles go beyond staffing and strategy: Her financial
predicament is
\href{https://www.nytimes.com/interactive/2019/10/16/us/elections/democratic-q3-fundraising.html}{dire}.
The campaign has not taken a poll or been able to afford TV advertising
since September, and it has all but quit buying Facebook ads in the last
two months. Her advisers, after months of resistance, have only now
signaled their desire for a group of former aides to begin a super PAC
to finance an independent political effort on her behalf.

To some Democrats who know Ms. Harris, her struggles indicate larger
limitations.

``You can't run the country if you can't run your campaign,'' said Gil
Duran, a former aide to Ms. Harris and other California Democrats who's
now the editorial page editor of the Sacramento Bee.

\includegraphics{https://static01.nyt.com/images/2019/11/30/us/politics/29harris2/merlin_159211332_f302e082-6a1c-47e3-80df-4968a583f73a-articleLarge.jpg?quality=75\&auto=webp\&disable=upscale}

Some of her problems have been beyond her control. Health care policy
and the identity of the Democratic Party became much-debated issues this
year, but she had never given the details of either matter extensive
thought as she rose from local prosecutor to California attorney general
to the Senate. And her supporters believe that as a black woman, Ms.
Harris has run into difficulty with some voters over one of the defining
issues of the race: assumptions about who can and cannot defeat
President Trump.

Ms. Harris is now attempting a pivot, taking a less scripted approach to
campaigning. On a conference call with donors after the last debate in
mid-November, Jim Margolis, a senior campaign adviser, pointed to her
improved performance as a case study in letting ``Kamala be Kamala,''
according to one person who participated in the call --- a reference to
Ms. Harris's strengths when she is listening to her competitors'
comments and reacting freely*.*

It was her abundant political skills --- strong on the stump, a warm
manner with voters and ferocity with the opposition that seemed to spell
trouble for Mr. Trump --- that convinced many Democrats of Ms. Harris's
potential.

Yet it has come to this: After beginning her candidacy with a speech
before
\href{https://www.nytimes.com/2019/01/27/us/politics/kamala-harris-rally-2020.html}{20,000
people in Oakland}, some of Ms. Harris's longtime supporters believe she
should consider dropping out in late December --- the deadline for
taking her name off the California primary ballot --- if she does not
show political momentum. Some advisers are already bracing for a primary
challenge, potentially from the billionaire Tom Steyer, should she run
for re-election to the Senate in 2022. Her senior aides plan to assess
next month whether she's made sufficient progress to remain in the race.

``For her to lose California would be really hard and it's not looking
good,'' said Susie Buell, a longtime Harris donor from the Bay Area.

\hypertarget{a-team-of-rivals-with-no-clear-message}{%
\subsection{A team of rivals with no clear
message}\label{a-team-of-rivals-with-no-clear-message}}

The fact that Ms. Harris is now banking on an Iowa-or-bust strategy
highlights a major strategic miscalculation early on that set her off on
the wrong track.

When she entered the race in January, she bet that the early voting
states of Iowa and New Hampshire would matter less to her political
fortunes than South Carolina, with its predominantly black Democratic
electorate. In this view, a strong showing in South Carolina, which
votes fourth, would vault her into racially diverse Super Tuesday states
like California that would propel her candidacy.

So for much of the year, she focused on competing against
\href{https://www.nytimes.com/interactive/2020/us/elections/joe-biden.html}{Joseph
R. Biden Jr.} in South Carolina and beyond. What her campaign did not
anticipate was that Mr. Biden would remain strong with many black
voters, and that Senator
\href{https://www.nytimes.com/interactive/2020/us/elections/elizabeth-warren.html}{Elizabeth
Warren} and Mayor
\href{https://www.nytimes.com/interactive/2020/us/elections/pete-buttigieg.html}{Pete
Buttigieg} would rise as threats in Iowa and New Hampshire.

Then there was Ms. Harris's campaign message. Extensive polling led her
to believe that there was great value in the word ``truth,'' so she
titled her 2019 memoir ``The Truths We Hold'' and made a similar phrase
the centerpiece of her early stump speech: ``Let's speak truth.'' But
she dropped the saying out of a belief that voters wanted something less
gauzy.

Her assumptions about the issues that would inspire Democrats were also
muddled: she began running on a tax cut aimed at lower- and
middle-income voters and then turned to a pay raise for teachers.

But those proposals also did little to animate voters, especially those
riveted by the ambitious policies of Ms. Warren and Senator
\href{https://www.nytimes.com/interactive/2020/us/elections/bernie-sanders.html}{Bernie
Sanders}, and before long Ms. Harris was downplaying what were her
signature proposals.

For a time, she sought to highlight a pragmatic agenda, about matters
she said voters thought about while lying awake at 3 a.m. Today, her
aides are given to gallows humor about just how many slogans and
one-liners she has cycled through, with one recalling how ```speak
truth' spring'' gave way to ```3 a.m.' summer'' before the current,
Trump-focused ```justice' winter.''

From the start, the campaign structure seemed ripe for conflict. Ms.
Harris divided her campaign between two coasts, basing her operation in
Baltimore but retaining some key advisers in the Bay Area. She
bifurcated the leadership between two decidedly different loyalists: her
sister, the chair, and Mr. Rodriguez, a trusted lieutenant who had
managed her 2016 Senate campaign. Mr. Rodriguez was a central figure at
the San Francisco-based consulting firm, SCRB, that had helped direct
Ms. Harris's rise for a decade; all of the firm's partners were lined up
to advise the presidential race.

The two camps were soon competing, each stocked with people who shared a
tight bond with Ms. Harris but who regarded each other with suspicion or
worse. The setup cost Ms. Harris opportunities to recruit some of her
party's most sought-after outside strategists and left her reliant on a
team less experienced in national politics than in California, an
overwhelmingly blue state where campaigns often turn on factional
infighting within the Democratic Party.

Dan Sena, a former executive director of the Democratic Congressional
Campaign Committee, met early with Ms. Harris's team and came away
concerned that they were overly reliant on political thinking shaped in
California's idiosyncratic political system.

``Winning in California requires a different road map, between a top-two
candidate system and the expensive TV markets,'' Mr. Sena said. ``When
it comes to winning there is a right way, the wrong way and the
California way.''

It was not only political tactics that divided the campaign: In the
spring, Maya Harris and the consulting team
\href{https://www.nytimes.com/2019/05/08/us/politics/kamala-harris-2020-trump.html}{were
at war} over whether the senator should embrace or downplay her record
as a prosecutor, which some on the left have criticized, a dilemma the
campaign has never resolved.

One campaign strategist said it was impossible to tell if
\href{https://www.nytimes.com/2019/05/08/us/politics/kamala-harris-2020-trump.html}{Maya
Harris} was speaking for herself, as an adviser, or as her sister's
representative. She has exercised broad influence over even logistical
details of the campaign, like the scheduling of fund-raising events, and
over hiring*.* ****** The uncertainty over who has final signoff has
made it more difficult for the campaign to quickly execute decisions and
Maya Harris's dual roles as relative and adviser prompted the
candidate's staff to be more restrained about the advice they offer.

Image

Ms. Harris speaks with her sister, Maya Harris, her campaign chairwoman,
at an Iowa picnic in July.Credit...Hilary Swift for The New York Times

There are also generational fissures*.* One adviser said the fixation
that some younger staffers have with liberals on Twitter distorted their
view of what issues and moments truly mattered, joking that it was not
President Trump's account that should be taken offline, as Ms. Harris
has urged, but rather those of their own trigger-happy communications
team.

In Baltimore, though, the consensus is that the fault lies with Mr.
Rodriguez.

Messages from bookkeepers warning of financial strain went unheeded,
according to his critics, until cutbacks were inevitable.

When those cuts arrived, Ms. Harris and other members of the senior
staff were enraged because they did not know the extent of the layoffs
until after they happened. Some aides were informed about the mass
firing of New Hampshire staff from junior aides and members of the press
rather than Mr. Rodriguez. Ms. Harris called him, infuriated.

Advisers close to Mr. Rodriguez said the cash flow problems were so
intense he had to move swiftly and denied he ever disregarded financial
warnings. They argued that the animus toward him,
\href{https://www.politico.com/news/2019/11/15/kamala-harris-campaign-2020-071105}{first
reported by Politico}, stems from the raw emotions of staffers seeing
their colleagues pushed out.

Some of Ms. Harris's aides said she had better instincts than her brain
trust. One official recalled that during the flight from Oakland to Iowa
on the night she announced her campaign in January, Ms. Harris told
senior members of her campaign team that she wanted to ``go stealth.''
However, instead of pursuing retail politics and introducing herself to
voters in more intimate settings, as Ms. Harris suggested she preferred,
her senior aides determined it was more important to cement herself in
the top tier and play for ``big, television moments,'' as one put it.

``If you go big like that, you'll never get a real understanding of the
American people,'' said Minyon Moore, a former senior adviser to Hillary
Clinton and a longtime admirer of Ms. Harris. ``Because we don't live up
there.''

Image

Supporters of Ms. Harris turned out to see her at a steak fry in Des
Moines in September.Credit...Hilary Swift for The New York Times

\hypertarget{she-lost-me-today}{%
\subsection{`She lost me today'}\label{she-lost-me-today}}

The organizational unsteadiness of Ms. Harris's campaign reflects a
longtime personal trait, according to allies: she is a candidate who
seeks input from a stable of advisers, but her personal political
convictions can be unclear.

In June, her gifts and liabilities were both on display. She scored the
campaign's biggest debate moment in her confrontation with Mr. Biden
over his record on school busing --- but also stepped into a morass of
hazy talk on health care and the current desegregation of schools.

``I'm cool with the T-shirts, but you also have to have a strategy,''
said Bakari Sellers, a former lawmaker in South Carolina and one of Ms.
Harris's top surrogates there, referring to the merchandise Ms. Harris's
campaign had marketed after that first debate.

On criminal justice, one of Ms. Harris's
\href{https://www.nytimes.com/2019/01/21/us/politics/kamala-harris-2020-president.html}{calling
cards}, she did not unveil her own proposals until months after she
began meeting with activists. Ms. Harris said she was being deliberate,
but several aides familiar with the process said she was knocked off
kilter by criticism from progressives and spent months torn between
embracing her prosecutor record and acknowledging some faults.

At times, she avoided the topic, even initially rejecting her current
campaign slogan, ``Justice Is On The Ballot,'' when it was presented to
her earlier in the summer. At one point during the preparations,
tensions flared so high that one senior aide pleaded with the candidate
to provide some direction. ``You know this stuff better than us!'' the
aide said, according to those present.

It was hardly the only time Ms. Harris has appeared uneasy or indecisive
about whether to go on the offensive. In the July debate, Ms. Harris did
not respond sharply to an attack on her prosecutorial record from
Representative
\href{https://www.nytimes.com/interactive/2020/us/elections/tulsi-gabbard.html}{Tulsi
Gabbard} of Hawaii, even after Ms. Harris had been prepped for the
topic.

On a conference call after the debate, several of Ms. Harris's donors
were alarmed and urged the campaign to strike back at Ms. Gabbard more
aggressively, two people on the call said.

Image

Many advisers to Ms. Harris point to the July Democratic debate, and her
weak response to an attack by Representative Tulsi Gabbard, right, as
accelerating Ms. Harris's decline as a candidate.Credit...Brittainy
Newman/The New York Times

Ms. Harris also knew her response had been insufficient, a view quickly
reinforced by her advisers. In interviews, many of them point to that
debate moment as accelerating Ms. Harris's decline and are so
exasperated that they bluntly acknowledge in private that Ms. Harris
struggles to carry a message beyond the initial script.

What she does seem more comfortable with, on the campaign trail and at
the November debate, is making the case against Mr. Trump, which is now
her core campaign message. After months of uncertainty, she's back to
embracing her role as a prosecutor.

``She should lean into it,'' said the radio host Charlamagne tha God,
who has campaigned with Ms. Harris in his native South Carolina. ``She
should say, `I'm a prosecutor and Donald Trump is a criminal and I'm
going to lock his ass up.'''

The question is whether it's too late.

Two women arrived at a recent event Ms. Harris held in Mason City, Iowa,
torn between supporting her or Mr. Buttigieg, who has emerged as a
front-runner in the state.

They were left so dissatisfied, they said, that they now are backing Mr.
Buttigieg.

Laurie Davis, one of the voters, said Ms. Harris's lack of policy
specifics in her remarks was disappointing. Asked when she realized she
wouldn't be voting for Ms. Harris, she paused.

``Right now, I guess,'' she said. ``She lost me today.''

\emph{Shane Goldmacher and Jennifer Medina contributed reporting.}

\hypertarget{our-2020-election-guide}{%
\section{Our 2020 Election Guide}\label{our-2020-election-guide}}

Updated July 31, 2020

\begin{itemize}
\item
  \begin{center}\rule{0.5\linewidth}{\linethickness}\end{center}

  \hypertarget{the-latest}{%
  \subsection{The Latest}\label{the-latest}}

  \begin{itemize}
  \tightlist
  \item
    President Trump's assault on the Postal Service is intersecting with
    his attacks on mail-in voting.
    \href{https://www.nytimes.com/2020/07/31/us/politics/trump-usps-mail-delays.html?action=click\&pgtype=Article\&state=default\&region=BELOW_MAIN_CONTENT\&context=storylines_guide}{Voting
    rights groups say it is a recipe for disaster.}
  \end{itemize}
\item
  \begin{center}\rule{0.5\linewidth}{\linethickness}\end{center}

  \hypertarget{bidens-vp-search}{%
  \subsection{Biden's V.P. Search}\label{bidens-vp-search}}

  \begin{itemize}
  \tightlist
  \item
    \href{https://www.nytimes.com/article/biden-vice-president-2020.html?action=click\&pgtype=Article\&state=default\&region=BELOW_MAIN_CONTENT\&context=storylines_guide}{Here
    are 13 women} who have been under consideration to be Joe Biden's
    running mate, and why each might be chosen --- and might not be.
  \end{itemize}
\item
  \begin{center}\rule{0.5\linewidth}{\linethickness}\end{center}

  \hypertarget{keep-up-with-our-coverage}{%
  \subsection{Keep Up With Our
  Coverage}\label{keep-up-with-our-coverage}}

  \begin{itemize}
  \tightlist
  \item
    Get an
    \href{https://www.nytimes.com/newsletters/politics?action=click\&pgtype=Article\&state=default\&region=BELOW_MAIN_CONTENT\&context=storylines_guide}{email}
    recapping the day's news
  \end{itemize}

  \begin{itemize}
  \tightlist
  \item
    Download our mobile app on
    \href{https://apps.apple.com/us/app/nytimes/id284862083?ls=1\&mat_click_id=5c79ae7455014fd1bd66b5610c05b8f2-20191112-16948\&referrer=mat_click_id\%3D5c79ae7455014fd1bd66b5610c05b8f2-20191112-16948\%26link_click_id\%3D722930677036718082}{iOS}
    and
    \href{http://a.localytics.com/android?id=com.nytimes.android\&referrer=utm_source\%3Dother_nyt_mobile_web\%26utm_medium\%3DWeb\%2520page\%26utm_term\%3DGeneral\%2520Mobile\%2520Page\%26utm_campaign\%3DNYT\%2520Mobile\%2520General\%2520Page}{Android}
    and turn on Breaking News and Politics alerts
  \end{itemize}
\end{itemize}

Advertisement

\protect\hyperlink{after-bottom}{Continue reading the main story}

\hypertarget{site-index}{%
\subsection{Site Index}\label{site-index}}

\hypertarget{site-information-navigation}{%
\subsection{Site Information
Navigation}\label{site-information-navigation}}

\begin{itemize}
\tightlist
\item
  \href{https://help.nytimes.com/hc/en-us/articles/115014792127-Copyright-notice}{©~2020~The
  New York Times Company}
\end{itemize}

\begin{itemize}
\tightlist
\item
  \href{https://www.nytco.com/}{NYTCo}
\item
  \href{https://help.nytimes.com/hc/en-us/articles/115015385887-Contact-Us}{Contact
  Us}
\item
  \href{https://www.nytco.com/careers/}{Work with us}
\item
  \href{https://nytmediakit.com/}{Advertise}
\item
  \href{http://www.tbrandstudio.com/}{T Brand Studio}
\item
  \href{https://www.nytimes.com/privacy/cookie-policy\#how-do-i-manage-trackers}{Your
  Ad Choices}
\item
  \href{https://www.nytimes.com/privacy}{Privacy}
\item
  \href{https://help.nytimes.com/hc/en-us/articles/115014893428-Terms-of-service}{Terms
  of Service}
\item
  \href{https://help.nytimes.com/hc/en-us/articles/115014893968-Terms-of-sale}{Terms
  of Sale}
\item
  \href{https://spiderbites.nytimes.com}{Site Map}
\item
  \href{https://help.nytimes.com/hc/en-us}{Help}
\item
  \href{https://www.nytimes.com/subscription?campaignId=37WXW}{Subscriptions}
\end{itemize}
