Sections

SEARCH

\protect\hyperlink{site-content}{Skip to
content}\protect\hyperlink{site-index}{Skip to site index}

\href{https://www.nytimes.com/section/nyregion}{New York}

\href{https://myaccount.nytimes.com/auth/login?response_type=cookie\&client_id=vi}{}

\href{https://www.nytimes.com/section/todayspaper}{Today's Paper}

\href{/section/nyregion}{New York}\textbar{}Amazon Grows in New York,
Reviving Debate Over Abandoned Queens Project

\url{https://nyti.ms/2Yrw8AE}

\begin{itemize}
\item
\item
\item
\item
\item
\end{itemize}

Advertisement

\protect\hyperlink{after-top}{Continue reading the main story}

Supported by

\protect\hyperlink{after-sponsor}{Continue reading the main story}

\hypertarget{amazon-grows-in-new-york-reviving-debate-over-abandoned-queens-project}{%
\section{Amazon Grows in New York, Reviving Debate Over Abandoned Queens
Project}\label{amazon-grows-in-new-york-reviving-debate-over-abandoned-queens-project}}

Representative Alexandria Ocasio-Cortez and other critics of an earlier
proposal said a move to add offices in Manhattan proved them right.

\includegraphics{https://static01.nyt.com/images/2019/12/06/nyregion/06amazon/06amazon-articleLarge.jpg?quality=75\&auto=webp\&disable=upscale}

By Ed Shanahan

\begin{itemize}
\item
  Published Dec. 6, 2019Updated Dec. 9, 2019
\item
  \begin{itemize}
  \item
  \item
  \item
  \item
  \item
  \end{itemize}
\end{itemize}

Amazon said on Friday that it would lease office space in Midtown
Manhattan for more than 1,500 employees, increasing its presence in New
York City less than a year after
\href{https://www.nytimes.com/2019/02/14/nyregion/amazon-hq2-queens.html}{abandoning
plans for a second headquarters in Queens} in the face of stiff local
opposition.

The retail giant said it had a signed a lease for 350,000 square feet in
a 10th Avenue building near the Hudson Yards development. The offices
will be occupied by workers in Amazon's consumer and advertising units,
the company said.

The Manhattan lease does not qualify for the kind of tax credits and
other government sweeteners the company had won in exchange for building
a huge campus in Long Island City that was to employ more than 25,000
people.

The incentives for the Queens project
\href{https://www.nytimes.com/2019/02/14/upshot/amazon-foxconn-subsidies-critics.html}{totaled
\$3 billion}, touching off a contentious debate about the value of using
outsize public subsidies to woo wealthy companies --- and about whether
such bait was necessary in a talent-rich city like New York.

Word that the company was adding office space in Manhattan to its
existing operations elsewhere in Midtown, which was first reported by
The Wall Street Journal, led leading critics of the Long Island City
venture to take what amounted to a second victory lap.

``Won't you look at that,'' Representative Alexandria Ocasio-Cortez, who
represents an area in Queens near where the campus was to be built,
\href{https://twitter.com/AOC/status/1203083485252112384}{wrote on
Twitter}. ``Amazon is coming to NYC anyway - *without* requiring the
public to finance shady deals, helipad handouts for Jeff Bezos, \&
corporate giveaways.''

\href{https://www.nytimes.com/2019/02/04/nyregion/amazon-hq2-board-veto.html}{State
Senator Michael Gianaris}, a Queens Democrat who initially supported the
project but changed his mind after learning the details of the incentive
package, said in a statement, ``Amazon is coming to New York, just as
they always planned.''

``Fortunately,'' he added, ``we dodged a \$3 billion bullet by not
agreeing to their subsidy shakedown earlier this year.''

Others were quick to argue that the Manhattan office space and the 1,500
employees it will house pale next to the vast campus proposed for Queens
and the 25,000 people that Amazon pledged to employ there.

``This is a tiny fraction of the jobs, with no help for public housing
residents or locals, in a place that was going to be developed and have
jobs anyway,'' Eric Phillips, a former press secretary for Mayor Bill de
Blasio,
\href{https://twitter.com/EricFPhillips/status/1203107707428974594}{wrote
on Twitter}.

In an interview, Mr. Gianaris discounted that argument. He predicted
that Amazon would continue to expand in New York steadily to the point
that the size of its local force would ultimately equal what had been
promised in Queens.

``This is where the talent is,'' Mr. Gianaris said, noting that both
Google and Facebook had announced substantial expansions in Manhattan in
the past year. ``They can't sacrifice the talent to the competition.''

Amazon's growth in New York has been healthy in recent years.

In 2017 --- as it searched for where to put a second headquarters to
augment its Seattle base --- Amazon said separately that it planned to
bring more than 2,000 jobs to New York City over three years and to
create about 6,000 jobs in New York State by 2019.

When it dropped the Queens project in February, the company said it had
more than 5,000 employees in Brooklyn, Manhattan and Staten Island,
\href{https://www.nytimes.com/2019/03/20/business/economy/amazon-warehouse-labor.html}{where
it operates a warehouse}, and that it planned ``to continue growing
these teams.''

On Friday, it put the size of its New York City work force at over
8,000, with 3,500 of those employed at what it calls its New York City
tech hub. It was unclear whether the 1,500 people expected to work in
the 10th Avenue building starting in late 2021 would fill newly created
positions.

``We plan to continue to hire and grow organically across our 18 tech
hubs, including New York City,'' an Amazon spokeswoman said.

The company's decision to leave Queens behind came several months after
\href{https://www.nytimes.com/2018/11/05/nyregion/amazon-hq2-long-island-city.html}{it
ended its much-publicized second-headquarters search} by splitting what
had been envisioned as a single huge, 50,000-employee campus between
Long Island City and Arlington, Va.

\href{https://www.nytimes.com/2019/02/14/nyregion/amazon-long-island-city.html}{The
proposal had supporters}, including Mr. de Blasio and
\href{https://www.nytimes.com/2019/02/28/nyregion/amazon-hq2-nyc.html}{Gov.
Andrew M. Cuomo}, but
\href{https://www.nytimes.com/2019/02/12/nyregion/amazon-nyc-hq2.html}{the
opposition} rallied by Ms. Ocasio-Cortez, Mr. Gianaris, progressive
activists, union leaders and others made Amazon, which has encountered
resistance in its hometown, wary of pressing the issue.

Advertisement

\protect\hyperlink{after-bottom}{Continue reading the main story}

\hypertarget{site-index}{%
\subsection{Site Index}\label{site-index}}

\hypertarget{site-information-navigation}{%
\subsection{Site Information
Navigation}\label{site-information-navigation}}

\begin{itemize}
\tightlist
\item
  \href{https://help.nytimes.com/hc/en-us/articles/115014792127-Copyright-notice}{©~2020~The
  New York Times Company}
\end{itemize}

\begin{itemize}
\tightlist
\item
  \href{https://www.nytco.com/}{NYTCo}
\item
  \href{https://help.nytimes.com/hc/en-us/articles/115015385887-Contact-Us}{Contact
  Us}
\item
  \href{https://www.nytco.com/careers/}{Work with us}
\item
  \href{https://nytmediakit.com/}{Advertise}
\item
  \href{http://www.tbrandstudio.com/}{T Brand Studio}
\item
  \href{https://www.nytimes.com/privacy/cookie-policy\#how-do-i-manage-trackers}{Your
  Ad Choices}
\item
  \href{https://www.nytimes.com/privacy}{Privacy}
\item
  \href{https://help.nytimes.com/hc/en-us/articles/115014893428-Terms-of-service}{Terms
  of Service}
\item
  \href{https://help.nytimes.com/hc/en-us/articles/115014893968-Terms-of-sale}{Terms
  of Sale}
\item
  \href{https://spiderbites.nytimes.com}{Site Map}
\item
  \href{https://help.nytimes.com/hc/en-us}{Help}
\item
  \href{https://www.nytimes.com/subscription?campaignId=37WXW}{Subscriptions}
\end{itemize}
