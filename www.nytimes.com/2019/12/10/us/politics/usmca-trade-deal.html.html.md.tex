Sections

SEARCH

\protect\hyperlink{site-content}{Skip to
content}\protect\hyperlink{site-index}{Skip to site index}

\href{https://www.nytimes.com/section/politics}{Politics}

\href{https://myaccount.nytimes.com/auth/login?response_type=cookie\&client_id=vi}{}

\href{https://www.nytimes.com/section/todayspaper}{Today's Paper}

\href{/section/politics}{Politics}\textbar{}Trump Aides and Democrats
Agree on Trade Pact With Mexico and Canada

\url{https://nyti.ms/38nF12x}

\begin{itemize}
\item
\item
\item
\item
\item
\item
\end{itemize}

Advertisement

\protect\hyperlink{after-top}{Continue reading the main story}

Supported by

\protect\hyperlink{after-sponsor}{Continue reading the main story}

\hypertarget{trump-aides-and-democrats-agree-on-trade-pact-with-mexico-and-canada}{%
\section{Trump Aides and Democrats Agree on Trade Pact With Mexico and
Canada}\label{trump-aides-and-democrats-agree-on-trade-pact-with-mexico-and-canada}}

The new provisions, solidified after months of negotiations between
House Democrats and the Trump administration, would strengthen the trade
deal's protections for workers

\includegraphics{https://static01.nyt.com/images/2019/12/03/us/politics/00dc-usmcahfo/merlin_144886743_050b4fa0-880e-428f-825e-2f7862b899cd-articleLarge.jpg?quality=75\&auto=webp\&disable=upscale}

\href{https://www.nytimes.com/by/emily-cochrane}{\includegraphics{https://static01.nyt.com/images/2018/11/28/multimedia/author-emily-cochrane/author-emily-cochrane-thumbLarge-v3.png}}\href{https://www.nytimes.com/by/ana-swanson}{\includegraphics{https://static01.nyt.com/images/2018/12/10/multimedia/author-ana-swanson/author-ana-swanson-thumbLarge.png}}

By \href{https://www.nytimes.com/by/emily-cochrane}{Emily Cochrane} and
\href{https://www.nytimes.com/by/ana-swanson}{Ana Swanson}

\begin{itemize}
\item
  Published Dec. 10, 2019Updated Jan. 29, 2020
\item
  \begin{itemize}
  \item
  \item
  \item
  \item
  \item
  \item
  \end{itemize}
\end{itemize}

WASHINGTON --- The White House and House Democrats reached an agreement
to strengthen labor, environmental, pharmaceutical and enforcement
provisions in President Trump's North American trade pact, a significant
development that made it all but certain that the signature trade deal
would become law.

The agreement on a revised
\href{https://www.nytimes.com/2020/01/29/business/economy/usmca-deal.html}{United
States-Mexico-Canada Agreement} was announced on Tuesday by Speaker
Nancy Pelosi after months of negotiations, handing Mr. Trump one of his
biggest legislative victories less than an hour after she unveiled
articles of impeachment.

Ms. Pelosi went from a news conference on impeachment to another on the
trade deal, where she and top Democrats, including Representative
Richard E. Neal of Massachusetts, pointed to concessions they had
secured in closed-door negotiations with the administration.

``We're declaring victory for the American worker,'' Ms. Pelosi said.
``It is infinitely better than what was initially proposed by the
administration.''

The timing of the handshake agreement offers Mr. Trump a crucial victory
to promote on the campaign trail during his re-election bid and House
Democrats tangible proof that they are able to legislate while preparing
to vote on charges of abuse of power and obstruction of Congress against
the president.

Mr. Trump, who spent weeks blaming Ms. Pelosi for standing in the way of
a trade deal that he said would help workers, played up the progress and
suggested the House speaker did so to ``smother the impeachment crap.''

``We've been waiting a long time for Nancy Pelosi to announce
U.S.M.C.A.,''
\href{https://www.nytimes.com/2019/12/10/us/politics/trump-impeachment-rally-pennsylvania.html}{he
said at a rally} on Tuesday night in Hershey, Pa. ``And she did it on
the same day that they announced that they are going to impeach the 45th
president of the United States, and your favorite president.''

In a statement, Mr. Trump's top trade adviser, Robert E. Lighthizer,
called the announcement a victory for Mr. Trump.

``After working with Republicans, Democrats, and many other stakeholders
for the past two years, we have created a deal that will benefit
American workers, farmers and ranchers for years to come,'' he said.

Ms. Pelosi repeatedly rebuffed Republican suggestions that Democrats had
timed the announcement to try to minimize any negative fallout from the
impeachment proceedings.

``Not any one of us is important enough to hold up a trade agreement
that is important for American workers,'' she said.

The administration
\href{https://www.nytimes.com/2018/11/30/world/americas/trump-trudeau-canada-mexico.html}{agreed
with Canada and Mexico on revisions} to the North American Free Trade
Agreement one year ago, but the deal requires the approval of Congress,
including the Democratic-controlled House. Ms. Pelosi and her colleagues
have used that vote as leverage to secure long-sought policy changes to
a long-maligned trade deal.

``Make no mistake,'' Representative Earl Blumenauer, Democrat of Oregon,
said Tuesday. ``This is a Democrat's agreement that we fought for, and
it's going to be the template going forward for writing new trade
agreements.''

Ms. Pelosi was more candid in a private meeting with her caucus on
Tuesday morning. ``These have been the fights,'' she said, referring to
the changes they secured. ``And we stayed on this and we ate their
lunch.''

Mr. Neal, the chairman of the Ways and Means Committee, said he remained
hopeful that the House could vote on the agreement before the end of the
year. Senator Mitch McConnell of Kentucky, the majority leader, said
that the Senate would not bring the deal for a vote before Dec. 20, when
lawmakers are scheduled to leave for a holiday break.

``That'll have to come up, in all likelihood, after a trial is finished
in the Senate,'' he said, referring to the impeachment proceedings.

Among the biggest victories was an agreement to remove intellectual
property protections for the pharmaceutical industry, which Democrats
warned could undermine efforts to make health care more affordable.
Democrats also persuaded the White House to strengthen the deal's
enforcement provisions, and obtained commitments to ensure Mexico is
adhering to labor reforms.

Those changes were critical to winning the support of labor unions,
including the influential AFL-CIO, which endorsed the revised pact just
moments before Ms. Pelosi's announcement.

In fact, the deal addressed so many of the Democrats' concerns that some
Republicans appeared skeptical of the final agreement and suggested that
Mr. Lighthizer had given away too much.

Senator John Cornyn, Republican of Texas, voiced concern that Mr.
Lighthizer had potentially spent more time talking with House Democrats
than Republicans on the final product. And Senator Patrick J. Toomey,
Republican of Pennsylvania and one of the most ardent critics of the
deal, railed against both the original deal and the new changes,
including the removal of the pharmaceutical provision.

``It's clearly moved way to the left,'' Mr. Toomey told reporters. ``It
seemed to be just a one-way direction in the direction of Democrats.''

The changes must now be woven into implementing legislation that the
House and
\href{https://www.nytimes.com/2020/01/16/us/politics/senate-usmca-approval-trump.html}{Senate
will both vote on}. The pact will also need to secure the president's
signature and the final approval of the Mexican and Canadian
legislatures.

In Mexico City, President Andrés Manuel López Obrador attended a signing
agreement at the National Palace. The event was attended by Mr.
Lighthizer and Jared Kushner, the president's senior adviser and
son-in-law, as well as Chrystia Freeland, who negotiated the pact on
behalf of Canada.

Mr. Lighthizer called the agreement ``the first truly bipartisan
agreement,'' saying it was ``nothing short of a miracle that we have all
come together.''

Mr. Lighthizer on Tuesday briefed groups of House and Senate Republicans
by phone on the changes. While some expressed concern, most Republicans
appeared to maintain their support for the new trade pact, even with the
new changes negotiated by Democrats.

Senator Rob Portman, Republican of Ohio, declared ``relief'' in an
interview, and noted that such a compromise in a divided government ``is
a rare feat around here, and we should celebrate it.''

And as Mr. Neal left the news conference, Representative Steve Scalise
of Louisiana, the Republican whip, shook his hand. A spokeswoman said
Mr. Scalise had promised Mr. Lighthizer strong Republican support for
the deal.

``There's a Republican leader saying it was good,'' Mr. Neal said as he
entered an elevator. ``That wasn't staged.''

The agreement came as a huge relief to industries that have grown up
around NAFTA and rely on tariff-free trade across Canada, Mexico and the
United States. The lack of movement in Congress, combined with Mr.
Trump's threats to walk away from the original NAFTA pact, had created
crippling uncertainty among businesses.

``This is finally good news on the trade front after a long, hard
year,'' said Rufus H. Yerxa, the president of the National Foreign Trade
Council, which represents major exporters. ``We believe this agreement
will further strengthen the North American region, bringing about the
commercial stability and certainty that our companies need to remain
competitive in the global economy.''

\includegraphics{https://static01.nyt.com/images/2019/11/21/us/politics/00dc-usmcaHFO2/merlin_164672949_60a0cfd6-c61e-4d4a-9b28-581aec016bc7-articleLarge.jpg?quality=75\&auto=webp\&disable=upscale}

The administration and Republicans in both chambers have hammered Ms.
Pelosi and her caucus to take action. Even within Ms. Pelosi's majority,
several moderate members and a number of the freshmen who flipped
Republican-held seats in 2018 had begun pressuring leadership for a vote
on the pact before the end of the year.

The deal announced Tuesday **** offered Ms. Pelosi and her core allies
justification for the delay by establishing what she said would be a
legacy agreement that set the standard for future trade deals.

In addition to updating rules for digital commerce, Mr. Trump's
U.S.M.C.A. raised the threshold for the proportion of a car's value that
must be made in North America in order to qualify for the pact's zero
tariffs. It also rolls back a special system of arbitration for
corporations long opposed by Democrats.

One of the most significant revisions will roll back protections for new
pharmaceutical products, in particular an advanced class of drugs called
biologics, which were initially given 10 years of protection from
cheaper alternatives. It also removed language that would ensure patent
protections when drug companies find new uses for their existing
products, a process known as ``evergreening.''

Those changes are a big departure from past trade agreements, which
sought to lock in stronger protections for intellectual property, long
seen as a competitive advantage for the American economy.

Just three years ago, Republicans blocked the progress of the
Trans-Pacific Partnership, a 12-country trade deal negotiated by
President Barack Obama, over complaints that similar protections for
drug companies were not strong enough. The pact never gained enough
support for a congressional vote under Mr. Obama, and Mr. Trump pulled
the United States out of the deal during his first week in office.

Mr. Blumenauer said the pharmaceutical revisions would ``change the
landscape'' on trade agreements. ``If we go back and review the other
trade agreements we've had, they are replete with pharmaceutical
protections,'' he said. ``This is a very significant shift.''

The drug industry was not pleased.

``The announcement made today puts politics over patients,'' Stephen J.
Ubl, the president and chief executive of the Pharmaceutical Research
and Manufacturers of America, said in a statement. ``The only winners
today are foreign governments who want to steal American intellectual
property and free ride on America's global leadership in
biopharmaceutical research and development.''

The revisions also beefed up labor protections, especially in Mexico.
While Mexican negotiators succeeded in rebuffing Democrats' demand for
American inspections of Mexican factories, they agreed to additional
funding and oversight to ensure that Mexico proceeds with strengthening
its labor laws and unions. The United States will also be allowed to
block goods from specific Mexican factories if companies are found in
violation of labor rules.

Democrats also said they had succeeded in bolstering enforcement of the
trade pact by stripping out a provision --- added by Mr. Lighthizer ---
which had curbed the ability of countries to bring disputes against one
another.

In a loss for Ms. Pelosi, the pact will still contain certain legal
protections that may shield online platforms like Facebook and Twitter
from some lawsuits over content posted by their users.

Ms. Pelosi acknowledged the inclusion of those provisions was a
``disappointment,'' adding, ``I mean, I lost.''

Catie Edmondson and David McCabe contributed reporting from Washington,
and Elisabeth Malkin from Mexico City.

Advertisement

\protect\hyperlink{after-bottom}{Continue reading the main story}

\hypertarget{site-index}{%
\subsection{Site Index}\label{site-index}}

\hypertarget{site-information-navigation}{%
\subsection{Site Information
Navigation}\label{site-information-navigation}}

\begin{itemize}
\tightlist
\item
  \href{https://help.nytimes.com/hc/en-us/articles/115014792127-Copyright-notice}{©~2020~The
  New York Times Company}
\end{itemize}

\begin{itemize}
\tightlist
\item
  \href{https://www.nytco.com/}{NYTCo}
\item
  \href{https://help.nytimes.com/hc/en-us/articles/115015385887-Contact-Us}{Contact
  Us}
\item
  \href{https://www.nytco.com/careers/}{Work with us}
\item
  \href{https://nytmediakit.com/}{Advertise}
\item
  \href{http://www.tbrandstudio.com/}{T Brand Studio}
\item
  \href{https://www.nytimes.com/privacy/cookie-policy\#how-do-i-manage-trackers}{Your
  Ad Choices}
\item
  \href{https://www.nytimes.com/privacy}{Privacy}
\item
  \href{https://help.nytimes.com/hc/en-us/articles/115014893428-Terms-of-service}{Terms
  of Service}
\item
  \href{https://help.nytimes.com/hc/en-us/articles/115014893968-Terms-of-sale}{Terms
  of Sale}
\item
  \href{https://spiderbites.nytimes.com}{Site Map}
\item
  \href{https://help.nytimes.com/hc/en-us}{Help}
\item
  \href{https://www.nytimes.com/subscription?campaignId=37WXW}{Subscriptions}
\end{itemize}
