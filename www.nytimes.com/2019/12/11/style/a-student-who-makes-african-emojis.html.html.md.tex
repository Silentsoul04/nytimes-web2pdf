Sections

SEARCH

\protect\hyperlink{site-content}{Skip to
content}\protect\hyperlink{site-index}{Skip to site index}

\href{https://www.nytimes.com/section/style}{Style}

\href{https://myaccount.nytimes.com/auth/login?response_type=cookie\&client_id=vi}{}

\href{https://www.nytimes.com/section/todayspaper}{Today's Paper}

\href{/section/style}{Style}\textbar{}A Student Who Makes African Emojis

\href{https://nyti.ms/35fcdqS}{https://nyti.ms/35fcdqS}

\begin{itemize}
\item
\item
\item
\item
\item
\end{itemize}

Advertisement

\protect\hyperlink{after-top}{Continue reading the main story}

Supported by

\protect\hyperlink{after-sponsor}{Continue reading the main story}

\hypertarget{a-student-who-makes-african-emojis}{%
\section{A Student Who Makes African
Emojis}\label{a-student-who-makes-african-emojis}}

O'Plérou Grebet, an artist from the Ivory Coast, learned to make emojis
from (where else?) YouTube.

\includegraphics{https://static01.nyt.com/images/2019/12/12/fashion/11upnext1/merlin_165312447_756ca93e-c726-404c-b327-fd3a0ba958d1-articleLarge.jpg?quality=75\&auto=webp\&disable=upscale}

\href{https://www.nytimes.com/by/alex-hawgood}{\includegraphics{https://static01.nyt.com/images/2019/02/20/multimedia/author-alex-hawgood/author-alex-hawgood-thumbLarge.png}}

By \href{https://www.nytimes.com/by/alex-hawgood}{Alex Hawgood}

\begin{itemize}
\item
  Dec. 11, 2019
\item
  \begin{itemize}
  \item
  \item
  \item
  \item
  \item
  \end{itemize}
\end{itemize}

\textbf{Name:} O'Plérou Grebet

\textbf{Age:} 22

\textbf{Hometown:} Abidjan, Ivory Coast

\textbf{Now lives:} With his family in a single-story house in Abidjan.

\begin{quote}
\end{quote}

\textbf{Claim to fame:} Using text messages as his medium, Mr. Grebet is
an Ivorian digital artist who has created more than 365
\href{https://www.instagram.com/creativorian/}{free emojis} that portray
contemporary African life, including a
\href{https://www.instagram.com/p/BfbuSBlgqlO/}{zebra-striped plastic
teakettle sold in Senegalese markets},
\href{https://www.instagram.com/p/BgE6VRoBi9O/}{hair braids} and a
\href{https://www.pinterest.com/pin/425449496045347582/}{shekere}, a
West African percussion instrument made with a dried gourd. The emojis,
which he calls
\href{https://apps.apple.com/fr/app/zouzoukwa/id1453717366}{Zouzoukwa}
(which translates roughly as ``image'' in the regional Bété language),
have been downloaded on more than 120,000 Android phones and iPhones
since their release last year.

\textbf{Big break:} As an arts student at the
\href{http://istc-gouv-ci.net}{Institute of Sciences and Communication
Techniques} in Abidjan, he noticed that he and his peers were consuming
mostly Western images. A YouTube tutorial on
\href{https://www.youtube.com/watch?v=D5ovtITGrSw}{how to create an
emoji} on Photoshop inspired him to translate those techniques for
everyday African culture. His first emoji depicted
\href{https://www.instagram.com/creativorian/p/Bdaxpv7gc8M/}{foutou}, a
staple \href{https://www.youtube.com/watch?v=n-7U7ZK1h3M}{Ivorian dish}
often made of mashed plantains and ground cassava flour. It clearly
struck a chord; the app was
\href{https://www.instagram.com/p/BsMQhzmHjIu/?igshid=182u8hb857e9}{downloaded
10,000 times} in the first three days.

Image

A shekere, a West African percussion instrument made with a dried
gourd.Credit...O'Plérou Grebet

Image

Foulard gele, a head scarf.Credit...O'Plérou Grebet

\textbf{Latest project:} The African Talents Awards, which recognize
young Africans in creative fields, recently named Zouzoukwa the
\href{http://africantalentsawards.com/?page_id=2560}{best app of} 2019.
Mr. Grebet also uploaded new emojis based on
\href{https://www.instagram.com/p/BwgshCsHnjW/}{Paquinou}, an Easter
festival celebrated in the Ivory Coast. ``People reach out to tell me
what they think I should design, or what they think is missing,'' he
said. ``Those interactions are some of the aspects I like best about
what I do.''

\textbf{Next thing:} After mastering two-dimensional avatars, he wants
to start ``sharing African culture'' using augmented reality. He also
hopes to start an e-commerce site, he said, ``so people could buy
clothes, phone cases and other objects made from my work.''

Image

Zouglou, an Ivorian dance style of musicCredit...O'Plérou Grebet

\textbf{Truth in Pixels:} His oeuvre may be cartoonish, but he aims for
verisimilitude. ``I have to travel and discover other African
countries,'' he said. ``I have to immerse myself in their cultures in
order to create emojis that truly represent them, instead of looking for
pictures on the internet about meals I never tasted or places I never
went to.''

Advertisement

\protect\hyperlink{after-bottom}{Continue reading the main story}

\hypertarget{site-index}{%
\subsection{Site Index}\label{site-index}}

\hypertarget{site-information-navigation}{%
\subsection{Site Information
Navigation}\label{site-information-navigation}}

\begin{itemize}
\tightlist
\item
  \href{https://help.nytimes.com/hc/en-us/articles/115014792127-Copyright-notice}{©~2020~The
  New York Times Company}
\end{itemize}

\begin{itemize}
\tightlist
\item
  \href{https://www.nytco.com/}{NYTCo}
\item
  \href{https://help.nytimes.com/hc/en-us/articles/115015385887-Contact-Us}{Contact
  Us}
\item
  \href{https://www.nytco.com/careers/}{Work with us}
\item
  \href{https://nytmediakit.com/}{Advertise}
\item
  \href{http://www.tbrandstudio.com/}{T Brand Studio}
\item
  \href{https://www.nytimes.com/privacy/cookie-policy\#how-do-i-manage-trackers}{Your
  Ad Choices}
\item
  \href{https://www.nytimes.com/privacy}{Privacy}
\item
  \href{https://help.nytimes.com/hc/en-us/articles/115014893428-Terms-of-service}{Terms
  of Service}
\item
  \href{https://help.nytimes.com/hc/en-us/articles/115014893968-Terms-of-sale}{Terms
  of Sale}
\item
  \href{https://spiderbites.nytimes.com}{Site Map}
\item
  \href{https://help.nytimes.com/hc/en-us}{Help}
\item
  \href{https://www.nytimes.com/subscription?campaignId=37WXW}{Subscriptions}
\end{itemize}
