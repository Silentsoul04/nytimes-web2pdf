Sections

SEARCH

\protect\hyperlink{site-content}{Skip to
content}\protect\hyperlink{site-index}{Skip to site index}

\href{/section/world/americas}{Americas}\textbar{}He Was One of Mexico's
Deadliest Assassins. Then He Turned on His Cartel.

\url{https://nyti.ms/38zzPJ0}

\begin{itemize}
\item
\item
\item
\item
\item
\item
\end{itemize}

\includegraphics{https://static01.nyt.com/images/2019/12/15/world/jpSICARIO1/xxSicario-articleLarge.jpg?quality=75\&auto=webp\&disable=upscale}

\hypertarget{he-was-one-of-mexicos-deadliest-assassins-then-he-turned-on-his-cartel}{%
\section{He Was One of Mexico's Deadliest Assassins. Then He Turned on
His
Cartel.}\label{he-was-one-of-mexicos-deadliest-assassins-then-he-turned-on-his-cartel}}

``They took away everything left in me that was human and made me a
monster,'' said the hit man.

The former assassin in a jail that serves as part of a makeshift witness
protection program.Credit...Alexandra Garcia/The New York Times

Supported by

\protect\hyperlink{after-sponsor}{Continue reading the main story}

By \href{https://www.nytimes.com/by/azam-ahmed}{Azam Ahmed} and Paulina
Villegas

\begin{itemize}
\item
  Published Dec. 14, 2019Updated Feb. 28, 2020
\item
  \begin{itemize}
  \item
  \item
  \item
  \item
  \item
  \item
  \end{itemize}
\end{itemize}

JOJUTLA, Mexico --- The recruits filed into a clearing, where a group of
trainers with the stern bearing of drill sergeants stood in a tight row,
hiding something.

``How many of you have killed someone before?'' one of the instructors
asked. A few hands shot up.

The trainers separated, revealing a naked corpse face up in the grass.
One thrust a machete into the nearest man's hand.

``Dismember that body,'' he ordered.

The recruit froze. The instructor waited, then walked up behind the
terrified recruit and fired a bullet into his head, killing him. Next,
he passed the blade to a lanky teenager while the others watched,
dumbfounded.

The teenager didn't hesitate. Offered the chance to prove that he could
be an assassin --- a sicario --- he seized it, he said. A chance at
money, power and what he craved most, respect. To be feared in a place
where fear was currency.

``I wanted to be a psychopath, to kill without mercy and be the most
feared sicario in the world,'' he said, describing the scene.

Like the other recruits, he had been sent by a drug cartel known as
Guerreros Unidos to a training camp in the mountains. He envisioned
field exercises, morning runs, target practice. Now, standing over the
body, he was just trying to suppress an urge to vomit.

He closed his eyes and struck blindly. To survive, he needed to stay the
course. The training would do the rest, purging him of fear and empathy.

``They took away everything left in me that was human and made me a
monster,'' he said.

Within a few years, he became one of the deadliest assassins in the
Mexican state of Morelos, an instrument of the cartels tearing the
nation apart. By 2017, at only 22 years old, he had taken part in more
than 100 murders, he said. The authorities have confirmed nearly two
dozen of them in Morelos alone.

When the police caught him that year, he could have faced more than 200
years in prison. But instead of prosecuting him, the authorities saw an
opportunity, a chance to pick apart the cartel from the inside. They
made him the centerpiece of an off-the-books police operation that
dismantled the cartel in southern Morelos, resulting in the arrest and
conviction of dozens of its operatives.

For investigators, he was a gold mine, a complete reference book on the
state's murder industry. For the sicario, the government was a lifeline.

\begin{center}\rule{0.5\linewidth}{\linethickness}\end{center}

\textbf{\href{https://www.nytimes.com/2020/02/28/the-weekly/mexico-cartels-police.html}{Watch
`The Weekly,' The Times's TV Show on FX and Hulu}}\\
An epidemic of violence in Mexico and endemic corruption pushed a police
chief to try something new --- an off-the-books witness protection
program for assassins willing to turn on their cartels.

\includegraphics{https://static01.nyt.com/images/2020/02/28/autossell/28theweekly-sicario-promopic/NYTW_SICARIO_JPEG-022720-videoSixteenByNine3000.jpg}

Of course, Mexico's legal system wasn't set up for this kind of
arrangement.

The nation has only one official witness protection program, at the
federal level, and few in law enforcement actually trust it. Leaks,
corruption and incompetence have left it in shambles.

\includegraphics{https://static01.nyt.com/images/2019/12/15/world/jpSICARIO3/15Sicario-07-articleLarge.jpg?quality=75\&auto=webp\&disable=upscale}

The police chief in Morelos at the time, Alberto Capella, wanted a
witness protection program that worked, one he could use to smash
organized crime in his state. So he simply created a clandestine one of
his own --- an improvised strategy that former justice officials
describe as a legal stretch.

But if working around the edges of the law was the only way to tackle
the scourge of organized crime, Mr. Capella figured, it seemed a small
price to pay for justice.

``We had to try something,'' said Mr. Capella, who had survived an
all-out gun battle with assassins years earlier, hardening his resolve.
``We couldn't just sit there and do nothing.''

The sicario's journey from hit man to state witness --- drawn from
public records, at least a dozen visits to the program and 17 months of
interviews with him, his family, officials and other assassins ---
offers a rare glimpse into the world of Mexico's ultraviolent killers
and the lengths to which the authorities will go to stop them.

More killings take place in Mexico today than at any time in
\href{https://www.nytimes.com/2017/08/04/world/americas/mexicos-drug-killings.html}{the
last two decades}, when the nation started collecting homicide
statistics. Cartels fight one another for control of local drug sales
and smuggling routes to the United States, while Mexico's armed forces
battle them all.

100 miles

Mexico City

MORELOS

Jojutla

MEXICO

Pacific

Ocean

Acapulco

U. S.

Gulf of

Mexico

MEXICO

Pacific Ocean

Detail

area

400 miles

By The New York Times

The violence is the worst it has been since
\href{https://www.nytimes.com/2019/12/11/world/americas/mexico-garcia-luna-indictment.html}{the
American-backed drug war began} 13 years ago, and assassins like the one
Mr. Capella built his program around embody the crisis, responsible for
a disproportionate share of murders nationwide.

Killings have become so common, so expected, that the country has grown
increasingly numb to them. Each passing year brings record levels of
violence --- with more harrowing expressions of it --- and the nation's
institutions are so ill-equipped to stem the tide that Mr. Capella felt
he had little choice but to invent a workaround to the country's broken
rule of law.

The deal was simple: The sicario testified against his former comrades
and bosses, detailing the inner workings of a notoriously ruthless
cartel. In return, he could walk free, without facing any charges.

No paperwork. No signatures. No legislation authorizing a witness
protection program in the state. Just a gentleman's agreement, those
involved called it.

``There was nothing to think about,'' the sicario recalled. ``I didn't
want to spend my whole life in prison.''

Through early 2019, the sicario proved so valuable that the police
erected an even bigger wildcat program around him, recruiting more than
a dozen cartel henchmen and housing them in a small, worn-down building
attached to the local prison.

Together, their testimony led to 100 convictions and helped cut
homicides, kidnappings and extortion in the state, at least for a time,
officials said. Even as violence soared across Mexico, it was down in
southern Morelos.

Countrywide, nearly 100 people were being killed every day, often in
horrible ways that stretched the bounds of human imagination. Fewer than
5 percent of those cases were ever solved.

With such dismal conviction rates, Mr. Capella felt, Mexico was
practically issuing licenses to kill. His program, explicitly authorized
by law or not, was a chance to do what hundreds of other officers could
only dream of: pinpoint and lock up the assassins driving the country's
homicide crisis.

The unchecked power of organized crime was on full display this October,
when hundreds of gunmen for the Sinaloa Cartel
\href{https://www.nytimes.com/2019/10/20/world/americas/culiacan-mexico-chapo-son.html}{laid
siege to the city of Culiacán} in broad daylight, forcing the government
to surrender a notable cartel figure --- the son of Joaquín Guzmán
Loera, the drug lord known as ``El Chapo'' --- and
\href{https://www.nytimes.com/2019/10/18/world/americas/mexico-cartel-chapo-son-guzman.html}{set
him loose}, right back into the underworld.

\begin{center}\rule{0.5\linewidth}{\linethickness}\end{center}

\hypertarget{watch-the-full-episode-of-the-weekly-the-timess-new-tv-show}{%
\subsubsection{Watch the full episode of ``The Weekly,'' The Times's new
TV
show}\label{watch-the-full-episode-of-the-weekly-the-timess-new-tv-show}}

One of Mexico's most notorious drug cartels turned a city into a war
zone for a day to rescue El Chapo's son.
\href{https://www.nytimes.com/2019/11/15/the-weekly/el-chapo-guzman-son.html}{Watch
the ```The Siege of Culiacán''} to see how gunmen took on the army ---
and won.

Video

transcript

Back

bars

0:00/1:08

-0:00

transcript

\begin{itemize}
\tightlist
\item
  {[}SHOUTING{]} ``On Thursday, October 17th, Mexican forces arrested
  the son of former drug lord, El Chapo.'' ``An image appeared online,
  the expressionless face or almost a slight smirk like, `Yeah, you want
  to take my picture? Sure. We'll see how this ends.''' We suddenly just
  saw this eruption of forceful violence. It was an interesting prism
  through which to watch a conflict between cartel gunmen and armed
  forces. It wasn't told through any one narrator. It was told through
  the eyes of a thousand people.'' ``Minute by minute, hour by hour, and
  a fuller story emerges.'' {[}GUNSHOTS{]} ``It is a story about how the
  Sinaloa cartel took on the Mexican government -- and won.'' {[}MUSIC
  PLAYING{]}
\end{itemize}

\begin{center}\rule{0.5\linewidth}{\linethickness}\end{center}

Soon after, a different cartel
\href{https://www.nytimes.com/2019/11/07/world/americas/mexico-mormon-massacre.html?action=click\&module=News\&pgtype=Homepage}{gunned
down nine Mormon mothers and children}, another haunting reminder of the
toll taken on innocent civilians. In the aftermath, President Trump
\href{https://www.nytimes.com/2019/12/06/us/trump-drug-cartels-terrorists.html}{threatened}
to designate the cartels as
\href{https://www.nytimes.com/2019/11/27/world/mexico-trump-terrorist-cartel.html}{terrorist
groups}.

Mr. Capella was well aware that his own solution to the cartels was
dangerous, particularly because it relied on the unsavory prospect of
setting a prolific killer free.

``It's something few have dared to do,'' the police chief acknowledged,
``but it is worth the risk.''

But no one, least of all the sicario, expected how the arrangement would
end.

Mr. Capella moved on to another job almost 1,000 miles away, and the
program slowly collapsed. With no legal mandate or official support, it
buckled this year under the change in political winds. Some of the
witnesses left and returned to lives of crime. At least one was
murdered.

The sicario stayed until the summer when, fearful the police were going
to hand him over to his cartel enemies, he fled.

Gunmen were not far behind. His brother --- who studiously avoided crime
and had enlisted in Mexico's armed forces --- was killed days later. His
parents found a note attached to the body: This is what happens to
snitches, it warned.

``This is the way things work in Mexico,'' the sicario, who asked that
his name not be used for his family's safety, said while on the run.
``And I want the world to see it.''

Image

The body of a man who was murdered was found in the water off this
popular beach in central Acapulco.Credit...Tyler Hicks/The New York
Times

\hypertarget{the-making-of-a-sicario}{%
\subsection{The Making of a Sicario}\label{the-making-of-a-sicario}}

The cartel bosses huddled in a small group, taunting him. Sure, he could
rob, even fight, his fellow gangsters teased him. But he couldn't kill,
they said. He didn't have the heart.

They snickered, pushing to see how far he would go. He knew it was a
test.

He was 17 and working for Guerreros Unidos, a cartel that operated
across several states and smuggled heroin to the United States. Right
away, he distinguished himself as smart and naturally violent. A
prospect in their world.

He snapped back. They didn't know what he was capable of, he said. In
truth, he didn't either.

His fellow gangsters pointed down the street at two young men --- a pair
of unwitting targets. He took off toward them, wondering if his bosses
were right, that he couldn't take a life. Then, as if someone else was
controlling his movements, he pulled a small knife from his pocket and,
without any warning, slit the throat of the young man closest to him.

As the blood spewed, he recalled, he buried his fear, determined to
prove he was merciless, the essence of a sicario.

``I blocked myself, my own emotions, and told myself it was someone else
doing it,'' he said.

He later discovered that the two men were innocent, part of a game his
bosses were playing. They hadn't expected him to actually kill anyone.

When word spread, and the glow of admiration came from friends and
others, his guilt subsided. No one would question him again. He was on
the path now, brutal and immutable, to becoming a professional killer.

``They liked this,'' he recalled. ``This opened up a career for me.''

Image

A river in Morelos where the sicario dumped a body before he was
arrested.Credit...Alexandra Garcia/The New York Times

In more than a dozen interviews, the sicario said his childhood was
normal, even good. His parents were together. They taught him to care
for others.

``I was taught values, principles,'' he said.

Tall and slender, with a round face and hooded eyes, he moved with the
economy of an athlete, which he was. He once hoped to play professional
soccer, but he skipped school to hang out with a small gang, smoking pot
and getting into fights. Eventually, he dropped out.

Some days, he followed his father to work, joining him on his rounds for
the local water company. For a while, he thought about making a life of
such work, however mundane and underpaid.

Then his father lost his job, plunging the family toward financial ruin.
His mother began working from dusk until dawn for a few dollars a day.
With growing resentment, he watched the humiliation and low pay of day
labor, while local gangsters made big money, enjoying a respect that
bordered on fear.

``That's when I chose to live day by day,'' he said. ``I became a
criminal.''

He worked his way up, from a small-time lookout for Guerreros Unidos to
robbery and drug sales. The leaders noticed his ambition. After that
first killing, the cartel leader offered him a slot in the sicario
training camp.

It was 2012, and Mexico's war on drugs was in its sixth year. Violence
had reached record highs as the military took to the streets to combat
organized crime and the cartels battled one another for supremacy.

Murder became a form of messaging, a spectacle of sadism --- bodies
hanging from bridges, chopped in pieces, deposited in public plazas,
each grisly crime scene a warning, a way of saying the cartel's violence
knew no limits.

As the drug market churned, with new players rising and falling,
training camps became academies for the industry's enforcers. The
sicario saw an opportunity.

For six months, he lived in austerity with dozens of other men in the
mountains of southern Mexico, he said, through terror, starvation and
cold. Everywhere the specter of death.

They hunted and killed rival cartel members, and were killed themselves,
often by their own trainers who disposed of them for disobeying orders
or showing hesitation, he said. Trainees who ran afoul of the
instructors were strung up from trees and used for target practice, he
recalled --- a claim that experts on cartels found plausible.

Knowing he might die for failing to follow orders --- whether killing a
farmer, cutting up a body or torturing a friend --- was all the
incentive he needed to do the unthinkable. At least that's how he
justified it.

``They turned me into an animal,'' he said.

But behind every decision, every inhuman act, was a truth he could not
escape. He chose this life. It was what he wanted.

Image

A murdered taxi driver's wife and daughter at the crime
scene.Credit...Tyler Hicks/The New York Times

\hypertarget{the-murder-business}{%
\subsection{The Murder Business}\label{the-murder-business}}

In a year, he had transformed into a skilled assassin --- battle-tested
and not yet 20 years old.

After the training camp, he was sent to Acapulco, he said, to fight
other cartels for the lucrative drug market in tourist districts.

A year or so later, he returned, but to a very different Morelos. His
old boss had been gunned down and his old cartel, Guerreros Unidos, was
nearly vanquished there, swallowed up by its one-time allies, Los Rojos.

The sicario no longer had a champion, or any allegiance at all.

Some of his old comrades had switched sides, which happened in cartel
warfare, the winners subsuming the losers.

The Rojos leader, Santiago Mazari Hernández, known on the street as El
Carrete, sent an emissary to recruit the sicario. He wanted him to help
set up drug operations across southern Morelos state. The past was the
past, he said.

``It was join them or be killed,'' the sicario recalled.

They began selling drugs in Jojutla, then spread to Tlaltizapan,
Tlaquiltenango, Zacatepec, fighting off other groups in the small towns
across southern Morelos.

As their business expanded, so did their influence, especially on local
government. They had local officials everywhere on the payroll, the
sicario said, to prevent surprises like arrests or seizures.

Expanding operations meant cleaning out the competition, not just other
cartels, but also local criminals --- thieves, rapists, small-time drug
dealers and snitches. Anyone who drew police scrutiny.

Murder was rarely for sport, the sicario said. He studied his victims at
length, investigating the complaints against them. Once confirmed, he
warned them to stop, mostly to keep them from drawing too much attention
from the authorities. If they didn't, he planned the killings
meticulously, carrying them out only with approval from above.

``For me to kill someone, I had to have permission,'' he explained.
``Why do I want to kill that person? Not because I just don't like them.
That's not how it works.''

He followed a code, he said. He didn't recruit children, and wouldn't
harm women or working people, if he could avoid it. But the workings of
organized crime were rarely orderly. He did kill women and innocent
civilians. For all the talk of honoring a code, it was often just that:
talk. Business always came first.

The New York Times confirmed many of his homicides with the authorities
and attempted to speak with the victims' families in several cases. All
refused. Having lost their daughters, sons and fathers to the cartel,
they were fearful of reprisals.

Of all the people the sicario killed in his five-year run, only a few
haunted him, he said. One in particular.

It was during a routine operation, he recalled, when his bosses sent him
to eliminate a group of local kidnappers. After he arrived, he said, he
found a college student with them. The sicario said he knew instantly
the student was innocent: the look of terror on his face, his body
language, even his clothes. They were all wrong.

Following protocol, the sicario tied everyone up and called his boss. He
wanted to let the young man go. He was unaffiliated. There was no need
to kill him. But the boss said no. Any witness was a liability.

As the boy begged for his life, the sicario said he looked away and told
him he was sorry before slitting his throat.

``That student still haunts me,'' he said, weeping. ``I see his face,
that kid begging me for his life. I will never forget his eyes. He was
the only one who ever looked at me that way.''

Image

Mexican police near the city of Cancún.Credit...Tyler Hicks/The New York
Times

\hypertarget{betrayal-and-capture}{%
\subsection{Betrayal and Capture}\label{betrayal-and-capture}}

Sometimes, in the dark, the sicario's mother quietly knelt beside his
bed, whispering over him as he slept. She knew he worked for the
cartels, even if she didn't know how exactly. Prayer was all she had
left.

``Stop doing that,'' he recalled telling her one night. ``Your God can't
save me.''

By late 2016, he had grown numb to killing, hunting for targets with a
mechanical indifference. Life mattered even less to him, his own
included.

He received a promotion, which brought higher pay, more responsibilities
and the envy of others. He still worked for El Carrete, who ran Los
Rojos cartel, but he was consumed by paranoia, and for good reason.

The deeper he descended into the underworld, the more he understood the
petty rivalries among the leadership. Their lives were steeped in
mistrust. The work demanded it. Friends betrayed friends, right-hand men
killed bosses.

He was told to kill members of his own team by leaders who worried they
were growing too influential or undisciplined. He said he killed so many
that he began to reconsider whom he hired.

``I almost never recruited within my friendship circles,'' he said. ``I
would recruit whatever guy wanted easy money.''

But that left him vulnerable, unable to trust his team. It proved to be
his undoing.

In May of 2017, the police took one of his partners into custody. To
avoid prison, he promised them the sicario.

Image

Another participant in the makeshift witness protection program sleeping
in a jail.Credit...Alexandra Garcia/The New York Times

On May 15, the partner called the sicario. They had work to do, he said.
It was bright outside, odd working hours for the men, but there was an
emergency, his partner said.

They met up at a safe house and left together, heading toward their
motorcycles parked down the street. The sicario heard the police before
he saw them, the screech of tires, the revved engines. It was over in
less than a minute.

He cursed himself on the way to the station. For years, he had survived
on suspicion, yet somehow missed this easy setup. He wondered whether
dumb luck alone had saved him all these years.

At the station in Jojutla, a small white building facing the district
prison, police commanders confiscated his phone. It contained enough
evidence to put him away for life.

While he sat handcuffed to a chair, the officers watched a snuff film of
his work, which he had recorded on his phone. In it, one of the cartel's
lawyers, who had gone missing, sat in the shallow eddy of a river,
bloody and terror-stricken, confessing a betrayal.

The police called his mother, who refused to believe them. Yes, she knew
her son was a criminal, she recalled. But she refused to believe he was
a killer --- until an officer made her watch an interview in which her
son confessed to his myriad homicides.

``We never taught him these things,'' she said, sobbing. ``He didn't
learn that malice from us. We gave him love and support.''

The police began adding up what they knew, starting with several
homicides that traced back to him. He faced 240 years in prison for
those alone.

But the police chief, Mr. Capella, had grown weary of the state's
limited tools and ambitions. Sloppy forensics, corrupt officers and
haphazard investigations left few cases solved.

He had previously been a police chief in Tijuana, where the local press
nicknamed him Rambo in 2007 for fighting off dozens of cartel assassins
in an all-out battle that riddled his home with bullets.

Now, as the commander in Morelos, he wanted results. As the sicario sat
in a ripped vinyl chair in the precinct, one of Mr. Capella's deputies
explained the arrangement.

The sicario would testify against his former comrades, detailing the
many murders they had committed. But instead of describing the sicario
in court or in case files as one of the killers or main conspirators,
the state authorities listed him as a witness --- someone with no real
involvement in the crime.

The sicario, then 22, agreed to live in a building next to the prison
for his own protection and be shuttled to public hearings. The state
authorities did not charge him with any of the killings, choosing to
wait until he was done testifying. Then, they could decide how to
prosecute him, if at all.

By law, cartel cases in Mexico are supposed to be handled at the federal
level, by a division tasked with investigating organized crime. The
group can use its plea bargain powers to persuade witnesses to come
forward, though few do. It is widely distrusted.

At the state level, no such program exists, and officials have often
found their own ways of chasing justice, sometimes by breaking the law
entirely. Many have held suspects in detention for years before trial as
a form of punishment, knowing they didn't have the evidence for a
conviction. Others have opted for a more brutal solution: the
extrajudicial killing of suspected criminals.

Mr. Capella tried a very different approach --- looking for convictions
in court, and ginning up a new set of rules to secure them. Tired of
Mexico's feeble rule of law, Mr. Capella decided to create his own
version of it.

His unorthodox methods and unapologetic manner have brought him
controversy, and plenty of enemies. The current government of Morelos
has accused him of misappropriating funds in a separate matter, which he
strongly denies.

Some former justice officials in Mexico call his witness protection
program a stretch, operating well outside of legal norms. Others say it
is so unusual that they are not quite sure. Even state officials in
Morelos who supported the program acknowledged that it operated in a
legal gray area, though, like Mr. Capella, they called it legal,
defensible --- and highly effective.

``I'd rather make a big mistake than be guilty of inaction,'' Mr.
Capella said. ``Mexico is tired of this institutional paralysis.''

Image

The sicario, center in pink shirt, at a religious service held for
members of a makeshift witness protection program.Credit...Tyler
Hicks/The New York Times

\hypertarget{its-a-miracle-i-survived}{%
\subsection{`It's a Miracle I
Survived'}\label{its-a-miracle-i-survived}}

For five years, the sicario lived as two different people: the son who
dropped off groceries for his mother and had a baby of his own with his
girlfriend; and the ``monster,'' as he called himself, who killed for a
few hundred dollars a week.

After his arrest, the wall between them began to crack. He suffered what
seemed like psychotic episodes, he said, sleepless nights of strange
voices and shadows collapsing on him. He knew he deserved no pity, that
he alone was to blame. He took some comfort in that.

``I was at the point of going crazy,'' he said. ``I would spend two or
three days crying.''

Eventually, a pastor --- an uneducated, reformed convict himself ---
came to see him. At first, the sicario worried the man was a spy sent by
his enemies. Eventually, he began to speak to him and, before long,
could hardly stop.

The pastor was caught off guard by the torrent of confessions as the
sicario gave himself over to the Bible with a fervor he once held for
violence, a conversion so common it is almost a cliché in the world of
gangs and cartels.

``That other person is dead,'' the sicario said as if, with repetition,
it might become true.

He found new purpose in confinement, helping solve cold cases,
testifying against cartel players and paving the way for some two dozen
convictions. The police said they saw a real transformation in him,
though they had their own reasons to believe it, too.

By October of 2018, the police had expanded the program to include a
dozen cooperating witnesses. With no other place to put them, the
authorities housed the young men right next door to the jail that held
the cartel members they were testifying against. Every few weeks, the
police ferried them to court to provide evidence in cases.

The witnesses slept on thin mattresses on the floor, ate at a cracked
plastic table and sat in chairs shorn of their backs. Large blue tubs
overflowed with water used for bathing and flushing.

There were small comforts --- a television, a microwave and an electric
keyboard on which the sicario taught himself to play the theme song to
the movie ``Titanic.'' And every weekday, the makeshift wing of the
prison turned into an evangelical revival.

A pastor strummed an old guitar and led them in hymns. When the singing
stopped, they took turns confessing --- the soulless acts of violence
they had committed, their temptation to return, their gratitude for
having been saved.

``Sixteen years ago, I was like you boys,'' the pastor said, the guitar
resting against his belly. ``It's a miracle I survived.'' Several began
to cry unprompted.

The sicario, whose crimes far exceeded those of the others, was the
natural leader. He became a parental figure for the group, and enforced
his will by wielding a large wooden stick.

Eventually, the young men earned the trust of their wardens, and were
allowed an almost comic level of autonomy. By early 2019, they were
running their own security, locking and unlocking the barred entry for
visitors, monitoring comings and goings in the ward. A few even started
their own business, washing the government cars in the lot.

The police knew the risks were big, as was the possibility of failure.
But their confidence grew by the day. Mr. Capella, the police chief,
boasted of the change the sicario's testimony had made on the streets.
One deputy said the sicario would walk free with a clean rap sheet.

``We have achieved what we set out to achieve,'' Mr. Capella said.

Image

Police on patrol in Quintana Roo, where the former police chief in
Morelos, Alberto Capella, is now chief.Credit...Tyler Hicks/The New York
Times

\hypertarget{you-wont-stand-a-chance}{%
\subsection{`You Won't Stand a Chance'}\label{you-wont-stand-a-chance}}

The unwinding came sooner than expected. More than a year into the
program, Mr. Capella got a new job as police chief in the state of
Quintana Roo. Home to the neon hum of Cancún and boho-chic of Tulum, it
was a much bigger post than Morelos.

With his departure, the witness protection program lost its steward. It
was expensive, and off the books. No one wanted to oversee someone
else's pet project.

The young men continued to attend their court dates, the pastor kept
turning up and the sicario's girlfriend gave birth to their second
child, a girl. But the energy of even a few months earlier began to
vanish.

Nearly half of the witnesses were gone. Some had finished their court
appearances and left of their own volition. Others had skipped out,
content to risk the death sentence that awaited them on the street. Many
had grown accustomed to the idea of an early death. To them, the program
was a brief respite.

The sicario talked less about what came next. Before, he practically
counted the days until his departure. Now he merely shrugged when asked.

In truth, he had grown used to the facility. He liked the respect from
the guards, the prosecutors and his fellow witnesses. It was a sanctuary
from the outside world, which frightened him. Not only did he worry
about the cartel and a life on the run, but he also feared the
temptation --- that for all his talk of change, he would wind up right
back where he started.

``I know that being released and forming part of society again is harder
than being locked up in here,'' he said after a prayer session. ``The
truth is, I'd rather be in here, in pain, for 10 years than out there on
my own.''

By the summer of 2019, the program was in rank disrepair --- dirty
dishes piled up, water pooled on the floor, toilets were left uncleaned.
The lights didn't even function properly anymore.

``Everything is coming to an end,'' he said one day. ``Just take a look
around you. The world is upside down.''

He was practically alone now. Only one other witness remained. His
friends came by periodically, to smoke weed or listen to music in the
dark. He used them to ferry messages to people on the outside, including
drug dealers.

The police had all but abandoned the program. Most officials were happy
to see it empty out, eager to be done with the burden.

In the void, the sicario returned to what he knew: selling drugs. While
still inside, he recruited former witnesses who had left the program,
forming a team of marijuana dealers from the same youth he had once
vowed to rescue.

The pastor found out and pressed him to stop.

``I realized how many people I was dragging to their doom again,'' the
sicario said. ``I led my friends toward the Bible, and now I'm making
them sell drugs.''

His relapse seemed almost inevitable. How could the state expect to
change someone so stripped of his humanity in just two years, with an
unpaid, uneducated pastor as his only source of inspiration?

Perhaps it never intended to. The sicario had helped dismantle his
former cartel, leaving it in shambles. He was no longer of much use to
the police.

On the outside, his enemies would see him as weak, no longer under the
protection of the police. He liked to claim that his reputation on the
streets kept his family safe, but that wasn't entirely true, either.
Even the police knew as much. The sicario had softened since joining the
program. He cared about his family, his children, the prospect of a new
life. Hope was a liability in his old world.

One of the police officers had warned him about leaving.

```You won't stand a chance out there,''' he recalled the officer
saying. ```You aren't the same person anymore.'''

``He got it right,'' the sicario said. ``It's true.''

Image

The sicario with his son after a visit to church.Credit...Alexandra
Garcia/The New York Times

\hypertarget{justice-for-me-would-be-death}{%
\subsection{`Justice For Me Would Be
Death'}\label{justice-for-me-would-be-death}}

On a sunny afternoon in August, the sicario fled. A tipster warned him
that the police were planning to arrest him and bring charges. True or
not, he didn't take the chance.

He had been careless before, when he was caught the first time. But now,
after all the people he had helped lock up, going to prison for real ---
with inmates, not cooperating witnesses --- would mean certain death. He
would be killed the moment he entered.

He slipped out of the facility and checked into a small roadside hotel.
After nearly two years under police protection, he was on his own.

A few days later, on Aug. 5, a pair of gunmen posing as customers came
to his parents' taco stand and shot his brother four times. As the
killers fled, they left a note: ``Let's see if you all learn this way.''

The brothers looked alike, so the gunmen may have thought they had
killed the sicario. When he found out about the shooting, he wished they
had.

His brother was innocent, the family insisted. He had never associated
with organized crime, on the sicario's orders. He finished high school,
lived at home with his parents, had enlisted to join the Mexican armed
forces and was scheduled to head out soon, his mother said.

The sicario knew he didn't deserve freedom. ``Justice for me,'' he
sometimes said, ``would be death.'' But his brother was different.

``They hit me where it hurt most,'' the sicario said, crying, not long
after the murder. ``The thing I loved most in the world, they took from
me.''

Still, he insisted that he would not seek revenge. Nothing would change
because of it. His brother would still be dead. The killings would
continue, even escalate, sucking in the rest of his family, in the kind
of unending cycle Mexico itself is trapped in. Murder was inevitable, he
said. His involvement didn't have to be.

``This will never end, no matter what I do,'' he said. ``But I just
won't be a part of it anymore.''

Advertisement

\protect\hyperlink{after-bottom}{Continue reading the main story}

\hypertarget{site-index}{%
\subsection{Site Index}\label{site-index}}

\hypertarget{site-information-navigation}{%
\subsection{Site Information
Navigation}\label{site-information-navigation}}

\begin{itemize}
\tightlist
\item
  \href{https://help.nytimes.com/hc/en-us/articles/115014792127-Copyright-notice}{©~2020~The
  New York Times Company}
\end{itemize}

\begin{itemize}
\tightlist
\item
  \href{https://www.nytco.com/}{NYTCo}
\item
  \href{https://help.nytimes.com/hc/en-us/articles/115015385887-Contact-Us}{Contact
  Us}
\item
  \href{https://www.nytco.com/careers/}{Work with us}
\item
  \href{https://nytmediakit.com/}{Advertise}
\item
  \href{http://www.tbrandstudio.com/}{T Brand Studio}
\item
  \href{https://www.nytimes.com/privacy/cookie-policy\#how-do-i-manage-trackers}{Your
  Ad Choices}
\item
  \href{https://www.nytimes.com/privacy}{Privacy}
\item
  \href{https://help.nytimes.com/hc/en-us/articles/115014893428-Terms-of-service}{Terms
  of Service}
\item
  \href{https://help.nytimes.com/hc/en-us/articles/115014893968-Terms-of-sale}{Terms
  of Sale}
\item
  \href{https://spiderbites.nytimes.com}{Site Map}
\item
  \href{https://help.nytimes.com/hc/en-us}{Help}
\item
  \href{https://www.nytimes.com/subscription?campaignId=37WXW}{Subscriptions}
\end{itemize}
