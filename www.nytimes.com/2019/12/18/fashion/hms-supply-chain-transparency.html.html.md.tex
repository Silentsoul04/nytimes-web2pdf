Sections

SEARCH

\protect\hyperlink{site-content}{Skip to
content}\protect\hyperlink{site-index}{Skip to site index}

\href{/section/fashion}{Fashion}\textbar{}H\&M's Different Kind of
Clickbait

\url{https://nyti.ms/38Ocqnf}

\begin{itemize}
\item
\item
\item
\item
\item
\item
\end{itemize}

\includegraphics{https://static01.nyt.com/images/2019/12/18/fashion/18HM-consumption-spinning/merlin_163482792_f01470dd-eb2b-4565-babb-16c17020f161-articleLarge.jpg?quality=75\&auto=webp\&disable=upscale}

\hypertarget{hms-different-kind-of-clickbait}{%
\section{H\&M's Different Kind of
Clickbait}\label{hms-different-kind-of-clickbait}}

The Swedish retailer now lets customers know where nearly every garment
it sells is made. Is that enough?

Spinning yarn inside the DBL factory.Credit...Fabeha Monir for The New
York Times

Supported by

\protect\hyperlink{after-sponsor}{Continue reading the main story}

\href{https://www.nytimes.com/by/elizabeth-paton}{\includegraphics{https://static01.nyt.com/images/2019/12/05/reader-center/author-elizabeth-paton/author-elizabeth-paton-thumbLarge.png}}\href{https://www.nytimes.com/by/sapna-maheshwari}{\includegraphics{https://static01.nyt.com/images/2018/02/20/multimedia/author-sapna-maheshwari/author-sapna-maheshwari-thumbLarge.jpg}}

By \href{https://www.nytimes.com/by/elizabeth-paton}{Elizabeth Paton}
and \href{https://www.nytimes.com/by/sapna-maheshwari}{Sapna Maheshwari}

\begin{itemize}
\item
  Dec. 18, 2019
\item
  \begin{itemize}
  \item
  \item
  \item
  \item
  \item
  \item
  \end{itemize}
\end{itemize}

\href{https://www.nytimes.com/2018/03/27/business/hm-clothes-stock-sales.html}{The
H\&M group} sells an estimated three billion articles of clothing per
year. Its revenue makes it among the top three fashion retailers in the
world.

Clothing for its brands, including H\&M, Arket and \& Other Stories, is
manufactured in 40 countries, the company said; in Bangladesh alone, it
sources from 275 factories that employ half a million workers.

As it sprawls ever farther around the globe, hopping from trend to
trend, how can H\&M keep track of how the skirts, pants and sweaters it
sells are made? How, for example, can it monitor whether, in faraway
countries,
\href{https://www.nytimes.com/2019/09/03/books/review/how-fast-fashion-is-destroying-the-planet.html}{workers
are being paid less than they need to live}, forced to work hours of
overtime in precarious conditions?

This spring, after almost three years of preparation and coordination by
40 team members from Hong Kong to Stockholm, and at a time when
\href{https://www.nytimes.com/interactive/2019/climate/sustainable-clothing.html}{scrutiny
of the global fashion industry} and its shadowy supply chain is greater
than ever, H\&M introduced an effort to do exactly that --- and to make
it public for shoppers.

Now, the company says, it can be held accountable for the origins of its
products. If consumers care to look.

\includegraphics{https://static01.nyt.com/images/2019/12/18/fashion/18HM-consumption-inspection/merlin_163482507_ec29ed2c-e01e-4d13-8818-c3103d896403-articleLarge.jpg?quality=75\&auto=webp\&disable=upscale}

\hypertarget{they-made-your-leggings}{%
\subsection{They Made Your Leggings}\label{they-made-your-leggings}}

Browsing the H\&M website this month, you may find yourself taken with a
\href{https://www2.hm.com/en_gb/search-results.html?q=0796179002}{ladies'
amber sweater with ``Hiver''} written on the front, or else a pair of
\href{https://www2.hm.com/en_gb/productpage.0808051001.html\#modalSuppliers}{pink
children's leggings}, with smiling bunny faces and ears that stick out
from the knees for \$4.99.

Click on the ``product sustainability'' tab on the page, and you will
learn they were made in Bangladesh by some of the 13,000 workers at the
Jinnat Apparels \& Fashion plant in Gazipur, a dense manufacturing
neighborhood near Dhaka.

This is part of the company's new ``consumer-facing transparency
layer.'' H\&M shoppers can now find out not only the country where
clothing was manufactured, but also details on materials and recycling,
the name of the supplier or authorized subcontractor where a garment was
made; the factory address; and the number of workers employed there.

Customers shopping in physical stores can also have access to this
information by using the H\&M app to scan the product price tag.

There are limits to how much information you'll get, of course. The
sustainability tab won't tell you that Jinnat sprawls over seven floors,
each the size of a football field, or that employees perch in front of
whirring sewing machines making white cotton T-shirts, monitoring 337
high-tech embroidery appliances and snipping at stray threads.

And you won't find out that this single company makes 400,000 pieces
(roughly 110 tons) of clothing per day, or around 10 to 12 million units
per month, up to a quarter of which will be bound for H\&M.

Nevertheless, it is the first effort of its kind by a retailer of this
scale.

H\&M created the system by building a bridge between its supplier and
production databases and then linking it to its retail interfaces. (The
company declined to say what the project cost.)

Pascal Brun, the head of sustainability for the H\&M brand, said the new
public transparency layer showed that the company had nothing to hide
regarding labor or environmental practices, or how H\&M products were
made.

``It is not going to change the world,'' he said. ``But it is about
building a foundation for real change, given we can't build this
industry from the ground up all over again.''

Image

Yarn holders are piled for spinning.Credit...Fabeha Monir for The New
York Times

\hypertarget{seeing-through-transparency}{%
\subsection{Seeing Through
Transparency}\label{seeing-through-transparency}}

``Transparency has become the key driver of change in the fashion
industry, which used to be about as untransparent an industry as it
could possibly be,'' said David Savman, the head of production for the
H\&M group, from a factory floor in Dhaka.

Tanned and golden haired, the Swede filed between rows of workers and
inspected sequined T-shirts, asking line managers about different cotton
hybrids and admiring fire doors.

Change came crashing down on the industry with the
\href{https://www.nytimes.com/2015/06/02/world/asia/bangladesh-rana-plaza-murder-charges.html}{Rana
Plaza disaster in Bangladesh in 2013}, a factory collapse that led to
the death of more than 1,000 workers, with scores more disfigured or
disabled for life.

In the wake of the catastrophe, several Western retailers found they had
sold clothes sourced from the factory, or had little to no idea where
the clothes they sold were sourced from. All have since come under
increasing public pressure to investigate, police and invest in exactly
where and how their products were made.

There is also pressure for them to be as transparent about their
findings as possible (though some have been far more forthcoming than
others about taking action).

The creation in Bangladesh in 2013 of two five-year fire and safety
monitoring agreements between retailers and unions made significant
improvements and reforms.

The Accord on Fire and Building Safety, which is legally binding, was
signed by more than 200 retailers including H\&M and Inditex (neither of
which had any ties to Rana Plaza, but plenty of other alleged supply
chain abuses). The other agreement is the nonbinding Alliance for
Bangladesh Worker Safety, which was signed by Walmart, Gap and Target.

Both have spurred improved working conditions in many Bangladeshi
factories, and calls for other countries to adopt similar standards.

These agreements,
\href{https://cleanclothes.org/news/2019/questions-raised-after-agreement-reached-on-bangladesh-accord}{now
up for renewal}, have sidelined some of the country's most dangerous
factories, and cut their ties to most Western retailers, though not all.
A
\href{https://www.wsj.com/video/unsafe-factories-in-bangladesh-are-supplying-amazon-sellers/120C9E33-4C91-43D9-AE28-47B42DF47405.html}{Wall
Street Journal investigation} in October found that Amazon continued to
sell clothes from Bangladeshi factories that other retailers had
blacklisted because of their inability to pass safety requirements.

Pressure from consumers has also prompted brands like H\&M to
proactively support local suppliers who create safe and profitable
businesses in places like Bangladesh.

``We choose not to work with a lot of suppliers that other rivals work
with so they can save on costs,'' said Karl-Johan Persson this fall. (In
2018 six suppliers in Bangladesh were phased out by H\&M because of
their poor sustainability performance.)

Mr. Persson, the billionaire chief executive of H\&M, sat in the
\href{https://www.nytimes.com/2016/12/24/fashion/wintering-the-danish-way-learning-about-hygge.html}{``hygge''-style
library} for the company's army of young designers in Stockholm as he
defended his family company's business model and its contributions.

He declined to specify how much H\&M spent annually on transparency
efforts, other than to say the investment had continually hurt
short-term profit in order to ensure the long-term survival and growth
of the company.

His argument is that by working in low-cost areas, H\&M is creating jobs
and investing in the economy; by making its partnerships public, it is
accepting its own liability.

``But often,'' Mr. Persson said, ``the focus ends up on what we don't
do.''

The new ``transparency layer'' project has been cautiously applauded by
some human rights and fashion advocacy groups and union leaders. But
many have also said that H\&M's efforts do not go far enough,
questioning whether improvements like this are worthwhile if they merely
prolong the existence of a system where profits and shareholder
interests are continually placed ahead of employees, suppliers and the
environment.

Currently, customers do not have access to information on workers' wages
at individual factories, or local minimum fair living wage commitments
and calculation methodology. Nor does the transparency layer offer a
breakdown of the pricing structure that could specify how labor costs
are calculated.

``Transparency is primarily a means to an end, and mere information
about where a garment is produced does not automatically guarantee
meaningful changes in factory labor conditions,'' said Aruna Kashyap,
senior counsel for the women's rights division at Human Rights Watch,
which is part of a coalition that started the
\href{https://www.hrw.org/news/2017/04/20/more-brands-should-reveal-where-their-clothes-are-made}{Transparency
Pledge} (of which H\&M is a signatory).

``H\&M is among the leaders on supplier transparency, and other
companies should follow this practice,'' Ms. Kashyap said. ``But that
doesn't mean that H\&M and other companies that are transparent have
fixed an industry model that is replete with problems.''

Image

A Bangladeshi employee working at the fabric knitting section of a
factory that produces garments for H\&M's more than 4,000 stores around
the world.Credit...Fabeha Monir for The New York Times

\hypertarget{the-model-and-the-problems}{%
\subsection{The Model and the
Problems}\label{the-model-and-the-problems}}

Even after the Rana Plaza tragedy, the global business
\href{https://www.nytimes.com/2018/04/24/style/survivors-of-rana-plaza-disaster.html}{model
for producing low-cost clothing} remains the same. Most brands don't own
their own production facilities, but instead contract with independent
factories to make their garments. Generally, in these factories, located
in mostly developing economies, very low wages are paid to workers using
manufacturing processes that are geared toward expediency rather than
the environment.

Subcontraction or homeworking remain common, and make it even harder to
track where clothes come from.

The industry is operating at an almighty scale. In total, across the
fashion industry, 80 billion garments are produced each year,
\href{https://www.greenpeace.org/international/story/7539/fast-fashion-is-drowning-the-world-we-need-a-fashion-revolution/}{according
to Greenpeace}, with consumer demand and appetite for trend-fueled
fashion only growing stronger, in part thanks to a
\href{https://www.nytimes.com/2014/04/10/fashion/fashion-in-the-age-of-instagram.html}{digital
culture powered by social media} and the wallets of a young emerging
global middle class.

The worldwide apparel and footwear market's expected growth, pegged at
roughly 5 percent through 2030 by Euromonitor analysts, would risk
``exerting an unprecedented strain on planetary resources'' by raising
annual production of fashion to more than 100 million tons, according to
a
\href{https://fashinnovation.com/fashions-sustainability-push-isnt-keeping-up-with-growth/}{Euromonitor
report}.

The pressure to meet those demands, and the demand for ever-cheaper
labor, are at odds with the move toward transparency and tightly managed
supply chains. Many major brands in Europe and North America continue to
have limited information about the factories and workers producing their
wares.

Inspections are usually delegated to third-party auditors, which have
proven to be far from foolproof and at the mercy of the often uneven
tides of developing nations.

Revelations of egregious failures within the garment industry still
emerge on a regular basis. A Guardian story in October reported that the
active wear company
\href{https://www.theguardian.com/global-development/2019/oct/14/workers-making-lululemon-leggings-claim-they-are-beaten}{Lululemon
had been sourcing clothing from a factory} where Bangladeshi female
factory workers said they were assaulted.

This month, in Delhi, India, a fire broke out in a factory that made
school bags and killed 43 workers, including children, who were asleep
on the floors inside.

Last year, Transparentem, a nonprofit focused on investigating human and
environmental abuses in the apparel industry, published
\href{https://www.transparentem.com/projects/}{a report about abusive
conditions} and forced labor at a set of Malaysian apparel factories
that made wares for brands in North America and Europe such as Primark,
Asics, Nike and Under Armour.

Image

A firefighter on duty at a DBL factory in Bangladesh; after the Rana
Plaza disaster, Western retailers woke up to their own
responsibilities.Credit...Fabeha Monir for The New York Times

\hypertarget{servitude-and-lack-of-a-living-wage}{%
\subsection{Servitude and Lack of a Living
Wage}\label{servitude-and-lack-of-a-living-wage}}

According to the Transparentem report, many workers, often migrants from
Bangladesh and Nepal, said that they paid steep recruitment fees to
acquire jobs. These could take years to pay back, resulting in ``debt
bondage,'' a common form of modern slavery that occurs when a person is
forced to work to pay off debts for little or no pay.

Factories limited employees' movements by withholding their passports;
it wasn't unusual for them to live jammed together in squalid
conditions. Many also had to pay a government levy on foreign workers
out of their own paychecks (a practice that was legal when Transparentem
interviewed workers in 2016 and 2017).

``The physical distance, cultural distance, and often time zone
difference have all meant that there are inherent challenges in
understanding the labor conditions in any manufacturer supply chain,''
said Benjamin Skinner, the founder and president of Transparentem.

Brands have largely trusted suppliers to follow certain rules with
employees and the environment and then verified that those policies were
being followed, Mr. Skinner said.

But based on his organization's work, he added, ``the `verify' part can
be pretty weak.'' Because auditors would alert factory owners to their
visits, or only interview workers in the presence of their bosses, it
created an
\href{https://www.nytimes.com/2019/02/06/fashion/india-fast-fashion-homeworkers.html}{environment
where noncompliance was easy to hide}.

This gap between intent and reality also emerged in a May report from
University of Sheffield researchers in Britain on apparel companies not
delivering on promises to pay workers a living wage.

Generally set by governments (sometimes with input from foreign and
local businesses, unions and NGOs),
\href{https://www.nytimes.com/2019/06/05/smarter-living/what-a-living-wage-actually-means.html}{living
wages can differ significantly between countries}, with benchmarks
sometimes geared to maintaining a country's competitiveness as a
low-cost manufacturing destination rather than the needs of workers.

The wages can also be significantly less --- sometimes even falling
below the poverty line --- than the living wage as defined by outside
groups, which broadly incorporates food, housing, medical care, clothing
and transportation.

Many companies, including Adidas and Puma, referred to components of a
living wage in their supplier codes of conduct, the researchers said,
but the wording around requirements was ``very vague,'' leaving
fulfillment an option and the legal minimum wage the only requirement.

On top of all this, the researchers noted that companies relied heavily
on outside auditors to ensure codes of conduct were being followed,
running into the same issues outlined by Mr. Skinner.

Many of these firms are ``beholden by financial conflict of interest
since they are hired by companies who could decide not to continue to
hire them if they identify too many problems,'' they wrote. Often, they
visited only top suppliers, leaving out the many subcontractors where
abuses can be the worst.

Image

Client binders lining file cabinets inside the Jinnat Apparel \& Fashion
factory near Dhaka.Credit...Fabeha Monir for The New York Times

\hypertarget{who-polices-the-supply-chain}{%
\subsection{Who Polices the Supply
Chain?}\label{who-polices-the-supply-chain}}

After Transparentem revealed the Malaysian abuses to 23 companies with
direct or indirect buying relationships with the factories, most said
that they would take action.

Buyers and suppliers were able to negotiate the return of passports and
secure the reimbursement of recruitment fees for workers at several
facilities. (By November 2018, the total amount of fees paid and
scheduled to be paid exceeded \$1.4 million.)

Still, under the current system, the industry status quo means major
garment manufacturers are mopping up mistakes, rather than not making
them at all. This is the problem H\&M is trying to solve.

Mr. Savman of H\&M said that because H\&M did not own factories, all
sustainability efforts and investments like a Dhaka training center
ultimately focused on supporting and promoting processes and mechanisms
between suppliers, unions and workers that made them self-sufficient
when it came to problem solving.

A self-reporting system called the Supplier Partnership Impact Program
allowed H\&M to see issues and regulate what sort of monitoring was
needed and where. National Monitoring Committees --- round table
discussions between H\&M employees, union representatives and factory
owners --- attempted to resolve pay disputes and abuse allegations at
factory level.

Alongside regular auditing by independent groups, Mr. Savman said, H\&M
still frequently sent its own employees to monitor factories, sometimes
by prearrangement but often unannounced.

His colleague Payal Jain, the sustainability manager for H\&M's global
supply chain who started her career as a factory worker in India, said
that H\&M visited its factories several times per week, and 2,500 audits
were made in the country per year.

That may sound like a lot, but it is an average of 10 per factory --- in
365 days. Or less than once per month. The company
\href{https://www.reuters.com/article/us-workers-garment-abuse/hm-accused-of-failing-to-ensure-fair-wages-for-global-factory-workers-idUSKCN1M41GR}{was
also criticized by the Clean Clothes campaign} last year, which said
H\&M had not met a 2013 commitment made to ensure suppliers would pay a
living wage to 850,000 textile workers by 2018.

(H\&M said it had reached at least 600 factories and 930,000 garment
workers with its fair living wage strategy, and did not share the Clean
Clothes Campaign's view of how to create change in the textile
industry.)

Additionally, some factory owners say that despite support from H\&M's
sustainability teams, they experience pressure from the company or from
production teams who still want more product at a cheaper price --- or
they threaten to pull their business and go to even less expensive hubs,
like Ethiopia.

Ms. Jain said cost of labor was not a negotiable part of a supplier
contract. But if suppliers are paid less, or overtime is required to
complete a contract, the likelihood is that shortfall will get passed
down the chain.

``Brands like H\&M offer training, help union members establish
themselves in my factory and guide us on investing in the business,
which are all very good and important things,'' said Lutful Matin, the
manager of Natural Denims, another factory near Dhaka. It employs 6,900
workers to make garments for H\&M, Zara, Mango and Esprit.

``But then their buying teams still drive down order values and I feel
such pressure,'' Mr. Matin said.

He had proudly shown off the conditions and quality of his products.
But, he said, while ``I know I've invested more in my factory than
competitors, they still get orders. There are always new certificates
and alliances that need to be passed. Globally the trading market is
getting tougher. Sometimes I don't know how easy it will be to
survive.''

Image

Bobbins piled inside a yarn and dyeing area within a
factory.Credit...Fabeha Monir for The New York Times

\hypertarget{the-shoppers-role}{%
\subsection{The Shopper's Role}\label{the-shoppers-role}}

While the work it does is recognized by its recognition in projects like
\href{https://issuu.com/fashionrevolution/docs/fashion_transparency_index_2019?e=25766662/69342298}{Fashion
Revolution's Transparency Index}, H\&M believes the best way to get
consumers thinking about who made their clothes is to talk to them close
to the point of sale.

``Consumers have a lack of trust and say they don't always know how to
make the right choices,'' said Anna Gedda, the head of sustainability
for the H\&M group. She added that it was ``a constant struggle'' to
work out how much information a customer may want versus what might make
them switch off or walk away from a sale.

From Dhaka, Mr. Savman was more forthright. ``We are still at the stage
where if you put two T-shirts, one cotton and one recycled cotton, which
is 30 percent more expensive, the majority of consumers will still take
the first option,'' he said. ``We put a lot of information out there,
like the product transparency layer. But how much do customers engage
with it? Not a lot --- yet.''

Nearby, the managers and owners were keen to show off the scope and
quality of their Jinnat complex, from their high-quality Italian
knitting machines and subsidized food store and medical facilities to
the anonymous complaint boxes on every floor and payment system so that
workers can be compensated directly and efficiently.

As tens of thousands of workers streamed back into the steamy streets
for their lunch break, Abdul Wahed, the chairman, looked on.

``We are extremely proud of the factory here, and the work we have
done,'' he said. ``People can know when and where we make their
clothes.'' The onus is on them to click.

Advertisement

\protect\hyperlink{after-bottom}{Continue reading the main story}

\hypertarget{site-index}{%
\subsection{Site Index}\label{site-index}}

\hypertarget{site-information-navigation}{%
\subsection{Site Information
Navigation}\label{site-information-navigation}}

\begin{itemize}
\tightlist
\item
  \href{https://help.nytimes.com/hc/en-us/articles/115014792127-Copyright-notice}{©~2020~The
  New York Times Company}
\end{itemize}

\begin{itemize}
\tightlist
\item
  \href{https://www.nytco.com/}{NYTCo}
\item
  \href{https://help.nytimes.com/hc/en-us/articles/115015385887-Contact-Us}{Contact
  Us}
\item
  \href{https://www.nytco.com/careers/}{Work with us}
\item
  \href{https://nytmediakit.com/}{Advertise}
\item
  \href{http://www.tbrandstudio.com/}{T Brand Studio}
\item
  \href{https://www.nytimes.com/privacy/cookie-policy\#how-do-i-manage-trackers}{Your
  Ad Choices}
\item
  \href{https://www.nytimes.com/privacy}{Privacy}
\item
  \href{https://help.nytimes.com/hc/en-us/articles/115014893428-Terms-of-service}{Terms
  of Service}
\item
  \href{https://help.nytimes.com/hc/en-us/articles/115014893968-Terms-of-sale}{Terms
  of Sale}
\item
  \href{https://spiderbites.nytimes.com}{Site Map}
\item
  \href{https://help.nytimes.com/hc/en-us}{Help}
\item
  \href{https://www.nytimes.com/subscription?campaignId=37WXW}{Subscriptions}
\end{itemize}
