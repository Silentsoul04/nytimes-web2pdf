Sections

SEARCH

\protect\hyperlink{site-content}{Skip to
content}\protect\hyperlink{site-index}{Skip to site index}

\href{https://myaccount.nytimes.com/auth/login?response_type=cookie\&client_id=vi}{}

\href{https://www.nytimes.com/section/todayspaper}{Today's Paper}

\href{/section/opinion}{Opinion}\textbar{}Presidential Debates Could Be
Much More Imaginative

\url{https://nyti.ms/38XZoU6}

\begin{itemize}
\item
\item
\item
\item
\item
\end{itemize}

Advertisement

\protect\hyperlink{after-top}{Continue reading the main story}

\href{/section/opinion}{Opinion}

Supported by

\protect\hyperlink{after-sponsor}{Continue reading the main story}

\hypertarget{presidential-debates-could-be-much-more-imaginative}{%
\section{Presidential Debates Could Be Much More
Imaginative}\label{presidential-debates-could-be-much-more-imaginative}}

In the age of quizzes, ``The Great British Baking Show'' and access to
experts all over the world, we don't have to settle for the traditional
televised debate format.

By Malka Older

Dr. Older is a sociologist and science fiction author.

\begin{itemize}
\item
  Dec. 19, 2019
\item
  \begin{itemize}
  \item
  \item
  \item
  \item
  \item
  \end{itemize}
\end{itemize}

\includegraphics{https://static01.nyt.com/images/2019/12/19/opinion/19Older2/merlin_157865358_af8444a3-cd5c-45a0-9e16-f5f2fc875fc8-articleLarge.jpg?quality=75\&auto=webp\&disable=upscale}

The candidate debates have become such a fixture of presidential
elections in the United States that it's easy to imagine them as
inevitable and unchangeable components of our democratic process. But as
we examine our democratic traditions in this fraught campaign cycle, we
should take a step back and consider whether, in this age of online
quizzes, audience-voted competition shows and
\href{https://www.reddit.com/r/IAmA/}{Reddit AMAs}, we could find better
ways to figure out which politicians we want to vote for.

In theory, debates are supposed to help us distinguish among
presidential candidates. There is some expectation that, forced to
respond immediately to questions and pushed into adversarial situations,
the candidates will reveal something more about themselves or their
policies than they do in prepared speeches, taped spots or tweets. In
this idealization, a debate should allow different viewpoints not only
to be heard but also to challenge one another.

Unfortunately, today's debates rarely accomplish that --- they are run
as spectacles, curated more for television ratings than for any benefit
to democracy. The debates stud our presidential elections like commas in
a run-on sentence, punctuating without full stops. Every month or so, we
are treated to a ramping up of excitement, breathless speculation and
predictions, and a selection of customized graphics that swoop onto
television screens. Then, during the debates, candidates are given
extremely short response times (75 seconds!) and questions designed to
elicit controversial responses and clickbait sound bites. The events
themselves are almost immediately overwritten by successive layers of
spin and punditry --- in real-time on Twitter, immediately post-debate
on television and online, the next morning in print and later in the
polls.

The paradox is that the value of debates comes from that very condition
of spectacle. In this era of the attention economy, an event, however
manufactured, is one of the few ways to ensure any kind of collective
focus. The scarcity of camera time to be divided among the candidates
increases the value of that time. It has also led to one of the more
persistent and baffling metrics to come out of these debates, as media
sites and pundits obsessively measure and rank the minutes each
candidate has spoken. (Is it a good thing to talk a lot, because you're
dominating the conversation? A bad thing, because you're dominating the
conversation? Or entirely the fault of the presumably biased
moderators?)

With many millions of dollars traded for the privileges of hosting the
debates and advertising during them, the point becomes more about making
the audience available to corporations than about making the candidates
available to the audience. The debates are, in part, advertisements for
the medium. The moderators are almost always journalists from the
channel or outlet hosting the debate, and not, for example,
constitutional lawyers, or presidential historians, or economists, or
tax policy experts, or foreign policy academics or climate scientists.
Why not have a debate moderated by a panel of governors and mayors, or
former congressional aides, or soldiers or data privacy activists?

It's true that we occasionally do get seemingly unscripted moments in
presidential debates. But those moments reveal, at best, candidates'
capacity to think on their feet, not in a moment of national crisis, but
on television under high-powered lights, a live audience and time
limits. It's not an entirely irrelevant skill for a chief executive, but
it's also not the only one we should be testing. What about the need to
read and absorb a great deal of information quickly and make decisions
about it? Or manage a staff of experts? Or communicate diplomatically
with foreign leaders? Imagine a series of debates in which each
candidate, surrounded by a handful of chosen staff members, competed to
prepare and persuade us of a policy proposal based on a surprise
scenario rolled out by the moderators, a kind of ``The Great British
Baking Show'' for politics. Or a briefing book challenge, ranking
candidates by their executive summaries after 15 minutes of on-camera
skimming.

There are so many possible solutions to the problems that plague today's
debates. They could be hosted and managed by organizations that are not
in the profit business (hello, C-Span). We could remove the video
component of debates, using still photos or nothing at all, to remove
the attention to clothes (especially those of female candidates),
hairstyles, shakes and sweats. We could make the debates boring,
allowing candidates to drone on and on. Alternatively, we could ensure
that all existing information about their platforms and proposals is
readily available online, by mail or to watch at an earlier time, and
allow only talks that add something new to the discussion.

Part of the problem is that we haven't really decided what we're looking
for in a president, a position that combines head of state with chief
executive. We are interested in certain kinds of personality or
character traits while disclaiming the importance of others. We claim to
want managerial competence, but evidence of that rarely makes it into
any part of a campaign. We might care about their policy choices, but
it's hard for most laypeople to gauge how successful a candidate is
likely to be at instituting those policies in our complicated government
structure. We want candidates who are polished and media-ready, but also
distrust them, worrying that they've been through too many focus groups
and consultants to show us their real selves.

Televised debates thrive in this gray area, playing off the celebrity
aura of candidates and sensationalizing superficial flaws while getting
virtue points for participating in the democratic process. They pretend
they are doing a civic service, while pivoting the process entirely to
their own interests --- something that could be said about many of the
actors in our political ecosystem. And they're staggeringly
unimaginative about how they do it. Our world is full of dazzling
potential for effectively communicating information; we should be
harnessing some of it for our elections.

Malka Older (\href{https://twitter.com/m_older}{@m\_older}) is an
affiliated research fellow with the Center for the Sociology of
Organizations at the Paris Institute of Political Studies. She is the
author of
\href{https://publishing.tor.com/infomocracy-malkaolder/9780765392367/}{``The
Centenal Cycle'' trilogy} and the short story collection
\href{http://www.masonjarpress.xyz/chapbooks-1/and-other-disasters}{``
\ldots{} and Other Disasters.''}

\emph{The Times is committed to publishing}
\href{https://www.nytimes.com/2019/01/31/opinion/letters/letters-to-editor-new-york-times-women.html}{\emph{a
diversity of letters}} \emph{to the editor. We'd like to hear what you
think about this or any of our articles. Here are some}
\href{https://help.nytimes.com/hc/en-us/articles/115014925288-How-to-submit-a-letter-to-the-editor}{\emph{tips}}\emph{.
And here's our email:}
\href{mailto:letters@nytimes.com}{\emph{letters@nytimes.com}}\emph{.}

\emph{Follow The New York Times Opinion section on}
\href{https://www.facebook.com/nytopinion}{\emph{Facebook}}\emph{,}
\href{http://twitter.com/NYTOpinion}{\emph{Twitter (@NYTopinion)}}
\emph{and}
\href{https://www.instagram.com/nytopinion/}{\emph{Instagram}}\emph{.}

Advertisement

\protect\hyperlink{after-bottom}{Continue reading the main story}

\hypertarget{site-index}{%
\subsection{Site Index}\label{site-index}}

\hypertarget{site-information-navigation}{%
\subsection{Site Information
Navigation}\label{site-information-navigation}}

\begin{itemize}
\tightlist
\item
  \href{https://help.nytimes.com/hc/en-us/articles/115014792127-Copyright-notice}{©~2020~The
  New York Times Company}
\end{itemize}

\begin{itemize}
\tightlist
\item
  \href{https://www.nytco.com/}{NYTCo}
\item
  \href{https://help.nytimes.com/hc/en-us/articles/115015385887-Contact-Us}{Contact
  Us}
\item
  \href{https://www.nytco.com/careers/}{Work with us}
\item
  \href{https://nytmediakit.com/}{Advertise}
\item
  \href{http://www.tbrandstudio.com/}{T Brand Studio}
\item
  \href{https://www.nytimes.com/privacy/cookie-policy\#how-do-i-manage-trackers}{Your
  Ad Choices}
\item
  \href{https://www.nytimes.com/privacy}{Privacy}
\item
  \href{https://help.nytimes.com/hc/en-us/articles/115014893428-Terms-of-service}{Terms
  of Service}
\item
  \href{https://help.nytimes.com/hc/en-us/articles/115014893968-Terms-of-sale}{Terms
  of Sale}
\item
  \href{https://spiderbites.nytimes.com}{Site Map}
\item
  \href{https://help.nytimes.com/hc/en-us}{Help}
\item
  \href{https://www.nytimes.com/subscription?campaignId=37WXW}{Subscriptions}
\end{itemize}
