\href{/section/style}{Style}\textbar{}Who Made Your Clothes?

\url{https://nyti.ms/34F28lY}

\begin{itemize}
\item
\item
\item
\item
\item
\end{itemize}

\includegraphics{https://static01.nyt.com/images/2019/12/19/fashion/19workers-Rumsinah/merlin_165538491_5437fcd5-6ffa-4eb3-b12a-66814c35a829-articleLarge.jpg?quality=75\&auto=webp\&disable=upscale}

Sections

\protect\hyperlink{site-content}{Skip to
content}\protect\hyperlink{site-index}{Skip to site index}

\hypertarget{who-made-your-clothes}{%
\section{Who Made Your Clothes?}\label{who-made-your-clothes}}

Garment workers around the world make everything from luxury handbags to
fast fashion leggings. Here are some of their stories.

Credit...Kemal Jufri for The New York Times

Supported by

\protect\hyperlink{after-sponsor}{Continue reading the main story}

By \href{https://www.nytimes.com/by/sanam-yar}{Sanam Yar}

\begin{itemize}
\item
  Published Dec. 19, 2019Updated Dec. 23, 2019
\item
  \begin{itemize}
  \item
  \item
  \item
  \item
  \item
  \end{itemize}
\end{itemize}

\hypertarget{its-a-good-factory-so-no-one-really-quits}{%
\subsection{`It's a good factory, so no one really
quits.'}\label{its-a-good-factory-so-no-one-really-quits}}

\hypertarget{rumsinah-44}{%
\subsubsection{Rumsinah, 44}\label{rumsinah-44}}

\textbf{Role:} Zipper operator at PT. Fajarindo Faliman Zipper, which
focuses largely on in-house brands

\textbf{Where:} Tangerang, Indonesia

``Most of my co-workers and I are all old-timers,'' said Ms. Rumsinah,
who has been working at the same factory for 26 years. ``It's a good
factory, so no one really quits. There's seldom any job openings ---
only if someone retires.''

She is paid about 3.4 million rupiah, or \$241, per month, which she
said is tight as a single parent. Her son recently finished high school.
``He can't work at my factory because there's no openings,'' she said.
``He wants to be a teacher, but we don't have enough money to send him
to go to university.''

Though her job is tiring, ``all jobs are tiring,'' she said. ``At least
weekends are off, and the hours are not too bad.''

\hypertarget{sometimes-needles-break-and-get-stuck-in-your-bone}{%
\subsection{`Sometimes needles break and get stuck in your
bone.'}\label{sometimes-needles-break-and-get-stuck-in-your-bone}}

\hypertarget{waheed-38}{%
\subsubsection{Waheed, 38}\label{waheed-38}}

\textbf{Role:} Sewing bedsheets and curtains at a textile mill

\textbf{Where:} Pakistan

Waheed, who is being identified only by his first name, has been in the
textile industry for 20 years and works seven days a week to support his
wife and two young sons. They share a house with his parents, his
sisters and his brothers.

``Most factories place a lot of restrictions on garment workers. Once
they come in for their shift around 8 in the morning, there's no knowing
when supervisors will let them out. It may be 8 p.m. or 10 p.m. by the
time they are allowed to leave for the day.

Workers at my factory don't have it as bad. That's why I've been here
for the past 10 years. It's a nice place to work. But some of the
resources that workers really need aren't provided, such as first-aid
kits or pension cards.

It's pretty common to get your fingers injured --- sometimes needles
break and get stuck in your bone if your hand gets in the way of the
machine. Then you have to go to the hospital and get X-rays yourself.

It's difficult to manage on the salary I earn. My expenses amount to
about 2,000 rupees a day, including the cost of my children's clothes,
their education, my family's groceries and other bills. But I barely
make 1,000 rupees a day.''

\includegraphics{https://static01.nyt.com/images/2019/12/19/fashion/19workers-seak/merlin_164596662_2dd900dc-b6e6-4868-ba5d-852df6ec0589-articleLarge.jpg?quality=75\&auto=webp\&disable=upscale}

\hypertarget{i-feel-tired-but-i-have-no-choice-i-have-to-work}{%
\subsection{`I feel tired, but I have no choice. I have to
work.'}\label{i-feel-tired-but-i-have-no-choice-i-have-to-work}}

\hypertarget{seak-hong-36}{%
\subsubsection{Seak Hong, 36}\label{seak-hong-36}}

\textbf{Role:} Sews outdoor apparel and bags at Horizon Outdoor

\textbf{Where:} Khum Longvek, Kampong Chhnang, Cambodia

Six days a week, Ms. Hong wakes up at 4:35 a.m. to catch the truck to
work from her village. Her workday begins at 7 and usually lasts nine
hours, with a lunch break. During the peak season, which lasts two to
three months, she works until 8:30 p.m.

Ms. Hong has been in the garment business for 22 years. She earns the
equivalent of about \$230 a month and supports her father, her sister,
her brother (who is on disability) and her 12-year-old son.

She hopes he will not end up in a factory, too, but the price of a
quality education --- about \$20 per month --- is beyond her means.
While she is at work, her sister manages the household, taking care of
their oxen and rice farming their land for extra food.

``I feel tired, but I have no choice,'' Ms. Hong said. ``I have to
work.''

Image

Credit...Nadège Mazars for The New York Times

\hypertarget{they-spoil-us-a-lot-here}{%
\subsection{`They spoil us a lot
here.'}\label{they-spoil-us-a-lot-here}}

\hypertarget{yurani-tascon-34}{%
\subsubsection{Yurani Tascon, 34}\label{yurani-tascon-34}}

\textbf{Role:} Tracks daily production numbers at Supertex, which works
with major active wear brands

\textbf{Where:} Yumbo, Colombia

``They spoil us a lot here,'' Ms. Tascon said. ``It's a job with good
stability.'' Her workplace blasts music --- usually salsa or something
traditional --- from speakers throughout the day while employees make
coats, bathing suits and sportswear.

At 11 a.m., employees get ``pausas activas'': active breaks with music.

Image

Credit...Kemal Jufri for The New York Times

\hypertarget{you-have-to-dare-to-dream-how-to-get-there-is-a-question-for-a-different-time}{%
\subsection{`You have to dare to dream, how to get there is a question
for a different
time.'}\label{you-have-to-dare-to-dream-how-to-get-there-is-a-question-for-a-different-time}}

\hypertarget{sarjimin-39}{%
\subsubsection{Sarjimin, 39}\label{sarjimin-39}}

\textbf{Role:} Makes shoes for a comfort footwear brand at PT. Dwi Naga
Sakti Abadi

\textbf{Where:} Tangerang, Indonesia

Mr. Sarjimin has worked at the same factory for about 12 years. The job
is relatively stable, and his workplace is spacious, bright and safe.

He earns the equivalent of \$250 a month, and his wife also works at a
factory. The family is able to send their children, a 13-year-old and a
9-year-old, to good schools. They recently purchased a computer for
their older son, who is passionate about technology.

Mr. Sarjimin farms catfish to supplement his family's grocery money. He
started six months ago, filling a big empty drum with starter fish as an
experiment. Now he has two drums with 300 fish each, and he sells them
to friends, family and neighbors.

One day, he would like to raise catfish full time. ``There's a
motivational speaker I heard once, `You have to dare to dream, how to
get there is a question for a different time,''' he said. ``I like
remembering those words.''

\hypertarget{we-live-overdrawn}{%
\subsection{`We live overdrawn.'}\label{we-live-overdrawn}}

\hypertarget{saida-38}{%
\subsubsection{Saida, 38}\label{saida-38}}

\textbf{Role:} Sewing machine operator at Pinehurst Manufacturing, which
works with major active wear brands

\textbf{Where:} San Pedro Sula, Honduras

The factory where Saida has worked for the last 12 years is one of the
few in the area. She earns about 8,200 lempira each month, roughly
\$331. ``It doesn't cover everything,'' she said. ``Vivimos
sobregirados.'' (``We live overdrawn.'')

Saida lives with her mother and her 19-year-old daughter, who goes to
school. ``I am the one who provides everything at home. The house, the
water, the electricity,'' she said. ``You have to stop buying certain
things to be able to cover the necessities.''

Her unit currently has one primary client, a major sportswear brand.
This is a source of anxiety for her and her co-workers because they fear
mass layoffs if the client leaves the company. ``It's really difficult
having one client,'' she said.

Image

Credit...Minzayar Oo for The New York Times

\hypertarget{i-can-finish-1000-to-1200-pieces-a-day-depending-on-the-difficulty}{%
\subsection{`I can finish 1,000 to 1,200 pieces a day, depending on the
difficulty.'}\label{i-can-finish-1000-to-1200-pieces-a-day-depending-on-the-difficulty}}

\hypertarget{bui-chi-thang-35}{%
\subsubsection{Bui Chi Thang, 35}\label{bui-chi-thang-35}}

\textbf{Role:} Stitching denim together for sustainability-focused
brands at Saitex International

\textbf{Where:} Bien Hoa, Vietnam

Mr. Bui has been at his factory for seven years. ``It matches my
skill,'' he said, ``and the salary is enough for my family.'' He earns
nine to 10 million dong a month (roughly \$388 to \$432), which he uses
to support his mother, wife and son.

During the average nine-hour workday, ``I can finish 1,000 to 1,200
pieces a day, depending on the difficulty,'' he said.

Image

Credit...Rozette Rago for The New York Times

\hypertarget{im-always-trying-to-figure-out-how-to-save-money-how-to-buy-food-how-to-not-eat-out-too-much}{%
\subsection{`I'm always trying to figure out how to save money, how to
buy food, how to not eat out too
much.'}\label{im-always-trying-to-figure-out-how-to-save-money-how-to-buy-food-how-to-not-eat-out-too-much}}

\hypertarget{santiago-48}{%
\subsubsection{Santiago, 48}\label{santiago-48}}

\textbf{Role:} Sews clasps and zippers onto dresses, blouses and pants
at a factory

\textbf{Where:} Los Angeles

``I'm from Guatemala. I've been doing garment work for 16 years. I
started because it was the only thing I knew how to do after leaving my
home country,'' Santiago said. ``I came here because there were not as
many opportunities back home, and with six children, there are a lot of
expenses.''

In the last five years, he has worked in five to eight factories. They
are often windowless and dirty, with little ventilation, he said.

When he first moved to Los Angeles, Santiago was working 11-hour shifts,
seven days a week. Now he works about 50 hours a week, taking home up to
\$350. The majority of his co-workers --- around 30 other people --- are
Spanish speakers from Guatemala, El Salvador and Mexico.

``I'm just making ends meet,'' he said. ``I'm always trying to figure
out how to save money, how to buy food, how to not eat out too much.''
Still, he said it is better than what he was earning in Guatemala.

Image

Credit...Maria Magdalena Arrellaga for The New York Times

\hypertarget{you-basically-have-to-kill-yourself-in-front-of-a-sewing-machine-in-order-to-provide-for-your-family}{%
\subsection{`You basically have to kill yourself in front of a sewing
machine in order to provide for your
family.'}\label{you-basically-have-to-kill-yourself-in-front-of-a-sewing-machine-in-order-to-provide-for-your-family}}

\hypertarget{maria-valdinete-da-silva-46}{%
\subsubsection{Maria Valdinete da Silva,
46}\label{maria-valdinete-da-silva-46}}

\textbf{Role:} Self-employed seamstress

\textbf{Where:} Caruaru, Brazil

The last factory Ms. da Silva worked at produced men's street wear. She
spent eight years there, stitching side seams together in an assembly
line with an hourly quota.

``Some companies, like the one I worked for, no longer have employees
inside the factory and the seamstresses work from home,'' she said.
``They establish small groups, tiny factories, and they are paid per
item, so they basically have the same production without any costs.''

In order to make minimum wage, outsourced employees ``have to work from
day to night,'' she said.

Ms. da Silva now makes women's clothing independently, producing fewer
pieces and selling them locally. She makes ``maybe half'' of minimum
wage, but she said it's worth it to work at her own pace. ``I love what
I do,'' she said. ``I no longer see myself in that situation of sitting
in front of a machine doing the same thing every day.''

She is planning on taking fashion design courses soon. ``Seamstresses
are the key element in the fashion chain, we are the ones who put the
clothes together,'' she said. ``You basically have to kill yourself in
front of a sewing machine in order to provide for your family.''

Image

Credit...Susan Wright for The New York Times

\hypertarget{its-not-so-much-the-salary-its-that-i-am-here-because-were-all-one-family}{%
\subsection{`It's not so much the salary, it's that I am here because
we're all one
family.'}\label{its-not-so-much-the-salary-its-that-i-am-here-because-were-all-one-family}}

\hypertarget{antonio-ripani-72}{%
\subsubsection{Antonio Ripani, 72}\label{antonio-ripani-72}}

\textbf{Role:} Leather quality control at Tod's Group

\textbf{Where:} Casette d'Ete, ****** Italy

Mr. Ripani, who began working with leather at 14, has been employed by
Tod's for more than 40 years, where he assesses ``practically all the
hides that arrive'' for quality.

``Alone it's hard to do everything, so I have a group of ragazzi
{[}guys{]} under me and I have taught them everything I've been able to
understand after all these years,'' he said.

Mr. Ripani doesn't earn much, he said, but he sets his own schedule,
often working eight to 12 hours a day. He has assistants and has
received
\href{https://www.corriereadriatico.it/fermo/fermo_cesetti_provincia_tod_39_s-469005.html}{awards}
for his highly specialized work.

``It's not so much the salary, it's that I am here because we're all one
family,'' he said. ``When I started, I had long hair. Now, I am bald.''

Image

Credit...Saiyna Bashir for The New York Times

\hypertarget{are-we-supposed-to-choose-between-buying-food-and-roti-or-paying-for-clothes-and-medicine}{%
\subsection{`Are we supposed to choose between buying food and roti or
paying for clothes and
medicine?'}\label{are-we-supposed-to-choose-between-buying-food-and-roti-or-paying-for-clothes-and-medicine}}

\hypertarget{rukhsana-48}{%
\subsubsection{Rukhsana, 48}\label{rukhsana-48}}

\textbf{Role:} Security at Sitara Textile Industries

\textbf{Where:} Faisalabad, Pakistan

Rukhsana began working in the garment industry shortly after her husband
died seven years ago. She works seven days a week.

``The hardest thing about working in a textile mill is that management
kind of cuts you off from the world for the duration of your shift. If
anyone calls you from home --- with good news or bad news --- you can't
take the call and management doesn't tell you until the day is over.

Two years ago, my nephew died in an accident when I was working. My
brother tried calling me, but management didn't tell me about it until
my family had already held his funeral. I was so upset, I quit my job.

Now that I'm in security, I know when someone comes to the mill and
tries to contact a worker. But I'm still not allowed to tell the worker
their relative has been trying to reach them.

It's not just difficult, it's impossible to survive on the salary the
textile mills pay. Are we supposed to choose between buying food and
roti or paying for clothes and medicine? And there's always rent to pay
in addition to that.''

(Employees store their phones in a locker before beginning their shift,
a company spokesman said in a phone interview, and they aren't allowed
to leave the organization ``without any written acknowledgment from the
manager.''

He said that family can reach employees on their cellphones or by
calling the factory directly, and that he was not aware of any incidents
in which family was prevented or delayed from contacting an employee
during an emergency. )

Image

Credit...Linh Pham for The New York Times

\hypertarget{my-favorite-time-is-at-3-pm-when-we-have-an-exercise-session}{%
\subsection{`My favorite time is at 3 p.m., when we have an exercise
session.'}\label{my-favorite-time-is-at-3-pm-when-we-have-an-exercise-session}}

\hypertarget{vu-hoang-quan-21}{%
\subsubsection{Vu Hoang Quan, 21}\label{vu-hoang-quan-21}}

\textbf{Role:} Sews dress shirts for mass retailers at TAL Apparel

\textbf{Where:} Binh Xuyen, Vinh Phuc, Vietnam

Mr. Vu has spent the last four years working on a production line with
about 30 other employees, each overseeing parts of the sewing process.
On average, he earns about 10 to 12 million dong (about \$432 to \$518)
monthly. He sends most of it back to his family.

``My favorite time is at 3 p.m., when we have an exercise session,'' he
said. ``We stay at our work spot. We pause our work process, line up and
follow the exercise instructions of team leaders.''

He recently participated in a talent show hosted by the company, where
he performed modern dance. ``I don't have plans to leave this job
anytime soon,'' he said. ``I'm quite satisfied with it.''

Image

Credit...Julien Mignot for The New York Times

\hypertarget{it-is-my-passion}{%
\subsection{`It is my passion.'}\label{it-is-my-passion}}

\hypertarget{catherine-gamet-48}{%
\subsubsection{Catherine Gamet, 48}\label{catherine-gamet-48}}

\textbf{Role:} Leather goods artisan at Louis Vuitton

\textbf{Where:} Saint-Pourçain-sur-Sioule, France

Ms. Gamet began working with leather when she was 16 years old and has
been employed by Vuitton for 23 years. ``To be able to build bags and
all, and to be able to sew behind the machine, to do hand-sewn products,
it is my passion,'' she said. ``That's how I got into it.''

About 800 employees work in Saint-Pourçain, spread out across four
sites. Ms. Gamet said the workshops are well organized, bright and
modern. ``The time flies by,'' she said.

Image

Credit...Rebecca Conway for The New York Times

\hypertarget{we-dont-even-have-the-freedom-to-drink-water}{%
\subsection{`We don't even have the freedom to drink
water.'}\label{we-dont-even-have-the-freedom-to-drink-water}}

\hypertarget{s-33}{%
\subsubsection{S, 33}\label{s-33}}

\textbf{Role:} Tailor making pants and socks for fast fashion and active
wear brands at Shahi Exports

\textbf{Where:} India

S.'s shift begins at 9 a.m. She feels a lot of pressure from supervisors
to reach quotas of about 90 to 120 pieces per hour and said many workers
are afraid to take breaks or use the restroom because it will waste
time.

Employees who can't keep up are often pulled aside at the end of each
hour, she said, and supervisors will yell at them and bang on tables.
Many workers spend most of their 30-minute lunch breaks scrambling to
finish more pieces to get back on track.

``We don't even have the freedom to drink water,'' S. said, adding that
management doesn't allow employees to bring in water bottles.

Instead, water is handed out by the factory. In the spring of 2018, the
supplied water was making workers sick, and when employees gave
management a letter with a variety of basic requests, including clean
water, they were beaten in response. Their clothes were torn, and many
of their valuables, including phones and jewelry, were taken.

The employees took their complaint to the labor department. The issues
were resolved three months after the incident, after the factory faced
public pressure from a report by an American watchdog group, social
media and brands that worked with the factory.

Some conditions have improved: Employees get mineral water now. But the
pay is still bad, S. said, and the main work space doesn't have windows,
air-conditioning or heaters.

``We want to ask for more salary, but people are scared after what
happened last year to ask again,'' she said.

(In an email, a spokesman from Shahi Exports acknowledged the 2018
incident and forwarded a statement outlining
\href{https://www.shahi.co.in/blog/?p=697}{the preventive measures} the
company has since enacted.

In a separate email, a spokesman said that berating employees in any way
``constitutes misconduct,'' and instances brought to management's
attention would ``initiate action'' against the perpetrator.

``While we do strive to drive efficiencies, there is no scope to berate
any employee on account of non-performance or deficient performance,''
he said. The spokesman added that there ``is adequate ventilation''
within the work space and that the entire factory is ``in compliance
with the law.'')

S. is a single parent and picks up extra work in the evenings, along
with taking out loans, to support herself and her daughter. ``There are
thousands of people'' in her city in the same situation, she said. ``My
story is just one of them.''

Image

Credit...Saumya Khandelwal for The New York Times

\hypertarget{there-are-some-plants-and-trees-also-you-know-the-kind-that-are-meant-for-decoration}{%
\subsection{`There are some plants and trees also, you know, the kind
that are meant for
decoration.'}\label{there-are-some-plants-and-trees-also-you-know-the-kind-that-are-meant-for-decoration}}

\hypertarget{phool-bano-38}{%
\subsubsection{Phool Bano, 38}\label{phool-bano-38}}

\textbf{Role:} Tailor at Friends Factory

\textbf{Where:} Noida, India

Ms. Bano has been a tailor for about 22 years and works at a progressive
factory that makes small batches of garments for high-end independent
brands. The building has little luxuries like air purifiers.

``It feels nice working here,'' Ms. Bano said. ``It's clean. There are
some plants and trees also, you know, the kind that are meant for
decoration.''

Image

Credit...Maria Magdalena Arrellaga for The New York Times

\hypertarget{my-dream-is-to-have-my-own-atelier-at-home}{%
\subsection{`My dream is to have my own atelier at
home.'}\label{my-dream-is-to-have-my-own-atelier-at-home}}

\hypertarget{helena-luxfacia-santos-da-conceiuxe7uxe3o-da-silva-54}{%
\subsubsection{Helena Lúcia Santos da Conceição da Silva,
54}\label{helena-luxfacia-santos-da-conceiuxe7uxe3o-da-silva-54}}

\textbf{Role:} Seamstress at Fantasia D!kas Roupas

\textbf{Where:} Nova Friburgo, Brazil

``I've always thought of myself as a seamstress. I even made my
daughter's sweet-16 dress. It looks like overlapping petals. It's my
greatest pride.

I start work at 7 a.m. We make everything: pants, shorts, tops. I work
eight hours a day Mondays to Fridays with a one-hour lunch break. It's a
small company: me and five other seamstresses. We don't have a quota.
Here they value quality over quantity. I don't even know how many pieces
I work on in a given day. We don't keep track.''

Ms. da Silva does not make enough money from her day job, so she picks
up extra work from private clients to complete on evenings and weekends,
sometimes working until 10 p.m.

``I prefer working for this manufacturer because I'm on the payroll, I'm
entitled to vacations. It's more secure. But my dream is to have my own
atelier at home.''

\begin{center}\rule{0.5\linewidth}{\linethickness}\end{center}

Knvul Sheikh contributed reporting.

Advertisement

\protect\hyperlink{after-bottom}{Continue reading the main story}

\hypertarget{site-index}{%
\subsection{Site Index}\label{site-index}}

\hypertarget{site-information-navigation}{%
\subsection{Site Information
Navigation}\label{site-information-navigation}}

\begin{itemize}
\tightlist
\item
  \href{https://help.nytimes.com/hc/en-us/articles/115014792127-Copyright-notice}{©~2020~The
  New York Times Company}
\end{itemize}

\begin{itemize}
\tightlist
\item
  \href{https://www.nytco.com/}{NYTCo}
\item
  \href{https://help.nytimes.com/hc/en-us/articles/115015385887-Contact-Us}{Contact
  Us}
\item
  \href{https://www.nytco.com/careers/}{Work with us}
\item
  \href{https://nytmediakit.com/}{Advertise}
\item
  \href{http://www.tbrandstudio.com/}{T Brand Studio}
\item
  \href{https://www.nytimes.com/privacy/cookie-policy\#how-do-i-manage-trackers}{Your
  Ad Choices}
\item
  \href{https://www.nytimes.com/privacy}{Privacy}
\item
  \href{https://help.nytimes.com/hc/en-us/articles/115014893428-Terms-of-service}{Terms
  of Service}
\item
  \href{https://help.nytimes.com/hc/en-us/articles/115014893968-Terms-of-sale}{Terms
  of Sale}
\item
  \href{https://spiderbites.nytimes.com}{Site Map}
\item
  \href{https://help.nytimes.com/hc/en-us}{Help}
\item
  \href{https://www.nytimes.com/subscription?campaignId=37WXW}{Subscriptions}
\end{itemize}
