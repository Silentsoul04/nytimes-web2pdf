Sections

SEARCH

\protect\hyperlink{site-content}{Skip to
content}\protect\hyperlink{site-index}{Skip to site index}

\href{https://www.nytimes.com/section/technology}{Technology}

\href{https://myaccount.nytimes.com/auth/login?response_type=cookie\&client_id=vi}{}

\href{https://www.nytimes.com/section/todayspaper}{Today's Paper}

\href{/section/technology}{Technology}\textbar{}Many Facial-Recognition
Systems Are Biased, Says U.S. Study

\url{https://nyti.ms/36SVOZE}

\begin{itemize}
\item
\item
\item
\item
\item
\end{itemize}

Advertisement

\protect\hyperlink{after-top}{Continue reading the main story}

Supported by

\protect\hyperlink{after-sponsor}{Continue reading the main story}

\hypertarget{many-facial-recognition-systems-are-biased-says-us-study}{%
\section{Many Facial-Recognition Systems Are Biased, Says U.S.
Study}\label{many-facial-recognition-systems-are-biased-says-us-study}}

Algorithms falsely identified African-American and Asian faces 10 to 100
times more than Caucasian faces, researchers for the National Institute
of Standards and Technology found.

\includegraphics{https://static01.nyt.com/images/2019/12/19/business/19facial/19facial-articleLarge.jpg?quality=75\&auto=webp\&disable=upscale}

By \href{https://www.nytimes.com/by/natasha-singer}{Natasha Singer} and
\href{https://www.nytimes.com/by/cade-metz}{Cade Metz}

\begin{itemize}
\item
  Dec. 19, 2019
\item
  \begin{itemize}
  \item
  \item
  \item
  \item
  \item
  \end{itemize}
\end{itemize}

The majority of commercial facial-recognition systems exhibit bias,
according to a study from a federal agency released on Thursday,
underscoring questions about a technology increasingly used by police
departments and federal agencies to identify suspected criminals.

The systems
\href{https://www.nytimes.com/2019/07/08/us/detroit-facial-recognition-cameras.html}{falsely
identified African-American} and Asian faces 10 times to 100 times more
than Caucasian faces, the National Institute of Standards and Technology
reported on Thursday. Among a database of photos used by law enforcement
agencies in the United States, the highest error rates came in
identifying Native Americans, the study found.

The technology also had more difficulty identifying women than men. And
it falsely identified older adults up to 10 times more than middle-aged
adults.

The new report comes at a time of mounting concern from lawmakers and
civil rights groups over the proliferation of facial recognition.
Proponents view it as an important tool for catching criminals and
tracking terrorists. Tech companies market it as a convenience that can
be used to help identify people in photos or in lieu of a password to
unlock smartphones.

Civil liberties experts, however, warn that the technology --- which can
be used to track people at a distance without their knowledge --- has
the potential to lead to ubiquitous surveillance, chilling freedom of
movement and speech. This year, San Francisco, Oakland and Berkeley in
California and the Massachusetts communities Somerville and Brookline
banned government use of the technology.

``One false match can lead to missed flights, lengthy interrogations,
watch list placements, tense police encounters, false arrests or
worse,'' Jay Stanley, a policy analyst at the American Civil Liberties
Union, said in a statement. ``Government agencies including the F.B.I.,
Customs and Border Protection and local law enforcement must immediately
halt the deployment of this dystopian technology.''

The federal report is one of the largest studies of its kind. The
researchers had access to more than 18 million photos of about 8.5
million people from United States mug shots, visa applications and
border-crossing databases.

The National Institute of Standards and Technology tested 189
\href{https://www.nytimes.com/2019/01/24/technology/amazon-facial-technology-study.html}{facial-recognition
algorithms} from 99 developers, representing the majority of commercial
developers. They included systems from Microsoft, biometric technology
companies like Cognitec, and Megvii, an artificial intelligence company
in China.

The agency did not test systems from Amazon, Apple, Facebook and Google
because they did not submit their algorithms for the federal study.

The federal report confirms earlier studies from M.I.T. that reported
that
\href{https://www.nytimes.com/2019/05/15/business/facial-recognition-software-controversy.html}{facial-recognition
systems} from some large tech companies had much lower accuracy rates in
identifying the female and darker-skinned faces
\href{https://www.nytimes.com/2018/02/09/technology/facial-recognition-race-artificial-intelligence.html}{than
the white male faces}.

``While some biometric researchers and vendors have attempted to claim
algorithmic bias is not an issue or has been overcome, this study
provides a comprehensive rebuttal,'' Joy Buolamwini, a researcher at the
M.I.T. Media Lab who led one of the facial studies, said in an email.
``We must safeguard the public interest and halt the proliferation of
face surveillance.''

Although the use of facial recognition by law enforcement is not new,
new uses are proliferating with little independent oversight or public
scrutiny. China has used the technology to surveil and control ethnic
minority groups like the Uighurs. This year, United States Immigration
and Customs Enforcement officials came under fire for using the
technology to analyze the drivers' licenses of millions of people
without their knowledge.

Biased facial recognition technology is particularly problematic in law
enforcement because errors could lead to false accusations and arrests.
The new federal study found that the kind of facial matching algorithms
used in law enforcement had the highest error rates for African-American
females.

``The consequences could be significant,'' said Patrick Grother, a
computer scientist at N.I.S.T. who was the primary author of the new
report. He said he hoped it would spur people who develop facial
recognition algorithms to ``look at the problems they may have and how
they might fix it.''

But ensuring that these systems are fair is only part of the task, said
Maria De-Arteaga, a researcher at Carnegie Mellon University who
specializes in algorithmic systems. As facial recognition becomes more
powerful, she said, companies and governments must be careful about
when, where, and how they are deployed.

``We have to think about whether we really want these technologies in
our society,'' she said.

Advertisement

\protect\hyperlink{after-bottom}{Continue reading the main story}

\hypertarget{site-index}{%
\subsection{Site Index}\label{site-index}}

\hypertarget{site-information-navigation}{%
\subsection{Site Information
Navigation}\label{site-information-navigation}}

\begin{itemize}
\tightlist
\item
  \href{https://help.nytimes.com/hc/en-us/articles/115014792127-Copyright-notice}{©~2020~The
  New York Times Company}
\end{itemize}

\begin{itemize}
\tightlist
\item
  \href{https://www.nytco.com/}{NYTCo}
\item
  \href{https://help.nytimes.com/hc/en-us/articles/115015385887-Contact-Us}{Contact
  Us}
\item
  \href{https://www.nytco.com/careers/}{Work with us}
\item
  \href{https://nytmediakit.com/}{Advertise}
\item
  \href{http://www.tbrandstudio.com/}{T Brand Studio}
\item
  \href{https://www.nytimes.com/privacy/cookie-policy\#how-do-i-manage-trackers}{Your
  Ad Choices}
\item
  \href{https://www.nytimes.com/privacy}{Privacy}
\item
  \href{https://help.nytimes.com/hc/en-us/articles/115014893428-Terms-of-service}{Terms
  of Service}
\item
  \href{https://help.nytimes.com/hc/en-us/articles/115014893968-Terms-of-sale}{Terms
  of Sale}
\item
  \href{https://spiderbites.nytimes.com}{Site Map}
\item
  \href{https://help.nytimes.com/hc/en-us}{Help}
\item
  \href{https://www.nytimes.com/subscription?campaignId=37WXW}{Subscriptions}
\end{itemize}
