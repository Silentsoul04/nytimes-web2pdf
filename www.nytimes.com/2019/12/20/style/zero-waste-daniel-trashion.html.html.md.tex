\href{/section/style}{Style}\textbar{}The Future Is Trashion

\url{https://nyti.ms/35HeDij}

\begin{itemize}
\item
\item
\item
\item
\item
\end{itemize}

\includegraphics{https://static01.nyt.com/images/2019/12/20/fashion/20Ragpicker-zero-waste-daniel/20Ragpicker-zero-waste-daniel-articleLarge-v3.jpg?quality=75\&auto=webp\&disable=upscale}

Sections

\protect\hyperlink{site-content}{Skip to
content}\protect\hyperlink{site-index}{Skip to site index}

\hypertarget{the-future-is-trashion}{%
\section{The Future Is Trashion}\label{the-future-is-trashion}}

We make too much and buy too much. But maybe there is a way not to waste
too much. The ragpicker of Brooklyn has an idea.

Daniel Silverstein of Zero Waste Daniel.Credit...Vincent Tullo for The
New York Times

Supported by

\protect\hyperlink{after-sponsor}{Continue reading the main story}

\href{https://www.nytimes.com/by/vanessa-friedman}{\includegraphics{https://static01.nyt.com/images/2018/06/12/multimedia/vanessa-friedman/vanessa-friedman-thumbLarge.png}}

By \href{https://www.nytimes.com/by/vanessa-friedman}{Vanessa Friedman}

\begin{itemize}
\item
  Published Dec. 20, 2019Updated Dec. 26, 2019
\item
  \begin{itemize}
  \item
  \item
  \item
  \item
  \item
  \end{itemize}
\end{itemize}

The ragpicker of Brooklyn works out of a 750-square-foot storefront a
few blocks east of the Brooklyn-Queens Expressway, down a mostly
residential side street in Williamsburg, where Hasidim and hipsters mix.

The ragpicker of Brooklyn sews in the back, behind a makeshift wall
sprouting a riot of scraps. Under the pattern-cutting table there are
bins of scraps of scraps, sorted by color (red and yellow and blue and
black), and on one wall are shelves of Mason jars containing
gumball-size scraps of scraps of scraps; up front are clothing rails,
and a dressing room canopied by a lavish waterfall of castoff cuttings
that flows down onto the floor like a Gaudí sand castle.

The ragpicker of Brooklyn, whose name is Daniel Silverstein and whose
nom de style is Zero Waste Daniel, looks like a fashion kid, which he is
(or was). He is 30 and tends to dress all in black, with a black knit
cap on his head, and went to the Fashion Institute of Technology and
interned at Carolina Herrera and even was on a fashion reality TV show.

And the ragpicker of Brooklyn would rather not be called that at all.

``I prefer to think of it as Rumpelstiltskin, spinning straw into
gold,'' Mr. Silverstein said one day in early November. He was on West
35th Street, in the garment district, with his partner and husband,
Mario DeMarco (also all in black). They were hauling home sacks of
cuttings from their own production run at HD Fashion, which also makes
clothes for Rag \& Bone and Donna Karan's Urban Zen line.

Mr. Silverstein's straw is more formally known as preconsumer,
postproduction waste, which is a fancy way of saying he works with the
fabrics that other designers and costume departments and factories would
normally throw out.

\includegraphics{https://static01.nyt.com/images/2019/12/24/fashion/24insider1/merlin_163871316_66c5dcf5-5593-475a-a0e6-8a481621491a-articleLarge.jpg?quality=75\&auto=webp\&disable=upscale}

His gold is street wear: sweatshirts and pants and T-shirts, the
occasional anorak, collaged together from rolls of old fabric, mostly
black and gray, often containing brightly colored geometric patchwork
inserts of smaller, brighter bits, like an exclamation point, or an
Easter egg.

Those patchwork inserts have been put together from the castoffs of the
bigger pieces, and then the castoffs from the inserts are saved and
pieced together into mosaic appliqués (the hands from the Sistine Chapel
and Earth as seen from above, for example). The appliqués can be
custom-made and attached to any piece. Leftovers, all the way.

As fashion comes to grips with its own culpability in the climate
crisis, the concept of upcycling, whether remaking old clothes or
re-engineering used fabric or simply using what would otherwise be
tossed into landfill, has begun to trickle out to many layers of the
fashion world.

That includes the high end, via the work of designers like
\href{https://www.thecut.com/2019/11/marine-serre-upcycling-videos.html}{Marine
Serre},
\href{https://www.nytimes.com/2019/11/18/style/bode-emily-bode-shopping-brick-and-mortar-is-dead-lets-open-a-store.html}{Emily
Bode} and
\href{https://www.nytimes.com/2019/09/11/style/proenza-schouler-gabriela-hearst-new-york-fashion-week.html}{Gabriela
Hearst}, and brands like Hermès, as well as the outdoor space, with the
Patagonia
\href{https://mountainculturegroup.com/patagonia-worn-wear-program-review/}{WornWear}
and \href{https://wornwear.patagonia.com/shop/recrafted}{Recrafted}
programs (to name a few).

And yet, because there are few economies of scale and even fewer
production systems, such clothing remains for many designers an
experiment rather than a strategy, and for many consumers, a luxury
rather than a choice.

Mr. Silverstein, whose clothes range from \$25 for a patch to \$595 for
an anorak made from what was a New York City Sanitation Department tent,
and who works only with fabric that would otherwise be thrown away, is
one of several new designers trying to change that.

How he got there, with lots of false starts and belly flops, is perhaps
as representative as anything of the way fashion may be stumbling toward
its future. We make too much, and we buy too much, but that doesn't have
to mean we waste too much.

Welcome to the growing world of trashion.

Image

Jars of tiny clothing scraps organized by color on a shelf in the back
of the Zero Waste Daniel store.Credit...Vincent Tullo for The New York
Times

\hypertarget{saved-by-the-dumpster}{%
\subsection{Saved By the Dumpster}\label{saved-by-the-dumpster}}

``I came to New York for that fashion dream --- what I'd been watching
on TV,'' Mr. Silverstein said a few weeks before his garment district
scrap-saving trip. ``I wanted that life so badly.''

He was sitting in the back of what he calls his ``make/shop,'' which he
and Mr. DeMarco renovated in 2017 using materials from Big Reuse, a
Brooklyn nonprofit. The make/shop has three sewing machines but no
garbage can.

Mr. Silverstein was born in Pennsylvania, and when he was 10, his
parents moved to New Jersey so their fashion-aware son could be closer
to New York. Mr. Silverstein's father owned a swimming pool and hot tub
supply company, and his mother worked part-time in the business. (She is
also a therapist.) As a family, they did some recycling but were not
particularly attuned to the environment.

Mr. Silverstein always knew he wanted to be a designer. When he was 4,
he started making clothes for his sister's Barbies out of tissue paper
and tinfoil. By the time he was 14, he was taking weekend classes at
F.I.T. and making his friends' prom dresses.

His Damascene moment was more like a series of cold-water splashes. For
a senior-year competition for the Clinton Global Initiative, he designed
a pair of sustainable jeans, which became his first zero-waste pattern.
He didn't win, but his teacher told him to hold on to the idea. ```You
have something there,''' he recalled the teacher saying.

After graduating, he found himself working as a temp at Victoria's
Secret, making knitwear. He would scroll through
\href{http://style.com/}{style.com} looking at recent runway shows, find
a sweater he liked, then create a technical design packet for a similar
style for VS.

One of the patterns involved an asymmetric cut with a long triangular
piece in front. Because of the irregular shape, the fabric ``had an
insanely poor yield,'' Mr. Silverstein said, meaning that only a portion
of every yard was used for the garment; almost half was waste.

He did the math and realized, he said, ``that if this is yielding only
47 percent per each sweater, and we are cutting 10,000 sweaters, then we
are knitting, milling, dying and finishing 5,000 yards of fabric just to
throw out.''

The next day, he said, he left VS to focus on a business he and a friend
had started based on his zero-waste patterns. They were making classic
ready-to-wear --- cocktail dresses and suits and such --- but with no
waste left on the cutting-room floor. One of their first customers was
Jennifer Hudson, who wore a turquoise dress that ended up in the pages
of Us Weekly.

Stores like Fred Segal in Los Angeles and e-tail sites like Master \&
Muse picked up the line, which was called 100\% (for the amount of
fabric used), and Mr. Silverstein spent a season on the
\href{https://www.nytimes.com/2012/03/13/arts/television/fashion-star-designer-competition-series-on-nbc.html}{``Fashion
Star},'' ending his tenure as second runner-up.

Image

A coat made out of a former DSNY promotional tent hanging in the Zero
Waster Daniel Brooklyn store.Credit...Vincent Tullo for The New York
Times

Still, the economics of fashion, in which stores pay after delivery,
were working against him. In 2015, after American Apparel, which had
bought Oak NYC, a store known for its edgy choices and one of his
wholesale accounts, declared bankruptcy, he was left with \$30,000 worth
of unpaid orders. He decided to quit.

Mr. Silverstein got a part-time job helping students get their art
portfolios together and, he said, ``lay on the couch for a while.''
Finally he boxed up his studio and threw all of his leftover fabric in a
garbage bag. He was set to haul it to a dumpster, only to have the bag
break, spilling its contents onto the floor.

``I thought, `I can't throw this out --- it's the antithesis of my
mission,''' he said. ``So I took the afternoon and made myself a shirt
and put it on my Instagram. I had maybe 2,000 followers, and probably
the most likes I had ever gotten was 95. I posted this dumb selfie of a
shirt I'd made out of my own trash because I was too poor to go
shopping, and it instantly got 200 likes. It was the most popular thing
I'd ever done.''

It occurred to him this may be a better way to go. He made ``a bunch of
scrappy shirts'' and became Zero Waste Daniel, his Instagram name (which
he had chosen because Daniel Silverstein was already taken).

He rented a booth at a flea market and sold them all. Johnny Wujek, Katy
Perry's stylist, bought one. Chris Anderson, a mentor who ran Dress for
Success in Morris County, N.J., where Mr. Silverstein had interned
during high school, said she would back him.

His father put in some money, too, as did Tuomo Tiisala, a professor at
New York University who saw his work at a market. Mr. Silverstein got a
small space at Manufacture New York, a group incubator in Sunset Park
(it disbanded after a year), and made a deal with a factory that
supplied the Marshalls chain to pick up its scraps.

Fabric dumping, though less discussed than the clothes consumers throw
out, is just as much a byproduct of fashion production, and just as
culpable in the landfill crisis. Reverse Resources, a group that has
created an online marketplace to connect factories and designers who
want to reuse their scraps, released
\href{https://reverseresources.net/news/how-much-does-garment-industry-actually-waste}{a
study in 2016}that estimated that the garment industry creates almost
enough leftover textile per year to cover the entire republic of Estonia
with waste.

That was a best-case scenario. Worst case would be enough to cover North
Korea.

At that stage, Mr. Silverstein was mostly making sweatshirts, piecing
them together by hand, but, he said, ``people started making little
videos about my work and putting up posts, and I started getting more
orders than I could keep up with.''

In 2017, he met Mr. DeMarco, who worked in hospitality. This year he
joined the business full-time.

In many ways, social media has also been their door to a customer base.
Just as it creates pressure to buy new stuff, it can create pressure to
buy new old stuff.

Image

Mr. Silverstein sewing one of the DSNY shirts in his make/shop. The
front is the retail space; the back is his atelier.Credit...Vincent
Tullo for The New York Times

\hypertarget{message-vs-money}{%
\subsection{Message vs. Money}\label{message-vs-money}}

``My freshman year at F.I.T., one of my teachers said there are good
designers and there are great designers,'' Mr. Silverstein said. ``Good
designers have careers and see their stuff in stores, and great
designers change the way people dress.'' And, perhaps, think about
dress.''

He was driving a small U-Haul truck. He had spent the morning with Mr.
DeMarco in FabScrap, a concrete loft in the erstwhile Army Terminal
complex in Sunset Park filled with trash bags and storage boxes bulging
at the seams with fabric waste. They were on the hunt for 400 or so
yards of random black remnants with some stretch.

Mr. Silverstein doesn't ragpick in the 19th-century way (the way that
gave birth to the term), sifting through garbage on the streets. He
picks through giant boxes and metal shelves of castoff fabric rolls and
then sews his finds together to make new rolls.

He doesn't really have seasons or shows by a traditional definition,
though he flirts with the idea. In 2018, the 1 Hotel Brooklyn Bridge
invited him to do a show for New York Fashion Week, and instead of a
runway, he decided to do a one-man stand-up routine called ``Sustainable
Fashion Is Hilarious,'' which was more about concept than clothes.

The hotel sold tickets online, and all of the proceeds went to Fashion
Revolution, a nonprofit that advocates industry reform. In September, he
did the same at the
\href{https://www.papermag.com/zero-waste-daniel-2640345826.html}{Ace
hotel} in Manhattan.

Mr. Silverstein is planning a performance for February at Arcadia Earth,
the climate installation museum in downtown New York, which also sells
some of his work.

Last year the Sanitation Department came calling. It had done a
collaboration with the
\href{https://www.nytimes.com/2016/09/10/fashion/new-york-fashion-week-heron-preston.html}{designer
Heron Preston} and was looking for another partner. While Mr. Preston
saw the opportunity as a way to elevate the role of the sanitation
worker in a one-off show, Mr. Silverstein saw it as a great partnership
for raw material.

The department's dead-stock T-shirts, tents and tablecloths have proven
something of a treasure trove for him.

Over Thanksgiving weekend, Mr. Silverstein was one of the star companies
in an American Express showcase on Small Business Saturday. He is also
teaming up with a former mentor at Swimwear Anywhere for a line of
bathing suits made in Taiwan, which will be his first foray into
offshore production. (The scraps will be sent back along with the trunks
and one-pieces, which are made from recycled ocean fishing nets.)

Recently Lin-Manuel Miranda wore a Zero Waste Daniel sweatshirt at an
Amex event. The drag queen Pattie Gonia wore a long mosaic gown based on
Botticelli's ``Birth of Venus'' at the Tony Awards in June and made
Vogue's
\href{https://www.vogue.com/slideshow/best-red-carpet-moments-fashion-tony-awards}{best-dressed
slide show}, albeit without identification.

The company has been profitable for a year, Mr. Silverstein said, and
ships across the United States as well as to Canada, Britain, Brazil and
Germany.

Now Mr. Silverstein is at another turning point. Does he get bigger?
Does he train other ragpickers to do what he does? Does he open another
outlet? Does he really get in the game?

He is not sure. ``I can't clothe the world, and maybe the world doesn't
need me to,'' he said. Maybe the drive to clothe the world is part of
what created the problem he is now trying to solve in the first place.
``When I think about what I want in terms of brand recognition, I would
love to see this brand as a household name. But I think that's very
different than dollars. And I don't want to be any bigger than I can
guarantee it's a zero-waste product or that I feel happy.''

He was gathering pieces for a Freddie Mercury mosaic. ``Right now,'' he
said, surveying his mountain of scraps, ``I am so happy.''

Advertisement

\protect\hyperlink{after-bottom}{Continue reading the main story}

\hypertarget{site-index}{%
\subsection{Site Index}\label{site-index}}

\hypertarget{site-information-navigation}{%
\subsection{Site Information
Navigation}\label{site-information-navigation}}

\begin{itemize}
\tightlist
\item
  \href{https://help.nytimes.com/hc/en-us/articles/115014792127-Copyright-notice}{©~2020~The
  New York Times Company}
\end{itemize}

\begin{itemize}
\tightlist
\item
  \href{https://www.nytco.com/}{NYTCo}
\item
  \href{https://help.nytimes.com/hc/en-us/articles/115015385887-Contact-Us}{Contact
  Us}
\item
  \href{https://www.nytco.com/careers/}{Work with us}
\item
  \href{https://nytmediakit.com/}{Advertise}
\item
  \href{http://www.tbrandstudio.com/}{T Brand Studio}
\item
  \href{https://www.nytimes.com/privacy/cookie-policy\#how-do-i-manage-trackers}{Your
  Ad Choices}
\item
  \href{https://www.nytimes.com/privacy}{Privacy}
\item
  \href{https://help.nytimes.com/hc/en-us/articles/115014893428-Terms-of-service}{Terms
  of Service}
\item
  \href{https://help.nytimes.com/hc/en-us/articles/115014893968-Terms-of-sale}{Terms
  of Sale}
\item
  \href{https://spiderbites.nytimes.com}{Site Map}
\item
  \href{https://help.nytimes.com/hc/en-us}{Help}
\item
  \href{https://www.nytimes.com/subscription?campaignId=37WXW}{Subscriptions}
\end{itemize}
