Sections

SEARCH

\protect\hyperlink{site-content}{Skip to
content}\protect\hyperlink{site-index}{Skip to site index}

\href{/section/technology}{Technology}\textbar{}At the Edge of the
World, a New Battleground for the U.S. and China

\url{https://nyti.ms/34E4MZj}

\begin{itemize}
\item
\item
\item
\item
\item
\item
\end{itemize}

\includegraphics{https://static01.nyt.com/images/2019/12/20/business/00huawei-faroe1/merlin_166187199_2c87d378-88a4-4518-8429-ac249a722ae6-articleLarge.jpg?quality=75\&auto=webp\&disable=upscale}

\hypertarget{at-the-edge-of-the-world-a-new-battleground-for-the-us-and-china}{%
\section{At the Edge of the World, a New Battleground for the U.S. and
China}\label{at-the-edge-of-the-world-a-new-battleground-for-the-us-and-china}}

The Faroe Islands have become perhaps the most unexpected place for the
United States and China to tussle over the Chinese tech giant Huawei.

Salmon farms belonging to Hidden Fjord seen from the island of Streymoy
with the island of Hestur behind.Credit...Ben Quinton for The New York
Times

Supported by

\protect\hyperlink{after-sponsor}{Continue reading the main story}

By \href{https://www.nytimes.com/by/adam-satariano}{Adam Satariano}

\begin{itemize}
\item
  Published Dec. 20, 2019Updated Dec. 21, 2019
\item
  \begin{itemize}
  \item
  \item
  \item
  \item
  \item
  \item
  \end{itemize}
\end{itemize}

TORSHAVN, Faroe Islands --- The mere existence of the Faroe Islands is a
wonder. Tall peaks of snow-patched volcanic rock jut out from the North
Atlantic Ocean. Steep cliffs plunge into the deep waters of narrow
fjords.

The remote collection of 18 small islands, which sit between Iceland and
Norway, is known for a robust puffin population and periodic whale
hunts. The semiautonomous Danish territory also has a thriving salmon
industry.

Technology is not a common conversation topic among its 50,000
residents. Yet in recent weeks, the Faroe Islands have turned into a new
and unlikely battleground in the
\href{https://www.nytimes.com/2018/03/23/technology/trump-china-tariffs-tech-cold-war.html}{technological
Cold War} between the United States and China.

The dispute started because of a contract. The Faroe Islands wanted to
build a new ultrafast wireless network with
\href{https://www.nytimes.com/2018/12/31/technology/personaltech/5g-what-you-need-to-know.html}{fifth-generation
technology, known as 5G}. To create that new network, the territory
planned to award the job to a technology supplier.

That was when the United States began urging the archipelago nation not
to give the contract to a particular company: the Chinese
telecommunications giant Huawei. American officials have long said
Huawei is beholden to Beijing and
\href{https://www.nytimes.com/2019/08/07/business/huawei-us-ban.html}{poses
national security concerns}.

Then Chinese officials got involved. A senior Faroe Islands government
official was recently caught on tape saying that the Chinese had offered
to boost trade between the territory and China --- as long as Huawei got
the 5G network assignment.

``Commercially, the Faroe Islands cannot be very important to Huawei or
anybody else,'' Sjurdur Skaale, who represents the territory in the
Danish parliament, said over breakfast in the capital of Torshavn this
week. ``The fact that the Chinese and American embassies are fighting
over this as hard they are, there is something else on the table. It is
about something else than purely business.''

\includegraphics{https://static01.nyt.com/images/2019/12/20/business/00huawei-faroe2/merlin_166187493_1cb3806b-7b83-4f85-9b07-ec51d5708f80-articleLarge.jpg?quality=75\&auto=webp\&disable=upscale}

Image

Sjurdur Skaale, a member of the Danish parliament who represents the
Faroe Islands, with the Faroese government buildings behind
him.Credit...Ben Quinton for The New York Times

No location is now too small for the United States and China to focus on
as they tussle over the future of technology. The Faroe Islands, whose
proximity to the arctic gives it added military importance, joins
countries across Europe caught in the middle of the two superpowers over
Huawei, the crown jewel of the Chinese tech sector.

For more than a year,
\href{https://www.nytimes.com/2019/01/26/us/politics/huawei-china-us-5g-technology.html}{American
officials have applied pressure} on Britain, Germany, Poland and others
to follow its lead in banning Huawei from new 5G networks. They argue
the company can be used by China's Communist Party to spy or sabotage
critical networks. Huawei has denied that it helps Beijing.

But if the European nations side with Washington, they risk harming
their economic ties to China, which has a growing appetite for German
cars, French airplanes and British pharmaceuticals.

In the Faroe Islands, Bardur Nielsen, the prime minister, has tried
defusing the conflict. In a statement, he said his government ``has not
been pressured or threatened by foreign authorities in relation to the
development of a 5G network in the Faroe Islands.''

Any decision about awarding a contract to Huawei, he said, would be made
by the local telecommunications company, Foroya Tele.

Foroya Tele said in a statement that it is testing different
technologies. The choice of a 5G network provider, it said, ``requires
significant considerations given the scale and importance of the
investment for the Faroe Islands.''

For the people of the Faroe Islands, the debate over Huawei and 5G is
rooted in salmon more than in download speeds.

Salmon is central to the territory's economy. More than 90 percent of
the Faroe Islands' exports are fish, including salmon, mackerel, herring
and cod. In the surrounding waters, thousands of salmon can be seen
splashing inside large netted rings, where they are bred for meals in
Paris, Moscow, New York --- and, increasingly, Beijing.

After 2010, the islands' salmon exports to China picked up. At the time,
the Chinese government had slowed the purchase of the fish from Norway
in response to the awarding of the
\href{https://www.nytimes.com/2010/12/11/world/europe/11nobel.html}{Nobel
Peace Prize to Chinese human rights activist Liu Xiaobo} in Oslo.

Image

Salmon has become central to the economy of the Faroe
Islands.Credit...Ben Quinton for The New York Times

Image

Runi Dam, a consultant for fishing companies in the Faroe Islands. ``We
have the perfect environment,'' he said.Credit...Ben Quinton for The New
York Times

China now makes up about 7 percent of the Faroe Islands' salmon sales.
The Faroese government this year opened an
\href{https://www.faroeislands.fo/the-big-picture/news/faroese-representation-to-open-in-beijing/}{office
in Beijing} to further expand trade.

In 2014, the islands' salmon sales to Russia exploded after the European
Union limited what fish other countries could export there. Those rules
do not apply to the Faroe Islands because it is not a part of the
European bloc.

In all, salmon exports from the Faroe Islands are expected to top \$550
million this year, up from roughly \$190 million a decade ago.

``This is the home place of Atlantic salmon,'' Runi Dam, a consultant
for local fishing companies, said while standing over giant pens filled
with about 15,000 salmon each. ``We have the perfect environment.''

Now the salmon business has become entangled in the fight over the 5G
wireless network.

Image

The Faroese island of Koltur, where steep cliffs fall into deep
waters.Credit...Ben Quinton for The New York Times

Image

A mountain top road on the island of Streymoy.Credit...Ben Quinton for
The New York Times

Image

A freshly caught salmon, taken out of one of Hidden Fjord's holding
nets.Credit...Ben Quinton for The New York Times

Last month, America's ambassador to Denmark, Carla Sands, went public
with warnings against Huawei. In an
\href{https://www.tidende.dk/tidende/indland/2019/11/29/usa-skraemmer-faeroeerne-fra-at-koebe-5g-kinesiske-huawei/}{opinion
piece} in the local Faroe Islands newspaper, Ms. Sands said there could
be ``dangerous consequences'' if the company was allowed to build the 5G
network. When countries let Huawei in, she said, ``they agree to work
under Chinese communist rules.''

In another
\href{https://www.dr.dk/nyheder/penge/usa-ambassadoer-i-danmark-angriber-huawei-direktoer-han-arbejder-de-kinesiske}{interview}
with Danish Broadcasting this week, Ms. Sands accused a Huawei executive
responsible for the Nordic region of ``working for the Chinese
communists,'' who are ``exporting their spying, their corruption and
bribery around the world.''

Ms. Sands declined to be interviewed.

At the same time, China's ambassador to Denmark visited the Faroe
Islands at least twice in the past two months.

This month, the Danish national newspaper,
\href{https://www.berlingske.dk/internationalt/banned-recording-reveals-china-ambassador-threatened-faroese-leader}{Berlingske},
published the
\href{https://www.berlingske.dk/globalt/documentation-read-the-printout-of-the-banned-recording}{transcript}
of an audio recording in which a senior Faroe Islands official is
summarizing one of the meetings. Herálvur Joensen, a senior aide in the
Faroese government, was caught on tape saying China's ambassador had
threatened to block a trade deal --- and more fish sales --- if Huawei
was not used for the 5G network.

``If Foroya Tele signed agreement with Huawei, then all doors would be
open for a free-trade agreement with China,'' he said in the recording.
``If this doesn't happen, then there won't be a trade agreement.''

A spokesman for the prime minister said Mr. Joensen had not attended the
meeting with the Chinese ambassador and was not available for an
interview.

Image

Rógvi Olavson, a lecturer at Faroe Islands University.Credit...Ben
Quinton for The New York Times

Image

Sissal Kristiansen, the co-owner of a local clothing company called
Sisha Brand.Credit...Ben Quinton for The New York Times

Huawei's critics jumped on the revelations, saying the leaked recording
showed the close links between Huawei and the Chinese government.

China's ambassador, Feng Tie, wrote in
\href{https://www.berlingske.dk/kommentatorer/kinesisk-ambassadoer-i-danmark-usa-er-den-stoerste-usikkerhedsfaktor}{Berlingske}
that the country did not pressure the Faroe Islands. ``It's my duty to
secure that Huawei is treated fair and without discrimination in
Denmark,'' he said. ``It's not at all in Chinese culture to promote
threats. Promoting threats is more known from the U.S.''

Huawei said in a statement it was not involved in any talks between the
two governments.

In villages and harbors around the islands, people said they were
bewildered about being thrust into a battle between China and the United
States.

Image

Workers at a fish processing plant and farm on the Faroese island of
Streymoy.Credit...Ben Quinton for The New York Times

``It is a lice between two nails,'' said Rógvi Olavson, who lives in
Torshavn and is a lecturer at the local university. ``You're squeezed by
the U.S. on the one hand and China on the other.''

While many residents said the Faroe Islands prefer the United States
over China, several expressed anger at American officials for demanding
that Huawei be banned. They said the company helped build the existing
4G network, which they use to make phone calls or share photos from some
of the more far-flung areas of the islands.

Sissal Kristiansen, who designs sweaters and other clothing from Faroese
wool, said she had listened to a recent interview with Ms. Sands.

``It awoke this, `Oh bugger off' feeling in me,'' she said. ``We make
our own decisions.''

Image

Antennas near Torshavn, the capital of the islands.Credit...Ben Quinton
for The New York Times

Others are wary about harming economic ties with China, which they fear
will retaliate if Huawei is not selected for the 5G network. Many locals
remember an economic crisis in the 1990s, when about 10 percent of
Faroese residents ended up moving abroad.

Today, unemployment on the islands is almost nonexistent --- just 183
people were out of work as of Friday, according to government
\href{https://www.als.fo/}{statistics}. Like other Nordic countries,
health care, education and other social services are free. There is
virtually no crime.

``China is not just a nice customer, it is a necessity,'' said Martin
Breum, an arctic expert who has
\href{https://www.mqup.ca/cold-rush-products-9780773553637.php?page_id=73\&}{written
about the Faroe Islands}. The Faroese, he added, ``have nothing else to
sell to the rest of the world. They live off their fish.''

Martin Selsoe Sorensen contributed reporting from Copenhagen.

Advertisement

\protect\hyperlink{after-bottom}{Continue reading the main story}

\hypertarget{site-index}{%
\subsection{Site Index}\label{site-index}}

\hypertarget{site-information-navigation}{%
\subsection{Site Information
Navigation}\label{site-information-navigation}}

\begin{itemize}
\tightlist
\item
  \href{https://help.nytimes.com/hc/en-us/articles/115014792127-Copyright-notice}{©~2020~The
  New York Times Company}
\end{itemize}

\begin{itemize}
\tightlist
\item
  \href{https://www.nytco.com/}{NYTCo}
\item
  \href{https://help.nytimes.com/hc/en-us/articles/115015385887-Contact-Us}{Contact
  Us}
\item
  \href{https://www.nytco.com/careers/}{Work with us}
\item
  \href{https://nytmediakit.com/}{Advertise}
\item
  \href{http://www.tbrandstudio.com/}{T Brand Studio}
\item
  \href{https://www.nytimes.com/privacy/cookie-policy\#how-do-i-manage-trackers}{Your
  Ad Choices}
\item
  \href{https://www.nytimes.com/privacy}{Privacy}
\item
  \href{https://help.nytimes.com/hc/en-us/articles/115014893428-Terms-of-service}{Terms
  of Service}
\item
  \href{https://help.nytimes.com/hc/en-us/articles/115014893968-Terms-of-sale}{Terms
  of Sale}
\item
  \href{https://spiderbites.nytimes.com}{Site Map}
\item
  \href{https://help.nytimes.com/hc/en-us}{Help}
\item
  \href{https://www.nytimes.com/subscription?campaignId=37WXW}{Subscriptions}
\end{itemize}
