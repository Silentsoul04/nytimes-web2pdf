\href{/section/opinion}{Opinion}\textbar{}`You Are Killing Us? We Will
Make You a Joke.' Meet Ahmed Albasheer.

\href{https://nyti.ms/2ZqyFeF}{https://nyti.ms/2ZqyFeF}

\begin{itemize}
\item
\item
\item
\item
\item
\item
\end{itemize}

\includegraphics{https://static01.nyt.com/images/2019/12/24/opinion/24Fixes2/24Fixes2-articleLarge.jpg?quality=75\&auto=webp\&disable=upscale}

Sections

\protect\hyperlink{site-content}{Skip to
content}\protect\hyperlink{site-index}{Skip to site index}

\href{/section/opinion}{Opinion}

\hypertarget{you-are-killing-us-we-will-make-you-a-joke-meet-ahmed-albasheer}{%
\section{`You Are Killing Us? We Will Make You a Joke.' Meet Ahmed
Albasheer.}\label{you-are-killing-us-we-will-make-you-a-joke-meet-ahmed-albasheer}}

The Iraqi comedian's TV show is fueling demonstrations in his home
country.

Iraqi comedian, journalist and director Ahmad Al-Basheer.Credit...Chad
Batka for The New York Times

Supported by

\protect\hyperlink{after-sponsor}{Continue reading the main story}

\includegraphics{https://static01.nyt.com/images/2019/02/13/opinion/tina-rosenberg/tina-rosenberg-thumbLarge-v2.png}

By Tina Rosenberg

Ms. Rosenberg is a co-founder of the Solutions Journalism Network, which
supports rigorous reporting about responses to social problems.

\begin{itemize}
\item
  Dec. 26, 2019
\item
  \begin{itemize}
  \item
  \item
  \item
  \item
  \item
  \item
  \end{itemize}
\end{itemize}

Last week in the studio where he tapes the
``\href{https://www.youtube.com/channel/UCjxrFnMg_scE7fkw_Ip0_yA}{Albasheer
Show},'' Ahmed Albasheer put on a dark presidential hat and a jacket
covered in an absurd amount of medals and gold braid, and sat at his
desk in an office adorned with the seal of the president of the Republic
of Albasheer.

The republic is his invention of course, but Iraqis know what he is
mocking. Mr. Albasheer, a 35-year-old journalist, fights for his country
with his sense of humor. He has a repertoire of slightly deranged
expressions and inspired comic timing, in Arabic (I'm told) and, more
surprisingly, in English --- a language he didn't really speak until
recently.

Since it began airing in 2014, Mr. Albasheer's weekly show has become
one of the most popular shows in Iraq, airing on YouTube and satellite
television. In the past few weeks, the show has taken on new importance.
Thousands of young Iraqis are demonstrating, in ways the country has
never seen before. The ``Albasheer Show'' is deeply intertwined with the
protests. Mr. Albasheer exposes the workings of power in Iraq, covers
the protests and the government's brutal response, exhorts the
protesters to stay peaceful, amplifies their voices and boosts their
morale. Some in Iraq believe the protests wouldn't be happening without
him.

\includegraphics{https://static01.nyt.com/images/2019/12/24/opinion/24Fixes1/24Fixes1-articleLarge.jpg?quality=75\&auto=webp\&disable=upscale}

Since Oct. 1, young people have been protesting in Baghdad and other
cities in southern Iraq, with thousands camped in Baghdad's Tahrir
Square since Oct. 25.

Previous protests in Iraq were party-led and focused on electricity and
jobs. These are different --- truly grass roots, leaderless, party-less,
uniting rich and poor, Sunni and Shiite, men and women. A protester and
blogger who goes by the name Hayder Hamzoz --- he's also organizer of
the Iraqi Network for Social Media, an association of bloggers and
citizen journalists --- said that about a quarter of protesters during
the day are female.

The protests aim at the heart of Iraqi power, a system largely designed
by the American occupiers of Iraq in 2003. The system divides spoils and
patronage among various religious sects and political parties.

The protesters' argument --- also Mr. Albasheer's argument --- is that
the government creates these divisions to keep itself in power. So far,
the protests
\href{https://www.nytimes.com/2019/11/29/world/middleeast/iraq-prime-minister-mahdi-resign.html}{have
toppled} the prime minister and led to the
\href{https://www.nytimes.com/2019/12/24/world/middleeast/iraq-election-law.html}{passage}
of a new election law. But there is no new prime minister, and it's not
clear how the election law would actually work. Protesters also want a
new government and a new election under United Nations supervision. They
want Iran, the main foreign backer of the system, out. Their slogan is
``We want a country.''

So far, the Iraqi government has responded with extreme brutality and
acted with impunity. Security forces and masked gunmen have
\href{https://www.nytimes.com/2019/12/21/world/middleeast/Iraq-protests-Iran.html}{killed
more than 500 people}, wounded thousands more and kidnapped many others.
Suha Oda, an Iraqi journalist and activist on women's issues, said that
the government targets women, ``specifically kidnapping and killing
female protesters, to pressure families to prevent women from
participating.'' Ms. Oda said --- and everyone I talked to concurred ---
that among protesters there has not been a single report of a sexual
assault. ``You are girls and boys in the same place with cooperation,
and no harassment,'' Mr. Albasheer has said on his show. ``I am very
proud. I am in love with that.''

``All of our revolutions before were party-oriented,'' said Zainab
Salbi, an Iraqi-American campaigner for women's rights and peace. ``This
is the first time it's the people. They're saying, `We're Muslims, and
we're done with the divisions.' We are all shocked this has happened. My
father is 77, and he calls me crying, saying, `In all my life I've never
seen such a thing.'''

``You are strong because you have no leader,'' Mr. Albasheer often says
on the show, speaking to the protesters. ``This is the people's voice.
No one can stop this at any moment. No one can claim he is the reason
people are out on the street.''

Tahrir Square has become its own city. The protesters have organized
security, cleaning, electrical repair, food, showers, toilets, a small
hospital, a radio station, a newspaper, civics classes, a library and
medical care, including psychotherapy. Protesters play music, shoot
videos and paint
\href{https://www.theguardian.com/global-development/gallery/2019/nov/26/murals-of-baghdad-the-protest-art-in-pictures}{murals}.
A big screen shows football games and YouTube videos --- including, of
course, the ``Albasheer Show.''

The show is an hour of Mr. Albasheer at his desk, mixed with video clips
and
\href{https://www.youtube.com/watch?v=uXgT-hvf_1o\&feature=youtu.be}{original
songs}. He mocks the Godlike deference accorded Iraq's presidents, the
country's corruption, the use of religion for political gain. When he
did a show in a
\href{https://www.youtube.com/watch?v=RQ54rlGp4WM\&feature=youtu.be}{cleric's
costume} (split into white on one side and black on the other to
indicate different religious affiliations), he gravely announced that
since he was now a cleric, he had to form his own militia.

The show begins with video of young Iraqis greeting the show and often
ends with Mr. Albasheer talking directly to the protesters. ``He'll say,
`Wave if you can see me,''' Mr. Hamzoz said. ``He can't see us, of
course, but we'll wave at him.''

In the protests' first, most violent week, when
\href{https://www.aljazeera.com/news/2019/10/iraq-abdul-mahdi-orders-probe-protester-deaths-191012152657389.html}{more
than 100 people were killed}, Mr. Albasheer told no jokes. ``The first
ones were a huge responsibility,'' he said. ``I would show video of
young men getting killed by a sniper and start to cry.''

Now he is funny again. ``We need to humiliate the politicians,'' he
said. ``You are killing us? We will make you a joke.''

Mr. Albasheer is now shooting the show in New Haven, Conn., where he has
been on a fellowship at Yale. His crew of about 25 people work from a
country he asked me not to name. Half of them work uncredited because of
the danger.

Iraqis watch the show on satellite TV on Germany's Deutsche Welle Arabic
channel. It repeats about once a day at different times to try to escape
government jamming. The show also has 3.8 million followers on YouTube.
It is, of course, in Arabic, but Mr. Albasheer also taped two short
programs in English (one
\href{https://www.youtube.com/watch?v=V-KVPWXWlNk}{pretty funny}, one
\href{https://www.youtube.com/watch?v=Lc0EOr_Rq_w}{sober}) to tell the
world about the protests. (He said that his income comes from the
program and from a company he owns that does other TV production.)

Mr. Albasheer's model was Jon Stewart (he calls George Carlin his idol),
and the two have been compared. The comparison goes only so far. Mr.
Stewart's programs were not jammed. Unlike Mr. Albasheer, Mr. Stewart
did not get daily death threats; has not had to flee his country; has
not seen his father, brother, best friend and countless other family
members and friends murdered; and has not been the victim of a suicide
bomber.

Mr. Albasheer was once an ordinary correspondent, anchor and talk-show
host at several of Iraq's highly controlled TV news stations. Then, at a
poetry celebration for the Prophet Muhammad in 2011, a man burst in
wearing a suicide vest. ``I looked him in the eye trying to explode
himself,'' he said. ``Those two seconds were very long seconds. I had
time to think: If I am dead now, who am I?''

He ran and found shelter behind a wall. But his best friend and several
other friends were killed.

He didn't leave his house for the next six months. ``I'm just sitting at
home getting fatter,'' he said. ``I was afraid of everything. I was
waiting for I don't know what. Just to die, maybe.''

But he remembered his long two seconds thinking: Who am I? He made a
decision: He was someone who would say whatever he wanted.

He moved his family to Jordan, and while working as a journalist, he
tried comedy news. In 2014 he began the ``Albasheer Show.''

The show has fueled the protests by engaging young people with the core
structures of power in Iraq. Mr. Albasheer is not the only one to talk
about them, but because he's funny, he's the one whom people watch ---
especially young people.

His decision to speak his mind earned him trust. ``He became important
when he started to be very clear with his messages,'' Mr. Hamzoz said.
``Before, he'd say, `There's a militia group behind that,' and make fun
of them. But he didn't name it, or its leader.''

That changed three years ago, with the cleric episode. He began to name
people, even militia leaders. ``It's very difficult to say that --- 100
percent to be killed or kidnapped,'' Mr. Hamzoz said. ``His show broke
the fear.''

The show also gives the protests more staying power. For people risking
bullets daily, it matters to know that the government cannot hide its
brutality. Just as important, protests last only as long as people feel
heroic and proud, and feel as if they're making a difference ---
feelings the show reinforces every week.

``Iraq is not going to change only with new ministers,'' Mr. Albasheer
tells them. ``The whole system must change. You will change Iraq.''

Tina Rosenberg (\href{https://twitter.com/tirosenberg}{@tirosenberg})
won a Pulitzer Prize for her book ``The Haunted Land: Facing Europe's
Ghosts After Communism.'' She is a former editorial writer for The Times
and the author, most recently, of ``Join the Club: How Peer Pressure Can
Transform the World'' and the World War II spy story e-book ``D for
Deception.''

\emph{To receive email alerts for Fixes columns, sign up}
\href{http://eepurl.com/ABIxL}{\emph{here.}}

\emph{The Times is committed to publishing}
\href{https://www.nytimes.com/2019/01/31/opinion/letters/letters-to-editor-new-york-times-women.html}{\emph{a
diversity of letters}} \emph{to the editor. We'd like to hear what you
think about this or any of our articles. Here are some}
\href{https://help.nytimes.com/hc/en-us/articles/115014925288-How-to-submit-a-letter-to-the-editor}{\emph{tips}}\emph{.
And here's our email:}
\href{mailto:letters@nytimes.com}{\emph{letters@nytimes.com}}\emph{.}

\emph{Follow The New York Times Opinion section on}
\href{https://www.facebook.com/nytopinion}{\emph{Facebook}}\emph{,}
\href{http://twitter.com/NYTOpinion}{\emph{Twitter (@NYTopinion)}}
\emph{and}
\href{https://www.instagram.com/nytopinion/}{\emph{Instagram}}\emph{.}

Advertisement

\protect\hyperlink{after-bottom}{Continue reading the main story}

\hypertarget{site-index}{%
\subsection{Site Index}\label{site-index}}

\hypertarget{site-information-navigation}{%
\subsection{Site Information
Navigation}\label{site-information-navigation}}

\begin{itemize}
\tightlist
\item
  \href{https://help.nytimes.com/hc/en-us/articles/115014792127-Copyright-notice}{©~2020~The
  New York Times Company}
\end{itemize}

\begin{itemize}
\tightlist
\item
  \href{https://www.nytco.com/}{NYTCo}
\item
  \href{https://help.nytimes.com/hc/en-us/articles/115015385887-Contact-Us}{Contact
  Us}
\item
  \href{https://www.nytco.com/careers/}{Work with us}
\item
  \href{https://nytmediakit.com/}{Advertise}
\item
  \href{http://www.tbrandstudio.com/}{T Brand Studio}
\item
  \href{https://www.nytimes.com/privacy/cookie-policy\#how-do-i-manage-trackers}{Your
  Ad Choices}
\item
  \href{https://www.nytimes.com/privacy}{Privacy}
\item
  \href{https://help.nytimes.com/hc/en-us/articles/115014893428-Terms-of-service}{Terms
  of Service}
\item
  \href{https://help.nytimes.com/hc/en-us/articles/115014893968-Terms-of-sale}{Terms
  of Sale}
\item
  \href{https://spiderbites.nytimes.com}{Site Map}
\item
  \href{https://help.nytimes.com/hc/en-us}{Help}
\item
  \href{https://www.nytimes.com/subscription?campaignId=37WXW}{Subscriptions}
\end{itemize}
