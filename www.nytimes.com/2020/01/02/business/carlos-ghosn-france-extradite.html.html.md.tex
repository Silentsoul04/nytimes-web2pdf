Sections

SEARCH

\protect\hyperlink{site-content}{Skip to
content}\protect\hyperlink{site-index}{Skip to site index}

\href{https://www.nytimes.com/section/business}{Business}

\href{https://myaccount.nytimes.com/auth/login?response_type=cookie\&client_id=vi}{}

\href{https://www.nytimes.com/section/todayspaper}{Today's Paper}

\href{/section/business}{Business}\textbar{}Turkey Questions Pilots
About Carlos Ghosn's Escape From Japan

\url{https://nyti.ms/2rK0D8K}

\begin{itemize}
\item
\item
\item
\item
\item
\item
\end{itemize}

Advertisement

\protect\hyperlink{after-top}{Continue reading the main story}

Supported by

\protect\hyperlink{after-sponsor}{Continue reading the main story}

\hypertarget{turkey-questions-pilots-about-carlos-ghosns-escape-from-japan}{%
\section{Turkey Questions Pilots About Carlos Ghosn's Escape From
Japan}\label{turkey-questions-pilots-about-carlos-ghosns-escape-from-japan}}

Mr. Ghosn, the former Nissan and Renault chief, fled Japan to avoid
trial and arrived in Lebanon this week.

\includegraphics{https://static01.nyt.com/images/2020/01/06/world/02jpjapan-ghosn-print/merlin_147107115_87c1e26e-519e-4662-89fd-4c660529615f-articleLarge.jpg?quality=75\&auto=webp\&disable=upscale}

\href{https://www.nytimes.com/by/elian-peltier}{\includegraphics{https://static01.nyt.com/images/2019/07/03/reader-center/author-elian-peltier/165383d8b7284129a185b6ca96e2a52e-thumbLarge.png}}

By \href{https://www.nytimes.com/by/elian-peltier}{Elian Peltier}

\begin{itemize}
\item
  Jan. 2, 2020
\item
  \begin{itemize}
  \item
  \item
  \item
  \item
  \item
  \item
  \end{itemize}
\end{itemize}

Turkish authorities on Thursday questioned seven people, including four
pilots, about the role they may have played in helping
\href{https://www.nytimes.com/2020/01/02/business/carlos-ghosn-movie.html}{Carlos
Ghosn} make his escape from Tokyo to Beirut, offering new clues to his
mysterious flight.

Elsewhere, prosecutors raided
\href{https://www.nytimes.com/2020/01/02/business/carlos-ghosn-movie.html}{Mr.
Ghosn's} home in Tokyo, a Lebanese government minister said the public
prosecutor had received a ``red notice'' ---~an alert that's akin to a
wanted poster ---~ from Interpol, and a French official said authorities
there would not extradite Mr. Ghosn if he were to travel to the country.

Four days after
\href{https://www.nytimes.com/2020/01/02/business/carlos-ghosn-movie.html}{Mr.
Ghosn} triumphantly announced his arrival in Beirut, law enforcement
officials and authorities were left grappling with the legal
implications of the former automotive executive's stunning escape, whose
details remain shrouded in mystery.

Mr. Ghosn, the former chief executive of Nissan and Renault, left Japan
on Sunday to avoid trial on financial misconduct charges there, though
his movements were supposed to be strictly limited while he was free on
bail. He turned up in Lebanon, saying he had escaped the ``rigged
Japanese justice system.''

The Lebanese justice minister, Albert Serhan, said on Thursday that the
public prosecutor had received a red notice from Interpol related to Mr.
Ghosn's case, according to the state-run National News Agency. Such a
notice is issued for individuals wanted for prosecution or to serve a
sentence.

An Interpol red notice is often portrayed as an international arrest
warrant, but it is essentially a diplomatic request seeking help
apprehending a fugitive. Interpol itself has no arrest powers and
foreign governments are not obligated to comply. Interpol does not
comment on specific cases, the agency said.

A red notice ``doesn't carry any weight other than being a notice,''
said Carl Tobias, a professor of law at the University of Richmond.
``It's not a charge, it's not saying Lebanon needs to arrest Mr.
Ghosn.''

Mr. Ghosn, who has not appeared in public since he announced he was in
Lebanon, issued a statement Thursday that seemed aimed to protect his
family from any legal jeopardy.

``There has been speculation in the media that my wife Carole, and other
members of my family played a role in my departure from Japan,'' the
statement said. ``All such speculation is inaccurate and false. I alone
arranged for my departure. My family had no role whatsoever.''

Mr. Ghosn was joyous and especially happy to be reunited with Carole,
said a friend of Mr. Ghosn's, speaking on the condition of anonymity to
discuss a private conversation. He said Mr. Ghosn's children were
expected to gather in Beirut and that ``some of them are there''
already. He said it was a relief to see Mr. Ghosn so happy because ``he
had not been for a long time.''

Hours earlier, a French government minister said authorities there would
not extradite Mr. Ghosn, a citizen of France, if he arrived there,
``because France never extradites its nationals.''

``That's a rule of the game,'' Agnès Pannier-Runacher, a junior economy
minister,
\href{https://www.bfmtv.com/mediaplayer/video/agnes-pannier-runacher-affirme-que-si-carlos-ghosn-venait-en-france-il-ne-serait-pas-extrade-1212117.html}{told
the news channel BFM}.

In Turkey, the authorities detained seven people suspected of helping
Mr. Ghosn escape, according to news outlets there. He reportedly left
Japan late Sunday aboard a business jet from Osaka to Istanbul Ataturk
Airport, where he quickly switched to another plane and flew to Beirut.

Much about his flight
\href{https://www.nytimes.com/2019/12/31/business/carlos-ghosn.html?action=click\&module=RelatedLinks\&pgtype=Article}{remains
unknown}, including how he was able to escape surveillance in Japan, how
he arranged his flights to Lebanon, and whether he was helped by any
other countries. The French foreign ministry declined to comment on
reports that Mr. Ghosn had used a French passport to enter Lebanon.

Mr. Ghosn, who has been charged in Japan with an array of financial
crimes while chairman of Nissan, was born in Brazil to a Lebanese
family, grew up mostly in Lebanon and has lived most of his adult life
in France. He has passports from all three countries, though his lawyers
in Japan have said that they held the documents.

Turkish news organizations, including the state-run Anadolu news agency,
reported that the planes that delivered Mr. Ghosn to Istanbul and Beirut
were operated by MNG Jet, a Turkish company that offers chartered
flights on business aircraft. Flight tracking websites confirm MNG
flights matching Mr. Ghosn's reported path.

Four of the seven people detained in Turkey were pilots employed by a
private aviation company, two were employees of a company that provides
ground services to aircraft, and one was a manager of a private cargo
company, according to the Turkish reports.

An official at Havas, a ground services company that operates at
Istanbul Ataturk Airport, confirmed that two of its employees were in
custody for questioning in the case but said that they were expected to
be released later in the day. A person who answered the phone at MNG
said no one was available to comment.

It was not clear whether anyone in Turkey knowingly aided Mr. Ghosn, or
if he used some kind of subterfuge to avoid detection, like traveling
under an alias.

The Bombardier Global Express jet that reportedly carried him to
Istanbul is owned by a Turkish company, STE Havacilik, which denied any
involvement in his escape. An executive of the company said that when it
was not using the plane, it rents the jet to MNG, which uses it for
chartered flights. Such arrangements are common with business jets.

In Japan, prosecutors on Thursday raided Mr. Ghosn's two-story house in
an exclusive neighborhood of central Tokyo. After about four hours,
around a dozen men --- most of them wearing black suits and surgical
masks --- carried out heavy black briefcases and other bags, ignoring
questions from journalists who followed them.

Officials in Japan have expressed their outrage over his escape, but Mr.
Ghosn said he would speak to the news media ``starting next week.''

\includegraphics{https://static01.nyt.com/images/2020/01/02/world/02ghosn2/merlin_166569594_72292731-f7b6-4a76-81ce-53dd92adbd86-articleLarge.jpg?quality=75\&auto=webp\&disable=upscale}

In Lebanon, Mr. Ghosn may face legal trouble for having visited Israel,
an enemy state. Under Lebanese law, it is illegal for citizens to visit
Israel, and even foreigners who have been there are supposed to be
banned.

On Thursday, three lawyers informed Lebanon's public prosecutor that Mr.
Ghosn had committed a crime by visiting Israel. He reportedly
\href{https://www.timesofisrael.com/lebanese-lawyers-want-ex-chief-of-nissan-prosecuted-over-2008-israel-trip/}{visited
Israel in 2008}, while an executive for the Renault-Nissan alliance.

One of the lawyers, Jad Tomeh, said the three were ``shocked'' that
Lebanese parties that back the ``resistance,'' or the struggle against
Israel, had been silent about ``this type of security breach.''

It was not immediately clear if the authorities would respond. A
\href{https://www.nytimes.com/2017/09/22/world/middleeast/lebanon-director-treason.html}{Lebanese-French-American
filmmaker} was briefly detained for visiting Israel when he arrived in
Lebanon in 2017, although he was not charged with a crime. But in
November, Lebanon
\href{https://www.timesofisrael.com/us-citizen-jailed-in-lebanon-20-years-after-fleeing-with-israel-allied-militia/}{jailed
an American-Lebanese} man who had joined an Israeli-backed militia in
Lebanon during the country's civil war and was accused of running a
prison notorious for torture.

In Lebanon, which does not have an extradition treaty with Japan, Mr.
Ghosn is seen by many as
\href{https://www.nytimes.com/2019/12/31/business/carlos-ghosn-lebanon.html}{a
folk hero}, a favorite son who studied in France's most prestigious
schools before embracing a successful career in the automobile industry.

Mr. Ghosn remains widely respected in France despite the accusations
that he
\href{https://www.nytimes.com/2019/03/06/business/carlos-ghosn-nissan.html}{underreported
his compensation}, shifted personal financial losses to Nissan, and used
funds from Renault to organize parties at the Palace of Versailles.
French officials would not comment on how Mr. Ghosn was able to flee
Japan or whether he had a second French passport.

On extradition, Ms. Pannier-Runacher said, the same rules apply to Mr.
Ghosn as to any French person. Nobody is above the law, she added, but
``French citizenship protects, and is protective of its citizens.''

A flight to France would be risky: Mr. Ghosn would have to pass through
the airspace of several countries that could arrest him.

Asked if Mr. Ghosn had fled to save his life, Ms. Pannier-Runacher said
that although his living conditions in Japan were unpleasant, his life
had not been threatened. Even so, she seemed amazed by the unfolding
drama.

``I'm hesitating between novel-like and \ldots{} I don't have the words
to describe this escape,'' she said.

\emph{Matt Apuzzo, Ben Dooley, Emily Flitter, Ben Hubbard and Amie Tsang
contributed reporting.}

Advertisement

\protect\hyperlink{after-bottom}{Continue reading the main story}

\hypertarget{site-index}{%
\subsection{Site Index}\label{site-index}}

\hypertarget{site-information-navigation}{%
\subsection{Site Information
Navigation}\label{site-information-navigation}}

\begin{itemize}
\tightlist
\item
  \href{https://help.nytimes.com/hc/en-us/articles/115014792127-Copyright-notice}{©~2020~The
  New York Times Company}
\end{itemize}

\begin{itemize}
\tightlist
\item
  \href{https://www.nytco.com/}{NYTCo}
\item
  \href{https://help.nytimes.com/hc/en-us/articles/115015385887-Contact-Us}{Contact
  Us}
\item
  \href{https://www.nytco.com/careers/}{Work with us}
\item
  \href{https://nytmediakit.com/}{Advertise}
\item
  \href{http://www.tbrandstudio.com/}{T Brand Studio}
\item
  \href{https://www.nytimes.com/privacy/cookie-policy\#how-do-i-manage-trackers}{Your
  Ad Choices}
\item
  \href{https://www.nytimes.com/privacy}{Privacy}
\item
  \href{https://help.nytimes.com/hc/en-us/articles/115014893428-Terms-of-service}{Terms
  of Service}
\item
  \href{https://help.nytimes.com/hc/en-us/articles/115014893968-Terms-of-sale}{Terms
  of Sale}
\item
  \href{https://spiderbites.nytimes.com}{Site Map}
\item
  \href{https://help.nytimes.com/hc/en-us}{Help}
\item
  \href{https://www.nytimes.com/subscription?campaignId=37WXW}{Subscriptions}
\end{itemize}
