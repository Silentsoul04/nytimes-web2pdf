Sections

SEARCH

\protect\hyperlink{site-content}{Skip to
content}\protect\hyperlink{site-index}{Skip to site index}

\href{https://www.nytimes.com/section/business}{Business}

\href{https://myaccount.nytimes.com/auth/login?response_type=cookie\&client_id=vi}{}

\href{https://www.nytimes.com/section/todayspaper}{Today's Paper}

\href{/section/business}{Business}\textbar{}Japan Issues Arrest Warrant
for Carlos Ghosn's Wife

\url{https://nyti.ms/2QZ4JCO}

\begin{itemize}
\item
\item
\item
\item
\item
\item
\end{itemize}

Advertisement

\protect\hyperlink{after-top}{Continue reading the main story}

Supported by

\protect\hyperlink{after-sponsor}{Continue reading the main story}

\hypertarget{japan-issues-arrest-warrant-for-carlos-ghosns-wife}{%
\section{Japan Issues Arrest Warrant for Carlos Ghosn's
Wife}\label{japan-issues-arrest-warrant-for-carlos-ghosns-wife}}

The authorities said they suspect Carole Ghosn gave false testimony in
April related to the allegations against her husband.

\includegraphics{https://static01.nyt.com/images/2020/01/08/world/08ghosn-1/merlin_156570825_048a7f7d-0b95-4998-89c4-b164c7160bee-articleLarge.jpg?quality=75\&auto=webp\&disable=upscale}

By Makiko Inoue, Eimi Yamamitsu and
\href{https://www.nytimes.com/by/carlotta-gall}{Carlotta Gall}

\begin{itemize}
\item
  Published Jan. 7, 2020Updated Jan. 8, 2020
\item
  \begin{itemize}
  \item
  \item
  \item
  \item
  \item
  \item
  \end{itemize}
\end{itemize}

TOKYO --- The Japanese authorities said on Tuesday that they had issued
a warrant for the arrest of Carole Ghosn, the wife of Carlos Ghosn,
taking direct aim at the family of the fallen auto magnate as they
sought to bring him back to the country to face criminal charges.

\href{https://www.nytimes.com/2020/01/08/business/carlos-ghosn-beirut.html}{{[}Latest
updates on Mr. Ghosn's press conference in Beirut.{]}}

Prosecutors in Tokyo said they had obtained an arrest warrant for Mrs.
Ghosn, 53, on suspicion of giving false testimony nine months ago. In a
statement, they said Mrs. Ghosn had testified that she did not know a
person who was involved in Mr. Ghosn's case, even though she was in
communication with that person while the person was wiring money between
companies at Mr. Ghosn's request.

The statement did not disclose the identities of the person or the
companies.

The arrest warrant is the latest twist in an international tale of
intrigue. Mr. Ghosn, the architect of the Nissan-Renault-Mitsubishi auto
empire, faces charges of financial wrongdoing in Japan. But he fled the
country on Dec. 29, flying on private jets first to Turkey and then to
Lebanon. Mr. Ghosn is a Lebanese national, and the nation does not
extradite its citizens.

In an additional development, the air charter company that flew Mr.
Ghosn said in a statement on Tuesday that it was paid only half of the
\$350,000 fee for the Japan-to-Turkey flight, and that it had received
no compensation at all for the second flight from Turkey to Lebanon.

The Ghosn family could not be reached for comment. Mrs. Ghosn, who is a
citizen of both Lebanon and the United States, denounced the Japanese
arrest warrant in
\href{http://www.leparisien.fr/economie/carole-ghosn-mon-mari-est-victime-d-un-complot-industriel-07-01-2020-8231064.php}{an
interview} published Tuesday in the French newspaper Le Parisien,
calling it ``an act of revenge by the prosecutors'' meant to put
pressure on her husband.

``I find this a belittling act from an alleged great democracy,'' she
said in the interview, which the newspaper said was conducted in an
exclusive hotel in Achrafieh, a neighborhood in eastern Beirut. ``I have
already been humiliated in Japan, where I have been accused of running
away from justice, when this is absolutely false.''

It is not clear how the Japanese arrest warrant would affect her ability
to return to the United States, which has an extradition agreement with
Japan, or to travel to other countries with extensive ties to Japan.

If she were to travel to the United States, an extradition request from
Japan would have to go through diplomatic channels. The Justice and
State Departments would review the request and then present it to a
federal magistrate for a hearing.

In France, Mr. Ghosn is a citizen but his wife is not. France does not
extradite its citizens, but a spokeswoman for the Ministry of Justice,
speaking generally, said that ``it's a different case'' for someone who
is not a French citizen.

While she is in Lebanon, Japan's options are limited. The authorities in
Tokyo have pressed Lebanon to return Mr. Ghosn, a Lebanese citizen,
though they acknowledged that Lebanese law forbids the extradition of a
citizen.

The Japanese ambassador on Tuesday met with Lebanon's president, Michel
Aoun, but afterward there was no sign that they had made progress toward
resolving the issue. A Japanese Justice Ministry official said on
Tuesday that the authorities were reviewing Lebanese law and working
with Japan's Foreign Ministry.

Japanese officials prompted Interpol, the international criminal
information clearinghouse, to issue what is known as a red notice
regarding Mr. Ghosn; such notices are issued internationally for people
wanted for prosecution or to serve a sentence. But red notices
essentially function as a diplomatic request for help, not as an
international arrest warrant, and they do not obligate governments to
comply.

Mrs. Ghosn has been a vocal defender of her husband. In April, in an
interview with The New York Times, she described how the Japanese
authorities
\href{https://www.nytimes.com/2019/04/04/business/carlos-ghosn-carole-wife-japan-nissan-arrest.html}{treated
her ``like a terrorist''} when they arrested him again in April at a
home in Tokyo where they were staying while he was free on bail on
earlier charges.

On Wednesday, Mr. Ghosn is expected to speak for himself at a news
conference in Beirut.

Mrs. Ghosn was formerly involved
\href{https://www.elledecor.com/shopping/home-accessories/a3167/caftans-for-a-cause-a-68942/}{in
the fashion industry}, and she was previously married to Marwan Marshi,
a banker of Lebanese-Palestinian origin in New York, according to a
relative of Mr. Marshi who asked not to be named in order to discuss the
couple's personal life. The couple was involved in philanthropy, and
Mrs. Ghosn met Mr. Ghosn at a charity event in New York, according to a
friend of Mr. Ghosn who also asked not to be named.

Also on Tuesday, Japan's land and transportation minister, Kazuyoshi
Akaba, said that major airports with terminals for private jets would be
required to inspect large baggage items that pass through them. The
stepped-up measures
\href{https://www.nytimes.com/2019/12/31/business/carlos-ghosn.html}{followed
reports} that Mr. Ghosn escaped through Kansai International Airport in
Osaka, Japan, while hiding in a large box that was loaded onto a private
aircraft.

Japanese officials have also said that they confiscated the 1.5 billion
yen, or nearly \$14 million, in bail that Mr. Ghosn forfeited when he
fled the country.

Last week, MNG Jet, the air charter company that operated the flights
that Mr. Ghosn used to leave Japan, brought a criminal complaint against
its operations manager, Okan Kosemen. It said Mr. Kosemen had deceived
the company after the news broke that Mr. Ghosn had flown on two of the
private jets it managed. MNG Jet is a subsidiary of MNG, a large Turkish
construction conglomerate.

Mr. Kosemen, who has been detained by the Turkish authorities on charges
of falsifying records and transporting an undocumented migrant, arranged
the contracts for both flights, according to MNG Jet. A lawyer for Mr.
Kosemen, Mehmet Fatih Danaci, said he believed his client was innocent.

Documents made available to The New York Times showed an invoice dated
Dec. 24 from MNG Jet to ``Al Nitaq Al Akhdhar for General Trade
Limited'' for ``TC-TSR Aviation and Logistic Services'' for the price of
\$175,000. (TC-TSR refers to the Bombardier aircraft that flew Mr. Ghosn
to Turkey.) The invoice, which appeared to be signed by Mr. Kosemen,
listed Al Nitaq Al Akhdhar as being based in Dubai.

A second document, dated Dec. 26, showed a payment order by Al Nitaq Al
Akhdhar to MNG Jet for \$175,000. The rest of the money was due upon the
completion of the flight.

Efforts to reach Al Nitaq Al Akhdhar were unsuccessful.

A spokesman for MNG said that it did not have access to the contract for
the Istanbul-Beirut flight because it was on Mr. Kosemen's computer,
which had been seized by investigators, and that it had not received any
payment for it at all.

The spokesman, who asked to remain anonymous citing company protocol,
said the company had no contact with Al Nitaq Al Akhdhar.

``This was managed by our rogue employee,'' the spokesman wrote in an
email.

Mr. Ghosn was first arrested by the Japanese authorities in November
2018 and was ultimately charged with four counts of financial wrongdoing
while running the vast automotive empire. Mr. Ghosn has denied the
allegations and said he was set up by Nissan executives who feared that
he would more closely combine the operations of the major Japanese
automaker with its French partner, Renault.

On Tuesday, Nissan broke its silence on Mr. Ghosn's flight, saying
\href{https://global.nissannews.com/en/releases/200107-00-e}{in a
statement} that an internal investigation had found ``numerous acts of
misconduct'' and that the company would continue to cooperate with the
authorities in investigating him.

Nissan kept monitoring Mr. Ghosn even after he was released on bail, the
Japanese news media has reported. The reports said that the surveillance
stopped shortly before Mr. Ghosn fled.

Junichiro Hironaka, one of Mr. Ghosn's lawyers in Japan, said he noted
that the former Nissan executive was under surveillance and filed a
criminal complaint with the police on Dec. 27, without specifying a
target of the complaint. In a brief interview, he said he did not know
whether his complaint led to the cancellation of the surveillance.

Nissan declined to comment.

Makiko Inoue and Eimi Yamamitsu reported from Tokyo, and Carlotta Gall
from Istanbul. Reporting was contributed by Aurelien Breeden and Liz
Alderman from Paris; Vivian Yee from Beirut, Lebanon; Motoko Rich from
Hong Kong; and Amy Chozick from New York.

Advertisement

\protect\hyperlink{after-bottom}{Continue reading the main story}

\hypertarget{site-index}{%
\subsection{Site Index}\label{site-index}}

\hypertarget{site-information-navigation}{%
\subsection{Site Information
Navigation}\label{site-information-navigation}}

\begin{itemize}
\tightlist
\item
  \href{https://help.nytimes.com/hc/en-us/articles/115014792127-Copyright-notice}{©~2020~The
  New York Times Company}
\end{itemize}

\begin{itemize}
\tightlist
\item
  \href{https://www.nytco.com/}{NYTCo}
\item
  \href{https://help.nytimes.com/hc/en-us/articles/115015385887-Contact-Us}{Contact
  Us}
\item
  \href{https://www.nytco.com/careers/}{Work with us}
\item
  \href{https://nytmediakit.com/}{Advertise}
\item
  \href{http://www.tbrandstudio.com/}{T Brand Studio}
\item
  \href{https://www.nytimes.com/privacy/cookie-policy\#how-do-i-manage-trackers}{Your
  Ad Choices}
\item
  \href{https://www.nytimes.com/privacy}{Privacy}
\item
  \href{https://help.nytimes.com/hc/en-us/articles/115014893428-Terms-of-service}{Terms
  of Service}
\item
  \href{https://help.nytimes.com/hc/en-us/articles/115014893968-Terms-of-sale}{Terms
  of Sale}
\item
  \href{https://spiderbites.nytimes.com}{Site Map}
\item
  \href{https://help.nytimes.com/hc/en-us}{Help}
\item
  \href{https://www.nytimes.com/subscription?campaignId=37WXW}{Subscriptions}
\end{itemize}
