Sections

SEARCH

\protect\hyperlink{site-content}{Skip to
content}\protect\hyperlink{site-index}{Skip to site index}

\href{https://myaccount.nytimes.com/auth/login?response_type=cookie\&client_id=vi}{}

\href{https://www.nytimes.com/section/todayspaper}{Today's Paper}

\href{/section/opinion}{Opinion}\textbar{}Before the `Final Solution'
There Was a `Test Killing'

\href{https://nyti.ms/2NlyxZl}{https://nyti.ms/2NlyxZl}

\begin{itemize}
\item
\item
\item
\item
\item
\item
\end{itemize}

Advertisement

\protect\hyperlink{after-top}{Continue reading the main story}

\href{/section/opinion}{Opinion}

Supported by

\protect\hyperlink{after-sponsor}{Continue reading the main story}

\hypertarget{before-the-final-solution-there-was-a-test-killing}{%
\section{Before the `Final Solution' There Was a `Test
Killing'}\label{before-the-final-solution-there-was-a-test-killing}}

Too few know the history of the Nazi methodical mass murder of disabled
people. That is why I write.

By Kenny Fries

Mr. Fries is the author, most recently, of
``\href{https://uwpress.wisc.edu/books/5638.htm}{In the Province of the
Gods}.''

\begin{itemize}
\item
  Jan. 8, 2020
\item
  \begin{itemize}
  \item
  \item
  \item
  \item
  \item
  \item
  \end{itemize}
\end{itemize}

\includegraphics{https://static01.nyt.com/images/2020/01/06/opinion/06disability-t4-1/06disability-t4-1-articleLarge.jpg?quality=75\&auto=webp\&disable=upscale}

My first visit to the Aktion T4 killing site at Brandenburg an der Havel
was in autumn. My destination, where 9,000 disabled people were murdered
as part of the Nazi ``euthanasia'' program, is embedded in the
activities of the town --- trams and buses, stores, a bank, a cafe.

The buildings that were once the old prison were mostly destroyed during
the war. If not for dark gray letters painted on one side of the light
gray building --- GEDENKSTÄTTE, on one side, and its English
translation, MEMORIAL, on another --- the site could easily be passed
unnoticed. From a distance, it looks prefab, temporary, perhaps an ad
hoc extension to an overcrowded school or municipal department.

Though it was October, I was thinking of winter. At the Nuremberg
``Doctors' Trial'' in 1947, Viktor Brack --- the economist, SS officer
and head of the office of the Chancellery of the Führer who was in
charge of Aktion T4 --- testified that the first of the mass murders of
disabled people happened ``in snow-covered Brandenburg on a winter's day
in December 1939 or January 1940.'' The exact date of this ``test
killing'' has not yet been determined.

No documents from the ``test killing'' have been preserved. According to
information at the memorial, ``Who the murdered patients were and where
they came from is unknown.'' What is known comes primarily from postwar
testimony of those involved, or thought to be involved, in what took
place that day.

Unlike the Holocaust, there are no T4 survivors. We know about T4 and
its aftermath mainly through medical records and from the perpetrators.
Aktion T4 does not have its Elie Wiesel or Primo Levi.

That is the main reason I write about what happened to disabled people
during the Third Reich. I want to be what Susanne C. Knittel and other
scholars call a ``vicarious witness.'' Ms. Knittel describes this not as
``an act of speaking for and thus appropriating the memory and story of
someone else but rather an attempt to bridge the silence through
narrative means.'' This is my way of bridging the silence, of keeping
alive something that is too often forgotten.

I'm not surprised that some of the perpetrators' testimony is
contradictory. In his diary, Dr. Irmfried Eberl, the medical director at
Brandenburg, mentions Jan. 18, 1940, as the date of the ``test
killing.'' However, Dr. Horst Schumann, whom we know to have been
present at the event, was on that day at Grafeneck, where he would
oversee mass killings, the first of which occurred on Jan. 18. Another
T4 employee said the murder of patients in Grafeneck started ``about 14
days'' after the ``test killing'' in Brandenburg. It seems Eberl mixed
up the dates of the two killings.

After he was arrested in 1959, Werner Heyde, a psychiatrist and the
medical director of the T4 program, placed the ``test killing'' at the
``beginning of January 1940.'' Heyde confessed only to being an
observer.

The German Meteorological Office records the first major snowfall of the
1939-40 winter in Brandenburg on New Year's Eve, 1939; December had been
relatively dry. Brack, in his testimony, was very clear about the snow
on the ground at Brandenburg for the ``test killing.'' By deduction, it
seems that the first Brandenburg mass murder took place during the first
days of January 1940.

Though the exact date is somewhat speculative, the words of those
responsible for the murder of 70,000 disabled people in Aktion T4, and
the 230,000 killed after the program's official end, clearly speak to
the main cause for what happened: the disvaluing of disabled lives.
Eugenics, which was rampant before and during the Reich, provided the
rationale for the killings, stigmatizing those with disabilities as not
human.

Dr. Albert Widmann, a chemist, forensic scientist and head of the
chemical department of the central offices of the Reich Detective
Forces, testified that he was asked to procure poison in large
quantities. At a meeting with an unidentified representative of the
Chancellery of the Führer, Widmann asked, ``What for? To kill people?''

``No,'' was the reply. ``Animals in the form of humans.''

It was the police chemist Dr. August Becker who prepared the carbon
monoxide gas for what he called the ``euthanasia experiment.''
Testifying in the 1960s, Becker also echoed eugenic depictions of the
disabled. He recalled looking through the gas chamber peephole and
observing ``the behavior of the delinquents,'' as the gas filled up the
chamber and the victims' lungs. Becker's depiction likens disabled
people to the immoral and illegal.

Becker described, in detail, the gas chamber as ``a room similar to a
shower room, lined with tiles about three by five meter{[}s{]}, and
three meters high in size.'' According to Becker, between 18 and 20
patients were led by nurses into this ``shower room.'' These men had to
``undress in an anteroom, so they were totally naked.'' Becker pointed
to Widmann as the one who ``operated the gas installation.'' But Widmann
always denied taking part.

When interrogated in 1947, Richard von Hegener, deputy head of the
killing of disabled children, named ``the chemist in charge, Dr.
Becker'' as the one ``who let the CO gas into the room.'' Von Hegener
said there were 30 patients ``dressed only in institutional clothing,''
who ``were led in and they calmly took a seat on the benches in the room
without any resistance.''

Heyde stated there were ``10, at most 15 --- the figure was more than 10
--- mentally ill patients.'' He said, ``I don't really know who let the
gas in.''

According to Brack, there were ``four such patients,'' all men, whom he
described, in another eugenic nod, as ``incurable.'' When asked about
their ages or from which institutions they came he replied, ``I really
don't have any memory of that any more.''

The more I learn, the more I understand the connection between Aktion T4
and what happened later to Jews and others deemed ``undesirable.'' The
Brandenburg ``test killing'' demonstrated that gassing was a
``suitable'' means for mass murder.

And as the text at the memorial emphasizes, ``it also gave the future
`killing doctors' the chance to familiarize with the method.'' After
recommending carbon monoxide for the mass murder of the disabled,
Widmann developed the gas wagons that were used for the subsequent mass
murder of Jews on the war's eastern front. Becker helped design these
mobile killing units, including those used by the notorious
\href{https://www.nytimes.com/2002/06/30/books/himmler-s-willing-executioners.html}{Einsatzgruppen}
in the Nazi-occupied areas of the Soviet Union. Eberl later worked at
the Chelmno and Treblinka extermination camps during Operation Reinhard,
the ``Final Solution.''

Of those whose testimonies are highlighted at the Brandenburg memorial,
Brack, in 1948, was the only one executed. Von Hegener was arrested in
1949 and sentenced to life imprisonment but was released early. Becker
had a stroke in 1959 and was deemed unfit to stand trial. Heyde was
arrested in 1959 and committed suicide before his trial. In both 1962
and 1967 Widmann was convicted to serve several years in prison but was
released upon payment of a fine.

Outside the memorial building, there is no cemetery. Across a parking
lot lies a large plot of gray gravel, interrupted only by the
reddish-brown brick foundations of what was the prison barn, which
housed the gas chamber. There are circles of piled leaves among the
gravel --- as if these random forms were gathered in a subliminal ritual
of mourning.

\begin{center}\rule{0.5\linewidth}{\linethickness}\end{center}

Kenny Fries is the author, most recently, of
``\href{https://www.kennyfries.com/works}{In the Province of the Gods}''
and is currently writing a book about disability and the Holocaust.

\emph{Disability is a series of essays, art and opinion by and about
people living with disabilities.}

\emph{\textbf{Now in print:}}
\emph{``}\href{http://bit.ly/2WTWIVv}{\emph{About Us: Essays From The
New York Times Disability Series}}\emph{,'' edited by Peter Catapano and
Rosemarie Garland-Thomson, published by Liveright.}

\emph{The Times is committed to publishing}
\href{https://www.nytimes.com/2019/01/31/opinion/letters/letters-to-editor-new-york-times-women.html}{\emph{a
diversity of letters}} \emph{to the editor. We'd like to hear what you
think about this or any of our articles. Here are some}
\href{https://help.nytimes.com/hc/en-us/articles/115014925288-How-to-submit-a-letter-to-the-editor}{\emph{tips}}\emph{.
And here's our email:}
\href{mailto:letters@nytimes.com}{\emph{letters@nytimes.com}}\emph{.}

\emph{Follow The New York Times Opinion section on}
\href{https://www.facebook.com/nytopinion}{\emph{Facebook}}\emph{,}
\href{http://twitter.com/NYTOpinion}{\emph{Twitter (@NYTopinion)}}
\emph{and}
\href{https://www.instagram.com/nytopinion/}{\emph{Instagram}}\emph{.}

Advertisement

\protect\hyperlink{after-bottom}{Continue reading the main story}

\hypertarget{site-index}{%
\subsection{Site Index}\label{site-index}}

\hypertarget{site-information-navigation}{%
\subsection{Site Information
Navigation}\label{site-information-navigation}}

\begin{itemize}
\tightlist
\item
  \href{https://help.nytimes.com/hc/en-us/articles/115014792127-Copyright-notice}{©~2020~The
  New York Times Company}
\end{itemize}

\begin{itemize}
\tightlist
\item
  \href{https://www.nytco.com/}{NYTCo}
\item
  \href{https://help.nytimes.com/hc/en-us/articles/115015385887-Contact-Us}{Contact
  Us}
\item
  \href{https://www.nytco.com/careers/}{Work with us}
\item
  \href{https://nytmediakit.com/}{Advertise}
\item
  \href{http://www.tbrandstudio.com/}{T Brand Studio}
\item
  \href{https://www.nytimes.com/privacy/cookie-policy\#how-do-i-manage-trackers}{Your
  Ad Choices}
\item
  \href{https://www.nytimes.com/privacy}{Privacy}
\item
  \href{https://help.nytimes.com/hc/en-us/articles/115014893428-Terms-of-service}{Terms
  of Service}
\item
  \href{https://help.nytimes.com/hc/en-us/articles/115014893968-Terms-of-sale}{Terms
  of Sale}
\item
  \href{https://spiderbites.nytimes.com}{Site Map}
\item
  \href{https://help.nytimes.com/hc/en-us}{Help}
\item
  \href{https://www.nytimes.com/subscription?campaignId=37WXW}{Subscriptions}
\end{itemize}
