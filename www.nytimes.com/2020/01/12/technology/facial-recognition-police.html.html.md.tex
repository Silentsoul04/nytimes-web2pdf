\href{/section/technology}{Technology}\textbar{}How the Police Use
Facial Recognition, and Where It Falls Short

\url{https://nyti.ms/2Nkhh6K}

\begin{itemize}
\item
\item
\item
\item
\item
\end{itemize}

\includegraphics{https://static01.nyt.com/images/2020/01/11/multimedia/11pinellas/00pinellas-superJumbo.jpg}

Credit...Yoshi Sodeoka

Sections

\protect\hyperlink{site-content}{Skip to
content}\protect\hyperlink{site-index}{Skip to site index}

\hypertarget{how-the-police-use-facial-recognition-and-where-it-falls-short}{%
\section{How the Police Use Facial Recognition, and Where It Falls
Short}\label{how-the-police-use-facial-recognition-and-where-it-falls-short}}

Records from Florida, where law enforcement has long used the
controversial technology, offer an inside look at its risks and rewards.

Credit...Yoshi Sodeoka

Supported by

\protect\hyperlink{after-sponsor}{Continue reading the main story}

By \href{https://www.nytimes.com/by/jennifer-valentino-devries}{Jennifer
Valentino-DeVries}

\begin{itemize}
\item
  Jan. 12, 2020
\item
  \begin{itemize}
  \item
  \item
  \item
  \item
  \item
  \end{itemize}
\end{itemize}

After a high-speed chase north of Orlando, Fla., sheriff's deputies
punctured the tires of a stolen Dodge Magnum and brought it to a stop.
They arrested the driver, but couldn't determine who he was. The man had
no identification card. He passed out after stuffing something into his
mouth. And his fingerprints, the deputies reported, appeared to have
been
\href{https://www.documentcloud.org/documents/6581927-Seminolecountycarchase-Redacted.html}{chewed
off}.

So investigators turned to one of the oldest and largest facial
recognition systems in the country: a statewide program based in
Pinellas County, Fla., that began almost 20 years ago, when law
enforcement agencies were just starting to use the technology. Officers
ran a photo of the man through a huge database, found a likely match and
marked the 2017 case as one of the system's more than 400 successful
``\href{https://www.documentcloud.org/documents/6586379-FACESlist-Redacted.html}{outcomes}''
since 2014.

A review of these Florida records --- the most comprehensive analysis of
a local law enforcement facial recognition system to date --- offers a
rare look at the technology's potential and its limitations.

Officials in Florida say that they query the system 4,600 times a month.
But the technology is no magic bullet: Only a small percentage of the
queries break open investigations of unknown suspects, the documents
indicate. The tool has been effective with clear images --- identifying
recalcitrant detainees, people using fake IDs and photos from anonymous
social media accounts --- but when investigators have tried to put a
name to a suspect glimpsed in grainy surveillance footage, it has
produced significantly fewer results.

The Florida program also underscores concerns about new technologies'
potential to violate due process. The system operates with little
oversight, and its role in legal cases is not always disclosed to
defendants, records show. Although officials said investigators could
not rely on facial recognition results to make an arrest, documents
suggested that on occasion officers gathered no other evidence.

``It's really being sold as this tool accurate enough to do all sorts of
crazy stuff,'' said Clare Garvie, a senior associate at the Center on
Privacy and Technology at Georgetown Law. ``It's not there yet.''

Facial recognition has set off controversy in recent years, even as it
has become an everyday tool for unlocking cellphones and tagging photos
on social media. The industry has drawn in new players
\href{https://www.cnet.com/news/facial-recognition-overkill-how-deputies-solved-a-12-shoplifting-case/}{like
Amazon}, which has
\href{https://www.nytimes.com/2018/05/22/technology/amazon-facial-recognition.html}{courted
police departments}, and the technology is used by law enforcement in
New York, Los Angeles, Chicago and elsewhere, as well as by the F.B.I.
and other federal agencies. Data on such systems is scarce, but
\href{https://www.perpetuallineup.org/}{a 2016 study} found that half of
American adults were in a law enforcement facial recognition database.

Police officials
\href{https://www.nytimes.com/2019/06/09/opinion/facial-recognition-police-new-york-city.html}{have
argued} that facial recognition
\href{https://www.nytimes.com/2019/05/18/us/facial-recognition-police.html}{makes
the public safer}. But a few cities, including San Francisco, have
\href{https://www.nytimes.com/2019/05/14/us/facial-recognition-ban-san-francisco.html}{barred
law enforcement from using the tool}, amid concerns about privacy and
false matches. Civil liberties advocates warn of the pernicious uses of
the technology, pointing to
\href{https://www.nytimes.com/2019/04/14/technology/china-surveillance-artificial-intelligence-racial-profiling.html}{China},
where the government has deployed it as a
\href{https://www.nytimes.com/2019/12/17/technology/china-surveillance.html}{tool
for authoritarian control}.

In Florida, facial recognition has long been part of daily policing. The
sheriff's office in Pinellas County, on the west side of Tampa Bay,
wrangled federal money two decades ago to try the technology and now
serves as the de facto facial recognition service for the state. It
enables access to more than 30 million images, including driver's
licenses, mug shots and juvenile booking photos.

``People think this is something new,'' the county sheriff, Bob
Gualtieri, said of facial recognition. ``But what everybody is getting
into now, we did it a long time ago.''

\hypertarget{a-question-of-due-process}{%
\subsection{A Question of Due Process}\label{a-question-of-due-process}}

Only one American court is known to have ruled on the use of facial
recognition by law enforcement, and it gave credence to the idea that a
defendant's right to the information was limited.

\href{https://www.nydailynews.com/news/crime/50-crack-sale-florida-lead-facial-recognition-change-article-1.3888958}{Willie
Allen Lynch} was accused in 2015 of selling \$50 worth of crack cocaine,
after the Pinellas facial recognition system suggested him as a likely
match. Mr. Lynch, who claimed he had been misidentified, sought the
images of the other possible matches;
\href{https://law.justia.com/cases/florida/first-district-court-of-appeal/2018/16-3290.html}{a
Florida appeals court ruled} against it. He is serving an eight-year
prison sentence.

Any technological findings presented as evidence are subject to analysis
through special hearings, but facial recognition results have never been
deemed reliable enough to stand up to such questioning. The results
still can play a significant role in investigations, though, without the
judicial scrutiny applied to more proven forensic technologies.

Laws and courts
\href{https://law.justia.com/cases/new-jersey/appellate-division-published/1992/254-n-j-super-754-1.html}{differ}
by \href{http://fprints.nwlean.net/Brady.pdf}{state} on what
investigative materials must be shared with the defense. This has led
some law enforcement officials to argue that they aren't required to
disclose the use of facial recognition.

In some of the Florida cases The Times reviewed, the technology was not
mentioned in initial warrants or affidavits. Instead, detectives noted
``\href{https://www.documentcloud.org/documents/6586485-Kohlsshoplifting-Redacted.html}{investigative
means}'' or an
``\href{https://www.documentcloud.org/documents/6591394-PasadenaLiquors-Redacted.html}{attempt
to identify}'' in court documents, while logging the matters as facial
recognition wins in the Pinellas County records. Defense lawyers said in
interviews that the use of facial recognition was sometimes mentioned
later in the discovery process, but not always.

Aimee Wyant, a senior assistant public defender in the judicial circuit
that includes Pinellas County, said defense lawyers should be provided
with all the information turned up in an investigation.

``Once the cops find a suspect, they're like a dog with a bone: That's
their suspect,'' she said. ``So we've got to figure out where they got
that name to start.''

Law enforcement officials in Florida and elsewhere emphasized that
facial recognition should not be relied on to put anyone in jail. ``No
one can be arrested on the basis of the computer match alone,'' the New
York police commissioner, James O'Neill,
\href{https://www.nytimes.com/2019/06/09/opinion/facial-recognition-police-new-york-city.html}{wrote
in a June op-ed}.

In most of the Florida cases The Times reviewed, investigators
\href{https://www.documentcloud.org/documents/6591395-Armedrobbery-Redacted.html}{followed}
\href{https://www.documentcloud.org/documents/6591396-Shoes-Redacted.html}{similar}
guidelines. But in a
\href{https://www.documentcloud.org/documents/6586485-Kohlsshoplifting-Redacted.html}{few}
\href{https://www.documentcloud.org/documents/6587265-Shoesshoplifting-Redacted.html}{instances},
court records suggest, facial recognition was the primary basis for an
arrest.

Last April, for example, a Tallahassee police officer investigating the
theft of an \$80 cellphone obtained a store surveillance image and
received a likely match from the facial recognition system, according to
the Pinellas list. The investigator then ``reviewed the surveillance
video and positively identified'' the suspect,
\href{https://www.documentcloud.org/documents/6586364-Targettheft-Redacted.html}{she
wrote} in a court document.

A police department spokeswoman suggested that this step provided a
check on the facial recognition system. ``What we can't do is just say,
`Oh, it's this guy,' and not even look at it,'' she said, adding that in
this instance ``it was a very clear photo.'' The case is proceeding.

\hypertarget{no-more-name-game}{%
\subsection{No More `Name Game'}\label{no-more-name-game}}

Pinellas County's Face Analysis Comparison \& Examination System, or
FACES, was started with a \$3.5 million federal grant arranged in 2000
by Representative Bill Young, a Florida Republican who led the House
Appropriations Committee.

\includegraphics{https://static01.nyt.com/images/2020/01/10/business/00Pinellas-2/merlin_166142886_d3380cf4-fa71-4f87-810b-da6c0cce5f34-articleLarge.jpg?quality=75\&auto=webp\&disable=upscale}

Earlier tests with law enforcement agencies elsewhere had produced
meager results, including systems in California that had
\href{https://www.nytimes.com/2001/05/03/technology/those-dimples-may-be-digits.html}{led
to one arrest in four years}. Still, the potential was tantalizing.
Pinellas's first planned use for facial recognition was in the local
jail's mug shot system. After Sept. 11, the program was expanded to
include the airport. Eventually, sheriff's deputies were able to upload
photos taken with digital cameras while on patrol.

The program received more than \$15 million in federal grants until
2014, when the county took over the annual maintenance costs, now about
\$100,000 a year, the sheriff's office said.

The first arrest attributed to the Florida program came in 2004, after a
woman who was wanted on a probation violation gave deputies a false
name, local
\href{https://www.orlandosentinel.com/news/os-xpm-2004-09-15-0409150165-story.html}{news
outlets reported}.

The number of arrests ticked up as the system spread across the state
and the pool of images grew to include the driver's license system. By
2009, the sheriff's office had credited it with nearly 500 arrests.
\href{https://www.washingtonpost.com/business/technology/state-photo-id-databases-become-troves-for-police/2013/06/16/6f014bd4-ced5-11e2-8845-d970ccb04497_story.html}{By
2013}, the number was approaching 1,000. Details on only a small number
of cases were disclosed publicly.

The latest list, of more than 400 successes since 2014, which The Times
obtained after a records request, is flawed: Not all successful
identifications are logged, and questionable or negative results are not
recorded. Still, together with related court documents --- records were
readily available for about half the cases --- the list offers insights
into which crimes facial recognition is best suited to help solve:
shoplifting,
\href{https://www.documentcloud.org/documents/6591405-Forgery-Redacted.html}{check
forgery}, ID fraud.

In case after case on the list, officers were seeking
\href{https://www.documentcloud.org/documents/6591397-Falsename-Redacted.html}{ID
checks}. ``We call it the name game,'' Sheriff Gualtieri said. ``We stop
somebody on the street, and they say, `My name is John Doe and I don't
have any identification.'''

In about three dozen court cases, facial recognition was crucial despite
being used with poorer-quality images. Nearly 20 of these involved minor
theft; **** others were more significant**.**

After
\href{https://www.tampabay.com/news/publicsafety/crime/deputies-brandon-man-faces-charges-after-robbing-victim-at-an-atm-at/2314761/}{a
2017 armed robbery} at an A.T.M. in nearby Hillsborough County, the
Pinellas records show, investigators used facial recognition to identify
a suspect. They showed the A.T.M. surveillance video to his girlfriend,
who confirmed it was him, according to
\href{https://www.documentcloud.org/documents/6587337-ATMcase-Redacted.html}{an
affidavit}. He pleaded guilty.

Image

Investigators in Florida used facial recognition software on an A.T.M.
surveillance video to identify a robbery suspect in 2017, according to
the Pinellas County records.Credit...Hillsborough County Sheriff's
Office

Instances of violent crime in which the system was helpful --- such as
the F.B.I.'s tracking
\href{https://www.bostonglobe.com/metro/2016/06/16/after-years-run-dorchester-man-arrested/Gpv7i9fJEP9pMNn00sQn4I/story.html}{a
fugitive} accused of child rape --- typically involved not surveillance
images but people with fake IDs or aliases.

In nearly 20 of the instances on the Pinellas list, investigators were
trying to identify people who could not identify themselves, including
Alzheimer's patients and murder victims. The sheriff's office said the
technology was also sometimes used to help identify witnesses.

The most cutting-edge applications of facial recognition in the area ---
at the airport, for instance --- never showed significant results and
were scrapped.

``For me it was a bridge too far and too Big Brother-ish,'' Sheriff
Gualtieri said.

\hypertarget{garbage-in-garbage-out}{%
\subsection{Garbage in, Garbage Out}\label{garbage-in-garbage-out}}

``It comes down to image quality,'' said Jake Ruberto, a technical
support specialist in the Pinellas County Sheriff's Office who helps run
the facial recognition program. ``If you put garbage into the system,
you're going to get garbage back.''

The software for FACES is developed by Idemia, a France-based company
whose prototype algorithms did well in several
\href{https://www.nist.gov/system/files/documents/2019/09/11/nistir_8271_20190911.pdf}{recent
tests} by the National Institute of Standards and Technology.

But the systems used by law enforcement agencies don't always have the
latest algorithms; Pinellas's, for example, was last overhauled in 2014,
although the county has been evaluating other, more recent, products.
Idemia declined to comment on it.

The gains in quality of the best facial recognition technology in recent
years have been astounding. In government tests, facial recognition
algorithms compared photos with a database of 1.6 million mug shots. In
2010, the error rate was just under 8 percent in ideal conditions ---
good lighting and high-resolution, front-facing photos. In 2018, it was
0.3 percent. But in surveillance situations, law enforcement hasn't been
able to count on that level of reliability.

Perhaps the biggest controversy in facial recognition has been its
uneven performance with people of different races. The findings of
\href{https://nvlpubs.nist.gov/nistpubs/ir/2019/NIST.IR.8280.pdf}{government
tests} released in December show that the type of facial recognition
used in police investigations tends to produce more false positive
results when evaluating images of black women. Law enforcement officials
in Florida said the technology's performance was not a sign that it
somehow harbored racial prejudice.

Officials in Pinellas and elsewhere also stressed the role of human
review. But
\href{https://journals.plos.org/plosone/article?id=10.1371/journal.pone.0139827}{tests
using passport images} have shown that human reviewers also have trouble
identifying the correct person on a list of similar-looking facial
recognition results. In those experiments, passport-system employees
chose wrong about half the time.

Poorer-quality images are known to contribute to mismatches, and dim
lighting, faces turned at an angle, or minimal disguises such as
baseball caps or sunglasses can hamper accuracy.

In China, law enforcement tries to get around this problem by installing
intrusive high-definition cameras with bright lights at face level, and
by tying facial recognition systems to other technology that scans
cellphones in an area. If a face and a phone are detected in the same
place, the system becomes more confident in a match, a Times
\href{https://www.nytimes.com/2019/12/17/technology/china-surveillance.html}{investigation}
found.

In countries with stronger civil liberties laws, the shortcomings of
facial recognition have proved problematic, particularly for systems
intended to
\href{https://www.orlandoweekly.com/Blogs/archives/2019/07/18/orlando-cancels-amazon-rekognition-capping-15-months-of-glitches-and-controversy}{spot
criminals in a crowd}.
\href{https://48ba3m4eh2bf2sksp43rq8kk-wpengine.netdna-ssl.com/wp-content/uploads/2019/07/London-Met-Police-Trial-of-Facial-Recognition-Tech-Report.pdf}{A
study of one such program} in London, which has an extensive network of
CCTV cameras, found that of the 42 matches the tool suggested during
tests, only eight were verifiably correct.

Current and former Pinellas County officials said they weren't
surprised. ``If you're going to get into bank robberies and convenience
store robberies, no --- no, it doesn't work that well,'' said Jim Main,
who handled technical aspects of the facial recognition program for the
sheriff's office until he retired in 2014. ``You can't ask, like:
`Please stop for a second. Let me get your photo.'''

Kitty Bennett contributed research.

Advertisement

\protect\hyperlink{after-bottom}{Continue reading the main story}

\hypertarget{site-index}{%
\subsection{Site Index}\label{site-index}}

\hypertarget{site-information-navigation}{%
\subsection{Site Information
Navigation}\label{site-information-navigation}}

\begin{itemize}
\tightlist
\item
  \href{https://help.nytimes.com/hc/en-us/articles/115014792127-Copyright-notice}{©~2020~The
  New York Times Company}
\end{itemize}

\begin{itemize}
\tightlist
\item
  \href{https://www.nytco.com/}{NYTCo}
\item
  \href{https://help.nytimes.com/hc/en-us/articles/115015385887-Contact-Us}{Contact
  Us}
\item
  \href{https://www.nytco.com/careers/}{Work with us}
\item
  \href{https://nytmediakit.com/}{Advertise}
\item
  \href{http://www.tbrandstudio.com/}{T Brand Studio}
\item
  \href{https://www.nytimes.com/privacy/cookie-policy\#how-do-i-manage-trackers}{Your
  Ad Choices}
\item
  \href{https://www.nytimes.com/privacy}{Privacy}
\item
  \href{https://help.nytimes.com/hc/en-us/articles/115014893428-Terms-of-service}{Terms
  of Service}
\item
  \href{https://help.nytimes.com/hc/en-us/articles/115014893968-Terms-of-sale}{Terms
  of Sale}
\item
  \href{https://spiderbites.nytimes.com}{Site Map}
\item
  \href{https://help.nytimes.com/hc/en-us}{Help}
\item
  \href{https://www.nytimes.com/subscription?campaignId=37WXW}{Subscriptions}
\end{itemize}
