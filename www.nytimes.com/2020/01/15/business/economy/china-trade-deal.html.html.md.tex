Sections

SEARCH

\protect\hyperlink{site-content}{Skip to
content}\protect\hyperlink{site-index}{Skip to site index}

\href{https://www.nytimes.com/section/business/economy}{Economy}

\href{https://myaccount.nytimes.com/auth/login?response_type=cookie\&client_id=vi}{}

\href{https://www.nytimes.com/section/todayspaper}{Today's Paper}

\href{/section/business/economy}{Economy}\textbar{}Trump Signs China
Trade Deal, Putting Economic Conflict on Pause

\url{https://nyti.ms/2NvyD0j}

\begin{itemize}
\item
\item
\item
\item
\item
\item
\end{itemize}

Advertisement

\protect\hyperlink{after-top}{Continue reading the main story}

Supported by

\protect\hyperlink{after-sponsor}{Continue reading the main story}

\hypertarget{trump-signs-china-trade-deal-putting-economic-conflict-on-pause}{%
\section{Trump Signs China Trade Deal, Putting Economic Conflict on
Pause}\label{trump-signs-china-trade-deal-putting-economic-conflict-on-pause}}

An initial pact, cooling tensions in an election year, follows months of
escalating tariffs and a trade war that seemed as if it would never end.

\includegraphics{https://static01.nyt.com/images/2020/01/15/business/15dc-chinadeal/15dc-chinadeal-videoSixteenByNine3000.jpg}

\href{https://www.nytimes.com/by/ana-swanson}{\includegraphics{https://static01.nyt.com/images/2018/12/10/multimedia/author-ana-swanson/author-ana-swanson-thumbLarge.png}}\href{https://www.nytimes.com/by/alan-rappeport}{\includegraphics{https://static01.nyt.com/images/2018/06/12/multimedia/author-alan-rappeport/author-alan-rappeport-thumbLarge-v2.png}}

By \href{https://www.nytimes.com/by/ana-swanson}{Ana Swanson} and
\href{https://www.nytimes.com/by/alan-rappeport}{Alan Rappeport}

\begin{itemize}
\item
  Published Jan. 15, 2020Updated June 23, 2020
\item
  \begin{itemize}
  \item
  \item
  \item
  \item
  \item
  \item
  \end{itemize}
\end{itemize}

WASHINGTON --- President Trump signed an
\href{https://www.nytimes.com/2020/06/23/business/economy/trump-navarro-china-trade-deal.html}{initial
trade deal with China} on Wednesday, bringing
\href{https://www.nytimes.com/2020/01/15/upshot/china-us-trade-peace.html}{the
first chapter} of a protracted and economically damaging fight with one
of the world's largest economies to a close.

The pact is intended to open Chinese markets to more American companies,
increase farm and energy exports and
\href{https://www.nytimes.com/2020/01/14/business/economy/trump-china-trade-deal.html}{provide
greater protection} for American technology and trade secrets. China has
committed to buying an additional \$200 billion worth of American goods
and services by 2021 and is expected to ease some of the tariffs it has
placed on American products.

But the agreement preserves the bulk of the tariffs that Mr. Trump has
placed on \$360 billion worth of
\href{https://www.nytimes.com/2018/08/21/business/economy/trump-china-tariffs-consumers.html}{Chinese
goods}, and it maintains the threat of additional punishment if Beijing
does not live up to the terms of the deal.

``Today we take a momentous step, one that has never been taken before
with China toward a future of fair and reciprocal trade with China,''
Mr. Trump said at a ceremony at the White House. ``Together we are
righting the wrongs of the past.''

The deal caps more than two years of tense negotiations and
\href{https://www.nytimes.com/2019/08/23/business/china-tariffs-trump.html}{escalating
threats} that at times seemed destined to plunge the United States and
China into a permanent economic war. Mr. Trump, who campaigned for
president in 2016 on a promise to get tough on China, pushed his
negotiators to rewrite trade terms that he said had destroyed American
industry and jobs, and he imposed record tariffs on Chinese goods in a
gamble to get Beijing to accede to his demands.

``As a candidate for president, I vowed strong action,'' Mr. Trump said.
``Unlike those who came before me, I kept my promise.''

The agreement is a significant turning point in American trade policy
and the types of free-trade agreements that the United States has
typically supported. Rather than lowering tariffs to allow for the flow
of goods and services to meet market demand, this deal leaves a record
level of tariffs in place and forces China to buy \$200 billion worth of
specific products within two years.

To Mr. Trump and other supporters, the approach corrects for past trade
deals that enabled corporate outsourcing and led to lost jobs and
industries. To critics, it is the type of managed trade approach that
the United States has long criticized, especially with regard to China
and its control over its economy.

While other presidents have tried to change China's economic approach,
Mr. Trump has leaned into it. The agreement stipulates that ``China
shall ensure'' that its purchases meet the \$200 billion figure by 2021,
all but guaranteeing an export boom as Mr. Trump heads into the 2020
election.

``Although the administration claims it wants to enhance market forces
in China, the purchase commitments hailed by the president will only
strengthen the role of the state in the economy,'' said Daniel Price, a
former George W. Bush administration official and the managing director
of Rock Creek Global Advisors.

The president's approach may pay off politically. He will head into a
re-election campaign with a commitment from China to strengthen its
intellectual-property protections, make large purchases of American
products and pursue other economic changes that will benefit American
business.

China's leader, Xi Jinping, said in a message conveyed to Mr. Trump that
the deal is ``beneficial to both China, the U.S. --- and the world.''
Mr. Xi also said the agreement showed that the two countries, ``based on
equality and mutual respect, through dialogue and consultations,'' can
find proper and effective solutions to problems.

At a lavish White House ceremony crowded with cabinet members, lawmakers
and executives from America's biggest companies, Mr. Trump seized on the
signing as a counterweight to impeachment proceedings that were taking
place across town, where lawmakers were about to vote to approve House
prosecutors for a Senate trial.

``They have a hoax going on over there --- let's take care of it,'' he
said.

But the agreement has plenty of critics in both parties, who say that
Mr. Trump's tactics have been economically damaging and that the deal
leaves many important economic issues unresolved.

Those include cybersecurity and China's tight controls over how
companies handle data and cloud computing. China rejected demands that
the text include promises to refrain from hacking American companies,
insisting it was not a trade issue.

And the deal does little to resolve more pernicious structural issues
surrounding China's approach, particularly its pattern of subsidizing
and supporting crucial industries that compete with American companies,
like solar energy and steel. American businesses blame those economic
practices for allowing cheap Chinese goods to flood the United States.

``A ceremony at the White House can't hide the stark truth about the
`Phase 1' China trade deal: The deal does absolutely nothing to curtail
China's subsidies to its manufacturers,'' Scott Paul, president of the
Alliance for American Manufacturing, which includes manufacturers and
the United Steelworkers union,
\href{https://twitter.com/ScottPaulAAM/status/1217491025079492608}{said
in a tweet}. ``All those `forgotten men and women' in U.S. factories
have, once again, been forgotten.''

The administration has said it will address some of these changes in
Phase 2 of the negotiations and is keeping tariffs in place in part to
maintain leverage for the next round of talks. Mr. Trump said he would
remove all tariffs if the two sides reach agreement on the next phase.

``I will agree to take those tariffs off if we're able to do Phase 2,''
he said.

But Mr. Trump has already kicked the deadline for another agreement past
the November election, and there is deep skepticism that the two
countries will reach another deal anytime soon.

As part of the deal, Mr. Trump agreed to reduce the rate on tariffs
imposed in September and forgo additional import taxes in the future.
But the United States will continue to maintain tariffs covering 65
percent of American imports from China,
\href{https://www.piie.com/blogs/trade-and-investment-policy-watch/phase-one-china-deal-steep-tariffs-are-new-normal}{according
to tracking} by Chad Bown, a senior fellow at the Peterson Institute of
International Economics. That leaves the United States with an overall
tariff rate higher than that of any other advanced nation, as well as
China, India and Turkey.

China will still tax 57 percent of imports from the United States in
retaliation, according to Mr. Bown, though it's possible some of those
levies may be waived in the weeks to come.

The two sides did not immediately distribute copies of the agreement in
Chinese, raising the question of whether translation issues had been
fully resolved and whether the final text would be as demanding of the
Beijing government in the Chinese version as in the English version.

``We also need to be sure that the wording of the agreement is the same
in both the Chinese and English versions --- history has shown that
mismatches become easily exploited loopholes,'' said Ker Gibbs, the
president of the American Chamber of Commerce in Shanghai.

\includegraphics{https://static01.nyt.com/images/2020/01/15/us/politics/15dc-chinadeal2/15dc-chinadeal2-articleLarge.jpg?quality=75\&auto=webp\&disable=upscale}

While updates about the trade war transfixed investors for much of the
last two years, the official signing of the deal was greeted with
something of a shrug. The S\&P 500 rose roughly 0.2 percent. A gauge of
semiconductor companies, which have been particularly sensitive to the
trade war, fell more than 1 percent.

The deal came under fire from top Democrats, including Senator Chuck
Schumer of New York, who criticized the agreement for failing to address
China's state-owned enterprises and industrial subsidies. He suggested
that President Xi was privately laughing at the United States and that
China has ``taken President Trump to the cleaners.''

``This Phase 1 deal is an extreme disappointment to me and to millions
and millions of Americans who want to see us make China play fair,'' Mr.
Schumer said on the Senate floor.

Wendy Cutler, a vice president at the Asia Society Policy Institute who
negotiated trade pacts for the Obama administration, called the gains
``meaningful, but modest.''

``Because the United States was willing to compromise with China and not
press them on the most difficult issues, they were able to reach
positive ground,'' she said.

The trade deal contains
\href{https://www.nytimes.com/2020/01/14/business/economy/trump-china-trade-deal.html}{a
variety of victories} for American industry, including opening up
markets for biotechnology, beef and poultry. Banks, insurers,
\href{https://www.nytimes.com/2019/08/27/business/china-cheap-drug-imports.html}{drug
companies} and the energy industry are also big beneficiaries.

China has also agreed not to force American companies to hand over their
technology as a condition of doing business there, under penalty of
further tariffs. And it will refrain from directing its companies to
obtain sensitive foreign technology through acquisitions. The agreement
also includes a pledge by both countries not to devalue their currencies
to
\href{https://www.nytimes.com/2019/08/05/business/china-currency.html}{gain
an advantage} in export markets.

The president trumpeted many of China's concessions during the signing
ceremony, singling out audience members who will benefit. He called out
a litany of Wall Street executives, many of whom have been pressing for
greater access to China's financial services market, including Stephen
A. Schwarzman, the chief executive of the private equity firm the
Blackstone Group, and Kenneth C. Griffin, the billionaire founder of the
hedge fund Citadel. He also mentioned the chiefs of Boeing, Citibank,
Visa and the American International Group, and the chip makers Micron
and Qualcomm.

Referring to the energy purchases in the agreement, Mr. Trump told
Senator Joni Ernst, the Iowa Republican, who was in attendance: ``You
got ethanol, so you can't be complaining.''

But those victories have come at a heavy price. The uncertainty created
by Mr. Trump's tariff threats and approach to trade has weighed on the
economy,
\href{https://www.nytimes.com/2020/01/06/business/economy/trade-war-tariffs.html}{raising
prices for businesses and consumers}, delaying corporate investments and
slowing growth around the globe. Businesses with exposure to China, like
Deere \& Company and Caterpillar, have laid off some workers and lowered
revenue expectations, in part citing the trade war.

And other sources of tension remain in the United States-China
relationship. The Trump administration has taken a tougher approach to
scrutinizing Chinese investments and technology purchases for national
security threats, including blacklisting Chinese companies like Huawei,
the telecom firm.

``I think it's maybe a useful pause in the downward spiral of U.S.-China
relations,'' Susan Shirk, a professor at the University of California,
San Diego, said of the trade deal.

Keith Bradsher contributed reporting from Beijing

Advertisement

\protect\hyperlink{after-bottom}{Continue reading the main story}

\hypertarget{site-index}{%
\subsection{Site Index}\label{site-index}}

\hypertarget{site-information-navigation}{%
\subsection{Site Information
Navigation}\label{site-information-navigation}}

\begin{itemize}
\tightlist
\item
  \href{https://help.nytimes.com/hc/en-us/articles/115014792127-Copyright-notice}{©~2020~The
  New York Times Company}
\end{itemize}

\begin{itemize}
\tightlist
\item
  \href{https://www.nytco.com/}{NYTCo}
\item
  \href{https://help.nytimes.com/hc/en-us/articles/115015385887-Contact-Us}{Contact
  Us}
\item
  \href{https://www.nytco.com/careers/}{Work with us}
\item
  \href{https://nytmediakit.com/}{Advertise}
\item
  \href{http://www.tbrandstudio.com/}{T Brand Studio}
\item
  \href{https://www.nytimes.com/privacy/cookie-policy\#how-do-i-manage-trackers}{Your
  Ad Choices}
\item
  \href{https://www.nytimes.com/privacy}{Privacy}
\item
  \href{https://help.nytimes.com/hc/en-us/articles/115014893428-Terms-of-service}{Terms
  of Service}
\item
  \href{https://help.nytimes.com/hc/en-us/articles/115014893968-Terms-of-sale}{Terms
  of Sale}
\item
  \href{https://spiderbites.nytimes.com}{Site Map}
\item
  \href{https://help.nytimes.com/hc/en-us}{Help}
\item
  \href{https://www.nytimes.com/subscription?campaignId=37WXW}{Subscriptions}
\end{itemize}
