Sections

SEARCH

\protect\hyperlink{site-content}{Skip to
content}\protect\hyperlink{site-index}{Skip to site index}

\href{https://www.nytimes.com/section/world/europe}{Europe}

\href{https://myaccount.nytimes.com/auth/login?response_type=cookie\&client_id=vi}{}

\href{https://www.nytimes.com/section/todayspaper}{Today's Paper}

\href{/section/world/europe}{Europe}\textbar{}Europe's Gamble: Can It
Save Iran Deal by Threatening to Kill It?

\url{https://nyti.ms/2tnJ6E9}

\begin{itemize}
\item
\item
\item
\item
\item
\end{itemize}

Advertisement

\protect\hyperlink{after-top}{Continue reading the main story}

Supported by

\protect\hyperlink{after-sponsor}{Continue reading the main story}

news analysis

\hypertarget{europes-gamble-can-it-save-iran-deal-by-threatening-to-kill-it}{%
\section{Europe's Gamble: Can It Save Iran Deal by Threatening to Kill
It?}\label{europes-gamble-can-it-save-iran-deal-by-threatening-to-kill-it}}

A risky strategy to keep the nuclear agreement alive could backfire.
Badly. But no one else is even trying, Europeans argue.

\includegraphics{https://static01.nyt.com/images/2020/01/15/world/15iran-europe/merlin_167116737_9b8e04c8-5a46-4e66-8cb4-319ace91ddfe-articleLarge.jpg?quality=75\&auto=webp\&disable=upscale}

\href{https://www.nytimes.com/by/steven-erlanger}{\includegraphics{https://static01.nyt.com/images/2018/10/10/multimedia/author-steven-erlanger/author-steven-erlanger-thumbLarge.png}}

By \href{https://www.nytimes.com/by/steven-erlanger}{Steven Erlanger}

\begin{itemize}
\item
  Jan. 15, 2020
\item
  \begin{itemize}
  \item
  \item
  \item
  \item
  \item
  \end{itemize}
\end{itemize}

BRUSSELS --- Europe is gambling on keeping the 2015 Iran nuclear deal
alive by threatening to destroy it --- a risky, oddly timed strategy
that could backfire badly, European officials and analysts say.

The
\href{https://www.nytimes.com/2020/01/14/world/europe/iran-nuclear-deal.html?searchResultPosition=1}{decision
by France, Germany and Britain}on Tuesday to challenge Iran's breaches
of the nuclear agreement and trigger what is known as the dispute
resolution mechanism starts a clock that the Europeans may not be able
to control, subject to unpredictable actions by the leaders of both Iran
and the United States.

Already the move has angered Iran, which contends its breaches of the
accord are justified and that the Europeans are bending toward President
Trump and his ``maximum pressure'' campaign of sanctions on Tehran.

The Europeans insist otherwise. But watching Iran and the United States
head for a possible military escalation, they are trying at least to
avoid an outcome in which Iran moves down the North Korean road toward a
nuclear bomb.

``If there is a European strategy here, it's essentially to try to buy
time,'' said Nathalie Tocci, a former adviser to the European Union's
former foreign policy chief, Federica Mogherini, who helped negotiate
the 2015 deal with Iran.

``The optimistic scenario is that they've done this to keep Trump happy
and hope to stretch out the dispute process until the November U.S.
elections,'' Ms. Tocci said. The Europeans, she said, are wagering that
``despite all the pressure on Iran, it won't use the freedom it's
granted itself'' to enrich uranium to bomb-ready levels.

The Europeans also are hoping to induce the United States and Iran to
somehow engage in negotiations on an enhanced deal that Mr. Trump can
call his own, even if it differs little from the current one, negotiated
by President Barack Obama, that limited Iran's nuclear activities --- a
deal Mr. Trump has called the worst in history.

``But it's a very high-risk strategy,'' Ms. Tocci said. ``It's hard to
see Tehran playing ball all the way until November.''

Wendy R. Sherman, a key American negotiator of the 2015 accord, also
described the use of the dispute mechanism as ``incredibly risky'' and
said ``it will increase the likelihood of the demise'' of the deal.

It is hard to see those around Mr. Trump, who have opposed the nuclear
deal and supported his maximum pressure campaign, granting Iran any
concession to get talks started, said Ian Bond, director of foreign
policy at the\href{https://www.cer.eu/}{Center for European Reform}.

Mr. Trump's subordinates appear to believe their strategy is working.
Iran's government and economy are weakened, they note. And they say the
United States killing of a top Iranian commander, Maj. Gen. Qassim
Suleimani, in Baghdad nearly two weeks ago has deterred Iran in the
region, rather than prompting further attempts to expel the Americans
from Iraq and Syria.

But others,
like\href{https://www.crisisgroup.org/who-we-are/people/robert-malley-0}{Robert
Malley}, an American who helped negotiate the nuclear deal and now runs
the International Crisis Group, say they are not so sure. They expect
further retaliation from Iran and from the Iraqi militias it supports,
including one led by a commander killed alongside General Suleimani.

In its zeal to pressure the Europeans, a Trump administration official
even threatened them with 25 percent tariffs on automobile exports if
they did not invoke the dispute provision with Iran, according to a
European official --- which would be an extraordinary use of economic
leverage for a foreign policy goal. The warning, first
\href{https://www.washingtonpost.com/world/national-security/days-before-europeans-warned-iran-of-nuclear-deal-violations-trump-secretly-threatened-to-impose-25percent-tariff-on-european-autos-if-they-didnt/2020/01/15/0a3ea8ce-37a9-11ea-a01d-b7cc8ec1a85d_story.html}{reported
by The Washington Post}, was conveyed in a single phone call and was
regarded by the Europeans as counterproductive, the European official
said.

An indication of the American administration's mood came on Wednesday
from Treasury Secretary Steven Mnuchin, who said that he and Secretary
of State Mike Pompeo believed United Nations sanctions on Iran would be
swiftly reimposed now that France, Britain and Germany had triggered the
dispute-resolution mechanism. The Europeans made that move in response
to
\href{https://www.nytimes.com/2020/01/05/world/middleeast/trump-iran-nuclear-agreement.html?searchResultPosition=2}{Iran's
declarations}that it would no longer honor the nuclear accord's limits
on Iranian enrichment of uranium --- potential fuel for a bomb.

\includegraphics{https://static01.nyt.com/images/2020/01/15/world/15iran-europe2/merlin_167172009_c6b38389-0f25-489b-a45b-f44f78bee7a0-articleLarge.jpg?quality=75\&auto=webp\&disable=upscale}

``I've had very direct discussions --- as well as Secretary Pompeo has
--- with our counterparts,'' Mr. Mnuchin told CNBC. ``We look forward to
working with them quickly and would expect that the U.N. sanctions will
snap back into place.''

But that outcome, which would likely terminate the agreement, is
precisely what the Europeans say they are trying to avoid. There is
nervousness that Prime Minister Boris Johnson of Britain, with his eye
on a post-Brexit trade deal with Washington, might crack, but European
officials consider that possibility unlikely.

Mr. Johnson and his government support the nuclear deal and collaborated
on
\href{https://www.diplomatie.gouv.fr/en/country-files/iran/news/article/joint-statement-by-the-foreign-ministers-of-france-germany-and-the-united}{a
statement} with France and Germany on Tuesday that said ``we are not
joining a campaign to implement maximum pressure on Iran.'' On Tuesday,
Mr. Johnson urged new talks with Washington and Tehran to try to
negotiate a ``Trump deal'' to supplant the current one.

Under the dispute mechanism, explained Ellie Geranmayeh, an Iran expert
with the European Council on Foreign Relations, time limits on
discussions can be extended by consent. Any party to the deal can go
directly to the United Nations Security Council to request the
re-imposition of United Nations sanctions, but no one is expected to do
that, unless Mr. Johnson unexpectedly capitulates to Washington.

John R. Bolton, Mr. Trump's former national security adviser, has argued
that the Americans can make this request themselves. The Europeans
disagree, saying Mr. Trump's repudiation of the nuclear accord last year
means the United States is no longer a party to it.

Still, Ms. Geranmayeh said, both Iran and the United States are
unpredictable. ``If there is no diplomacy or something else poisons it
militarily, the Europeans have triggered a clock that could end up more
quickly at the Security Council,'' she said.

The political unrest in Iran, following the killing of General Suleimani
and Iran's accidental shooting down of a civilian airliner, creates
enormous uncertainty. The government in Iran is cracking down on
protests, and with parliamentary elections next month, hard-line
rhetoric is bound to increase.

While Iran insists its nuclear work will remain peaceful, its decisions
to disregard the nuclear deal's limits on both the volume and purity of
Iranian nuclear fuel have raised worries that the country could amass
enough enriched uranium to create a bomb in a matter of months.

``I think the attack on Suleimani will make the Iranians want to
accelerate their nuclear program,'' Mr. Bond said. ``They've seen Kim
Jong-un and the lesson from Trump, which is if you have nuclear weapons
you can get love letters from the president, and if you don't, your
generals can get killed.''

Iran's first reaction has been relative calm, Ms. Geranmayeh said. ``But
given the stalemate and the European inability to deliver the economic
benefits it promised, Iran could end up by lashing out, and could do so
on the nuclear side,'' she said.

Image

A demonstration outside the British Embassy in Tehran on
Sunday.Credit...Arash Khamooshi for The New York Times

For example, Iran could restrict access to inspectors from the
International Atomic Energy Agency or even expel them. ``So it could all
be worse, and you can't separate that from the U.S.-Iran
confrontation,'' she said.

Which is why the three European countries are expected to continue to
focus on finding an opportunity for a diplomatic breakthrough, as
President\href{https://www.nytimes.com/2019/09/30/world/middleeast/iran-trump-rouhani-call-macron.html?searchResultPosition=1}{Emmanuel
Macron of France has been trying to do}with Washington and Iran since
the late summer.

Any new talks would face extraordinary obstacles. Iran says they would
require sanctions relief as a precondition, and the Trump administration
is instead increasing sanctions on Iran.

Some European officials have speculated that President Vladimir V. Putin
of Russia, who has enjoyed some diplomatic successes where Washington
has created vacuums, might try to mediate himself.

Mr. Bond said Russia was happy to be aligned with the Europeans on at
least this issue, and has benefited from Europe's split on Iran with the
United States. And Moscow, which is a party to the nuclear deal, has
made clear it does not want a nuclear-armed Iran.

François Heisbourg, a French defense analyst, said that he could imagine
a phased negotiation --- first to determine what sanctions Washington
could lift as a sign of good faith, then some high-level meeting, much
like Mr. Trump had with Mr. Kim.

While Mr. Heisbourg expressed skepticism, he praised the European
countries for at least trying to de-escalate the tension while
attempting to keep Iran from going down the North Korean path, risking a
real war with the United States or Israel.

``This is an unholy mess,'' he said. ``The chance of the Europeans
succeeding in a meaningful way to improve matters is very low. But we
are in a position to try this. And no one else is even trying.''

Michael Crowley contributed reporting from Washington.

Advertisement

\protect\hyperlink{after-bottom}{Continue reading the main story}

\hypertarget{site-index}{%
\subsection{Site Index}\label{site-index}}

\hypertarget{site-information-navigation}{%
\subsection{Site Information
Navigation}\label{site-information-navigation}}

\begin{itemize}
\tightlist
\item
  \href{https://help.nytimes.com/hc/en-us/articles/115014792127-Copyright-notice}{©~2020~The
  New York Times Company}
\end{itemize}

\begin{itemize}
\tightlist
\item
  \href{https://www.nytco.com/}{NYTCo}
\item
  \href{https://help.nytimes.com/hc/en-us/articles/115015385887-Contact-Us}{Contact
  Us}
\item
  \href{https://www.nytco.com/careers/}{Work with us}
\item
  \href{https://nytmediakit.com/}{Advertise}
\item
  \href{http://www.tbrandstudio.com/}{T Brand Studio}
\item
  \href{https://www.nytimes.com/privacy/cookie-policy\#how-do-i-manage-trackers}{Your
  Ad Choices}
\item
  \href{https://www.nytimes.com/privacy}{Privacy}
\item
  \href{https://help.nytimes.com/hc/en-us/articles/115014893428-Terms-of-service}{Terms
  of Service}
\item
  \href{https://help.nytimes.com/hc/en-us/articles/115014893968-Terms-of-sale}{Terms
  of Sale}
\item
  \href{https://spiderbites.nytimes.com}{Site Map}
\item
  \href{https://help.nytimes.com/hc/en-us}{Help}
\item
  \href{https://www.nytimes.com/subscription?campaignId=37WXW}{Subscriptions}
\end{itemize}
