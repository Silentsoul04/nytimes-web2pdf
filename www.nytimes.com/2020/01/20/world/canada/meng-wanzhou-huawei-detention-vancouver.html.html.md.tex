Sections

SEARCH

\protect\hyperlink{site-content}{Skip to
content}\protect\hyperlink{site-index}{Skip to site index}

\href{https://www.nytimes.com/section/world/canada}{Canada}

\href{https://myaccount.nytimes.com/auth/login?response_type=cookie\&client_id=vi}{}

\href{https://www.nytimes.com/section/todayspaper}{Today's Paper}

\href{/section/world/canada}{Canada}\textbar{}Extradition Hearings Begin
for Meng Wanzhou, Huawei Officer Held in Canada

\url{https://nyti.ms/38lYr76}

\begin{itemize}
\item
\item
\item
\item
\item
\end{itemize}

Advertisement

\protect\hyperlink{after-top}{Continue reading the main story}

Supported by

\protect\hyperlink{after-sponsor}{Continue reading the main story}

\hypertarget{extradition-hearings-begin-for-meng-wanzhou-huawei-officer-held-in-canada}{%
\section{Extradition Hearings Begin for Meng Wanzhou, Huawei Officer
Held in
Canada}\label{extradition-hearings-begin-for-meng-wanzhou-huawei-officer-held-in-canada}}

A Chinese executive wanted by the U.S. may be passing her days in a
Vancouver mansion, but friends say she is ``trapped'' in a gilded cage.

\includegraphics{https://static01.nyt.com/images/2020/01/20/world/20canada-meng/merlin_167287725_23fd2062-701b-4c72-bb0e-6ca44f87d437-articleLarge.jpg?quality=75\&auto=webp\&disable=upscale}

\href{https://www.nytimes.com/by/dan-bilefsky}{\includegraphics{https://static01.nyt.com/images/2019/01/10/multimedia/author-dan-bilefsky/author-dan-bilefsky-thumbLarge.png}}

By \href{https://www.nytimes.com/by/dan-bilefsky}{Dan Bilefsky}

\begin{itemize}
\item
  Jan. 20, 2020
\item
  \begin{itemize}
  \item
  \item
  \item
  \item
  \item
  \end{itemize}
\end{itemize}

\href{https://cn.nytimes.com/world/20200121/meng-wanzhou-huawei-detention-vancouver/}{阅读简体中文版}\href{https://cn.nytimes.com/world/20200121/meng-wanzhou-huawei-detention-vancouver/zh-hant/}{閱讀繁體中文版}

VANCOUVER, British Columbia --- She has passed her days painting
flowers, conferring with her lawyers, reading books and improving her
English, ensconced in two different multimillion-dollar mansions in
exclusive sections of Vancouver.

But on Friday, after the city experienced a rare snowstorm, Meng
Wanzhou, the Huawei chief financial officer detained in Vancouver and
awaiting an extradition hearing on fraud charges, could be seen
playfully throwing a snowball outside her house.

Then she spotted a photographer. Her smile turned into a frown.

In the 13 months since Ms. Meng, 47, was arrested at the Vancouver
airport, her life has been circumscribed by the terms of her \$10
million bail. On Monday, it entered a new phase when her extradition
hearing formally began.

This part of
\href{https://www.nytimes.com/2020/01/19/world/canada/19meng-wanzhou-extradition-huawei.html}{the
hearing}, which could last a week, will examine whether the crime Ms.
Meng is accused of constitutes a crime in Canada, a prerequisite under
Canadian law for her extradition to proceed. This is known as the legal
concept of ``double criminality.''

The United States, in a January 2019
\href{https://www.nytimes.com/2019/01/28/us/politics/meng-wanzhou-huawei-iran.html}{indictment},
charged, among other things, that Ms. Meng had deceived four banks into
clearing transactions in Iran through Skycom, a subsidiary, violating
American sanctions against Iran.

Prosecutors say Ms. Meng lied to representatives of the bank HSBC in
2013 about Huawei's relationship with Skycom, telling them it was a
partner when it was, in fact, a subsidiary. That put HSBC at risk of
sanctions.

On Monday, at the beginning of the proceedings, Richard Peck, a member
of Ms. Meng's defense team, said the case should be dismissed because it
was founded on the accusation of breach of United States sanctions
against Iran --- sanctions that Canada does not accept. The claim of
fraud, he said, was ``a facade.''

In the courtroom, Ms. Meng, sitting in a bulletproof defendant's box at
the back of the court, appeared composed and smiling as she conferred
with her legal team.

If her case goes ahead, the court will also hear her lawyers' arguments
that her rights were violated when she was arrested.

Ms. Meng's arrest has pushed Canada into the center of a tense
diplomatic battle between China and the United States. President Trump
once said he might consider
\href{https://www.nytimes.com/2018/12/12/us/politics/trump-meng-wanzhou-huawei-extradition.html?module=inline}{interceding
in the case} if that helped him reach a trade deal with China.

In what many called a retaliation for the arrest of Ms. Meng --- one of
China's most prominent business executives --- China has arrested two
Canadians,
\href{https://www.nytimes.com/2019/03/04/world/asia/china-canada-michael-kovrig-huawei.html?module=inline}{accused
them of espionage} and detained them in secret jails, deprived of access
to lawyers and their families.

It is unclear when a final ruling in Ms. Meng's extradition case will be
made. In the meantime, she could face many more months in Vancouver.

\includegraphics{https://static01.nyt.com/images/2020/01/20/world/20canada-meng2/merlin_167283477_52a7c02a-2030-4425-8259-be69978c07f6-articleLarge.jpg?quality=75\&auto=webp\&disable=upscale}

\href{https://huawei.eu/profile/vincent-peng}{Vincent Pang}, a Huawei
board member and longtime friend of Ms. Meng's, visits her about once a
month. He said in an interview on Friday that he and the company
``believe she is innocent, and so does she.''

Nevertheless, Mr. Pang said, Ms. Meng is facing enormous emotional
stress because of the uncertainty about her future and from being
separated for long periods from her four children, all of whom live in
China.

He said her two youngest children were in Vancouver schools at the time
of her arrest, which upended their lives. ``She feels sorry for her
family,'' Mr. Pang said.

According to one of the security guards monitoring her house, her family
came to see her during the Christmas holidays. Her mother and husband
have taken turns keeping her company.

Ms. Meng lives in a seven-bedroom gated mansion in Vancouver's exclusive
Shaughnessy neighborhood, valued at about 14 million Canadian dollars.

Under the terms of her bail, she is able to travel relatively freely
about the city, although she has an 11 p.m. curfew and a GPS tracker
around her ankle. Her bail conditions, which give her the freedom to go
shopping, are in stark contrast with the two Canadians detained in
China, and
\href{https://www.nytimes.com/2019/03/04/world/canada/huawei-canada-meng-wanzhou.html}{have
been criticized} in Canada.

But she seldom leaves the house, said one of two unarmed guards, who was
standing near a makeshift tent outside Ms. Meng's mansion on Friday.
When she does, he said, it is usually to meet with her lawyers.

Image

Ms. Meng's home is in the quiet Shaughnessy neighborhood of
Vancouver.Credit...Jackie Dives for The New York Times

Mr. Pang said that while Ms. Meng lived in a mansion, it was effectively
a gilded cage. He said she had lost a lot of weight.

``It's not about how big her home is but what's in her heart,'' he said,
adding, ``She feels trapped.''

Despite entreaties by him and others friends to go out more, Ms. Meng
prefers to stay at home, Mr. Pang said. ``If you're detained, you don't
feel like shopping or eating,'' he said.

Still, on one occasion, Ms. Meng was spotted browsing at Holt Renfrew,
the luxury retailer. And on China's National Day, she wore a bright red
Gucci dress to court, adorned with an enamel Chinese flag pin.

Ms. Meng has hired an A-list legal team led by
\href{https://martinandassociates.ca/experience/}{David Martin}, one of
Canada's most high-powered lawyers, who specializes in white-collar
crime. He has argued cases at the country's Supreme Court and has
defended members of foreign governments and ultrarich captains of
industry seeking to avoid extradition.

While Ms. Meng retains her chief financial officer title at Huawei and
continues to take a role in the business, Mr. Pang said her
responsibilities had been significantly scaled back, given that
Shenzhen, where Huawei has its headquarters, is 16 hours ahead of
Vancouver.

Image

Ms. Meng, right, throwing a snowball in front her house after a major
snowfall last week.Credit...Jackie Dives for The New York Times

In December, on the first anniversary of her arrest, Ms. Meng wrote a
reflective
\href{https://www.huawei.com/en/facts/voices-of-huawei/your-warmth-is-a-beacon-that-lights-my-way-forward}{letter},
published on Huawei's website, in which she said she had experienced
moments of ``fear,'' ``pain'' and ``disappointment'' --- but also
acceptance.

``When I was in Shenzhen, time used to pass by very quickly,'' she
wrote. She continued, ``Right now, time seems to pass slowly.''

After the letter was released, there was a backlash on social media
against Huawei and sly allusions to the story of a former company
employee, Li Hongyuan, who was jailed for more than eight months after
he demanded severance pay when his contract wasn't renewed. Mr. Li was
eventually released with no charges, but some compared his detainment
with Ms. Meng's luxurious surroundings.

In recent interviews with international news outlets, Ms. Meng's father,
Ren Zhengfei, who is Huawei's billionaire founder, suggested that her
detention had brought the two closer together.

Before her arrest, Mr. Ren said, Ms. Meng rarely called or texted him.
Now the two speak regularly by phone.

When Mr. Ren turned 75 in October, Ms. Meng posted a
\href{http://xinsheng.huawei.com/cn/index.php?app=forum\&mod=Detail\&act=index\&id=4471927\&p=1}{handwritten
birthday greeting} on social media. The post included a photo of her
standing outdoors, wearing her ankle monitor. The letter was signed
``Piggy'' --- a nickname apparently given to her when she was a chubby
child.

Given that Ms. Meng is wanted by the United States on fraud charges, the
irony has not gone unnoticed among local residents in her neighborhood
that her 8,047-square-foot house is just a few doors down from the
residence of the United States consul general, where an American flag
flaps in the wind.

Image

Ms. Meng is under detention just a few houses down from the American
consul general's residence.Credit...Jackie Dives for The New York Times

A short walk away is the hulking Consulate General of the People's
Republic of China.

Ji Jiahao, 22, came to Vancouver from Shanghai to study English and
lives in a mansion on the same street as Ms. Meng. He said Ms. Meng had
not broken any laws and should be released.

``Canada has always been the little brother of United States,'' he said.
``It does whatever the United States does.''

\href{https://residentdoctorsbc.ca/resident-spotlight-dr-hayden-rubensohn/}{Hayden
Rubensohn}, a doctor who works in the area, said Ms. Meng's bail
conditions showed that Canada was a humane country. But he said they did
spur some resentment in a city with a vast economic gaps.

``It's hard for people to see foreigners living with so much wealth,''
he said, ``when people in Vancouver are struggling to make ends meet.''

Tracy Sherlock and Winston Szeto contributed reporting from Vancouver,
and Raymond Zhong from Beijing.

Advertisement

\protect\hyperlink{after-bottom}{Continue reading the main story}

\hypertarget{site-index}{%
\subsection{Site Index}\label{site-index}}

\hypertarget{site-information-navigation}{%
\subsection{Site Information
Navigation}\label{site-information-navigation}}

\begin{itemize}
\tightlist
\item
  \href{https://help.nytimes.com/hc/en-us/articles/115014792127-Copyright-notice}{©~2020~The
  New York Times Company}
\end{itemize}

\begin{itemize}
\tightlist
\item
  \href{https://www.nytco.com/}{NYTCo}
\item
  \href{https://help.nytimes.com/hc/en-us/articles/115015385887-Contact-Us}{Contact
  Us}
\item
  \href{https://www.nytco.com/careers/}{Work with us}
\item
  \href{https://nytmediakit.com/}{Advertise}
\item
  \href{http://www.tbrandstudio.com/}{T Brand Studio}
\item
  \href{https://www.nytimes.com/privacy/cookie-policy\#how-do-i-manage-trackers}{Your
  Ad Choices}
\item
  \href{https://www.nytimes.com/privacy}{Privacy}
\item
  \href{https://help.nytimes.com/hc/en-us/articles/115014893428-Terms-of-service}{Terms
  of Service}
\item
  \href{https://help.nytimes.com/hc/en-us/articles/115014893968-Terms-of-sale}{Terms
  of Sale}
\item
  \href{https://spiderbites.nytimes.com}{Site Map}
\item
  \href{https://help.nytimes.com/hc/en-us}{Help}
\item
  \href{https://www.nytimes.com/subscription?campaignId=37WXW}{Subscriptions}
\end{itemize}
