Sections

SEARCH

\protect\hyperlink{site-content}{Skip to
content}\protect\hyperlink{site-index}{Skip to site index}

\href{https://myaccount.nytimes.com/auth/login?response_type=cookie\&client_id=vi}{}

\href{https://www.nytimes.com/section/todayspaper}{Today's Paper}

\href{/section/opinion}{Opinion}\textbar{}Jeff Bezos' Phone Hack Should
Terrify Everyone

\url{https://nyti.ms/37oujrL}

\begin{itemize}
\item
\item
\item
\item
\item
\item
\end{itemize}

Advertisement

\protect\hyperlink{after-top}{Continue reading the main story}

\href{/section/opinion}{Opinion}

Supported by

\protect\hyperlink{after-sponsor}{Continue reading the main story}

\hypertarget{jeff-bezos-phone-hack-should-terrify-everyone}{%
\section{Jeff Bezos' Phone Hack Should Terrify
Everyone}\label{jeff-bezos-phone-hack-should-terrify-everyone}}

Those with the most to lose don't always safeguard their privacy very
well. You can do better.

\href{https://www.nytimes.com/by/charlie-warzel}{\includegraphics{https://static01.nyt.com/images/2019/03/15/opinion/charlie-warzel/charlie-warzel-thumbLarge-v3.png}}

By \href{https://www.nytimes.com/by/charlie-warzel}{Charlie Warzel}

Mr. Warzel is an opinion writer at large

\begin{itemize}
\item
  Jan. 24, 2020
\item
  \begin{itemize}
  \item
  \item
  \item
  \item
  \item
  \item
  \end{itemize}
\end{itemize}

\includegraphics{https://static01.nyt.com/images/2020/01/27/opinion/24bezos_iphone/24bezos_iphone-articleLarge-v5.jpg?quality=75\&auto=webp\&disable=upscale}

If the Saudi crown prince, Mohammed bin Salman, wants to chat on
WhatsApp, politely decline.

That's the lesson from a series of reports this week based off a
forensic examination of Jeff Bezos' communications with the crown
prince. The investigation --- conducted at Mr. Bezos' request by FTI
Consulting --- found that his phone had most likely been attacked in
2018 after he received a WhatsApp message from the prince's personal
account. While my colleagues in the Times newsroom have
\href{https://www.nytimes.com/2020/01/22/technology/jeff-bezos-hack-iphone.html}{pieced
together details of the investigation}, there's still a great deal
unknown. And cybersecurity experts have serious questions about FTI's
report, which, according to CyberScoop,
``\href{https://www.cyberscoop.com/jeff-bezos-mbs-hack-fti-report-questions/}{has
not impressed the information security community.}''

Still, the story seems to have everything: from lighthearted,
embarrassingly earnest texts, ``divorced guy'' memes and world leaders
who awkwardly sign their text messages with their full names to the
deeply problematic issue of revenge porn and stealing of private nude
photos. Though it's a gilded example of digital theft, there's something
troubling and relatable about it all. Billionaires, they're just like
us!

Or maybe not. Looked at one way, the attack on Mr. Bezos' phone could be
seen as yet more proof of what my colleague Kara Swisher called
``\href{https://www.nytimes.com/2019/01/18/opinion/amazon-jeff-bezos-affair.html}{the
death of privacy}.'' If the richest man in the world --- the man who
sells listening devices used in millions of homes and whose servers
create the
\href{https://twitter.com/alexstamos/status/1219768146615128064?s=20}{internet's
infrastructure} --- can be hacked, what hope is there for us mere
mortals?

Turns out, there's some. Yes, your personal privacy and security are
constantly under threat. And yes, you should be trying to safeguard
against malware, phishing and bulk data collection. But the Bezos attack
is an example of extremely targeted surveillance, the potentially
expensive and risky kind that is aimed at high-value targets like
executives, government officials, celebrities and billionaires. And as
it turns out, many of those with the most to lose are woefully inept at
safeguarding their privacy.

Not long after the Bezos news broke this week, I spoke to Christopher
Pierson, who founded BlackCloak, a cybersecurity company for
high-net-worth and high-profile individuals --- executives, celebrities
and billionaires. According to Mr. Pierson, few people take their
digital lives as seriously as they should.

``The majority of clients we onboard come on in some kind of hacked
state,'' he told me. ``Their computers are compromised or their login
credentials are available on dark web. Their home camera systems are
accessible to people on internet or their entire home and appliances are
vulnerable and viewable by persons remotely.'' Mr. Pierson suggests
that's in part because high-value targets choose to focus on physical
security over digital and invest in private bodyguards, camera systems
and protections like kidnapping insurance.

How bad is it? ``We see passwords in little black books on desks by the
machines and in files on the computers. We see passwords that are the
same everywhere. We absolutely do not see good use of dual factor
authentication on email, health care and financial accounts. I'd say we
see less than 1 percent of high-net-worth individuals using dual
factor.''

Mr. Pierson said BlackCloak has found more than 82 percent of its
clients' current passwords on the dark web when it ran an initial
search. ``In the case of high-net-worth individuals, the same
compromised password is frequently used by 20 to 40 different accounts
--- some of those are personal, some are in the office.''

What Mr. Pierson describes is low-hanging fruit --- the kind of security
flaws that can quickly be fixed with a little knowledge and attention to
detail. Even then, he said, it takes time for the true nature of
clients' vulnerability to sink in. ``They're shocked when we give them
their password and tell them where we found it, but it doesn't hit as
hard as when we tell them their entire home automation system has been
potentially online and viewable for three or five or eight years,'' he
said.

When it comes to a Bezos-style breach --- potentially at the hands of a
nation-state's intelligence service --- high-profile targets would most
likely be even less prepared. As Mr. Bezos' lengthy investigation into
the 2018 attack shows, it's difficult to get straight answers even when
you have the money and resources to run full forensics.

Of course, it's not just wealth that turns somebody into a person of
interest for hackers. Journalists, government employees, workers at
energy companies and utilities could all be targets for someone. Those
who work for financial companies, airlines, hospitals, universities,
Hollywood studios and tech businesses are all potentially at risk. You
can take steps to secure yourself from corporate data collection by
\href{https://www.nytimes.com/interactive/2019/12/19/opinion/location-tracking-privacy-tips.html}{using
privacy settings on your phone}. And to protect yourself from
cyberattacks there are helpful guides you can use that have been
\href{https://securityplanner.org/\#/tool/advanced-anonymity-security-guides}{vetted
by security professionals.}

For most of us, the attack against Mr. Bezos isn't the death of privacy,
but a reminder of the risks of living a connected life. It should be a
moment to think as critically about what you do online as you might in
the real world. Invest in a password manager. Turn on dual factor
authentication. Be skeptical of any communication that looks out of
place.

For the ultrarich and influential, the Bezos hack should be a terrifying
revelation. As the former State Department employee and whistle-blower
John Napier Tye
\href{https://www.nytimes.com/2019/11/12/opinion/whistleblower.html}{told
me last autumn}, ``For someone who's truly a high-value target, there is
no way to safely use a digital device.'' The stakes are astronomically
high. Not just personally, as Mr. Bezos found, but professionally.
Company secrets, matters of national security, access to critical
infrastructure and the safety of employees could all be compromised by
lax security at the top.

The internet has long been thought of as a truly democratic tool,
flattening and democratizing the ability to publish and communicate.
It's also the great privacy equalizer. Money can buy a lot of things.
But on a dangerous internet full of exploits, flawed code, shady actors
and absent-minded humans, total, foolproof security is not one of them.

\emph{Like other media companies, The Times collects data on its
visitors when they read stories like this one. For more detail please
see}
\href{https://help.nytimes.com/hc/en-us/articles/115014892108-Privacy-policy?module=inline}{\emph{our
privacy policy}} \emph{and}
\href{https://www.nytimes.com/2019/04/10/opinion/sulzberger-new-york-times-privacy.html?rref=collection\%2Fspotlightcollection\%2Fprivacy-project-does-privacy-matter\&action=click\&contentCollection=opinion\&region=stream\&module=stream_unit\&version=latest\&contentPlacement=8\&pgtype=collection}{\emph{our
publisher's description}} \emph{of The Times's practices and continued
steps to increase transparency and protections.}

\emph{Follow}
\href{https://twitter.com/privacyproject}{\emph{@privacyproject}}
\emph{on Twitter and The New York Times Opinion Section on}
\href{https://www.facebook.com/nytopinion}{\emph{Facebook}}
\emph{and}\href{https://www.instagram.com/nytopinion/}{\emph{Instagram}}\emph{.}

\hypertarget{glossary-replacer}{%
\subsection{glossary replacer}\label{glossary-replacer}}

Advertisement

\protect\hyperlink{after-bottom}{Continue reading the main story}

\hypertarget{site-index}{%
\subsection{Site Index}\label{site-index}}

\hypertarget{site-information-navigation}{%
\subsection{Site Information
Navigation}\label{site-information-navigation}}

\begin{itemize}
\tightlist
\item
  \href{https://help.nytimes.com/hc/en-us/articles/115014792127-Copyright-notice}{©~2020~The
  New York Times Company}
\end{itemize}

\begin{itemize}
\tightlist
\item
  \href{https://www.nytco.com/}{NYTCo}
\item
  \href{https://help.nytimes.com/hc/en-us/articles/115015385887-Contact-Us}{Contact
  Us}
\item
  \href{https://www.nytco.com/careers/}{Work with us}
\item
  \href{https://nytmediakit.com/}{Advertise}
\item
  \href{http://www.tbrandstudio.com/}{T Brand Studio}
\item
  \href{https://www.nytimes.com/privacy/cookie-policy\#how-do-i-manage-trackers}{Your
  Ad Choices}
\item
  \href{https://www.nytimes.com/privacy}{Privacy}
\item
  \href{https://help.nytimes.com/hc/en-us/articles/115014893428-Terms-of-service}{Terms
  of Service}
\item
  \href{https://help.nytimes.com/hc/en-us/articles/115014893968-Terms-of-sale}{Terms
  of Sale}
\item
  \href{https://spiderbites.nytimes.com}{Site Map}
\item
  \href{https://help.nytimes.com/hc/en-us}{Help}
\item
  \href{https://www.nytimes.com/subscription?campaignId=37WXW}{Subscriptions}
\end{itemize}
