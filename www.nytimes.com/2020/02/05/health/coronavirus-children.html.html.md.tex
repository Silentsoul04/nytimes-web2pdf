Sections

SEARCH

\protect\hyperlink{site-content}{Skip to
content}\protect\hyperlink{site-index}{Skip to site index}

\href{https://www.nytimes.com/section/health}{Health}

\href{https://myaccount.nytimes.com/auth/login?response_type=cookie\&client_id=vi}{}

\href{https://www.nytimes.com/section/todayspaper}{Today's Paper}

\href{/section/health}{Health}\textbar{}Why the New Coronavirus (Mostly)
Spares Children

\url{https://nyti.ms/2Sol5VV}

\begin{itemize}
\item
\item
\item
\item
\item
\end{itemize}

\href{https://www.nytimes.com/news-event/coronavirus?action=click\&pgtype=Article\&state=default\&region=TOP_BANNER\&context=storylines_menu}{The
Coronavirus Outbreak}

\begin{itemize}
\tightlist
\item
  live\href{https://www.nytimes.com/2020/08/02/world/coronavirus-updates.html?action=click\&pgtype=Article\&state=default\&region=TOP_BANNER\&context=storylines_menu}{Latest
  Updates}
\item
  \href{https://www.nytimes.com/interactive/2020/us/coronavirus-us-cases.html?action=click\&pgtype=Article\&state=default\&region=TOP_BANNER\&context=storylines_menu}{Maps
  and Cases}
\item
  \href{https://www.nytimes.com/interactive/2020/science/coronavirus-vaccine-tracker.html?action=click\&pgtype=Article\&state=default\&region=TOP_BANNER\&context=storylines_menu}{Vaccine
  Tracker}
\item
  \href{https://www.nytimes.com/interactive/2020/07/29/us/schools-reopening-coronavirus.html?action=click\&pgtype=Article\&state=default\&region=TOP_BANNER\&context=storylines_menu}{What
  School May Look Like}
\item
  \href{https://www.nytimes.com/live/2020/07/31/business/stock-market-today-coronavirus?action=click\&pgtype=Article\&state=default\&region=TOP_BANNER\&context=storylines_menu}{Economy}
\end{itemize}

Advertisement

\protect\hyperlink{after-top}{Continue reading the main story}

Supported by

\protect\hyperlink{after-sponsor}{Continue reading the main story}

\hypertarget{why-the-new-coronavirus-mostly-spares-children}{%
\section{Why the New Coronavirus (Mostly) Spares
Children}\label{why-the-new-coronavirus-mostly-spares-children}}

So far, very few young children seem to be falling ill. The pattern was
seen in outbreaks of SARS and MERS, too.

\includegraphics{https://static01.nyt.com/images/2020/02/05/science/05VIRUS-CHILDREN/05VIRUS-CHILDREN-articleLarge.jpg?quality=75\&auto=webp\&disable=upscale}

By Apoorva Mandavilli

\begin{itemize}
\item
  Published Feb. 5, 2020Updated May 11, 2020
\item
  \begin{itemize}
  \item
  \item
  \item
  \item
  \item
  \end{itemize}
\end{itemize}

\href{https://cn.nytimes.com/health/20200206/coronavirus-children/}{阅读简体中文版}\href{https://cn.nytimes.com/health/20200206/coronavirus-children/zh-hant/}{閱讀繁體中文版}

The new coronavirus has infected nearly 90,000 people, and more than
3,000 have died. But relatively few
\href{https://www.nytimes.com/2020/05/11/health/coronavirus-children-icu.html}{children}
appear to have developed severe symptoms so far, according to available
data.

``Disease in children appears to be relatively rare and mild,'' with
those under 19 years making up only 2.4 percent of the total cases,
according to a
\href{https://www.who.int/docs/default-source/coronaviruse/who-china-joint-mission-on-covid-19-final-report.pdf}{report}
published Feb. 28 by the World Health Organization.

So why aren't more children getting sick?

``My strong, educated guess is that younger people are getting infected,
but they get the relatively milder disease,'' Dr. Malik Peiris, chief of
virology at the University of Hong Kong, said last month. Dr. Peiris has
developed a diagnostic test for the new coronavirus.

The numbers so far support that theory: According to the W.H.O., only
2.5 percent of those under 19 have developed severe disease and only 0.2
percent had critical disease. There have been no deaths recorded in
children under 9.

Scientists may not be seeing more infected children because ``we don't
have data on the milder cases,'' Dr. Peiris said.

Without more information, it is also unclear whether
\href{https://www.nytimes.com/2020/05/11/health/coronavirus-children-icu.html}{children}
can transmit the disease to others. The W.H.O. report said that its team
sent to China, the epicenter of the outbreak, ``could not recall
episodes in which transmission occurred from a child to an adult.''

Still, children who are detected as infected must be shedding some virus
or they wouldn't be detected, noted Dr. Marc Lipsitch, an epidemiologist
at the Harvard T.H. Chan School of Public Health. But whether their
infectiousness is high is as yet unknown. ``It's a very high priority to
do studies to find it out,'' he said.

\hypertarget{latest-updates-global-coronavirus-outbreak}{%
\section{\texorpdfstring{\href{https://www.nytimes.com/2020/08/01/world/coronavirus-covid-19.html?action=click\&pgtype=Article\&state=default\&region=MAIN_CONTENT_1\&context=storylines_live_updates}{Latest
Updates: Global Coronavirus
Outbreak}}{Latest Updates: Global Coronavirus Outbreak}}\label{latest-updates-global-coronavirus-outbreak}}

Updated 2020-08-02T17:52:35.962Z

\begin{itemize}
\tightlist
\item
  \href{https://www.nytimes.com/2020/08/01/world/coronavirus-covid-19.html?action=click\&pgtype=Article\&state=default\&region=MAIN_CONTENT_1\&context=storylines_live_updates\#link-34047410}{The
  U.S. reels as July cases more than double the total of any other
  month.}
\item
  \href{https://www.nytimes.com/2020/08/01/world/coronavirus-covid-19.html?action=click\&pgtype=Article\&state=default\&region=MAIN_CONTENT_1\&context=storylines_live_updates\#link-780ec966}{Top
  U.S. officials work to break an impasse over the federal jobless
  benefit.}
\item
  \href{https://www.nytimes.com/2020/08/01/world/coronavirus-covid-19.html?action=click\&pgtype=Article\&state=default\&region=MAIN_CONTENT_1\&context=storylines_live_updates\#link-2bc8948}{Its
  outbreak untamed, Melbourne goes into even greater lockdown.}
\end{itemize}

\href{https://www.nytimes.com/2020/08/01/world/coronavirus-covid-19.html?action=click\&pgtype=Article\&state=default\&region=MAIN_CONTENT_1\&context=storylines_live_updates}{See
more updates}

More live coverage:
\href{https://www.nytimes.com/live/2020/07/31/business/stock-market-today-coronavirus?action=click\&pgtype=Article\&state=default\&region=MAIN_CONTENT_1\&context=storylines_live_updates}{Markets}

One way to find out, he said, is to look at outbreaks such as the one at
the church in South Korea. ``If there were children among those
people,'' he said, ``that would be a goldmine of data.''

The other approach is to conduct household studies, where multiple
members of a family might be infected.

In one such published case study of a family, a 10-year-old traveled to
Wuhan, China, with his family. Upon returning to Shenzhen, the other
infected family members, ranging in age from 36 to 66,
\href{https://www.thelancet.com/journals/lancet/article/PIIS0140-6736(20)30154-9/fulltext}{developed
fever, sore throat, diarrhea and pneumonia}.

The child, too, had signs of viral pneumonia in the lungs, doctors found
--- but no outward symptoms. Some scientists suspect that this is
typical of coronavirus infection in children.

``It's certainly true that children can be either asymptomatically
infected or have very mild infection,'' Dr. Raina MacIntyre said last
month. Dr. MacIntyre is an epidemiologist at the University of New South
Wales in Sydney, Australia, who has been studying the spread of the new
coronavirus.

In many ways, this pattern parallels that seen during outbreaks of SARS
and MERS, also coronaviruses. The MERS epidemics in Saudi Arabia in 2012
and in South Korea in 2015 together claimed more than 800 lives. Most
children who were infected never developed symptoms.

No children died during the SARS epidemic in 2003, and the majority of
the 800 deaths in the outbreak were in people over age 45, with men more
at risk.

Among the more than 8,000 cases of SARS, researchers at the Centers for
Disease Control and Prevention were able to identify 135 infected
children in published reports.

Children under age 12 were much less likely to be admitted to a hospital
or to need oxygen or other treatment, the researchers found. Children
over age 12 had symptoms much like those of adults.

``We don't fully understand the reason for this age-related increase of
severity,'' Dr. Peiris said. ``But we see that now --- and with SARS,
you could see that much more clearly.''

It's not unusual for viruses to trigger only mild infections in children
and much more severe illnesses in adults. Chickenpox, for example, can
be largely inconsequential in children, yet catastrophic in adults.

\href{https://www.nytimes.com/news-event/coronavirus?action=click\&pgtype=Article\&state=default\&region=MAIN_CONTENT_3\&context=storylines_faq}{}

\hypertarget{the-coronavirus-outbreak-}{%
\subsubsection{The Coronavirus Outbreak
›}\label{the-coronavirus-outbreak-}}

\hypertarget{frequently-asked-questions}{%
\paragraph{Frequently Asked
Questions}\label{frequently-asked-questions}}

Updated July 27, 2020

\begin{itemize}
\item ~
  \hypertarget{should-i-refinance-my-mortgage}{%
  \paragraph{Should I refinance my
  mortgage?}\label{should-i-refinance-my-mortgage}}

  \begin{itemize}
  \tightlist
  \item
    \href{https://www.nytimes.com/article/coronavirus-money-unemployment.html?action=click\&pgtype=Article\&state=default\&region=MAIN_CONTENT_3\&context=storylines_faq}{It
    could be a good idea,} because mortgage rates have
    \href{https://www.nytimes.com/2020/07/16/business/mortgage-rates-below-3-percent.html?action=click\&pgtype=Article\&state=default\&region=MAIN_CONTENT_3\&context=storylines_faq}{never
    been lower.} Refinancing requests have pushed mortgage applications
    to some of the highest levels since 2008, so be prepared to get in
    line. But defaults are also up, so if you're thinking about buying a
    home, be aware that some lenders have tightened their standards.
  \end{itemize}
\item ~
  \hypertarget{what-is-school-going-to-look-like-in-september}{%
  \paragraph{What is school going to look like in
  September?}\label{what-is-school-going-to-look-like-in-september}}

  \begin{itemize}
  \tightlist
  \item
    It is unlikely that many schools will return to a normal schedule
    this fall, requiring the grind of
    \href{https://www.nytimes.com/2020/06/05/us/coronavirus-education-lost-learning.html?action=click\&pgtype=Article\&state=default\&region=MAIN_CONTENT_3\&context=storylines_faq}{online
    learning},
    \href{https://www.nytimes.com/2020/05/29/us/coronavirus-child-care-centers.html?action=click\&pgtype=Article\&state=default\&region=MAIN_CONTENT_3\&context=storylines_faq}{makeshift
    child care} and
    \href{https://www.nytimes.com/2020/06/03/business/economy/coronavirus-working-women.html?action=click\&pgtype=Article\&state=default\&region=MAIN_CONTENT_3\&context=storylines_faq}{stunted
    workdays} to continue. California's two largest public school
    districts --- Los Angeles and San Diego --- said on July 13, that
    \href{https://www.nytimes.com/2020/07/13/us/lausd-san-diego-school-reopening.html?action=click\&pgtype=Article\&state=default\&region=MAIN_CONTENT_3\&context=storylines_faq}{instruction
    will be remote-only in the fall}, citing concerns that surging
    coronavirus infections in their areas pose too dire a risk for
    students and teachers. Together, the two districts enroll some
    825,000 students. They are the largest in the country so far to
    abandon plans for even a partial physical return to classrooms when
    they reopen in August. For other districts, the solution won't be an
    all-or-nothing approach.
    \href{https://bioethics.jhu.edu/research-and-outreach/projects/eschool-initiative/school-policy-tracker/}{Many
    systems}, including the nation's largest, New York City, are
    devising
    \href{https://www.nytimes.com/2020/06/26/us/coronavirus-schools-reopen-fall.html?action=click\&pgtype=Article\&state=default\&region=MAIN_CONTENT_3\&context=storylines_faq}{hybrid
    plans} that involve spending some days in classrooms and other days
    online. There's no national policy on this yet, so check with your
    municipal school system regularly to see what is happening in your
    community.
  \end{itemize}
\item ~
  \hypertarget{is-the-coronavirus-airborne}{%
  \paragraph{Is the coronavirus
  airborne?}\label{is-the-coronavirus-airborne}}

  \begin{itemize}
  \tightlist
  \item
    The coronavirus
    \href{https://www.nytimes.com/2020/07/04/health/239-experts-with-one-big-claim-the-coronavirus-is-airborne.html?action=click\&pgtype=Article\&state=default\&region=MAIN_CONTENT_3\&context=storylines_faq}{can
    stay aloft for hours in tiny droplets in stagnant air}, infecting
    people as they inhale, mounting scientific evidence suggests. This
    risk is highest in crowded indoor spaces with poor ventilation, and
    may help explain super-spreading events reported in meatpacking
    plants, churches and restaurants.
    \href{https://www.nytimes.com/2020/07/06/health/coronavirus-airborne-aerosols.html?action=click\&pgtype=Article\&state=default\&region=MAIN_CONTENT_3\&context=storylines_faq}{It's
    unclear how often the virus is spread} via these tiny droplets, or
    aerosols, compared with larger droplets that are expelled when a
    sick person coughs or sneezes, or transmitted through contact with
    contaminated surfaces, said Linsey Marr, an aerosol expert at
    Virginia Tech. Aerosols are released even when a person without
    symptoms exhales, talks or sings, according to Dr. Marr and more
    than 200 other experts, who
    \href{https://academic.oup.com/cid/article/doi/10.1093/cid/ciaa939/5867798}{have
    outlined the evidence in an open letter to the World Health
    Organization}.
  \end{itemize}
\item ~
  \hypertarget{what-are-the-symptoms-of-coronavirus}{%
  \paragraph{What are the symptoms of
  coronavirus?}\label{what-are-the-symptoms-of-coronavirus}}

  \begin{itemize}
  \tightlist
  \item
    Common symptoms
    \href{https://www.nytimes.com/article/symptoms-coronavirus.html?action=click\&pgtype=Article\&state=default\&region=MAIN_CONTENT_3\&context=storylines_faq}{include
    fever, a dry cough, fatigue and difficulty breathing or shortness of
    breath.} Some of these symptoms overlap with those of the flu,
    making detection difficult, but runny noses and stuffy sinuses are
    less common.
    \href{https://www.nytimes.com/2020/04/27/health/coronavirus-symptoms-cdc.html?action=click\&pgtype=Article\&state=default\&region=MAIN_CONTENT_3\&context=storylines_faq}{The
    C.D.C. has also} added chills, muscle pain, sore throat, headache
    and a new loss of the sense of taste or smell as symptoms to look
    out for. Most people fall ill five to seven days after exposure, but
    symptoms may appear in as few as two days or as many as 14 days.
  \end{itemize}
\item ~
  \hypertarget{does-asymptomatic-transmission-of-covid-19-happen}{%
  \paragraph{Does asymptomatic transmission of Covid-19
  happen?}\label{does-asymptomatic-transmission-of-covid-19-happen}}

  \begin{itemize}
  \tightlist
  \item
    So far, the evidence seems to show it does. A widely cited
    \href{https://www.nature.com/articles/s41591-020-0869-5}{paper}
    published in April suggests that people are most infectious about
    two days before the onset of coronavirus symptoms and estimated that
    44 percent of new infections were a result of transmission from
    people who were not yet showing symptoms. Recently, a top expert at
    the World Health Organization stated that transmission of the
    coronavirus by people who did not have symptoms was ``very rare,''
    \href{https://www.nytimes.com/2020/06/09/world/coronavirus-updates.html?action=click\&pgtype=Article\&state=default\&region=MAIN_CONTENT_3\&context=storylines_faq\#link-1f302e21}{but
    she later walked back that statement.}
  \end{itemize}
\end{itemize}

Influenza is unusual in that it has evolved with humans over thousands
of years and infects millions worldwide each year. Still, even though
thousands of young children end up in the hospital each year with
influenza, just a small percentage of them die, said Dr. Krys Johnson,
an epidemiologist at Temple University in Philadelphia.

This trend is generally true of respiratory illnesses because children
tend to eat well, and to get plenty of exercise and rest --- none of
which may be true of adults. ``The younger, most healthy segment of the
population are able to fight it off,'' she said. Adults may also be more
susceptible because they are more likely to have other diseases, such as
diabetes, high blood pressure or heart disease, that weaken their
ability to stave off infections.

The body's innate immunity, which is critical for fighting viruses, also
deteriorates with age, and particularly after middle age.

``Something happens at age 50,'' Dr. MacIntyre said. ``It declines, and
it declines exponentially, which is why for most infections we see the
highest incidence in the elderly.''

A key question about the new coronavirus is whether children who are
infected and asymptomatic are able to pass the virus to others.

``We know that young people in general --- not just kids, but young
adults and teenagers --- have the most intense contact in society,'' Dr.
MacIntyre said. Young people who don't realize they are sick may
contribute to the epidemic's momentum, she said.

To understand the epidemic fully, she and other scientists said they
need detailed data: when people were first exposed to the virus, when
they first began to show symptoms, how many and which people have mild
symptoms versus more severe disease.

With detailed data, some observations, such as the higher risk in men,
may change. Still, Dr. Mark Denison, a pediatric infectious diseases
specialist at Vanderbilt University in Nashville, said last month that
he does not expect to see a sudden uptick in infected children.

``It's hard for me to imagine that there's such a degree of
underreporting of clinical illness in children that we're only hearing
about two or three cases,'' he said.

``I think it means that there are many, many less children'' who are
infected in China, he said, ``and that they're not as much at risk.''

Advertisement

\protect\hyperlink{after-bottom}{Continue reading the main story}

\hypertarget{site-index}{%
\subsection{Site Index}\label{site-index}}

\hypertarget{site-information-navigation}{%
\subsection{Site Information
Navigation}\label{site-information-navigation}}

\begin{itemize}
\tightlist
\item
  \href{https://help.nytimes.com/hc/en-us/articles/115014792127-Copyright-notice}{©~2020~The
  New York Times Company}
\end{itemize}

\begin{itemize}
\tightlist
\item
  \href{https://www.nytco.com/}{NYTCo}
\item
  \href{https://help.nytimes.com/hc/en-us/articles/115015385887-Contact-Us}{Contact
  Us}
\item
  \href{https://www.nytco.com/careers/}{Work with us}
\item
  \href{https://nytmediakit.com/}{Advertise}
\item
  \href{http://www.tbrandstudio.com/}{T Brand Studio}
\item
  \href{https://www.nytimes.com/privacy/cookie-policy\#how-do-i-manage-trackers}{Your
  Ad Choices}
\item
  \href{https://www.nytimes.com/privacy}{Privacy}
\item
  \href{https://help.nytimes.com/hc/en-us/articles/115014893428-Terms-of-service}{Terms
  of Service}
\item
  \href{https://help.nytimes.com/hc/en-us/articles/115014893968-Terms-of-sale}{Terms
  of Sale}
\item
  \href{https://spiderbites.nytimes.com}{Site Map}
\item
  \href{https://help.nytimes.com/hc/en-us}{Help}
\item
  \href{https://www.nytimes.com/subscription?campaignId=37WXW}{Subscriptions}
\end{itemize}
