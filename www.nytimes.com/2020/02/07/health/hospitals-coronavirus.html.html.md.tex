Sections

SEARCH

\protect\hyperlink{site-content}{Skip to
content}\protect\hyperlink{site-index}{Skip to site index}

\href{https://www.nytimes.com/section/health}{Health}

\href{https://myaccount.nytimes.com/auth/login?response_type=cookie\&client_id=vi}{}

\href{https://www.nytimes.com/section/todayspaper}{Today's Paper}

\href{/section/health}{Health}\textbar{}Inundated With Flu Patients,
U.S. Hospitals Brace for Coronavirus

\url{https://nyti.ms/3buYNLc}

\begin{itemize}
\item
\item
\item
\item
\item
\end{itemize}

\href{https://www.nytimes.com/news-event/coronavirus?action=click\&pgtype=Article\&state=default\&region=TOP_BANNER\&context=storylines_menu}{The
Coronavirus Outbreak}

\begin{itemize}
\tightlist
\item
  live\href{https://www.nytimes.com/2020/08/02/world/coronavirus-updates.html?action=click\&pgtype=Article\&state=default\&region=TOP_BANNER\&context=storylines_menu}{Latest
  Updates}
\item
  \href{https://www.nytimes.com/interactive/2020/us/coronavirus-us-cases.html?action=click\&pgtype=Article\&state=default\&region=TOP_BANNER\&context=storylines_menu}{Maps
  and Cases}
\item
  \href{https://www.nytimes.com/interactive/2020/science/coronavirus-vaccine-tracker.html?action=click\&pgtype=Article\&state=default\&region=TOP_BANNER\&context=storylines_menu}{Vaccine
  Tracker}
\item
  \href{https://www.nytimes.com/interactive/2020/07/29/us/schools-reopening-coronavirus.html?action=click\&pgtype=Article\&state=default\&region=TOP_BANNER\&context=storylines_menu}{What
  School May Look Like}
\item
  \href{https://www.nytimes.com/live/2020/07/31/business/stock-market-today-coronavirus?action=click\&pgtype=Article\&state=default\&region=TOP_BANNER\&context=storylines_menu}{Economy}
\end{itemize}

Advertisement

\protect\hyperlink{after-top}{Continue reading the main story}

Supported by

\protect\hyperlink{after-sponsor}{Continue reading the main story}

\hypertarget{inundated-with-flu-patients-us-hospitals-brace-for-coronavirus}{%
\section{Inundated With Flu Patients, U.S. Hospitals Brace for
Coronavirus}\label{inundated-with-flu-patients-us-hospitals-brace-for-coronavirus}}

Resources are already stretched during flu season. With so much medical
equipment and drugs made in China, public health experts are anxiously
watching the global supply chain.

\includegraphics{https://static01.nyt.com/images/2020/02/07/science/07virus-prepare01/merlin_167531781_6afccfbd-ae7c-4c1d-b8f9-b396611c2d31-articleLarge.jpg?quality=75\&auto=webp\&disable=upscale}

By \href{https://www.nytimes.com/by/reed-abelson}{Reed Abelson} and
\href{https://www.nytimes.com/by/katie-thomas}{Katie Thomas}

\begin{itemize}
\item
  Feb. 7, 2020
\item
  \begin{itemize}
  \item
  \item
  \item
  \item
  \item
  \end{itemize}
\end{itemize}

\href{https://www.nytimes.com/2020/01/08/health/flu-season-severity.html}{With
an intense flu season in full swing}, hundreds of thousands of coughing
and feverish patients have already overwhelmed emergency rooms around
the United States. Now, hospitals are bracing for the potential spread
of coronavirus that could bring another surge of patients.

So far, only
\href{https://www.cdc.gov/coronavirus/2019-ncov/cases-in-us.html}{a
dozen people in the United States} have become infected with the novel
coronavirus, but an outbreak could severely strain the nation's
hospitals.

``We're talking about the possibility of a double flu pandemic,'' where
a second wave starts before the first is over, said Dr. Eric Toner, a
senior scholar at the Johns Hopkins Center for Health Security.

Public health experts are also closely watching reserves of vital
medical supplies and medications, many of which are made in China. Some
hospitals in the United States are already ``critically low'' on
respirator masks, according to Premier Inc., which secures medical
supplies and equipment on behalf of hospitals and health systems. And
China is the dominant supplier of the raw ingredients needed for
penicillin, ibuprofen and even aspirin --- drugs taken daily by millions
of Americans and dispensed routinely to hospital patients.

``All the hospitals are taxed with a large flu season and other bugs,''
said Dr. Mark Jarrett, the chief quality officer for Northwell Health,
which operates 23 hospitals across Long Island and elsewhere in New
York. About 400 patients are coming to its emergency rooms each day with
flulike symptoms.

``Everybody is at maximum capacity,'' Dr. Jarrett said.

Northwell activated its emergency operations center earlier this month
in response to concerns that the virus, now in epidemic proportions
across China, could become a pandemic, similar to the hospital system's
reaction to the threat of swine flu in 2009 and Ebola in 2014.

Although the Trump administration declared a public health emergency for
coronavirus late last week, Dr. Robert Redfield, the head of the Centers
for Disease Control and Prevention, stressed on Friday at a
\href{https://www.youtube.com/watch?v=NKcUSGk6bdI}{news briefing} that
the risk for Americans was still low. Hundreds of people who returned to
the United States from Wuhan have been quarantined at military bases,
and others possibly exposed have been asked to self-quarantine or stay
at home.

But Dr. Redfield also said that officials were beginning to actively
discuss what other steps would have to be taken in the event that the
new coronavirus spread in this country. Asked about emergency funding,
Alex M. Azar II, the secretary for health and human services, said he
did not believe agencies required additional funds right now.

Still, across the country, hospitals are taking steps to prepare for the
potential threat.

Some are already scrambling to find sufficient supplies of medical face
masks,
\href{https://www.fda.gov/medical-devices/personal-protective-equipment-infection-control/masks-and-n95-respirators}{especially
so-called N95 respirators}, which are more effective for preventing
infection.
\href{https://www.nytimes.com/2020/02/06/business/coronavirus-face-masks.html?action=click\&module=Top\%20Stories\&pgtype=Homepage}{Manufacturers
are struggling to keep up with global demand} and the desperate need for
supplies in China.

The mask shortage highlights just how dependent the United States health
care system is on goods from China. Premier was told last week that a
Taiwanese factory it had a contract with was halting shipments to the
United States. In addition, Chaun Powell, the group vice president of
strategic supplier engagement for Premier, said masks that are made in
China are being diverted for use there. As a result, ``there's not as
much supply to ship,'' he said.

\hypertarget{latest-updates-global-coronavirus-outbreak}{%
\section{\texorpdfstring{\href{https://www.nytimes.com/2020/08/01/world/coronavirus-covid-19.html?action=click\&pgtype=Article\&state=default\&region=MAIN_CONTENT_1\&context=storylines_live_updates}{Latest
Updates: Global Coronavirus
Outbreak}}{Latest Updates: Global Coronavirus Outbreak}}\label{latest-updates-global-coronavirus-outbreak}}

Updated 2020-08-02T17:52:35.962Z

\begin{itemize}
\tightlist
\item
  \href{https://www.nytimes.com/2020/08/01/world/coronavirus-covid-19.html?action=click\&pgtype=Article\&state=default\&region=MAIN_CONTENT_1\&context=storylines_live_updates\#link-34047410}{The
  U.S. reels as July cases more than double the total of any other
  month.}
\item
  \href{https://www.nytimes.com/2020/08/01/world/coronavirus-covid-19.html?action=click\&pgtype=Article\&state=default\&region=MAIN_CONTENT_1\&context=storylines_live_updates\#link-780ec966}{Top
  U.S. officials work to break an impasse over the federal jobless
  benefit.}
\item
  \href{https://www.nytimes.com/2020/08/01/world/coronavirus-covid-19.html?action=click\&pgtype=Article\&state=default\&region=MAIN_CONTENT_1\&context=storylines_live_updates\#link-2bc8948}{Its
  outbreak untamed, Melbourne goes into even greater lockdown.}
\end{itemize}

\href{https://www.nytimes.com/2020/08/01/world/coronavirus-covid-19.html?action=click\&pgtype=Article\&state=default\&region=MAIN_CONTENT_1\&context=storylines_live_updates}{See
more updates}

More live coverage:
\href{https://www.nytimes.com/live/2020/07/31/business/stock-market-today-coronavirus?action=click\&pgtype=Article\&state=default\&region=MAIN_CONTENT_1\&context=storylines_live_updates}{Markets}

Hospitals in the United States have also increased their requests, he
said, adding that orders for respirator masks in the first five days of
February had completely outstripped the demand for a typical month.

``We're trying to get as many masks as we can, as is everybody else,''
Dr. Jarrett of Northwell said.

On Friday, Dr. Stephen Hahn, commissioner of the Food and Drug
Administration, said the agency was closely monitoring the supply chain
from Chinese pharmaceutical and medical supply factories, but said the
F.D.A. had received no reports of disruption so far.

Many of those factories closed during the Lunar New Year holidays and
remained closed for additional days..

\includegraphics{https://static01.nyt.com/images/2020/02/07/science/07virus-prepare02/merlin_168100131_8ea237bb-229b-4450-b5cb-349450082140-articleLarge.jpg?quality=75\&auto=webp\&disable=upscale}

``Many of us are holding our breath to see the downstream effect on
pharmaceuticals and other medical supplies because of this outbreak in
China,'' said Dr. Paul Biddinger, who helps oversee emergency
preparedness for Partners Healthcare, the Boston hospital group that
includes Massachusetts General.

Experts like Dr. Toner say supplies could easily become depleted,
especially at smaller hospitals that tend to have less inventory of
basic items like masks, gowns and gloves. Hospitals
\href{https://www.nytimes.com/2019/10/29/health/drug-shortages-generics.html}{have
long struggled with shortages of injectable medications} and staples
like saline. In 2017, Hurricane Maria knocked out power to several
pharmaceutical factories in Puerto Rico,
\href{https://www.nytimes.com/2017/10/23/health/puerto-rico-hurricane-maria-drug-shortage.html}{leading
to a shortage of saline bags}.

Hospitals will ``burn through their supplies very quickly if there are a
lot of these patients,'' Dr. Toner said.

Image

A notice issued by CommonSpirit Health, a large Catholic hospital
system, to help people distinguish between the symptoms of coronavirus
and the seasonal flu.Credit...CommonSpirit Health

Some hospital executives downplayed concerns. ``This is something we
have a tremendous amount of experience with,'' said Dr. Paul Stefanacci,
the chief medical officer for acute care of Universal Health Services, a
for-profit hospital chain. His company has increased its orders of
supplies by 10 percent to handle any uptick in patients.

David Gillan, senior vice president of sourcing operations at Vizient,
another purchasing group, described the supply chain as strained, and
said he worries about too much ``anticipatory behavior,'' such as
hospitals ordering more than they need.

Public health officials are also watching whether the shutdown of
manufacturing plants could lead to shortages in a wide variety of drugs
containing active ingredients made in China. Some drug makers reported
that their facilities had reopened this week. But even temporary delays
in manufacturing --- or problems transporting materials --- could lead
to problems, some experts said.

``\href{https://cen.acs.org/pharmaceuticals/pharmaceutical-chemicals/Coronavirus-puts-drug-chemical-industry/98/web/2020/02}{A
number of companies} are panicking, for sure,'' said James Bruno, a
consultant who advises small and midsize drug makers on manufacturing
issues. ``Especially the smaller ones who don't have the wherewithal to
have backup suppliers.''

Allen Goldberg, a spokesman for the Association for Accessible
Medicines, a generic industry trade group, said companies typically have
several years' worth of supplies of ingredients and that there were
``redundancies built into the system.''

Rosemary Gibson, an expert on China's drug supply who is a senior
adviser at the Hastings Center, a nonpartisan bioethics research
institute, said she had learned of companies having trouble getting
shipments of their products. ``If there were exports from China that
were no longer coming to the U.S., our hospitals and our health systems
would cease to function within a couple of months,'' said Ms. Gibson,
who
\href{https://www.amazon.com/China-Rx-Exposing-Americas-Dependence/dp/1633883817}{also
wrote a book}, ``China Rx,'' about the American dependence on medicines
from China.

The United States has already seen examples of how quickly a disruption
in supply from China can lead to shortages. In 2018,
\href{https://www.nytimes.com/2018/07/16/health/fda-blood-pressure-valsartan.html}{widespread
recalls of the blood pressure drug valsartan} were traced to problems at
a single Chinese factory that made the drug's active ingredient, which
was contaminated with a possible carcinogen.

Image

The growing shortage of masks highlights just how dependent the health
care system in the United States is on goods from China. Credit...Agence
France-Presse --- Getty Images

In 2016,
\href{https://www.theguardian.com/society/2017/jul/01/antibiotic-shortage-puts-patients-at-risk}{a
chemical explosion at a raw ingredient factory} in China led to a global
shortage of piperacillin-tazobactam, an antibiotic used to treat a
number of life-or-death conditions, including sepsis.

\href{https://www.nytimes.com/news-event/coronavirus?action=click\&pgtype=Article\&state=default\&region=MAIN_CONTENT_3\&context=storylines_faq}{}

\hypertarget{the-coronavirus-outbreak-}{%
\subsubsection{The Coronavirus Outbreak
›}\label{the-coronavirus-outbreak-}}

\hypertarget{frequently-asked-questions}{%
\paragraph{Frequently Asked
Questions}\label{frequently-asked-questions}}

Updated July 27, 2020

\begin{itemize}
\item ~
  \hypertarget{should-i-refinance-my-mortgage}{%
  \paragraph{Should I refinance my
  mortgage?}\label{should-i-refinance-my-mortgage}}

  \begin{itemize}
  \tightlist
  \item
    \href{https://www.nytimes.com/article/coronavirus-money-unemployment.html?action=click\&pgtype=Article\&state=default\&region=MAIN_CONTENT_3\&context=storylines_faq}{It
    could be a good idea,} because mortgage rates have
    \href{https://www.nytimes.com/2020/07/16/business/mortgage-rates-below-3-percent.html?action=click\&pgtype=Article\&state=default\&region=MAIN_CONTENT_3\&context=storylines_faq}{never
    been lower.} Refinancing requests have pushed mortgage applications
    to some of the highest levels since 2008, so be prepared to get in
    line. But defaults are also up, so if you're thinking about buying a
    home, be aware that some lenders have tightened their standards.
  \end{itemize}
\item ~
  \hypertarget{what-is-school-going-to-look-like-in-september}{%
  \paragraph{What is school going to look like in
  September?}\label{what-is-school-going-to-look-like-in-september}}

  \begin{itemize}
  \tightlist
  \item
    It is unlikely that many schools will return to a normal schedule
    this fall, requiring the grind of
    \href{https://www.nytimes.com/2020/06/05/us/coronavirus-education-lost-learning.html?action=click\&pgtype=Article\&state=default\&region=MAIN_CONTENT_3\&context=storylines_faq}{online
    learning},
    \href{https://www.nytimes.com/2020/05/29/us/coronavirus-child-care-centers.html?action=click\&pgtype=Article\&state=default\&region=MAIN_CONTENT_3\&context=storylines_faq}{makeshift
    child care} and
    \href{https://www.nytimes.com/2020/06/03/business/economy/coronavirus-working-women.html?action=click\&pgtype=Article\&state=default\&region=MAIN_CONTENT_3\&context=storylines_faq}{stunted
    workdays} to continue. California's two largest public school
    districts --- Los Angeles and San Diego --- said on July 13, that
    \href{https://www.nytimes.com/2020/07/13/us/lausd-san-diego-school-reopening.html?action=click\&pgtype=Article\&state=default\&region=MAIN_CONTENT_3\&context=storylines_faq}{instruction
    will be remote-only in the fall}, citing concerns that surging
    coronavirus infections in their areas pose too dire a risk for
    students and teachers. Together, the two districts enroll some
    825,000 students. They are the largest in the country so far to
    abandon plans for even a partial physical return to classrooms when
    they reopen in August. For other districts, the solution won't be an
    all-or-nothing approach.
    \href{https://bioethics.jhu.edu/research-and-outreach/projects/eschool-initiative/school-policy-tracker/}{Many
    systems}, including the nation's largest, New York City, are
    devising
    \href{https://www.nytimes.com/2020/06/26/us/coronavirus-schools-reopen-fall.html?action=click\&pgtype=Article\&state=default\&region=MAIN_CONTENT_3\&context=storylines_faq}{hybrid
    plans} that involve spending some days in classrooms and other days
    online. There's no national policy on this yet, so check with your
    municipal school system regularly to see what is happening in your
    community.
  \end{itemize}
\item ~
  \hypertarget{is-the-coronavirus-airborne}{%
  \paragraph{Is the coronavirus
  airborne?}\label{is-the-coronavirus-airborne}}

  \begin{itemize}
  \tightlist
  \item
    The coronavirus
    \href{https://www.nytimes.com/2020/07/04/health/239-experts-with-one-big-claim-the-coronavirus-is-airborne.html?action=click\&pgtype=Article\&state=default\&region=MAIN_CONTENT_3\&context=storylines_faq}{can
    stay aloft for hours in tiny droplets in stagnant air}, infecting
    people as they inhale, mounting scientific evidence suggests. This
    risk is highest in crowded indoor spaces with poor ventilation, and
    may help explain super-spreading events reported in meatpacking
    plants, churches and restaurants.
    \href{https://www.nytimes.com/2020/07/06/health/coronavirus-airborne-aerosols.html?action=click\&pgtype=Article\&state=default\&region=MAIN_CONTENT_3\&context=storylines_faq}{It's
    unclear how often the virus is spread} via these tiny droplets, or
    aerosols, compared with larger droplets that are expelled when a
    sick person coughs or sneezes, or transmitted through contact with
    contaminated surfaces, said Linsey Marr, an aerosol expert at
    Virginia Tech. Aerosols are released even when a person without
    symptoms exhales, talks or sings, according to Dr. Marr and more
    than 200 other experts, who
    \href{https://academic.oup.com/cid/article/doi/10.1093/cid/ciaa939/5867798}{have
    outlined the evidence in an open letter to the World Health
    Organization}.
  \end{itemize}
\item ~
  \hypertarget{what-are-the-symptoms-of-coronavirus}{%
  \paragraph{What are the symptoms of
  coronavirus?}\label{what-are-the-symptoms-of-coronavirus}}

  \begin{itemize}
  \tightlist
  \item
    Common symptoms
    \href{https://www.nytimes.com/article/symptoms-coronavirus.html?action=click\&pgtype=Article\&state=default\&region=MAIN_CONTENT_3\&context=storylines_faq}{include
    fever, a dry cough, fatigue and difficulty breathing or shortness of
    breath.} Some of these symptoms overlap with those of the flu,
    making detection difficult, but runny noses and stuffy sinuses are
    less common.
    \href{https://www.nytimes.com/2020/04/27/health/coronavirus-symptoms-cdc.html?action=click\&pgtype=Article\&state=default\&region=MAIN_CONTENT_3\&context=storylines_faq}{The
    C.D.C. has also} added chills, muscle pain, sore throat, headache
    and a new loss of the sense of taste or smell as symptoms to look
    out for. Most people fall ill five to seven days after exposure, but
    symptoms may appear in as few as two days or as many as 14 days.
  \end{itemize}
\item ~
  \hypertarget{does-asymptomatic-transmission-of-covid-19-happen}{%
  \paragraph{Does asymptomatic transmission of Covid-19
  happen?}\label{does-asymptomatic-transmission-of-covid-19-happen}}

  \begin{itemize}
  \tightlist
  \item
    So far, the evidence seems to show it does. A widely cited
    \href{https://www.nature.com/articles/s41591-020-0869-5}{paper}
    published in April suggests that people are most infectious about
    two days before the onset of coronavirus symptoms and estimated that
    44 percent of new infections were a result of transmission from
    people who were not yet showing symptoms. Recently, a top expert at
    the World Health Organization stated that transmission of the
    coronavirus by people who did not have symptoms was ``very rare,''
    \href{https://www.nytimes.com/2020/06/09/world/coronavirus-updates.html?action=click\&pgtype=Article\&state=default\&region=MAIN_CONTENT_3\&context=storylines_faq\#link-1f302e21}{but
    she later walked back that statement.}
  \end{itemize}
\end{itemize}

About 20 F.D.A. employees assigned to the agency's
\href{https://www.nytimes.com/2008/11/19/world/asia/19iht-beijing.1.17954828.html}{offices
in China}, and who were responsible for product inspections there, have
been evacuated.

``It will be some matter of time before we understand what the
disruptions are,'' said Michael Wessel, a member of the U.S.-China
Economic and Security Review Commission
\href{https://www.uscc.gov/hearings/exploring-growing-us-reliance-chinas-biotech-and-pharmaceutical-products}{who
has studied} the growing reliance by the United States on Chinese
pharmaceutical products.

\textbf{\emph{{[}}\href{http://on.fb.me/1paTQ1h}{\emph{Like the Science
Times page on Facebook.}}} ****** \emph{\textbar{} Sign up for the}
\textbf{\href{http://nyti.ms/1MbHaRU}{\emph{Science Times
newsletter.}}\emph{{]}}}

Meanwhile, hospitals are training their staffs and circulating the
latest guidance for screening patients and limiting the spread of the
virus. At CommonSpirit Health --- a large Catholic hospital system with
142 hospitals created through the merger of Dignity Health, based in San
Francisco, and Catholic Health Initiatives, based in Chicago ---
patients and staff members are being instructed on how to distinguish
coronavirus from the flu. It has already embedded screening questions in
its electronic health records.

``It has not impacted our facilities at this time,'' said Roy
Boukidjian, vice president for infection prevention for CommonSpirit.

Learning from a 2014 experience with Ebola
\href{https://www.nytimes.com/2014/10/29/us/ebola-outbreak-dallas-nurse-amber-joy-vinson.html}{when
two nurses in Dallas were infected while caring for a patient},
hospitals understand they need to actively prepare their medical staffs
for an influx of cases from a contagious disease, instructing them on
how to safely put on and take off clothing and other protective
equipment, said Julie Fischer, an assistant professor of microbiology at
Georgetown University. ``Personnel were not trained, systems were not
tested,'' she said. ``That's where the breakdown happened.''

Health care workers are always at high risk, and
\href{https://www.nytimes.com/2020/02/06/world/asia/chinese-doctor-Li-Wenliang-coronavirus.html}{the
death of a doctor} treating patients in China reinforced the dangers.

Some nurses say they are better prepared now to protect workers. ``We
all went through the Ebola virus a couple of years ago,'' said Barbara
Rosen, an official with Health Professionals and Allied Employees, a New
Jersey union. ``For this round, everything seems very prepared.''

But Bonnie Castillo, the executive director of National Nurses United,
which represents roughly 155,000 nurses, said some hospitals may not be
fully prepared.

``Our concern is that there is inadequate planning and training and also
just basic staffing,'' as well as the availability of supplies that
protect nurses and patients from infection, she said.

Because the nature of the virus is still unknown, public health
officials said it's unclear what future challenges hospitals will face
if the coronavirus spreads into an epidemic in the United States. While
the current government guidelines call for patients to be treated in
specialized isolation rooms, experts say it is unlikely that there will
be enough isolation rooms at individual facilities.

And to manage the health of patients with severe respiratory illnesses,
which coronavirus can cause, hospitals may find themselves without
enough beds in their intensive care units or sufficient ventilators and
other breathing equipment.

``If it's a pandemic, it's going to stretch our resources very thin,''
said Nicole Errett, a researcher at the University of Washington School
of Public Health.

Advertisement

\protect\hyperlink{after-bottom}{Continue reading the main story}

\hypertarget{site-index}{%
\subsection{Site Index}\label{site-index}}

\hypertarget{site-information-navigation}{%
\subsection{Site Information
Navigation}\label{site-information-navigation}}

\begin{itemize}
\tightlist
\item
  \href{https://help.nytimes.com/hc/en-us/articles/115014792127-Copyright-notice}{©~2020~The
  New York Times Company}
\end{itemize}

\begin{itemize}
\tightlist
\item
  \href{https://www.nytco.com/}{NYTCo}
\item
  \href{https://help.nytimes.com/hc/en-us/articles/115015385887-Contact-Us}{Contact
  Us}
\item
  \href{https://www.nytco.com/careers/}{Work with us}
\item
  \href{https://nytmediakit.com/}{Advertise}
\item
  \href{http://www.tbrandstudio.com/}{T Brand Studio}
\item
  \href{https://www.nytimes.com/privacy/cookie-policy\#how-do-i-manage-trackers}{Your
  Ad Choices}
\item
  \href{https://www.nytimes.com/privacy}{Privacy}
\item
  \href{https://help.nytimes.com/hc/en-us/articles/115014893428-Terms-of-service}{Terms
  of Service}
\item
  \href{https://help.nytimes.com/hc/en-us/articles/115014893968-Terms-of-sale}{Terms
  of Sale}
\item
  \href{https://spiderbites.nytimes.com}{Site Map}
\item
  \href{https://help.nytimes.com/hc/en-us}{Help}
\item
  \href{https://www.nytimes.com/subscription?campaignId=37WXW}{Subscriptions}
\end{itemize}
