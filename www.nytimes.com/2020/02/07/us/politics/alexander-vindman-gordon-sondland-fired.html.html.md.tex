Sections

SEARCH

\protect\hyperlink{site-content}{Skip to
content}\protect\hyperlink{site-index}{Skip to site index}

\href{https://www.nytimes.com/section/politics}{Politics}

\href{https://myaccount.nytimes.com/auth/login?response_type=cookie\&client_id=vi}{}

\href{https://www.nytimes.com/section/todayspaper}{Today's Paper}

\href{/section/politics}{Politics}\textbar{}Trump Fires Impeachment
Witnesses Gordon Sondland and Alexander Vindman in Post-Acquittal Purge

\url{https://nyti.ms/3bqvDwP}

\begin{itemize}
\item
\item
\item
\item
\item
\item
\end{itemize}

Advertisement

\protect\hyperlink{after-top}{Continue reading the main story}

Supported by

\protect\hyperlink{after-sponsor}{Continue reading the main story}

\hypertarget{trump-fires-impeachment-witnesses-gordon-sondland-and-alexander-vindman-in-post-acquittal-purge}{%
\section{Trump Fires Impeachment Witnesses Gordon Sondland and Alexander
Vindman in Post-Acquittal
Purge}\label{trump-fires-impeachment-witnesses-gordon-sondland-and-alexander-vindman-in-post-acquittal-purge}}

Emboldened by his victory and determined to strike back, the president
removed Mr. Sondland as ambassador to the European Union after the White
House earlier on Friday dismissed Colonel Vindman.

\includegraphics{https://static01.nyt.com/images/2020/02/07/us/07dc-revenge2/07dc-revenge2-articleLarge-v2.jpg?quality=75\&auto=webp\&disable=upscale}

\href{https://www.nytimes.com/by/peter-baker}{\includegraphics{https://static01.nyt.com/images/2018/06/13/multimedia/peter-baker/peter-baker-thumbLarge-v2.png}}\href{https://www.nytimes.com/by/maggie-haberman}{\includegraphics{https://static01.nyt.com/images/2018/07/12/multimedia/author-maggie-haberman/author-maggie-haberman-thumbLarge.png}}\href{https://www.nytimes.com/by/danny-hakim}{\includegraphics{https://static01.nyt.com/images/2018/10/18/multimedia/author-danny-hakim/author-danny-hakim-thumbLarge.png}}\href{https://www.nytimes.com/by/michael-s-schmidt}{\includegraphics{https://static01.nyt.com/images/2018/06/12/multimedia/author-michael-s-schmidt/author-michael-s-schmidt-thumbLarge.png}}

By \href{https://www.nytimes.com/by/peter-baker}{Peter Baker},
\href{https://www.nytimes.com/by/maggie-haberman}{Maggie Haberman},
\href{https://www.nytimes.com/by/danny-hakim}{Danny Hakim} and
\href{https://www.nytimes.com/by/michael-s-schmidt}{Michael S. Schmidt}

\begin{itemize}
\item
  Published Feb. 7, 2020Updated July 8, 2020
\item
  \begin{itemize}
  \item
  \item
  \item
  \item
  \item
  \item
  \end{itemize}
\end{itemize}

WASHINGTON ---
\href{https://www.nytimes.com/2020/02/14/podcasts/the-daily/trump-acquittal.html?action=click\&module=Briefings\&pgtype=Homepage}{President
Trump} wasted little time on Friday opening a campaign of retribution
against those he blames for his impeachment, firing two of the most
prominent witnesses in the House inquiry against him barely 48 hours
after being
\href{https://www.nytimes.com/2020/02/05/us/politics/trump-acquitted-impeachment.html}{acquitted
by the Senate}.

Emboldened by his victory and determined to strike back, Mr. Trump
ordered Gordon D. Sondland, the founder of a hotel chain who donated \$1
million to the president's inaugural committee, recalled from his post
as the ambassador to the European Union on the same day that
\href{https://www.nytimes.com/2020/07/08/us/politics/vindman-trump-ukraine-impeachment.html}{Lt.
Col. Alexander S. Vindman}, a decorated Iraq war veteran on the National
Security Council staff, was marched out of the White House by security
guards.

The ousters of Mr. Sondland and Colonel Vindman --- along with Mr.
Vindman's brother, Lt. Col. Yevgeny Vindman, an Army officer who also
worked on the National Security Council staff --- may only presage a
broader effort to even accounts with the president's perceived enemies.
In the two days since his acquittal in the Senate, Mr. Trump has
\href{https://www.nytimes.com/2020/02/06/us/politics/trump-impeachment.html}{railed
about those who stood against him}, calling them ``evil,'' ``corrupt''
and ``crooked,'' while his press secretary declared that those who hurt
the president
``\href{https://www.foxnews.com/media/stephanie-grisham-dems-must-be-held-accountable-for-corrupt-impeachment}{should
pay for}'' it.

Even as he began purging administration officials who testified in the
House impeachment inquiry, Mr. Trump assailed a Democratic senator who
he had hoped would side with him during the trial but did not and called
on the House to ``expunge'' his impeachment because he deems it
illegitimate.

The flurry of actions and outbursts drew quick condemnation from
Democrats, who said the president was demonstrating that he feels
unleashed, and complicated the politics of impeachment for moderate
Republicans who stood by him while arguing that he had learned his
lesson and would be more restrained in the future.

``There is no question in the mind of any American why this man's job is
over, why this country now has one less soldier serving it at the White
House,'' David Pressman, Colonel Vindman's lawyer, said in a statement.
``Lt. Col. Vindman was asked to leave for telling the truth. His honor,
his commitment to right, frightened the powerful.''

Colonel Vindman spoke publicly only once, after being ordered to under
subpoena, Mr. Pressman added. ``And for that, the most powerful man in
the world --- buoyed by the silent, the pliable and the complicit ---
has decided to exact revenge.''

\includegraphics{https://static01.nyt.com/images/2017/01/29/podcasts/the-daily-album-art/the-daily-album-art-articleInline-v2.jpg?quality=75\&auto=webp\&disable=upscale}

\hypertarget{listen-to-the-daily-the-post-acquittal-presidency}{%
\subsubsection{Listen to `The Daily': The Post-Acquittal
Presidency}\label{listen-to-the-daily-the-post-acquittal-presidency}}

How has President Trump's acquittal in the Senate impeachment trial
emboldened him in the Oval Office?

transcript

Back to The Daily

bars

0:00/28:08

-28:08

transcript

\hypertarget{listen-to-the-daily-the-post-acquittal-presidency-1}{%
\subsection{Listen to `The Daily': The Post-Acquittal
Presidency}\label{listen-to-the-daily-the-post-acquittal-presidency-1}}

\hypertarget{hosted-by-michael-barbaro-produced-by-eric-krupke-adizah-eghan-and-jonathan-wolfe-and-edited-by-lisa-chow}{%
\subsubsection{Hosted by Michael Barbaro, produced by Eric Krupke,
Adizah Eghan, and Jonathan Wolfe, and edited by Lisa
Chow}\label{hosted-by-michael-barbaro-produced-by-eric-krupke-adizah-eghan-and-jonathan-wolfe-and-edited-by-lisa-chow}}

\hypertarget{how-has-president-trumps-acquittal-in-the-senate-impeachment-trial-emboldened-him-in-the-oval-office}{%
\paragraph{How has President Trump's acquittal in the Senate impeachment
trial emboldened him in the Oval
Office?}\label{how-has-president-trumps-acquittal-in-the-senate-impeachment-trial-emboldened-him-in-the-oval-office}}

\begin{itemize}
\item
  {[}music{]}
\item
  michael barbaro\\
  From The New York Times, I'm Michael Barbaro. This is ``The Daily.''

  Today: President Trump has undertaken a campaign of retribution
  against those who crossed him during the impeachment inquiry and
  favors for those who have tried to protect him. Peter Baker on the
  post-acquittal presidency.

  It's Friday, February 14.
\item
  archived recording\\
  {[}HORNS{]} Ladies and gentlemen, the President of the United States.
\end{itemize}

michael barbaro

Peter, I want to begin with retribution. How does that start?

\begin{itemize}
\tightlist
\item
  archived recording (donald trump)\\
  Well, thank you very much, everybody. Wow.
\end{itemize}

peter baker

The day after his acquittal in the Senate, the president gathers people
in the East Room of the White House for an event. It's not quite a press
conference. It's not quite a speech. It's really kind of a mix, a mix of
a celebration of his acquittal but a venting session of his grievances.

\begin{itemize}
\tightlist
\item
  archived recording (donald trump)\\
  I want to start by thanking some of --- and I call them friends,
  because you develop friendships and relationships when you're in
  battle and war.
\end{itemize}

peter baker

And he wants to thank the people who stood behind him, names them in the
audience.

\begin{itemize}
\tightlist
\item
  archived recording (donald trump)\\
  Mitch McConnell, I want to tell you, you did a fantastic job.
  {[}APPLAUSE{]}
\end{itemize}

peter baker

Mitch McConnell, the Republican leader who did more than anybody to
secure his acquittal in the trial. And he mentions Jim Jordan ---

\begin{itemize}
\tightlist
\item
  archived recording (donald trump)\\
  When I first got to know Jim, I said, huh, he never wears a jacket.
  What the hell's going on? He's obviously very proud of his body.
  {[}LAUGHTER{]}
\end{itemize}

peter baker

--- and other members of the House, the Freedom Caucus, the conservative
Republicans who always stood by him in the most aggressive and assertive
and staunch way. And then, of course, he turns to his enemies.

The people he blames for his ordeal, the people he thinks have treated
him so unfairly, have plotted against him, been disloyal or what have
you. And he names ones that you would expect, of course.

\begin{itemize}
\tightlist
\item
  archived recording (donald trump)\\
  Nancy Pelosi is a horrible person.
\end{itemize}

peter baker

Nancy Pelosi, he says she's a horrible person.

\begin{itemize}
\tightlist
\item
  archived recording (donald trump)\\
  A corrupt politician named Adam Schiff made up my statement to the
  Ukrainian president. He brought it out of thin air --- just made it
  up. They say he's a screenwriter, a failed screenwriter.
\end{itemize}

peter baker

He names, of course, Adam Schiff, the lead House prosecutor.

\begin{itemize}
\tightlist
\item
  archived recording (donald trump)\\
  And then you have some that used religion as a crutch. They never used
  it before.
\end{itemize}

peter baker

He names Mitt Romney, the Republican, the only Republican senator to
vote for conviction.

\begin{itemize}
\tightlist
\item
  archived recording (donald trump)\\
  But, you know, it's a failed presidential candidate, so things can
  happen when you fail so badly running for president.
\end{itemize}

peter baker

These two now, of course, are really at odds. And you see the visceral
anger in the president in this moment.

And he mentions Colonel Alexander Vindman, a member of his own staff, a
detailee from the Pentagon working on Ukraine issues, and his twin
brother Yevgeny Vindman, who also works at the N.S.C. staff. He says it
almost in passing.

\begin{itemize}
\tightlist
\item
  archived recording (donald trump)\\
  Lieutenant Colonel Vindman and his twin brother, right?
\end{itemize}

peter baker

And he says it with such dripping disdain in his voice. You'd get the
sense immediately, of course, that this is somebody who's really angry
at the president, and he's got his attention.

michael barbaro

And remind us what puts Vindman in this list of enemies.

peter baker

Colonel Vindman was one of the members of the White House staff, the
National Security Council staff who were subpoenaed by the House to
testify in the impeachment inquiry. He didn't come forward voluntarily.
He was required to by law to give his testimony to the committee. And
during his testimony, he told about being on the famous July 25 call
between the president and President Zelensky of Ukraine when the
president asked him to investigate Joe Biden and the Democrats. And
Colonel Vindman told the committee that he thought that was
inappropriate, and he reported it to his superiors at the N.S.C. And for
that, he has been on the target list of President Trump and his allies
ever since. Painted as disloyal, painted as even treasonous to the
country. His patriotism questioned even though he's a decorated veteran
of the Iraq War, injured in battle, and really, a kind of a symbol to
both sides of sort of where this fight has evolved.

\begin{itemize}
\tightlist
\item
  archived recording (donald trump)\\
  Our country is just respected again, and it's an honor to be with the
  people in this room. Thank you very much, everybody. Thank you. Thank
  you very much. Thank you.
\end{itemize}

peter baker

And so he comes to the end of this sort of rambling, meandering talk
that goes on for an hour and two minutes. And you get the sense that
this is not the end and that there's more to come.

\begin{itemize}
\tightlist
\item
  archived recording\\
  Well, President Trump has begun his revenge in the wake of his
  impeachment trial.
\end{itemize}

peter baker

Colonel Vindman, the same witness he had just talked about so
dismissively at the East Room event finds himself escorted out of the
White House by security guards and told his services are no longer
needed ---

michael barbaro

Wow.

peter baker

--- exiled back to the Pentagon from which he came. Not just him ---

\begin{itemize}
\tightlist
\item
  archived recording\\
  Escorted out of his job and off the White House grounds, as was his
  twin brother, who was also assigned to the N.S.C.
\end{itemize}

peter baker

His brother Yevgeny Vindman --- who didn't do anything, had nothing to
do with the impeachment hearings at all, except to show up and sit
behind his brother just as a matter of family support --- also dismissed
from his post at the National Security Council, marched out at the same
time by security and sent back to the Pentagon.

\begin{itemize}
\tightlist
\item
  archived recording\\
  Today, Vindman's lawyer issued a statement saying, quote, ``The truth
  has cost him his job, his career and his privacy.''
\end{itemize}

peter baker

You can understand why a president might not want somebody on his staff
who had testified an impeachment hearing against him. But it was handled
in a way that was meant to maximize the public message, right? I'll tell
you what I mean by that. The N.S.C. is currently undergoing a
downsizing. And in fact, the plan was to move Colonel Vindman out as
part of that, or at least to use that as the cover to to say, it's not
about reprisal. It's not about his role in impeachment. It's just part
of this overall restructuring. And that's frankly how other presidents
might have handled a situation like that.

michael barbaro

Come up with a rationale.

peter baker

Come up with a rationale, come up with a public face-saving kind of
storyline, a narrative, at least, that even though people would see
through it, would at least have the veneer of looking professional
rather than vindictive. That was not what the president wanted. He made
sure they did this separate from that reorganization. They did not
explain it as part of that reorganization. And they did not deny when we
called them that day that this was what it looked like, which was, of
course, an act of retribution.

michael barbaro

OK, so what happens next?

peter baker

Well, we thought that was the story for the day, these two brothers
being marched out of there.

michael barbaro

Right.

peter baker

And then we discover as the evening arrives that it's not over.

\begin{itemize}
\tightlist
\item
  archived recording\\
  Now we're getting word that the U.S. ambassador to the European Union,
  Gordon Sondland, he is out as well.
\end{itemize}

peter baker

Gordon Sondland, you may remember him. He was the ambassador to the
European Union, who had been deeply involved in the Ukraine pressure
campaign, on the phone with the president and required to testify,
became a key witness in the House hearings. He said that they were
operating on the order of the president himself. He said that it was
clearly a quid pro quo, and he said that everyone was in the loop.
Suddenly, it turns out he's out as well. Now, as with Vindman, there was
a way to do this that would have minimized the public kerfuffle. Gordon
Sondland actually was ready to leave. He had told his superiors at the
State Department that he was ready to step down on his own. And he got
word that Friday you have to resign today, they told him. But he says,
no. I don't want to resign on the same day that you're pushing out these
Vindmans as if I'm part of some sort of purge.

michael barbaro

Wow.

peter baker

If you want me today, you're going to have to fire me. And they called
back and said, OK, you're fired.

michael barbaro

So at this point, it's clear that this is a vindictive purge of anyone
who did anything that put the president in a negative light during the
impeachment process. And what is the reaction to that, that very clear
and deliberate message from the president inside Washington?

peter baker

Certainly among Democrats, even among a few Republicans who say what's
the message you're sending? If you respond to a subpoena, as ordered by
the law, and you give your testimony, you shouldn't be punished for
doing that. The president's view is, why should I have people I can't
trust working for me? It's my right as the president to have a staff
that serves my interests that I believe is loyal. And he's made clear
that loyalty is a number one when it comes to this president. There's no
other quality that matters more to him.

michael barbaro

And, Peter, as somebody who's covered many White Houses, is he right
about that? Is it ultimately a presidential prerogative to decide if
someone testified against you, that, you know, you no longer want them
around, you don't want them in those jobs anymore, especially
presidential appointments?

peter baker

It's a good question, right? Because it does feel like it would be
untenable to have testified and provided damaging testimony against the
president, and then come to work every day afterwards. You would think,
in fact, you might not want to necessarily do that. But the question
isn't what's the right place then for that person to work. The question
is what the message the president is trying to send by what he's doing,
right?

michael barbaro

Right.

peter baker

This president has made a point of making sure everybody knows these
people are out, and they're out because of him and because he will not
tolerate disloyalty.

michael barbaro

OK, so that is the campaign of retribution so far, post acquittal, which
brings us to the campaign of protection for the president's allies.

peter baker

Right. It's not enough just to go after his perceived enemies. Now it's
time to do something to protect his friends. And for him, this is going
to start with a colorful character and longtime friend and adviser named
Roger Stone, who's about to go to prison.

michael barbaro

We'll be right back.

So, Peter, before we get to how the president is trying to protect Roger
Stone, remind us who Roger Stone is.

peter baker

Roger Stone has been in American politics going back decades.

He is somebody who calls himself a dirty trickster.

\begin{itemize}
\tightlist
\item
  archived recording (roger stone)\\
  I'm certainly guilty of bluffing and posturing and punking the
  Democrats. Unless they pass some law against {[}BLEEP{]} and I missed
  it, I'm engaging in tradecraft. It's politics.
\end{itemize}

peter baker

He's a self-proclaimed fan of Richard Nixon. Even to this day, he has a
Richard Nixon tattoo.

michael barbaro

Right.

peter baker

He's somebody who's involved early on in some of the Reagan and Dole
campaigns, but over the years kind of drifted off into the side, really
kind of more of a fringe character, a conspiracy theorist, a
provocateur.

\begin{itemize}
\item
  archived recording 1\\
  In 1980, Stone began a lobbying firm with Paul Manafort that
  unapologetically catered to human rights abusers.
\item
  archived recording 2\\
  He has these maxims on how he conducts his political strategy. One of
  his rules is never turn down an opportunity to have sex or be on
  television.
\item
  archived recording 3\\
  We've seen a lot of colorful characters in the world of political
  consulting, none more colorful than Roger Stone. And that is the most
  charitable adjective you can apply to the single weirdest man possibly
  in the history of political consulting.
\end{itemize}

peter baker

He'd been friends for years with Donald Trump. And like Roger Stone,
Trump comes from the outside, right? He was not part of the Republican
establishment. But suddenly, he's powering forward toward a presidential
bid. And he brings with him people like Roger Stone, who had not been in
the center of American politics now for years.

michael barbaro

Right. And my recollection is that it's during that campaign that Roger
Stone gets into very significant trouble.

peter baker

Right. He becomes wrapped up in the whole story about the Russian
hacking of the Democratic emails.

\begin{itemize}
\tightlist
\item
  archived recording\\
  Hillary Clinton's campaign dealing with more email problems. The email
  account of campaign chairman John Podesta was hacked and many of the
  emails released.
\end{itemize}

peter baker

Things he said gave the impression that he might have known about it in
advance.

michael barbaro

Right.

\begin{itemize}
\item
  archived recording\\
  So were you surprised when John Podesta's emails came out, as you
  seemed to predict ahead of time?
\item
  archived recording (roger stone)\\
  I was interested, like the rest of the country.
\item
  archived recording\\
  Were you surprised?
\item
  archived recording (roger stone)\\
  No, I wouldn't say that I was surprised.
\end{itemize}

peter baker

And that puts him right in the heart of this. Is he a link between the
Trump campaign and Russia through perhaps WikiLeaks, which is the cutout
that the Russians used to get these emails out. And so, once the
president wins and comes into office, his friend Roger Stone finds
himself under investigation for what he knew and when he knew it. And
then Congress jumps in. They call Stone to testify at the House
Intelligence Committee. And this is where he really gets into trouble.

\begin{itemize}
\tightlist
\item
  archived recording (roger stone)\\
  We had a very frank exchange. I answered all of the questions. I made
  the case that the accusation that I knew about John Podesta's email
  hack in advance was false, that I knew about the content and source of
  the WikiLeaks disclosures regarding Hillary Clinton was false.
\end{itemize}

peter baker

He starts telling things that are demonstrably not true. And he
ultimately ends up getting charged with lying to Congress. He also tries
to get an associate of his to not tell the truth, threatens him even,
threatens to kill his dog.

michael barbaro

Whoa.

peter baker

And he was put on trial. And last fall Roger Stone was convicted of
seven crimes, seven felonies, including lying to Congress and witness
intimidation.

michael barbaro

And these are conditions on very serious charges of obstructing a
congressional investigation into Russian meddling in the 2016 election.
That's right. I remember thinking when that happened, like, whoa. This
is the big leagues for Roger Stone.

peter baker

Exactly. And the question is, why is he lying? Why is he obstructing? Is
he trying to protect the president? This is how this all fits together,
right? This goes back to the whole Russian interference. This goes back
to the Mueller probe. This goes back to the things that have dominated
this presidency for three years and frustrated this president for three
years. So he sees Roger Stone's conviction as an illegitimate shot at
him, at himself, the president. A way of trying to take him down because
they couldn't take him down any other way.

michael barbaro

OK. So Peter, how does the president try to protect Stone after this
conviction?

peter baker

So even as he's in the middle of this campaign of retribution against
the Vindman brothers and Gordon Sondland, he is increasingly aware that
the sentencing for Roger Stone is coming up. And then, when Monday comes
around and the prosecutors present their recommendation for a sentence
to the court, the prosecutors ask for seven to nine years behind bars.
That's the normal sentence that would be required under the sentencing
guidelines passed by Congress for crimes of the type that Roger Stone
was convicted of. So they didn't go outside of those guidelines. They
simply said we want to sentence him to what the guidelines say. That
doesn't mean the judge would go along with it, but that was their
recommendation. Well, that set the president off.

\begin{itemize}
\tightlist
\item
  archived recording\\
  The president expressed his outrage on Twitter, calling it a very
  unfair situation, adding, ``Cannot allow this miscarriage of
  justice!''
\end{itemize}

peter baker

In the middle of the night, he starts sending out tweets, angry tweets.
How can this happen? Nine years, this is outrageous. And they're going
after him. How come they don't go after my enemies but they go after
him? And that just sort of sets the town ablaze.

\begin{itemize}
\tightlist
\item
  archived recording\\
  Controversy in the nation's capital now over a sentencing
  recommendation for President Trump's longtime friend Roger Stone.
\end{itemize}

peter baker

Here's a president weighing in directly on a court case involving a
friend of his. This is something that we have not seen really since
Watergate. Presidents don't, especially publicly, weigh in on
prosecutions of people that they are personally connected to, at least
except in the venue of issuing pardons at some point, which they
sometimes do. So this has shocked a lot of people. But what really
shocked a lot of people in Washington was when they woke up a few hours
later on Tuesday and they saw not only these tweets, but they saw that
the attorney general of the United States, Bill Barr, had essentially
overruled the career prosecutors.

\begin{itemize}
\tightlist
\item
  archived recording\\
  Breaking news involving President Trump. A stunning reversal in the
  sentencing recommendation for Trump confidant Roger Stone.
\end{itemize}

peter baker

And said, no, we're not going to ask for a sentence this heavy. We're
going to ask for something lighter.

michael barbaro

So not seven to nine years, something less.

peter baker

Not seven to nine years, something less. It doesn't specify what, but
something below what the guidelines would normally call for. And so this
is causing a huge furor in the U.S. attorney's office in Washington.

\begin{itemize}
\tightlist
\item
  archived recording\\
  What is going on? President Trump knows how to get away with stuff
  when we're not watching.
\end{itemize}

peter baker

The four career prosecutors who worked on the Stone case, all four of
them, quit.

\begin{itemize}
\tightlist
\item
  archived recording\\
  We're following some truly stunning, breaking news, still developing
  by the minute this hour. Federal prosecutors in the Roger Stone
  criminal case have resigned this afternoon.
\end{itemize}

peter baker

One after the other. One, two, three, four, just like that.

\begin{itemize}
\tightlist
\item
  archived recording\\
  This does not happen. Prosecutors don't resign just days before they
  go to sentencing after a case that they've worked so hard on.
\end{itemize}

peter baker

One of them actually quits his job altogether, leaves the Justice
Department as a whole.

\begin{itemize}
\tightlist
\item
  archived recording\\
  In protest.
\end{itemize}

peter baker

Well, they don't say it, but that's the obvious conclusion. Yes, they're
protesting the overruling of their recommendation. And I think that they
felt like they had an ethical obligation. If they had told the court
this is the sentence we think is appropriate, and then suddenly a day
later the same department is coming and saying, no, we don't --- how is
that tenable for them to continue on that case?

michael barbaro

And, Peter, given what has just happened --- the firing of Vindman,
Sondland, Vindman's brother --- what is the reaction to this
intervention? Not just the retribution, but this protection?

peter baker

Well, in effect, the Democrats are saying we told you so, right?

\begin{itemize}
\tightlist
\item
  archived recording (chuck schumer)\\
  No serious person believes President Trump has learned any lesson. He
  doesn't learn any lessons. He does just what he wants, what suits his
  ego at the moment.
\end{itemize}

peter baker

Senator Schumer, the Democratic leader in the upper chamber goes to the
floor and gives a pretty passionate speech in which he says that the
natural consequence of acquitting the president on the Ukraine matter
means that he feels completely unleashed and empowered to do whatever he
thinks is right for his own political interests.

\begin{itemize}
\tightlist
\item
  archived recording (chuck schumer)\\
  We are witnessing a crisis in the rule of law in America, unlike one
  we have ever seen before. It's a crisis of President Trump's making,
  but it was enabled and emboldened by every Senate Republican.
\end{itemize}

peter baker

Even amongst some Republicans, you're seeing you know some discomfort,
particularly among moderate Republicans who tried to give the president
the benefit of the doubt by standing with him in the impeachment trial.
A couple of them had said even, well, maybe he'll have learned a lesson
from all of this and he'll be more measured, he'll be more restrained in
the future and that that would be a good thing. Well, what you're
hearing a lot of people saying is that doesn't seem to be the case. And
I think that the question going forward is going to be, is it just a
burst of energy and lashing out in the days after the acquittal, or is
this the beginning of a month's long recalibration of his
administration? What is he going to do going forward?

michael barbaro

Right. Is this the post-aquittal presidency, one in which enemies are
punished and allies are at all costs protected?

peter baker

Right, exactly. And that the instruments of government are to serve the
president's interests, not just the public's interests.

michael barbaro

Peter, what you have described here is what an old school political
terms might be called a strategy of carrots and sticks, but on steroids,
right? You protect those who have done right by you, and you punish
those who have somehow wronged you. And in the case of the president,
that ``steroided up'' strategy clearly worked when it came to
impeachment. And we talked to you. We talked to many of our colleagues
about the fact that there was genuine fear of crossing this president,
and that that influenced how the Senate voted in the impeachment trial.
So if this strategy is working --- and by all accounts, it is working
--- why shouldn't the president keep it up?

peter baker

Well, it's a great question. I think one of the things we've learned
about the last three years is that the norms, the standards, the lines
that we used to think of that constrained a president were more
aspirational and conceptual than they were legal. You go look. You go
back far enough, you're going to find plenty of presidents who punished
their enemies and protected their friends. But in the post-Watergate
period in particular, when we put in new guardrails, we put a new laws,
we put in new systems, we thought that that had been minimized at the
very least, right? That, yeah, you're going to probably give an
appointment to somebody who's been good to you. And you're going to
maybe take away a grant from the state of somebody who crossed you on a
vote. These things happen. They happen under any presidency. This is
that, as you put it, on steroids. And it's overt. It's right out there
in the open. He wants everybody to know what he's doing. He wants
everybody to understand. You are loyal to this president or you should
get out. And that's true of people in government. That's true of people
even in Congress. He's made very clear that the Republican Party has no
room for anybody who is not on his side. You're either in his camp or
you're not.

michael barbaro

And, of course, there's a larger context here, which is we're in the
middle of a presidential election. And I wonder how this behavior by the
president fits into his re-election strategy.

peter baker

You say that all this fits into a broader approach by this presidential
politics. It's not about unifying. It's about dividing. It's about us
versus them. And this is what the appeal is to his constituents. It is:
I am fighting for you. And they are trying to stop me. It's the deep
state, it's the Democrats, it's the fake news media. They're all trying
to stop me, and by extension, you. And that's why you should stick with
me in this election this fall. So this idea that Washington is all
alarmed by retributions and protections of friends because it violates
norms doesn't hurt his appeal to many of his voters out there, because
it's part of this larger argument that he's making. And the larger
argument is I am a force of disruption. I am a force that is shaking
things up. And the reason why you're seeing things in the news that are
bad about me is because they're fighting back. And you should stay with
me because it's not just me. It's about you too.

michael barbaro

Peter, thank you.

peter baker

OK, thank you.

michael barbaro

On Thursday, in an interview with ABC News, Attorney General Bill Barr
said that the president's interference in cases like Roger Stone's was
making it all but impossible for him to run the Department of Justice.

\begin{itemize}
\tightlist
\item
  archived recording (william barr)\\
  To have public statements and tweets made about the department, about
  people in the department, our men and women here, about cases pending
  in the department and about judges before whom we have cases, make it
  impossible for me to do my job and to assure the courts and the
  prosecutors and the department that we're doing our work with
  integrity.
\end{itemize}

michael barbaro

But Barr did not directly criticize the president, and confirmed in the
interview that he had overruled prosecutors to recommend a more lenient
sentence for Stone.

We'll be right back.

Here's what else you need to know today.

\begin{itemize}
\tightlist
\item
  archived recording\\
  Are there any senators in the chamber wishing to change their vote? If
  not, the yeas are 55, the nays are 45. The joint resolution as amended
  is passed.
\end{itemize}

michael barbaro

On Thursday, a bipartisan majority in the Senate passed a resolution
requiring President Trump to seek authorization from Congress before
taking further military action against Iran. The legislation, which was
already passed by the House, is an unusual move to restrain presidential
power and reflected the growing unease within Congress over Trump's
approach to Iraq, which many fear could lead to all-out war. It follows
Trump's decision six weeks ago to kill Qassim Suleimani, a top Iranian
military commander, without the authorization of Congress.

``The Daily'' is made by Theo Balcomb, Andy Mills, Lisa Tobin, Rachel
Quester, Lynsea Garrison, Annie Brown, Clare Toeniskoetter, Paige
Cowett, Michael Simon Johnson, Brad Fisher, Larissa Anderson, Wendy
Dorr, Chris Wood, Jessica Cheung, Alexandra Leigh Young, Jonathan Wolfe,
Lisa Chow, Eric Krupke, Marc Georges, Luke Vander Ploeg, Adizah Eghan,
Kelly Prime, Julia Longoria, Donna Summer, Jazmin Aguilera, M.J. Davis
Lin, Austin Mitchell, Sayre Quevedo, Neena Pathak, Dan Jimison. Dave
Shaw, Sydney Harper, Daniel Guillematte, Hans Buetow and Robert Jimison.
Our theme music is by Jim Brunberg and Ben Landsverk of Wonderly.
Special thanks to Sam Dolnick, Mikayla Bouchard, Stella Tan, Lauren
Jackson, Julia Simon, Mahima Chablani and Nora Keller. That's it for
``The Daily.'' I'm Michael Barbaro. See you on Tuesday after the
holiday.

\includegraphics{https://static01.nyt.com/images/2020/02/07/us/politics/07dc-vindman/07dc-vindman-articleLarge.jpg?quality=75\&auto=webp\&disable=upscale}

Mr. Sondland took a more measured approach, confirming that he had been
dismissed without offering any protest.

``I was advised today that the president intends to recall me effective
immediately as United States ambassador to the European Union,'' he said
in a statement hours after Colonel Vindman's dismissal. ``I am grateful
to President Trump for having given me the opportunity to serve, to
Secretary Pompeo for his consistent support and to the exceptional and
dedicated professionals at the U.S. Mission to the European Union.''

Mr. Sondland and Colonel Vindman were key witnesses in the House
impeachment hearings. Mr. Sondland, who was deeply involved in the
effort to pressure Ukraine to announce investigations into Mr. Trump's
Democratic rivals, testified that ``we followed the president's orders''
and that ``everyone was in the loop.'' Colonel Vindman, who was on Mr.
Trump's July 25 phone call with Ukraine's president, testified that it
was
\href{https://www.nytimes.com/2019/11/19/us/politics/impeachment-hearings.html}{``improper
for the president''} to coerce a foreign country to investigate a
political opponent.

It may have been untenable for them to keep working for a president with
whom they broke so publicly, but the White House made no effort to
portray the ousters as anything other than reprisals. Mr. Trump
foreshadowed Colonel Vindman's fate hours ahead of time when asked if he
would be pushed out. ``Well, I'm not happy with him,'' the president
told reporters. ``You think I'm supposed to be happy with him? I'm
not.''

The president continued to assail lawmakers who voted for conviction,
targeting Senator Joe Manchin III of West Virginia, who bitterly
disappointed Mr. Trump by sticking with his party. ``I was told by many
that Manchin was just a puppet for Schumer \& Pelosi,''
\href{https://twitter.com/realDonaldTrump/status/1225901404600049664}{Mr.
Trump wrote on Twitter}, referring to Senator Chuck Schumer of New York,
the minority leader, and Speaker Nancy Pelosi. ``That's all he is!''

Even as Mr. Trump flew to North Carolina to
\href{https://www.nytimes.com/2020/02/07/us/politics/trump-north-carolina.html}{highlight
his economic record}, he called on the House to ``expunge'' his
impeachment, an idea with no precedent or basis in the Constitution.
``They should because it was a hoax,'' he told reporters. ``It was a
total political hoax.'' And he accused Ms. Pelosi of committing a crime
by
\href{https://www.nytimes.com/2020/02/04/us/politics/pelosi-trump-handshake.html}{ripping
up a copy} of his State of the Union address. ``She broke the law,'' he
asserted.

The president's critics had warned that
\href{https://www.nytimes.com/2020/02/01/us/politics/trump-impeachment-trial.html}{he
would feel unbound if acquitted}, and some said the dismissals proved
their point, quickly calling them ``the Friday night massacre,'' as
Senator Mark Warner, Democrat of Virginia, put it.

``These are the actions of a man who
\href{https://twitter.com/RepAdamSchiff/status/1225952767803523073}{believes
he is above the law},'' said Representative Adam B. Schiff, Democrat of
California and the lead House impeachment manager. Mr. Schumer said the
White House was running from the truth. ``This action is not a sign of
strength,'' he said. ``It only shows President Trump's weakness.'' Ms.
Pelosi said, ``This goes too far.'' At the
\href{https://www.nytimes.com/2020/02/07/us/politics/democratic-debate-tonight.html?action=click\&module=Top\%20Stories\&pgtype=Homepage}{Democratic
presidential debate in New Hampshire}, former Vice President Joseph R.
Biden Jr. asked the audience to stand in support of Colonel Vindman.

The White House would not discuss the Vindman decision. ``We do not
comment on personnel matters,'' said John Ullyot, a spokesman for the
National Security Council.

Donald Trump Jr., the president's eldest son, celebrated the dismissals,
offering mock thanks to Mr. Schiff. ``Were it not for his crack
investigation skills, @realDonaldTrump might have had a tougher time
unearthing who all needed to be fired,''
\href{https://twitter.com/DonaldJTrumpJr/status/1225941861765918720}{he
tweeted}.

``The president had every right to make the moves that he did today,''
Representative Lee Zeldin, Republican of New York, said in an interview.
``Moving Lt. Col. Vindman, for example, is a good move based on the fact
that there is a lack of trust. He disagrees with the president's
policies.'' As for Mr. Sondland, ``the president can recall an
ambassador at any time with or without cause, and in the case of Gordon
Sondland, the guy was a hot mess, anyway.''

Representative Matt Gaetz, Republican of Florida, expressed no regrets
over Mr. Sondland's dismissal. ``Somehow I think America will be able to
deliver foreign policy without Gordon Sondland,'' he said by text
message.

Other witnesses have left with less drama recently. Marie L.
Yovanovitch, the ambassador to Ukraine who was recalled from her post
last spring because she was seen as an obstacle to the president's
plans,
\href{https://www.nytimes.com/2020/01/31/us/politics/ambassador-ukraine-impeachment-retires.html}{retired
last month from the Foreign Service}. William B. Taylor Jr., who
replaced her in an acting capacity,
\href{https://www.nytimes.com/2019/12/17/world/europe/william-taylor-ukraine.html}{was
essentially brought back early}, as well. And Jennifer Williams, a
career official working for Vice President Mike Pence, quietly returned
to the Defense Department.

Several had already left government, like Fiona Hill, the Europe policy
chief at the National Security Council, and Kurt D. Volker, the special
envoy for Ukraine, who
\href{https://www.nytimes.com/2019/09/27/us/politics/volker-ukraine-resigns.html}{resigned
days before testifying}. But others remain, including
\href{https://www.nytimes.com/2019/11/13/us/politics/who-is-george-kent.html}{George
P. Kent} at the State Department,
\href{https://www.nytimes.com/2019/11/20/us/politics/laura-cooper-impeachment.html}{Laura
Cooper} at the Defense Department and
\href{https://www.nytimes.com/2019/11/21/us/politics/david-holmes-impeachment.html}{David
Holmes} at the embassy in Ukraine.

Mr. Sondland began discussions with senior officials about leaving
shortly after he testified in November, according to two people briefed
on the matter. He believed that remaining would be unrealistic given his
role in impeachment and hoped to exit gracefully.

A decision on timing was postponed until after impeachment, but on
Friday, State Department officials told Mr. Sondland that they wanted
him to resign, the people said. Mr. Sondland relayed to them that he
would not step down amid what was clearly a purge of impeachment
witnesses and that he would have to be fired, they said. In response,
State Department officials recalled him.

Colonel Vindman's brother seemed to be collateral damage. Yevgeny
Vindman, who goes by Eugene, worked as a lawyer for the National
Security Council and had no role in the impeachment hearings other than
showing up to sit behind his brother when he appeared in November. He
was given no explanation for his dismissal ``despite over two decades of
loyal service to this country,'' said Mr. Pressman, the lawyer. ``He
deeply regrets that he will not be able to continue his service at the
White House.''

Both Vindmans, whose White House tours were scheduled to last until
July, will retain their Army ranks and return to military service.
Alexander Vindman, who had been expecting the move and had begun
removing personal items, was told he would go to the Pentagon before
moving to the National War College in July as originally planned.
Yevgeny Vindman was more surprised and was told he would report to the
office of the Army general counsel.

Defense Secretary Mark T. Esper said service members who return to the
military would be welcomed back. ``We protect all of our persons,
service members, from retribution or anything like that,'' he told
reporters.

Mr. Trump, on the other hand, has made clear his personal antipathy for
both Vindmans. ``Lieutenant Colonel Vindman and his twin brother,
right?'' he said on Thursday during a
\href{https://www.nytimes.com/2020/02/06/us/politics/trump-impeachment.html}{rambling
hourlong venting session} at the White House, his voice dripping with
disdain. ``We had some people that --- really amazing.''

On Friday, Mr. Trump
\href{https://twitter.com/TomFitton/status/1196980591398457346}{retweeted
a message} from a supporter advocating Alexander Vindman's dismissal:
``Vindman's behavior is a scandal. He should be removed from the
@RealDonaldTrump White House ASAP to protect our foreign policy from his
machinations.''

Senator Susan Collins of Maine, a Republican who voted to acquit the
president but expressed hope that he would learn a lesson from the
impeachment, said witnesses should not be punished. ``I obviously am not
in favor of any kind of retribution against anyone who came forward with
evidence,'' she said in Maine,
\href{https://www.pressherald.com/2020/02/07/collins-says-retribution-after-impeachment-acquital-would-not-be-appropriate/}{according
to The Portland Press Herald}.

Colonel Vindman has been subjected
\href{https://www.nytimes.com/2019/11/06/us/politics/trump-vindman-twitter.html}{to
virulent attacks} on his patriotism on Fox News and social media. The
president called him a ``Never Trumper,'' a term the colonel rejected.
\href{https://www.nytimes.com/2019/11/20/us/alexander-vindman-fox-news-espionage.html}{Fox
aired a segment} suggesting his service in the White House might amount
to ``espionage.'' And Senator Marsha Blackburn, Republican of Tennessee,
\href{https://twitter.com/MarshaBlackburn/status/1220452721616216087?ref_src=twsrc\%5Etfw\%7Ctwcamp\%5Etweetembed\%7Ctwterm\%5E1220452721616216087\&ref_url=https\%3A\%2F\%2Fwww.washingtonpost.com\%2Fpolitics\%2F2020\%2F01\%2F23\%2Fgop-senator-flimsily-impugns-key-impeachment-witness-again\%2F}{attacked
him on Twitter}: ``How patriotic is it to badmouth and ridicule our
great nation in front of Russia, America's greatest enemy?''

With impeachment over, Mr. Trump is debating additional personnel
changes. Some advisers are encouraging him to part ways with his acting
chief of staff, Mick Mulvaney, who was involved in freezing
\href{https://www.nytimes.com/2019/12/29/us/politics/trump-ukraine-military-aid.html}{security
aid} to Ukraine, which paved the way for impeachment.

Other advisers are telling Mr. Trump to delay major changes until after
the November election. Some hope that Representative Mark Meadows,
Republican of North Carolina, will join the White House as a senior
adviser. Mr. Meadows traveled with the president on Friday to North
Carolina.

Mr. Trump denied that Mr. Mulvaney would be pushed out in favor of Mr.
Meadows. ``I have a great relationship with Mick,'' the president told
reporters on Friday. ``I have a great relationship with Mark. And it's
false.''

Peter Baker and Michael S. Schmidt reported from Washington, and Maggie
Haberman and Danny Hakim from New York. Lola Fadulu contributed
reporting from Charlotte, N.C.

Advertisement

\protect\hyperlink{after-bottom}{Continue reading the main story}

\hypertarget{site-index}{%
\subsection{Site Index}\label{site-index}}

\hypertarget{site-information-navigation}{%
\subsection{Site Information
Navigation}\label{site-information-navigation}}

\begin{itemize}
\tightlist
\item
  \href{https://help.nytimes.com/hc/en-us/articles/115014792127-Copyright-notice}{©~2020~The
  New York Times Company}
\end{itemize}

\begin{itemize}
\tightlist
\item
  \href{https://www.nytco.com/}{NYTCo}
\item
  \href{https://help.nytimes.com/hc/en-us/articles/115015385887-Contact-Us}{Contact
  Us}
\item
  \href{https://www.nytco.com/careers/}{Work with us}
\item
  \href{https://nytmediakit.com/}{Advertise}
\item
  \href{http://www.tbrandstudio.com/}{T Brand Studio}
\item
  \href{https://www.nytimes.com/privacy/cookie-policy\#how-do-i-manage-trackers}{Your
  Ad Choices}
\item
  \href{https://www.nytimes.com/privacy}{Privacy}
\item
  \href{https://help.nytimes.com/hc/en-us/articles/115014893428-Terms-of-service}{Terms
  of Service}
\item
  \href{https://help.nytimes.com/hc/en-us/articles/115014893968-Terms-of-sale}{Terms
  of Sale}
\item
  \href{https://spiderbites.nytimes.com}{Site Map}
\item
  \href{https://help.nytimes.com/hc/en-us}{Help}
\item
  \href{https://www.nytimes.com/subscription?campaignId=37WXW}{Subscriptions}
\end{itemize}
