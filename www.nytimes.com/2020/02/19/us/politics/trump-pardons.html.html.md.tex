Sections

SEARCH

\protect\hyperlink{site-content}{Skip to
content}\protect\hyperlink{site-index}{Skip to site index}

\href{https://www.nytimes.com/section/politics}{Politics}

\href{https://myaccount.nytimes.com/auth/login?response_type=cookie\&client_id=vi}{}

\href{https://www.nytimes.com/section/todayspaper}{Today's Paper}

\href{/section/politics}{Politics}\textbar{}The 11 Criminals Granted
Clemency by Trump Had One Thing in Common: Connections

\url{https://nyti.ms/38Jqez0}

\begin{itemize}
\item
\item
\item
\item
\item
\item
\end{itemize}

Advertisement

\protect\hyperlink{after-top}{Continue reading the main story}

Supported by

\protect\hyperlink{after-sponsor}{Continue reading the main story}

\hypertarget{the-11-criminals-granted-clemency-by-trump-had-one-thing-in-common-connections}{%
\section{The 11 Criminals Granted Clemency by Trump Had One Thing in
Common:
Connections}\label{the-11-criminals-granted-clemency-by-trump-had-one-thing-in-common-connections}}

The process bypassed the formal procedures used by past presidents and
was driven instead by friendship, fame, personal empathy and a shared
sense of persecution.

\includegraphics{https://static01.nyt.com/images/2020/02/19/us/politics/19dc-pardon1/merlin_169111542_af3b58f0-a87a-4194-b641-8c3436aaf4ce-articleLarge.jpg?quality=75\&auto=webp\&disable=upscale}

By \href{https://www.nytimes.com/by/peter-baker}{Peter Baker},
\href{https://www.nytimes.com/by/j-david-goodman}{J. David Goodman},
\href{https://www.nytimes.com/by/michael-rothfeld}{Michael Rothfeld} and
\href{https://www.nytimes.com/by/elizabeth-williamson}{Elizabeth
Williamson}

\begin{itemize}
\item
  Feb. 19, 2020
\item
  \begin{itemize}
  \item
  \item
  \item
  \item
  \item
  \item
  \end{itemize}
\end{itemize}

WASHINGTON --- Early Tuesday morning, Bernard B. Kerik's telephone rang.
On the line was David Safavian, a friend and fellow former government
official who like Mr. Kerik was once imprisoned for misconduct. Mr.
Safavian had life-changing news.

Mr. Safavian, who had ties to the White House, said that he was putting
together a letter asking President Trump to pardon Mr. Kerik, the former
New York City police commissioner who pleaded guilty to tax fraud and
other charges. Mr. Safavian needed names of supporters to sign the
letter. By noon.

Mr. Kerik hit the phones. Shortly after 10 a.m., he reached Geraldo
Rivera, the Fox News correspondent and a friend of Mr. Trump's. Mr.
Rivera, who described Mr. Kerik as ``an American hero,'' instantly
agreed to sign the one-page letter. Mr. Kerik called Representative
Peter T. King, Republican of New York, and when Mr. Safavian reached Mr.
King around 10:30, he too agreed to sign.

At 11:57 a.m., Mr. Kerik's phone rang again. This time it was the
president.

``He said, `As we speak, I am signing a full presidential pardon on your
behalf,''' Mr. Kerik recalled in an interview on Wednesday. ``Once he
started talking and I realized what we were talking about, I got
emotional.''

At 1:41 p.m., Mr. Trump approached reporters before boarding Air Force
One and mentioned that he had pardoned Mr. Kerik. At 2:10, the White
House announced that Mr. Safavian had been pardoned as well.

\href{https://www.nytimes.com/2020/02/18/us/politics/trump-pardon-blagojevich-debartolo.html?action=click\&module=Top\%20Stories\&pgtype=Homepage}{The
clemency orders that the president issued} that day to celebrity felons
like Mr. Kerik,
\href{https://www.nytimes.com/2020/02/18/us/politics/trump-pardon-blagojevich-debartolo.html?action=click\&module=Top\%20Stories\&pgtype=Homepage}{Rod
R. Blagojevich} and Michael R. Milken came about through a typically
Trumpian process, an ad hoc scramble that bypassed the formal procedures
used by past presidents and was driven instead by friendship, fame,
personal empathy and a shared sense of persecution. While aides said the
timing was random, it reinforced Mr. Trump's antipathy toward the law
enforcement establishment.

\href{https://www.nytimes.com/2020/02/18/us/politics/trump-pardons.html}{All
11 recipients} had an inside connection or were promoted on Fox News.
Some were vocal supporters of Mr. Trump, donated to his campaign or in
one case had a son who weekended in the Hamptons with the president's
eldest son. Even three obscure women serving time on drug or fraud
charges got on Mr. Trump's radar screen through a personal connection.

While \href{https://www.justice.gov/pardon/clemency-statistics}{14,000
clemency petitions} sit unaddressed at the Justice Department's Office
of the Pardon Attorney, Mr. Trump eagerly granted relief to a former
football team owner who hosted a pre-inauguration party, a onetime
contestant on ``Celebrity Apprentice'' and an infamous investor
championed both by Rudolph W. Giuliani, the president's personal lawyer,
and by the billionaire who hosted a \$10 million fund-raiser for Mr.
Trump just last weekend.

\includegraphics{https://static01.nyt.com/images/2020/02/19/us/politics/19dc-pardon2/merlin_169163676_ccbe11ad-e508-47cb-a192-d744c97cd2fa-articleLarge.jpg?quality=75\&auto=webp\&disable=upscale}

``There is now no longer any pretense of regularity,'' said Margaret
Love, who served as pardon attorney under President Bill Clinton and now
represents clients seeking clemency. ``The president seems proud to
declare that he makes his own decisions without relying on any official
source of advice, but acts on the recommendation of friends, colleagues
and political allies.''

Mr. Trump's advisers acknowledged that the process was unique to this
president, but stressed that he had become personally committed to
countering the excesses of the criminal justice system, a mission fueled
by his own scalding encounters with investigations since taking office.
In addition to his pardons, Mr. Trump in 2018 signed the First Step Act
providing sentencing relief for many criminals.

``The president seems to be someone who's willing to listen to people's
appeals,'' said Robert Blagojevich, who lobbied for a commutation for
his brother, Rod Blagojevich, the former governor of Illinois
\href{https://www.nytimes.com/2011/12/08/us/blagojevich-expresses-remorse-in-courtroom-speech.html}{sentenced
to 14 years} for trying to essentially sell the Senate seat vacated by
President Barack Obama. ``I think he's just got an antenna to listen to
people who have been truly wronged by the system.''

Indeed, Mr. Trump takes personal pleasure in dispensing mercy. He called
Patti Blagojevich, who is married to the former governor, right after
signing the papers on Tuesday. He likewise called Ricky Munoz to tell
him that his wife, Crystal Munoz, was coming home.

Advisers said there is little rhyme or reason to how Mr. Trump chooses
clemency recipients. He meets with advisers every few weeks to discuss
various cases. Once he makes a decision, he tends to announce them right
away, without bothering to draft a communications strategy, reasoning
that there is no point in anyone sitting in prison longer than needed.

Mr. Trump recognizes that his friends-and-family approach generates
criticism, but has repeatedly cited
\href{https://www.nytimes.com/2018/04/13/us/politics/trump-pardon-scooter-libby.html}{his
2018 pardon of I. Lewis Libby Jr.} as proof that he is willing to absorb
attacks that others would not. President George W. Bush refused to
pardon Mr. Libby, who served as chief of staff to Vice President Dick
Cheney and was convicted of lying to the authorities.

Mr. Trump has known some of those he favored this week for years,
including Mr. Kerik and Mr. Milken, the so-called junk bond king who
tried at least twice to obtain a pardon from Mr. Bush without success.
Mr. Trump called Mr. Milken ``a brilliant guy'' in his first memoir and
has hosted him at his Mar-a-Lago estate in Florida. He called Mr. Kerik
``a friend of mine'' and ``a great guy'' in 2004 when Mr. Kerik
\href{https://www.nytimes.com/2004/12/11/politics/kerik-pulls-out-as-bush-nominee-for-homeland-security-job.html}{was
forced to withdraw his nomination} for Mr. Bush's secretary of homeland
security because of ethics issues.

In addition to Mr. Giuliani, Mr. Milken's pardon was supported by Mr.
Trump's son-in-law Jared Kushner and his developer friends Howard Lorber
and Richard LeFrak. Also supportive was Treasury Secretary Steven
Mnuchin, a longtime friend who
\href{https://www.nytimes.com/2019/01/18/us/politics/mnuchin-private-flight-michael-milken.html}{last
year flew on Mr. Milken's private jet} from Washington to Los Angeles
and helped secure a real estate tax break that could benefit Mr. Milken.

Paul Pogue, the former owner of a Texas construction company, was
pardoned for tax charges after his family contributed more than
\$200,000 in the last six months to help re-elect Mr. Trump. In August,
his son Benjamin and daughter-in-law Ashleigh posted a picture on
Instagram of themselves with Donald Trump Jr. and his girlfriend,
Kimberly Guilfoyle, in the Hamptons. ``What an experience spending the
weekend with these two and more!'' Ms. Pogue wrote.

In announcing his pardon, the White House cited Paul Pogue's charitable
work around the world, including the creation of two nonprofit
organizations that help rebuild churches and provide aid to people after
natural disasters.

Ariel Friedler, the former executive of a software development company
who pleaded guilty to conspiring to hack a competitor, found his way in
the door through Chris Christie, the former governor of New Jersey and a
close ally of Mr. Trump's.

Mr. Christie said on Wednesday that he met with Mr. Friedler in person
and agreed to represent him in a pardon application after being referred
by a former prosecutor he knew. Mr. Christie said he heard nothing since
2018 about the case until Mr. Trump called him out of the blue last
Thursday to ask about it.

``He said, `Listen, I've reviewed the application, but tell me what you
think about this guy and what happened to him,''' Mr. Christie said. A
former prosecutor himself, Mr. Christie said he told the president that
the government had overreached.

``Do you really think this guy has a good heart?'' he recalled Mr. Trump
asking.

``I'm not soft,'' Mr. Christie said he replied, ``but this is over the
top.''

Angela Stanton, an author and television personality with a record
stemming from a stolen-vehicle ring, was championed by Alveda King, a
niece of the Rev. Dr. Martin Luther King Jr. A Fox News contributor and
outspoken Trump supporter, Ms. King appeared with Ms. Stanton at a
``Women for Trump'' summit meeting in 2018.

While most of this week's recipients had political ties, Mr. Trump's
defenders pointed to three women whose sentences he commuted without any
notable political background. But even those three --- Ms. Munoz, 40,
Tynice Nichole Hall, 36, and Judith Negron, 48 --- came to his attention
because of someone he already knew, Alice Marie Johnson.

Mr. Trump
\href{https://www.nytimes.com/2018/06/06/us/politics/trump-alice-johnson-sentence-commuted-kim-kardashian-west.html}{commuted
Ms. Johnson's life sentence} for a nonviolent drug conviction in 2018
after the reality television star Kim Kardashian West made a personal
plea. Since then, Ms. Johnson has become his prison reform whisperer and
\href{https://www.nytimes.com/2020/02/04/us/politics/trump-super-bowl-ad.html}{appeared
in a multimillion-dollar Super Bowl ad} for his campaign.

During an October appearance at Benedict College, a historically black
school in South Carolina, Mr. Trump told Ms. Johnson to give him the
names of others who had been mistreated. Ms. Johnson then traveled to
Washington to meet with prisoner advocates and they identified about 10
women for the White House.

Ms. Johnson served in prison with all three of those released this week
by Mr. Trump. ``They don't have Kim Kardashian, but they have me to
fight for them,'' she said in an interview. She was especially close to
Ms. Munoz. ``Crystal was like my daughter in prison,'' she said. ``In
fact, I called her my prison daughter.''

Ms. Negron, who was sentenced to 35 years for Medicare fraud, filed a
clemency petition years ago but it ``disappeared into the bowels of the
government,'' according to her lawyer, Bill Norris. She was stunned to
learn that the president had suddenly ordered her freed. ``I'm indebted
to him,'' she said on Wednesday. ``He gave us our dream come true. He
gave me back my family. He gave me back our home. Just a new life. The
nightmare is over.''

Ms. Munoz, serving nearly two decades on a marijuana charge, said that
she was called to the office of her case manager and counselor on
Tuesday. ``When I went into their office, they said, `Who do you know?
Do you know some people?''' She did not understand at first. But the
person she knew had secured her a commutation.

Advocates for justice overhaul said Mr. Trump should be praised for his
interventions. ``Some people are trying to bash Trump for letting people
circumvent the process and go directly to the White House,'' said Amy
Ralston Povah, the founder of the Clemency for All Nonviolent Drug
Offenders Foundation. ``But the system is broken.''

Image

Judith Negron, center, hugging her sons after Mr. Trump commuted her
sentence on Tuesday in Orlando, Fla.Credit...Phelan M. Ebenhack for The
New York Times

Among those activists these days is Mr. Safavian, the government's top
procurement official under Mr. Bush who was sentenced to a year in
prison for covering up ties to the corrupt lobbyist Jack Abramoff. Now
the general counsel for the American Conservative Union, Mr. Safavian
lobbies for legislation and programs granting leniency and job training
for lower-level drug offenders as well as white-collar former convicts
like himself.

Not everyone believes in his conversion. Walter Shaub, the former
director of the Office of Government Ethics, said Mr. Trump's pardon of
Mr. Safavian sent a message to dishonest officials to ``wait long enough
and a corrupt president may bless your corruption.''

But others, including the liberal CNN commentator Van Jones, praised Mr.
Safavian's work to redeem the system, calling him ``a quiet wonder'' and
declining to second-guess the pardon. ``I'm not going to criticize
freedom,'' Mr. Jones said. ``I want more people to be able to come
home.''

As with the others, Mr. Safavian had friends in the right places. The
head of the conservative union, Matt Schlapp, is a strong supporter of
Mr. Trump, and his wife, Mercedes Schlapp, worked as the White House
strategic communications director before moving to the president's
campaign.

As he pushed for Mr. Kerik's pardon, Mr. Safavian said he did not
realize that he would receive one himself. ``Quite frankly, it was out
of the blue for me,'' he said. ``I was in the drive-through window at
McDonald's when I got the call that the president had just signed my
pardon.

``I had zero role in the pardon process,'' he added. ``None. I didn't
ask for it.''

Peter Baker and Elizabeth Williamson reported from Washington, and J.
David Goodman and Michael Rothfeld from New York. Reporting was
contributed by Annie Karni, Zach Montague, Alan Rappeport and Michael D.
Shear in Washington; Maggie Haberman and Jesse Drucker in New York;
Mitch Smith in Chicago; Patricia Mazzei and Jack Begg in Miami; and
Manny Fernandez in Houston.

Advertisement

\protect\hyperlink{after-bottom}{Continue reading the main story}

\hypertarget{site-index}{%
\subsection{Site Index}\label{site-index}}

\hypertarget{site-information-navigation}{%
\subsection{Site Information
Navigation}\label{site-information-navigation}}

\begin{itemize}
\tightlist
\item
  \href{https://help.nytimes.com/hc/en-us/articles/115014792127-Copyright-notice}{©~2020~The
  New York Times Company}
\end{itemize}

\begin{itemize}
\tightlist
\item
  \href{https://www.nytco.com/}{NYTCo}
\item
  \href{https://help.nytimes.com/hc/en-us/articles/115015385887-Contact-Us}{Contact
  Us}
\item
  \href{https://www.nytco.com/careers/}{Work with us}
\item
  \href{https://nytmediakit.com/}{Advertise}
\item
  \href{http://www.tbrandstudio.com/}{T Brand Studio}
\item
  \href{https://www.nytimes.com/privacy/cookie-policy\#how-do-i-manage-trackers}{Your
  Ad Choices}
\item
  \href{https://www.nytimes.com/privacy}{Privacy}
\item
  \href{https://help.nytimes.com/hc/en-us/articles/115014893428-Terms-of-service}{Terms
  of Service}
\item
  \href{https://help.nytimes.com/hc/en-us/articles/115014893968-Terms-of-sale}{Terms
  of Sale}
\item
  \href{https://spiderbites.nytimes.com}{Site Map}
\item
  \href{https://help.nytimes.com/hc/en-us}{Help}
\item
  \href{https://www.nytimes.com/subscription?campaignId=37WXW}{Subscriptions}
\end{itemize}
