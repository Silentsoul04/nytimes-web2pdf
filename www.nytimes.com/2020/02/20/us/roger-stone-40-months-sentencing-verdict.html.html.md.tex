Sections

SEARCH

\protect\hyperlink{site-content}{Skip to
content}\protect\hyperlink{site-index}{Skip to site index}

\href{https://www.nytimes.com/section/us}{U.S.}

\href{https://myaccount.nytimes.com/auth/login?response_type=cookie\&client_id=vi}{}

\href{https://www.nytimes.com/section/todayspaper}{Today's Paper}

\href{/section/us}{U.S.}\textbar{}Roger Stone Is Sentenced to Over 3
Years in Prison

\url{https://nyti.ms/37PBWad}

\begin{itemize}
\item
\item
\item
\item
\item
\item
\end{itemize}

\begin{itemize}
\item
  \href{https://www.nytimes.com/2020/07/31/us/elections/biden-vs-trump.html?action=click\&pgtype=Article\&state=default\&region=TOP_BANNER\&context=storylines_menu}{Election
  Updates}
\item
  \href{https://www.nytimes.com/article/biden-vice-president-2020.html?action=click\&pgtype=Article\&state=default\&region=TOP_BANNER\&context=storylines_menu}{Biden's
  V.P. Search}
\item
  \href{https://www.nytimes.com/interactive/2020/07/24/us/politics/trump-biden-campaign-donors.html?action=click\&pgtype=Article\&state=default\&region=TOP_BANNER\&context=storylines_menu}{Map
  of Donations}
\item
  \href{https://www.nytimes.com/interactive/2020/us/elections/delegate-count-primary-results.html?action=click\&pgtype=Article\&state=default\&region=TOP_BANNER\&context=storylines_menu}{Delegate
  Count}
\item
  \href{https://www.nytimes.com/interactive/2019/us/politics/2020-presidential-candidates.html?action=click\&pgtype=Article\&state=default\&region=TOP_BANNER\&context=storylines_menu}{The
  Candidates}
\item
  \href{https://www.nytimes.com/newsletters/politics?action=click\&pgtype=Article\&state=default\&region=TOP_BANNER\&context=storylines_menu}{Politics
  Newsletter}
\end{itemize}

Advertisement

\protect\hyperlink{after-top}{Continue reading the main story}

Supported by

\protect\hyperlink{after-sponsor}{Continue reading the main story}

\hypertarget{roger-stone-is-sentenced-to-over-3-years-in-prison}{%
\section{Roger Stone Is Sentenced to Over 3 Years in
Prison}\label{roger-stone-is-sentenced-to-over-3-years-in-prison}}

The sentencing played out amid extraordinary upheaval at the Justice
Department and a virtual standoff between the president and the attorney
general.

\includegraphics{https://static01.nyt.com/images/2020/02/20/us/politics/20dc-stone-sub2/20dc-stone-sub2-videoSixteenByNine3000.jpg}

\href{https://www.nytimes.com/by/sharon-lafraniere}{\includegraphics{https://static01.nyt.com/images/2018/07/12/multimedia/author-sharon-lafraniere/author-sharon-lafraniere-thumbLarge.png}}

By \href{https://www.nytimes.com/by/sharon-lafraniere}{Sharon
LaFraniere}

\begin{itemize}
\item
  Published Feb. 20, 2020Updated July 19, 2020
\item
  \begin{itemize}
  \item
  \item
  \item
  \item
  \item
  \item
  \end{itemize}
\end{itemize}

WASHINGTON ---
\href{https://www.nytimes.com/2020/07/19/us/politics/roger-stone-mo-kelly-slur.html}{Roger
J. Stone Jr.}, a longtime friend and adviser of President Trump, was
sentenced Thursday to more than three years in prison in a politically
fraught case that put the president at odds with his attorney general,
stirred widespread consternation in the Justice Department and provoked
the judge in the case to denounce pressure on the justice system.

In announcing the 40-month sentence, Judge Amy Berman Jackson of United
States District Court in Washington suggested that attacks on federal
judges, prosecutors and juries should be a wake-up call about the
threats now endangering an independent justice system. While she never
mentioned Mr. Trump by name, her remarks seemed directed at him.

``The dismay and the disgust at the attempts by others to defend his
actions as just business as usual in our polarized climate should
transcend party,'' the judge said of Mr. Stone. ``The dismay and disgust
at any attempt to interfere with the efforts of prosecutors and members
of the judiciary to fulfill their duty should transcend party.''

The case
\href{https://www.nytimes.com/2020/02/11/us/politics/roger-stone-sentencing.html}{was
thrown into disarray} last week when Attorney General William P. Barr
overruled
\href{https://www.nytimes.com/2020/02/10/us/roger-stone-prison-sentence.html}{a
sentencing recommendation} by four career prosecutors, who then quit the
case in protest. Mr. Barr said he decided on his own that the
prosecutors' request for a prison term of seven to nine years was too
harsh. But his move coincided with
\href{https://www.nytimes.com/2020/02/14/us/politics/trump-william-barr.html}{Mr.
Trump's public complaints} about the prosecutors' recommendation and
\href{https://www.nytimes.com/2020/02/12/us/politics/justice-department-roger-stone-sentencing.html}{elicited
widespread criticism} that he had bent to the president's will.

The attorney general, facing a backlash within the department,
\href{https://www.nytimes.com/2020/02/13/us/politics/william-barr-trump.html}{asked
Mr. Trump in a nationally televised interview} to cease his running
commentary about the department's criminal cases.

Yet less than three hours after Mr. Stone was sentenced, the president
declared he should be ``exonerated,'' echoing the defense team's
arguments in detail. Speaking in Las Vegas, he said Mr. Stone was the
victim of ``a bad jury'' led by an anti-Trump activist And he suggested
that he would use his clemency power to spare Mr. Stone if Judge Jackson
did not agree to a new trial sought by defense lawyers.

A jury in November
\href{https://www.nytimes.com/2019/11/15/us/politics/roger-stone-trial-guilty.html}{convicted
Mr. Stone of seven felony charges}, including lying under oath to a
congressional committee and threatening a witness whose testimony would
have exposed those lies. In biting tones, Judge Jackson dismissed any
notion that the case lacked merit.

She said that Mr. Stone hindered a congressional inquiry of national
importance because the truth would have embarrassed the president and
his 2016 campaign. The documentary evidence alone, she said, proved that
Mr. Stone deceived the House Intelligence Committee about his efforts to
obtain information from WikiLeaks about Democratic emails that had been
stolen by Russian operatives who sought to influence the 2016
presidential election.

``He was not prosecuted, as some have complained, for standing up for
the president. He was prosecuted for covering up for the president,''
the judge said. In government inquiries, she added, ``the truth still
exists. The truth still matters.'' Otherwise, she said, ``everyone
loses.''

Judge Jackson took special umbrage at the defense team's argument that
Mr. Stone's deception was of no real consequence. ``Sure, defense is
free to say, `Who cares?''' she said. ``But I will say this: Congress
cared.'' So, too, she said, did the Justice Department and U.S.
attorney's office, which brought the case, and the jurors who heard the
evidence.

\hypertarget{latest-updates-2020-election}{%
\section{\texorpdfstring{\href{https://www.nytimes.com/2020/07/31/us/elections/biden-vs-trump.html?action=click\&pgtype=Article\&state=default\&region=MAIN_CONTENT_1\&context=storylines_live_updates}{Latest
Updates: 2020
Election}}{Latest Updates: 2020 Election}}\label{latest-updates-2020-election}}

Updated 2020-08-01T01:26:45.732Z

\begin{itemize}
\tightlist
\item
  \href{https://www.nytimes.com/2020/07/31/us/elections/biden-vs-trump.html?action=click\&pgtype=Article\&state=default\&region=MAIN_CONTENT_1\&context=storylines_live_updates\#link-29fdff45}{Kamala
  Harris, a top vice-presidential contender, confronts double
  standards.}
\item
  \href{https://www.nytimes.com/2020/07/31/us/elections/biden-vs-trump.html?action=click\&pgtype=Article\&state=default\&region=MAIN_CONTENT_1\&context=storylines_live_updates\#link-13ec3d9c}{Karen
  Bass and Susan Rice are rising on Biden's vice-presidential
  shortlist.}
\item
  \href{https://www.nytimes.com/2020/07/31/us/elections/biden-vs-trump.html?action=click\&pgtype=Article\&state=default\&region=MAIN_CONTENT_1\&context=storylines_live_updates\#link-49e9a016}{Trump
  says Russian bounties to kill U.S. troops `never took place.'}
\end{itemize}

\href{https://www.nytimes.com/2020/07/31/us/elections/biden-vs-trump.html?action=click\&pgtype=Article\&state=default\&region=MAIN_CONTENT_1\&context=storylines_live_updates}{See
more updates}

``The American people cared. And I care,'' she declared.

In his remarks in Las Vegas, Mr. Trump made clear he was not satisfied
with the outcome of the case.

``I'm not going to do anything in terms of the great powers bestowed on
a president of the United States,'' he said. ``I want the process to
play out. I think that's the best thing to do because I'd love to see
Roger exonerated --- and I'd love to see it happen --- because
personally I think he was treated unfairly. ''

He said he would wait to see how the case was ultimately resolved.

``We will watch the process and watch it very closely,'' the president
added. ``And at some point, I will make a determination. But Roger Stone
and everybody has to be treated fairly. And this has not been a fair
process, OK?''

Judge Jackson, who spent six years in the U.S. attorney's office in
Washington that handled the case, interrogated the prosecutor who
replaced the four who quit over the change in sentencing recommendation
ordered by Mr. Barr. In a second sentencing recommendation, prosecutors
said ``far less'' of a prison term than seven to nine years was
warranted, but left the length of incarceration up to the judge.

Why, she asked, had the prosecutors scrapped Justice Department policy
--- as well as the usual practice of the office --- and sought a more
lenient punishment than advisory sentencing guidelines suggested?

``As I understand it, you are representing the United States of
America,'' she told John Crabb Jr., an assistant United States attorney,
with a trace of sarcasm. ``I fear you know less about the case, saw less
of the testimony and exhibits than just about every other person in this
courtroom.''

``Is there anything you would like to say about why you are the one
standing here?'' she asked.

Clearly uncomfortable, Mr. Crabb apologized to the judge for ``the
confusion the government has caused.'' He went out of his way to
compliment her, saying the Justice Department trusted her to decide
because of her expertise in such cases and ``this court's record of
thoughtful analysis and fair sentences.''

Mr. Crabb defended the prosecution as a ``righteous'' effort to hold Mr.
Stone to account for ``serious'' crimes. And he said the prosecutors who
resigned from the case were not to blame for the confusion, instead
blaming ``a miscommunication'' between Mr. Barr and Timothy Shea, the
newly appointed United States attorney for the District of Columbia.

``The Department of Justice and the United States attorney's office is
committed to enforcing the law without fear, favor or political
influence,'' he said.

But Mr. Crabb raised yet more questions by simultaneously defending the
argument for a stiff sentence laid out in the office's first sentencing
memo without disavowing the second sentencing memo that argued for a
lighter punishment.

``I'm not at liberty to discuss internal deliberations,'' he said when
the judge pressed him for details. ``I apologize.''

Judge Jackson said she agreed with the Justice Department's second memo
that the sentencing guidelines were too harsh. But she said she wondered
if the government would adopt that same stance in future cases or
whether it was reserved only for Mr. Stone.

The Stone case was one of the last prosecutions arising from the
investigation by the special counsel, Robert S. Mueller III, into
Russian interference in the 2016 presidential race. Despite the
controversy surrounding it, Judge Jackson said repeatedly on Thursday
that she was treating the case like any ordinary prosecution.

She said she took into account that Mr. Stone is 67 and had no criminal
record. Working against Mr. Stone, she said, was his ``belligerence''
and ``incendiary'' conduct after he was indicted. On social media, Mr.
Stone
\href{https://www.nytimes.com/2019/02/19/us/roger-stone-instagram-judge.html}{posted
a photograph of the judge} with the image of a cross hairs near her
head.

In response, Judge Jackson imposed a strict gag order on Mr. Stone, but
she said he violated it even during trial when he appeared to ask the
president, through a proxy, for a pardon. Flouting court orders and
threatening the security of court personnel ``is intolerable to the
administration of justice,'' she declared.

Much of the debate over Mr. Stone's sentence revolved around whether he
merely tried to influence a witness not to cooperate or issued a threat
of violence, which under the sentencing guidelines would prompt a far
more substantial penalty.

Initially, the prosecutors asserted that Mr. Stone's threat of bodily
harm justified a stiffer sentence. So, too, did the fact that he
prevented congressional investigators from discovering the truth and
carried out an extensive scheme over many months to frustrate them, they
said.

Then they appeared to back off, noting that the witness, a New York
radio host named Randy Credico, had written a letter to the judge saying
he never feared that Mr. Stone himself would actually harm him.

Seth Ginsberg, Mr. Stone's defense lawyer, said that Mr. Stone was
simply engaging in his usual banter. ``Mr. Stone is know for using
rough, hyperbolic language,'' he said. ``There was no threat at all.''

Judge Jackson said Mr. Credico was an ``extremely nervous'' witness
whose accounts had varied over time. She noted that he had testified to
a grand jury that he had left his home and was wearing a disguise
because he feared he would be in danger if Mr. Stone identified him as
``a rat.''

During the trial, Mr. Credico said that he feared Mr. Stone, a
well-known political commentator, would create havoc in his life and
that of a close friend's if he testified before the House committee. He
ultimately took the Fifth Amendment.

The judge said she could not simply overlook Mr. Stone's texts to Mr.
Credico, including one that read, ``Prepare to die.''

While Mr. Stone clearly enjoyed ``adolescent mind games,'' she said,
``nothing about this case was a joke. It wasn't a stunt and it wasn't a
prank.''

Zach Montague contributed reporting.

\hypertarget{our-2020-election-guide}{%
\section{Our 2020 Election Guide}\label{our-2020-election-guide}}

Updated July 31, 2020

\begin{itemize}
\item
  \begin{center}\rule{0.5\linewidth}{\linethickness}\end{center}

  \hypertarget{the-latest}{%
  \subsection{The Latest}\label{the-latest}}

  \begin{itemize}
  \tightlist
  \item
    President Trump's assault on the Postal Service is intersecting with
    his attacks on mail-in voting.
    \href{https://www.nytimes.com/2020/07/31/us/politics/trump-usps-mail-delays.html?action=click\&pgtype=Article\&state=default\&region=BELOW_MAIN_CONTENT\&context=storylines_guide}{Voting
    rights groups say it is a recipe for disaster.}
  \end{itemize}
\item
  \begin{center}\rule{0.5\linewidth}{\linethickness}\end{center}

  \hypertarget{bidens-vp-search}{%
  \subsection{Biden's V.P. Search}\label{bidens-vp-search}}

  \begin{itemize}
  \tightlist
  \item
    \href{https://www.nytimes.com/article/biden-vice-president-2020.html?action=click\&pgtype=Article\&state=default\&region=BELOW_MAIN_CONTENT\&context=storylines_guide}{Here
    are 13 women} who have been under consideration to be Joe Biden's
    running mate, and why each might be chosen --- and might not be.
  \end{itemize}
\item
  \begin{center}\rule{0.5\linewidth}{\linethickness}\end{center}

  \hypertarget{keep-up-with-our-coverage}{%
  \subsection{Keep Up With Our
  Coverage}\label{keep-up-with-our-coverage}}

  \begin{itemize}
  \tightlist
  \item
    Get an
    \href{https://www.nytimes.com/newsletters/politics?action=click\&pgtype=Article\&state=default\&region=BELOW_MAIN_CONTENT\&context=storylines_guide}{email}
    recapping the day's news
  \end{itemize}

  \begin{itemize}
  \tightlist
  \item
    Download our mobile app on
    \href{https://apps.apple.com/us/app/nytimes/id284862083?ls=1\&mat_click_id=5c79ae7455014fd1bd66b5610c05b8f2-20191112-16948\&referrer=mat_click_id\%3D5c79ae7455014fd1bd66b5610c05b8f2-20191112-16948\%26link_click_id\%3D722930677036718082}{iOS}
    and
    \href{http://a.localytics.com/android?id=com.nytimes.android\&referrer=utm_source\%3Dother_nyt_mobile_web\%26utm_medium\%3DWeb\%2520page\%26utm_term\%3DGeneral\%2520Mobile\%2520Page\%26utm_campaign\%3DNYT\%2520Mobile\%2520General\%2520Page}{Android}
    and turn on Breaking News and Politics alerts
  \end{itemize}
\end{itemize}

Advertisement

\protect\hyperlink{after-bottom}{Continue reading the main story}

\hypertarget{site-index}{%
\subsection{Site Index}\label{site-index}}

\hypertarget{site-information-navigation}{%
\subsection{Site Information
Navigation}\label{site-information-navigation}}

\begin{itemize}
\tightlist
\item
  \href{https://help.nytimes.com/hc/en-us/articles/115014792127-Copyright-notice}{©~2020~The
  New York Times Company}
\end{itemize}

\begin{itemize}
\tightlist
\item
  \href{https://www.nytco.com/}{NYTCo}
\item
  \href{https://help.nytimes.com/hc/en-us/articles/115015385887-Contact-Us}{Contact
  Us}
\item
  \href{https://www.nytco.com/careers/}{Work with us}
\item
  \href{https://nytmediakit.com/}{Advertise}
\item
  \href{http://www.tbrandstudio.com/}{T Brand Studio}
\item
  \href{https://www.nytimes.com/privacy/cookie-policy\#how-do-i-manage-trackers}{Your
  Ad Choices}
\item
  \href{https://www.nytimes.com/privacy}{Privacy}
\item
  \href{https://help.nytimes.com/hc/en-us/articles/115014893428-Terms-of-service}{Terms
  of Service}
\item
  \href{https://help.nytimes.com/hc/en-us/articles/115014893968-Terms-of-sale}{Terms
  of Sale}
\item
  \href{https://spiderbites.nytimes.com}{Site Map}
\item
  \href{https://help.nytimes.com/hc/en-us}{Help}
\item
  \href{https://www.nytimes.com/subscription?campaignId=37WXW}{Subscriptions}
\end{itemize}
