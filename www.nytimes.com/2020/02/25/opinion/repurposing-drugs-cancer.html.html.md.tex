Sections

SEARCH

\protect\hyperlink{site-content}{Skip to
content}\protect\hyperlink{site-index}{Skip to site index}

\href{https://myaccount.nytimes.com/auth/login?response_type=cookie\&client_id=vi}{}

\href{https://www.nytimes.com/section/todayspaper}{Today's Paper}

\href{/section/opinion}{Opinion}\textbar{}Repurposing Drugs to Fight
Cancer

\href{https://nyti.ms/3913Kth}{https://nyti.ms/3913Kth}

\begin{itemize}
\item
\item
\item
\item
\item
\item
\end{itemize}

Advertisement

\protect\hyperlink{after-top}{Continue reading the main story}

\href{/section/opinion}{Opinion}

Supported by

\protect\hyperlink{after-sponsor}{Continue reading the main story}

FixeS

\hypertarget{repurposing-drugs-to-fight-cancer}{%
\section{Repurposing Drugs to Fight
Cancer}\label{repurposing-drugs-to-fight-cancer}}

While doctors are allowed to use drugs approved for one disease to treat
another condition, many don't because approval to do so doesn't appear
on the label. But that may be changing.

By Sophie Cousins

Ms. Cousins writes about health issues and gender inequality in South
Asia.

\begin{itemize}
\item
  Feb. 25, 2020
\item
  \begin{itemize}
  \item
  \item
  \item
  \item
  \item
  \item
  \end{itemize}
\end{itemize}

\includegraphics{https://static01.nyt.com/images/2020/02/25/opinion/25Fixes-cousins/25Fixes-cousins-articleLarge.jpg?quality=75\&auto=webp\&disable=upscale}

Metformin, the world's most widely prescribed diabetes drug, might do
more than control diabetes. A
\href{https://www.ncbi.nlm.nih.gov/pmc/articles/PMC558205/?report=reader}{study}
in Scotland in 2005 found that people taking metformin were 23 percent
less likely to get cancer, suggesting that it might also prevent cancer.
Other
\href{https://www.cancer.gov/about-cancer/treatment/clinical-trials/intervention/metformin-hydrochloride}{studies}
are exploring metformin's ability to treat cancer, alone and in
combination with other therapies.

But if you have cancer, it's likely you aren't taking metformin, even
though it's inexpensive, safe and combines readily with other therapies.
That's because metformin's label doesn't specify it as a cancer drug.

And while doctors are free to use drugs approved for one disease for
another disease, many don't.

That could be changing. Given the enormous cost of developing new drugs,
various organizations and individuals are going back to basics and
championing the cause of drug repurposing in attempts to get new,
affordable treatments to patients as quickly as possible.

\includegraphics{https://static01.nyt.com/images/2020/02/11/opinion/00Fixes-Cousins2/00Fixes-Cousins2-articleLarge.jpg?quality=75\&auto=webp\&disable=upscale}

``When we started repurposing a few years ago it was incredibly niche,''
said Pan Pantziarka, head of drug repurposing at the
\href{https://www.anticancerfund.org/}{Anticancer Fund} in Brussels,
Belgium. ``Now you can't go to a big cancer conference without there
being talks on repurposing.''

Drug repurposing is not new. Viagra, in fact, was repurposed twice. It
was originally researched to treat hypertension and then repurposed to
treat angina. It was repurposed again when male volunteers who enrolled
in trials experienced side-effects that hinted at a possibility of using
it to treat sexual difficulties. ``We were getting very interesting
feedback from men,'' said Dr. David Brown, a co-inventor of Viagra. He
is a co-founder and the chairman of \href{https://healx.io/}{Healx}, an
international biotechnology company with a main office in Cambridge,
England. The organization uses artificial intelligence and machine
learning to establish links between existing therapies and rare
diseases.

Image

Dr. David Brown, the man who repurposed Viagra, which started out as a
heart medication.Credit...Tom Jamieson for The New York Times

``Drug discovery is very costly and very slow,'' he said. ``We've shown
we can be a lot faster and cost-effective. We have to be. We don't want
to be charging six or seven figures to treat a rare disease. That just
bankrupts health systems and families.''

Pfizer, which held the patent for Viagra, became eager to repurpose it
to treat erectile dysfunction; it knew that such a pill would
immediately be in wide demand. But that level of enthusiasm is rare. In
most cases, by the time there's evidence of a new use for a drug, it's
already off-patent --- so companies have little or no financial
incentive to investigate its potential to treat or prevent other
diseases.

Without evidence of that potential, doctors are reluctant to prescribe.

While precise figures are unavailable, hundreds of drugs have been
approved for one disease but show promise in treating a range of others,
from cancer to rare tropical diseases. Without interest from
pharmaceutical companies, exploration has been left largely to nonprofit
organizations and even patients themselves.

The Anticancer Fund has trawled published scientific literature going as
far back as the 1960s, and has identified almost 300 drugs that studies
say hold anticancer properties, but have not been specifically approved
for cancer. Most have lost patent protection.

The fund is interested now in repurposing drugs that might prevent the
recurrence of cancer after surgery when used in combination with
checkpoint inhibitors --- a form of cancer immunotherapy --- to improve
and sustain the immune system's response.

One example for which it has high hopes is propranolol, a generic 1960s
blood pressure drug, to help treat pancreatic cancer and cancer of the
inner lining of blood vessels --- known as angiosarcoma --- in
combination with other cancer therapies.

\href{https://www.cancerresearchuk.org/}{Cancer Research UK}, a charity
based in London, is now conducting trials of aspirin to see if it can
prevent cancer from recurring; an antifungal medication to treat bowel
cancer; and metformin to help treat early prostate cancer.

``There's no doubt at all about the benefits of using an existing drug,
generic or not generic, because it's a short cut to patient benefits,''
said Hamish Ryder, director of therapeutic discovery laboratories at
Cancer Research UK. ``It could easily take off 10 years to getting a new
therapy for patients.''

Drug repurposing still faces major barriers. The big one is the expense
of clinical trials, which are too much for medical charities to handle
--- perhaps too much for anyone to handle if a drug is off patent and
there is no money to be made. ``In the end, drug repurposing has to be
commercially viable in some way,'' Dr. Brown said.

Another barrier is excitement. ``These are old unsexy drugs. If you're
doing work with a new molecule there's a huge amount of interest,'' Mr.
Pantziarka said. ``But if you're doing something with aspirin, it's not
that interesting. You don't get that hype. We like to think medicine
isn't fashion driven, but there is a degree of that going on.''

Doctor and patient acceptance is another concern. While off-label
prescribing is entirely legal --- the Food and Drug Administration
doesn't have the legal authority to regulate the practice of medicine
--- many doctors still fear lawsuits.

As I was told by one woman who preferred not to be identified in order
to protect her privacy found out, many doctors disagree with
experimenting with off-label drugs.

Almost four years ago, that woman was diagnosed with an incurable, Stage
4 cancer of the bile ducts, known as cholangiocarcinoma. The cancer,
which is extremely rare, had spread and 24 tumors had grown throughout
her body.

The patient, who had three children younger than 5 at the time of her
diagnosis, was told she had six to 12 months to live. She quit her job
and dedicated her time to finding a way --- any way --- of surviving.
She saw as many doctors as possible to gain insight into her rare
cancer, but some were unwilling to prescribe drugs outside the standard
cancer treatments.

``I saw a doctor who thought more out of the box, and she recommended me
to look into metformin,'' the patient said. Metformin does appear to
inhibit the proliferation of
\href{https://www.spandidos-publications.com/10.3892/ol.2017.7412}{tumor}
cells, but the mechanism underpinning that isn't yet fully understood.
Nevertheless, she added metformin to her chemotherapy for six months.

Her search for a cure didn't end there, however. It led next to
fenbendazole, an anti-parasitic drug for animals. Joe Tippens, an
American who was diagnosed with Stage 4 small-cell lung cancer in 2017,
\href{https://www.mycancerstory.rocks/single-post/2016/08/22/Shake-up-your-life-how-to-change-your-own-perspective}{claims}
that taking fenbendazole saved his life; he has been cancer-free now for
a year, and his story has gone
\href{https://www.msn.com/en-ae/lifestyle/wellbeing/man-claims-drug-for-dogs-cured-him-of-cancer/ar-AAB50rx?li=BBqrPye}{viral}
on the internet. While a few studies in mice and petri dishes have
suggested that the drug may have anticancer properties, studies on
humans is lacking.

``Dog deworming tablets are really cheap and have low toxicity,'' the
patient said. ``Yes, they're meant for dogs, but for human use it's
fine.''

The woman with bile duct cancer went into remission for seven months
while she was using metformin in combination with chemotherapy, but the
cancer came back. She is continuing chemotherapy and continues to search
for repurposed drugs. Four years after her diagnosis, she hasn't lost
hope.

But how many patients, even on their deathbed, are willing to go beyond
their prescribed treatment? Few can dedicate significant time to reading
the scientific literature, especially without any prior medical
knowledge. And health insurers are reluctant to pay for drugs that
haven't been approved for the disease in question.

There are no numbers on how many people have turned to metformin and
other such drugs in their desperation to be cured. Mr. Pantziarka
believes the key to expanding the use of repurposed drugs is in the
hands of drug regulators, who have the power to approve and relabel them
for new diseases. When a regulator licenses a drug, clinical guidelines
are updated and doctors gain experience prescribing it.

``The ideal solution is for repurposed drugs to be on label, Mr.
Pantziarka said. ``We want to establish a pathway that once you have the
efficacy from clinical trials then you can go all the way to the
market,'' he said. ``At the end of the day, repurposing is not just
science and medicine but also policymaking.''

In an encouraging step toward overcoming the institutional barriers
surrounding drug repurposing, such issues are under
\href{https://ec.europa.eu/health/sites/health/files/files/committee/stamp/stamp_8_repurposing_established_medicines_background.pdf}{review}
by an expert group on repurposing within the European Commission.
Currently only the original developers or makers of drugs are permitted
to adjust a drug's label. But the review is looking at creating a way
for third parties to apply for drug label extensions.

``We think today of today's generic drugs, but there's tomorrow's
generic drugs,'' said Mr. Ryder of Cancer Research UK. ``I don't think
we should limit our thinking to aspirin and metformin. I think that
we're at the tip of the iceberg in terms of potential for repurposing
therapies.''

Sophie Cousins is an Australian writer and freelance journalist.

\emph{To receive email alerts for Fixes columns, sign up}
\href{http://eepurl.com/ABIxL}{\emph{here.}}

\emph{The Times is committed to publishing}
\href{https://www.nytimes.com/2019/01/31/opinion/letters/letters-to-editor-new-york-times-women.html}{\emph{a
diversity of letters}} \emph{to the editor. We'd like to hear what you
think about this or any of our articles. Here are some}
\href{https://help.nytimes.com/hc/en-us/articles/115014925288-How-to-submit-a-letter-to-the-editor}{\emph{tips}}\emph{.
And here's our email:}
\href{mailto:letters@nytimes.com}{\emph{letters@nytimes.com}}\emph{.}

\emph{Follow The New York Times Opinion section on}
\href{https://www.facebook.com/nytopinion}{\emph{Facebook}}\emph{,}
\href{http://twitter.com/NYTOpinion}{\emph{Twitter (@NYTopinion)}}
\emph{and}
\href{https://www.instagram.com/nytopinion/}{\emph{Instagram}}\emph{.}

Advertisement

\protect\hyperlink{after-bottom}{Continue reading the main story}

\hypertarget{site-index}{%
\subsection{Site Index}\label{site-index}}

\hypertarget{site-information-navigation}{%
\subsection{Site Information
Navigation}\label{site-information-navigation}}

\begin{itemize}
\tightlist
\item
  \href{https://help.nytimes.com/hc/en-us/articles/115014792127-Copyright-notice}{©~2020~The
  New York Times Company}
\end{itemize}

\begin{itemize}
\tightlist
\item
  \href{https://www.nytco.com/}{NYTCo}
\item
  \href{https://help.nytimes.com/hc/en-us/articles/115015385887-Contact-Us}{Contact
  Us}
\item
  \href{https://www.nytco.com/careers/}{Work with us}
\item
  \href{https://nytmediakit.com/}{Advertise}
\item
  \href{http://www.tbrandstudio.com/}{T Brand Studio}
\item
  \href{https://www.nytimes.com/privacy/cookie-policy\#how-do-i-manage-trackers}{Your
  Ad Choices}
\item
  \href{https://www.nytimes.com/privacy}{Privacy}
\item
  \href{https://help.nytimes.com/hc/en-us/articles/115014893428-Terms-of-service}{Terms
  of Service}
\item
  \href{https://help.nytimes.com/hc/en-us/articles/115014893968-Terms-of-sale}{Terms
  of Sale}
\item
  \href{https://spiderbites.nytimes.com}{Site Map}
\item
  \href{https://help.nytimes.com/hc/en-us}{Help}
\item
  \href{https://www.nytimes.com/subscription?campaignId=37WXW}{Subscriptions}
\end{itemize}
