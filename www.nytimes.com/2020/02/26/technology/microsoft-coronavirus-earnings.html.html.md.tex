Sections

SEARCH

\protect\hyperlink{site-content}{Skip to
content}\protect\hyperlink{site-index}{Skip to site index}

\href{https://www.nytimes.com/section/technology}{Technology}

\href{https://myaccount.nytimes.com/auth/login?response_type=cookie\&client_id=vi}{}

\href{https://www.nytimes.com/section/todayspaper}{Today's Paper}

\href{/section/technology}{Technology}\textbar{}Microsoft Issues
Financial Warning Because of Coronavirus

\url{https://nyti.ms/32u0jch}

\begin{itemize}
\item
\item
\item
\item
\item
\end{itemize}

\href{https://www.nytimes.com/news-event/coronavirus?action=click\&pgtype=Article\&state=default\&region=TOP_BANNER\&context=storylines_menu}{The
Coronavirus Outbreak}

\begin{itemize}
\tightlist
\item
  live\href{https://www.nytimes.com/2020/08/04/world/coronavirus-covid-19.html?action=click\&pgtype=Article\&state=default\&region=TOP_BANNER\&context=storylines_menu}{Latest
  Updates}
\item
  \href{https://www.nytimes.com/interactive/2020/us/coronavirus-us-cases.html?action=click\&pgtype=Article\&state=default\&region=TOP_BANNER\&context=storylines_menu}{Maps
  and Cases}
\item
  \href{https://www.nytimes.com/interactive/2020/science/coronavirus-vaccine-tracker.html?action=click\&pgtype=Article\&state=default\&region=TOP_BANNER\&context=storylines_menu}{Vaccine
  Tracker}
\item
  \href{https://www.nytimes.com/2020/08/02/us/covid-college-reopening.html?action=click\&pgtype=Article\&state=default\&region=TOP_BANNER\&context=storylines_menu}{College
  Reopening}
\item
  \href{https://www.nytimes.com/live/2020/08/03/business/stock-market-today-coronavirus?action=click\&pgtype=Article\&state=default\&region=TOP_BANNER\&context=storylines_menu}{Economy}
\end{itemize}

Advertisement

\protect\hyperlink{after-top}{Continue reading the main story}

Supported by

\protect\hyperlink{after-sponsor}{Continue reading the main story}

\hypertarget{microsoft-issues-financial-warning-because-of-coronavirus}{%
\section{Microsoft Issues Financial Warning Because of
Coronavirus}\label{microsoft-issues-financial-warning-because-of-coronavirus}}

The tech giant said the virus was causing issues with its supply chain,
about a week after Apple said it was facing similar problems.

\includegraphics{https://static01.nyt.com/images/2020/02/26/business/26microsoft/merlin_162038613_aa6b899a-0db7-4075-9025-27dddc922959-articleLarge.jpg?quality=75\&auto=webp\&disable=upscale}

\href{https://www.nytimes.com/by/karen-weise}{\includegraphics{https://static01.nyt.com/images/2019/04/11/multimedia/author-karen-weise/author-karen-weise-thumbLarge.png}}

By \href{https://www.nytimes.com/by/karen-weise}{Karen Weise}

\begin{itemize}
\item
  Feb. 26, 2020
\item
  \begin{itemize}
  \item
  \item
  \item
  \item
  \item
  \end{itemize}
\end{itemize}

SEATTLE --- Microsoft on Wednesday said its sales in the current quarter
would be lower than it had previously predicted because of
coronavirus-related disruptions in Chinese manufacturing.

While its fast-growing cloud computing business is not affected, the
company said its personal computing business, which includes Windows
installations and its Surface laptops and tablets, would record lower
sales than it told investors to expect last month.

``Although we see strong Windows demand in line with our expectations,
the supply chain is returning to normal operations at a slower pace than
anticipated,'' the company
\href{https://news.microsoft.com/2020/02/26/microsoft-update-on-q3-fy20-guidance/}{said}
in a statement.

Roughly a week ago,
\href{https://www.nytimes.com/2020/02/17/technology/apple-coronavirus-economy.html}{Apple
warned it was cutting sales projections}because of the public health
crisis from the coronavirus.

Apple said the factories that make its iPhones were slower to reopen
after the Lunar New Year than it had expected, and the company
experienced lower demand for its products from Chinese consumers. At the
time, all of its stores in China were closed, though some have since
\href{https://www.bloomberg.com/news/articles/2020-02-24/apple-reopens-more-than-half-of-its-retail-stores-in-china}{begun
to reopen}.

The financial warnings from Microsoft and Apple --- two of the most
valuable publicly traded companies in the world --- underscore the
vulnerability of technology supply chains in China, said Dan Ives, a
managing director at Wedbush Securities. Many tech companies in the
United States rely on large factories in China, though some have started
to shift to other countries like Vietnam.

``When bellwethers like Microsoft come out and talk about the supply
chain and how it will negatively impact PC demand, it fans the flames of
some of the worries out there for the broader supply chain,'' he said.

Microsoft's stock fell about 1 percent in after-market trading on
Wednesday evening.

Personal computing, which includes hardware sales as well as Windows
installed on computers that other companies produce and sell, accounts
for roughly a third of Microsoft's revenue.

In January, the company said it expected to have between \$10.75 billion
and \$11.15 billion in sales for the segment, a wider-than-normal range
reflecting the uncertainty at the time. The company did not provide a
new sales estimate.

Unlike Apple, Microsoft does not sell much in China, and the market
accounts for
\href{https://www.geekwire.com/2020/microsoft-president-brad-smith-tech-cold-war-u-s-china-relations/}{less
than 2 percent} of the company's revenue. While its Windows and Office
products are used in China, the software is often pirated.

Having two giants disclose that the coronavirus is squeezing their
finances will put pressure on competitors to follow suit, said David
Larcker, a professor of corporate governance at the Stanford Graduate
School of Business.

When companies make unusual financial disclosures, they need to juggle
their duty to inform shareholders, their desire to get ahead of a big
surprise and the need to keep some information secret from competitors,
he said.

``It may be they are still trying to understand this,'' Mr. Larcker
said. ``It may not be a satisfying answer, but it's a truthful one.''

Advertisement

\protect\hyperlink{after-bottom}{Continue reading the main story}

\hypertarget{site-index}{%
\subsection{Site Index}\label{site-index}}

\hypertarget{site-information-navigation}{%
\subsection{Site Information
Navigation}\label{site-information-navigation}}

\begin{itemize}
\tightlist
\item
  \href{https://help.nytimes.com/hc/en-us/articles/115014792127-Copyright-notice}{©~2020~The
  New York Times Company}
\end{itemize}

\begin{itemize}
\tightlist
\item
  \href{https://www.nytco.com/}{NYTCo}
\item
  \href{https://help.nytimes.com/hc/en-us/articles/115015385887-Contact-Us}{Contact
  Us}
\item
  \href{https://www.nytco.com/careers/}{Work with us}
\item
  \href{https://nytmediakit.com/}{Advertise}
\item
  \href{http://www.tbrandstudio.com/}{T Brand Studio}
\item
  \href{https://www.nytimes.com/privacy/cookie-policy\#how-do-i-manage-trackers}{Your
  Ad Choices}
\item
  \href{https://www.nytimes.com/privacy}{Privacy}
\item
  \href{https://help.nytimes.com/hc/en-us/articles/115014893428-Terms-of-service}{Terms
  of Service}
\item
  \href{https://help.nytimes.com/hc/en-us/articles/115014893968-Terms-of-sale}{Terms
  of Sale}
\item
  \href{https://spiderbites.nytimes.com}{Site Map}
\item
  \href{https://help.nytimes.com/hc/en-us}{Help}
\item
  \href{https://www.nytimes.com/subscription?campaignId=37WXW}{Subscriptions}
\end{itemize}
