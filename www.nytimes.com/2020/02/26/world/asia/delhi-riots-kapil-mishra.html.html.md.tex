Sections

SEARCH

\protect\hyperlink{site-content}{Skip to
content}\protect\hyperlink{site-index}{Skip to site index}

\href{https://www.nytimes.com/section/world/asia}{Asia Pacific}

\href{https://myaccount.nytimes.com/auth/login?response_type=cookie\&client_id=vi}{}

\href{https://www.nytimes.com/section/todayspaper}{Today's Paper}

\href{/section/world/asia}{Asia Pacific}\textbar{}The Roots of the Delhi
Riots: A Fiery Speech and an Ultimatum

\url{https://nyti.ms/37YsQbb}

\begin{itemize}
\item
\item
\item
\item
\item
\end{itemize}

Advertisement

\protect\hyperlink{after-top}{Continue reading the main story}

Supported by

\protect\hyperlink{after-sponsor}{Continue reading the main story}

\hypertarget{the-roots-of-the-delhi-riots-a-fiery-speech-and-an-ultimatum}{%
\section{The Roots of the Delhi Riots: A Fiery Speech and an
Ultimatum}\label{the-roots-of-the-delhi-riots-a-fiery-speech-and-an-ultimatum}}

A local Hindu politician told the police to evict a group of Muslim
protesters or he and his men would. Now, 25 have died in some of the
worst violence in years.

\includegraphics{https://static01.nyt.com/images/2020/02/26/world/26INDIA-RIOTS-01/merlin_169567503_4f368702-94ae-405e-b3dc-075149a6bb1f-articleLarge.jpg?quality=75\&auto=webp\&disable=upscale}

\href{https://www.nytimes.com/by/jeffrey-gettleman}{\includegraphics{https://static01.nyt.com/images/2018/10/10/multimedia/author-jeffrey-gettleman/author-jeffrey-gettleman-thumbLarge.png}}\href{https://www.nytimes.com/by/suhasini-raj}{\includegraphics{https://static01.nyt.com/images/2019/11/22/reader-center/author-Suhasini-Raj/author-Suhasini-Raj-thumbLarge.png}}\href{https://www.nytimes.com/by/sameer-yasir}{\includegraphics{https://static01.nyt.com/images/2019/11/22/reader-center/author-sameer-yasir/author-sameer-yasir-thumbLarge.png}}

By \href{https://www.nytimes.com/by/jeffrey-gettleman}{Jeffrey
Gettleman}, \href{https://www.nytimes.com/by/suhasini-raj}{Suhasini Raj}
and \href{https://www.nytimes.com/by/sameer-yasir}{Sameer Yasir}

\begin{itemize}
\item
  Feb. 26, 2020
\item
  \begin{itemize}
  \item
  \item
  \item
  \item
  \item
  \end{itemize}
\end{itemize}

NEW DELHI --- To many in the eastern Delhi neighborhood where a
convulsion of religious violence erupted this week, it all began with
one man.

Kapil Mishra, a local politician with India's leading Hindu nationalist
party, had just lost an election. Acquaintances in the area, which now
feels like a war zone, said he had been looking for a way to bounce
back.

Mr. Mishra, 39, is known for his outspoken views and flexible politics.
As an upper-caste Hindu from a political family, he had worked for
Amnesty International and Greenpeace, and risen in the ranks of one of
India's most progressive political organizations. But several years ago
he shifted allegiance across the political spectrum to the Bharatiya
Janata Party, India's current governing party, which has deep roots in
Hindu supremacist ideology.

On Sunday, he appeared at a rally against a group of protesters (most of
them women) who were objecting to
\href{https://www.nytimes.com/2019/12/16/world/asia/india-citizenship-protests.html}{a
new citizenship law} widely seen as discriminatory toward Muslims. There
he vented his anger in a fiery speech in which he issued an ultimatum to
the police: either clear out the demonstrators, who were blocking a main
road, or he and his followers would do it themselves.

Image

Kapil Mishra addressed a rally in New Delhi last year.Credit...Sonu
Mehta/Hindustan Times

Within hours, the worst Hindu-Muslim violence in India in years was
exploding. Gangs of Hindus and Muslims fought each other with swords and
bats, shops burst into flames, chunks of bricks sailed through the air,
and mobs rained blows on cornered men.

\emph{{[}Update:}
\href{http://www.nytimes.com/2020/02/27/world/asia/india-violence-hindu-muslim.html}{\emph{Violence
continues in New Delhi, and the police are criticized}}\emph{.{]}}

Many Indians, including Hindus, **** believe that Mr. Mishra and his
Hindu nationalist supporters have weaponized a very dangerous mood. In a
Hindu majority nation, with a Hindu nationalist government that has
allowed the
\href{https://www.nytimes.com/2019/02/18/world/asia/india-cow-religious-attacks.html}{killers
of Muslims to go unpunished,} fear has been growing that violent Hindu
extremism could spin out of control.

``Kapil Mishra should be in jail,'' said Rupesh Bathla, a businessman
who says he has known Mr. Mishra since they were teenagers. ``He started
communal riots. He planted hatred in other people's hearts.''

By Wednesday, at least 25 people had died, hospital officials said, most
from gunshot wounds. Several witnesses said that the live fire came from
the direction of the police officers, and the dead included Hindus as
well as Muslims.

Though property belonging to Hindus was burned, the destruction was much
heavier on the Muslim side. In Muslim areas, shop after shop was
destroyed and entire markets were burned down. Dozens of Muslim
residents have accused police officers of standing passively by while
the destruction was underway.

On Wednesday, the few people out on the streets walked quietly past the
blackened car hulks and smashed homes. The whiff of charred materials
still hung in the air, in what some scholars said was an eerie echo of
previous religious bloodletting in India.

\includegraphics{https://static01.nyt.com/images/2020/03/26/world/26india-riots-2sub/merlin_169575168_7617361c-f461-495a-b18f-4e90866be30f-articleLarge.jpg?quality=75\&auto=webp\&disable=upscale}

``On the whole, the Delhi riots of this week are now beginning to look
like a pogrom, à la Gujarat 2002 and Delhi 1984,'' said Ashutosh
Varshney, the director of the Center for Contemporary South Asia at
Brown University.

While the death toll is nowhere near that of those earlier bouts, the
episodes shared a disturbing similarity, Mr. Varshney said, with ``mobs
unleashing savage violence while the cops look away, or join the mob,
instead of neutrally intervening to crush the riot.''

With the violence cooling down for the moment, Prime Minister Narendra
Modi, who hosted President Trump as the fighting raged, broke his
silence on Wednesday after Mr. Trump had departed, urging
people\href{https://twitter.com/narendramodi/status/1232581653916155912}{in
a Twitter post} to ``maintain peace and brotherhood at all times.'' He
added, ``Peace and harmony are central to our ethos.''

As night fell on Wednesday, a few sporadic attacks were reported, but no
large-scale mayhem. The police, armed now with assault rifles, had been
reinforced with paramilitary troops.

In the area that suffered the worst in the fighting, many residents laid
blame on Mr. Mishra, who declined a request for an interview. But
\href{https://twitter.com/KapilMishra_IND/status/1232292773581217794?s=20}{in
a Twitter post}, he said, ``It's not a crime to ask for blocked roads to
be opened. It's not a crime to tell the truth. I don't fear this massive
hate campaign against me.''

\includegraphics{https://static01.nyt.com/images/2020/02/26/world/26india1/merlin_169553361_221053fb-584c-4b7b-b3f3-a3ce0f6b36f1-videoSixteenByNine3000.jpg}

Mr. Mishra had said in his speech that he did not want to create trouble
before Mr. Trump left the country on Tuesday night. But as Sunday
evening approached, gangs of Hindu and Muslim men began throwing rocks
at one another, and that quickly spawned far greater violence.

At a court hearing on the riots on Wednesday, a judge pressed police
officials about why they had not watched videos of Mr. Mishra's
incendiary speech --- an indication, the judge implied, that they had
not seriously investigated the sources of the violence.

``This is really concerning,'' said the judge, S. Muralidhar, according
to
\href{https://twitter.com/LiveLawIndia/status/1232617825820381184}{LiveLaw},
a legal news website. ``There are so many TVs in your office, how can a
police officer say that he hasn't watched the videos? I'm really
appalled by the state of affairs of the Delhi Police.''

Mr. Mishra's supporters said the majority of people in the community had
backed his effort to evict the protesters. ``How could our kids get to
school with those protesters blocking the road?'' said Alok Kumar Gupta,
a retired military officer who lives near the protest area. ``Kapil
Mishra was only trying to help.''

Image

A mosque burned by rioters in Mustafabad in East Delhi.Credit...Atul
Loke for The New York Times

But others wonder if Mr. Mishra was trying to make a name for himself in
Hindu nationalist circles. He had been elected to the local assembly in
Delhi in 2015 from the progressive Aam Aadmi Party, but eventually fell
out with his colleagues and defected to Mr. Modi's Bharatiya Janata
Party, or B.J.P.

He then started espousing Hindu nationalist views and vilifying Muslims,
more out of political expediency than true belief, argued Mr. Bathla,
who claims to have known Mr. Mishra for 30 years.

``When he was younger he wasn't like that,'' he said. ``He was chill.''

Just a few weeks before the Feb. 8 local assembly elections, Mr. Mishra
posted what was widely viewed as an incendiary Twitter message, framing
the contest as
\href{https://twitter.com/KapilMishra_IND/status/1220213605359992833}{``India
vs Pakistan.''}

After he lost the race, several people said he took the hard line
against the mostly Muslim protesters as a way to improve his standing in
the B.J.P.

``He wasn't getting much attention from the higher-ups,'' said Hasrat
Ali, a legal officer who lives the same area where Mr. Mishra's family
lived for many years. ``This was all a plan to get a firmer position.''

At least one other politician in Mr. Modi's party is now distancing
himself from Mr. Mishra. ``Whoever has done this, strict action must be
taken,'' said the politician, Gautam Gambhir, a member of Parliament
from the area. ``Kapil Mishra's speech is not acceptable.''

\href{https://www.nytimes.com/2020/01/17/world/asia/india-protests-aishe-ghosh.html}{Protests
against the new citizenship law}, which makes it easier for non-Muslim
migrants to become full-fledged Indian citizens, have flared
intermittently since December. There has been a subtext of religious
differences, with most of India's Muslims objecting to the law and many
Hindus supporting it. But this past week was the first time the protests
turned large numbers of Hindus and Muslims violently against one
another.

Image

The body of a man killed in the clashes was moved in New Delhi on
Wednesday.Credit...Adnan Abidi/Reuters

Most of India's Muslims distrust the B.J.P. and point to the
\href{https://www.nytimes.com/interactive/2014/04/06/world/asia/modi-gujarat-riots-timeline.html\#/\#time287_8514}{sectarian
killing frenzy} that claimed the lives of more than 1,000 people, almost
800 of them Muslims, in Gujarat State in 2002 when Mr. Modi was its
chief minister.

Mr. Modi and his state government were accused of quietly ordering the
police to stand by as the violence raged. He has denied those
accusations, and in 2012, an investigative panel for the Supreme Court
\href{http://archive.indianexpress.com/news/SIT-report-clears-Modi--61-others/935226}{found
no evidence to support them.}

But until he became prime minister in 2014, Mr. Modi was banned from
entering the United States. And since he came to power, violence against
Muslims, including
\href{https://www.nytimes.com/2019/02/18/world/asia/india-cow-religious-attacks.html}{mob
lynchings}, has increased sharply.

Muslim families in northeast Delhi are now abandoning their homes.
Several said in interviews that they no longer felt safe.

Asgar Ali, whose grocery shop was burned to the ground on Tuesday, said
there was no difference between police officers and Hindu mobs. He said
he was fleeing his home, where he had lived for 20 years, knowing that
he might never return.

``I built this house with my blood and sweat,'' Mr. Ali said. ``Now, I
have been reduced to a homeless pauper. I have lost everything.''

Image

Mourning a victim of the clashes outside a mortuary in New Delhi on
Wednesday.Credit...Altaf Qadri/Associated Press

Shalini Venugopal contributed reporting.

Advertisement

\protect\hyperlink{after-bottom}{Continue reading the main story}

\hypertarget{site-index}{%
\subsection{Site Index}\label{site-index}}

\hypertarget{site-information-navigation}{%
\subsection{Site Information
Navigation}\label{site-information-navigation}}

\begin{itemize}
\tightlist
\item
  \href{https://help.nytimes.com/hc/en-us/articles/115014792127-Copyright-notice}{©~2020~The
  New York Times Company}
\end{itemize}

\begin{itemize}
\tightlist
\item
  \href{https://www.nytco.com/}{NYTCo}
\item
  \href{https://help.nytimes.com/hc/en-us/articles/115015385887-Contact-Us}{Contact
  Us}
\item
  \href{https://www.nytco.com/careers/}{Work with us}
\item
  \href{https://nytmediakit.com/}{Advertise}
\item
  \href{http://www.tbrandstudio.com/}{T Brand Studio}
\item
  \href{https://www.nytimes.com/privacy/cookie-policy\#how-do-i-manage-trackers}{Your
  Ad Choices}
\item
  \href{https://www.nytimes.com/privacy}{Privacy}
\item
  \href{https://help.nytimes.com/hc/en-us/articles/115014893428-Terms-of-service}{Terms
  of Service}
\item
  \href{https://help.nytimes.com/hc/en-us/articles/115014893968-Terms-of-sale}{Terms
  of Sale}
\item
  \href{https://spiderbites.nytimes.com}{Site Map}
\item
  \href{https://help.nytimes.com/hc/en-us}{Help}
\item
  \href{https://www.nytimes.com/subscription?campaignId=37WXW}{Subscriptions}
\end{itemize}
