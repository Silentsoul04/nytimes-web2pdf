Sections

SEARCH

\protect\hyperlink{site-content}{Skip to
content}\protect\hyperlink{site-index}{Skip to site index}

\href{https://www.nytimes.com/section/world}{World}

\href{https://myaccount.nytimes.com/auth/login?response_type=cookie\&client_id=vi}{}

\href{https://www.nytimes.com/section/todayspaper}{Today's Paper}

\href{/section/world}{World}\textbar{}`When Can We Go to School?' Nearly
300 Million Children Are Missing Class

\url{https://nyti.ms/3czip1k}

\begin{itemize}
\item
\item
\item
\item
\item
\end{itemize}

\hypertarget{schools-reopening}{%
\subsubsection{\texorpdfstring{\href{https://www.nytimes.com/spotlight/schools-reopening?name=styln-coronavirus-schools-reopening\&region=TOP_BANNER\&variant=undefined\&block=storyline_menu_recirc\&action=click\&pgtype=Article\&impression_id=c895cd30-e108-11ea-b063-bb1d4dad1224}{Schools
Reopening}}{Schools Reopening}}\label{schools-reopening}}

\begin{itemize}
\tightlist
\item
  \href{https://www.nytimes.com/2020/08/17/us/k-12-schools-reopening.html?name=styln-coronavirus-schools-reopening\&region=TOP_BANNER\&variant=undefined\&block=storyline_menu_recirc\&action=click\&pgtype=Article\&impression_id=c895cd31-e108-11ea-b063-bb1d4dad1224}{State
  of Play for K-12}
\item
  \href{https://www.nytimes.com/2020/08/15/us/covid-college-tuition.html?name=styln-coronavirus-schools-reopening\&region=TOP_BANNER\&variant=undefined\&block=storyline_menu_recirc\&action=click\&pgtype=Article\&impression_id=c895cd32-e108-11ea-b063-bb1d4dad1224}{College
  Costs}
\item
  \href{https://www.nytimes.com/2020/08/14/us/covid-schools-learning-pods.html?name=styln-coronavirus-schools-reopening\&region=TOP_BANNER\&variant=undefined\&block=storyline_menu_recirc\&action=click\&pgtype=Article\&impression_id=c895cd33-e108-11ea-b063-bb1d4dad1224}{Priced
  Out of Learning Pods}
\item
  \href{https://www.nytimes.com/2020/08/14/nyregion/school-reopening-nyc.html?name=styln-coronavirus-schools-reopening\&region=TOP_BANNER\&variant=undefined\&block=storyline_menu_recirc\&action=click\&pgtype=Article\&impression_id=c895cd34-e108-11ea-b063-bb1d4dad1224}{N.Y.C.
  Schools Not Ready}
\item
  \href{https://www.nytimes.com/2020/08/05/parenting/parents-distance-learning.html?name=styln-coronavirus-schools-reopening\&region=TOP_BANNER\&variant=undefined\&block=storyline_menu_recirc\&action=click\&pgtype=Article\&impression_id=c895f440-e108-11ea-b063-bb1d4dad1224}{Prepare
  for Distance Learning}
\end{itemize}

Advertisement

\protect\hyperlink{after-top}{Continue reading the main story}

Supported by

\protect\hyperlink{after-sponsor}{Continue reading the main story}

\hypertarget{when-can-we-go-to-school-nearly-300-million-children-are-missing-class}{%
\section{`When Can We Go to School?' Nearly 300 Million Children Are
Missing
Class}\label{when-can-we-go-to-school-nearly-300-million-children-are-missing-class}}

The global scale and speed of the educational disruption from the
coronavirus epidemic is ``unparalleled,'' the United Nations said.

\includegraphics{https://static01.nyt.com/images/2020/03/04/world/04virus-schools01/merlin_169998780_4d9e99a8-52b1-44d4-881c-3dba3e96b813-articleLarge.jpg?quality=75\&auto=webp\&disable=upscale}

By \href{https://www.nytimes.com/by/vivian-wang}{Vivian Wang} and Makiko
Inoue

\begin{itemize}
\item
  March 4, 2020
\item
  \begin{itemize}
  \item
  \item
  \item
  \item
  \item
  \end{itemize}
\end{itemize}

\href{https://cn.nytimes.com/education/20200305/coronavirus-schools-closed/}{阅读简体中文版}\href{https://cn.nytimes.com/education/20200305/coronavirus-schools-closed/zh-hant/}{閱讀繁體中文版}\href{https://www.nytimes.com/es/2020/03/05/espanol/mundo/suspension-clases-coronavirus.html}{Leer
en español}

HONG KONG --- The coronavirus epidemic has reached deeper into daily
life across the world, with a sweeping shutdown of all schools in Italy,
a suspension of classes in India's capital and warnings of school
closures in the United States, intensifying the educational upheaval of
nearly 300 million students globally.

Only a few weeks ago, China, where the outbreak began, was the only
country to suspend classes. But the virus has spread so quickly that by
Wednesday, 22 countries on three continents had announced school
closures of varying degrees, leading
\href{https://en.unesco.org/news/290-million-students-out-school-due-covid-19-unesco-releases-first-global-numbers-and-mobilizes}{the
United Nations to warn} that ``the global scale and speed of the current
educational disruption is unparalleled.''

Students are now out of school in South Korea, Iran, Japan, France,
Pakistan and elsewhere --- some for only a few days, others for weeks on
end. In India on Thursday, all public and private schools through the
fifth grade
\href{https://www.nytimes.com/2020/03/05/world/coronavirus-news.html?action=click\&module=Top\%20Stories\&pgtype=Homepage\#link-4c11ca8b}{were
ordered closed} through March in the capital, New Delhi, affecting more
than two million children.

\emph{{[}Read:
`}\href{https://www.nytimes.com/2020/03/10/world/asia/south-korea-coronavirus-shincheonji.html}{\emph{Proselytizing
robots': Inside South Korean church at outbreak's center}}\emph{.{]}}

In Italy, suffering one of the deadliest outbreaks outside China,
officials said Wednesday that they would extend school closures beyond
the north, where the government has imposed a lockdown on several towns,
to the entire nation. All schools and universities will remain closed
until March 15, officials said.

On the West Coast of the United States, the region with the most
American infections so far, Los Angeles declared a state of emergency on
Wednesday, advising parents to steel themselves for school closures in
the nation's second-largest public school district. Washington State,
which has reported at least 10 deaths from the outbreak, has closed some
schools, while on the other side of the country in New York, newly
diagnosed cases have led to the closure of several schools as well.

The speed and scale of the educational tumult --- which now affects
290.5 million students worldwide, the United Nations says --- has little
parallel in modern history, educators and economists contend. Schools
provide structure and support for families, communities and entire
economies. The effect of closing them for days, weeks and sometimes even
months could have untold repercussions for children and societies at
large.

``They're always saying, `When can we go out to play? When can we go to
school?''' said Gao Mengxian, a security guard in Hong Kong whose two
daughters have been stuck at home because school has been suspended
since January.

In some countries, older students have missed crucial study sessions for
college admissions exams, while younger ones have risked falling behind
in reading and math. Parents have lost wages, tried to work at home or
scrambled to find child care. Some have moved children to new schools in
areas unaffected by the coronavirus, and lost milestones like graduation
ceremonies or last days of school.

``I don't have data to offer, but can't think of any instances in modern
times where advanced economies shut down schools nationally for
prolonged periods of time,'' said Jacob Kirkegaard, a senior fellow at
the \href{https://www.piie.com/}{Peterson Institute for International
Economics}in Washington.

In Hong Kong, families like Ms. Gao's have struggled to maintain some
semblance of normalcy.

Ms. Gao, 48, stopped working to watch her daughters and started
scrimping on household expenses. She ventures outside just once a week
and spends the most time helping her girls, 10 and 8, with online
classes, fumbling through technology that leaves her confused and her
daughters frustrated.

\includegraphics{https://static01.nyt.com/images/2020/03/05/world/05virus-schools-p1/merlin_169931925_ed241136-2542-4690-8e48-6be8f1794c15-articleLarge.jpg?quality=75\&auto=webp\&disable=upscale}

Governments are trying to help. Japan is offering subsidies to help
companies offset the cost of parents' taking time off. France has
promised 14 days of paid sick leave to parents of children who must
self-isolate, if they have no choice but to watch their children.

But the burdens are widespread, touching corners of society seemingly
unconnected to education. In Japan, schools have canceled bulk food
deliveries for lunches they will no longer serve, hurting farmers and
suppliers. In Hong Kong, an army of domestic helpers has been left
unemployed after wealthy families enrolled their children in schools
overseas.

Julia Bossard, a 39-year-old mother of two in France, said she had been
forced to rethink her entire routine since her older son's school was
closed for two weeks for disinfection. Her days now consist of helping
her children with homework and scouring supermarkets for
fast-disappearing pasta, rice and canned food. ``We had to reorganize
ourselves,'' she said.

Image

Attending an online class at home in Fuyang, China, on
Monday.Credit...China Daily/Reuters

\hypertarget{online-and-alone}{%
\subsection{Online and Alone}\label{online-and-alone}}

School and government officials have sought to keep children learning
--- and occupied --- at home. The Italian government created a
\href{https://www.istruzione.it/coronavirus/didattica-a-distanza.html}{web
page} to give teachers access to videoconference tools and ready-made
lesson plans. Mongolian television stations are airing classes. Iran's
government has made all children's internet content free.

Students even take online physical education: At least one school in
Hong Kong requires students --- in gym uniform --- to follow along as an
instructor demonstrates push-ups onscreen. Each student's webcam
provides proof.

The offline reality, though, is challenging. Technological hurdles and
unavoidable distractions pop up when children and teenagers are left to
their own devices --- literally.

Thira Pang, a 17-year-old high school student in Hong Kong, has been
repeatedly late for class because her internet connection is slow. She
now logs on 15 minutes early.

``It's just a bit of luck to see whether you can get in,'' she said.

The new classroom at home poses greater problems for younger students,
and their older caregivers. Ruby Tan, a teacher in Chongqing, a city in
southwestern China that suspended school last month, said many
grandparents were helping with child care so that the parents can go to
work. But the grandparents do not always know the technology.

``They don't have any way of supervising the children's learning, and
instead let them develop bad habits of not being able to focus during
study time,'' Ms. Tan said.

Some interruptions are unavoidable. Posts on Chinese social media show
teachers and students climbing onto rooftops or hovering outside
neighbors' homes in search of a stronger internet signal. One family in
Inner Mongolia \href{https://m.weibo.cn/status/4478077407118555?}{packed
up its yurt} and migrated elsewhere in the grasslands for a better web
connection, a Chinese magazine reported.

The closings have also altered the normal milestones of education. In
Japan, the school year typically ends in March.
\href{https://www.nytimes.com/2020/02/27/world/asia/japan-schools-coronavirus.html}{Many
schools} are now restricting the ceremonies to teachers and students.

When Satoko Morita's son graduated from high school in Akita Prefecture,
in northern Japan, on March 1, she was not there. It will be the same
for her daughter's ceremony at elementary school.

``My daughter asked me, `What's the point of attending and delivering
speeches in the ceremony without parents?''' she said.

For Chloe Lau, a Hong Kong student, the end of her high school education
came abruptly. Her last day was supposed to be April 2, but schools in
Hong Kong will not resume until at least April 20.

Image

An employee in Tokyo working as her son completed his schoolwork on
Monday.Credit...Eugene Hoshiko/Associated Press

\hypertarget{a-burden-on-women}{%
\subsection{A Burden on Women}\label{a-burden-on-women}}

With the closings, families must rethink how they support themselves and
split household responsibilities. The burden has fallen particularly
hard on women, who across the world are still largely responsible for
child care.

Babysitters are in short supply or leery of taking children from
hard-hit regions.

The 11-year-old son of Lee Seong-yeon, a health information manager at a
hospital in Seoul, South Korea, has been out of class since the
government suspended schools nationwide on Monday.
\href{https://www.nytimes.com/2020/02/23/world/asia/south-korea-coronavirus-moon.html}{South
Korea has the highest number of coronavirus cases} outside China.

Working from home was never an option for Ms. Lee: She and her husband,
also a hospital employee, have more work duties than ever. So Ms. Lee's
son spends each weekday alone, eating lunchboxes of sausage and kimchi
fried rice premade by Ms. Lee.

``I think I would have quit my job if my son were younger, because I
wouldn't have been able to leave him alone at home,'' Ms. Lee said.

\href{https://www.nytimes.com/spotlight/schools-reopening?action=click\&pgtype=Article\&state=default\&region=MAIN_CONTENT_3\&context=storylines_keepup}{}

\hypertarget{schools-reopening-}{%
\subsubsection{Schools Reopening ›}\label{schools-reopening-}}

\hypertarget{back-to-school}{%
\paragraph{Back to School}\label{back-to-school}}

Updated Aug. 17, 2020

The latest on how schools are navigating an uncertain season.

\begin{itemize}
\item
  \begin{itemize}
  \tightlist
  \item
    Universities across the country are facing
    \href{https://www.nytimes.com/2020/08/15/us/covid-college-tuition.html?action=click\&pgtype=Article\&state=default\&region=MAIN_CONTENT_3\&context=storylines_keepup}{a
    rising demand for tuition rebates} as students ask if college is
    becoming ``glorified Skype.''
  \item
    In Los Angeles, the nation's second-largest school district has
    \href{https://www.nytimes.com/2020/08/16/us/los-angeles-schools-virus-testing.html?action=click\&pgtype=Article\&state=default\&region=MAIN_CONTENT_3\&context=storylines_keepup}{perhaps
    the most ambitious plan in the country} to test for the coronavirus.
  \item
    Families
    \href{https://www.nytimes.com/2020/08/14/us/covid-schools-learning-pods.html?action=click\&pgtype=Article\&state=default\&region=MAIN_CONTENT_3\&context=storylines_keepup}{priced
    out of ``learning pods'' are seeking alternatives}.
  \item
    How are campus newspapers covering back to school?
    \href{https://www.nytimes.com/2020/08/17/us/student-newspaper-schools-reopening.html?action=click\&pgtype=Article\&state=default\&region=MAIN_CONTENT_3\&context=storylines_keepup}{We
    want to hear from student journalists}.
  \end{itemize}
\end{itemize}

Still, she feels her career will suffer. ``I try to get off work at 6
p.m. sharp, even when others at the office are still at their desks, and
I run home to my son and make him dinner,'' she said. ``So I know there
is no way I am ever going to be acknowledged for my career at work.''

For mothers with few safety nets, options are even more limited.

In Athens, Anastasia Moschos said she had been lucky. When her
6-year-old son's school was closed for a week, Ms. Moschos, 47, an
insurance broker, left her son with her father, who was visiting. But if
the schools stay closed, she may have to scramble for help.

``The assumption is that everyone has someone to assist,'' she said.
``That's not the case with me. I'm a single mother, and I don't have
help at home.''

Even mothers able to leave affected areas have trouble finding child
care. Cristina Tagliabue, a communications entrepreneur from Milan,
\href{https://www.nytimes.com/2020/02/24/world/europe/24coronavirus-milan-italy.html?action=click\&module=RelatedLinks\&pgtype=Article}{the
center of Italy's outbreak}, recently moved with her 2-year-old son to
her second home in Rome. But no day care facility would accept her son
because other parents did not want anyone from Milan near their
children, Ms. Tagliabue said.

The closings in Italy --- which include day care in addition to schools
and universities --- are likely to create problems for parents
nationwide.

Ms. Tagliabue has turned down several job proposals, she said, since she
is unable to work at home without a babysitter for her young child.

``It's right to close schools, but that has a cost,'' she said. ``The
government could have done something for mothers --- we are also in
quarantine.''

Image

Spraying disinfectant in a high school classroom in Athens,
Greece.Credit...Yorgos Karahalis/Associated Press

\hypertarget{beyond-the-classroom}{%
\subsection{Beyond the Classroom}\label{beyond-the-classroom}}

The epidemic has shaken entire industries that rely on the rituals of
students in school and parents at work.

School administrators in Japan, surprised by the abrupt decision to
close schools, have rushed to cancel orders for cafeteria lunches,
stranding suppliers with unwanted groceries and temporarily unneeded
employees.

Kazuo Tanaka, deputy director of the Yachimata School Lunch Center in
central Japan, said it scrapped orders for ingredients to make about
5,000 lunches for 13 schools. It would cost the center about 20 million
yen, nearly \$200,000, each month that school was out, he said.

``Bakeries are blown,'' said Yuzo Kojima, secretary general at the
National School Lunch Association. ``Dairy farmers and vegetable farmers
will be hit. The workers at the school lunch centers cannot work.''

To blunt the effects, Japan's government is
\href{https://www.japantimes.co.jp/news/2020/02/29/national/science-health/shinzo-abe-coronavirus/\#.Xl-dEBMzb-Y}{offering
financial help} to parents, small businesses and health care providers.
But school lunch officials said they had not heard about compensation
for their workers.

In Hong Kong, many among its
\href{https://www.legco.gov.hk/research-publications/english/1617rb04-foreign-domestic-helpers-and-evolving-care-duties-in-hong-kong-20170720-e.pdf}{sizable
population of domestic helpers} have been jobless as affluent parents
have enrolled children overseas.

Demand for nannies had already dropped by a third when the outbreak
began, because many companies allowed parents to work from home, said
Felix Choi, the director of Babysitter.hk, a nanny service. Now some
expatriate families have left the city rather than wait out the
closings.

``Over 30 percent of our client base is Western expat families, and I'm
not seeing many of them coming back to Hong Kong at this moment,'' Mr.
Choi said. ``Most of them informed us they will only come back after
school restarts.''

Image

The University of Milan was closed in February. The surrounding region
has a large coronavirus outbreak.Credit...Andrea Mantovani for The New
York Times

Vivian Wang reported from Hong Kong, and Makiko Inoue from Tokyo.
Reporting was contributed by Su-Hyun Lee from Seoul, South Korea;
Constant Méheut from Paris; Elisabetta Povoledo from Rome; Niki
Kitsantonis from Athens; and Farnaz Fassihi and Rick Gladstone from New
York.

Advertisement

\protect\hyperlink{after-bottom}{Continue reading the main story}

\hypertarget{site-index}{%
\subsection{Site Index}\label{site-index}}

\hypertarget{site-information-navigation}{%
\subsection{Site Information
Navigation}\label{site-information-navigation}}

\begin{itemize}
\tightlist
\item
  \href{https://help.nytimes.com/hc/en-us/articles/115014792127-Copyright-notice}{©~2020~The
  New York Times Company}
\end{itemize}

\begin{itemize}
\tightlist
\item
  \href{https://www.nytco.com/}{NYTCo}
\item
  \href{https://help.nytimes.com/hc/en-us/articles/115015385887-Contact-Us}{Contact
  Us}
\item
  \href{https://www.nytco.com/careers/}{Work with us}
\item
  \href{https://nytmediakit.com/}{Advertise}
\item
  \href{http://www.tbrandstudio.com/}{T Brand Studio}
\item
  \href{https://www.nytimes.com/privacy/cookie-policy\#how-do-i-manage-trackers}{Your
  Ad Choices}
\item
  \href{https://www.nytimes.com/privacy}{Privacy}
\item
  \href{https://help.nytimes.com/hc/en-us/articles/115014893428-Terms-of-service}{Terms
  of Service}
\item
  \href{https://help.nytimes.com/hc/en-us/articles/115014893968-Terms-of-sale}{Terms
  of Sale}
\item
  \href{https://spiderbites.nytimes.com}{Site Map}
\item
  \href{https://help.nytimes.com/hc/en-us}{Help}
\item
  \href{https://www.nytimes.com/subscription?campaignId=37WXW}{Subscriptions}
\end{itemize}
