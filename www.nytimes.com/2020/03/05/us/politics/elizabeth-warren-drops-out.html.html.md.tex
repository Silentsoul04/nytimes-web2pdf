Sections

SEARCH

\protect\hyperlink{site-content}{Skip to
content}\protect\hyperlink{site-index}{Skip to site index}

\href{https://www.nytimes.com/section/politics}{Politics}

\href{https://myaccount.nytimes.com/auth/login?response_type=cookie\&client_id=vi}{}

\href{https://www.nytimes.com/section/todayspaper}{Today's Paper}

\href{/section/politics}{Politics}\textbar{}Elizabeth Warren, Once a
Front-Runner, Drops Out of Presidential Race

\url{https://nyti.ms/2VP4MVA}

\begin{itemize}
\item
\item
\item
\item
\item
\item
\end{itemize}

\begin{itemize}
\item
  \href{https://www.nytimes.com/2020/07/31/us/elections/biden-vs-trump.html?action=click\&pgtype=Article\&state=default\&region=TOP_BANNER\&context=storylines_menu}{Election
  Updates}
\item
  \href{https://www.nytimes.com/article/biden-vice-president-2020.html?action=click\&pgtype=Article\&state=default\&region=TOP_BANNER\&context=storylines_menu}{Biden's
  V.P. Search}
\item
  \href{https://www.nytimes.com/interactive/2020/07/24/us/politics/trump-biden-campaign-donors.html?action=click\&pgtype=Article\&state=default\&region=TOP_BANNER\&context=storylines_menu}{Map
  of Donations}
\item
  \href{https://www.nytimes.com/interactive/2020/us/elections/delegate-count-primary-results.html?action=click\&pgtype=Article\&state=default\&region=TOP_BANNER\&context=storylines_menu}{Delegate
  Count}
\item
  \href{https://www.nytimes.com/interactive/2019/us/politics/2020-presidential-candidates.html?action=click\&pgtype=Article\&state=default\&region=TOP_BANNER\&context=storylines_menu}{The
  Candidates}
\item
  \href{https://www.nytimes.com/newsletters/politics?action=click\&pgtype=Article\&state=default\&region=TOP_BANNER\&context=storylines_menu}{Politics
  Newsletter}
\end{itemize}

Advertisement

\protect\hyperlink{after-top}{Continue reading the main story}

Supported by

\protect\hyperlink{after-sponsor}{Continue reading the main story}

\hypertarget{elizabeth-warren-once-a-front-runner-drops-out-of-presidential-race}{%
\section{Elizabeth Warren, Once a Front-Runner, Drops Out of
Presidential
Race}\label{elizabeth-warren-once-a-front-runner-drops-out-of-presidential-race}}

Ms. Warren, a senator and former law professor, staked her campaign on
fighting corruption and changing the rules of the economy.

\includegraphics{https://static01.nyt.com/images/2020/03/01/us/03vid-Warren-Live/03vid-Warren-Live-videoSixteenByNine3000-v2.jpg}

\href{https://www.nytimes.com/by/shane-goldmacher}{\includegraphics{https://static01.nyt.com/images/2018/07/27/multimedia/author-shane-goldmacher/author-shane-goldmacher-thumbLarge.png}}\href{https://www.nytimes.com/by/astead-w-herndon}{\includegraphics{https://static01.nyt.com/images/2018/09/14/us/author-head-astead/author-head-astead-thumbLarge-v2.png}}

By \href{https://www.nytimes.com/by/shane-goldmacher}{Shane Goldmacher}
and \href{https://www.nytimes.com/by/astead-w-herndon}{Astead W.
Herndon}

\begin{itemize}
\item
  Published March 5, 2020Updated March 10, 2020
\item
  \begin{itemize}
  \item
  \item
  \item
  \item
  \item
  \item
  \end{itemize}
\end{itemize}

CAMBRIDGE, Mass. ---
\href{https://www.nytimes.com/2020/03/10/podcasts/the-daily/warren.html?action=click\&module=Briefings\&pgtype=Homepage}{Senator
Elizabeth Warren} entered the 2020 race with expansive plans to use the
federal government to remake American society, pressing to strip power
and wealth from a moneyed class that she saw as fundamentally corrupting
the country's economic and political order.

She exited on Thursday after her avalanche of progressive policy
proposals, which briefly elevated her to front-runner status last
fall,~failed to attract a broader political coalition in a Democratic
Party increasingly, if not singularly, focused on defeating President
Trump.

Her departure means that a Democratic field that began as the most
diverse in American history --- and included six women --- is now
essentially down to two white men: former Vice President Joseph R. Biden
Jr. and Senator Bernie Sanders.

Ms. Warren said that from the start, she had been told there were only
two true lanes in the 2020 contest: a liberal one dominated by Mr.
Sanders, 78, and a moderate one led by Mr. Biden, 77.

``I thought that wasn't right,'' Ms. Warren said in front of her house
in Cambridge as she suspended her campaign, ``But evidently I was
wrong.''

Though her vision energized many liberals ---~the unlikely chant of
``big, structural change'' rang out at her rallies --- it did not find a
wide enough audience among the party's working-class and diverse base.
Now her potential endorsement is highly sought, and both Mr. Sanders and
Mr. Biden have spoken with her in the days since Super Tuesday losses
sealed her political fate, though she revealed precious little of her
intentions on Thursday.

``I need some space around this,'' she said.

Ms. Warren's impact on the race was far greater than just the outcome
for her own candidacy. Her policy plans drove the agenda. She
effectively pushed
\href{https://www.nytimes.com/2020/03/04/us/politics/michael-bloomberg-drops-out.html}{former
Mayor Michael R. Bloomberg of New York, a centrist billionaire, out of
the race} with a dominant debate performance last month.

And her ability to raise well over \$100 million and fully fund a
presidential campaign without holding high-dollar fund-raisers
demonstrated that other candidates, beyond Mr. Sanders and his intensely
loyal small-dollar donors, could do so in the future.

\includegraphics{https://static01.nyt.com/images/2020/03/06/us/politics/06warren-out-p1/05warren-out4-articleLarge.jpg?quality=75\&auto=webp\&disable=upscale}

\includegraphics{https://static01.nyt.com/images/2017/01/29/podcasts/the-daily-album-art/the-daily-album-art-articleInline-v2.jpg?quality=75\&auto=webp\&disable=upscale}

\hypertarget{listen-to-the-field-what-happened-to-elizabeth-warren}{%
\subsubsection{Listen to `The Field': What Happened to Elizabeth
Warren?}\label{listen-to-the-field-what-happened-to-elizabeth-warren}}

We went to Massachusetts to explore how the most diverse slate of
candidates in U.S. history become a contest, again, between two men.

transcript

Back to The Daily

bars

0:00/36:42

-36:42

transcript

\hypertarget{listen-to-the-field-what-happened-to-elizabeth-warren-1}{%
\subsection{Listen to `The Field': What Happened to Elizabeth
Warren?}\label{listen-to-the-field-what-happened-to-elizabeth-warren-1}}

\hypertarget{hosted-by-michael-barbaro-produced-by-austin-mitchell-and-jessica-cheung-and-edited-by-lisa-tobin-and-mike-benoist}{%
\subsubsection{Hosted by Michael Barbaro, produced by Austin Mitchell
and Jessica Cheung, and edited by Lisa Tobin and Mike
Benoist}\label{hosted-by-michael-barbaro-produced-by-austin-mitchell-and-jessica-cheung-and-edited-by-lisa-tobin-and-mike-benoist}}

\hypertarget{we-went-to-massachusetts-to-explore-how-the-most-diverse-slate-of-candidates-in-us-history-become-a-contest-again-between-two-men}{%
\paragraph{We went to Massachusetts to explore how the most diverse
slate of candidates in U.S. history become a contest, again, between two
men.}\label{we-went-to-massachusetts-to-explore-how-the-most-diverse-slate-of-candidates-in-us-history-become-a-contest-again-between-two-men}}

\begin{itemize}
\item
  jessica cheung\\
  Hello.
\item
  austin mitchell\\
  Hello!
\item
  astead herndon\\
  Hey, how are you?
\item
  austin mitchell\\
  Buddy.
\item
  astead herndon\\
  Yikes. What a day already. Um ---
\item
  austin mitchell\\
  Yeah, what do you mean when you say, ``What a day already``?
\item
  astead herndon\\
  Well, we just reported that Elizabeth Warren is dropping out of the
  presidential race. We're here in Boston outside of her house. Not in
  Boston, actually. Cambridge. Waiting for her to speak to a horde of
  media, both local and national.
\item
  speaker 1\\
  We have ---
\item
  speaker 2\\
  I understand that, but I was here before you guys all jumped in front
  of me.
\item
  speaker 1\\
  Because we just found out we have to move.
\end{itemize}

astead herndon

From The New York Times, this is ``The Field.'' I'm Astead Herndon in
Massachusetts.

\begin{itemize}
\item
  {[}cheering{]}\\
  Around 12:30 on Thursday, Senator Elizabeth Warren came out of a side
  door of her house with her husband and her golden retriever and
  addressed the media.
\item
  archived recording (elizabeth warren)\\
  All right. So I announced this morning that I am suspending my
  campaign for president. I say this with a deep sense of gratitude.
\end{itemize}

astead herndon

It's been two days since Super Tuesday, where Elizabeth Warren's best
finish was in third place, including in her home state of Massachusetts.
That put her behind her campaign's already lowered expectations and made
a gathering like today feel almost inevitable.

\begin{itemize}
\tightlist
\item
  archived recording (elizabeth warren)\\
  For every single person ---
\end{itemize}

astead herndon

She thanks her supporters and her staff and takes questions.

\begin{itemize}
\item
  archived recording (reporter)\\
  Senator, will you be making an endorsement today? We know that you
  spoke with both Joe Biden and Bernie Sanders yesterday.
\item
  archived recording (elizabeth warren)\\
  Not today. Not today. I need some space around this.
\end{itemize}

astead herndon

And when it comes to why she has to drop out?

\begin{itemize}
\item
  archived recording (elizabeth warren)\\
  You know, I was told at the beginning of this whole undertaking that
  there are two lanes, a progressive lane that Bernie Sanders is the
  incumbent for, and a moderate lane that Joe Biden is the incumbent
  for. And there's no room for anyone else in this. I thought that
  wasn't right, but evidently I was wrong.
\item
  archived recording (reporters)\\
  Senator, why do you think ---
\end{itemize}

astead herndon

And on the question of gender?

\begin{itemize}
\item
  archived recording (reporter)\\
  And I wonder what the message would be to the women and girls who feel
  like we're left with two white men to decide between?
\item
  archived recording (elizabeth warren)\\
  I know. One of the hardest parts of this is all those little girls who
  are going to have to wait four more years. That's going to be hard.
\item
  archived recording (reporters)\\
  Senator, why do you think ---
\end{itemize}

astead herndon

She gets emotional, but there are clearly things that she's left unsaid.

But when you ask her supporters who have come to the house to watch this
speech, they go there.

\begin{itemize}
\item
  warren supporter 1\\
  Oh, I'm so sad. Yesterday, I so sad I couldn't --- I couldn't move.
\item
  warren supporter 2\\
  I'm frustrated, I'm disappointed, and sad.
\item
  warren supporter 3\\
  I'm heartbroken that very clearly most qualified candidate is out of
  the race.
\item
  warren supporter 4\\
  Sadly, too many people in this country aren't ready for a woman
  president, which is an unfortunate thing.
\item
  warren supporter 5\\
  Very disappointed, but I guess there's never going to be a time for a
  woman. She's my generation, and we're not going to see it now. This
  was our --- it's not going to happen.
\item
  toddler\\
  (CRYING) Grandma, Grandma, I wanna go home!
\item
  warren supporter 5\\
  Maybe her generation. All right, we gotta go. Look at this little girl
  looking at the doggie. She likes the doggie.
\end{itemize}

astead herndon

Today, millions of voters across six states will cast their ballot for
the two viable Democratic candidates left: Joe Biden and Bernie Sanders.
What began as a contest with historic diversity, along racial and gender
lines, has now come down to two men, 70 plus, both white. And as someone
who covered senator Kamala Harris and Elizabeth Warren, and with Warren
especially, who once led in national polling, I'm left to wonder how did
we get here? How did it end up this way?

\begin{itemize}
\item
  {[}doorbell ringing{]}
\item
  astead herndon\\
  Wow. Oh, those are one of the fancy doorbells, where you can look at
  it with your phone. Hello, how are you?
\item
  lyn licciardello\\
  Hello, a party.
\item
  astead herndon\\
  Hi, my name's Astead. I'm a politics reporter at The Times.
\item
  lyn licciardello\\
  Hi, nice to meet you.
\item
  astead herndon\\
  It's nice to meet you.
\item
  lyn licciardello\\
  I'm Lyn. What did you say your name is?
\item
  astead herndon\\
  Astead.
\item
  lyn licciardello\\
  Astead?
\item
  astead herndon\\
  Yes.
\item
  jessica cheung\\
  I'm Jessica ---
\end{itemize}

astead herndon

So the next day, me and producers Austin Mitchell and Jessica Cheung go
to North Andover, Massachusetts.

\begin{itemize}
\item
  austin mitchell\\
  Should we take our shoes off?
\item
  lyn licciardello\\
  Oh, you don't have to.
\item
  jessica cheung\\
  You sure?
\end{itemize}

astead herndon

To meet with a pretty typical Warren supporter, Lyn Licciardello.

\begin{itemize}
\item
  astead herndon\\
  I only have one sock on.
\item
  lyn licciardello\\
  It's fine. {[}LAUGHTER{]}
\item
  jessica cheung\\
  You have what?
\item
  astead herndon\\
  I only have one sock on.
\item
  {[}laughter{]}
\item
  austin mitchell\\
  How does that happen?
\end{itemize}

astead herndon

Her husband Tom is there, too.

\begin{itemize}
\item
  lyn licciardello\\
  Tommy, will you pour water, please?
\item
  tom licciardello\\
  Yes, I knew there was a role for me somewhere.
\end{itemize}

astead herndon

And her cousin Kathleen.

\begin{itemize}
\tightlist
\item
  kathleen lambert\\
  Let me take your coats.
\end{itemize}

astead herndon

And we all sit down in their living room.

\begin{itemize}
\item
  astead herndon\\
  What do you do?
\item
  lyn licciardello\\
  I am a nurse, but I teach exercise now.
\item
  astead herndon\\
  Oh, very nice. Give me --- what exercise?
\item
  lyn licciardello\\
  Oh, well, I teach a class that's about the first half is aerobic, and
  then there's some stretching and strength training. It's about an hour
  class at the senior center in Lawrence.
\item
  astead herndon\\
  Awesome. How long have you been in North Andover?
\item
  lyn licciardello\\
  Since 1949.
\item
  astead herndon\\
  Hm. And when did you first notice Senator Warren?
\item
  lyn licciardello\\
  I noticed Senator Warren years ago. I think it was around 2012. I
  happened to be reading the paper one morning. And I noticed that a
  congressman named Todd Akin had said a horrible thing about women and
  pregnancy, saying that if a woman were to get pregnant as a result of
  rape, then her body has a way of getting rid of that.
\item
  archived recording (todd akin)\\
  It seems to me, first of all, from what I understand from doctors,
  that's really rare. If it's a legitimate rape, the female body has
  ways to try to shut that whole thing down.
\item
  lyn licciardello\\
  And my eyes just flew open, and I said, oh my god, this man is in our
  Congress. I was flabbergasted, and I said to my husband, who is that
  woman? There was a --- at the same time, I was not happy with Scott
  Brown, who was our senator at the time.
\item
  archived recording\\
  He doesn't stand up for women's reproductive rights and economic
  security. He co-sponsored legislation to let employers deny women
  coverage for birth control or even mammograms. He had two
  opportunities ---
\item
  lyn licciardello\\
  And I said, who's that woman that's going to run against Scott Brown?
  I heard that a woman is going to run against Scott Brown.
\item
  archived recording (elizabeth warren)\\
  I'm Elizabeth Warren. I'm running for the United States Senate. And
  before you hear a bunch of ridiculous attack ads, I want to tell you
  who I am. Like a lot of you, I came up the hard way.
\item
  lyn licciardello\\
  And I said, I have to do something to help her get elected.
\item
  archived recording (elizabeth warren)\\
  But Washington is still rigged for the big guys, and that's gotta
  change. I'm Elizabeth Warren, and I approve this message, because I
  want Massachusetts families to have a level playing field.
\end{itemize}

astead herndon

And so Lyn becomes a volunteer for the Warren Senate campaign.

\begin{itemize}
\item
  lyn licciardello\\
  I was always on board with Elizabeth right after that.
\item
  astead herndon\\
  Why do you think you felt so drawn to Elizabeth Warren?
\item
  lyn licciardello\\
  Because in many ways, she's me. She's me. She has the same feelings
  that I have. She's actually very close to my age. She has a wonderful
  way of kind of looking into your heart and mind. She's interested in
  you. She's interested in the people.
\item
  archived recording (elizabeth warren)\\
  Hi, I'm Elizabeth Warren. It's very nice to see you.
\item
  speaker\\
  I'm going to vote for you.
\item
  archived recording (elizabeth warren)\\
  Wonderful, say that again. Fabulous. We're here for the chicken.
  {[}LAUGHTER{]} Good to see you. I like your shirt. Very handsome.
\item
  lyn licciardello\\
  Elizabeth has that heart. And, she's brilliant.
\item
  archived recording (elizabeth warren)\\
  And despite the odds, you elected the first woman senator to the state
  of Massachusetts. {[}CHEERING{]}
\item
  astead herndon\\
  How did it feel? I mean, she wins the race, obviously. How did that
  feel?
\item
  lyn licciardello\\
  Oh, my god.

  It was so exciting. I still can't say that without crying. It was so
  exciting.
\end{itemize}

astead herndon

So then, in 2016.

\begin{itemize}
\item
  lyn licciardello\\
  Well, I was for Hillary. Hillary, I was very invested in having
  Hillary be president. She had all the qualifications. She was more
  qualified than anybody who's ever been president, in my opinion. But
  because she was a woman, I knew it would be difficult. But I still
  thought she could win.
\item
  astead herndon\\
  Did you know people or did you hear people say, I won't vote for
  Clinton because she's a woman?
\item
  lyn licciardello\\
  Not like that. But here's what I did hear. I was talking to a woman
  who was kind of a stranger, but we were chatting. And we were talking
  about politics and about how we feel about certain things. So I guess
  it was like immigration, climate change and things like that. And this
  woman was on board with all of the Democratic ideals. And then I
  mentioned Hillary Clinton, and she said, ``Oh, I hate her.'' And I
  said, ``Really? Because she's the one who stands for all of these
  things that we're talking about.'' ``No, no. I can't stand her.'' I
  said, ``Well, why don't you like her?'' ``Oh, I have no idea.'' I
  said, ``Is it because she's an aggressive woman? Is she too
  aggressive? Is she too loud? Does she express herself too much?'' And
  her reply was, ``No. I don't know why.'' So like a minute later, I
  mentioned Elizabeth Warren. She said, ``Oh, I can't stand her
  either.'' So I was like, ``She agrees with you about everything! All
  the things that you're saying you believe in, she is promoting.''
  ``No, well, I can't stand her.'' So, I know. I mean, I've been a woman
  my whole life. So I know very well that that is the reason. Even women
  will vote against women, because they're women.
\item
  archived recording (joy behar)\\
  There are people out there who have this idea that you're not
  trustworthy, that they don't like you for some reason. What is that
  about, in your opinion?
\item
  archived recording (hillary clinton)\\
  You know, Joy, obviously I've thought a lot about it, because I don't
  like to hear it.
\item
  archived recording (joy behar)\\
  Yeah.
\item
  archived recording (hillary clinton)\\
  So I need to figure out what's behind it. You know, I am perhaps a
  more serious person, a more reserved person than is in the public
  arena these days. So I think people then say, ``Well, she's serious.
  She's reserved. Can I really like her?''
\item
  archived recording (joy behar)\\
  But what is inauthentic? What's inauthentic mean?
\item
  archived recording (hillary clinton)\\
  I don't understand that. I don't understand that. Because I've been
  pretty much the same person my entire life, for better or worse,
  right?
\item
  archived recording (joy behar)\\
  Mm-hm.
\item
  lyn licciardello\\
  Clinton losing made a difference. Clinton losing did make it harder
  for me to think that a woman could win.
\end{itemize}

astead herndon

So this year, when Elizabeth Warren announces that she's running, Lyn
has mixed feelings.

\begin{itemize}
\item
  lyn licciardello\\
  I love that people were getting to know her all across the country.
\item
  archived recording (elizabeth warren)\\
  Hi.
\item
  archived recording (raelyn)\\
  Hi, my name's Raelyn.
\item
  archived recording (elizabeth warren)\\
  Hi, Raelyn.
\item
  archived recording (raelyn)\\
  I was wondering if there was ever a time in your life where somebody
  you really looked up to maybe didn't accept you as much and how you
  dealt with that?
\item
  archived recording (elizabeth warren)\\
  Yeah.

  My mother and I had very different views of how to build a future. She
  wanted me to marry well.

  And I really tried, and it just didn't work out. But I also know it
  was the right thing to do. And sometimes, you just gotta do what's
  right inside. You gotta take care of yourself first and do this. Give
  me a hug. {[}APPLAUSE{]}
\item
  lyn licciardello\\
  And it was a pleasure for me to see that, although I was a little bit
  worried.
\item
  archived recording 1\\
  In the wake of her third place finish in Iowa and fourth place finish
  in New Hampshire, Warren said, ``My job is to persist.'' But
  persisting and winning are two very different things.
\item
  archived recording 2\\
  Senator Elizabeth Warren, she came in fifth place in South Carolina.
\item
  archived recording 3\\
  Elizabeth Warren trailing in fourth place in Nevada.
\item
  archived recording 4\\
  I think the biggest question that Elizabeth Warren has to answer is,
  where does she win?
\end{itemize}

astead herndon

Lyn voted early in Massachusetts for Senator Warren. But in the days
leading up to Super Tuesday, she's questioning her vote and wondering if
she did the right thing.

\begin{itemize}
\item
  lyn licciardello\\
  But I didn't feel --- I didn't feel bad that I had already voted for
  her at all, because like she said, vote with your heart. And she was
  my heart.
\item
  jessica cheung\\
  Did you know anyone personally in your life who were for Warren and
  then jumped ship?
\item
  lyn licciardello\\
  Yes, my cousin Kathleen.
\item
  astead herndon\\
  Raising her hand.
\item
  kathleen lambert\\
  Yes, actually as soon as Buttigieg and Klobuchar backed Biden --- and
  I was kind of waiting to see how it all shook out a little bit --- I
  voted for Biden because we have to stop Sanders, in my opinion. But I
  would have voted for Warren, because I voted for Hillary.
\end{itemize}

astead herndon

As Kathleen is talking, Tom nods and raises his hand.

\begin{itemize}
\item
  austin mitchell\\
  Yeah, did you raise your hand, too?
\item
  tom licciardello\\
  Me too. Yeah, yeah.

  Yeah, actually as the sole old white guy in the room, I, too, did vote
  for Biden, though I love Elizabeth Warren, and she would have made an
  extraordinarily wonderful president.
\item
  astead herndon\\
  By the time Massachusetts was voting, if she looked more electable, if
  she was a front-runner, you all would have stuck with her?
\item
  tom licciardello\\
  Yeah, you bet.
\item
  kathleen lambert\\
  Absolutely, yeah.
\item
  astead herndon\\
  Did you see when she --- obviously when she comes out and talks at her
  house, did you see that?
\item
  lyn licciardello\\
  Oh, yes. I was here in the living room. I was in my own living room
  listening to it, watching. And yeah, I cried through the whole thing.
  It's heartbreaking.
\item
  astead herndon\\
  Do you remember something she said that day that stuck with you or
  maybe caused you to feel that emotion?
\item
  lyn licciardello\\
  Well, one of the things was that, you know how she talked about the
  pinkie swear that she does with the little kids? The first time I met
  her, which was way back when, before she was elected senator for the
  first time, she was doing that with little girls that were there.
\item
  jessica cheung\\
  What is the pinkie swear? What is she promising girls?
\item
  lyn licciardello\\
  She says that girls can be president. She gets right down to their
  level. She gets right down to their eye level and talks to the girls
  like that.
\item
  astead herndon\\
  The race is now down to two guys, after starting with such a diverse
  field, gender ratio, all of that. How does it feel for it to be down
  to two men, when you had four women senators at the start, all of whom
  you liked?
\item
  lyn licciardello\\
  I know. It's sick, isn't it? {[}CHUCKLES{]}

  You know, it was almost inevitable. I think people pan --- not
  panicked, but I think that people are very, very concerned that we
  have to beat Trump.
\item
  astead herndon\\
  But why does the feeling of ``need to beat Trump'' then translate to
  men? So you're saying, I think a lot of people are scared, I think a
  lot of people just want to beat Trump, and that's why it came down to
  two men. What is necessarily male about those qualities?
\item
  lyn licciardello\\
  Because there are so many people in the country that just would not
  vote for a woman, like that woman that I was talking about before. And
  we really feel the need to prevail this time, especially.
\item
  astead herndon\\
  So the idea that a woman candidate is a risk because of other people's
  or the country's sexism?
\item
  lyn licciardello\\
  Yes, that's how I feel. It's a terrible thing to have to feel, but I
  do feel that way. Right now. I don't think it's always going to be
  like that, but I think it's the way it is right now.
\end{itemize}

{[}music{]}

\begin{itemize}
\item
  jessica cheung\\
  All right.
\item
  austin mitchell\\
  Thank you. You can go skiing now.
\item
  lyn licciardello\\
  Oh, good. I'm sorry you can't come with us. That was too bad.
\item
  jessica cheung\\
  Tell us where you're going.
\item
  lyn licciardello\\
  Warren, Vermont. {[}LAUGHS{]}

  It's very funny, because we have our ski place right next to our
  daughter and son-in-law in Warren, Vermont. And our son-in-law's
  parents live in Warren, Rhode Island.
\item
  laughter
\item
  astead herndon\\
  A Warren household to the end.
\item
  lyn licciardello\\
  Yes.
\end{itemize}

astead herndon

It's one voter's view that sexism is what consumed the Elizabeth Warren
campaign. And certainly that's one that's popular among her most
die-hard supporters. But I'm wondering, is this the view inside her own
campaign? Do they think that the barriers that gender placed on them
were too big to overcome?

The same day we met with Lyn, we headed to the Charlestown neighborhood
of Boston, to Elizabeth Warren's campaign headquarters.

\begin{itemize}
\tightlist
\item
  astead herndon\\
  It's kind of like a warehousey building, nondescript, very on brand.
  There is a Dunkin' Donuts right across the street.
\end{itemize}

astead herndon

To meet with someone who's worked for the Warren campaign from the
start.

\begin{itemize}
\tightlist
\item
  kristen orthman\\
  I'm Kristen Orthman, the communications director for Elizabeth Warren.
\end{itemize}

astead herndon

As communications director, Kristen's in charge of trying to best
translate the candidate to the rest of the country, particularly through
the media. And over the course of this campaign, we got to know each
other pretty well. Now that the campaign has ended, I'm wondering
whether Kristen will speak more candidly about what went wrong and about
what role gender specifically played in the campaign's demise.

\begin{itemize}
\item
  astead herndon\\
  Is there something different about planning communications and media
  for women politicians?
\item
  kristen orthman\\
  Yes.
\item
  astead herndon\\
  {[}LAUGHS{]} Yeah, in what way?
\item
  kristen orthman\\
  I think that there can be more caution when you're working for a
  woman, because you're viewing things through the lens of much more of
  ``how will this be perceived?'' And I think I've an appreciation for
  the challenges of the Clinton presidential campaign probably now than
  I did when I first started.
\item
  astead herndon\\
  In what way?
\item
  kristen orthman\\
  I think that the fact that there were stories when Elizabeth first ran
  --- there was an infamous one like the day she announced.
\end{itemize}

astead herndon

This is December 31, 2018, when Warren first announced her intentions to
seek the presidency.

\begin{itemize}
\item
  kristen orthman\\
  Talking about like, I think it was the likability factor, her versus
  Hillary. Because they were two women who ran for president, two white
  women who ran for president that had blond hair. I mean, I guess I'm
  not quite sure what else warranted necessarily that.
\item
  astead herndon\\
  So when it becomes clear she's running for president, how forefront of
  mind was gender and the need to define her on her own, outside of
  Clinton or whatever terms?
\item
  kristen orthman\\
  I think when you're running for president, male, female, you have to
  be yourself. So what I was always, and what our team and her always
  wanted to make sure is, you are showing what you would hope is the
  truest version of yourself.
\end{itemize}

astead herndon

One of the critiques of the Clinton campaign, fair or not, was that many
voters felt like they never really knew her authentic self, that there
was a barrier between candidate and voter built up over all those years
in the public eye. So Kristen and her team tried to go in the opposite
direction. To distinguish Warren, both from Clinton but also from
everyone else.

\begin{itemize}
\item
  kristen orthman\\
  You know, she runs out on stage.
\item
  archived recording (elizabeth warren)\\
  Hello, Indianola!
\item
  kristen orthman\\
  She dances.
\item
  archived recording\\
  {[}CHEERING{]}
\item
  kristen orthman\\
  She stays for hours for photos.
\item
  archived recording (elizabeth warren)\\
  So we just finished our event here in New York City, and I got a lot
  of notes and a {[}INAUDIBLE{]}.
\item
  kristen orthman\\
  She is just like the compassionate and joyful person that I know
  behind the scenes, was the person that was on a town hall stage. Or
  the fighter that I've also seen behind the scenes, and that many
  people saw, whether in the hearing room or otherwise, was the person
  on the debate stage. There is a vulnerability that comes with that,
  being a female candidate versus being a male candidate. I always was
  thinking through the risks in that, because I just knew that the
  ``mistakes'' that --- I'm using quotes, because I don't always they
  were, but they were perceived as mistakes --- that female candidates
  make. It's like a higher bar.
\end{itemize}

astead herndon

But to her, these risks were necessary components of running an
authentic campaign.

\begin{itemize}
\item
  kristen orthman\\
  Let's do the things that have now become like signatures, were
  signatures of the campaign. It's like, well, OK, she's running for
  president to say something and do something. So let's start doing
  that.
\item
  archived recording\\
  2020 Democratic presidential candidate Elizabeth Warren is leading the
  pack when it comes to policy proposals.
\item
  archived recording (elizabeth warren)\\
  So I've got a plan for 3.2 million new housing units in America. I've
  got a plan to put \$800 billion new federal into our public schools.
  Student loan debt, I've got a plan for that. And corruption.
\end{itemize}

astead herndon

She became known as the ``I have a plan for that'' candidate.

\begin{itemize}
\item
  kristen orthman\\
  The ``I have a plan for that'' just happened organically. Time
  Magazine put it on the cover, and that's when it became more of a
  thing. I think it happened grassroots level before that. People start
  talking about it, because we were doing it.
\item
  astead herndon\\
  There was like a whole meme section of like, Warren plan.
\item
  kristen orthman\\
  Exactly, that's what I mean. Yeah, exactly, exactly.
\end{itemize}

astead herndon

And by the end of last fall, Warren has crossed into front-runner
status. She's leading in some national polls. But this is also about the
time that I noticed a shift in the race.

\begin{itemize}
\item
  astead herndon\\
  The primary change from a contest to ideas, to one of just as
  obsession about who can win. Do you think that's true?
\item
  kristen orthman\\
  I think the primary campaign has always been about who can win.
\item
  astead herndon\\
  How does that impact the women, specifically, who are running?
\item
  kristen orthman\\
  I do think I need distance in order to fully formulate, but I don't
  think there's any question that electability was viewed through a lens
  that probably hurt the women candidates. Because there was a
  perception that, after 2016, even though the female candidate got 3
  million more votes, is a woman not electable? And she has said before
  publicly that she would hear that from people in the early states.
  Like, ``I'm worried about who can beat Trump.'' But I don't think all
  of a sudden in October it was like, ``Oh, let's make this primary
  about who's going to beat Donald Trump.'' That's what it's been.
\end{itemize}

astead herndon

I would largely agree. Certainly, electability has never been far from
mind for most Democratic voters. But as more and more people tuned into
the race, particularly around this time in the late fall, it shifted its
tone. Policy ideas took a back seat to that electability question. And
the candidates who had most clearly articulated their path to victory
started to rise, while Elizabeth Warren started to fall. This coincided
with rival candidates like Pete Buttigieg and Amy Klobuchar increasingly
casting Warren's campaign as out of touch with the mainstream Democratic
Party and a real general election risk. And so to respond to that
scrutiny, the Warren campaign tries to reposition itself as a unity
candidate, someone who can actually bring all sides together.

\begin{itemize}
\item
  astead herndon\\
  There's sometime during the unity candidate phase, where I did feel
  like it was different from the fight that we had heard before.
\item
  kristen orthman\\
  Mm-hm.
\item
  astead herndon\\
  How do you square those two versions that we did see just this year?
\item
  kristen orthman\\
  I mean, I think you can both --- unity doesn't mean not fighting. I
  never didn't think she was herself.
\item
  astead herndon\\
  Take us into debate prep, for instance. Like, are you sitting there
  thinking, ``We have to package a candidate, and there are concerns
  about how she'll be perceived if she attacks too much or attacks too
  little.'' Like, how much is gender a concern as you are thinking about
  the big national combative moments?
\item
  kristen orthman\\
  Mm-hm. You know, I don't want to make a big statement saying it's
  easier for men to attack than women. I do think that there are
  probably greater consequences to a failed attack by a female than a
  failed attack by a male. Because obviously she had a debate
  performance a couple --- two debates ago --- where she doing some
  level of contrasts with Mayor Bloomgberg.
\item
  laughter
\item
  astead herndon\\
  Understatement of human history.
\item
  archived recording (elizabeth warren)\\
  I'd like to talk about who we're running against --- a billionaire who
  calls women fat broads and horse-faced lesbians. And, no, I'm not
  talking about Donald Trump. I'm talking about Mayor Bloomberg.
  Democrats are not going to ---
\item
  kristen orthman\\
  She was really strong. She was tough, and she was dynamic.
\item
  archived recording (elizabeth warren)\\
  Democrats take a huge risk if we just substitute one arrogant
  billionaire for another. This country has worked ---
\item
  kristen orthman\\
  They were contrasting with each other, and I think overall, it was
  agreed upon that she did well in that exchange.
\end{itemize}

astead herndon

But this came in Nevada, after two straight disappointing finishes in
Iowa and New Hampshire.

\begin{itemize}
\item
  astead herndon\\
  I mean, if we're just going to take Nevada, I feel like that is a
  reason to ask why then and not previously? Were you all in rooms
  saying, ``Well, we can't attack yet``? One of things I remember from
  the second this campaign started was the way that supporters were so
  eager to see her cut down the Bidens and the Bernies and everyone
  else. It really didn't get that payoff until Nevada.
\item
  kristen orthman\\
  Mm-hm.
\item
  astead herndon\\
  Why?
\item
  kristen orthman\\
  So I like reject the premise of your question, because I think that
  you can be both advocating for yourself and creating contrast in ways
  that are not --- that don't always need to be how it happened in
  Nevada.
\item
  archived recording (moderator)\\
  Senator Warren, what did you think when Senator Sanders told you a
  woman could not win the election?
\item
  archived recording\\
  {[}LAUGHTER{]}
\item
  archived recording (elizabeth warren)\\
  I disagreed. Bernie is my friend, and I am not here to try to fight
  with Bernie. But look ---
\item
  kristen orthman\\
  Now, some would say about --- and this is what I heard frequently from
  reporters --- ``Oh, you guys are doing subtle contrasts. People don't
  get that. People don't know that.'' I don't necessarily agree with
  that. I think that you're always looking at the bar of how do we
  balance advocating for yourself and your ideas versus contrasting with
  other people.
\item
  jessica cheung\\
  You characterize Warren as contrasting with Bloomberg.
\item
  kristen orthman\\
  Yes.
\item
  jessica cheung\\
  Others might characterize that as she was attacking Bloomberg or going
  after Bloomberg.
\item
  kristen orthman\\
  Yeah.
\item
  jessica cheung\\
  And I wonder --- yeah, I wonder if in your role, you're choosing your
  words carefully, because you know that male politicians are treated
  differently or characterized differently than female politicians.
\item
  kristen orthman\\
  I very specifically used the word ``contrast.'' So you are correct
  that I was specifically choosing those words. That was your question,
  right? Yeah.
\item
  astead herndon\\
  There is a prevailing view from Warren supporters that gender was the
  foremost reason that she wasn't successful in this race. Do you share
  that feeling?
\item
  kristen orthman\\
  You think that's a prevailing view?
\item
  astead herndon\\
  Yeah, definitely.
\item
  kristen orthman\\
  I just think I probably need more time to think about it. And I'm not
  trying to not answer your question. I think that there were --- what
  she said yesterday around, like, there were basically two ideological,
  I think she called them poles. We could call them lanes. That's not
  necessarily genderized. That is an ideological reason. And then I
  think electability, the idea of electability, was the other reason.
  Now, the idea of electability through the eyes of, ``can a woman
  win``? Certainly, that's gender. And I would add that it wasn't just
  ``can a woman win?'' It was, can a non-white or non-male candidate
  win?
\item
  astead herndon\\
  For that last point, it seems as if, then, this kind of place that
  we've ended, with two people on the ideological poles, both of them
  being white males, was that inevitable?
\item
  kristen orthman\\
  I mean, maybe.

  I don't know.
\end{itemize}

{[}music{]}

\begin{itemize}
\item
  astead herndon\\
  Yeah, we'll see you later.
\item
  jessica cheung\\
  Bye, we'll see you later.
\item
  astead herndon\\
  It's so hard to get people --- I think this is true. I mean,
  particularly being a male reporter asking women about sexism, you need
  people to --- like, oh, were you thinking about this, then? But that's
  not really how biases work, right? Like, does she say ``contrast'' and
  not ``attack'' because of sexism? Maybe. But it's so deeply pervasive
  that it's not something you actively think about as you're doing it.
  And I feel like that makes sometimes the reporting challenge
  difficult, because you're asking these candidates, like, wasn't that
  sexist? Wasn't that racist? Wasn't that blah, blah, blah. And it's
  like, maybe? I think there were a lot of challenges in the race, and
  ideology was one, name recognition is one, fund-raising is one, and
  sexism and gender pervades all of those things. And does it define all
  of those things? Maybe it informs all of those things is a better way
  to put it, but you know, how do you now?
\item
  archived recording (elizabeth warren)\\
  Gender in this race, you know that is the trap question for every
  woman. If you say, yeah, there was sexism in this race, everyone says
  ``whiner.'' And if you say, no, there was no sexism, about a bazillion
  women think, what planet do you live on? I promise you this. I will
  have a lot more to say on that subject later on.
\item
  archived recording\\
  Senator, advice to your supporters right now looking for a candidate.
  What is your advice to them?
\end{itemize}

{[}music{]}

michael barbaro

For an update on the economic fallout from the coronavirus, which
triggered historic declines in global financial markets on Monday,
listen to ``The Latest.'' You can find it on ``The Daily'' feed or by
searching for ``The Latest'' wherever you listen.

That's it for ``The Daily.'' I'm Michael Barbaro. See you tomorrow.

\hypertarget{latest-updates-2020-election}{%
\section{\texorpdfstring{\href{https://www.nytimes.com/2020/07/31/us/elections/biden-vs-trump.html?action=click\&pgtype=Article\&state=default\&region=MAIN_CONTENT_1\&context=storylines_live_updates}{Latest
Updates: 2020
Election}}{Latest Updates: 2020 Election}}\label{latest-updates-2020-election}}

Updated 2020-08-01T01:26:45.732Z

\begin{itemize}
\tightlist
\item
  \href{https://www.nytimes.com/2020/07/31/us/elections/biden-vs-trump.html?action=click\&pgtype=Article\&state=default\&region=MAIN_CONTENT_1\&context=storylines_live_updates\#link-29fdff45}{Kamala
  Harris, a top vice-presidential contender, confronts double
  standards.}
\item
  \href{https://www.nytimes.com/2020/07/31/us/elections/biden-vs-trump.html?action=click\&pgtype=Article\&state=default\&region=MAIN_CONTENT_1\&context=storylines_live_updates\#link-13ec3d9c}{Karen
  Bass and Susan Rice are rising on Biden's vice-presidential
  shortlist.}
\item
  \href{https://www.nytimes.com/2020/07/31/us/elections/biden-vs-trump.html?action=click\&pgtype=Article\&state=default\&region=MAIN_CONTENT_1\&context=storylines_live_updates\#link-49e9a016}{Trump
  says Russian bounties to kill U.S. troops `never took place.'}
\end{itemize}

\href{https://www.nytimes.com/2020/07/31/us/elections/biden-vs-trump.html?action=click\&pgtype=Article\&state=default\&region=MAIN_CONTENT_1\&context=storylines_live_updates}{See
more updates}

Ms. Warren's political demise was a death by a thousand cuts, not a
dramatic implosion but a steady decline. In the fall, most national
polls showed that Ms. Warren was the national pacesetter in the
Democratic field. By December, she had fallen to the edge of the top
tier, wounded by an October debate during which her opponents
relentlessly attacked her, particularly on her embrace of ``Medicare for
all.''

She invested heavily in the early states, with a ground game that was
the envy of her rivals. But it did not pay off: In Iowa, where she had
bet much of her candidacy~--- she had to take out a \$3 million line of
credit before the caucuses to ensure she could pay her bills in late
January --- she wound up in a disappointing third place.

Ms. Warren slid to fourth in
\href{https://www.nytimes.com/interactive/2020/02/11/us/elections/results-new-hampshire-primary-election.html}{New
Hampshire} and
\href{https://www.nytimes.com/interactive/2020/02/22/us/elections/results-nevada-caucus.html}{Nevada},
and to fifth in
\href{https://www.nytimes.com/interactive/2020/02/29/us/elections/results-south-carolina-primary-election.html}{South
Carolina}. By Super Tuesday, her campaign was effectively over --- with
the final blow losing her home state, Massachusetts.

The California results strikingly laid bare
\href{https://www.nytimes.com/2020/03/03/us/politics/elizabeth-warren-super-tuesday.html}{the
demographic cul-de-sac her candidacy had become} as Ms. Warren struggled
to win over voters beyond college-educated white people, in particular
white women. She was poised to win delegates in only a handful of highly
educated enclaves: places like San Francisco, Santa Monica and West
Hollywood.

Though the campaign failed to generate the widespread backing necessary
to win the nomination, Ms. Warren retained a core of fierce loyalists
dedicated to her promise of wholesale change.

Her selfie lines were filled with well-wishers --- young girls seeking
her trademark pinkie promise (``I'm running for president because that's
what girls do''), cutouts of Ms. Warren's likeness, and tattoos of her
adopted slogan: ``Nevertheless, she persisted.'' When her staff gathered
Thursday, many were clad in liberty green, the color her campaign
adopted to symbolize its togetherness.

``One of the hardest parts of this is all those pinkie promises,'' a
visibly emotional Ms. Warren said, describing the ``trap'' of gender for
female candidates.

``If you say, `Yeah, there was sexism in this race,' everyone says,
`Whiner!''' Ms. Warren said. ``If you say, `No, there was no sexism,'
about a bazillion women think, `What planet do you live on?'''

Image

Ms. Warren often made a pinkie promise with young girls at her events,
saying,~``I'm running for president because that's what girls
do.''Credit...Ruth Fremson/The New York Times

Before her exit, Ms. Warren accumulated the second-largest number of
Democratic delegates of any woman to run for president in history,
behind only Hillary Clinton, the 2016 nominee.

The party's left lane is now clearer for Mr. Sanders. His supporters and
other progressives have spent the last two days gingerly reaching out to
Ms. Warren's orbit and plotting in private conversations about how to
keep the two liberal standard-bearers aligned.

In January, Mr. Sanders and Ms. Warren clashed in a deeply personal way
after she confirmed a report that in a private meeting before the
campaign began,
\href{https://www.nytimes.com/2020/01/13/us/politics/bernie-sanders-elizabeth-warren-woman-president.html}{he
told her he believed that a woman could not win the White House in
2020}. During a debate, Mr. Sanders strongly denied having made the
remark, and Ms. Warren confronted him onstage afterward, accusing him of
calling her a ``liar.'' Relations have been chilly since.

In her call with Mr. Biden, Ms. Warren revealed so little of her
endorsement plans that a person familiar with the call remarked on her
``great poker face.''

Ms. Warren arrived on the political scene in the aftermath of the 2008
financial collapse and shot to stardom with her indictments of Wall
Street and unfettered capitalism.

In 2016, some progressive organizations mounted ``Run Warren Run''
campaigns and Mr. Sanders floated her as a possible challenger to Mrs.
Clinton, but Ms. Warren declined to run.

Joining the 2020 race, she found a changed political terrain. Mr.
Sanders's political stock had soared after his 2016 run, giving him an
immediate advantage in fund-raising and name recognition that
complicated Ms. Warren's electoral path.

Mr. Trump's election seemed to shock the Democratic base into an acute
focus on electability. Voters frequently second-guessed their electoral
choices as they tried to game out which candidate would be best equipped
to beat him.

Mr. Biden, in particular, has capitalized on this anxiety.

Ms. Warren's allies and supporters said the electability question ---
who would be the surest bet to defeat the president ---
disproportionately hurt female candidates after Mrs. Clinton's
unexpected loss in 2016.

``All they heard all along was what a risk the women were,'' said
Christina Reynolds, a vice president of Emily's List, a leading
Democratic women's group that endorsed Ms. Warren this week, only after
Senator Amy Klobuchar withdrew.

Ms. Reynolds said that evaluation was as wrong as it was widespread.
``The idea that that doesn't hang around the women's necks is crazy,''
she said.

Ms. Warren's campaign was slow to directly address questions of
electability, seeming to believe her rise in the polls last year spoke
for itself. But as the calendar turned to 2020, it was apparent that the
issue was hobbling her candidacy as precinct captains and volunteers
warned Ms. Warren that it was what they were hearing about from voters.

Ms. Warren's decline had begun in earnest at the October debate, when
she was pressed on how she would pay for Medicare for all~and had no
answer. It took weeks to detail her plan, but by then her perceived
trustworthiness seemed to have taken a hit: The candidate with a plan
for everything did not have one to finance the biggest issue of the
campaign. (Mr. Sanders, despite releasing fewer details on paying for
Medicare for all, has faced fewer questions.)

When she did roll out details, she was criticized by those on the left
for compromising too much and by centrists for the sheer size of the
plan. The episode captured a fundamental pain point for her candidacy:
She was too much of an insider for those demanding revolution, and too
much of an outsider for those who wanted to tinker with the system and
focus on beating Mr. Trump.

As the race intensified in the fall, Ms. Warren was reluctant to strike
back at her opponents, even as they undermined her image. Pete Buttigieg
made deep incursions into her support among educated white voters but
she did not call him out
\href{https://www.nytimes.com/2019/12/05/us/politics/elizabeth-warren-pete-buttigieg.html}{in
earnest until December}, even as he flooded the Iowa airwaves with a
moderate message undercutting her progressive platform.

Image

Before her exit, Ms. Warren accumulated the second-largest number of
Democratic delegates of any woman to run for president in history,
behind only Hillary Clinton, the 2016 nominee.Credit...Ruth Fremson/The
New York Times

While most campaigns used the megaphone of mass television ads to cut
through the media filter, Ms. Warren's braintrust was cool to the power
of commercials from the start, preferring on-the-ground and digital
organizing.

At times, Ms. Warren's campaign did not reflect the urgency of a
candidacy trying to make history and promote a program of systemic
upheaval that included government-run health care, free public college,
student debt cancellation, breaking up Big Tech, universal child care,
and tax increases on the wealthy.

But after weak finishes in Iowa and New Hampshire, Ms. Warren charged
into the February debate planning to confront Mr. Bloomberg in his first
appearance onstage. In Mr. Bloomberg, she found a rare rival she seemed
truly comfortable attacking, an embodiment of the influence of money.

She slashed. He stumbled. Mr. Bloomberg would never recover. Ms.
Warren's donations surged, but her vote count did not.

She would bend a principled stand that week as well,
\href{https://twitter.com/ShaneGoldmacher/status/1230633667195621376}{declining
to disavow} a new super PAC that would air nearly \$15 million in
pro-Warren advertising, saying she did not want to unilaterally disarm.
The irony was not lost on her opponents: The anti-big money candidate
wound up with the biggest super PAC in the race to date.

In recent days, Ms. Warren had taken to speaking to voters directly
about their electability fears, imploring them to tune out pundits.

``Cast a vote from your heart,'' she said Tuesday.

In speeches over the course of her campaign, Ms. Warren sought to
elevate the stories of women, often women of color. Her final major
address, in East Los Angeles on Monday, was devoted to Latina janitors
who organized for better working conditions.

Aimee Allison, the founder and president of She The People, a political
advocacy organization for women of color, praised Ms. Warren for her
campaign's intentional inclusivity. ``She really comes up as the first
white candidate for president who had an intersectional politics,'' she
said.

But Ms. Allison acknowledged that pitch
\href{https://www.nytimes.com/2020/02/28/us/politics/elizabeth-warren-black-vote.html}{did
not find favor in the broader minority electorate, even as it won
plaudits from academics and activists}.

``Black voters really were looking for a return to normalcy,'' she said.
``It was a rejection from what was perceived as riskier politics and a
broader and more courageous political vision.''

Ms. Warren's supporters were devoted to making the party more
progressive to the end. In Illinois, where Ms. Warren's campaign was
scheduled to hold a post-Super Tuesday phone banking session, staff and
supporters refused to cancel. They used their time to support Marie
Newman, the local challenger running against an incumbent Democrat
opposed to abortion rights.

``Our work continues,'' Ms. Warren told her staff in the call informing
them she was quitting the race. ``The fight goes on, and big dreams
never die.''

Astead W. Herndon reported from Cambridge, and Shane Goldmacher from New
York. Jonathan Martin contributed reporting from New York.

\hypertarget{our-2020-election-guide}{%
\section{Our 2020 Election Guide}\label{our-2020-election-guide}}

Updated July 31, 2020

\begin{itemize}
\item
  \begin{center}\rule{0.5\linewidth}{\linethickness}\end{center}

  \hypertarget{the-latest}{%
  \subsection{The Latest}\label{the-latest}}

  \begin{itemize}
  \tightlist
  \item
    President Trump's assault on the Postal Service is intersecting with
    his attacks on mail-in voting.
    \href{https://www.nytimes.com/2020/07/31/us/politics/trump-usps-mail-delays.html?action=click\&pgtype=Article\&state=default\&region=BELOW_MAIN_CONTENT\&context=storylines_guide}{Voting
    rights groups say it is a recipe for disaster.}
  \end{itemize}
\item
  \begin{center}\rule{0.5\linewidth}{\linethickness}\end{center}

  \hypertarget{bidens-vp-search}{%
  \subsection{Biden's V.P. Search}\label{bidens-vp-search}}

  \begin{itemize}
  \tightlist
  \item
    \href{https://www.nytimes.com/article/biden-vice-president-2020.html?action=click\&pgtype=Article\&state=default\&region=BELOW_MAIN_CONTENT\&context=storylines_guide}{Here
    are 13 women} who have been under consideration to be Joe Biden's
    running mate, and why each might be chosen --- and might not be.
  \end{itemize}
\item
  \begin{center}\rule{0.5\linewidth}{\linethickness}\end{center}

  \hypertarget{keep-up-with-our-coverage}{%
  \subsection{Keep Up With Our
  Coverage}\label{keep-up-with-our-coverage}}

  \begin{itemize}
  \tightlist
  \item
    Get an
    \href{https://www.nytimes.com/newsletters/politics?action=click\&pgtype=Article\&state=default\&region=BELOW_MAIN_CONTENT\&context=storylines_guide}{email}
    recapping the day's news
  \end{itemize}

  \begin{itemize}
  \tightlist
  \item
    Download our mobile app on
    \href{https://apps.apple.com/us/app/nytimes/id284862083?ls=1\&mat_click_id=5c79ae7455014fd1bd66b5610c05b8f2-20191112-16948\&referrer=mat_click_id\%3D5c79ae7455014fd1bd66b5610c05b8f2-20191112-16948\%26link_click_id\%3D722930677036718082}{iOS}
    and
    \href{http://a.localytics.com/android?id=com.nytimes.android\&referrer=utm_source\%3Dother_nyt_mobile_web\%26utm_medium\%3DWeb\%2520page\%26utm_term\%3DGeneral\%2520Mobile\%2520Page\%26utm_campaign\%3DNYT\%2520Mobile\%2520General\%2520Page}{Android}
    and turn on Breaking News and Politics alerts
  \end{itemize}
\end{itemize}

Advertisement

\protect\hyperlink{after-bottom}{Continue reading the main story}

\hypertarget{site-index}{%
\subsection{Site Index}\label{site-index}}

\hypertarget{site-information-navigation}{%
\subsection{Site Information
Navigation}\label{site-information-navigation}}

\begin{itemize}
\tightlist
\item
  \href{https://help.nytimes.com/hc/en-us/articles/115014792127-Copyright-notice}{©~2020~The
  New York Times Company}
\end{itemize}

\begin{itemize}
\tightlist
\item
  \href{https://www.nytco.com/}{NYTCo}
\item
  \href{https://help.nytimes.com/hc/en-us/articles/115015385887-Contact-Us}{Contact
  Us}
\item
  \href{https://www.nytco.com/careers/}{Work with us}
\item
  \href{https://nytmediakit.com/}{Advertise}
\item
  \href{http://www.tbrandstudio.com/}{T Brand Studio}
\item
  \href{https://www.nytimes.com/privacy/cookie-policy\#how-do-i-manage-trackers}{Your
  Ad Choices}
\item
  \href{https://www.nytimes.com/privacy}{Privacy}
\item
  \href{https://help.nytimes.com/hc/en-us/articles/115014893428-Terms-of-service}{Terms
  of Service}
\item
  \href{https://help.nytimes.com/hc/en-us/articles/115014893968-Terms-of-sale}{Terms
  of Sale}
\item
  \href{https://spiderbites.nytimes.com}{Site Map}
\item
  \href{https://help.nytimes.com/hc/en-us}{Help}
\item
  \href{https://www.nytimes.com/subscription?campaignId=37WXW}{Subscriptions}
\end{itemize}
