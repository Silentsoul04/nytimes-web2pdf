Sections

SEARCH

\protect\hyperlink{site-content}{Skip to
content}\protect\hyperlink{site-index}{Skip to site index}

\href{https://www.nytimes.com/section/world/asia}{Asia Pacific}

\href{https://myaccount.nytimes.com/auth/login?response_type=cookie\&client_id=vi}{}

\href{https://www.nytimes.com/section/todayspaper}{Today's Paper}

\href{/section/world/asia}{Asia Pacific}\textbar{}How Delhi's Police
Turned Against Muslims

\url{https://nyti.ms/2IFoxXY}

\begin{itemize}
\item
\item
\item
\item
\item
\item
\end{itemize}

Advertisement

\protect\hyperlink{after-top}{Continue reading the main story}

Supported by

\protect\hyperlink{after-sponsor}{Continue reading the main story}

\hypertarget{how-delhis-police-turned-against-muslims}{%
\section{How Delhi's Police Turned Against
Muslims}\label{how-delhis-police-turned-against-muslims}}

More evidence has emerged that the Indian police took part in violence
against Muslims or stood aside during fighting in the capital last
month.

\includegraphics{https://static01.nyt.com/images/2020/03/09/world/09india-police-sub1/merlin_170232576_4457d28f-c3f8-4f80-91ba-a8455ed199e0-articleLarge.jpg?quality=75\&auto=webp\&disable=upscale}

By \href{https://www.nytimes.com/by/jeffrey-gettleman}{Jeffrey
Gettleman}, \href{https://www.nytimes.com/by/sameer-yasir}{Sameer
Yasir}, \href{https://www.nytimes.com/by/suhasini-raj}{Suhasini Raj} and
\href{https://www.nytimes.com/by/hari-kumar}{Hari Kumar}

Photographs by Atul Loke

\begin{itemize}
\item
  March 12, 2020
\item
  \begin{itemize}
  \item
  \item
  \item
  \item
  \item
  \item
  \end{itemize}
\end{itemize}

NEW DELHI --- Kaushar Ali, a house painter, was trying to get home when
he ran into a battle.

Hindu and Muslim mobs were hurling rocks at each other, blocking a
street he needed to cross to get to his children. Mr. Ali, who is
Muslim, said that he turned to some police officers for help. That was
his mistake.

The officers threw him onto the ground, he said, and cracked him on the
head. They started beating him and several other Muslims. As the men lay
bleeding, begging for mercy --- one of them died two days later from
internal injuries --- the officers laughed, jabbed them with their
sticks and made them sing the national anthem. That abuse, on Feb. 24,
was
\href{https://www.altnews.in/video-verification-delhi-cops-beating-injured-men-forcing-them-to-sing-national-anthem/}{captured
on video}.

``The police were toying with us,'' Mr. Ali said. He recalled them
saying, ``Even if we kill you, nothing will happen to us.''

So far, they have been right.

India has suffered its worst sectarian bloodshed in years, in what many
here see as the inevitable result of
\href{https://www.nytimes.com/2020/03/01/world/asia/india-modi-hindus.html}{Hindu
extremism} that has flourished under the government of Prime Minister
Narendra Modi. His party has embraced a militant brand of Hindu
nationalism and its leaders have openly vilified Indian Muslims. In
recent months Mr. Modi has presided over a raft of policies widely seen
as anti-Muslim, such as erasing the statehood of what had been India's
only Muslim-majority state, Jammu and Kashmir.

Now, more evidence is emerging that the Delhi police, who are under the
direct command of Mr. Modi's government and have very few Muslim
officers, concertedly moved against Muslims and at times actively helped
the Hindu mobs that rampaged in New Delhi in late February, burning down
Muslim homes and targeting Muslim families.

\begin{itemize}
\tightlist
\item
  \href{https://scroll.in/video/955020/the-police-told-us-to-throw-stones-bbc-meets-delhi-residents-who-took-part-in-communal-violence}{Several
  videos showed police officers beating and throwing rocks at Muslim
  protesters} and waving on Hindu mobs to join them.
\end{itemize}

\begin{itemize}
\tightlist
\item
  A police commander said that as the violence erupted --- at that point
  mostly by Hindu mobs --- officers in the affected areas were ordered
  to deposit their guns at the station houses. Several officers during
  the violence were later overheard by New York Times journalists
  yelling to one another that they had only sticks and that they needed
  guns to confront the growing mobs. Some researchers accuse the police
  force of deliberately putting too few officers on the streets, with
  inadequate firepower, as the violence morphed from clashes between
  rival protesters into targeted killings of Muslims.
\end{itemize}

\begin{itemize}
\item
  Two thirds of the more than 50 people who were killed and have been
  identified were Muslim. Human rights activists are calling it an
  organized massacre.
\item
  Though India's population is 14 percent Muslim and New Delhi's is 13
  percent, the total Muslim representation on the Delhi police force is
  less than 2 percent.
\end{itemize}

\includegraphics{https://static01.nyt.com/images/2020/03/09/world/09india-police-2sub/merlin_170104365_120bfef2-f0c9-4d5e-9d44-2cc1973cffff-articleLarge.jpg?quality=75\&auto=webp\&disable=upscale}

India's policing culture has long been brutal, biased, anti-minority and
almost colonial in character, a holdover from the days of British rule
when the police had no illusions of serving the public but were used to
suppress a restive population.

But what seems to be different now, observers contend, is how profoundly
India's law enforcement machinery has been politicized by the Bharatiya
Janata Party, Prime Minister Narendra Modi's Hindu-nationalist governing
bloc.

Police officials, especially in states controlled by Mr. Modi's party,
have been highly selective in their targets, like
\href{https://scroll.in/article/951915/karnatakas-sedition-case-against-parent-and-teacher-for-a-school-play-is-absurd-and-illegal}{a
Muslim school principal in Karnataka who was jailed for more than two
weeks on sedition charges} after her students performed a play about a
new immigration law that police officials said was critical of Mr. Modi.

Some judges have also seemed to be caught up --- or pushed out --- by a
Hindu-nationalist wave.

A
\href{https://www.nytimes.com/2020/02/26/world/asia/india-hindu-muslim-violence-modi.html}{Delhi
judge who expressed disbelief} that the police had yet to investigate
members of Mr. Modi's party who have been
\href{https://www.nytimes.com/2020/02/26/world/asia/delhi-riots-kapil-mishra.html}{widely
accused of instigating} the recent violence in Delhi was taken off the
case and transferred to another state. And at the same time that the
Supreme Court has made a string of rulings in the government's favor,
one of the judges, Arun Mishra, publicly praised
\href{https://in.video.search.yahoo.com/yhs/search?fr=yhs-trp-001\&hsimp=yhs-001\&hspart=trp\&p=video+of+justice+arun+mishra+parising+narendra+modi\#id=51\&vid=86959a2615bb2d949c65c611e8811221\&action=click}{Mr.
Modi as a ``visionary genius.''}

All of this is emboldening Hindu extremists on the street.

The religiously mixed and extremely crowded neighborhoods in
northeastern Delhi that were on fire in late February have cooled. But
some Hindu politicians continue to lead so-called peace marches,
trotting out casualties of the violence with their heads wrapped in
white medical tape, trying to upend the narrative and make Hindus seem
like the victims, which is stoking more anti-Muslim hatred.

Some Muslims are leaving their neighborhoods, having lost all faith in
the police. More than 1,000 have piled into a camp for internally
displaced people that is rising on Delhi's outskirts.

Image

A relief camp in New Mustafabad in northeastern Delhi, on Thursday.

Muslim leaders see the violence as a state-sanctioned campaign to teach
them a lesson. After years of staying quiet as
\href{https://www.hrw.org/report/2019/02/18/violent-cow-protection-india/vigilante-groups-attack-minorities}{Hindu
lynch mobs killed Muslims with impunity} and Mr. Modi's government
chipped away at their political power, India's Muslim population awoke
in December and poured into the streets,
\href{https://www.nytimes.com/2020/01/17/world/asia/india-protests-aishe-ghosh.html}{along
with many other Indians}, to protest the new immigration law, which
favors migrants belonging to every major religion in South Asia ---
\href{https://www.nytimes.com/2019/12/09/world/asia/india-muslims-citizenship-narendra-modi.html}{except
for Muslims.}

Mr. Modi's government, Muslim leaders say, is now trying to drive the
whole community back into silence.

``There's a method to this madness,'' said Umar Khalid, a Muslim
activist. ``The government wants to bring the entire Muslim community to
their knees, to beg for their lives and beg for their livelihoods.''

``You can read it in their books," he said, referring to foundational
texts by Hindu nationalists. ``They believe India's Muslims should live
in perpetual fear.''

Mr. Modi has said little since the bloodshed erupted, except for a
\href{https://twitter.com/narendramodi/status/1232581653916155912?lang=en}{few
anodyne tweets} urging peace. Delhi police officials deny an anti-Muslim
bias and said they ``acted swiftly to control law and order,'' which
both Muslims and Hindus in those neighborhoods have said was not true.
The police responded ``without favoring any person on religious lines or
otherwise,'' according to a written reply to questions, provided by M.S.
Randhawa, a police spokesman.

Police officials said that Mr. Ali and the other Muslim men were hurt by
protesters and rescued by the police --- though videos clearly show them
being hit by police officers. Police officials also pointed out that one
officer was killed and more than 80 injured;
\href{https://www.facebook.com/520830984697574/posts/2861380863975896/?vh=e\&d=n}{videos
show a huge crowd of Muslim protesters attacking outnumbered officers}.

The violence in New Delhi fits a pattern, experts say, of chaos being
allowed to rage for a few days --- with minorities being killed ---
before the government brings it under control.

Image

Police patrolling in Mustafabad on Feb. 28.

In 1984, under the Congress party, which often bills itself as
representing the interests of minorities, the police in New Delhi stood
back for several days as mobs massacred 3,000 Sikhs.

In 1993, again under a Congress government, riots swept Mumbai and
hundreds of Muslims were killed.

In 2002 in Gujarat, when Mr. Modi was the state's chief minister,
\href{https://www.nytimes.com/2002/07/27/world/religious-riots-loom-over-indian-politics.html}{Hindu
mobs massacred hundreds of Muslims}. Mr. Modi was accused of complicity,
though he was cleared by a court.

Several retired Indian police commanders said that the rule in quelling
communal violence was to deploy maximum force and make many arrests,
neither of which happened in Delhi.

Ajai Raj Sharma, a former commissioner, called the performance
``unexplainable.'' ``It can't be forgiven,'' he said.

When the violence started on Feb. 23 --- as Hindu men gathered to
forcibly eject a peaceful Muslim protest near their neighborhood ---
much of it became two-sided. By day's end, both Muslims and Hindus had
been attacked, and dozens had been shot, apparently with small-bore
homemade guns.

But by Feb. 25 the direction had changed. Hindu mobs fanned out and
targeted Muslim families. Violence crackled in the air.

Police officers watched as mobs of Hindus, their foreheads marked by
saffron stripes, prowled the streets with baseball bats and rusty bars,
looking for Muslims to kill. The sky was filled with smoke. Muslim
homes, shops and mosques were burned down.

When a reporter for The New York Times tried to speak to residents
standing near police officers that day, a mob of men with darting eyes
surrounded him and ripped the notebook out of his hands. When the
reporter asked police officers for help, one said: ``I can't. These
young men are very volatile.''

The home ministry, which controls Delhi's police force and is led by
Amit Shah, one of the most combative Hindu nationalists in the B.J.P.,
has come under heavy criticism for the policing failures. Delhi police
officials denied being instructed by the central government to go easy
on the troublemakers. The home ministry did not respond to repeated
requests for comment.

On Thursday,
\href{https://twitter.com/bjp4india/status/1238089841469874178?s=21}{during
a debate in parliament}, Mr. Shah vowed to bring the culprits to
justice, ``regardless of their caste, religion or political
affiliations.'' He has defended the police and called the violence a
conspiracy, saying investigators found links to the Islamic State. Many
observers question how much, if at all, the Islamic State had anything
to do with what unfolded.

And then there's the composition of the police. The Delhi force,
numbering around 80,000, has fewer than 2,000 Muslim officers and just a
handful of Muslim commanders, according to an
\href{https://indianexpress.com/article/opinion/columns/muslims-in-delhi-police-jobs-delhi-police-recruitment-crime-in-delhi-national-crime-records-bureau-4936111/lite/}{analysis
done in 2017} by the Commonwealth Human Rights Initiative. Delhi police
officials did not deny this, and Muslim leaders said that police
behavior was biased across India.

``Indian police are extremely colonial and caste-ist,'' said Shahid
Siddiqui, a former member of Parliament. Police behavior, he said, is
always ``more violent and aggressive toward the weak.''

India's population is about 80 percent Hindu, and gangs of Hindus
threatened Muslims in several Delhi neighborhoods to leave before the
Hindu holiday Holi that was celebrated this week.

One Muslim woman, who goes by the name Baby, opened her door a few days
ago to find 50 men outside with a notebook in their hands, listing the
addresses of Muslims. She packed up. She may be leaving soon.

``O, Allah, why didn't you make me a Hindu?'' she said, her voice
quavering. ``Is it my fault that I was born a Muslim?''

Image

Friday prayer at a mosque in Mustafabad on Feb. 28.~

Shalini Venugopal contributed reporting.

Advertisement

\protect\hyperlink{after-bottom}{Continue reading the main story}

\hypertarget{site-index}{%
\subsection{Site Index}\label{site-index}}

\hypertarget{site-information-navigation}{%
\subsection{Site Information
Navigation}\label{site-information-navigation}}

\begin{itemize}
\tightlist
\item
  \href{https://help.nytimes.com/hc/en-us/articles/115014792127-Copyright-notice}{©~2020~The
  New York Times Company}
\end{itemize}

\begin{itemize}
\tightlist
\item
  \href{https://www.nytco.com/}{NYTCo}
\item
  \href{https://help.nytimes.com/hc/en-us/articles/115015385887-Contact-Us}{Contact
  Us}
\item
  \href{https://www.nytco.com/careers/}{Work with us}
\item
  \href{https://nytmediakit.com/}{Advertise}
\item
  \href{http://www.tbrandstudio.com/}{T Brand Studio}
\item
  \href{https://www.nytimes.com/privacy/cookie-policy\#how-do-i-manage-trackers}{Your
  Ad Choices}
\item
  \href{https://www.nytimes.com/privacy}{Privacy}
\item
  \href{https://help.nytimes.com/hc/en-us/articles/115014893428-Terms-of-service}{Terms
  of Service}
\item
  \href{https://help.nytimes.com/hc/en-us/articles/115014893968-Terms-of-sale}{Terms
  of Sale}
\item
  \href{https://spiderbites.nytimes.com}{Site Map}
\item
  \href{https://help.nytimes.com/hc/en-us}{Help}
\item
  \href{https://www.nytimes.com/subscription?campaignId=37WXW}{Subscriptions}
\end{itemize}
