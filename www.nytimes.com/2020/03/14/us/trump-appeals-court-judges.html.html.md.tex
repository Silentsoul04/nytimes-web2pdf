Sections

SEARCH

\protect\hyperlink{site-content}{Skip to
content}\protect\hyperlink{site-index}{Skip to site index}

\href{/section/us}{U.S.}\textbar{}A Conservative Agenda Unleashed on the
Federal Courts

\url{https://nyti.ms/3aOHDXA}

\begin{itemize}
\item
\item
\item
\item
\item
\item
\end{itemize}

\includegraphics{https://static01.nyt.com/images/2020/03/13/multimedia/00Fed-judges-new/00Fed-judges-new-mobileMasterAt3x-v6.jpg}\includegraphics{https://static01.nyt.com/images/2020/03/13/multimedia/00Fed-judges-new-02/00Fed-judges-new-02-mobileMasterAt3x-v7.jpg}\includegraphics{https://static01.nyt.com/images/2020/03/13/multimedia/00Fed-judges-new-03/00Fed-judges-new-03-mobileMasterAt3x-v6.jpg}\includegraphics{https://static01.nyt.com/images/2020/03/13/multimedia/00Fed-judges-new-04/00Fed-judges-new-04-mobileMasterAt3x-v7.jpg}\includegraphics{https://static01.nyt.com/images/2020/03/13/multimedia/00Fed-judges-new-05/00Fed-judges-new-05-mobileMasterAt3x-v6.jpg}

President Trump has appointed judges to the federal appeals courts at a
record-setting pace.

The Trump appointees are far less diverse than Mr. Obama's, with
two-thirds of them white men.

The new judges have been selected for their rock-solid conservative
credentials, including at least seven that had previous jobs with Mr.
Trump's campaign or his administration.

All but eight had ties to the Federalist Society, a legal group with
views once considered on ``the fringe.''

Now, as he seeks a second term, Mr. Trump can boast of having named more
than a quarter of all judges on the appeals courts, 51 to date.

Supported by

\protect\hyperlink{after-sponsor}{Continue reading the main story}

\hypertarget{a-conservative-agenda-unleashed-on-the-federal-courts}{%
\section{A Conservative Agenda Unleashed on the Federal
Courts}\label{a-conservative-agenda-unleashed-on-the-federal-courts}}

President Trump's imprint on the nation's appeals courts has been swift
and historic. He has named judges with records on a range of issues
important to Republicans --- and to his re-election.

By \href{https://www.nytimes.com/by/rebecca-r-ruiz}{Rebecca R. Ruiz},
\href{https://www.nytimes.com/by/robert-gebeloff}{Robert Gebeloff},
\href{https://www.nytimes.com/by/steve-eder}{Steve Eder} and
\href{https://www.nytimes.com/by/ben-protess}{Ben Protess}

\begin{itemize}
\item
  Published March 14, 2020Updated March 16, 2020
\item
  \begin{itemize}
  \item
  \item
  \item
  \item
  \item
  \item
  \end{itemize}
\end{itemize}

As a Republican candidate for the Texas Supreme Court, Don R. Willett
flaunted his uncompromising conservatism, boasting of endorsements from
groups with ``pro-life, pro-faith, pro-family'' credentials.

``I intend to build such a fiercely conservative record on the court
that I will be unconfirmable for any future federal judicial post ---
and proudly so,'' a Republican rival quoted him telling party leaders.

Judge Willett served a dozen years on the Texas bench. But rather than
disqualifying him, his record there propelled him to the very job he had
deemed beyond reach. President Trump nominated him to a federal appeals
court, and Republicans in the Senate narrowly confirmed him on a
party-line vote.

As Mr. Trump seeks re-election, his rightward
\href{https://www.nytimes.com/2020/06/29/us/politics/supreme-court-trump-biden.html}{overhaul
of the federal judiciary} --- in particular, the highly influential
appeals courts --- has been invoked as one of his most enduring
accomplishments. While individual nominees have drawn scrutiny, The New
York Times conducted a deep examination of all 51 new appellate judges
to obtain a collective portrait of the Trump-populated bench.

The review shows that the Trump class of appellate judges, much like the
president himself, breaks significantly with the norms set by his
Democratic and Republican predecessors, Barack Obama and George W. Bush.

The lifetime appointees --- who make up more than a quarter of the
entire appellate bench --- were more openly engaged in causes important
to Republicans, such as opposition to gay marriage and to government
funding for abortion.

They more typically held a political post in the federal government and
donated money to political candidates and causes. Just four had no
discernible political activity in their past, and several were confirmed
in spite of an unfavorable rating from the American Bar Association ---
the first time that had happened at the appellate level in decades.

Two-thirds are white men, and as a group, they are much younger than the
Obama and Bush appointees.

Once on the bench, the Trump appointees have stood out from their fellow
judges, according to an analysis by The Times of more than 10,000
published decisions and dissents through December.

When ruling on cases, they have been notably more likely than other
Republican appointees to disagree with peers selected by Democratic
presidents, and more likely to agree with those Republican appointees,
suggesting they are more consistently conservative. Among the dozen or
so judges that most fit the pattern, The Times found, are three Mr.
Trump has signaled were on his Supreme Court shortlist.

While the appellate courts favor consensus and disagreement remains
relatively rare --- there were 125 instances when a Trump appointee
wrote the majority opinion or dissent in a split decision --- the new
judges have ruled on disputed cases across a range of contentious
issues, including abortion, immigration, L.G.B.T. rights and lobbying
requirements, the examination shows.

One new judge, who had held a political job in the Trump administration,
dissented on an issue of particular importance to the president:
disclosure of his financial records. The judge, Neomi Rao, opposed a
decision requiring the release of the documents to a congressional
committee, a mandate the president continues to resist and is now before
the Supreme Court.

``They have long records of standing up, and they're not afraid of being
unpopular,'' said Carrie Severino, president of the Judicial Crisis
Network, a conservative advocacy group that has pushed for the
mold-breaking appointments. Ms. Severino once served as a law clerk for
Justice Clarence Thomas, one of the Supreme Court's most reliably
conservative members.

Stephen B. Burbank, a law professor at the University of Pennsylvania,
said Mr. Trump's appointments reflected attempts by recent presidents to
draw the federal judiciary --- a constitutionally independent branch of
government --- into policy debates more appropriate in Congress and the
White House.

``The problem as I see it is not that judges differ ideologically --- of
course they do --- nor is it that a Republican president would look for
someone with congenial ideological preferences,'' Mr. Burbank said.
``It's that in recent decades the search has been for hard-wired
ideologues because they're reliable policy agents.''

Mr. Trump has appointed more judges to the appeals courts, where eight
of the nine current Supreme Court Justices served, than any other
president during the first three years in office. Also known as
circuits, the 13 courts are the
\href{https://www.uscourts.gov/statistics-reports/appellate-courts-and-cases-journalists-guide}{last
stop for federal cases before the Supreme Court,} and nearly all federal
litigation ends there.

The Times examination was based on interviews with dozens of people
close to the nomination process, including some of Mr. Trump's
appointees; the analysis of thousands of published decisions and
dissents since Mr. Trump became president; a review of detailed
biographical and financial questionnaires submitted by all 168 appellate
judges named by Mr. Trump, Mr. Obama and Mr. Bush, as well as their
records, public statements and campaign contributions since 1989.

Judicial appointments, a standard measure of a president's legacy,
almost always draw partisan scrutiny, as Republicans tend to appoint
conservative lawyers who interpret the Constitution according to what
they say was its original meaning, and Democrats lean toward liberal
appointees with a more expansive view. But Mr. Trump's record is
particularly striking because of the divisive atmosphere, the
examination shows, and the president's disruptive approach to governing.
The White House did not respond to requests for comment, and none of the
judges contacted by The Times would agree to be quoted.

\includegraphics{https://static01.nyt.com/images/2020/03/15/us/politics/15trump-judiciary-oak-01/merlin_170299731_2d1b5f19-1642-4b9c-a8ef-5b97d1b336bc-articleLarge.jpg?quality=75\&auto=webp\&disable=upscale}

When Mr. Trump took office there were
\href{https://www.brookings.edu/blog/fixgov/2020/01/28/judicial-appointments-in-trumps-first-three-years-myths-and-realities/}{103
unfilled federal court openings}, in addition to a Supreme Court seat,
in part because
\href{https://www.brookings.edu/blog/fixgov/2018/06/04/senate-obstructionism-handed-judicial-vacancies-to-trump/}{Senator
Mitch McConnell of Kentucky}, the Republican leader of the Senate, and
allies had refused to proceed with confirming many of Mr. Obama's
nominees. The last time so many vacancies had been left to a successor
of the opposing party was when the federal bench was expanded by dozens
of judges under President George H.W. Bush.

Mr. Trump wasted no time in seizing the opportunity. During his first
three years in office, with Mr. McConnell's assistance, he was able to
name nearly as many appellate judges as Mr. Obama had appointed over two
terms.

And he did so with great political flourish. More than one-third of the
Trump appointees have filled seats previously occupied by judges
appointed by Democrats, tipping the balance toward conservatives in some
circuits that include largely Democratic states like New York and
Connecticut. Even in the San Francisco-based Ninth Circuit, a reliably
liberal appeals court, Mr. Trump has significantly narrowed the gap
between judges appointed by Democratic and Republican presidents.

With Republicans and Democrats in Congress retreating to their corners,
many of the Trump appointees have benefited from Republicans' decision
to extend a contentious and partisan confirmation path that upended
bipartisan Senate practices.

Two-thirds of the new appellate judges failed to win the support of 60
senators, historically a requirement of consensus that was first
\href{https://www.nytimes.com/2013/11/22/us/politics/reid-sets-in-motion-steps-to-limit-use-of-filibuster.html}{jettisoned
by the Democratic-controlled Senate} midway through the Obama
administration because Republicans were blocking nominees to the D.C.
Circuit. After he became majority leader, Mr. McConnell followed suit
when Democrats initially blocked Mr. Trump's first Supreme Court
nominee, Neil Gorsuch.

About a third did not receive the signoff of both home-state senators, a
courtesy for a nomination to move forward that was tossed aside in late
2017 by Senator Charles E. Grassley of Iowa, then the Judiciary
Committee's Republican chairman. Senator Lindsey Graham of South
Carolina, Mr. Grassley's successor in that role, carried the decision
forward. Crucially, that meant Mr. Trump did not have to compromise on
his appellate picks in states with a Democratic senator.

Just two found unanimous support across the aisle, a sharp drop from
both the Obama and Bush nominees.

According to
\href{https://www.heritage.org/courts/commentary/filling-the-judicial-confirmation-stocking}{a
tally by the Heritage Foundation}, a conservative policy group, Mr.
Trump's appointees across the judiciary have drawn three times more
``no'' votes in the Senate than all confirmed judges in the 20th century
combined. So far, Mr. Trump has appointed more than 185 federal judges.

On the appellate bench, Mr. Trump's appointees have drawn nearly twice
as many ``no'' votes as did those of Mr. Bush and Mr. Obama, The Times's
analysis shows.

In a history-making intervention, one of Mr. Trump's appellate picks was
confirmed only when Vice President Mike Pence broke a 50-50 deadlock. It
was Mr. Pence's 12th tiebreaking vote in the Senate, the most of anyone
in his office since the 1870s, and the only time a vice president
installed a nominee to the bench.

The judge, Jonathan A. Kobes, had been working on Capitol Hill as an
aide to a Republican senator. He was rated unqualified by the American
Bar Association, which
\href{https://www.americanbar.org/content/dam/aba/uncategorized/GAO/2018-9-14ChairtoGrassleyFeinstein-statementon-JonathanKobesnominee-EighthCCA.pdf}{questioned
his ability} to reflect ``complex legal analysis'' and ``knowledge of
the law'' in his writing.

He got the job anyway, with Mr. Grassley
\href{https://twitter.com/chuckgrassley/status/1072638424664952835?lang=en}{proclaiming
on Twitter}in December 2018 that the confirmation had ``made HISTORY.''
Judge Kobes became Mr. Trump's 30th confirmed appointee to the appellate
bench.

Democrats, powerless to block the nominees, have been sidelined as angry
bystanders.

Senator Dianne Feinstein of California, the ranking Democrat on the
Judiciary Committee, called Mr. Trump's appellate appointees ``far
outside the judicial mainstream,'' adding that she believed Republicans
were using them to advance ``a particular agenda.'' She voted against
all but 14 of the appellate nominees.

''Americans are certainly aware of Supreme Court nominations,'' Ms.
Feinstein said in a statement to The Times, ``but most don't pay close
attention to the lower courts, which can have an even more direct effect
on their lives.''

\hypertarget{from-fringe-to-mainstream}{%
\subsection{From Fringe to Mainstream}\label{from-fringe-to-mainstream}}

Mr. Trump has staked his presidency on upending conventions, and his
approach to the judiciary breaks sharply with that of past presidents.

He unapologetically views judges as agents of the presidents who
appointed them ---
\href{https://www.nytimes.com/2018/11/20/us/politics/trump-appeals-court-ninth-circuit.html}{calling
out an ``Obama judge},'' for instance, for ruling against the Trump
administration in an immigration case. He frequently attributes his
popularity among Republicans to his judicial appointments. And he has
not been shy about politicizing the process.

``95\% Approval Rating in the Republican Party,'' he wrote
\href{https://twitter.com/realdonaldtrump/status/1221469014427348992?lang=en}{on
Twitter in January}. ``Thank you! 191 Federal Judges (a record), and two
Supreme Court Justices, approved. Best Economy \& Employment Numbers
EVER. Thank you to our great New, Smart and Nimble REPUBLICAN PARTY.
Join now, it's where people want to be!''

In his State of the Union address in February, he bragged about his
judicial appointments, promising, ``We have many in the pipeline.'' A
week later, the Senate approved his 51st nominee to the appeals bench;
41 others now await votes for the lower courts.

While federal judges of all stripes take an oath of impartiality and
reject the notion that they do a president's bidding --- Chief Justice
John G. Roberts Jr. recently described an independent judiciary as
\href{https://www.supremecourt.gov/publicinfo/year-end/2019year-endreport.pdf}{``a
key source of national unity and stability''} --- the examination by The
Times shows that the Trump administration has filled the appellate
courts with formidable allies who fought for a range of issues important
to Republicans.

Democratic presidents have also sought out reliable political allies
when filling some judicial posts, but Mr. Trump's approach has left
little to chance.

His appointees include former litigators who argued against legalizing
same-sex marriage; advocated blocking Medicaid reimbursements to health
care providers performing abortions; argued that corporations with
religious owners could not be required to pay for insurance coverage of
certain forms of birth control; and supported the Trump administration's
choice to include a question about citizenship on the census.

In the past, many conservatives have been left disappointed when judges
appointed by Republican presidents were seen to have lost their resolve
on the bench. Now what matters most with Mr. Trump's appointees, said
Josh Blackman, a professor at the South Texas College of Law Houston, is
that they come with rock-solid conservative résumés.

``You have to demand a paper trail --- no more skeleton nominees,'' said
Mr. Blackman, who advised the presidential campaign of Senator Ted Cruz
of Texas, a Republican, and is a strong supporter of the Trump approach.

One standout appointee, Kyle Duncan, now an appellate judge in New
Orleans, fought to uphold Louisiana's gay-marriage ban before the
Supreme Court, defended a North Carolina law restricting transgender
people from using their preferred bathrooms and represented Hobby Lobby
when it sued the federal government over the requirement that it provide
employees with insurance coverage for some birth control.

Image

Lawyers for Hobby Lobby outside the Supreme Court in 2014.Credit...Doug
Mills/The New York Times

He had worked as general counsel of the
\href{https://www.becketlaw.org/about-us/history/}{Becket Fund for
Religious Liberty,} a legal advocacy group that has been a
\href{https://www.washingtonpost.com/politics/becket-fund-law-firm-gaining-a-reputation-as-powerhouse-after-hobby-lobby-win/2014/07/20/c28931a4-104c-11e4-8936-26932bcfd6ed_story.html}{strong
defender} of the religious right.

Responding to questions from senators about the North Carolina law, he
said he had been ``advancing not my own personal beliefs but legal
arguments on behalf of my client's interests, just as I have done in
every case to the best of my ability.''

Judge Stephanos Bibas, a Trump appointee to the federal appeals court in
Philadelphia, last September
\href{https://www.wsj.com/articles/judges-say-they-arent-extensions-of-presidents-who-appointed-them-11568566598}{emphasized
the independence of judges} once they took the bench, saying, ``We
certainly are not viewing ourselves as members of teams or camps or
parties.''

Many of the appointees have elite credentials, with nearly half having
trained as lawyers at Harvard, Stanford, the University of Chicago or
Yale, and more than a third having clerked for a Supreme Court justice,
surpassing the appointees of both Mr. Obama and Mr. Bush.

But Mr. Trump's appellate picks often have less judicial experience, The
Times found. About 40 percent previously served as a judge, compared
with more than half of the Bush and Obama appointees.

Mr. Trump named some of his judges before they received a rating from
the American Bar Association, which Republicans have long viewed as
biased against their nominees. Three deemed unqualified were confirmed
--- a step not taken at the appellate level since at least 1975, when a
former governor of Connecticut
\href{https://timesmachine.nytimes.com/timesmachine/1975/03/22/92190255.html?pageNumber=20}{nominated}
by former Presidents Richard Nixon and Gerald Ford joined the bench,
according to Sheldon Goldman, a political scientist focusing on the
judiciary.

Mr. Trump is betting that the judges will back Republican priorities for
a long time: The median age of the appointees is five-and-a-half years
younger than it was under Mr. Obama, and three-and-a-half years younger
than under Mr. Bush. Thirty-three percent were under 45 when appointed,
compared with just five percent under Mr. Obama and 19 percent under Mr.
Bush. And countering a trend of increasing diversity on the appellate
bench under Mr. Obama, two-thirds of Mr. Trump's appointees are white
men.

They are also well off: Their median net worth is nearly \$2 million ---
adjusted for inflation, that is on a par with the worth of Obama
appointees, and about a half-million dollars more than that of Bush
appointees.

Perhaps most telling, all but eight of the new judges have had ties to
the Federalist Society, a legal group that has been central to the White
House's appointment process and ascendant in Republican circles in
recent years for its advocacy of strictly interpreting the Constitution.

Nearly twice as many appointees have had ties to the group as did those
of Mr. Trump's most recent Republican predecessor, Mr. Bush. Early this
year, a proposal was circulated among federal judges by the court
system's ethical advisory arm that would ban membership in the group.

Image

Donald F. McGahn II, a former White House counsel, at the Federalist
Society's Antonin Scalia Memorial Dinner in November.Credit...T.J.
Kirkpatrick for The New York Times

The Trump appointees turned out in big numbers at its national
convention in Washington in November. Many participated in panel
discussions and attended a black-tie dinner, where Donald F. McGahn II,
a former White House counsel for Mr. Trump, extolled the group's
extraordinary trajectory.

``We have seen our views go from the fringe, views that in years past
would inhibit someone's chances to be considered for the federal
bench,'' he said, ``to being the center of the conversation.''

\hypertarget{battle-tested-conservatives}{%
\subsection{Battle-Tested
Conservatives}\label{battle-tested-conservatives}}

Judge Willett, the former Texas Supreme Court justice, had a paper trail
replete with political connections and ties to prominent Texas
Republicans when he was nominated to the federal bench in 2017.

He had raised over \$4 million for two campaigns for the state bench,
more than half of it from lawyers, lobbyists and oil interests,
according to the National Institute on Money in Politics. He also had
more than 25,000 posts on Twitter that often focused on current affairs
and Republican politics, even some jabbing Mr. Trump as a candidate.

During the last Republican administration, under Mr. Bush, he had
advised judicial nominees ``to bob and weave, be the teeniest tiniest
target you can be,'' Judge Willett said during a speech in 2010, adding,
``You want to be as bland, forgettable and unremarkable as possible.''

No more. The Trump approach has translated into a new breed of appellate
appointees with open experience in ideological and political warfare.

John Malcolm, a conservative legal scholar at the Heritage Foundation,
said he was looking for ``people who have the strength of their
convictions.'' He
\href{https://www.heritage.org/crime-and-justice/impact/heritage-expert-helps-shape-supreme-court-nominee-list}{drew
up a list} in 2016 of recommended Supreme Court nominees that was
embraced by Mr. Trump.

About three-quarters of Mr. Trump's appellate appointees donated to
political candidates and causes, a significantly higher proportion than
Mr. Obama's and slightly ahead of Mr. Bush's, according to an analysis
of data from the National Institute on Money in Politics and the Center
for Responsive Politics.

They were also more likely than the Obama and Bush nominees to have been
affiliated with an election campaign in the decade before their
appointment.

Judge Duncan, previously a renowned conservative litigator, had
volunteered for the 2016 presidential campaign of Senator Marco Rubio of
Florida and was a donor and poll watcher for Mitt Romney's 2012
presidential bid.

At least seven of his fellow appointees had ties to the Trump
administration itself. Judge Rao had run a regulatory office in the
White House. Judges Steven J. Menashi and Gregory G. Katsas had worked
in the office of the White House counsel. Judge Patrick J. Bumatay had
been a counselor to the attorney general, while Judge Lawrence J.C.
VanDyke had been tapped for the **** Justice Department's Environment
and Natural Resources Division**.** Judges Katsas and Andrew L. Brasher
had volunteered for Mr. Trump's transition team. And Judge Chad A.
Readler had done legal work for Mr. Trump's 2016 presidential campaign.

He later became acting head of the Justice Department's civil division,
putting him in charge of defending nearly every high-profile
presidential policy that came under attack in the courts.

Judge L. Steven Grasz, in the year he was nominated to the appeals court
in Nebraska, had sat on the board of the anti-abortion Nebraska Family
Alliance and served as assistant secretary of Nebraskans for the Death
Penalty.

Other appointees had held state jobs that showcased their conservative
--- and sometimes partisan --- credentials.

Nearly a quarter of them worked in the office of a Republican state
attorney general. That was about triple the percentage of Bush nominees,
The Times found. Mostly, they served as solicitors general or their
deputies, putting them on the front lines in court battles over
contentious state laws.

At least eight actively fought against legalizing gay marriage, and at
least as many argued for immigration positions now embraced by the Trump
administration. At least 18 sought to limit access to abortion or
contraception.

Some nominees amassed their conservative credentials by filing
friend-of-the-court briefs, weighing in on cases especially important to
Republicans.

Judge VanDyke, appointed to the Ninth Circuit in Nevada, had been
prolific. As solicitor general of Montana, according to published
emails, he encouraged the state's attorney general to support a 20-week
abortion ban in Arizona, to defend a professional photographer's refusal
to shoot a same-sex commitment ceremony in New Mexico and to challenge a
ban on semiautomatic weapons in New York.

The fact that Montana was not directly affected by the cases did not
matter. In an email to the solicitor general in Alabama --- who would
also be named to the appellate bench by Mr. Trump --- he wrote about the
New York ban: ``Semiautomatic firearms are fun to hunt elk with, as the
attached picture attests :).'' The Great Falls Tribune,
\href{https://www.greatfallstribune.com/story/news/local/2014/09/17/vandyke-politician-nature/15812491/}{which
obtained the emails}, published a photo of the now-judge in hunting
garb.

Judge Michael H. Park, another appointee, had come to the support of the
Trump administration in its
\href{https://www.nytimes.com/2019/06/27/us/politics/census-citizenship-question-supreme-court.html}{unsuccessful
effort} to add a citizenship question to the census. As a private lawyer
representing the Project on Fair Representation, a conservative group,
he argued in 2018 that the question was justified, calling it
``immensely helpful to redistricting and voting rights litigation.''

The Supreme Court disagreed. Four months after he weighed in, it was
announced that Mr. Trump intended to nominate him to the U.S. Court of
Appeals for the Second Circuit in New York.

\hypertarget{judicial-courage}{%
\subsection{`Judicial Courage'}\label{judicial-courage}}

Some of Mr. Trump's choices for the appeals courts had already landed on
his short list for Supreme Court nominations. Once on the bench, they
quickly confirmed that they could shake things up.

Image

Judge Amul Thapar, who has been on Mr. Trump's Supreme Court short list,
has shown a penchant for disagreeing with Democratic appointees on the
appellate bench.Credit...Gabriella Demczuk for The New York Times

Three on that short list --- Judges Joan Larsen, David R. Stras and Amul
Thapar --- were among those identified by The Times as having a penchant
for disagreeing with Democratic nominees. They voted differently from
those judges 23 percent of the time, but from judges appointed by
Republican presidents only four percent of the time.

Judge Larsen and Judge Stras had been state supreme court justices named
by Republican governors in their home states. Judge Thapar was elevated
by Mr. Trump from a federal district court in Kentucky, where he had
been appointed by Mr. Bush.

Unlike lower courts, the appellate courts, which review other courts'
decisions, do not have juries. Instead, cases are largely decided by
panels of three judges, usually selected randomly from all of the judges
in the circuit.

There is a culture of consensus in most circuits, and in the cases
reviewed by The Times, appellate judges of both parties agreed with one
another the vast majority of times. But when they did not, the Trump
appointees stood out.

On panels that had members appointed by presidents of the same party,
dissent occurred just 7 percent of the time. The rate jumped to 12
percent on panels that included a mix of judges appointed by both
Democrats and Republicans.

But when a Trump appointee wrote an opinion for a panel with a lone
Democrat, or served as the only Republican appointee, the dissent rate
rose to 17 percent --- meaning the likelihood of dissent was nearly 1.5
times higher if a Trump appointee was involved.

Writing a dissent marks a bold break from fellow members of the bench,
experts say, and by definition sets the judge apart. The dissenting
opinions can also inform future legal arguments and cases.

``You're going to get some judges who will bite their tongue and say,
`These are my colleagues --- I'm not going to rock the boat unless I
feel strongly about it,''' said Russell Wheeler, a visiting fellow at
the Brookings Institution and a former deputy director of the Federal
Judicial Center, the research and education arm of the federal court
system.

In other instances, however, judges ``go in slashing and burning'' with
no regard for comity --- or with an eye to drawing attention to
themselves, he said. ``Some of them obviously are going to be thinking
about the next vacancy on the Supreme Court,'' Mr. Wheeler said.

In a speech in 2017, Mr. McGahn, the former White House counsel and a
main driver of the Trump selection process, said ``judicial courage''
was as important as judicial independence.

The Times analysis included 10,025 opinions of three-judge panels from
2017 through 2019 that were tagged as ``published, written and signed''
in the federal center's integrated case database.

It covered more than 1,975 cases involving at least one Trump appointee.
Because many of Mr. Trump's earliest appointments occurred in appellate
courts dominated by judges named by Republicans, more than half of those
cases did not involve panels with judges appointed by a Democrat.

Of the 125 cases in which a Trump appointee wrote a dissent or an
opinion eliciting dissent, about half involved civil rights or criminal
matters. The others touched on a wide variety of topics, from
transgender rights to pregnancy discrimination to the limits of police
powers.

In one instance, a Trump appointee joined with a Bush appointee to
strike down a key part of the Affordable Care Act. A
Democratic-nominated judge dissented.

Amy Coney Barrett, another judge on Mr. Trump's Supreme Court shortlist,
was among the new appointees who wrote a
\href{http://media.ca7.uscourts.gov/cgi-bin/rssExec.pl?Submit=Display\&Path=Y2019/D03-15/C:18-1478:J:Flaum:aut:T:fnOp:N:2309276:S:0}{dissent}
\href{https://www.nationalreview.com/bench-memos/judge-barretts-dissent-in-second-amendment-case/}{cited
by conservatives}.

Judge Barrett, a noted originalist, once served as a clerk to former
Justice Antonin Scalia. But she stood out among the Trump appointees not
by disagreeing with Democratic appointees but by taking on two judges
named by former President Ronald Reagan, a Republican.

The subject was Second Amendment gun rights, and Ms. Barrett took a
broader view than her colleagues.

The owner of a therapeutic shoe-insert company had pleaded guilty to
mail fraud and, as a felon, was barred from owning a gun. He objected,
claiming the penalty was unconstitutional.

The two Reagan appointees upheld a lower-court ruling against the man.
Their decision was based in part on the notion that governments banned
felons from owning firearms because they were considered more likely to
abuse them.

Image

Judge Amy Coney Barrett, also on the Supreme Court short list, took a
different view from fellow Republican appointees in a recent gun rights
case.Credit...Samuel Corum for The New York Times

But in dissenting, Judge Barrett argued that lawmakers could prohibit
only violent people from owning firearms, and that the government had
not proved that a nonviolent felon would turn violent.

``History does not support the proposition that felons lose their Second
Amendment rights solely because of their status as felons,'' she wrote.

A dissent by another new appointee, Judge Rao, the former Trump
administration official, staked out territory important to the president
and his allies.

Two appellate judges, both appointed by Democrats,
\href{https://www.nytimes.com/2019/10/11/us/politics/mazars-trump-tax-returns.html}{ruled}
that Mr. Trump's accountants had to comply with a congressional subpoena
for eight years of his financial records.

Judge Rao, who holds the appellate seat vacated by Justice Brett M.
Kavanaugh, questioned whether Democrats in the House of Representatives
had overshot their authority. At a hearing on the case, she suggested
the Democrats were, in effect, seeking the ``regulation of the
president.''

In her 67-page dissent, she wrote that the subpoena went beyond
Congress's authority, and that the documents in question could be
obtained only in an impeachment inquiry.

She also
\href{https://www.cadc.uscourts.gov/internet/opinions.nsf/20C16C3C5721030C85258490004DE33C/$file/19-5142-1810450.pdf}{chided}
her fellow judges for allowing Congress to conduct ``a roving
inquisition over a co-equal branch of government,'' suggesting they had
chosen to fixate on worst-case scenarios.

The case
\href{https://www.nytimes.com/2019/12/05/us/trump-supreme-court-mazars.html}{has
been appealed} to the Supreme Court, where two Trump appointees, Justice
Kavanaugh and Justice Gorsuch, will help determine its fate.

\hypertarget{tipping-the-balance}{%
\subsection{Tipping the Balance}\label{tipping-the-balance}}

The push in the Senate last November to confirm a White House lawyer for
a top federal judgeship in New York unnerved Democrats.

The lawyer, Steven Menashi had a trail of inflammatory writings about
feminism and multiculturalism. He had declined to answer specific
questions about his role in the Trump administration on family
detentions and education policy.

And he had managed to get a confirmation vote only because Republicans
did away with a courtesy rule letting home-state senators --- in this
case, Kirsten Gillibrand and Chuck Schumer --- block nominees they found
unworthy. Mr. Schumer had described him as ``a textbook example of
someone who does not deserve to sit on the federal bench.''

Not only did he get his seat on the Second Circuit, but his appointment
marked a signature moment in Mr. Trump's bid to tilt the nation's
appellate courts to the right: Judge Menashi's confirmation flipped the
balance toward Republican appointees in a circuit encompassing three
states --- New York, Connecticut and Vermont --- dominated at nearly
every level by Democrats.

Judges named by Mr. Trump have forged new majorities in two other
circuits --- the Third and the 11th. And they have come close in the
nation's largest appeals court, the Ninth, based in San Francisco, which
has long issued rulings favorable to liberal causes.

The unequal split between Democratic and Republican appointees can give
the circuits distinct reputations as liberal or conservative.

Mr. Trump has called the Ninth Circuit ``out of control'' and a
``complete \& total disaster,'' and he has suggested that some of its
decisions related to immigration and the border have threatened national
security.

In one case, the court ruled that the Trump administration could not
erase Obama-era protections for so-called Dreamers, children brought
into the
\href{https://www.nytimes.com/2018/11/08/us/daca-dreamers-9th-circuit.html}{United
States illegally}.

In another, it
\href{https://www.nytimes.com/2019/03/07/us/asylum-seekers-ninth-circuit.html}{blocked
the administration's effort} to speed up the deportation of asylum
seekers. It also
\href{https://www.nytimes.com/2017/12/22/us/travel-ban-court.html}{rejected
Mr. Trump's policy} restricting travel from eight countries, six of them
largely Muslim.

``Every case that gets filed in the Ninth Circuit, we get beaten,'' Mr.
Trump complained in 2018. ``It's a disgrace.''

At the close of the Obama administration, 18 judges on the circuit had
been appointed by Democratic presidents, seven had been named by
Republicans and four seats were vacant. Through his appointments, Mr.
Trump has whittled the majority held by Democratic appointees to just
three making it less likely that a liberal philosophy can dominate so
thoroughly.

Allies of the president have celebrated. In emails to supporters, the
National Organization for Marriage, a group established to fight the
legalization of same-sex marriage, lauded the change ``from the most
liberal court in the country to one that is much more balanced.''

Brian S. Brown, president of the group, heralded that Mr. Trump was
remaking the court and others across the country. ``Judges will be with
us for a lot longer than any politician who holds office,'' he wrote.

Conservatives have also celebrated Trump appointees in circuits where
the balance of power has not shifted. Their votes have proved
significant in so-called en-banc hearings --- when a decision by a panel
of appellate judges is reviewed by a larger group of judges.

In 2014, a Louisiana law required doctors performing abortions to be
able to admit patients to a hospital within 30 miles of their clinic.
Opponents of the law predicted a chilling effect on access to abortions.
Those in favor argued that it protected women seeking abortions by
making sure doctors were competent.

Image

Supreme Court Justice Brett M. Kavanaugh was a speaker at the Federalist
Society dinner in November.Credit...T.J. Kirkpatrick for The New York
Times

The case,
\href{https://www.nytimes.com/2020/03/04/us/supreme-court-abortion.html}{now
being decided} by the Supreme Court, has been playing out in federal
courts for years. A district court judge struck down the law, but was
reversed by a divided three-judge panel of the Fifth Circuit. When
opponents of the law asked all judges of the circuit to hear an appeal,
the
\href{http://www.ca5.uscourts.gov/opinions/pub/17/17-30397-CV1.pdf}{request
was denied}, 9-6, with four judges appointed by Mr. Trump joining the
majority.

At an en-banc hearing in Missouri,
\href{https://ecf.ca8.uscourts.gov/opndir/19/11/172654P.pdf}{four
Trump-appointed judges}in the Eighth Circuit joined a 6-5 decision that
loosened disclosure requirements for political activists.

In the case, a man who ran a nonprofit advocating conservative causes
had sued Missouri officials over a lobbying registration law he deemed
unconstitutional. He was not a lobbyist, he argued, addressing lawmakers
often but not spending or receiving money for it. A lower court and a
three-judge appeals panel had sided with the state, requiring him to
register out of transparency.

Among the four Trump appointees who overturned that ruling were Judge
Kobes, the former Senate aide who had been confirmed when Mr. Pence
offered up a tiebreaking vote; Judge Stras, who was on Mr. Trump's
shortlist for the Supreme Court; and Judge Grasz of Nebraska.

As he seeks re-election, Mr. Trump has showcased his role in fulfilling
the Republican judicial agenda. One afternoon last November, he gathered
an array of Republican leaders and conservative judicial activists to
celebrate his success.

``I've always heard, actually, that when you become president, the most
--- single most --- important thing you can do is federal judges,'' he
said.

\begin{center}\rule{0.5\linewidth}{\linethickness}\end{center}

\hypertarget{how-the-numbers-were-calculated}{%
\subsubsection{How the Numbers Were
Calculated}\label{how-the-numbers-were-calculated}}

In examining the president's judicial appointments, The Times compiled
two databases of information about judges who were named to the U.S.
Court of Appeals by Mr. Trump and his predecessors.

One focused on the professional and political backgrounds of judges
appointed by Mr. Trump, Mr. Obama and Mr. Bush. The other analyzed
published opinions in the court's 12 regional circuits to gain insights
into ruling patterns and rates of dissent.

\hypertarget{the-biographies}{%
\subsubsection{The Biographies}\label{the-biographies}}

The Times compared the appellate judges' experiences outside the court.
All told, there were 168 appointees --- 51 by Mr. Trump, 55 by Mr. Obama
and 62 by Mr. Bush.

The database drew primarily on biographical questionnaires the
appointees had submitted to the Senate Judiciary Committee, obtained
from staff members, the Congressional Record and other sources. They
listed jobs and internships held since college, judicial clerkships,
club memberships, affiliations with political campaigns and other
information. Some judges volunteered more detail than others.

Separately, campaign finance data was compiled from two sources: the
National Institute on Money in Politics, which has access to state
donations since 2000 and federal ones since 2010, and the Center for
Responsive Politics, which tracks federal donations beginning in 1989.
In searching the donations, The Times sometimes found matches by using
variations of judges' names, including maiden names, as well as other
relevant information like employment.

Calculations of partisan donations were based on federal contributions
to political candidates or causes of the same party as the judge's
appointing president. Past political activity was measured more broadly
and included work for politicians of any party; volunteer or paid work
for political campaigns of any party; memberships affiliated with any
party; donations to campaigns of any party; participation as a candidate
for any party; references to ``Republican'' or ``Democrat'' in any
answer in the questionnaire; and work in a political post in the federal
government, including political duties assigned to a federal employee.

The age of judges on their appointment date was based on years of birth
provided by the \href{https://www.fjc.gov/history/judges}{Federal
Judicial Center}, the official clearinghouse for court research.

\hypertarget{the-rulings-and-dissents}{%
\subsubsection{The Rulings and
Dissents}\label{the-rulings-and-dissents}}

The database includes more than 10,000 opinions published from 2017
through last year in the 12 regional circuit courts. The 13th appeals
court, the Federal Circuit, hears mostly intellectual property cases and
has no Trump appointees.

The case list was published by the
\href{https://www.fjc.gov/research/idb/appellate-cases-filed-terminated-and-pending-fy-2008-present}{Federal
Judicial Center}. Only cases designated ``published, written and
signed'' were included in the analysis, because they carry the weight of
precedence and represent the most legally impactful work. For
consistency, all of the cases involved a standard three-judge panel with
a named opinion author.

For every case, The Times parsed the text of the opinion to identify the
judges, whose names are redacted from the judicial center's data.
Additional information about the judges was obtained by joining the case
data to a separate \href{https://www.fjc.gov/history/judges}{biography
data set} kept by the center.

The data was analyzed in two ways: first, to determine how often cases
involved a dissent, and second, to determine how often individual judges
agreed or disagreed with their two colleagues on a panel.

On the case level, the data showed that when a judge named by Mr. Trump
served in a pivotal role --- as the author of an opinion on a panel with
only one Democratic appointee, or as the only Republican appointee on a
panel --- the rate of dissent increased significantly.

For individual judges, the analysis split each panel into three
pairings. If the case was unanimously decided, all judges were deemed to
have agreed. If one judge dissented, that judge was deemed to have
disagreed with the other two. While judges appointed by presidents of
different parties were more likely to disagree than judges appointed by
presidents of the same party, the difference was far more pronounced for
many, though not all, of the new Trump appointees, the analysis found.

There were caveats to the findings. Some of the circuits have a higher
dissent rate overall, for example, and some circuits appear more often
in the database because they conduct a higher share of their work in the
form of published opinions.

Even accounting for those factors, the findings were supported by a
separate regression analysis, which accounted for other variables,
including the circuit hearing the case, the topic before the court, the
type of appeal and whether the ruling affirmed or overturned a decision
by a lower court.

\begin{center}\rule{0.5\linewidth}{\linethickness}\end{center}

Hillary Flynn and Jaclyn Peiser contributed reporting. Jack Begg
contributed research.

Advertisement

\protect\hyperlink{after-bottom}{Continue reading the main story}

\hypertarget{site-index}{%
\subsection{Site Index}\label{site-index}}

\hypertarget{site-information-navigation}{%
\subsection{Site Information
Navigation}\label{site-information-navigation}}

\begin{itemize}
\tightlist
\item
  \href{https://help.nytimes.com/hc/en-us/articles/115014792127-Copyright-notice}{©~2020~The
  New York Times Company}
\end{itemize}

\begin{itemize}
\tightlist
\item
  \href{https://www.nytco.com/}{NYTCo}
\item
  \href{https://help.nytimes.com/hc/en-us/articles/115015385887-Contact-Us}{Contact
  Us}
\item
  \href{https://www.nytco.com/careers/}{Work with us}
\item
  \href{https://nytmediakit.com/}{Advertise}
\item
  \href{http://www.tbrandstudio.com/}{T Brand Studio}
\item
  \href{https://www.nytimes.com/privacy/cookie-policy\#how-do-i-manage-trackers}{Your
  Ad Choices}
\item
  \href{https://www.nytimes.com/privacy}{Privacy}
\item
  \href{https://help.nytimes.com/hc/en-us/articles/115014893428-Terms-of-service}{Terms
  of Service}
\item
  \href{https://help.nytimes.com/hc/en-us/articles/115014893968-Terms-of-sale}{Terms
  of Sale}
\item
  \href{https://spiderbites.nytimes.com}{Site Map}
\item
  \href{https://help.nytimes.com/hc/en-us}{Help}
\item
  \href{https://www.nytimes.com/subscription?campaignId=37WXW}{Subscriptions}
\end{itemize}
