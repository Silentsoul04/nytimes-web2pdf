Sections

SEARCH

\protect\hyperlink{site-content}{Skip to
content}\protect\hyperlink{site-index}{Skip to site index}

\href{https://www.nytimes.com/section/health}{Health}

\href{https://myaccount.nytimes.com/auth/login?response_type=cookie\&client_id=vi}{}

\href{https://www.nytimes.com/section/todayspaper}{Today's Paper}

\href{/section/health}{Health}\textbar{}Trump's Embrace of Unproven
Drugs to Treat Coronavirus Defies Science

\url{https://nyti.ms/2QzfDjg}

\begin{itemize}
\item
\item
\item
\item
\item
\end{itemize}

\href{https://www.nytimes.com/news-event/coronavirus?action=click\&pgtype=Article\&state=default\&region=TOP_BANNER\&context=storylines_menu}{The
Coronavirus Outbreak}

\begin{itemize}
\tightlist
\item
  live\href{https://www.nytimes.com/2020/08/04/world/coronavirus-cases.html?action=click\&pgtype=Article\&state=default\&region=TOP_BANNER\&context=storylines_menu}{Latest
  Updates}
\item
  \href{https://www.nytimes.com/interactive/2020/us/coronavirus-us-cases.html?action=click\&pgtype=Article\&state=default\&region=TOP_BANNER\&context=storylines_menu}{Maps
  and Cases}
\item
  \href{https://www.nytimes.com/interactive/2020/science/coronavirus-vaccine-tracker.html?action=click\&pgtype=Article\&state=default\&region=TOP_BANNER\&context=storylines_menu}{Vaccine
  Tracker}
\item
  \href{https://www.nytimes.com/2020/08/02/us/covid-college-reopening.html?action=click\&pgtype=Article\&state=default\&region=TOP_BANNER\&context=storylines_menu}{College
  Reopening}
\item
  \href{https://www.nytimes.com/live/2020/08/04/business/stock-market-today-coronavirus?action=click\&pgtype=Article\&state=default\&region=TOP_BANNER\&context=storylines_menu}{Economy}
\end{itemize}

Advertisement

\protect\hyperlink{after-top}{Continue reading the main story}

Supported by

\protect\hyperlink{after-sponsor}{Continue reading the main story}

\hypertarget{trumps-embrace-of-unproven-drugs-to-treat-coronavirus-defies-science}{%
\section{Trump's Embrace of Unproven Drugs to Treat Coronavirus Defies
Science}\label{trumps-embrace-of-unproven-drugs-to-treat-coronavirus-defies-science}}

Doctors and patients also worry that the president's rosy outlook for
the treatments will exacerbate shortages of old malaria drugs relied on
by patients with lupus and other debilitating conditions.

\includegraphics{https://static01.nyt.com/images/2020/03/20/science/20VIRUS-DRUGS2/20VIRUS-DRUGS2-videoSixteenByNine3000.jpg}

By \href{https://www.nytimes.com/by/katie-thomas}{Katie Thomas} and
\href{https://www.nytimes.com/by/denise-grady}{Denise Grady}

\begin{itemize}
\item
  March 20, 2020
\item
  \begin{itemize}
  \item
  \item
  \item
  \item
  \item
  \end{itemize}
\end{itemize}

At a long-winded White House briefing on Friday, President Trump
enthusiastically and repeatedly promoted the promise of two long-used
\href{https://www.nytimes.com/2020/04/01/health/hydroxychloroquine-coronavirus-malaria.html}{malaria
drugs} that are still unproven against the coronavirus, but being tested
in clinical trials.

``I'm a smart guy,'' he said, while acknowledging he couldn't predict
the drugs would work. ``I feel good about it. And we're going to see.
You're going to see soon enough.''

But the nation's leading infectious disease expert, Dr. Anthony S.
Fauci, delicately --- yet forcefully --- pushed back from the same
stage, explaining that there was only anecdotal evidence that the drugs,
chloroquine and hydroxychloroquine, may be effective.

``The president feels optimistic about something, has feelings about
it,'' said Dr. Fauci, director of the National Institute of Allergy and
Infectious Diseases, emphasizing that he was a scientist. ``I am saying
it may be effective.''

Mr. Trump's boosterish attitude toward the drugs has deepened worries
among doctors and patients with lupus and other diseases who rely on the
drugs, because the idea that the old malaria drugs could work against
the coronavirus has circulated widely in recent weeks and fueled
shortages that have already left people rushing to fill their
prescriptions.

``Rheumatologists are furious about the hype going on over this drug,''
said Dr. Michael Lockshin, of the Hospital for Special Surgery in
Manhattan. ``There is a run on it and we're getting calls every few
minutes, literally, from patients who are trying to stay on the drug and
finding it in short supply.''

The moment of discord between Mr. Trump and one of the nation's most
trusted authorities on the coronavirus was a clash between opinion and
fact. It threw Mr. Trump's faith in his own instincts into conflict with
the careful, evidence-based approach of scientists like Dr. Fauci, who
has held his position since the presidency of Ronald Reagan. Mr. Trump
appeared eager to sweep aside long-established standards for evaluating
drugs in order to champion the remedy he favors.

\hypertarget{latest-updates-global-coronavirus-outbreak}{%
\section{\texorpdfstring{\href{https://www.nytimes.com/2020/08/04/world/coronavirus-cases.html?action=click\&pgtype=Article\&state=default\&region=MAIN_CONTENT_1\&context=storylines_live_updates}{Latest
Updates: Global Coronavirus
Outbreak}}{Latest Updates: Global Coronavirus Outbreak}}\label{latest-updates-global-coronavirus-outbreak}}

Updated 2020-08-05T04:01:36.184Z

\begin{itemize}
\tightlist
\item
  \href{https://www.nytimes.com/2020/08/04/world/coronavirus-cases.html?action=click\&pgtype=Article\&state=default\&region=MAIN_CONTENT_1\&context=storylines_live_updates\#link-762df92}{As
  talks drag on, McConnell signals openness to jobless aid extension,
  and negotiators agree on a deadline.}
\item
  \href{https://www.nytimes.com/2020/08/04/world/coronavirus-cases.html?action=click\&pgtype=Article\&state=default\&region=MAIN_CONTENT_1\&context=storylines_live_updates\#link-1228a480}{Novavax
  sees encouraging results from two studies of its experimental
  vaccine.}
\item
  \href{https://www.nytimes.com/2020/08/04/world/coronavirus-cases.html?action=click\&pgtype=Article\&state=default\&region=MAIN_CONTENT_1\&context=storylines_live_updates\#link-794484ed}{Mississippians
  must now wear masks in public, governor says.}
\end{itemize}

\href{https://www.nytimes.com/2020/08/04/world/coronavirus-cases.html?action=click\&pgtype=Article\&state=default\&region=MAIN_CONTENT_1\&context=storylines_live_updates}{See
more updates}

More live coverage:
\href{https://www.nytimes.com/live/2020/08/04/business/stock-market-today-coronavirus?action=click\&pgtype=Article\&state=default\&region=MAIN_CONTENT_1\&context=storylines_live_updates}{Markets}

The excitement about the drugs is based largely on reports from China
and France that they seem to help patients. But researchers and Dr.
Fauci have stressed that the reports are not based on carefully
controlled studies, which are the only way to be sure a treatment really
works.

In
\href{https://twitter.com/realDonaldTrump/status/1241367239900778501}{a
tweet on Saturday}, Mr. Trump called attention to yet another unapproved
treatment for Covid-19, this time a combination of hydroxychloroquine
and azithromycin, a common anti-bacterial agent. He cited a report by
French researchers in a scientific journal that was not a controlled
clinical trial, and studied only 20 patients.

As word of the drugs' possible effects have spread around the globe,
demand has surged, with hospitals ordering the treatments in a desperate
effort to treat severely ill patients.

Two large generic manufacturers, Teva and Mylan, have said they are
ramping up production of hydroxychloroquine, and Teva has said it will
donate millions of pills to the U.S. government. The sole manufacturer
of chloroquine, Rising Pharmaceuticals, has also said it is increasing
production. In addition, the German company Bayer announced this week it
was donating millions of pills of chloroquine to the U.S. government and
would seek approval from the F.D.A. for its products to be used in the
United States.

Hydroxychloroquine is especially important for people with lupus, which
can be life-threatening, Dr. Lockshin said. The drug can lower the risk
of dying from lupus and prevent organ damage, and is considered the
standard of care. If patients stop taking it after using it regularly
for a long time, they can gradually become quite ill. He said it was
particularly disturbing to think that people known to benefit from the
drug could lose access to it because it is being diverted to a disease
for which there is no solid evidence that it actually works.

``If there were justification for everyone taking it, that would be one
thing,'' Dr. Lockshin said. ``It's not hard to do the studies even in
the midst of this crisis. We could have answers in a few weeks. But it's
being prescribed right and left.''

Dana Olita, 50, of Los Angeles, raced Thursday and Friday to refill her
prescription for hydroxychloroquine, which she has taken for a decade to
treat rheumatoid arthritis. Her pharmacist at CVS initially told her it
was unavailable; then she was told on Friday that they had located
enough to fill her prescription. ``There are lots of people that
desperately need this, and if we stop taking it, the problem to us is
overwhelming,'' she said.

A spokesman for CVS said the company was ``closely monitoring the global
pharmaceutical manufacturing environment and working with our suppliers
to ensure we can continue filling prescriptions for pharmacy patients
and plan members.'' The company said it had adequate stocks of
hydroxychloroquine but described the supply of chloroquine as ``tight''
and said it was taking steps to address the issue.

``I would hope that doctors would stick to the science and try to keep a
cool head,'' said Dr. Percio S. Gulko, chief of the division of
rheumatology at the Icahn School of Medicine at Mount Sinai. ``Somebody
is prescribing it for people who are trying to get it, in some instances
preventively. They may just be depriving the patients who do need it for
an established indication, for a possibility or a speculation.''

\href{https://www.nytimes.com/news-event/coronavirus?action=click\&pgtype=Article\&state=default\&region=MAIN_CONTENT_3\&context=storylines_faq}{}

\hypertarget{the-coronavirus-outbreak-}{%
\subsubsection{The Coronavirus Outbreak
›}\label{the-coronavirus-outbreak-}}

\hypertarget{frequently-asked-questions}{%
\paragraph{Frequently Asked
Questions}\label{frequently-asked-questions}}

Updated August 4, 2020

\begin{itemize}
\item ~
  \hypertarget{i-have-antibodies-am-i-now-immune}{%
  \paragraph{I have antibodies. Am I now
  immune?}\label{i-have-antibodies-am-i-now-immune}}

  \begin{itemize}
  \tightlist
  \item
    As of right
    now,\href{https://www.nytimes.com/2020/07/22/health/covid-antibodies-herd-immunity.html?action=click\&pgtype=Article\&state=default\&region=MAIN_CONTENT_3\&context=storylines_faq}{that
    seems likely, for at least several months.} There have been
    frightening accounts of people suffering what seems to be a second
    bout of Covid-19. But experts say these patients may have a
    drawn-out course of infection, with the virus taking a slow toll
    weeks to months after initial exposure. People infected with the
    coronavirus typically
    \href{https://www.nature.com/articles/s41586-020-2456-9}{produce}
    immune molecules called antibodies, which are
    \href{https://www.nytimes.com/2020/05/07/health/coronavirus-antibody-prevalence.html?action=click\&pgtype=Article\&state=default\&region=MAIN_CONTENT_3\&context=storylines_faq}{protective
    proteins made in response to an
    infection}\href{https://www.nytimes.com/2020/05/07/health/coronavirus-antibody-prevalence.html?action=click\&pgtype=Article\&state=default\&region=MAIN_CONTENT_3\&context=storylines_faq}{.
    These antibodies may} last in the body
    \href{https://www.nature.com/articles/s41591-020-0965-6}{only two to
    three months}, which may seem worrisome, but that's perfectly normal
    after an acute infection subsides, said Dr. Michael Mina, an
    immunologist at Harvard University. It may be possible to get the
    coronavirus again, but it's highly unlikely that it would be
    possible in a short window of time from initial infection or make
    people sicker the second time.
  \end{itemize}
\item ~
  \hypertarget{im-a-small-business-owner-can-i-get-relief}{%
  \paragraph{I'm a small-business owner. Can I get
  relief?}\label{im-a-small-business-owner-can-i-get-relief}}

  \begin{itemize}
  \tightlist
  \item
    The
    \href{https://www.nytimes.com/article/small-business-loans-stimulus-grants-freelancers-coronavirus.html?action=click\&pgtype=Article\&state=default\&region=MAIN_CONTENT_3\&context=storylines_faq}{stimulus
    bills enacted in March} offer help for the millions of American
    small businesses. Those eligible for aid are businesses and
    nonprofit organizations with fewer than 500 workers, including sole
    proprietorships, independent contractors and freelancers. Some
    larger companies in some industries are also eligible. The help
    being offered, which is being managed by the Small Business
    Administration, includes the Paycheck Protection Program and the
    Economic Injury Disaster Loan program. But lots of folks have
    \href{https://www.nytimes.com/interactive/2020/05/07/business/small-business-loans-coronavirus.html?action=click\&pgtype=Article\&state=default\&region=MAIN_CONTENT_3\&context=storylines_faq}{not
    yet seen payouts.} Even those who have received help are confused:
    The rules are draconian, and some are stuck sitting on
    \href{https://www.nytimes.com/2020/05/02/business/economy/loans-coronavirus-small-business.html?action=click\&pgtype=Article\&state=default\&region=MAIN_CONTENT_3\&context=storylines_faq}{money
    they don't know how to use.} Many small-business owners are getting
    less than they expected or
    \href{https://www.nytimes.com/2020/06/10/business/Small-business-loans-ppp.html?action=click\&pgtype=Article\&state=default\&region=MAIN_CONTENT_3\&context=storylines_faq}{not
    hearing anything at all.}
  \end{itemize}
\item ~
  \hypertarget{what-are-my-rights-if-i-am-worried-about-going-back-to-work}{%
  \paragraph{What are my rights if I am worried about going back to
  work?}\label{what-are-my-rights-if-i-am-worried-about-going-back-to-work}}

  \begin{itemize}
  \tightlist
  \item
    Employers have to provide
    \href{https://www.osha.gov/SLTC/covid-19/standards.html}{a safe
    workplace} with policies that protect everyone equally.
    \href{https://www.nytimes.com/article/coronavirus-money-unemployment.html?action=click\&pgtype=Article\&state=default\&region=MAIN_CONTENT_3\&context=storylines_faq}{And
    if one of your co-workers tests positive for the coronavirus, the
    C.D.C.} has said that
    \href{https://www.cdc.gov/coronavirus/2019-ncov/community/guidance-business-response.html}{employers
    should tell their employees} -\/- without giving you the sick
    employee's name -\/- that they may have been exposed to the virus.
  \end{itemize}
\item ~
  \hypertarget{should-i-refinance-my-mortgage}{%
  \paragraph{Should I refinance my
  mortgage?}\label{should-i-refinance-my-mortgage}}

  \begin{itemize}
  \tightlist
  \item
    \href{https://www.nytimes.com/article/coronavirus-money-unemployment.html?action=click\&pgtype=Article\&state=default\&region=MAIN_CONTENT_3\&context=storylines_faq}{It
    could be a good idea,} because mortgage rates have
    \href{https://www.nytimes.com/2020/07/16/business/mortgage-rates-below-3-percent.html?action=click\&pgtype=Article\&state=default\&region=MAIN_CONTENT_3\&context=storylines_faq}{never
    been lower.} Refinancing requests have pushed mortgage applications
    to some of the highest levels since 2008, so be prepared to get in
    line. But defaults are also up, so if you're thinking about buying a
    home, be aware that some lenders have tightened their standards.
  \end{itemize}
\item ~
  \hypertarget{what-is-school-going-to-look-like-in-september}{%
  \paragraph{What is school going to look like in
  September?}\label{what-is-school-going-to-look-like-in-september}}

  \begin{itemize}
  \tightlist
  \item
    It is unlikely that many schools will return to a normal schedule
    this fall, requiring the grind of
    \href{https://www.nytimes.com/2020/06/05/us/coronavirus-education-lost-learning.html?action=click\&pgtype=Article\&state=default\&region=MAIN_CONTENT_3\&context=storylines_faq}{online
    learning},
    \href{https://www.nytimes.com/2020/05/29/us/coronavirus-child-care-centers.html?action=click\&pgtype=Article\&state=default\&region=MAIN_CONTENT_3\&context=storylines_faq}{makeshift
    child care} and
    \href{https://www.nytimes.com/2020/06/03/business/economy/coronavirus-working-women.html?action=click\&pgtype=Article\&state=default\&region=MAIN_CONTENT_3\&context=storylines_faq}{stunted
    workdays} to continue. California's two largest public school
    districts --- Los Angeles and San Diego --- said on July 13, that
    \href{https://www.nytimes.com/2020/07/13/us/lausd-san-diego-school-reopening.html?action=click\&pgtype=Article\&state=default\&region=MAIN_CONTENT_3\&context=storylines_faq}{instruction
    will be remote-only in the fall}, citing concerns that surging
    coronavirus infections in their areas pose too dire a risk for
    students and teachers. Together, the two districts enroll some
    825,000 students. They are the largest in the country so far to
    abandon plans for even a partial physical return to classrooms when
    they reopen in August. For other districts, the solution won't be an
    all-or-nothing approach.
    \href{https://bioethics.jhu.edu/research-and-outreach/projects/eschool-initiative/school-policy-tracker/}{Many
    systems}, including the nation's largest, New York City, are
    devising
    \href{https://www.nytimes.com/2020/06/26/us/coronavirus-schools-reopen-fall.html?action=click\&pgtype=Article\&state=default\&region=MAIN_CONTENT_3\&context=storylines_faq}{hybrid
    plans} that involve spending some days in classrooms and other days
    online. There's no national policy on this yet, so check with your
    municipal school system regularly to see what is happening in your
    community.
  \end{itemize}
\end{itemize}

``If it does turn out to be a success, we understand that there will be
a need for more than has ever been available for patients with
autoimmune diseases,'' said Dr. David R. Karp, the president-elect of
the American College of Rheumatology and chief of the rheumatic disease
division at the University of Texas Southwestern Medical Center in
Dallas. ``We hope there will be a way for our patients to continue to
access these medicines they've been taking for many years. Other
medications can be used, but the safety profile gets much worse, and
patients will likely have side effects.''

Judie Stein, of Sun Prairie, Wis., said she was stunned when she heard
Mr. Trump pronounce the name of the drug she has taken for two years to
treat rheumatoid arthritis. ``When I was first prescribed this, nobody
had heard of it,'' said Ms. Stein, who is 59. She said she has a
one-month supply of the drug and she quickly tried to refill her
prescription yesterday. She has not yet received a confirmation. ``If
it's readily available and has other uses, then fine, but when I say I
really need it, I need it,'' she said.

Dr. Michael Belmont, medical director of NYU Langone Orthopedic
Hospital, said a number of his lupus patients had requested 90-day
prescriptions for hydroxychloroquine, rather than the usual 30 days'
worth.

Noting that hydroxychloroquine was being widely used in coronavirus
patients outside of controlled studies, he said, ``It would be a shame
if we use a lot of this and after all is said and done we are not able
to determine with accuracy whether it had an effect or not.''

Onisis Stefas, the chief pharmacy officer for Northwell Health's 23
hospitals, said the system began stocking up on hydroxychloroquine
several weeks ago. He said the drug was being given to many coronavirus
patients, but that Northwell's 10 pharmacies were also setting aside
supplies for patients who had been taking it regularly for lupus or
rheumatoid arthritis.

Concerned about shortages, Mr. Stefas said, ``The last thing I want to
happen now is that, especially since President Trump and others have
been mentioning this by name, is that people will go out and ask their
doctors to write prescriptions, just in case.''

On Friday, Mr. Trump appeared to encourage Americans to do just that,
arguing that there was little downside to taking a malaria drug that is
already on the market.

``If you wanted, you can have a prescription. You get a prescription,''
he said. ``You know the expression, what the hell do you have to lose?''

\includegraphics{https://static01.nyt.com/images/2020/03/20/science/20VIRUS-DRUGS/20VIRUS-DRUGS-articleLarge.jpg?quality=75\&auto=webp\&disable=upscale}

Advertisement

\protect\hyperlink{after-bottom}{Continue reading the main story}

\hypertarget{site-index}{%
\subsection{Site Index}\label{site-index}}

\hypertarget{site-information-navigation}{%
\subsection{Site Information
Navigation}\label{site-information-navigation}}

\begin{itemize}
\tightlist
\item
  \href{https://help.nytimes.com/hc/en-us/articles/115014792127-Copyright-notice}{©~2020~The
  New York Times Company}
\end{itemize}

\begin{itemize}
\tightlist
\item
  \href{https://www.nytco.com/}{NYTCo}
\item
  \href{https://help.nytimes.com/hc/en-us/articles/115015385887-Contact-Us}{Contact
  Us}
\item
  \href{https://www.nytco.com/careers/}{Work with us}
\item
  \href{https://nytmediakit.com/}{Advertise}
\item
  \href{http://www.tbrandstudio.com/}{T Brand Studio}
\item
  \href{https://www.nytimes.com/privacy/cookie-policy\#how-do-i-manage-trackers}{Your
  Ad Choices}
\item
  \href{https://www.nytimes.com/privacy}{Privacy}
\item
  \href{https://help.nytimes.com/hc/en-us/articles/115014893428-Terms-of-service}{Terms
  of Service}
\item
  \href{https://help.nytimes.com/hc/en-us/articles/115014893968-Terms-of-sale}{Terms
  of Sale}
\item
  \href{https://spiderbites.nytimes.com}{Site Map}
\item
  \href{https://help.nytimes.com/hc/en-us}{Help}
\item
  \href{https://www.nytimes.com/subscription?campaignId=37WXW}{Subscriptions}
\end{itemize}
