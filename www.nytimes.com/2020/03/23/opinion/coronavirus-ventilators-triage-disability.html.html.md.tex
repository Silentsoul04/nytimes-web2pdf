Sections

SEARCH

\protect\hyperlink{site-content}{Skip to
content}\protect\hyperlink{site-index}{Skip to site index}

\href{https://myaccount.nytimes.com/auth/login?response_type=cookie\&client_id=vi}{}

\href{https://www.nytimes.com/section/todayspaper}{Today's Paper}

\href{/section/opinion}{Opinion}\textbar{}`I Will Not Apologize for My
Needs'

\href{https://nyti.ms/3bgjlpN}{https://nyti.ms/3bgjlpN}

\begin{itemize}
\item
\item
\item
\item
\item
\end{itemize}

Advertisement

\protect\hyperlink{after-top}{Continue reading the main story}

\href{/section/opinion}{Opinion}

Supported by

\protect\hyperlink{after-sponsor}{Continue reading the main story}

disability

\hypertarget{i-will-not-apologize-for-my-needs}{%
\section{`I Will Not Apologize for My
Needs'}\label{i-will-not-apologize-for-my-needs}}

Even in a crisis, doctors should not abandon the principle of
nondiscrimination.

By Ari Ne'eman

Mr. Ne'eman is a disability rights activist and author.

\begin{itemize}
\item
  March 23, 2020
\item
  \begin{itemize}
  \item
  \item
  \item
  \item
  \item
  \end{itemize}
\end{itemize}

\includegraphics{https://static01.nyt.com/images/2020/03/23/opinion/23disability-neeman/23disability-neeman-articleLarge.jpg?quality=75\&auto=webp\&disable=upscale}

Times of crisis ask us who we are as a country. As hospitals prepare for
shortages in ventilators and other scarce medical resources, many people
with disabilities are worried about the answer to that question.

\href{http://www.siaarti.it/SiteAssets/News/COVID19\%20-\%20documenti\%20SIAARTI/SIAARTI\%20-\%20Covid19\%20-\%20Raccomandazioni\%20di\%20etica\%20clinica.pdf}{In
Italy, doctors}
\href{https://www.theatlantic.com/ideas/archive/2020/03/who-gets-hospital-bed/607807/}{are
already rationing access to care} on the basis of age and disability.
The Washington Post
\href{https://www.washingtonpost.com/health/2020/03/15/coronavirus-rationing-us/}{reports}
that many states are considering how to implement similar rationing
measures here. Though almost everyone would agree doctors may deny care
that is unlikely to benefit a patient, there may soon be too many
patients in urgent need of lifesaving treatment and too few resources to
treat them all.

When that happens, some are proposing to send the disabled to the back
of the line. States across the country are
\href{https://www.washingtonpost.com/health/2020/03/15/coronavirus-rationing-us/?fbclid=IwAR057A8cmTE1u90ZhjhyWMIkK7gnDmOE1l6FrKBOZE8BUSEgYTtFr7Nl9_w}{looking
to their Crisis Standards of Care plans} --- documents that explain how
medical care changes amid the shortages of an unprecedented catastrophe.
While each is different, many have a concerning common attribute: When
there isn't enough lifesaving care to go around, those who need more
than others may be in trouble.

Some plans single out particularly severe conditions, like
\href{http://www.adph.org/CEP/assets/VENTTRIAGE.pdf}{Alabama}'s decision
that people with severe or profound intellectual disability ``are
unlikely candidates for ventilator support'' or
\href{https://www.tn.gov/content/dam/tn/health/documents/2016_Guidance_for_the_Ethical_Allocation_of_Scarce_Resources.pdf}{Tennessee}'s
listing people with spinal muscular atrophy who need assistance with
activities of daily living among those excluded from critical care.

Others just lay out a broad goal, counting on clinical judgment to do
the rest. Newly issued
\href{https://covid-19.uwmedicine.org/Screening\%20and\%20Testing\%20Algorithms/Other\%20Inpatient\%20Clinical\%20Guidance/Clinical\%20Care\%20in\%20ICU/Material\%20Resource\%20Allocation.COVID19.docx}{allocation
guidelines} from the University of Washington Medical Center argue for
``weighting the survival of young otherwise-healthy patients more
heavily than that of older, chronically debilitated patients.'' The
existence of the nonelderly disabled, a group increasingly in fear for
their lives, goes unacknowledged.

People with disabilities have a long and complicated history with the
medical profession. While many disabled people need ongoing medical
care, many doctors view life with certain disabilities as unworthy of
living. Disabled people who require ongoing ventilator care and other
forms of expensive lifelong assistance are used to being asked by
medical professionals if they would rather abandon life-sustaining
treatment --- often with the clear implication that ``yes'' is the right
answer.

When my friends with some of these needs go into the hospital, even
under normal circumstances, those of us who love them try to organize
lots of calls and visits. These aren't just to keep the patient's
spirits up. They are designed to send a message to treating
professionals: ``Someone cares if this person lives or dies. You are
being watched.''

With visiting restrictions in place and many prominent authorities
explicitly allowing the denial of care to disabled people, will that
message now get through? I worry it will not.

Even when discrimination is not based on perceptions of quality of life,
but instead on seemingly ``rational'' considerations of resource
intensity, we should object to abandoning the disabled to second-class
medical status.

Italian clinical guidelines have called for ``the presence of
comorbidity and functional status'' to be evaluated as considerations in
the allocation of resources, as ``a relatively brief progression in
healthy patients could become longer and thus more resource-consuming on
the health care system in the case of elderly patients, fragile patients
or patients with severe comorbidity.''

This idea is both straightforward and concerning: Patients with
disabilities may require more resources than the nondisabled. In a
crisis, the nondisabled can be saved more efficiently. As a result, when
doctors must choose between a disabled and a nondisabled patient with
similarly urgent levels of need, the nondisabled patients should get
priority, since they will recover more quickly, freeing up scarce
resources.

Adopting such an approach would be a mistake. Even in a crisis,
authorities should not abandon nondiscrimination. By permitting
clinicians to discriminate against those who require more resources,
perhaps more lives would be saved. But the ranks of the survivors would
look very different, biased toward those who lacked disabilities before
the pandemic. Equity would have been sacrificed in the name of
efficiency.

Not only is such an approach poor ethics --- it can also interfere with
efforts to combat the pandemic.

In 2015, the New York State Department of Health developed
\href{https://www.health.ny.gov/regulations/task_force/reports_publications/docs/ventilator_guidelines.pdf}{guidelines}
on how to allocate ventilators in a crisis. Among other things, they
permit hospitals to take away ventilators from those who use them on an
ongoing basis in the community or at a long-term care facility if they
seek hospital care. Not only is this a concerning precedent, it also
interferes with the trust in the medical system that we need to combat
the virus: Chronic ventilator users may have reason to avoid seeking
needed hospital care if they become infected, based on a well-founded
fear of being sacrificed ``for the greater good.''

I spoke to a colleague of mine, Alice Wong of the
\href{https://disabilityvisibilityproject.com/}{Disability Visibility
Project}, on these issues. As a 46-year-old who uses a ventilator on a
regular basis, she has a lot at stake.

``My vent is part of my body --- I cannot be without it for more than an
hour at the most due to my neuromuscular disability. For clinicians to
take my vent away from me would be an assault on my personhood and lead
to my death,'' Alice writes. ``I deserve the same treatments as any
patient. As a disabled person, I've been clawing my way into existence
ever since I was born. I will not apologize for my needs.''

She is correct. To allow discrimination against the disabled, even when
there isn't enough to go around, is simply wrong. Disability advocates
are mobilizing to defend this position --- on Thursday, the
\href{https://www.aapd.com/wp-content/uploads/2020/03/COVID-19-Response-Package.pdf}{American
Association of People with Disabilities sent a letter to Congress}
urging ``a statutory prohibition on the rationing of scarce medical
resources on the basis of anticipated or demonstrated resource-intensity
needs.''

\href{https://www.nytimes.com/2020/03/12/opinion/coronavirus-hospital-shortage.html}{Though
some insist otherwise}, we should maintain a broad approach of ``first
come first served'' when it comes to lifesaving care, even scarce
medical resources like ventilators. We certainly should not remove
ventilators from those who are already using them in the name of
allocating more ``efficiently.''

This is a sacrifice --- but not so great as some might imagine.
Maintaining nondiscrimination does not require hospitals to treat those
who would die anyway. Even under nondisaster situations, clinicians can
withhold care that is
\href{https://depts.washington.edu/bhdept/ethics-medicine/bioethics-topics/detail/65}{deemed
futile} --- medically ineffective. But those who can be helped should
not be given lower priority because of pre-existing disabilities, even
those that will require more scarce resources.

I recognize that this approach imposes a cost. By maintaining ``first
come first served'' for the provision of nonfutile lifesaving care, we
may save fewer lives than through ruthlessly efficient optimization. If
someone needs twice the average amount of time on a ventilator,
maintaining that we shouldn't turn them away --- or deprive them of a
ventilator they are already using --- means that we are potentially
costing the lives of two people who come into the I.C.U. after them.

But even in a crisis, can we not ascribe some value to maintaining our
principles? I argue yes --- though it may cost lives. This is an
unorthodox position, and one that may earn me the ire of the esteemed
bioethicists who crafted the rationing protocols now on the verge of
deployment.

But I fight for it, because I believe that nondiscrimination is not just
a tool to accomplish an end --- it also is an end in and of itself.
Federal authorities, like the
\href{https://www.hhs.gov/ocr/index.html}{Health and Human Services
Office of Civil Rights}, must defend the equality of disabled Americans,
even now.

At its core, these debates are about value --- the value we place on
disabled life and the value we place on disability nondiscrimination.
When Congress passed the \href{https://www.ada.gov/}{Americans With
Disabilities Act} 30 years ago, did it do so as a form of charity
limited to times of plenty? Or was our country serious about disability
as a civil rights issue? Charity can end when resources are scarce ---
civil rights must continue, even if doing so imposes a cost in time,
money and even lives. People with disabilities have an equal right to
society's scarce resources, even in a time of crisis.

Ari Ne'eman is a visiting scholar at the Lurie Institute for Disability
Policy at Brandeis University and a doctoral student in health policy at
Harvard University. He is at work on a book on the history of American
disability advocacy.

\emph{Disability is a series of essays, art and opinion by and about
people living with disabilities.}

\emph{A collection of 60 essays from this series is now available in
book, e-book and audiobook form:
``}\href{https://www.aboutusbook.com/}{\emph{About Us: Essays From the
Disability Series of The New York Times}}\emph{,'' edited by Peter
Catapano and Rosemarie Garland-Thomson, published by Liveright.}

\emph{The Times is committed to publishing}
\href{https://www.nytimes.com/2019/01/31/opinion/letters/letters-to-editor-new-york-times-women.html}{\emph{a
diversity of letters}} \emph{to the editor. We'd like to hear what you
think about this or any of our articles. Here are some}
\href{https://help.nytimes.com/hc/en-us/articles/115014925288-How-to-submit-a-letter-to-the-editor}{\emph{tips}}\emph{.
And here's our email:}
\href{mailto:letters@nytimes.com}{\emph{letters@nytimes.com}}\emph{.}

\emph{Follow The New York Times Opinion section on}
\href{https://www.facebook.com/nytopinion}{\emph{Facebook}}\emph{,}
\href{http://twitter.com/NYTOpinion}{\emph{Twitter (@NYTopinion)}}
\emph{and}
\href{https://www.instagram.com/nytopinion/}{\emph{Instagram}}\emph{.}

Advertisement

\protect\hyperlink{after-bottom}{Continue reading the main story}

\hypertarget{site-index}{%
\subsection{Site Index}\label{site-index}}

\hypertarget{site-information-navigation}{%
\subsection{Site Information
Navigation}\label{site-information-navigation}}

\begin{itemize}
\tightlist
\item
  \href{https://help.nytimes.com/hc/en-us/articles/115014792127-Copyright-notice}{©~2020~The
  New York Times Company}
\end{itemize}

\begin{itemize}
\tightlist
\item
  \href{https://www.nytco.com/}{NYTCo}
\item
  \href{https://help.nytimes.com/hc/en-us/articles/115015385887-Contact-Us}{Contact
  Us}
\item
  \href{https://www.nytco.com/careers/}{Work with us}
\item
  \href{https://nytmediakit.com/}{Advertise}
\item
  \href{http://www.tbrandstudio.com/}{T Brand Studio}
\item
  \href{https://www.nytimes.com/privacy/cookie-policy\#how-do-i-manage-trackers}{Your
  Ad Choices}
\item
  \href{https://www.nytimes.com/privacy}{Privacy}
\item
  \href{https://help.nytimes.com/hc/en-us/articles/115014893428-Terms-of-service}{Terms
  of Service}
\item
  \href{https://help.nytimes.com/hc/en-us/articles/115014893968-Terms-of-sale}{Terms
  of Sale}
\item
  \href{https://spiderbites.nytimes.com}{Site Map}
\item
  \href{https://help.nytimes.com/hc/en-us}{Help}
\item
  \href{https://www.nytimes.com/subscription?campaignId=37WXW}{Subscriptions}
\end{itemize}
