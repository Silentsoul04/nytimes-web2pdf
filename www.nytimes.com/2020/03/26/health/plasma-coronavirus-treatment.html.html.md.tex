Sections

SEARCH

\protect\hyperlink{site-content}{Skip to
content}\protect\hyperlink{site-index}{Skip to site index}

\href{https://www.nytimes.com/section/health}{Health}

\href{https://myaccount.nytimes.com/auth/login?response_type=cookie\&client_id=vi}{}

\href{https://www.nytimes.com/section/todayspaper}{Today's Paper}

\href{/section/health}{Health}\textbar{}Blood Plasma From Survivors Will
Be Given to Coronavirus Patients

\url{https://nyti.ms/3dyIIFi}

\begin{itemize}
\item
\item
\item
\item
\item
\item
\end{itemize}

\href{https://www.nytimes.com/news-event/coronavirus?action=click\&pgtype=Article\&state=default\&region=TOP_BANNER\&context=storylines_menu}{The
Coronavirus Outbreak}

\begin{itemize}
\tightlist
\item
  live\href{https://www.nytimes.com/2020/08/04/world/coronavirus-cases.html?action=click\&pgtype=Article\&state=default\&region=TOP_BANNER\&context=storylines_menu}{Latest
  Updates}
\item
  \href{https://www.nytimes.com/interactive/2020/us/coronavirus-us-cases.html?action=click\&pgtype=Article\&state=default\&region=TOP_BANNER\&context=storylines_menu}{Maps
  and Cases}
\item
  \href{https://www.nytimes.com/interactive/2020/science/coronavirus-vaccine-tracker.html?action=click\&pgtype=Article\&state=default\&region=TOP_BANNER\&context=storylines_menu}{Vaccine
  Tracker}
\item
  \href{https://www.nytimes.com/2020/08/02/us/covid-college-reopening.html?action=click\&pgtype=Article\&state=default\&region=TOP_BANNER\&context=storylines_menu}{College
  Reopening}
\item
  \href{https://www.nytimes.com/live/2020/08/04/business/stock-market-today-coronavirus?action=click\&pgtype=Article\&state=default\&region=TOP_BANNER\&context=storylines_menu}{Economy}
\end{itemize}

Advertisement

\protect\hyperlink{after-top}{Continue reading the main story}

Supported by

\protect\hyperlink{after-sponsor}{Continue reading the main story}

\hypertarget{blood-plasma-from-survivors-will-be-given-to-coronavirus-patients}{%
\section{Blood Plasma From Survivors Will Be Given to Coronavirus
Patients}\label{blood-plasma-from-survivors-will-be-given-to-coronavirus-patients}}

In people who have recovered, plasma is teeming with antibodies that may
fight the virus. But the treatment beginning in New York is
experimental.

\includegraphics{https://static01.nyt.com/images/2020/03/26/science/26VIRUS-PLASMA1/merlin_171006354_f51ef70b-96e3-4ca1-a8bd-ff30f6030cb3-articleLarge.jpg?quality=75\&auto=webp\&disable=upscale}

By \href{https://www.nytimes.com/by/denise-grady}{Denise Grady}

\begin{itemize}
\item
  Published March 26, 2020Updated Aug. 4, 2020
\item
  \begin{itemize}
  \item
  \item
  \item
  \item
  \item
  \item
  \end{itemize}
\end{itemize}

Can blood from
\href{https://www.nytimes.com/2020/08/04/health/trump-plasma.html}{coronavirus}
survivors help other people fight the illness?

Doctors in New York will soon be testing the idea in hospitalized
patients who are seriously ill.

\href{https://www.nytimes.com/2020/08/04/health/trump-plasma.html}{Blood}
from people who have recovered can be a rich source of antibodies,
proteins made by the immune system to attack the virus. The part of the
blood that contains antibodies, so-called convalescent plasma, has been
used for decades to treat infectious diseases, including Ebola and
influenza.

``It's kind of difficult scientifically to know how valuable it is in
any disease until you try,'' said Dr. David L. Reich, president and
chief operating officer of the Mount Sinai Hospital, which will be using
the treatment. ``It's not exactly a shot in the dark, but it's not tried
and true.''

Dr. Reich said it would be tried as a treatment for hospitalized
patients who had a moderate form of the disease and had trouble
breathing, but not for those who are in advanced stages of the disease.

``The idea is to get to the right patients at the right time,'' he said.
``But it's experimental.''

Researchers at Mount Sinai were among the first in the United States to
develop a test that can detect antibodies in recovering patients, an
essential part of this treatment strategy.

On Tuesday, the Food and Drug Administration gave permission for the
plasma to be used experimentally on an emergency basis to treat
coronavirus patients, and hospitals in New York quickly began asking to
participate, said Dr. Bruce Sachais, chief medical officer of the New
York Blood Center, which will collect, test and distribute the plasma.

``Our main focus is, how do we implement this quickly to help the
hospitals get product to their patients,'' Dr. Sachais said. ``We have
blood centers in New England, Delaware and the Midwest, so we can do the
same thing in other regions. We're working with other blood centers and
hospitals that may collect their own blood and want to do this. We may
not be able to collect enough plasma in New York to help the entire
country, so we want to share with other centers to help them.''

Dr. Reich said that an email asking Mount Sinai staff members who had
recovered to consider donating plasma went ``a little viral,'' and
quickly drew 2,000 responses.

But volunteers will have to be carefully screened to meet strict
criteria. The donors will include people who tested positive for the
virus when they were ill, recovered, have had no symptoms for 14 days,
now test negative --- and have high levels, also called titers, of
antibodies that fight the virus. Dr. Reich said that because there were
delays and shortages in testing, the number of people who qualify may be
low at first.

\hypertarget{latest-updates-global-coronavirus-outbreak}{%
\section{\texorpdfstring{\href{https://www.nytimes.com/2020/08/04/world/coronavirus-cases.html?action=click\&pgtype=Article\&state=default\&region=MAIN_CONTENT_1\&context=storylines_live_updates}{Latest
Updates: Global Coronavirus
Outbreak}}{Latest Updates: Global Coronavirus Outbreak}}\label{latest-updates-global-coronavirus-outbreak}}

Updated 2020-08-05T04:01:36.184Z

\begin{itemize}
\tightlist
\item
  \href{https://www.nytimes.com/2020/08/04/world/coronavirus-cases.html?action=click\&pgtype=Article\&state=default\&region=MAIN_CONTENT_1\&context=storylines_live_updates\#link-762df92}{As
  talks drag on, McConnell signals openness to jobless aid extension,
  and negotiators agree on a deadline.}
\item
  \href{https://www.nytimes.com/2020/08/04/world/coronavirus-cases.html?action=click\&pgtype=Article\&state=default\&region=MAIN_CONTENT_1\&context=storylines_live_updates\#link-1228a480}{Novavax
  sees encouraging results from two studies of its experimental
  vaccine.}
\item
  \href{https://www.nytimes.com/2020/08/04/world/coronavirus-cases.html?action=click\&pgtype=Article\&state=default\&region=MAIN_CONTENT_1\&context=storylines_live_updates\#link-794484ed}{Mississippians
  must now wear masks in public, governor says.}
\end{itemize}

\href{https://www.nytimes.com/2020/08/04/world/coronavirus-cases.html?action=click\&pgtype=Article\&state=default\&region=MAIN_CONTENT_1\&context=storylines_live_updates}{See
more updates}

More live coverage:
\href{https://www.nytimes.com/live/2020/08/04/business/stock-market-today-coronavirus?action=click\&pgtype=Article\&state=default\&region=MAIN_CONTENT_1\&context=storylines_live_updates}{Markets}

``Our expectation, based on reports from the Chinese experience, is that
most people who get better have high-titer antibodies,'' Dr. Sachais
said. ``Most patients who recover will have good antibodies in a
month.''

\includegraphics{https://static01.nyt.com/images/2020/03/26/science/26VIRUS-PLASMA2/merlin_170554578_9942a545-6a08-43c8-9ae3-d2e3caa4fffc-articleLarge.jpg?quality=75\&auto=webp\&disable=upscale}

People who qualify will then be sent to blood centers to donate plasma.
The procedure, called apheresis, is similar to giving blood, except that
the blood drawn from the patient is run through a machine to extract the
plasma, and the red and white cells are then returned to the donor.
Needles go into both arms: Blood flows out of one arm, passes through
the machine and goes back into the other arm. The process usually takes
60 to 90 minutes, and can yield enough plasma to treat three patients,
Dr. Sachais said.

People who have recovered have antibodies to spare, and removing some
will not endanger the donors or diminish their own resistance to the
virus, Dr. Sachais said. ``We may get rid of 20 percent of their
antibodies, and a couple days later they'll be back.''

The plasma will be tested to make sure it is not carrying infections
like hepatitis or H.I.V., or certain proteins that could set off immune
reactions in the recipient. If it passes the tests, it can then be
frozen, or used right away. Each patient to be treated will receive one
unit, about a cup, which will be dripped in like a blood transfusion. As
with blood transfusions, plasma donors and recipients must have matching
types, but the rules are not the same as those for transfusions.

``We think this is going to be an effective treatment for at least some
patients, but we don't really know yet,'' Dr. Sachais said. ``Hopefully,
we'll get some data in the next few weeks from the first patients, to
see if we're on the right track.''

``In other coronavirus epidemics I don't think we have strong
evidence,'' he said. ``We don't have controlled data. There were reports
from SARS and MERS that patients improved.''

He said the decision to try this approach was based in part on reports
from China that it seemed to help patients. But the reports are not
based on controlled studies or definitive data.

\href{https://www.nytimes.com/news-event/coronavirus?action=click\&pgtype=Article\&state=default\&region=MAIN_CONTENT_3\&context=storylines_faq}{}

\hypertarget{the-coronavirus-outbreak-}{%
\subsubsection{The Coronavirus Outbreak
›}\label{the-coronavirus-outbreak-}}

\hypertarget{frequently-asked-questions}{%
\paragraph{Frequently Asked
Questions}\label{frequently-asked-questions}}

Updated August 4, 2020

\begin{itemize}
\item ~
  \hypertarget{i-have-antibodies-am-i-now-immune}{%
  \paragraph{I have antibodies. Am I now
  immune?}\label{i-have-antibodies-am-i-now-immune}}

  \begin{itemize}
  \tightlist
  \item
    As of right
    now,\href{https://www.nytimes.com/2020/07/22/health/covid-antibodies-herd-immunity.html?action=click\&pgtype=Article\&state=default\&region=MAIN_CONTENT_3\&context=storylines_faq}{that
    seems likely, for at least several months.} There have been
    frightening accounts of people suffering what seems to be a second
    bout of Covid-19. But experts say these patients may have a
    drawn-out course of infection, with the virus taking a slow toll
    weeks to months after initial exposure. People infected with the
    coronavirus typically
    \href{https://www.nature.com/articles/s41586-020-2456-9}{produce}
    immune molecules called antibodies, which are
    \href{https://www.nytimes.com/2020/05/07/health/coronavirus-antibody-prevalence.html?action=click\&pgtype=Article\&state=default\&region=MAIN_CONTENT_3\&context=storylines_faq}{protective
    proteins made in response to an
    infection}\href{https://www.nytimes.com/2020/05/07/health/coronavirus-antibody-prevalence.html?action=click\&pgtype=Article\&state=default\&region=MAIN_CONTENT_3\&context=storylines_faq}{.
    These antibodies may} last in the body
    \href{https://www.nature.com/articles/s41591-020-0965-6}{only two to
    three months}, which may seem worrisome, but that's perfectly normal
    after an acute infection subsides, said Dr. Michael Mina, an
    immunologist at Harvard University. It may be possible to get the
    coronavirus again, but it's highly unlikely that it would be
    possible in a short window of time from initial infection or make
    people sicker the second time.
  \end{itemize}
\item ~
  \hypertarget{im-a-small-business-owner-can-i-get-relief}{%
  \paragraph{I'm a small-business owner. Can I get
  relief?}\label{im-a-small-business-owner-can-i-get-relief}}

  \begin{itemize}
  \tightlist
  \item
    The
    \href{https://www.nytimes.com/article/small-business-loans-stimulus-grants-freelancers-coronavirus.html?action=click\&pgtype=Article\&state=default\&region=MAIN_CONTENT_3\&context=storylines_faq}{stimulus
    bills enacted in March} offer help for the millions of American
    small businesses. Those eligible for aid are businesses and
    nonprofit organizations with fewer than 500 workers, including sole
    proprietorships, independent contractors and freelancers. Some
    larger companies in some industries are also eligible. The help
    being offered, which is being managed by the Small Business
    Administration, includes the Paycheck Protection Program and the
    Economic Injury Disaster Loan program. But lots of folks have
    \href{https://www.nytimes.com/interactive/2020/05/07/business/small-business-loans-coronavirus.html?action=click\&pgtype=Article\&state=default\&region=MAIN_CONTENT_3\&context=storylines_faq}{not
    yet seen payouts.} Even those who have received help are confused:
    The rules are draconian, and some are stuck sitting on
    \href{https://www.nytimes.com/2020/05/02/business/economy/loans-coronavirus-small-business.html?action=click\&pgtype=Article\&state=default\&region=MAIN_CONTENT_3\&context=storylines_faq}{money
    they don't know how to use.} Many small-business owners are getting
    less than they expected or
    \href{https://www.nytimes.com/2020/06/10/business/Small-business-loans-ppp.html?action=click\&pgtype=Article\&state=default\&region=MAIN_CONTENT_3\&context=storylines_faq}{not
    hearing anything at all.}
  \end{itemize}
\item ~
  \hypertarget{what-are-my-rights-if-i-am-worried-about-going-back-to-work}{%
  \paragraph{What are my rights if I am worried about going back to
  work?}\label{what-are-my-rights-if-i-am-worried-about-going-back-to-work}}

  \begin{itemize}
  \tightlist
  \item
    Employers have to provide
    \href{https://www.osha.gov/SLTC/covid-19/standards.html}{a safe
    workplace} with policies that protect everyone equally.
    \href{https://www.nytimes.com/article/coronavirus-money-unemployment.html?action=click\&pgtype=Article\&state=default\&region=MAIN_CONTENT_3\&context=storylines_faq}{And
    if one of your co-workers tests positive for the coronavirus, the
    C.D.C.} has said that
    \href{https://www.cdc.gov/coronavirus/2019-ncov/community/guidance-business-response.html}{employers
    should tell their employees} -\/- without giving you the sick
    employee's name -\/- that they may have been exposed to the virus.
  \end{itemize}
\item ~
  \hypertarget{should-i-refinance-my-mortgage}{%
  \paragraph{Should I refinance my
  mortgage?}\label{should-i-refinance-my-mortgage}}

  \begin{itemize}
  \tightlist
  \item
    \href{https://www.nytimes.com/article/coronavirus-money-unemployment.html?action=click\&pgtype=Article\&state=default\&region=MAIN_CONTENT_3\&context=storylines_faq}{It
    could be a good idea,} because mortgage rates have
    \href{https://www.nytimes.com/2020/07/16/business/mortgage-rates-below-3-percent.html?action=click\&pgtype=Article\&state=default\&region=MAIN_CONTENT_3\&context=storylines_faq}{never
    been lower.} Refinancing requests have pushed mortgage applications
    to some of the highest levels since 2008, so be prepared to get in
    line. But defaults are also up, so if you're thinking about buying a
    home, be aware that some lenders have tightened their standards.
  \end{itemize}
\item ~
  \hypertarget{what-is-school-going-to-look-like-in-september}{%
  \paragraph{What is school going to look like in
  September?}\label{what-is-school-going-to-look-like-in-september}}

  \begin{itemize}
  \tightlist
  \item
    It is unlikely that many schools will return to a normal schedule
    this fall, requiring the grind of
    \href{https://www.nytimes.com/2020/06/05/us/coronavirus-education-lost-learning.html?action=click\&pgtype=Article\&state=default\&region=MAIN_CONTENT_3\&context=storylines_faq}{online
    learning},
    \href{https://www.nytimes.com/2020/05/29/us/coronavirus-child-care-centers.html?action=click\&pgtype=Article\&state=default\&region=MAIN_CONTENT_3\&context=storylines_faq}{makeshift
    child care} and
    \href{https://www.nytimes.com/2020/06/03/business/economy/coronavirus-working-women.html?action=click\&pgtype=Article\&state=default\&region=MAIN_CONTENT_3\&context=storylines_faq}{stunted
    workdays} to continue. California's two largest public school
    districts --- Los Angeles and San Diego --- said on July 13, that
    \href{https://www.nytimes.com/2020/07/13/us/lausd-san-diego-school-reopening.html?action=click\&pgtype=Article\&state=default\&region=MAIN_CONTENT_3\&context=storylines_faq}{instruction
    will be remote-only in the fall}, citing concerns that surging
    coronavirus infections in their areas pose too dire a risk for
    students and teachers. Together, the two districts enroll some
    825,000 students. They are the largest in the country so far to
    abandon plans for even a partial physical return to classrooms when
    they reopen in August. For other districts, the solution won't be an
    all-or-nothing approach.
    \href{https://bioethics.jhu.edu/research-and-outreach/projects/eschool-initiative/school-policy-tracker/}{Many
    systems}, including the nation's largest, New York City, are
    devising
    \href{https://www.nytimes.com/2020/06/26/us/coronavirus-schools-reopen-fall.html?action=click\&pgtype=Article\&state=default\&region=MAIN_CONTENT_3\&context=storylines_faq}{hybrid
    plans} that involve spending some days in classrooms and other days
    online. There's no national policy on this yet, so check with your
    municipal school system regularly to see what is happening in your
    community.
  \end{itemize}
\end{itemize}

Dr. Sachais said an article in a journal that was not peer-reviewed
described treating 10 patients in China with one unit each of
convalescent plasma, and said it appeared safe and seemed to quickly
lower their virus levels.

``It's anecdotal,'' he said.

A researcher not associated with the new treatment plans said there was
evidence to support using plasma from survivors.

``Four to six or eight weeks after infection, their blood should be full
of antibodies that will neutralize the virus and that will theoretically
limit the infection,'' said Vineet Menachery, a virologist at the
University of Texas Medical Branch.

In studies in mice, he said, ``If you can drive the virus replication
down tenfold to hundredfold, that can be the difference between life and
death.''

He described the use of convalescent plasma as ``a classic approach that
is a really effective way to treat'' --- if there are enough donors with
enough of the right antibodies.

A potential risk, he said, is that the patient's immune system could
react against something in the plasma, and cause additional illness.

Although hospitals will gather information about the patients being
treated, the procedure is not being done as part of a clinical trial.
There will not be a placebo group or the other measures needed to
determine whether a treatment works.

``People are so desperately ill now, it isn't the right time,'' Dr.
Reich said. ``They're in the hospital, they're sick, in intensive care,
on ventilators. Some get sick so quickly, and it's such a severe illness
in some people, we feel it's not the right moment.''

He said the doctors were relying on science and evidence as much as
possible.

But he added: ``You see this steamroller coming at you, and you don't
want to sit there passively and let it roll over you. So you put
together everything you have to try to fight it. This has the potential
to help and also the potential to harm, but we just won't know until
it's later in the process of the disease and people have had an
opportunity to try different things.''

Survivors seem eager to help.

``We're getting a lot of requests,'' Dr. Sachais said. ``One center sent
a survey to patients who are getting better, and there were hundreds of
responses saying they were interested in being donors. This is going to
bring people together. People who've survived will want to do something
for their fellow New Yorkers.''

Advertisement

\protect\hyperlink{after-bottom}{Continue reading the main story}

\hypertarget{site-index}{%
\subsection{Site Index}\label{site-index}}

\hypertarget{site-information-navigation}{%
\subsection{Site Information
Navigation}\label{site-information-navigation}}

\begin{itemize}
\tightlist
\item
  \href{https://help.nytimes.com/hc/en-us/articles/115014792127-Copyright-notice}{©~2020~The
  New York Times Company}
\end{itemize}

\begin{itemize}
\tightlist
\item
  \href{https://www.nytco.com/}{NYTCo}
\item
  \href{https://help.nytimes.com/hc/en-us/articles/115015385887-Contact-Us}{Contact
  Us}
\item
  \href{https://www.nytco.com/careers/}{Work with us}
\item
  \href{https://nytmediakit.com/}{Advertise}
\item
  \href{http://www.tbrandstudio.com/}{T Brand Studio}
\item
  \href{https://www.nytimes.com/privacy/cookie-policy\#how-do-i-manage-trackers}{Your
  Ad Choices}
\item
  \href{https://www.nytimes.com/privacy}{Privacy}
\item
  \href{https://help.nytimes.com/hc/en-us/articles/115014893428-Terms-of-service}{Terms
  of Service}
\item
  \href{https://help.nytimes.com/hc/en-us/articles/115014893968-Terms-of-sale}{Terms
  of Sale}
\item
  \href{https://spiderbites.nytimes.com}{Site Map}
\item
  \href{https://help.nytimes.com/hc/en-us}{Help}
\item
  \href{https://www.nytimes.com/subscription?campaignId=37WXW}{Subscriptions}
\end{itemize}
