Sections

SEARCH

\protect\hyperlink{site-content}{Skip to
content}\protect\hyperlink{site-index}{Skip to site index}

\href{https://www.nytimes.com/section/world/africa}{Africa}

\href{https://myaccount.nytimes.com/auth/login?response_type=cookie\&client_id=vi}{}

\href{https://www.nytimes.com/section/todayspaper}{Today's Paper}

\href{/section/world/africa}{Africa}\textbar{}With Most Coronavirus
Cases in Africa, South Africa Locks Down

\url{https://nyti.ms/2UDFw2x}

\begin{itemize}
\item
\item
\item
\item
\item
\end{itemize}

\href{https://www.nytimes.com/news-event/coronavirus?action=click\&pgtype=Article\&state=default\&region=TOP_BANNER\&context=storylines_menu}{The
Coronavirus Outbreak}

\begin{itemize}
\tightlist
\item
  live\href{https://www.nytimes.com/2020/08/01/world/coronavirus-covid-19.html?action=click\&pgtype=Article\&state=default\&region=TOP_BANNER\&context=storylines_menu}{Latest
  Updates}
\item
  \href{https://www.nytimes.com/interactive/2020/us/coronavirus-us-cases.html?action=click\&pgtype=Article\&state=default\&region=TOP_BANNER\&context=storylines_menu}{Maps
  and Cases}
\item
  \href{https://www.nytimes.com/interactive/2020/science/coronavirus-vaccine-tracker.html?action=click\&pgtype=Article\&state=default\&region=TOP_BANNER\&context=storylines_menu}{Vaccine
  Tracker}
\item
  \href{https://www.nytimes.com/interactive/2020/07/29/us/schools-reopening-coronavirus.html?action=click\&pgtype=Article\&state=default\&region=TOP_BANNER\&context=storylines_menu}{What
  School May Look Like}
\item
  \href{https://www.nytimes.com/live/2020/07/31/business/stock-market-today-coronavirus?action=click\&pgtype=Article\&state=default\&region=TOP_BANNER\&context=storylines_menu}{Economy}
\end{itemize}

Advertisement

\protect\hyperlink{after-top}{Continue reading the main story}

Supported by

\protect\hyperlink{after-sponsor}{Continue reading the main story}

\hypertarget{with-most-coronavirus-cases-in-africa-south-africa-locks-down}{%
\section{With Most Coronavirus Cases in Africa, South Africa Locks
Down}\label{with-most-coronavirus-cases-in-africa-south-africa-locks-down}}

South Africa is now the epicenter of the pandemic in Africa, with more
than 1,000 confirmed cases across the country's nine provinces.

\includegraphics{https://static01.nyt.com/images/2020/03/27/world/27virus-southafrica/merlin_171021426_db3a27c6-9c30-4fca-b23a-90510cfa0b6e-articleLarge.jpg?quality=75\&auto=webp\&disable=upscale}

By Lynsey Chutel and
\href{https://www.nytimes.com/by/abdi-latif-dahir}{Abdi Latif Dahir}

\begin{itemize}
\item
  Published March 27, 2020Updated June 29, 2020
\item
  \begin{itemize}
  \item
  \item
  \item
  \item
  \item
  \end{itemize}
\end{itemize}

JOHANNESBURG, South Africa --- When the clock struck midnight on Friday,
South Africa,
\href{https://www.nytimes.com/2020/06/29/world/africa/Africa-middle-class-coronavirus.html}{Africa}'s
most industrialized nation, ordered most of its 59 million people to
stay at home for three weeks --- the biggest and most restrictive action
in the African continent to contain the spread of the coronavirus.

The lockdown was precipitated by an alarming increase in confirmed
coronavirus cases across the nation's nine provinces. Three weeks after
the first infection was discovered in South Africa, the country is now
the epicenter of the outbreak in the continent,
\href{https://twitter.com/nicd_sa/status/1243438789692919808}{with more
than 1,000 confirmed cases}, double the cases in Egypt.

In Johannesburg, the biggest city, shops and offices were shuttered in
observance of the lockdown, announced on Tuesday. A few delivery trucks,
minibus taxis and ambulances drove through roads normally clogged with
rush-hour traffic.

\includegraphics{https://static01.nyt.com/images/2020/03/27/world/27safrica2/merlin_171021855_63c4db00-2bd7-42c0-bf76-0dd4521305f7-articleLarge.jpg?quality=75\&auto=webp\&disable=upscale}

``People didn't have enough time to prepare,'' said Dineo Mafoho, 25,
sitting outside a taxi stand trying to get home to Diepsloot, a township
in the city's outskirts.

As a cleaner, she's considered essential personnel, and so allowed to be
out. Wearing pink lipstick, but not the face mask or gloves that
essential workers have been asked to wear, she said she ``just can't get
used to it.''

While the deadly virus
\href{https://www.nytimes.com/2020/02/06/world/africa/africa-coronavirus-china.html}{was
slow to take hold} across Africa, the number of confirmed coronavirus
cases and deaths there
\href{https://www.nytimes.com/2020/02/28/world/africa/nigeria-coronavirus.html}{has
gradually increased} in recent days,
\href{https://www.nytimes.com/2020/03/17/world/africa/coronavirus-africa-burkina-faso.html}{raising
fears} about the continent's readiness to deal with a pandemic.

\hypertarget{latest-updates-global-coronavirus-outbreak}{%
\section{\texorpdfstring{\href{https://www.nytimes.com/2020/08/01/world/coronavirus-covid-19.html?action=click\&pgtype=Article\&state=default\&region=MAIN_CONTENT_1\&context=storylines_live_updates}{Latest
Updates: Global Coronavirus
Outbreak}}{Latest Updates: Global Coronavirus Outbreak}}\label{latest-updates-global-coronavirus-outbreak}}

Updated 2020-08-02T06:58:18.835Z

\begin{itemize}
\tightlist
\item
  \href{https://www.nytimes.com/2020/08/01/world/coronavirus-covid-19.html?action=click\&pgtype=Article\&state=default\&region=MAIN_CONTENT_1\&context=storylines_live_updates\#link-34047410}{The
  U.S. reels as July cases more than double the total of any other
  month.}
\item
  \href{https://www.nytimes.com/2020/08/01/world/coronavirus-covid-19.html?action=click\&pgtype=Article\&state=default\&region=MAIN_CONTENT_1\&context=storylines_live_updates\#link-780ec966}{Top
  U.S. officials work to break an impasse over the federal jobless
  benefit.}
\item
  \href{https://www.nytimes.com/2020/08/01/world/coronavirus-covid-19.html?action=click\&pgtype=Article\&state=default\&region=MAIN_CONTENT_1\&context=storylines_live_updates\#link-2bc8948}{Its
  outbreak untamed, Melbourne goes into even greater lockdown.}
\end{itemize}

\href{https://www.nytimes.com/2020/08/01/world/coronavirus-covid-19.html?action=click\&pgtype=Article\&state=default\&region=MAIN_CONTENT_1\&context=storylines_live_updates}{See
more updates}

More live coverage:
\href{https://www.nytimes.com/live/2020/07/31/business/stock-market-today-coronavirus?action=click\&pgtype=Article\&state=default\&region=MAIN_CONTENT_1\&context=storylines_live_updates}{Markets}

To date, 46 African states
\href{https://twitter.com/AfricaCDC/status/1243555244572844037}{have
reported} a total of 3,426 positive cases and 94 deaths, according to
the Africa Centers for Disease Control and Prevention. Besides South
Africa and Egypt, the countries of Algeria, Morocco, Tunisia, Burkina
Faso, Ghana and Senegal have all reported over 100 cases, mostly
imported by visitors from Europe.

So far the virus has spread fastest in some of Africa's most
economically developed countries, like South Africa and Egypt, which
have more air connections and commerce with Europe and China, and have
the capacity to do the testing to confirm positive cases.

The spike in numbers has pushed other African countries to also
undertake strict measures. Kenya, Egypt, and Senegal have imposed
overnight curfews; Uganda has restricted visitors from high-risk
countries; and Rwanda has
\href{https://twitter.com/PrimatureRwanda/status/1241412264193937412}{banned
inter-country travel}.

In Zimbabwe, nurses in state hospitals walked off their jobs for lack of
protective equipment even as the southern African state was shaken by
its first death from the virus,
\href{https://www.cnn.com/2020/03/23/africa/zimbabwe-broadcaster-dies-coronavirus/index.html}{a
prominent television journalist}.

In Burkina Faso, five government ministers and two ambassadors ---
including the American ambassador, Andrew Young --- tested positive for
coronavirus. In the Democratic Republic of Congo, a senior aide to
President Felix Tshisekedi died of the virus this week.

South Africa is one of the world's most unequal societies, with millions
of people living in cramped, unhygienic conditions in townships with no
clean water or public health care. For many of these people, the
lockdown will impose great hardships.

Image

Streets across the country were largely silent.Credit...Kim
Ludbrook/EPA, via Shutterstock

In informal settlements and rural areas, residents usually have to stand
close to one another to collect water or queue to use shared latrines,
making it difficult to maintain a physical distance, said Alana Potter,
director of research and advocacy at the nonprofit Socio-Economic Rights
Institute of South Africa.

Also, the vast majority of poor people, she said, generate their
livelihoods in an informal economy. Under lockdown, ``street vendors
can't trade, which will destroy their livelihoods --- and low-income
households that rely on vendors for food supply will now have to pay
more to access food,'' she said.

South Africa also has
\href{https://apps.who.int/iris/bitstream/handle/10665/255007/ccs_zaf_2016_2020.pdf?sequence=1}{a
significant percentage of its population} living with chronic,
underlying conditions including H.I.V., tuberculosis, diabetes, and
asthma --- putting them at risk of developing serious complications from
Covid-19.

``South Africa's medical system is overburdened even in normal times,''
said Atiya Mosam, a medical doctor and co-founder of Public Health
Action Team, a group of doctors working to improve South Africa's health
care system.

``If the virus spreads like it has in China or Italy or the United
States,'' she continued, ``it's going to be very difficult for South
Africa to respond. We cannot afford that.''

\href{https://www.nytimes.com/news-event/coronavirus?action=click\&pgtype=Article\&state=default\&region=MAIN_CONTENT_3\&context=storylines_faq}{}

\hypertarget{the-coronavirus-outbreak-}{%
\subsubsection{The Coronavirus Outbreak
›}\label{the-coronavirus-outbreak-}}

\hypertarget{frequently-asked-questions}{%
\paragraph{Frequently Asked
Questions}\label{frequently-asked-questions}}

Updated July 27, 2020

\begin{itemize}
\item ~
  \hypertarget{should-i-refinance-my-mortgage}{%
  \paragraph{Should I refinance my
  mortgage?}\label{should-i-refinance-my-mortgage}}

  \begin{itemize}
  \tightlist
  \item
    \href{https://www.nytimes.com/article/coronavirus-money-unemployment.html?action=click\&pgtype=Article\&state=default\&region=MAIN_CONTENT_3\&context=storylines_faq}{It
    could be a good idea,} because mortgage rates have
    \href{https://www.nytimes.com/2020/07/16/business/mortgage-rates-below-3-percent.html?action=click\&pgtype=Article\&state=default\&region=MAIN_CONTENT_3\&context=storylines_faq}{never
    been lower.} Refinancing requests have pushed mortgage applications
    to some of the highest levels since 2008, so be prepared to get in
    line. But defaults are also up, so if you're thinking about buying a
    home, be aware that some lenders have tightened their standards.
  \end{itemize}
\item ~
  \hypertarget{what-is-school-going-to-look-like-in-september}{%
  \paragraph{What is school going to look like in
  September?}\label{what-is-school-going-to-look-like-in-september}}

  \begin{itemize}
  \tightlist
  \item
    It is unlikely that many schools will return to a normal schedule
    this fall, requiring the grind of
    \href{https://www.nytimes.com/2020/06/05/us/coronavirus-education-lost-learning.html?action=click\&pgtype=Article\&state=default\&region=MAIN_CONTENT_3\&context=storylines_faq}{online
    learning},
    \href{https://www.nytimes.com/2020/05/29/us/coronavirus-child-care-centers.html?action=click\&pgtype=Article\&state=default\&region=MAIN_CONTENT_3\&context=storylines_faq}{makeshift
    child care} and
    \href{https://www.nytimes.com/2020/06/03/business/economy/coronavirus-working-women.html?action=click\&pgtype=Article\&state=default\&region=MAIN_CONTENT_3\&context=storylines_faq}{stunted
    workdays} to continue. California's two largest public school
    districts --- Los Angeles and San Diego --- said on July 13, that
    \href{https://www.nytimes.com/2020/07/13/us/lausd-san-diego-school-reopening.html?action=click\&pgtype=Article\&state=default\&region=MAIN_CONTENT_3\&context=storylines_faq}{instruction
    will be remote-only in the fall}, citing concerns that surging
    coronavirus infections in their areas pose too dire a risk for
    students and teachers. Together, the two districts enroll some
    825,000 students. They are the largest in the country so far to
    abandon plans for even a partial physical return to classrooms when
    they reopen in August. For other districts, the solution won't be an
    all-or-nothing approach.
    \href{https://bioethics.jhu.edu/research-and-outreach/projects/eschool-initiative/school-policy-tracker/}{Many
    systems}, including the nation's largest, New York City, are
    devising
    \href{https://www.nytimes.com/2020/06/26/us/coronavirus-schools-reopen-fall.html?action=click\&pgtype=Article\&state=default\&region=MAIN_CONTENT_3\&context=storylines_faq}{hybrid
    plans} that involve spending some days in classrooms and other days
    online. There's no national policy on this yet, so check with your
    municipal school system regularly to see what is happening in your
    community.
  \end{itemize}
\item ~
  \hypertarget{is-the-coronavirus-airborne}{%
  \paragraph{Is the coronavirus
  airborne?}\label{is-the-coronavirus-airborne}}

  \begin{itemize}
  \tightlist
  \item
    The coronavirus
    \href{https://www.nytimes.com/2020/07/04/health/239-experts-with-one-big-claim-the-coronavirus-is-airborne.html?action=click\&pgtype=Article\&state=default\&region=MAIN_CONTENT_3\&context=storylines_faq}{can
    stay aloft for hours in tiny droplets in stagnant air}, infecting
    people as they inhale, mounting scientific evidence suggests. This
    risk is highest in crowded indoor spaces with poor ventilation, and
    may help explain super-spreading events reported in meatpacking
    plants, churches and restaurants.
    \href{https://www.nytimes.com/2020/07/06/health/coronavirus-airborne-aerosols.html?action=click\&pgtype=Article\&state=default\&region=MAIN_CONTENT_3\&context=storylines_faq}{It's
    unclear how often the virus is spread} via these tiny droplets, or
    aerosols, compared with larger droplets that are expelled when a
    sick person coughs or sneezes, or transmitted through contact with
    contaminated surfaces, said Linsey Marr, an aerosol expert at
    Virginia Tech. Aerosols are released even when a person without
    symptoms exhales, talks or sings, according to Dr. Marr and more
    than 200 other experts, who
    \href{https://academic.oup.com/cid/article/doi/10.1093/cid/ciaa939/5867798}{have
    outlined the evidence in an open letter to the World Health
    Organization}.
  \end{itemize}
\item ~
  \hypertarget{what-are-the-symptoms-of-coronavirus}{%
  \paragraph{What are the symptoms of
  coronavirus?}\label{what-are-the-symptoms-of-coronavirus}}

  \begin{itemize}
  \tightlist
  \item
    Common symptoms
    \href{https://www.nytimes.com/article/symptoms-coronavirus.html?action=click\&pgtype=Article\&state=default\&region=MAIN_CONTENT_3\&context=storylines_faq}{include
    fever, a dry cough, fatigue and difficulty breathing or shortness of
    breath.} Some of these symptoms overlap with those of the flu,
    making detection difficult, but runny noses and stuffy sinuses are
    less common.
    \href{https://www.nytimes.com/2020/04/27/health/coronavirus-symptoms-cdc.html?action=click\&pgtype=Article\&state=default\&region=MAIN_CONTENT_3\&context=storylines_faq}{The
    C.D.C. has also} added chills, muscle pain, sore throat, headache
    and a new loss of the sense of taste or smell as symptoms to look
    out for. Most people fall ill five to seven days after exposure, but
    symptoms may appear in as few as two days or as many as 14 days.
  \end{itemize}
\item ~
  \hypertarget{does-asymptomatic-transmission-of-covid-19-happen}{%
  \paragraph{Does asymptomatic transmission of Covid-19
  happen?}\label{does-asymptomatic-transmission-of-covid-19-happen}}

  \begin{itemize}
  \tightlist
  \item
    So far, the evidence seems to show it does. A widely cited
    \href{https://www.nature.com/articles/s41591-020-0869-5}{paper}
    published in April suggests that people are most infectious about
    two days before the onset of coronavirus symptoms and estimated that
    44 percent of new infections were a result of transmission from
    people who were not yet showing symptoms. Recently, a top expert at
    the World Health Organization stated that transmission of the
    coronavirus by people who did not have symptoms was ``very rare,''
    \href{https://www.nytimes.com/2020/06/09/world/coronavirus-updates.html?action=click\&pgtype=Article\&state=default\&region=MAIN_CONTENT_3\&context=storylines_faq\#link-1f302e21}{but
    she later walked back that statement.}
  \end{itemize}
\end{itemize}

In
\href{http://www.thepresidency.gov.za/speeches/statement-president-cyril-ramaphosa-escalation-measures-combat-covid-19-epidemic\%2C-union}{announcing
the lockdown}, President Cyril Ramaphosa of South Africa said the
measures were aimed at preventing ``a human catastrophe of enormous
proportions.'' And although he acknowledged that they would affect the
South African economy, he said ``the human cost of delaying this action
would be far, far greater.''

Ronak Gopaldas, director of the Cape Town-based consultancy Signal Risk,
said that in general, ``Coronavirus will undoubtedly have a
contractionary impact on what is a stagnant
economy''\href{https://www.businessinsider.co.za/south-african-economy-in-deep-trouble-2020-2}{in
South Africa.}

The country, he said, will particularly be affected
\href{https://www.nytimes.com/2020/03/16/business/coronavirus-china-economy.html}{by
the economic slowdown in China}, the country's largest trading partner.
Diminished demand from China, he said, will likely drive down exports,
affecting sectors from mining and manufacturing to tourism.

Image

People were arrested for defying the lockdown order in
Johannesburg.Credit...Michele Spatari/Agence France-Presse --- Getty
Images

Across South Africa, people had been bracing for the lockdown. ​Some had
piled shopping carts high with bottles of beer and wine, preparing for a
much-debated feature of the lockdown --- a ban on the sale of alcohol
and tobacco. Anyone defying the ban would face a penalty, the
authorities said.

In Johannesburg on Thursday, hours before the lockdown took effect, a
line stretched outside Makro, a wholesale store. Tshidi Molubi, a
51-year-old resident of the Soweto neighborhood, joined the queue before
the store opened at 9 a.m.

She was laid off from a bank a few months earlier, and said she was
using her savings to buy essentials like rice, flour and eggs.

``If you can't go out, at least we can make a dumpling,'' Ms. Molubi
said.

Akhona Makasi, a 35-year-old freelancer in the film industry, left
Johannesburg on Wednesday to visit her grandparents in the Eastern Cape
province. But she said she had rushed home ``without calculating the
risk.''

Few people wore masks on the journey home and when she used hand
sanitizer and disinfected her seat, commuters in the packed bus
complained about the smell.

At home, her grandparents refused to self-isolate or ask that visitors
sanitize their hands. Villagers gathered for a funeral and slaughtered a
cow.

``If I had a basic income, I would have stayed in Johannesburg and
self-isolated, not risking my grandparents' lives,'' she said.

Lynsey Chutel reported from Johannesburg, and Abdi Latif Dahir from
Nairobi, Kenya. Ruth Maclean contributed reporting from Dakar, Senegal.

Advertisement

\protect\hyperlink{after-bottom}{Continue reading the main story}

\hypertarget{site-index}{%
\subsection{Site Index}\label{site-index}}

\hypertarget{site-information-navigation}{%
\subsection{Site Information
Navigation}\label{site-information-navigation}}

\begin{itemize}
\tightlist
\item
  \href{https://help.nytimes.com/hc/en-us/articles/115014792127-Copyright-notice}{©~2020~The
  New York Times Company}
\end{itemize}

\begin{itemize}
\tightlist
\item
  \href{https://www.nytco.com/}{NYTCo}
\item
  \href{https://help.nytimes.com/hc/en-us/articles/115015385887-Contact-Us}{Contact
  Us}
\item
  \href{https://www.nytco.com/careers/}{Work with us}
\item
  \href{https://nytmediakit.com/}{Advertise}
\item
  \href{http://www.tbrandstudio.com/}{T Brand Studio}
\item
  \href{https://www.nytimes.com/privacy/cookie-policy\#how-do-i-manage-trackers}{Your
  Ad Choices}
\item
  \href{https://www.nytimes.com/privacy}{Privacy}
\item
  \href{https://help.nytimes.com/hc/en-us/articles/115014893428-Terms-of-service}{Terms
  of Service}
\item
  \href{https://help.nytimes.com/hc/en-us/articles/115014893968-Terms-of-sale}{Terms
  of Sale}
\item
  \href{https://spiderbites.nytimes.com}{Site Map}
\item
  \href{https://help.nytimes.com/hc/en-us}{Help}
\item
  \href{https://www.nytimes.com/subscription?campaignId=37WXW}{Subscriptions}
\end{itemize}
