Sections

SEARCH

\protect\hyperlink{site-content}{Skip to
content}\protect\hyperlink{site-index}{Skip to site index}

\href{https://www.nytimes.com/section/us}{U.S.}

\href{https://myaccount.nytimes.com/auth/login?response_type=cookie\&client_id=vi}{}

\href{https://www.nytimes.com/section/todayspaper}{Today's Paper}

\href{/section/us}{U.S.}\textbar{}Rhode Island Pulls Over New Yorkers to
Keep the Virus at Bay

\url{https://nyti.ms/3bwvBSZ}

\begin{itemize}
\item
\item
\item
\item
\item
\end{itemize}

\href{https://www.nytimes.com/news-event/coronavirus?action=click\&pgtype=Article\&state=default\&region=TOP_BANNER\&context=storylines_menu}{The
Coronavirus Outbreak}

\begin{itemize}
\tightlist
\item
  live\href{https://www.nytimes.com/2020/08/01/world/coronavirus-covid-19.html?action=click\&pgtype=Article\&state=default\&region=TOP_BANNER\&context=storylines_menu}{Latest
  Updates}
\item
  \href{https://www.nytimes.com/interactive/2020/us/coronavirus-us-cases.html?action=click\&pgtype=Article\&state=default\&region=TOP_BANNER\&context=storylines_menu}{Maps
  and Cases}
\item
  \href{https://www.nytimes.com/interactive/2020/science/coronavirus-vaccine-tracker.html?action=click\&pgtype=Article\&state=default\&region=TOP_BANNER\&context=storylines_menu}{Vaccine
  Tracker}
\item
  \href{https://www.nytimes.com/interactive/2020/07/29/us/schools-reopening-coronavirus.html?action=click\&pgtype=Article\&state=default\&region=TOP_BANNER\&context=storylines_menu}{What
  School May Look Like}
\item
  \href{https://www.nytimes.com/live/2020/07/31/business/stock-market-today-coronavirus?action=click\&pgtype=Article\&state=default\&region=TOP_BANNER\&context=storylines_menu}{Economy}
\end{itemize}

Advertisement

\protect\hyperlink{after-top}{Continue reading the main story}

Supported by

\protect\hyperlink{after-sponsor}{Continue reading the main story}

\hypertarget{rhode-island-pulls-over-new-yorkers-to-keep-the-virus-at-bay}{%
\section{Rhode Island Pulls Over New Yorkers to Keep the Virus at
Bay}\label{rhode-island-pulls-over-new-yorkers-to-keep-the-virus-at-bay}}

Police and National Guard troops are deployed to warn visitors they must
self-quarantine, as local resentment builds toward people fleeing
coronavirus hot spots.

\includegraphics{https://static01.nyt.com/images/2020/03/28/us/28virus-rhodeisland-1/merlin_171069546_9a2d030e-17ac-4ee8-99cb-ddc6c72746ea-articleLarge.jpg?quality=75\&auto=webp\&disable=upscale}

By \href{https://www.nytimes.com/by/nicholas-bogel-burroughs}{Nicholas
Bogel-Burroughs}

\begin{itemize}
\item
  Published March 28, 2020Updated April 10, 2020
\item
  \begin{itemize}
  \item
  \item
  \item
  \item
  \item
  \end{itemize}
\end{itemize}

Police officers were watching Rhode Island's highways, bridges and bus
stops. National Guard troops were trudging through resort towns with
clipboards, knocking on doors. They were all hunting for
\href{https://www.nytimes.com/2020/04/10/nyregion/coronavirus-second-homes-travel.html}{fleeing
New Yorkers} and their telltale Empire State license plates.

The states are increasingly finding themselves pitted against one
another, as they bid for scarce medical equipment, angle for federal aid
and
\href{https://www.nytimes.com/2020/03/25/us/coronavirus-united-states.html}{demand
that nonresidents self-quarantine}.

Few, though, have gone yet to the lengths Rhode Island has to try to
keep the potentially infected at bay, especially those from
\href{https://www.nytimes.com/2020/04/10/nyregion/coronavirus-second-homes-travel.html}{New
York}, the biggest hot spot in the country.

Gov. Gina Raimondo, a Democrat, said on Friday that if New Yorkers
entering the state did not quarantine themselves for 14 days after
arrival, they would be fined and, if they continued to flout the order,
arrested.

``That's a law ---~that's an order,'' Ms. Raimondo said. ``It comes with
penalties. It's not a suggestion.''

She sent troops and police officers to the main highways entering the
state, as well as to Amtrak stations and the main airport, to stop and
warn people coming from New York State about the quarantine order. And
on Saturday, she extended the order to cover travelers from any state.

She was not alone in establishing checkpoints to stop travelers from
places where the virus is widespread. Gov. Ron DeSantis of Florida, a
Republican, said on Saturday that his administration would try to set up
a checkpoint on Interstate 95 near the Georgia border to stop drivers
from the New York area and tell them they had to quarantine themselves
for two weeks.

\includegraphics{https://static01.nyt.com/images/2020/03/28/us/28virus-rhodeisland-2/merlin_171071865_2e78ad42-459e-4d7a-b863-409cc40234bf-articleLarge.jpg?quality=75\&auto=webp\&disable=upscale}

``It's not fair to the people of Florida'' that outsiders keep coming
in, Governor DeSantis said. ``It would make it a lot easier if we didn't
have folks coming in from hot zones where they may be very well carrying
the virus.''

On Friday, Mr. DeSantis extended his quarantine order to cover travelers
from Louisiana as well, and authorized highway checkpoints in the
Panhandle to intercept and warn them.

\hypertarget{latest-updates-global-coronavirus-outbreak}{%
\section{\texorpdfstring{\href{https://www.nytimes.com/2020/08/01/world/coronavirus-covid-19.html?action=click\&pgtype=Article\&state=default\&region=MAIN_CONTENT_1\&context=storylines_live_updates}{Latest
Updates: Global Coronavirus
Outbreak}}{Latest Updates: Global Coronavirus Outbreak}}\label{latest-updates-global-coronavirus-outbreak}}

Updated 2020-08-02T10:04:29.623Z

\begin{itemize}
\tightlist
\item
  \href{https://www.nytimes.com/2020/08/01/world/coronavirus-covid-19.html?action=click\&pgtype=Article\&state=default\&region=MAIN_CONTENT_1\&context=storylines_live_updates\#link-34047410}{The
  U.S. reels as July cases more than double the total of any other
  month.}
\item
  \href{https://www.nytimes.com/2020/08/01/world/coronavirus-covid-19.html?action=click\&pgtype=Article\&state=default\&region=MAIN_CONTENT_1\&context=storylines_live_updates\#link-780ec966}{Top
  U.S. officials work to break an impasse over the federal jobless
  benefit.}
\item
  \href{https://www.nytimes.com/2020/08/01/world/coronavirus-covid-19.html?action=click\&pgtype=Article\&state=default\&region=MAIN_CONTENT_1\&context=storylines_live_updates\#link-2bc8948}{Its
  outbreak untamed, Melbourne goes into even greater lockdown.}
\end{itemize}

\href{https://www.nytimes.com/2020/08/01/world/coronavirus-covid-19.html?action=click\&pgtype=Article\&state=default\&region=MAIN_CONTENT_1\&context=storylines_live_updates}{See
more updates}

More live coverage:
\href{https://www.nytimes.com/live/2020/07/31/business/stock-market-today-coronavirus?action=click\&pgtype=Article\&state=default\&region=MAIN_CONTENT_1\&context=storylines_live_updates}{Markets}

President Trump said on Saturday that concerns raised by states like
Rhode Island and Florida about travelers from New York City had prompted
him to consider imposing a federal quarantine of New York, New Jersey
and part of Connecticut. He later backed off the idea.

Legal experts said that states were on shaky ground pulling people over
just for their license plates. And Gov. Andrew M. Cuomo said on Saturday
that Rhode Island's action was ``at the point of absurdity.''

``If they don't roll back that policy, I'm going to sue Rhode Island,
because that clearly is unconstitutional,'' Governor Cuomo said on CNN,
though he added that he didn't think it would come to that. ``We'll work
it out amicably, I'm sure,'' he said.

Even so, Rhode Island's measures were welcomed by many ``year-rounders''
who live in the state's summer resort communities near the Connecticut
border. They have been growing increasingly frustrated with an influx of
New Yorkers fleeing the city to second homes and rental properties in
the area, and possibly bringing the virus with them.

Local residents have been posting videos and photographs in local
Facebook groups of cars with New York tags being pulled over.

John Austin witnessed a stop on Friday in Westerly, a town of about
23,000 in the corner of the state nearest to New York, when a trooper's
flashing lights followed a driver into the parking lot of Sandy's Fine
Food Emporium, where Mr. Austin is the store manager.

``It is live, and it is happening,'' Mr. Austin said. ``It's happening
in our backyard.''

He said that police officers were stationed all along Route 78 from the
state line at the Pawcatuck River~to his store. ``It's not easy for any
one of us,'' he said. ``Let's pray for a resolution quick.''

\href{https://www.nytimes.com/news-event/coronavirus?action=click\&pgtype=Article\&state=default\&region=MAIN_CONTENT_3\&context=storylines_faq}{}

\hypertarget{the-coronavirus-outbreak-}{%
\subsubsection{The Coronavirus Outbreak
›}\label{the-coronavirus-outbreak-}}

\hypertarget{frequently-asked-questions}{%
\paragraph{Frequently Asked
Questions}\label{frequently-asked-questions}}

Updated July 27, 2020

\begin{itemize}
\item ~
  \hypertarget{should-i-refinance-my-mortgage}{%
  \paragraph{Should I refinance my
  mortgage?}\label{should-i-refinance-my-mortgage}}

  \begin{itemize}
  \tightlist
  \item
    \href{https://www.nytimes.com/article/coronavirus-money-unemployment.html?action=click\&pgtype=Article\&state=default\&region=MAIN_CONTENT_3\&context=storylines_faq}{It
    could be a good idea,} because mortgage rates have
    \href{https://www.nytimes.com/2020/07/16/business/mortgage-rates-below-3-percent.html?action=click\&pgtype=Article\&state=default\&region=MAIN_CONTENT_3\&context=storylines_faq}{never
    been lower.} Refinancing requests have pushed mortgage applications
    to some of the highest levels since 2008, so be prepared to get in
    line. But defaults are also up, so if you're thinking about buying a
    home, be aware that some lenders have tightened their standards.
  \end{itemize}
\item ~
  \hypertarget{what-is-school-going-to-look-like-in-september}{%
  \paragraph{What is school going to look like in
  September?}\label{what-is-school-going-to-look-like-in-september}}

  \begin{itemize}
  \tightlist
  \item
    It is unlikely that many schools will return to a normal schedule
    this fall, requiring the grind of
    \href{https://www.nytimes.com/2020/06/05/us/coronavirus-education-lost-learning.html?action=click\&pgtype=Article\&state=default\&region=MAIN_CONTENT_3\&context=storylines_faq}{online
    learning},
    \href{https://www.nytimes.com/2020/05/29/us/coronavirus-child-care-centers.html?action=click\&pgtype=Article\&state=default\&region=MAIN_CONTENT_3\&context=storylines_faq}{makeshift
    child care} and
    \href{https://www.nytimes.com/2020/06/03/business/economy/coronavirus-working-women.html?action=click\&pgtype=Article\&state=default\&region=MAIN_CONTENT_3\&context=storylines_faq}{stunted
    workdays} to continue. California's two largest public school
    districts --- Los Angeles and San Diego --- said on July 13, that
    \href{https://www.nytimes.com/2020/07/13/us/lausd-san-diego-school-reopening.html?action=click\&pgtype=Article\&state=default\&region=MAIN_CONTENT_3\&context=storylines_faq}{instruction
    will be remote-only in the fall}, citing concerns that surging
    coronavirus infections in their areas pose too dire a risk for
    students and teachers. Together, the two districts enroll some
    825,000 students. They are the largest in the country so far to
    abandon plans for even a partial physical return to classrooms when
    they reopen in August. For other districts, the solution won't be an
    all-or-nothing approach.
    \href{https://bioethics.jhu.edu/research-and-outreach/projects/eschool-initiative/school-policy-tracker/}{Many
    systems}, including the nation's largest, New York City, are
    devising
    \href{https://www.nytimes.com/2020/06/26/us/coronavirus-schools-reopen-fall.html?action=click\&pgtype=Article\&state=default\&region=MAIN_CONTENT_3\&context=storylines_faq}{hybrid
    plans} that involve spending some days in classrooms and other days
    online. There's no national policy on this yet, so check with your
    municipal school system regularly to see what is happening in your
    community.
  \end{itemize}
\item ~
  \hypertarget{is-the-coronavirus-airborne}{%
  \paragraph{Is the coronavirus
  airborne?}\label{is-the-coronavirus-airborne}}

  \begin{itemize}
  \tightlist
  \item
    The coronavirus
    \href{https://www.nytimes.com/2020/07/04/health/239-experts-with-one-big-claim-the-coronavirus-is-airborne.html?action=click\&pgtype=Article\&state=default\&region=MAIN_CONTENT_3\&context=storylines_faq}{can
    stay aloft for hours in tiny droplets in stagnant air}, infecting
    people as they inhale, mounting scientific evidence suggests. This
    risk is highest in crowded indoor spaces with poor ventilation, and
    may help explain super-spreading events reported in meatpacking
    plants, churches and restaurants.
    \href{https://www.nytimes.com/2020/07/06/health/coronavirus-airborne-aerosols.html?action=click\&pgtype=Article\&state=default\&region=MAIN_CONTENT_3\&context=storylines_faq}{It's
    unclear how often the virus is spread} via these tiny droplets, or
    aerosols, compared with larger droplets that are expelled when a
    sick person coughs or sneezes, or transmitted through contact with
    contaminated surfaces, said Linsey Marr, an aerosol expert at
    Virginia Tech. Aerosols are released even when a person without
    symptoms exhales, talks or sings, according to Dr. Marr and more
    than 200 other experts, who
    \href{https://academic.oup.com/cid/article/doi/10.1093/cid/ciaa939/5867798}{have
    outlined the evidence in an open letter to the World Health
    Organization}.
  \end{itemize}
\item ~
  \hypertarget{what-are-the-symptoms-of-coronavirus}{%
  \paragraph{What are the symptoms of
  coronavirus?}\label{what-are-the-symptoms-of-coronavirus}}

  \begin{itemize}
  \tightlist
  \item
    Common symptoms
    \href{https://www.nytimes.com/article/symptoms-coronavirus.html?action=click\&pgtype=Article\&state=default\&region=MAIN_CONTENT_3\&context=storylines_faq}{include
    fever, a dry cough, fatigue and difficulty breathing or shortness of
    breath.} Some of these symptoms overlap with those of the flu,
    making detection difficult, but runny noses and stuffy sinuses are
    less common.
    \href{https://www.nytimes.com/2020/04/27/health/coronavirus-symptoms-cdc.html?action=click\&pgtype=Article\&state=default\&region=MAIN_CONTENT_3\&context=storylines_faq}{The
    C.D.C. has also} added chills, muscle pain, sore throat, headache
    and a new loss of the sense of taste or smell as symptoms to look
    out for. Most people fall ill five to seven days after exposure, but
    symptoms may appear in as few as two days or as many as 14 days.
  \end{itemize}
\item ~
  \hypertarget{does-asymptomatic-transmission-of-covid-19-happen}{%
  \paragraph{Does asymptomatic transmission of Covid-19
  happen?}\label{does-asymptomatic-transmission-of-covid-19-happen}}

  \begin{itemize}
  \tightlist
  \item
    So far, the evidence seems to show it does. A widely cited
    \href{https://www.nature.com/articles/s41591-020-0869-5}{paper}
    published in April suggests that people are most infectious about
    two days before the onset of coronavirus symptoms and estimated that
    44 percent of new infections were a result of transmission from
    people who were not yet showing symptoms. Recently, a top expert at
    the World Health Organization stated that transmission of the
    coronavirus by people who did not have symptoms was ``very rare,''
    \href{https://www.nytimes.com/2020/06/09/world/coronavirus-updates.html?action=click\&pgtype=Article\&state=default\&region=MAIN_CONTENT_3\&context=storylines_faq\#link-1f302e21}{but
    she later walked back that statement.}
  \end{itemize}
\end{itemize}

When the police officers stop New Yorkers or
\href{https://www.providencejournal.com/news/20200327/not-rsquoall-that-neighborlyrsquo-police-national-guard-out-in-westerly-looking-for-new-yorkers}{find
them in their summer homes}, they hand out copies of the governor's
\href{http://www.governor.ri.gov/documents/orders/Executive-Order-20-12.pdf}{executive
order} and collect contact information. Any who manage to avoid the
officers may face another penalty: public shaming.

``Leaving Walmart \emph{filled} to the top,'' one woman seethed in a
posting to a local Facebook group, alongside photos of a Volkswagen with
New York plates and a trunk loaded with paper towels. Another woman
responded, ``They need to \emph{go home}!''

Nearly
\href{https://www.nytimes.com/interactive/2020/us/coronavirus-us-cases.html}{half
of the country's more than 119,000 coronavirus cases} have been
confirmed in New York State, in part because of widespread testing
there. Rhode Island, the smallest state by area, had 239 cases on
Saturday evening, and had reported its
\href{https://www.providencejournal.com/news/20200328/with-first-2-ri-coronavirus-deaths-raimondo-issues-stay-at-home-order-closes-non-essential-retail}{first
two deaths}.

Texas, Maryland and South Carolina are among other states that have
ordered people arriving from New York to self-quarantine. In Texas, for
instance,
\href{https://www.dps.texas.gov/director_staff/media_and_communications/pr/2020/0327a}{the
Department of Public Safety said} Friday that its agents would make
surprise visits to see whether travelers were adhering to the state's
mandate, and they warned that violators could be fined \$1,000 and
jailed for up to 180 days.

The orders have not come without controversy: Steven Brown, the
executive director of the American Civil Liberties Union of Rhode
Island, said outright that the police stops in that state were
unconstitutional.

``Under the Fourth Amendment, having a New York state license plate
simply does not, and cannot, constitute `probable cause' to allow police
to stop a car and interrogate the driver, no matter how laudable the
goal of the stop may be,'' Mr. Brown said in a statement.

Lawrence O. Gostin, the chair of global health law at Georgetown
University, agreed that the directive appeared ``too arbitrary and
capricious'' to be upheld by a judge. License plates were far from
foolproof indicators of whether someone had been exposed to the virus,
he noted.

The legal principle barring states from treating residents of other
states differently from its own citizens is one of the oldest in the
country, rooted in the Privileges and Immunities Clause of the
Constitution. That clause forbids discrimination based on state
residence in most cases, underpinning the right of unfettered interstate
travel.

Neil MacFarquhar, Patricia Mazzei and Alan Blinder contributed
reporting.

Advertisement

\protect\hyperlink{after-bottom}{Continue reading the main story}

\hypertarget{site-index}{%
\subsection{Site Index}\label{site-index}}

\hypertarget{site-information-navigation}{%
\subsection{Site Information
Navigation}\label{site-information-navigation}}

\begin{itemize}
\tightlist
\item
  \href{https://help.nytimes.com/hc/en-us/articles/115014792127-Copyright-notice}{©~2020~The
  New York Times Company}
\end{itemize}

\begin{itemize}
\tightlist
\item
  \href{https://www.nytco.com/}{NYTCo}
\item
  \href{https://help.nytimes.com/hc/en-us/articles/115015385887-Contact-Us}{Contact
  Us}
\item
  \href{https://www.nytco.com/careers/}{Work with us}
\item
  \href{https://nytmediakit.com/}{Advertise}
\item
  \href{http://www.tbrandstudio.com/}{T Brand Studio}
\item
  \href{https://www.nytimes.com/privacy/cookie-policy\#how-do-i-manage-trackers}{Your
  Ad Choices}
\item
  \href{https://www.nytimes.com/privacy}{Privacy}
\item
  \href{https://help.nytimes.com/hc/en-us/articles/115014893428-Terms-of-service}{Terms
  of Service}
\item
  \href{https://help.nytimes.com/hc/en-us/articles/115014893968-Terms-of-sale}{Terms
  of Sale}
\item
  \href{https://spiderbites.nytimes.com}{Site Map}
\item
  \href{https://help.nytimes.com/hc/en-us}{Help}
\item
  \href{https://www.nytimes.com/subscription?campaignId=37WXW}{Subscriptions}
\end{itemize}
