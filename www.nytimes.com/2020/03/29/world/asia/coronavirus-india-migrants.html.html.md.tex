Sections

SEARCH

\protect\hyperlink{site-content}{Skip to
content}\protect\hyperlink{site-index}{Skip to site index}

\href{https://www.nytimes.com/section/world/asia}{Asia Pacific}

\href{https://myaccount.nytimes.com/auth/login?response_type=cookie\&client_id=vi}{}

\href{https://www.nytimes.com/section/todayspaper}{Today's Paper}

\href{/section/world/asia}{Asia Pacific}\textbar{}India's Coronavirus
Lockdown Leaves Vast Numbers Stranded and Hungry

\url{https://nyti.ms/33RzrDI}

\begin{itemize}
\item
\item
\item
\item
\item
\end{itemize}

\href{https://www.nytimes.com/news-event/coronavirus?action=click\&pgtype=Article\&state=default\&region=TOP_BANNER\&context=storylines_menu}{The
Coronavirus Outbreak}

\begin{itemize}
\tightlist
\item
  live\href{https://www.nytimes.com/2020/08/02/world/coronavirus-updates.html?action=click\&pgtype=Article\&state=default\&region=TOP_BANNER\&context=storylines_menu}{Latest
  Updates}
\item
  \href{https://www.nytimes.com/interactive/2020/us/coronavirus-us-cases.html?action=click\&pgtype=Article\&state=default\&region=TOP_BANNER\&context=storylines_menu}{Maps
  and Cases}
\item
  \href{https://www.nytimes.com/interactive/2020/science/coronavirus-vaccine-tracker.html?action=click\&pgtype=Article\&state=default\&region=TOP_BANNER\&context=storylines_menu}{Vaccine
  Tracker}
\item
  \href{https://www.nytimes.com/interactive/2020/07/29/us/schools-reopening-coronavirus.html?action=click\&pgtype=Article\&state=default\&region=TOP_BANNER\&context=storylines_menu}{What
  School May Look Like}
\item
  \href{https://www.nytimes.com/live/2020/07/31/business/stock-market-today-coronavirus?action=click\&pgtype=Article\&state=default\&region=TOP_BANNER\&context=storylines_menu}{Economy}
\end{itemize}

Advertisement

\protect\hyperlink{after-top}{Continue reading the main story}

Supported by

\protect\hyperlink{after-sponsor}{Continue reading the main story}

\hypertarget{indias-coronavirus-lockdown-leaves-vast-numbers-stranded-and-hungry}{%
\section{India's Coronavirus Lockdown Leaves Vast Numbers Stranded and
Hungry}\label{indias-coronavirus-lockdown-leaves-vast-numbers-stranded-and-hungry}}

The sudden shutdown of businesses has upended the lives of millions of
migrant laborers in Indian cities. More than a dozen migrants have died,
and anger is rising.

\includegraphics{https://static01.nyt.com/images/2020/03/29/world/29virus-india04/merlin_170956968_de6d65c7-4b99-4453-9fc7-26d6585395e8-articleLarge.jpg?quality=75\&auto=webp\&disable=upscale}

\href{https://www.nytimes.com/by/maria-abi-habib}{\includegraphics{https://static01.nyt.com/images/2018/10/08/multimedia/author-maria-abi-habib/author-maria-abi-habib-thumbLarge.png}}\href{https://www.nytimes.com/by/sameer-yasir}{\includegraphics{https://static01.nyt.com/images/2019/11/22/reader-center/author-sameer-yasir/author-sameer-yasir-thumbLarge.png}}

By \href{https://www.nytimes.com/by/maria-abi-habib}{Maria Abi-Habib}
and \href{https://www.nytimes.com/by/sameer-yasir}{Sameer Yasir}

\begin{itemize}
\item
  March 29, 2020
\item
  \begin{itemize}
  \item
  \item
  \item
  \item
  \item
  \end{itemize}
\end{itemize}

NEW DELHI --- In one of the biggest migrations in India's modern
history, hundreds of thousands of migrant laborers have begun long
journeys on foot to get home, having been rendered homeless and jobless
by Prime Minister Narendra Modi's
\href{https://www.nytimes.com/2020/03/24/world/asia/india-coronavirus-lockdown.html}{nationwide
lockdown} to contain the spread of the coronavirus.

With businesses shut down in cities across the country, vast numbers of
migrants --- many of whom lived and ate where they worked --- were
suddenly without food and shelter. Soup kitchens in Delhi, the capital,
have been overwhelmed. So far, more than a dozen migrant laborers have
lost their lives in different parts of the country as they tried to
return to their home, hospital officials said.

Thousands of migrants in Delhi, including whole families, packed their
pots, pans and blankets into rucksacks, some balancing children on their
shoulders as they walked along interstate highways. Some planned to walk
hundreds of miles. But as they reached the Delhi border, many were
beaten back by the police.

``You fear the disease, living on the streets. But I fear hunger more,
not corona,'' said Papu, 32, who came to Delhi three weeks ago for work
and was now trying to return to his home in Saharanpur in the state of
Uttar Pradesh, 125 miles away.

\includegraphics{https://static01.nyt.com/images/2020/03/29/world/29virus-india1/merlin_171069630_bdefe6fd-682e-47ee-8ca4-5cfd8cbf7330-articleLarge.jpg?quality=75\&auto=webp\&disable=upscale}

While dozens of countries across the world are under lockdown to contain
the virus's spread, in crowded and impoverished places like India, many
fear that the measures could spark social unrest. Millions of people
live in Indian slums, and staying at home for three weeks --- as Mr.
Modi has ordered --- is a daunting prospect in such places, where dozens
of family members often share a few rooms.

Migrant laborers have been protesting the lockdown across India. On
Saturday, thousands came out to the streets in the southern state of
Kerala, saying they had not eaten in days. The authorities urged them to
disperse for their own safety, but they ignored the commands.

As of Sunday morning, just one of India's 36 state and territorial
governments, Uttar Pradesh, had made arrangements to bring migrants
home, commissioning about 1,000 buses. On Saturday, migrants waited in
lines miles long on the outskirts of Delhi to board a few buses, and the
overwhelming majority were turned away.

But by Sunday afternoon, the central government had ordered states to
reverse course and seal their borders, ordering migrants to stay where
they are. The reversal added to the already confused rollout of the
lockdown, which has prompted state government actions often at odds with
the central government's orders. The police, often confused, have
resorted to violence.

India already had one of the world's largest homeless populations, and
the lockdown has swelled its numbers exponentially, workers for
nongovernmental organizations say. A 2011 government census put the
number of homeless at 1.7 million, almost certainly a vast underestimate
in this country of 1.3 billion, experts say.

Image

``You fear the disease, living on the streets,'' said Papu, a migrant
worker in Delhi. ``But I fear hunger more, not
corona.''~Credit...Rebecca Conway for The New York Times

Mr. Modi announced the lockdown, which includes a ban on interstate
travel, with just four hours' notice on Tuesday, leaving the enormous
migrant population stranded in big cities. Jobs lure at least 45 million
people to cities from the countryside every year, according to
government estimates.

Many of those migrants are fed and housed at the shops and construction
sites where they work, and as businesses closed, hundreds of thousands
--- if not millions --- were suddenly without their homes and a regular
source of food.

\hypertarget{latest-updates-global-coronavirus-outbreak}{%
\section{\texorpdfstring{\href{https://www.nytimes.com/2020/08/01/world/coronavirus-covid-19.html?action=click\&pgtype=Article\&state=default\&region=MAIN_CONTENT_1\&context=storylines_live_updates}{Latest
Updates: Global Coronavirus
Outbreak}}{Latest Updates: Global Coronavirus Outbreak}}\label{latest-updates-global-coronavirus-outbreak}}

Updated 2020-08-02T17:52:35.962Z

\begin{itemize}
\tightlist
\item
  \href{https://www.nytimes.com/2020/08/01/world/coronavirus-covid-19.html?action=click\&pgtype=Article\&state=default\&region=MAIN_CONTENT_1\&context=storylines_live_updates\#link-34047410}{The
  U.S. reels as July cases more than double the total of any other
  month.}
\item
  \href{https://www.nytimes.com/2020/08/01/world/coronavirus-covid-19.html?action=click\&pgtype=Article\&state=default\&region=MAIN_CONTENT_1\&context=storylines_live_updates\#link-780ec966}{Top
  U.S. officials work to break an impasse over the federal jobless
  benefit.}
\item
  \href{https://www.nytimes.com/2020/08/01/world/coronavirus-covid-19.html?action=click\&pgtype=Article\&state=default\&region=MAIN_CONTENT_1\&context=storylines_live_updates\#link-2bc8948}{Its
  outbreak untamed, Melbourne goes into even greater lockdown.}
\end{itemize}

\href{https://www.nytimes.com/2020/08/01/world/coronavirus-covid-19.html?action=click\&pgtype=Article\&state=default\&region=MAIN_CONTENT_1\&context=storylines_live_updates}{See
more updates}

More live coverage:
\href{https://www.nytimes.com/live/2020/07/31/business/stock-market-today-coronavirus?action=click\&pgtype=Article\&state=default\&region=MAIN_CONTENT_1\&context=storylines_live_updates}{Markets}

A group of 13 men walking along a Delhi highway last week, bound for
their homes in Uttar Pradesh, said they had not eaten in nearly two
days. They had about \$3 between them, they said.

``This may have been a good decision for the wealthy, but not those of
us with no money,'' said Deepak Kumar, a 28-year-old truck driver,
referring to the lockdown.

Sirens approached in the distance, and the men ran away, worried it was
the police. It turned out to be an ambulance, and the men regrouped and
set off again.

Aid workers warn that the situation could deteriorate into violence if
the desperation increases and people continue to go without food.

Soup kitchens across Delhi are unable to cope with the demand, which aid
workers estimate has tripled. Fights have been breaking out. The
government has given the police no explicit policy for dealing with
stranded migrants, and many officers have lashed out.

``In the absence of a clear policy, the migrants have been left to the
whims of police. And there are instances where the police treat them
inhumanely,'' said Ashwin Parulkar, a senior researcher for the Center
for Policy Research in Delhi who studies India's homeless population.

Usually, the homeless are fed by India's array of religious
institutions: Hindu temples, Sikh gurdwaras and mosques. But now,
everything is closed, and shelters are feeling the strain.

Image

Giving out food outside a government-run shelter in New Delhi on
Saturday. Soup kitchens have been overwhelmed.Credit...Yawar Nazir/Getty
Images

``The pressure has increased drastically. People can't walk the streets,
and if it remains like this, the situation will explode,'' said Nishu
Tripathi, 29, a supervisor at a soup kitchen opened by Safe Approach, a
Delhi-based nongovernmental organization.

``Every time we start distributing food, we are charged by the crowd,''
he said.

Safe Approach started an open-air soup kitchen in northeast Delhi last
week. It now serves 8,000 people. As people lined up for food there on
Thursday, police cars circled, sirens blaring.

``Leave this place! Go inside. Separate! Separate! Maintain distance!''
the police yelled through a loudspeaker.

\href{https://www.nytimes.com/news-event/coronavirus?action=click\&pgtype=Article\&state=default\&region=MAIN_CONTENT_3\&context=storylines_faq}{}

\hypertarget{the-coronavirus-outbreak-}{%
\subsubsection{The Coronavirus Outbreak
›}\label{the-coronavirus-outbreak-}}

\hypertarget{frequently-asked-questions}{%
\paragraph{Frequently Asked
Questions}\label{frequently-asked-questions}}

Updated July 27, 2020

\begin{itemize}
\item ~
  \hypertarget{should-i-refinance-my-mortgage}{%
  \paragraph{Should I refinance my
  mortgage?}\label{should-i-refinance-my-mortgage}}

  \begin{itemize}
  \tightlist
  \item
    \href{https://www.nytimes.com/article/coronavirus-money-unemployment.html?action=click\&pgtype=Article\&state=default\&region=MAIN_CONTENT_3\&context=storylines_faq}{It
    could be a good idea,} because mortgage rates have
    \href{https://www.nytimes.com/2020/07/16/business/mortgage-rates-below-3-percent.html?action=click\&pgtype=Article\&state=default\&region=MAIN_CONTENT_3\&context=storylines_faq}{never
    been lower.} Refinancing requests have pushed mortgage applications
    to some of the highest levels since 2008, so be prepared to get in
    line. But defaults are also up, so if you're thinking about buying a
    home, be aware that some lenders have tightened their standards.
  \end{itemize}
\item ~
  \hypertarget{what-is-school-going-to-look-like-in-september}{%
  \paragraph{What is school going to look like in
  September?}\label{what-is-school-going-to-look-like-in-september}}

  \begin{itemize}
  \tightlist
  \item
    It is unlikely that many schools will return to a normal schedule
    this fall, requiring the grind of
    \href{https://www.nytimes.com/2020/06/05/us/coronavirus-education-lost-learning.html?action=click\&pgtype=Article\&state=default\&region=MAIN_CONTENT_3\&context=storylines_faq}{online
    learning},
    \href{https://www.nytimes.com/2020/05/29/us/coronavirus-child-care-centers.html?action=click\&pgtype=Article\&state=default\&region=MAIN_CONTENT_3\&context=storylines_faq}{makeshift
    child care} and
    \href{https://www.nytimes.com/2020/06/03/business/economy/coronavirus-working-women.html?action=click\&pgtype=Article\&state=default\&region=MAIN_CONTENT_3\&context=storylines_faq}{stunted
    workdays} to continue. California's two largest public school
    districts --- Los Angeles and San Diego --- said on July 13, that
    \href{https://www.nytimes.com/2020/07/13/us/lausd-san-diego-school-reopening.html?action=click\&pgtype=Article\&state=default\&region=MAIN_CONTENT_3\&context=storylines_faq}{instruction
    will be remote-only in the fall}, citing concerns that surging
    coronavirus infections in their areas pose too dire a risk for
    students and teachers. Together, the two districts enroll some
    825,000 students. They are the largest in the country so far to
    abandon plans for even a partial physical return to classrooms when
    they reopen in August. For other districts, the solution won't be an
    all-or-nothing approach.
    \href{https://bioethics.jhu.edu/research-and-outreach/projects/eschool-initiative/school-policy-tracker/}{Many
    systems}, including the nation's largest, New York City, are
    devising
    \href{https://www.nytimes.com/2020/06/26/us/coronavirus-schools-reopen-fall.html?action=click\&pgtype=Article\&state=default\&region=MAIN_CONTENT_3\&context=storylines_faq}{hybrid
    plans} that involve spending some days in classrooms and other days
    online. There's no national policy on this yet, so check with your
    municipal school system regularly to see what is happening in your
    community.
  \end{itemize}
\item ~
  \hypertarget{is-the-coronavirus-airborne}{%
  \paragraph{Is the coronavirus
  airborne?}\label{is-the-coronavirus-airborne}}

  \begin{itemize}
  \tightlist
  \item
    The coronavirus
    \href{https://www.nytimes.com/2020/07/04/health/239-experts-with-one-big-claim-the-coronavirus-is-airborne.html?action=click\&pgtype=Article\&state=default\&region=MAIN_CONTENT_3\&context=storylines_faq}{can
    stay aloft for hours in tiny droplets in stagnant air}, infecting
    people as they inhale, mounting scientific evidence suggests. This
    risk is highest in crowded indoor spaces with poor ventilation, and
    may help explain super-spreading events reported in meatpacking
    plants, churches and restaurants.
    \href{https://www.nytimes.com/2020/07/06/health/coronavirus-airborne-aerosols.html?action=click\&pgtype=Article\&state=default\&region=MAIN_CONTENT_3\&context=storylines_faq}{It's
    unclear how often the virus is spread} via these tiny droplets, or
    aerosols, compared with larger droplets that are expelled when a
    sick person coughs or sneezes, or transmitted through contact with
    contaminated surfaces, said Linsey Marr, an aerosol expert at
    Virginia Tech. Aerosols are released even when a person without
    symptoms exhales, talks or sings, according to Dr. Marr and more
    than 200 other experts, who
    \href{https://academic.oup.com/cid/article/doi/10.1093/cid/ciaa939/5867798}{have
    outlined the evidence in an open letter to the World Health
    Organization}.
  \end{itemize}
\item ~
  \hypertarget{what-are-the-symptoms-of-coronavirus}{%
  \paragraph{What are the symptoms of
  coronavirus?}\label{what-are-the-symptoms-of-coronavirus}}

  \begin{itemize}
  \tightlist
  \item
    Common symptoms
    \href{https://www.nytimes.com/article/symptoms-coronavirus.html?action=click\&pgtype=Article\&state=default\&region=MAIN_CONTENT_3\&context=storylines_faq}{include
    fever, a dry cough, fatigue and difficulty breathing or shortness of
    breath.} Some of these symptoms overlap with those of the flu,
    making detection difficult, but runny noses and stuffy sinuses are
    less common.
    \href{https://www.nytimes.com/2020/04/27/health/coronavirus-symptoms-cdc.html?action=click\&pgtype=Article\&state=default\&region=MAIN_CONTENT_3\&context=storylines_faq}{The
    C.D.C. has also} added chills, muscle pain, sore throat, headache
    and a new loss of the sense of taste or smell as symptoms to look
    out for. Most people fall ill five to seven days after exposure, but
    symptoms may appear in as few as two days or as many as 14 days.
  \end{itemize}
\item ~
  \hypertarget{does-asymptomatic-transmission-of-covid-19-happen}{%
  \paragraph{Does asymptomatic transmission of Covid-19
  happen?}\label{does-asymptomatic-transmission-of-covid-19-happen}}

  \begin{itemize}
  \tightlist
  \item
    So far, the evidence seems to show it does. A widely cited
    \href{https://www.nature.com/articles/s41591-020-0869-5}{paper}
    published in April suggests that people are most infectious about
    two days before the onset of coronavirus symptoms and estimated that
    44 percent of new infections were a result of transmission from
    people who were not yet showing symptoms. Recently, a top expert at
    the World Health Organization stated that transmission of the
    coronavirus by people who did not have symptoms was ``very rare,''
    \href{https://www.nytimes.com/2020/06/09/world/coronavirus-updates.html?action=click\&pgtype=Article\&state=default\&region=MAIN_CONTENT_3\&context=storylines_faq\#link-1f302e21}{but
    she later walked back that statement.}
  \end{itemize}
\end{itemize}

As a group of men and boys, some disabled and hobbling on makeshift
crutches, walked along the highway toward the soup kitchen, police
officers suddenly began beating them with bamboo sticks. ``Maintain
social distance!'' they yelled.

A boy of about 15 was hit in the mouth, his wails exposing his
blood-soaked teeth. An angry crowd formed to console him. ``Why would
they do that!'' screamed a man waiting for food. ``He was walking here.
Why would they treat us like this!''

Mr. Tripathi, the supervisor, turned to reporters. ``Go, we cannot
ensure your safety,'' he said.

Despite government orders to allow the transportation of essential items
like food and medicine during the lockdown, vendors complain their
delivery trucks are being harassed by the police and their stores forced
to shut.

Image

A shelter for homeless women and children in Delhi.Credit...Rebecca
Conway for The New York Times

``I've never seen such desperation,'' said Ricky Chandael, a supervisor
at another shelter. ``Before, charitable people would come and donate to
our shelter, but they can't reach us because of the lockdown. And every
day, there are at least 100 new people showing up here for food.''

As lunchtime neared and the crowd grew, Mr. Chandael, like Mr. Tripathi,
advised reporters to leave for their safety.

On Thursday, the government announced a \$22.5 billion relief package to
support the millions made unemployed by the lockdown. But it is unclear
how much that will help migrants and others in India's enormous
off-the-books work force --- believed to make up 80 percent of India's
470 million workers --- who are likely to have trouble getting access to
the benefits.

The aid, including cash and food handouts, is tied to registration in
national labor databases, which omit many migrant workers, or a home
address, which many migrants do not have.

Mr. Modi has said that shutting down for three weeks is India's only
hope to prevent a devastating epidemic. As of Sunday, 980 people in the
country had tested positive for the coronavirus, with 24 dead.

Supervisors at a shelter for women and children in Nizamuddin, a
neighborhood in Central Delhi, said the government had given them soap
for the first time, and that they were under orders to teach those
seeking shelter about the coronavirus, and to force them to wash their
hands and take showers.

``It's hard; they aren't used to washing all the time,'' said Rajesh
Kumar, the shelter's supervisor.

Image

Migrant workers headed on foot from Delhi, hoping to reach their homes
in the neighboring state of Uttar Pradesh.Credit...Rebecca Conway for
The New York Times

The previous night, he said, about 70 women with dozens of children had
started banging on the gate to the shelter, begging to be let in, saying
they had been beaten by the police for sleeping on the road. But the
shelter was full and Mr. Kumar had to turn them away.

Mr. Kumar said most homeless people he encountered had known nothing
about the coronavirus, and had awakened one day to find the police
shooing them off the streets, ordering them to practice social
distancing --- a new catchword in India, as in most of the world.

``But where do the homeless go?'' he asked.

Advertisement

\protect\hyperlink{after-bottom}{Continue reading the main story}

\hypertarget{site-index}{%
\subsection{Site Index}\label{site-index}}

\hypertarget{site-information-navigation}{%
\subsection{Site Information
Navigation}\label{site-information-navigation}}

\begin{itemize}
\tightlist
\item
  \href{https://help.nytimes.com/hc/en-us/articles/115014792127-Copyright-notice}{©~2020~The
  New York Times Company}
\end{itemize}

\begin{itemize}
\tightlist
\item
  \href{https://www.nytco.com/}{NYTCo}
\item
  \href{https://help.nytimes.com/hc/en-us/articles/115015385887-Contact-Us}{Contact
  Us}
\item
  \href{https://www.nytco.com/careers/}{Work with us}
\item
  \href{https://nytmediakit.com/}{Advertise}
\item
  \href{http://www.tbrandstudio.com/}{T Brand Studio}
\item
  \href{https://www.nytimes.com/privacy/cookie-policy\#how-do-i-manage-trackers}{Your
  Ad Choices}
\item
  \href{https://www.nytimes.com/privacy}{Privacy}
\item
  \href{https://help.nytimes.com/hc/en-us/articles/115014893428-Terms-of-service}{Terms
  of Service}
\item
  \href{https://help.nytimes.com/hc/en-us/articles/115014893968-Terms-of-sale}{Terms
  of Sale}
\item
  \href{https://spiderbites.nytimes.com}{Site Map}
\item
  \href{https://help.nytimes.com/hc/en-us}{Help}
\item
  \href{https://www.nytimes.com/subscription?campaignId=37WXW}{Subscriptions}
\end{itemize}
