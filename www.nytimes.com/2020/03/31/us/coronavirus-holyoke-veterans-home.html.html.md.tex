Sections

SEARCH

\protect\hyperlink{site-content}{Skip to
content}\protect\hyperlink{site-index}{Skip to site index}

\href{https://www.nytimes.com/section/us}{U.S.}

\href{https://myaccount.nytimes.com/auth/login?response_type=cookie\&client_id=vi}{}

\href{https://www.nytimes.com/section/todayspaper}{Today's Paper}

\href{/section/us}{U.S.}\textbar{}Coronavirus Spreads in Veterans' Home,
Leaving `Shuddering Loss for Us All'

\url{https://nyti.ms/2QZ0aJt}

\begin{itemize}
\item
\item
\item
\item
\item
\end{itemize}

\href{https://www.nytimes.com/news-event/coronavirus?action=click\&pgtype=Article\&state=default\&region=TOP_BANNER\&context=storylines_menu}{The
Coronavirus Outbreak}

\begin{itemize}
\tightlist
\item
  live\href{https://www.nytimes.com/2020/08/01/world/coronavirus-covid-19.html?action=click\&pgtype=Article\&state=default\&region=TOP_BANNER\&context=storylines_menu}{Latest
  Updates}
\item
  \href{https://www.nytimes.com/interactive/2020/us/coronavirus-us-cases.html?action=click\&pgtype=Article\&state=default\&region=TOP_BANNER\&context=storylines_menu}{Maps
  and Cases}
\item
  \href{https://www.nytimes.com/interactive/2020/science/coronavirus-vaccine-tracker.html?action=click\&pgtype=Article\&state=default\&region=TOP_BANNER\&context=storylines_menu}{Vaccine
  Tracker}
\item
  \href{https://www.nytimes.com/interactive/2020/07/29/us/schools-reopening-coronavirus.html?action=click\&pgtype=Article\&state=default\&region=TOP_BANNER\&context=storylines_menu}{What
  School May Look Like}
\item
  \href{https://www.nytimes.com/live/2020/07/31/business/stock-market-today-coronavirus?action=click\&pgtype=Article\&state=default\&region=TOP_BANNER\&context=storylines_menu}{Economy}
\end{itemize}

Advertisement

\protect\hyperlink{after-top}{Continue reading the main story}

Supported by

\protect\hyperlink{after-sponsor}{Continue reading the main story}

\hypertarget{coronavirus-spreads-in-veterans-home-leaving-shuddering-loss-for-us-all}{%
\section{Coronavirus Spreads in Veterans' Home, Leaving `Shuddering Loss
for Us
All'}\label{coronavirus-spreads-in-veterans-home-leaving-shuddering-loss-for-us-all}}

The mayor of Holyoke in Massachusetts confronted the superintendent of
the Holyoke Soldiers' Home after hearing rumors that infections were
spreading.

\includegraphics{https://static01.nyt.com/images/2020/03/31/us/31virus-veterans/merlin_171144408_0c5b1aa0-b12c-4003-99a9-500bfaa1783e-articleLarge.jpg?quality=75\&auto=webp\&disable=upscale}

\href{https://www.nytimes.com/by/ellen-barry}{\includegraphics{https://static01.nyt.com/images/2018/10/08/multimedia/author-ellen-barry/author-ellen-barry-thumbLarge.png}}

By \href{https://www.nytimes.com/by/ellen-barry}{Ellen Barry}

\begin{itemize}
\item
  Published March 31, 2020Updated May 21, 2020
\item
  \begin{itemize}
  \item
  \item
  \item
  \item
  \item
  \end{itemize}
\end{itemize}

NEWTON, Mass. --- The mayor of Holyoke, Mass., got an unsigned letter
over the weekend that deeply disturbed him.

``Are you aware of the horrific circumstances at the Soldiers' Home?''
the letter read, and went on to describe serious breaches, like a
resident suspected of having the coronavirus, awaiting the results of a
test, being sent back to a dementia ward with 20 other veterans.

``Where is the state in addressing what is truly happening in this
building?'' the letter concluded.

The mayor, Alex Morse, reached out to Bennett Walsh, the superintendent
of the Holyoke Soldiers' Home, a 247-bed, state-managed
\href{https://www.nytimes.com/article/coronavirus-nursing-homes-racial-disparity.html}{nursing
home} for veterans, to figure out what was going on.

But by then, Mr. Morse said, the damage was far more than he had
imagined: In a matter of five days, eight veterans had died, apparently
without being reported to either state or local officials. Others were
sick with the coronavirus; staff members were too.

Mr. Walsh's explanations left the mayor ``incredibly disappointed,'' and
so did a conversation with Mr. Walsh's superior, Francisco Urena,
Massachusetts' Secretary of Veterans' Services. Frustrated and ``with a
sense of disappointment at the lack of urgency,'' Mr. Morse contacted
Lt. Gov. Karyn Polito.

By Monday, state officials had announced a series of major moves.

Mr. Walsh was placed on administrative leave. A new command structure
was put in place. The National Guard was brought in to speed up testing
of staff and patients.

And that, Mr. Morse said, is increasingly the role of local government
in the coronavirus crisis: To keep watch.

``Mayors should make themselves available, should be vigilant in getting
as much information as possible,'' he said. ``Mayors have to know what's
going on within their community. I don't have oversight over the
facility, but it's still my city.''

By Tuesday, 10 residents and seven staff members had tested positive for
the coronavirus, with 25 more awaiting test results. Among 13 recent
deaths, tests had come back positive for the virus in six cases, while
five were still pending, another was inconclusive, and another came back
negative.

Flags in Holyoke, a city of 40,000 around 90 miles west of Boston, were
lowered to half-staff on Tuesday in honor of the veterans who died.

``These are people who gave their all, who risked their lives to protect
all of us, and they deserved better, frankly,'' Mr. Morse said.

State Representative Aaron Vega, whose district includes Holyoke, said
he was still trying to understand how the virus could have moved so
swiftly through the home's population without word getting out to local
officials.

``All of us in Western Mass support that home, and nobody knew
anything,'' he said. ``The fact that nobody knew anything until it was
in the news is trouble.''

Gov. Charlie Baker, in a news conference, said he had not learned of the
deaths until Sunday night, when he spoke with Mr. Morse.

``In the short term, our primary focus is going to be on stabilizing and
supporting the health and safety of the residents and their families,''
he said. ``And we will get to the bottom of what happened and when ---
and by who.''

\href{https://www.nytimes.com/news-event/coronavirus?action=click\&pgtype=Article\&state=default\&region=MAIN_CONTENT_3\&context=storylines_faq}{}

\hypertarget{the-coronavirus-outbreak-}{%
\subsubsection{The Coronavirus Outbreak
›}\label{the-coronavirus-outbreak-}}

\hypertarget{frequently-asked-questions}{%
\paragraph{Frequently Asked
Questions}\label{frequently-asked-questions}}

Updated July 27, 2020

\begin{itemize}
\item ~
  \hypertarget{should-i-refinance-my-mortgage}{%
  \paragraph{Should I refinance my
  mortgage?}\label{should-i-refinance-my-mortgage}}

  \begin{itemize}
  \tightlist
  \item
    \href{https://www.nytimes.com/article/coronavirus-money-unemployment.html?action=click\&pgtype=Article\&state=default\&region=MAIN_CONTENT_3\&context=storylines_faq}{It
    could be a good idea,} because mortgage rates have
    \href{https://www.nytimes.com/2020/07/16/business/mortgage-rates-below-3-percent.html?action=click\&pgtype=Article\&state=default\&region=MAIN_CONTENT_3\&context=storylines_faq}{never
    been lower.} Refinancing requests have pushed mortgage applications
    to some of the highest levels since 2008, so be prepared to get in
    line. But defaults are also up, so if you're thinking about buying a
    home, be aware that some lenders have tightened their standards.
  \end{itemize}
\item ~
  \hypertarget{what-is-school-going-to-look-like-in-september}{%
  \paragraph{What is school going to look like in
  September?}\label{what-is-school-going-to-look-like-in-september}}

  \begin{itemize}
  \tightlist
  \item
    It is unlikely that many schools will return to a normal schedule
    this fall, requiring the grind of
    \href{https://www.nytimes.com/2020/06/05/us/coronavirus-education-lost-learning.html?action=click\&pgtype=Article\&state=default\&region=MAIN_CONTENT_3\&context=storylines_faq}{online
    learning},
    \href{https://www.nytimes.com/2020/05/29/us/coronavirus-child-care-centers.html?action=click\&pgtype=Article\&state=default\&region=MAIN_CONTENT_3\&context=storylines_faq}{makeshift
    child care} and
    \href{https://www.nytimes.com/2020/06/03/business/economy/coronavirus-working-women.html?action=click\&pgtype=Article\&state=default\&region=MAIN_CONTENT_3\&context=storylines_faq}{stunted
    workdays} to continue. California's two largest public school
    districts --- Los Angeles and San Diego --- said on July 13, that
    \href{https://www.nytimes.com/2020/07/13/us/lausd-san-diego-school-reopening.html?action=click\&pgtype=Article\&state=default\&region=MAIN_CONTENT_3\&context=storylines_faq}{instruction
    will be remote-only in the fall}, citing concerns that surging
    coronavirus infections in their areas pose too dire a risk for
    students and teachers. Together, the two districts enroll some
    825,000 students. They are the largest in the country so far to
    abandon plans for even a partial physical return to classrooms when
    they reopen in August. For other districts, the solution won't be an
    all-or-nothing approach.
    \href{https://bioethics.jhu.edu/research-and-outreach/projects/eschool-initiative/school-policy-tracker/}{Many
    systems}, including the nation's largest, New York City, are
    devising
    \href{https://www.nytimes.com/2020/06/26/us/coronavirus-schools-reopen-fall.html?action=click\&pgtype=Article\&state=default\&region=MAIN_CONTENT_3\&context=storylines_faq}{hybrid
    plans} that involve spending some days in classrooms and other days
    online. There's no national policy on this yet, so check with your
    municipal school system regularly to see what is happening in your
    community.
  \end{itemize}
\item ~
  \hypertarget{is-the-coronavirus-airborne}{%
  \paragraph{Is the coronavirus
  airborne?}\label{is-the-coronavirus-airborne}}

  \begin{itemize}
  \tightlist
  \item
    The coronavirus
    \href{https://www.nytimes.com/2020/07/04/health/239-experts-with-one-big-claim-the-coronavirus-is-airborne.html?action=click\&pgtype=Article\&state=default\&region=MAIN_CONTENT_3\&context=storylines_faq}{can
    stay aloft for hours in tiny droplets in stagnant air}, infecting
    people as they inhale, mounting scientific evidence suggests. This
    risk is highest in crowded indoor spaces with poor ventilation, and
    may help explain super-spreading events reported in meatpacking
    plants, churches and restaurants.
    \href{https://www.nytimes.com/2020/07/06/health/coronavirus-airborne-aerosols.html?action=click\&pgtype=Article\&state=default\&region=MAIN_CONTENT_3\&context=storylines_faq}{It's
    unclear how often the virus is spread} via these tiny droplets, or
    aerosols, compared with larger droplets that are expelled when a
    sick person coughs or sneezes, or transmitted through contact with
    contaminated surfaces, said Linsey Marr, an aerosol expert at
    Virginia Tech. Aerosols are released even when a person without
    symptoms exhales, talks or sings, according to Dr. Marr and more
    than 200 other experts, who
    \href{https://academic.oup.com/cid/article/doi/10.1093/cid/ciaa939/5867798}{have
    outlined the evidence in an open letter to the World Health
    Organization}.
  \end{itemize}
\item ~
  \hypertarget{what-are-the-symptoms-of-coronavirus}{%
  \paragraph{What are the symptoms of
  coronavirus?}\label{what-are-the-symptoms-of-coronavirus}}

  \begin{itemize}
  \tightlist
  \item
    Common symptoms
    \href{https://www.nytimes.com/article/symptoms-coronavirus.html?action=click\&pgtype=Article\&state=default\&region=MAIN_CONTENT_3\&context=storylines_faq}{include
    fever, a dry cough, fatigue and difficulty breathing or shortness of
    breath.} Some of these symptoms overlap with those of the flu,
    making detection difficult, but runny noses and stuffy sinuses are
    less common.
    \href{https://www.nytimes.com/2020/04/27/health/coronavirus-symptoms-cdc.html?action=click\&pgtype=Article\&state=default\&region=MAIN_CONTENT_3\&context=storylines_faq}{The
    C.D.C. has also} added chills, muscle pain, sore throat, headache
    and a new loss of the sense of taste or smell as symptoms to look
    out for. Most people fall ill five to seven days after exposure, but
    symptoms may appear in as few as two days or as many as 14 days.
  \end{itemize}
\item ~
  \hypertarget{does-asymptomatic-transmission-of-covid-19-happen}{%
  \paragraph{Does asymptomatic transmission of Covid-19
  happen?}\label{does-asymptomatic-transmission-of-covid-19-happen}}

  \begin{itemize}
  \tightlist
  \item
    So far, the evidence seems to show it does. A widely cited
    \href{https://www.nature.com/articles/s41591-020-0869-5}{paper}
    published in April suggests that people are most infectious about
    two days before the onset of coronavirus symptoms and estimated that
    44 percent of new infections were a result of transmission from
    people who were not yet showing symptoms. Recently, a top expert at
    the World Health Organization stated that transmission of the
    coronavirus by people who did not have symptoms was ``very rare,''
    \href{https://www.nytimes.com/2020/06/09/world/coronavirus-updates.html?action=click\&pgtype=Article\&state=default\&region=MAIN_CONTENT_3\&context=storylines_faq\#link-1f302e21}{but
    she later walked back that statement.}
  \end{itemize}
\end{itemize}

Messages for Mr. Walsh were not immediately answered, and no one
representing the Soldiers' Home could be immediately reached for
comment.

Brooke Karanovich, a spokeswoman from the state's Executive Office of
Health and Human Services, said the state ``took immediate action'' as
soon as it learned the extent of the coronavirus outbreak.

Most of those who have died were not identified.

But one of the dead was Theodore A. Monette, 74, a former senior
official at the Federal Emergency Management Agency, who helped
coordinate the emergency response in Lower Manhattan after the World
Trade Center attacks, and in New Orleans after Hurricane Katrina.

A retired U.S. Army colonel who had served in the Vietnam War and
Persian Gulf war, Mr. Monette had moved into the facility two months ago
after the death of his wife, who had been his caretaker.

``They told me he was probably the highest-ranked guy there,'' but so
self-deprecating that he would rarely tell anybody his rank, said his
daughter, Aimee Monette.

She recalled that one of her father's physical therapists once Googled
him and returned to her afterward in wide-eyed amazement.

Ms. Monette said she had initially learned from a community message
board on Facebook that some veterans at the Soldiers' Home had
contracted the virus.

An anxious nurse called her last week, she said, to report that her
father's oxygen levels had dropped. Ms. Monette said they should
transfer him to a hospital.

``They're all dealing with something completely new, and everyone's
scared,'' she said. ``I don't want to place blame, but the protocol
should have happened faster.'' Mr. Monette died on Monday, and no
memorial is yet scheduled, she said.

``He deserves the full-on Army taps and the flag and everything,'' she
said. ``But we have to wait.''

Advertisement

\protect\hyperlink{after-bottom}{Continue reading the main story}

\hypertarget{site-index}{%
\subsection{Site Index}\label{site-index}}

\hypertarget{site-information-navigation}{%
\subsection{Site Information
Navigation}\label{site-information-navigation}}

\begin{itemize}
\tightlist
\item
  \href{https://help.nytimes.com/hc/en-us/articles/115014792127-Copyright-notice}{©~2020~The
  New York Times Company}
\end{itemize}

\begin{itemize}
\tightlist
\item
  \href{https://www.nytco.com/}{NYTCo}
\item
  \href{https://help.nytimes.com/hc/en-us/articles/115015385887-Contact-Us}{Contact
  Us}
\item
  \href{https://www.nytco.com/careers/}{Work with us}
\item
  \href{https://nytmediakit.com/}{Advertise}
\item
  \href{http://www.tbrandstudio.com/}{T Brand Studio}
\item
  \href{https://www.nytimes.com/privacy/cookie-policy\#how-do-i-manage-trackers}{Your
  Ad Choices}
\item
  \href{https://www.nytimes.com/privacy}{Privacy}
\item
  \href{https://help.nytimes.com/hc/en-us/articles/115014893428-Terms-of-service}{Terms
  of Service}
\item
  \href{https://help.nytimes.com/hc/en-us/articles/115014893968-Terms-of-sale}{Terms
  of Sale}
\item
  \href{https://spiderbites.nytimes.com}{Site Map}
\item
  \href{https://help.nytimes.com/hc/en-us}{Help}
\item
  \href{https://www.nytimes.com/subscription?campaignId=37WXW}{Subscriptions}
\end{itemize}
