Sections

SEARCH

\protect\hyperlink{site-content}{Skip to
content}\protect\hyperlink{site-index}{Skip to site index}

\href{https://www.nytimes.com/section/business}{Business}

\href{https://myaccount.nytimes.com/auth/login?response_type=cookie\&client_id=vi}{}

\href{https://www.nytimes.com/section/todayspaper}{Today's Paper}

\href{/section/business}{Business}\textbar{}Banks Warn of `Overwhelming'
Demand and Messy Start for Small Business Loans

\url{https://nyti.ms/2xLm6kd}

\begin{itemize}
\item
\item
\item
\item
\item
\end{itemize}

\href{https://www.nytimes.com/news-event/coronavirus?action=click\&pgtype=Article\&state=default\&region=TOP_BANNER\&context=storylines_menu}{The
Coronavirus Outbreak}

\begin{itemize}
\tightlist
\item
  live\href{https://www.nytimes.com/2020/08/04/world/coronavirus-covid-19.html?action=click\&pgtype=Article\&state=default\&region=TOP_BANNER\&context=storylines_menu}{Latest
  Updates}
\item
  \href{https://www.nytimes.com/interactive/2020/us/coronavirus-us-cases.html?action=click\&pgtype=Article\&state=default\&region=TOP_BANNER\&context=storylines_menu}{Maps
  and Cases}
\item
  \href{https://www.nytimes.com/interactive/2020/science/coronavirus-vaccine-tracker.html?action=click\&pgtype=Article\&state=default\&region=TOP_BANNER\&context=storylines_menu}{Vaccine
  Tracker}
\item
  \href{https://www.nytimes.com/2020/08/02/us/covid-college-reopening.html?action=click\&pgtype=Article\&state=default\&region=TOP_BANNER\&context=storylines_menu}{College
  Reopening}
\item
  \href{https://www.nytimes.com/live/2020/08/03/business/stock-market-today-coronavirus?action=click\&pgtype=Article\&state=default\&region=TOP_BANNER\&context=storylines_menu}{Economy}
\end{itemize}

Advertisement

\protect\hyperlink{after-top}{Continue reading the main story}

Supported by

\protect\hyperlink{after-sponsor}{Continue reading the main story}

\hypertarget{banks-warn-of-overwhelming-demand-and-messy-start-for-small-business-loans}{%
\section{Banks Warn of `Overwhelming' Demand and Messy Start for Small
Business
Loans}\label{banks-warn-of-overwhelming-demand-and-messy-start-for-small-business-loans}}

The government has promised to lend \$349 billion to small businesses
starting Friday, but banks and owners have no idea how it will play out.

\includegraphics{https://static01.nyt.com/images/2020/04/03/business/02JPvirus-smallbizloans1-print/merlin_171190623_d1139f31-6038-4830-9dd7-133ec4261708-articleLarge.jpg?quality=75\&auto=webp\&disable=upscale}

\href{https://www.nytimes.com/by/stacy-cowley}{\includegraphics{https://static01.nyt.com/images/2018/10/03/multimedia/author-stacy-cowley/author-stacy-cowley-thumbLarge.png}}

By \href{https://www.nytimes.com/by/stacy-cowley}{Stacy Cowley}

\begin{itemize}
\item
  Published April 2, 2020Updated Aug. 3, 2020
\item
  \begin{itemize}
  \item
  \item
  \item
  \item
  \item
  \end{itemize}
\end{itemize}

{[}\href{https://www.nytimes.com/article/small-business-loans-stimulus-grants-freelancers-coronavirus.html}{\emph{Read
our Coronavirus Relief Small Business F.A.Q.}}{]}

Small business owners, desperate for help amid the economic meltdown
wrought by the coronavirus pandemic, are eagerly awaiting the start of a
\$349 billion government relief program.

But just one day before the program's launch on Friday, the banks and
other lenders that the government is relying on to fund loans and vet
applicants were still waiting for much of the information they need to
participate. They are also nervous about how they --- and the government
--- will handle what is expected to be a huge crush of demand.

``The response is overwhelming --- it's unlike anything I've ever seen
in my career,'' said Craig Street, the chief lending officer of United
Midwest Savings Bank, a community bank in Columbus, Ohio. ``We're
talking about attempting to do 10 times our normal monthly loan volume,
and maybe more than that.''

The so-called
\href{https://www.nytimes.com/2020/04/20/business/shake-shack-returning-loan-ppp-coronavirus.html}{paycheck
protection program}, part of the \$2 trillion stimulus package enacted
last week, offers companies and nonprofits with up to 500 workers a
low-interest loan to cover up to two months of payroll and other
expenses. Most --- and in some cases, all --- of the loan will be
forgiven if the borrower retains its workers and doesn't cut their
wages. (The government will repay lenders for the forgiven portions of
the loans.)

That's an appealing deal for many companies that would otherwise be
leery of taking on debt in the midst of a global crisis. Jason
Dolmetsch, the president of MSK Engineering \& Design in Bennington,
Vt., said he was eager to apply. His engineering firm and its affiliated
architectural company are trying to hold on to their 23 workers despite
a rash of canceled and postponed projects.

When he called his business's banker on Monday, he was told to be
patient and wait. The bank had no information yet about how the program
would work.

Late Tuesday, the Treasury Department and the Small Business
Administration released
\href{https://www.sba.gov/funding-programs/loans/paycheck-protection-program-ppp}{an
overview for borrowers} and a sample loan application. The S.B.A., which
is backing the loans, has waived most of its usual requirements --- the
loans do not require collateral or detailed financial records --- and is
encouraging lenders to take applications digitally and make quick
decisions.

``This will be up and running tomorrow,'' Treasury Secretary Steven
Mnuchin said on Thursday at a White House briefing. He added that loan
checks could be disbursed ``the same day'' that borrowers applied.

But on Thursday evening, lenders were still waiting for technical
information about how to underwrite the loans --- which will be break
even, at best, for most lenders --- and collect reimbursement on those
that qualify for forgiveness. A trade group, the National Association of
Government Guaranteed Lenders, had to postpone a training call for 1,500
lenders on Thursday because it did not have the needed information from
the S.B.A.

``I've asked for the information twice today, and I still have
nothing,'' Tony Wilkinson, the group's chief executive, said on
Wednesday. ``I worry that they're asking lenders to make loans without
the information they need to understand the rules of engagement.''

\hypertarget{latest-updates-economy}{%
\section{\texorpdfstring{\href{https://www.nytimes.com/live/2020/08/03/business/stock-market-today-coronavirus?action=click\&pgtype=Article\&state=default\&region=MAIN_CONTENT_1\&context=storylines_live_updates}{Latest
Updates:
Economy}}{Latest Updates: Economy}}\label{latest-updates-economy}}

\href{https://www.nytimes.com/live/2020/08/03/business/stock-market-today-coronavirus?action=click\&pgtype=Article\&state=default\&region=MAIN_CONTENT_1\&context=storylines_live_updates\#the-chicago-fed-president-says-its-up-to-congress-to-save-the-economy}{13h
ago}

\href{https://www.nytimes.com/live/2020/08/03/business/stock-market-today-coronavirus?action=click\&pgtype=Article\&state=default\&region=MAIN_CONTENT_1\&context=storylines_live_updates\#the-chicago-fed-president-says-its-up-to-congress-to-save-the-economy}{The
Chicago Fed president says it's up to Congress to save the economy.}

\href{https://www.nytimes.com/live/2020/08/03/business/stock-market-today-coronavirus?action=click\&pgtype=Article\&state=default\&region=MAIN_CONTENT_1\&context=storylines_live_updates\#faa-says-boeing-has-effectively-mitigated-defects-in-the-737-max}{14h
ago}

\href{https://www.nytimes.com/live/2020/08/03/business/stock-market-today-coronavirus?action=click\&pgtype=Article\&state=default\&region=MAIN_CONTENT_1\&context=storylines_live_updates\#faa-says-boeing-has-effectively-mitigated-defects-in-the-737-max}{F.A.A.
says Boeing has `effectively mitigated' defects in the 737 Max.}

\href{https://www.nytimes.com/live/2020/08/03/business/stock-market-today-coronavirus?action=click\&pgtype=Article\&state=default\&region=MAIN_CONTENT_1\&context=storylines_live_updates\#small-businesses-got-emergency-loans-but-not-what-they-expected}{16h
ago}

\href{https://www.nytimes.com/live/2020/08/03/business/stock-market-today-coronavirus?action=click\&pgtype=Article\&state=default\&region=MAIN_CONTENT_1\&context=storylines_live_updates\#small-businesses-got-emergency-loans-but-not-what-they-expected}{Small
businesses got emergency loans, but not what they expected.}

\href{https://www.nytimes.com/live/2020/08/03/business/stock-market-today-coronavirus?action=click\&pgtype=Article\&state=default\&region=MAIN_CONTENT_1\&context=storylines_live_updates}{See
more updates}

More live coverage:
\href{https://www.nytimes.com/2020/08/04/world/coronavirus-covid-19.html?action=click\&pgtype=Article\&state=default\&region=MAIN_CONTENT_1\&context=storylines_live_updates}{Global}

Bank lobbyist groups have warned the Treasury Department that the
program as designed will not be workable, expressing alarm about their
own legal liability as they try to rush money to borrowers and keep tabs
on potential fraud. The Independent Community Bankers of America sent a
letter to Mr. Mnuchin on Wednesday complaining that guidelines calling
for low-interest loans could mean ``unacceptable losses'' for lenders.

\includegraphics{https://static01.nyt.com/images/2020/04/02/business/02virus-smallbizhelp2/merlin_171190803_14d4daac-0ea4-4e21-bcf2-f8a8a0b93fd8-articleLarge.jpg?quality=75\&auto=webp\&disable=upscale}

S.B.A. representatives did not respond to questions about when guidance
for lenders would be available.

Although the government has scrambled to pull aid together quickly, the
program's slow rollout has frustrated business owners facing a daily
fight to salvage their companies. Paul Caragiulo is an owner of a group
of restaurants in Sarasota, Fla., that employ around 150 people. He is
loath to lay off anyone --- even though his restaurants' sales have
cratered --- but he's also hesitant about borrowing what could be
millions of dollars from a program whose details are being worked out on
the fly.

The information sheets posted by the Treasury Department and the S.B.A.
have not reassured him. ``Those are bullet points, not term sheets,'' he
said. ``We're not used to having debt, and we don't look at that
lightly.''

The Trump administration has said it wants the paycheck protection loans
to be easy to obtain;
\href{https://www.sba.gov/sites/default/files/2020-03/Borrower\%20Paycheck\%20Protection\%20Program\%20Application_0.pdf}{a
sample application} posted on Tuesday is a four-page form that can be
completed in less than 10 minutes. But the fine print contains a line
that gave Mr. Caragiulo pause: Borrowers must promise to buy only
American-made equipment and products ``to the extent feasible.''

Mr. Caragiulo, who uses Italian pizza ovens, said the requirement seemed
like an absurd bureaucratic tripwire. When asked about it, an S.B.A.
spokeswoman pointed to
\href{https://www.congress.gov/bill/102nd-congress/house-bill/4111/text}{a
1992 law} that requires the agency to ``encourage'' business owners
receiving financial help to buy American goods. She did not respond to
questions about how --- or if --- that will be enforced.

Other federal small business aid efforts have been generous but chaotic.
A program offering
\href{https://www.nytimes.com/2020/08/03/business/small-business-loans-coronavirus.html}{low-interest
disaster loans} funded directly by the government has already had more
than 100,000 applicants, according to one person familiar with its
operations.

The S.B.A. started taking applications weeks ago, but last Friday's
stimulus bill added a new sweetener: Applicants, including those who are
rejected for loans, are eligible for up to \$10,000 in cash grants. (The
funds are described on the S.B.A. website as a ``loan advance,'' but an
agency spokeswoman confirmed that it does not have to be repaid.)

Abninder Mundra, who owns a franchise of the UPS Store in Portola
Valley, Calif., applied for a disaster loan on March 20 and was approved
four days later for \$210,000. Then the stimulus bill introduced the
grants. Mr. Mundra said an S.B.A. representative had told him to fill
out a second loan application if he wanted the grant funds. He was still
waiting for both his disaster loan check and a response to the grant
application.

Image

Abninder Mundra at the UPS store he owns with his wife in Portola
Valley, Calif.Credit...Jim Wilson/The New York Times

Mr. Mundra said he could afford to wait a few weeks and was grateful for
the aid. He also plans to seek a paycheck protection loan as soon as his
bank starts taking applications. He had to cut his three employees'
hours to offset a drop in foot traffic, and hopes the loan will help
restore them.

``I think the government really understood that small businesses are the
backbone of the economy,'' he said. ``If we stop employing people, they
won't have money to pay their bills.''

But with job losses
\href{https://www.nytimes.com/2020/03/26/business/economy/coronavirus-unemployment-claims.html}{already
setting records} and certain to worsen, lenders fear that the \$349
billion Congress allocated for the paycheck program will quickly run
out. Senior officials from the Treasury and S.B.A. told reporters on
Tuesday that they were prepared to ask Congress for more money if
needed.

Jim Donnelly, the chief commercial officer of Bangor Savings Bank in
Maine, said his small staff was working around the clock to accommodate
the pent-up demand. In a typical year, his bank handles hundreds of
business loans. He expects to process thousands in the coming months.

And even though his bank was still waiting for critical technical
information, it planned to start taking loan applications on Friday.

``We have local businesses like restaurants that have shut down and are
looking at these loans as a way to reopen their doors,'' he said.

Many of the nation's largest banks said they planned to offer the loans,
though some will restrict which applicants they will work with.

JPMorgan Chase, for example, said
\href{https://recovery.chase.com/cares}{it would make the loans
available} to customers with Chase business checking accounts as of Feb.
15. Bank of America and Citi both said they planned to participate but
did not yet have details.

The Treasury has encouraged non-bank lenders to also offer the loans,
but some that want to do so say the process has been maddening. Kabbage,
one of the biggest online lenders, said the system for becoming an
approved lender was opaque.

Mr. Street, at United Midwest Savings Bank, was also desperate for more
information, including details about how thoroughly banks are expected
to scrutinize potential borrowers. Any choice involves trade-offs: Quick
approvals and cash disbursements raise the risk of mistakes and borrower
fraud, but rigorous underwriting takes time that desperate business
owners and overtaxed bankers don't have to spare.

Mr. Street hopes the S.B.A. and the banking industry's regulators will
give lenders leeway to err on the side of speed.

``We're trying to set things up so that we can crank these things out,''
he said. ``We had calls starting Monday morning from people who wanted
to borrow right away. It was hard telling people they had to wait.
Nobody can afford to wait.''

Advertisement

\protect\hyperlink{after-bottom}{Continue reading the main story}

\hypertarget{site-index}{%
\subsection{Site Index}\label{site-index}}

\hypertarget{site-information-navigation}{%
\subsection{Site Information
Navigation}\label{site-information-navigation}}

\begin{itemize}
\tightlist
\item
  \href{https://help.nytimes.com/hc/en-us/articles/115014792127-Copyright-notice}{©~2020~The
  New York Times Company}
\end{itemize}

\begin{itemize}
\tightlist
\item
  \href{https://www.nytco.com/}{NYTCo}
\item
  \href{https://help.nytimes.com/hc/en-us/articles/115015385887-Contact-Us}{Contact
  Us}
\item
  \href{https://www.nytco.com/careers/}{Work with us}
\item
  \href{https://nytmediakit.com/}{Advertise}
\item
  \href{http://www.tbrandstudio.com/}{T Brand Studio}
\item
  \href{https://www.nytimes.com/privacy/cookie-policy\#how-do-i-manage-trackers}{Your
  Ad Choices}
\item
  \href{https://www.nytimes.com/privacy}{Privacy}
\item
  \href{https://help.nytimes.com/hc/en-us/articles/115014893428-Terms-of-service}{Terms
  of Service}
\item
  \href{https://help.nytimes.com/hc/en-us/articles/115014893968-Terms-of-sale}{Terms
  of Sale}
\item
  \href{https://spiderbites.nytimes.com}{Site Map}
\item
  \href{https://help.nytimes.com/hc/en-us}{Help}
\item
  \href{https://www.nytimes.com/subscription?campaignId=37WXW}{Subscriptions}
\end{itemize}
