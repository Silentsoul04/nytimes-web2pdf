Sections

SEARCH

\protect\hyperlink{site-content}{Skip to
content}\protect\hyperlink{site-index}{Skip to site index}

\href{https://www.nytimes.com/section/world/asia}{Asia Pacific}

\href{https://myaccount.nytimes.com/auth/login?response_type=cookie\&client_id=vi}{}

\href{https://www.nytimes.com/section/todayspaper}{Today's Paper}

\href{/section/world/asia}{Asia Pacific}\textbar{}`Muslims Are
Foreigners': Inside India's Campaign to Decide Who Is a Citizen

\url{https://nyti.ms/2xPAYhy}

\begin{itemize}
\item
\item
\item
\item
\item
\end{itemize}

Advertisement

\protect\hyperlink{after-top}{Continue reading the main story}

Supported by

\protect\hyperlink{after-sponsor}{Continue reading the main story}

\hypertarget{muslims-are-foreigners-inside-indias-campaign-to-decide-who-is-a-citizen}{%
\section{`Muslims Are Foreigners': Inside India's Campaign to Decide Who
Is a
Citizen}\label{muslims-are-foreigners-inside-indias-campaign-to-decide-who-is-a-citizen}}

Tribunal members in the state of Assam say they felt pressured to
declare Muslims noncitizens as the government seeks to expel illegal
migrants. Some politicians have vowed to take the process nationwide.

\includegraphics{https://static01.nyt.com/images/2020/04/06/world/asia/monavra-still/monavra-still-videoSixteenByNineJumbo1600.jpg}

\href{https://www.nytimes.com/by/karan-deep-singh}{\includegraphics{https://static01.nyt.com/images/2019/12/02/reader-center/author-karan-deep-singh/author-karan-deep-singh-thumbLarge.png}}\href{https://www.nytimes.com/by/suhasini-raj}{\includegraphics{https://static01.nyt.com/images/2019/11/22/reader-center/author-Suhasini-Raj/author-Suhasini-Raj-thumbLarge.png}}

By \href{https://www.nytimes.com/by/karan-deep-singh}{Karan Deep Singh}
and \href{https://www.nytimes.com/by/suhasini-raj}{Suhasini Raj}

\begin{itemize}
\item
  April 4, 2020
\item
  \begin{itemize}
  \item
  \item
  \item
  \item
  \item
  \end{itemize}
\end{itemize}

JORHAT, India --- For nearly two years, Mamoni Rajkumari, a lawyer,
spent her days deciding who was an Indian citizen and who was not, as
part of a tribunal reviewing suspected foreigners in the state of Assam.
Then, she says, she was dismissed for not declaring enough Muslims to be
noncitizens.

``I was punished,'' she said.

Ms. Rajkumari, 54, has found herself on the front line of India's
\href{https://www.nytimes.com/2019/12/22/world/asia/modi-india-citizenship-law.html}{citizenship
wars}. In addition to the tribunals, which Assam has operated for
decades, the state has also recently completed a broader, separate
\href{https://www.nytimes.com/2019/08/17/world/asia/india-muslims-narendra-modi.html}{review
of every resident's paperwork} to determine if they were citizens.

That review found that nearly two million of Assam's 33 million
residents, many of them desperately poor, were possibly foreigners. Now
this group --- \href{https://www.youtube.com/watch?v=F1eAFpLLcXk}{which
is disproportionately Muslim} --- is potentially stateless.

What's happening in Assam is a preview of what may be coming to India as
a whole as Prime Minister Narendra Modi tries to pull the country away
from its foundation as a secular, multicultural nation and turn it into
a more overtly Hindu state.

CHINA

NEPAL

ASSAM

PAKISTAN

New

Delhi

INDIA

MYANMAR

BANGLADESH

Bay of Bengal

SRI LANKA

500 MILES

By The New York Times

The New York Times interviewed one current and five former members of
the Assam tribunals that review suspected foreigners. The five former
members said they had felt pressured by the government to declare
Muslims to be noncitizens. Three of them, including Ms. Rajkumari, said
they were fired because they did not do so.

State and central government officials declined to comment.

Mr. Modi's Bharatiya Janata Party has its roots in a Hindu nationalist
worldview, and during last year's national elections, party leaders
vowed to apply the same type of citizenship checks used in Assam to the
rest of India. Mr. Modi has recently denied he has any such plans.

Like Assam, India is majority Hindu, with a large Muslim minority. In
December, India's national government
\href{https://www.nytimes.com/2019/12/11/world/asia/india-muslims-citizenship-narendra-modi.html}{passed
a sweeping new immigration law} that gives a fast track to citizenship
for undocumented migrants from nearby countries as long as they are
Hindu or one of five other religions. Muslims are excluded.

The upshot is that any Hindus left off Assam's citizenship lists after
its broad review, or declared by tribunals to be foreigners, will likely
be affirmed as citizens because of the new immigration law. Muslims may
not.

``Increasingly, it is looking like Muslims are becoming a target,'' said
Binod Khadria, an expert on migration who is a former professor at the
Jawaharlal Nehru University in New Delhi. ``It's a charged situation.''

\includegraphics{https://static01.nyt.com/images/2020/04/05/world/05india-assam1/merlin_166026678_a2aae689-ef07-4101-9396-ea6a5fc49f49-articleLarge.jpg?quality=75\&auto=webp\&disable=upscale}

Even before the citizenship review, an indigenous rights movement in
Assam, in northeast India on the border of Bangladesh, had been
agitating for the government to expel foreigners.

The police --- sometimes acting on reports from private citizens --- had
referred more than 433,000 residents as ``suspected foreigners,''
according to parliamentary documents, and sent them to tribunals like
the one Ms. Rajkumari sat on to produce documents or witnesses to prove
they are truly Indian.

Now, the citizenship review has produced 1.9 million new ``suspected
foreigners.'' So Assam is adding more foreigner tribunals to adjudicate
their cases.

The whole tribunal process has troubled Ms. Rajkumari and some others
who have served as tribunal members, generally hearing cases on their
own.

Many poor Indians lack the required paperwork to prove citizenship, like
parents' voting records and land ownership documents that have been
certified by authorities as authentic.

What's more, the choice of who is labeled a suspected foreigner seems to
have a religious bias to it, with a much higher percentage of Muslims
sent to the tribunals than Hindus, according to Ms. Rajkumari and the
tribunal members interviewed. Some of those current and former tribunal
members spoke on condition of anonymity because they feared reprisals
from the government.

Although the tribunals are not technically courts, they function as if
they were. If they find that someone cannot prove his or her
citizenship, that person can be sent to detention, often within a jail.

Image

A government official in Kharupetia, in the Indian state of Assam,
collected documents from people hoping to be included on an official
list of Indian citizens last year.Credit...Saumya Khandelwal for The New
York Times

Kartik Roy, a lawyer and another former tribunal member, said ``most of
the references'' that police officers made to his tribunal to
investigate suspected foreigners ``were against Muslims.''

He said the pressure was clear: ``You have to declare `foreigners' means
you have to declare the Muslims,'' he said.

Ms. Rajkumari agreed, saying state officials ``think Muslims are
foreigners.''

Some of the tribunal members interviewed said they felt pressure in
general to find more ``foreigners,'' with a monthly requirement to
report how many cases they had heard and of those, how many people had
been declared foreigners.

Two other former members said officials in Assam's Home and Political
Department, which from 2016 has been controlled by Mr. Modi's political
party, had complained that they were not declaring enough people
noncitizens.

The former tribunal members said the complaints relayed to them were a
form of indirect but heavy pressure.

Tribunal members who declared more people foreigners had their
performance rated as ``good,'' which increased their chances of keeping
their jobs, according to court documents viewed by The Times. The
performance of those who didn't declare enough people foreigners was
marked as ``not satisfactory.''

Image

Mamoni Rajkumari was part of a tribunal reviewing suspected foreigners
in the state of Assam.Credit...Karan Deep Singh/The New York Times

Both Ms. Rajkumari's and Mr. Roy's names appeared on that review list
with a note against their names, saying they ``may be terminated.''

That is exactly what happened. The terms of Ms. Rajkumari and Mr. Roy
were not renewed in 2017.

They both said that because the bulk of people in front of the tribunals
were Muslims, the expectation was that they would declare Muslims as
foreigners, paving the way to deport them, incarcerate them or take away
fundamental rights.

The director general of police in Assam and other state officials
declined to comment.

Mr. Modi and top officials in his party have denied targeting Muslims in
the Assam citizenship check, saying it was meant purely to identify
illegal migrants.

Image

Relatives grieving Ishwar Nayak, who was shot by the police in Guwahati
in December while protesting the new citizenship law.Credit...Ahmer Khan
for The New York Times

The Home Ministry in New Delhi, which ultimately oversees citizenship
and residency rules in India, also declined to comment, citing the
demands of the coronavirus crisis.

Most of the migrants in Assam came from Bangladesh, at one time or
another. Many have lived in Assam for generations, the descendants of
economic migrants from decades ago.

And many are illiterate and poor, often with no idea how to read the
papers vital to proving a citizenship claim, and keeping them out of
jail.

Bangladesh, a predominantly Muslim country, and one of the poorest and
most densely populated nations in the world, has expressed zero
enthusiasm in taking migrants back.

The passage of the national citizenship law sparked
\href{https://www.nytimes.com/video/world/100000006877118/protests-india.html}{protests}
in Assam and across the country, and they continued to flare up until
Mr. Modi imposed a coronavirus
\href{https://www.nytimes.com/2020/03/24/world/asia/india-coronavirus-lockdown.html}{lockdown}
across India in late March.

Dozens of people in Assam whose citizenship has been questioned have
killed themselves, according to
\href{https://www.thecitizen.in/index.php/en/NewsDetail/index/15/18425/Suicides-over-the-NRC--Trend-Analysis}{Indian
media reports}. Countless others fear being expelled from India or
thrown in jail.

Mr. Modi's government doesn't seem to be devising any plans to deport
millions of people.

But it is expanding its capacity to incarcerate foreigners; an enormous
\href{https://www.nytimes.com/2019/08/17/world/asia/india-muslims-narendra-modi.html}{detention}
facility is under construction in the Goalpara district of Assam, where
up to 3,000 people are likely to be held.

The compound, set to open in a few months, has thick, high walls on the
periphery, watchtowers in every corner, separate sections for men and
women, and an infirmary.

Image

An under-construction detention center for people not included in
Assam's citizens register, in the village of Kadamtola
Gopalpur.Credit...Biju Boro/Agence France-Presse --- Getty Images

In July 2017, Ms. Rajkumari, Mr. Roy and 12 other former tribunal
members sued the government for wrongful dismissal, disputing the
government's rating of their work. They lost the case.

Ms. Rajkumari, who continues to practice law, said that what the
government pressured her to do was wrong. A few weeks before she was
fired, she recalled, she visited her mother.

``Ma, how will I accomplish this task?'' Ms. Rajkumari remembered asking
her, tears streaming down her face. ``Because it is very illegal.''

Anupam Chakravartty contributed reporting from Guwahati, India.

Advertisement

\protect\hyperlink{after-bottom}{Continue reading the main story}

\hypertarget{site-index}{%
\subsection{Site Index}\label{site-index}}

\hypertarget{site-information-navigation}{%
\subsection{Site Information
Navigation}\label{site-information-navigation}}

\begin{itemize}
\tightlist
\item
  \href{https://help.nytimes.com/hc/en-us/articles/115014792127-Copyright-notice}{©~2020~The
  New York Times Company}
\end{itemize}

\begin{itemize}
\tightlist
\item
  \href{https://www.nytco.com/}{NYTCo}
\item
  \href{https://help.nytimes.com/hc/en-us/articles/115015385887-Contact-Us}{Contact
  Us}
\item
  \href{https://www.nytco.com/careers/}{Work with us}
\item
  \href{https://nytmediakit.com/}{Advertise}
\item
  \href{http://www.tbrandstudio.com/}{T Brand Studio}
\item
  \href{https://www.nytimes.com/privacy/cookie-policy\#how-do-i-manage-trackers}{Your
  Ad Choices}
\item
  \href{https://www.nytimes.com/privacy}{Privacy}
\item
  \href{https://help.nytimes.com/hc/en-us/articles/115014893428-Terms-of-service}{Terms
  of Service}
\item
  \href{https://help.nytimes.com/hc/en-us/articles/115014893968-Terms-of-sale}{Terms
  of Sale}
\item
  \href{https://spiderbites.nytimes.com}{Site Map}
\item
  \href{https://help.nytimes.com/hc/en-us}{Help}
\item
  \href{https://www.nytimes.com/subscription?campaignId=37WXW}{Subscriptions}
\end{itemize}
