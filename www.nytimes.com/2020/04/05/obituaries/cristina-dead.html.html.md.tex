Sections

SEARCH

\protect\hyperlink{site-content}{Skip to
content}\protect\hyperlink{site-index}{Skip to site index}

\href{https://www.nytimes.com/section/obituaries}{Obituaries}

\href{https://myaccount.nytimes.com/auth/login?response_type=cookie\&client_id=vi}{}

\href{https://www.nytimes.com/section/todayspaper}{Today's Paper}

\href{/section/obituaries}{Obituaries}\textbar{}Cristina, Cult Downtown
New York Singer, Dies at 64

\url{https://nyti.ms/2V6qmCS}

\begin{itemize}
\item
\item
\item
\item
\item
\item
\end{itemize}

\href{https://www.nytimes.com/news-event/coronavirus?action=click\&pgtype=Article\&state=default\&region=TOP_BANNER\&context=storylines_menu}{The
Coronavirus Outbreak}

\begin{itemize}
\tightlist
\item
  live\href{https://www.nytimes.com/2020/08/03/world/coronavirus-covid-19.html?action=click\&pgtype=Article\&state=default\&region=TOP_BANNER\&context=storylines_menu}{Latest
  Updates}
\item
  \href{https://www.nytimes.com/interactive/2020/us/coronavirus-us-cases.html?action=click\&pgtype=Article\&state=default\&region=TOP_BANNER\&context=storylines_menu}{Maps
  and Cases}
\item
  \href{https://www.nytimes.com/interactive/2020/science/coronavirus-vaccine-tracker.html?action=click\&pgtype=Article\&state=default\&region=TOP_BANNER\&context=storylines_menu}{Vaccine
  Tracker}
\item
  \href{https://www.nytimes.com/2020/08/02/us/covid-college-reopening.html?action=click\&pgtype=Article\&state=default\&region=TOP_BANNER\&context=storylines_menu}{College
  Reopening}
\item
  \href{https://www.nytimes.com/live/2020/08/03/business/stock-market-today-coronavirus?action=click\&pgtype=Article\&state=default\&region=TOP_BANNER\&context=storylines_menu}{Economy}
\end{itemize}

Advertisement

\protect\hyperlink{after-top}{Continue reading the main story}

Supported by

\protect\hyperlink{after-sponsor}{Continue reading the main story}

Those WE've Lost

\hypertarget{cristina-cult-downtown-new-york-singer-dies-at-64}{%
\section{Cristina, Cult Downtown New York Singer, Dies at
64}\label{cristina-cult-downtown-new-york-singer-dies-at-64}}

A fixture in New York's dance-music nightlife of the late '70s and early
'80s, she gained new attention in 2004 when her albums were reissued.

\includegraphics{https://static01.nyt.com/images/2020/04/11/obituaries/11Cristina-obit1/merlin_171229230_064ff997-2ee1-497b-ab4c-2f1795c1bf5d-articleLarge.jpg?quality=75\&auto=webp\&disable=upscale}

By \href{https://www.nytimes.com/by/jon-caramanica}{Jon Caramanica}

\begin{itemize}
\item
  Published April 5, 2020Updated May 12, 2020
\item
  \begin{itemize}
  \item
  \item
  \item
  \item
  \item
  \item
  \end{itemize}
\end{itemize}

\emph{This obituary is part of a series about}
\href{https://www.nytimes.com/series/people-who-have-died-of-the-coronavirus}{\emph{people
who have died in the coronavirus pandemic}}\emph{.}

Cristina, a cult singer who brought avant-garde sensibilities to New
York's dance-music nightlife at the turn of the 1980s, died on March 31
in New York. She was 64.

Her daughter, Lucinda Zilkha Francis, said she had been suffering from
several autoimmune disorders, including relapsing polychondritis, for
approximately two decades. On Friday, her family learned she had tested
positive for the coronavirus.

In the fertile anything-goes downtown New York of the late 1970s and
early 1980s, Cristina cut a unique figure --- a hyperliterary,
well-to-do, seen-it-all performer who taunted club music culture with
songs that could be read as wry parody or progressive updates.

``My strength is not in my voice, nor do I have sexy ankles,'' she
\href{https://www.redoverwhite.org/cristina/rocks_cristina.html}{told
the Boston Globe} in 1980. ``I have an analytical brain, and maybe
that's a liability in rock n' roll, but if I play it right, it will
translate musical principles into theatrical terms, which is what I have
to do anyway, given my lack of technical expertise in music.''

Cristina Monet Palaci was born on Jan. 17, 1956, in Manhattan to Dorothy
Monet, a writer and illustrator, and Jacques Palaci, a psychoanalyst.
She attended Harvard, where she studied playwriting under William
Alfred. She took a year off from college and came to New York, where she
wrote freelance theater reviews for The Village Voice. It was there that
she met Michael Zilkha, who became her boyfriend and, later, the engine
behind her music career.

Mr. Zilkha was starting Ze Records with a partner. Cristina, who didn't
have particular aspirations to be a singer, nevertheless became his
first artist with the 1978 single ``Disco Clone.'' Produced by John
Cale, it was a deceptively slick dismantling of disco's sameness, sung
in an aspirated and shrill voice: ``If you like the way I shake it/And
you think you want to make it/There's 50 just like me.''

``I thought it was so bad that it could be a Brechtian pastiche,'' she
\href{https://www.redoverwhite.org/cristina/queen_cristina.html}{told
Time Out New York} in 2004. ``It turned out to be an eccentric and funny
record --- insane, enthusiastic, impassioned, amateurish.'' (One version
of the song features Kevin Kline on accompanying vocals.)

According to a
\href{https://books.google.com/books?id=C-YCAAAAMBAJ\&pg=PA54\&lpg=PA54\&dq=Harvard+\%22Cristina+Monet\%22\&source=bl\&ots=DyhnXwdtrl\&sig=ACfU3U06gcTYJ9qThF562RojNY1vXS4gEA\&hl=en\&ppis=_c\&sa=X\&ved=2ahUKEwj2io3oxsjoAhVMlnIEHfdNAtYQ6AEwCHoECD8QKA\#v=onepage\&q=Monet\%22\&f=false}{1984
New York magazine article}, after her first live performance, at the
Squat Theater in Chelsea, Cristina's mother told her, ``You were always
a brilliant writer. A good artist \ldots{} a good actress. How could you
be so self-destructive as to sing?''

``Disco Clone'' was successful enough that Cristina continued recording.
Her self-titled debut album, released in 1980, was produced by August
Darnell (who performed as Kid Creole), and featured disco paired with
heavy Latin percussion. ``The first theatrical, cinematic, nostalgic
disco record, at a time when there wasn't a lot of humor in disco,''
Cristina said in an early 2000s interview with the zine
\href{http://festivefanzine.blogspot.com/2010/12/from-vaults-merry-cristina-mas.html}{Festive!}

Consistently, Cristina injected a wry, burned-out, misadventuring
patrician sensibility into her lyrics and delivery, especially on her
rendition, that same year, of the Leiber and Stoller song, ``Is That All
There Is?'' (originally made famous by Peggy Lee).

She tweaked the lyrics, making them both more whimsical and more
terrifying: ``I remember when I was a little girl, my mother set the
house on fire --- she was like that.'' (Lieber and Stoller protested,
and insisted the song be withdrawn from release.)

The Cristina that appeared on records was unfiltered. ``It was entirely
her, there's no confection or construction in it all,'' Mr. Zilkha said
in an interview.

She and Mr. Zilkha married in 1983, and in 1984 she released her second
album, ``Sleep It Off,'' produced by Don Was, with cover design by
Jean-Paul Goude. On this album, she leaned into new wave, rendered again
with savage satirical energy on songs like the punkish ``Don't Mutilate
My Mink'' and ``What's a Girl to Do.'' (``If you loved the Pulitzer
divorce trial, you'll love this record,'' crowed Rolling Stone.)

``The one thing that pop music has lost lately is its sense of irony,''
Cristina said when the album was released. ``People either write
dumb-funny novelty songs or dead-earnest serious songs. There's nothing
around that combines elements of both.''

She and Mr. Zilkha moved to Texas soon after that album's release,
effectively ending her music career. ``I believed the idea that Michael
had bought me a career to such an extent that I felt sheepish and
guilty, which I shouldn't have been,'' she told Time Out New York. After
she and Mr. Zilkha divorced in 1990, she returned to New York.

In its day, Cristina's work was very much a product of its demimonde.
But in 2004, her albums were reissued to great acclaim and wide
attention. Her only other musical recording was a 2006 collaboration
with Ursula 1000, ``Urgent/Anxious,'' which took advantage of the
implied eye-rolling in her voice, which hadn't diminished at all.

Outside of music, Cristina retained an avid interest in theater and
books. She was especially passionate about 19th-century literature.
``When I was a child, she would read me Dickens, doing all the voices,''
Ms. Francis recalled. She contributed occasional book and film reviews
to the Times Literary Supplement, as well as articles to Tatler and
London Literary Review.

Her medical conditions were often debilitating: ``It's hard to plan a
new album when you don't know if you will make it down to the end of the
street from one day to the next,'' she told Time Out New York. But in
recent years, she had recovered enough to begin traveling.

In addition to her daughter, Cristina is survived by two granddaughters
and her longtime companion, Stephen Graham.

\href{https://www.nytimes.com/interactive/2020/obituaries/people-died-coronavirus-obituaries.html?action=click\&pgtype=Article\&state=default\&region=BELOW_MAIN_CONTENT\&context=covid_obits_promo}{}

\hypertarget{those-weve-lost}{%
\section{Those We've Lost}\label{those-weve-lost}}

The coronavirus pandemic has taken an incalculable death toll. This
series is designed to put names and faces to the numbers.

Read more

\includegraphics{https://static01.nyt.com/images/2020/07/30/obituaries/30Pedro/30Pedro-square640.jpg}

\hypertarget{bernaldina-josuxe9-pedro}{%
\section{Bernaldina José Pedro}\label{bernaldina-josuxe9-pedro}}

d. Boa Vista, Brazil

Leader among the Indigenous Macuxi

\includegraphics{https://static01.nyt.com/images/2020/07/31/obituaries/31Swing/merlin_175167783_8913bc90-0d64-43f3-a655-1bb1bf1601c9-square640.jpg}

\hypertarget{john-eric-swing}{%
\section{John Eric Swing}\label{john-eric-swing}}

d. Fountain Valley, Calif.

Champion of Filipino-Americans

\includegraphics{https://static01.nyt.com/images/2020/07/27/obituaries/27Victor/merlin_175001436_38b11f8e-227a-4e2c-9821-7618af9b2524-square640.jpg}

\hypertarget{victor-victor}{%
\section{Victor Victor}\label{victor-victor}}

d. Santo Domingo, Dominican Republic

Beloved musician of the Dominican Republic

\includegraphics{https://static01.nyt.com/images/2020/07/31/obituaries/31Negron/merlin_175160169_516322ae-fd23-4969-b6b2-193ced371105-square640.jpg}

\hypertarget{dr-eddie-negruxf3n}{%
\section{Dr. Eddie Negrón}\label{dr-eddie-negruxf3n}}

d. Fort Walton Beach, Fla.

Internist on Florida's Emerald Coast

\includegraphics{https://static01.nyt.com/images/2020/07/30/obituaries/30Dobson/merlin_175115928_f6b9271c-8f05-4fe1-a38a-5ca4a58f8935-square640.jpg}

\hypertarget{dobby-dobson}{%
\section{Dobby Dobson}\label{dobby-dobson}}

d. Coral Springs, Fla.

Jamaican singer and songwriter

\includegraphics{https://static01.nyt.com/images/2020/08/01/obituaries/28Gonzalez/merlin_175002771_beb57888-3951-409a-ae13-03a94b2e962e-square640.jpg}

\hypertarget{waldemar-gonzalez}{%
\section{Waldemar Gonzalez}\label{waldemar-gonzalez}}

d. White Plains, N.Y.

Teacher and social worker

Advertisement

\protect\hyperlink{after-bottom}{Continue reading the main story}

\hypertarget{site-index}{%
\subsection{Site Index}\label{site-index}}

\hypertarget{site-information-navigation}{%
\subsection{Site Information
Navigation}\label{site-information-navigation}}

\begin{itemize}
\tightlist
\item
  \href{https://help.nytimes.com/hc/en-us/articles/115014792127-Copyright-notice}{©~2020~The
  New York Times Company}
\end{itemize}

\begin{itemize}
\tightlist
\item
  \href{https://www.nytco.com/}{NYTCo}
\item
  \href{https://help.nytimes.com/hc/en-us/articles/115015385887-Contact-Us}{Contact
  Us}
\item
  \href{https://www.nytco.com/careers/}{Work with us}
\item
  \href{https://nytmediakit.com/}{Advertise}
\item
  \href{http://www.tbrandstudio.com/}{T Brand Studio}
\item
  \href{https://www.nytimes.com/privacy/cookie-policy\#how-do-i-manage-trackers}{Your
  Ad Choices}
\item
  \href{https://www.nytimes.com/privacy}{Privacy}
\item
  \href{https://help.nytimes.com/hc/en-us/articles/115014893428-Terms-of-service}{Terms
  of Service}
\item
  \href{https://help.nytimes.com/hc/en-us/articles/115014893968-Terms-of-sale}{Terms
  of Sale}
\item
  \href{https://spiderbites.nytimes.com}{Site Map}
\item
  \href{https://help.nytimes.com/hc/en-us}{Help}
\item
  \href{https://www.nytimes.com/subscription?campaignId=37WXW}{Subscriptions}
\end{itemize}
