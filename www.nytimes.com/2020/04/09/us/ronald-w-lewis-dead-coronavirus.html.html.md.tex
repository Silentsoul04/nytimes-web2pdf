Sections

SEARCH

\protect\hyperlink{site-content}{Skip to
content}\protect\hyperlink{site-index}{Skip to site index}

\href{https://www.nytimes.com/section/us}{U.S.}

\href{https://myaccount.nytimes.com/auth/login?response_type=cookie\&client_id=vi}{}

\href{https://www.nytimes.com/section/todayspaper}{Today's Paper}

\href{/section/us}{U.S.}\textbar{}Ronald Lewis, Preserver of New Orleans
Black Culture, Dies at 68

\url{https://nyti.ms/2UUBrs6}

\begin{itemize}
\item
\item
\item
\item
\item
\end{itemize}

\href{https://www.nytimes.com/news-event/coronavirus?action=click\&pgtype=Article\&state=default\&region=TOP_BANNER\&context=storylines_menu}{The
Coronavirus Outbreak}

\begin{itemize}
\tightlist
\item
  live\href{https://www.nytimes.com/2020/08/03/world/coronavirus-covid-19.html?action=click\&pgtype=Article\&state=default\&region=TOP_BANNER\&context=storylines_menu}{Latest
  Updates}
\item
  \href{https://www.nytimes.com/interactive/2020/us/coronavirus-us-cases.html?action=click\&pgtype=Article\&state=default\&region=TOP_BANNER\&context=storylines_menu}{Maps
  and Cases}
\item
  \href{https://www.nytimes.com/interactive/2020/science/coronavirus-vaccine-tracker.html?action=click\&pgtype=Article\&state=default\&region=TOP_BANNER\&context=storylines_menu}{Vaccine
  Tracker}
\item
  \href{https://www.nytimes.com/2020/08/02/us/covid-college-reopening.html?action=click\&pgtype=Article\&state=default\&region=TOP_BANNER\&context=storylines_menu}{College
  Reopening}
\item
  \href{https://www.nytimes.com/live/2020/08/03/business/stock-market-today-coronavirus?action=click\&pgtype=Article\&state=default\&region=TOP_BANNER\&context=storylines_menu}{Economy}
\end{itemize}

Advertisement

\protect\hyperlink{after-top}{Continue reading the main story}

Supported by

\protect\hyperlink{after-sponsor}{Continue reading the main story}

Those We've Lost

\hypertarget{ronald-lewis-preserver-of-new-orleans-black-culture-dies-at-68}{%
\section{Ronald Lewis, Preserver of New Orleans Black Culture, Dies at
68}\label{ronald-lewis-preserver-of-new-orleans-black-culture-dies-at-68}}

His colorful museum, the House of Dance and Feathers, was a monument to
the rich street culture of African-Americans. Mr. Lewis died of the
coronavirus.

\includegraphics{https://static01.nyt.com/images/2020/04/25/obituaries/07Lewis/07Lewis-articleLarge-v2.jpg?quality=75\&auto=webp\&disable=upscale}

\href{https://www.nytimes.com/by/steven-kurutz}{\includegraphics{https://static01.nyt.com/images/2018/09/25/multimedia/author-steven-kurutz/author-steven-kurutz-thumbLarge.png}}

By \href{https://www.nytimes.com/by/steven-kurutz}{Steven Kurutz}

\begin{itemize}
\item
  Published April 9, 2020Updated April 24, 2020
\item
  \begin{itemize}
  \item
  \item
  \item
  \item
  \item
  \end{itemize}
\end{itemize}

\emph{This obituary is part of a series about}
\href{https://www.nytimes.com/series/people-who-have-died-of-the-coronavirus}{\emph{people
who have died in the coronavirus pandemic}}\emph{.}

Ronald W. Lewis, whose colorful museum in the Lower Ninth Ward of
\href{https://www.nytimes.com/2020/04/13/us/coronavirus-new-orleans-mardi-gras.html}{New
Orleans} preserved the performance traditions and rich street culture of
its African-American population, died on March 20 in that city. He was
68.

The cause was the
\href{https://www.nytimes.com/2020/04/13/us/coronavirus-new-orleans-mardi-gras.html}{coronavirus},
said Brent Taylor, a nephew. Mr. Lewis had bypass surgery last year and
had been well until he contracted the virus in early March. He died at
Ochsner Hospital.

\href{houseofdanceandfeathers.org/}{The House of Dance and Feathers}, as
Mr. Lewis named his cultural institution, was an astonishing
treasure-trove of local history focused on the Lower Ninth Ward and
\href{https://www.neworleans.com/things-to-do/music/history-and-traditions/mardi-gras-indians/}{the
Mardi Gras Indians}, who dress in feathers, bangles and dazzling,
hand-sewn costumes as they dance through the city's black neighborhoods
on special occasions. (The group dates to the 1800s, formed by
African-Americans to pay tribute to the Native Americans who had helped
them during the time of slavery.)

The museum, about the size of a trailer and located in Mr. Lewis's
backyard on Tupelo Street, was packed floor to ceiling with costumes,
parade ephemera, photographs and memorabilia from African-American
social clubs.

Mr. Lewis spent his entire life in the Lower Ninth Ward, a working-class
black neighborhood, except for a difficult year in which he lived in
nearby Thibodaux, La., after being displaced in 2005 by Hurricane
Katrina. As he wrote in
\href{https://www.neighborhoodstoryproject.org/product-page/the-house-of-dance-feathers}{``The
House of Dance and Feathers''} (2009), a book he put together with
Rachel Breunlin, a cultural anthropologist, he believed that the history
of his culture should be told by ``someone who has lived the culture.''

Ronald William Lewis was born in New Orleans on July 17, 1951. He was
raised by his aunt and uncle, Rebecca and Irvin Dickerson, in a loving
but no-nonsense home on Deslonde Street in the Lower Ninth Ward. He
attended George Washington Carver High School but left before receiving
a diploma.

In the book
``\href{https://www.nytimes.com/2009/02/18/books/chapters/excerpt-the-nine.html?action=click\&module=RelatedCoverage\&pgtype=Article\&region=Footer}{Nine
Lives},'' an account of post-Katrina New Orleans as seen through the
eyes of nine residents, the journalist Dan Baum wrote that the Lower
Ninth Ward of Mr. Lewis's childhood had been made up of tidy houses
occupied by country transplants like his aunt (she was born on a sugar
plantation in Thibodaux) who held ``good waterfront jobs,'' harvested
``tomatoes from the garden and eggs from their chickens'' and gathered
at dinner tables crowded elbow-to-elbow with siblings and relatives
``while Mahalia Jackson sang from the radio.''

Mr. Lewis would remain in that tight-knit African-American world as he
became a coach and father figure to countless boys, co-founded the first
social and pleasure club in the Lower Ninth Ward, shared his table with
family, friends and strangers, and sought to tell the world about his
home.

``Right here in the Ninth Ward was where our people chased the American
dream,'' he told The Advocate.

Mr. Lewis is survived by his wife, Charlotte Hill Lewis; two sons,
Renaldo and Rashad; his stepsiblings, Larry Dickerson, Stella Keasley,
Cedric Lewis, Ebimola Ojusoku, Richard Lewis, Irvin Dickerson, Walter
Jones and Larry Lewis; seven grandchildren; two great-grandchildren; and
many nieces and nephews.

\href{https://www.nytimes.com/interactive/2020/obituaries/people-died-coronavirus-obituaries.html?action=click\&pgtype=Article\&state=default\&region=BELOW_MAIN_CONTENT\&context=covid_obits_promo}{}

\hypertarget{those-weve-lost}{%
\section{Those We've Lost}\label{those-weve-lost}}

The coronavirus pandemic has taken an incalculable death toll. This
series is designed to put names and faces to the numbers.

Read more

\includegraphics{https://static01.nyt.com/images/2020/07/30/obituaries/30Pedro/30Pedro-square640.jpg}

\hypertarget{bernaldina-josuxe9-pedro}{%
\section{Bernaldina José Pedro}\label{bernaldina-josuxe9-pedro}}

d. Boa Vista, Brazil

Leader among the Indigenous Macuxi

\includegraphics{https://static01.nyt.com/images/2020/07/31/obituaries/31Swing/merlin_175167783_8913bc90-0d64-43f3-a655-1bb1bf1601c9-square640.jpg}

\hypertarget{john-eric-swing}{%
\section{John Eric Swing}\label{john-eric-swing}}

d. Fountain Valley, Calif.

Champion of Filipino-Americans

\includegraphics{https://static01.nyt.com/images/2020/07/27/obituaries/27Victor/merlin_175001436_38b11f8e-227a-4e2c-9821-7618af9b2524-square640.jpg}

\hypertarget{victor-victor}{%
\section{Victor Victor}\label{victor-victor}}

d. Santo Domingo, Dominican Republic

Beloved musician of the Dominican Republic

\includegraphics{https://static01.nyt.com/images/2020/07/31/obituaries/31Negron/merlin_175160169_516322ae-fd23-4969-b6b2-193ced371105-square640.jpg}

\hypertarget{dr-eddie-negruxf3n}{%
\section{Dr. Eddie Negrón}\label{dr-eddie-negruxf3n}}

d. Fort Walton Beach, Fla.

Internist on Florida's Emerald Coast

\includegraphics{https://static01.nyt.com/images/2020/07/30/obituaries/30Dobson/merlin_175115928_f6b9271c-8f05-4fe1-a38a-5ca4a58f8935-square640.jpg}

\hypertarget{dobby-dobson}{%
\section{Dobby Dobson}\label{dobby-dobson}}

d. Coral Springs, Fla.

Jamaican singer and songwriter

\includegraphics{https://static01.nyt.com/images/2020/08/01/obituaries/28Gonzalez/merlin_175002771_beb57888-3951-409a-ae13-03a94b2e962e-square640.jpg}

\hypertarget{waldemar-gonzalez}{%
\section{Waldemar Gonzalez}\label{waldemar-gonzalez}}

d. White Plains, N.Y.

Teacher and social worker

Advertisement

\protect\hyperlink{after-bottom}{Continue reading the main story}

\hypertarget{site-index}{%
\subsection{Site Index}\label{site-index}}

\hypertarget{site-information-navigation}{%
\subsection{Site Information
Navigation}\label{site-information-navigation}}

\begin{itemize}
\tightlist
\item
  \href{https://help.nytimes.com/hc/en-us/articles/115014792127-Copyright-notice}{©~2020~The
  New York Times Company}
\end{itemize}

\begin{itemize}
\tightlist
\item
  \href{https://www.nytco.com/}{NYTCo}
\item
  \href{https://help.nytimes.com/hc/en-us/articles/115015385887-Contact-Us}{Contact
  Us}
\item
  \href{https://www.nytco.com/careers/}{Work with us}
\item
  \href{https://nytmediakit.com/}{Advertise}
\item
  \href{http://www.tbrandstudio.com/}{T Brand Studio}
\item
  \href{https://www.nytimes.com/privacy/cookie-policy\#how-do-i-manage-trackers}{Your
  Ad Choices}
\item
  \href{https://www.nytimes.com/privacy}{Privacy}
\item
  \href{https://help.nytimes.com/hc/en-us/articles/115014893428-Terms-of-service}{Terms
  of Service}
\item
  \href{https://help.nytimes.com/hc/en-us/articles/115014893968-Terms-of-sale}{Terms
  of Sale}
\item
  \href{https://spiderbites.nytimes.com}{Site Map}
\item
  \href{https://help.nytimes.com/hc/en-us}{Help}
\item
  \href{https://www.nytimes.com/subscription?campaignId=37WXW}{Subscriptions}
\end{itemize}
