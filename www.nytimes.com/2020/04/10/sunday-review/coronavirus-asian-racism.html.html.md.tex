Sections

SEARCH

\protect\hyperlink{site-content}{Skip to
content}\protect\hyperlink{site-index}{Skip to site index}

\href{https://www.nytimes.com/section/opinion/sunday}{Sunday Review}

\href{https://myaccount.nytimes.com/auth/login?response_type=cookie\&client_id=vi}{}

\href{https://www.nytimes.com/section/todayspaper}{Today's Paper}

\href{/section/opinion/sunday}{Sunday Review}\textbar{}When
Asian-Americans Have to Prove We Belong

\url{https://nyti.ms/2XtvuE3}

\begin{itemize}
\item
\item
\item
\item
\item
\end{itemize}

\href{https://www.nytimes.com/news-event/coronavirus?action=click\&pgtype=Article\&state=default\&region=TOP_BANNER\&context=storylines_menu}{The
Coronavirus Outbreak}

\begin{itemize}
\tightlist
\item
  live\href{https://www.nytimes.com/2020/08/01/world/coronavirus-covid-19.html?action=click\&pgtype=Article\&state=default\&region=TOP_BANNER\&context=storylines_menu}{Latest
  Updates}
\item
  \href{https://www.nytimes.com/interactive/2020/us/coronavirus-us-cases.html?action=click\&pgtype=Article\&state=default\&region=TOP_BANNER\&context=storylines_menu}{Maps
  and Cases}
\item
  \href{https://www.nytimes.com/interactive/2020/science/coronavirus-vaccine-tracker.html?action=click\&pgtype=Article\&state=default\&region=TOP_BANNER\&context=storylines_menu}{Vaccine
  Tracker}
\item
  \href{https://www.nytimes.com/interactive/2020/07/29/us/schools-reopening-coronavirus.html?action=click\&pgtype=Article\&state=default\&region=TOP_BANNER\&context=storylines_menu}{What
  School May Look Like}
\item
  \href{https://www.nytimes.com/live/2020/07/31/business/stock-market-today-coronavirus?action=click\&pgtype=Article\&state=default\&region=TOP_BANNER\&context=storylines_menu}{Economy}
\end{itemize}

Advertisement

\protect\hyperlink{after-top}{Continue reading the main story}

Supported by

\protect\hyperlink{after-sponsor}{Continue reading the main story}

news analysis

\hypertarget{when-asian-americans-have-to-prove-we-belong}{%
\section{When Asian-Americans Have to Prove We
Belong}\label{when-asian-americans-have-to-prove-we-belong}}

This isn't the first time we've been treated as a threat.

\includegraphics{https://static01.nyt.com/images/2020/04/12/opinion/12Yang1/12Yang1-articleLarge.jpg?quality=75\&auto=webp\&disable=upscale}

\includegraphics{https://static01.nyt.com/images/2020/04/24/us/author-jia-lynn-yang/author-jia-lynn-yang-thumbLarge.png}

By Jia Lynn Yang

Ms. Yang is the author of ``One Mighty and Irresistible Tide: The Epic
Struggle Over American Immigration, 1924-1965.''

\begin{itemize}
\item
  April 10, 2020
\item
  \begin{itemize}
  \item
  \item
  \item
  \item
  \item
  \end{itemize}
\end{itemize}

\href{https://cn.nytimes.com/opinion/20200420/coronavirus-asian-racism/}{阅读简体中文版}\href{https://cn.nytimes.com/opinion/20200420/coronavirus-asian-racism/zh-hant/}{閱讀繁體中文版}

The
\href{https://www.nytimes.com/2020/06/02/us/politics/african-americans-china-coronavirus.html}{coronavirus}
pandemic has unleashed a torrent of anti-Asian
\href{https://www.nytimes.com/2020/06/02/us/politics/african-americans-china-coronavirus.html}{racism}
in America that shows no signs of abating. Asian-Americans have been
spat on in the streets, harassed and insulted. Even children have been
attacked as our fellow citizens blame us for a virus that threatens our
families no less than any other household.

This is not the first season of darkness for Asian-Americans in this
country. Nearly 80 years ago, Japanese-Americans were forced from their
homes into barren internment camps. It did not matter how long they had
lived here or what they had contributed. They were still considered
foreign --- dangerous to their neighbors and a threat that had to be
contained.

Today, as then, Asian-Americans are wondering how to respond. The former
presidential candidate Andrew Yang on April 1
\href{https://www.washingtonpost.com/opinions/2020/04/01/andrew-yang-coronavirus-discrimination/}{called
on} the community to ``show our American-ness'' by pitching in to fight
the pandemic, invoking the example set by Japanese-Americans who proved
their loyalty to white America by volunteering to fight in World War II.

Books and movies have memorialized the dramatic story of those soldiers,
who made up the 442nd Regiment of the U.S. Army, one of the most
decorated American units from the war. Less known is what happened after
their sacrifice, when Japanese-American leaders leveraged the regiment's
heroism to end a ban on Asian immigration and to win naturalization
rights for all Asians.

But that political fight did not have a happy ending. Appeasing white
America did in fact achieve some victories --- even major ones --- for
the Asian-American community. As a strategy to defeat racism in the long
run, though, it fell painfully short.

The chief architect of this plan was Mike Masaoka, a son of Japanese
immigrants who grew up in Utah. He was 26 years old when the U.S.
military placed his family in an internment camp. Working for a group
called the Japanese American Citizens League, he told terrified families
to cooperate with evacuation orders and lobbied the Pentagon to allow
second-generation Japanese-Americans to enlist in the military.

Not all Japanese-Americans were on board with Masaoka's strategy. After
the Japanese American Citizens League's leadership endorsed military
service, fights broke out at an internment camp in Manzanar, Calif. But
Masaoka dismissed these protests, arguing ``the most effective weapon''
against Japanese-Americans being persecuted would be ``a record of
having fought valiantly for our country side by side with Americans of
other racial extraction.''

\hypertarget{latest-updates-global-coronavirus-outbreak}{%
\section{\texorpdfstring{\href{https://www.nytimes.com/2020/08/01/world/coronavirus-covid-19.html?action=click\&pgtype=Article\&state=default\&region=MAIN_CONTENT_1\&context=storylines_live_updates}{Latest
Updates: Global Coronavirus
Outbreak}}{Latest Updates: Global Coronavirus Outbreak}}\label{latest-updates-global-coronavirus-outbreak}}

Updated 2020-08-02T10:04:29.623Z

\begin{itemize}
\tightlist
\item
  \href{https://www.nytimes.com/2020/08/01/world/coronavirus-covid-19.html?action=click\&pgtype=Article\&state=default\&region=MAIN_CONTENT_1\&context=storylines_live_updates\#link-34047410}{The
  U.S. reels as July cases more than double the total of any other
  month.}
\item
  \href{https://www.nytimes.com/2020/08/01/world/coronavirus-covid-19.html?action=click\&pgtype=Article\&state=default\&region=MAIN_CONTENT_1\&context=storylines_live_updates\#link-780ec966}{Top
  U.S. officials work to break an impasse over the federal jobless
  benefit.}
\item
  \href{https://www.nytimes.com/2020/08/01/world/coronavirus-covid-19.html?action=click\&pgtype=Article\&state=default\&region=MAIN_CONTENT_1\&context=storylines_live_updates\#link-2bc8948}{Its
  outbreak untamed, Melbourne goes into even greater lockdown.}
\end{itemize}

\href{https://www.nytimes.com/2020/08/01/world/coronavirus-covid-19.html?action=click\&pgtype=Article\&state=default\&region=MAIN_CONTENT_1\&context=storylines_live_updates}{See
more updates}

More live coverage:
\href{https://www.nytimes.com/live/2020/07/31/business/stock-market-today-coronavirus?action=click\&pgtype=Article\&state=default\&region=MAIN_CONTENT_1\&context=storylines_live_updates}{Markets}

Among the first to join the 442nd Regiment, Masaoka participated in one
of the unit's most daring episodes: the 1944 rescue of a battalion of
Texans in the Vosges Mountains in eastern France, in which the regiment
suffered about 800 casualties. The dead included Masaoka's older brother
Ben.

When the war was over, and with Japanese-American sacrifices on the
battlefield still fresh, Masaoka believed there would never be a better
time to demand equal treatment.

It is hard to imagine now, with Asians the fastest-growing racial group
in America, but in the first half of the 20th century they were largely
blocked from entering the country and prevented from becoming citizens
after they arrived. The Chinese Exclusion Act of 1882 had banned the
immigration of Chinese laborers, and in 1924, Congress enacted a new set
of ethnic quotas dreamed up by eugenicists aimed at maintaining their
conception of America as a white and Anglo-Saxon nation. By designating
some races as more desirable than others, the law sharply restricted
Jewish and Italian immigration --- and banned nearly all Asians.

In the years that followed, a small group of Jewish lawmakers fought to
abolish the quotas. In 1952, when Congress embarked on its most
ambitious overhaul of the country's immigration system in decades, they
recognized their best opportunity in a generation.

Masaoka joined the fray, lobbying for a provision that gave Asians the
right to naturalize and for an easing of the nearly comprehensive ban on
Asian immigration.

But defeating the overall quota system proved more difficult. With the
Red Scare at its zenith, lawmakers were wary of admitting Eastern and
Southern European immigrants, whom they associated with radical
political activity. And so to the dismay of Jewish leaders, lawmakers
refused to abandon ethnic quotas giving preference to countries like
Britain.

Nor were African-American leaders pleased. By the end of World War II,
more than 250,000 black immigrants from the Caribbean had settled in the
United States, mostly in New York City and Chicago, counted within the
large quotas of their colonizers, the British, the French and the Dutch.
But the 1952 bill aimed to cap this flow of immigrants at 100 a year
from each of these European colonies.

\href{https://www.nytimes.com/news-event/coronavirus?action=click\&pgtype=Article\&state=default\&region=MAIN_CONTENT_3\&context=storylines_faq}{}

\hypertarget{the-coronavirus-outbreak-}{%
\subsubsection{The Coronavirus Outbreak
›}\label{the-coronavirus-outbreak-}}

\hypertarget{frequently-asked-questions}{%
\paragraph{Frequently Asked
Questions}\label{frequently-asked-questions}}

Updated July 27, 2020

\begin{itemize}
\item ~
  \hypertarget{should-i-refinance-my-mortgage}{%
  \paragraph{Should I refinance my
  mortgage?}\label{should-i-refinance-my-mortgage}}

  \begin{itemize}
  \tightlist
  \item
    \href{https://www.nytimes.com/article/coronavirus-money-unemployment.html?action=click\&pgtype=Article\&state=default\&region=MAIN_CONTENT_3\&context=storylines_faq}{It
    could be a good idea,} because mortgage rates have
    \href{https://www.nytimes.com/2020/07/16/business/mortgage-rates-below-3-percent.html?action=click\&pgtype=Article\&state=default\&region=MAIN_CONTENT_3\&context=storylines_faq}{never
    been lower.} Refinancing requests have pushed mortgage applications
    to some of the highest levels since 2008, so be prepared to get in
    line. But defaults are also up, so if you're thinking about buying a
    home, be aware that some lenders have tightened their standards.
  \end{itemize}
\item ~
  \hypertarget{what-is-school-going-to-look-like-in-september}{%
  \paragraph{What is school going to look like in
  September?}\label{what-is-school-going-to-look-like-in-september}}

  \begin{itemize}
  \tightlist
  \item
    It is unlikely that many schools will return to a normal schedule
    this fall, requiring the grind of
    \href{https://www.nytimes.com/2020/06/05/us/coronavirus-education-lost-learning.html?action=click\&pgtype=Article\&state=default\&region=MAIN_CONTENT_3\&context=storylines_faq}{online
    learning},
    \href{https://www.nytimes.com/2020/05/29/us/coronavirus-child-care-centers.html?action=click\&pgtype=Article\&state=default\&region=MAIN_CONTENT_3\&context=storylines_faq}{makeshift
    child care} and
    \href{https://www.nytimes.com/2020/06/03/business/economy/coronavirus-working-women.html?action=click\&pgtype=Article\&state=default\&region=MAIN_CONTENT_3\&context=storylines_faq}{stunted
    workdays} to continue. California's two largest public school
    districts --- Los Angeles and San Diego --- said on July 13, that
    \href{https://www.nytimes.com/2020/07/13/us/lausd-san-diego-school-reopening.html?action=click\&pgtype=Article\&state=default\&region=MAIN_CONTENT_3\&context=storylines_faq}{instruction
    will be remote-only in the fall}, citing concerns that surging
    coronavirus infections in their areas pose too dire a risk for
    students and teachers. Together, the two districts enroll some
    825,000 students. They are the largest in the country so far to
    abandon plans for even a partial physical return to classrooms when
    they reopen in August. For other districts, the solution won't be an
    all-or-nothing approach.
    \href{https://bioethics.jhu.edu/research-and-outreach/projects/eschool-initiative/school-policy-tracker/}{Many
    systems}, including the nation's largest, New York City, are
    devising
    \href{https://www.nytimes.com/2020/06/26/us/coronavirus-schools-reopen-fall.html?action=click\&pgtype=Article\&state=default\&region=MAIN_CONTENT_3\&context=storylines_faq}{hybrid
    plans} that involve spending some days in classrooms and other days
    online. There's no national policy on this yet, so check with your
    municipal school system regularly to see what is happening in your
    community.
  \end{itemize}
\item ~
  \hypertarget{is-the-coronavirus-airborne}{%
  \paragraph{Is the coronavirus
  airborne?}\label{is-the-coronavirus-airborne}}

  \begin{itemize}
  \tightlist
  \item
    The coronavirus
    \href{https://www.nytimes.com/2020/07/04/health/239-experts-with-one-big-claim-the-coronavirus-is-airborne.html?action=click\&pgtype=Article\&state=default\&region=MAIN_CONTENT_3\&context=storylines_faq}{can
    stay aloft for hours in tiny droplets in stagnant air}, infecting
    people as they inhale, mounting scientific evidence suggests. This
    risk is highest in crowded indoor spaces with poor ventilation, and
    may help explain super-spreading events reported in meatpacking
    plants, churches and restaurants.
    \href{https://www.nytimes.com/2020/07/06/health/coronavirus-airborne-aerosols.html?action=click\&pgtype=Article\&state=default\&region=MAIN_CONTENT_3\&context=storylines_faq}{It's
    unclear how often the virus is spread} via these tiny droplets, or
    aerosols, compared with larger droplets that are expelled when a
    sick person coughs or sneezes, or transmitted through contact with
    contaminated surfaces, said Linsey Marr, an aerosol expert at
    Virginia Tech. Aerosols are released even when a person without
    symptoms exhales, talks or sings, according to Dr. Marr and more
    than 200 other experts, who
    \href{https://academic.oup.com/cid/article/doi/10.1093/cid/ciaa939/5867798}{have
    outlined the evidence in an open letter to the World Health
    Organization}.
  \end{itemize}
\item ~
  \hypertarget{what-are-the-symptoms-of-coronavirus}{%
  \paragraph{What are the symptoms of
  coronavirus?}\label{what-are-the-symptoms-of-coronavirus}}

  \begin{itemize}
  \tightlist
  \item
    Common symptoms
    \href{https://www.nytimes.com/article/symptoms-coronavirus.html?action=click\&pgtype=Article\&state=default\&region=MAIN_CONTENT_3\&context=storylines_faq}{include
    fever, a dry cough, fatigue and difficulty breathing or shortness of
    breath.} Some of these symptoms overlap with those of the flu,
    making detection difficult, but runny noses and stuffy sinuses are
    less common.
    \href{https://www.nytimes.com/2020/04/27/health/coronavirus-symptoms-cdc.html?action=click\&pgtype=Article\&state=default\&region=MAIN_CONTENT_3\&context=storylines_faq}{The
    C.D.C. has also} added chills, muscle pain, sore throat, headache
    and a new loss of the sense of taste or smell as symptoms to look
    out for. Most people fall ill five to seven days after exposure, but
    symptoms may appear in as few as two days or as many as 14 days.
  \end{itemize}
\item ~
  \hypertarget{does-asymptomatic-transmission-of-covid-19-happen}{%
  \paragraph{Does asymptomatic transmission of Covid-19
  happen?}\label{does-asymptomatic-transmission-of-covid-19-happen}}

  \begin{itemize}
  \tightlist
  \item
    So far, the evidence seems to show it does. A widely cited
    \href{https://www.nature.com/articles/s41591-020-0869-5}{paper}
    published in April suggests that people are most infectious about
    two days before the onset of coronavirus symptoms and estimated that
    44 percent of new infections were a result of transmission from
    people who were not yet showing symptoms. Recently, a top expert at
    the World Health Organization stated that transmission of the
    coronavirus by people who did not have symptoms was ``very rare,''
    \href{https://www.nytimes.com/2020/06/09/world/coronavirus-updates.html?action=click\&pgtype=Article\&state=default\&region=MAIN_CONTENT_3\&context=storylines_faq\#link-1f302e21}{but
    she later walked back that statement.}
  \end{itemize}
\end{itemize}

Masaoka did not relish having the interests of Japanese-Americans pitted
against those of other immigrants. But to secure gains for his
community, he decided to abandon the other groups to support what became
known as the McCarran-Walter Act.

Winning the right to naturalize was a watershed moment in Asian-American
history. But the fight left others bitter. ``It is impossible to compute
the amount of harm which the Japanese American Citizens League and
Masaoka caused to effective opposition to this legislation,'' concluded
an analysis conducted by the American Jewish Congress.

It would take 13 more years of pressure from Jewish lawmakers and
activists and support from the Irish Catholic Kennedy family before
race-based quotas were finally abolished from the country's immigration
system. The success of the black civil rights movement also provided
moral momentum for the cause. In 1965, President Lyndon Johnson signed
the Immigration and Nationality Act, abolishing the race-based quotas
once and for all.

The 1965 law proved much more transformational for Asian-Americans than
Masaoka's 1952 effort, opening the gates to non-European immigration,
especially from Asia, in numbers that this country had never seen
before.

But Masaoka's battle was not inconsequential. Once Asian immigrants
became citizens, they were able to take advantage of the law's
preference for keeping families together by bringing relatives living
abroad to American shores. If Masaoka had not previously won Asians the
right to naturalize, far fewer of them would have been able to settle in
this country.

Yet because he decided to go it alone in his fight, Asian-Americans lost
a precious opportunity to build an alliance across racial groups --- a
pattern that persists to this day. The recent attacks have exposed the
fact that Asian-Americans remain dangerously isolated politically,
estranged from one another and from potential allies.

This current spasm of racism offers an opportunity for the tenuous
Asian-American community to come together as never before and demand
true equality --- for itself and for others. The battle cannot be won
alone.

Jia Lynn Yang is a deputy national editor at The New York Times and the
author of the forthcoming book ``\href{https://www.jialynnyang.com/}{One
Mighty and Irresistible Tide: The Epic Struggle Over American
Immigration, 1924-1965},'' from which this essay is adapted.

\emph{The Times is committed to publishing}
\href{https://www.nytimes.com/2019/01/31/opinion/letters/letters-to-editor-new-york-times-women.html}{\emph{a
diversity of letters}} \emph{to the editor. We'd like to hear what you
think about this or any of our articles. Here are some}
\href{https://help.nytimes.com/hc/en-us/articles/115014925288-How-to-submit-a-letter-to-the-editor}{\emph{tips}}\emph{.
And here's our email:}
\href{mailto:letters@nytimes.com}{\emph{letters@nytimes.com}}\emph{.}

\emph{Follow The New York Times Opinion section on}
\href{https://www.facebook.com/nytopinion}{\emph{Facebook}}\emph{,}
\href{http://twitter.com/NYTOpinion}{\emph{Twitter (@NYTopinion)}}
\emph{and}
\href{https://www.instagram.com/nytopinion/}{\emph{Instagram}}\emph{.}

Advertisement

\protect\hyperlink{after-bottom}{Continue reading the main story}

\hypertarget{site-index}{%
\subsection{Site Index}\label{site-index}}

\hypertarget{site-information-navigation}{%
\subsection{Site Information
Navigation}\label{site-information-navigation}}

\begin{itemize}
\tightlist
\item
  \href{https://help.nytimes.com/hc/en-us/articles/115014792127-Copyright-notice}{©~2020~The
  New York Times Company}
\end{itemize}

\begin{itemize}
\tightlist
\item
  \href{https://www.nytco.com/}{NYTCo}
\item
  \href{https://help.nytimes.com/hc/en-us/articles/115015385887-Contact-Us}{Contact
  Us}
\item
  \href{https://www.nytco.com/careers/}{Work with us}
\item
  \href{https://nytmediakit.com/}{Advertise}
\item
  \href{http://www.tbrandstudio.com/}{T Brand Studio}
\item
  \href{https://www.nytimes.com/privacy/cookie-policy\#how-do-i-manage-trackers}{Your
  Ad Choices}
\item
  \href{https://www.nytimes.com/privacy}{Privacy}
\item
  \href{https://help.nytimes.com/hc/en-us/articles/115014893428-Terms-of-service}{Terms
  of Service}
\item
  \href{https://help.nytimes.com/hc/en-us/articles/115014893968-Terms-of-sale}{Terms
  of Sale}
\item
  \href{https://spiderbites.nytimes.com}{Site Map}
\item
  \href{https://help.nytimes.com/hc/en-us}{Help}
\item
  \href{https://www.nytimes.com/subscription?campaignId=37WXW}{Subscriptions}
\end{itemize}
