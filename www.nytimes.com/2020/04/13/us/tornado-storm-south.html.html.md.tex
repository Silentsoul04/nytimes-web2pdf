Sections

SEARCH

\protect\hyperlink{site-content}{Skip to
content}\protect\hyperlink{site-index}{Skip to site index}

\href{https://www.nytimes.com/section/us}{U.S.}

\href{https://myaccount.nytimes.com/auth/login?response_type=cookie\&client_id=vi}{}

\href{https://www.nytimes.com/section/todayspaper}{Today's Paper}

\href{/section/us}{U.S.}\textbar{}Dozens Are Killed as Tornadoes and
Severe Weather Strike Southern States

\url{https://nyti.ms/2y9Pu40}

\begin{itemize}
\item
\item
\item
\item
\item
\end{itemize}

Advertisement

\protect\hyperlink{after-top}{Continue reading the main story}

Supported by

\protect\hyperlink{after-sponsor}{Continue reading the main story}

\hypertarget{dozens-are-killed-as-tornadoes-and-severe-weather-strike-southern-states}{%
\section{Dozens Are Killed as Tornadoes and Severe Weather Strike
Southern
States}\label{dozens-are-killed-as-tornadoes-and-severe-weather-strike-southern-states}}

The storm carved a destructive path across six states on Sunday and
Monday, causing widespread damage and cutting power to tens of thousands
of customers.

\includegraphics{https://static01.nyt.com/images/2020/05/12/multimedia/12xp-weather-update-top/merlin_171556503_069ce5d8-8f92-4a20-8998-ec05838f28e8-articleLarge.jpg?quality=75\&auto=webp\&disable=upscale}

By Ellen Ann Fentress and
\href{https://www.nytimes.com/by/richard-fausset}{Richard Fausset}

\begin{itemize}
\item
  April 13, 2020
\item
  \begin{itemize}
  \item
  \item
  \item
  \item
  \item
  \end{itemize}
\end{itemize}

BASSFIELD, Miss. --- Like most Americans, Mamie Harper and her husband
had tucked themselves away at home in an effort to keep the coronavirus
at bay. On Easter Sunday, they listened to an audio feed of their church
service while huddled indoors.

But a different kind of trouble soon found them. A tornado --- one of
dozens that tore across the Southeast this weekend --- roared over their
street on Sunday afternoon, snapping trees, blowing away keepsakes and
launching cars from their parking spots. When it was over, debris
temporarily kept emergency medical workers from driving onto Ms.
Harper's block. Her daughter pulled her from the wreckage of her small
white house, and soon other neighbors, many of them relatives, also came
out to try to help.

Ms. Harper, 68, felt an odd mix of gratitude and wariness. ``You don't
know,'' she said Monday, standing on the shoulder of her ruined street
in a surgical mask because of the coronavirus. ``Even though they're
coming out of the goodness of their heart, they may not know they've got
it.''

The devastating weather system started Sunday and barreled across the
region into Monday, leaving destruction, blackouts and heartbreak in its
path. More than 30 people died --- including at least 11 in Mississippi,
nine in South Carolina and eight in Georgia --- making it one of the
most significant natural disasters in the country since government
officials began ordering people to stay home and away from one another
in an effort to stop the spread of the virus.

\includegraphics{https://static01.nyt.com/images/2020/04/13/us/13STORMS/merlin_171557859_2565cf48-9913-46b5-9263-be88ff2ef91f-articleLarge.jpg?quality=75\&auto=webp\&disable=upscale}

The response laid bare the new complications the pandemic may create for
neighbors, victims and disaster response officials alike in coming
months as a shuttered nation braces for the looming seasons of floods,
fires and storms.

Emergency agencies are now being forced into new realms of improvisation
and creativity as they attempt to provide shelter and succor, while
simultaneously minding the presence of a quieter killer. Already,
officials are doing what they can to avoid housing evacuees in large
shelters, which could prove as dangerous for spreading the coronavirus
as any cruise ship.

``This is a collision course of conflicting strategies to deal with the
natural disasters and the pandemic simultaneously,'' said Irwin
Redlener, the director of the National Center for Disaster Preparedness
at Columbia University.

The ``bull's-eye'' of the storm system encircled a swath of the South
and brought twisters, high winds and intense rain into parts of southern
Kentucky, eastern Georgia, Florida, Mississippi, Louisiana and Arkansas,
said Katie Martin, a meteorologist with the National Weather Service.

In Alabama, Gov. Kay Ivey suspended social distancing and other
coronavirus-related orders, but only if they might get in the way of an
effective response.

Though Alabama escaped without any reported fatalities, neighboring
Mississippi appeared hardest hit, with at least 11 deaths as of Monday
afternoon. The American Red Cross, which runs most of the temporary
shelters in the nation, opened a couple of Mississippi sites that
attracted dozens of evacuees.

But the organization moved all of those people into hotel rooms before
Monday morning in an effort to comply with the spirit of social
distancing rules, said Trevor Riggen, the Red Cross's senior vice
president of Disaster Cycle Services.

In Jones County, in southeast Mississippi, more than 50 people showed up
to take shelter in the region's fortified safe room, built to federal
specifications to withstand big storms. Paul Sheffield, the executive
director of the county emergency management office, said the evacuees
were greeted by volunteers who offered hand sanitizer and insisted they
wear masks. They rode out the storm in marked-off areas that kept
families about seven and a half feet away from each other.

Image

The sisters Aula Montgomery and Juanita Rushing and their cousin Ellis
Ratcliff stand outside Ms. Rushing's tornado-ravaged home near
Tylertown, Miss., on Monday.Credit...Caleb Mccluskey/The
Enterprise-Journal, via Associated Press

Experts say these efforts, though admirable, may not work in the event
of a disaster on the scale of Hurricane Katrina, in which case there may
be no choice but to house people in larger shelters.

``The shelter environment is the last thing we want,'' Mr. Redlener
said. ``It's virtually impossible to sustain social distancing and
enforce appropriate public health measures in those circumstances.''

Plus, he said, there is the likelihood that many people in shelters
would be there without the medications they would need for underlying
health conditions.

Hurricane season begins June 1, and experts said it may be more active
than usual because of warm water temperatures in the Gulf of Mexico,
Atlantic and Caribbean. That has officials like Shannon Scaff, emergency
management director for the coastal city of Charleston, S.C.,
particularly stressed.

On Monday, Mr. Scaff said coronavirus-related stay-at-home measures have
meant his team has been unable to hold community meetings to advise
people how to prepare for a hurricane evacuation.

And he was still not sure how social distancing might work if an
evacuation was large-scale.

``Here's what I know,'' he said. ``If an evacuation order is given
because of a hurricane, Covid or not, I'm telling you to get out of
here.''

Across the country, emergency officials said that, no matter the
circumstance, they would encourage people to adhere to social distancing
measures as best they could.

Officials in California said they were evaluating how to take into
account the virus in responding to wildfires, earthquakes, floods and
other disasters that might arise in the coming months.

``We're disaster-prone, so you have to be prepared for multiple things
at multiple times,'' said Kim Zagaris, a former state fire and rescue
chief for the California governor's Office of Emergency Services,
\href{https://ktla.com/news/california/california-agencies-trying-to-figure-out-earthquake-wildfire-flood-evacuation-plans-amid-pandemic/}{according
to The Associated Press}.

James Kendra, director of the Disaster Research Center at the University
of Delaware, said he is worried that the Federal Emergency Management
Agency, which has been tasked with taking a lead on the pandemic
response, could find itself stretched thin if natural disasters pile up.

FEMA has struggled with understaffing in recent years, according to a
recent article in Insurance Journal, though the agency has contended
that it is prepared for the challenges ahead.

Image

A destroyed house in Chattanooga, Tenn., after severe storms hit the
area.Credit...Terry Stolt/Chattanooga Times Free Press, via Associated
Press

In a statement Monday, a FEMA spokesperson said that more than 2,900
employees out of 20,500 were supporting the pandemic response, and that
it could call in other federal employees to help as part of a ``surge
capacity force.''

Nongovernmental groups are also feeling challenged. Samaritan's Purse,
the North Carolina-based Christian relief group headed by the Rev.
Franklin Graham, is planning a vigorous response to the Southeastern
storms, sending in crews to help people repair and rebuild homes.

On Monday, Mr. Graham said his volunteers would wear masks and gloves
and try to abide by social distancing rules as they worked. The problem,
for now, was finding enough of them.

``There are going to be volunteers who normally would come but may be a
little reluctant because of the coronavirus,'' he said.

The scenes of double crises were commonplace on Monday. Around
Chattanooga, Tenn., where two people died, residents of battered
neighborhoods visited home improvement stores in masks and gloves,
picking up tarps and plywood to cover up shattered windows and doors.

Also hard-hit was Walterboro, S.C., a historic, moss-draped community of
5,400 residents, about 45 minutes west of Charleston. A tornado tore
through the heart of downtown early Monday, dropping trees on two dozen
buildings and killing a woman who had been sheltering with her family.
The storm also damaged or destroyed better than half of the 52 airplanes
at the Lowcountry regional private airport, including a Douglas C-54
that delivered supplies during the Berlin airlift of 1948-49.

At the Colleton Courtyard assisted living facility downtown, Maxwell
Lockwood, its manager, said he was thankful the tornado had not done
much damage. And that none of the 34 residents of the home, he said,
have tested positive for the virus.

Ellen Ann Fentress reported from Bassfield, Miss., and Richard Fausset
from Atlanta. Reporting was contributed by Rick Rojas from Atlanta;
Chris Dixon from Walterboro, S.C.; Cari Gervin from Chattanooga, Tenn.;
Christine Hauser from Nantucket, Mass.; and Sandra E. Garcia, Aimee
Ortiz, Mihir Zaveri and Jenny Gross from New York.

Advertisement

\protect\hyperlink{after-bottom}{Continue reading the main story}

\hypertarget{site-index}{%
\subsection{Site Index}\label{site-index}}

\hypertarget{site-information-navigation}{%
\subsection{Site Information
Navigation}\label{site-information-navigation}}

\begin{itemize}
\tightlist
\item
  \href{https://help.nytimes.com/hc/en-us/articles/115014792127-Copyright-notice}{©~2020~The
  New York Times Company}
\end{itemize}

\begin{itemize}
\tightlist
\item
  \href{https://www.nytco.com/}{NYTCo}
\item
  \href{https://help.nytimes.com/hc/en-us/articles/115015385887-Contact-Us}{Contact
  Us}
\item
  \href{https://www.nytco.com/careers/}{Work with us}
\item
  \href{https://nytmediakit.com/}{Advertise}
\item
  \href{http://www.tbrandstudio.com/}{T Brand Studio}
\item
  \href{https://www.nytimes.com/privacy/cookie-policy\#how-do-i-manage-trackers}{Your
  Ad Choices}
\item
  \href{https://www.nytimes.com/privacy}{Privacy}
\item
  \href{https://help.nytimes.com/hc/en-us/articles/115014893428-Terms-of-service}{Terms
  of Service}
\item
  \href{https://help.nytimes.com/hc/en-us/articles/115014893968-Terms-of-sale}{Terms
  of Sale}
\item
  \href{https://spiderbites.nytimes.com}{Site Map}
\item
  \href{https://help.nytimes.com/hc/en-us}{Help}
\item
  \href{https://www.nytimes.com/subscription?campaignId=37WXW}{Subscriptions}
\end{itemize}
