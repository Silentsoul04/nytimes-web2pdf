Sections

SEARCH

\protect\hyperlink{site-content}{Skip to
content}\protect\hyperlink{site-index}{Skip to site index}

\href{https://www.nytimes.com/section/books}{Books}

\href{https://myaccount.nytimes.com/auth/login?response_type=cookie\&client_id=vi}{}

\href{https://www.nytimes.com/section/todayspaper}{Today's Paper}

\href{/section/books}{Books}\textbar{}Henry F. Graff, Columbia Historian
of Presidents, Dies at 98

\url{https://nyti.ms/3erfcll}

\begin{itemize}
\item
\item
\item
\item
\item
\end{itemize}

\href{https://www.nytimes.com/news-event/coronavirus?action=click\&pgtype=Article\&state=default\&region=TOP_BANNER\&context=storylines_menu}{The
Coronavirus Outbreak}

\begin{itemize}
\tightlist
\item
  live\href{https://www.nytimes.com/2020/08/03/world/coronavirus-covid-19.html?action=click\&pgtype=Article\&state=default\&region=TOP_BANNER\&context=storylines_menu}{Latest
  Updates}
\item
  \href{https://www.nytimes.com/interactive/2020/us/coronavirus-us-cases.html?action=click\&pgtype=Article\&state=default\&region=TOP_BANNER\&context=storylines_menu}{Maps
  and Cases}
\item
  \href{https://www.nytimes.com/interactive/2020/science/coronavirus-vaccine-tracker.html?action=click\&pgtype=Article\&state=default\&region=TOP_BANNER\&context=storylines_menu}{Vaccine
  Tracker}
\item
  \href{https://www.nytimes.com/2020/08/02/us/covid-college-reopening.html?action=click\&pgtype=Article\&state=default\&region=TOP_BANNER\&context=storylines_menu}{College
  Reopening}
\item
  \href{https://www.nytimes.com/live/2020/08/03/business/stock-market-today-coronavirus?action=click\&pgtype=Article\&state=default\&region=TOP_BANNER\&context=storylines_menu}{Economy}
\end{itemize}

Advertisement

\protect\hyperlink{after-top}{Continue reading the main story}

Supported by

\protect\hyperlink{after-sponsor}{Continue reading the main story}

Those We've Lost

\hypertarget{henry-f-graff-columbia-historian-of-presidents-dies-at-98}{%
\section{Henry F. Graff, Columbia Historian of Presidents, Dies at
98}\label{henry-f-graff-columbia-historian-of-presidents-dies-at-98}}

A professor and author, he had translated decrypted Japanese messages in
World War II that revealed German defenses for D-Day and Tokyo's
imminent capitulation.

\includegraphics{https://static01.nyt.com/images/2020/04/25/obituaries/14Graff1/14Graff1-articleLarge.jpg?quality=75\&auto=webp\&disable=upscale}

\href{https://www.nytimes.com/by/sam-roberts}{\includegraphics{https://static01.nyt.com/images/2018/02/20/multimedia/author-sam-roberts/author-sam-roberts-thumbLarge.jpg}}

By \href{https://www.nytimes.com/by/sam-roberts}{Sam Roberts}

\begin{itemize}
\item
  Published April 15, 2020Updated April 24, 2020
\item
  \begin{itemize}
  \item
  \item
  \item
  \item
  \item
  \end{itemize}
\end{itemize}

\emph{This obituary is part of a series about people who died in the
coronavirus pandemic. Read about others}
\href{https://www.nytimes.com/series/people-who-have-died-of-the-coronavirus}{\emph{here}}\emph{.}

Henry Graff, a Columbia University professor who studied the past and
present as a scholar of the presidency and, as an Army translator during
World War II, foreshadowed the future from decrypted Japanese diplomatic
messages, died on April 7 in a hospital in Greenwich, Conn. He was 98.

The cause was complications of the new coronavirus, said Molly Morse,
his granddaughter. He lived in Scarsdale, N.Y.

An author of 12 books and countless articles and a regular contributor
to The New York Times Book Review, Professor Graff was best known as a
keen observer of the men who occupied the White House --- 17 of whom
presided during his lifetime.

He knew several personally, including Harry S. Truman and Gerald R.
Ford, who sat in on his popular seminar at Columbia; and Lyndon B.
Johnson and Bill Clinton, both of whom appointed him to presidential
panels.

Playing Boswell to Johnson's advisers on Vietnam, Professor Graff wrote
the book, ``The Tuesday Cabinet: Deliberation and Decision on Peace and
War Under Lyndon B. Johnson'' (1970), which he later described as ``an
effort at explaining the administration's Vietnam policy as the
president and his chief aides said they understood it.''

While he rhapsodized about teaching at Columbia, which he did from 1946
until he retired in 1991, he exulted in his exploits as an Army
translator shortly after Pearl Harbor. He was assigned to the Signal
Intelligence Service in Washington, a precursor of the National Security
Agency, because he understood Japanese. He had studied the language more
or less by chance as a prerequisite to minoring in Asian history during
a semester at Columbia when Chinese language courses were not being
offered.

In November 1943, he translated part of a message deciphered from
Japan's complex Purple code that had been sent by Hiroshi Oshima, the
Japanese ambassador in Berlin, to the foreign office in Tokyo detailing
German plans to repel the expected Allied invasion of northern France on
D-Day.

Professor Graff quoted General George C. Marshall as saying ``that
message was worth 25,000 men's lives.''

``I, a kid, would come in and look to see if we had any messages from
Oshima to the Japanese Foreign Office,''
\href{https://www.oldnewyorkstories.com/post/11670862477/henry-graff}{he
recalled}. ``And I was reading messages that reported his conversation
with Hitler the day before. I cannot tell you other than I felt that I
was at the center of the universe.''

Nine months later, he translated another intercepted message, this one
from Japan to the Soviet Union.

``I was the first American, the first member of the Allied side, to know
Japan was going to get out of the War,'' he said, ``because I was
working at two in the morning in 1945 shortly after Hiroshima, and I got
this message asking Bern, Switzerland, to help get them out of the
war.''

If Japan had not surrendered, Professor Graff was expected to be
deployed with the Allied invasion forces.

Henry Franklin Graff was born on Aug. 11, 1921, in Manhattan to Samuel
F. Graff, a salesman in the Garment District, and Florence (Morris)
Graff, both descendants of Jewish immigrants from Germany. His maternal
grandmother's family had a clothing store in East Harlem.

Raised in the Inwood section of Manhattan, he graduated from George
Washington High School at 16 and earned his bachelor of social science
degree from City College in 1941. He was working toward his master's at
Columbia (``I was the first Jew in the Columbia History Department,'' he
said) when he enlisted. After the war, he taught history at City College
before joining the Columbia faculty in 1946 and earning his doctorate in
1949.

He married Edith Krantz in 1946; she died in 2019. He is survived by
their two daughters, Iris Morse and Ellen Graff; five grandchildren; and
five great-grandchildren. His twin sister, Myra Balber, died.

Professor Graff wrote ``The Modern Researcher'' (1957) with the
historian and cultural critic
\href{https://www.nytimes.com/2012/10/26/arts/jacques-barzun-historian-and-scholar-dies-at-104.html}{Jacques
Barzun}, a colleague at Columbia; and ``The Presidents: A Reference
History'' (1984). He was honored with Columbia's Great Teacher Award and
with its Mark Van Doren Award for Teaching.

Kenneth T. Jackson, one of his successors as chairman of the history
department, said that Professor Graff was that rare Columbia professor
who also received an honorary doctorate from the university.

Professor Graff was chairman of the juries for the Pulitzer Prize in
American history and the Bancroft Prize in history by Columbia
University Libraries. He was appointed by President Johnson to the
National Historical Publications Commission, and by President Clinton to
the President John F. Kennedy Assassination Records Review Board.

Professor Graff once said that American scholars of the United States
are twice blessed, because the nation is young enough so that its
historical record is largely intact, and because historians have the
academic freedom to analyze that record critically.

He regarded the presidency as ``the litmus paper for testing the
nation's aims and character.''

``When wearing their historical laurels and burdens,'' he wrote, ``they
symbolize even better than the Caesars the fascinating disparity between
vast opportunity for personal glory and the uncommonness of the gift to
use it wisely.''

But, he added, ``in offering themselves for posterity's judgment, they
confront a standard no one has defined.''

\href{https://www.nytimes.com/interactive/2020/obituaries/people-died-coronavirus-obituaries.html?action=click\&pgtype=Article\&state=default\&region=BELOW_MAIN_CONTENT\&context=covid_obits_promo}{}

\hypertarget{those-weve-lost}{%
\section{Those We've Lost}\label{those-weve-lost}}

The coronavirus pandemic has taken an incalculable death toll. This
series is designed to put names and faces to the numbers.

Read more

\includegraphics{https://static01.nyt.com/images/2020/07/30/obituaries/30Pedro/30Pedro-square640.jpg}

\hypertarget{bernaldina-josuxe9-pedro}{%
\section{Bernaldina José Pedro}\label{bernaldina-josuxe9-pedro}}

d. Boa Vista, Brazil

Leader among the Indigenous Macuxi

\includegraphics{https://static01.nyt.com/images/2020/07/31/obituaries/31Swing/merlin_175167783_8913bc90-0d64-43f3-a655-1bb1bf1601c9-square640.jpg}

\hypertarget{john-eric-swing}{%
\section{John Eric Swing}\label{john-eric-swing}}

d. Fountain Valley, Calif.

Champion of Filipino-Americans

\includegraphics{https://static01.nyt.com/images/2020/07/27/obituaries/27Victor/merlin_175001436_38b11f8e-227a-4e2c-9821-7618af9b2524-square640.jpg}

\hypertarget{victor-victor}{%
\section{Victor Victor}\label{victor-victor}}

d. Santo Domingo, Dominican Republic

Beloved musician of the Dominican Republic

\includegraphics{https://static01.nyt.com/images/2020/07/31/obituaries/31Negron/merlin_175160169_516322ae-fd23-4969-b6b2-193ced371105-square640.jpg}

\hypertarget{dr-eddie-negruxf3n}{%
\section{Dr. Eddie Negrón}\label{dr-eddie-negruxf3n}}

d. Fort Walton Beach, Fla.

Internist on Florida's Emerald Coast

\includegraphics{https://static01.nyt.com/images/2020/07/30/obituaries/30Dobson/merlin_175115928_f6b9271c-8f05-4fe1-a38a-5ca4a58f8935-square640.jpg}

\hypertarget{dobby-dobson}{%
\section{Dobby Dobson}\label{dobby-dobson}}

d. Coral Springs, Fla.

Jamaican singer and songwriter

\includegraphics{https://static01.nyt.com/images/2020/08/01/obituaries/28Gonzalez/merlin_175002771_beb57888-3951-409a-ae13-03a94b2e962e-square640.jpg}

\hypertarget{waldemar-gonzalez}{%
\section{Waldemar Gonzalez}\label{waldemar-gonzalez}}

d. White Plains, N.Y.

Teacher and social worker

Advertisement

\protect\hyperlink{after-bottom}{Continue reading the main story}

\hypertarget{site-index}{%
\subsection{Site Index}\label{site-index}}

\hypertarget{site-information-navigation}{%
\subsection{Site Information
Navigation}\label{site-information-navigation}}

\begin{itemize}
\tightlist
\item
  \href{https://help.nytimes.com/hc/en-us/articles/115014792127-Copyright-notice}{©~2020~The
  New York Times Company}
\end{itemize}

\begin{itemize}
\tightlist
\item
  \href{https://www.nytco.com/}{NYTCo}
\item
  \href{https://help.nytimes.com/hc/en-us/articles/115015385887-Contact-Us}{Contact
  Us}
\item
  \href{https://www.nytco.com/careers/}{Work with us}
\item
  \href{https://nytmediakit.com/}{Advertise}
\item
  \href{http://www.tbrandstudio.com/}{T Brand Studio}
\item
  \href{https://www.nytimes.com/privacy/cookie-policy\#how-do-i-manage-trackers}{Your
  Ad Choices}
\item
  \href{https://www.nytimes.com/privacy}{Privacy}
\item
  \href{https://help.nytimes.com/hc/en-us/articles/115014893428-Terms-of-service}{Terms
  of Service}
\item
  \href{https://help.nytimes.com/hc/en-us/articles/115014893968-Terms-of-sale}{Terms
  of Sale}
\item
  \href{https://spiderbites.nytimes.com}{Site Map}
\item
  \href{https://help.nytimes.com/hc/en-us}{Help}
\item
  \href{https://www.nytimes.com/subscription?campaignId=37WXW}{Subscriptions}
\end{itemize}
