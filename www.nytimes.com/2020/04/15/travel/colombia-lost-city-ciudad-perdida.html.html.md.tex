Sections

SEARCH

\protect\hyperlink{site-content}{Skip to
content}\protect\hyperlink{site-index}{Skip to site index}

\href{/section/travel}{Travel}\textbar{}A Visual Trek Through the
Sweltering Jungle: In Search of Colombia's `Lost City'

\url{https://nyti.ms/34E5A2b}

\begin{itemize}
\item
\item
\item
\item
\item
\item
\end{itemize}

\href{https://www.nytimes.com/spotlight/at-home?action=click\&pgtype=Article\&state=default\&region=TOP_BANNER\&context=at_home_menu}{At
Home}

\begin{itemize}
\tightlist
\item
  \href{https://www.nytimes.com/2020/07/28/books/time-for-a-literary-road-trip.html?action=click\&pgtype=Article\&state=default\&region=TOP_BANNER\&context=at_home_menu}{Take:
  A Literary Road Trip}
\item
  \href{https://www.nytimes.com/2020/07/29/magazine/bored-with-your-home-cooking-some-smoky-eggplant-will-fix-that.html?action=click\&pgtype=Article\&state=default\&region=TOP_BANNER\&context=at_home_menu}{Cook:
  Smoky Eggplant}
\item
  \href{https://www.nytimes.com/2020/07/27/travel/moose-michigan-isle-royale.html?action=click\&pgtype=Article\&state=default\&region=TOP_BANNER\&context=at_home_menu}{Look
  Out: For Moose}
\item
  \href{https://www.nytimes.com/interactive/2020/at-home/even-more-reporters-editors-diaries-lists-recommendations.html?action=click\&pgtype=Article\&state=default\&region=TOP_BANNER\&context=at_home_menu}{Explore:
  Reporters' Obsessions}
\end{itemize}

\includegraphics{https://static01.nyt.com/images/2020/04/15/travel/15travel-colombia-header-1f/15travel-colombia-header-1f-videoSixteenByNineJumbo1600.jpg}

The World Through a Lens

\hypertarget{a-visual-trek-through-the-sweltering-jungle-in-search-of-colombias-lost-city}{%
\section{A Visual Trek Through the Sweltering Jungle: In Search of
Colombia's `Lost
City'}\label{a-visual-trek-through-the-sweltering-jungle-in-search-of-colombias-lost-city}}

Ciudad Perdida, an ancient city that predates Machu Picchu by several
hundred years, has become one of South America's most rewarding
adventure destinations.

Credit...

Supported by

\protect\hyperlink{after-sponsor}{Continue reading the main story}

\href{https://www.nytimes.com/by/stephen-hiltner}{\includegraphics{https://static01.nyt.com/images/2018/06/13/multimedia/author-stephen-hiltner/author-stephen-hiltner-thumbLarge-v2.jpg}}

Photographs and Text by
\href{https://www.nytimes.com/by/stephen-hiltner}{Stephen Hiltner}

\begin{itemize}
\item
  Published April 15, 2020Updated May 7, 2020
\item
  \begin{itemize}
  \item
  \item
  \item
  \item
  \item
  \item
  \end{itemize}
\end{itemize}

\emph{With travel restrictions in place worldwide, we're turning to
photojournalists who can help transport you, virtually, to some of our
planet's most beautiful and intriguing places. We're calling this new
series}
\href{https://www.nytimes.com/column/the-world-through-a-lens}{\emph{``The
World Through a Lens.''}} \emph{This week, Stephen Hiltner, an editor on
the Travel desk, invites you to join him on an arduous multiday hike to
an archaeological site in Colombia.}

\begin{center}\rule{0.5\linewidth}{\linethickness}\end{center}

It was the third day of our trek through the Colombian jungle, just
before 5 a.m., when Ailyn Paul, one of our guides, came by to rouse us
from our narrow bunks.

``Sudados!'' she said, calling out our group's nickname --- The Sweaty
Ones --- through the scant privacy of our mosquito netting. ``Wake up!
It's time to visit the Lost City.''

A little over an hour later --- after reluctantly pulling on a damp
long-sleeved shirt and gulping down eggs and
\href{https://cooking.nytimes.com/recipes/1015180-colombian-corn-and-cheese-arepas}{arepas}
at our campsite --- I hopped across the Buritaca River and found myself
staring up at the base of some 1,200 stone steps. At the top lay our
destination: Ciudad Perdida, Colombia's ``Lost City,'' the home of an
ancient people, the Tairona, who occupied this pocket of South America
for more than a millennium before the first Spanish settlements appeared
here in the early 1500s.

Image

Image

Lost to memory for 400 years before its accidental rediscovery in the
1970s, Ciudad Perdida is stunning in its scale and complexity: an
80-acre site --- parts of which date to the seventh century --- with
terraces, plazas, canals, storehouses, stone paths and staircases, many
of them remarkably preserved.

At its peak, archaeologists have deduced, about 2,500 people may have
lived here. But exploring Ciudad Perdida is a hard-earned prize: The
only way to reach the site is by completing the nearly 30-mile
round-trip trek through the unbearably hot, mountainous,
mosquito-swirling Colombian rainforest that surrounds it.

\includegraphics{https://static01.nyt.com/images/2020/04/12/travel/15travel-colombia-04/15travel-colombia-04-mobileMasterAt3x.jpg}\includegraphics{https://static01.nyt.com/images/2020/03/03/travel/15travel-colombia-05/15travel-colombia-05-mobileMasterAt3x.jpg}\includegraphics{https://static01.nyt.com/images/2020/03/03/travel/15travel-colombia-10/15travel-colombia-10-mobileMasterAt3x.jpg}

The trail to Ciudad Perdida follows the Buritaca River, whose waters
offer trekkers a chance to cool off during unforgivingly hot days.

Facilities along the route are basic, and ongoing construction at many
of the camps and snack shops hints at increasing numbers of visitors.

Armed Colombian soldiers --- stationed at the site for security purposes
--- are a fixture in and around the ancient city.

Before the coronavirus pandemic, tourism at Ciudad Perdida had increased
dramatically since 2008, though its popularity as an adventure
destination and archaeological site is still dwarfed by its main South
American rival, Machu Picchu, which in 2019 drew thousands of tourists
per day --- most of whom opted not to hike there but to arrive instead
by train and bus.

Ciudad Perdida, by comparison, where hiking remains the only way in and
out, drew about 70 people per day last year. And so far, the various
groups who hold sway over the area --- including four Indigenous groups,
the \href{https://www.icanh.gov.co/}{Colombian Institute of Anthropology
and History} and the
\href{https://globalheritagefund.org/what-we-do/projects-and-programs/ciudad-perdida/}{Global
Heritage Fund} --- have resisted plans to ease access. (A proposed cable
car that would have facilitated entry, for example, has been rejected on
multiple occasions.) ``The trek,'' said
\href{https://thecitypaperbogota.com/features/giraldo-living-among-ruins/8199}{Santiago
Giraldo}, an anthropologist and archaeologist who has worked in the
region for more than 20 years, ``is the first line of conservation
defense.''

Even so, ubiquitous construction at snack huts and overnight camps hints
at both increasing numbers of visitors and a greater local dependence on
tourism. These trends are mirrored in Colombia more broadly, where
international tourism nearly tripled between 2010 and 2018, from 1.4
million to about 3.9 million, according to figures from
\href{https://data.worldbank.org/indicator/ST.INT.ARVL?locations=CO}{The
World Bank}.

Image

Tairona architecture is characterized by circular designs and the use of
open spaces between buildings.

Image

Ciudad Perdida, just one of several hundred ancient Taironan settlements
in the area, extends over the crest and slopes of a hill that rises from
the Buritaca River. It was rediscovered by looters and heavily raided
before one of the looters' patrons alerted an official at the Gold
Museum in
\href{https://www.nytimes.com/2018/12/27/travel/what-to-do-in-bogota.html}{Bogotá},
sparking a visit by archaeologists from the Colombian Institute of
Anthropology in 1976. (The
\href{https://popular-archaeology.com/article/a-tale-of-cities-lost-and-found/}{longer
version of its rediscovery} is worth reading.)

There are several distinct sectors at the site, and the many complex,
multilevel terraces and other stone structures, archaeologists
speculate, served a range of functions: social, commercial, political,
residential, ritualistic. The ascending tiered terraces of the central
axis span a narrow ridgeline; the larger terraces were likely used as
public spaces for civil or political events. Viewed from the top, these
pristine patches appear to have sprouted miraculously from the
encroaching jungle.

Caribbean Sea

90

Santa Marta

colombia

Buritaca River

Ciudad Perdida

Sierra Nevada

de Santa Marta

10 miles

Caribbean

Sea

Cartagena

venezuela

colombia

Bogotá

200 miles

By The New York Times

What's remarkable (and a little disconcerting)~about the site, from a
tourist's perspective, is that visitors are free to roam its mostly
vacant grounds. And that's partly a consequence of its layout. ``It's an
architecture that's very alien to us,'' Mr. Giraldo explained. ``There's
really no such thing as private or public space, as we understand it.
That can be a bit unsettling for many people --- and it makes it
difficult to tease out what belonged to whom.''

The city's past is rich and intriguing. Ongoing archaeological research
has identified structures buried many feet below the visible terraces,
suggesting that the area was initially settled sometime around the
seventh century. (It likely began acquiring its current form sometime
around the 12th century and was abandoned --- due to a large number of
epidemic cycles --- in the late 16th century.)

\includegraphics{https://static01.nyt.com/images/2020/04/03/travel/15travel-colombia-42/00travel-colombia-42-videoSixteenByNineJumbo1600.jpg}

The rise in tourism at Ciudad Perdida is generally attributed to
demobilization among the rebel groups who long controlled the area. For
years, the threat of violence --- much of it tied to the cultivation of
coca plants and the production of cocaine ---~helped keep people out.

In 2003, for example, members of the National Liberation Army, or ELN, a
Marxist guerrilla group,
\href{https://www.theguardian.com/travel/2009/oct/24/colombia-lost-city-kidnapping}{kidnapped
eight visitors to the site}, holding some of them for 101 days.
(Ironically, as our lead guide, Iderle Muñoz, explained, international
coverage of the kidnapping eventually led to a surge in visitors ---~an
unlikely marketing campaign.)

Violence in the area is no longer a serious threat to trekkers. The
Colombian army maintains several outposts in and around the site,~as
much to aid with accidents along the trail, it seems, as to protect the
place.

\includegraphics{https://static01.nyt.com/images/2020/04/12/travel/15travel-colombia-02/15travel-colombia-02-mobileMasterAt3x.jpg}\includegraphics{https://static01.nyt.com/images/2020/03/03/travel/15travel-colombia-34/15travel-colombia-34-mobileMasterAt3x.jpg}\includegraphics{https://static01.nyt.com/images/2020/03/03/travel/15travel-colombia-33/15travel-colombia-33-mobileMasterAt3x.jpg}

The Sierra Nevada de Santa Marta, which surrounds Ciudad Perdida, is one
of the world's highest coastal mountain ranges.

In addition to its Indigenous reservations, the mountains are dotted
with non-reservation areas --- mostly coffee and livestock farms.

Of the four Indigenous groups who occupy the area, two --- the Kogi and
Wiwa --- are regularly encountered on the trails.

In many respects, Ciudad Perdida offers a model of sustainable tourism.
Solo, unguided hikes here are forbidden. Instead, would-be visitors must
pay 1,150,000 Colombian pesos (about \$300) to join a four- or five-day
guided tour, the fee for which includes meals (carried in on mules) and
basic accommodation at simple camps. (I used
\href{https://expotur-eco.com/en/}{Expotur} and was continually
impressed with the knowledge and expertise of the guides.) All of the
guides are locals, or based in nearby
\href{https://www.nytimes.com/2010/09/19/travel/19nextstop.html}{Santa
Marta} --- as are the cooks, porters and mule drivers. The campsites,
too, are locally owned. Money from trekkers, in other words, has flowed
back to the local communities.

\includegraphics{https://static01.nyt.com/images/2020/03/03/travel/15travel-colombia-ants-image/00travel-colombia-ants-image-videoSixteenByNineJumbo1600.png}

By some estimates, the mountain range surrounding Ciudad Perdida --- the
Sierra Nevada de Santa Marta --- is home to around 60,000 Indigenous
people, along with 350,000 campesinos, or rural farmers.

Guide companies work to facilitate interactions with the communities,
and meaningful exchanges do occur. Twice en route, for example, local
men displayed and discussed their
\href{https://www.youtube.com/watch?v=43M-J5ReoqQ}{poporos}, intensely
personal devices used to store burned and crushed seashells, which, when
mixed in the mouth with chewed coca leaves, help stimulate the coca
plant's active ingredients. Guides are also eager to stress that tourism
helps provide around 600 local families with a steady income.

\includegraphics{https://static01.nyt.com/images/2020/03/03/travel/15travel-colombia-35/00travel-colombia-35-articleLarge.jpg?quality=75\&auto=webp\&disable=upscale}

There's no doubt, though, that the site's growing popularity has caused
friction with local inhabitants. Exchanges are sometimes fraught. Some
locals actively engage with trekkers by selling supplies at shacks along
the way, and greeting those whom they pass on the trail. But others,
understandably, seem to be exasperated by the steady stream of gawking
tourists, an increasing number of whom are clogging trails, leaving
behind waste, and introducing unsanctioned technologies into largely
off-the-grid Indigenous cultures.

Moreover, many visitors (most of them are international) belong to
socioeconomic classes that are disproportionately contributing to
climate change --- an
\href{https://www.nationalgeographic.com/history/2019/11/indigenous-protectors-sacred-peaks-secret-until-now/}{existential
threat} to Indigenous ways of life. The
\href{https://www.nytimes.com/2019/06/03/travel/traveling-climate-change.html}{moral
dilemma posed by international travel} has never felt so immediate to me
as when, on the final night of our trek, a Kogi elder implored us to
respect Mother Earth.

Image

At the Piedras Sector of Ciudad Perdida, a girl sells bracelets made by
a Kogi mamo, or priest.

Image

A Kogi elder, holding his poporo, addresses our tour group on the final
night of the trek.

Image

An Indigenous guide stands atop a terrace at Ciudad Perdida.

Image

Cultural, historical and archaeological draws aside, perhaps the most
thrilling aspect of trekking to Ciudad Perdida --- a destination which,
on most tours, you'll have just three hours to explore --- is that the
site pulls its visitors through the lush beauty of the Colombian
rainforest.

The Sierra Nevada de Santa Marta is one of the most biologically diverse
mountain ranges on the planet. A staggering array of plants and animals
can be found here,~including around
\href{https://www.npr.org/sections/parallels/2016/04/27/475716092/as-colombia-grows-safer-tourists-especially-bird-lovers-flock-back}{630
species of birds} --- many of which are endemic, or~found nowhere else
on earth.

All along the edges of the trail, the jungle, folded in its tangles and
thickets, stands like an impenetrable wall. More than once, staring into
its depths and transfixed by a melodic bird call or by an impossibly
vibrant flower, I glanced back toward the hiking path only to realize
that I'd fallen more than an hour behind my group. I'd then skitter
ahead to regain ground.

Image

Image

Image

The Sierra Nevada de Santa Marta is one of the most biologically diverse
mountain ranges on the planet.

Skittering, though, wasn't always possible. At certain points the trek
was a grueling slog: sweltering heat, steep dirt trails, direct exposure
to the tropical sun, and all of it with a continual swirl of mosquitoes
menacing about my head and neck and arms and legs. I sweated through my
clothes within the first 10 minutes on the very first day. I had a
couple backup shirts tucked away in my pack, but my hiking pants
---~which I hung up hopelessly each night in the damp, warm air ---
never completely dried. The fact that I hardly minded is a testament to
the enchantment of the jungle.

Image

The trek also enforced a welcome disconnection from all the screens
whose ubiquitous glow often fills my waking hours --- a reality that
now, in the midst of the coronavirus pandemic, when almost every one of
my daily routines hinges on digital connectivity, seems difficult to
conjure.

At our final camp, after three days without scrolling, I handed my phone
to a woman working the snack shop; for 5,000 Colombian pesos (\$1.25),
she entered the camp's Wi-Fi password. Mostly I was hoping to back up
some of my images. But suddenly the world came crashing back with a
vengeance: texts from friends and family, an early Covid-19 warning from
the C.D.C., news about a dip in the markets.

The government! The markets! How absurdly remote it all seemed! If
anything makes you realize just how fantastically intangible stocks are,
I thought, it's the visceral reality of the jungle, where you shake out
your boots in the mornings to be sure they're free of scorpions.

Image

The route passes several Indigenous settlements, including this Wiwa
village, nestled in a narrow valley.

Image

Of course, the trek, which I made in February, now feels like a lifetime
ago --- a different world, a different era. I spoke by phone this week
with Ailyn, one of my guides, who said that tours have been suspended
indefinitely. Her most immediate concern was for the well-being of the
Indigenous groups; they could be especially vulnerable if exposed to the
virus, she said. But as with
\href{https://www.nytimes.com/2020/03/25/travel/coronavirus-travel-hospitality-workers.html}{many
on the front lines of the travel industry}, she was also concerned about
the welfare of her fellow guides, cooks and porters, all of whom have
come to depend on the trekkers for their livelihoods.

As for the site itself, there's little cause for concern: Ciudad Perdida
has a long history of surviving dormancy. And so the great Taironan city
is once again hidden away in the jungle ---~lost for now to adventurous
discovery, if not to memory.

Image

Our lead guide, Iderle Muñoz, holding up the rear, watches as our group
descends the steps leading away from Ciudad Perdida.

Advertisement

\protect\hyperlink{after-bottom}{Continue reading the main story}

\hypertarget{site-index}{%
\subsection{Site Index}\label{site-index}}

\hypertarget{site-information-navigation}{%
\subsection{Site Information
Navigation}\label{site-information-navigation}}

\begin{itemize}
\tightlist
\item
  \href{https://help.nytimes.com/hc/en-us/articles/115014792127-Copyright-notice}{©~2020~The
  New York Times Company}
\end{itemize}

\begin{itemize}
\tightlist
\item
  \href{https://www.nytco.com/}{NYTCo}
\item
  \href{https://help.nytimes.com/hc/en-us/articles/115015385887-Contact-Us}{Contact
  Us}
\item
  \href{https://www.nytco.com/careers/}{Work with us}
\item
  \href{https://nytmediakit.com/}{Advertise}
\item
  \href{http://www.tbrandstudio.com/}{T Brand Studio}
\item
  \href{https://www.nytimes.com/privacy/cookie-policy\#how-do-i-manage-trackers}{Your
  Ad Choices}
\item
  \href{https://www.nytimes.com/privacy}{Privacy}
\item
  \href{https://help.nytimes.com/hc/en-us/articles/115014893428-Terms-of-service}{Terms
  of Service}
\item
  \href{https://help.nytimes.com/hc/en-us/articles/115014893968-Terms-of-sale}{Terms
  of Sale}
\item
  \href{https://spiderbites.nytimes.com}{Site Map}
\item
  \href{https://help.nytimes.com/hc/en-us}{Help}
\item
  \href{https://www.nytimes.com/subscription?campaignId=37WXW}{Subscriptions}
\end{itemize}
