Sections

SEARCH

\protect\hyperlink{site-content}{Skip to
content}\protect\hyperlink{site-index}{Skip to site index}

\href{/section/world/asia}{Asia Pacific}\textbar{}`Will We Die Hungry?'
A Teeming Manila Slum Chafes Under Lockdown

\url{https://nyti.ms/2XF6jOW}

\begin{itemize}
\item
\item
\item
\item
\item
\end{itemize}

\href{https://www.nytimes.com/news-event/coronavirus?action=click\&pgtype=Article\&state=default\&region=TOP_BANNER\&context=storylines_menu}{The
Coronavirus Outbreak}

\begin{itemize}
\tightlist
\item
  live\href{https://www.nytimes.com/2020/08/03/world/coronavirus-covid-19.html?action=click\&pgtype=Article\&state=default\&region=TOP_BANNER\&context=storylines_menu}{Latest
  Updates}
\item
  \href{https://www.nytimes.com/interactive/2020/us/coronavirus-us-cases.html?action=click\&pgtype=Article\&state=default\&region=TOP_BANNER\&context=storylines_menu}{Maps
  and Cases}
\item
  \href{https://www.nytimes.com/interactive/2020/science/coronavirus-vaccine-tracker.html?action=click\&pgtype=Article\&state=default\&region=TOP_BANNER\&context=storylines_menu}{Vaccine
  Tracker}
\item
  \href{https://www.nytimes.com/2020/08/02/us/covid-college-reopening.html?action=click\&pgtype=Article\&state=default\&region=TOP_BANNER\&context=storylines_menu}{College
  Reopening}
\item
  \href{https://www.nytimes.com/live/2020/08/03/business/stock-market-today-coronavirus?action=click\&pgtype=Article\&state=default\&region=TOP_BANNER\&context=storylines_menu}{Economy}
\end{itemize}

\includegraphics{https://static01.nyt.com/images/2020/04/15/world/15virus-philippines-lockdown-span/merlin_171600018_b7fea218-0d5d-40b2-9f71-b7e40b6d83b0-articleLarge.jpg?quality=75\&auto=webp\&disable=upscale}

\hypertarget{will-we-die-hungry-a-teeming-manila-slum-chafes-under-lockdown}{%
\section{`Will We Die Hungry?' A Teeming Manila Slum Chafes Under
Lockdown}\label{will-we-die-hungry-a-teeming-manila-slum-chafes-under-lockdown}}

Life was already a struggle in the crowded shanties of San Roque. Then
came the coronavirus.

Crime, overcrowding and shortages are common in the San Roque slum in
the Philippines.Credit...

Supported by

\protect\hyperlink{after-sponsor}{Continue reading the main story}

By \href{https://www.nytimes.com/by/jason-gutierrez}{Jason Gutierrez}

Photographs by Jes Aznar

\begin{itemize}
\item
  Published April 15, 2020Updated April 17, 2020
\item
  \begin{itemize}
  \item
  \item
  \item
  \item
  \item
  \end{itemize}
\end{itemize}

MANILA --- Even before the coronavirus arrived in Manila, a saying in
the capital's sprawling San Roque slum --- ``no one dies from a fever''
--- crystallized the many threats that its residents faced in their
daily lives.

Drug-fueled petty crime. Food shortages. Overcrowding and poor
sanitation. Fever, body aches and coughs were commonplace long before
the virus came.

President Rodrigo Duterte's
\href{https://www.rappler.com/nation/256432-duterte-extends-luzon-lockdown-april-30-2020-coronavirus-pandemic}{lockdown
of Luzon}, the Philippines' largest island and home to Manila, is moving
into its second month, plunging San Roque's people even deeper into
poverty as the virus continues to rage. Yet the restrictions have not
stopped runny-nosed children from playing tag in the slum's labyrinth of
alleyways, as parents shout halfhearted admonitions to stay away from
one another.

Home to roughly 6,000 families --- conservatively, about 35,000 people
--- San Roque, in Manila's northern suburb of Quezon City, has for years
been home to some of the poorest people on the fringes of Philippine
society.

\includegraphics{https://static01.nyt.com/images/2020/04/15/world/15virus-philippines-lockdown-2/merlin_171283785_5d21a131-e0c8-4688-82c1-628167bbce38-articleLarge.jpg?quality=75\&auto=webp\&disable=upscale}

Image

Residents of the slum in Manila's northern suburbs flock to the market
to buy food before each evening's curfew begins.

Image

Susana Baldoza has lived for decades in San Roque, where she says people
are more likely to die of hunger than the coronavirus.

Many of the men are day laborers who work at construction sites in the
ever-expanding metropolis. Others are provincial migrants whose journeys
took them to the slum's squalid shanties, made from dilapidated
cardboard and rusting iron sheet roofing.

``Now it is a nightmare for people like us,'' said Susana Baldoza, a
grandmother of four who has lived nearly half of her 59 years in San
Roque, subsisting on odd jobs. ``Now that there is a lockdown, we can't
go outside to look for jobs, to survive.''

She says she does not doubt that
\href{https://www.nytimes.com/2020/02/02/world/asia/philippines-coronavirus-china.html}{the
virus is a killer}, but believes that many are likelier to die of
hunger, because government aid has been slow to trickle in. Now,
neighbors are helping neighbors, as the community turns inward to feed
its poorest residents.

\hypertarget{latest-updates-global-coronavirus-outbreak}{%
\section{\texorpdfstring{\href{https://www.nytimes.com/2020/08/03/world/coronavirus-covid-19.html?action=click\&pgtype=Article\&state=default\&region=MAIN_CONTENT_1\&context=storylines_live_updates}{Latest
Updates: Global Coronavirus
Outbreak}}{Latest Updates: Global Coronavirus Outbreak}}\label{latest-updates-global-coronavirus-outbreak}}

Updated 2020-08-04T04:02:32.475Z

\begin{itemize}
\tightlist
\item
  \href{https://www.nytimes.com/2020/08/03/world/coronavirus-covid-19.html?action=click\&pgtype=Article\&state=default\&region=MAIN_CONTENT_1\&context=storylines_live_updates\#link-4547638f}{Fauci
  defends Birx after she is criticized by Trump.}
\item
  \href{https://www.nytimes.com/2020/08/03/world/coronavirus-covid-19.html?action=click\&pgtype=Article\&state=default\&region=MAIN_CONTENT_1\&context=storylines_live_updates\#link-15e7f995}{Trump
  derides Democrats as lawmakers and administration officials try to
  break stimulus impasse.}
\item
  \href{https://www.nytimes.com/2020/08/03/world/coronavirus-covid-19.html?action=click\&pgtype=Article\&state=default\&region=MAIN_CONTENT_1\&context=storylines_live_updates\#link-e5a2cda}{The
  deadline for 2020 census counting has been moved up by a month.}
\end{itemize}

\href{https://www.nytimes.com/2020/08/03/world/coronavirus-covid-19.html?action=click\&pgtype=Article\&state=default\&region=MAIN_CONTENT_1\&context=storylines_live_updates}{See
more updates}

More live coverage:
\href{https://www.nytimes.com/live/2020/08/03/business/stock-market-today-coronavirus?action=click\&pgtype=Article\&state=default\&region=MAIN_CONTENT_1\&context=storylines_live_updates}{Markets}

Frustration over the lockdown recently exploded into violence. An April
1 gathering in San Roque became an impromptu rally, with dozens taking
to the streets demanding answers from the government about when they
would receive promised relief.

Police officers in riot gear and fatigues responded with force,
scuffling with protesters and sending 21 people to jail. Mr. Duterte
accused Kadamay, a group that advocates for the poor, of inciting the
violence, and warned that his government would not be lenient toward
those who challenged it.

Image

Firefighters spraying disinfectant in Manila last week.

Image

Police officers enforcing the lockdown in Manila, where it is entering
its second month.

Image

People marking Holy Thursday last week in Manila.

``Now is the time to set an example to everybody,'' Mr. Duterte said,
telling the police to
``\href{https://www.nytimes.com/reuters/2020/04/02/world/asia/02reuters-health-coronavirus-philippines-duterte.html?searchResultPosition=6}{shoot
them dead}'' if they believed protesters were endangering their lives.
``I am not used to being challenged,'' he said. ``Not me. Let this be a
warning to all.''

So far, there have been no confirmed cases of the coronavirus in San
Roque, though Ms. Baldoza is almost sure that residents have been
infected. ``I pray to God that there won't be any, but how could there
be none?'' she said.

As of Wednesday, 349 people had died in the Philippines from Covid-19,
the disease caused by the virus, and 5,453 infections had been
confirmed. But that figure is likely to rise sharply, with the
Philippine government having
\href{https://www.nytimes.com/reuters/2020/04/14/world/asia/14reuters-health-coronavirus-philippines.html?searchResultPosition=21}{just
begun mass testing this week}.

Image

A temperature check in a Manila slum.

Image

Physical distancing is essentially impossible for residents of densely
packed shanties.

Image

Many people in San Roque, like Cecille Carino and her family, rely on
food donated by charities and other aid groups.

Community leaders in San Roque have been tacking up cardboard signs
reminding people not to spit. Some people have started wearing face
masks, but most don't. Wearing them in the city's stifling heat can be
suffocating, some said; others said they would rather spend what little
money they had on food.

Yumi Castillo, a volunteer social worker with Kadamay, said it was hard
to explain the concept of social distancing to people who spend their
lives crammed into small, makeshift spaces.

Her group had printed out information about the virus for volunteers to
distribute. But judging from the many children playing in congested
alleyways and streets, the message didn't seem to be getting through.

``There are practically no health services here. No one teaches them,''
Ms. Castillo said at a community center where rice, food, drinking water
and rubbing alcohol were sorted and stored.

\href{https://www.nytimes.com/news-event/coronavirus?action=click\&pgtype=Article\&state=default\&region=MAIN_CONTENT_3\&context=storylines_faq}{}

\hypertarget{the-coronavirus-outbreak-}{%
\subsubsection{The Coronavirus Outbreak
›}\label{the-coronavirus-outbreak-}}

\hypertarget{frequently-asked-questions}{%
\paragraph{Frequently Asked
Questions}\label{frequently-asked-questions}}

Updated August 3, 2020

\begin{itemize}
\item ~
  \hypertarget{im-a-small-business-owner-can-i-get-relief}{%
  \paragraph{I'm a small-business owner. Can I get
  relief?}\label{im-a-small-business-owner-can-i-get-relief}}

  \begin{itemize}
  \tightlist
  \item
    The
    \href{https://www.nytimes.com/article/small-business-loans-stimulus-grants-freelancers-coronavirus.html?action=click\&pgtype=Article\&state=default\&region=MAIN_CONTENT_3\&context=storylines_faq}{stimulus
    bills enacted in March} offer help for the millions of American
    small businesses. Those eligible for aid are businesses and
    nonprofit organizations with fewer than 500 workers, including sole
    proprietorships, independent contractors and freelancers. Some
    larger companies in some industries are also eligible. The help
    being offered, which is being managed by the Small Business
    Administration, includes the Paycheck Protection Program and the
    Economic Injury Disaster Loan program. But lots of folks have
    \href{https://www.nytimes.com/interactive/2020/05/07/business/small-business-loans-coronavirus.html?action=click\&pgtype=Article\&state=default\&region=MAIN_CONTENT_3\&context=storylines_faq}{not
    yet seen payouts.} Even those who have received help are confused:
    The rules are draconian, and some are stuck sitting on
    \href{https://www.nytimes.com/2020/05/02/business/economy/loans-coronavirus-small-business.html?action=click\&pgtype=Article\&state=default\&region=MAIN_CONTENT_3\&context=storylines_faq}{money
    they don't know how to use.} Many small-business owners are getting
    less than they expected or
    \href{https://www.nytimes.com/2020/06/10/business/Small-business-loans-ppp.html?action=click\&pgtype=Article\&state=default\&region=MAIN_CONTENT_3\&context=storylines_faq}{not
    hearing anything at all.}
  \end{itemize}
\item ~
  \hypertarget{what-are-my-rights-if-i-am-worried-about-going-back-to-work}{%
  \paragraph{What are my rights if I am worried about going back to
  work?}\label{what-are-my-rights-if-i-am-worried-about-going-back-to-work}}

  \begin{itemize}
  \tightlist
  \item
    Employers have to provide
    \href{https://www.osha.gov/SLTC/covid-19/standards.html}{a safe
    workplace} with policies that protect everyone equally.
    \href{https://www.nytimes.com/article/coronavirus-money-unemployment.html?action=click\&pgtype=Article\&state=default\&region=MAIN_CONTENT_3\&context=storylines_faq}{And
    if one of your co-workers tests positive for the coronavirus, the
    C.D.C.} has said that
    \href{https://www.cdc.gov/coronavirus/2019-ncov/community/guidance-business-response.html}{employers
    should tell their employees} -\/- without giving you the sick
    employee's name -\/- that they may have been exposed to the virus.
  \end{itemize}
\item ~
  \hypertarget{should-i-refinance-my-mortgage}{%
  \paragraph{Should I refinance my
  mortgage?}\label{should-i-refinance-my-mortgage}}

  \begin{itemize}
  \tightlist
  \item
    \href{https://www.nytimes.com/article/coronavirus-money-unemployment.html?action=click\&pgtype=Article\&state=default\&region=MAIN_CONTENT_3\&context=storylines_faq}{It
    could be a good idea,} because mortgage rates have
    \href{https://www.nytimes.com/2020/07/16/business/mortgage-rates-below-3-percent.html?action=click\&pgtype=Article\&state=default\&region=MAIN_CONTENT_3\&context=storylines_faq}{never
    been lower.} Refinancing requests have pushed mortgage applications
    to some of the highest levels since 2008, so be prepared to get in
    line. But defaults are also up, so if you're thinking about buying a
    home, be aware that some lenders have tightened their standards.
  \end{itemize}
\item ~
  \hypertarget{what-is-school-going-to-look-like-in-september}{%
  \paragraph{What is school going to look like in
  September?}\label{what-is-school-going-to-look-like-in-september}}

  \begin{itemize}
  \tightlist
  \item
    It is unlikely that many schools will return to a normal schedule
    this fall, requiring the grind of
    \href{https://www.nytimes.com/2020/06/05/us/coronavirus-education-lost-learning.html?action=click\&pgtype=Article\&state=default\&region=MAIN_CONTENT_3\&context=storylines_faq}{online
    learning},
    \href{https://www.nytimes.com/2020/05/29/us/coronavirus-child-care-centers.html?action=click\&pgtype=Article\&state=default\&region=MAIN_CONTENT_3\&context=storylines_faq}{makeshift
    child care} and
    \href{https://www.nytimes.com/2020/06/03/business/economy/coronavirus-working-women.html?action=click\&pgtype=Article\&state=default\&region=MAIN_CONTENT_3\&context=storylines_faq}{stunted
    workdays} to continue. California's two largest public school
    districts --- Los Angeles and San Diego --- said on July 13, that
    \href{https://www.nytimes.com/2020/07/13/us/lausd-san-diego-school-reopening.html?action=click\&pgtype=Article\&state=default\&region=MAIN_CONTENT_3\&context=storylines_faq}{instruction
    will be remote-only in the fall}, citing concerns that surging
    coronavirus infections in their areas pose too dire a risk for
    students and teachers. Together, the two districts enroll some
    825,000 students. They are the largest in the country so far to
    abandon plans for even a partial physical return to classrooms when
    they reopen in August. For other districts, the solution won't be an
    all-or-nothing approach.
    \href{https://bioethics.jhu.edu/research-and-outreach/projects/eschool-initiative/school-policy-tracker/}{Many
    systems}, including the nation's largest, New York City, are
    devising
    \href{https://www.nytimes.com/2020/06/26/us/coronavirus-schools-reopen-fall.html?action=click\&pgtype=Article\&state=default\&region=MAIN_CONTENT_3\&context=storylines_faq}{hybrid
    plans} that involve spending some days in classrooms and other days
    online. There's no national policy on this yet, so check with your
    municipal school system regularly to see what is happening in your
    community.
  \end{itemize}
\item ~
  \hypertarget{is-the-coronavirus-airborne}{%
  \paragraph{Is the coronavirus
  airborne?}\label{is-the-coronavirus-airborne}}

  \begin{itemize}
  \tightlist
  \item
    The coronavirus
    \href{https://www.nytimes.com/2020/07/04/health/239-experts-with-one-big-claim-the-coronavirus-is-airborne.html?action=click\&pgtype=Article\&state=default\&region=MAIN_CONTENT_3\&context=storylines_faq}{can
    stay aloft for hours in tiny droplets in stagnant air}, infecting
    people as they inhale, mounting scientific evidence suggests. This
    risk is highest in crowded indoor spaces with poor ventilation, and
    may help explain super-spreading events reported in meatpacking
    plants, churches and restaurants.
    \href{https://www.nytimes.com/2020/07/06/health/coronavirus-airborne-aerosols.html?action=click\&pgtype=Article\&state=default\&region=MAIN_CONTENT_3\&context=storylines_faq}{It's
    unclear how often the virus is spread} via these tiny droplets, or
    aerosols, compared with larger droplets that are expelled when a
    sick person coughs or sneezes, or transmitted through contact with
    contaminated surfaces, said Linsey Marr, an aerosol expert at
    Virginia Tech. Aerosols are released even when a person without
    symptoms exhales, talks or sings, according to Dr. Marr and more
    than 200 other experts, who
    \href{https://academic.oup.com/cid/article/doi/10.1093/cid/ciaa939/5867798}{have
    outlined the evidence in an open letter to the World Health
    Organization}.
  \end{itemize}
\end{itemize}

Ms. Baldoza, the grandmother of four, was volunteering as a cook for a
community kitchen in San Roque, serving fried herring over rice,
courtesy of the Catholic Church and a civic group that has been helping
residents weather the crisis.

``People here are very poor, as you can see,'' said Ms. Baldoza, frying
fish outdoors in a wok. ``We don't have money and the luxury of going to
the supermarkets. We haven't received help from the government, no help
from the outside except the donations that they give us. And people
can't work.''

Image

Medical workers at a Manila hospital on Tuesday with oxygen tanks for
patients believed to have contracted the coronavirus.

Image

Checking a patient for a Covid-19 infection at an observation tent at
Manila's Santa Ana Hospital.

Image

Transporting the body of a coronavirus victim in Quezon City, the
Philippines, last week. By mid-April at least 349 people had died of the
virus in the country.

Her neighbor Analyn Mikunog was waiting for the food to be served. Ms.
Mikunog's partner has no permanent job, though sometimes he is lucky
enough to find work on construction sites. He had just been hired as a
day laborer when Mr. Duterte imposed the lockdown.

Now, the family's future is bleak. With four young children to feed, the
gaunt-looking Ms. Mikunog, 28, said she was at her wits' end trying to
figure out how they would survive.

``Sometimes we talk, and wonder how long this lockdown will last,'' she
said. ``Will we die hungry?''

Priests in clerical collars and rugged jeans were busily making lunches,
but their camaraderie belied the seriousness of the situation. As the
meals were being prepared, riot police officers moved in to break up the
feeding program. They accused leftist organizations of using it to
recruit people to campaign against the government.

The officers, some in black uniforms and others in combat fatigues,
carried batons and long firearms. They confiscated signs that read,
``Help, Not Jail.'' After tense negotiations, a commander, who refused
to identify himself, finally relented. But he warned the group to break
up after the food was distributed, and to practice social distancing.

``We are just serving the people,'' said King Garcia, a 39-year-old
priest.

``The government has left them in the fringes at a time when they needed
help the most,'' he said. ``If the virus does not kill them, hunger
will.''

Image

For many in Manila, the future is bleak.~``Now it is a nightmare for
people like us,'' Ms. Baldoza said.

Advertisement

\protect\hyperlink{after-bottom}{Continue reading the main story}

\hypertarget{site-index}{%
\subsection{Site Index}\label{site-index}}

\hypertarget{site-information-navigation}{%
\subsection{Site Information
Navigation}\label{site-information-navigation}}

\begin{itemize}
\tightlist
\item
  \href{https://help.nytimes.com/hc/en-us/articles/115014792127-Copyright-notice}{©~2020~The
  New York Times Company}
\end{itemize}

\begin{itemize}
\tightlist
\item
  \href{https://www.nytco.com/}{NYTCo}
\item
  \href{https://help.nytimes.com/hc/en-us/articles/115015385887-Contact-Us}{Contact
  Us}
\item
  \href{https://www.nytco.com/careers/}{Work with us}
\item
  \href{https://nytmediakit.com/}{Advertise}
\item
  \href{http://www.tbrandstudio.com/}{T Brand Studio}
\item
  \href{https://www.nytimes.com/privacy/cookie-policy\#how-do-i-manage-trackers}{Your
  Ad Choices}
\item
  \href{https://www.nytimes.com/privacy}{Privacy}
\item
  \href{https://help.nytimes.com/hc/en-us/articles/115014893428-Terms-of-service}{Terms
  of Service}
\item
  \href{https://help.nytimes.com/hc/en-us/articles/115014893968-Terms-of-sale}{Terms
  of Sale}
\item
  \href{https://spiderbites.nytimes.com}{Site Map}
\item
  \href{https://help.nytimes.com/hc/en-us}{Help}
\item
  \href{https://www.nytimes.com/subscription?campaignId=37WXW}{Subscriptions}
\end{itemize}
