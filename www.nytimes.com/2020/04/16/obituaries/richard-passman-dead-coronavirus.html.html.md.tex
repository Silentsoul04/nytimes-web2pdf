Sections

SEARCH

\protect\hyperlink{site-content}{Skip to
content}\protect\hyperlink{site-index}{Skip to site index}

\href{https://www.nytimes.com/section/obituaries}{Obituaries}

\href{https://myaccount.nytimes.com/auth/login?response_type=cookie\&client_id=vi}{}

\href{https://www.nytimes.com/section/todayspaper}{Today's Paper}

\href{/section/obituaries}{Obituaries}\textbar{}Richard Passman,
Space-Age Engineer Who Kept His Secrets, Dies at 94

\url{https://nyti.ms/2yYTvZo}

\begin{itemize}
\item
\item
\item
\item
\item
\end{itemize}

\href{https://www.nytimes.com/news-event/coronavirus?action=click\&pgtype=Article\&state=default\&region=TOP_BANNER\&context=storylines_menu}{The
Coronavirus Outbreak}

\begin{itemize}
\tightlist
\item
  live\href{https://www.nytimes.com/2020/08/03/world/coronavirus-covid-19.html?action=click\&pgtype=Article\&state=default\&region=TOP_BANNER\&context=storylines_menu}{Latest
  Updates}
\item
  \href{https://www.nytimes.com/interactive/2020/us/coronavirus-us-cases.html?action=click\&pgtype=Article\&state=default\&region=TOP_BANNER\&context=storylines_menu}{Maps
  and Cases}
\item
  \href{https://www.nytimes.com/interactive/2020/science/coronavirus-vaccine-tracker.html?action=click\&pgtype=Article\&state=default\&region=TOP_BANNER\&context=storylines_menu}{Vaccine
  Tracker}
\item
  \href{https://www.nytimes.com/2020/08/02/us/covid-college-reopening.html?action=click\&pgtype=Article\&state=default\&region=TOP_BANNER\&context=storylines_menu}{College
  Reopening}
\item
  \href{https://www.nytimes.com/live/2020/08/03/business/stock-market-today-coronavirus?action=click\&pgtype=Article\&state=default\&region=TOP_BANNER\&context=storylines_menu}{Economy}
\end{itemize}

Advertisement

\protect\hyperlink{after-top}{Continue reading the main story}

Supported by

\protect\hyperlink{after-sponsor}{Continue reading the main story}

Those We've Lost

\hypertarget{richard-passman-space-age-engineer-who-kept-his-secrets-dies-at-94}{%
\section{Richard Passman, Space-Age Engineer Who Kept His Secrets, Dies
at
94}\label{richard-passman-space-age-engineer-who-kept-his-secrets-dies-at-94}}

Mr. Passman had a leadership role in many of the technologies that gave
the United States strength in aircraft, spy satellites and missiles.

\includegraphics{https://static01.nyt.com/images/2020/04/18/obituaries/14passman-virus-lost1/14passman-virus-lost1-articleLarge.jpg?quality=75\&auto=webp\&disable=upscale}

\href{https://www.nytimes.com/by/john-schwartz}{\includegraphics{https://static01.nyt.com/images/2018/02/16/multimedia/author-john-schwartz/author-john-schwartz-thumbLarge.jpg}}

By \href{https://www.nytimes.com/by/john-schwartz}{John Schwartz}

\begin{itemize}
\item
  April 16, 2020
\item
  \begin{itemize}
  \item
  \item
  \item
  \item
  \item
  \end{itemize}
\end{itemize}

\emph{This obituary is part of a series about}
\href{https://www.nytimes.com/series/people-who-have-died-of-the-coronavirus}{\emph{people
who have died in the coronavirus pandemic}}\emph{.}

Richard Passman, an aeronautical engineer whose wide-ranging career took
him through the early stages of supersonic flight, spy satellites and
intercontinental ballistic missiles, died on April 1 in Silver Spring,
Md. He was 94.

The cause was complications of the new coronavirus, his son William
said.

Mr. Passman was involved in crucial space-age projects, many of them
secret --- unlike the work of the civilian space program, which made
public figures of those who blasted into space and some of those whose
work got them there, Dwayne A. Day, a space historian, said.

``There was a classified space program and there were people equally
smart,'' Mr. Day said, ``and yet we don't know their names.''

Richard A. Passman was born on June 30, 1925, in Cedarhurst, N.Y., on
Long Island, to Matthew and Ethel Passman. (The middle initial didn't
stand for anything, his son William said.) His father co-owned an
insurance company, and his mother was a homemaker.

At the University of Michigan, he earned a bachelor of science degree in
aeronautical engineering in 1944, a bachelor's degree in mathematics in
1946 and a master's in aeronautical engineering in 1947.

He joined Bell Aircraft as the company was creating the Bell X-1, the
first aircraft to fly faster than the speed of sound, or Mach 1 (about
770 miles per hour). He served as chief aerodynamicist on its successor,
the X-2, which
\href{https://www.nasa.gov/centers/armstrong/news/FactSheets/FS-079-DFRC.html}{reached
Mach 3}. He was aircraft designer for Bell's X-16, a spy plane, but the
government instead chose Lockheed's U-2.

He then moved from Bell to
\href{https://minutemanmissile.com/documents/GEReentryVehicles.pdf}{General
Electric}, and raised his horizons from aircraft to spacecraft. He was a
manager in a part of the company responsible for creating systems that
allowed objects sent plunging through the atmosphere to withstand the
blazing heat of re-entry.

There was the Corona, the first spy satellite: The orbiter took
high-resolution photographs and ejected the film in a heat-shielding
``bucket.'' The container re-entered the atmosphere, deployed a
parachute and was snagged out of the air by military aircraft. G.E. was
responsible for the bucket.

He also helped lead efforts to create heat shielding technology for
intercontinental ballistic missiles and multiple-warhead missiles. He
continued to work on projects for the space program; he was G.E.'s
\href{https://aero.engin.umich.edu/wp-content/uploads/sites/2/2017/06/riding-the-crest-a-history-of-michigan2019s-aerospace-engineering-department.pdf}{general
manager for space activities} and was developing the
\href{https://www.nasa.gov/feature/50-years-ago-nasa-benefits-from-mol-cancellation}{Manned
Orbiting Laboratory}, an Air Force project that would have put a manned
spy satellite into orbit, when it was canceled during the Nixon
administration.

Bill Passman said that he knew very little of his father's secret work;
the Corona system wasn't declassified until 1995. ``These guys could
keep a secret,'' he said. ``They were old school.''

Image

``X-15: The World's Fastest Rocket Plane and the Pilots Who Ushered in
the Space Age'' by John Anderson and Richard Passman.

In retirement, Mr. Passman volunteered to work with John D. Anderson,
the curator of aerodynamics at the Smithsonian National Air and Space
Museum. He and Dr. Anderson co-wrote
``\href{https://www.thespacereview.com/article/2448/1}{X-15: The World's
Fastest Rocket Plane and the Pilots Who Ushered in the Space Age},''
which was published in 2014 by the Smithsonian.

In addition to his son William, Mr. Passman is survived by his wife of
70 years, Minna (Hocky) Passman; two other sons, Henry and Don; and
numerous grandchildren and great-grandchildren.

Mr. and Mrs. Passman were living in a retirement community in Florida
when the coronavirus began its spread. Their family, based in the
Washington area, persuaded them to relocate to a retirement home closer
to them, in Silver Spring, on March 15.

``They locked it down the same day,'' Bill Passman said. Two weeks
later, his father was dead. The family held a virtual shiva via Zoom.

\href{https://www.nytimes.com/interactive/2020/obituaries/people-died-coronavirus-obituaries.html?action=click\&pgtype=Article\&state=default\&region=BELOW_MAIN_CONTENT\&context=covid_obits_promo}{}

\hypertarget{those-weve-lost}{%
\section{Those We've Lost}\label{those-weve-lost}}

The coronavirus pandemic has taken an incalculable death toll. This
series is designed to put names and faces to the numbers.

Read more

\includegraphics{https://static01.nyt.com/images/2020/07/30/obituaries/30Pedro/30Pedro-square640.jpg}

\hypertarget{bernaldina-josuxe9-pedro}{%
\section{Bernaldina José Pedro}\label{bernaldina-josuxe9-pedro}}

d. Boa Vista, Brazil

Leader among the Indigenous Macuxi

\includegraphics{https://static01.nyt.com/images/2020/07/31/obituaries/31Swing/merlin_175167783_8913bc90-0d64-43f3-a655-1bb1bf1601c9-square640.jpg}

\hypertarget{john-eric-swing}{%
\section{John Eric Swing}\label{john-eric-swing}}

d. Fountain Valley, Calif.

Champion of Filipino-Americans

\includegraphics{https://static01.nyt.com/images/2020/07/27/obituaries/27Victor/merlin_175001436_38b11f8e-227a-4e2c-9821-7618af9b2524-square640.jpg}

\hypertarget{victor-victor}{%
\section{Victor Victor}\label{victor-victor}}

d. Santo Domingo, Dominican Republic

Beloved musician of the Dominican Republic

\includegraphics{https://static01.nyt.com/images/2020/07/31/obituaries/31Negron/merlin_175160169_516322ae-fd23-4969-b6b2-193ced371105-square640.jpg}

\hypertarget{dr-eddie-negruxf3n}{%
\section{Dr. Eddie Negrón}\label{dr-eddie-negruxf3n}}

d. Fort Walton Beach, Fla.

Internist on Florida's Emerald Coast

\includegraphics{https://static01.nyt.com/images/2020/07/30/obituaries/30Dobson/merlin_175115928_f6b9271c-8f05-4fe1-a38a-5ca4a58f8935-square640.jpg}

\hypertarget{dobby-dobson}{%
\section{Dobby Dobson}\label{dobby-dobson}}

d. Coral Springs, Fla.

Jamaican singer and songwriter

\includegraphics{https://static01.nyt.com/images/2020/08/01/obituaries/28Gonzalez/merlin_175002771_beb57888-3951-409a-ae13-03a94b2e962e-square640.jpg}

\hypertarget{waldemar-gonzalez}{%
\section{Waldemar Gonzalez}\label{waldemar-gonzalez}}

d. White Plains, N.Y.

Teacher and social worker

Advertisement

\protect\hyperlink{after-bottom}{Continue reading the main story}

\hypertarget{site-index}{%
\subsection{Site Index}\label{site-index}}

\hypertarget{site-information-navigation}{%
\subsection{Site Information
Navigation}\label{site-information-navigation}}

\begin{itemize}
\tightlist
\item
  \href{https://help.nytimes.com/hc/en-us/articles/115014792127-Copyright-notice}{©~2020~The
  New York Times Company}
\end{itemize}

\begin{itemize}
\tightlist
\item
  \href{https://www.nytco.com/}{NYTCo}
\item
  \href{https://help.nytimes.com/hc/en-us/articles/115015385887-Contact-Us}{Contact
  Us}
\item
  \href{https://www.nytco.com/careers/}{Work with us}
\item
  \href{https://nytmediakit.com/}{Advertise}
\item
  \href{http://www.tbrandstudio.com/}{T Brand Studio}
\item
  \href{https://www.nytimes.com/privacy/cookie-policy\#how-do-i-manage-trackers}{Your
  Ad Choices}
\item
  \href{https://www.nytimes.com/privacy}{Privacy}
\item
  \href{https://help.nytimes.com/hc/en-us/articles/115014893428-Terms-of-service}{Terms
  of Service}
\item
  \href{https://help.nytimes.com/hc/en-us/articles/115014893968-Terms-of-sale}{Terms
  of Sale}
\item
  \href{https://spiderbites.nytimes.com}{Site Map}
\item
  \href{https://help.nytimes.com/hc/en-us}{Help}
\item
  \href{https://www.nytimes.com/subscription?campaignId=37WXW}{Subscriptions}
\end{itemize}
