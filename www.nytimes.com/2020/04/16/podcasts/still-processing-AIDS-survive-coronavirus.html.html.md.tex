Sections

SEARCH

\protect\hyperlink{site-content}{Skip to
content}\protect\hyperlink{site-index}{Skip to site index}

\href{https://www.nytimes.com/spotlight/podcasts}{Podcasts}

\href{https://myaccount.nytimes.com/auth/login?response_type=cookie\&client_id=vi}{}

\href{https://www.nytimes.com/section/todayspaper}{Today's Paper}

\href{/spotlight/podcasts}{Podcasts}\textbar{}How to Learn From a Plague

\url{https://nyti.ms/2ynQt0p}

\begin{itemize}
\item
\item
\item
\item
\item
\end{itemize}

Advertisement

\protect\hyperlink{after-top}{Continue reading the main story}

transcript

Back to Still Processing

bars

0:00/0:00

-0:00

transcript

\hypertarget{how-to-learn-from-a-plague}{%
\subsection{How to Learn From a
Plague}\label{how-to-learn-from-a-plague}}

\hypertarget{hosted-by-wesley-morris-and-jenna-wortham-produced-by-hans-buetow}{%
\subsubsection{Hosted by Wesley Morris and Jenna Wortham. Produced by
Hans
Buetow.}\label{hosted-by-wesley-morris-and-jenna-wortham-produced-by-hans-buetow}}

\hypertarget{we-study-the-aids-epidemic-documentary-how-to-survive-a-plague--and-apply-its-lessons-to-the-covid-19-crisis}{%
\paragraph{We study the AIDS-epidemic documentary ``How to Survive a
Plague'' --- and apply its lessons to the Covid-19
crisis.}\label{we-study-the-aids-epidemic-documentary-how-to-survive-a-plague--and-apply-its-lessons-to-the-covid-19-crisis}}

Thursday, April 16th, 2020

\begin{itemize}
\item
  wesley morris\\
  Can I tell you, Jenna, about my Aunt Geri?
\item
  jenna wortham\\
  Please. I would love to hear more.
\item
  wesley morris\\
  Well, you know that she died a couple weeks ago of Covid-19. She was
  in a nursing home for a while. And like many people who are in nursing
  homes right now, it is not the greatest place to be with this disease
  out there. And she was one of the unfortunate people who got it. She
  was 90 years old, and her life was good. It was very good. Aunt Geri
  was so full of life, and so full of energy, and so funny, and she
  loved having people around her. Let me --- OK, Jenna, let me just ask
  you a question. Like, what is the thing where like, you don't like
  Thanksgiving, but you will put up with it because there's this one
  thing you get to eat every year? What is it?
\item
  jenna wortham\\
  100 percent. 100 percent it's stuffing.
\item
  wesley morris\\
  Aunt Geri was the stuffing on your Thanksgiving plate. My personal
  favorite thing is macaroni and cheese with cranberry sauce. My Aunt
  Geri was the mac and cheese and the cranberry sauce. She was the
  person who, whenever my family got together, even if I didn't even
  think about her being there, the minute I get in the house, Aunt Geri
  is there, and I'm happy. She told funny stories. She was a great side
  person in somebody else's story if she was present when my grandmother
  was holding court. My Aunt Geri was basically like the Pips to my
  grandmother's Gladys Knight. She was always there and just be like,
  leaving ---
\item
  jenna wortham\\
  Oh, my god. So good.
\item
  wesley morris\\
  On that midnight train. Yeah, I mean, that was my Aunt Geri. She was
  there to back up any story being told that she knew anything about.
  And whatever your favorite thing is, that was Aunt Geri. She was one
  of my favorite things. And it's just wild to me that this vivacious,
  humorous, vinegary woman who loved so many people, who was loved by so
  many people, who lived such a full and rich life, would have to spend
  the last of those days by herself, untouched, unspoken to,
  essentially, by anybody who loved her, by the people she raised. I
  think all of us in my family would have loved to have given her that.
\item
  jenna wortham\\
  It's really hard to grapple with the reality that in order to take
  care of someone, and in order to love them, you have to stay away from
  them, because you could get sick. I mean, their bodies are dangerous.
  They are carrying the virus.
\item
  wesley morris\\
  And where my brain goes thinking about where to put my sadness, or
  like how to think about it in some other context, is to the 1980s and
  1990s, when people were also dying stigmatized deaths. And it isn't so
  much that my Aunt Geri died stigmatized, but she kind of did, Jenna.
  She was forced to die in a way that feels, emotionally, to me,
  shameful, because it's not the way human beings commemorate each
  other.
\item
  jenna wortham\\
  Right.
\item
  wesley morris\\
  We couldn't be in space with her. We couldn't talk her through her
  death. We couldn't comfort her as she died. She just had to do it on
  her own. And so there is this --- there's this spot on Aunt Geri's
  death in a way that feels, kinda makes me feel worse.
\item
  jenna wortham\\
  But that last act of Aunt Geri was one of love, you know? She died
  alone to prevent anyone else from getting sick, and it's not right. It
  isn't right.
\item
  wesley morris\\
  But that's where we are.
\item
  jenna wortham\\
  Yeah. And that's where we are.
\item
  {[}music{]}
\item
  wesley morris\\
  I'm Wesley Morris.
\item
  jenna wortham\\
  I'm Jenna Wortham.
\item
  wesley morris\\
  We're two New York Times writers hunkered down in our living rooms.
\item
  jenna wortham\\
  This is ``Still Processing.''
\item
  wesley morris\\
  Jenna, you and I watched this film, ``How to Survive a Plague.'' I
  think this movie, which was made in 2012 by David France, is a really
  instructive blueprint for how we might proceed, and the ways in which
  the era of the AIDS crisis, which lasted for the 1980s and 1990s, the
  high point of it is just really useful to compare these two areas.
\item
  jenna wortham\\
  Sure.
\item
  wesley morris\\
  This one that we've just embarked on, and this one that was well under
  way when the movie starts. And essentially, David France made this
  film out of lots of video footage that existed from the time of ACT UP
  and the fight to get some sort of drug treatment for these people who
  were dying and who contracted the disease.
\item
  archived recording\\
  Welcome to ACT UP. We are the AIDS Coaltion to Unleash Power, a
  diverse, non-partisan group of individuals united in anger and
  committed to direct action to end the AIDS crisis.
\end{itemize}

wesley morris

And they were formed in 1987, and they essentially harassed and
basically terrorized the government into giving them the drugs.

\begin{itemize}
\tightlist
\item
  archived recording\\
  Fight back. Fight AIDS. Act up. Fight back. Fight AIDS.
\end{itemize}

wesley morris

I think part of the power comes from the fear of the people organizing,
and the uncertainty, and the anger that they have that there is
something that can be done, but the government, the U.S. government in
this case, isn't doing enough, if anything, to fight it. And I think the
reason I wanted us to watch it was because it just seemed like a film
that nobody's really talking about in terms of what lessons we can
learn. Everybody's watching ``Contagion.''

jenna wortham

And ``28 Days Later,'' I just keep bringing it up.

wesley morris

You really keep bringing it up.

{[}laughter{]}

wesley morris

I know it's not just you.

jenna wortham

My biggest regret. My biggest regret. Go on.

wesley morris

But I think ``How to Survive a Plague'' is the most instructive movie
manual we might have. I mean, there are all kinds of books we can read
as well, but movie-wise I actually wonder if, at this point where we are
now --- ``Contagion'' made sense for the first phase of this.

jenna wortham

Right.

wesley morris

I feel like we're entering a more existential phase, and a more angry
and political phase.

jenna wortham

We're more engaged.

wesley morris

Yeah.

jenna wortham

Like, this is the new reality. Now we want to know what to do with it.

wesley morris

Right, right. I feel like there are a lot of ways in which this moment
is similar to what was happening 25, 30, 35 years ago, and one of those
ways is what we were talking about with my aunt, which involves the
danger of a vector and the fact that we can't touch each other. I'm
thinking about this moment in ``How to Survive a Plague'' where Peter
Staley, who's one of the most prominent members of ACT UP, is on
``Crossfire,'' CNN's ``Crossfire,'' back when ``Crossfire'' was not the
way all cable news was now, which is people yelling at each other.

\begin{itemize}
\tightlist
\item
  archived recording\\
  Tonight from Washington, ``Crossfire,'' against all odds. On the left,
  Tom Braden. On the right, Pat Buchanan. In the crossfire, Peter Staley
  of the New York AIDS Coalition To Unleash Power.
\end{itemize}

wesley morris

Pat Buchanan, conservative Republican on one side, and Tom Braden, I
don't know what political affiliation Tom Braden was. He basically is
famous for having written the book that became the television show
``Eight is Enough.'' Anyway, at some point Pat Buchanan says to Peter
Staley, you know, I want you to look into the camera, and what would you
tell a young kid who wants to have sex?

jenna wortham

Well, he said your brother. He was like ---

wesley morris

Yeah, or your brother. Yeah, your young brother, yeah.

jenna wortham

If you have a brother, he's 21, what would you say to him? Would you
want him to avoid this life that you're living? Peter Staley is HIV
positive.

\begin{itemize}
\item
  archived recording (pat buchanan)\\
  What would you tell him if you wanted him to live a long life?
\item
  archived recording (peter staley)\\
  Use a condom. And also to use a lubricant, by the way, that has the
  medicine that can ---
\item
  archived recording (pat buchanan)\\
  This is Russian roulette.
\item
  archived recording (peter staley)\\
  It is not Russian roulette. It is Russian roulette to not give people
  this information when human nature dictates that they're going to go
  out there and they're going to have sex anyway.
\item
  archived recording (pat buchanan)\\
  You mean celibacy is impossible?
\item
  archived recording (peter staley)\\
  It's just not going to work. People aren't going to do it, and lots,
  lots of people are going to die.
\end{itemize}

jenna wortham

It's incredible because the two men on either side of him cannot deal.
They're both just sputtering. They literally --- if they were a cartoon,
there'd be like smoke coming out of their ears.

\begin{itemize}
\tightlist
\item
  archived recording (pat buchanan)\\
  I think that --- well, thank you very much, Peter Staley. Thanks for
  being in our studio. Mr. Braden and I will be back in a minute.
\end{itemize}

{[}music{]}

wesley morris

What I hear Peter Staley saying in that moment is that we need the tools
to be safe. Because the truth is, and you can see it in 2020, you can
see it right now anytime you leave the house with the uncertainty around
how to be safe around each other. And right now the solution is just to
quarantine and stay in the house and not go anywhere near each other.
But I think part of the anger that is bubbling up among so many of us
right now is that there is potentially a way we can begin to be safe
around each other, and that is through testing. Those guys wanted
condoms. We want tests. And it is a thing that is going to help us be
safe when it's time to actually do what comes natural to us, which is
not stay in the house. A very simple clarification that could be made
for us medically, and at this point on behalf of the government, is
testing. Where are the tests?

jenna wortham

Mmm.

wesley morris

And there are two things that really have stayed with me in this outing
with ``How to Survive a Plague,'' and the one I just talked about with
Peter Staley and Pat Buchanan and Tom Braden on ``Crossfire.'' And the
other one is this moment at which things are really tense among ACT UP
members. They've reached a sort of stage where, I don't know how else to
put this except like, some pettiness has begun to creep into the
operation. And there are people who think that the people at the front
of the organization, at the top of ACT UP, are getting too much
attention, and they're a little too famous.

jenna wortham

A common activist, organizer dilemma. In every movement there comes a
moment. Go on.

wesley morris

Yes. Well, there's a little --- somebody in the peanut gallery is trying
to chirp up and cause a little bit of disruption.

\begin{itemize}
\item
  archived recording 1\\
  Bill, you're going to have a chance to talk, all right? Everybody ---
\item
  archived recording 2\\
  Don't lecture me, you stupid, lazy, incompetent shithead.
\item
  archived recording 3\\
  Bill, everybody ---
\end{itemize}

wesley morris

Up at the front of the room in this space is Larry Kramer, the
playwright activist, author of ``The Normal Heart,'' the play,
``Faggots,'' the novel, and a real figurehead in the AIDS crisis. Larry
Kramer leans forward and he says, plague. He screams it.

\begin{itemize}
\item
  archived recording\\
  You're making the same point towards ---
\item
  archived recording (larry kramer)\\
  Plague! We are in the middle of a fucking plague, and you behave like
  this?

  Plague! 40 million infected people is a fucking plague!
\end{itemize}

jenna wortham

He's like parting the Red Sea when he screams it. It's like ---

wesley morris

Yes. It is --- that is great Jenna.

jenna wortham

It's deeply theatrical.

wesley morris

Yes. It is an Old Testament moment.

jenna wortham

It is.

\begin{itemize}
\tightlist
\item
  archived recording (larry kramer)\\
  We are in the worst shape we have ever, ever, ever been in. All those
  pills we're shoveling down our throats? Forget it. ACT UP has been
  taken over by a lunatic fringe. They can't get together. Nobody agrees
  with anything. All we can do is field a couple of hundred people in a
  demonstration. That's not going to make anybody pay attention, not
  until we get millions out there. We can't do that. All we do is pick
  at each other and yell at each other.
\end{itemize}

wesley morris

He screams this, and the point of his screaming it is that he wants this
infighting to stop, because the stakes are too high. People are dying
right now, and we don't have time for these little petty power
squabbles. There's a larger mission, and our job right now is to figure
out how to execute it. I will never forget the sight of those people in
Wisconsin trying to vote ---

jenna wortham

Heartbreaking.

wesley morris

The other week. And seemingly endless lines in the rain, in a hailstorm.
What would have been all right for that moment would have been somebody
just standing outside some polling station and just saying, ``Plague!
This is a plague! And this is not a time to play games with people's
lives. This primary can happen somewhere down the line. It does not have
to happen today when there are lives on the line, when people are
risking their lives to cast a vote. This is not the day for that.'' And
I just imagine somebody --- I mean, maybe it's just Larry Kramer saying
to these people, this is not a game. This is a life or death thing. This
primary, I mean, in the scheme of things is probably a little --- it's
crucial, but there should be people alive to govern and not risking
their lives in order to vote to be governed in a moment like this. And
basically, what people were doing that day in leaving their houses to
stand in line --- not six feet apart from people, by the way, according
to some of the pictures I saw --- in order to vote because they had to,
for me, it felt like a form of protest.

jenna wortham

Absolutely. I've also been thinking a lot about how workers are
organizing right now in this moment. So there are these incredible
demonstrations where Whole Foods workers and some Amazon workers,
they're calling in sick instead of showing up for work. And they're
essentially protesting inadequate protections for workers who are still
showing up for work, bagging groceries, packing boxes and sending them
out into the world. Essential jobs, no doubt about it, but they don't
feel that they have what they need to care for themselves. And in
January, Whole Foods stopped offering health care for its part-time
employees, and that means anybody who works under 30 hours a week. And
so those are really, really, really hard choices to have to make. Do I
show up and try to make money to take care of myself and take care of my
family, and put myself in harm's way, and potentially contract or spread
the virus? Thinking about those labor uprisings and those worker
movements, A, it's really inspiring, but it also makes me think a lot
about the die-ins that were being held during the AIDS epidemic. In the
peak of the AIDS crisis, where people were going into public spaces ---
sometimes they were churches, sometimes they were in front of federal
buildings --- and they were just laying on the ground and yelling,
``You're killing us. You're killing us because you won't protect us.''

\begin{itemize}
\tightlist
\item
  archived recording\\
  Stop killing us! Stop killing us! Stop killing us! We're not going to
  take it anymore. You're killing us. Stop it! Stop it! Stop it! Stop
  it!
\end{itemize}

jenna wortham

But none of that is possible right now. That's not what protest can look
like right now in 2020 during Covid-19, but it does look like standing
in line to vote in Wisconsin. It also looks like organizing and staging
sick-outs to demand that your workplace is safer, because people can't
organize and have meetings. People can't come together and hold signs
and storm a federal building, although I'm sure we would if we could.

wesley morris

There is this inside-out way that the risk of your life in order to
survive in some ways is a kind of protest. Right? Like, having to leave
your house to file for unemployment benefits, or go to a food bank, is a
form of protest.

jenna wortham

It's interesting, though, because I think there's been this narrative
over the course of the last four years about the death of the American
protest. Or that Americans aren't in the streets enough, and Americans
aren't holding these endless vigils, like people are doing in South
Korea and other countries who are unhappy with their government and just
refusing to stay inside.

wesley morris

Hong Kong.

jenna wortham

And Hong Kong. Yeah, Hong Kong is a great example as well. And the
images that we're receiving of that need laid so bare, so desperate, and
so real is becoming this unexpected demonstration. It's a demonstration
of the stark inequalities of this country. It's puncturing through any
fantasy that people have had. And I actually think that right now, we're
building to the point where the documentary starts, which is six years
into the AIDS epidemic, and there are hundreds of thousands of people
who have died from AIDS. There was a feeling of negligence and death all
around you. We're not there yet, and that is something I'm grateful for
in some ways. But it was also a reality check that we can't let it get
to that point as well. Like, there's no reason for this to go on longer
than it needs to go. And so it was kind of a shorthand to kind of like,
activate now. Like, I felt like I was being powered up like a
transformer. It was like, assemble! {[}LAUGHTER{]} You don't want to get
to the point ---

wesley morris

That's Voltron.

jenna wortham

That's Vol --- for sure.

You know, Wesley, there is this scene in the documentary that I just
can't get out of my mind. It takes place during a moment when the AIDS
quilt is coming to Washington D.C.

wesley morris

Oh yeah.

jenna wortham

And the film shows these huge swaths of cloth being laid out over the
national lawn. And you know what it looks like. It's beautiful. It's
these hand-hewn squares with names on them and pictures and hearts and
other bits of memorabilia. And it's emotional and it's peaceful and it's
evocative, and that's one end of a grieving spectrum. At the other end
of that spectrum, there is an entirely different tone of emotion being
uncorked, and it's a lot of ACT UP activists and other folks who are
also in Washington D.C. They go to the White House, and they bring the
cremated remains of their loved ones, and they're literally taking
handfuls of ash, handfuls of charred bone, and they're tossing it on the
lawn, and they're yelling out their loved one's names.

wesley morris

On the White House lawn.

jenna wortham

And it's so emotional, and I cried, and it's so powerful, and it's so
visceral, and it's so real. And I can see how the AIDS quilt in some
ways was offensive, because it didn't --- it wasn't a strong enough act
to commemorate the lives that were lost unnecessarily. And so in a way
---

wesley morris

It was kind of covering something up.

jenna wortham

It is covering something up and it was kind of --- yeah, it was a
blanket of sorts, right? Like, it was a pacifying gesture.

wesley morris

Right, right.

jenna wortham

And so I was thinking a lot about this film's choice too, and the
choices that were made, and the figures that were included. And like, in
some ways, that watching ``How to Survive a Plague,'' it was so
overwhelmingly white. I have no doubt that those core members of ACT UP
were the ones to focus on and that they were instrumental in pushing
forward the biomedical progress that people suffering from AIDS needed.
And I also couldn't help but feel like, even though there are a few
artists of color, including Ray Navarro, there are a few black people
who are shown who are sick, there are a few black people at the ACT UP
meetings who were shown speaking out, but we don't get their names. We
don't know if they lived or died. We don't know who they were. We don't
know who they are. And ---

wesley morris

Is that the movie's fault? Or is that ACT UP's fault?

jenna wortham

Good question.

wesley morris

Because the movie seems to know what ACT UP's blind spots are, because
they kind of keep recurring in this questioning of where the people of
color are. And it's hard not to watch this movie, once you notice that,
and not think about how there weren't blacks and Latinos in positions of
power pretty much anywhere during this era. And to flash forward to now,
the people who are still suffering most greatly with AIDS and HIV are
black people and Latinos, and those are also the same people dying
disproportionately right now of Covid-19.

jenna wortham

That's right. Black and brown communities are still dealing with the
highest infection rates, right? The New York Times --- we'll put it in
the show notes --- has a great piece about it a few years ago about how
infection rates of HIV are still tremendously high in the south and how
devastating that is. And a lot of it has to do with the whiteness of the
portrayal of the crisis. And so, you know, I was thinking a lot about
that and the similarities to Covid.

wesley morris

The vexing thing about all of this is, whenever these crises happen,
things that were set in motion 400 years ago, essentially ---

jenna wortham

Yes!

wesley morris

Always continue to bear the same results. Right? There isn't anything
you can do. When stuff is supposed to go wrong for everybody, it goes
way more wrong for some people over here in the United States. And that
inequality is still bearing rotten fruit in the 21st century.

jenna wortham

I've been really meditating on the whiteness of ``How to Survive a
Plague,'' and for once it didn't anger me. It actually --- well it did
anger me, but it was also really enlightening because it's really
helping me understand how filtered a lot of the coverage and a lot of
the perception of what the pandemic means for so many. Right now, in
particular, I'm thinking a lot about how my Twitter feed in particular
is just full of people talking about their boredom and loneliness and
what they're cooking. And all those things are valid, but they're really
only one subset of an experience right now, and it's an experience of
people with the means and who have the means to be bored. And for a lot
of people, they're dealing with living in overcrowded apartments.
They're dealing with waiting in line for food. They're dealing with
medical racism. The Times had a really great story about a lot of folks
in Queens. Queens is an area that's been hit the hardest in New York.
And when you watch that film, it's very clear who's not being
represented and who was left out of the narrative. And it is such a
stark reminder who not to leave out this time, and that we have to be
really careful of those deletions, because they do have a historical
impact.

wesley morris

I've been finding that coverage to be heartening on the one hand, but
also like, well, welcome to a thing that has been going on --- what?
There's no internet at some of the homes of these kids who now have to
do distance learning?

jenna wortham

Exactly.

wesley morris

Shocker.

jenna wortham

Right.

wesley morris

It is hitting people --- This is happening --- Wait, it's happening
again? Racism is happening by accident again? It is news to people in a
way that for some reason a hurricane isn't, and it isn't just the Flint
water crisis. There is something about the fact that this is happening
in every state in the country. And in every state in which there is a
black or Latino person, they are suffering this far more greatly than
other people, by and large. And I think it is really forcing people to
think about what racism --- like, what invisible racism is, what ancient
racism is, what skeletal, morally skeletal racism is. It is who is being
forced to get sick in order to stay alive, and it's that racism that's
coming to bear on people and their consciousnesses right now.

jenna wortham

We've never been collectively at a moment in society, in my lifetime,
where everyone is susceptible to the same sickness, right? And all we
want is the same return to health. I just hope that people come out of
this moment, and the desire to go back to the new normal is a slightly
tweaked normal, that what we think of as health shifts, and that we
understand that health is not a given, and that some people are always
ill and always unwell because of the societal conditions under which
they live.

{[}music{]}

wesley morris

That is definitely true. And I have something to offer your hope when we
come back.

{[}music{]}

wesley morris

You know, Jenna, I've been thinking a lot about this moment that you and
I have been talking about --- this `80s and `90s AIDS HIV era. And I've
been thinking a lot about the culture that came out of that moment. The
painters Keith Herring and David Wojnarowicz on the one hand, and then
somebody like the poet Tom Gunn on the other, who wrote this great
collection of poems called ``The Man With Night Sweats'' about AIDS in
that time of crisis, in and around San Francisco especially. And the
filmmaker, the great filmmaker, Marlon Riggs, who was also working
during this period and making really great experimental work. But the
thing, a thing that I've been thinking about, given what we've been
talking about, in --- really in the last couple of weeks is this REM
song that was recorded well after the height of the crisis. The song is
called ``Hope.''

{[}music - "hope," rem{]}

And it's on their album ``Up'' which came out in 1998. And there's
something about this song that is just so resonant with me. It's got
this sort of stripped down beat, and it is essentially Michael Stipe
narrating the story of going to visit his friend who is sick in the
hospital, and the toll that the sickness is taking on him and maybe
their relationship, and the way that you care for sick people and the
way you care for dying people.

He's saying, ``You want to go out Friday. You want to go forever. You
know that it sounds childish that you've dreamt of alligators. You hope
that we are with you, and you hope you're recognized. You want to go
forever. You see it in my eyes. I'm lost in the confusion, and it
doesn't seem to matter. You really can't believe it, and you hope it's
getting better.''

And I don't know. I read a verse like that, and I think about Geri and
how much she had that wanting to go forever in her eyes, and how unlike
in this song and unlike people who are sick and dying under ordinary
circumstances, she can't see anything in anybody else's eyes. I think
this is one of the most beautiful songs that REM's ever written. I think
that it is one of the most beautiful songs ever written about dying or
the uncertainty around dying. But I also think that part of the thing
that makes the song beautiful is the way that the death is sort of
interlaced with this belief that things can get better, even if it
doesn't seem in the moment that they will.

\begin{itemize}
\tightlist
\item
  music - "hope," rem\\
  You want to trust religion, and you know it's allegory, but the people
  who are followers have written their own story. So you look up to the
  heavens, and you hope that it's a space ship, and it's something from
  your childhood. You're thinking don't be frightened.
\end{itemize}

wesley morris

I think about this song all the time for both how specifically detailed
it is, but also how powerfully broad its resonance is. I mean, this is
an experience that any two people can have in any hospital room, except
for right now. And it just stands in for me as a proxy for all of these
experiences that lots and lots of other people can't have with their
loved ones too, but I have hope. I have hope. I have hope.

{[}music{]}

That's our show. And you guys don't have any homework for next week,
because we are going to watch something together.

jenna wortham

Ooh.

wesley morris

We'll explain it on the spot, but just go try to enjoy yourselves as
much as you possibly can. And there's one other thing.

jenna wortham

We really want to hear from you all about how taking care is changing in
your life under Covid. Tell us a story. How are you taking care of the
people in your life, and how are they taking care of you? You can record
yourself using the voice memo app or the voice recorder app, and email
the file to
\href{mailto:StillProcessing@NYTimes.com}{\nolinkurl{StillProcessing@NYTimes.com}}.
We may use what you send us in an upcoming episode. Either way, thank
you so much for your time.

wesley morris

``Still Processing'' is a product of The New York Times, and it was
recorded in our living rooms.

jenna wortham

It is produced by Hans Buetow.

wesley morris

Our editors are Sarah Sarasohn, Sasha Weiss, Wendy Dorr, and Lisa Tobin.

jenna wortham

Our engineer is Jake Gorski.

wesley morris

And our theme music's by Kindness. It's called ``World Restart'' from
the album ``Otherness.''

jenna wortham

You can find all of our episodes and various fun things at
NYTimes.com/StillProcessing.

wesley morris

Thanks for listening, everybody.

jenna wortham

See you next week.

\href{https://www.nytimes.com/column/still-processing-podcast}{\includegraphics{https://static01.nyt.com/images/2019/09/15/podcasts/still-processing-album-art-2/still-processing-album-art-2-square320.jpg}Still
Processing}Subscribe:

\begin{itemize}
\tightlist
\item
  \href{https://itunes.apple.com/us/podcast/id1151436460}{Apple
  Podcasts}
\item
  \href{https://www.google.com/podcasts?feed=aHR0cHM6Ly9yc3MuYXJ0MTkuY29tL255dC1zdGlsbC1wcm9jZXNzaW5n}{Google
  Podcasts}
\end{itemize}

\hypertarget{how-to-learn-from-a-plague-1}{%
\section{How to Learn From a
Plague}\label{how-to-learn-from-a-plague-1}}

\hypertarget{we-study-the-aids-epidemic-documentary-how-to-survive-a-plague--and-apply-its-lessons-to-the-covid-19-crisis-1}{%
\subsection{We study the AIDS-epidemic documentary ``How to Survive a
Plague'' --- and apply its lessons to the Covid-19
crisis.}\label{we-study-the-aids-epidemic-documentary-how-to-survive-a-plague--and-apply-its-lessons-to-the-covid-19-crisis-1}}

Hosted by Wesley Morris and Jenna Wortham. Produced by Hans Buetow.

Transcript

transcript

Back to Still Processing

bars

0:00/0:00

-0:00

transcript

\hypertarget{how-to-learn-from-a-plague-2}{%
\subsection{How to Learn From a
Plague}\label{how-to-learn-from-a-plague-2}}

\hypertarget{hosted-by-wesley-morris-and-jenna-wortham-produced-by-hans-buetow-1}{%
\subsubsection{Hosted by Wesley Morris and Jenna Wortham. Produced by
Hans
Buetow.}\label{hosted-by-wesley-morris-and-jenna-wortham-produced-by-hans-buetow-1}}

\hypertarget{we-study-the-aids-epidemic-documentary-how-to-survive-a-plague--and-apply-its-lessons-to-the-covid-19-crisis-2}{%
\paragraph{We study the AIDS-epidemic documentary ``How to Survive a
Plague'' --- and apply its lessons to the Covid-19
crisis.}\label{we-study-the-aids-epidemic-documentary-how-to-survive-a-plague--and-apply-its-lessons-to-the-covid-19-crisis-2}}

Thursday, April 16th, 2020

\begin{itemize}
\item
  wesley morris\\
  Can I tell you, Jenna, about my Aunt Geri?
\item
  jenna wortham\\
  Please. I would love to hear more.
\item
  wesley morris\\
  Well, you know that she died a couple weeks ago of Covid-19. She was
  in a nursing home for a while. And like many people who are in nursing
  homes right now, it is not the greatest place to be with this disease
  out there. And she was one of the unfortunate people who got it. She
  was 90 years old, and her life was good. It was very good. Aunt Geri
  was so full of life, and so full of energy, and so funny, and she
  loved having people around her. Let me --- OK, Jenna, let me just ask
  you a question. Like, what is the thing where like, you don't like
  Thanksgiving, but you will put up with it because there's this one
  thing you get to eat every year? What is it?
\item
  jenna wortham\\
  100 percent. 100 percent it's stuffing.
\item
  wesley morris\\
  Aunt Geri was the stuffing on your Thanksgiving plate. My personal
  favorite thing is macaroni and cheese with cranberry sauce. My Aunt
  Geri was the mac and cheese and the cranberry sauce. She was the
  person who, whenever my family got together, even if I didn't even
  think about her being there, the minute I get in the house, Aunt Geri
  is there, and I'm happy. She told funny stories. She was a great side
  person in somebody else's story if she was present when my grandmother
  was holding court. My Aunt Geri was basically like the Pips to my
  grandmother's Gladys Knight. She was always there and just be like,
  leaving ---
\item
  jenna wortham\\
  Oh, my god. So good.
\item
  wesley morris\\
  On that midnight train. Yeah, I mean, that was my Aunt Geri. She was
  there to back up any story being told that she knew anything about.
  And whatever your favorite thing is, that was Aunt Geri. She was one
  of my favorite things. And it's just wild to me that this vivacious,
  humorous, vinegary woman who loved so many people, who was loved by so
  many people, who lived such a full and rich life, would have to spend
  the last of those days by herself, untouched, unspoken to,
  essentially, by anybody who loved her, by the people she raised. I
  think all of us in my family would have loved to have given her that.
\item
  jenna wortham\\
  It's really hard to grapple with the reality that in order to take
  care of someone, and in order to love them, you have to stay away from
  them, because you could get sick. I mean, their bodies are dangerous.
  They are carrying the virus.
\item
  wesley morris\\
  And where my brain goes thinking about where to put my sadness, or
  like how to think about it in some other context, is to the 1980s and
  1990s, when people were also dying stigmatized deaths. And it isn't so
  much that my Aunt Geri died stigmatized, but she kind of did, Jenna.
  She was forced to die in a way that feels, emotionally, to me,
  shameful, because it's not the way human beings commemorate each
  other.
\item
  jenna wortham\\
  Right.
\item
  wesley morris\\
  We couldn't be in space with her. We couldn't talk her through her
  death. We couldn't comfort her as she died. She just had to do it on
  her own. And so there is this --- there's this spot on Aunt Geri's
  death in a way that feels, kinda makes me feel worse.
\item
  jenna wortham\\
  But that last act of Aunt Geri was one of love, you know? She died
  alone to prevent anyone else from getting sick, and it's not right. It
  isn't right.
\item
  wesley morris\\
  But that's where we are.
\item
  jenna wortham\\
  Yeah. And that's where we are.
\item
  {[}music{]}
\item
  wesley morris\\
  I'm Wesley Morris.
\item
  jenna wortham\\
  I'm Jenna Wortham.
\item
  wesley morris\\
  We're two New York Times writers hunkered down in our living rooms.
\item
  jenna wortham\\
  This is ``Still Processing.''
\item
  wesley morris\\
  Jenna, you and I watched this film, ``How to Survive a Plague.'' I
  think this movie, which was made in 2012 by David France, is a really
  instructive blueprint for how we might proceed, and the ways in which
  the era of the AIDS crisis, which lasted for the 1980s and 1990s, the
  high point of it is just really useful to compare these two areas.
\item
  jenna wortham\\
  Sure.
\item
  wesley morris\\
  This one that we've just embarked on, and this one that was well under
  way when the movie starts. And essentially, David France made this
  film out of lots of video footage that existed from the time of ACT UP
  and the fight to get some sort of drug treatment for these people who
  were dying and who contracted the disease.
\item
  archived recording\\
  Welcome to ACT UP. We are the AIDS Coaltion to Unleash Power, a
  diverse, non-partisan group of individuals united in anger and
  committed to direct action to end the AIDS crisis.
\end{itemize}

wesley morris

And they were formed in 1987, and they essentially harassed and
basically terrorized the government into giving them the drugs.

\begin{itemize}
\tightlist
\item
  archived recording\\
  Fight back. Fight AIDS. Act up. Fight back. Fight AIDS.
\end{itemize}

wesley morris

I think part of the power comes from the fear of the people organizing,
and the uncertainty, and the anger that they have that there is
something that can be done, but the government, the U.S. government in
this case, isn't doing enough, if anything, to fight it. And I think the
reason I wanted us to watch it was because it just seemed like a film
that nobody's really talking about in terms of what lessons we can
learn. Everybody's watching ``Contagion.''

jenna wortham

And ``28 Days Later,'' I just keep bringing it up.

wesley morris

You really keep bringing it up.

{[}laughter{]}

wesley morris

I know it's not just you.

jenna wortham

My biggest regret. My biggest regret. Go on.

wesley morris

But I think ``How to Survive a Plague'' is the most instructive movie
manual we might have. I mean, there are all kinds of books we can read
as well, but movie-wise I actually wonder if, at this point where we are
now --- ``Contagion'' made sense for the first phase of this.

jenna wortham

Right.

wesley morris

I feel like we're entering a more existential phase, and a more angry
and political phase.

jenna wortham

We're more engaged.

wesley morris

Yeah.

jenna wortham

Like, this is the new reality. Now we want to know what to do with it.

wesley morris

Right, right. I feel like there are a lot of ways in which this moment
is similar to what was happening 25, 30, 35 years ago, and one of those
ways is what we were talking about with my aunt, which involves the
danger of a vector and the fact that we can't touch each other. I'm
thinking about this moment in ``How to Survive a Plague'' where Peter
Staley, who's one of the most prominent members of ACT UP, is on
``Crossfire,'' CNN's ``Crossfire,'' back when ``Crossfire'' was not the
way all cable news was now, which is people yelling at each other.

\begin{itemize}
\tightlist
\item
  archived recording\\
  Tonight from Washington, ``Crossfire,'' against all odds. On the left,
  Tom Braden. On the right, Pat Buchanan. In the crossfire, Peter Staley
  of the New York AIDS Coalition To Unleash Power.
\end{itemize}

wesley morris

Pat Buchanan, conservative Republican on one side, and Tom Braden, I
don't know what political affiliation Tom Braden was. He basically is
famous for having written the book that became the television show
``Eight is Enough.'' Anyway, at some point Pat Buchanan says to Peter
Staley, you know, I want you to look into the camera, and what would you
tell a young kid who wants to have sex?

jenna wortham

Well, he said your brother. He was like ---

wesley morris

Yeah, or your brother. Yeah, your young brother, yeah.

jenna wortham

If you have a brother, he's 21, what would you say to him? Would you
want him to avoid this life that you're living? Peter Staley is HIV
positive.

\begin{itemize}
\item
  archived recording (pat buchanan)\\
  What would you tell him if you wanted him to live a long life?
\item
  archived recording (peter staley)\\
  Use a condom. And also to use a lubricant, by the way, that has the
  medicine that can ---
\item
  archived recording (pat buchanan)\\
  This is Russian roulette.
\item
  archived recording (peter staley)\\
  It is not Russian roulette. It is Russian roulette to not give people
  this information when human nature dictates that they're going to go
  out there and they're going to have sex anyway.
\item
  archived recording (pat buchanan)\\
  You mean celibacy is impossible?
\item
  archived recording (peter staley)\\
  It's just not going to work. People aren't going to do it, and lots,
  lots of people are going to die.
\end{itemize}

jenna wortham

It's incredible because the two men on either side of him cannot deal.
They're both just sputtering. They literally --- if they were a cartoon,
there'd be like smoke coming out of their ears.

\begin{itemize}
\tightlist
\item
  archived recording (pat buchanan)\\
  I think that --- well, thank you very much, Peter Staley. Thanks for
  being in our studio. Mr. Braden and I will be back in a minute.
\end{itemize}

{[}music{]}

wesley morris

What I hear Peter Staley saying in that moment is that we need the tools
to be safe. Because the truth is, and you can see it in 2020, you can
see it right now anytime you leave the house with the uncertainty around
how to be safe around each other. And right now the solution is just to
quarantine and stay in the house and not go anywhere near each other.
But I think part of the anger that is bubbling up among so many of us
right now is that there is potentially a way we can begin to be safe
around each other, and that is through testing. Those guys wanted
condoms. We want tests. And it is a thing that is going to help us be
safe when it's time to actually do what comes natural to us, which is
not stay in the house. A very simple clarification that could be made
for us medically, and at this point on behalf of the government, is
testing. Where are the tests?

jenna wortham

Mmm.

wesley morris

And there are two things that really have stayed with me in this outing
with ``How to Survive a Plague,'' and the one I just talked about with
Peter Staley and Pat Buchanan and Tom Braden on ``Crossfire.'' And the
other one is this moment at which things are really tense among ACT UP
members. They've reached a sort of stage where, I don't know how else to
put this except like, some pettiness has begun to creep into the
operation. And there are people who think that the people at the front
of the organization, at the top of ACT UP, are getting too much
attention, and they're a little too famous.

jenna wortham

A common activist, organizer dilemma. In every movement there comes a
moment. Go on.

wesley morris

Yes. Well, there's a little --- somebody in the peanut gallery is trying
to chirp up and cause a little bit of disruption.

\begin{itemize}
\item
  archived recording 1\\
  Bill, you're going to have a chance to talk, all right? Everybody ---
\item
  archived recording 2\\
  Don't lecture me, you stupid, lazy, incompetent shithead.
\item
  archived recording 3\\
  Bill, everybody ---
\end{itemize}

wesley morris

Up at the front of the room in this space is Larry Kramer, the
playwright activist, author of ``The Normal Heart,'' the play,
``Faggots,'' the novel, and a real figurehead in the AIDS crisis. Larry
Kramer leans forward and he says, plague. He screams it.

\begin{itemize}
\item
  archived recording\\
  You're making the same point towards ---
\item
  archived recording (larry kramer)\\
  Plague! We are in the middle of a fucking plague, and you behave like
  this?

  Plague! 40 million infected people is a fucking plague!
\end{itemize}

jenna wortham

He's like parting the Red Sea when he screams it. It's like ---

wesley morris

Yes. It is --- that is great Jenna.

jenna wortham

It's deeply theatrical.

wesley morris

Yes. It is an Old Testament moment.

jenna wortham

It is.

\begin{itemize}
\tightlist
\item
  archived recording (larry kramer)\\
  We are in the worst shape we have ever, ever, ever been in. All those
  pills we're shoveling down our throats? Forget it. ACT UP has been
  taken over by a lunatic fringe. They can't get together. Nobody agrees
  with anything. All we can do is field a couple of hundred people in a
  demonstration. That's not going to make anybody pay attention, not
  until we get millions out there. We can't do that. All we do is pick
  at each other and yell at each other.
\end{itemize}

wesley morris

He screams this, and the point of his screaming it is that he wants this
infighting to stop, because the stakes are too high. People are dying
right now, and we don't have time for these little petty power
squabbles. There's a larger mission, and our job right now is to figure
out how to execute it. I will never forget the sight of those people in
Wisconsin trying to vote ---

jenna wortham

Heartbreaking.

wesley morris

The other week. And seemingly endless lines in the rain, in a hailstorm.
What would have been all right for that moment would have been somebody
just standing outside some polling station and just saying, ``Plague!
This is a plague! And this is not a time to play games with people's
lives. This primary can happen somewhere down the line. It does not have
to happen today when there are lives on the line, when people are
risking their lives to cast a vote. This is not the day for that.'' And
I just imagine somebody --- I mean, maybe it's just Larry Kramer saying
to these people, this is not a game. This is a life or death thing. This
primary, I mean, in the scheme of things is probably a little --- it's
crucial, but there should be people alive to govern and not risking
their lives in order to vote to be governed in a moment like this. And
basically, what people were doing that day in leaving their houses to
stand in line --- not six feet apart from people, by the way, according
to some of the pictures I saw --- in order to vote because they had to,
for me, it felt like a form of protest.

jenna wortham

Absolutely. I've also been thinking a lot about how workers are
organizing right now in this moment. So there are these incredible
demonstrations where Whole Foods workers and some Amazon workers,
they're calling in sick instead of showing up for work. And they're
essentially protesting inadequate protections for workers who are still
showing up for work, bagging groceries, packing boxes and sending them
out into the world. Essential jobs, no doubt about it, but they don't
feel that they have what they need to care for themselves. And in
January, Whole Foods stopped offering health care for its part-time
employees, and that means anybody who works under 30 hours a week. And
so those are really, really, really hard choices to have to make. Do I
show up and try to make money to take care of myself and take care of my
family, and put myself in harm's way, and potentially contract or spread
the virus? Thinking about those labor uprisings and those worker
movements, A, it's really inspiring, but it also makes me think a lot
about the die-ins that were being held during the AIDS epidemic. In the
peak of the AIDS crisis, where people were going into public spaces ---
sometimes they were churches, sometimes they were in front of federal
buildings --- and they were just laying on the ground and yelling,
``You're killing us. You're killing us because you won't protect us.''

\begin{itemize}
\tightlist
\item
  archived recording\\
  Stop killing us! Stop killing us! Stop killing us! We're not going to
  take it anymore. You're killing us. Stop it! Stop it! Stop it! Stop
  it!
\end{itemize}

jenna wortham

But none of that is possible right now. That's not what protest can look
like right now in 2020 during Covid-19, but it does look like standing
in line to vote in Wisconsin. It also looks like organizing and staging
sick-outs to demand that your workplace is safer, because people can't
organize and have meetings. People can't come together and hold signs
and storm a federal building, although I'm sure we would if we could.

wesley morris

There is this inside-out way that the risk of your life in order to
survive in some ways is a kind of protest. Right? Like, having to leave
your house to file for unemployment benefits, or go to a food bank, is a
form of protest.

jenna wortham

It's interesting, though, because I think there's been this narrative
over the course of the last four years about the death of the American
protest. Or that Americans aren't in the streets enough, and Americans
aren't holding these endless vigils, like people are doing in South
Korea and other countries who are unhappy with their government and just
refusing to stay inside.

wesley morris

Hong Kong.

jenna wortham

And Hong Kong. Yeah, Hong Kong is a great example as well. And the
images that we're receiving of that need laid so bare, so desperate, and
so real is becoming this unexpected demonstration. It's a demonstration
of the stark inequalities of this country. It's puncturing through any
fantasy that people have had. And I actually think that right now, we're
building to the point where the documentary starts, which is six years
into the AIDS epidemic, and there are hundreds of thousands of people
who have died from AIDS. There was a feeling of negligence and death all
around you. We're not there yet, and that is something I'm grateful for
in some ways. But it was also a reality check that we can't let it get
to that point as well. Like, there's no reason for this to go on longer
than it needs to go. And so it was kind of a shorthand to kind of like,
activate now. Like, I felt like I was being powered up like a
transformer. It was like, assemble! {[}LAUGHTER{]} You don't want to get
to the point ---

wesley morris

That's Voltron.

jenna wortham

That's Vol --- for sure.

You know, Wesley, there is this scene in the documentary that I just
can't get out of my mind. It takes place during a moment when the AIDS
quilt is coming to Washington D.C.

wesley morris

Oh yeah.

jenna wortham

And the film shows these huge swaths of cloth being laid out over the
national lawn. And you know what it looks like. It's beautiful. It's
these hand-hewn squares with names on them and pictures and hearts and
other bits of memorabilia. And it's emotional and it's peaceful and it's
evocative, and that's one end of a grieving spectrum. At the other end
of that spectrum, there is an entirely different tone of emotion being
uncorked, and it's a lot of ACT UP activists and other folks who are
also in Washington D.C. They go to the White House, and they bring the
cremated remains of their loved ones, and they're literally taking
handfuls of ash, handfuls of charred bone, and they're tossing it on the
lawn, and they're yelling out their loved one's names.

wesley morris

On the White House lawn.

jenna wortham

And it's so emotional, and I cried, and it's so powerful, and it's so
visceral, and it's so real. And I can see how the AIDS quilt in some
ways was offensive, because it didn't --- it wasn't a strong enough act
to commemorate the lives that were lost unnecessarily. And so in a way
---

wesley morris

It was kind of covering something up.

jenna wortham

It is covering something up and it was kind of --- yeah, it was a
blanket of sorts, right? Like, it was a pacifying gesture.

wesley morris

Right, right.

jenna wortham

And so I was thinking a lot about this film's choice too, and the
choices that were made, and the figures that were included. And like, in
some ways, that watching ``How to Survive a Plague,'' it was so
overwhelmingly white. I have no doubt that those core members of ACT UP
were the ones to focus on and that they were instrumental in pushing
forward the biomedical progress that people suffering from AIDS needed.
And I also couldn't help but feel like, even though there are a few
artists of color, including Ray Navarro, there are a few black people
who are shown who are sick, there are a few black people at the ACT UP
meetings who were shown speaking out, but we don't get their names. We
don't know if they lived or died. We don't know who they were. We don't
know who they are. And ---

wesley morris

Is that the movie's fault? Or is that ACT UP's fault?

jenna wortham

Good question.

wesley morris

Because the movie seems to know what ACT UP's blind spots are, because
they kind of keep recurring in this questioning of where the people of
color are. And it's hard not to watch this movie, once you notice that,
and not think about how there weren't blacks and Latinos in positions of
power pretty much anywhere during this era. And to flash forward to now,
the people who are still suffering most greatly with AIDS and HIV are
black people and Latinos, and those are also the same people dying
disproportionately right now of Covid-19.

jenna wortham

That's right. Black and brown communities are still dealing with the
highest infection rates, right? The New York Times --- we'll put it in
the show notes --- has a great piece about it a few years ago about how
infection rates of HIV are still tremendously high in the south and how
devastating that is. And a lot of it has to do with the whiteness of the
portrayal of the crisis. And so, you know, I was thinking a lot about
that and the similarities to Covid.

wesley morris

The vexing thing about all of this is, whenever these crises happen,
things that were set in motion 400 years ago, essentially ---

jenna wortham

Yes!

wesley morris

Always continue to bear the same results. Right? There isn't anything
you can do. When stuff is supposed to go wrong for everybody, it goes
way more wrong for some people over here in the United States. And that
inequality is still bearing rotten fruit in the 21st century.

jenna wortham

I've been really meditating on the whiteness of ``How to Survive a
Plague,'' and for once it didn't anger me. It actually --- well it did
anger me, but it was also really enlightening because it's really
helping me understand how filtered a lot of the coverage and a lot of
the perception of what the pandemic means for so many. Right now, in
particular, I'm thinking a lot about how my Twitter feed in particular
is just full of people talking about their boredom and loneliness and
what they're cooking. And all those things are valid, but they're really
only one subset of an experience right now, and it's an experience of
people with the means and who have the means to be bored. And for a lot
of people, they're dealing with living in overcrowded apartments.
They're dealing with waiting in line for food. They're dealing with
medical racism. The Times had a really great story about a lot of folks
in Queens. Queens is an area that's been hit the hardest in New York.
And when you watch that film, it's very clear who's not being
represented and who was left out of the narrative. And it is such a
stark reminder who not to leave out this time, and that we have to be
really careful of those deletions, because they do have a historical
impact.

wesley morris

I've been finding that coverage to be heartening on the one hand, but
also like, well, welcome to a thing that has been going on --- what?
There's no internet at some of the homes of these kids who now have to
do distance learning?

jenna wortham

Exactly.

wesley morris

Shocker.

jenna wortham

Right.

wesley morris

It is hitting people --- This is happening --- Wait, it's happening
again? Racism is happening by accident again? It is news to people in a
way that for some reason a hurricane isn't, and it isn't just the Flint
water crisis. There is something about the fact that this is happening
in every state in the country. And in every state in which there is a
black or Latino person, they are suffering this far more greatly than
other people, by and large. And I think it is really forcing people to
think about what racism --- like, what invisible racism is, what ancient
racism is, what skeletal, morally skeletal racism is. It is who is being
forced to get sick in order to stay alive, and it's that racism that's
coming to bear on people and their consciousnesses right now.

jenna wortham

We've never been collectively at a moment in society, in my lifetime,
where everyone is susceptible to the same sickness, right? And all we
want is the same return to health. I just hope that people come out of
this moment, and the desire to go back to the new normal is a slightly
tweaked normal, that what we think of as health shifts, and that we
understand that health is not a given, and that some people are always
ill and always unwell because of the societal conditions under which
they live.

{[}music{]}

wesley morris

That is definitely true. And I have something to offer your hope when we
come back.

{[}music{]}

wesley morris

You know, Jenna, I've been thinking a lot about this moment that you and
I have been talking about --- this `80s and `90s AIDS HIV era. And I've
been thinking a lot about the culture that came out of that moment. The
painters Keith Herring and David Wojnarowicz on the one hand, and then
somebody like the poet Tom Gunn on the other, who wrote this great
collection of poems called ``The Man With Night Sweats'' about AIDS in
that time of crisis, in and around San Francisco especially. And the
filmmaker, the great filmmaker, Marlon Riggs, who was also working
during this period and making really great experimental work. But the
thing, a thing that I've been thinking about, given what we've been
talking about, in --- really in the last couple of weeks is this REM
song that was recorded well after the height of the crisis. The song is
called ``Hope.''

{[}music - "hope," rem{]}

And it's on their album ``Up'' which came out in 1998. And there's
something about this song that is just so resonant with me. It's got
this sort of stripped down beat, and it is essentially Michael Stipe
narrating the story of going to visit his friend who is sick in the
hospital, and the toll that the sickness is taking on him and maybe
their relationship, and the way that you care for sick people and the
way you care for dying people.

He's saying, ``You want to go out Friday. You want to go forever. You
know that it sounds childish that you've dreamt of alligators. You hope
that we are with you, and you hope you're recognized. You want to go
forever. You see it in my eyes. I'm lost in the confusion, and it
doesn't seem to matter. You really can't believe it, and you hope it's
getting better.''

And I don't know. I read a verse like that, and I think about Geri and
how much she had that wanting to go forever in her eyes, and how unlike
in this song and unlike people who are sick and dying under ordinary
circumstances, she can't see anything in anybody else's eyes. I think
this is one of the most beautiful songs that REM's ever written. I think
that it is one of the most beautiful songs ever written about dying or
the uncertainty around dying. But I also think that part of the thing
that makes the song beautiful is the way that the death is sort of
interlaced with this belief that things can get better, even if it
doesn't seem in the moment that they will.

\begin{itemize}
\tightlist
\item
  music - "hope," rem\\
  You want to trust religion, and you know it's allegory, but the people
  who are followers have written their own story. So you look up to the
  heavens, and you hope that it's a space ship, and it's something from
  your childhood. You're thinking don't be frightened.
\end{itemize}

wesley morris

I think about this song all the time for both how specifically detailed
it is, but also how powerfully broad its resonance is. I mean, this is
an experience that any two people can have in any hospital room, except
for right now. And it just stands in for me as a proxy for all of these
experiences that lots and lots of other people can't have with their
loved ones too, but I have hope. I have hope. I have hope.

{[}music{]}

That's our show. And you guys don't have any homework for next week,
because we are going to watch something together.

jenna wortham

Ooh.

wesley morris

We'll explain it on the spot, but just go try to enjoy yourselves as
much as you possibly can. And there's one other thing.

jenna wortham

We really want to hear from you all about how taking care is changing in
your life under Covid. Tell us a story. How are you taking care of the
people in your life, and how are they taking care of you? You can record
yourself using the voice memo app or the voice recorder app, and email
the file to
\href{mailto:StillProcessing@NYTimes.com}{\nolinkurl{StillProcessing@NYTimes.com}}.
We may use what you send us in an upcoming episode. Either way, thank
you so much for your time.

wesley morris

``Still Processing'' is a product of The New York Times, and it was
recorded in our living rooms.

jenna wortham

It is produced by Hans Buetow.

wesley morris

Our editors are Sarah Sarasohn, Sasha Weiss, Wendy Dorr, and Lisa Tobin.

jenna wortham

Our engineer is Jake Gorski.

wesley morris

And our theme music's by Kindness. It's called ``World Restart'' from
the album ``Otherness.''

jenna wortham

You can find all of our episodes and various fun things at
NYTimes.com/StillProcessing.

wesley morris

Thanks for listening, everybody.

jenna wortham

See you next week.

Previous

More episodes ofStill Processing

\href{https://www.nytimes.com/2020/07/23/podcasts/hamilton-ziwe-discomfort.html?action=click\&module=audio-series-bar\&region=header\&pgtype=Article}{\includegraphics{https://static01.nyt.com/images/2020/07/23/multimedia/23stillprocessing-pix/23stillprocessing-pix-thumbLarge.jpg}}

July 23, 2020~~•~ 38:10Ziwe May Destroy Hamilton

\href{https://www.nytimes.com/2020/07/16/podcasts/reparations-for-aunt-jemima.html?action=click\&module=audio-series-bar\&region=header\&pgtype=Article}{\includegraphics{https://static01.nyt.com/images/2020/07/18/multimedia/16stillprocessing-pix/16stillprocessing-pix-thumbLarge.jpg}}

July 16, 2020~~•~ 35:35Reparations for Aunt Jemima!

\href{https://www.nytimes.com/2020/07/09/podcasts/still-processing-black-lives-matter.html?action=click\&module=audio-series-bar\&region=header\&pgtype=Article}{\includegraphics{https://static01.nyt.com/images/2020/07/12/podcasts/09stillprocessing-image/xx-stillprocessing-thumbLarge.jpg}}

July 9, 2020~~•~ 26:29So Y'all Finally Get It

\href{https://www.nytimes.com/2020/05/14/podcasts/still-processing-westworld-hollywood-utopia-dystopia.html?action=click\&module=audio-series-bar\&region=header\&pgtype=Article}{\includegraphics{https://static01.nyt.com/images/2020/05/16/podcasts/14stillprocessing-image/14stillprocessing-image-thumbLarge-v2.jpg}}

May 14, 2020New Loop, America

\href{https://www.nytimes.com/2020/05/07/podcasts/still-processing-internet-vulnerability-sondheim-parks-recreation.html?action=click\&module=audio-series-bar\&region=header\&pgtype=Article}{\includegraphics{https://static01.nyt.com/images/2020/04/28/pageoneplus/28sondheimjp-sp/28sondheimjp-sp-thumbLarge-v4.jpg}}

May 7, 2020Does This Phone Make Me Look Human?

\href{https://www.nytimes.com/2020/04/30/podcasts/still-processing-fiona-apple-fetch-bolt-cutters.html?action=click\&module=audio-series-bar\&region=header\&pgtype=Article}{\includegraphics{https://static01.nyt.com/images/2020/05/03/multimedia/30stillpro-image/30stillpro-image-thumbLarge.jpg}}

May 1, 2020Fiona Ex Machina

\href{https://www.nytimes.com/2020/04/23/podcasts/still-processing-halle-berry-sharon-stone-catwoman-quarantine.html?action=click\&module=audio-series-bar\&region=header\&pgtype=Article}{\includegraphics{https://static01.nyt.com/images/2020/04/25/arts/23stillprocessing/23stillprocessing-thumbLarge-v3.jpg}}

April 23, 2020Halle Berry? Hallelujah.

\href{https://www.nytimes.com/2020/04/16/podcasts/still-processing-AIDS-survive-coronavirus.html?action=click\&module=audio-series-bar\&region=header\&pgtype=Article}{\includegraphics{https://static01.nyt.com/images/2020/04/20/us/16stillprocessing/16stillprocessing-thumbLarge-v3.jpg}}

April 16, 2020How to Learn From a Plague

\href{https://www.nytimes.com/2020/04/09/podcasts/still-processing-tiger-king.html?action=click\&module=audio-series-bar\&region=header\&pgtype=Article}{\includegraphics{https://static01.nyt.com/images/2020/04/11/podcasts/09stillprocessing-image2/09stillprocessing-image2-thumbLarge-v2.jpg}}

April 9, 2020~~•~ 39:49Frosted Flakes

\href{https://www.nytimes.com/2020/04/02/podcasts/high-fidelity-zoe-kravitz.html?action=click\&module=audio-series-bar\&region=header\&pgtype=Article}{\includegraphics{https://static01.nyt.com/images/2020/04/05/arts/02still-processing-highfidelity/13highfidelity-thumbLarge.jpg}}

April 2, 2020~~•~ 40:55Delicious Vinyl

\href{https://www.nytimes.com/2020/03/26/podcasts/still-processing-quarantine.html?action=click\&module=audio-series-bar\&region=header\&pgtype=Article}{\includegraphics{https://static01.nyt.com/images/2020/03/29/podcasts/26stillprocessing1/26stillprocessing1-thumbLarge.jpg}}

March 26, 2020~~•~ 30:47A Pod From Both Our Houses

\href{https://www.nytimes.com/2019/11/07/podcasts/still-processing-parasite-watchmen-bong-joon-ho.html?action=click\&module=audio-series-bar\&region=header\&pgtype=Article}{\includegraphics{https://static01.nyt.com/images/2019/11/08/arts/07stilpr-parasite/00parasite-1-thumbLarge.jpg}}

November 7, 2019Wake

\href{https://www.nytimes.com/column/still-processing-podcast}{See All
Episodes ofStill Processing}

Next

Published April 16, 2020Updated May 12, 2020

\begin{itemize}
\item
\item
\item
\item
\item
\end{itemize}

By \href{https://www.nytimes.com/by/wesley-morris}{Wesley Morris} and
\href{https://www.nytimes.com/by/jenna-wortham}{Jenna Wortham}

Activists stood up against the AIDS epidemic in the 1980s and 1990s, but
the tools they used to make themselves heard are unavailable during our
coronavirus pandemic. Still, many of that era's strategies and warning
signs seem alarmingly relevant now.

\includegraphics{https://static01.nyt.com/images/2020/04/20/us/16stillprocessing/16stillprocessing-articleLarge-v3.jpg?quality=75\&auto=webp\&disable=upscale}

\textbf{Discussed this week:}

\begin{itemize}
\item
  ``\href{https://www.imdb.com/title/tt2124803/}{How to Survive a
  Plague}'' (directed by David France, 2012)
\item
  \href{https://actupny.org/}{ACT UP New York}
\item
  ``\href{https://www.nytimes.com/interactive/2020/04/13/t-magazine/act-up-aids.html}{How
  ACT UP Remade Political Organizing in America}'' (David France, The
  New York Times, April, 2020)
\item
  ``\href{https://www.nytimes.com/2020/04/09/nyregion/coronavirus-queens-corona-jackson-heights-elmhurst.html}{`A
  Tragedy Is Unfolding': Inside New York's Virus Epicenter}'' (Annie
  Correal, Andrew Jacobs and Ryan Christopher Jones, The New York Times,
  April, 2020)
\item
  ``\href{https://www.nytimes.com/2017/06/06/magazine/americas-hidden-hiv-epidemic.html}{America's
  Hidden H.I.V. Epidemic}'' (Linda Villarosa, The New York Times, June,
  2017)
\item
  ``\href{https://www.bloomberg.com/news/articles/2019-09-12/amazon-s-whole-foods-to-cut-benefits-for-part-timers-report}{Amazon's
  Whole Foods to Cut Medical Benefits for Part-Timers}'' (Spencer Soper,
  Bloomberg, September, 2019)
\end{itemize}

``Still Processing'' is produced by Hans Buetow and edited by Sara
Sarasohn and Sasha Weiss, with editorial oversight from Wendy Dorr and
Lisa Tobin. Our engineer is Jake Gorski. Our theme music is by Kindness.
It's called ``World Restart,'' from the album ``Otherness.''

Advertisement

\protect\hyperlink{after-bottom}{Continue reading the main story}

\hypertarget{site-index}{%
\subsection{Site Index}\label{site-index}}

\hypertarget{site-information-navigation}{%
\subsection{Site Information
Navigation}\label{site-information-navigation}}

\begin{itemize}
\tightlist
\item
  \href{https://help.nytimes.com/hc/en-us/articles/115014792127-Copyright-notice}{©~2020~The
  New York Times Company}
\end{itemize}

\begin{itemize}
\tightlist
\item
  \href{https://www.nytco.com/}{NYTCo}
\item
  \href{https://help.nytimes.com/hc/en-us/articles/115015385887-Contact-Us}{Contact
  Us}
\item
  \href{https://www.nytco.com/careers/}{Work with us}
\item
  \href{https://nytmediakit.com/}{Advertise}
\item
  \href{http://www.tbrandstudio.com/}{T Brand Studio}
\item
  \href{https://www.nytimes.com/privacy/cookie-policy\#how-do-i-manage-trackers}{Your
  Ad Choices}
\item
  \href{https://www.nytimes.com/privacy}{Privacy}
\item
  \href{https://help.nytimes.com/hc/en-us/articles/115014893428-Terms-of-service}{Terms
  of Service}
\item
  \href{https://help.nytimes.com/hc/en-us/articles/115014893968-Terms-of-sale}{Terms
  of Sale}
\item
  \href{https://spiderbites.nytimes.com}{Site Map}
\item
  \href{https://help.nytimes.com/hc/en-us}{Help}
\item
  \href{https://www.nytimes.com/subscription?campaignId=37WXW}{Subscriptions}
\end{itemize}
