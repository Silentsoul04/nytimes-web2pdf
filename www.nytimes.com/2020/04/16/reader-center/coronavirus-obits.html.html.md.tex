Sections

SEARCH

\protect\hyperlink{site-content}{Skip to
content}\protect\hyperlink{site-index}{Skip to site index}

\href{https://www.nytimes.com/section/reader-center}{Reader Center}

\href{https://myaccount.nytimes.com/auth/login?response_type=cookie\&client_id=vi}{}

\href{https://www.nytimes.com/section/todayspaper}{Today's Paper}

\href{/section/reader-center}{Reader Center}\textbar{}Faces That Can't
Be Forgotten

\url{https://nyti.ms/3bs188Y}

\begin{itemize}
\item
\item
\item
\item
\item
\end{itemize}

\href{https://www.nytimes.com/news-event/coronavirus?action=click\&pgtype=Article\&state=default\&region=TOP_BANNER\&context=storylines_menu}{The
Coronavirus Outbreak}

\begin{itemize}
\tightlist
\item
  live\href{https://www.nytimes.com/2020/08/03/world/coronavirus-covid-19.html?action=click\&pgtype=Article\&state=default\&region=TOP_BANNER\&context=storylines_menu}{Latest
  Updates}
\item
  \href{https://www.nytimes.com/interactive/2020/us/coronavirus-us-cases.html?action=click\&pgtype=Article\&state=default\&region=TOP_BANNER\&context=storylines_menu}{Maps
  and Cases}
\item
  \href{https://www.nytimes.com/interactive/2020/science/coronavirus-vaccine-tracker.html?action=click\&pgtype=Article\&state=default\&region=TOP_BANNER\&context=storylines_menu}{Vaccine
  Tracker}
\item
  \href{https://www.nytimes.com/2020/08/02/us/covid-college-reopening.html?action=click\&pgtype=Article\&state=default\&region=TOP_BANNER\&context=storylines_menu}{College
  Reopening}
\item
  \href{https://www.nytimes.com/live/2020/08/03/business/stock-market-today-coronavirus?action=click\&pgtype=Article\&state=default\&region=TOP_BANNER\&context=storylines_menu}{Economy}
\end{itemize}

Advertisement

\protect\hyperlink{after-top}{Continue reading the main story}

Supported by

\protect\hyperlink{after-sponsor}{Continue reading the main story}

Times INsider

\hypertarget{faces-that-cant-be-forgotten}{%
\section{Faces That Can't Be
Forgotten}\label{faces-that-cant-be-forgotten}}

How the Obituaries desk is memorializing victims of the coronavirus
pandemic.

\includegraphics{https://static01.nyt.com/images/2020/04/17/pageoneplus/17insider-web/17insider-web-articleLarge-v2.jpg?quality=75\&auto=webp\&disable=upscale}

\href{https://www.nytimes.com/by/daniel-j-wakin}{\includegraphics{https://static01.nyt.com/images/2016/08/03/admin/dan-wakin-headshot/dan-wakin-headshot-thumbLarge.jpg}}

By \href{https://www.nytimes.com/by/daniel-j-wakin}{Daniel J. Wakin}

\begin{itemize}
\item
  April 16, 2020
\item
  \begin{itemize}
  \item
  \item
  \item
  \item
  \item
  \end{itemize}
\end{itemize}

\href{https://www.nytimes.com/series/times-insider}{\emph{Times
Insider}} \emph{explains who we are and what we do, and delivers
behind-the-scenes insights into how our journalism comes together.}

Colleges were sending students packing. Wall Street was on its way to
its worst decline since Black Monday in 1987. Broadway went dark,
northern Italian hospitals fell into critical condition, and the World
Health Organization declared the spread of the coronavirus a pandemic.

The prospect of death on a global scale was growing real.

That day, March 12, Sam Roberts, a veteran Times reporter and editor,
emailed his editors on the Obituaries desk --- me among them: Should we
be thinking about a Portraits of Grief for victims of the virus?

I knew exactly what he had in mind.

\href{https://www.poynter.org/reporting-editing/2011/portraits-of-grief-10-years-later-lessons-from-the-original-new-york-times-911-coverage/}{Portraits
of Grief}were short written sketches in The Times, accompanied by
photographs, of the victims of the Sept. 11, 2001, terror attacks.
Beginning as a series that ran for months, the profiles eventually
numbered some 2,400 and were published in a book. It was a powerful
experience for all who were involved --- the editors and writers, the
readers and especially the family members of the victims. Like many
other Times journalists of a certain generation, I had contributed to
the series.

But this time there was no finite number of the dead. No geographical
point united them. Their backgrounds were of infinite variety. They did
not die all at once on a bright blue morning.

\hypertarget{latest-updates-global-coronavirus-outbreak}{%
\section{\texorpdfstring{\href{https://www.nytimes.com/2020/08/03/world/coronavirus-covid-19.html?action=click\&pgtype=Article\&state=default\&region=MAIN_CONTENT_1\&context=storylines_live_updates}{Latest
Updates: Global Coronavirus
Outbreak}}{Latest Updates: Global Coronavirus Outbreak}}\label{latest-updates-global-coronavirus-outbreak}}

Updated 2020-08-04T05:55:16.339Z

\begin{itemize}
\tightlist
\item
  \href{https://www.nytimes.com/2020/08/03/world/coronavirus-covid-19.html?action=click\&pgtype=Article\&state=default\&region=MAIN_CONTENT_1\&context=storylines_live_updates\#link-4547638f}{Fauci
  defends Birx after she is criticized by Trump.}
\item
  \href{https://www.nytimes.com/2020/08/03/world/coronavirus-covid-19.html?action=click\&pgtype=Article\&state=default\&region=MAIN_CONTENT_1\&context=storylines_live_updates\#link-15e7f995}{Trump
  derides Democrats as lawmakers and administration officials try to
  break stimulus impasse.}
\item
  \href{https://www.nytimes.com/2020/08/03/world/coronavirus-covid-19.html?action=click\&pgtype=Article\&state=default\&region=MAIN_CONTENT_1\&context=storylines_live_updates\#link-e5a2cda}{The
  deadline for 2020 census counting has been moved up by a month.}
\end{itemize}

\href{https://www.nytimes.com/2020/08/03/world/coronavirus-covid-19.html?action=click\&pgtype=Article\&state=default\&region=MAIN_CONTENT_1\&context=storylines_live_updates}{See
more updates}

More live coverage:
\href{https://www.nytimes.com/live/2020/08/03/business/stock-market-today-coronavirus?action=click\&pgtype=Article\&state=default\&region=MAIN_CONTENT_1\&context=storylines_live_updates}{Markets}

The same day that Mr. Roberts made his suggestion, I received an email
from my friend Donald G. McNeil Jr., the Times's global health writer
and one of the lead reporters on the pandemic. You should prepare for
waves and waves of obituaries, he told me.

After some discussion among the editors in Obits, we agreed that it was
time for another generation's Portraits of Grief. The executive editor,
Dean Baquet, immediately lent his support.

So began a frantic week of planning. The obituaries editor, Bill
McDonald, sent an all-points bulletin soliciting contributions from The
Times's bureaus across the country and around the world. He asked me to
coordinate the project. Amy Padnani, the leader of the department's
signature
\href{https://www.nytimes.com/interactive/2018/obituaries/overlooked.html}{Overlooked}
series of unsung figures from the past, helped guide the project's
online presentation. Both Peter Keepnews, the deputy editor, and Mr.
McDonald offered counsel and editing help while we all did our day jobs.
Julia Carmel, a news assistant, kept us organized. Our Interactive news
team came up with an elegant vehicle to display the
\href{https://www.nytimes.com/interactive/2020/obituaries/people-died-coronavirus-obituaries.html}{profiles
online}.

We came up with a name: Those We've Lost. The purpose was to convey the
human toll of Covid-19 by putting faces and names to the growing numbers
of the dead, and to portray them in all of their variety.

By March 27, we had our first group of obituaries, most written by Mr.
Roberts:
\href{https://www.nytimes.com/2020/03/27/obituaries/jenny-polanco-dead-coronavirus.html}{Jenny
Polanco}, a fashion designer;
\href{https://www.nytimes.com/2020/03/27/obituaries/dave-edwards-dead-coronavirus.html}{Dave
Edwards}, a college basketball star;
\href{https://www.nytimes.com/2020/03/27/obituaries/dez-ann-roman-dead-coronavirus.html}{Dez-Ann
Romain}, a New York City school principal; and one of our own,
\href{https://www.nytimes.com/2020/03/27/obituaries/alan-finder-dead-coronavirus.html}{Alan
Finder}, a retired Times reporter. Frances Robles, a reporter based in
Florida, contributed a piece about
\href{https://www.nytimes.com/2020/03/27/obituaries/arnold-obey-dead-coronavirus.html}{Arnold
Obey}, another school principal with a remarkable streak of marathons,
and Joseph Berger, a former colleague, wrote about a mohel and savior of
Jews during World War II,
\href{https://www.nytimes.com/2020/03/27/obituaries/romi-cohn-dead-coronavirus.html}{Romi
Cohn}.

\href{https://www.nytimes.com/news-event/coronavirus?action=click\&pgtype=Article\&state=default\&region=MAIN_CONTENT_3\&context=storylines_faq}{}

\hypertarget{the-coronavirus-outbreak-}{%
\subsubsection{The Coronavirus Outbreak
›}\label{the-coronavirus-outbreak-}}

\hypertarget{frequently-asked-questions}{%
\paragraph{Frequently Asked
Questions}\label{frequently-asked-questions}}

Updated August 3, 2020

\begin{itemize}
\item ~
  \hypertarget{im-a-small-business-owner-can-i-get-relief}{%
  \paragraph{I'm a small-business owner. Can I get
  relief?}\label{im-a-small-business-owner-can-i-get-relief}}

  \begin{itemize}
  \tightlist
  \item
    The
    \href{https://www.nytimes.com/article/small-business-loans-stimulus-grants-freelancers-coronavirus.html?action=click\&pgtype=Article\&state=default\&region=MAIN_CONTENT_3\&context=storylines_faq}{stimulus
    bills enacted in March} offer help for the millions of American
    small businesses. Those eligible for aid are businesses and
    nonprofit organizations with fewer than 500 workers, including sole
    proprietorships, independent contractors and freelancers. Some
    larger companies in some industries are also eligible. The help
    being offered, which is being managed by the Small Business
    Administration, includes the Paycheck Protection Program and the
    Economic Injury Disaster Loan program. But lots of folks have
    \href{https://www.nytimes.com/interactive/2020/05/07/business/small-business-loans-coronavirus.html?action=click\&pgtype=Article\&state=default\&region=MAIN_CONTENT_3\&context=storylines_faq}{not
    yet seen payouts.} Even those who have received help are confused:
    The rules are draconian, and some are stuck sitting on
    \href{https://www.nytimes.com/2020/05/02/business/economy/loans-coronavirus-small-business.html?action=click\&pgtype=Article\&state=default\&region=MAIN_CONTENT_3\&context=storylines_faq}{money
    they don't know how to use.} Many small-business owners are getting
    less than they expected or
    \href{https://www.nytimes.com/2020/06/10/business/Small-business-loans-ppp.html?action=click\&pgtype=Article\&state=default\&region=MAIN_CONTENT_3\&context=storylines_faq}{not
    hearing anything at all.}
  \end{itemize}
\item ~
  \hypertarget{what-are-my-rights-if-i-am-worried-about-going-back-to-work}{%
  \paragraph{What are my rights if I am worried about going back to
  work?}\label{what-are-my-rights-if-i-am-worried-about-going-back-to-work}}

  \begin{itemize}
  \tightlist
  \item
    Employers have to provide
    \href{https://www.osha.gov/SLTC/covid-19/standards.html}{a safe
    workplace} with policies that protect everyone equally.
    \href{https://www.nytimes.com/article/coronavirus-money-unemployment.html?action=click\&pgtype=Article\&state=default\&region=MAIN_CONTENT_3\&context=storylines_faq}{And
    if one of your co-workers tests positive for the coronavirus, the
    C.D.C.} has said that
    \href{https://www.cdc.gov/coronavirus/2019-ncov/community/guidance-business-response.html}{employers
    should tell their employees} -\/- without giving you the sick
    employee's name -\/- that they may have been exposed to the virus.
  \end{itemize}
\item ~
  \hypertarget{should-i-refinance-my-mortgage}{%
  \paragraph{Should I refinance my
  mortgage?}\label{should-i-refinance-my-mortgage}}

  \begin{itemize}
  \tightlist
  \item
    \href{https://www.nytimes.com/article/coronavirus-money-unemployment.html?action=click\&pgtype=Article\&state=default\&region=MAIN_CONTENT_3\&context=storylines_faq}{It
    could be a good idea,} because mortgage rates have
    \href{https://www.nytimes.com/2020/07/16/business/mortgage-rates-below-3-percent.html?action=click\&pgtype=Article\&state=default\&region=MAIN_CONTENT_3\&context=storylines_faq}{never
    been lower.} Refinancing requests have pushed mortgage applications
    to some of the highest levels since 2008, so be prepared to get in
    line. But defaults are also up, so if you're thinking about buying a
    home, be aware that some lenders have tightened their standards.
  \end{itemize}
\item ~
  \hypertarget{what-is-school-going-to-look-like-in-september}{%
  \paragraph{What is school going to look like in
  September?}\label{what-is-school-going-to-look-like-in-september}}

  \begin{itemize}
  \tightlist
  \item
    It is unlikely that many schools will return to a normal schedule
    this fall, requiring the grind of
    \href{https://www.nytimes.com/2020/06/05/us/coronavirus-education-lost-learning.html?action=click\&pgtype=Article\&state=default\&region=MAIN_CONTENT_3\&context=storylines_faq}{online
    learning},
    \href{https://www.nytimes.com/2020/05/29/us/coronavirus-child-care-centers.html?action=click\&pgtype=Article\&state=default\&region=MAIN_CONTENT_3\&context=storylines_faq}{makeshift
    child care} and
    \href{https://www.nytimes.com/2020/06/03/business/economy/coronavirus-working-women.html?action=click\&pgtype=Article\&state=default\&region=MAIN_CONTENT_3\&context=storylines_faq}{stunted
    workdays} to continue. California's two largest public school
    districts --- Los Angeles and San Diego --- said on July 13, that
    \href{https://www.nytimes.com/2020/07/13/us/lausd-san-diego-school-reopening.html?action=click\&pgtype=Article\&state=default\&region=MAIN_CONTENT_3\&context=storylines_faq}{instruction
    will be remote-only in the fall}, citing concerns that surging
    coronavirus infections in their areas pose too dire a risk for
    students and teachers. Together, the two districts enroll some
    825,000 students. They are the largest in the country so far to
    abandon plans for even a partial physical return to classrooms when
    they reopen in August. For other districts, the solution won't be an
    all-or-nothing approach.
    \href{https://bioethics.jhu.edu/research-and-outreach/projects/eschool-initiative/school-policy-tracker/}{Many
    systems}, including the nation's largest, New York City, are
    devising
    \href{https://www.nytimes.com/2020/06/26/us/coronavirus-schools-reopen-fall.html?action=click\&pgtype=Article\&state=default\&region=MAIN_CONTENT_3\&context=storylines_faq}{hybrid
    plans} that involve spending some days in classrooms and other days
    online. There's no national policy on this yet, so check with your
    municipal school system regularly to see what is happening in your
    community.
  \end{itemize}
\item ~
  \hypertarget{is-the-coronavirus-airborne}{%
  \paragraph{Is the coronavirus
  airborne?}\label{is-the-coronavirus-airborne}}

  \begin{itemize}
  \tightlist
  \item
    The coronavirus
    \href{https://www.nytimes.com/2020/07/04/health/239-experts-with-one-big-claim-the-coronavirus-is-airborne.html?action=click\&pgtype=Article\&state=default\&region=MAIN_CONTENT_3\&context=storylines_faq}{can
    stay aloft for hours in tiny droplets in stagnant air}, infecting
    people as they inhale, mounting scientific evidence suggests. This
    risk is highest in crowded indoor spaces with poor ventilation, and
    may help explain super-spreading events reported in meatpacking
    plants, churches and restaurants.
    \href{https://www.nytimes.com/2020/07/06/health/coronavirus-airborne-aerosols.html?action=click\&pgtype=Article\&state=default\&region=MAIN_CONTENT_3\&context=storylines_faq}{It's
    unclear how often the virus is spread} via these tiny droplets, or
    aerosols, compared with larger droplets that are expelled when a
    sick person coughs or sneezes, or transmitted through contact with
    contaminated surfaces, said Linsey Marr, an aerosol expert at
    Virginia Tech. Aerosols are released even when a person without
    symptoms exhales, talks or sings, according to Dr. Marr and more
    than 200 other experts, who
    \href{https://academic.oup.com/cid/article/doi/10.1093/cid/ciaa939/5867798}{have
    outlined the evidence in an open letter to the World Health
    Organization}.
  \end{itemize}
\end{itemize}

But then Mr. McNeil's prediction came true. Major figures in the news
began dying of the novel coronavirus as well, including the Tony-winning
playwright
\href{https://www.nytimes.com/2020/03/24/theater/terrence-mcnally-dead-coronavirus.html}{Terrence
McNally}. They joined our accounting of the dead, rubbing shoulders with
everyone else. We also asked for help from readers with a
\href{https://www.nytimes.com/2020/04/01/obituaries/coronavirus-victims.html}{callout}
for candidates.

Times writers from other departments, including Styles, Culture and
Climate, joined in. Obituaries arrived from correspondents in London,
Tokyo, Rome and Nairobi. We started with a page of ``Lost'' obituaries
in print, appearing at first on Saturdays. But it soon became clear that
we'd need at least one more page of them a week, acknowledging that even
then they'd only scratch the surface of the mounting death toll.
Reinforcements began arriving from other departments.

Those who remember the Portraits of Grief, which emerged from the Metro
desk, will see something quite different here. Rather than vignettes,
these are more rounded obituaries, much as we write for heads of state,
Hall of Famers and trailblazers.

Most obituary writers and editors will agree that standard professional
detachment is the first line of defense when death is your beat.
Embracing the fascinating histories of the people we write about; using
graceful prose to capture them as best we can for the ages; and
employing journalistic chops to dig deep and uncover stories also help
keep morbidness at bay.

It's tougher when the story is all around you.

\begin{center}\rule{0.5\linewidth}{\linethickness}\end{center}

Follow the \href{https://twitter.com/readercenter}{@ReaderCenter} on
Twitter for more coverage highlighting your perspectives and experiences
and for insight into how we work.

Advertisement

\protect\hyperlink{after-bottom}{Continue reading the main story}

\hypertarget{site-index}{%
\subsection{Site Index}\label{site-index}}

\hypertarget{site-information-navigation}{%
\subsection{Site Information
Navigation}\label{site-information-navigation}}

\begin{itemize}
\tightlist
\item
  \href{https://help.nytimes.com/hc/en-us/articles/115014792127-Copyright-notice}{©~2020~The
  New York Times Company}
\end{itemize}

\begin{itemize}
\tightlist
\item
  \href{https://www.nytco.com/}{NYTCo}
\item
  \href{https://help.nytimes.com/hc/en-us/articles/115015385887-Contact-Us}{Contact
  Us}
\item
  \href{https://www.nytco.com/careers/}{Work with us}
\item
  \href{https://nytmediakit.com/}{Advertise}
\item
  \href{http://www.tbrandstudio.com/}{T Brand Studio}
\item
  \href{https://www.nytimes.com/privacy/cookie-policy\#how-do-i-manage-trackers}{Your
  Ad Choices}
\item
  \href{https://www.nytimes.com/privacy}{Privacy}
\item
  \href{https://help.nytimes.com/hc/en-us/articles/115014893428-Terms-of-service}{Terms
  of Service}
\item
  \href{https://help.nytimes.com/hc/en-us/articles/115014893968-Terms-of-sale}{Terms
  of Sale}
\item
  \href{https://spiderbites.nytimes.com}{Site Map}
\item
  \href{https://help.nytimes.com/hc/en-us}{Help}
\item
  \href{https://www.nytimes.com/subscription?campaignId=37WXW}{Subscriptions}
\end{itemize}
