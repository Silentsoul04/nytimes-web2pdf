Sections

SEARCH

\protect\hyperlink{site-content}{Skip to
content}\protect\hyperlink{site-index}{Skip to site index}

\href{https://www.nytimes.com/section/us}{U.S.}

\href{https://myaccount.nytimes.com/auth/login?response_type=cookie\&client_id=vi}{}

\href{https://www.nytimes.com/section/todayspaper}{Today's Paper}

\href{/section/us}{U.S.}\textbar{}After Virus Delays, Census Must
Scramble to Avoid Undercount

\url{https://nyti.ms/2KhjlKM}

\begin{itemize}
\item
\item
\item
\item
\item
\end{itemize}

\href{https://www.nytimes.com/news-event/coronavirus?action=click\&pgtype=Article\&state=default\&region=TOP_BANNER\&context=storylines_menu}{The
Coronavirus Outbreak}

\begin{itemize}
\tightlist
\item
  live\href{https://www.nytimes.com/2020/08/04/world/coronavirus-cases.html?action=click\&pgtype=Article\&state=default\&region=TOP_BANNER\&context=storylines_menu}{Latest
  Updates}
\item
  \href{https://www.nytimes.com/interactive/2020/us/coronavirus-us-cases.html?action=click\&pgtype=Article\&state=default\&region=TOP_BANNER\&context=storylines_menu}{Maps
  and Cases}
\item
  \href{https://www.nytimes.com/interactive/2020/science/coronavirus-vaccine-tracker.html?action=click\&pgtype=Article\&state=default\&region=TOP_BANNER\&context=storylines_menu}{Vaccine
  Tracker}
\item
  \href{https://www.nytimes.com/2020/08/02/us/covid-college-reopening.html?action=click\&pgtype=Article\&state=default\&region=TOP_BANNER\&context=storylines_menu}{College
  Reopening}
\item
  \href{https://www.nytimes.com/live/2020/08/04/business/stock-market-today-coronavirus?action=click\&pgtype=Article\&state=default\&region=TOP_BANNER\&context=storylines_menu}{Economy}
\end{itemize}

Advertisement

\protect\hyperlink{after-top}{Continue reading the main story}

Supported by

\protect\hyperlink{after-sponsor}{Continue reading the main story}

\hypertarget{after-virus-delays-census-must-scramble-to-avoid-undercount}{%
\section{After Virus Delays, Census Must Scramble to Avoid
Undercount}\label{after-virus-delays-census-must-scramble-to-avoid-undercount}}

Slammed by a pandemic, the Census Bureau postponed crucial portions of
the count for the third time in a month, further raising the bar for an
accurate count.

\includegraphics{https://static01.nyt.com/images/2020/04/19/us/19VIRUS-CENSUS/merlin_170821071_9242b790-70fd-4396-8b4d-04409de8f2a8-articleLarge.jpg?quality=75\&auto=webp\&disable=upscale}

By \href{https://www.nytimes.com/by/michael-wines}{Michael Wines}

\begin{itemize}
\item
  Published April 18, 2020Updated July 28, 2020
\item
  \begin{itemize}
  \item
  \item
  \item
  \item
  \item
  \end{itemize}
\end{itemize}

WASHINGTON --- Barely a month into the
\href{https://www.nytimes.com/2020/07/28/us/trump-census.html}{2020
census}, the 1,600 residents of Tract 1847 in Royal Oak, Mich., a
comfortable Detroit suburb, are taking the rest of the country to
school. Nationally, half of all households have filled out their census
forms. Royal Oak is up to seven in 10, virtually all of them filing
online.

Twenty miles due south, in Tract 5246 in the faded industrial community
of Delray, the return rate is barely one in five, perhaps because half
the households have no internet service. The Census Bureau says young
children there are at a very high risk of not being counted at all.

Places like Delray were always going to be hard targets for this census.
But this week they have become much harder: Slammed by a pandemic, the
Census Bureau postponed crucial portions of the count for the third time
in a month, asking Congress for permission to push final population
totals and even reapportionment of Congress far into 2021.

The unprecedented delay buys time for census strategists to try to
figure out how a head count built around engaging the public --- through
advertising, crowd-drawing events and knocking on millions of doors ---
can succeed in a nation locked down by the coronavirus pandemic.

The answer could be beyond their control.

``The truth is that the only thing in charge of the census right now is
the virus,'' said Kenneth Prewitt, a Columbia University professor who
directed the Census Bureau during the 2000 count. ``Not the bureau, not
the president. And the virus will be in charge until it isn't.''

Mr. Prewitt, among others, said he had faith that the bureau would
persevere, and its director, Steve Dillingham,
\href{https://m.youtube.com/watch?v=CGP1Yblqasc}{went on YouTube this
weekend} to urge a full count of the country ``in unprecedented times.''

But the obstacles are enormous, and the cost of failure would be large.
Most critically, the task of counting those who were already hardest to
count --- chiefly minorities, the poor, children and those who were born
elsewhere --- keeps getting harder.

Those groups are regularly undercounted in censuses, but the size of the
shortfall has shrunk over the decades. Experts have long feared that the
2020 census, playing out amid the Trump administration's crackdown on
undocumented immigrants and a bitter battle over whether the census
should ask whether respondents are citizens, would reverse that trend.

The pandemic erupted precisely as a national campaign to allay those
fears and boost response had begun to gain steam. Many experts
anticipate that it will only cause the undercounts to balloon even
further.

\hypertarget{latest-updates-global-coronavirus-outbreak}{%
\section{\texorpdfstring{\href{https://www.nytimes.com/2020/08/04/world/coronavirus-cases.html?action=click\&pgtype=Article\&state=default\&region=MAIN_CONTENT_1\&context=storylines_live_updates}{Latest
Updates: Global Coronavirus
Outbreak}}{Latest Updates: Global Coronavirus Outbreak}}\label{latest-updates-global-coronavirus-outbreak}}

Updated 2020-08-04T19:15:37.345Z

\begin{itemize}
\tightlist
\item
  \href{https://www.nytimes.com/2020/08/04/world/coronavirus-cases.html?action=click\&pgtype=Article\&state=default\&region=MAIN_CONTENT_1\&context=storylines_live_updates\#link-4825b93}{Public
  and private schools in Maryland and elsewhere are divided over
  in-person instruction.}
\item
  \href{https://www.nytimes.com/2020/08/04/world/coronavirus-cases.html?action=click\&pgtype=Article\&state=default\&region=MAIN_CONTENT_1\&context=storylines_live_updates\#link-4d1eafa8}{N.Y.C.'s
  health commissioner resigns after clashing with the mayor over the
  virus.}
\item
  \href{https://www.nytimes.com/2020/08/04/world/coronavirus-cases.html?action=click\&pgtype=Article\&state=default\&region=MAIN_CONTENT_1\&context=storylines_live_updates\#link-6b644638}{`Long
  days, long nights': Washington prepares for a prolonged fight over
  virus relief.}
\end{itemize}

\href{https://www.nytimes.com/2020/08/04/world/coronavirus-cases.html?action=click\&pgtype=Article\&state=default\&region=MAIN_CONTENT_1\&context=storylines_live_updates}{See
more updates}

More live coverage:
\href{https://www.nytimes.com/live/2020/08/04/business/stock-market-today-coronavirus?action=click\&pgtype=Article\&state=default\&region=MAIN_CONTENT_1\&context=storylines_live_updates}{Markets}

``In 2010 there was almost a 5 percent undercount of children'' ---
\href{https://www.census.gov/content/dam/Census/library/working-papers/2014/demo/2014-undercount-children.pdf}{nearly
one million youngsters} --- ``aged 0 to 5,'' said Robert L. Santos, the
vice president of the Urban Institute and president-elect of the
American Statistical Association. ``This time, even with the extra
months the Census Bureau has built in, that type of risk will remain ---
except in supercharged form.''

The good news is that the public response to date has been encouraging.
The half of known households that have already submitted census forms is
more than the bureau's analysts had expected by now, and filing by the
internet --- a new option that some feared would be a risky gamble ---
has so far proved a welcome success. (Those who have not yet responded
\href{https://2020census.gov}{can do so here} even without mailed
instructions at hand*.*)

If that continues, the bureau expects to surpass its goal of getting
roughly six in 10 households to complete forms before it has to deploy
an army of door-knockers to track down the rest, the agency's spokesman,
Michael C. Cook, said on Thursday. And that is vital: The more
households that send in forms on their own, the less it will cost to
find those who do not respond.

``This is the critical period, the period where the only way to answer
is on your own,'' said Joseph J. Salvo, the chief demographer for New
York City. ``If we can get those self-response rates up, we'll be OK.
The problem is getting them up with one hand tied behind your back.''

The pandemic has upended hundreds of millions of dollars of publicity
campaigns --- by state and local governments, nonprofit organizations
and advocacy groups, philanthropies and businesses --- that envisioned
using door-to-door canvasses and community events like summer festivals
to proselytize the civic benefits of the census. Many if not most were
aimed at neighborhoods where response to past censuses has been poor.

\includegraphics{https://static01.nyt.com/images/2020/04/17/us/00virus-census-wines/merlin_171194388_efe135a1-36e2-49ea-afad-5e25337ad3b7-articleLarge.jpg?quality=75\&auto=webp\&disable=upscale}

The number of households that respond to the census by telephone is
behind expectations because social-distancing requirements cut into the
bureau's phone banks. Responses by mail, the standard method in decades
past, are muddled because social distancing has reduced the staff that
processes forms. The bureau had planned to send a new batch of forms by
now to every household that had not responded to earlier mailings, but
virus-caused slowdowns at the bureau and the Postal Service have delayed
that until the month's end.

Places hardest hit by the pandemic also are taking a census hit, Mr.
Cook said. And while the bureau is targeting those hot spots and other
low-response areas with extra encouragements to fill out census forms,
he said, it also must avoid competing with a public-health message that
has to take precedence.

Mr. Cook said the bureau had not parsed its responses to determine in
detail who was and was not submitting forms. But the trends are telling:
Households that can afford internet service are exceeding expectations;
phone and mail responses, likely to be used by the less well-off, are
stable or down.

\href{https://www.nytimes.com/news-event/coronavirus?action=click\&pgtype=Article\&state=default\&region=MAIN_CONTENT_3\&context=storylines_faq}{}

\hypertarget{the-coronavirus-outbreak-}{%
\subsubsection{The Coronavirus Outbreak
›}\label{the-coronavirus-outbreak-}}

\hypertarget{frequently-asked-questions}{%
\paragraph{Frequently Asked
Questions}\label{frequently-asked-questions}}

Updated August 4, 2020

\begin{itemize}
\item ~
  \hypertarget{i-have-antibodies-am-i-now-immune}{%
  \paragraph{I have antibodies. Am I now
  immune?}\label{i-have-antibodies-am-i-now-immune}}

  \begin{itemize}
  \tightlist
  \item
    As of right
    now,\href{https://www.nytimes.com/2020/07/22/health/covid-antibodies-herd-immunity.html?action=click\&pgtype=Article\&state=default\&region=MAIN_CONTENT_3\&context=storylines_faq}{that
    seems likely, for at least several months.} There have been
    frightening accounts of people suffering what seems to be a second
    bout of Covid-19. But experts say these patients may have a
    drawn-out course of infection, with the virus taking a slow toll
    weeks to months after initial exposure. People infected with the
    coronavirus typically
    \href{https://www.nature.com/articles/s41586-020-2456-9}{produce}
    immune molecules called antibodies, which are
    \href{https://www.nytimes.com/2020/05/07/health/coronavirus-antibody-prevalence.html?action=click\&pgtype=Article\&state=default\&region=MAIN_CONTENT_3\&context=storylines_faq}{protective
    proteins made in response to an
    infection}\href{https://www.nytimes.com/2020/05/07/health/coronavirus-antibody-prevalence.html?action=click\&pgtype=Article\&state=default\&region=MAIN_CONTENT_3\&context=storylines_faq}{.
    These antibodies may} last in the body
    \href{https://www.nature.com/articles/s41591-020-0965-6}{only two to
    three months}, which may seem worrisome, but that's perfectly normal
    after an acute infection subsides, said Dr. Michael Mina, an
    immunologist at Harvard University. It may be possible to get the
    coronavirus again, but it's highly unlikely that it would be
    possible in a short window of time from initial infection or make
    people sicker the second time.
  \end{itemize}
\item ~
  \hypertarget{im-a-small-business-owner-can-i-get-relief}{%
  \paragraph{I'm a small-business owner. Can I get
  relief?}\label{im-a-small-business-owner-can-i-get-relief}}

  \begin{itemize}
  \tightlist
  \item
    The
    \href{https://www.nytimes.com/article/small-business-loans-stimulus-grants-freelancers-coronavirus.html?action=click\&pgtype=Article\&state=default\&region=MAIN_CONTENT_3\&context=storylines_faq}{stimulus
    bills enacted in March} offer help for the millions of American
    small businesses. Those eligible for aid are businesses and
    nonprofit organizations with fewer than 500 workers, including sole
    proprietorships, independent contractors and freelancers. Some
    larger companies in some industries are also eligible. The help
    being offered, which is being managed by the Small Business
    Administration, includes the Paycheck Protection Program and the
    Economic Injury Disaster Loan program. But lots of folks have
    \href{https://www.nytimes.com/interactive/2020/05/07/business/small-business-loans-coronavirus.html?action=click\&pgtype=Article\&state=default\&region=MAIN_CONTENT_3\&context=storylines_faq}{not
    yet seen payouts.} Even those who have received help are confused:
    The rules are draconian, and some are stuck sitting on
    \href{https://www.nytimes.com/2020/05/02/business/economy/loans-coronavirus-small-business.html?action=click\&pgtype=Article\&state=default\&region=MAIN_CONTENT_3\&context=storylines_faq}{money
    they don't know how to use.} Many small-business owners are getting
    less than they expected or
    \href{https://www.nytimes.com/2020/06/10/business/Small-business-loans-ppp.html?action=click\&pgtype=Article\&state=default\&region=MAIN_CONTENT_3\&context=storylines_faq}{not
    hearing anything at all.}
  \end{itemize}
\item ~
  \hypertarget{what-are-my-rights-if-i-am-worried-about-going-back-to-work}{%
  \paragraph{What are my rights if I am worried about going back to
  work?}\label{what-are-my-rights-if-i-am-worried-about-going-back-to-work}}

  \begin{itemize}
  \tightlist
  \item
    Employers have to provide
    \href{https://www.osha.gov/SLTC/covid-19/standards.html}{a safe
    workplace} with policies that protect everyone equally.
    \href{https://www.nytimes.com/article/coronavirus-money-unemployment.html?action=click\&pgtype=Article\&state=default\&region=MAIN_CONTENT_3\&context=storylines_faq}{And
    if one of your co-workers tests positive for the coronavirus, the
    C.D.C.} has said that
    \href{https://www.cdc.gov/coronavirus/2019-ncov/community/guidance-business-response.html}{employers
    should tell their employees} -\/- without giving you the sick
    employee's name -\/- that they may have been exposed to the virus.
  \end{itemize}
\item ~
  \hypertarget{should-i-refinance-my-mortgage}{%
  \paragraph{Should I refinance my
  mortgage?}\label{should-i-refinance-my-mortgage}}

  \begin{itemize}
  \tightlist
  \item
    \href{https://www.nytimes.com/article/coronavirus-money-unemployment.html?action=click\&pgtype=Article\&state=default\&region=MAIN_CONTENT_3\&context=storylines_faq}{It
    could be a good idea,} because mortgage rates have
    \href{https://www.nytimes.com/2020/07/16/business/mortgage-rates-below-3-percent.html?action=click\&pgtype=Article\&state=default\&region=MAIN_CONTENT_3\&context=storylines_faq}{never
    been lower.} Refinancing requests have pushed mortgage applications
    to some of the highest levels since 2008, so be prepared to get in
    line. But defaults are also up, so if you're thinking about buying a
    home, be aware that some lenders have tightened their standards.
  \end{itemize}
\item ~
  \hypertarget{what-is-school-going-to-look-like-in-september}{%
  \paragraph{What is school going to look like in
  September?}\label{what-is-school-going-to-look-like-in-september}}

  \begin{itemize}
  \tightlist
  \item
    It is unlikely that many schools will return to a normal schedule
    this fall, requiring the grind of
    \href{https://www.nytimes.com/2020/06/05/us/coronavirus-education-lost-learning.html?action=click\&pgtype=Article\&state=default\&region=MAIN_CONTENT_3\&context=storylines_faq}{online
    learning},
    \href{https://www.nytimes.com/2020/05/29/us/coronavirus-child-care-centers.html?action=click\&pgtype=Article\&state=default\&region=MAIN_CONTENT_3\&context=storylines_faq}{makeshift
    child care} and
    \href{https://www.nytimes.com/2020/06/03/business/economy/coronavirus-working-women.html?action=click\&pgtype=Article\&state=default\&region=MAIN_CONTENT_3\&context=storylines_faq}{stunted
    workdays} to continue. California's two largest public school
    districts --- Los Angeles and San Diego --- said on July 13, that
    \href{https://www.nytimes.com/2020/07/13/us/lausd-san-diego-school-reopening.html?action=click\&pgtype=Article\&state=default\&region=MAIN_CONTENT_3\&context=storylines_faq}{instruction
    will be remote-only in the fall}, citing concerns that surging
    coronavirus infections in their areas pose too dire a risk for
    students and teachers. Together, the two districts enroll some
    825,000 students. They are the largest in the country so far to
    abandon plans for even a partial physical return to classrooms when
    they reopen in August. For other districts, the solution won't be an
    all-or-nothing approach.
    \href{https://bioethics.jhu.edu/research-and-outreach/projects/eschool-initiative/school-policy-tracker/}{Many
    systems}, including the nation's largest, New York City, are
    devising
    \href{https://www.nytimes.com/2020/06/26/us/coronavirus-schools-reopen-fall.html?action=click\&pgtype=Article\&state=default\&region=MAIN_CONTENT_3\&context=storylines_faq}{hybrid
    plans} that involve spending some days in classrooms and other days
    online. There's no national policy on this yet, so check with your
    municipal school system regularly to see what is happening in your
    community.
  \end{itemize}
\end{itemize}

That has some experts concerned. In Texas, a fifth of the state's 254
counties --- heavily Latino areas along the border with Mexico, sparsely
populated poor areas in the state's interior --- are responding to the
census at rates half that of the state average, which is itself among
the lowest in the country. In part, that is because a Census Bureau
effort to deliver forms to about five million households without postal
addresses, most of them poor, was barely begun before being halted by
the virus.

``The response rates have really dropped since census day'' on April 1,
said Lila Valencia, senior demographer and the point person on census
issues at the Texas Demographic Center in Austin. ``We need to get
self-response kick-started, right now.''

Amid that, the bureau has
\href{https://2020census.gov/content/dam/2020census/materials/news/2020-census\%20operational-adjustments-long\%20version.pdf}{begun
to lay plans} for restarting parts of the count that have been stopped
in their tracks. National field offices, shuttered in March, are slated
to start reopening on June 1. Deliveries of forms and instructions to
those five million households lacking addresses are now set to resume in
mid-June, three months late. In-person visits by census takers to group
quarters like nursing homes, once scheduled to begin at the start of
this month, have been rescheduled to start July 1.

Some other aspects of the census --- a count of roughly a half-million
homeless people, pop-up centers that help people complete the census at
places like grocery stores, the count of all 1.7 million households in
Puerto Rico --- remain on hold until the bureau can figure out how to
conduct them safely.

Most important, the strategists are betting that the virus's grip will
weaken enough by mid-August to safely deploy hundreds of thousands of
temporary field workers to track down the millions who still have not
sent in forms. Without the success of that exercise --- known in
census-speak by the acronym NRFU (``ner-foo''), for nonresponse
follow-up --- the census will be compromised.

Experts say that effort, which is set to run through October, is likely
to be the diciest aspect of the entire reboot. The census is supposed to
be a snapshot of the nation at the beginning of April; the door-knocking
was originally supposed to begin in May. But by autumn, the national
mosaic will have reshuffled.

``The farther you get from April 1, the less accurate the data is,''
said Jeri Green, a veteran Census Bureau employee who now is the senior
adviser on the census for the National Urban League. ``Imagine in
October that a household gets a knock on the door and someone in a mask
asks who lived there on April 1. In some communities people may be one
stimulus check from getting off someone's couch. Weddings are coming up.
People are going to move out of their parents' homes.''

There are other concerns as well: The rescheduled follow-up would take
place at the peak of summer heat in the Southwest and hurricane season
in the South. A prolonged economic collapse could trigger huge
migrations of job-seekers. And, of course, the pandemic may not abate
enough to allow hordes of door-knockers to trek through neighborhoods,
much less persuade residents to open their doors.

Mr. Cook, of the Census Bureau, said the agency was ready to change
plans again if the need arises. Veterans of past censuses say doomsday
scenarios are most likely just that. The pandemic could also fade. Local
campaigns to drive up response could resume. And there are ways to
discern who lives in households that do not respond --- data from
neighbors, government records and the makeup of nearby households ---
that would allow the bureau to at least make an educated guess.

In the most nervous corners, there are whispers in hushed tones of a
failed census and what to do about a count that everyone knows is off.
But experts and officials say there are tools like reliance on existing
data that could make this count sufficient, if not perfect --- even in
the pandemic.

Barbara Anderson, a professor at the University of Michigan Population
Studies Center and a past chair of the bureau's Census Scientific
Advisory Committee, said that the bureau's decision to delay some of the
count could help circumvent the virus and ``make the undercount much
less than it would have been otherwise.''

She said: ``Even if visiting individual homes is impeded, there are
other things they can do to try to improve the count, and I think they
will.''

Advertisement

\protect\hyperlink{after-bottom}{Continue reading the main story}

\hypertarget{site-index}{%
\subsection{Site Index}\label{site-index}}

\hypertarget{site-information-navigation}{%
\subsection{Site Information
Navigation}\label{site-information-navigation}}

\begin{itemize}
\tightlist
\item
  \href{https://help.nytimes.com/hc/en-us/articles/115014792127-Copyright-notice}{©~2020~The
  New York Times Company}
\end{itemize}

\begin{itemize}
\tightlist
\item
  \href{https://www.nytco.com/}{NYTCo}
\item
  \href{https://help.nytimes.com/hc/en-us/articles/115015385887-Contact-Us}{Contact
  Us}
\item
  \href{https://www.nytco.com/careers/}{Work with us}
\item
  \href{https://nytmediakit.com/}{Advertise}
\item
  \href{http://www.tbrandstudio.com/}{T Brand Studio}
\item
  \href{https://www.nytimes.com/privacy/cookie-policy\#how-do-i-manage-trackers}{Your
  Ad Choices}
\item
  \href{https://www.nytimes.com/privacy}{Privacy}
\item
  \href{https://help.nytimes.com/hc/en-us/articles/115014893428-Terms-of-service}{Terms
  of Service}
\item
  \href{https://help.nytimes.com/hc/en-us/articles/115014893968-Terms-of-sale}{Terms
  of Sale}
\item
  \href{https://spiderbites.nytimes.com}{Site Map}
\item
  \href{https://help.nytimes.com/hc/en-us}{Help}
\item
  \href{https://www.nytimes.com/subscription?campaignId=37WXW}{Subscriptions}
\end{itemize}
