Sections

SEARCH

\protect\hyperlink{site-content}{Skip to
content}\protect\hyperlink{site-index}{Skip to site index}

\href{https://www.nytimes.com/section/world/europe}{Europe}

\href{https://myaccount.nytimes.com/auth/login?response_type=cookie\&client_id=vi}{}

\href{https://www.nytimes.com/section/todayspaper}{Today's Paper}

\href{/section/world/europe}{Europe}\textbar{}A Deep-Diving Sub. A
Deadly Fire. And Russia's Secret Undersea Agenda.

\url{https://nyti.ms/2Klw4Ms}

\begin{itemize}
\item
\item
\item
\item
\item
\item
\end{itemize}

Advertisement

\protect\hyperlink{after-top}{Continue reading the main story}

Supported by

\protect\hyperlink{after-sponsor}{Continue reading the main story}

The Great Read

\hypertarget{a-deep-diving-sub-a-deadly-fire-and-russias-secret-undersea-agenda}{%
\section{A Deep-Diving Sub. A Deadly Fire. And Russia's Secret Undersea
Agenda.}\label{a-deep-diving-sub-a-deadly-fire-and-russias-secret-undersea-agenda}}

Few want to talk about how 14 sailors met their deaths on a Russian
engineering marvel. Fewer still want to talk about what they were doing
off Norway's waters.

Like the shell of an egg, the Losharik's titanium spheres can resist
terrific pressure.

Like the shell of an egg, the Losharik's titanium spheres can resist
terrific pressure.

Like the shell of an egg, the Losharik's titanium spheres can resist
terrific pressure.

Like the shell of an egg, the Losharik's

titanium spheres can resist terrific pressure.

Like the shell of an egg, the Losharik's

titanium spheres can resist terrific pressure.

By Mika Gröndahl

By \href{https://www.nytimes.com/by/james-glanz}{James Glanz} and Thomas
Nilsen

\begin{itemize}
\item
  Published April 20, 2020Updated April 21, 2020
\item
  \begin{itemize}
  \item
  \item
  \item
  \item
  \item
  \item
  \end{itemize}
\end{itemize}

\href{https://www.nytimes.com/ru/2020/04/20/world/europe/russia-submarine-losharik-norway.html}{Читать
статью
по-русски}\href{https://cn.nytimes.com/world/20200421/russian-submarine-fire-losharik/}{阅读简体中文版}\href{https://cn.nytimes.com/world/20200421/russian-submarine-fire-losharik/zh-hant/}{閱讀繁體中文版}

OFF THE COAST OF NORWAY --- There could hardly have been a more
terrifying place to fight a fire than in the belly of the Losharik, a
mysterious deep-diving Russian submarine.

Something, it appears, had gone terribly wrong in the battery
compartment as the sub made its way through Russian waters 250 miles
north of the Arctic Circle on the First of July.

A fire on any submarine may be a mariner's worst nightmare, but a fire
on the Losharik was a threat of another order altogether. The vessel is
able to dive far deeper than almost any other sub, but the feats of
engineering that allow it do so may have helped seal the fate of the 14
sailors killed in the disaster.

The only thing more mysterious than what exactly went wrong that day is
what the sub was doing in a thousand feet of water just 60 nautical
miles east of Norway in the first place.

The extraordinary incident may offer yet another clue to Russia's
military ambitions in the deep sea, and how they figure into a plan to
leverage Arctic naval power to achieve its strategic goals around the
globe --- including the ability to choke off vital international
communication channels at will.

Moscow has been unforthcoming about
\href{https://www.nytimes.com/2019/07/04/world/europe/russia-nuclear-sub-fire.html}{the
Losharik disaster}, and insists that the sub was merely a research
vessel. The Norwegian military, whose observation posts, navy and
surveillance aircraft track Russia's Northern Fleet for NATO, refuses to
say what it may have seen. The only civilian witnesses to the rescue
that followed the fire may have been a ragtag band of Russians fishing
illegally in the area.

But it was clearly a mission of the highest sensitivity, and
\href{https://function.mil.ru/news_page/country/more.htm?id=12239672@egNews}{the
roster of the dead} included some of the most decorated and experienced
officers of the Russian submarine corps.

\includegraphics{https://static01.nyt.com/images/2020/04/05/world/05russia-sub-03/05russia-sub-03-articleLarge.jpg?quality=75\&auto=webp\&disable=upscale}

To understand why these men may have found themselves on a submarine
that can dive to perhaps 20,000 feet --- more than 10 times deeper than
crewed American subs are believed to operate --- consider what
crisscrosses the floor of the North Atlantic: endless miles of
fiber-optic cables that carry a large fraction of the world's internet
traffic, including trillions of dollars in financial transactions. There
are also cables linking the sonar listening devices that litter the
ocean floor.

Russia's president, Vladimir V. Putin, and his commanders have
increasingly stressed the importance of controlling the flow of
information to keep the upper hand in a conflict, said Katarzyna Zysk,
head of the Center for Security Policy at the Norwegian Institute for
Defense Studies in Oslo.

No matter where in the world a conflict might be brewing, cutting those
undersea cables, Professor Zysk said, might force an adversary to think
twice before risking an escalation of the dispute.

``The Russian understanding is that the level of unacceptable damage is
much lower in Europe and the West than during the Cold War,'' she said.
``So you might not have to do too much.''

Not just any submarine can do that --- at least, not across nearly the
entire expanse of the sea bottom.

But the Losharik is not just any submarine. Its inner hull is thought to
consist of a series of titanium spheres holding the control room, the
bunks, the nuclear reactor and other equipment. Its name, it appears,
was taken from an old Russian cartoon character, a horse assembled from
small spheres.

Sealevel

The mothership, a Delta IV class sub called Podmoskovye

Manned

U.S. submarines

1,200 feet

Losharik docked underneath

Length about 200 feet

Space between hull and spheres contains seawater and equipment such as
ballast tanks

Atlantic Ocean

average depth

about 12,000 feet

Connecting

tunnels

Control room

Crew

Nuclear reactor

Propulsion system

Batteries most likely placed

in two compartments

Retrieval arm

Retractable skids for protection

when resting on the seabed

Losharik

down to 20,000 feet

Deepest point in

the Atlantic Ocean,

Puerto Rico Trench

about 28,000 feet

The mothership

Delta IV class sub called Podmoskovye

Losharik docked underneath

Length about 200 feet

Inside the Losharik

Space between hull and spheres contains

seawater and equipment such as ballast tanks

Control room

Nuclear reactor

Retractable skids

Batteries most likely placed in two compartments

Below the sealevel

Manned U.S. submarines

1,200 feet

Atlantic Ocean average depth

12,000 feet

Losharik down to 20,000 feet

Deepest point in the Atlantic Ocean,

Puerto Rico Trench

28,000 feet

Sources: Revised edition of ``Cold War Submarines'' by Norman Polmar and
K.J. Moore; NOAA

By Mika Gröndahl

The spheres are cramped, and they are joined by even smaller
passageways.

A common procedure when there is a fire on a sub is to close the hatches
to slow its spread. If that was done on the Losharik, the crew members
may have found themselves trapped in small, dim, smoke-filled chambers.

And if they were in the chamber containing the battery compartment where
the trouble appears to have started, they may have been battling flames
raging in spaces as narrow as a couple of feet, said
\href{https://lynceans.org/author/pete/}{Peter Lobner}, a former
electrical officer on a United States submarine.

``That's the creepiest place you ever want to be on a submarine,'' Mr.
Lobner said.

Image

A funeral for the crew members of the submarine was held in St.
Petersburg.Credit...Anatoly Maltsev/EPA, via Shutterstock

\hypertarget{a-very-russian-story}{%
\subsection{`A Very Russian Story'}\label{a-very-russian-story}}

The Russian fishermen were out in a small boat, moving eastward,
probably in restricted waters, when a submarine burst from the water in
front of them, one later told a local newspaper in Murmansk, The
SeverPost.

``We were heading towards Kildin,'' a nearby island, the fisherman told
a SeverPost reporter in a phone call, ``and then, about half past nine
in the evening, a submarine surfaces. Suddenly and completely surfaces.
I have never seen anything like it in my life. On the deck, people were
running around making a fuss.''

The submarine they saw was not the Losharik but a much larger vessel:
its mothership. The Losharik is designed to fasten to its underside, so
it can be carried along for servicing, transport over long distances or
--- as may have happened on July 1 off Norway --- rescue.

Why Russia did not secure the area is unknown, but if the fisherman's
account is accurate, it appears they were the only outside witnesses to
the secret rescue operation. They were fishing in a restricted area ---
but they decided to talk about what they saw anyway.

``This is a very Russian story,'' said
\href{https://www.csis.org/people/jeffrey-mankoff}{Jeffrey Mankoff}, a
senior fellow with the Russia and Eurasia Program at the Center for
Strategic and International Studies in Washington.

The submarine sped away, but there was no immediate alert from Russia to
the Norwegian Radiation and Nuclear Safety Authority about a possible
nuclear incident in the Barents Sea, said Astrid Liland, head of the
nuclear preparedness section.

TASS, the official Russian news agency, reported the accident the
following day without mentioning that the submarine was nuclear powered.
The SeverPost story appeared the next morning.

Russia and Norway, Ms. Liland said, have an agreement to notify each
other in the case of incidents involving nuclear installations.
``Unfortunately,'' she said, ``Russia interprets that agreement not to
include military installations such as submarines.''

Image

A fjord near Kirkenes, a small Norwegian town close to the Russian
border.Credit...Mathias Svold for The New York Times

As convoluted as it is in so many ways, the tale of the Losharik, and
the growing power of Russia's Northern Fleet, begins with at least one
very simple explanation, said Professor Zysk, the Norwegian analyst.

``There's a special place in Putin's heart for the navy,'' she said.
``That's one of the key symbols of a great power.''

The Northern Fleet is at the top of Mr. Putin's military budget, which
included top-drawer items like the most advanced surface vessels and
cruise missiles. In 2014, the Northern Fleet put the Arctic brigades
under its command; soldiers equipped with the latest gear for cold
climate warfare. New generations of ballistic-missile and attack
submarines are also being deployed.

With all that naval power, the quickest way for Russia to surprise the
United States would be to steam from the Arctic to the North Atlantic,
said Heather A. Conley, senior vice president for Europe, Eurasia, and
the Arctic at the Center for Strategic and International Studies.

``It's really becoming a much more dynamic area,'' Ms. Conley said. ``It
does feel like we're updating
`\href{https://www.nytimes.com/1985/04/21/books/fiction-in-short.html}{The
Hunt for Red October}.'''

There is also an eye toward economic benefit, Ms. Conley said: Russia
has made no secret of its desire to control a northern shipping lane
through the Arctic as ice recedes because of climate change and to
expand its oil and gas production.

Over the last five years, 14 airfields have been opened or rebuilt along
the Northern Sea Route; three fully autonomous bases have opened on
Arctic archipelagoes. Billions of dollars have been spent on fields for
gas production on the Yamal Peninsula, where total volumes are estimated
at almost 17 trillion cubic meters. The natural gas from the Yamal will
ultimately feed the pipeline now being built through the Baltic Sea to
supply Western Europe.

Still, with the extreme difficulty of recovering oil and gas north of
the Yamal, and the unknowns of tourism and foreign shipping, the
economics may not add up for another half-century --- if then, said
Andreas Osthagen, a senior research fellow at the Fridtjof Nansen
Institute, near Oslo, and author of
``\href{https://www.palgrave.com/gp/book/9789811507533}{Coast Guards and
Ocean Politics in the Arctic}.''

Beyond Russia's need to protect the nuclear deterrent itself, the key to
understanding Russian's keen interest in the Arctic, Professor Zysk
said, is to bear in mind what Moscow does not want to do: become
directly involved in any extended conflict with NATO. Russia knows it
does not have the resources to win that kind of conflict, Professor Zysk
said.

For that reason, no matter where a conflict begins, she said, ``Russia
would do anything to maintain the strategic initiative.'' She said,
``The information superiority comes here.''

Russian generals, for example, speak openly of sowing chaos in the
government financial system of an adversary, Professor Zysk said, and
disrupting seabed cables ``would certainly fit into the objective.''

Major operating and planned submarine cable systems

Barents Sea

Arctic Circle

Atlantic

Ocean

Pacific

Ocean

Indian Ocean

Barents Sea

Arctic Circle

Atlantic

Ocean

Pacific

Ocean

Barents Sea

Arctic Circle

Atlantic Ocean

Pacific

Ocean

Barents Sea

Arctic Circle

North Pacific

Ocean

North Atlantic

Ocean

Indian Ocean

South Atlantic

Ocean

South Pacific

Ocean

Source: TeleGeography

By The New York Times

A 2017 report by Policy Exchange, a research and educational institute
in the United Kingdom, found that seabed cables carry 97 percent of the
data in communications globally, including roughly \$10 trillion in
financial transactions a day. The cables are largely unprotected and
easy to find. As recently as a few years ago, American military and
intelligence officials
\href{https://www.nytimes.com/2015/10/26/world/europe/russian-presence-near-undersea-cables-concerns-us.html}{reported
that} Russian submarines had often been operating near them.

Because the internet can reroute data when cables are damaged, Western
analysts have often dismissed the dangers of sabotage. But considering
the vital role of data in Western institutions of all kinds, Professor
Zysk said, simply applying pressure by degrading the network could be
enough.

``When people lose Facebook and Twitter --- oh, my God!'' she said, not
entirely facetiously.

\href{https://www.chathamhouse.org/expert/mathieu-boulegue}{Mathieu
Boulègue}, a research fellow in the Russia and Eurasia program at
Chatham House, in the United Kingdom, said a specialized craft like the
Losharik might help test the West's ability to respond if cables were
cut.

``This is part of Russia's newfound capability of messing with us,'' Mr.
Boulègue said.

Image

The coast of Norway, near the Russian~ border, as seen though binoculars
from a Norwegian Coast Guard ship.Credit...Mathias Svold for The New
York Times

\hypertarget{an-uncrackable-egg}{%
\subsection{An Uncrackable Egg}\label{an-uncrackable-egg}}

As for the accident itself, few expressed surprise that a jewel of the
Russian submarine fleet might catch fire not very far from its home base
--- probably in water no more than 1,000 feet deep --- leaving most of
its crew dead. The Russians, some experts said, seem to have a greater
tolerance for risk than the West.

The Losharik was designed in the 1980s but, delayed by the fall of the
Soviet Union, it was not launched until 2003, according to a forthcoming
revised edition of ``Cold War Submarines'' by the historians Norman
Polmar and K.J. Moore.

In 2012, the Losharik was part of a scientific operation to drill two
miles into the Arctic crust and retrieve rock samples. The
\href{https://barentsobserver.com/en/security/2015/01/car-magazine-captured-photo-secret-sub-15-01}{best
public view of the sub} came a few years later, in 2015, when it
surfaced during a photo shoot of a Mercedes S.U.V. by the Russian
edition of ``Top Gear.''

Like the shell of an egg, the vessel's titanium spheres resist terrific
pressure much more readily than a traditional, elongated hull, Mr.
Polmar said. ``It can go slowly to the bottom and it won't crack,'' he
said.

Mr. Polmar said there was ``nothing in the U.S. fleet to match'' the
depths that the Losharik can take its crew. Various reports, he said,
place the mysterious craft's maximum depth at anywhere from 8,200 to
20,000 feet.

Mr. Lobner, the former American submarine officer, said ``we have
nothing except unmanned vehicles'' operating at such depths.

Still, while some see an engineering marvel, others see evidence that
Russia may be unable to build the kind of sophisticated, autonomous
underwater drones the United States appears to rely on.

``They would rather adapt existing systems, modernize them, and try to
muddle through,'' Mr. Boulègue said. ``So, no wonder these things keep
exploding,'' he said. Mr. Boulègue believes accidents have been far more
common than publicly known.

John Pike,
\href{https://www.globalsecurity.org/org/staff/pike.htm}{director of the
think tank GlobalSecurity.org}, said the Losharik fire suggested that
the Russian military was still contending with some longstanding issues:
corrupt contractors, and problems with quality control in manufacturing,
spare parts supply chains and maintenance.

``I assume that every other sub in the Russian fleet has similar
problems,'' Mr. Pike said. ``I just think the whole thing is held
together with a lot of baling wire and spit.''

A Russian business newspaper, Kommersant, citing sources close to an
investigation into the Losharik incident, said that when smoke was first
detected in the sub, it did not appear to be catastrophic. The Losharik
may have been docked with its mothership at the time, Kommersant said.

After a partial evacuation, 10 crew members stayed to fight the fire
along with four reinforcements from the mothership, the situation became
more and more dire as oxygen was depleted from two emergency breathing
systems aboard the sub, Kommersant reported. The crew began succumbing
to smoke inhalation, and there may have been an explosion in the battery
compartment, the newspaper said.

Mr. Lobner said that even in an ordinary nuclear submarine, clearances
in the battery compartment are so narrow that a routine inspection often
requires shimmying through in a prone or supine position. The crew
quarters would be small and could quickly fill with smoke, he said.

``This wouldn't be like going into a burning house, '' Mr. Lobner said.

Image

Pvt. Sander Badar of the Norwegian Army watching the waters off Russia's
northern coast.Credit...Mathias Svold for The New York Times

\hypertarget{eyes-open-mouths-shut}{%
\subsection{Eyes Open. Mouths Shut.}\label{eyes-open-mouths-shut}}

The Russians are not the only ones who don't want to talk about the
Losharik.

Adm. James G. Foggo III, commander of the United States Sixth Fleet,
whose area of operations includes Europe, declined to be interviewed for
this article. So did Haakon Bruun-Hanssen, chief of defense for the
Norwegian Armed Forces.

Even Pvt. Sander Badar, a young conscript in the Norwegian Army, guarded
his words carefully as he trained a pair of huge binoculars on the
waters off Russia's northern coast from his observation post on a ridge
nearly a thousand feet above the Barents Sea. It was in that direction,
on the other side of a stretch of coastline called the Fisherman's
Peninsula, that the Losharik burned.

``It's not a secret that we are watching over their border and seeing
what's happening there,'' Private Badar said early one October
afternoon, the Arctic light already fading.

With outposts like Private Badar's, as well as surveillance aircraft and
navy ships, the Norwegian military serves as NATO's eyes and ears on
Russia's doorstep. But when asked about Russian submarines, Private
Badar declined to reveal what he may have seen.

When TASS, the Russian news agency, first reported the Losharik fire, it
said 14 sailors had been killed aboard a ``deep-sea station,'' without
mentioning its nuclear reactor. The next day, a spokesman for Mr. Putin
said information on the accident ``belongs to the category of top-secret
data.''

In the following days, Mr. Putin posthumously conferred the nation's
highest honor, Hero of the Russian Federation, to four of the crew
members and lesser awards to the other 10. At the funeral in St.
Petersburg, a navy officer said the crew had ``prevented a planetary
catastrophe.''

Image

Russian sailors at a memorial service for the men who died on the
Losharik.Credit...Anatoly Maltsev/EPA, via Shutterstock

Russia says it plans to fully restore the sub and put it back into
service. Not everyone seems worried about that.

One retired American rear admiral, John B. Padgett III, a former
commander of the Pacific submarine force, said in a phone interview that
he had no concerns about the United States losing ground to subs like
the Losharik.

``We go as deep as we need to go, as fast as we need to go,'' Admiral
Padgett said.

But Col. Eystein Kvarving, chief of public affairs at Norwegian Joint
Headquarters made clear that the stakes are high.

The Norwegian military, Colonel Kvarving said, has a direct Skype line
to the commander of Russia's Northern Fleet, and tests it once a week.
In the months since the fire, he said, the Russians have carried out
their largest naval exercises since the Cold War.

How might the Losharik fit in?

``You go deep, you go silent,'' Colonel Kvarving said. ``Undetected is
the key word. If they can go undetected where they please, that is a big
concern.''

Image

Col. Eystein Kvarving of the Norwegian Army.Credit...Mathias Svold for
The New York Times

Thomas Nilsen is the editor of The Independent Barents Observer.

Advertisement

\protect\hyperlink{after-bottom}{Continue reading the main story}

\hypertarget{site-index}{%
\subsection{Site Index}\label{site-index}}

\hypertarget{site-information-navigation}{%
\subsection{Site Information
Navigation}\label{site-information-navigation}}

\begin{itemize}
\tightlist
\item
  \href{https://help.nytimes.com/hc/en-us/articles/115014792127-Copyright-notice}{©~2020~The
  New York Times Company}
\end{itemize}

\begin{itemize}
\tightlist
\item
  \href{https://www.nytco.com/}{NYTCo}
\item
  \href{https://help.nytimes.com/hc/en-us/articles/115015385887-Contact-Us}{Contact
  Us}
\item
  \href{https://www.nytco.com/careers/}{Work with us}
\item
  \href{https://nytmediakit.com/}{Advertise}
\item
  \href{http://www.tbrandstudio.com/}{T Brand Studio}
\item
  \href{https://www.nytimes.com/privacy/cookie-policy\#how-do-i-manage-trackers}{Your
  Ad Choices}
\item
  \href{https://www.nytimes.com/privacy}{Privacy}
\item
  \href{https://help.nytimes.com/hc/en-us/articles/115014893428-Terms-of-service}{Terms
  of Service}
\item
  \href{https://help.nytimes.com/hc/en-us/articles/115014893968-Terms-of-sale}{Terms
  of Sale}
\item
  \href{https://spiderbites.nytimes.com}{Site Map}
\item
  \href{https://help.nytimes.com/hc/en-us}{Help}
\item
  \href{https://www.nytimes.com/subscription?campaignId=37WXW}{Subscriptions}
\end{itemize}
