Sections

SEARCH

\protect\hyperlink{site-content}{Skip to
content}\protect\hyperlink{site-index}{Skip to site index}

\href{https://www.nytimes.com/section/realestate}{Real Estate}

\href{https://myaccount.nytimes.com/auth/login?response_type=cookie\&client_id=vi}{}

\href{https://www.nytimes.com/section/todayspaper}{Today's Paper}

\href{/section/realestate}{Real Estate}\textbar{}Stonington, Conn.: A
Waterfront Community With a Colonial Vibe

\url{https://nyti.ms/2Vs8p3j}

\begin{itemize}
\item
\item
\item
\item
\item
\item
\end{itemize}

Advertisement

\protect\hyperlink{after-top}{Continue reading the main story}

Supported by

\protect\hyperlink{after-sponsor}{Continue reading the main story}

Living in

\hypertarget{stonington-conn-a-waterfront-community-with-a-colonial-vibe}{%
\section{Stonington, Conn.: A Waterfront Community With a Colonial
Vibe}\label{stonington-conn-a-waterfront-community-with-a-colonial-vibe}}

New Yorkers seeking respite from the city have long gravitated to
Connecticut's easternmost shoreline town, with its 18th- and
19th-century homes.

\href{https://www.nytimes.com/slideshow/2020/04/22/realestate/living-in-stonington-conn.html}{}

\hypertarget{living-in--stonington-conn}{%
\subsection{Living In ... Stonington,
Conn.}\label{living-in--stonington-conn}}

13 Photos

View Slide Show ›

\includegraphics{https://static01.nyt.com/images/2020/04/26/realestate/22LIVING-STONINGTONCT-slide-D2JA/22LIVING-STONINGTONCT-slide-D2JA-articleLarge.jpg?quality=75\&auto=webp\&disable=upscale}

Jane Beiles for The New York Times

By Lisa Prevost

\begin{itemize}
\item
  April 22, 2020
\item
  \begin{itemize}
  \item
  \item
  \item
  \item
  \item
  \item
  \end{itemize}
\end{itemize}

Karen von Ruffer Hills and Francis Hills discovered the coastal town of
Stonington, in southeastern Connecticut, when friends invited them for a
weekend getaway at a local inn in 2009.

``We got there on a Friday, and by Sunday we were at Seaboard Properties
asking, `What's for rent?''' said Ms. von Ruffer Hills, 48, who works as
the marketing director for the photography studio that Mr. Hills, 53,
has in New York. ``My husband is British, and Stonington has a little
bit of that English countryside feel to it and those nice old homes. It
immediately spoke to him.''

NEW LONDON

CONNECTICUT

CONN.

Stonington

Stonington

R.I.

Mystic R.

Pawcatuck

Mystic Aquarium

Mystic Seaport Museum ~

AMTRAk

Stone Acres

Farm

Mystic

Stonington

Borough

Mystic station

Pawcatuck R.

MASONS

ISLAND

Stonington

Lighthouse Museum

WATCH

HILL

Block Island Sound

1 mile

By The New York Times

The couple rented in Stonington for many summers, stealing time away
from their apartment in New York, about 135 miles southwest, as they
could. Then about five years ago, they decided to reverse the
arrangement, making Stonington their home base.

``It was a lifestyle choice: We're both midlife and sort of
re-evaluating,'' Ms. von Ruffer Hills said. ``It was a nice way to
gently remove ourselves from the chaos of New York.''

\includegraphics{https://static01.nyt.com/images/2020/04/22/realestate/22LIVING-STONINGTONCT-slide-IH7Q/22LIVING-STONINGTONCT-slide-IH7Q-articleLarge.jpg?quality=75\&auto=webp\&disable=upscale}

Now they live in a renovated Colonial-era home with a gambrel roof in
the heart of Stonington Borough, a densely developed village district on
a peninsula jutting into Stonington Harbor. At one end of their street
is a tiny sandy beach where they can ease into the waters of Long Island
Sound. At the other is the compact commercial strip of Water Street,
where they can walk to grab a coffee, browse the windows of an art
gallery or catch up on local gossip at Tom's News.

They have developed a ``beautiful network of friends,'' Ms. von Ruffer
Hills said, and thrown themselves into community life. Her husband
curated a project for the Stonington Historical Society last year. And
she is now a burgess in the borough government.

``I am sanitation commissioner, responsible for sanitation and
streetlights,'' she said. ``It's not the sexiest of titles, but it's
fun, actually.''

Danielle Chesebrough, 36, also immersed herself in Stonington's inner
workings after moving from New York with her husband, Sam, to the Mystic
section four years ago. Formerly a senior analyst with the United
Nations Global Compact, Ms. Chesebrough served on a couple of town
boards before becoming the first woman to be elected First Selectman
last November.

Image

The Stonington Free Library, in Wadawanuck Square, is a private
association that serves the entire town.~Credit...Jane Beiles for The
New York Times

Four months into the job, she now finds herself trying to help lead the
community through the pandemic crisis --- with three young children at
home. While the economic upheaval facing the heavily tourist-dependent
town has at times seemed overwhelming, Ms. Chesebrough said she is
heartened to see so many residents pull together.

``Some residents started a GoFundMe page to raise money to buy gift
certificates to local restaurants doing takeout,'' she said. ``They're
donating the certificates to our human services department to distribute
to households in need.''

As of mid-April, three weeks later, the group,
\href{https://www.facebook.com/groups/2583619385099118/}{Stonington
Feeds Stonington}, had raised over \$15,000.

Image

6 COVE HILL ROAD \textbar{} A five-bedroom, four-bathroom house with a
saltwater swimming pool on Masons Island, built in 2000 on 1.36 acres,
listed for \$2.9 million. 860-460-8002Credit...Jane Beiles for The New
York Times

\hypertarget{what-youll-find}{%
\subsection{What You'll Find}\label{what-youll-find}}

Situated between the Mystic River to the west and the Pawcatuck River to
the east, Stonington is Connecticut's easternmost shoreline town,
bumping up against Rhode Island. Largely oriented around the water, it
is perhaps most widely known for the Mystic Seaport Museum and Mystic
Aquarium, popular tourist attractions on the Stonington side of the
normally bustling Mystic village (the rest of the village is in Groton).

But the town of roughly 18,000 has a much more varied landscape, from
its pastoral northern uplands crisscrossed with old stone walls, to the
industrial-era village of Pawcatuck and the suburban-flavored
neighborhood of Masons Island, reached by a causeway.

New Yorkers seeking a weekend respite have long gravitated toward
historic Stonington Borough, which served as a summer destination for
city dwellers during the steamship era (as did Watch Hill, R.I., to the
east). The well-protected harbor, a haven for boating enthusiasts, is
home to a yacht club and a commercial fishing fleet.

In the heart of the borough, the narrow streets are lined with
colorfully painted 18th- and 19th-century homes that hug the sidewalks.
While there are no historic district protections in the borough, ``you
would get a lot of pushback if you wanted to knock down a historic home
and redo it,'' said James H. Michalove, president and founder of
Seaboard Properties. (Other sections of town
\href{https://www.theday.com/article/20190321/NWS05/190329861}{have not
fared as well} at protecting historic buildings from demolition.)

During the high season, the shops and restaurants on and around Water
Street normally attract day-trippers, many of whom come by boat. The
street ends at the peninsula's point, where a small beach and parking
area provide visitors with panoramic views.

Howard Taylor, a yacht broker who has lived in town for about 25 years,
said he has noticed more New York families moving to Stonington
year-round, as professionals have become increasingly able to work from
home.

``What we're going through now is going to exaggerate that,'' he said.

Image

330 NORTH MAIN STREET \textbar{} A four-bedroom, five-bathroom
waterfront home with a dock on Quanaduck Cove, built in 1950 on 2.2
acres, listed for \$2.35 million. 860-535-8364

\hypertarget{what-youll-pay}{%
\subsection{What You'll Pay}\label{what-youll-pay}}

The median sale price for a single-family home in the town of Stonington
in the 12-month period before March 31 was \$336,000, about 5 percent
higher than in the previous year, according to data provided by William
Pitt Sotheby's International Realty; for a single-family home in the
borough, the median sale price was \$510,000. The median sale price for
a condo in the year ending March 31 was \$292,000, compared with
\$338,500 the previous year.

As of April 14, 140 properties were listed for sale in Stonington,
ranging from a two-bedroom mobile home in a park on the Mystic River
asking \$48,000 to a five-bedroom waterfront home on 1.36 acres on
Masons Island asking \$2.9 million.

``Inventory is nice and low right now,'' said Melinda Carlisle, a sales
associate with Randall Realtors. ``We've had a couple of really good
years, and people weren't putting much stuff on.''

Historic homes in good repair in the borough typically sell for upward
of \$600,000, with waterfront properties drawing \$2 million or more,
Mr. Michalove said.

On Masons Island, many homes belong to the children or grandchildren of
the original owners, Ms. Carlisle said. While it is sometimes possible
to find a smaller, unimproved home in the \$400,000s, the five homes she
currently has listed there are priced from \$899,000 to \$2.9 million.

One of the newer rental complexes in town is Threadmill Apartments, a
converted factory building in Pawcatuck. The 58 one-bedroom apartments,
with exposed brick walls and ceiling beams, rent for \$1,500 to \$1,900
a month, said Mary Ann Agostini, an agent with William Pitt Sotheby's.

Image

435 NORTH MAIN STREET \textbar{} A six-bedroom house with five full and
two half bathrooms, built in 1938 on 3.4 acres, listed for \$1.395
million. 860-912-1221

\hypertarget{the-vibe}{%
\subsection{The Vibe}\label{the-vibe}}

While activity ramps up in Stonington during the summer months, the
atmosphere is low-key. The borough's Fourth of July parade is a
community highlight, said Chelsea Mitchell, the library director at the
Stonington Historical Society. Children ride on decorated bicycles,
spectators are encouraged to join the parade, and the Declaration of
Independence is read aloud --- ``at the end of which, everyone yells, `A
pox on King George!''' she said.

The \href{http://www.holyghostclub.com/}{Portuguese Holy Ghost Society},
a social club based in the borough, draws residents from around the area
to its year-round events, which include Friday night fish-and-chips
dinners and the Labor Day weekend Feast of the Holy Ghost.

``We have over 400 members from all backgrounds: lawyers, medical
doctors, fishermen, carpenters and scientists,'' said Mr. Taylor, the
yacht broker, who is also the club's vice president.

Residents support several farms in town, including
\href{https://www.stoneacresfarm.com/}{Stone Acres}, which hosts farm
dinners, weddings and educational events. The farm stand, which sells
produce, flowers and locally produced artisanal foods, is already in
high gear, said Jane Meiser, the director of operations and a descendant
of the farm's original Colonial-era owners.

``Everyone's pretty much here in their second homes now,'' she said,
because of the pandemic lockdown. ``So our farm stand has been
incredibly busy with the heightened demand on local, sustainable, really
healthy organic food.''

Image

Social Coffee Roastery in the borough is open for business during the
shutdown.Credit...Jane Beiles for The New York Times

\hypertarget{the-schools}{%
\subsection{The Schools}\label{the-schools}}

About 2,000 students are served by Stonington's four public schools.
Students in prekindergarten through fifth grade attend Deans Mill or
West Vine Street elementary schools, both of which recently underwent
extensive renovations. West Vine Street's principal, Alicia Sweet Dawe,
was recognized last year as Elementary Principal of the Year by the
Connecticut Association of Schools and the National Association of
Elementary School Principals.

Students in sixth through eighth grade attend Stonington Middle School,
in Mystic.

Stonington High School, in Pawcatuck, has about 650 students. Mean SAT
scores for the 2019 graduating class were 551 in evidence-based reading
and writing, and 532 in math; statewide means were 529 and 516. The
school also has career and technical education programs to train
students for postgraduate employment.

\hypertarget{the-commute}{%
\subsection{The Commute}\label{the-commute}}

From the Shoreline East rail station in New London, about 20 minutes
away, commuters can get a train to the Metro-North Railroad station in
New Haven for \$10.25 one-way; a round-trip peak ticket from there to
Grand Central Terminal is \$47. The commute takes roughly three hours.

Amtrak trains stop at the station in Mystic. The trip to Penn Station
takes around three hours; round-trip fares range from \$122 to \$240,
depending on seat selection.

All rail services are currently operating on reduced schedules because
of the pandemic.

The 135-mile drive to New York City on Interstate 95 takes two and a
half hours or longer, depending on traffic.

Image

The Captain Nathaniel B. Palmer House, built in 1852, has an octagonal
cupola once used for viewing ships coming in to Stonington
Harbor.~Credit...Jane Beiles for The New York Times

\hypertarget{the-history}{%
\subsection{The History}\label{the-history}}

The stone lighthouse tower that stands at the borough's point dates to
1840. Listed on the National Register of Historic Places, it is attached
to a modest residence that housed the beacon's keepers for its nearly 50
years in operation. The Stonington Historical Society bought the
abandoned lighthouse in 1925 for \$3,650 and converted it to a museum
for lighthouse artifacts. The museum is currently closed for restoration
and is tentatively scheduled to reopen in the summer.

For weekly email updates on residential real estate news,
\href{http://www.nytimes.com/newsletters/realestate/}{sign up here}.
Follow us on Twitter:
\href{https://twitter.com/nytrealestate}{@nytrealestate}.

Advertisement

\protect\hyperlink{after-bottom}{Continue reading the main story}

\hypertarget{site-index}{%
\subsection{Site Index}\label{site-index}}

\hypertarget{site-information-navigation}{%
\subsection{Site Information
Navigation}\label{site-information-navigation}}

\begin{itemize}
\tightlist
\item
  \href{https://help.nytimes.com/hc/en-us/articles/115014792127-Copyright-notice}{©~2020~The
  New York Times Company}
\end{itemize}

\begin{itemize}
\tightlist
\item
  \href{https://www.nytco.com/}{NYTCo}
\item
  \href{https://help.nytimes.com/hc/en-us/articles/115015385887-Contact-Us}{Contact
  Us}
\item
  \href{https://www.nytco.com/careers/}{Work with us}
\item
  \href{https://nytmediakit.com/}{Advertise}
\item
  \href{http://www.tbrandstudio.com/}{T Brand Studio}
\item
  \href{https://www.nytimes.com/privacy/cookie-policy\#how-do-i-manage-trackers}{Your
  Ad Choices}
\item
  \href{https://www.nytimes.com/privacy}{Privacy}
\item
  \href{https://help.nytimes.com/hc/en-us/articles/115014893428-Terms-of-service}{Terms
  of Service}
\item
  \href{https://help.nytimes.com/hc/en-us/articles/115014893968-Terms-of-sale}{Terms
  of Sale}
\item
  \href{https://spiderbites.nytimes.com}{Site Map}
\item
  \href{https://help.nytimes.com/hc/en-us}{Help}
\item
  \href{https://www.nytimes.com/subscription?campaignId=37WXW}{Subscriptions}
\end{itemize}
