Sections

SEARCH

\protect\hyperlink{site-content}{Skip to
content}\protect\hyperlink{site-index}{Skip to site index}

\href{https://www.nytimes.com/section/climate}{Climate}

\href{https://myaccount.nytimes.com/auth/login?response_type=cookie\&client_id=vi}{}

\href{https://www.nytimes.com/section/todayspaper}{Today's Paper}

\href{/section/climate}{Climate}\textbar{}John Houghton, Who Sounded
Alarm on Climate Change, Dies at 88

\url{https://nyti.ms/3554JYj}

\begin{itemize}
\item
\item
\item
\item
\item
\end{itemize}

\href{https://www.nytimes.com/section/climate?action=click\&pgtype=Article\&state=default\&region=TOP_BANNER\&context=storylines_menu}{Climate
and Environment}

\begin{itemize}
\tightlist
\item
  \href{https://www.nytimes.com/2020/07/30/climate/sea-level-inland-floods.html?action=click\&pgtype=Article\&state=default\&region=TOP_BANNER\&context=storylines_menu}{Rising
  Seas}
\item
  \href{https://www.nytimes.com/interactive/2020/climate/trump-environment-rollbacks.html?action=click\&pgtype=Article\&state=default\&region=TOP_BANNER\&context=storylines_menu}{Trump's
  Changes}
\item
  \href{https://www.nytimes.com/interactive/2020/04/19/climate/climate-crash-course-1.html?action=click\&pgtype=Article\&state=default\&region=TOP_BANNER\&context=storylines_menu}{Climate
  101}
\item
  \href{https://www.nytimes.com/interactive/2018/08/30/climate/how-much-hotter-is-your-hometown.html?action=click\&pgtype=Article\&state=default\&region=TOP_BANNER\&context=storylines_menu}{Is
  Your Hometown Hotter?}
\item
  \href{https://www.nytimes.com/newsletters/climate-change?action=click\&pgtype=Article\&state=default\&region=TOP_BANNER\&context=storylines_menu}{Newsletter}
\end{itemize}

Advertisement

\protect\hyperlink{after-top}{Continue reading the main story}

Supported by

\protect\hyperlink{after-sponsor}{Continue reading the main story}

Those We've Lost

\hypertarget{john-houghton-who-sounded-alarm-on-climate-change-dies-at-88}{%
\section{John Houghton, Who Sounded Alarm on Climate Change, Dies at
88}\label{john-houghton-who-sounded-alarm-on-climate-change-dies-at-88}}

He led the United Nations panel on global warming that won a Nobel Peace
Prize; he died from complications of the coronavirus

\includegraphics{https://static01.nyt.com/images/2020/04/26/obituaries/26hougton-obit2/23hougton1-articleLarge.jpg?quality=75\&auto=webp\&disable=upscale}

\href{https://www.nytimes.com/by/john-schwartz}{\includegraphics{https://static01.nyt.com/images/2018/02/16/multimedia/author-john-schwartz/author-john-schwartz-thumbLarge.jpg}}

By \href{https://www.nytimes.com/by/john-schwartz}{John Schwartz}

\begin{itemize}
\item
  April 23, 2020
\item
  \begin{itemize}
  \item
  \item
  \item
  \item
  \item
  \end{itemize}
\end{itemize}

\emph{This obituary is part of a series about people who have died in
the coronavirus pandemic. Read about others}
\href{https://www.nytimes.com/series/people-who-have-died-of-the-coronavirus}{\emph{here}}\emph{.}

John Houghton, a climate scientist and influential figure in the United
Nations panel that brought the threat of climate change to the world's
attention and received a Nobel Prize, died on April 15 in Dolgellau,
Wales. He was 88.

The cause was complications of the novel coronavirus, according to his
granddaughter Hannah Malcolm, who
\href{https://twitter.com/hannahmmalcolm/status/1250778555505655808}{announced
the death, at a hospital, on Twitter}.

A key participant in the United Nations Intergovernmental Panel on
Climate Change, Dr. Houghton was the lead editor of the organization's
first three reports, issued in 1990, 1995 and 2001. With each report,
the evidence underpinning global warming
\href{https://archive.ipcc.ch/publications_and_data/ar4/wg1/en/ch9s9-1-3.html}{and
the role humans play in causing it grew more ineluctable},
\href{https://www.nytimes.com/2000/10/26/us/a-shift-in-stance-on-global-warming-theory.html}{and
the calls for international action became more pressing.} The group
\href{https://www.nobelprize.org/prizes/peace/2007/summary/}{received
the 2007 Nobel Peace Prize jointly with Al Gore}, the former vice
president and climate campaigner.

\href{https://www.nytimes.com/1994/09/20/science/emissions-must-be-cut-to-avert-shift-in-climate-panel-says.html}{Speaking
about climate change in 1994}, Dr. Houghton said that delay served no
one. ``We should start to do what we can do now and also begin to plan
to do more,'' he said, and ``not wait 10 or 20 years till things are
more clear.''

Mr. Gore recalled Dr. Houghton in a statement as ``a critical voice
bringing the urgency of the climate crisis to the attention of
policymakers.''

``He took seriously the responsibility of scientists to not only produce
research,'' Mr. Gore added, ``but also to help ensure that the public
world understood the implications of that research.''

\href{https://www.nytimes.com/section/climate?action=click\&pgtype=Article\&state=default\&region=MAIN_CONTENT_1\&context=storylines_keepup}{}

\hypertarget{climate-and-environment-}{%
\subsubsection{Climate and Environment
›}\label{climate-and-environment-}}

\hypertarget{keep-up-on-the-latest-climate-news}{%
\paragraph{Keep Up on the Latest Climate
News}\label{keep-up-on-the-latest-climate-news}}

Updated July 30, 2020

Here's what you need to know about the latest climate change news this
week:

\begin{itemize}
\item
  \begin{itemize}
  \tightlist
  \item
    \href{https://www.nytimes.com/2020/07/30/climate/bangladesh-floods.html?action=click\&pgtype=Article\&state=default\&region=MAIN_CONTENT_1\&context=storylines_keepup}{Floods
    in}\href{https://www.nytimes.com/2020/07/30/climate/bangladesh-floods.html?action=click\&pgtype=Article\&state=default\&region=MAIN_CONTENT_1\&context=storylines_keepup}{Bangladesh}
    are punishing the people least responsible for climate change.
  \item
    As climate change raises sea levels,
    \href{https://www.nytimes.com/2020/07/30/climate/sea-level-inland-floods.html?action=click\&pgtype=Article\&state=default\&region=MAIN_CONTENT_1\&context=storylines_keepup}{storm
    surges and high tides} are likely to push farther inland.
  \item
    The E.P.A. inspector general plans to investigate whether a rollback
    of fuel efficiency standards
    \href{https://www.nytimes.com/2020/07/27/climate/trump-fuel-efficiency-rule.html?action=click\&pgtype=Article\&state=default\&region=MAIN_CONTENT_1\&context=storylines_keepup}{violated
    government rules}.
  \end{itemize}
\end{itemize}

Peter Gleick, a climate scientist and member of the U.S. National
Academy of Sciences, said in an email: ``He understood earlier than
most, and was willing to tell the politicians, that climate change was
real and a threat not just to the richer countries, but especially to
the poorer ones.''

Religion was central to Dr. Houghton's life. In his autobiography
``\href{https://www.amazon.com/Eye-Storm-Autobiography-John-Houghton/dp/0745955843/ref=tmm_pap_swatch_0?_encoding=UTF8\&qid=\&sr=\#reader_0745955843}{In
the Eye of the Storm''} (2013, with Gill Tavner), he said: ``It was
increasingly clear to me that the universe is God's creation. As science
was the means by which I would be able to explore and describe God's
creative work, I could not see how there could possibly be conflict
between science and faith.''

Dr. Houghton provided a spark that led to a climate movement within the
evangelical community. In 2002, the Rev. Rich Cizik, an American
evangelical leader, heard Dr. Houghton speak at the University of Oxford
in England and had a ``conversion on climate change so profound that he
likened it to an `altar call,' when nonbelievers accept Jesus as their
savior,''
\href{https://www.nytimes.com/2005/03/10/us/evangelical-leaders-swing-influence-behind-effort-to-combat-global.html}{The
New York Times} wrote in 2005.

\includegraphics{https://static01.nyt.com/images/2020/04/26/obituaries/26Houghton/23Houghton-articleLarge.jpg?quality=75\&auto=webp\&disable=upscale}

John Theodore Houghton was born in Dyserth, Wales, on Dec. 30, 1931, to
Sidney and Miriam (Yarwood) Houghton. His father was a history teacher,
and his mother taught mathematics before becoming a homemaker. At 16,
John received a scholarship to Jesus College, Oxford, in 1948.

``Not only was I 16,'' he wrote in the autobiography, ``but I was a
rather young 16 from a strict Christian background, with very little
experience of anything other than home.''

But he made his way, studying mathematics and physics. He graduated from
Oxford with a bachelor's degree in 1951 and a doctorate in atmospheric,
oceanic and planetary physics in 1955. He began to teach at Oxford in
1958.

In the 1970s, Dr. Houghton worked with NASA on the remote sensing
instruments that allowed its
\href{https://www.nasa.gov/content/goddard/nimbus-nasa-remembers-first-earth-observations}{Nimbus
satellites} to explore the earth's atmosphere; the instrumentation
helped transform the study of weather systems and the environment. In
1972, he became a fellow of the Royal Society, the British scientific
society.

Dr. Houghton was director general and chief executive of the United
Kingdom's Meteorological Office from 1983 to 1991 and joined the effort
to form the Intergovernmental Panel on Climate Change, or I.P.C.C. He
served as a chairman of the panel's scientific assessment working group
from 1988 to 2002. In 1990, he helped set up the
\href{https://www.metoffice.gov.uk/weather/climate-change/organisations-and-reports/met-office-hadley-centre}{Hadley
Center for Climate Prediction and Research}, a leading center of climate
work in Britain. He was knighted in 1991.

Dr. Houghton's other books include
``\href{https://www.amazon.com/gp/product/1107463793?ie=UTF8\&tag=thewaspos09-20\&camp=1789\&linkCode=xm2\&creativeASIN=1107463793}{Global
Warming: The Complete Briefing},'' first published in 1994, and
\href{https://www.amazon.com/Does-Play-Dice-Story-Universe/dp/0310515718}{``Does
God Play Dice? A Look at the Story of the Universe''} (1988).

He married Margaret Broughton in 1962; she died in 1986. In 1988, he
married Sheila Thompson. In addition to his granddaughter Hannah, he is
survived by his wife; two children from his first marriage, Peter
Houghton and Janet Malcolm; a younger brother, Paul; and six other
grandchildren. An older brother,
\href{https://rmets.onlinelibrary.wiley.com/doi/pdf/10.1002/wea.2566}{David,
died in 2015}.

In recent years Dr. Houghton had retired to the Welsh seaside and was
fading into dementia, Hannah
\href{https://twitter.com/hannahmmalcolm/status/1250778581476749314}{Malcolm
said, adding}: ``But the sea remained with him. A good life.''

As a leader of the I.P.C.C., he had the skills of a statesman, said
Jean-Pascal van Ypersele of the Université Catholique de Louvain ******
in Belgium. He recalled watching Dr. Houghton co-chair a meeting in 1995
in Madrid that led to a statement that the smoking gun of climate change
had been found: The influence of human activities on climate was
becoming discernible in observations of the present, not just in
projections of the future.

``Fossil fuel companies and oil-dependent countries were intensely
lobbying at that I.P.C.C. meeting to try to dilute the message,'' Dr.
van Ypersele said. But Dr. Houghton, he added, ``had a deep
understanding of the science,'' and ``he was also a British gentleman,
able to listen patiently to the views of vested interests, and manage
the meeting so that scientists would have the last word, as it should
be.''

After a marathon session that was still going at 4 a.m., the tough
language was approved.

Despite such efforts, however, effective global action to blunt the
effects of a warming world has yet to happen. In a
\href{https://twitter.com/hannahmmalcolm/status/1250778576405880833}{series
of Twitter messages about her grandfather,} Ms. Malcolm said: ``When I
was younger, my consistent memory of him was warnings over the
devastation waiting us if we didn't act on climate change. And I
remember thinking how glad I was that scientists like him were in
charge. But of course it isn't the scientists in charge.''

Advertisement

\protect\hyperlink{after-bottom}{Continue reading the main story}

\hypertarget{site-index}{%
\subsection{Site Index}\label{site-index}}

\hypertarget{site-information-navigation}{%
\subsection{Site Information
Navigation}\label{site-information-navigation}}

\begin{itemize}
\tightlist
\item
  \href{https://help.nytimes.com/hc/en-us/articles/115014792127-Copyright-notice}{©~2020~The
  New York Times Company}
\end{itemize}

\begin{itemize}
\tightlist
\item
  \href{https://www.nytco.com/}{NYTCo}
\item
  \href{https://help.nytimes.com/hc/en-us/articles/115015385887-Contact-Us}{Contact
  Us}
\item
  \href{https://www.nytco.com/careers/}{Work with us}
\item
  \href{https://nytmediakit.com/}{Advertise}
\item
  \href{http://www.tbrandstudio.com/}{T Brand Studio}
\item
  \href{https://www.nytimes.com/privacy/cookie-policy\#how-do-i-manage-trackers}{Your
  Ad Choices}
\item
  \href{https://www.nytimes.com/privacy}{Privacy}
\item
  \href{https://help.nytimes.com/hc/en-us/articles/115014893428-Terms-of-service}{Terms
  of Service}
\item
  \href{https://help.nytimes.com/hc/en-us/articles/115014893968-Terms-of-sale}{Terms
  of Sale}
\item
  \href{https://spiderbites.nytimes.com}{Site Map}
\item
  \href{https://help.nytimes.com/hc/en-us}{Help}
\item
  \href{https://www.nytimes.com/subscription?campaignId=37WXW}{Subscriptions}
\end{itemize}
