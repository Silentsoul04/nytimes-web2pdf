Sections

SEARCH

\protect\hyperlink{site-content}{Skip to
content}\protect\hyperlink{site-index}{Skip to site index}

\href{https://www.nytimes.com/section/nyregion}{New York}

\href{https://myaccount.nytimes.com/auth/login?response_type=cookie\&client_id=vi}{}

\href{https://www.nytimes.com/section/todayspaper}{Today's Paper}

\href{/section/nyregion}{New York}\textbar{}Iris Love, Stylish
Archaeologist and Dog Breeder, Dies at 86

\url{https://nyti.ms/3bu09W1}

\begin{itemize}
\item
\item
\item
\item
\item
\end{itemize}

\href{https://www.nytimes.com/news-event/coronavirus?action=click\&pgtype=Article\&state=default\&region=TOP_BANNER\&context=storylines_menu}{The
Coronavirus Outbreak}

\begin{itemize}
\tightlist
\item
  live\href{https://www.nytimes.com/2020/08/03/world/coronavirus-covid-19.html?action=click\&pgtype=Article\&state=default\&region=TOP_BANNER\&context=storylines_menu}{Latest
  Updates}
\item
  \href{https://www.nytimes.com/interactive/2020/us/coronavirus-us-cases.html?action=click\&pgtype=Article\&state=default\&region=TOP_BANNER\&context=storylines_menu}{Maps
  and Cases}
\item
  \href{https://www.nytimes.com/interactive/2020/science/coronavirus-vaccine-tracker.html?action=click\&pgtype=Article\&state=default\&region=TOP_BANNER\&context=storylines_menu}{Vaccine
  Tracker}
\item
  \href{https://www.nytimes.com/2020/08/02/us/covid-college-reopening.html?action=click\&pgtype=Article\&state=default\&region=TOP_BANNER\&context=storylines_menu}{College
  Reopening}
\item
  \href{https://www.nytimes.com/live/2020/08/03/business/stock-market-today-coronavirus?action=click\&pgtype=Article\&state=default\&region=TOP_BANNER\&context=storylines_menu}{Economy}
\end{itemize}

Advertisement

\protect\hyperlink{after-top}{Continue reading the main story}

Supported by

\protect\hyperlink{after-sponsor}{Continue reading the main story}

Those We've Lost

\hypertarget{iris-love-stylish-archaeologist-and-dog-breeder-dies-at-86}{%
\section{Iris Love, Stylish Archaeologist and Dog Breeder, Dies at
86}\label{iris-love-stylish-archaeologist-and-dog-breeder-dies-at-86}}

She was once known as the archaeologist in a miniskirt, a scion of old
New York whose second career was raising Westminster Kennel Club
champions.

\includegraphics{https://static01.nyt.com/images/2020/04/24/obituaries/24love2/23Irislove3-articleLarge.jpg?quality=75\&auto=webp\&disable=upscale}

\href{https://www.nytimes.com/by/penelope-green}{\includegraphics{https://static01.nyt.com/images/2018/07/18/multimedia/author-penelope-green/author-penelope-green-thumbLarge-v3.png}}

By \href{https://www.nytimes.com/by/penelope-green}{Penelope Green}

\begin{itemize}
\item
  April 23, 2020
\item
  \begin{itemize}
  \item
  \item
  \item
  \item
  \item
  \end{itemize}
\end{itemize}

\emph{This obituary is part of a series about people who have died in
the coronavirus pandemic. Read about others}
\href{https://www.nytimes.com/series/people-who-have-died-of-the-coronavirus}{\emph{here}}\emph{.}

She was Indiana Jones in a miniskirt, a celebrity archaeologist hatched
out of old New York aristocracy. Iris Love, art historian, champion dog
breeder and the longtime romantic partner of the gossip columnist
\href{https://www.nytimes.com/2017/11/12/arts/liz-smith-dead.html}{Liz
Smith}, was just as comfortable in the ancient world as in the society
pages.

Ms. Love died of the novel coronavirus on April 17 at
NewYork-Presbyterian/Weill Cornell Medical Center in Manhattan, a
friend, Carri Lyon, said. She was 86.

Sunburned, leggy and with a mop of cropped blonde hair, Ms. Love was
catnip to the press. When, in 1971,
\href{https://www.nytimes.com/1971/03/07/archives/an-archeological-find-named-iris-love-archeological-find.html}{The
New York Times wrote about her}for the third time, she was 38 and
several years into what would become an 11-year dig at Knidos, an
ancient Greek city that is now part of Turkey. There she discovered a
temple to Aphrodite on the same summer day in 1969 that Neil Armstrong
walked on the moon.

``A previous reporter from a woman's magazine has been disappointed to
learn that Miss Love can't wear skin creams at Knidos because the dust
would cling to her face,'' the Times reporter wrote on a visit to her
Upper East Side apartment in Manhattan. ``A grocery carton bulging with
the week's fan mail occupies the center of the carpet like an icon.''

Ms. Love had already made headlines when she was a graduate student at
the Institute of Fine Arts, New York University, for outing as forgeries
a prized group of Etruscan warriors at the Metropolitan Museum of Art.
She made headlines again when, on a visit to the British Museum's
collection of antiquities, she identified a crumbling marble head
stashed in its basement as being a remnant of Praxiteles' lost statue of
Aphrodite.

\includegraphics{https://static01.nyt.com/images/2020/04/24/obituaries/24love1/merlin_125200502_4203f985-f8d9-47ef-ab22-6a637e3223df-articleLarge.jpg?quality=75\&auto=webp\&disable=upscale}

Neither storied institution was pleased. Chalk it up perhaps to the
sexism of the time, and the parochialism of her field. Also, though she
had completed the course work for a doctorate, Ms. Love never wrote a
thesis, and as The New Yorker
\href{https://www.newyorker.com/magazine/1978/07/17/the-dig-at-cnidus}{noted
in a profile of her in 1978}, her degree-less status further irritated
jealous peers, who had derided her for her skill at fund-raising, not to
mention her gender.

``Amazons,'' one archaeologist scoffed, referring to Ms. Love's mostly
female crew at Knidos. ``Beautiful girls in bikinis,'' said another.

Ms. Love's Turkish workers, however, called her Mister Director.

``She had a formidable energy and enthusiasm that separated her from the
more cautious of her peers,'' said Maxwell Anderson, a past curator of
the department of Greek and Roman Art at the Met. ``Archaeology relies
on facts, and Iris was given to informed and colorful speculation, which
added coloratura to the discipline. She was a public intellectual in a
way that was not typical of archaeology.''

Iris Cornelia Love was born on Aug. 1, 1933, in New York City. Her
father,
\href{https://www.nytimes.com/1971/09/07/archives/c-ruxton-love-jr-of-stock-exchange.html}{Cornelius
Ruxton Love Jr.}, was a diplomat, an investment banker employed by his
father-in-law, a collector and a descendant of Alexander Hamilton. Her
mother,
\href{https://www.nytimes.com/2003/11/27/nyregion/audrey-b-love-100-a-patron-of-the-arts.html}{Audrey
B. (Josephthal) Love}, was an heiress and arts patron, the daughter of
Edyth Guggenheim and Louis Josephthal, an admiral and the founder of a
brokerage firm.

Her parents were remote figures, as was the custom of the time for her
demographic, but luckily she had a British governess, Katie Wray, who
happened to be a classicist. Iris learned Latin before first grade and
would grow up to be a polylinguist. She spoke Greek, French, German,
Italian and Turkish and could make her way in Mandarin, Russian and
Arabic. At her death she was studying Portuguese.

She was famously loquacious in English, too. Ms. Smith used to chastise
Ms. Love, as she noted in her memoir, ``Natural Blonde'' (2000): ``Don't
begin the story back when they invented language. Get to the bottom
line.''

Ms. Love attended the Brearley School in Manhattan and the Madeira
School in Virginia, where classmates taunted her for being Jewish, a
lineage she had not understood was hers until then.

She graduated from Smith College in 1955; Sylvia Plath was a classmate.
She earned a master's degree from N.Y.U.'s Institute of Fine Arts and
had finished Ph.D. classes there, but not her thesis, because as she
often said, she was too busy with Knidos, overseeing the dig each summer
and fund-raising most winters, to write it.

``She brought archaeology and ancient art to a whole new strata of
society,'' Carlos Picon, an antiquities expert who was curator of Greek
and Roman art at the Met for 28 years, said in a phone interview. ``She
popularized it and warmed it up, and it seemed like everybody knew her
name. You could go to the middle of the most faraway city and they would
have heard of Iris. There are enough Ph.D.s, and whether we gained
another book or not doesn't matter in the long run. More than once Iris
helped me secure objects and funding for the museum.''

Image

Ms. Love in 2012. She liked to name her dachsunds after figures in
ancient Greek mythology. She was in the home of a friend with
Euphrosyne, left, and Diomedes (on her lap).Credit...Emily Berl for The
New York Times

In her memoir, Ms. Smith recalled falling for Ms. Love --- ``a
Givenchy-clad scientist with a name like a movie star'' --- at a dinner
party in 1977. She said she had been taken by Ms. Love's guilelessness
and energy, her complete lack of interest in pop culture, her intellect
and her love of a good party.

They traveled the world together, and Ms. Love and her many dachshunds
moved into Ms. Smith's apartment. By the late 1980s, she had begun to
breed dogs in earnest from her property in Vermont, including a number
of Westminster Kennel Club champions. She liked to name the dogs for
figures in Greek mythology, like Achilles and Tyche.

Ms. Smith was proud of her companion's new métier, though it came with
complications. Ms. Love, always peripatetic, spent months in Italy,
often with another longtime partner, Bice Brichetto, an Italian
baroness, artist and costume designer, leaving Ms. Smith, as she wrote,
to take care of ``Iris business'' and the dogs. After 15 years, Ms.
Smith had had enough, she wrote, though they remained friends until Ms.
Smith's death in 2017. Ms. Love left no immediate survivors.

``I had lovely times with Iris, who might have been a headache, but
literally never was a bore to me,'' Ms. Smith wrote.

Their annual Westminster dog party at Tavern on the Green in Manhattan,
with a guest list typically exceeding 500 people, was
\href{https://www.nytimes.com/1996/02/18/style/in-westminster-show-season-dogs-are-party-animals.html}{a
canine extravaganza}. There were pate molds shaped like dachshunds, ice
sculptures shaped like fire hydrants and everyone, including the dogs,
in costume. Ms. Love appeared, variously, as Alexander Hamilton,
Cleopatra and a Viking.

``She rose every morning convinced she could move the world if only she
had a lever,'' Ms. Smith wrote of her friend.

\href{https://www.nytimes.com/interactive/2020/obituaries/people-died-coronavirus-obituaries.html?action=click\&pgtype=Article\&state=default\&region=BELOW_MAIN_CONTENT\&context=covid_obits_promo}{}

\hypertarget{those-weve-lost}{%
\section{Those We've Lost}\label{those-weve-lost}}

The coronavirus pandemic has taken an incalculable death toll. This
series is designed to put names and faces to the numbers.

Read more

\includegraphics{https://static01.nyt.com/images/2020/07/30/obituaries/30Pedro/30Pedro-square640.jpg}

\hypertarget{bernaldina-josuxe9-pedro}{%
\section{Bernaldina José Pedro}\label{bernaldina-josuxe9-pedro}}

d. Boa Vista, Brazil

Leader among the Indigenous Macuxi

\includegraphics{https://static01.nyt.com/images/2020/07/31/obituaries/31Swing/merlin_175167783_8913bc90-0d64-43f3-a655-1bb1bf1601c9-square640.jpg}

\hypertarget{john-eric-swing}{%
\section{John Eric Swing}\label{john-eric-swing}}

d. Fountain Valley, Calif.

Champion of Filipino-Americans

\includegraphics{https://static01.nyt.com/images/2020/07/27/obituaries/27Victor/merlin_175001436_38b11f8e-227a-4e2c-9821-7618af9b2524-square640.jpg}

\hypertarget{victor-victor}{%
\section{Victor Victor}\label{victor-victor}}

d. Santo Domingo, Dominican Republic

Beloved musician of the Dominican Republic

\includegraphics{https://static01.nyt.com/images/2020/07/31/obituaries/31Negron/merlin_175160169_516322ae-fd23-4969-b6b2-193ced371105-square640.jpg}

\hypertarget{dr-eddie-negruxf3n}{%
\section{Dr. Eddie Negrón}\label{dr-eddie-negruxf3n}}

d. Fort Walton Beach, Fla.

Internist on Florida's Emerald Coast

\includegraphics{https://static01.nyt.com/images/2020/07/30/obituaries/30Dobson/merlin_175115928_f6b9271c-8f05-4fe1-a38a-5ca4a58f8935-square640.jpg}

\hypertarget{dobby-dobson}{%
\section{Dobby Dobson}\label{dobby-dobson}}

d. Coral Springs, Fla.

Jamaican singer and songwriter

\includegraphics{https://static01.nyt.com/images/2020/08/01/obituaries/28Gonzalez/merlin_175002771_beb57888-3951-409a-ae13-03a94b2e962e-square640.jpg}

\hypertarget{waldemar-gonzalez}{%
\section{Waldemar Gonzalez}\label{waldemar-gonzalez}}

d. White Plains, N.Y.

Teacher and social worker

Advertisement

\protect\hyperlink{after-bottom}{Continue reading the main story}

\hypertarget{site-index}{%
\subsection{Site Index}\label{site-index}}

\hypertarget{site-information-navigation}{%
\subsection{Site Information
Navigation}\label{site-information-navigation}}

\begin{itemize}
\tightlist
\item
  \href{https://help.nytimes.com/hc/en-us/articles/115014792127-Copyright-notice}{©~2020~The
  New York Times Company}
\end{itemize}

\begin{itemize}
\tightlist
\item
  \href{https://www.nytco.com/}{NYTCo}
\item
  \href{https://help.nytimes.com/hc/en-us/articles/115015385887-Contact-Us}{Contact
  Us}
\item
  \href{https://www.nytco.com/careers/}{Work with us}
\item
  \href{https://nytmediakit.com/}{Advertise}
\item
  \href{http://www.tbrandstudio.com/}{T Brand Studio}
\item
  \href{https://www.nytimes.com/privacy/cookie-policy\#how-do-i-manage-trackers}{Your
  Ad Choices}
\item
  \href{https://www.nytimes.com/privacy}{Privacy}
\item
  \href{https://help.nytimes.com/hc/en-us/articles/115014893428-Terms-of-service}{Terms
  of Service}
\item
  \href{https://help.nytimes.com/hc/en-us/articles/115014893968-Terms-of-sale}{Terms
  of Sale}
\item
  \href{https://spiderbites.nytimes.com}{Site Map}
\item
  \href{https://help.nytimes.com/hc/en-us}{Help}
\item
  \href{https://www.nytimes.com/subscription?campaignId=37WXW}{Subscriptions}
\end{itemize}
