Sections

SEARCH

\protect\hyperlink{site-content}{Skip to
content}\protect\hyperlink{site-index}{Skip to site index}

\href{https://www.nytimes.com/section/health}{Health}

\href{https://myaccount.nytimes.com/auth/login?response_type=cookie\&client_id=vi}{}

\href{https://www.nytimes.com/section/todayspaper}{Today's Paper}

\href{/section/health}{Health}\textbar{}F.D.A. Warns of Heart Problems
From Malaria Drugs Used for Coronavirus

\url{https://nyti.ms/3eJ2rTj}

\begin{itemize}
\item
\item
\item
\item
\item
\end{itemize}

\href{https://www.nytimes.com/news-event/coronavirus?action=click\&pgtype=Article\&state=default\&region=TOP_BANNER\&context=storylines_menu}{The
Coronavirus Outbreak}

\begin{itemize}
\tightlist
\item
  live\href{https://www.nytimes.com/2020/08/01/world/coronavirus-covid-19.html?action=click\&pgtype=Article\&state=default\&region=TOP_BANNER\&context=storylines_menu}{Latest
  Updates}
\item
  \href{https://www.nytimes.com/interactive/2020/us/coronavirus-us-cases.html?action=click\&pgtype=Article\&state=default\&region=TOP_BANNER\&context=storylines_menu}{Maps
  and Cases}
\item
  \href{https://www.nytimes.com/interactive/2020/science/coronavirus-vaccine-tracker.html?action=click\&pgtype=Article\&state=default\&region=TOP_BANNER\&context=storylines_menu}{Vaccine
  Tracker}
\item
  \href{https://www.nytimes.com/interactive/2020/07/29/us/schools-reopening-coronavirus.html?action=click\&pgtype=Article\&state=default\&region=TOP_BANNER\&context=storylines_menu}{What
  School May Look Like}
\item
  \href{https://www.nytimes.com/live/2020/07/31/business/stock-market-today-coronavirus?action=click\&pgtype=Article\&state=default\&region=TOP_BANNER\&context=storylines_menu}{Economy}
\end{itemize}

Advertisement

\protect\hyperlink{after-top}{Continue reading the main story}

Supported by

\protect\hyperlink{after-sponsor}{Continue reading the main story}

\hypertarget{fda-warns-of-heart-problems-from-malaria-drugs-used-for-coronavirus}{%
\section{F.D.A. Warns of Heart Problems From Malaria Drugs Used for
Coronavirus}\label{fda-warns-of-heart-problems-from-malaria-drugs-used-for-coronavirus}}

The drugs, hydroxychloroquine and chloroquine, have been repeatedly
promoted by President Trump. But they should be used only in clinical
trials or hospitals, the agency said.

\includegraphics{https://static01.nyt.com/images/2020/04/24/science/24VIRUS-FDAHCQ/merlin_170736384_4a604b19-280e-43d2-a889-13e90cd406f9-articleLarge.jpg?quality=75\&auto=webp\&disable=upscale}

By \href{https://www.nytimes.com/by/denise-grady}{Denise Grady}

\begin{itemize}
\item
  Published April 24, 2020Updated May 29, 2020
\item
  \begin{itemize}
  \item
  \item
  \item
  \item
  \item
  \end{itemize}
\end{itemize}

The \href{https://www.fda.gov/media/137250/download}{Food and Drug
Administration issued a safety warning} on Friday about
\href{https://www.nytimes.com/2020/05/29/health/coronavirus-hydroxychloroquine.html}{hydroxychloroquine}
and chloroquine, malaria drugs that have been promoted by
\href{https://www.nytimes.com/2020/05/19/us/politics/hydroxychloroquine-trump-coronavirus.html}{President
Trump} repeatedly and
\href{https://www.nytimes.com/2020/04/01/health/hydroxychloroquine-coronavirus-malaria.html}{widely
used to treat coronavirus patients} despite the lack of evidence that
they work.

The drugs can cause dangerous abnormalities in heart rhythm in
coronavirus patients, and should be used only in clinical trials or
hospitals where patients can be closely monitored for heart problems,
the Food and Drug Administration warned in a safety communication issued
on Friday.

Top health experts have regularly contradicted the president. . This
week, a federal official abruptly removed from his post as head of an
agency involved in development of a vaccine,
\href{https://www.nytimes.com/2020/04/23/us/politics/rick-bright-trump-hydroxychloroquine.html}{complained
that he was pressured to endorse these drugs without adequate science.}
Administration officials said he had been dismissed for other reasons.

With no vaccine or effective treatment for Covid-19, the disease caused
by the coronavirus,
\href{https://www.nytimes.com/2020/04/17/health/trump-hydroxychloroquine-coronavirus.html?searchResultPosition=2}{many
hospitals have been using hydroxychloroquine}, sometimes with the
antibiotic azithromycin, in the hope that they might help.
\href{https://www.nytimes.com/2020/04/17/health/trump-hydroxychloroquine-coronavirus.html}{In
recent interviews, doctors around the country} have cited their
desperate need to help patients and try potential treatments despite
limited evidence.

Results of
\href{https://www.nytimes.com/2020/04/12/health/chloroquine-coronavirus-trump.html?searchResultPosition=3}{small
studies have} trickled out in the last few weeks that signaled more
problems with using the malaria drugs to treat coronavirus patients. The
latest report, on Friday in the journal Nature Medicine, describes
abnormal heart rhythms in 84 patients treated with the drugs.

Several medical societies, including the Infectious Diseases Society of
America, the American Thoracic Society and the American College of
Cardiology, have warned of the risks of using malaria drugs with
azithromycin to treat patients with Covid-19 outside of a clinical trial
or without close monitoring.

``The F.D.A. is aware of reports of serious heart rhythm problems in
patients with Covid-19 treated with hydroxychloroquine or chloroquine,
often in combination with azithromycin'' and other drugs that can
disrupt heart rhythm, the statement said. It also noted that many people
were getting outpatient prescriptions for the drugs in the hopes of
preventing the infection or treating it themselves.

\hypertarget{latest-updates-global-coronavirus-outbreak}{%
\section{\texorpdfstring{\href{https://www.nytimes.com/2020/08/01/world/coronavirus-covid-19.html?action=click\&pgtype=Article\&state=default\&region=MAIN_CONTENT_1\&context=storylines_live_updates}{Latest
Updates: Global Coronavirus
Outbreak}}{Latest Updates: Global Coronavirus Outbreak}}\label{latest-updates-global-coronavirus-outbreak}}

Updated 2020-08-02T07:42:09.613Z

\begin{itemize}
\tightlist
\item
  \href{https://www.nytimes.com/2020/08/01/world/coronavirus-covid-19.html?action=click\&pgtype=Article\&state=default\&region=MAIN_CONTENT_1\&context=storylines_live_updates\#link-34047410}{The
  U.S. reels as July cases more than double the total of any other
  month.}
\item
  \href{https://www.nytimes.com/2020/08/01/world/coronavirus-covid-19.html?action=click\&pgtype=Article\&state=default\&region=MAIN_CONTENT_1\&context=storylines_live_updates\#link-780ec966}{Top
  U.S. officials work to break an impasse over the federal jobless
  benefit.}
\item
  \href{https://www.nytimes.com/2020/08/01/world/coronavirus-covid-19.html?action=click\&pgtype=Article\&state=default\&region=MAIN_CONTENT_1\&context=storylines_live_updates\#link-2bc8948}{Its
  outbreak untamed, Melbourne goes into even greater lockdown.}
\end{itemize}

\href{https://www.nytimes.com/2020/08/01/world/coronavirus-covid-19.html?action=click\&pgtype=Article\&state=default\&region=MAIN_CONTENT_1\&context=storylines_live_updates}{See
more updates}

More live coverage:
\href{https://www.nytimes.com/live/2020/07/31/business/stock-market-today-coronavirus?action=click\&pgtype=Article\&state=default\&region=MAIN_CONTENT_1\&context=storylines_live_updates}{Markets}

The warning is based on reports from multiple sources that described
adverse events, including several types of abnormal heart rhythm, ``and
in some cases death,'' the F.D.A. said.

The message is the second warning about the drugs this week from a
federal health agency. On Tuesday,
\href{https://www.nytimes.com/2020/04/21/health/nih-covid-19-treatment.html?searchResultPosition=1}{guidelines
from the National Institute of Allergy and Infectious Diseases}
cautioned that patients receiving hydroxychloroquine or chloroquine
should be monitored for adverse effects, particularly an abnormality in
heart rhythm called prolonged QTc interval. And at a White House
briefing that day, the F.D.A. commissioner, Dr. Stephen Hahn, emphasized
that the agency wanted data from randomized clinical trials before
considering the drugs as a valid treatment.

Rick Bright, who led the Biomedical Advanced Research and Development
Authority, the agency seeding money to companies working on vaccines,
\href{https://www.nytimes.com/2020/04/22/us/politics/rick-bright-trump-hydroxychloroquine-coronavirus.html}{said
on Wednesday that he was removed from his post} after he pressed for
rigorous vetting of hydroxychloroquine, and that the administration had
put ``politics and cronyism ahead of science.''

There is no proven treatment for the coronavirus, and there is no proof
that hydroxychloroquine and chloroquine can help coronavirus patients.
Those two drugs are approved to treat malaria and the autoimmune
diseases lupus and rheumatoid arthritis*.* But earlier reports from
France and China suggesting a benefit led to interest in the drugs, even
though the reports lacked the scientific controls needed to determine
whether the drugs actually worked. The French study was later
discredited.

Scientists have urged that the drugs be tested in controlled clinical
trials to find out definitively whether they can fight the coronavirus
or quell overreactions by the immune system that can become
life-threatening. Those studies are underway in the United States and
around the world.

Dr. Anthony Fauci, director of the National Institute of Allergy and
Infectious Diseases and a member of the president's coronavirus task
force, has not endorsed the drugs, but has consistently said that
scientific evidence is essential to find out whether they work.

``I have been very clear of the importance of doing randomized
controlled trials to definitively prove whether something is both safe
and effective,'' he said in an interview.

\href{https://www.nytimes.com/news-event/coronavirus?action=click\&pgtype=Article\&state=default\&region=MAIN_CONTENT_3\&context=storylines_faq}{}

\hypertarget{the-coronavirus-outbreak-}{%
\subsubsection{The Coronavirus Outbreak
›}\label{the-coronavirus-outbreak-}}

\hypertarget{frequently-asked-questions}{%
\paragraph{Frequently Asked
Questions}\label{frequently-asked-questions}}

Updated July 27, 2020

\begin{itemize}
\item ~
  \hypertarget{should-i-refinance-my-mortgage}{%
  \paragraph{Should I refinance my
  mortgage?}\label{should-i-refinance-my-mortgage}}

  \begin{itemize}
  \tightlist
  \item
    \href{https://www.nytimes.com/article/coronavirus-money-unemployment.html?action=click\&pgtype=Article\&state=default\&region=MAIN_CONTENT_3\&context=storylines_faq}{It
    could be a good idea,} because mortgage rates have
    \href{https://www.nytimes.com/2020/07/16/business/mortgage-rates-below-3-percent.html?action=click\&pgtype=Article\&state=default\&region=MAIN_CONTENT_3\&context=storylines_faq}{never
    been lower.} Refinancing requests have pushed mortgage applications
    to some of the highest levels since 2008, so be prepared to get in
    line. But defaults are also up, so if you're thinking about buying a
    home, be aware that some lenders have tightened their standards.
  \end{itemize}
\item ~
  \hypertarget{what-is-school-going-to-look-like-in-september}{%
  \paragraph{What is school going to look like in
  September?}\label{what-is-school-going-to-look-like-in-september}}

  \begin{itemize}
  \tightlist
  \item
    It is unlikely that many schools will return to a normal schedule
    this fall, requiring the grind of
    \href{https://www.nytimes.com/2020/06/05/us/coronavirus-education-lost-learning.html?action=click\&pgtype=Article\&state=default\&region=MAIN_CONTENT_3\&context=storylines_faq}{online
    learning},
    \href{https://www.nytimes.com/2020/05/29/us/coronavirus-child-care-centers.html?action=click\&pgtype=Article\&state=default\&region=MAIN_CONTENT_3\&context=storylines_faq}{makeshift
    child care} and
    \href{https://www.nytimes.com/2020/06/03/business/economy/coronavirus-working-women.html?action=click\&pgtype=Article\&state=default\&region=MAIN_CONTENT_3\&context=storylines_faq}{stunted
    workdays} to continue. California's two largest public school
    districts --- Los Angeles and San Diego --- said on July 13, that
    \href{https://www.nytimes.com/2020/07/13/us/lausd-san-diego-school-reopening.html?action=click\&pgtype=Article\&state=default\&region=MAIN_CONTENT_3\&context=storylines_faq}{instruction
    will be remote-only in the fall}, citing concerns that surging
    coronavirus infections in their areas pose too dire a risk for
    students and teachers. Together, the two districts enroll some
    825,000 students. They are the largest in the country so far to
    abandon plans for even a partial physical return to classrooms when
    they reopen in August. For other districts, the solution won't be an
    all-or-nothing approach.
    \href{https://bioethics.jhu.edu/research-and-outreach/projects/eschool-initiative/school-policy-tracker/}{Many
    systems}, including the nation's largest, New York City, are
    devising
    \href{https://www.nytimes.com/2020/06/26/us/coronavirus-schools-reopen-fall.html?action=click\&pgtype=Article\&state=default\&region=MAIN_CONTENT_3\&context=storylines_faq}{hybrid
    plans} that involve spending some days in classrooms and other days
    online. There's no national policy on this yet, so check with your
    municipal school system regularly to see what is happening in your
    community.
  \end{itemize}
\item ~
  \hypertarget{is-the-coronavirus-airborne}{%
  \paragraph{Is the coronavirus
  airborne?}\label{is-the-coronavirus-airborne}}

  \begin{itemize}
  \tightlist
  \item
    The coronavirus
    \href{https://www.nytimes.com/2020/07/04/health/239-experts-with-one-big-claim-the-coronavirus-is-airborne.html?action=click\&pgtype=Article\&state=default\&region=MAIN_CONTENT_3\&context=storylines_faq}{can
    stay aloft for hours in tiny droplets in stagnant air}, infecting
    people as they inhale, mounting scientific evidence suggests. This
    risk is highest in crowded indoor spaces with poor ventilation, and
    may help explain super-spreading events reported in meatpacking
    plants, churches and restaurants.
    \href{https://www.nytimes.com/2020/07/06/health/coronavirus-airborne-aerosols.html?action=click\&pgtype=Article\&state=default\&region=MAIN_CONTENT_3\&context=storylines_faq}{It's
    unclear how often the virus is spread} via these tiny droplets, or
    aerosols, compared with larger droplets that are expelled when a
    sick person coughs or sneezes, or transmitted through contact with
    contaminated surfaces, said Linsey Marr, an aerosol expert at
    Virginia Tech. Aerosols are released even when a person without
    symptoms exhales, talks or sings, according to Dr. Marr and more
    than 200 other experts, who
    \href{https://academic.oup.com/cid/article/doi/10.1093/cid/ciaa939/5867798}{have
    outlined the evidence in an open letter to the World Health
    Organization}.
  \end{itemize}
\item ~
  \hypertarget{what-are-the-symptoms-of-coronavirus}{%
  \paragraph{What are the symptoms of
  coronavirus?}\label{what-are-the-symptoms-of-coronavirus}}

  \begin{itemize}
  \tightlist
  \item
    Common symptoms
    \href{https://www.nytimes.com/article/symptoms-coronavirus.html?action=click\&pgtype=Article\&state=default\&region=MAIN_CONTENT_3\&context=storylines_faq}{include
    fever, a dry cough, fatigue and difficulty breathing or shortness of
    breath.} Some of these symptoms overlap with those of the flu,
    making detection difficult, but runny noses and stuffy sinuses are
    less common.
    \href{https://www.nytimes.com/2020/04/27/health/coronavirus-symptoms-cdc.html?action=click\&pgtype=Article\&state=default\&region=MAIN_CONTENT_3\&context=storylines_faq}{The
    C.D.C. has also} added chills, muscle pain, sore throat, headache
    and a new loss of the sense of taste or smell as symptoms to look
    out for. Most people fall ill five to seven days after exposure, but
    symptoms may appear in as few as two days or as many as 14 days.
  \end{itemize}
\item ~
  \hypertarget{does-asymptomatic-transmission-of-covid-19-happen}{%
  \paragraph{Does asymptomatic transmission of Covid-19
  happen?}\label{does-asymptomatic-transmission-of-covid-19-happen}}

  \begin{itemize}
  \tightlist
  \item
    So far, the evidence seems to show it does. A widely cited
    \href{https://www.nature.com/articles/s41591-020-0869-5}{paper}
    published in April suggests that people are most infectious about
    two days before the onset of coronavirus symptoms and estimated that
    44 percent of new infections were a result of transmission from
    people who were not yet showing symptoms. Recently, a top expert at
    the World Health Organization stated that transmission of the
    coronavirus by people who did not have symptoms was ``very rare,''
    \href{https://www.nytimes.com/2020/06/09/world/coronavirus-updates.html?action=click\&pgtype=Article\&state=default\&region=MAIN_CONTENT_3\&context=storylines_faq\#link-1f302e21}{but
    she later walked back that statement.}
  \end{itemize}
\end{itemize}

A report on Friday, from doctors in New York, adds to concerns about
combining hydroxychloroquine and azithromycin. In 84 hospitalized
patients receiving the drugs, electrocardiograms found a rhythm
disruption called a prolonged QTc interval a few days after the
treatment began. In nine cases the disorder was severe, reaching levels
known to increase the risk of sudden death. None of the patients died
from heart problems, however.

Patients given the combination should be carefully monitored, especially
if they have other chronic conditions and if they are also receiving
other drugs known to affect heart rhythm, the doctors, from NYU Langone
Health, said in a
\href{https://www.nature.com/articles/s41591-020-0888-2}{letter to the
journal Nature Medicine.}

Many of the 84 patients had other health problems, including 65 percent
with high blood pressure and 20 percent with diabetes. Their ages ranged
from 18 to 88, with an average of 63, and 74 percent were male. Many
hospitals are reporting that the disease appears more serious in men
than in women.

The doctors suggested that the underlying illnesses and the severity of
the coronavirus infection may have made the patients especially
vulnerable to the cardiac effects of the combined drugs.

Their study was peer reviewed but did not include a comparison group of
patients who did not receive the drugs, to see if their heart rhythm
changed as the disease progressed.

Another
\href{https://www.medrxiv.org/content/10.1101/2020.04.16.20065920v1}{study,
analyzing the records of 368 Veterans Affairs patients}, posted on
Tuesday but not yet peer-reviewed, found that hydroxychloroquine, with
or without azithromycin, did not help patients avoid the need for
ventilators. And hydroxychloroquine alone was associated with an
increased risk of death.

But the study was not a controlled trial, was not peer-reviewed and
patients who received the drugs were sicker to begin with. The authors
wrote, ``These findings highlight the importance of awaiting the results
of ongoing prospective, randomized, controlled studies before widespread
adoption of these drugs.''

Advertisement

\protect\hyperlink{after-bottom}{Continue reading the main story}

\hypertarget{site-index}{%
\subsection{Site Index}\label{site-index}}

\hypertarget{site-information-navigation}{%
\subsection{Site Information
Navigation}\label{site-information-navigation}}

\begin{itemize}
\tightlist
\item
  \href{https://help.nytimes.com/hc/en-us/articles/115014792127-Copyright-notice}{©~2020~The
  New York Times Company}
\end{itemize}

\begin{itemize}
\tightlist
\item
  \href{https://www.nytco.com/}{NYTCo}
\item
  \href{https://help.nytimes.com/hc/en-us/articles/115015385887-Contact-Us}{Contact
  Us}
\item
  \href{https://www.nytco.com/careers/}{Work with us}
\item
  \href{https://nytmediakit.com/}{Advertise}
\item
  \href{http://www.tbrandstudio.com/}{T Brand Studio}
\item
  \href{https://www.nytimes.com/privacy/cookie-policy\#how-do-i-manage-trackers}{Your
  Ad Choices}
\item
  \href{https://www.nytimes.com/privacy}{Privacy}
\item
  \href{https://help.nytimes.com/hc/en-us/articles/115014893428-Terms-of-service}{Terms
  of Service}
\item
  \href{https://help.nytimes.com/hc/en-us/articles/115014893968-Terms-of-sale}{Terms
  of Sale}
\item
  \href{https://spiderbites.nytimes.com}{Site Map}
\item
  \href{https://help.nytimes.com/hc/en-us}{Help}
\item
  \href{https://www.nytimes.com/subscription?campaignId=37WXW}{Subscriptions}
\end{itemize}
