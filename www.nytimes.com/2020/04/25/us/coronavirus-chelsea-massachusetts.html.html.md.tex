Sections

SEARCH

\protect\hyperlink{site-content}{Skip to
content}\protect\hyperlink{site-index}{Skip to site index}

\href{https://www.nytimes.com/section/us}{U.S.}

\href{https://myaccount.nytimes.com/auth/login?response_type=cookie\&client_id=vi}{}

\href{https://www.nytimes.com/section/todayspaper}{Today's Paper}

\href{/section/us}{U.S.}\textbar{}In a Crowded City, Leaders Struggle to
Separate the Sick From the Well

\url{https://nyti.ms/3aw9Wtq}

\begin{itemize}
\item
\item
\item
\item
\item
\end{itemize}

\href{https://www.nytimes.com/news-event/coronavirus?action=click\&pgtype=Article\&state=default\&region=TOP_BANNER\&context=storylines_menu}{The
Coronavirus Outbreak}

\begin{itemize}
\tightlist
\item
  live\href{https://www.nytimes.com/2020/08/01/world/coronavirus-covid-19.html?action=click\&pgtype=Article\&state=default\&region=TOP_BANNER\&context=storylines_menu}{Latest
  Updates}
\item
  \href{https://www.nytimes.com/interactive/2020/us/coronavirus-us-cases.html?action=click\&pgtype=Article\&state=default\&region=TOP_BANNER\&context=storylines_menu}{Maps
  and Cases}
\item
  \href{https://www.nytimes.com/interactive/2020/science/coronavirus-vaccine-tracker.html?action=click\&pgtype=Article\&state=default\&region=TOP_BANNER\&context=storylines_menu}{Vaccine
  Tracker}
\item
  \href{https://www.nytimes.com/interactive/2020/07/29/us/schools-reopening-coronavirus.html?action=click\&pgtype=Article\&state=default\&region=TOP_BANNER\&context=storylines_menu}{What
  School May Look Like}
\item
  \href{https://www.nytimes.com/live/2020/07/31/business/stock-market-today-coronavirus?action=click\&pgtype=Article\&state=default\&region=TOP_BANNER\&context=storylines_menu}{Economy}
\end{itemize}

Advertisement

\protect\hyperlink{after-top}{Continue reading the main story}

Supported by

\protect\hyperlink{after-sponsor}{Continue reading the main story}

\hypertarget{in-a-crowded-city-leaders-struggle-to-separate-the-sick-from-the-well}{%
\section{In a Crowded City, Leaders Struggle to Separate the Sick From
the
Well}\label{in-a-crowded-city-leaders-struggle-to-separate-the-sick-from-the-well}}

Chelsea, Mass., has an infection rate higher than any other community in
the state. With families in cramped housing, it is difficult to contain
the spread.

\includegraphics{https://static01.nyt.com/images/2020/04/22/us/00virus-chelsea01/merlin_171695355_c735aa1f-adf7-4683-933f-c52579dd6826-articleLarge.jpg?quality=75\&auto=webp\&disable=upscale}

\href{https://www.nytimes.com/by/ellen-barry}{\includegraphics{https://static01.nyt.com/images/2018/10/08/multimedia/author-ellen-barry/author-ellen-barry-thumbLarge.png}}

By \href{https://www.nytimes.com/by/ellen-barry}{Ellen Barry}

\begin{itemize}
\item
  Published April 25, 2020Updated April 28, 2020
\item
  \begin{itemize}
  \item
  \item
  \item
  \item
  \item
  \end{itemize}
\end{itemize}

CHELSEA, Mass. --- Paul Nowicki, the director of operations for the
housing authority in this small, crowded immigrant city, walked the
halls of the Buckley Apartments last week in a plastic face shield and
white gown, trying to stop an invisible predator.

Chelsea is the epicenter of
\href{https://www.nytimes.com/interactive/2020/us/massachusetts-coronavirus-cases.html}{the
coronavirus crisis in Massachusetts}, with rates of infection that
surged last week to 3,841 per 100,000 people, around
\href{https://www.mass.gov/doc/confirmed-covid-19-cases-in-ma-by-citytown-january-1-2020-april-22-2020-pdf/download}{six
times the statewide average}. And officials fear the virus is still
spreading.

Take Mr. Nowicki: There were nine confirmed cases of the virus in the
Buckley Apartments, tucked among eight floors of public housing. Mr.
Nowicki had ordered waves of deep-cleaning, wiping of railings and
elevator buttons. He watched the residents shuffle in and out of the
lobby, mostly grandparents, fragile and disabled. It was his job to
safeguard them.

But how could he do that when, because of medical privacy laws, he did
not know where the nine infected people lived? ``It's the specificity of
the floor you'd like to know,'' Mr. Nowicki said. ``Like, are the cases
on the 7th floor or the 9th floor? Are all the infections on one floor?
Or is it spread along all the floors? You'd like to know.''

He is not the only one. Residents call Mr. Nowicki's wife, Tracy, the
city's director of elder services, demanding to know who in their
building is positive, and she gently deters them.

``They want to make sure they don't knock on their door,'' she said. ``I
totally understand that. I totally understand why the residents that are
still healthy want to stay that way.''

As the virus spreads through American communities, many leaders will
face the same stubborn challenge: How, in a country that values its
citizens' medical privacy and autonomy, can authorities separate the
sick from the well?

The question is an urgent one if public life is to resume.

Chinese cities solved this problem by giving infected people no choice.
In the city of Wuhan, authorities realized that social distancing was
not enough to
\href{https://threader.app/thread/1251276061398220800}{rapidly bring the
virus's reproduction rate down to near zero}, which they felt was
necessary to reopen schools and businesses.

Household transmission represented the bulk of new cases. So when people
had mild symptoms, or were known to have been exposed, they were
\href{https://www.telegraph.co.uk/global-health/science-and-disease/isolate-isolate-isolate-chinas-approach-covid-19-quarantine/}{removed
to vast quarantine centers}. There, they were medically monitored and
provided with food, until two successive tests showed they were not
infectious.

Two months of this regimen brought the number of new confirmed cases to
nearly zero, said
\href{https://docs.google.com/presentation/d/1-rvZs0zsXF_0Tw8TNsBxKH4V1LQQXq7Az9kDfCgZDfE/edit\#slide=id.p1}{Xihong
Lin}, a biostatistician at Harvard's
\href{https://www.hsph.harvard.edu/xihong-lin/}{T.H. Chan School of
Public Health}. But the United States, she said, will have to encourage
sick people to separate from family voluntarily.

\hypertarget{latest-updates-global-coronavirus-outbreak}{%
\section{\texorpdfstring{\href{https://www.nytimes.com/2020/08/01/world/coronavirus-covid-19.html?action=click\&pgtype=Article\&state=default\&region=MAIN_CONTENT_1\&context=storylines_live_updates}{Latest
Updates: Global Coronavirus
Outbreak}}{Latest Updates: Global Coronavirus Outbreak}}\label{latest-updates-global-coronavirus-outbreak}}

Updated 2020-08-02T07:42:09.613Z

\begin{itemize}
\tightlist
\item
  \href{https://www.nytimes.com/2020/08/01/world/coronavirus-covid-19.html?action=click\&pgtype=Article\&state=default\&region=MAIN_CONTENT_1\&context=storylines_live_updates\#link-34047410}{The
  U.S. reels as July cases more than double the total of any other
  month.}
\item
  \href{https://www.nytimes.com/2020/08/01/world/coronavirus-covid-19.html?action=click\&pgtype=Article\&state=default\&region=MAIN_CONTENT_1\&context=storylines_live_updates\#link-780ec966}{Top
  U.S. officials work to break an impasse over the federal jobless
  benefit.}
\item
  \href{https://www.nytimes.com/2020/08/01/world/coronavirus-covid-19.html?action=click\&pgtype=Article\&state=default\&region=MAIN_CONTENT_1\&context=storylines_live_updates\#link-2bc8948}{Its
  outbreak untamed, Melbourne goes into even greater lockdown.}
\end{itemize}

\href{https://www.nytimes.com/2020/08/01/world/coronavirus-covid-19.html?action=click\&pgtype=Article\&state=default\&region=MAIN_CONTENT_1\&context=storylines_live_updates}{See
more updates}

More live coverage:
\href{https://www.nytimes.com/live/2020/07/31/business/stock-market-today-coronavirus?action=click\&pgtype=Article\&state=default\&region=MAIN_CONTENT_1\&context=storylines_live_updates}{Markets}

``Western countries are different from Asian countries,'' she said.
``One cannot force people to do things.''

Chelsea, a city of 40,000 people crammed into less than two square
miles,\href{https://www.mass.gov/doc/confirmed-covid-19-cases-in-ma-by-citytown-january-1-2020-april-22-2020-pdf/download}{has
1,447 confirmed cases,} according to state data, by far the highest rate
in Massachusetts.

And those confirmed cases represent only the tip of an iceberg. Last
weekend, when researchers from Massachusetts General Hospital
\href{https://www.bostonglobe.com/2020/04/17/business/nearly-third-200-blood-samples-taken-chelsea-show-exposure-coronavirus/}{conducted
antibodies tests on 200 apparently healthy pedestrians in Chelsea,
selected at random,} nearly
\href{http://chelsearecord.com/2020/04/23/mgh-researcher-finds-interesting-results-here/}{a
third of them tested positive}, suggesting that many had been infected
without knowing it.

``They may infect other people around them who are high risk,'' said
John Iafrate, vice chairman of Massachusetts General's pathology
department and \href{https://www.youtube.com/watch?v=GLQ9Alai5Ps}{the
study's principal investigator}. ``That is a very, very serious
infection control issue.''

This month, city officials began offering people who tested positive for
the virus the option of moving into a 157-room hotel in nearby Revere,
to avoid infecting their family members or housemates. Ten days later,
though, only 14 people from Chelsea are staying there.

``We were expecting the floodgates to open,'' said Alexander Train, the
assistant director of the city's Department of Planning and Development.
He said undocumented immigrants may be afraid to take advantage of the
offer, fearing it would lead to deportation.

``I think there is some uncertainty and anxiety that is inhibiting the
flow of guests to the hotel, because it is attributed to the
government,'' he said. ``It's about, `what if I don't make it back to my
family?'''

Most people, given a choice, will stay home, despite the risk of
infection, said Roy Avellaneda, the City Council president, who said he
tried in vain to persuade an employee at the restaurant he runs, who
risked infecting her family members, to check into a hotel.

``For all the love we have in this country,'' Mr. Avellaneda said, ``the
reason we're probably going to be hit sicker is that we still have a
government that cannot make those decisions for the benefit of its
residents.''

\includegraphics{https://static01.nyt.com/images/2020/04/26/us/26virus-chelsea02/merlin_171686601_7440f465-6d38-4502-ad64-5b6e9d75ad4c-articleLarge.jpg?quality=75\&auto=webp\&disable=upscale}

\hypertarget{the-air-would-have-been-contaminated}{%
\subsection{`The Air Would Have Been
Contaminated'}\label{the-air-would-have-been-contaminated}}

Separated from Boston by the Mystic River, Chelsea is a world apart, a
first stop for immigrant families --- Lithuanian, Polish, Irish, and
more recently Honduran and Guatemalan --- who cannot afford the bigger
city's sky-high rents.

It has a population density of nearly 17,000 people per square mile,
with whole families crowding into single rooms in triple-decker
rowhouses, buildings with high rates of lead paint, asbestos and air
pollution.

Katharine Robb, a researcher at the Harvard Kennedy School who spent a
summer following housing inspectors in Chelsea, was
\href{https://dash.harvard.edu/handle/1/40976724}{stunned by what she
found} --- families living on porches, in unfinished basements or even
closets, without access to running water, heat or sanitation.

``I didn't think conditions like this were happening in the 21st
century,'' she said. ``It reminded me of stories I heard of the late
1800s, at the beginning of sanitary reform, at the beginning of
urbanization.''

This spring, the fast-spreading virus collided disastrously with the
city's overcrowded housing. A warning flare came in the second week of
April, when, late at night, a young mother called the city housing
authority from the street; she had disclosed her test results to her
roommates, and they had kicked her out. ``It dawned on me that this
situation was going to replicate itself,'' said Thomas Ambrosino,
Chelsea's city manager, ``and we better have a solution.''

Over the weeks that came after, some of the sick isolated themselves.
One man, worried about infecting his family, slept in his car for two
days, until his relatives sought help from the city.

Gladys Vega, a longtime community activist, helped a man who had been
banished to a freezing, unfinished dirt basement, where he was riding
out the illness on a piece of cardboard. Another man had been sent to
sleep on a porch, despite temperatures that still dropped below freezing
at night.

``People are being treated as if they have leprosy,'' said Ms. Vega,
\href{https://www.chelseacollab.org/meet-the-team-1}{executive director
of the Chelsea Collaborative.}

Others did their best to ride out the virus in small spaces. Marisol
Lima, 35, was eight days from her move-out date, in the tiny room she
rented from a Colombian family, when she noticed that her downstairs
neighbor was coughing. Within days, six of the seven people in the
apartment were seriously ill, feverish, breathing with difficulty.

``I think it was impossible not to get the virus,'' she said. ``It was a
very small living space. The air would have been contaminated.''

\href{https://www.nytimes.com/news-event/coronavirus?action=click\&pgtype=Article\&state=default\&region=MAIN_CONTENT_3\&context=storylines_faq}{}

\hypertarget{the-coronavirus-outbreak-}{%
\subsubsection{The Coronavirus Outbreak
›}\label{the-coronavirus-outbreak-}}

\hypertarget{frequently-asked-questions}{%
\paragraph{Frequently Asked
Questions}\label{frequently-asked-questions}}

Updated July 27, 2020

\begin{itemize}
\item ~
  \hypertarget{should-i-refinance-my-mortgage}{%
  \paragraph{Should I refinance my
  mortgage?}\label{should-i-refinance-my-mortgage}}

  \begin{itemize}
  \tightlist
  \item
    \href{https://www.nytimes.com/article/coronavirus-money-unemployment.html?action=click\&pgtype=Article\&state=default\&region=MAIN_CONTENT_3\&context=storylines_faq}{It
    could be a good idea,} because mortgage rates have
    \href{https://www.nytimes.com/2020/07/16/business/mortgage-rates-below-3-percent.html?action=click\&pgtype=Article\&state=default\&region=MAIN_CONTENT_3\&context=storylines_faq}{never
    been lower.} Refinancing requests have pushed mortgage applications
    to some of the highest levels since 2008, so be prepared to get in
    line. But defaults are also up, so if you're thinking about buying a
    home, be aware that some lenders have tightened their standards.
  \end{itemize}
\item ~
  \hypertarget{what-is-school-going-to-look-like-in-september}{%
  \paragraph{What is school going to look like in
  September?}\label{what-is-school-going-to-look-like-in-september}}

  \begin{itemize}
  \tightlist
  \item
    It is unlikely that many schools will return to a normal schedule
    this fall, requiring the grind of
    \href{https://www.nytimes.com/2020/06/05/us/coronavirus-education-lost-learning.html?action=click\&pgtype=Article\&state=default\&region=MAIN_CONTENT_3\&context=storylines_faq}{online
    learning},
    \href{https://www.nytimes.com/2020/05/29/us/coronavirus-child-care-centers.html?action=click\&pgtype=Article\&state=default\&region=MAIN_CONTENT_3\&context=storylines_faq}{makeshift
    child care} and
    \href{https://www.nytimes.com/2020/06/03/business/economy/coronavirus-working-women.html?action=click\&pgtype=Article\&state=default\&region=MAIN_CONTENT_3\&context=storylines_faq}{stunted
    workdays} to continue. California's two largest public school
    districts --- Los Angeles and San Diego --- said on July 13, that
    \href{https://www.nytimes.com/2020/07/13/us/lausd-san-diego-school-reopening.html?action=click\&pgtype=Article\&state=default\&region=MAIN_CONTENT_3\&context=storylines_faq}{instruction
    will be remote-only in the fall}, citing concerns that surging
    coronavirus infections in their areas pose too dire a risk for
    students and teachers. Together, the two districts enroll some
    825,000 students. They are the largest in the country so far to
    abandon plans for even a partial physical return to classrooms when
    they reopen in August. For other districts, the solution won't be an
    all-or-nothing approach.
    \href{https://bioethics.jhu.edu/research-and-outreach/projects/eschool-initiative/school-policy-tracker/}{Many
    systems}, including the nation's largest, New York City, are
    devising
    \href{https://www.nytimes.com/2020/06/26/us/coronavirus-schools-reopen-fall.html?action=click\&pgtype=Article\&state=default\&region=MAIN_CONTENT_3\&context=storylines_faq}{hybrid
    plans} that involve spending some days in classrooms and other days
    online. There's no national policy on this yet, so check with your
    municipal school system regularly to see what is happening in your
    community.
  \end{itemize}
\item ~
  \hypertarget{is-the-coronavirus-airborne}{%
  \paragraph{Is the coronavirus
  airborne?}\label{is-the-coronavirus-airborne}}

  \begin{itemize}
  \tightlist
  \item
    The coronavirus
    \href{https://www.nytimes.com/2020/07/04/health/239-experts-with-one-big-claim-the-coronavirus-is-airborne.html?action=click\&pgtype=Article\&state=default\&region=MAIN_CONTENT_3\&context=storylines_faq}{can
    stay aloft for hours in tiny droplets in stagnant air}, infecting
    people as they inhale, mounting scientific evidence suggests. This
    risk is highest in crowded indoor spaces with poor ventilation, and
    may help explain super-spreading events reported in meatpacking
    plants, churches and restaurants.
    \href{https://www.nytimes.com/2020/07/06/health/coronavirus-airborne-aerosols.html?action=click\&pgtype=Article\&state=default\&region=MAIN_CONTENT_3\&context=storylines_faq}{It's
    unclear how often the virus is spread} via these tiny droplets, or
    aerosols, compared with larger droplets that are expelled when a
    sick person coughs or sneezes, or transmitted through contact with
    contaminated surfaces, said Linsey Marr, an aerosol expert at
    Virginia Tech. Aerosols are released even when a person without
    symptoms exhales, talks or sings, according to Dr. Marr and more
    than 200 other experts, who
    \href{https://academic.oup.com/cid/article/doi/10.1093/cid/ciaa939/5867798}{have
    outlined the evidence in an open letter to the World Health
    Organization}.
  \end{itemize}
\item ~
  \hypertarget{what-are-the-symptoms-of-coronavirus}{%
  \paragraph{What are the symptoms of
  coronavirus?}\label{what-are-the-symptoms-of-coronavirus}}

  \begin{itemize}
  \tightlist
  \item
    Common symptoms
    \href{https://www.nytimes.com/article/symptoms-coronavirus.html?action=click\&pgtype=Article\&state=default\&region=MAIN_CONTENT_3\&context=storylines_faq}{include
    fever, a dry cough, fatigue and difficulty breathing or shortness of
    breath.} Some of these symptoms overlap with those of the flu,
    making detection difficult, but runny noses and stuffy sinuses are
    less common.
    \href{https://www.nytimes.com/2020/04/27/health/coronavirus-symptoms-cdc.html?action=click\&pgtype=Article\&state=default\&region=MAIN_CONTENT_3\&context=storylines_faq}{The
    C.D.C. has also} added chills, muscle pain, sore throat, headache
    and a new loss of the sense of taste or smell as symptoms to look
    out for. Most people fall ill five to seven days after exposure, but
    symptoms may appear in as few as two days or as many as 14 days.
  \end{itemize}
\item ~
  \hypertarget{does-asymptomatic-transmission-of-covid-19-happen}{%
  \paragraph{Does asymptomatic transmission of Covid-19
  happen?}\label{does-asymptomatic-transmission-of-covid-19-happen}}

  \begin{itemize}
  \tightlist
  \item
    So far, the evidence seems to show it does. A widely cited
    \href{https://www.nature.com/articles/s41591-020-0869-5}{paper}
    published in April suggests that people are most infectious about
    two days before the onset of coronavirus symptoms and estimated that
    44 percent of new infections were a result of transmission from
    people who were not yet showing symptoms. Recently, a top expert at
    the World Health Organization stated that transmission of the
    coronavirus by people who did not have symptoms was ``very rare,''
    \href{https://www.nytimes.com/2020/06/09/world/coronavirus-updates.html?action=click\&pgtype=Article\&state=default\&region=MAIN_CONTENT_3\&context=storylines_faq\#link-1f302e21}{but
    she later walked back that statement.}
  \end{itemize}
\end{itemize}

Wendy Rosales, a restaurant kitchen manager, discovered she had the
virus when she went to the doctor for an earache, and tried to isolate
from her husband and daughters by staying in a bedroom. But the room had
no lock and her three-year-old toddled into the room any time her
father's attention drifted, running to hug her mother. When Ms. Rosales
pushed her out and shut the door, the girl would stand outside, crying.

``I didn't sleep,'' she said. ``I spent almost the whole night thinking
what to do, and thinking about the little one --- she's young and
doesn't understand what's happening.''

Image

Children watched as emergency medical workers responded to a call in
their neighborhood in Chelsea.Credit...Brian Snyder/Reuters

\hypertarget{a-safety-valve}{%
\subsection{A Safety Valve}\label{a-safety-valve}}

The opening of the Quality Inn in Revere last week provided the city
with a safety valve. Those who checked in would get three meals a day
and medical monitoring, but would not be allowed to leave their rooms,
or leave until cleared by medical staff.

Ms. Rosales was one of the first to check in.

After her night of crying, she stuffed a few possessions into a backpack
and left home without saying goodbye, or even looking back.
\href{https://www.cbsnews.com/news/contact-tracing-deployed-in-effort-to-fight-coronavirus/}{Now
in her seventh day at the Quality Inn}, she is both intensely homesick
and intensely relieved.

``I was terrified I was going to infect them,'' she said. ``It was the
best decision.''

But many in the city are choosing to ride out the illness at home.

For Mr. Ambrosino, the city manager, it is part of a bigger problem for
the city of Chelsea: Infected people must be persuaded to take difficult
steps --- like social distancing and isolating in the hotel ---
themselves. ``We're not going to engage in violent physical sealing of
doors, that's not how we operate in the U. S. of A.,'' he said.

He said compliance rates were very good, as high as 95 percent, but that
left a significant number to spread the virus.'' If the compliance rates
were 95 percent, that means that I have 2,000 nitwits out on the
street,'' he said.

Another obstacle, he said, are medical privacy laws that can prevent the
city's public health staff from disclosing who, in this crowded city,
has tested positive.

``She can't call me and say, `I've got someone in Unit 6 who is
positive,''' Mr. Ambrosino said. ``I can't know their names or
addresses. She can say, `the city is ready to help you, call this help
line,' but if that person doesn't call, because they don't understand,
or they're afraid of the government, there is no way for us to help
them.''

``Unless someone tells us they're Covid-19 positive, I have no way of
knowing,'' he said.

And many sick people are withholding their status out of fear. Earlier
this month, Maria Belen
Power,\href{http://www.greenrootschelsea.org/team}{a community
activist}, found herself begging an undocumented friend, Floridalma
Ochoa, to call 911. Ms. Power was weeping; her friend had spent the
night gasping for air.

``They just didn't want to call, because they were afraid,'' said Ms.
Power, associate executive director of
\href{http://www.greenrootschelsea.org/}{GreenRoots}, an environmental
justice organization. ``She kept saying, `but what if they ask me for
papers?' Honestly, I thought she could die. I was saying, `You have to
call. You're losing time.''

Ms. Lima, who was infected by one of her roommates, said many people she
knows don't want to reveal that they have the illness. ``Fear exists
heavily among the Latino people,'' she said. ``A lot of people do not
want to speak, or even accept that they are sick, because they are
scared of how the rest of the people will look at them.''

And many simply cannot fathom leaving sick relatives alone. Ruth
Gabriela Santos, 34, is counting the days until her mother, Ms. Ochoa,
is released from the hospital. She has been on a respirator for three
weeks, a period during which the city was transformed.

``The impact is terrible,'' said Ms. Santos. ``Knowing you are at the
epicenter, looking around and realizing how many people have died. And
that the numbers of infections are not only increasing, but doubling and
tripling. Learning that people you've known your whole life are
infected.''

Bringing her mother home, into a four-room apartment with eight other
people, comes with the risk that she is contagious. But Ms. Santos
cannot bear to think of her mother's loneliness if she should be moved
to the hotel. She wants to cook for her.

``Our parents had their time,'' Ms. Santos said. ``They gave everything
to us. Now it is my time to return it, and take care of them.''

Vanessa Swales contributed reporting from New York.

Advertisement

\protect\hyperlink{after-bottom}{Continue reading the main story}

\hypertarget{site-index}{%
\subsection{Site Index}\label{site-index}}

\hypertarget{site-information-navigation}{%
\subsection{Site Information
Navigation}\label{site-information-navigation}}

\begin{itemize}
\tightlist
\item
  \href{https://help.nytimes.com/hc/en-us/articles/115014792127-Copyright-notice}{©~2020~The
  New York Times Company}
\end{itemize}

\begin{itemize}
\tightlist
\item
  \href{https://www.nytco.com/}{NYTCo}
\item
  \href{https://help.nytimes.com/hc/en-us/articles/115015385887-Contact-Us}{Contact
  Us}
\item
  \href{https://www.nytco.com/careers/}{Work with us}
\item
  \href{https://nytmediakit.com/}{Advertise}
\item
  \href{http://www.tbrandstudio.com/}{T Brand Studio}
\item
  \href{https://www.nytimes.com/privacy/cookie-policy\#how-do-i-manage-trackers}{Your
  Ad Choices}
\item
  \href{https://www.nytimes.com/privacy}{Privacy}
\item
  \href{https://help.nytimes.com/hc/en-us/articles/115014893428-Terms-of-service}{Terms
  of Service}
\item
  \href{https://help.nytimes.com/hc/en-us/articles/115014893968-Terms-of-sale}{Terms
  of Sale}
\item
  \href{https://spiderbites.nytimes.com}{Site Map}
\item
  \href{https://help.nytimes.com/hc/en-us}{Help}
\item
  \href{https://www.nytimes.com/subscription?campaignId=37WXW}{Subscriptions}
\end{itemize}
