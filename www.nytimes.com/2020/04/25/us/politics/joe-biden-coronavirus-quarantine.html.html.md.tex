Sections

SEARCH

\protect\hyperlink{site-content}{Skip to
content}\protect\hyperlink{site-index}{Skip to site index}

\href{https://www.nytimes.com/section/politics}{Politics}

\href{https://myaccount.nytimes.com/auth/login?response_type=cookie\&client_id=vi}{}

\href{https://www.nytimes.com/section/todayspaper}{Today's Paper}

\href{/section/politics}{Politics}\textbar{}A Candidate in Isolation:
Inside Joe Biden's Cloistered Campaign

\url{https://nyti.ms/2yEzw26}

\begin{itemize}
\item
\item
\item
\item
\item
\end{itemize}

\begin{itemize}
\item
  \href{https://www.nytimes.com/2020/07/31/us/elections/biden-vs-trump.html?action=click\&pgtype=Article\&state=default\&region=TOP_BANNER\&context=storylines_menu}{Election
  Updates}
\item
  \href{https://www.nytimes.com/article/biden-vice-president-2020.html?action=click\&pgtype=Article\&state=default\&region=TOP_BANNER\&context=storylines_menu}{Biden's
  V.P. Search}
\item
  \href{https://www.nytimes.com/interactive/2020/07/24/us/politics/trump-biden-campaign-donors.html?action=click\&pgtype=Article\&state=default\&region=TOP_BANNER\&context=storylines_menu}{Map
  of Donations}
\item
  \href{https://www.nytimes.com/interactive/2020/us/elections/delegate-count-primary-results.html?action=click\&pgtype=Article\&state=default\&region=TOP_BANNER\&context=storylines_menu}{Delegate
  Count}
\item
  \href{https://www.nytimes.com/interactive/2019/us/politics/2020-presidential-candidates.html?action=click\&pgtype=Article\&state=default\&region=TOP_BANNER\&context=storylines_menu}{The
  Candidates}
\item
  \href{https://www.nytimes.com/newsletters/politics?action=click\&pgtype=Article\&state=default\&region=TOP_BANNER\&context=storylines_menu}{Politics
  Newsletter}
\end{itemize}

Advertisement

\protect\hyperlink{after-top}{Continue reading the main story}

Supported by

\protect\hyperlink{after-sponsor}{Continue reading the main story}

\hypertarget{a-candidate-in-isolation-inside-joe-bidens-cloistered-campaign}{%
\section{A Candidate in Isolation: Inside Joe Biden's Cloistered
Campaign}\label{a-candidate-in-isolation-inside-joe-bidens-cloistered-campaign}}

Walled off from voters in a distinctive kind of lockdown, Mr. Biden has
developed a routine, of sorts, as he seeks the presidency from his
basement.

\includegraphics{https://static01.nyt.com/images/2020/04/26/us/politics/26biden-quarantine-p1/merlin_171898821_4e9bb493-8d28-49d2-a57b-422e2a622321-articleLarge.jpg?quality=75\&auto=webp\&disable=upscale}

\href{https://www.nytimes.com/by/alexander-burns}{\includegraphics{https://static01.nyt.com/images/2018/09/25/multimedia/author-alexander-burns/author-alexander-burns-thumbLarge-v2.png}}\href{https://www.nytimes.com/by/shane-goldmacher}{\includegraphics{https://static01.nyt.com/images/2018/07/27/multimedia/author-shane-goldmacher/author-shane-goldmacher-thumbLarge.png}}\href{https://www.nytimes.com/by/katie-glueck}{\includegraphics{https://static01.nyt.com/images/2020/01/29/reader-center/author-katie-glueck/author-katie-glueck-thumbLarge.png}}

By \href{https://www.nytimes.com/by/alexander-burns}{Alexander Burns},
\href{https://www.nytimes.com/by/shane-goldmacher}{Shane Goldmacher} and
\href{https://www.nytimes.com/by/katie-glueck}{Katie Glueck}

\begin{itemize}
\item
  Published April 25, 2020Updated April 28, 2020
\item
  \begin{itemize}
  \item
  \item
  \item
  \item
  \item
  \end{itemize}
\end{itemize}

Joseph R. Biden Jr. usually rises before 8 a.m. at his home in
Wilmington, Del., and starts his day with a workout in an upstairs gym
that contains a Peloton bike, weights and a treadmill. He often enjoys a
protein shake for breakfast and puts on a suit ** or blazer much of the
time. In the evenings, he and his wife, Jill, sit down together for
dinner, a ritual that was absent for much of the last frenzied year on
the campaign trail.

In the intervening hours, Mr. Biden attempts to win the presidency
without leaving his house.

With the coronavirus outbreak freezing the country's public life, Mr.
Biden has been forced to adapt to a cloistered mode of campaigning never
before seen in modern American politics. He was unable to embark on a
victory tour after the Democratic primaries or hold unity rallies with
onetime rivals like Senators Bernie Sanders of Vermont and Elizabeth
Warren of Massachusetts. Instead, the former vice president is in a
distinctive kind of lockdown, walled off from voters, separated from his
top strategists and yet leading in the polls.

For a famous backslapper like Mr. Biden, this open-ended period of
captivity has tested both his patience and his political imagination. He
has lamented being deprived of human contact, and he has expressed
exasperation with media coverage critiquing his limited visibility
compared with President Trump's daily performances in the White House
briefing room. He does not make a habit of watching the president's
briefings in full; he is said to be fixated mainly on the eventual
challenge --- if he wins --- of governing amid a pandemic.

Interviews with dozens of people in touch with the presumptive
Democratic nominee and his advisers revealed a newly detailed picture of
Mr. Biden's life in seclusion, one spent in long-distance consultation
with a wide array of coalition leaders helping him map out the fall
campaign and a potential administration.

Mr. Biden has revived many of the rituals of the vice presidency,
including similarly formatted briefing memos and tour d'horizon-style
updates from aides on the virus and the economy --- all aimed at giving
him the information he would need to make the weighty decisions at hand
if he were in charge, except that he is not.

Fran Person, who served for years as a Biden aide and speaks with him
regularly, said the detached lifestyle was unnatural for Mr. Biden, an
extrovert who spent virtually his entire adult life in government.

``I can imagine, for him, you're watching this play out, you know what
needs to get done,'' Mr. Person said. ``You want to be right in the
middle of it.''

\includegraphics{https://static01.nyt.com/images/2020/04/24/us/politics/24biden-quarantine2/merlin_170317503_84db12f2-8733-4e14-b68e-26475901b822-articleLarge.jpg?quality=75\&auto=webp\&disable=upscale}

As the temperature of the campaign rises in public, increasingly
featuring caustic attacks on Mr. Biden from Mr. Trump and his allies and
blunt rebuttals from Mr. Biden's aides, the former vice president has
not attempted to match Mr. Trump blow for blow on television.

For the most part, Mr. Biden is seeking to run a campaign based on
something like digital-age fireside chats, offering himself as a calmly
authoritative figure rather than a brawler like his opponent. In
private, he voices a combination of optimism about American resilience
and recognition that the country is likely to be in a bleak state on
Inauguration Day.

It remains to be seen whether that approach will come to be viewed as
appropriately sober or perilously passive against a tenacious and
unpredictable opponent. Many Democrats remain anxious about the
limitations of Mr. Biden's position, even though Mr. Trump has slipped
markedly in the polls and faces growing disapproval of his response to
the pandemic.

Only a few people have seen Mr. Biden, 77, in the flesh in recent weeks.
He is guarded by the Secret Service, and a pair of trusted staffers
assist with his daily activities. The rare outside visitors don masks
and gloves as a safety measure.

\hypertarget{latest-updates-2020-election}{%
\section{\texorpdfstring{\href{https://www.nytimes.com/2020/07/31/us/elections/biden-vs-trump.html?action=click\&pgtype=Article\&state=default\&region=MAIN_CONTENT_1\&context=storylines_live_updates}{Latest
Updates: 2020
Election}}{Latest Updates: 2020 Election}}\label{latest-updates-2020-election}}

Updated 2020-08-01T01:26:45.732Z

\begin{itemize}
\tightlist
\item
  \href{https://www.nytimes.com/2020/07/31/us/elections/biden-vs-trump.html?action=click\&pgtype=Article\&state=default\&region=MAIN_CONTENT_1\&context=storylines_live_updates\#link-29fdff45}{Kamala
  Harris, a top vice-presidential contender, confronts double
  standards.}
\item
  \href{https://www.nytimes.com/2020/07/31/us/elections/biden-vs-trump.html?action=click\&pgtype=Article\&state=default\&region=MAIN_CONTENT_1\&context=storylines_live_updates\#link-13ec3d9c}{Karen
  Bass and Susan Rice are rising on Biden's vice-presidential
  shortlist.}
\item
  \href{https://www.nytimes.com/2020/07/31/us/elections/biden-vs-trump.html?action=click\&pgtype=Article\&state=default\&region=MAIN_CONTENT_1\&context=storylines_live_updates\#link-49e9a016}{Trump
  says Russian bounties to kill U.S. troops `never took place.'}
\end{itemize}

\href{https://www.nytimes.com/2020/07/31/us/elections/biden-vs-trump.html?action=click\&pgtype=Article\&state=default\&region=MAIN_CONTENT_1\&context=storylines_live_updates}{See
more updates}

Like many professionals these days, the former vice president fills his
time with conference calls. There are at least four standing calls on
his daily schedule, including one with Jennifer O'Malley Dillon, his new
campaign manager. There are daily briefings on the economy, public
health and electoral strategy, and a less frequent session on national
security.

Mr. Biden has used a television-quality video uplink from his
refurbished rec room for interviews and online campaign events. But for
private conversations, he prefers conferring by telephone, usually on
speakerphone in his study. At times, callers deduce from rowdy
background noise that Mr. Biden is working beside his German shepherds,
Major and Champ.

The former vice president also places calls to mayors and governors;
congressional leaders like Representative James E. Clyburn of South
Carolina; elder statesmen like Al Gore; potential running mates; donors;
and former rivals like Mr. Sanders and Ms. Warren. A few governors have
become favorite points of contact, including Andrew M. Cuomo of New
York, Jay Inslee of Washington and Gretchen Whitmer of Michigan.

At his request, Mr. Biden talks at least once daily to a voter or
campaign volunteer --- the kind of people he would meet constantly on
the trail. And he regularly phones allies to express sympathy or
support, including a call to Ms. Warren when he learned that one of her
brothers had died of the coronavirus.

Ms. Whitmer, a potential running mate for Mr. Biden, said the former
vice president had been deeply engaged with the details of the outbreak
in her state. He had offered advice and commiserated over the isolation
brought on by the virus, and how it had barred them from performing
consoling tasks like visiting mourners and medical workers.

``I think that's why he's calling and reaching out and trying to keep a
pulse on what's happening,'' Ms. Whitmer said. ``It's not a great
substitute for personal interaction, but it's a way to stay connected.''

The Biden campaign declined to make him available for an interview. But
the former vice president has at times spoken publicly about his
isolation. ``I'm chomping at the bit,'' Mr. Biden
\href{https://www.nytimes.com/2020/03/25/us/politics/bernie-sanders-joe-biden-next-debate.html}{told
reporters} a month ago. ``I wish I were still in the Senate, you know,
being able to impact on some of these things. But I am where I am.''

For a team that employed a relatively skeletal digital operation
throughout the primaries, the sudden shift toward online campaigning has
been abrupt. At times, Mr. Biden has appeared out of his comfort zone
and he continues to express a kind of chuckling disbelief that his
basement has become a makeshift studio. Advisers acknowledge that they
have considerable catching up to do on sites like Facebook and YouTube.

Mr. Biden is also facing pressure from donors to ramp up his at-home
fund-raising activities, and from leaders in the states who want to see
him beaming more often into key battlegrounds. To that end, he has
recently conducted a series of interviews with local television stations
in markets like Detroit and Pittsburgh, with more planned**.**

But Mr. Biden is burrowing in for the long haul, telling donors this
month he did not anticipate holding traditional public events anytime
soon.

``It's going to be this way,'' he said, ``for a little while.''

Image

Mr. Biden held a ``virtual rope line'' to speak with voters like Ashley
Ruiz in Fort Lauderdale, Fla.Credit...Calla Kessler/The New York Times

\hypertarget{an-extrovert-in-lockdown}{%
\subsection{An Extrovert in Lockdown}\label{an-extrovert-in-lockdown}}

The estate on which Mr. Biden is functionally trapped has long been a
personal refuge. Nestled along a lake and recessed from the road by a
long private drive, the 6,800-square-foot home took more than two years
to build and Mr. Biden has said he designed it himself.

It is a home the Bidens had talked about bequeathing to his son, Beau,
and that Mr. Biden later considered mortgaging or selling to help
support Beau's family as he suffered from cancer. It was at this home
where Mr. Biden worked to refine the 2016 presidential announcement
speech he never delivered.

Today, the **** house has become an almost sealed containment zone. Two
political aides regularly enter and leave the house, according to people
briefed on the safety restrictions put in place: Annie Tomasini, Mr.
Biden's traveling chief of staff, and Anthony Bernal, Jill Biden's chief
of staff, both of whom have worked for the Bidens on and off for more
than a decade.

But several people familiar with their roles said they are not staffing
the Bidens around the clock and it is not clear whether any other aides
assist the candidate at home. Much of the time Mr. Biden answers his own
telephone, and he frequently falls behind his limited public schedule.

The campaign has consulted physicians and health experts about
safeguarding Mr. Biden, who at 77 falls squarely into a high-risk group
for the coronavirus. Irwin Redlener, a clinical professor at Columbia
University's Mailman School of Public Health, said he had spoken with
the campaign about health precautions, including how to handle the
possibility that members of Mr. Biden's traveling staff had been
exposed.

``In terms of the safety of the staff, the candidate, what did they need
to know?'' said Dr. Redlener, who previously served on Mr. Biden's
public health advisory committee.

Mr. Biden has embraced the safety guidelines: He has described in
interviews a careful protocol that allows him to interact with some of
his grandchildren, who live nearby. They come over to play on his lawn,
allowing Mr. Biden and Jill Biden to talk to them and sometimes throw
them candy or ice cream from a short distance.

To interact with voters, his campaign has experimented with virtual town
halls and round tables, but Democrats in the states are anxious to see
more of the candidate.

Milwaukee Mayor Tom Barrett, who recently endorsed Mr. Biden, said he
had prodded the campaign to do more to put him directly in front of
Wisconsin voters.

``It is so critically important for him to have a presence here,'' Mr.
Barrett said. ``I think, in some ways, Zoom and FaceTime --- they're the
2020 counterpart to what President Trump used effectively for his base,
which is Twitter.''

Mr. Biden is working to adapt to those platforms; this past week he
spent half an hour on a Zoom call with a nurse in Wisconsin and then
contacted other members of her family by phone. But targeted
video-chatting offers Mr. Biden only so many opportunities to hear from
voters directly about their struggles and needs.

Ashley Ruiz, a voter in Fort Lauderdale, Fla., who recently participated
in a ``virtual rope line'' with Mr. Biden, said she had found him eager
to share his ideas about education and child care. But Mr. Biden grew
most animated when he detected the presence of her two sons --- ages 10
and 7 --- along with her red-nose pit bull, Kacie.

Mr. Biden, she said, was determined to communicate with her 7-year-old
son, who has autism and, like Mr. Biden, a stutter. ``He said to my son,
`I want you to know you can do anything,''' Ms. Ruiz said, recalling
that Mr. Biden had told her, ``When I'm president, I will care for your
family like they're my family.''

Defining the substance behind that promise is what mainly occupies Mr.
Biden's time.

Image

Mr. Biden regularly calls mayors, governors and former rivals, like
Senator Bernie Sanders, who endorsed him this month after leaving the
race.Credit...Erin Schaff/The New York Times

\hypertarget{seeking-bigger-ideas}{%
\subsection{Seeking Bigger Ideas}\label{seeking-bigger-ideas}}

Even before Mr. Biden entered his state of near-quarantine, he was
telling associates that he feared the onset of a national catastrophe.
In mid-March, Mr. Biden told one confidant that he was concerned that
the country could face another Great Depression, sharing that he had
discussed the possibility with Lawrence H. Summers, the former treasury
secretary.

That dark contingency now looks more plausible than ever. In the daily
briefings he receives about public health and the economy, Mr. Biden
seeks the kind of minute information he would need to make important
policy decisions --- if only he were in a position to do so.

Several participants in the briefings said Mr. Biden probes extensively
about the mechanics of how money and medical resources are being
distributed around the country. Spurred by beleaguered governors, he
regularly presses his team about the steps Washington might take to
shore up shattered state budgets.

``There is that sort of suspended quality to things in that you're not
making a decision that's urgent and that people have to carry out
today,'' said Senator Chris Coons of Delaware, a close Biden ally.
Still, he described ``a real sense of imminence'' because the aides
briefing Mr. Biden in lockdown today could well be managing the
government response in 10 months.

``It's like the relief pitcher warming up in the bullpen, knowing you
only get a couple more pitches and then you're going out on the mound,''
Mr. Coons said.

One of those advisers, Vivek Murthy, the former surgeon general, said
Mr. Biden wanted to stay on top of both the large-scale policies aimed
at containing the virus and on the precise efforts of local governments
and medical facilities. Though most people on the calls are former
government officials, a view from the front lines of medicine comes from
a member of the Biden family: Howard Krein, the former vice president's
son-in-law, who is a doctor in Philadelphia.

In one briefing, Dr. Murthy said it hit Mr. Biden hard to learn that
hospitals were barring people from visiting dying family members. ``He
knows what it's like to lose people and to have your life turned upside
down,'' Dr. Murthy said.

A daily call on the economy and a somewhat less frequent briefing on
national security are stocked with veterans of the Obama administration,
including Ben Harris and Jared Bernstein, who served as economic
advisers to Mr. Biden in the vice presidency, and Antony J. Blinken and
Jake Sullivan, his former national security advisers. Murmurs about Mr.
Summers's quiet role advising Mr. Biden have alarmed some progressives,
who saw the former Harvard president as closely aligned with Wall Street
during the last recession.

It is not clear, however, that any ideological camp has a full claim on
Mr. Biden's ear right now: On the economic calls, Mr. Biden regularly
seeks insight into the thinking of his party's populist wing, inquiring
by name about Ms. Warren, Mr. Sanders and a third liberal, Senator
Sherrod Brown of Ohio.

So far, Mr. Biden's policy huddles have yielded proposals to contain the
immediate damage of the pandemic. But his allies expect he will soon go
substantially further with a national-emergency agenda, likely to
include huge new promises on economic stimulus, infrastructure, climate
change and student debt.

The test ahead for him, however, is not just defining a bold agenda, but
also communicating it from a desk in his house as Mr. Trump makes
ruthless use of his bully pulpit.

Mr. Inslee, who endorsed Mr. Biden on Wednesday after conferring with
him privately about broadening his climate agenda, said he urged Mr.
Biden to put safety first. Democrats, he said, ``understand that we're
not going to hear from our candidate as much as we would if we didn't
have a pandemic.''

``It's really important that he take care of his health right now,'' Mr.
Inslee said. ``It's important for all of us.''

\hypertarget{our-2020-election-guide}{%
\section{Our 2020 Election Guide}\label{our-2020-election-guide}}

Updated July 31, 2020

\begin{itemize}
\item
  \begin{center}\rule{0.5\linewidth}{\linethickness}\end{center}

  \hypertarget{the-latest}{%
  \subsection{The Latest}\label{the-latest}}

  \begin{itemize}
  \tightlist
  \item
    President Trump's assault on the Postal Service is intersecting with
    his attacks on mail-in voting.
    \href{https://www.nytimes.com/2020/07/31/us/politics/trump-usps-mail-delays.html?action=click\&pgtype=Article\&state=default\&region=BELOW_MAIN_CONTENT\&context=storylines_guide}{Voting
    rights groups say it is a recipe for disaster.}
  \end{itemize}
\item
  \begin{center}\rule{0.5\linewidth}{\linethickness}\end{center}

  \hypertarget{bidens-vp-search}{%
  \subsection{Biden's V.P. Search}\label{bidens-vp-search}}

  \begin{itemize}
  \tightlist
  \item
    \href{https://www.nytimes.com/article/biden-vice-president-2020.html?action=click\&pgtype=Article\&state=default\&region=BELOW_MAIN_CONTENT\&context=storylines_guide}{Here
    are 13 women} who have been under consideration to be Joe Biden's
    running mate, and why each might be chosen --- and might not be.
  \end{itemize}
\item
  \begin{center}\rule{0.5\linewidth}{\linethickness}\end{center}

  \hypertarget{keep-up-with-our-coverage}{%
  \subsection{Keep Up With Our
  Coverage}\label{keep-up-with-our-coverage}}

  \begin{itemize}
  \tightlist
  \item
    Get an
    \href{https://www.nytimes.com/newsletters/politics?action=click\&pgtype=Article\&state=default\&region=BELOW_MAIN_CONTENT\&context=storylines_guide}{email}
    recapping the day's news
  \end{itemize}

  \begin{itemize}
  \tightlist
  \item
    Download our mobile app on
    \href{https://apps.apple.com/us/app/nytimes/id284862083?ls=1\&mat_click_id=5c79ae7455014fd1bd66b5610c05b8f2-20191112-16948\&referrer=mat_click_id\%3D5c79ae7455014fd1bd66b5610c05b8f2-20191112-16948\%26link_click_id\%3D722930677036718082}{iOS}
    and
    \href{http://a.localytics.com/android?id=com.nytimes.android\&referrer=utm_source\%3Dother_nyt_mobile_web\%26utm_medium\%3DWeb\%2520page\%26utm_term\%3DGeneral\%2520Mobile\%2520Page\%26utm_campaign\%3DNYT\%2520Mobile\%2520General\%2520Page}{Android}
    and turn on Breaking News and Politics alerts
  \end{itemize}
\end{itemize}

Advertisement

\protect\hyperlink{after-bottom}{Continue reading the main story}

\hypertarget{site-index}{%
\subsection{Site Index}\label{site-index}}

\hypertarget{site-information-navigation}{%
\subsection{Site Information
Navigation}\label{site-information-navigation}}

\begin{itemize}
\tightlist
\item
  \href{https://help.nytimes.com/hc/en-us/articles/115014792127-Copyright-notice}{©~2020~The
  New York Times Company}
\end{itemize}

\begin{itemize}
\tightlist
\item
  \href{https://www.nytco.com/}{NYTCo}
\item
  \href{https://help.nytimes.com/hc/en-us/articles/115015385887-Contact-Us}{Contact
  Us}
\item
  \href{https://www.nytco.com/careers/}{Work with us}
\item
  \href{https://nytmediakit.com/}{Advertise}
\item
  \href{http://www.tbrandstudio.com/}{T Brand Studio}
\item
  \href{https://www.nytimes.com/privacy/cookie-policy\#how-do-i-manage-trackers}{Your
  Ad Choices}
\item
  \href{https://www.nytimes.com/privacy}{Privacy}
\item
  \href{https://help.nytimes.com/hc/en-us/articles/115014893428-Terms-of-service}{Terms
  of Service}
\item
  \href{https://help.nytimes.com/hc/en-us/articles/115014893968-Terms-of-sale}{Terms
  of Sale}
\item
  \href{https://spiderbites.nytimes.com}{Site Map}
\item
  \href{https://help.nytimes.com/hc/en-us}{Help}
\item
  \href{https://www.nytimes.com/subscription?campaignId=37WXW}{Subscriptions}
\end{itemize}
