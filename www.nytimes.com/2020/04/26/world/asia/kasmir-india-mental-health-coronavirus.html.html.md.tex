Sections

SEARCH

\protect\hyperlink{site-content}{Skip to
content}\protect\hyperlink{site-index}{Skip to site index}

\href{https://www.nytimes.com/section/world/asia}{Asia Pacific}

\href{https://myaccount.nytimes.com/auth/login?response_type=cookie\&client_id=vi}{}

\href{https://www.nytimes.com/section/todayspaper}{Today's Paper}

\href{/section/world/asia}{Asia Pacific}\textbar{}Kashmir, Under Siege
and Lockdown, Faces a Mental Health Crisis

\url{https://nyti.ms/3eRJeyH}

\begin{itemize}
\item
\item
\item
\item
\item
\end{itemize}

\href{https://www.nytimes.com/news-event/coronavirus?action=click\&pgtype=Article\&state=default\&region=TOP_BANNER\&context=storylines_menu}{The
Coronavirus Outbreak}

\begin{itemize}
\tightlist
\item
  live\href{https://www.nytimes.com/2020/08/02/world/coronavirus-updates.html?action=click\&pgtype=Article\&state=default\&region=TOP_BANNER\&context=storylines_menu}{Latest
  Updates}
\item
  \href{https://www.nytimes.com/interactive/2020/us/coronavirus-us-cases.html?action=click\&pgtype=Article\&state=default\&region=TOP_BANNER\&context=storylines_menu}{Maps
  and Cases}
\item
  \href{https://www.nytimes.com/interactive/2020/science/coronavirus-vaccine-tracker.html?action=click\&pgtype=Article\&state=default\&region=TOP_BANNER\&context=storylines_menu}{Vaccine
  Tracker}
\item
  \href{https://www.nytimes.com/interactive/2020/07/29/us/schools-reopening-coronavirus.html?action=click\&pgtype=Article\&state=default\&region=TOP_BANNER\&context=storylines_menu}{What
  School May Look Like}
\item
  \href{https://www.nytimes.com/live/2020/07/31/business/stock-market-today-coronavirus?action=click\&pgtype=Article\&state=default\&region=TOP_BANNER\&context=storylines_menu}{Economy}
\end{itemize}

Advertisement

\protect\hyperlink{after-top}{Continue reading the main story}

Supported by

\protect\hyperlink{after-sponsor}{Continue reading the main story}

\hypertarget{kashmir-under-siege-and-lockdown-faces-a-mental-health-crisis}{%
\section{Kashmir, Under Siege and Lockdown, Faces a Mental Health
Crisis}\label{kashmir-under-siege-and-lockdown-faces-a-mental-health-crisis}}

Years of strife left a generation traumatized. India's clampdown
disrupted daily life. Now the battle against the coronavirus has further
isolated and scarred a people with little access to help.

\includegraphics{https://static01.nyt.com/images/2020/04/26/world/26india-kashmir1/merlin_170898000_82750348-b667-4a5d-b54c-aaf52901c955-articleLarge.jpg?quality=75\&auto=webp\&disable=upscale}

By \href{https://www.nytimes.com/by/sameer-yasir}{Sameer Yasir}

Photographs by Atul Loke

\begin{itemize}
\item
  April 26, 2020
\item
  \begin{itemize}
  \item
  \item
  \item
  \item
  \item
  \end{itemize}
\end{itemize}

PAHOO, Kashmir --- Sara Begum's suffering began on Aug. 3, when masked
policemen barged into her home, badly roughed up her son and whisked him
away.

Ms. Begum's son, Fayaz Ahmad Mir, 28, was one of thousands of civilians
arrested or detained by order of the Indian government after it
\href{https://www.nytimes.com/2019/08/05/world/asia/india-pakistan-kashmir-jammu.html}{moved
forcefully} to cement its control over Kashmir, a largely Muslim region
of about eight million people claimed by both India and Pakistan. The
clampdown has disrupted daily life, with many people feeling besieged
and afraid to leave their homes.

Since her son was arrested, Ms. Begum has become gaunt and unsteady, but
she and her family say her worst afflictions have been mental and
emotional. She now takes sertraline and lithium, both antidepressants.
She tried twice to commit suicide, once by consuming rat poison and
again by jumping into a river.

``When I close my eyes,'' Ms. Begum said, ``I see my son shouting,
`Mother, I want to see you.'''

Eight months after India revoked Kashmir's semiautonomous status and
brought the region fully under its authority, doctors here say a state
of hopelessness has morphed into a severe psychological crisis. Mental
health workers say Kashmir is witnessing an alarming increase in
instances of depression, anxiety and psychotic events.

Hard data is difficult to come by, but local medical professionals say
they are seeing a rise in suicides and an increase in already
disturbingly high rates of domestic abuse.

A nationwide lockdown that India imposed across the country in recent
weeks to fight the coronavirus has worsened the problem, the medical
professionals say. Police officers block roads with coils of glistening
concertina wire. Any residents who step out of their homes, especially
in Kashmir's towns and cities, risk getting beaten up.

\includegraphics{https://static01.nyt.com/images/2020/04/26/world/26india-kashmir2-SUB/merlin_170722131_554a778f-20ef-4eba-b201-a8fa5f5031e7-articleLarge.jpg?quality=75\&auto=webp\&disable=upscale}

Doctors and researchers say the Kashmir Valley, tucked into the
Himalayas, has few resources to cope. This area has been
\href{https://www.nytimes.com/2018/08/01/world/asia/kashmir-war-india-pakistan.html}{mired
in conflict for decades}, with its majority-Muslim population agitating
for independence or at least more autonomy from India, which is majority
Hindu and controls most of Kashmir. Pakistan controls a smaller slice.

Even before the events of recent months, decades of violence between
Indian security forces and Kashmiri militants had taken a physical and
mental toll on the region and its people. Nearly 1.8 million Kashmiris,
or nearly half of all adults, have some form of mental disorder, Doctors
Without Borders estimated after surveying 5,600 households in 2015. Nine
of 10 have experienced conflict-related traumas. The figures are much
higher than in India,
\href{https://www.thelancet.com/journals/lanpsy/article/PIIS2215-0366(19)30475-4/fulltext}{according
to other surveys}.

A leading psychiatrist said he was overwhelmed. Dr. Majid Shafi, a
government psychiatrist, said that last year he saw a hundred patients a
week. Now he sees more than 500. Overall, Kashmir has fewer than 60
psychiatrists.

\hypertarget{latest-updates-global-coronavirus-outbreak}{%
\section{\texorpdfstring{\href{https://www.nytimes.com/2020/08/01/world/coronavirus-covid-19.html?action=click\&pgtype=Article\&state=default\&region=MAIN_CONTENT_1\&context=storylines_live_updates}{Latest
Updates: Global Coronavirus
Outbreak}}{Latest Updates: Global Coronavirus Outbreak}}\label{latest-updates-global-coronavirus-outbreak}}

Updated 2020-08-02T17:52:35.962Z

\begin{itemize}
\tightlist
\item
  \href{https://www.nytimes.com/2020/08/01/world/coronavirus-covid-19.html?action=click\&pgtype=Article\&state=default\&region=MAIN_CONTENT_1\&context=storylines_live_updates\#link-34047410}{The
  U.S. reels as July cases more than double the total of any other
  month.}
\item
  \href{https://www.nytimes.com/2020/08/01/world/coronavirus-covid-19.html?action=click\&pgtype=Article\&state=default\&region=MAIN_CONTENT_1\&context=storylines_live_updates\#link-780ec966}{Top
  U.S. officials work to break an impasse over the federal jobless
  benefit.}
\item
  \href{https://www.nytimes.com/2020/08/01/world/coronavirus-covid-19.html?action=click\&pgtype=Article\&state=default\&region=MAIN_CONTENT_1\&context=storylines_live_updates\#link-2bc8948}{Its
  outbreak untamed, Melbourne goes into even greater lockdown.}
\end{itemize}

\href{https://www.nytimes.com/2020/08/01/world/coronavirus-covid-19.html?action=click\&pgtype=Article\&state=default\&region=MAIN_CONTENT_1\&context=storylines_live_updates}{See
more updates}

More live coverage:
\href{https://www.nytimes.com/live/2020/07/31/business/stock-market-today-coronavirus?action=click\&pgtype=Article\&state=default\&region=MAIN_CONTENT_1\&context=storylines_live_updates}{Markets}

A long line snakes out of Dr. Shafi's office --- teenagers traumatized
by violence; mothers too worried about their incarcerated children to
sleep; businesspeople owing a mountain of debt that is climbing higher
and higher under a lockdown that has shuttered nearly everything.

Image

Patients waiting to see Majid Shafi, a psychiatrist, in Kashmir last
month.

``This is just the tip of an iceberg,'' said Dr. Shafi, the single
government psychiatrist for around a million people in the district of
Pulwama. ``The crisis is growing.''

Every season of turmoil in Kashmir brings a new kind of pain. One season
is marked by the corpses of teenage boys felled by Indian forces.
Another brings an
\href{https://www.nytimes.com/2016/08/29/world/asia/pellet-guns-used-in-kashmir-protests-cause-dead-eyes-epidemic.html}{epidemic
of dead eyes}, as Kashmiris refer to protesters left blind after being
struck in the eyes by pellets fired by police officers.

This past year will be remembered for
\href{https://www.nytimes.com/2019/09/30/world/asia/Kashmir-lockdown-photos.html}{the
crackdown}. In August, the Indian government suddenly stripped away
statehood from Jammu and Kashmir, which had been India's one
Muslim-majority state.

Security forces flooded the area, cut off roads, shut down landlines,
cellphone lines and the internet, and arrested thousands of Kashmiris,
from students to top elected officials. Some have been released, but
many remain in jail. Though some phone and internet service have been
restored, they remain nothing close to pre-crackdown levels.

Many Kashmiris, who used social media to socialize because it was
dangerous to hang out in the streets, now feel completely isolated.
Children have remained
\href{https://www.nytimes.com/2019/10/31/world/asia/kashmir-school-children.html}{out
of school for months}. Because of the military crackdown and then the
coronavirus lockdown, students have been in school only a few weeks.

Ms. Begum's family said her son, a farmer, participated in protests
against the Indian government a few years ago, as did
\href{https://www.nytimes.com/2016/07/17/world/asia/how-killing-of-prominent-separatist-set-off-turmoil-in-kashmir.html}{thousands
of other Kashmiris}. They believe this is why he was arrested in the
roundups in August. Ms. Begum, who is in her 60s, raised Mr. Mir during
the heyday of the insurgency, shielding him from the wrath of security
forces and militants alike.

He is being held in a jail hundreds of miles away on vague charges of
being a ``threat to peace,'' family members said. They don't have money
to visit. When the cellphone network was
\href{https://www.nytimes.com/2019/10/14/business/kashmir-cellphone-service-restored.html}{switched
back} on in October, the authorities promised a video call. That has not
yet happened.

Image

Disinfecting at a hospital in Srinagar last month.

Ms. Begum, increasingly demoralized, said her son had been ``stolen''
from her. Ms. Begum says she sees her son in dreams, his face covered in
bandages, his hands shaking in fear. He begs for water, but she feels
chained, unable to move.

According to India's home ministry, Mr. Mir was among more than 7,000
people arrested in August. Of those, more than 450 are still in jail,
including Mr. Mir.

\href{https://www.nytimes.com/news-event/coronavirus?action=click\&pgtype=Article\&state=default\&region=MAIN_CONTENT_3\&context=storylines_faq}{}

\hypertarget{the-coronavirus-outbreak-}{%
\subsubsection{The Coronavirus Outbreak
›}\label{the-coronavirus-outbreak-}}

\hypertarget{frequently-asked-questions}{%
\paragraph{Frequently Asked
Questions}\label{frequently-asked-questions}}

Updated July 27, 2020

\begin{itemize}
\item ~
  \hypertarget{should-i-refinance-my-mortgage}{%
  \paragraph{Should I refinance my
  mortgage?}\label{should-i-refinance-my-mortgage}}

  \begin{itemize}
  \tightlist
  \item
    \href{https://www.nytimes.com/article/coronavirus-money-unemployment.html?action=click\&pgtype=Article\&state=default\&region=MAIN_CONTENT_3\&context=storylines_faq}{It
    could be a good idea,} because mortgage rates have
    \href{https://www.nytimes.com/2020/07/16/business/mortgage-rates-below-3-percent.html?action=click\&pgtype=Article\&state=default\&region=MAIN_CONTENT_3\&context=storylines_faq}{never
    been lower.} Refinancing requests have pushed mortgage applications
    to some of the highest levels since 2008, so be prepared to get in
    line. But defaults are also up, so if you're thinking about buying a
    home, be aware that some lenders have tightened their standards.
  \end{itemize}
\item ~
  \hypertarget{what-is-school-going-to-look-like-in-september}{%
  \paragraph{What is school going to look like in
  September?}\label{what-is-school-going-to-look-like-in-september}}

  \begin{itemize}
  \tightlist
  \item
    It is unlikely that many schools will return to a normal schedule
    this fall, requiring the grind of
    \href{https://www.nytimes.com/2020/06/05/us/coronavirus-education-lost-learning.html?action=click\&pgtype=Article\&state=default\&region=MAIN_CONTENT_3\&context=storylines_faq}{online
    learning},
    \href{https://www.nytimes.com/2020/05/29/us/coronavirus-child-care-centers.html?action=click\&pgtype=Article\&state=default\&region=MAIN_CONTENT_3\&context=storylines_faq}{makeshift
    child care} and
    \href{https://www.nytimes.com/2020/06/03/business/economy/coronavirus-working-women.html?action=click\&pgtype=Article\&state=default\&region=MAIN_CONTENT_3\&context=storylines_faq}{stunted
    workdays} to continue. California's two largest public school
    districts --- Los Angeles and San Diego --- said on July 13, that
    \href{https://www.nytimes.com/2020/07/13/us/lausd-san-diego-school-reopening.html?action=click\&pgtype=Article\&state=default\&region=MAIN_CONTENT_3\&context=storylines_faq}{instruction
    will be remote-only in the fall}, citing concerns that surging
    coronavirus infections in their areas pose too dire a risk for
    students and teachers. Together, the two districts enroll some
    825,000 students. They are the largest in the country so far to
    abandon plans for even a partial physical return to classrooms when
    they reopen in August. For other districts, the solution won't be an
    all-or-nothing approach.
    \href{https://bioethics.jhu.edu/research-and-outreach/projects/eschool-initiative/school-policy-tracker/}{Many
    systems}, including the nation's largest, New York City, are
    devising
    \href{https://www.nytimes.com/2020/06/26/us/coronavirus-schools-reopen-fall.html?action=click\&pgtype=Article\&state=default\&region=MAIN_CONTENT_3\&context=storylines_faq}{hybrid
    plans} that involve spending some days in classrooms and other days
    online. There's no national policy on this yet, so check with your
    municipal school system regularly to see what is happening in your
    community.
  \end{itemize}
\item ~
  \hypertarget{is-the-coronavirus-airborne}{%
  \paragraph{Is the coronavirus
  airborne?}\label{is-the-coronavirus-airborne}}

  \begin{itemize}
  \tightlist
  \item
    The coronavirus
    \href{https://www.nytimes.com/2020/07/04/health/239-experts-with-one-big-claim-the-coronavirus-is-airborne.html?action=click\&pgtype=Article\&state=default\&region=MAIN_CONTENT_3\&context=storylines_faq}{can
    stay aloft for hours in tiny droplets in stagnant air}, infecting
    people as they inhale, mounting scientific evidence suggests. This
    risk is highest in crowded indoor spaces with poor ventilation, and
    may help explain super-spreading events reported in meatpacking
    plants, churches and restaurants.
    \href{https://www.nytimes.com/2020/07/06/health/coronavirus-airborne-aerosols.html?action=click\&pgtype=Article\&state=default\&region=MAIN_CONTENT_3\&context=storylines_faq}{It's
    unclear how often the virus is spread} via these tiny droplets, or
    aerosols, compared with larger droplets that are expelled when a
    sick person coughs or sneezes, or transmitted through contact with
    contaminated surfaces, said Linsey Marr, an aerosol expert at
    Virginia Tech. Aerosols are released even when a person without
    symptoms exhales, talks or sings, according to Dr. Marr and more
    than 200 other experts, who
    \href{https://academic.oup.com/cid/article/doi/10.1093/cid/ciaa939/5867798}{have
    outlined the evidence in an open letter to the World Health
    Organization}.
  \end{itemize}
\item ~
  \hypertarget{what-are-the-symptoms-of-coronavirus}{%
  \paragraph{What are the symptoms of
  coronavirus?}\label{what-are-the-symptoms-of-coronavirus}}

  \begin{itemize}
  \tightlist
  \item
    Common symptoms
    \href{https://www.nytimes.com/article/symptoms-coronavirus.html?action=click\&pgtype=Article\&state=default\&region=MAIN_CONTENT_3\&context=storylines_faq}{include
    fever, a dry cough, fatigue and difficulty breathing or shortness of
    breath.} Some of these symptoms overlap with those of the flu,
    making detection difficult, but runny noses and stuffy sinuses are
    less common.
    \href{https://www.nytimes.com/2020/04/27/health/coronavirus-symptoms-cdc.html?action=click\&pgtype=Article\&state=default\&region=MAIN_CONTENT_3\&context=storylines_faq}{The
    C.D.C. has also} added chills, muscle pain, sore throat, headache
    and a new loss of the sense of taste or smell as symptoms to look
    out for. Most people fall ill five to seven days after exposure, but
    symptoms may appear in as few as two days or as many as 14 days.
  \end{itemize}
\item ~
  \hypertarget{does-asymptomatic-transmission-of-covid-19-happen}{%
  \paragraph{Does asymptomatic transmission of Covid-19
  happen?}\label{does-asymptomatic-transmission-of-covid-19-happen}}

  \begin{itemize}
  \tightlist
  \item
    So far, the evidence seems to show it does. A widely cited
    \href{https://www.nature.com/articles/s41591-020-0869-5}{paper}
    published in April suggests that people are most infectious about
    two days before the onset of coronavirus symptoms and estimated that
    44 percent of new infections were a result of transmission from
    people who were not yet showing symptoms. Recently, a top expert at
    the World Health Organization stated that transmission of the
    coronavirus by people who did not have symptoms was ``very rare,''
    \href{https://www.nytimes.com/2020/06/09/world/coronavirus-updates.html?action=click\&pgtype=Article\&state=default\&region=MAIN_CONTENT_3\&context=storylines_faq\#link-1f302e21}{but
    she later walked back that statement.}
  \end{itemize}
\end{itemize}

Many, including
\href{https://www.outlookindia.com/website/story/six-months-after-detention-prospects-of-3-ex-jk-cms-release-still-bleak/346801}{three
former chief ministers}, were detained under India's Public Safety Act,
a law that allows the authorities to jail suspects without charge for up
to two years.

People who were recently released spoke of the humiliation and fear they
experienced behind bars.

Bilal Sultan, a politician, was placed with hardened criminals, he said.
He slipped into depression. In February, he said, he was set free
because his doctors were afraid he might kill himself.

Mr. Sultan takes sleeping pills and complains of a recurring dream in
which he is traveling to a tourist resort before soldiers stop him and
shoot him between the eyes.

``I was a very strong man,'' Mr. Sultan, 55, said recently at his house
in Srinagar. ``Now I fear my own children.''

Kashmir's new generation, long accustomed to violence and bloodshed, may
be the hardest hit.

Image

Nida Rehman, who tried to spread mental health awareness to help others,
was diagnosed with depression.

Before the August clampdown, Nida Rehman, 28, wanted to lift the spirits
of others by setting up a nonprofit organization to raise awareness for
mental health issues. But she says she has failed to learn the lessons
she gave to others.

In pictures, she looks happy. But the trauma has withered her. She has
lost weight, her cheeks sunken, her eyes ringed by dark circles.

Last year, Ms. Rehman visited a psychiatrist, who diagnosed acute
depression. She doesn't sleep for days. She found some relief by
spending hours with a caged parrot, Noor, that her family kept.

``I felt I was living in a cage, like Noor,'' Ms. Rehman said. ``That
happy world slipped out of my hands.''

After her relatives found her talking to the parrot, they grew worried
and embarrassed, persuading her to release it. One evening, in October,
Ms. Rehman set Noor free.

It didn't help. Ms. Rehman is now taking regular doses of
antidepressants.

So is Ms. Begum, the mother of the imprisoned Mr. Mir. As days get
longer and weather begins to warm, she finds herself staring at a
tractor parked outside her house. Mr. Mir bought it on a loan. The bank
is now threatening to seize the family's home.

Ms. Begum's hands tremble and her lips quiver as she speaks. The
desperation has incapacitated her. She can't even cook, she said.

``I may never see my son again,'' she said. ``I feel I will die before
he comes back home.''

Image

The Dal Lake in Srinagar, normally full of tourists, without customers
last month.

Iqbal Kirmani contributed reporting.

Advertisement

\protect\hyperlink{after-bottom}{Continue reading the main story}

\hypertarget{site-index}{%
\subsection{Site Index}\label{site-index}}

\hypertarget{site-information-navigation}{%
\subsection{Site Information
Navigation}\label{site-information-navigation}}

\begin{itemize}
\tightlist
\item
  \href{https://help.nytimes.com/hc/en-us/articles/115014792127-Copyright-notice}{©~2020~The
  New York Times Company}
\end{itemize}

\begin{itemize}
\tightlist
\item
  \href{https://www.nytco.com/}{NYTCo}
\item
  \href{https://help.nytimes.com/hc/en-us/articles/115015385887-Contact-Us}{Contact
  Us}
\item
  \href{https://www.nytco.com/careers/}{Work with us}
\item
  \href{https://nytmediakit.com/}{Advertise}
\item
  \href{http://www.tbrandstudio.com/}{T Brand Studio}
\item
  \href{https://www.nytimes.com/privacy/cookie-policy\#how-do-i-manage-trackers}{Your
  Ad Choices}
\item
  \href{https://www.nytimes.com/privacy}{Privacy}
\item
  \href{https://help.nytimes.com/hc/en-us/articles/115014893428-Terms-of-service}{Terms
  of Service}
\item
  \href{https://help.nytimes.com/hc/en-us/articles/115014893968-Terms-of-sale}{Terms
  of Sale}
\item
  \href{https://spiderbites.nytimes.com}{Site Map}
\item
  \href{https://help.nytimes.com/hc/en-us}{Help}
\item
  \href{https://www.nytimes.com/subscription?campaignId=37WXW}{Subscriptions}
\end{itemize}
