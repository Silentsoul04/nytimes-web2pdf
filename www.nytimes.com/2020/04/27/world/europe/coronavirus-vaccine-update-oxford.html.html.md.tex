Sections

SEARCH

\protect\hyperlink{site-content}{Skip to
content}\protect\hyperlink{site-index}{Skip to site index}

\href{https://www.nytimes.com/section/world/europe}{Europe}

\href{https://myaccount.nytimes.com/auth/login?response_type=cookie\&client_id=vi}{}

\href{https://www.nytimes.com/section/todayspaper}{Today's Paper}

\href{/section/world/europe}{Europe}\textbar{}In Race for a Coronavirus
Vaccine, an Oxford Group Leaps Ahead

\url{https://nyti.ms/3cWsEf9}

\begin{itemize}
\item
\item
\item
\item
\item
\item
\end{itemize}

\href{https://www.nytimes.com/news-event/coronavirus?action=click\&pgtype=Article\&state=default\&region=TOP_BANNER\&context=storylines_menu}{The
Coronavirus Outbreak}

\begin{itemize}
\tightlist
\item
  live\href{https://www.nytimes.com/2020/08/02/world/coronavirus-updates.html?action=click\&pgtype=Article\&state=default\&region=TOP_BANNER\&context=storylines_menu}{Latest
  Updates}
\item
  \href{https://www.nytimes.com/interactive/2020/us/coronavirus-us-cases.html?action=click\&pgtype=Article\&state=default\&region=TOP_BANNER\&context=storylines_menu}{Maps
  and Cases}
\item
  \href{https://www.nytimes.com/interactive/2020/science/coronavirus-vaccine-tracker.html?action=click\&pgtype=Article\&state=default\&region=TOP_BANNER\&context=storylines_menu}{Vaccine
  Tracker}
\item
  \href{https://www.nytimes.com/interactive/2020/07/29/us/schools-reopening-coronavirus.html?action=click\&pgtype=Article\&state=default\&region=TOP_BANNER\&context=storylines_menu}{What
  School May Look Like}
\item
  \href{https://www.nytimes.com/live/2020/07/31/business/stock-market-today-coronavirus?action=click\&pgtype=Article\&state=default\&region=TOP_BANNER\&context=storylines_menu}{Economy}
\end{itemize}

Advertisement

\protect\hyperlink{after-top}{Continue reading the main story}

Supported by

\protect\hyperlink{after-sponsor}{Continue reading the main story}

\hypertarget{in-race-for-a-coronavirus-vaccine-an-oxford-group-leaps-ahead}{%
\section{In Race for a Coronavirus Vaccine, an Oxford Group Leaps
Ahead}\label{in-race-for-a-coronavirus-vaccine-an-oxford-group-leaps-ahead}}

As scientists at the Jenner Institute prepare for mass clinical trials,
new tests show their vaccine to be effective in monkeys.

\includegraphics{https://static01.nyt.com/images/2020/05/03/world/27virus-vaccine/merlin_171899610_d411bbb0-be14-4e31-89a9-28d47efd9cf6-articleLarge.jpg?quality=75\&auto=webp\&disable=upscale}

\href{https://www.nytimes.com/by/david-d-kirkpatrick}{\includegraphics{https://static01.nyt.com/images/2018/10/15/multimedia/author-david-d-kirkpatrick/author-david-d-kirkpatrick-thumbLarge-v2.png}}

By \href{https://www.nytimes.com/by/david-d-kirkpatrick}{David D.
Kirkpatrick}

\begin{itemize}
\item
  Published April 27, 2020Updated May 2, 2020
\item
  \begin{itemize}
  \item
  \item
  \item
  \item
  \item
  \item
  \end{itemize}
\end{itemize}

OXFORD, England --- In the worldwide race for a
\href{https://www.nytimes.com/2020/04/08/health/coronavirus-vaccines.html}{vaccine
to stop the coronavirus}, the laboratory sprinting fastest is at Oxford
University.

Most other teams have had to start with small clinical trials of a few
hundred participants to demonstrate safety. But scientists at the
university's \href{https://www.jenner.ac.uk/}{Jenner Institute} had a
head start on a vaccine, having proved in previous trials that similar
inoculations --- including one last year against an earlier coronavirus
--- were harmless to humans.

That has enabled them to leap ahead and schedule tests of their new
coronavirus
\href{https://www.nytimes.com/2020/05/15/us/politics/coronavirus-vaccine-timeline.html}{vaccine}
involving more than 6,000 people by the end of next month, hoping to
show not only that it is safe, but also that it works.

The Oxford scientists now say that with an emergency approval from
regulators, the first few million doses of their vaccine could be
available by September --- at least several months ahead of any of the
other announced efforts --- if it proves to be effective.

Now, they have received promising news suggesting that it might.

Scientists at the National Institutes of Health's
\href{https://www.niaid.nih.gov/about/rocky-mountain-overview}{Rocky
Mountain Laboratory} in Montana last month inoculated six rhesus macaque
monkeys with single doses of the Oxford vaccine. The animals were then
exposed to heavy quantities of the virus that is causing the pandemic
--- exposure that had consistently sickened other monkeys in the lab.
But more than 28 days later all six were healthy, said Vincent Munster,
the researcher who conducted the test.

``The rhesus macaque is pretty much the closest thing we have to
humans,'' Dr. Munster said, noting that scientists were still analyzing
the result. He said he expected to
\href{https://www.nytimes.com/2020/04/01/world/europe/coronavirus-science-research-cooperation.html}{share
it with other scientists} next week and then submit it to a
peer-reviewed journal.

\emph{{[}}\href{https://www.nytimes.com/interactive/2020/science/coronavirus-vaccine-tracker.html}{\emph{Follow
our Live Coronavirus Vaccine Tracker}}\emph{.{]}}

Immunity in monkeys is no guarantee that a vaccine will provide the same
degree of protection for humans. A Chinese company that recently started
a clinical trial with 144 participants,
\href{https://www.sciencemag.org/news/2020/04/covid-19-vaccine-protects-monkeys-new-coronavirus-chinese-biotech-reports}{SinoVac},
has also said that its vaccine was effective in rhesus macaques. But
with dozens of efforts now underway to find a vaccine, the monkey
results are the latest indication that Oxford's accelerated venture is
emerging as a bellwether.

``It is a very, very fast clinical program,'' said Emilio Emini, a
director of the vaccine program at the Bill and Melinda Gates
Foundation, which is providing financial support to many competing
efforts.

\includegraphics{https://static01.nyt.com/images/2020/04/27/world/27vaccine2/merlin_171900807_ab1a20ca-e3d4-40f9-8b35-6dd047d03650-articleLarge.jpg?quality=75\&auto=webp\&disable=upscale}

Which potential vaccine will emerge from the scramble as the most
successful is impossible to know until clinical trial data becomes
available.

\hypertarget{latest-updates-global-coronavirus-outbreak}{%
\section{\texorpdfstring{\href{https://www.nytimes.com/2020/08/01/world/coronavirus-covid-19.html?action=click\&pgtype=Article\&state=default\&region=MAIN_CONTENT_1\&context=storylines_live_updates}{Latest
Updates: Global Coronavirus
Outbreak}}{Latest Updates: Global Coronavirus Outbreak}}\label{latest-updates-global-coronavirus-outbreak}}

Updated 2020-08-02T17:52:35.962Z

\begin{itemize}
\tightlist
\item
  \href{https://www.nytimes.com/2020/08/01/world/coronavirus-covid-19.html?action=click\&pgtype=Article\&state=default\&region=MAIN_CONTENT_1\&context=storylines_live_updates\#link-34047410}{The
  U.S. reels as July cases more than double the total of any other
  month.}
\item
  \href{https://www.nytimes.com/2020/08/01/world/coronavirus-covid-19.html?action=click\&pgtype=Article\&state=default\&region=MAIN_CONTENT_1\&context=storylines_live_updates\#link-780ec966}{Top
  U.S. officials work to break an impasse over the federal jobless
  benefit.}
\item
  \href{https://www.nytimes.com/2020/08/01/world/coronavirus-covid-19.html?action=click\&pgtype=Article\&state=default\&region=MAIN_CONTENT_1\&context=storylines_live_updates\#link-2bc8948}{Its
  outbreak untamed, Melbourne goes into even greater lockdown.}
\end{itemize}

\href{https://www.nytimes.com/2020/08/01/world/coronavirus-covid-19.html?action=click\&pgtype=Article\&state=default\&region=MAIN_CONTENT_1\&context=storylines_live_updates}{See
more updates}

More live coverage:
\href{https://www.nytimes.com/live/2020/07/31/business/stock-market-today-coronavirus?action=click\&pgtype=Article\&state=default\&region=MAIN_CONTENT_1\&context=storylines_live_updates}{Markets}

More than one vaccine would be needed in any case, Dr. Emini argued.
Some may work more effectively than others in groups like children or
older people, or at different costs and dosages. Having more than one
variety of vaccine in production will also help avoid bottlenecks in
manufacturing, he said.

But as the first to reach such a relatively large scale, the Oxford
trial, even if it fails, will provide lessons about the nature of the
coronavirus and about the immune system's responses that can inform
governments, donors, drug companies and other scientists hunting for a
vaccine.

``This big U.K. study,'' Dr. Emini said, ``is actually going to
translate to learning a lot about some of the others as well.''

All of the others will face the same challenges, including obtaining
millions of dollars in funding, persuading regulators to approve human
tests, demonstrating a vaccine's safety and --- after all of that ---
proving its effectiveness in protecting people from the coronavirus.

Paradoxically, the growing success of efforts to contain the spread of
Covid-19, the disease caused by the virus, may present yet another
hurdle.

``We're the only people in the country who want the number of new
infections to stay up for another few weeks, so we can test our
vaccine,'' Prof. Adrian Hill, the Jenner Institute's director and one of
five researchers involved in the effort, said in an interview in a
laboratory building emptied by Britain's monthlong lockdown.

Ethics rules, as a general principle, forbid seeking to infect human
test participants with a serious disease. That means the only way to
prove that a vaccine works is to inoculate people in a place where the
virus is spreading naturally around them.

If social distancing measures or other factors continue to slow the rate
of new infections in Britain, he said, the trial might not be able to
show that the vaccine makes a difference: Participants who received a
placebo might not be infected any more frequently than those who have
been given the vaccine. The scientists would have to try again
elsewhere, a dilemma that every other vaccine effort will face as well.

Image

Social distancing at Oxford last week.Credit...Mary Turner for The New
York Times

The Jenner Institute's coronavirus efforts grew out of Professor Hill's
so-far unsuccessful pursuit of a vaccine against a different scourge,
malaria.

He developed a fascination with malaria and other tropical diseases as a
medical student in Dublin in the early 1980s, when he visited an uncle
who was a priest working in a hospital during the civil war in what is
now Zimbabwe.

``I came back wondering, `What do you see in these hospitals in England
and Ireland?''' Professor Hill said. ``They don't have any of these
diseases.''

The major drug companies typically see little profit in epidemics that
afflict mainly developing countries or run their course before a vaccine
can hit the market. So after training in tropical medicine and a
doctorate in molecular genetics, Professor Hill, 61, helped build
Oxford's institute into one of the largest academic centers dedicated to
nonprofit vaccine research, with its own pilot manufacturing facility
capable of producing a batch of up to 1,000 doses.

The institute's effort against the coronavirus uses a technology that
centers on altering the genetic code of a familiar virus. A classic
vaccine uses a weakened version of a virus to trigger an immune
response. But in the technology that the institute is using, a different
virus is modified first to neutralize its effects and then to make it
mimic the one scientists seek to stop --- in this case, the virus that
causes Covid-19. Injected into the body, the harmless impostor can
induce the immune system to fight and kill the targeted virus, providing
protection.

Professor Hill has worked with that technology for decades to try to
tweak a respiratory virus found in chimpanzees in order to elicit a
human immune response against malaria and other diseases. Over the last
20 years, the institute has conducted more than 70 clinical trials of
potential vaccines against the parasite that causes malaria. None have
yet yielded a successful inoculation.

In 2014, however, a vaccine based on the chimp virus that Professor Hill
had tested was manufactured in a large enough scale to provide a million
doses. That created a template for mass production of the coronavirus
vaccine, should it prove effective.

A longtime colleague, Prof. Sarah Gilbert, 58, modified the same
chimpanzee virus to make a vaccine against an earlier coronavirus, MERS.
After a clinical trial in Britain demonstrated its safety, another test
began in December in Saudi Arabia, where outbreaks of the deadly disease
are still common.

When she heard in January that Chinese scientists had identified the
genetic code of a mysterious virus in Wuhan, she thought she might have
a chance to prove the speed and versatility of their approach.

\href{https://www.nytimes.com/news-event/coronavirus?action=click\&pgtype=Article\&state=default\&region=MAIN_CONTENT_3\&context=storylines_faq}{}

\hypertarget{the-coronavirus-outbreak-}{%
\subsubsection{The Coronavirus Outbreak
›}\label{the-coronavirus-outbreak-}}

\hypertarget{frequently-asked-questions}{%
\paragraph{Frequently Asked
Questions}\label{frequently-asked-questions}}

Updated July 27, 2020

\begin{itemize}
\item ~
  \hypertarget{should-i-refinance-my-mortgage}{%
  \paragraph{Should I refinance my
  mortgage?}\label{should-i-refinance-my-mortgage}}

  \begin{itemize}
  \tightlist
  \item
    \href{https://www.nytimes.com/article/coronavirus-money-unemployment.html?action=click\&pgtype=Article\&state=default\&region=MAIN_CONTENT_3\&context=storylines_faq}{It
    could be a good idea,} because mortgage rates have
    \href{https://www.nytimes.com/2020/07/16/business/mortgage-rates-below-3-percent.html?action=click\&pgtype=Article\&state=default\&region=MAIN_CONTENT_3\&context=storylines_faq}{never
    been lower.} Refinancing requests have pushed mortgage applications
    to some of the highest levels since 2008, so be prepared to get in
    line. But defaults are also up, so if you're thinking about buying a
    home, be aware that some lenders have tightened their standards.
  \end{itemize}
\item ~
  \hypertarget{what-is-school-going-to-look-like-in-september}{%
  \paragraph{What is school going to look like in
  September?}\label{what-is-school-going-to-look-like-in-september}}

  \begin{itemize}
  \tightlist
  \item
    It is unlikely that many schools will return to a normal schedule
    this fall, requiring the grind of
    \href{https://www.nytimes.com/2020/06/05/us/coronavirus-education-lost-learning.html?action=click\&pgtype=Article\&state=default\&region=MAIN_CONTENT_3\&context=storylines_faq}{online
    learning},
    \href{https://www.nytimes.com/2020/05/29/us/coronavirus-child-care-centers.html?action=click\&pgtype=Article\&state=default\&region=MAIN_CONTENT_3\&context=storylines_faq}{makeshift
    child care} and
    \href{https://www.nytimes.com/2020/06/03/business/economy/coronavirus-working-women.html?action=click\&pgtype=Article\&state=default\&region=MAIN_CONTENT_3\&context=storylines_faq}{stunted
    workdays} to continue. California's two largest public school
    districts --- Los Angeles and San Diego --- said on July 13, that
    \href{https://www.nytimes.com/2020/07/13/us/lausd-san-diego-school-reopening.html?action=click\&pgtype=Article\&state=default\&region=MAIN_CONTENT_3\&context=storylines_faq}{instruction
    will be remote-only in the fall}, citing concerns that surging
    coronavirus infections in their areas pose too dire a risk for
    students and teachers. Together, the two districts enroll some
    825,000 students. They are the largest in the country so far to
    abandon plans for even a partial physical return to classrooms when
    they reopen in August. For other districts, the solution won't be an
    all-or-nothing approach.
    \href{https://bioethics.jhu.edu/research-and-outreach/projects/eschool-initiative/school-policy-tracker/}{Many
    systems}, including the nation's largest, New York City, are
    devising
    \href{https://www.nytimes.com/2020/06/26/us/coronavirus-schools-reopen-fall.html?action=click\&pgtype=Article\&state=default\&region=MAIN_CONTENT_3\&context=storylines_faq}{hybrid
    plans} that involve spending some days in classrooms and other days
    online. There's no national policy on this yet, so check with your
    municipal school system regularly to see what is happening in your
    community.
  \end{itemize}
\item ~
  \hypertarget{is-the-coronavirus-airborne}{%
  \paragraph{Is the coronavirus
  airborne?}\label{is-the-coronavirus-airborne}}

  \begin{itemize}
  \tightlist
  \item
    The coronavirus
    \href{https://www.nytimes.com/2020/07/04/health/239-experts-with-one-big-claim-the-coronavirus-is-airborne.html?action=click\&pgtype=Article\&state=default\&region=MAIN_CONTENT_3\&context=storylines_faq}{can
    stay aloft for hours in tiny droplets in stagnant air}, infecting
    people as they inhale, mounting scientific evidence suggests. This
    risk is highest in crowded indoor spaces with poor ventilation, and
    may help explain super-spreading events reported in meatpacking
    plants, churches and restaurants.
    \href{https://www.nytimes.com/2020/07/06/health/coronavirus-airborne-aerosols.html?action=click\&pgtype=Article\&state=default\&region=MAIN_CONTENT_3\&context=storylines_faq}{It's
    unclear how often the virus is spread} via these tiny droplets, or
    aerosols, compared with larger droplets that are expelled when a
    sick person coughs or sneezes, or transmitted through contact with
    contaminated surfaces, said Linsey Marr, an aerosol expert at
    Virginia Tech. Aerosols are released even when a person without
    symptoms exhales, talks or sings, according to Dr. Marr and more
    than 200 other experts, who
    \href{https://academic.oup.com/cid/article/doi/10.1093/cid/ciaa939/5867798}{have
    outlined the evidence in an open letter to the World Health
    Organization}.
  \end{itemize}
\item ~
  \hypertarget{what-are-the-symptoms-of-coronavirus}{%
  \paragraph{What are the symptoms of
  coronavirus?}\label{what-are-the-symptoms-of-coronavirus}}

  \begin{itemize}
  \tightlist
  \item
    Common symptoms
    \href{https://www.nytimes.com/article/symptoms-coronavirus.html?action=click\&pgtype=Article\&state=default\&region=MAIN_CONTENT_3\&context=storylines_faq}{include
    fever, a dry cough, fatigue and difficulty breathing or shortness of
    breath.} Some of these symptoms overlap with those of the flu,
    making detection difficult, but runny noses and stuffy sinuses are
    less common.
    \href{https://www.nytimes.com/2020/04/27/health/coronavirus-symptoms-cdc.html?action=click\&pgtype=Article\&state=default\&region=MAIN_CONTENT_3\&context=storylines_faq}{The
    C.D.C. has also} added chills, muscle pain, sore throat, headache
    and a new loss of the sense of taste or smell as symptoms to look
    out for. Most people fall ill five to seven days after exposure, but
    symptoms may appear in as few as two days or as many as 14 days.
  \end{itemize}
\item ~
  \hypertarget{does-asymptomatic-transmission-of-covid-19-happen}{%
  \paragraph{Does asymptomatic transmission of Covid-19
  happen?}\label{does-asymptomatic-transmission-of-covid-19-happen}}

  \begin{itemize}
  \tightlist
  \item
    So far, the evidence seems to show it does. A widely cited
    \href{https://www.nature.com/articles/s41591-020-0869-5}{paper}
    published in April suggests that people are most infectious about
    two days before the onset of coronavirus symptoms and estimated that
    44 percent of new infections were a result of transmission from
    people who were not yet showing symptoms. Recently, a top expert at
    the World Health Organization stated that transmission of the
    coronavirus by people who did not have symptoms was ``very rare,''
    \href{https://www.nytimes.com/2020/06/09/world/coronavirus-updates.html?action=click\&pgtype=Article\&state=default\&region=MAIN_CONTENT_3\&context=storylines_faq\#link-1f302e21}{but
    she later walked back that statement.}
  \end{itemize}
\end{itemize}

``We thought, `Well, should we have a go?''' she recalled. ```It'll be a
little lab project and we'll publish a paper.'''

It did not stay a ``little lab project'' for long.

Image

Professor Sarah Gilbert, a vaccinologist at the institute, has also
worked on developing a vaccine for MERS, an earlier
coronavirus.Credit...Mary Turner for The New York Times

As the pandemic exploded, grant money poured in. All other vaccines were
soon put into the freezer so that the institute's laboratory could focus
full-time on Covid-19. Then the lockdown forced everyone not working on
Covid-19 to stay home altogether.

``The whole world doesn't usually stand up and say, `How can we help? Do
you want some money?''' Professor Hill said.

``Vaccines are good for pandemics,'' he added, ``and pandemics are good
for vaccines.''

Other scientists involved in the project are working with a half dozen
drug manufacturing companies across Europe and Asia to prepare to churn
out billions of doses as quickly as possible if the vaccine is approved.
None have been granted exclusive marketing rights, and one is the giant
Serum Institute of India, the world's largest supplier of vaccines.

Donors are currently spending tens of millions of dollars to start the
manufacturing process at facilities in Britain and the Netherlands even
before the vaccine is proven to work, said Sandy Douglas, 37, a doctor
at Oxford overseeing vaccine production.

``There is no alternative,'' he said.

But the team has not yet reached an agreement with a North American
manufacturer, in part because the major pharmaceutical companies there
typically demand exclusive worldwide rights before investing in a
potential medicine.

``I personally don't believe that in a time of pandemic there should be
exclusive licenses,'' Professor Hill said. ``So we are asking a lot of
them. Nobody is going to make a lot of money off this.''

The Jenner Institute's vaccine effort is not the only one showing
promise. Two American companies,
\href{https://www.nytimes.com/2020/03/16/health/coronavirus-vaccine.html}{Moderna}
and
\href{https://www.nytimes.com/2020/01/28/health/coronavirus-vaccine.html}{Inovio},
have started small clinical trials with **** technologies involving
modified or otherwise manipulated genetic material. They are seeking
both to demonstrate their safety and to learn more about dosing and
other variables. Neither technology has ever produced a licensed drug or
been manufactured at scale.

A Chinese company, CanSino, has also started clinical trials in China
using a technology similar to the Oxford institute's, using a strain of
the same respiratory virus that is found in humans, not chimps. But
demonstrating the effectiveness of a vaccine in China may be difficult
because Covid-19 infections there have plummeted.

Armed with safety data from their human trials of similar vaccines for
Ebola, MERS and malaria, though, the scientists at Oxford's institute
persuaded British regulators to allow unusually accelerated trials while
the epidemic is still hot around them.

The institute last week began a Phase I clinical trial involving 1,100
people. Crucially, next month it will begin a combined Phase II and
Phase III trial involving another 5,000. Unlike any other vaccine
project now underway, that trial is designed to prove effectiveness as
well as safety.

The scientists would declare victory if as many as a dozen participants
who are given a placebo become sick with Covid-19 compared with only one
or two who receive the inoculation. ``Then we have a party and tell the
world,'' Professor Hill said. Everyone who had received only the placebo
would also be vaccinated immediately.

If too few participants are infected in Britain, the institute is
planning other trials where the coronavirus may still be spreading,
possibly in Africa or India.

``We'll have to chase the epidemic,'' Professor Hill said. ``If it is
still raging in certain states, it is not inconceivable we end up
testing in the United States in November.''

Carl Zimmer contributed reporting.

Advertisement

\protect\hyperlink{after-bottom}{Continue reading the main story}

\hypertarget{site-index}{%
\subsection{Site Index}\label{site-index}}

\hypertarget{site-information-navigation}{%
\subsection{Site Information
Navigation}\label{site-information-navigation}}

\begin{itemize}
\tightlist
\item
  \href{https://help.nytimes.com/hc/en-us/articles/115014792127-Copyright-notice}{©~2020~The
  New York Times Company}
\end{itemize}

\begin{itemize}
\tightlist
\item
  \href{https://www.nytco.com/}{NYTCo}
\item
  \href{https://help.nytimes.com/hc/en-us/articles/115015385887-Contact-Us}{Contact
  Us}
\item
  \href{https://www.nytco.com/careers/}{Work with us}
\item
  \href{https://nytmediakit.com/}{Advertise}
\item
  \href{http://www.tbrandstudio.com/}{T Brand Studio}
\item
  \href{https://www.nytimes.com/privacy/cookie-policy\#how-do-i-manage-trackers}{Your
  Ad Choices}
\item
  \href{https://www.nytimes.com/privacy}{Privacy}
\item
  \href{https://help.nytimes.com/hc/en-us/articles/115014893428-Terms-of-service}{Terms
  of Service}
\item
  \href{https://help.nytimes.com/hc/en-us/articles/115014893968-Terms-of-sale}{Terms
  of Sale}
\item
  \href{https://spiderbites.nytimes.com}{Site Map}
\item
  \href{https://help.nytimes.com/hc/en-us}{Help}
\item
  \href{https://www.nytimes.com/subscription?campaignId=37WXW}{Subscriptions}
\end{itemize}
