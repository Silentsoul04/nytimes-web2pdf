Sections

SEARCH

\protect\hyperlink{site-content}{Skip to
content}\protect\hyperlink{site-index}{Skip to site index}

\href{https://www.nytimes.com/section/nyregion}{New York}

\href{https://myaccount.nytimes.com/auth/login?response_type=cookie\&client_id=vi}{}

\href{https://www.nytimes.com/section/todayspaper}{Today's Paper}

\href{/section/nyregion}{New York}\textbar{}Samuel Hargress Jr., Owner
of a Beloved Harlem Bar, Dies at 84

\url{https://nyti.ms/2KHR9Rm}

\begin{itemize}
\item
\item
\item
\item
\item
\item
\end{itemize}

\href{https://www.nytimes.com/news-event/coronavirus?action=click\&pgtype=Article\&state=default\&region=TOP_BANNER\&context=storylines_menu}{The
Coronavirus Outbreak}

\begin{itemize}
\tightlist
\item
  live\href{https://www.nytimes.com/2020/08/03/world/coronavirus-covid-19.html?action=click\&pgtype=Article\&state=default\&region=TOP_BANNER\&context=storylines_menu}{Latest
  Updates}
\item
  \href{https://www.nytimes.com/interactive/2020/us/coronavirus-us-cases.html?action=click\&pgtype=Article\&state=default\&region=TOP_BANNER\&context=storylines_menu}{Maps
  and Cases}
\item
  \href{https://www.nytimes.com/interactive/2020/science/coronavirus-vaccine-tracker.html?action=click\&pgtype=Article\&state=default\&region=TOP_BANNER\&context=storylines_menu}{Vaccine
  Tracker}
\item
  \href{https://www.nytimes.com/2020/08/02/us/covid-college-reopening.html?action=click\&pgtype=Article\&state=default\&region=TOP_BANNER\&context=storylines_menu}{College
  Reopening}
\item
  \href{https://www.nytimes.com/live/2020/08/03/business/stock-market-today-coronavirus?action=click\&pgtype=Article\&state=default\&region=TOP_BANNER\&context=storylines_menu}{Economy}
\end{itemize}

Advertisement

\protect\hyperlink{after-top}{Continue reading the main story}

Supported by

\protect\hyperlink{after-sponsor}{Continue reading the main story}

Those We've Lost

\hypertarget{samuel-hargress-jr-owner-of-a-beloved-harlem-bar-dies-at-84}{%
\section{Samuel Hargress Jr., Owner of a Beloved Harlem Bar, Dies at
84}\label{samuel-hargress-jr-owner-of-a-beloved-harlem-bar-dies-at-84}}

He ran Paris Blues, a throwback to the '60s that attracted locals and
tourists alike and seemed to be an extension of himself. He died of the
coronavirus.

\includegraphics{https://static01.nyt.com/images/2020/04/30/obituaries/30hargress1/30hargress1-articleLarge-v2.jpg?quality=75\&auto=webp\&disable=upscale}

\href{https://www.nytimes.com/by/steven-kurutz}{\includegraphics{https://static01.nyt.com/images/2018/09/25/multimedia/author-steven-kurutz/author-steven-kurutz-thumbLarge.png}}

By \href{https://www.nytimes.com/by/steven-kurutz}{Steven Kurutz}

\begin{itemize}
\item
  April 29, 2020
\item
  \begin{itemize}
  \item
  \item
  \item
  \item
  \item
  \item
  \end{itemize}
\end{itemize}

\emph{This obituary is part of a series about people who have died in
the coronavirus pandemic. Read about others}
\href{https://www.nytimes.com/series/people-who-have-died-of-the-coronavirus}{\emph{here}}\emph{.}

At Paris Blues, a neighborhood bar at Adam Clayton Powell Jr. Boulevard
and 121st Street in Harlem, jazz or blues bands played on a tiny stage
under a string of blue lights and a photograph of Malcolm X. There was
always free food, like chicken and rice, in a crockpot on a table. And
sitting at the bar or outside on the patio greeting customers was the
owner, Samuel Hargress Jr., elegant in his signature three-piece suit,
fedora and dark sunglasses.

Tourists and locals alike appreciated how Paris Blues evoked the Harlem
of the 1950s, '60s and '70s. But Mr. Hargress didn't intentionally
create a time capsule. He embodied that lost world, and remained loyal
to it as the city changed around him.

``This is what he put his blood, sweat and tears into,'' said his son,
Sam Hargress III. ``He made the bar almost an extension of himself.''

Mr. Hargress died on April 10 at Mount Sinai Morningside Hospital in
Manhattan. He was 84. His son said the cause was complications of the
novel coronavirus.

For Mr. Hargress, Paris Blues was quite literally an extension of his
home. He lived in an apartment above the bar. Downstairs, patrons found
within its wood-paneled walls the comfort and intimacy of a cozy living
room, where they would crowd around the bar and fill a row of wooden
booths that Mr. Hargress had built himself.

He fostered an egalitarian and family spirit there, employing the
sisters Judith Escalante and Esther Stokes as bartenders for many years
and making Sue Kelly the day manager. He once
\href{https://archive.nytimes.com/query.nytimes.com/gst/fullpage-9A0CEFDB173CF932A05756C0A9669D8B63.html}{had
business cards printed} listing the names of every employee, including
one identified as ``Disco \#1 Man.''

Mr. Hargress displayed a chalkboard of his regular customers' birthdays,
so they could be celebrated with cake and a singalong. Once, after a
late night, he went and got his Cadillac Escalade and drove one of the
regulars, Enrique Justiniano, home to his wife.

``Sam was the custodian of, the soul ambassador of, that culture of
community,'' the chef Marcus Samuelsson said in a phone interview. When
Mr. Samuelsson moved to Harlem in 2003 with plans to open his restaurant
\href{https://www.nytimes.com/2011/03/09/dining/reviews/09rest.html}{Red
Rooster Harlem}, he sought out Mr. Hargress. ``It didn't matter if you
came from downtown, Asia, Africa, Brooklyn,'' he said. ``Once you were
in the bar, you were in Sam's house.''

Mr. Samuelsson added: ``And Sam was no fool. Sam owned the building. He
saw ahead.''

Mr. Hargress's enviable position as his own landlord --- he bought the
five-story building decades ago, reportedly for \$38,000 --- gave him
leeway when business was slow and protected him from the real-estate
pressures that had sunk his competitors. He turned down
multimillion-dollar offers to sell out. And as Harlem gentrified, he
greeted the changes philosophically.

``It's not good or bad,'' he said in a short 2010
\href{https://viewing.nyc/mr-blues-a-profile-of-harlems-paris-blues-bar-and-owner-samuel-hargress-jr/}{documentary
about the bar}. ``It just happened. And you cannot stop it.''

As landmarks like Seville Lounge, St. Nick's Jazz Pub and
\href{https://www.nytimes.com/2012/12/08/nyregion/harlem-to-say-goodbye-to-the-lenox-lounge.html}{Lenox
Lounge closed,} Paris Blues appeared only more unique. Performances by
groups like \href{https://www.youtube.com/watch?v=oqxQsgDyj1g}{the Les
Goodson Band}, which played every Wednesday night for years, packed the
joint. It became a lively hangout for local musicians, and a destination
for tourists, especially Europeans, who had a romanticized view of
Harlem that Paris Blues fulfilled.

When Christina Kallas, a Greek-born, Harlem-based filmmaker, moved to
the city eight years ago, it hardly matched the New York she had seen in
movies. Then she walked into Paris Blues.

``It instantly reminded me of that place of my imagination,''
\href{http://www.talkhouse.com/harlem-blues-remembering-sam-hargress-jr/}{said
Ms. Kallas, who began filming inside the bar} for a movie, ``Paris Is in
Harlem.'' ``It was the perfect bar.''

\includegraphics{https://static01.nyt.com/images/2020/04/30/obituaries/30hargress2/merlin_35746201_e7dce34e-3f55-4e69-a0ea-9052763642c4-articleLarge.jpg?quality=75\&auto=webp\&disable=upscale}

Samuel Hargress Jr. was born on April 9, 1936, in Demopolis, in
west-central Alabama, to the Rev. Samuel Hargress Sr., a Baptist
minister, and Kate Hargress.

Mr. Hargress was drafted into the Army in 1959, and upon his discharge
migrated north to New York and entered the bar business, first as a
bartender. He opened Paris Blues in 1969, the name partly inspired by
the
\href{https://www.smithsonianmag.com/history/one-hundred-years-ago-harlem-hellfighters-bravely-led-us-wwi-180968977/}{Harlem
Hellfighters}, a celebrated African-American infantry regiment in World
War I that had been honored by France.

``Black soldiers who served in France were treated so much better there
than at home,'' Mr. Hargress was quoted as saying
\href{http://amsterdamnews.com/news/2020/apr/23/paris-blues-owner-dies/}{in
an obituary in The New York Amsterdam News}. ``I named Paris Blues to
honor the city, the soldiers and the music I grew up listening to and
love.''

In the 1970s and '80s, Paris Blues was a mainly a hangout for people
from the neighborhood, providing music from a jukebox. Over time, as Mr.
Hargress booked live bands and welcomed newcomers, the bar blossomed
into ``an international, integrated scene that was authentic,'' said
\href{https://rakiemwalker.com/}{Rakiem Walker}, one of the many
musicians Mr. Hargress hired and supported.

Mr. Hargress's influence extended beyond the bar. He helped fund block
parties and other community events, and he counseled Mr. Walker and
others about life matters. ``He was the elder black man who could give
you the shortcut knowledge about the better choice,'' Mr. Walker said.

Mr. Hargress settled on a look back in the '70s --- three-piece suits in
a variety of colors, snakeskin or alligator shoes, a thin mustache ---
and happily stuck with it, though he did sometimes substitute a big
Stetson for a fedora, or what he called a ``godfather hat,'' of which he
claimed to own as many as 46.

Despite working for more than 50 years in the nightlife business, Mr.
Hargress, a man of few words, revealed in the documentary: ``I don't
drink. Never drank. Don't smoke. Never smoked.''

In addition to his son, he is survived by another son, Franklin; a
daughter, Samantha Hargress; and a stepson, Michael Stewart.

Sam Hargress III said that in February and early March, as New York's
bars and restaurants temporarily closed amid the coronavirus pandemic,
his father found it upsetting not to be downstairs in his beloved
lounge. Though he was dutifully staying home, he fell ill.

With Mr. Hargress's death, patrons are mourning not only a man but also
a place, so intertwined were the two. His son plans to keep Paris Blues
going. But no one disagrees that it won't be the same.

\href{https://www.nytimes.com/interactive/2020/obituaries/people-died-coronavirus-obituaries.html?action=click\&pgtype=Article\&state=default\&region=BELOW_MAIN_CONTENT\&context=covid_obits_promo}{}

\hypertarget{those-weve-lost}{%
\section{Those We've Lost}\label{those-weve-lost}}

The coronavirus pandemic has taken an incalculable death toll. This
series is designed to put names and faces to the numbers.

Read more

\includegraphics{https://static01.nyt.com/images/2020/07/30/obituaries/30Pedro/30Pedro-square640.jpg}

\hypertarget{bernaldina-josuxe9-pedro}{%
\section{Bernaldina José Pedro}\label{bernaldina-josuxe9-pedro}}

d. Boa Vista, Brazil

Leader among the Indigenous Macuxi

\includegraphics{https://static01.nyt.com/images/2020/07/31/obituaries/31Swing/merlin_175167783_8913bc90-0d64-43f3-a655-1bb1bf1601c9-square640.jpg}

\hypertarget{john-eric-swing}{%
\section{John Eric Swing}\label{john-eric-swing}}

d. Fountain Valley, Calif.

Champion of Filipino-Americans

\includegraphics{https://static01.nyt.com/images/2020/07/27/obituaries/27Victor/merlin_175001436_38b11f8e-227a-4e2c-9821-7618af9b2524-square640.jpg}

\hypertarget{victor-victor}{%
\section{Victor Victor}\label{victor-victor}}

d. Santo Domingo, Dominican Republic

Beloved musician of the Dominican Republic

\includegraphics{https://static01.nyt.com/images/2020/07/31/obituaries/31Negron/merlin_175160169_516322ae-fd23-4969-b6b2-193ced371105-square640.jpg}

\hypertarget{dr-eddie-negruxf3n}{%
\section{Dr. Eddie Negrón}\label{dr-eddie-negruxf3n}}

d. Fort Walton Beach, Fla.

Internist on Florida's Emerald Coast

\includegraphics{https://static01.nyt.com/images/2020/07/30/obituaries/30Dobson/merlin_175115928_f6b9271c-8f05-4fe1-a38a-5ca4a58f8935-square640.jpg}

\hypertarget{dobby-dobson}{%
\section{Dobby Dobson}\label{dobby-dobson}}

d. Coral Springs, Fla.

Jamaican singer and songwriter

\includegraphics{https://static01.nyt.com/images/2020/08/01/obituaries/28Gonzalez/merlin_175002771_beb57888-3951-409a-ae13-03a94b2e962e-square640.jpg}

\hypertarget{waldemar-gonzalez}{%
\section{Waldemar Gonzalez}\label{waldemar-gonzalez}}

d. White Plains, N.Y.

Teacher and social worker

Advertisement

\protect\hyperlink{after-bottom}{Continue reading the main story}

\hypertarget{site-index}{%
\subsection{Site Index}\label{site-index}}

\hypertarget{site-information-navigation}{%
\subsection{Site Information
Navigation}\label{site-information-navigation}}

\begin{itemize}
\tightlist
\item
  \href{https://help.nytimes.com/hc/en-us/articles/115014792127-Copyright-notice}{©~2020~The
  New York Times Company}
\end{itemize}

\begin{itemize}
\tightlist
\item
  \href{https://www.nytco.com/}{NYTCo}
\item
  \href{https://help.nytimes.com/hc/en-us/articles/115015385887-Contact-Us}{Contact
  Us}
\item
  \href{https://www.nytco.com/careers/}{Work with us}
\item
  \href{https://nytmediakit.com/}{Advertise}
\item
  \href{http://www.tbrandstudio.com/}{T Brand Studio}
\item
  \href{https://www.nytimes.com/privacy/cookie-policy\#how-do-i-manage-trackers}{Your
  Ad Choices}
\item
  \href{https://www.nytimes.com/privacy}{Privacy}
\item
  \href{https://help.nytimes.com/hc/en-us/articles/115014893428-Terms-of-service}{Terms
  of Service}
\item
  \href{https://help.nytimes.com/hc/en-us/articles/115014893968-Terms-of-sale}{Terms
  of Sale}
\item
  \href{https://spiderbites.nytimes.com}{Site Map}
\item
  \href{https://help.nytimes.com/hc/en-us}{Help}
\item
  \href{https://www.nytimes.com/subscription?campaignId=37WXW}{Subscriptions}
\end{itemize}
