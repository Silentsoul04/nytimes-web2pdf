Sections

SEARCH

\protect\hyperlink{site-content}{Skip to
content}\protect\hyperlink{site-index}{Skip to site index}

\href{https://www.nytimes.com/section/arts}{Arts}

\href{https://myaccount.nytimes.com/auth/login?response_type=cookie\&client_id=vi}{}

\href{https://www.nytimes.com/section/todayspaper}{Today's Paper}

\href{/section/arts}{Arts}\textbar{}Germano Celant, Curator Behind
Italy's Arte Povera, Dies at 79

\url{https://nyti.ms/2W8UNJm}

\begin{itemize}
\item
\item
\item
\item
\item
\end{itemize}

\href{https://www.nytimes.com/news-event/coronavirus?action=click\&pgtype=Article\&state=default\&region=TOP_BANNER\&context=storylines_menu}{The
Coronavirus Outbreak}

\begin{itemize}
\tightlist
\item
  live\href{https://www.nytimes.com/2020/08/03/world/coronavirus-covid-19.html?action=click\&pgtype=Article\&state=default\&region=TOP_BANNER\&context=storylines_menu}{Latest
  Updates}
\item
  \href{https://www.nytimes.com/interactive/2020/us/coronavirus-us-cases.html?action=click\&pgtype=Article\&state=default\&region=TOP_BANNER\&context=storylines_menu}{Maps
  and Cases}
\item
  \href{https://www.nytimes.com/interactive/2020/science/coronavirus-vaccine-tracker.html?action=click\&pgtype=Article\&state=default\&region=TOP_BANNER\&context=storylines_menu}{Vaccine
  Tracker}
\item
  \href{https://www.nytimes.com/2020/08/02/us/covid-college-reopening.html?action=click\&pgtype=Article\&state=default\&region=TOP_BANNER\&context=storylines_menu}{College
  Reopening}
\item
  \href{https://www.nytimes.com/live/2020/08/03/business/stock-market-today-coronavirus?action=click\&pgtype=Article\&state=default\&region=TOP_BANNER\&context=storylines_menu}{Economy}
\end{itemize}

Advertisement

\protect\hyperlink{after-top}{Continue reading the main story}

Supported by

\protect\hyperlink{after-sponsor}{Continue reading the main story}

Those we've lost

\hypertarget{germano-celant-curator-behind-italys-arte-povera-dies-at-79}{%
\section{Germano Celant, Curator Behind Italy's Arte Povera, Dies at
79}\label{germano-celant-curator-behind-italys-arte-povera-dies-at-79}}

He identified the ``poor art'' avant-garde movement and was an
internationally renowned curator at the Guggenheim Museum in New York
and the Fondazione Prada in Milan.

\includegraphics{https://static01.nyt.com/images/2020/05/03/obituaries/03Celant-obit1/merlin_172040721_65dc2aae-56f2-4870-bcad-12834acef83d-articleLarge.jpg?quality=75\&auto=webp\&disable=upscale}

By \href{https://www.nytimes.com/by/jason-farago}{Jason Farago}

\begin{itemize}
\item
  Published April 30, 2020Updated May 4, 2020
\item
  \begin{itemize}
  \item
  \item
  \item
  \item
  \item
  \end{itemize}
\end{itemize}

\emph{This obituary is part of a series about people who have died in
the coronavirus pandemic. Read about others}
\href{https://www.nytimes.com/series/people-who-have-died-of-the-coronavirus}{\emph{here}}\emph{.}

Germano Celant, an influential curator, critic and art historian who
brought postwar Italian art to international prominence, died on
Wednesday in Milan. He was 79.

The cause was complications of the new coronavirus. His death, at San
Raffaele Hospital, was confirmed by the Fondazione Prada, the Milan art
foundation that Mr. Celant collaborated with for more than two decades.

In 1967, Mr. Celant (pronounced CHAY-lant) wrote a lasting page in art
history when, as a 27-year-old curator in Genoa, he mounted an
exhibition of five young Italian artists making provisional assemblages
of humble materials, which he grouped under the term
\href{https://www.moma.org/collection/terms/7}{Arte Povera} (``poor
art'').

These artists, including Alghiero Boetti, Jannis Kounellis and Luciano
Fabro, bridled against the conventions of the Italian academies (and
American Pop art), and made a virtue of simple everyday objects: melted
wax, rusting iron, fallen leaves, ground coffee,
even\href{https://www.nytimes.com/2015/06/26/arts/design/review-art-that-snorts-from-jannis-kounellis-at-gavin-browns-enterprise.html}{horses
munching hay}.

Nourished by the anti-establishment sentiment that crested in the
student protests of 1968, the young Italians embraced a newly pragmatic
art that paid particular attention to the body and the environment. Mr.
Celant was in their corner with exhibitions, magazine
\href{https://flash---art.com/article/germano-celant-arte-povera-notes-on-a-guerrilla-war/}{articles}
and an influential 1969 book, ``Arte Povera,'' which collected and
analyzed the art of
\href{https://www.nytimes.com/2012/06/29/arts/design/alighiero-boetti-retrospective-at-museum-of-modern-art.html}{Mr.
Boetti},
\href{https://www.nytimes.com/2017/02/21/arts/design/jannis-kounellis-died-sculptor.html}{Mr.
Kounellis}, Giuseppe Penone, Giovanni Anselmo and others who went on to
international acclaim.

Each of these artists, Mr. Celant wrote in the introduction, ``has
chosen to live within direct experience'' and ``feels the necessity of
leaving intact the value of the existence of things.''

Energetic and urbane, with a mane of silver hair in vivid contrast with
his usual all-black wardrobe, Mr. Celant championed this generation of
artists throughout his 50-year career, presenting large exhibitions of
Arte Povera at the Centre Pompidou in Paris and
\href{https://www.nytimes.com/1985/10/13/arts/conceptual-art-italian-style-makes-a-statement-at-ps-1.html}{at
P.S. 1 in New York City}.

He was also a key theorist of experimental design. In the 1970s, he
advocated the ``radical architecture'' of Archizoom, Superstudio and
other boundary-breaking firms in Florence. At the Venice Biennale of
1976, he
\href{https://www.nytimes.com/1976/07/25/archives/the-biennale-a-show-of-compromises.html}{assembled
the groundbreaking presentation ``Arte e Ambiente''} (``Art and
Space''), recreating environments by earlier artists like Wassily
Kandinsky, and turning over entire rooms to living artists like Sol
LeWitt.

Mr. Celant was the author or editor of more than 200 books, including
catalogs raisonnés of Piero Manzoni, Mimmo Rotella and Carla Accardi.
And from 1995 until his death, he steered the programming at Fondazione
Prada, the art foundation created by the fashion designer Miuccia Prada
and her husband, Patrizio Bertelli, which has become one of Europe's
most important centers for contemporary art.

Yet for all his activity, Mr. Celant almost never worked in a full-time
capacity in a single institution, holding on as well as he could to the
freedom he celebrated with Arte Povera.

``This is a way of being,'' he wrote in 1967, ``that asks only for
essential information, that refuses dialogue with both the social and
the cultural systems, and that aspires to present itself as something
sudden and unforeseen.''

\includegraphics{https://static01.nyt.com/images/2020/05/03/obituaries/03Celant-obit2/merlin_169691913_fe1809dc-1c48-4cd7-a486-f8302911c40f-articleLarge.jpg?quality=75\&auto=webp\&disable=upscale}

Germano Celant was born on Sept. 11, 1940, in Genoa, where his father
worked in that port city's import-export industry. He studied art
history at the University of Genoa under Eugenio Battisti, a scholar of
Renaissance and Baroque painting who created the short-lived Museo
Sperimentale, Genoa's first significant contemporary art space.

Mr. Celant got his first taste of the art world there and at
\href{http://www.capti.it/index.php?ParamCatID=10\&IDFascicolo=192\&artgal=38\&lang=EN}{Marcatr}è,
a culture magazine founded by Mr. Battisti. Mr. Celant soon became the
magazine's editor, and he began traveling to Milan, Rome and especially
Turin, the epicenter of new Italian art, where he befriended artists
like Mr. Kounellis and Michelangelo Pistoletto, and dealers like Gian
Enzo Sperone, the first Italian to show Andy Warhol and other Americans.

With his exhibitions and books, Mr. Celant gave form to a new Italian
avant-garde, although, as
\href{https://www.repubblica.it/cultura/2017/05/07/news/germano_celant_non_dite_che_ho_inventato_l_arte_povera_e_un_espressione_cosi_ampia_da_non_significare_nulla_-164857041/}{he
told the newspaper La Repubblica} in 2017: ``I didn't invent anything.
Arte Povera is an expression so broad that it means nothing. It doesn't
specify a pictorial language, but an attitude.''

In 1988 the Solomon R. Guggenheim Museum in New York made him a curator
of contemporary art. (``I have often been accused in Europe of being
pro-American and in the United States of being pro-European,''
\href{https://www.nytimes.com/1988/12/01/arts/guggenheim-names-curator.html}{he
told The New York Times} upon his appointment.) His most significant
exhibition there was
``\href{https://www.nytimes.com/1994/10/07/arts/art-review-from-postwar-italy-with-style.html}{The
Italian Metamorphosis, 1943--1968},'' an immense survey of high and
popular art from the postwar republic to the swinging years of la dolce
vita ** and Arte Povera. He remained with the Guggenheim until 2008.

Though he once said, ``I hate group shows,'' he assisted in the 1982
edition of the megashow Documenta in Kassel, Germany, and in 1997 he
curated a widely-panned
\href{https://www.nytimes.com/1997/06/16/arts/another-venice-biennale-shuffles-to-life.html}{edition
of the Venice Biennale}. A 2015 exhibition at the
\href{https://www.nytimes.com/2015/05/04/arts/international/at-milan-worlds-fair-164-years-of-food-and-art.html}{Milan
Expo, on the subject of art and food}, was also criticized, though
mostly for the eye-watering 750,000 euros he was paid.

More recently he lent his credibility to exhibitions by KAWS, a hugely
popular street artist, in Qatar and Hong Kong. (Francesco Bonami, an
Italian
curator,\href{https://www.instagram.com/p/BvbdW6_F3Hl/?hl=en}{acidly
analogized} it to the superstar Zinedine Zidane's playing for a B-league
soccer team.)

Image

Part of the group show ``Post Zang Tumb Tuuum'' in Milan. It was one of
the most important exhibitions of Mr. Celant's
career.Credit...Fondazione Prada

Despite his aversion to group shows, Mr. Celant completed one of the
most important exhibitions of his career in 2018:
``\href{https://www.nytimes.com/2018/03/21/arts/design/italian-art-fondazione-prada-palazzo-strozzi.html}{Post
Zang Tumb Tuuum},'' a sprawling, year-by-year study of the art of
Fascist Italy that took over the entire Milan campus of the Fondazione
Prada.

Tremendous in both size and ambition, this final major show intermingled
the hushed still lifes of Giorgio Morandi with propaganda and garish
marble supermen, revealing the surprising diversity of Italian art of
the interwar years and the flexibility of Fascist ideology.

Mr. Celant is survived by his wife, Paris Murray Celant, and their son,
Argento.

In 2017,
\href{https://www.artagencypartners.com/episode-19-transcript-germano-celant-and-allan-schwartzman/}{speaking
on a podcast with Charlotte Burns}, Mr. Celant recalled his early
encounters with Luciano Fabro and others of the new Italian avant-garde,
and outlined the unstoppable curiosity that drove his curatorial and
critical career:

``What I learn from an artist like Fabro, he said: `In order to judge an
artist, you need 30 years to see if it is important or not.' I'm not
waiting 30 years. I don't have this kind of time.''

\href{https://www.nytimes.com/interactive/2020/obituaries/people-died-coronavirus-obituaries.html?action=click\&pgtype=Article\&state=default\&region=BELOW_MAIN_CONTENT\&context=covid_obits_promo}{}

\hypertarget{those-weve-lost}{%
\section{Those We've Lost}\label{those-weve-lost}}

The coronavirus pandemic has taken an incalculable death toll. This
series is designed to put names and faces to the numbers.

Read more

\includegraphics{https://static01.nyt.com/images/2020/07/30/obituaries/30Pedro/30Pedro-square640.jpg}

\hypertarget{bernaldina-josuxe9-pedro}{%
\section{Bernaldina José Pedro}\label{bernaldina-josuxe9-pedro}}

d. Boa Vista, Brazil

Leader among the Indigenous Macuxi

\includegraphics{https://static01.nyt.com/images/2020/07/31/obituaries/31Swing/merlin_175167783_8913bc90-0d64-43f3-a655-1bb1bf1601c9-square640.jpg}

\hypertarget{john-eric-swing}{%
\section{John Eric Swing}\label{john-eric-swing}}

d. Fountain Valley, Calif.

Champion of Filipino-Americans

\includegraphics{https://static01.nyt.com/images/2020/07/27/obituaries/27Victor/merlin_175001436_38b11f8e-227a-4e2c-9821-7618af9b2524-square640.jpg}

\hypertarget{victor-victor}{%
\section{Victor Victor}\label{victor-victor}}

d. Santo Domingo, Dominican Republic

Beloved musician of the Dominican Republic

\includegraphics{https://static01.nyt.com/images/2020/07/31/obituaries/31Negron/merlin_175160169_516322ae-fd23-4969-b6b2-193ced371105-square640.jpg}

\hypertarget{dr-eddie-negruxf3n}{%
\section{Dr. Eddie Negrón}\label{dr-eddie-negruxf3n}}

d. Fort Walton Beach, Fla.

Internist on Florida's Emerald Coast

\includegraphics{https://static01.nyt.com/images/2020/07/30/obituaries/30Dobson/merlin_175115928_f6b9271c-8f05-4fe1-a38a-5ca4a58f8935-square640.jpg}

\hypertarget{dobby-dobson}{%
\section{Dobby Dobson}\label{dobby-dobson}}

d. Coral Springs, Fla.

Jamaican singer and songwriter

\includegraphics{https://static01.nyt.com/images/2020/08/01/obituaries/28Gonzalez/merlin_175002771_beb57888-3951-409a-ae13-03a94b2e962e-square640.jpg}

\hypertarget{waldemar-gonzalez}{%
\section{Waldemar Gonzalez}\label{waldemar-gonzalez}}

d. White Plains, N.Y.

Teacher and social worker

Advertisement

\protect\hyperlink{after-bottom}{Continue reading the main story}

\hypertarget{site-index}{%
\subsection{Site Index}\label{site-index}}

\hypertarget{site-information-navigation}{%
\subsection{Site Information
Navigation}\label{site-information-navigation}}

\begin{itemize}
\tightlist
\item
  \href{https://help.nytimes.com/hc/en-us/articles/115014792127-Copyright-notice}{©~2020~The
  New York Times Company}
\end{itemize}

\begin{itemize}
\tightlist
\item
  \href{https://www.nytco.com/}{NYTCo}
\item
  \href{https://help.nytimes.com/hc/en-us/articles/115015385887-Contact-Us}{Contact
  Us}
\item
  \href{https://www.nytco.com/careers/}{Work with us}
\item
  \href{https://nytmediakit.com/}{Advertise}
\item
  \href{http://www.tbrandstudio.com/}{T Brand Studio}
\item
  \href{https://www.nytimes.com/privacy/cookie-policy\#how-do-i-manage-trackers}{Your
  Ad Choices}
\item
  \href{https://www.nytimes.com/privacy}{Privacy}
\item
  \href{https://help.nytimes.com/hc/en-us/articles/115014893428-Terms-of-service}{Terms
  of Service}
\item
  \href{https://help.nytimes.com/hc/en-us/articles/115014893968-Terms-of-sale}{Terms
  of Sale}
\item
  \href{https://spiderbites.nytimes.com}{Site Map}
\item
  \href{https://help.nytimes.com/hc/en-us}{Help}
\item
  \href{https://www.nytimes.com/subscription?campaignId=37WXW}{Subscriptions}
\end{itemize}
