Sections

SEARCH

\protect\hyperlink{site-content}{Skip to
content}\protect\hyperlink{site-index}{Skip to site index}

\href{https://www.nytimes.com/section/business/economy}{Economy}

\href{https://myaccount.nytimes.com/auth/login?response_type=cookie\&client_id=vi}{}

\href{https://www.nytimes.com/section/todayspaper}{Today's Paper}

\href{/section/business/economy}{Economy}\textbar{}Stymied in Seeking
Benefits, Millions of Unemployed Go Uncounted

\url{https://nyti.ms/2Wguq4p}

\begin{itemize}
\item
\item
\item
\item
\item
\item
\end{itemize}

\href{https://www.nytimes.com/news-event/coronavirus?action=click\&pgtype=Article\&state=default\&region=TOP_BANNER\&context=storylines_menu}{The
Coronavirus Outbreak}

\begin{itemize}
\tightlist
\item
  live\href{https://www.nytimes.com/2020/08/04/world/coronavirus-cases.html?action=click\&pgtype=Article\&state=default\&region=TOP_BANNER\&context=storylines_menu}{Latest
  Updates}
\item
  \href{https://www.nytimes.com/interactive/2020/us/coronavirus-us-cases.html?action=click\&pgtype=Article\&state=default\&region=TOP_BANNER\&context=storylines_menu}{Maps
  and Cases}
\item
  \href{https://www.nytimes.com/interactive/2020/science/coronavirus-vaccine-tracker.html?action=click\&pgtype=Article\&state=default\&region=TOP_BANNER\&context=storylines_menu}{Vaccine
  Tracker}
\item
  \href{https://www.nytimes.com/2020/08/02/us/covid-college-reopening.html?action=click\&pgtype=Article\&state=default\&region=TOP_BANNER\&context=storylines_menu}{College
  Reopening}
\item
  \href{https://www.nytimes.com/live/2020/08/04/business/stock-market-today-coronavirus?action=click\&pgtype=Article\&state=default\&region=TOP_BANNER\&context=storylines_menu}{Economy}
\end{itemize}

Advertisement

\protect\hyperlink{after-top}{Continue reading the main story}

Supported by

\protect\hyperlink{after-sponsor}{Continue reading the main story}

\hypertarget{stymied-in-seeking-benefits-millions-of-unemployed-go-uncounted}{%
\section{Stymied in Seeking Benefits, Millions of Unemployed Go
Uncounted}\label{stymied-in-seeking-benefits-millions-of-unemployed-go-uncounted}}

As state agencies grapple with new guidelines and sheer volume, many
workers are frustrated in filing claims and omitted from jobless
tallies.

6

million

30,307,000

5

Claims were filed in

the last six weeks

4

Initial jobless claims, per week

Seasonally adjusted

3

2

RECESSION

1

0

'04

'08

'09

'12

'16

'20

6

million

30,307,000

5

Claims were filed in

the last six weeks

4

3

RECESSION

2

Initial jobless claims, per week

Seasonally adjusted

1

0

'04

'08

'09

'12

'16

'20

30,307,000

6

million

Claims were filed in

the last six weeks

5

Initial jobless claims, per week

4

Seasonally adjusted

3

RECESSION

2

1

0

'04

'08

'09

'12

'16

'20

30,307,000

6

million

Claims were filed in

the last six weeks

5

4

3

RECESSION

2

Initial jobless claims, per week

Seasonally adjusted

1

0

'04

'08

'09

'12

'16

'20

Source: Department of Labor

By The New York Times

By \href{https://www.nytimes.com/by/nelson-d-schwartz}{Nelson D.
Schwartz}, \href{https://www.nytimes.com/by/tiffany-hsu}{Tiffany Hsu}
and \href{https://www.nytimes.com/by/patricia-cohen}{Patricia Cohen}

\begin{itemize}
\item
  Published April 30, 2020Updated June 11, 2020
\item
  \begin{itemize}
  \item
  \item
  \item
  \item
  \item
  \item
  \end{itemize}
\end{itemize}

With a flood of
\href{https://www.nytimes.com/2020/05/21/business/economy/coronavirus-unemployment-claims.html}{unemployment}
claims continuing to overwhelm many state agencies, economists say the
job losses may be far worse than government tallies indicate.

The Labor Department said Thursday that 3.8 million workers filed for
\href{https://www.nytimes.com/2020/06/11/us/politics/unemployment-benefits-coronavirus.html}{unemployment
benefits} last week, bringing the six-week total to 30 million. But
researchers say that as the economy staggers under the weight of the
coronavirus pandemic, millions of others have lost jobs but have yet to
see benefits.

A
\href{https://www.epi.org/blog/unemployment-filing-failures-new-survey-confirms-that-millions-of-jobless-were-unable-to-file-an-unemployment-insurance-claim/}{study
by the Economic Policy Institute} found that roughly 50 percent more
people than counted as filing claims in a recent four-week period may
have qualified for benefits --- with the difference representing those
who were stymied in applying or didn't even try because the process was
too formidable.

``The problem is even bigger than the data suggest,'' said Elise Gould,
a senior economist with the institute, a left-leaning research group.
``We're undercounting the economic pain.''

Alexander Bick of Arizona State University and Adam Blandin of Virginia
Commonwealth University found that 42 percent of those working in
February
\href{https://alexbick.weebly.com/uploads/1/0/1/3/101306056/bb_covid.pdf}{had
lost their jobs or suffered a reduction in earnings}. By April 18, they
found, up to eight million workers were unemployed but not reflected in
the
\href{https://www.nytimes.com/2020/06/04/business/economy/coronavirus-unemployment-claims.html}{weekly
claims data}.

The difficulties at the state level largely flow from the sheer volume
of claims, which few agencies were prepared to handle. Many were
burdened by aging computer systems that were hard to reconfigure for new
federal guidelines.

``We've known that the state
\href{https://www.nytimes.com/2020/05/28/business/economy/coronavirus-unemployment-claims.html}{unemployment}
insurance systems were not up to the task, yet those investments were
not made,'' Ms. Gould said. ``The result is that the state systems are
buckling under the weight of these claims.''

The crush of claims is a major reason --- but not the only one --- that
states are backlogged. Frustrated applicants who refile their
applications, some as many as 20 times, slow the system as processors
weed out duplicates.

Some applications are missing information. New York analyzed a million
claims and found many had been delayed because of a missing employer
identification number. In such cases, each applicant has to be called
back. Callers looking for updates also flood the system, increasing the
wait for those who need to correct a mistake.

\hypertarget{latest-updates-economy}{%
\section{\texorpdfstring{\href{https://www.nytimes.com/live/2020/08/04/business/stock-market-today-coronavirus?action=click\&pgtype=Article\&state=default\&region=MAIN_CONTENT_1\&context=storylines_live_updates}{Latest
Updates:
Economy}}{Latest Updates: Economy}}\label{latest-updates-economy}}

\href{https://www.nytimes.com/live/2020/08/04/business/stock-market-today-coronavirus?action=click\&pgtype=Article\&state=default\&region=MAIN_CONTENT_1\&context=storylines_live_updates\#disney-lost-4-7-billion-last-quarter-but-its-newest-business-was-a-big-hit}{32m
ago}

\href{https://www.nytimes.com/live/2020/08/04/business/stock-market-today-coronavirus?action=click\&pgtype=Article\&state=default\&region=MAIN_CONTENT_1\&context=storylines_live_updates\#disney-lost-4-7-billion-last-quarter-but-its-newest-business-was-a-big-hit}{Disney
lost \$4.7 billion last quarter, but its newest business was a big hit.}

\href{https://www.nytimes.com/live/2020/08/04/business/stock-market-today-coronavirus?action=click\&pgtype=Article\&state=default\&region=MAIN_CONTENT_1\&context=storylines_live_updates\#the-ad-giant-publicis-has-parted-ways-with-an-executive-over-his-virus-tweets}{2h
ago}

\href{https://www.nytimes.com/live/2020/08/04/business/stock-market-today-coronavirus?action=click\&pgtype=Article\&state=default\&region=MAIN_CONTENT_1\&context=storylines_live_updates\#the-ad-giant-publicis-has-parted-ways-with-an-executive-over-his-virus-tweets}{The
ad giant Publicis has `parted ways' with an executive over his virus
tweets.}

\href{https://www.nytimes.com/live/2020/08/04/business/stock-market-today-coronavirus?action=click\&pgtype=Article\&state=default\&region=MAIN_CONTENT_1\&context=storylines_live_updates\#nbcuniversal-to-cut-about-10-percent-of-its-work-force}{3h
ago}

\href{https://www.nytimes.com/live/2020/08/04/business/stock-market-today-coronavirus?action=click\&pgtype=Article\&state=default\&region=MAIN_CONTENT_1\&context=storylines_live_updates\#nbcuniversal-to-cut-about-10-percent-of-its-work-force}{NBCUniversal
to cut about 10 percent of its work force.}

\href{https://www.nytimes.com/live/2020/08/04/business/stock-market-today-coronavirus?action=click\&pgtype=Article\&state=default\&region=MAIN_CONTENT_1\&context=storylines_live_updates}{See
more updates}

More live coverage:
\href{https://www.nytimes.com/2020/08/04/world/coronavirus-cases.html?action=click\&pgtype=Article\&state=default\&region=MAIN_CONTENT_1\&context=storylines_live_updates}{Global}

The seasonally adjusted number of people filing initial unemployment
claims is down from late March and early April, when more than six
million people applied for benefits two weeks in a row. But that's a
small consolation in light of the larger economic picture, economists
said. Before the pandemic, just over 200,000 people a week applied for
new unemployment benefits.

``It is declining, but the level is still breathtakingly high,'' said
Ian Shepherdson, chief economist at Pantheon Macroeconomics. ``Claims
could stay in the millions for several more weeks, which is almost
unfathomable.''

Mr. Shepherdson said job cuts now extended far beyond the industries
initially hit by the pandemic and the ensuing lockdown in most states,
like leisure and hospitality.

``You can't close a bar twice,'' he said. ``Layoffs are now working
their way through management and supply chains and business services.''

Millions who have managed to keep their jobs face salary cuts or
furloughs, a sign of employers' uncertainty. Given the trillions spent,
``we would have hoped that federal efforts would have been more
effective at stemming job losses,'' said Michael Gapen, chief U.S.
economist at Barclays.

Mr. Gapen said he expected the unemployment rate to hit 19.5 percent in
April, a level unseen since the Depression.

The federal stimulus efforts include an additional \$600 in weekly
unemployment benefits through one program, known as Federal Pandemic
Unemployment Compensation. Another, Pandemic Unemployment Assistance, is
aimed at independent contractors and so-called gig workers who don't
qualify for traditional unemployment coverage. Washington is also paying
for 13 weeks of benefits once state payments run out, an initiative
called Pandemic Emergency Unemployment Compensation.

According to the Labor Department, all 50 states are paying the \$600
weekly supplement, but only 23 have begun benefits under the program for
independent contractors, and only nine have started the 13-week extended
payments.

New Jersey has struggled to answer phone calls from filers like David
Schoonover, an actor from Jersey City who first applied for benefits on
March 23 after his show in New York City closed. All seemed normal at
the beginning, but his case was marked pending week after week when he
checked online.

\includegraphics{https://static01.nyt.com/images/2020/04/30/business/30virus-jobless2/merlin_172046613_22540f12-3d62-48dc-88e7-d6ebf9e569bf-articleLarge.jpg?quality=75\&auto=webp\&disable=upscale}

Unable to get through by phone, he searched for email addresses for
officials from the New Jersey Department of Labor and Workforce
Development and sent them messages. One responded, and his claim status
switched this week to filed from pending. The department scheduled a
call with him on June 3, and Mr. Schoonover said there was little he
could do to expedite the process, heightening the financial pressure on
him and his wife.

``Every week or so we get the calculator out and ask how much longer we
can go if we don't get unemployment benefits by June,'' Mr. Schoonover,
37, said. ``We're pinching every penny.''

New York has had fewer problems than some states, but the volume of
applicants is ``simply heartbreaking,'' Roberta Reardon, New York's
labor commissioner, said in a call Wednesday with news organizations.

The state is calling back everyone who has a problem with an
application, she said.

New York has started processing claims from gig workers and freelancers,
but one of those, Seth Flicker of Brooklyn, hasn't had any luck.

``Not a phone call nor an email, nothing,'' said Mr. Flicker, 52, who
applied in mid-March after his work as a handyman came to a halt. ``We
are stuck with absolutely nowhere to turn,'' he said, calling his
situation ``a Dante-esque limbo.''

Image

Seth Flicker of Brooklyn lost his work as a handyman and has been
stymied in filing for unemployment benefits. He is worried about making
his May rent.Credit...Laylah Amatullah Barrayn for The New York Times

Mr. Flicker was able to delay paying his electric bill without a penalty
and sent a check to the phone company, but he is worried about covering
May's rent. ``I haven't figured it out yet,'' he said. ``It's
nerve-racking.''

In Kentucky, where roughly a quarter of the work force is out of a job,
unemployed workers have faced waits of six hours or more when calling
about benefits.

One of those frustrated is Lauren Standifur, 30, who lost her front-desk
job at a hotel in Lexington, Ky. She says she has been unable to get
through to state agencies for weeks and has collected no benefits.

``My whole call log is filled,'' she said. ``It's close to 40 hours a
week --- if I got paid for making calls, I could do it as a full-time
job. But I haven't talked to a single human being.''

Ms. Standifur, who was furloughed on March 13, says she immediately
applied for benefits, only to be asked to check her status in two weeks.

When she did, she was told to come back in early April. Her online
profile with the Kentucky unemployment agency lists an expected \$291
weekly payment from the state and a weekly \$600 federal stimulus
payment. But nearly seven weeks after filing, she says, she has received
nothing.

Image

Lexington, Ky., where Lauren Standifur lost her job at a hotel. Her
calls to state agencies about benefits have been futile.Credit...Luke
Sharrett for The New York Times

Ms. Standifur says she calls various government numbers every day,
including the governor's office. But the call volume is always too high,
or a recording says the number is faulty, or she hits a busy signal. On
Tuesday, she started calling at 6:34 a.m.

Ms. Standifur, who lives with her mother and three nephews, said she had
applied for odd jobs and had strained her credit card to buy food,
settling for meals like peanut-butter-and-jelly sandwiches.

``I'm trying everything I can to get some money in so that when this is
all over, we don't have all these bills stacked up,'' she said. ``Every
day we go to bed and pray that it gets better. But every day, it feels
like it's getting worse.''

Advertisement

\protect\hyperlink{after-bottom}{Continue reading the main story}

\hypertarget{site-index}{%
\subsection{Site Index}\label{site-index}}

\hypertarget{site-information-navigation}{%
\subsection{Site Information
Navigation}\label{site-information-navigation}}

\begin{itemize}
\tightlist
\item
  \href{https://help.nytimes.com/hc/en-us/articles/115014792127-Copyright-notice}{©~2020~The
  New York Times Company}
\end{itemize}

\begin{itemize}
\tightlist
\item
  \href{https://www.nytco.com/}{NYTCo}
\item
  \href{https://help.nytimes.com/hc/en-us/articles/115015385887-Contact-Us}{Contact
  Us}
\item
  \href{https://www.nytco.com/careers/}{Work with us}
\item
  \href{https://nytmediakit.com/}{Advertise}
\item
  \href{http://www.tbrandstudio.com/}{T Brand Studio}
\item
  \href{https://www.nytimes.com/privacy/cookie-policy\#how-do-i-manage-trackers}{Your
  Ad Choices}
\item
  \href{https://www.nytimes.com/privacy}{Privacy}
\item
  \href{https://help.nytimes.com/hc/en-us/articles/115014893428-Terms-of-service}{Terms
  of Service}
\item
  \href{https://help.nytimes.com/hc/en-us/articles/115014893968-Terms-of-sale}{Terms
  of Sale}
\item
  \href{https://spiderbites.nytimes.com}{Site Map}
\item
  \href{https://help.nytimes.com/hc/en-us}{Help}
\item
  \href{https://www.nytimes.com/subscription?campaignId=37WXW}{Subscriptions}
\end{itemize}
