Sections

SEARCH

\protect\hyperlink{site-content}{Skip to
content}\protect\hyperlink{site-index}{Skip to site index}

\href{https://www.nytimes.com/section/world/europe}{Europe}

\href{https://myaccount.nytimes.com/auth/login?response_type=cookie\&client_id=vi}{}

\href{https://www.nytimes.com/section/todayspaper}{Today's Paper}

\href{/section/world/europe}{Europe}\textbar{}Global Backlash Builds
Against China Over Coronavirus

\url{https://nyti.ms/3d90hdP}

\begin{itemize}
\item
\item
\item
\item
\item
\end{itemize}

\href{https://www.nytimes.com/news-event/coronavirus?action=click\&pgtype=Article\&state=default\&region=TOP_BANNER\&context=storylines_menu}{The
Coronavirus Outbreak}

\begin{itemize}
\tightlist
\item
  live\href{https://www.nytimes.com/2020/08/04/world/coronavirus-cases.html?action=click\&pgtype=Article\&state=default\&region=TOP_BANNER\&context=storylines_menu}{Latest
  Updates}
\item
  \href{https://www.nytimes.com/interactive/2020/us/coronavirus-us-cases.html?action=click\&pgtype=Article\&state=default\&region=TOP_BANNER\&context=storylines_menu}{Maps
  and Cases}
\item
  \href{https://www.nytimes.com/interactive/2020/science/coronavirus-vaccine-tracker.html?action=click\&pgtype=Article\&state=default\&region=TOP_BANNER\&context=storylines_menu}{Vaccine
  Tracker}
\item
  \href{https://www.nytimes.com/2020/08/02/us/covid-college-reopening.html?action=click\&pgtype=Article\&state=default\&region=TOP_BANNER\&context=storylines_menu}{College
  Reopening}
\item
  \href{https://www.nytimes.com/live/2020/08/04/business/stock-market-today-coronavirus?action=click\&pgtype=Article\&state=default\&region=TOP_BANNER\&context=storylines_menu}{Economy}
\end{itemize}

Advertisement

\protect\hyperlink{after-top}{Continue reading the main story}

Supported by

\protect\hyperlink{after-sponsor}{Continue reading the main story}

\hypertarget{global-backlash-builds-against-china-over-coronavirus}{%
\section{Global Backlash Builds Against China Over
Coronavirus}\label{global-backlash-builds-against-china-over-coronavirus}}

As calls for inquiries and reparations spread, Beijing has responded
aggressively, mixing threats with aid and adding to a growing mistrust
of China.

\includegraphics{https://static01.nyt.com/images/2020/05/01/world/00virus-chinablowback-wuhan/merlin_170325123_cd8d9bda-c882-4736-ad63-2b042fee38a9-articleLarge.jpg?quality=75\&auto=webp\&disable=upscale}

\href{https://www.nytimes.com/by/steven-erlanger}{\includegraphics{https://static01.nyt.com/images/2018/10/10/multimedia/author-steven-erlanger/author-steven-erlanger-thumbLarge.png}}

By \href{https://www.nytimes.com/by/steven-erlanger}{Steven Erlanger}

\begin{itemize}
\item
  Published May 3, 2020Updated June 17, 2020
\item
  \begin{itemize}
  \item
  \item
  \item
  \item
  \item
  \end{itemize}
\end{itemize}

\href{https://cn.nytimes.com/world/20200506/backlash-china-coronavirus/}{阅读简体中文版}\href{https://cn.nytimes.com/world/20200506/backlash-china-coronavirus/zh-hant/}{閱讀繁體中文版}

BRUSSELS --- Australia has called for an inquiry into the origin of the
virus. Germany and Britain are hesitating anew about inviting in the
Chinese tech giant Huawei. President Trump
has\href{https://www.nytimes.com/2020/04/30/us/politics/trump-administration-intelligence-coronavirus-china.html?action=click\&module=Spotlight\&pgtype=Homepage}{blamed}\href{https://www.nytimes.com/2020/05/21/business/economy/coronavirus-china-economy.html}{China}
for the contagion and is seeking to punish it. Some governments want to
sue
\href{https://www.nytimes.com/2020/06/19/world/asia/coronavirus-china-beijing.html}{Beijing}
for damages and reparations.

Across the globe a backlash is building against
\href{https://www.nytimes.com/2020/05/11/world/australia/coronavirus-china-inquiry.html}{China}
for its initial mishandling of the crisis that helped loose the
\href{https://www.nytimes.com/2020/05/21/business/economy/coronavirus-china-economy.html}{coronavirus}
on the world, creating a deeply polarizing battle of narratives and
setting back China's ambition to fill the leadership vacuum
\href{https://www.nytimes.com/2020/05/03/us/coronavirus-updates.html}{left
by the United States}.

China, never receptive to outside criticism and wary of damage to its
domestic control and long economic reach, has
\href{https://www.nytimes.com/2020/05/11/world/australia/coronavirus-china-inquiry.html}{responded
aggressively}, combining medical aid to other countries with harsh
nationalist rhetoric, and mixing demands for gratitude with economic
threats.

The result has only added momentum to the blowback and the growing
mistrust of China in Europe and Africa, undermining China's desired
image as a generous global actor.

\includegraphics{https://static01.nyt.com/images/2020/05/01/world/00virus-chinablowback-aid/merlin_171206412_63b0be77-a171-425a-b5c1-30a091c14934-articleLarge.jpg?quality=75\&auto=webp\&disable=upscale}

Even before the virus, Beijing displayed a fierce approach to public
relations, an aggressive style called ``Wolf Warrior'' diplomacy, named
after two ultrapatriotic Chinese films featuring the evil plots and
fiery demise of American-led foreign mercenaries.

With clear encouragement from President Xi Jinping and the powerful
Propaganda Department of the Chinese Communist Party, a younger
generation of Chinese diplomats have been proving their loyalty with
defiantly nationalist and sometimes threatening messages in the
countries where they are based.

Image

A video screen in Beijing in March showing President Xi Jinping of China
with army officers and other officials.Credit...Gilles Sabrie for The
New York Times

``You have a new brand of Chinese diplomats who seem to compete with
each other to be more radical and eventually insulting to the country
where they happen to be posted,'' said François Godement, a senior
adviser for Asia at the Paris-based Institut Montaigne. ``They've gotten
into fights with every northern European country with whom they should
have an interest, and they've alienated every one of them.''

Since the virus, the tone has only toughened, a measure of just how
serious a danger China's leaders consider the virus to their standing at
home, where it has fueled anger and destroyed economic growth, as well
as abroad.

\hypertarget{latest-updates-global-coronavirus-outbreak}{%
\section{\texorpdfstring{\href{https://www.nytimes.com/2020/08/04/world/coronavirus-cases.html?action=click\&pgtype=Article\&state=default\&region=MAIN_CONTENT_1\&context=storylines_live_updates}{Latest
Updates: Global Coronavirus
Outbreak}}{Latest Updates: Global Coronavirus Outbreak}}\label{latest-updates-global-coronavirus-outbreak}}

Updated 2020-08-04T20:57:54.346Z

\begin{itemize}
\tightlist
\item
  \href{https://www.nytimes.com/2020/08/04/world/coronavirus-cases.html?action=click\&pgtype=Article\&state=default\&region=MAIN_CONTENT_1\&context=storylines_live_updates\#link-1228a480}{Novavax
  sees encouraging results from two studies of its experimental
  vaccine.}
\item
  \href{https://www.nytimes.com/2020/08/04/world/coronavirus-cases.html?action=click\&pgtype=Article\&state=default\&region=MAIN_CONTENT_1\&context=storylines_live_updates\#link-4825b93}{Public
  and private schools in Maryland and elsewhere are divided over
  in-person instruction.}
\item
  \href{https://www.nytimes.com/2020/08/04/world/coronavirus-cases.html?action=click\&pgtype=Article\&state=default\&region=MAIN_CONTENT_1\&context=storylines_live_updates\#link-50f7386d}{The
  United Nations calls on policymakers to `plan thoroughly for school
  reopenings.'}
\end{itemize}

\href{https://www.nytimes.com/2020/08/04/world/coronavirus-cases.html?action=click\&pgtype=Article\&state=default\&region=MAIN_CONTENT_1\&context=storylines_live_updates}{See
more updates}

More live coverage:
\href{https://www.nytimes.com/live/2020/08/04/business/stock-market-today-coronavirus?action=click\&pgtype=Article\&state=default\&region=MAIN_CONTENT_1\&context=storylines_live_updates}{Markets}

In the past several weeks, at least seven Chinese ambassadors --- to
France, Kazakhstan, Nigeria, Kenya, Uganda, Ghana and the African Union
--- have been summoned by their hosts to answer accusations ranging from
spreading misinformation to the ``racist mistreatment'' of Africans in
Guangzhou.

Image

Chinese flags lining a street in Guangzhou, where Africans say they have
been evicted and forced into quarantine.Credit...Alex Plavevski/EPA, via
Shutterstock

Just last week,
\href{https://www.nytimes.com/2020/06/15/world/asia/beijing-coronavirus-outbreak.html}{China}
threatened to withhold medical aid from the Netherlands for changing the
name of its representative office in Taiwan to include the word Taipei.
And before that, the Chinese Embassy in Berlin sparred publicly with the
German newspaper Bild after the tabloid demanded \$160 billion in
compensation from China for damages to Germany from the virus.

Mr. Trump said last week that his administration was conducting
``serious investigations'' into Beijing's handling of the coronavirus
outbreak.

He has pressed American intelligence agencies to find the source of the
virus, suggesting
\href{https://www.nytimes.com/2020/04/30/us/politics/trump-administration-intelligence-coronavirus-china.html}{it
might have emerged accidentally from a Wuhan weapons lab}, although most
intelligence agencies remain skeptical. And he has expressed interest in
trying to sue Beijing for damages, with the United States seeking \$10
million for every American death.

Republicans in the United States have moved to support Mr. Trump's
attacks on China. Missouri's attorney general, Eric Schmitt, filed a
lawsuit in federal court seeking to hold Beijing responsible for the
outbreak.

A Chinese Foreign Ministry spokesman, Geng Shuang, called the suit
``frivolous,'' adding that it had ``no factual and legal basis'' and
``only invites ridicule.''

Image

Attorney General Eric Schmitt of Missouri has filed a lawsuit in federal
court seeking to hold Beijing responsible for the coronavirus
outbreak.Credit...Manuel Balce Ceneta/Associated Press

The suit seems to aim less at securing victory in court, which is
unlikely, than at prodding Congress to pass legislation to make it
easier for U.S. citizens to sue foreign states for damages.

``From Beijing's point of view, this contemporary call is a historic
echo of the reparations paid after the Boxer Rebellion,'' said Theresa
Fallon, director of the Centre for Russia Europe Asia Studies, referring
to the anti-imperialist, anti-Christian and ultranationalist uprising
around 1899-1901 in China that ended in defeat, with huge reparations
for eight nations over the next decades. ``The party's cultivation of
the humiliation narrative makes it politically impossible for Xi to ever
agree to pay any reparations.''

Instead, it has been imperative for Mr. Xi to turn the narrative around,
steering it from a story of incompetence and failure --- including the
suppression of early warnings about the virus --- into one of victory
over the illness, a victory achieved through the unity of the party.

In the latest iteration of the new Chinese narrative, the enemy --- the
virus --- did not even come from China,
\href{https://www.nytimes.com/2020/03/13/world/asia/coronavirus-china-conspiracy-theory.html}{but
from the U.S. military}, an unsubstantiated accusation made by China's
combative Foreign Ministry spokesman, Zhao Lijian.

Chinese diplomats are encouraged to be combative by Beijing, said Susan
Shirk, a China scholar and director of the 21st Century China Center at
the University of California, San Diego. The promotion of Mr. Zhao to
spokesman and his statement about the U.S. Army ``signals to everyone in
China that this is the official line, so you get this megaphone
effect,'' she said, adding that it makes any negotiations more
difficult.

But in the longer run, China is seeding mistrust and damaging its own
interests, said Ms. Shirk, who is working on a book called
``Overreach,'' about how China's domestic politics have derailed its
ambitions for a peaceful rise as a global superpower.

\href{https://www.nytimes.com/news-event/coronavirus?action=click\&pgtype=Article\&state=default\&region=MAIN_CONTENT_3\&context=storylines_faq}{}

\hypertarget{the-coronavirus-outbreak-}{%
\subsubsection{The Coronavirus Outbreak
›}\label{the-coronavirus-outbreak-}}

\hypertarget{frequently-asked-questions}{%
\paragraph{Frequently Asked
Questions}\label{frequently-asked-questions}}

Updated August 4, 2020

\begin{itemize}
\item ~
  \hypertarget{i-have-antibodies-am-i-now-immune}{%
  \paragraph{I have antibodies. Am I now
  immune?}\label{i-have-antibodies-am-i-now-immune}}

  \begin{itemize}
  \tightlist
  \item
    As of right
    now,\href{https://www.nytimes.com/2020/07/22/health/covid-antibodies-herd-immunity.html?action=click\&pgtype=Article\&state=default\&region=MAIN_CONTENT_3\&context=storylines_faq}{that
    seems likely, for at least several months.} There have been
    frightening accounts of people suffering what seems to be a second
    bout of Covid-19. But experts say these patients may have a
    drawn-out course of infection, with the virus taking a slow toll
    weeks to months after initial exposure. People infected with the
    coronavirus typically
    \href{https://www.nature.com/articles/s41586-020-2456-9}{produce}
    immune molecules called antibodies, which are
    \href{https://www.nytimes.com/2020/05/07/health/coronavirus-antibody-prevalence.html?action=click\&pgtype=Article\&state=default\&region=MAIN_CONTENT_3\&context=storylines_faq}{protective
    proteins made in response to an
    infection}\href{https://www.nytimes.com/2020/05/07/health/coronavirus-antibody-prevalence.html?action=click\&pgtype=Article\&state=default\&region=MAIN_CONTENT_3\&context=storylines_faq}{.
    These antibodies may} last in the body
    \href{https://www.nature.com/articles/s41591-020-0965-6}{only two to
    three months}, which may seem worrisome, but that's perfectly normal
    after an acute infection subsides, said Dr. Michael Mina, an
    immunologist at Harvard University. It may be possible to get the
    coronavirus again, but it's highly unlikely that it would be
    possible in a short window of time from initial infection or make
    people sicker the second time.
  \end{itemize}
\item ~
  \hypertarget{im-a-small-business-owner-can-i-get-relief}{%
  \paragraph{I'm a small-business owner. Can I get
  relief?}\label{im-a-small-business-owner-can-i-get-relief}}

  \begin{itemize}
  \tightlist
  \item
    The
    \href{https://www.nytimes.com/article/small-business-loans-stimulus-grants-freelancers-coronavirus.html?action=click\&pgtype=Article\&state=default\&region=MAIN_CONTENT_3\&context=storylines_faq}{stimulus
    bills enacted in March} offer help for the millions of American
    small businesses. Those eligible for aid are businesses and
    nonprofit organizations with fewer than 500 workers, including sole
    proprietorships, independent contractors and freelancers. Some
    larger companies in some industries are also eligible. The help
    being offered, which is being managed by the Small Business
    Administration, includes the Paycheck Protection Program and the
    Economic Injury Disaster Loan program. But lots of folks have
    \href{https://www.nytimes.com/interactive/2020/05/07/business/small-business-loans-coronavirus.html?action=click\&pgtype=Article\&state=default\&region=MAIN_CONTENT_3\&context=storylines_faq}{not
    yet seen payouts.} Even those who have received help are confused:
    The rules are draconian, and some are stuck sitting on
    \href{https://www.nytimes.com/2020/05/02/business/economy/loans-coronavirus-small-business.html?action=click\&pgtype=Article\&state=default\&region=MAIN_CONTENT_3\&context=storylines_faq}{money
    they don't know how to use.} Many small-business owners are getting
    less than they expected or
    \href{https://www.nytimes.com/2020/06/10/business/Small-business-loans-ppp.html?action=click\&pgtype=Article\&state=default\&region=MAIN_CONTENT_3\&context=storylines_faq}{not
    hearing anything at all.}
  \end{itemize}
\item ~
  \hypertarget{what-are-my-rights-if-i-am-worried-about-going-back-to-work}{%
  \paragraph{What are my rights if I am worried about going back to
  work?}\label{what-are-my-rights-if-i-am-worried-about-going-back-to-work}}

  \begin{itemize}
  \tightlist
  \item
    Employers have to provide
    \href{https://www.osha.gov/SLTC/covid-19/standards.html}{a safe
    workplace} with policies that protect everyone equally.
    \href{https://www.nytimes.com/article/coronavirus-money-unemployment.html?action=click\&pgtype=Article\&state=default\&region=MAIN_CONTENT_3\&context=storylines_faq}{And
    if one of your co-workers tests positive for the coronavirus, the
    C.D.C.} has said that
    \href{https://www.cdc.gov/coronavirus/2019-ncov/community/guidance-business-response.html}{employers
    should tell their employees} -\/- without giving you the sick
    employee's name -\/- that they may have been exposed to the virus.
  \end{itemize}
\item ~
  \hypertarget{should-i-refinance-my-mortgage}{%
  \paragraph{Should I refinance my
  mortgage?}\label{should-i-refinance-my-mortgage}}

  \begin{itemize}
  \tightlist
  \item
    \href{https://www.nytimes.com/article/coronavirus-money-unemployment.html?action=click\&pgtype=Article\&state=default\&region=MAIN_CONTENT_3\&context=storylines_faq}{It
    could be a good idea,} because mortgage rates have
    \href{https://www.nytimes.com/2020/07/16/business/mortgage-rates-below-3-percent.html?action=click\&pgtype=Article\&state=default\&region=MAIN_CONTENT_3\&context=storylines_faq}{never
    been lower.} Refinancing requests have pushed mortgage applications
    to some of the highest levels since 2008, so be prepared to get in
    line. But defaults are also up, so if you're thinking about buying a
    home, be aware that some lenders have tightened their standards.
  \end{itemize}
\item ~
  \hypertarget{what-is-school-going-to-look-like-in-september}{%
  \paragraph{What is school going to look like in
  September?}\label{what-is-school-going-to-look-like-in-september}}

  \begin{itemize}
  \tightlist
  \item
    It is unlikely that many schools will return to a normal schedule
    this fall, requiring the grind of
    \href{https://www.nytimes.com/2020/06/05/us/coronavirus-education-lost-learning.html?action=click\&pgtype=Article\&state=default\&region=MAIN_CONTENT_3\&context=storylines_faq}{online
    learning},
    \href{https://www.nytimes.com/2020/05/29/us/coronavirus-child-care-centers.html?action=click\&pgtype=Article\&state=default\&region=MAIN_CONTENT_3\&context=storylines_faq}{makeshift
    child care} and
    \href{https://www.nytimes.com/2020/06/03/business/economy/coronavirus-working-women.html?action=click\&pgtype=Article\&state=default\&region=MAIN_CONTENT_3\&context=storylines_faq}{stunted
    workdays} to continue. California's two largest public school
    districts --- Los Angeles and San Diego --- said on July 13, that
    \href{https://www.nytimes.com/2020/07/13/us/lausd-san-diego-school-reopening.html?action=click\&pgtype=Article\&state=default\&region=MAIN_CONTENT_3\&context=storylines_faq}{instruction
    will be remote-only in the fall}, citing concerns that surging
    coronavirus infections in their areas pose too dire a risk for
    students and teachers. Together, the two districts enroll some
    825,000 students. They are the largest in the country so far to
    abandon plans for even a partial physical return to classrooms when
    they reopen in August. For other districts, the solution won't be an
    all-or-nothing approach.
    \href{https://bioethics.jhu.edu/research-and-outreach/projects/eschool-initiative/school-policy-tracker/}{Many
    systems}, including the nation's largest, New York City, are
    devising
    \href{https://www.nytimes.com/2020/06/26/us/coronavirus-schools-reopen-fall.html?action=click\&pgtype=Article\&state=default\&region=MAIN_CONTENT_3\&context=storylines_faq}{hybrid
    plans} that involve spending some days in classrooms and other days
    online. There's no national policy on this yet, so check with your
    municipal school system regularly to see what is happening in your
    community.
  \end{itemize}
\end{itemize}

``As China started getting control over the virus and started this
health diplomacy, it could have been the opportunity for China to
emphasize its compassionate side and rebuild trust and its reputation as
a responsible global power,'' she said. ``But that diplomatic effort got
hijacked by the Propaganda Department of the party, with a much more
assertive effort to leverage their assistance to get praise for China as
a country and a system and its performance in stopping the spread of the
virus.''

In recent days, Chinese state media has run numerous inflammatory
statements, saying that Australia, after announcing its desire for an
inquiry into the virus, was ``gum stuck to the bottom of China's shoe.''
Beijing warned that Australia risked long-term damage to its trading
partnership with China, which takes a third of Australia's exports.

``Maybe the ordinary people will say, `Why should we drink Australian
wine? Eat Australian beef?''' China's ambassador, Cheng Jingye, told The
Australian Financial Review. Australia's foreign minister, Marise Payne,
dismissed China's attempt as ``economic coercion.''

Image

The Sydney waterfront. Chinese state media recently assailed Australia,
after it announced its desire for an inquiry into the
virus.Credit...Matthew Abbott for The New York Times

Even in European countries like Germany, ``the mistrust of China has
accelerated so quickly with the virus that no ministry knows how to deal
with it,'' said Angela Stanzel, a China expert with the German Institute
for International and Security Affairs.

In Germany, as in Britain, in addition to new questions about the
advisability of using Huawei for new 5G systems, worries have also grown
about dependency on China for vital materials and pharmaceuticals.

France, which traditionally has good relations with Beijing, has also
been angered by critical statements by Chinese diplomats, including a
charge that the French had deliberately left their older residents to
die in nursing homes. That prompted a rebuke from France's foreign
minister, Jean-Yves Le Drian, and anger from legislators, despite an
early reciprocal exchange of medical aid like masks.

Recently, the German government complained that Chinese diplomats were
soliciting letters of support and gratitude for Beijing's aid and
efforts against the virus from government officials and the heads of
major German companies.

The same has been true in Poland, said the U.S. ambassador to Warsaw,
Georgette Mosbacher, in an interview, describing Chinese pressure on
President Andrzej Duda to call Mr. Xi and thank him for aid, a call the
\href{https://www.fmprc.gov.cn/mfa_eng/zxxx_662805/t1761096.shtml}{Chinese
heralded} at home.

``Poland wasn't going to get this stuff unless the phone call was made,
so they could use that phone call'' for propaganda, Ms. Mosbacher said.

There is some unhappiness in China with the current diplomatic rhetoric.
In
\href{http://chinaheritage.net/journal/1900-2020-an-old-anxiety-in-a-new-era/}{a
recent essay}, Zi Zhongyun, now 89, a longtime expert on America at the
Chinese Academy of Social Sciences, sees parallels in the harsh
nationalist and xenophobic rhetoric of the Wolf Warriors of today with
the period around the Boxer Rebellion against Western influence in
China.

Ms. Zi said such reactions risked getting out of hand.

``I can say without a doubt,'' she concluded, ``that as long as
Boxer-like activities are given the official stamp of approval as being
`patriotic,''' and as long as ``generation after generation of our
fellow Chinese are educated and inculcated with a Boxer-like mentality,
it will be impossible for China to take its place among the modern
civilized nations of the world.''

Isabella Kwai contributed reporting from Sydney, Australia. Monika
Pronczuk contributed research from Brussels.

Advertisement

\protect\hyperlink{after-bottom}{Continue reading the main story}

\hypertarget{site-index}{%
\subsection{Site Index}\label{site-index}}

\hypertarget{site-information-navigation}{%
\subsection{Site Information
Navigation}\label{site-information-navigation}}

\begin{itemize}
\tightlist
\item
  \href{https://help.nytimes.com/hc/en-us/articles/115014792127-Copyright-notice}{©~2020~The
  New York Times Company}
\end{itemize}

\begin{itemize}
\tightlist
\item
  \href{https://www.nytco.com/}{NYTCo}
\item
  \href{https://help.nytimes.com/hc/en-us/articles/115015385887-Contact-Us}{Contact
  Us}
\item
  \href{https://www.nytco.com/careers/}{Work with us}
\item
  \href{https://nytmediakit.com/}{Advertise}
\item
  \href{http://www.tbrandstudio.com/}{T Brand Studio}
\item
  \href{https://www.nytimes.com/privacy/cookie-policy\#how-do-i-manage-trackers}{Your
  Ad Choices}
\item
  \href{https://www.nytimes.com/privacy}{Privacy}
\item
  \href{https://help.nytimes.com/hc/en-us/articles/115014893428-Terms-of-service}{Terms
  of Service}
\item
  \href{https://help.nytimes.com/hc/en-us/articles/115014893968-Terms-of-sale}{Terms
  of Sale}
\item
  \href{https://spiderbites.nytimes.com}{Site Map}
\item
  \href{https://help.nytimes.com/hc/en-us}{Help}
\item
  \href{https://www.nytimes.com/subscription?campaignId=37WXW}{Subscriptions}
\end{itemize}
