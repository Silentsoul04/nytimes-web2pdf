Sections

SEARCH

\protect\hyperlink{site-content}{Skip to
content}\protect\hyperlink{site-index}{Skip to site index}

\href{https://www.nytimes.com/section/world/asia}{Asia Pacific}

\href{https://myaccount.nytimes.com/auth/login?response_type=cookie\&client_id=vi}{}

\href{https://www.nytimes.com/section/todayspaper}{Today's Paper}

\href{/section/world/asia}{Asia Pacific}\textbar{}Gas Leak in India at
LG Factory Kills 11 and Sickens Hundreds

\url{https://nyti.ms/2SItftA}

\begin{itemize}
\item
\item
\item
\item
\item
\item
\end{itemize}

Advertisement

\protect\hyperlink{after-top}{Continue reading the main story}

Supported by

\protect\hyperlink{after-sponsor}{Continue reading the main story}

\hypertarget{gas-leak-in-india-at-lg-factory-kills-11-and-sickens-hundreds}{%
\section{Gas Leak in India at LG Factory Kills 11 and Sickens
Hundreds}\label{gas-leak-in-india-at-lg-factory-kills-11-and-sickens-hundreds}}

Residents in eastern India woke up in the middle of the night surrounded
by a cloud of styrene vapor. Many couldn't breathe.

\includegraphics{https://static01.nyt.com/images/2020/05/07/world/07india-gas-1/merlin_172261560_ed6904ec-bec1-4a83-8372-443eda83e8f6-articleLarge.jpg?quality=75\&auto=webp\&disable=upscale}

\href{https://www.nytimes.com/by/jeffrey-gettleman}{\includegraphics{https://static01.nyt.com/images/2018/10/10/multimedia/author-jeffrey-gettleman/author-jeffrey-gettleman-thumbLarge.png}}\href{https://www.nytimes.com/by/suhasini-raj}{\includegraphics{https://static01.nyt.com/images/2019/11/22/reader-center/author-Suhasini-Raj/author-Suhasini-Raj-thumbLarge.png}}\href{https://www.nytimes.com/by/kai-schultz}{\includegraphics{https://static01.nyt.com/images/2019/11/22/reader-center/author-kai-schultz/author-kai-schultz-thumbLarge.png}}\href{https://www.nytimes.com/by/sameer-yasir}{\includegraphics{https://static01.nyt.com/images/2019/11/22/reader-center/author-sameer-yasir/author-sameer-yasir-thumbLarge.png}}

By \href{https://www.nytimes.com/by/jeffrey-gettleman}{Jeffrey
Gettleman}, \href{https://www.nytimes.com/by/suhasini-raj}{Suhasini
Raj}, \href{https://www.nytimes.com/by/kai-schultz}{Kai Schultz} and
\href{https://www.nytimes.com/by/sameer-yasir}{Sameer Yasir}

\begin{itemize}
\item
  Published May 7, 2020Updated May 8, 2020
\item
  \begin{itemize}
  \item
  \item
  \item
  \item
  \item
  \item
  \end{itemize}
\end{itemize}

NEW DELHI --- D.V.S.S. Ramana, who lives about a mile from an industrial
plant in eastern India, said he woke up early Thursday morning enveloped
in a strange white mist.

He started coughing. His eyes burned. He flipped on his TV to learn that
a cloud of toxic vapor had just escaped from a nearby plastics factory
owned by the South Korean industrial giant LG Corp. in the Indian
coastal city of Visakhapatnam.

Mr. Ramana jerked his wife and two children awake. As he rushed outside,
he heard sirens blaring and saw people staggering into the street. Some
collapsed right in front of him.

``We could smell the gas in our mouths,'' Mr. Ramana said, speaking by
telephone as he was driving off, trying to get his family as far away as
possible. ``It was terrifying.''

The leak, which officials said came from a tank of styrene, a liquid
material used in making plastics, sent out a cloud of toxic vapor that
drifted over the outskirts of Visakhapatnam, killing at least 11 people
and sickening hundreds.

CHINA

PAKISTAN

BHUTAN

NEPAL

New Delhi

Bhopal

INDIA

Visakhapatnam ~~~~

ANDHRA

PRADESH

Arabian

Sea

Bay of Bengal

SRI LANKA

400 Miles

By The New York Times

Dozens of men and women were left lying unconscious in the street.
Mothers ran to hospitals with limp children in their arms. Police
officers moved house to house to evacuate hundreds of people. Sometimes
they had to break down doors because the residents inside were
unconscious.

Indian officials said the accident happened around 2:30 a.m., as the
chemical plant was restarting operations after a six-week hiatus because
of
\href{https://www.nytimes.com/2020/05/01/world/asia/india-coronavirus-delhi.html}{India's
strict coronavirus lockdown}. The tanks of styrene had been left
unattended, Indian officials said, and in the course of the factory
resuming operations, a major leak erupted.

``It seems unskilled labor mishandled the maintenance work and because
of that, the gas leaked,'' said Srijana Gummalla, one of the top
government officials in Visakhapatnam.

\includegraphics{https://static01.nyt.com/images/2020/05/07/world/07india-gas-2/merlin_172261461_9d13a422-eb10-400a-bdb9-4096f492adf9-articleLarge.jpg?quality=75\&auto=webp\&disable=upscale}

The plant had been closed since India's coronavirus lockdown began in
late March, but this week the lockdown was eased and many businesses,
including heavy industries, have begun to reopen.

A former manager of LG Polymers, the Indian subsidiary of LG Chemical,
said in an interview that liquid styrene required careful attention, and
that because of the lockdown, workers were not going to the factory. He
said that may have played a role in the leak.

Indian news media broadcast
\href{https://timesofindia.indiatimes.com/city/visakhapatnam/vizag-over-1000-fall-sick-after-gas-leak-from-chemical-plant/articleshow/75590112.cms}{clips
of victims,} including villagers lying face down on a muddy road, their
mobile phones spilled out next to them. In one shot, a woman frantically
pounds the chest of another woman who had just collapsed. In another, a
stray dog struggles to stand and then crumples to the floor.

Though this accident was not close to the scale of the 1984 gas leak at
a Union Carbide pesticide plant in the Indian city of Bhopal, which left
nearly 4,000 dead and a half million poisoned, it immediately drew
comparisons, especially among Bhopal survivors.

``When I saw the images on television of people struggling to breathe
and laying on the roadside,'' said Rashida Bi, a Bhopal survivor,
``something hit me deep inside as if I was laying motionless, among
them, begging to save my life and struggling to breath.''

``At least they had people ready to take them to hospitals,'' she added.
``We had none. All of us were looking at each other, waiting for death
to come.''

While the Bhopal disaster brought to the fore India's dangerous
industrial practices, forcing the government to improve some safety
standards, Bhopal survivors and many others said India still hasn't
learned the right lessons.

For example, several union officials said that the warning sirens at the
Visakhapatnam factory failed to go off and that more people had died as
a result.

Mantri Rajsekhar, the local leader of a union representing LG Polymers,
accused the company of ignoring safety issues raised by the group,
including poor maintenance of equipment.

LG Chemical said in a statement that it was investigating how the leak
happened and ``the cause of deaths.''

``We are working together with the local authorities to assess the
damage caused to the local people and to take whatever it takes to
protect them and our workers,'' the statement said. ``The gas leak from
the factory is now under control. But the gas can cause vomiting and
dizziness when inhaled. We are doing all we can to ensure medical
treatment as quickly as possible.''

Medical officials in Visakhapatnam
\href{https://mumbaimirror.indiatimes.com/news/india/three-dead-many-trapped-in-houses-after-chemical-leaked-from-polymers-firm-in-visakhapatnam/articleshow/75590150.cms}{said
hundreds of patients ended up exposed to the styrene}vapor, which can
immobilize a person within minutes of inhalation and be deadly at high
concentrations. Doctors said that many patients were vomiting and
experiencing ``neurological deficiencies.'' At least a dozen remained in
critical condition.

India's National Disaster Response Force sent men in camouflaged hazmat
suits with oxygen tanks on their backs into the factory. By Thursday
afternoon, there was ``hardly any leakage,'' said S.N. Pradhan, the
disaster force's director.

``We will be there until that leakage is totally plugged,'' he said.

Image

The plant following the leak, which officials said exposed residents to
styrene vapor.Credit...Agence France-Presse --- Getty Images

South Korea's \href{http://www.lgpi.co.in/AboutLGPI.html}{LG Chemical
took over the}factory from a local Indian company in the late 1990s. It
employs several hundred people, the former manager said, and makes
polystyrene (a hard plastic used for appliances, toys, and electronics)
and expandable polystyrene (a foam material used as an insulator for
buildings and packaging for items.)

The factory sits at the edge of Visakhapatnam, surrounded by train
tracks, small villages, dirt roads, and neem and ficus trees. It is an
old port city now home to many industrial plants, and is one of the
biggest cities in southeastern India, in Andhra Pradesh State.

The accident happened when most people in the area were sleeping. Indian
officials said the cloud of toxic vapor spread over an area with a three
kilometer radius, exposing more than 1,000 people.

At least two children died, authorities said, and several other people
perished while trying to run away. Officials said that one villager who
was temporarily blinded by the vapor fell into a well.

Mr. Ramana, who works as a store manager at an ironware factory, said he
plans to stay at a relative's place a couple of hours drive from
Visakhapatnam.

By Thursday afternoon, speaking from further along the highway, far from
Visakhapatnam, he sounded much more relieved.

``I will drop my family at a relative's where we will be staying for a
couple of days and then go to my workplace,'' he said. ``Life must go
on.

Jeffrey Gettleman, Kai Schultz and Sameer Yasir reported from New Delhi,
and Suhasini Raj from Lucknow, India. Reporting was contributed by Hari
Kumar from New Delhi; S.M. Bilal from Hyderabad, India; Choe Sang-hun
from Seoul; and Maria Abi-Habib from Los Angeles.

Advertisement

\protect\hyperlink{after-bottom}{Continue reading the main story}

\hypertarget{site-index}{%
\subsection{Site Index}\label{site-index}}

\hypertarget{site-information-navigation}{%
\subsection{Site Information
Navigation}\label{site-information-navigation}}

\begin{itemize}
\tightlist
\item
  \href{https://help.nytimes.com/hc/en-us/articles/115014792127-Copyright-notice}{©~2020~The
  New York Times Company}
\end{itemize}

\begin{itemize}
\tightlist
\item
  \href{https://www.nytco.com/}{NYTCo}
\item
  \href{https://help.nytimes.com/hc/en-us/articles/115015385887-Contact-Us}{Contact
  Us}
\item
  \href{https://www.nytco.com/careers/}{Work with us}
\item
  \href{https://nytmediakit.com/}{Advertise}
\item
  \href{http://www.tbrandstudio.com/}{T Brand Studio}
\item
  \href{https://www.nytimes.com/privacy/cookie-policy\#how-do-i-manage-trackers}{Your
  Ad Choices}
\item
  \href{https://www.nytimes.com/privacy}{Privacy}
\item
  \href{https://help.nytimes.com/hc/en-us/articles/115014893428-Terms-of-service}{Terms
  of Service}
\item
  \href{https://help.nytimes.com/hc/en-us/articles/115014893968-Terms-of-sale}{Terms
  of Sale}
\item
  \href{https://spiderbites.nytimes.com}{Site Map}
\item
  \href{https://help.nytimes.com/hc/en-us}{Help}
\item
  \href{https://www.nytimes.com/subscription?campaignId=37WXW}{Subscriptions}
\end{itemize}
