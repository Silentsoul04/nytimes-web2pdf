Sections

SEARCH

\protect\hyperlink{site-content}{Skip to
content}\protect\hyperlink{site-index}{Skip to site index}

\href{https://www.nytimes.com/section/obituaries}{Obituaries}

\href{https://myaccount.nytimes.com/auth/login?response_type=cookie\&client_id=vi}{}

\href{https://www.nytimes.com/section/todayspaper}{Today's Paper}

\href{/section/obituaries}{Obituaries}\textbar{}Roy Horn, Who Dazzled
Audiences as Half of Siegfried \& Roy, Dies at 75

\url{https://nyti.ms/3dAjcPf}

\begin{itemize}
\item
\item
\item
\item
\item
\end{itemize}

\href{https://www.nytimes.com/news-event/coronavirus?action=click\&pgtype=Article\&state=default\&region=TOP_BANNER\&context=storylines_menu}{The
Coronavirus Outbreak}

\begin{itemize}
\tightlist
\item
  live\href{https://www.nytimes.com/2020/08/03/world/coronavirus-covid-19.html?action=click\&pgtype=Article\&state=default\&region=TOP_BANNER\&context=storylines_menu}{Latest
  Updates}
\item
  \href{https://www.nytimes.com/interactive/2020/us/coronavirus-us-cases.html?action=click\&pgtype=Article\&state=default\&region=TOP_BANNER\&context=storylines_menu}{Maps
  and Cases}
\item
  \href{https://www.nytimes.com/interactive/2020/science/coronavirus-vaccine-tracker.html?action=click\&pgtype=Article\&state=default\&region=TOP_BANNER\&context=storylines_menu}{Vaccine
  Tracker}
\item
  \href{https://www.nytimes.com/2020/08/02/us/covid-college-reopening.html?action=click\&pgtype=Article\&state=default\&region=TOP_BANNER\&context=storylines_menu}{College
  Reopening}
\item
  \href{https://www.nytimes.com/live/2020/08/03/business/stock-market-today-coronavirus?action=click\&pgtype=Article\&state=default\&region=TOP_BANNER\&context=storylines_menu}{Economy}
\end{itemize}

Advertisement

\protect\hyperlink{after-top}{Continue reading the main story}

Supported by

\protect\hyperlink{after-sponsor}{Continue reading the main story}

Those We've Lost

\hypertarget{roy-horn-who-dazzled-audiences-as-half-of-siegfried--roy-dies-at-75}{%
\section{Roy Horn, Who Dazzled Audiences as Half of Siegfried \& Roy,
Dies at
75}\label{roy-horn-who-dazzled-audiences-as-half-of-siegfried--roy-dies-at-75}}

He and his partner mesmerized crowds for decades, using a stunning mix
of magic, costumes and exotic animals. He died of complications of
Covid-19.

\includegraphics{https://static01.nyt.com/images/2020/05/10/us/politics/08horn-obit/merlin_10910529_bb270e2c-169e-4b6a-82f7-f252d2ac17a3-articleLarge.jpg?quality=75\&auto=webp\&disable=upscale}

By \href{https://www.nytimes.com/by/robert-d-mcfadden}{Robert D.
McFadden}

\begin{itemize}
\item
  Published May 8, 2020Updated May 18, 2020
\item
  \begin{itemize}
  \item
  \item
  \item
  \item
  \item
  \end{itemize}
\end{itemize}

\emph{This obituary is part of a series about people who have died in
the coronavirus pandemic. Read about others}
\href{https://www.nytimes.com/series/people-who-have-died-of-the-coronavirus}{\emph{here}}\emph{.}

Roy Horn, who levitated tigers, made elephants disappear, turned himself
into a python and mesmerized Las Vegas audiences for decades as half of
the famed illusionist team \href{http://siegfriedandroy.com/}{Siegfried
\& Roy}, died on Friday in Las Vegas. He was 75.

The cause was complications of Covid-19, the disease caused by the
coronavirus, according to his publicist, Dave Kirvin. Mr. Horn, who
lived in Las Vegas, tested positive for the virus last week and died at
MountainView Hospital, Mr. Kirvin said.

The German-born Mr. Horn and Siegfried Fischbacher's long-running
production, one of the most successful in Las Vegas history, ended on
Oct. 3, 2003, when Mr. Horn, on his 59th birthday, was
\href{https://www.nytimes.com/2003/10/06/us/onstage-attack-casts-pall-over-las-vegas-strip.html}{mauled
by a 400-pound white tiger} that lunged at his throat and dragged him
offstage before a stunned capacity crowd of 1,500 at
\href{https://mirage.mgmresorts.com/en.html}{MGM's Mirage hotel and
casino}.

An aide yanked the tiger's tail, leapt on its back and tried to pry open
its jaws. Another sprayed it with a fire extinguisher until it let go.
But Mr. Horn's windpipe had been crushed, and an artery carrying oxygen
to his brain was damaged. He suffered a stroke and partial paralysis on
his left side, underwent two operations at University Medical Center in
Las Vegas and was placed on life support.

After weeks in critical condition, however, Mr. Horn began a long
recovery, with rehabilitation at the U.C.L.A. Medical Center in Los
Angeles. In 2004 he returned to his home in Las Vegas, and within months
he was walking again with assistance. There was even talk of a comeback,
but medical experts and entertainment moguls considered it highly
unlikely.

In February 2009, Siegfried and Roy made one final appearance with a
tiger, a benefit performance for the Lou Ruvo Center for Brain Health in
Las Vegas. They officially retired from show business in 2010.

\hypertarget{a-sorcerers-extravaganza}{%
\subsection{A Sorcerer's Extravaganza}\label{a-sorcerers-extravaganza}}

Mr. Horn and Mr. Fischbacher dazzled Las Vegas crowds for 35 years with
\href{https://www.youtube.com/watch?v=jun11ng8WaM}{a sorcerer's
extravaganza} that combined the glitz of sequined costumes and feathered
headdresses with smoke-and-laser magic and the circus thrills of exotic
animals, including rare white tigers and white lions. Under their
spells, a white tiger turned into a beautiful woman,
\href{https://www.youtube.com/watch?v=Vk48Dry8KzY}{a six-ton elephant
vanished}, a tiger floated out over the audience, and Mr. Horn slithered
down and became a snake.

``The world has lost one of the greats of magic, but I have lost my best
friend,'' Mr. Fischbacher said in a statement on Friday. ``From the
moment we met, I knew Roy and I, together, would change the world. There
could be no Siegfried without Roy, and no Roy without Siegfried.''

The two showmen performed in Europe, Japan and elsewhere. They were
featured in a 1999 3-D Imax movie and a 1994 television special and
appeared at Radio City Music Hall in New York. They broke records for
the longest-running act in Las Vegas and were among the most popular and
highly paid performers on the Strip. They also wrote a book, ``Siegfried
and Roy: Mastering the Impossible'' (1992).

Mr. Horn and Mr. Fischbacher, who were domestic as well as professional
partners, kept their menageries, including dozens of exotic cats, at a
glass-enclosed tropically forested habitat at the Mirage; at Jungle
Paradise, their 88-acre estate outside town; and at Jungle Palace, their
\$10 million Spanish-style home in Las Vegas.

Acknowledging that their acts depended on some endangered species, they
were prominent in various animal conservation efforts, particularly for
the white tiger, native to Asia, and the white lion of Timbavati, in
South Africa. They raised many of their show animals from birth, and
they said the animals were not exploited and were never tranquilized.

Mr. Horn insisted that the white tiger that mauled him --- a 7-year-old
male who had been acquired in Mexico, trained by Mr. Horn and used in
performances for six and a half years --- not be harmed afterward. The
tiger was quarantined for a time, then returned to its habitat at the
Mirage, where many of the act's animals were kept on display after the
show ended. The tiger's name was reported to be Montecore at the time of
the mauling, but it was given by Mr. Horn as Mantecore when the animal
died in 2014 after a short illness.

In the years after the mauling, Mr. Horn and Mr. Fischbacher tried to
minimize what was widely reported as a ferocious attack. They said the
tiger had been unhinged by a woman in the front row with a beehive
hairdo and the sight of Mr. Horn tripping as he tried to step between
them, and that the tiger had picked Mr. Horn up by the neck, as a
tigress might a cub, and was attempting to carry him to safety.

That and other theories --- suggesting a provocation by animal-rights
activists or an act of economic terrorism against Las Vegas --- were
investigated by the police and federal officials.
\href{https://www.cbsnews.com/news/roy-horn-tiger-mauling-case-closed/}{A
comprehensive report by the United States Department of Agriculture}
discounted all such theories and called it a simple attack by the tiger.
But the department amended its safety regulations for the live
exhibition of big cats to stipulate minimal distances and barriers
between animals and the viewing public.

Investigators quoted witnesses as saying that Mr. Horn had ordered the
tiger to lie down, and that when it refused he hit it on the nose with
his microphone. The tiger then snagged his forearm, and when Mr. Horn
tried to beat it back with the microphone, it lunged at his throat and
dragged him off like a rag doll.

The cancellation of Siegfried \& Roy after the mauling left 267 cast
members and employees out of work, prompted refunds for shows booked
months in advance and led to millions in losses for the Mirage, which
had been selling out the show's performances for more than 13 years.
With ticket prices of \$110, the show, performed six times weekly for 45
weeks a year, brought in about \$44 million annually to the Mirage.

\includegraphics{https://static01.nyt.com/images/2020/05/10/obituaries/08horn-virus-lost/08horn-virus-lost-articleLarge.jpg?quality=75\&auto=webp\&disable=upscale}

``Throughout the history of Las Vegas, no artists have meant more to the
development of Las Vegas's global reputation as the entertainment
capital of the world than Siegfried and Roy,'' J. Terrence Lanni, who
was the chairman of MGM Mirage, said after the attack. ``They are so
much more than the stars of the Mirage. They are the very heart of our
resort.''

\hypertarget{how-it-all-began}{%
\subsection{How It All Began}\label{how-it-all-began}}

Roy Uwe Ludwig Horn was born on Oct. 3, 1944, in Nordenham, Germany,
near Bremen. Like Mr. Fischbacher, who was five years older and raised
in Rosenheim, a village in Bavaria, Mr. Horn grew up in the turmoil of
wartime and postwar Germany. While Mr. Fischbacher was drawn to magic,
Mr. Horn was taken with animals, including his wolfdog Hexe, and a
cheetah, Chico, at a zoo in Bremen where the boy took an after-school
job feeding animals and cleaning cages.

It was a chance meeting in 1957, when both were working on a German
cruise ship, that led to their partnership. Mr. Fischbacher, a steward,
was entertaining passengers with magic tricks, and Mr. Horn, a cabin
boy, caught his act.

``I told Siegfried if he could make rabbits come out of a hat, why
couldn't he make cheetahs appear?'' Mr. Horn recalled. He said he
smuggled Chico out of the zoo and aboard the ship in a laundry bag. The
new trick, he said, was a hit with passengers.

They formed a partnership in 1959. By 1964, Mr. Horn and Mr.
Fischbacher, still with Chico, were on the road, performing in cabarets
and theaters in Germany and Switzerland. The results were mixed ---
Chico ate steak, the men potatoes --- until Princess Grace of Monaco saw
them at a 1966 charity benefit in Monte Carlo and gave them a rave
notice.

A rush of publicity ensued. Adding animals and tricks, they were soon
playing nightclubs in Paris and other European cities. They made their
Las Vegas debut at the Tropicana in 1967, and by the early 1970s, having
made Las Vegas their base, were under contract at the MGM Grand. In
1981, they became the main act at the Frontier Hotel; in seven years
there, they performed before three million people.

In 1987, they signed a five-year, \$57.5 million contract with Steve
Wynn, owner of the planned \$640 million Mirage casino-hotel --- a deal
Variety called the largest in show business history. It included \$40
million more for a new theater for the show and an \$18 million ``Secret
Garden'' hotel habitat for the animals.

While the hotel was being built, the show went on a 10-month tour of
Japan, where patrons paid up to \$300 a ticket. And in 1989, Siegfried
\& Roy performed 32 shows over four weeks before packed houses at Radio
City Music Hall in New York. By then, they had added more handlers and
assistants and scores of exotic animals, including white tigers, lions,
panthers, elephants and pythons.

Opening night at the Mirage in 1990 marked the show's 10,000th
performance in Las Vegas. In the ensuing years there were thousands of
shows, which took in hundreds of millions of dollars. In 2001, after 20
years of sold-out performances, Mr. Horn and Mr. Fischbacher signed
lifetime contracts to work at the Mirage.

A 1994 ABC television special, ``Siegfried \& Roy: The Magic, the
Mystery,'' showed part of their act, but focused on the performers at
home and interacting with animals. The 1999 Imax film
\href{https://www.youtube.com/watch?v=hHiA1iDYkWU}{``Siegfried \& Roy:
The Magic Box''} detailed the men's partnership and featured 3-D shots
in which tigers and lions seemed to leap into the audience.

Animal rights activists generally oppose using wild animals in shows,
but few ever accused Siegfried \& Roy of mistreating animals. Mr. Horn
said theirs were exercised daily by handlers, fed special diets and even
had their teeth brushed three times a month. He draped big cats around
his neck for photographers and was seen cuddling and petting animals. He
called his training methods ``affection conditioning.''

The men shared their home with tigers, jaguars, mastiffs and other
creatures that often roamed their compound freely. Mr. Horn said he
slept with a tiger or a leopard every night. He said he and Mr.
Fischbacher kept separate quarters and took separate vacations. Mr.
Horn's mother, Johanna, also lived at their home for many years until
her death in 2000.

In addition to Mr. Fischbacher, Mr. Horn is survived by a brother,
Werner Horn.

Conservation had been on Mr. Horn's agenda for decades. In 1982, he and
Mr. Fischbacher obtained their first three white tiger cubs from the
Cincinnati Zoo, an important breeding site for white tigers at that
time. Over the years the partners multiplied their brood tenfold. They
eventually owned 10 percent of the world's white tigers.

White tigers, which have blue eyes and are larger than orange tigers,
possess a recessive genetic property that creates a virtual absence of
orange pigment in the fur, though most have dark stripes. Another
genetic condition renders the stripes pale, producing an almost
snow-white coat. White tigers are extremely rare in the wild; several
hundred are in captivity, about 100 of them in India. Nearly all are
descendants of a white cub found by the Maharajah of Rewa in India in
1951.

In 1995, Mr. Horn and Mr. Fischbacher obtained two white lion cubs from
the Johannesburg Zoological Society in South Africa. Only five white
lions were known to exist then. Dr. Patrick Condy of the zoological
society later told Cats magazine that breeding efforts by the two men
``virtually guarantee the white lions of Timbavati will not only
continue to exist, but flourish.''

Michael Levenson contributed reporting.

\href{https://www.nytimes.com/interactive/2020/obituaries/people-died-coronavirus-obituaries.html?action=click\&pgtype=Article\&state=default\&region=BELOW_MAIN_CONTENT\&context=covid_obits_promo}{}

\hypertarget{those-weve-lost}{%
\section{Those We've Lost}\label{those-weve-lost}}

The coronavirus pandemic has taken an incalculable death toll. This
series is designed to put names and faces to the numbers.

Read more

\includegraphics{https://static01.nyt.com/images/2020/07/30/obituaries/30Pedro/30Pedro-square640.jpg}

\hypertarget{bernaldina-josuxe9-pedro}{%
\section{Bernaldina José Pedro}\label{bernaldina-josuxe9-pedro}}

d. Boa Vista, Brazil

Leader among the Indigenous Macuxi

\includegraphics{https://static01.nyt.com/images/2020/07/31/obituaries/31Swing/merlin_175167783_8913bc90-0d64-43f3-a655-1bb1bf1601c9-square640.jpg}

\hypertarget{john-eric-swing}{%
\section{John Eric Swing}\label{john-eric-swing}}

d. Fountain Valley, Calif.

Champion of Filipino-Americans

\includegraphics{https://static01.nyt.com/images/2020/07/27/obituaries/27Victor/merlin_175001436_38b11f8e-227a-4e2c-9821-7618af9b2524-square640.jpg}

\hypertarget{victor-victor}{%
\section{Victor Victor}\label{victor-victor}}

d. Santo Domingo, Dominican Republic

Beloved musician of the Dominican Republic

\includegraphics{https://static01.nyt.com/images/2020/07/31/obituaries/31Negron/merlin_175160169_516322ae-fd23-4969-b6b2-193ced371105-square640.jpg}

\hypertarget{dr-eddie-negruxf3n}{%
\section{Dr. Eddie Negrón}\label{dr-eddie-negruxf3n}}

d. Fort Walton Beach, Fla.

Internist on Florida's Emerald Coast

\includegraphics{https://static01.nyt.com/images/2020/07/30/obituaries/30Dobson/merlin_175115928_f6b9271c-8f05-4fe1-a38a-5ca4a58f8935-square640.jpg}

\hypertarget{dobby-dobson}{%
\section{Dobby Dobson}\label{dobby-dobson}}

d. Coral Springs, Fla.

Jamaican singer and songwriter

\includegraphics{https://static01.nyt.com/images/2020/08/01/obituaries/28Gonzalez/merlin_175002771_beb57888-3951-409a-ae13-03a94b2e962e-square640.jpg}

\hypertarget{waldemar-gonzalez}{%
\section{Waldemar Gonzalez}\label{waldemar-gonzalez}}

d. White Plains, N.Y.

Teacher and social worker

Advertisement

\protect\hyperlink{after-bottom}{Continue reading the main story}

\hypertarget{site-index}{%
\subsection{Site Index}\label{site-index}}

\hypertarget{site-information-navigation}{%
\subsection{Site Information
Navigation}\label{site-information-navigation}}

\begin{itemize}
\tightlist
\item
  \href{https://help.nytimes.com/hc/en-us/articles/115014792127-Copyright-notice}{©~2020~The
  New York Times Company}
\end{itemize}

\begin{itemize}
\tightlist
\item
  \href{https://www.nytco.com/}{NYTCo}
\item
  \href{https://help.nytimes.com/hc/en-us/articles/115015385887-Contact-Us}{Contact
  Us}
\item
  \href{https://www.nytco.com/careers/}{Work with us}
\item
  \href{https://nytmediakit.com/}{Advertise}
\item
  \href{http://www.tbrandstudio.com/}{T Brand Studio}
\item
  \href{https://www.nytimes.com/privacy/cookie-policy\#how-do-i-manage-trackers}{Your
  Ad Choices}
\item
  \href{https://www.nytimes.com/privacy}{Privacy}
\item
  \href{https://help.nytimes.com/hc/en-us/articles/115014893428-Terms-of-service}{Terms
  of Service}
\item
  \href{https://help.nytimes.com/hc/en-us/articles/115014893968-Terms-of-sale}{Terms
  of Sale}
\item
  \href{https://spiderbites.nytimes.com}{Site Map}
\item
  \href{https://help.nytimes.com/hc/en-us}{Help}
\item
  \href{https://www.nytimes.com/subscription?campaignId=37WXW}{Subscriptions}
\end{itemize}
