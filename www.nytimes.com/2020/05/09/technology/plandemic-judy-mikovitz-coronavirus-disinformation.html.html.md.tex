Sections

SEARCH

\protect\hyperlink{site-content}{Skip to
content}\protect\hyperlink{site-index}{Skip to site index}

\href{https://www.nytimes.com/section/technology}{Technology}

\href{https://myaccount.nytimes.com/auth/login?response_type=cookie\&client_id=vi}{}

\href{https://www.nytimes.com/section/todayspaper}{Today's Paper}

\href{/section/technology}{Technology}\textbar{}Virus Conspiracists
Elevate a New Champion

\url{https://nyti.ms/3clD6Nv}

\begin{itemize}
\item
\item
\item
\item
\item
\end{itemize}

\href{https://www.nytimes.com/news-event/coronavirus?action=click\&pgtype=Article\&state=default\&region=TOP_BANNER\&context=storylines_menu}{The
Coronavirus Outbreak}

\begin{itemize}
\tightlist
\item
  live\href{https://www.nytimes.com/2020/08/01/world/coronavirus-covid-19.html?action=click\&pgtype=Article\&state=default\&region=TOP_BANNER\&context=storylines_menu}{Latest
  Updates}
\item
  \href{https://www.nytimes.com/interactive/2020/us/coronavirus-us-cases.html?action=click\&pgtype=Article\&state=default\&region=TOP_BANNER\&context=storylines_menu}{Maps
  and Cases}
\item
  \href{https://www.nytimes.com/interactive/2020/science/coronavirus-vaccine-tracker.html?action=click\&pgtype=Article\&state=default\&region=TOP_BANNER\&context=storylines_menu}{Vaccine
  Tracker}
\item
  \href{https://www.nytimes.com/interactive/2020/07/29/us/schools-reopening-coronavirus.html?action=click\&pgtype=Article\&state=default\&region=TOP_BANNER\&context=storylines_menu}{What
  School May Look Like}
\item
  \href{https://www.nytimes.com/live/2020/07/31/business/stock-market-today-coronavirus?action=click\&pgtype=Article\&state=default\&region=TOP_BANNER\&context=storylines_menu}{Economy}
\end{itemize}

Advertisement

\protect\hyperlink{after-top}{Continue reading the main story}

Supported by

\protect\hyperlink{after-sponsor}{Continue reading the main story}

\hypertarget{virus-conspiracists-elevate-a-new-champion}{%
\section{Virus Conspiracists Elevate a New
Champion}\label{virus-conspiracists-elevate-a-new-champion}}

A video showcasing baseless arguments by Dr. Judy Mikovits, including
attacks on Dr. Anthony Fauci, has been viewed more than eight million
times in the past week.

\includegraphics{https://static01.nyt.com/images/2020/05/11/business/00virus-mikovits-print/merlin_53613071_01d78a84-91a4-4066-91e7-cfa2e41b330f-articleLarge.jpg?quality=75\&auto=webp\&disable=upscale}

By \href{https://www.nytimes.com/by/davey-alba}{Davey Alba}

\begin{itemize}
\item
  May 9, 2020
\item
  \begin{itemize}
  \item
  \item
  \item
  \item
  \item
  \end{itemize}
\end{itemize}

\href{https://www.nytimes.com/es/2020/05/11/espanol/plandemia-judy-mikovitz-desinformacion.html}{Leer
en español}

In a video posted to YouTube on Monday, a woman animatedly described an
unsubstantiated secret plot by global elites like Bill Gates and Dr.
Anthony Fauci to use the coronavirus pandemic to profit and grab
political power.

In the 26-minute video, the woman asserted how Dr. Fauci, the director
of the National Institute of Allergy and Infectious Diseases and a
leading voice on the coronavirus, had buried her research about how
vaccines can damage people's immune systems. It is those weakened immune
systems, she declared, that have made people susceptible to illnesses
like Covid-19.

The video, a scene from a longer dubious documentary called
``\href{https://www.nytimes.com/2020/05/20/technology/plandemic-movie-youtube-facebook-coronavirus.html}{Plandemic},''
was quickly seized upon by anti-vaccinators, the conspiracy group QAnon
and activists from the Reopen America movement, generating more than
eight million views. And it has turned the woman --- Dr. Judy Mikovits,
62, a discredited scientist --- into a new star of virus disinformation.

Her ascent was powered not only by the YouTube video but also by a book
that she published in April, ``Plague of Corruption,'' which frames Dr.
Mikovits as a truth-teller fighting deception in science. In recent
weeks, she has become a darling of far-right publications like The Epoch
Times and The Gateway Pundit. Mentions of her on social media and
television have spiked to as high as 14,000 a day, according to the
media insights company Zignal Labs.

The rise of Dr. Mikovits is the latest twist in the virus disinformation
wars, which have swelled throughout the pandemic. Conspiracy theorists
have used the uncertainty and fear around the disease to mint many
villains. Those include
\href{https://www.nytimes.com/2020/03/28/technology/coronavirus-fauci-trump-conspiracy-target.html}{Dr.
Fauci after he appeared to slight President Trump} and
\href{https://www.nytimes.com/2020/04/17/technology/bill-gates-virus-conspiracy-theories.html}{Mr.
Gates, a co-founder of Microsoft, as someone who started the disease}.
They have also pushed the baseless idea that
\href{https://www.nytimes.com/2020/04/10/technology/coronavirus-5g-uk.html}{5G
wireless waves can help cause the disease}.

On the flip side, they have created their own heroes, like Dr. Mikovits.

The conspiracy theorists ``recast a pusher of discredited pseudoscience
as a whistle-blowing counterpoint to real expertise,'' said Renee
DiResta, a disinformation researcher at the Stanford Internet
Observatory.

Dr. Mikovits did not respond to requests for comment.

Judy Mikovits has a degree in biology from the University of Virginia
and a Ph.D. in molecular biology from George Washington University. From
1992 to 2001, she worked at the National Cancer Institute as a
postdoctoral fellow, a staff scientist and a lab director, then served
as research director of the Whittemore Peterson Institute for
Neuro-Immune Disease from 2006 to 2011. In 2011, after her research into
chronic fatigue syndrome was discredited, she was
\href{https://www.sciencemag.org/news/2011/10/chronic-fatigue-syndrome-researcher-fired-amidst-new-controversy}{fired}
from Whittemore.

Dr. Mikovits's rise to internet notoriety has been sudden. According to
data from Zignal Labs, she was rarely mentioned on social media
platforms in February.

\includegraphics{https://static01.nyt.com/images/2020/05/08/technology/09virus-mikovits/oakImage-1588977888511-articleLarge.png?quality=75\&auto=webp\&disable=upscale}

By April, coverage of Dr. Mikovits rose to 800 mentions a day. That
month, Darla Shine, the wife of Bill Shine, a former Fox News executive
and former top aide to Mr. Trump, promoted Dr. Mikovits's book
\href{https://twitter.com/DarlaShine/status/1246607755915526144}{in a
tweet}. Videos by The Epoch Times, a publication with ties to the Falun
Gong, and the conservative outlet ``The Next News Network'' interviewed
Dr. Mikovits about the pandemic, generating more than 1.5 million views
on social networks.

Then came the video from ``Plandemic,'' which made mentions of Dr.
Mikovits on social media spike far higher. The video was produced by
Mikki Willis, who was involved in making ``Bernie or Bust'' and ``Never
Hillary'' videos during the 2016 presidential campaign.

\hypertarget{latest-updates-economy}{%
\section{\texorpdfstring{\href{https://www.nytimes.com/live/2020/07/31/business/stock-market-today-coronavirus?action=click\&pgtype=Article\&state=default\&region=MAIN_CONTENT_1\&context=storylines_live_updates}{Latest
Updates:
Economy}}{Latest Updates: Economy}}\label{latest-updates-economy}}

\href{https://www.nytimes.com/live/2020/07/31/business/stock-market-today-coronavirus?action=click\&pgtype=Article\&state=default\&region=MAIN_CONTENT_1\&context=storylines_live_updates\#kodaks-chief-executive-was-given-stock-options-then-the-share-price-spiked-1000-percent}{31h
ago}

\href{https://www.nytimes.com/live/2020/07/31/business/stock-market-today-coronavirus?action=click\&pgtype=Article\&state=default\&region=MAIN_CONTENT_1\&context=storylines_live_updates\#kodaks-chief-executive-was-given-stock-options-then-the-share-price-spiked-1000-percent}{Kodak's
chief executive was given stock options. Then the share price spiked
1,000 percent.}

\href{https://www.nytimes.com/live/2020/07/31/business/stock-market-today-coronavirus?action=click\&pgtype=Article\&state=default\&region=MAIN_CONTENT_1\&context=storylines_live_updates\#fitch-ratings-downgrades-its-outlook-on-us-debt}{34h
ago}

\href{https://www.nytimes.com/live/2020/07/31/business/stock-market-today-coronavirus?action=click\&pgtype=Article\&state=default\&region=MAIN_CONTENT_1\&context=storylines_live_updates\#fitch-ratings-downgrades-its-outlook-on-us-debt}{Fitch
Ratings downgrades its outlook on U.S. debt.}

\href{https://www.nytimes.com/live/2020/07/31/business/stock-market-today-coronavirus?action=click\&pgtype=Article\&state=default\&region=MAIN_CONTENT_1\&context=storylines_live_updates\#us-sanctions-more-chinese-officials-over-human-rights-violations-as-tensions-flare}{41h
ago}

\href{https://www.nytimes.com/live/2020/07/31/business/stock-market-today-coronavirus?action=click\&pgtype=Article\&state=default\&region=MAIN_CONTENT_1\&context=storylines_live_updates\#us-sanctions-more-chinese-officials-over-human-rights-violations-as-tensions-flare}{U.S.
sanctions more Chinese officials over human rights violations as
tensions flare}

\href{https://www.nytimes.com/live/2020/07/31/business/stock-market-today-coronavirus?action=click\&pgtype=Article\&state=default\&region=MAIN_CONTENT_1\&context=storylines_live_updates}{See
more updates}

More live coverage:
\href{https://www.nytimes.com/2020/08/01/world/coronavirus-covid-19.html?action=click\&pgtype=Article\&state=default\&region=MAIN_CONTENT_1\&context=storylines_live_updates}{Global}

Her arguments also began to spill over into the real world, including
her baseless assertion that ``wearing the mask literally activates your
own virus.'' There is no evidence that wearing a mask can activate
viruses and make people sick. On Thursday in Sacramento, Calif.,
\href{https://twitter.com/mikeblountsac/status/1258439584507031552}{a
woman brandished a sign} in front of the state Capitol building that
read, ``Do you know who Dr. Judy Mikovits is? Then don't tell me I need
a silly mask.''

YouTube and Facebook have removed the ``Plandemic'' scene, saying that
it spread inaccurate information about Covid-19 that could be harmful to
the public. But the video continues to circulate, as people post new
copies. Twitter added an ``unsafe'' warning on
\href{https://twitter.com/DrJudyAMikovits/status/1257425999408611330}{at
least one link} featuring Dr. Mikovits on the social network, and
blocked the hashtags \#PlagueOfCorruption and \#Plandemicmovie from
trends and search.

Dr. Mikovits has attacked Dr. Fauci online since at least 2018. But her
claims did not gain much traction until this year, when the
\href{https://www.nytimes.com/2020/03/28/technology/coronavirus-fauci-trump-conspiracy-target.html}{narrative
that Dr. Fauci was secretly plotting to undermine and discredit the
president started spreading}.

Dr. Mikovits says Dr. Fauci's attacks on her work date back to the
1980s, when she
\href{https://nyaspubs.onlinelibrary.wiley.com/doi/abs/10.1111/j.1749-6632.1990.tb16119.x}{contributed
research} to the National Cancer Institute as a graduate student. In the
video being shared, Dr. Mikovits alleges that Dr. Fauci intercepted her
research on H.I.V. to make money off patents, threatened her and then
took undeserved credit for moving the field of H.I.V. treatment forward.

She also ties her professional downfall to Dr. Fauci. In 2009, Dr.
Mikovits published research in the journal Science claiming to show that
a mouse retrovirus caused chronic fatigue syndrome and other illnesses.
That research gained significant media attention, but it was discredited
a couple of years later, including with a retraction by the journal. Dr.
Mikovits was briefly jailed in California on charges of theft made by
Whittemore. The charges were later dropped.

Dr. Mikovits has sought to reframe the scandal as part of a broader
campaign of persecution, aimed at silencing her work questioning the
safety of vaccines.

There is no evidence that Dr. Fauci and Dr. Mikovits interacted. This
week, in a
\href{https://www.snopes.com/fact-check/scientist-vaccine-jailed/}{statement
to the fact-checking website Snopes}, Dr. Fauci denied ever having
threatened Dr. Mikovits. ``I have no idea what she is talking about,''
he wrote.

The National Cancer Institute referred an inquiry about Dr. Mikovits's
claims to the National Institutes of Health, the agency that oversees
the N.C.I.'s cancer research and training. Dr. Fauci came to the
National Institutes of Health as a clinical associate in 1968, and was
appointed director of the National Institute of Allergy and Infectious
Diseases at the N.I.H. by 1984.

In a statement, the agency said, ``The National Institutes of Health and
National Institute of Allergy and Infectious Diseases are focused on
critical research aimed at ending the Covid-19 pandemic and preventing
further deaths. We are not engaging in tactics by some seeking to derail
our efforts.''

Dr. Ian Lipkin, the director of the Center for Infection and Immunity at
Columbia University, said in an interview on Saturday morning that Dr.
Fauci had asked him in 2011 to design a study that would address whether
Dr. Mikovits and others could reproduce her research showing an
association between XMRV, the mouse retrovirus, and chronic fatigue
syndrome. He pointed to a
\href{https://www.mailman.columbia.edu/research/center-infection-and-immunity/cii-press-conference-multicenter-study-chronic-fatigue}{September
2012 news conference at Columbia} in which Dr. Mikovits admitted the
link her original research had made between the mouse retrovirus and
chronic fatigue syndrome was ``simply not there.''

``Now is the time to use'' the invalidating results that came out of the
effort to reproduce her research ``and move forward,''
\href{https://www.youtube.com/watch?list=PLm8X_9pQvTbVCBWor35cXIOJpsjGfCnKm\&time_continue=334\&v=Y3dJsNZuNUk\&feature=emb_title}{Dr.
Mikovits said at the time}. ``And that's what science is all about.''

Ivan Oransky, a co-founder of the academic watchdog
\href{https://retractionwatch.com/2020/05/06/who-is-judy-mikovits/}{Retraction
Watch}, which has followed Dr. Mikovits's work closely, said that when
he sees videos like the one posted in the past week, ``they tend to
coalesce around certain kinds of subjects, then the trajectory turns to
martyrhood really quickly.''

There is some evidence that prominent members of conspiracy groups have
tried to give her name and her story a lift online.

Zach Vorhies, a former YouTube employee who has recently promoted QAnon
conspiracy theories, posted a GoFundMe campaign on April 19 titled
``Help me amplify Pharma Whistleblower Judy Mikovits.'' The campaign was
first spotted by Ms. DiResta, of the Stanford Internet Observatory.

A day before the GoFundMe campaign began, a newly created account for
Dr. Mikovits
\href{https://twitter.com/DrJudyAMikovits/status/1251647330572476417}{tweeted
for the first time}. ``A big thanks goes out to Zach Vorhies
(@Perpetualmaniac) for helping me get on Twitter!'' It was retweeted 400
times and liked more than 2,200 times. The account has gained over
111,000 followers in less than a month.

GoFundMe removed the page on Friday, stating that the campaign violated
the website's terms of service for ``campaigns that are fraudulent,
misleading, inaccurate, dishonest, or impossible.''

Mr. Vorhies did not respond to requests for comment.

Dr. Mikovits's newfound notoriety has also lifted sales of her new book.
This week, ``Plague of Corruption'' shot to No. 1 on Amazon's print
best-seller list. The book was out of stock on Friday. Amazon said that
the book did not violate the company's content guidelines.

Skyhorse, the independent publishing company behind the book, defended
its decision to print Dr. Mikovits. ``The world should discuss the ideas
in this book, rather than allow censorship to prevail,'' a spokeswoman
for Skyhorse said.

Dr. Peter J. Hotez, dean of the National School of Tropical Medicine at
Baylor College of Medicine, said her rise illustrated how the
anti-vaccination movement had ``taken a new ominous twist'' with the
coronavirus.

``They've now aligned themselves with far-right groups,'' Dr. Hotez
said, ``and their weapons of choice are YouTube, Facebook and Amazon.''

Sheera Frenkel and Alexandra Alter contributed reporting. Ben Decker and
Jack Begg contributed research.

Advertisement

\protect\hyperlink{after-bottom}{Continue reading the main story}

\hypertarget{site-index}{%
\subsection{Site Index}\label{site-index}}

\hypertarget{site-information-navigation}{%
\subsection{Site Information
Navigation}\label{site-information-navigation}}

\begin{itemize}
\tightlist
\item
  \href{https://help.nytimes.com/hc/en-us/articles/115014792127-Copyright-notice}{©~2020~The
  New York Times Company}
\end{itemize}

\begin{itemize}
\tightlist
\item
  \href{https://www.nytco.com/}{NYTCo}
\item
  \href{https://help.nytimes.com/hc/en-us/articles/115015385887-Contact-Us}{Contact
  Us}
\item
  \href{https://www.nytco.com/careers/}{Work with us}
\item
  \href{https://nytmediakit.com/}{Advertise}
\item
  \href{http://www.tbrandstudio.com/}{T Brand Studio}
\item
  \href{https://www.nytimes.com/privacy/cookie-policy\#how-do-i-manage-trackers}{Your
  Ad Choices}
\item
  \href{https://www.nytimes.com/privacy}{Privacy}
\item
  \href{https://help.nytimes.com/hc/en-us/articles/115014893428-Terms-of-service}{Terms
  of Service}
\item
  \href{https://help.nytimes.com/hc/en-us/articles/115014893968-Terms-of-sale}{Terms
  of Sale}
\item
  \href{https://spiderbites.nytimes.com}{Site Map}
\item
  \href{https://help.nytimes.com/hc/en-us}{Help}
\item
  \href{https://www.nytimes.com/subscription?campaignId=37WXW}{Subscriptions}
\end{itemize}
