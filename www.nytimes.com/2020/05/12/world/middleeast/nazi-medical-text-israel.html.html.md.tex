Sections

SEARCH

\protect\hyperlink{site-content}{Skip to
content}\protect\hyperlink{site-index}{Skip to site index}

\href{https://www.nytimes.com/section/world/middleeast}{Middle East}

\href{https://myaccount.nytimes.com/auth/login?response_type=cookie\&client_id=vi}{}

\href{https://www.nytimes.com/section/todayspaper}{Today's Paper}

\href{/section/world/middleeast}{Middle East}\textbar{}In Israel, Modern
Medicine Grapples With Ghosts of the Third Reich

\url{https://nyti.ms/2Ws0pQ4}

\begin{itemize}
\item
\item
\item
\item
\item
\item
\end{itemize}

Advertisement

\protect\hyperlink{after-top}{Continue reading the main story}

Supported by

\protect\hyperlink{after-sponsor}{Continue reading the main story}

\hypertarget{in-israel-modern-medicine-grapples-with-ghosts-of-the-third-reich}{%
\section{In Israel, Modern Medicine Grapples With Ghosts of the Third
Reich}\label{in-israel-modern-medicine-grapples-with-ghosts-of-the-third-reich}}

A Palestinian surgeon, a Jewish patient, a Nazi medical text --- and an
unlikely bond.

\includegraphics{https://static01.nyt.com/images/2020/04/21/world/00israel-nazi04/merlin_171330726_de593ea3-7d90-4eb9-b14f-c01b37e1b762-articleLarge.jpg?quality=75\&auto=webp\&disable=upscale}

\href{https://www.nytimes.com/by/isabel-kershner}{\includegraphics{https://static01.nyt.com/images/2018/10/12/multimedia/author-isabel-kershner/author-isabel-kershner-thumbLarge.png}}

By \href{https://www.nytimes.com/by/isabel-kershner}{Isabel Kershner}

\begin{itemize}
\item
  Published May 12, 2020Updated May 14, 2020
\item
  \begin{itemize}
  \item
  \item
  \item
  \item
  \item
  \item
  \end{itemize}
\end{itemize}

\href{https://www.nytimes.com/es/2020/05/15/espanol/mundo/libro-nazi-israel-medicina.html}{Leer
en español}

JERUSALEM --- The explosion flung him skyward, legs first, before he
crashed to the ground.

It was June 2002, at the height of the second Palestinian intifada. Dvir
Musai, then a 13-year-old Israeli schoolboy from a religious Jewish
settlement, was on a class cherry-picking trip in the southern West
Bank. On his way back to the bus, he stepped on a mine laid by
Palestinian militants and was gravely wounded, along with two other
boys.

``There was a lot of smoke, clumps of earth falling, a smell of burning
and gunpowder,'' Mr. Musai, now 31, recalled.

Decades of agony followed. Mr. Musai's right foot felt as if it were
permanently afire. And then last year, a surgeon offered him hope ---
and a disquieting disclosure.

In pre-op at the Hadassah Medical Center in Jerusalem, Dr. Madi el-Haj
told his patient that the anatomical atlas he would use to guide him
through the intricate nerve pathways had been produced by Nazis. Its
illustrations are believed to be based on the dissected victims of the
Nazi court system under Hitler's Third Reich.

If there were objections, Dr. el-Haj told the Musai family, he could
operate without it --- but it would be harder. He noted that there was
rabbinical approval for the book's use.

Mr. Musai's mother, Chana, had lost relatives in the Holocaust.

``She said, `If it can help now, we'll use it,''' Mr. Musai recalled.

That gut-wrenching decision went to the heart of a longstanding debate
about the ethics of drawing on knowledge derived from the Nazis'
expansive medical and scientific experimentation --- and in this case,
the ethics of using the textbook,
``\href{https://www.nytimes.com/1996/11/26/science/doctors-question-use-of-nazi-s-medical-atlas.html}{Atlas
of Topographical and Applied Human Anatomy}.''

The book, by Eduard Pernkopf, stands out for its accuracy and detail,
and even in an age of state-of-the-art imaging, some surgeons, among
them those who perform peripheral nerve procedures, still find its
drawings invaluable.

In a perverse twist, the more advanced the relatively new field of
peripheral nerve surgery becomes, the more reliant on the atlas some of
its practitioners say they find themselves. That is because even
high-tech imaging is of limited use to the complex discipline, in which
doctors treat problems like chronic pain caused by nerves that are
damaged or trapped.

\includegraphics{https://static01.nyt.com/images/2020/04/21/world/00israel-nazi01/merlin_171330960_5ce1814f-a53c-4c98-bf2c-0b19454cc66a-articleLarge.jpg?quality=75\&auto=webp\&disable=upscale}

Pernkopf began work on the atlas at the University of Vienna, where he
became chairman of anatomy in 1933, the year he joined the Nazi party.
With Hitler's 1938 annexation of Austria, he became dean of the medical
faculty, then president of the university.

The illustrators to whom Pernkopf turned to produce the atlas were also
Nazi enthusiasts. Three of the four illustrators incorporated swastikas,
SS lightning bolts and other Nazi insignia into their signatures ---
hallmarks of evil airbrushed out of later editions.

Less is clear about the people whose bodies were dissected so that the
illustrators could produce their work. Over the years, there have been
questions about whether some had been killed in Hitler's death camps.
Those questions remain unresolved, but many experts believe that most of
the prisoners were Austrians condemned in the courts.

After the war, Pernkopf spent three years in an Allied prison camp but
was not charged with war crimes. He continued work on the atlas until
his death in 1955.

A two-volume edition was published in five languages, with the first
American edition coming out in 1963. Elsevier, a European scientific
publisher that currently holds the copyright, stopped printing it on
ethical grounds, but the volumes can be found in private collections and
purchased on eBay and Amazon.

Scholars first raised questions about the origins of the atlas in the
1980s as the Cold War's ``Great Silence'' about the Nazis' medical
legacy began to crack.

By the 1990s, the controversy was drawing wider public attention.

Dr. Howard Israel, an oral surgeon at Columbia University who had
routinely used the atlas, exposed the Nazi symbols in the artists'
signatures included in early editions of the book.

Then Dr. Israel and Dr. William Seidelman, a Toronto physician, turned
for help to Israel's official Holocaust memorial, Yad Vashem, asking it
to press the University of Vienna to investigate the background of the
atlas --- and of the dissected cadavers its authors used. After some
initial reluctance, the university agreed.

``Things started to unravel,'' recounted Dr. Seidelman, who now lives in
Jerusalem.

From 1938 to 1945, the university's anatomical institute received more
than 1,370 bodies of prisoners executed by the Vienna court system,
according to the findings of an investigative committee. More than half
had been political prisoners --- people targeted by the Nazi regime. At
that time in Austria, joking about Hitler was enough to warrant
execution, often by decapitation.

Dr. el-Haj, the Hadassah surgeon, said he was first introduced to the
atlas while studying under Dr. Susan Mackinnon, a pioneer in peripheral
nerve surgery, at Washington University in St. Louis.

``She knew I came from Israel --- she thought I was a Jewish guy,'' he
recalled.

That he was, in fact, an Arab Muslim from the Galilee changed nothing.

``I was shocked,'' he said. ``It's a matter of humanity.''

Image

Dvir Musai was 13 when he stepped on a mine.~Dr. el-Haj operated on him
years later.Credit...Dan Balilty for The New York Times

Dr. Mackinnon bought her first copy in the early 1980s as a young
plastic surgeon in Baltimore, and used it to guide many of her surgical
procedures.

But troubled by the provenance of the illustrations, Dr. Mackinnon
photocopied the first scholarly articles about Pernkopf's past a few
years later and tucked them into the book as a constant reminder.

In 2015, Dr. Mackinnon and her longtime associate Andrew Yee wanted to
share drawings from the atlas on an online teaching platform, and sought
an opinion from Dr. Sabine Hildebrandt, a Boston physician who has
studied the Third Reich.

An international effort was already underway to determine how to handle
unearthed human remains and medical specimens from the Holocaust era.

Dr. Hildebrandt took on Dr. Mackinnon's query and consulted with other
experts, giving rise to a special set of recommendations regarding the
Pernkopf atlas in a document known as the
``\href{https://www.bu.edu/jewishstudies/files/2018/08/HOW-TO-DEAL-WITH-HOLOCAUST-ERA-REMAINS.FINAL_.pdf}{Vienna
Protocol}.'' It was written by a prominent American rabbi and ethicist,
Joseph A. Polak, and formally adopted by a 2017 symposium of experts at
Yad Vashem. Under the protocol, the atlas can be used if there is full
disclosure about its origins.

In a recent survey of an international group of nerve surgeons, Dr.
Mackinnon and Mr. Yee found that 59 percent of the 182 respondents were
aware of the Pernkopf atlas, 41 percent had used it at some point and 13
percent were currently using it.

But the debate is hardly settled.

Dr. Justin M. Sacks, chief of the division of plastic and reconstructive
surgery at Washington University, said he had never come across the
atlas until he arrived at the department this year. He argued that it
was morally and ethically wrong to use it and that there were perfectly
adequate substitutes available in print or online.

``I'm not looking to stir a controversy,'' he said in an interview,
``but I'm looking to put it where it belongs: in a museum.''

Dr. el-Haj said that while the alternatives might be good enough in
other medical fields, when it came to peripheral nerve surgery, they
were no match for Pernkopf.

One of eight siblings, Dr. el-Haj grew up in a farming village and
aspired to become a nerve surgeon, he said, in the hope of helping his
father, who as a young man was left with a paralyzed arm and leg by a
work accident. After studying in the United States, Dr. el-Haj returned
to Jerusalem with his own Pernkopf volumes in August 2018.

Around the same time, Mr. Musai, who had undergone dozens of operations
since his injury, returned to his doctors. Now a married father of two,
he could barely walk. His foot could not bear the weight of a sheet at
night.

He was referred to Dr. el-Haj.

From his days as a medical student at Hadassah, Dr. el-Haj, 40,
remembered Mr. Musai as an angry teenager in terrible pain who harbored
a hatred of Arabs.

Mr. Musai acknowledges that was the case.

Image

The origins of the cadavers used for the book have caused a dilemma for
surgeons. Credit...Dan Balilty for The New York Times

``The truth is if they'd sent me to Madi at the beginning of my injury,
I would have said no,'' Mr. Musai said. ``Not because of the atlas, but
because I had a big problem with the Arab population. I saw in everyone
the terrorist who hurt me.''

But now, years later, Dr. el-Haj ran some tests and scheduled surgery.
Guided by Pernkopf's atlas, which he took into the operating room, he
found a necklace of shrapnel laced around the nerve, located the main
branches causing the pain and took them down, alleviating his suffering.

``It sounds like a good joke,'' Mr. Musai said. ``The Muslim surgeon
with the Nazi atlas operating on a Jew.''

The lives of Dr. el-Haj and Mr. Musai have since become intertwined.

Mr. Musai has visited the doctor's family in his village. And when Dr.
el-Haj's mother was hospitalized at Hadassah, Mr. Musai, who now works
as a guide there, visited her. Dr. el-Haj has taken his children to
visit the Musais in their West Bank settlement, too.

Dr. el-Haj said he had used the atlas in about 90 percent of his
operations, always explaining its background to the patients.

``No patient has ever refused,'' he said. ``Not ever. Because these
people can make a pact with the devil to get out of their pain.''

Advertisement

\protect\hyperlink{after-bottom}{Continue reading the main story}

\hypertarget{site-index}{%
\subsection{Site Index}\label{site-index}}

\hypertarget{site-information-navigation}{%
\subsection{Site Information
Navigation}\label{site-information-navigation}}

\begin{itemize}
\tightlist
\item
  \href{https://help.nytimes.com/hc/en-us/articles/115014792127-Copyright-notice}{©~2020~The
  New York Times Company}
\end{itemize}

\begin{itemize}
\tightlist
\item
  \href{https://www.nytco.com/}{NYTCo}
\item
  \href{https://help.nytimes.com/hc/en-us/articles/115015385887-Contact-Us}{Contact
  Us}
\item
  \href{https://www.nytco.com/careers/}{Work with us}
\item
  \href{https://nytmediakit.com/}{Advertise}
\item
  \href{http://www.tbrandstudio.com/}{T Brand Studio}
\item
  \href{https://www.nytimes.com/privacy/cookie-policy\#how-do-i-manage-trackers}{Your
  Ad Choices}
\item
  \href{https://www.nytimes.com/privacy}{Privacy}
\item
  \href{https://help.nytimes.com/hc/en-us/articles/115014893428-Terms-of-service}{Terms
  of Service}
\item
  \href{https://help.nytimes.com/hc/en-us/articles/115014893968-Terms-of-sale}{Terms
  of Sale}
\item
  \href{https://spiderbites.nytimes.com}{Site Map}
\item
  \href{https://help.nytimes.com/hc/en-us}{Help}
\item
  \href{https://www.nytimes.com/subscription?campaignId=37WXW}{Subscriptions}
\end{itemize}
