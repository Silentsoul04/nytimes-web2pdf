Sections

SEARCH

\protect\hyperlink{site-content}{Skip to
content}\protect\hyperlink{site-index}{Skip to site index}

\href{https://myaccount.nytimes.com/auth/login?response_type=cookie\&client_id=vi}{}

\href{https://www.nytimes.com/section/todayspaper}{Today's Paper}

\href{/section/opinion}{Opinion}\textbar{}The Bay Area Billionaires Are
Breaking My Heart

\href{https://nyti.ms/35VzzTG}{https://nyti.ms/35VzzTG}

\begin{itemize}
\item
\item
\item
\item
\item
\end{itemize}

\href{https://www.nytimes.com/interactive/2020/opinion/america-inequality-coronavirus.html?action=click\&pgtype=Article\&state=default\&region=TOP_BANNER\&context=storylines_menu}{\includegraphics{https://static01.nyt.com/newsgraphics/2020/04/10/storylines-op-inequality/128d73ea016db3e2791158d151b6485f57f635a8/NYTOpEd-Inequality-Icon.jpg}
The America We Need is a Times Opinion series on emerging from this
crisis with a fair, resilient society. }

debugid:204

Advertisement

\protect\hyperlink{after-top}{Continue reading the main story}

\href{/section/opinion}{Opinion}

Supported by

\protect\hyperlink{after-sponsor}{Continue reading the main story}

\hypertarget{the-bay-area-billionaires-are-breaking-my-heart}{%
\section{The Bay Area Billionaires Are Breaking My
Heart}\label{the-bay-area-billionaires-are-breaking-my-heart}}

Looking for hope in San Francisco.

\href{https://www.nytimes.com/by/farhad-manjoo}{\includegraphics{https://static01.nyt.com/images/2019/01/08/opinion/farhad-manjoo-opinion/farhad-manjoo-opinion-thumbLarge.png}}

By \href{https://www.nytimes.com/by/farhad-manjoo}{Farhad Manjoo}

Opinion Columnist

\begin{itemize}
\item
  May 13, 2020
\item
  \begin{itemize}
  \item
  \item
  \item
  \item
  \item
  \end{itemize}
\end{itemize}

\includegraphics{https://static01.nyt.com/images/2020/05/09/opinion/sunday/InequalityChapter2Icon-9-02/InequalityChapter2Icon-9-02-articleLarge.jpg?quality=75\&auto=webp\&disable=upscale}

One sun-drenched afternoon last month, I took a long solo bike ride
through the San Francisco Bay Area. I rode from my home to Mountain
View, near the once-desolate stretch of marsh that Google has leased
from NASA to build a monumental new campus. It looks like
\href{https://www.dezeen.com/2019/08/27/google-hq-big-heatherwick-the-111th/}{a
collection of lunar bases made out of origami}.

Construction has been paused under lockdown, and on the fetid plains
surrounding the million-square-foot project, birds sang and wildflowers
painted the horizon, and the trails that run beside the site were packed
to socially distant capacity with masked families on foot and wheel.

\includegraphics{https://static01.nyt.com/images/2020/05/11/opinion/00man1/merlin_172370514_18f2abeb-c495-41b6-9397-73e638727680-articleLarge.jpg?quality=75\&auto=webp\&disable=upscale}

Bicycles and pets, not sirens and fridge-truck morgues, have become the
unlikely icons of the pandemic in the Bay Area.
\href{https://www.sfexaminer.com/the-city/san-francisco-bike-shops-are-booming-during-the-pandemic/}{Bike
shops} and
\href{https://sfpublicpress.org/news/2020-04/want-to-foster-a-dog-get-on-the-waitlist-as-demand-soars-at-bay-area-shelters}{animal
shelters} say they've been inundated with demand. With the streets free
of cars and full of people, the air clean, the cavernous office
buildings empty and their endless parking lots turned into carefree
pedestrian plazas, you'd be forgiven for mistaking some areas of Silicon
Valley under lockdown for outtakes from the ``The Good Place.''

On my way to the Google lunar landing base, I passed by Santiago Villa,
one of the area's few remaining mobile-home parks. It was built in the
1960s as an affordable retirement community. In January, its residents,
who rent the space on which their mobile homes stand, petitioned the
City Council to
\href{https://mv-voice.com/news/2020/01/30/city-council-agrees-in-theory-to-extend-rent-control-to-mountain-view-mobile-home-parks}{include
trailer parks in Mountain View's rent-control} rules.

They're worried that wealthy Googlers
\href{https://www.theringer.com/2016/8/18/16039104/google-santiago-villa-housing-4ac4b1ca49fe}{looking
for a kitschy pied-à-terre} near the new campus will push them out. The
anger has been rising. Last year, the same City Council prohibited RVs
and trailers --- many of them used as homes ---~from parking on the
street; a petition to overturn the RV ban
\href{https://www.mv-voice.com/news/2020/01/15/mountain-view-voters-to-decide-on-rv-parking-ban}{will
be on the ballot in November}.

But as I rode past Santiago Villa, all that rancor felt like a remnant
of the Before Time. Everything was quiet --- then, from one of the
trailers, a jolly trumpet began blowing loud and out of tune.

It was then that I first had the ghoulish idea: Could the coronavirus
have an upside, at least in this one place? What if the pandemic and its
aftermath lead Googlers and trailer park residents to find common cause?
What if, after the virus, the Bay Area's wealthy gained a new
appreciation for those who live on its edges, and finally made room for
them in this digital wonderland?

Image

Construction cranes for Google's new campus loom over the Santiago Villa
trailer park.Credit...Nicholas Albrecht for The New York Times

Image

Santiago Villa was built in the 1960s as an affordable retirement
community.Credit...Nicholas Albrecht for The New York Times

I have lived in the Bay Area for almost 20 years, and for most of that
time, I've felt this place creaking steadily into uninhabitability for
all but the wealthiest few. We have
\href{https://www.sfchronicle.com/bayarea/article/billionaires-San-Francisco-world-report-wealth-x-13832316.php}{one
of the world's highest concentrations of billionaires}, and yet we have
not been able to marshal our immense wealth and ingenuity against our
most blatant and glaring challenges --- including the lack of affordable
housing and entrenched homelessness.

But in this crisis, the Bay Area's response was an unexpected success.
And that has given a lot of people, including me, new hope about what's
possible. Yes, it sounds hokey, but this might be a time for hokeyness.

The first big moment came on March 16, when the six counties around the
San Francisco Bay ordered the first shelter-in-place rules in the United
States. Google, Apple, Facebook and other large employers fell right
into step; they ordered all of their employees to work from home,
setting the pace for most other local businesses to close up shop. And
the tech giants set an important example --- they made a commitment to
keep paying their on-site service workers, even if they could no longer
come on-site to work.

San Francisco, Oakland and San Jose secured thousands of hotel rooms for
homeless people, away from the streets and the risk of the virus in
crowded shelters. Cities opened their streets to pedestrians and
bicycles and closed them to cars. Perhaps most important, officials in
the area were the picture of calm leadership.

When I despaired about our national failures, I found myself tuning into
hear the plain-spoken exhortations of San Francisco's mayor, London
Breed. ``This is going to take all of us,'' Breed
\href{https://www.youtube.com/watch?v=YkkcLEJ-nLY}{told the city late in
March}. ``This is going to take all of us coming together and
sacrificing so that we get through this.''

Image

Mayor London Breed has urged San Franciscans to look out for one
another.Credit...Nicholas Albrecht for The New York Times

And it worked. Thanks to some combination of early action, collective
adherence to public health guidelines, a concerted effort to help the
vulnerable, and
\href{https://www.nbcbayarea.com/news/local/difference-between-covid-19-cases-in-ca-vs-ny-is-likely-sheer-luck-experts/2269934/}{perhaps
just blind luck}, mass death missed the Bay. By the start of May, fewer
than 30 people had died of Covid-19 in San Francisco; in the greater Bay
Area, deaths stand around 350.

The toll is probably an undercount, and
\href{https://www.nytimes.com/2020/04/28/us/coronavirus-california-black-latinos.html}{blacks
and Latinos} are disproportionately represented in it. Still, compared
to many American metropolitan areas, this ranks as a near miracle. San
Francisco's death rate of four per 100,000 residents is one-fourth the
rate in Los Angeles, a fraction of the national average, and nowhere
near New York's.

In the absence of mass death, people around here have had time and
psychic space to imagine longer-term possibilities. If we could band
together so quickly to beat the virus, making so many big changes so
seamlessly, what else are we capable of doing?

I was not alone in my vague sense of optimism.

In an article that went viral among techies last month, the venture
capitalist Marc Andreessen characterized the pandemic as a
\href{https://a16z.com/2020/04/18/its-time-to-build/}{call to arms to
rebuild} American institutions, including our cities. Like many of the
Valley's tech princes, Andreessen has often been skeptical of government
and its champions, but now here he was, cheering them on: ``Demonstrate
that the public sector can build better hospitals, better schools,
better transportation, better cities, better housing,'' he wrote. ``Stop
trying to protect the old, the entrenched, the irrelevant; commit the
public sector fully to the future.''

I heard a similar urgency for grand reform from nearly every Bay Area
official, activist and resident I spoke to --- even those who had
clashed with the tech industry or those whose fights earlier seemed
unwinnable.

Libby Schaaf, the mayor of Oakland, opened up
\href{https://www.citylab.com/transportation/2020/04/slow-streets-oakland-car-free-roads-pedestrians-covid-19/609961/}{74
miles of city streets} for pedestrians and moved hundreds of homeless
people into hotels. She saw the crisis as an opportunity to make
permanent improvements.

Image

Mayor Libby Schaaf opened up 74 miles of city streets for pedestrians in
Oakland and moved hundreds of homeless people into hotels to shield them
from the virus.Credit...Nicholas Albrecht for The New York Times

One example: Schaaf required that the hotels which the city paid to
house the homeless during the pandemic offer the city long-term leases.
``I do not want, at the end of the health emergency, to turn homeless
people back out onto the streets,'' she said.

In April, Ro Khanna, who represents parts of Silicon Valley in the
House, introduced legislation to provide greater pay, health care and
labor protections to workers deemed ``essential'' during the pandemic.
``When we talk about who are the `essential workers,' very few people
are saying it's lawyers or middle or senior management,'' Khanna said.
``They're saying, we want the person who's delivering our groceries, the
person who's keeping the internet open, the electricity flowing, or the
person who's taking care of our kids.''

In a similar way, the crisis illustrated the importance of keeping
everyone healthy ---~even people who lack a place to live. ``Housing is
health care,'' explained Abigail Stewart-Kahn, director of the San
Francisco Department of Homelessness and Supportive Housing. ``That's
something, in my field, that people have been saying for a long time.''
Now, the connection was inescapable ---~people who lacked housing were
also outside of the health care system, and during a pandemic, their
presence on the streets created a risk for everyone else in the city.
``What this has shown us all is that everyone's health is intertwined,''
she said.

More from ``The America We Need''
\href{https://www.nytimes.com/2020/07/04/opinion/sunday/women-work-coronavirus.html?action=click\&pgtype=Article\&state=default\&region=MAIN_CONTENT_2\&context=storylines_related_links}{}

\includegraphics{https://static01.nyt.com/images/2020/07/05/opinion/05nashtop/05nashtop-threeByTwoSmallAt2X.jpg}

Women Ask Themselves, `How Can I Do This for One More Day?' By Leah Nash

\href{https://www.nytimes.com/2020/07/02/opinion/sunday/income-inequality-solutions.html?action=click\&pgtype=Article\&state=default\&region=MAIN_CONTENT_2\&context=storylines_related_links}{}

\includegraphics{https://static01.nyt.com/images/2020/07/02/opinion/02solutionWeb/02solutionWeb-threeByTwoSmallAt2X-v2.jpg}

America Needs Some Repairs. Here's Where to Start. By The Editorial
Board

\href{https://www.nytimes.com/2020/07/02/opinion/private-equity-inequality.html?action=click\&pgtype=Article\&state=default\&region=MAIN_CONTENT_2\&context=storylines_related_links}{}

\includegraphics{https://static01.nyt.com/images/2020/07/08/opinion/02-inequalityB/02-inequalityB-threeByTwoSmallAt2X-v2.jpg}

The Neoliberal Looting of America By Mehrsa Baradaran

These were all officials and experts ---~people who might be biased
toward finding ``silver linings'' in any crisis. But was anything really
changing for homeless people around the Bay Area? I contacted several
homeless people who have been placed in hotels during the pandemic. They
spoke rapturously about their sudden fortune in an otherwise grim time.

``Oh my God --- I can really breathe and be myself.'' That was the
reaction from a 33-year-old woman who had been living in a hotel for
weeks with her 12-year-son. She asked me not to use her name. Before the
virus, they had spent years bouncing from couch to couch around the Bay.
Under lockdown, their lives were, in many ways, freer than before. For
the first time in years, she no longer felt that crushing dependence on
other people. ``I can move as the adult I am, and no one dictates what I
do or how I move,'' she told me.

The hotel room has two beds and a private bathroom. It was starting to
feel like a studio apartment ---~like a kind of home, she told me. ``I
only wish we could have a deep fryer.'' It is only guaranteed for three
months, but she has begun to see the possibility of a new life in the
uncertain distance: ``I just know that I am on my way to my place.''

As the weeks of lockdown dragged on, San Francisco began to break my
heart again. While the number of coronavirus cases and deaths remained
low, the full gloom of the coming recession began to descend into
view,~and with it, the same ageless, endless political squabbles. The
basic problem is that despite the region's apparently limitless wealth,
there were not enough ready resources available to public officials to
reach everyone in need. And in the absence of more help from the state
and the federal government, or from the region's billionaires, the Bay
Area's needs simply outmatched its capacity to meet them.

Even after the huge effort to move people into hotels, there are still
thousands of homeless people on the Bay Area's streets, and little
prospect that many will be housed anytime soon. My hopes for inspiring
leadership began to fall apart when a fight broke out recently between
San Francisco's Board of Supervisors and the mayor over how many more
homeless people the city could house.

The board passed an ordinance to secure 7,000 hotel rooms for homeless
people who are now on the street, but the mayor refused to comply. She
said it was impossible; the city is straining against its limit already.
So far, San Francisco has placed 965 homeless people in hotel rooms, and
\href{https://www.wsj.com/articles/san-francisco-leaders-clash-over-hotel-rooms-for-homeless-population-11588696567}{has
signed contracts for 2,731 rooms for homeless people and essential
workers.}

Image

At a new community-based testing site, Robert Cox helped residents of
East Oakland get free, drop-in coronavirus tests.Credit...Nicholas
Albrecht for The New York Times

Image

Tenesha Williams helps residents of East Oakland gets coronavirus tests
at the area's first walk-in testing site, organized by the city of
Oakland, Roots Community Health Center and Community Organized Relief
Effort, a nonprofit group.Credit...Nicholas Albrecht for The New York
Times

This fight hinges on the usual things --- money, willpower, staffing and
basic municipal capacity. But it also lays bare how ephemeral our
coronavirus-inspired unity may be. ``To the extent we have restored
faith in what is possible, we have also underscored, sadly, our city's
limitations,'' Matt Haney, a member of the Board of Supervisors, told
me.

When I asked the mayor about her dispute with the supervisors, she was
cordial but clearly annoyed. Annoyed that the supervisors hadn't
considered the limits on the city's capacity. Annoyed that she agreed
with them --- more homeless people could be taken off the streets if
only she had the funds or the people to make it happen.

The federal government has promised to reimburse cities for part of the
cost of housing the homeless, but Breed says she is not sure whether
those funds will come through. ``There's a huge difference between what
we all want, which is to get every homeless person off the street, and
reality,'' she said.

And instead of bringing the region's wealthy and its needy together, she
suggested that the pandemic might pit the less needy against the more
needy. ``I think many people are like, `Well, wait a minute --- I lost
my job where I was making minimum wage. I can't pay my rent. I can
barely eat. Where's \emph{my} help from the city?''' Breed said.

When I asked if the virus had created much political room for bold
action to address inequality, she said, ``It's going to make it even
harder.''

Is this really the best the city can do? The further we move from the
initial crisis, the crazier my bike-riding optimism now sounds. Rather
than fostering some new sense of civic unity, the virus is just as
likely to worsen inequality further.

Margot Kushel, a physician and scholar of homelessness at the University
of California, San Francisco, suggested that this was the ``nightmare
scenario'' for inequality in San Francisco: low-income jobs disappear,
so more people lose their homes, but because the tech industry keeps
doing well, home prices remain high, and housing slips further out of
reach for everyone else. ``Those who are housed are fully aware that
they're one thread away from losing that housing,'' Kushel said.

Image

A view of downtown San Francisco. Rather than fostering some new sense
of civic unity, the virus could worsen inequality.Credit...Nicholas
Albrecht for The New York Times

\href{https://www.kqed.org/news/11809099/a-guide-to-bay-area-eviction-moratoriums-during-the-coronavirus-crisis}{San
Francisco and other Bay Area cities have imposed temporary moratoriums}
on evictions caused by virus-related economic disruptions. But those
will expire later in the year, at which time a wave of tenants may be
kicked out of their homes unless they can pay months of back rent. At
the same time, the virus has given more political ammo to those NIMBYs
who have long opposed urban density and blocked the construction of more
housing.

All is not lost. I do feel a renewed sense of pride and possibility
about the Bay Area ---~the way our leaders responded to the virus did
strengthen my faith in our local institutions, and we certainly seem
better equipped to address long-term challenges than I once thought we
were.

There might still be a window for substantive action: Our local
governments can use the new leverage to push for bold ideas --- among
other policies, a plan for rent relief, rather than simply an eviction
moratorium, so that more people don't lose their housing.

I'm also waiting on the city's billionaires to open up new floodgates of
generosity, at least for mitigating the immediate pain of the crisis.
Jack Dorsey, the chief executive of Twitter and Square, recently pledged
\$1 billion to coronavirus relief; but of the
\href{https://www.forbes.com/billionaires/\#2c1c274c251c}{nearly 100
billionaires} reportedly living in the Bay Area,
\href{https://sf.curbed.com/2020/4/30/21241539/sf-billionaires-donations-coronavirus-dorsey-benioff}{only
a handful} have donated to the city's coronavirus relief fund. Mary Kate
Bacalao, the director of external affairs at Compass Family Services, a
nonprofit group that helps homeless families, told me that with a few
big checks, the Bay's wealthiest could instantly make a difference.

But I wouldn't be surprised if we ---~the people of the Bay Area, our
lawmakers, our billionaires and our ordinary, overburdened citizens ---
end up squandering this moment. Rebuilding a fairer, more livable urban
environment will take years of difficult work. It will require
sacrifices from the wealthy. It will require a renewed federal interest
in addressing the problems of cities. It will require abandoning
pie-in-the-sky techno-optimism.

This isn't a problem that will be solved by flying cars; it will be
solved by better zoning laws, fairer taxes and, when we can make it safe
again, more public transportation. We will have to commit ourselves to
these and other boring but permanent civic solutions.

I'm hopeful we're up to the task. We cannot go back to the way things
were. But as the immediate danger of the pandemic recedes, it will be
all too easy for many of us to do exactly that.

\hypertarget{bay-area-residents-what-do-you-think}{%
\subsection{Bay Area residents, what do you
think?}\label{bay-area-residents-what-do-you-think}}

\emph{The Times is committed to publishing}
\href{https://www.nytimes.com/2019/01/31/opinion/letters/letters-to-editor-new-york-times-women.html}{\emph{a
diversity of letters}} \emph{to the editor. We'd like to hear what you
think about this or any of our articles. Here are some}
\href{https://help.nytimes.com/hc/en-us/articles/115014925288-How-to-submit-a-letter-to-the-editor}{\emph{tips}}\emph{.
And here's our email:}
\href{mailto:letters@nytimes.com}{\emph{letters@nytimes.com}}\emph{.}

\emph{Follow The New York Times Opinion section on}
\href{https://www.facebook.com/nytopinion}{\emph{Facebook}}\emph{,}
\href{http://twitter.com/NYTOpinion}{\emph{Twitter (@NYTopinion)}}
\emph{and}
\href{https://www.instagram.com/nytopinion/}{\emph{Instagram}}\emph{.}

Spot Illustration by Giacomo Bagnara

Advertisement

\protect\hyperlink{after-bottom}{Continue reading the main story}

\hypertarget{site-index}{%
\subsection{Site Index}\label{site-index}}

\hypertarget{site-information-navigation}{%
\subsection{Site Information
Navigation}\label{site-information-navigation}}

\begin{itemize}
\tightlist
\item
  \href{https://help.nytimes.com/hc/en-us/articles/115014792127-Copyright-notice}{©~2020~The
  New York Times Company}
\end{itemize}

\begin{itemize}
\tightlist
\item
  \href{https://www.nytco.com/}{NYTCo}
\item
  \href{https://help.nytimes.com/hc/en-us/articles/115015385887-Contact-Us}{Contact
  Us}
\item
  \href{https://www.nytco.com/careers/}{Work with us}
\item
  \href{https://nytmediakit.com/}{Advertise}
\item
  \href{http://www.tbrandstudio.com/}{T Brand Studio}
\item
  \href{https://www.nytimes.com/privacy/cookie-policy\#how-do-i-manage-trackers}{Your
  Ad Choices}
\item
  \href{https://www.nytimes.com/privacy}{Privacy}
\item
  \href{https://help.nytimes.com/hc/en-us/articles/115014893428-Terms-of-service}{Terms
  of Service}
\item
  \href{https://help.nytimes.com/hc/en-us/articles/115014893968-Terms-of-sale}{Terms
  of Sale}
\item
  \href{https://spiderbites.nytimes.com}{Site Map}
\item
  \href{https://help.nytimes.com/hc/en-us}{Help}
\item
  \href{https://www.nytimes.com/subscription?campaignId=37WXW}{Subscriptions}
\end{itemize}
