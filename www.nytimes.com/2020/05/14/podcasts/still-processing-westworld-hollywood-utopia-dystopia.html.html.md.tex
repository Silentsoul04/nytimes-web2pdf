Sections

SEARCH

\protect\hyperlink{site-content}{Skip to
content}\protect\hyperlink{site-index}{Skip to site index}

\href{https://www.nytimes.com/spotlight/podcasts}{Podcasts}

\href{https://myaccount.nytimes.com/auth/login?response_type=cookie\&client_id=vi}{}

\href{https://www.nytimes.com/section/todayspaper}{Today's Paper}

\href{/spotlight/podcasts}{Podcasts}\textbar{}New Loop, America

\url{https://nyti.ms/2T87qUf}

\begin{itemize}
\item
\item
\item
\item
\item
\end{itemize}

Advertisement

\protect\hyperlink{after-top}{Continue reading the main story}

transcript

Back to Still Processing

bars

0:00/0:00

-0:00

transcript

\hypertarget{new-loop-america}{%
\subsection{New Loop, America}\label{new-loop-america}}

\hypertarget{hosted-by-wesley-morris-and-jenna-wortham-produced-by-hans-buetow-and-sydney-harper}{%
\subsubsection{Hosted by Wesley Morris and Jenna Wortham. Produced by
Hans Buetow and Sydney
Harper.}\label{hosted-by-wesley-morris-and-jenna-wortham-produced-by-hans-buetow-and-sydney-harper}}

\hypertarget{how-dystopian-and-utopian-shows-like-westworld-and-hollywood-can-help-us-map-out-a-better-future}{%
\paragraph{How dystopian and utopian shows like ``Westworld'' and
``Hollywood'' can help us map out a better
future.}\label{how-dystopian-and-utopian-shows-like-westworld-and-hollywood-can-help-us-map-out-a-better-future}}

Thursday, May 14th, 2020

\begin{itemize}
\item
  jenna wortham\\
  Where's my potato salad?
\item
  wesley morris\\
  Jenna, I'm still building my potato arsenal. {[}LAUGHTER{]} It's not
  going to be easy.
\item
  jenna wortham\\
  I was promised potato salad.
\item
  wesley morris\\
  And the promise still stands. It's not time. It's a Memorial Day
  potato salad. OK, let's go.
\item
  jenna wortham\\
  Ready? Three, two, one, record.

  I'm Jenna Wortham.
\item
  wesley morris\\
  I'm Wesley Morris. We're two culture writers scratching our heads at
  what the hell ``Westworld's'' about. We still don't know.
\item
  jenna wortham\\
  This is ``Still Processing.''
\item
  wesley morris\\
  Jenna you know I love you, and I'll do anything you ask me to do. And
  you really didn't ask me to do it. As a matter of fact, I believe you
  warned me not to do it. But I watched all of season three of
  ``Westworld.'' And I just want to know, as a ``Westworld'' lover
  yourself, what the hell is going on with you?
\item
  jenna wortham\\
  Listen, I think ``Westworld lover'' is a bit strong of a title, OK?
\item
  wesley morris\\
  I'm sorry. I did not mean to impugn you.
\item
  jenna wortham\\
  ``Westworld'' obsessive, ``Westworld'' curious, I mean, all of those
  apply.
\item
  wesley morris\\
  I'm ``Westworld'' curious. You're a ``Westworld'' sexual.
\item
  jenna wortham\\
  Yes, I'm a West sexual. But since the show came out in 2016, I have
  found it so intriguing.
\item
  archived recording\\
  Welcome to Westworld. Live without limits.
\end{itemize}

jenna wortham

The show is about a very fancy high tech amusement park for the ultra
wealthy, called Westworld. And at that amusement park, there are all
these very human-like androids, who live in this imperialist fantasy, I
guess. And visitors to the park get to engage with these robots in
whatever way they choose. So they can hunt them. They can enlist them
for quests. They can drink with them in the local watering hole, the
local saloon. They can have sex with them. I mean, they can beat them. I
mean, they can do whatever they want to them. And the show is basically
about the ethics of that.

\begin{itemize}
\item
  archived recording 1\\
  No, no, please don't hurt him. I'll do whatever you say! Agh!
\item
  archived recording 2\\
  I didn't pay all this money because I want it easy. I want you to
  fight.
\end{itemize}

jenna wortham

And a select handful of them start to become sentient, and they start to
acquire a type of agency. And then the subsequent two seasons of the
show are about what they do with that consciousness, and how they make
use of it.

\begin{itemize}
\item
  archived recording 1\\
  You can do as you wish with them. The goal is to escape.
\item
  archived recording 2\\
  Go where?
\item
  archived recording 3\\
  I want to see their world.
\end{itemize}

jenna wortham

In the most recent season, a handful of these robots, notably women and
two black people, are fighting for their freedom. And I think the
question of what does freedom look like is really fascinating to me. And
then beyond that, they're also asking this question of what type of
world should we be living in? What type of world do we want to be a part
of?

wesley morris

And we've now learned that one of these, the most renegade of all the
renegade robots, her name is Delores. And in the first season she was a
simple country girl. And then things just take a turn, and she winds up
being a rifle-toting momma, who figures out a way by season 2 to get out
of that world, destroys that, and then she's coming to this world, the
world you and I allegedly live in. I mean, I don't know what city it is.
Is it San Frangeles?

jenna wortham

New San Frangeles, actually. It's half New York and half ---

wesley morris

Ding.

jenna wortham

Yes.

wesley morris

Ding, dingity ding. She's coming to new San Frangeles to destroy it. But
in order to liberate humankind.

\begin{itemize}
\item
  archived recording 1\\
  You can stop pressing the alert button in your pocket. That won't
  work, either. Bit of a tactical mistake, really. You want to be the
  dominant species, but you built your whole world with things more like
  me.
\item
  archived recording 2\\
  You're not going to hurt me, are you?
\end{itemize}

jenna wortham

I love science fiction. I love the questions it asks us about ourselves.
I love the way it tries to get us to look at what we see as normal, and
poke at it, and try to freak it, so that we can try to understand it
better and understand ourselves better. It draws a link between what
you're seeing in the future to what's happening right now, and tries to
get you to examine the connection between the two of those things, so
that you can map out or chart out a different path for yourself into a
much better future than the one you're watching on screen. Because most
science fiction is a little dystopic, right? Most science fiction is
trying to be a parable, or trying to teach you a lesson. It's trying to
tell you something, or throw up a giant yellow warning sign, like
caution, yield, before you head too far down this path. And I think
``Westworld'' is trying to do that. I think it's also trying to do way
too much. It's got so many other things folded in there and all around
it.

wesley morris

I'll say.

jenna wortham

But it is essentially --- {[}LAUGHTER{]} But at its heart, it really is
a show about what you do when you wake up. Everyone has to sort of
figure out how awake they want to be, and what they're going to do about
it. And so it's just functioning on a lot of really interesting levels
for me, that I have a lot more fun thinking about than I do watching the
show. {[}LAUGHTER{]}

wesley morris

I'm having way more fun listening to you talk about it than I do
watching it.

\begin{itemize}
\tightlist
\item
  archived recording\\
  Our most skilled guests will fight their way to the outer limits of
  the park, besting fearsome braves, seducing nubile maidens,
  befriending tragically ill-fated sidekicks, and of course, like all
  our best narratives over the years, our guests will have the privilege
  of getting to know the character they're most interested in:
  themselves. {[}APPLAUSE{]}
\end{itemize}

jenna wortham

When you're watching ``Westworld'' and you're seeing the ultra rich pay
to abuse people that look like regular humans --- that look like their
relatives, look like their co-workers, look like their children,
presumably --- you know that that's from the outset wrong. That does not
feel OK. That does not feel good. That does not feel like an appropriate
way to work out your anger issues. There's something called therapy for
that. We shouldn't treat any other entity like that, no matter what we
think their intelligence levels are, no matter what race we think they
are, no matter what species we think they belong to. Nothing deserves
that kind of treatment. And we should watch a show like ``Westworld''
and think about what are the things in our life right now that are
immediately obviously not OK? Are we treating other people as though
they're subhuman? Are we treating other people as though their lives are
expendable? It's helpful for just illuminating what are the elements of
our lives that we should be more suspicious of, that we take for
granted, that we normalize in this moment?

wesley morris

Yeah, I mean, I feel like what you're describing is the uses of a
dystopia, the ways in which a dystopia can ultimately inform our moment,
which also seems loosely dystopic, if not altogether dystopic.

jenna wortham

Mmm.

wesley morris

So speaking of dystopias in science fiction television, let's play a
game, Jenna.

jenna wortham

OK.

wesley morris

I am a young, hungry, aspiring TV writer.

jenna wortham

Mm-hm.

wesley morris

And you are the big, bad TV executive.

jenna wortham

Mmm.

wesley morris

I want to pitch you on my dystopic eight-part four season series. It's
set in a time when everything is pretty normal.

jenna wortham

Hmm.

wesley morris

And then all of a sudden, just like that, everything changes. Almost
overnight, a virus takes over the whole world. That's the inciting
incident. And it's got a society that is basically kept apart six feet
at every step.

jenna wortham

Wow.

wesley morris

But not equally, because the black people and Latinx people are being
arrested at a 90 percent higher rate than anybody else. And they are
also the people doing the most egregious dying of this disease.

jenna wortham

Mm-hmm.

wesley morris

And let's just forget about America's native people, who are dying at
alarmingly high rates, too. And then there'll be something called like,
I don't know, an immunity passport where the people who've had the
disease will be able to freely move about the country, and get all the
jobs, and have all the opportunities. And people are going to be spying
on each other, and tracking each other on social media, and then calling
911 to tell the police that their neighbor, some person they see out the
window, I don't know, a guy at the supermarket is not wearing a mask.
Police are going to come and arrest that person, too. Except I don't
know that I'd really have to pitch this show, because we're living in
it!

jenna wortham

I was going to say, it's just called America 2020. I mean, it's
basically just called ``Dateline.'' That show exists, and it's on my TV
in about 20 minutes. No, it's really chilling when you put it that way,
because when we think about classic definitions of a dystopia, it's some
place in the future that's undesired, where things are unlivable. It
feels like a waking nightmare. People are ruled by fear, and there's a
lot of dehumanization. I mean, that's basically right now. We're living
it.

wesley morris

So if right now is a dystopia, and ``Westworld,'' as we've been
discussing, is a dystopia, I'm wondering what you think the use of a
utopia would be right now.

jenna wortham

Hmm. I mean, interestingly enough, they kind of serve the same purpose
as thinking through a dystopia. Historically in science fiction, utopias
are always held up to remind you that nothing is perfect, nothing can be
perfect. And my mind immediately goes to a film like ``The Stepford
Wives,'' where you have this picture perfect town of all these 1950s
very happy housewives, and then once Nicole Kidman's character starts
scratching a little bit below the surface, it's revealed that all these
women are actually robots. Or a film like ``Elysium,'' where all the
people on poverty stricken Earth are trying to get up to this incredible
haven, safe haven, heaven in the clouds, this really fancy city. But
nothing up there is as it seems, either. And so both films are trying to
let you know that, if it seems too good to be true, it probably is. Be
suspicious, proceed with caution.

wesley morris

I'm just going to say, I love that you chose the Nicole Kidman
``Stepford Wives.''

jenna wortham

The only one.

wesley morris

Oh, wow.

jenna wortham

In my opinion. I love her! I love Nicki.

wesley morris

OK, well, I love the `75 ``Stepford Wives.''

jenna wortham

What? The new one has Bette Midler in it, too. I mean, come on.

wesley morris

Listen, I'm not arguing. But I think there is a version of a utopia that
stays a utopia for pretty much the entire time. And even though
television is generically full of these things, I think there's a one
thing that we can watch on one of these platforms right now that is
entirely utopic.

jenna wortham

OK.

wesley morris

And that is Ryan Murphy's show about old Hollywood called ``Hollywood.''
{[}LAUGHTER{]} So we're going to take a break. And when we come back,
we're going to talk about how one little movie is going to rewrite the
entire history of racism in this country to make it better for
everybody.

{[}music{]}

wesley morris

``Hollywood'' is a seven part Netflix series that is focused on the
production of a motion picture in 1946 or `47. The movie's called
``Meg,'' and it is essentially written by, starring, directed by young
people. Not just any young people, young non-white people, some of whom
are gay. The star is an African-American woman named Camille Washington,
who's played by Laura Harrier. The director of the movie is played by
Darren Criss, and he's Filipino and white, and very proud to totally
embrace his Filipino ancestry, comes out of the closet and everything.
He doesn't have to hide anymore. And the whole movie has been written by
a gay black man who's played by Jeremy Pope. And Ace Pictures is what
the studio on the show is called. And initially it's run by Rob Reiner.
Perfect. He's kind of bigoted. He has a heart attack. He's in a coma.
And while he's in the hospital, his wife, Patti LuPone, takes over the
studio and decides she's going to put this movie ``Meg'' into production
even if it ruins the studio. Because you know what, y'all? It is the
right thing to do. It's the right thing to do!

\begin{itemize}
\item
  archived recording1\\
  OK, we're doing it.
\item
  archived recording 2\\
  Mrs. Amberg, before you go and take ---
\item
  archived recording 3\\
  You're fired. I don't know if I can fire you, but if I can, you're
  fired. We're making ``Meg.'' Now, everyone, go home before I change my
  mind. We'll put out a press release first thing in the morning. Tell
  Camille Washington she has the role.
\item
  archived recording 4\\
  Well, this is wrong, surely.
\item
  archived recording 5\\
  Avis, I'm very proud of you.
\item
  archived recording 6\\
  You're good.
\item
  archived recording 7\\
  No, I'm just fed up.
\end{itemize}

wesley morris

This show ``Hollywood'' takes real people and makes them characters,
too. There's Rock Hudson, one of our great movie stars, who on this show
is just content to be an aspiring househusband. Hattie McDaniel, the
first black woman to win an Oscar. And Anna May Wong, the
Chinese-American actress who, in real life, lost a lot of starring roles
because she wasn't white. And this movie would have been in production
during the reign of the Hays Code, which is basically commandments that
movies would have to follow in order to be deemed producible. Can I just
go through some of what they are?

jenna wortham

Yeah. Because this is a real thing. Because ``Hollywood'' blends fact
and fiction, but the Hays Code is a real thing. It was an actual thing
that existed in Hollywood, right?

wesley morris

Oh, it was real.

jenna wortham

OK, that's what I thought.

wesley morris

It was real. So here are some things in the Hays Code. You can't make
fun of the clergy. You can't have any sex hygiene or venereal disease of
any kind. Even though, parenthetically, Hollywood was rampant with a lot
of poor sex hygiene.

jenna wortham

Interesting.

wesley morris

You can't have what they describe as miscegenation, which is basically
any interracial sexual relationship. No white slavery. You could show
any other race being enslaved all day long.

jenna wortham

Clearly.

wesley morris

Just not white people.

jenna wortham

Clearly.

wesley morris

Not white people!

jenna wortham

Fascinating.

wesley morris

No, quote, ``sex perversion,'' which is basically just gay people. You
can imagine how a show like Hollywood would be positioning itself
against something like the Hays Code.

jenna wortham

Mm-hmm.

wesley morris

It wants to break as many of those laws as it possibly can, and in doing
so, functions to me as a work of utopia.

jenna wortham

It functions the way utopias function in science fiction. It just is not
intentional in the architecture of the show, which is to say that for me
watching it, I understand that this is supposed to be utopic. Even the
title sequence, when the show starts, there all these actors, and
cross-race, gender, ethnicity, creed, and they're all climbing the
Hollywood sign. And it seems like they're neck and neck at first, but
then they're checking to make sure the others are coming along as well.
And so there are these close-ups of hands reaching to help pull other
people. One person slips, this person grabs them around their waist. And
as commentary on intersectionality, I was here for it. I was like,
``Yes, we do all need to work together, folks.'' I love that part of it.
But in terms of what the show is actually about, these imagined
fantasies around what it would look like if, let's say, the head of a
studio said, ``I made the wrong call in the past around something as
offensive as yellow face. I want to do right by this colored actress.''
These are the terms they use in the show. These are not my terms. ``I
want to do right by these people. Let's do the right thing.'' I
understand why you might want to present that as a blueprint for how we
can do things better in the future, but you actually don't have to
transport yourself back to the late 1940s and the early 1950s to make
that show. You can actually make a version of that, that exists right
now in Hollywood.

wesley morris

You could just set it at CBS!

jenna wortham

You could literally just set it at CBS! You could literally just set it
at Netflix! And so it's this kind of a fantasy of this nostalgic
optimism that if we could go back and do things the right way, we would.
But hello? Heller? In 2020, there aren't a lot of good parts for black
and Asian actors right now. Am I wrong? I don't know. You don't have to
go back to old Hollywood for that. And it's true we've had movies like
``Love Jones,'' obviously ``Love \& Basketball.'' We've had a lot of
really incredible films cast with non-white leads. But to me, it feels
like it's not enough.

wesley morris

But that is the whole point of this show, which is to make it so that if
you start at pretty much in the golden age of Hollywood, you don't have
to deal with the frustration of ``Love \& Basketball'' and ``Love
Jones'' being these atypical national black person event type movies.
And that is the goal of this show. And we should say that it comes from
Ryan Murphy. ``Hollywood'' is a Ryan Murphy production. He made the show
with this guy named Ian Brenner. And it's pretty much of a piece with
this version of Ryan Murphy, this man who has committed to these ideas
of so-called diversity and inclusion, and what it means to hire non cis
white men for jobs, essentially. And he is doing that work. He's given
us the Betty and Joan season of ``Feud'' and --- you know one of his
anthology shows on FX. He's made ``The People v. O.J. Simpson,'' which
is great. And ``Feud'' is also great.

jenna wortham

And ``Pose!'' Can't forget about ``Pose.''

wesley morris

And ``Pose.'' Who could forget? Who could forget ``Pose?'' A world made
up entirely of queer people and mostly trans people.

jenna wortham

Beautiful black people, and Latinx people.

wesley morris

And they're mostly black and Latinx, yes. So I mean, he's busy. He's got
a lot of hands in a lot of pots, and a lot of them involve bringing
people who don't get brought to things to the television screens of
people around the world. It is a worthy project, and this show is very
much of a piece with it.

\begin{itemize}
\item
  archived recording 1\\
  I got to say Mr. Samuels, sometimes I think folks in this town don't
  really understand the power they have. Movies don't just show us how
  the world is. They show us how the world can be. And if we change the
  way that movies are made, you take a chance and you make a different
  kind of story, I think you can change the world.
\item
  archived recording 2\\
  Here are some scripts that I'm --- that Mr. Amberg is excited about.
  Read them. If you find something you like, come back and let's talk.
\item
  archived recording 3\\
  OK, all of them?
\item
  archived recording 4\\
  All of them. {[}BELL RINGS{]}
\end{itemize}

jenna wortham

At some point, the film is released. It's a huge hit. It's a wide
release, so it's in the black theaters, the white theaters, all the
theaters. They've dropped the price, so anybody can go. And the film is
nominated for a whole host of Oscars. And while we're at the Oscars and
each character's category is being called, they cut to a shot of people
at home listening along. So when it's Anna May Wong's turn, it's like
``And the nominees for Best Supporting Actress are --- '' and they call
her name. They cut to a family of Chinese people crowded around a radio.
And in that moment, I was reminded, the Oscars do matter. These awards
do matter. Representation is so important. Visibility is so important.
And then a few seconds later, when Camille's category is called ---

wesley morris

Oh, lord.

jenna wortham

--- they cut to a black family in like a shack. I don't know. I was
like, what year is this? Did we go back in time? Is this the 1880s? I
don't know. I am still traumatized by that scene. Because I actually was
thinking a lot about the most recent Oscars. And I was thinking about
``Parasite.'' And I was thinking about that win, and how good it felt,
and it felt like a recognition. It felt like ---

wesley morris

Of course, yeah.

jenna wortham

--- we're acknowledging this South Korean film and this filmmaker. And
it did remind me of the ways in which these major international
acknowledgments, they do feel like advances. They do feel like progress,
and I don't want to undermine that at all. ``Hollywood'' itself, as this
Netflix show, didn't feel like it was trying to do the thing that the
show was about. It felt like it was patting itself on the back for
wanting to do those things, right?

\begin{itemize}
\tightlist
\item
  archived recording\\
  To be standing here as the first actress of Chinese descent to win
  this award, not for putting on yellow face and playing an oriental
  caricature, but for playing a woman. A complex woman with a heart and
  a soul. Thank you. {[}APPLAUSE{]}
\end{itemize}

wesley morris

So the thing about this show to me that feels especially fantastical in
terms of how utopic it is, is that it's sort of a luxury for white
people to be able to go back in the past and do whatever they want,
because they were fine back then. {[}LAUGHTER{]} When black people go
back into the past, it is to essentially operate around certain
histories that were essentially designed to oppress, and subjugate, and
essentially annihilate them. And the idea of going back there and
creating a utopia just seems to be a rather --- I mean, if not
distasteful thing to do, then slightly naive. And Ryan Murphy has this
luxury of revisiting a past in a way that a black person, or if you're
telling Anna May Wong's story, a Chinese-American person just wouldn't
be able to.

jenna wortham

Listen, for a show that wants me to take these tweaks seriously, and I
love Queen Latifah as much as the next person, but you mean to tell me
they couldn't find a single dark skinned actor to play Hattie McDaniel?
Like I get it, you wanted a big name actress, great. But like, Gabby
Sidibe wasn't available? That's a real sensitive part of history. And
she was the daughter of slaves, she wasn't allowed in a theater, and
you're going to cast a light-skinned actress to play her? That's
painful. Don't rewrite our history like that. You want to rewrite the
past so we can rethink the present? Then do right by the past. Because
that does not give me the confidence that you really take this
seriously.

wesley morris

I hear you and --- Hattie McDaniel, of course, won the academy award for
Best Supporting Actress for 1939's ``Gone With the Wind.'' She played
the character, the unforgettable iconic character Mammy. And the thing
about this show with respect to Hattie McDaniel and people like her, is
that it is full of all of these interesting real people, who lived
really interesting Hollywood lives. And why ``Hollywood'' isn't just an
anthology show about these great entertainers? What if they had just
made a movie about Oscar Micheaux? He's the first black filmmaker of any
significance in this country. I mean, he is somebody that you could have
had a lot of fun with taking from the margins. He was essentially the
Tyler Perry of the 1920s. What if Warner Brothers hired him to make a
major motion picture, and starring Hattie McDaniel? I just feel like the
show has an interesting imagination when it comes to what it wants to do
with the Hollywood equivalent of the Marvel universe, right? Which of
your favorite characters does it want to get together to do work
together?

jenna wortham

My gosh. This show, to me, starts to feel a little bit like --- it's a
liberation fantasy. It's like what would it have looked like back then
if white people were the saviors of all other people? And that fantasy
is not helpful for thinking about then. It's not helpful for thinking
about right now. And there's a moment in the Oscar scene of the seventh
episode of the show ``Hollywood,'' when Avis Amberg, who's running Ace
Studios, is watching ---

wesley morris

Patti LuPone's character.

jenna wortham

Patti LuPone's character --- is watching ``Meg.'' At first, it seems
like ``Meg'' is not going to win any awards, and she has this very
dramatic back of the hand to the forehead moment. ``Oh, it's because I'm
a woman.'' Is it because you're a woman? Or is it because they don't
want to reward this show about an interracial couple with a Chinese
woman it. I don't know if it's about you, white lady. I don't know if
it's about you right now in this moment. And them writing that in with
an ounce of sincerity, and I just felt like, this is the problem with
the show. It thinks it's all about whoever made the show, and not
actually about the lives and the trials and tribulations of these people
of color.

wesley morris

When you say show, do you mean ``Hollywood,'' or do you mean the people
who made ``Meg?''

jenna wortham

It's kind of both.

wesley morris

OK.

jenna wortham

It's kind of one and the same, to be honest with you. It's really nice
to imagine that after World War II, in this period of rebuilding and
regeneration, that something like the social and psychological legacy of
slavery could be easily wiped away by people in a boardroom saying,
``Let's make a movie. Let's make this movie starring a colored woman.''
Let's think about what was going on during that period of time. Jim
Crow, deep segregation is still active. And it's not just something
that's like colored water fountain and a white water fountain. No, this
is a system of segregation that is set up to demoralize one group of
people and maintain the hegemony of another group of people. That's
something that's very difficult to undo. It's not something like ironing
out wrinkles in a shirt. It's not a switch that you can flip. And the
government at this period of time has decided, you know, All
Japanese-Americans, you're just going to be hidden away and be in these
internment camps. I mean, there is a real practice, active practice of
racial terror that's happening across many different groups of people.
And that's not something that's just undone with a magic wand. I
understand the fantasy, and I understand the agenda. And again, I do
think it is a noble experiment, but it is also a little bit insulting to
just assume that at this moment in time, people would be ready for a
Chinese-American movie star. They would be ready for a beautiful black
woman to have all the opportunities that white women have. To kind of
allow that type of magical thinking to exist in this racial fantasy I
think is really harmful. And it's frustrating, and it's a free pass. And
it was hurtful. I actually was like --- I felt pained.

wesley morris

Yeah, no. I mean, it's reducing racism to something that can basically
be solved by one movie, that's presented as the solution to all of
America's racial problems. There is a moment in episode seven that just
cracked me up.

jenna wortham

Which one?

wesley morris

The movie opens. It's a huge hit. And three of the men involved with the
film go to a screening and watch. They sit in the balcony, and there's
newsreel footage of the premiere of the movie and the hoopla around it.

jenna wortham

Right, right.

wesley morris

And at some point, the announcer says, ``Racial protests around the
country simply melted away.''

\begin{itemize}
\item
  archived recording 1\\
  --- simply melted away, as thousands rushed out to see a new kind of
  motion picture, and moviegoers of every color have fallen in love.
\item
  archived recording 2\\
  Every woman I know who saw it cried. And we understood how she was
  feeling.
\item
  archived recording 3\\
  Could one movie change the way a nation sees itself? Who knows? But
  one thing's for sure, America's mad for ---
\item
  archived recording 4\\
  ``Meg!'' {[}LAUGHTER{]}
\end{itemize}

jenna wortham

Honestly, Wesley, I'm glad one of us is laughing right now, because I
had the exact opposite somatic response in my body. I was watching that
episode in the bathtub, and I jerked so much that a tidal wave of my
rose-scented magnesium salt-infused water splashed onto my floor, and I
had to scramble to find a towel to mop it up. But I ---

wesley morris

Oh my god.

jenna wortham

I mean, I'm being really serious. I cried. I was so hurt. That is a
really beautiful sentiment that feels so negligent, and so --- that is
in direct opposition to even what we're seeing right now. You have
anti-quarantine protests, which, protest wanting to go back to work.
Yeah, make yourself known. That is a thing you should be able to do. But
the white people in this country can't even show up in protest without
bringing Nazi insignia, and Confederate flags, and assault rifles. Those
are emblems of hate. I don't want to overestimate the power of popular
culture to shift societal expectations and norms. It is hugely
influential, hugely instrumental. But we also can't overstate its
ability. There has to be some balance ---

wesley morris

Yes, yes, I hear you.

jenna wortham

--- and part of the fantasy of this whole show is that Hollywood has a
lot more power than it thinks it does. It's really easy to look back and
to cast yourself backwards in time and say, ``If I had been around
during this particularly fraught era in history, I would have helped
slaves escape. I would have hid people in my home if I'd been in Nazi
Germany.'' All of these fantasies are easy to say, and this show extends
that. It frustrates me because I think that it allows for a type of
escapism that isn't useful right now. We are living in a state of
emergency.

wesley morris

That's why I find the distrust of humanity on a show like ``Westworld''
to be slightly more fascinating. Even though I don't quite know what the
hell is going on with that show, I think the use of its dystopic
elements are actually really imagining a better future in a way that
requires, as at some point Evan Rachel Wood's character Delores says,
that Free will exists, it's just {[}EXPLETIVE{]} hard. And I just find
the hard way more interesting than the easy way. In ``Hollywood'' the TV
show, it kind of makes it too easy. And we're in a moment in which we're
having utopias pushed on us all the time. Every time we turn on a
television show, there is the risk of being bombarded with utopia
propaganda, essentially.

jenna wortham

Mmm.

wesley morris

And there are all these commercials that basically insist that we're in
this together. ``We're all gonna make it!''

jenna wortham

``Everything's gonna be just fine. Just order our products, and you'll
live. You'll be OK.'' It's like that is the most dystopic thing, but
it's a dystopia masquerading as a utopia.

wesley morris

There are ``we're all in this together'' ads for John Deere. There are
``we're all in this together'' ads for places like Walmart.

jenna wortham

Yeah, and I've seen them for No Touch toilets, and I've seen them for
Budweiser. But I really want to talk about this Papa John's commercial,
because it is the most dystopic thing masquerading as a utopic thing.
And it opens with a young man by the name of T.J. Lam, who's a store
manager in Louisville, Ky., professing his deep devotion to delivering
pizzas to all the white families at home that can't be bothered to cook
for their kids anymore.

\begin{itemize}
\tightlist
\item
  archived recording\\
  I know for every pizza I deliver, that's a trip that family doesn't
  have to take out of their house. It relieves some stress off of me to
  let me know that I'm doing something good for the community, not just
  Papa John's.
\end{itemize}

jenna wortham

The utopic notion in this commercial is that the people who are forced
to keep working during this time want to be. They want to be in service
of those who can afford to stay home in their enormous mini mansions, by
the way. And that is the fantasy. That is so dystopic! I can imagine him
saying, ``I'm happily putting myself at risk every day. I may or may not
have a family of dependents, grandparents, immunocompromised folks that
I am putting at risk so I can do my job for you, Papa John's.'' I mean,
Jesus.

wesley morris

I mean, we can see he's not wearing any P.P.E., by the way. They don't
get that --- Papa John's not handing out the P.P.E.

jenna wortham

Why are we acting like that's an OK thing, that he shouldn't have a mask
on, or they shouldn't have a mask on?

wesley morris

Right. Part of the dystopic feeling of a commercial like that is that
you're being lied to. This is a fable about our moment. And I also
wonder if one of the crazy things about this moment is we are really
thinking about the ways in which we maybe thought we had been living in
a utopia, and have now been forced to reckon with the fact that we've
been in a dystopia the entire time. And the thing that has really
brought that home to me is how Ahmaud Arbery died. And the thing that
people keep saying is unimaginable is that this is how people used to
die. This is a death from the 1920s. This is an old death. The idea that
he was running for his --- literally running for his life from white men
in a pickup truck? What time? What wormhole did that truck drive out of?
And the truth is, no, there's no wormhole!

jenna wortham

Yeah, that's it. There's no movie. There's no robot. There's no
algorithm that changes that. And this has been going around via Twitter.
I'm just not claiming credit for this idea. It's been repeated a number
of times, different ways, but the police didn't arrest those men because
there was a video. They saw the video. They only had to take action ---

wesley morris

We saw the video.

jenna wortham

--- because we saw the video. And so when we think about like --- that's
the most dystopic thing ever. And I think that's why a show like
``Westworld'' titillates me, because what do you do when you realize the
levels of oppression that you're living under? And that's why a show
like ``Hollywood'' frustrates me, because there is no version of history
in which an Oscar-winning film fixes all of the problems that this
country has with structural oppression and racism. And you know how I
know the answer to that? Because we actually saw that when Halle Berry
won the award for ``Monster's Ball.'' And guess what? It didn't fix all
these problems, because some 20 years later, Ahmaud Arbery's still dead.
Nina Pop is still dead. Breonna Taylor's still dead. And those are just
the names that were turned into hashtags. So please don't go back to the
past and tell me, ``Hey, had `Meg' won all these awards, this reality
might be different.'' Because variations of that have happened. That's
the loop. And it's up to us to break it. It's up to us to wake up from
it. I just don't know if people want to.

wesley morris

That's our show. ``Still Processing'' is a product of The New York
Times. It was recorded, once again, and for the last time, in our living
rooms.

jenna wortham

It is produced by Hans Buetow and Sydney Harper.

wesley morris

Our editors are Sarah Sarasohn and Sasha Weiss, Wendy Dorr, and Lisa
Tobin.

jenna wortham

Our engineer is Jake Gorski.

wesley morris

Our theme music is by Kindness. It's called ``World Restart'' from the
album ``Otherness.''

jenna wortham

You can find all of our episodes and various fun things at
nytimes.com/stillprocessing.

wesley morris

Thanks for listening, you guys. Good luck.

jenna wortham

Stay safe, and stay healthy. And we'll be back soon.

wesley morris

We'll miss you.

jenna wortham

We will miss you.

wesley morris

We'll miss you.

jenna wortham

It sounds like you're a ghost.

wesley morris

We'll miss you. {[}LAUGHTER{]}

\href{https://www.nytimes.com/column/still-processing-podcast}{\includegraphics{https://static01.nyt.com/images/2019/09/15/podcasts/still-processing-album-art-2/still-processing-album-art-2-square320.jpg}Still
Processing}Subscribe:

\begin{itemize}
\tightlist
\item
  \href{https://itunes.apple.com/us/podcast/id1151436460}{Apple
  Podcasts}
\item
  \href{https://www.google.com/podcasts?feed=aHR0cHM6Ly9yc3MuYXJ0MTkuY29tL255dC1zdGlsbC1wcm9jZXNzaW5n}{Google
  Podcasts}
\end{itemize}

\hypertarget{new-loop-america-1}{%
\section{New Loop, America}\label{new-loop-america-1}}

\hypertarget{how-dystopian-and-utopian-shows-like-westworld-and-hollywood-can-help-us-map-out-a-better-future-1}{%
\subsection{How dystopian and utopian shows like ``Westworld'' and
``Hollywood'' can help us map out a better
future.}\label{how-dystopian-and-utopian-shows-like-westworld-and-hollywood-can-help-us-map-out-a-better-future-1}}

Hosted by Wesley Morris and Jenna Wortham. Produced by Hans Buetow and
Sydney Harper.

Transcript

transcript

Back to Still Processing

bars

0:00/0:00

-0:00

transcript

\hypertarget{new-loop-america-2}{%
\subsection{New Loop, America}\label{new-loop-america-2}}

\hypertarget{hosted-by-wesley-morris-and-jenna-wortham-produced-by-hans-buetow-and-sydney-harper-1}{%
\subsubsection{Hosted by Wesley Morris and Jenna Wortham. Produced by
Hans Buetow and Sydney
Harper.}\label{hosted-by-wesley-morris-and-jenna-wortham-produced-by-hans-buetow-and-sydney-harper-1}}

\hypertarget{how-dystopian-and-utopian-shows-like-westworld-and-hollywood-can-help-us-map-out-a-better-future-2}{%
\paragraph{How dystopian and utopian shows like ``Westworld'' and
``Hollywood'' can help us map out a better
future.}\label{how-dystopian-and-utopian-shows-like-westworld-and-hollywood-can-help-us-map-out-a-better-future-2}}

Thursday, May 14th, 2020

\begin{itemize}
\item
  jenna wortham\\
  Where's my potato salad?
\item
  wesley morris\\
  Jenna, I'm still building my potato arsenal. {[}LAUGHTER{]} It's not
  going to be easy.
\item
  jenna wortham\\
  I was promised potato salad.
\item
  wesley morris\\
  And the promise still stands. It's not time. It's a Memorial Day
  potato salad. OK, let's go.
\item
  jenna wortham\\
  Ready? Three, two, one, record.

  I'm Jenna Wortham.
\item
  wesley morris\\
  I'm Wesley Morris. We're two culture writers scratching our heads at
  what the hell ``Westworld's'' about. We still don't know.
\item
  jenna wortham\\
  This is ``Still Processing.''
\item
  wesley morris\\
  Jenna you know I love you, and I'll do anything you ask me to do. And
  you really didn't ask me to do it. As a matter of fact, I believe you
  warned me not to do it. But I watched all of season three of
  ``Westworld.'' And I just want to know, as a ``Westworld'' lover
  yourself, what the hell is going on with you?
\item
  jenna wortham\\
  Listen, I think ``Westworld lover'' is a bit strong of a title, OK?
\item
  wesley morris\\
  I'm sorry. I did not mean to impugn you.
\item
  jenna wortham\\
  ``Westworld'' obsessive, ``Westworld'' curious, I mean, all of those
  apply.
\item
  wesley morris\\
  I'm ``Westworld'' curious. You're a ``Westworld'' sexual.
\item
  jenna wortham\\
  Yes, I'm a West sexual. But since the show came out in 2016, I have
  found it so intriguing.
\item
  archived recording\\
  Welcome to Westworld. Live without limits.
\end{itemize}

jenna wortham

The show is about a very fancy high tech amusement park for the ultra
wealthy, called Westworld. And at that amusement park, there are all
these very human-like androids, who live in this imperialist fantasy, I
guess. And visitors to the park get to engage with these robots in
whatever way they choose. So they can hunt them. They can enlist them
for quests. They can drink with them in the local watering hole, the
local saloon. They can have sex with them. I mean, they can beat them. I
mean, they can do whatever they want to them. And the show is basically
about the ethics of that.

\begin{itemize}
\item
  archived recording 1\\
  No, no, please don't hurt him. I'll do whatever you say! Agh!
\item
  archived recording 2\\
  I didn't pay all this money because I want it easy. I want you to
  fight.
\end{itemize}

jenna wortham

And a select handful of them start to become sentient, and they start to
acquire a type of agency. And then the subsequent two seasons of the
show are about what they do with that consciousness, and how they make
use of it.

\begin{itemize}
\item
  archived recording 1\\
  You can do as you wish with them. The goal is to escape.
\item
  archived recording 2\\
  Go where?
\item
  archived recording 3\\
  I want to see their world.
\end{itemize}

jenna wortham

In the most recent season, a handful of these robots, notably women and
two black people, are fighting for their freedom. And I think the
question of what does freedom look like is really fascinating to me. And
then beyond that, they're also asking this question of what type of
world should we be living in? What type of world do we want to be a part
of?

wesley morris

And we've now learned that one of these, the most renegade of all the
renegade robots, her name is Delores. And in the first season she was a
simple country girl. And then things just take a turn, and she winds up
being a rifle-toting momma, who figures out a way by season 2 to get out
of that world, destroys that, and then she's coming to this world, the
world you and I allegedly live in. I mean, I don't know what city it is.
Is it San Frangeles?

jenna wortham

New San Frangeles, actually. It's half New York and half ---

wesley morris

Ding.

jenna wortham

Yes.

wesley morris

Ding, dingity ding. She's coming to new San Frangeles to destroy it. But
in order to liberate humankind.

\begin{itemize}
\item
  archived recording 1\\
  You can stop pressing the alert button in your pocket. That won't
  work, either. Bit of a tactical mistake, really. You want to be the
  dominant species, but you built your whole world with things more like
  me.
\item
  archived recording 2\\
  You're not going to hurt me, are you?
\end{itemize}

jenna wortham

I love science fiction. I love the questions it asks us about ourselves.
I love the way it tries to get us to look at what we see as normal, and
poke at it, and try to freak it, so that we can try to understand it
better and understand ourselves better. It draws a link between what
you're seeing in the future to what's happening right now, and tries to
get you to examine the connection between the two of those things, so
that you can map out or chart out a different path for yourself into a
much better future than the one you're watching on screen. Because most
science fiction is a little dystopic, right? Most science fiction is
trying to be a parable, or trying to teach you a lesson. It's trying to
tell you something, or throw up a giant yellow warning sign, like
caution, yield, before you head too far down this path. And I think
``Westworld'' is trying to do that. I think it's also trying to do way
too much. It's got so many other things folded in there and all around
it.

wesley morris

I'll say.

jenna wortham

But it is essentially --- {[}LAUGHTER{]} But at its heart, it really is
a show about what you do when you wake up. Everyone has to sort of
figure out how awake they want to be, and what they're going to do about
it. And so it's just functioning on a lot of really interesting levels
for me, that I have a lot more fun thinking about than I do watching the
show. {[}LAUGHTER{]}

wesley morris

I'm having way more fun listening to you talk about it than I do
watching it.

\begin{itemize}
\tightlist
\item
  archived recording\\
  Our most skilled guests will fight their way to the outer limits of
  the park, besting fearsome braves, seducing nubile maidens,
  befriending tragically ill-fated sidekicks, and of course, like all
  our best narratives over the years, our guests will have the privilege
  of getting to know the character they're most interested in:
  themselves. {[}APPLAUSE{]}
\end{itemize}

jenna wortham

When you're watching ``Westworld'' and you're seeing the ultra rich pay
to abuse people that look like regular humans --- that look like their
relatives, look like their co-workers, look like their children,
presumably --- you know that that's from the outset wrong. That does not
feel OK. That does not feel good. That does not feel like an appropriate
way to work out your anger issues. There's something called therapy for
that. We shouldn't treat any other entity like that, no matter what we
think their intelligence levels are, no matter what race we think they
are, no matter what species we think they belong to. Nothing deserves
that kind of treatment. And we should watch a show like ``Westworld''
and think about what are the things in our life right now that are
immediately obviously not OK? Are we treating other people as though
they're subhuman? Are we treating other people as though their lives are
expendable? It's helpful for just illuminating what are the elements of
our lives that we should be more suspicious of, that we take for
granted, that we normalize in this moment?

wesley morris

Yeah, I mean, I feel like what you're describing is the uses of a
dystopia, the ways in which a dystopia can ultimately inform our moment,
which also seems loosely dystopic, if not altogether dystopic.

jenna wortham

Mmm.

wesley morris

So speaking of dystopias in science fiction television, let's play a
game, Jenna.

jenna wortham

OK.

wesley morris

I am a young, hungry, aspiring TV writer.

jenna wortham

Mm-hm.

wesley morris

And you are the big, bad TV executive.

jenna wortham

Mmm.

wesley morris

I want to pitch you on my dystopic eight-part four season series. It's
set in a time when everything is pretty normal.

jenna wortham

Hmm.

wesley morris

And then all of a sudden, just like that, everything changes. Almost
overnight, a virus takes over the whole world. That's the inciting
incident. And it's got a society that is basically kept apart six feet
at every step.

jenna wortham

Wow.

wesley morris

But not equally, because the black people and Latinx people are being
arrested at a 90 percent higher rate than anybody else. And they are
also the people doing the most egregious dying of this disease.

jenna wortham

Mm-hmm.

wesley morris

And let's just forget about America's native people, who are dying at
alarmingly high rates, too. And then there'll be something called like,
I don't know, an immunity passport where the people who've had the
disease will be able to freely move about the country, and get all the
jobs, and have all the opportunities. And people are going to be spying
on each other, and tracking each other on social media, and then calling
911 to tell the police that their neighbor, some person they see out the
window, I don't know, a guy at the supermarket is not wearing a mask.
Police are going to come and arrest that person, too. Except I don't
know that I'd really have to pitch this show, because we're living in
it!

jenna wortham

I was going to say, it's just called America 2020. I mean, it's
basically just called ``Dateline.'' That show exists, and it's on my TV
in about 20 minutes. No, it's really chilling when you put it that way,
because when we think about classic definitions of a dystopia, it's some
place in the future that's undesired, where things are unlivable. It
feels like a waking nightmare. People are ruled by fear, and there's a
lot of dehumanization. I mean, that's basically right now. We're living
it.

wesley morris

So if right now is a dystopia, and ``Westworld,'' as we've been
discussing, is a dystopia, I'm wondering what you think the use of a
utopia would be right now.

jenna wortham

Hmm. I mean, interestingly enough, they kind of serve the same purpose
as thinking through a dystopia. Historically in science fiction, utopias
are always held up to remind you that nothing is perfect, nothing can be
perfect. And my mind immediately goes to a film like ``The Stepford
Wives,'' where you have this picture perfect town of all these 1950s
very happy housewives, and then once Nicole Kidman's character starts
scratching a little bit below the surface, it's revealed that all these
women are actually robots. Or a film like ``Elysium,'' where all the
people on poverty stricken Earth are trying to get up to this incredible
haven, safe haven, heaven in the clouds, this really fancy city. But
nothing up there is as it seems, either. And so both films are trying to
let you know that, if it seems too good to be true, it probably is. Be
suspicious, proceed with caution.

wesley morris

I'm just going to say, I love that you chose the Nicole Kidman
``Stepford Wives.''

jenna wortham

The only one.

wesley morris

Oh, wow.

jenna wortham

In my opinion. I love her! I love Nicki.

wesley morris

OK, well, I love the `75 ``Stepford Wives.''

jenna wortham

What? The new one has Bette Midler in it, too. I mean, come on.

wesley morris

Listen, I'm not arguing. But I think there is a version of a utopia that
stays a utopia for pretty much the entire time. And even though
television is generically full of these things, I think there's a one
thing that we can watch on one of these platforms right now that is
entirely utopic.

jenna wortham

OK.

wesley morris

And that is Ryan Murphy's show about old Hollywood called ``Hollywood.''
{[}LAUGHTER{]} So we're going to take a break. And when we come back,
we're going to talk about how one little movie is going to rewrite the
entire history of racism in this country to make it better for
everybody.

{[}music{]}

wesley morris

``Hollywood'' is a seven part Netflix series that is focused on the
production of a motion picture in 1946 or `47. The movie's called
``Meg,'' and it is essentially written by, starring, directed by young
people. Not just any young people, young non-white people, some of whom
are gay. The star is an African-American woman named Camille Washington,
who's played by Laura Harrier. The director of the movie is played by
Darren Criss, and he's Filipino and white, and very proud to totally
embrace his Filipino ancestry, comes out of the closet and everything.
He doesn't have to hide anymore. And the whole movie has been written by
a gay black man who's played by Jeremy Pope. And Ace Pictures is what
the studio on the show is called. And initially it's run by Rob Reiner.
Perfect. He's kind of bigoted. He has a heart attack. He's in a coma.
And while he's in the hospital, his wife, Patti LuPone, takes over the
studio and decides she's going to put this movie ``Meg'' into production
even if it ruins the studio. Because you know what, y'all? It is the
right thing to do. It's the right thing to do!

\begin{itemize}
\item
  archived recording1\\
  OK, we're doing it.
\item
  archived recording 2\\
  Mrs. Amberg, before you go and take ---
\item
  archived recording 3\\
  You're fired. I don't know if I can fire you, but if I can, you're
  fired. We're making ``Meg.'' Now, everyone, go home before I change my
  mind. We'll put out a press release first thing in the morning. Tell
  Camille Washington she has the role.
\item
  archived recording 4\\
  Well, this is wrong, surely.
\item
  archived recording 5\\
  Avis, I'm very proud of you.
\item
  archived recording 6\\
  You're good.
\item
  archived recording 7\\
  No, I'm just fed up.
\end{itemize}

wesley morris

This show ``Hollywood'' takes real people and makes them characters,
too. There's Rock Hudson, one of our great movie stars, who on this show
is just content to be an aspiring househusband. Hattie McDaniel, the
first black woman to win an Oscar. And Anna May Wong, the
Chinese-American actress who, in real life, lost a lot of starring roles
because she wasn't white. And this movie would have been in production
during the reign of the Hays Code, which is basically commandments that
movies would have to follow in order to be deemed producible. Can I just
go through some of what they are?

jenna wortham

Yeah. Because this is a real thing. Because ``Hollywood'' blends fact
and fiction, but the Hays Code is a real thing. It was an actual thing
that existed in Hollywood, right?

wesley morris

Oh, it was real.

jenna wortham

OK, that's what I thought.

wesley morris

It was real. So here are some things in the Hays Code. You can't make
fun of the clergy. You can't have any sex hygiene or venereal disease of
any kind. Even though, parenthetically, Hollywood was rampant with a lot
of poor sex hygiene.

jenna wortham

Interesting.

wesley morris

You can't have what they describe as miscegenation, which is basically
any interracial sexual relationship. No white slavery. You could show
any other race being enslaved all day long.

jenna wortham

Clearly.

wesley morris

Just not white people.

jenna wortham

Clearly.

wesley morris

Not white people!

jenna wortham

Fascinating.

wesley morris

No, quote, ``sex perversion,'' which is basically just gay people. You
can imagine how a show like Hollywood would be positioning itself
against something like the Hays Code.

jenna wortham

Mm-hmm.

wesley morris

It wants to break as many of those laws as it possibly can, and in doing
so, functions to me as a work of utopia.

jenna wortham

It functions the way utopias function in science fiction. It just is not
intentional in the architecture of the show, which is to say that for me
watching it, I understand that this is supposed to be utopic. Even the
title sequence, when the show starts, there all these actors, and
cross-race, gender, ethnicity, creed, and they're all climbing the
Hollywood sign. And it seems like they're neck and neck at first, but
then they're checking to make sure the others are coming along as well.
And so there are these close-ups of hands reaching to help pull other
people. One person slips, this person grabs them around their waist. And
as commentary on intersectionality, I was here for it. I was like,
``Yes, we do all need to work together, folks.'' I love that part of it.
But in terms of what the show is actually about, these imagined
fantasies around what it would look like if, let's say, the head of a
studio said, ``I made the wrong call in the past around something as
offensive as yellow face. I want to do right by this colored actress.''
These are the terms they use in the show. These are not my terms. ``I
want to do right by these people. Let's do the right thing.'' I
understand why you might want to present that as a blueprint for how we
can do things better in the future, but you actually don't have to
transport yourself back to the late 1940s and the early 1950s to make
that show. You can actually make a version of that, that exists right
now in Hollywood.

wesley morris

You could just set it at CBS!

jenna wortham

You could literally just set it at CBS! You could literally just set it
at Netflix! And so it's this kind of a fantasy of this nostalgic
optimism that if we could go back and do things the right way, we would.
But hello? Heller? In 2020, there aren't a lot of good parts for black
and Asian actors right now. Am I wrong? I don't know. You don't have to
go back to old Hollywood for that. And it's true we've had movies like
``Love Jones,'' obviously ``Love \& Basketball.'' We've had a lot of
really incredible films cast with non-white leads. But to me, it feels
like it's not enough.

wesley morris

But that is the whole point of this show, which is to make it so that if
you start at pretty much in the golden age of Hollywood, you don't have
to deal with the frustration of ``Love \& Basketball'' and ``Love
Jones'' being these atypical national black person event type movies.
And that is the goal of this show. And we should say that it comes from
Ryan Murphy. ``Hollywood'' is a Ryan Murphy production. He made the show
with this guy named Ian Brenner. And it's pretty much of a piece with
this version of Ryan Murphy, this man who has committed to these ideas
of so-called diversity and inclusion, and what it means to hire non cis
white men for jobs, essentially. And he is doing that work. He's given
us the Betty and Joan season of ``Feud'' and --- you know one of his
anthology shows on FX. He's made ``The People v. O.J. Simpson,'' which
is great. And ``Feud'' is also great.

jenna wortham

And ``Pose!'' Can't forget about ``Pose.''

wesley morris

And ``Pose.'' Who could forget? Who could forget ``Pose?'' A world made
up entirely of queer people and mostly trans people.

jenna wortham

Beautiful black people, and Latinx people.

wesley morris

And they're mostly black and Latinx, yes. So I mean, he's busy. He's got
a lot of hands in a lot of pots, and a lot of them involve bringing
people who don't get brought to things to the television screens of
people around the world. It is a worthy project, and this show is very
much of a piece with it.

\begin{itemize}
\item
  archived recording 1\\
  I got to say Mr. Samuels, sometimes I think folks in this town don't
  really understand the power they have. Movies don't just show us how
  the world is. They show us how the world can be. And if we change the
  way that movies are made, you take a chance and you make a different
  kind of story, I think you can change the world.
\item
  archived recording 2\\
  Here are some scripts that I'm --- that Mr. Amberg is excited about.
  Read them. If you find something you like, come back and let's talk.
\item
  archived recording 3\\
  OK, all of them?
\item
  archived recording 4\\
  All of them. {[}BELL RINGS{]}
\end{itemize}

jenna wortham

At some point, the film is released. It's a huge hit. It's a wide
release, so it's in the black theaters, the white theaters, all the
theaters. They've dropped the price, so anybody can go. And the film is
nominated for a whole host of Oscars. And while we're at the Oscars and
each character's category is being called, they cut to a shot of people
at home listening along. So when it's Anna May Wong's turn, it's like
``And the nominees for Best Supporting Actress are --- '' and they call
her name. They cut to a family of Chinese people crowded around a radio.
And in that moment, I was reminded, the Oscars do matter. These awards
do matter. Representation is so important. Visibility is so important.
And then a few seconds later, when Camille's category is called ---

wesley morris

Oh, lord.

jenna wortham

--- they cut to a black family in like a shack. I don't know. I was
like, what year is this? Did we go back in time? Is this the 1880s? I
don't know. I am still traumatized by that scene. Because I actually was
thinking a lot about the most recent Oscars. And I was thinking about
``Parasite.'' And I was thinking about that win, and how good it felt,
and it felt like a recognition. It felt like ---

wesley morris

Of course, yeah.

jenna wortham

--- we're acknowledging this South Korean film and this filmmaker. And
it did remind me of the ways in which these major international
acknowledgments, they do feel like advances. They do feel like progress,
and I don't want to undermine that at all. ``Hollywood'' itself, as this
Netflix show, didn't feel like it was trying to do the thing that the
show was about. It felt like it was patting itself on the back for
wanting to do those things, right?

\begin{itemize}
\tightlist
\item
  archived recording\\
  To be standing here as the first actress of Chinese descent to win
  this award, not for putting on yellow face and playing an oriental
  caricature, but for playing a woman. A complex woman with a heart and
  a soul. Thank you. {[}APPLAUSE{]}
\end{itemize}

wesley morris

So the thing about this show to me that feels especially fantastical in
terms of how utopic it is, is that it's sort of a luxury for white
people to be able to go back in the past and do whatever they want,
because they were fine back then. {[}LAUGHTER{]} When black people go
back into the past, it is to essentially operate around certain
histories that were essentially designed to oppress, and subjugate, and
essentially annihilate them. And the idea of going back there and
creating a utopia just seems to be a rather --- I mean, if not
distasteful thing to do, then slightly naive. And Ryan Murphy has this
luxury of revisiting a past in a way that a black person, or if you're
telling Anna May Wong's story, a Chinese-American person just wouldn't
be able to.

jenna wortham

Listen, for a show that wants me to take these tweaks seriously, and I
love Queen Latifah as much as the next person, but you mean to tell me
they couldn't find a single dark skinned actor to play Hattie McDaniel?
Like I get it, you wanted a big name actress, great. But like, Gabby
Sidibe wasn't available? That's a real sensitive part of history. And
she was the daughter of slaves, she wasn't allowed in a theater, and
you're going to cast a light-skinned actress to play her? That's
painful. Don't rewrite our history like that. You want to rewrite the
past so we can rethink the present? Then do right by the past. Because
that does not give me the confidence that you really take this
seriously.

wesley morris

I hear you and --- Hattie McDaniel, of course, won the academy award for
Best Supporting Actress for 1939's ``Gone With the Wind.'' She played
the character, the unforgettable iconic character Mammy. And the thing
about this show with respect to Hattie McDaniel and people like her, is
that it is full of all of these interesting real people, who lived
really interesting Hollywood lives. And why ``Hollywood'' isn't just an
anthology show about these great entertainers? What if they had just
made a movie about Oscar Micheaux? He's the first black filmmaker of any
significance in this country. I mean, he is somebody that you could have
had a lot of fun with taking from the margins. He was essentially the
Tyler Perry of the 1920s. What if Warner Brothers hired him to make a
major motion picture, and starring Hattie McDaniel? I just feel like the
show has an interesting imagination when it comes to what it wants to do
with the Hollywood equivalent of the Marvel universe, right? Which of
your favorite characters does it want to get together to do work
together?

jenna wortham

My gosh. This show, to me, starts to feel a little bit like --- it's a
liberation fantasy. It's like what would it have looked like back then
if white people were the saviors of all other people? And that fantasy
is not helpful for thinking about then. It's not helpful for thinking
about right now. And there's a moment in the Oscar scene of the seventh
episode of the show ``Hollywood,'' when Avis Amberg, who's running Ace
Studios, is watching ---

wesley morris

Patti LuPone's character.

jenna wortham

Patti LuPone's character --- is watching ``Meg.'' At first, it seems
like ``Meg'' is not going to win any awards, and she has this very
dramatic back of the hand to the forehead moment. ``Oh, it's because I'm
a woman.'' Is it because you're a woman? Or is it because they don't
want to reward this show about an interracial couple with a Chinese
woman it. I don't know if it's about you, white lady. I don't know if
it's about you right now in this moment. And them writing that in with
an ounce of sincerity, and I just felt like, this is the problem with
the show. It thinks it's all about whoever made the show, and not
actually about the lives and the trials and tribulations of these people
of color.

wesley morris

When you say show, do you mean ``Hollywood,'' or do you mean the people
who made ``Meg?''

jenna wortham

It's kind of both.

wesley morris

OK.

jenna wortham

It's kind of one and the same, to be honest with you. It's really nice
to imagine that after World War II, in this period of rebuilding and
regeneration, that something like the social and psychological legacy of
slavery could be easily wiped away by people in a boardroom saying,
``Let's make a movie. Let's make this movie starring a colored woman.''
Let's think about what was going on during that period of time. Jim
Crow, deep segregation is still active. And it's not just something
that's like colored water fountain and a white water fountain. No, this
is a system of segregation that is set up to demoralize one group of
people and maintain the hegemony of another group of people. That's
something that's very difficult to undo. It's not something like ironing
out wrinkles in a shirt. It's not a switch that you can flip. And the
government at this period of time has decided, you know, All
Japanese-Americans, you're just going to be hidden away and be in these
internment camps. I mean, there is a real practice, active practice of
racial terror that's happening across many different groups of people.
And that's not something that's just undone with a magic wand. I
understand the fantasy, and I understand the agenda. And again, I do
think it is a noble experiment, but it is also a little bit insulting to
just assume that at this moment in time, people would be ready for a
Chinese-American movie star. They would be ready for a beautiful black
woman to have all the opportunities that white women have. To kind of
allow that type of magical thinking to exist in this racial fantasy I
think is really harmful. And it's frustrating, and it's a free pass. And
it was hurtful. I actually was like --- I felt pained.

wesley morris

Yeah, no. I mean, it's reducing racism to something that can basically
be solved by one movie, that's presented as the solution to all of
America's racial problems. There is a moment in episode seven that just
cracked me up.

jenna wortham

Which one?

wesley morris

The movie opens. It's a huge hit. And three of the men involved with the
film go to a screening and watch. They sit in the balcony, and there's
newsreel footage of the premiere of the movie and the hoopla around it.

jenna wortham

Right, right.

wesley morris

And at some point, the announcer says, ``Racial protests around the
country simply melted away.''

\begin{itemize}
\item
  archived recording 1\\
  --- simply melted away, as thousands rushed out to see a new kind of
  motion picture, and moviegoers of every color have fallen in love.
\item
  archived recording 2\\
  Every woman I know who saw it cried. And we understood how she was
  feeling.
\item
  archived recording 3\\
  Could one movie change the way a nation sees itself? Who knows? But
  one thing's for sure, America's mad for ---
\item
  archived recording 4\\
  ``Meg!'' {[}LAUGHTER{]}
\end{itemize}

jenna wortham

Honestly, Wesley, I'm glad one of us is laughing right now, because I
had the exact opposite somatic response in my body. I was watching that
episode in the bathtub, and I jerked so much that a tidal wave of my
rose-scented magnesium salt-infused water splashed onto my floor, and I
had to scramble to find a towel to mop it up. But I ---

wesley morris

Oh my god.

jenna wortham

I mean, I'm being really serious. I cried. I was so hurt. That is a
really beautiful sentiment that feels so negligent, and so --- that is
in direct opposition to even what we're seeing right now. You have
anti-quarantine protests, which, protest wanting to go back to work.
Yeah, make yourself known. That is a thing you should be able to do. But
the white people in this country can't even show up in protest without
bringing Nazi insignia, and Confederate flags, and assault rifles. Those
are emblems of hate. I don't want to overestimate the power of popular
culture to shift societal expectations and norms. It is hugely
influential, hugely instrumental. But we also can't overstate its
ability. There has to be some balance ---

wesley morris

Yes, yes, I hear you.

jenna wortham

--- and part of the fantasy of this whole show is that Hollywood has a
lot more power than it thinks it does. It's really easy to look back and
to cast yourself backwards in time and say, ``If I had been around
during this particularly fraught era in history, I would have helped
slaves escape. I would have hid people in my home if I'd been in Nazi
Germany.'' All of these fantasies are easy to say, and this show extends
that. It frustrates me because I think that it allows for a type of
escapism that isn't useful right now. We are living in a state of
emergency.

wesley morris

That's why I find the distrust of humanity on a show like ``Westworld''
to be slightly more fascinating. Even though I don't quite know what the
hell is going on with that show, I think the use of its dystopic
elements are actually really imagining a better future in a way that
requires, as at some point Evan Rachel Wood's character Delores says,
that Free will exists, it's just {[}EXPLETIVE{]} hard. And I just find
the hard way more interesting than the easy way. In ``Hollywood'' the TV
show, it kind of makes it too easy. And we're in a moment in which we're
having utopias pushed on us all the time. Every time we turn on a
television show, there is the risk of being bombarded with utopia
propaganda, essentially.

jenna wortham

Mmm.

wesley morris

And there are all these commercials that basically insist that we're in
this together. ``We're all gonna make it!''

jenna wortham

``Everything's gonna be just fine. Just order our products, and you'll
live. You'll be OK.'' It's like that is the most dystopic thing, but
it's a dystopia masquerading as a utopia.

wesley morris

There are ``we're all in this together'' ads for John Deere. There are
``we're all in this together'' ads for places like Walmart.

jenna wortham

Yeah, and I've seen them for No Touch toilets, and I've seen them for
Budweiser. But I really want to talk about this Papa John's commercial,
because it is the most dystopic thing masquerading as a utopic thing.
And it opens with a young man by the name of T.J. Lam, who's a store
manager in Louisville, Ky., professing his deep devotion to delivering
pizzas to all the white families at home that can't be bothered to cook
for their kids anymore.

\begin{itemize}
\tightlist
\item
  archived recording\\
  I know for every pizza I deliver, that's a trip that family doesn't
  have to take out of their house. It relieves some stress off of me to
  let me know that I'm doing something good for the community, not just
  Papa John's.
\end{itemize}

jenna wortham

The utopic notion in this commercial is that the people who are forced
to keep working during this time want to be. They want to be in service
of those who can afford to stay home in their enormous mini mansions, by
the way. And that is the fantasy. That is so dystopic! I can imagine him
saying, ``I'm happily putting myself at risk every day. I may or may not
have a family of dependents, grandparents, immunocompromised folks that
I am putting at risk so I can do my job for you, Papa John's.'' I mean,
Jesus.

wesley morris

I mean, we can see he's not wearing any P.P.E., by the way. They don't
get that --- Papa John's not handing out the P.P.E.

jenna wortham

Why are we acting like that's an OK thing, that he shouldn't have a mask
on, or they shouldn't have a mask on?

wesley morris

Right. Part of the dystopic feeling of a commercial like that is that
you're being lied to. This is a fable about our moment. And I also
wonder if one of the crazy things about this moment is we are really
thinking about the ways in which we maybe thought we had been living in
a utopia, and have now been forced to reckon with the fact that we've
been in a dystopia the entire time. And the thing that has really
brought that home to me is how Ahmaud Arbery died. And the thing that
people keep saying is unimaginable is that this is how people used to
die. This is a death from the 1920s. This is an old death. The idea that
he was running for his --- literally running for his life from white men
in a pickup truck? What time? What wormhole did that truck drive out of?
And the truth is, no, there's no wormhole!

jenna wortham

Yeah, that's it. There's no movie. There's no robot. There's no
algorithm that changes that. And this has been going around via Twitter.
I'm just not claiming credit for this idea. It's been repeated a number
of times, different ways, but the police didn't arrest those men because
there was a video. They saw the video. They only had to take action ---

wesley morris

We saw the video.

jenna wortham

--- because we saw the video. And so when we think about like --- that's
the most dystopic thing ever. And I think that's why a show like
``Westworld'' titillates me, because what do you do when you realize the
levels of oppression that you're living under? And that's why a show
like ``Hollywood'' frustrates me, because there is no version of history
in which an Oscar-winning film fixes all of the problems that this
country has with structural oppression and racism. And you know how I
know the answer to that? Because we actually saw that when Halle Berry
won the award for ``Monster's Ball.'' And guess what? It didn't fix all
these problems, because some 20 years later, Ahmaud Arbery's still dead.
Nina Pop is still dead. Breonna Taylor's still dead. And those are just
the names that were turned into hashtags. So please don't go back to the
past and tell me, ``Hey, had `Meg' won all these awards, this reality
might be different.'' Because variations of that have happened. That's
the loop. And it's up to us to break it. It's up to us to wake up from
it. I just don't know if people want to.

wesley morris

That's our show. ``Still Processing'' is a product of The New York
Times. It was recorded, once again, and for the last time, in our living
rooms.

jenna wortham

It is produced by Hans Buetow and Sydney Harper.

wesley morris

Our editors are Sarah Sarasohn and Sasha Weiss, Wendy Dorr, and Lisa
Tobin.

jenna wortham

Our engineer is Jake Gorski.

wesley morris

Our theme music is by Kindness. It's called ``World Restart'' from the
album ``Otherness.''

jenna wortham

You can find all of our episodes and various fun things at
nytimes.com/stillprocessing.

wesley morris

Thanks for listening, you guys. Good luck.

jenna wortham

Stay safe, and stay healthy. And we'll be back soon.

wesley morris

We'll miss you.

jenna wortham

We will miss you.

wesley morris

We'll miss you.

jenna wortham

It sounds like you're a ghost.

wesley morris

We'll miss you. {[}LAUGHTER{]}

Previous

More episodes ofStill Processing

\href{https://www.nytimes.com/2020/07/23/podcasts/hamilton-ziwe-discomfort.html?action=click\&module=audio-series-bar\&region=header\&pgtype=Article}{\includegraphics{https://static01.nyt.com/images/2020/07/23/multimedia/23stillprocessing-pix/23stillprocessing-pix-thumbLarge.jpg}}

July 23, 2020~~•~ 38:10Ziwe May Destroy Hamilton

\href{https://www.nytimes.com/2020/07/16/podcasts/reparations-for-aunt-jemima.html?action=click\&module=audio-series-bar\&region=header\&pgtype=Article}{\includegraphics{https://static01.nyt.com/images/2020/07/18/multimedia/16stillprocessing-pix/16stillprocessing-pix-thumbLarge.jpg}}

July 16, 2020~~•~ 35:35Reparations for Aunt Jemima!

\href{https://www.nytimes.com/2020/07/09/podcasts/still-processing-black-lives-matter.html?action=click\&module=audio-series-bar\&region=header\&pgtype=Article}{\includegraphics{https://static01.nyt.com/images/2020/07/12/podcasts/09stillprocessing-image/xx-stillprocessing-thumbLarge.jpg}}

July 9, 2020~~•~ 26:29So Y'all Finally Get It

\href{https://www.nytimes.com/2020/05/14/podcasts/still-processing-westworld-hollywood-utopia-dystopia.html?action=click\&module=audio-series-bar\&region=header\&pgtype=Article}{\includegraphics{https://static01.nyt.com/images/2020/05/16/podcasts/14stillprocessing-image/14stillprocessing-image-thumbLarge-v2.jpg}}

May 14, 2020New Loop, America

\href{https://www.nytimes.com/2020/05/07/podcasts/still-processing-internet-vulnerability-sondheim-parks-recreation.html?action=click\&module=audio-series-bar\&region=header\&pgtype=Article}{\includegraphics{https://static01.nyt.com/images/2020/04/28/pageoneplus/28sondheimjp-sp/28sondheimjp-sp-thumbLarge-v4.jpg}}

May 7, 2020Does This Phone Make Me Look Human?

\href{https://www.nytimes.com/2020/04/30/podcasts/still-processing-fiona-apple-fetch-bolt-cutters.html?action=click\&module=audio-series-bar\&region=header\&pgtype=Article}{\includegraphics{https://static01.nyt.com/images/2020/05/03/multimedia/30stillpro-image/30stillpro-image-thumbLarge.jpg}}

May 1, 2020Fiona Ex Machina

\href{https://www.nytimes.com/2020/04/23/podcasts/still-processing-halle-berry-sharon-stone-catwoman-quarantine.html?action=click\&module=audio-series-bar\&region=header\&pgtype=Article}{\includegraphics{https://static01.nyt.com/images/2020/04/25/arts/23stillprocessing/23stillprocessing-thumbLarge-v3.jpg}}

April 23, 2020Halle Berry? Hallelujah.

\href{https://www.nytimes.com/2020/04/16/podcasts/still-processing-AIDS-survive-coronavirus.html?action=click\&module=audio-series-bar\&region=header\&pgtype=Article}{\includegraphics{https://static01.nyt.com/images/2020/04/20/us/16stillprocessing/16stillprocessing-thumbLarge-v3.jpg}}

April 16, 2020How to Learn From a Plague

\href{https://www.nytimes.com/2020/04/09/podcasts/still-processing-tiger-king.html?action=click\&module=audio-series-bar\&region=header\&pgtype=Article}{\includegraphics{https://static01.nyt.com/images/2020/04/11/podcasts/09stillprocessing-image2/09stillprocessing-image2-thumbLarge-v2.jpg}}

April 9, 2020~~•~ 39:49Frosted Flakes

\href{https://www.nytimes.com/2020/04/02/podcasts/high-fidelity-zoe-kravitz.html?action=click\&module=audio-series-bar\&region=header\&pgtype=Article}{\includegraphics{https://static01.nyt.com/images/2020/04/05/arts/02still-processing-highfidelity/13highfidelity-thumbLarge.jpg}}

April 2, 2020~~•~ 40:55Delicious Vinyl

\href{https://www.nytimes.com/2020/03/26/podcasts/still-processing-quarantine.html?action=click\&module=audio-series-bar\&region=header\&pgtype=Article}{\includegraphics{https://static01.nyt.com/images/2020/03/29/podcasts/26stillprocessing1/26stillprocessing1-thumbLarge.jpg}}

March 26, 2020~~•~ 30:47A Pod From Both Our Houses

\href{https://www.nytimes.com/2019/11/07/podcasts/still-processing-parasite-watchmen-bong-joon-ho.html?action=click\&module=audio-series-bar\&region=header\&pgtype=Article}{\includegraphics{https://static01.nyt.com/images/2019/11/08/arts/07stilpr-parasite/00parasite-1-thumbLarge.jpg}}

November 7, 2019Wake

\href{https://www.nytimes.com/column/still-processing-podcast}{See All
Episodes ofStill Processing}

Next

Published May 14, 2020Updated May 20, 2020

\begin{itemize}
\item
\item
\item
\item
\item
\end{itemize}

By \href{https://www.nytimes.com/by/wesley-morris}{Wesley Morris} and
\href{https://www.nytimes.com/by/jenna-wortham}{Jenna Wortham}

In our final episode from our living rooms, we visit the dystopia of
``Westworld'' and the utopia of ``Hollywood'' to see if we can glean
anything about what might be in store on the other side of this pandemic
--- and about who we want to be.

\includegraphics{https://static01.nyt.com/images/2020/05/16/podcasts/14stillprocessing-image/14stillprocessing-image-articleLarge.jpg?quality=75\&auto=webp\&disable=upscale}

Discussed this week:

\begin{itemize}
\item
  ``\href{https://www.hbo.com/westworld}{Westworld}'' (HBO, 2016-20)
\item
  ``\href{https://www.netflix.com/title/81088617}{Hollywood}'' (Netflix,
  2020)
\item
  \href{https://www.imdb.com/name/nm0614682/}{Ryan Murphy}
\item
  ``\href{https://www.imdb.com/title/tt0327162/}{The Stepford Wives}''
  (directed by Frank Oz, 2004)
\item
  ``\href{https://www.imdb.com/title/tt0073747/}{The Stepford Wives}''
  (directed by Bryan Forbes, 1975)
\item
  \href{https://www.asu.edu/courses/fms200s/total-readings/MotionPictureProductionCode.pdf}{The
  Motion Picture Production Code of 1930}
\item
  ``\href{https://www.imdb.com/title/tt0119572/}{Love Jones}'' (directed
  by Theodore Witcher, 1997)
\item
  ``\href{https://www.imdb.com/title/tt0199725/}{Love \& Basketball}''
  (directed by Gina Prince-Bythewood, 2000)
\item
  \href{https://www.imdb.com/name/nm0938923/}{Anna May Wong}
\item
  \href{https://www.imdb.com/name/nm0567408/}{Hattie McDaniel}
\item
  \href{https://www.imdb.com/name/nm0584778/}{Oscar Micheaux}
\item
  ``\href{https://www.youtube.com/watch?v=mczJdKcB1bc}{Delivering Thanks
  Team}'' (Papa John's, 2020)
\end{itemize}

``Still Processing'' is produced by Hans Buetow and Sydney Harper, and
edited by Sara Sarasohn and Sasha Weiss, with editorial oversight from
Wendy Dorr and Lisa Tobin. Our engineer is Jake Gorski. Our theme music
is by Kindness. It's called ``World Restart,'' from the album
``Otherness.''

Advertisement

\protect\hyperlink{after-bottom}{Continue reading the main story}

\hypertarget{site-index}{%
\subsection{Site Index}\label{site-index}}

\hypertarget{site-information-navigation}{%
\subsection{Site Information
Navigation}\label{site-information-navigation}}

\begin{itemize}
\tightlist
\item
  \href{https://help.nytimes.com/hc/en-us/articles/115014792127-Copyright-notice}{©~2020~The
  New York Times Company}
\end{itemize}

\begin{itemize}
\tightlist
\item
  \href{https://www.nytco.com/}{NYTCo}
\item
  \href{https://help.nytimes.com/hc/en-us/articles/115015385887-Contact-Us}{Contact
  Us}
\item
  \href{https://www.nytco.com/careers/}{Work with us}
\item
  \href{https://nytmediakit.com/}{Advertise}
\item
  \href{http://www.tbrandstudio.com/}{T Brand Studio}
\item
  \href{https://www.nytimes.com/privacy/cookie-policy\#how-do-i-manage-trackers}{Your
  Ad Choices}
\item
  \href{https://www.nytimes.com/privacy}{Privacy}
\item
  \href{https://help.nytimes.com/hc/en-us/articles/115014893428-Terms-of-service}{Terms
  of Service}
\item
  \href{https://help.nytimes.com/hc/en-us/articles/115014893968-Terms-of-sale}{Terms
  of Sale}
\item
  \href{https://spiderbites.nytimes.com}{Site Map}
\item
  \href{https://help.nytimes.com/hc/en-us}{Help}
\item
  \href{https://www.nytimes.com/subscription?campaignId=37WXW}{Subscriptions}
\end{itemize}
