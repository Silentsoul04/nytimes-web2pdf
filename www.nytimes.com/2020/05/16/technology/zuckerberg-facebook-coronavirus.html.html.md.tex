\href{/section/technology}{Technology}\textbar{}Now More Than Ever,
Facebook Is a `Mark Zuckerberg Production'

\begin{itemize}
\item
\item
\item
\item
\item
\end{itemize}

\href{https://www.nytimes.com/news-event/coronavirus?action=click\&pgtype=Article\&state=default\&region=TOP_BANNER\&context=storylines_menu}{The
Coronavirus Outbreak}

\begin{itemize}
\tightlist
\item
  live\href{https://www.nytimes.com/2020/08/01/world/coronavirus-covid-19.html?action=click\&pgtype=Article\&state=default\&region=TOP_BANNER\&context=storylines_menu}{Latest
  Updates}
\item
  \href{https://www.nytimes.com/interactive/2020/us/coronavirus-us-cases.html?action=click\&pgtype=Article\&state=default\&region=TOP_BANNER\&context=storylines_menu}{Maps
  and Cases}
\item
  \href{https://www.nytimes.com/interactive/2020/science/coronavirus-vaccine-tracker.html?action=click\&pgtype=Article\&state=default\&region=TOP_BANNER\&context=storylines_menu}{Vaccine
  Tracker}
\item
  \href{https://www.nytimes.com/interactive/2020/07/29/us/schools-reopening-coronavirus.html?action=click\&pgtype=Article\&state=default\&region=TOP_BANNER\&context=storylines_menu}{What
  School May Look Like}
\item
  \href{https://www.nytimes.com/live/2020/07/31/business/stock-market-today-coronavirus?action=click\&pgtype=Article\&state=default\&region=TOP_BANNER\&context=storylines_menu}{Economy}
\end{itemize}

\includegraphics{https://static01.nyt.com/images/2020/05/17/business/00zuck-sub/merlin_159071070_4d828857-4686-40bc-89f4-dcba28775807-articleLarge.jpg?quality=75\&auto=webp\&disable=upscale}

Sections

\protect\hyperlink{site-content}{Skip to
content}\protect\hyperlink{site-index}{Skip to site index}

\hypertarget{now-more-than-ever-facebook-is-a-mark-zuckerberg-production}{%
\section{Now More Than Ever, Facebook Is a `Mark Zuckerberg
Production'}\label{now-more-than-ever-facebook-is-a-mark-zuckerberg-production}}

For years, he was an obsessive C.E.O. in some ways, distant in others.
Then Facebook's problems became too acute to leave to anyone else.

Credit...Tom Brenner/The New York Times

Supported by

\protect\hyperlink{after-sponsor}{Continue reading the main story}

\href{https://www.nytimes.com/by/mike-isaac}{\includegraphics{https://static01.nyt.com/images/2018/02/16/multimedia/author-mike-isaac/author-mike-isaac-thumbLarge.jpg}}\href{https://www.nytimes.com/by/sheera-frenkel}{\includegraphics{https://static01.nyt.com/images/2018/06/14/multimedia/author-sheera-frenkel/author-sheera-frenkel-thumbLarge.png}}\href{https://www.nytimes.com/by/cecilia-kang}{\includegraphics{https://static01.nyt.com/images/2019/01/29/multimedia/author-cecilia-kang/author-cecilia-kang-thumbLarge.png}}

By \href{https://www.nytimes.com/by/mike-isaac}{Mike Isaac},
\href{https://www.nytimes.com/by/sheera-frenkel}{Sheera Frenkel} and
\href{https://www.nytimes.com/by/cecilia-kang}{Cecilia Kang}

\begin{itemize}
\item
  Published May 16, 2020Updated July 8, 2020
\item
  \begin{itemize}
  \item
  \item
  \item
  \item
  \item
  \end{itemize}
\end{itemize}

\hypertarget{listen-to-this-article}{%
\subsubsection{Listen to This Article}\label{listen-to-this-article}}

Audio Recording by Audm

\emph{To hear more audio stories from publishers like The New York
Times, download}
\href{https://www.audm.com/?utm_source=nyt\&utm_medium=embed\&utm_campaign=facebook_zuckerberg_production}{\emph{Audm
for iPhone or Android}}\emph{.}

SAN FRANCISCO --- On Jan. 27, at a regularly scheduled Monday morning
meeting with top executives at
\href{https://www.nytimes.com/2020/05/21/technology/facebook-remote-work-coronavirus.html}{Facebook},
\href{https://www.nytimes.com/2020/06/02/technology/zuckerberg-defends-facebook-trump-posts.html}{Mark
Zuckerberg} turned the agenda to the coronavirus. For weeks, he told his
staff, he had been hearing from global health care experts that the
virus had the makings of a pandemic, and now Facebook needed to prepare
for a worst-case scenario --- one in which the company's ability to
\href{https://www.nytimes.com/2020/03/08/technology/coronavirus-misinformation-social-media.html}{combat
misinformation}, scammers and conspiracy theorists would be tested as
never before.

To start, Mr. Zuckerberg said, the company should take some of the tools
it had developed to fight 2020 election garbage and attempt to retool
them for the pathogen. He asked executives in charge of every department
to develop plans for responding to a global outbreak by the end of the
week.

The meeting, described by two people who attended it, helped vault
Facebook ahead of other companies --- and even some governments --- in
preparing for Covid-19. And it exemplified a change in how the
36-year-old is running the company he founded.

Since the day he coded the words ``a Mark Zuckerberg production'' onto
every blue-and-white Facebook page, he has been the singular face of the
social network. But to an extent not widely appreciated outside Silicon
Valley, Mr. Zuckerberg has long been a kind of binary chief executive
--- extraordinarily involved in some aspects of the business, and
virtually hands-off in areas that he finds less interesting.

The beginning of the end of Mr. Zuckerberg's distanced leadership came
on Nov. 8, 2016, with the election of Donald Trump. From that moment, a
relentless series of crises --- his casual dismissal of concerns over
fake news as
``\href{https://www.theguardian.com/technology/2016/nov/10/facebook-fake-news-us-election-mark-zuckerberg-donald-trump}{a
pretty crazy idea}''; revelations that the platform had been used as a
plaything for
\href{https://www.nytimes.com/2017/09/07/us/politics/russia-facebook-twitter-election.html}{state-sponsored
espionage}; the
\href{https://www.nytimes.com/2018/04/04/us/politics/cambridge-analytica-scandal-fallout.html}{Cambridge
Analytica scandal} --- jolted Mr. Zuckerberg to tighten his grip.

Many of his consolidation tactics have been highly visible: He replaced
the outside founders of Instagram and WhatsApp with loyalists, and he
refashioned Facebook's already-friendly board to be even more
deferential, swapping out five of its nine members.

\includegraphics{https://static01.nyt.com/images/2020/05/17/business/00zuck3/merlin_152318454_1f864ee5-2334-49a4-8143-aadf99de5330-articleLarge.jpg?quality=75\&auto=webp\&disable=upscale}

With the attention of a quarter of the world's population to sell to
advertisers, Facebook is so colossal that org-chart moves have the
effect of creating powerful new characters on the global policy stage.
Mr. Zuckerberg has elevated lieutenants to win over hostile territories
--- the Republican operative Joel Kaplan in Washington, and the former
deputy prime minister of Britain, Sir Nicholas Clegg, in the eurozone.
And his more hands-on approach has caused, by the zero-sum logic of
corporate clout, an effective sidelining of Sheryl Sandberg, his chief
operating officer and the most high-profile woman in technology.

Now, the coronavirus has presented Mr. Zuckerberg with the opportunity
to demonstrate that he has grown into his responsibilities as a leader
--- a 180-degree turn from the aloof days of 2016. It's given him the
chance to lead 50,000 employees through a crisis that, for once, is not
of their own making. And seizing the moment might allow Mr. Zuckerberg
to prove a thesis that he truly believes: That if one sees past its
capacity for destruction, Facebook can be a force for good.

``Mark has taken an active role in the leadership of Facebook from its
founding through to today,'' Dave Arnold, a company spokesman, said in
an emailed statement. ``We're fortunate to have such engaged leaders,
including Mark, Sheryl and the entire leadership team. Facebook is a
better company for it.''

The revamp has not gone without incident. In early May, Facebook
struggled with how to handle a viral conspiracy video known as
``\href{https://www.nytimes.com/2020/05/09/technology/plandemic-judy-mikovitz-coronavirus-disinformation.html}{Plandemic},''
waffling as the footage spread to the screens of millions of users. Last
week, reporters at the Detroit Metro Times showed that the company was
blind to
\href{https://www.metrotimes.com/news-hits/archives/2020/05/11/whitmer-becomes-target-of-dozens-of-threats-on-private-facebook-groups-ahead-of-armed-rally-in-lansing}{assassination-stoking
activity} on pages with 400,000 members.

Still, for Mr. Zuckerberg, the pandemic has the potential to be a more
favorable backdrop than what 2020 would have ordinarily been dominated
by --- the presidential election and the difficulties of policing
political speech.

In theory, the crisis plays to some of his strengths. Through his
personal philanthropy, the Chan Zuckerberg Initiative, he has long been
interested in curing and preventing disease. Covid is borderless, like
Facebook itself, and will require a supranational response at a scale
few other organizations are equipped to handle. Solutions, if they ever
come, will be grounded in science and not emotion or politics.

\hypertarget{latest-updates-economy}{%
\section{\texorpdfstring{\href{https://www.nytimes.com/live/2020/07/31/business/stock-market-today-coronavirus?action=click\&pgtype=Article\&state=default\&region=MAIN_CONTENT_1\&context=storylines_live_updates}{Latest
Updates:
Economy}}{Latest Updates: Economy}}\label{latest-updates-economy}}

\href{https://www.nytimes.com/live/2020/07/31/business/stock-market-today-coronavirus?action=click\&pgtype=Article\&state=default\&region=MAIN_CONTENT_1\&context=storylines_live_updates\#kodaks-chief-executive-was-given-stock-options-then-the-share-price-spiked-1000-percent}{31h
ago}

\href{https://www.nytimes.com/live/2020/07/31/business/stock-market-today-coronavirus?action=click\&pgtype=Article\&state=default\&region=MAIN_CONTENT_1\&context=storylines_live_updates\#kodaks-chief-executive-was-given-stock-options-then-the-share-price-spiked-1000-percent}{Kodak's
chief executive was given stock options. Then the share price spiked
1,000 percent.}

\href{https://www.nytimes.com/live/2020/07/31/business/stock-market-today-coronavirus?action=click\&pgtype=Article\&state=default\&region=MAIN_CONTENT_1\&context=storylines_live_updates\#fitch-ratings-downgrades-its-outlook-on-us-debt}{34h
ago}

\href{https://www.nytimes.com/live/2020/07/31/business/stock-market-today-coronavirus?action=click\&pgtype=Article\&state=default\&region=MAIN_CONTENT_1\&context=storylines_live_updates\#fitch-ratings-downgrades-its-outlook-on-us-debt}{Fitch
Ratings downgrades its outlook on U.S. debt.}

\href{https://www.nytimes.com/live/2020/07/31/business/stock-market-today-coronavirus?action=click\&pgtype=Article\&state=default\&region=MAIN_CONTENT_1\&context=storylines_live_updates\#us-sanctions-more-chinese-officials-over-human-rights-violations-as-tensions-flare}{41h
ago}

\href{https://www.nytimes.com/live/2020/07/31/business/stock-market-today-coronavirus?action=click\&pgtype=Article\&state=default\&region=MAIN_CONTENT_1\&context=storylines_live_updates\#us-sanctions-more-chinese-officials-over-human-rights-violations-as-tensions-flare}{U.S.
sanctions more Chinese officials over human rights violations as
tensions flare}

\href{https://www.nytimes.com/live/2020/07/31/business/stock-market-today-coronavirus?action=click\&pgtype=Article\&state=default\&region=MAIN_CONTENT_1\&context=storylines_live_updates}{See
more updates}

More live coverage:
\href{https://www.nytimes.com/2020/08/01/world/coronavirus-covid-19.html?action=click\&pgtype=Article\&state=default\&region=MAIN_CONTENT_1\&context=storylines_live_updates}{Global}

Or the pandemic could take all that is dangerous about Facebook and
amplify it. When the stakes are not merely a presidential election but
global health, any role the company plays in elevating toxic information
has the potential to make all its prior harms seem trivial. And if Mr.
Zuckerberg is fully in control of his company in a way he wasn't before
--- as acknowledged by interviews with more than two dozen people ---
the success or failure of its response will reside entirely with him.

``I think it's going to piss off a lot of people,'' Mr. Zuckerberg said
of his new management style in an interview
\href{https://www.cnbc.com/2020/01/31/mark-zuckerberg-silicon-slopes-speech-honesty-will-piss-off-people.html}{at
a tech conference} earlier this year. ``But frankly, the old approach
was pissing off a lot of people, too.''

\hypertarget{until-now-ive-been-a-peacetime-leader}{%
\subsection{`Until now, I've been a peacetime
leader'}\label{until-now-ive-been-a-peacetime-leader}}

In Silicon Valley, there is a certain kind of company founder whose
title is C.E.O. but who presents himself as a ``product guy.'' A
product-guy C.E.O. feels more at home developing what is for sale than
actually running the company.

At Apple, Steve Jobs was a product guy, inventing the iPhone while
leaving the supply chain to his C.O.O. At Amazon, Jeff Bezos is a
product guy, obsessing about retail customers while others run the
profitable web-hosting division. And at Facebook, for more than a
decade, Mark Zuckerberg was a product guy's product guy.

In practice, this meant Mr. Zuckerberg dove into important new products,
giving direct orders to middle managers in charge of whatever feature he
was obsessed with that week. It also meant he was comfortable delegating
in areas that interested him less keenly --- including the advertising
machine that generated \$70 billion in revenue last year. Even less
compelling to Mr. Zuckerberg was the realm of Facebook policy around
what kind of speech was and was not permitted. Those subjects fell into
a specific category: Too important to ignore, but not exactly what a
young billionaire wants to spend all of his time on.

Oversight of those areas went to his trusted inner circle, known as the
M-Team. Short for ``Mark Team,'' its members knew they were never likely
to succeed him as chief executive, but they could remain powerful and
autonomous within their own departments. At the top was Ms. Sandberg,
Mr. Zuckerberg's second-in-command, whose portfolio spanned advertising,
marketing, regulation, communications and beyond.

The 2016 election made it clear to Mr. Zuckerberg that the accommodation
was no longer viable, as he and Ms. Sandberg were
\href{https://www.nytimes.com/2018/11/14/technology/facebook-data-russia-election-racism.html}{pilloried}
for being absent and distracted, if not willfully negligent. Afterward,
Mr. Zuckerberg spent a chunk of 2017 on a
\href{https://www.nytimes.com/2017/05/25/technology/zuckerberg-harvard-commencement-road-trip.html}{state-by-state
tour of America}, but it wasn't well received; mostly, his photogenic
purple-state antics --- sitting on tractors, attending church,
bottle-feeding calves --- just fed the rumor that he was making a run
for president. Mr. Zuckerberg resolved to take control of the global
superpower in which he already dominated the voting.

First, he made a show of owning up to its failures. ``It's clear now
that we didn't do enough,'' he told reporters on a
\href{https://about.fb.com/news/2018/04/hard-questions-protecting-peoples-information/}{conference
call} in 2018, reflecting on the company's string of missteps. ``We
didn't focus enough on preventing abuse and thinking through how people
could use these tools to do harm as well. We didn't take a broad enough
view of what our responsibility is, and that was a huge mistake.'' He
added: ``It was my mistake.''

Not long after, in July 2018, Mr. Zuckerberg called a meeting with his
top lieutenants. In the past, he had used the group's semiannual
gatherings to chart new courses for Facebook products, or discuss new
technology he was interested in capitalizing on. This time, he told his
executives that his focus was on himself. With Facebook constantly under
attack from outsiders, Mr. Zuckerberg said, he needed to reinvent
himself for ``wartime.''

``Up until now, I've been a peacetime leader,'' Mr. Zuckerberg said,
according to three people who were present but not authorized to discuss
the meeting publicly. ``That's going to change.'' Mr. Zuckerberg said he
would be making more decisions on his own, based on his instincts and
vision for the company. Wartime leaders were quicker and more decisive,
he said, and they didn't let fear of angering others paralyze them.
(Some details of the meeting were previously
\href{https://www.wsj.com/articles/with-facebook-at-war-zuckerberg-adopts-more-aggressive-style-1542577980}{reported}
by The Wall Street Journal.)

Mr. Zuckerberg directed Facebook's so-called ``family of apps'' ---
Instagram, Messenger, WhatsApp and Facebook proper --- to work more
closely together. Instagram had to start sending traffic back to the
flagship product; WhatsApp had to better integrate with its sister
social media services. Rather than execute Mr. Zuckerberg's vision, the
heads of Instagram, Kevin Systrom and Mike Krieger, left the company in
September 2018, after earlier departures by the disillusioned founders
of WhatsApp. Together, they forfeited more than a billion dollars in
compensation.

Image

Mr Zuckerberg speaking at Georgetown University last
October.Credit...Justin T. Gellerson for The New York Times

Mr. Zuckerberg also began to participate more directly in meetings that
had previously been Ms. Sandberg's domain --- from the nitty-gritty of
taking down disinformation campaigns, to winding philosophical
discussions on how Facebook ought to handle political ads. Employees
couldn't help but notice a shift in the balance of power in one of
technology's most lucrative partnerships.

Giving speeches and schmoozing policymakers were two of Ms. Sandberg's
specialties. Mr. Zuckerberg began to do more of that, too, starting with
a
\href{https://www.nytimes.com/2019/10/17/business/zuckerberg-facebook-free-speech.html}{lofty
public address} at Georgetown University's hallowed Gaston Hall, where
more than a century's worth of dignitaries had orated from the same
antique, carved-wood podium.

Mr. Zuckerberg continued the speaking tour with regulator-heavy
engagements in Utah, Belgium, Germany and elsewhere. In Europe, where
Facebook had an especially frosty relationship with government agencies,
he tapped Mr. Clegg, who has grown into a new role as the company's
diplomat-in-chief.

Publicly, Ms. Sandberg has said her role at Facebook is larger than
ever; she is directing a
\href{https://www.facebook.com/business/boost/grants}{\$100 million
grant program} for small businesses hurt by the pandemic. Many of the
new hires, including Mr. Clegg, report to her, and she has said she has
always wanted Mr. Zuckerberg to be more visible. ``I think we don't
spend that much time worrying about our public image,'' Ms. Sandberg
said in an
\href{https://www.nbcnews.com/podcast/byers-market/transcript-facebook-s-sheryl-sandberg-n1145051}{NBC
podcast interview} in February. ``The issue is not what people think of
me or Mark personally. What it is, is how are we doing as a company?''

But privately, Ms. Sandberg has worried that she was being pushed aside
and that her role at Facebook has become less important, said two people
who work within her department. Through a spokesperson, Ms. Sandberg
declined to comment.

Facebook disputes that the relationship has changed. ``There's a clear
structure. Mark is driving the product side of things, while Sheryl is
running the business side of things,'' David Fischer, Facebook's chief
revenue officer, said in an interview. ``It doesn't mean it's all or
nothing --- it's not zero-sum between them.''

\hypertarget{an-adroit-performer}{%
\subsection{`An adroit performer'}\label{an-adroit-performer}}

Facebook devoted 2019 to a full-out lobbying assault on Washington,
committing \$16.7 million to influence policymakers. Only two other
companies spent more. But even beyond cash, Facebook's most powerful
weapon was access to its C.E.O.

Mr. Kaplan --- a well-connected veteran of the George W. Bush
administration --- began arranging for Mr. Zuckerberg to host dinners
with influential conservatives, including Senator Lindsey Graham of
South Carolina and the Fox News host Tucker Carlson. Mr. Kaplan also
nurtured a relationship between Mr. Zuckerberg and Jared Kushner,
President Trump's son-in-law.

In September 2019, New York's attorney general announced a multistate
investigation into whether Facebook had broken antitrust laws. For Mr.
Zuckerberg, it was the clearest indication yet that politics and
government required his full attention --- a potentially existential
threat to his company that could no longer be delegated to others. A
week later, he traveled to Washington to court members of both parties.

In a private room at Ris, an upscale restaurant next to the
Ritz-Carlton, Mr. Zuckerberg dined with prominent Senate Democrats. The
group included Mark Warner of Virginia and Richard Blumenthal of
Connecticut --- both longtime critics of Facebook's security and privacy
practices --- as well as officials newer to tech policy, such as Jeanne
Shaheen of New Hampshire, Catherine Cortez Masto of Nevada and Angus
King, the independent from Maine.

Over grilled salmon, chicken potpie and roasted brussels sprouts, Mr.
Zuckerberg gamely did the kind of basic D.C. give-and-take he'd long
asked Ms. Sandberg to handle: He listened intently and made assurances
about a range of Facebook issues, from foreign election interference to
cryptocurrency.

``He's an adroit performer,'' Mr. Blumenthal said in an interview.
``Almost certainly a result of professional advice, and maybe coaching
and a lot of guidance from a heavy team of lobbyists here in
Washington.'' Mr. Warner added: ``For a while, I think Facebook, along
with a lot of tech companies in the Valley, thought that dealing with
Washington was sort of beneath them. I think Mr. Zuckerberg has realized
that it's to his benefit to engage with us directly.''

The Democratic dinner was just a warm-up for the really important
meeting, which came the next day: Mr. Kaplan and Mr. Kushner arranged
for Mr. Zuckerberg to sit down with the president. The two men had never
met. Ahead of the Sept. 19 session, Mr. Zuckerberg asked his Washington
staff to brief him about Mr. Trump's Facebook presence, so that he could
casually rattle off some statistics in the Oval Office.

Image

Mr. Zuckerberg on Capitol Hill. He is now gamely performing the kind of
basic D.C. give-and-take he'd long asked Ms. Sandberg to
handle.Credit...Samuel Corum/Getty Images

Wearing a dark blue suit and a burgundy tie, Mr. Zuckerberg sat between
Mr. Kushner and Mr. Kaplan, facing Mr. Trump and his jumbo glass of Diet
Coke. Mr. Zuckerberg quickly noted that the president had the highest
level of engagement of any world leader on the social network. Mr. Trump
--- who had previously savaged Facebook on a range of issues ---
immediately adopted a new tone, describing the conversation in social
media posts as ``nice.''

A month later, the president invited Mr. Zuckerberg --- along with
Facebook board member and Trump supporter Peter Thiel --- to a
\href{https://www.nbcnews.com/news/amp/ncna1087986}{private White House
dinner}, which went undisclosed for weeks. Mr. Zuckerberg's simple
flattery seems to have paid off. Mr. Trump hasn't publicly castigated
the company since, and months later, he continues to tell audiences that
he is ``No. 1'' on the world's largest social network.

Within Facebook, Mr. Zuckerberg's more engaged style was rankling
employees. The discontent boiled over later in October, after Mr.
Zuckerberg publicly laid out how Facebook would regulate political
speech on the platform. In the name of free speech, he had said, the
social network would not police what politicians said in political ads
--- even if they lied. Facebook was not in the business of being an
arbiter of truth, nor did it want to be, Mr. Zuckerberg said.

In response,
\href{https://www.nytimes.com/2019/10/28/technology/facebook-mark-zuckerberg-political-ads.html}{more
than 250 employees signed an internal memo} arguing that free speech and
paid speech were different and that misinformation was harmful to all.
Facebook's position on political advertising is ``a threat to what FB
stands for,'' the employees wrote. ``We strongly object to this policy
as it stands.''

Days later, on Halloween, Mr. Zuckerberg led a regular weekly
question-and-answer session with employees. Near the end, someone
dressed in an enormous, inflatable Pikachu costume lumbered toward the
microphone and pressed the C.E.O. on his policy, according to three
people who were present.

Mr. Zuckerberg, now less worried than ever about trying to make everyone
happy, reiterated his position. When versions of the same question kept
popping up during the session, he held firm.

``This is not a democracy,'' he said.

\hypertarget{the-bleach-test}{%
\subsection{The bleach test}\label{the-bleach-test}}

``Not a democracy'' could also describe Facebook's nine-person board of
directors. Mr. Zuckerberg chairs the group, holds a majority of voting
shares and controls its dynamics.

The board isn't exactly a check on his power. Last year, Kenneth
Chenault, the former chief executive of American Express, suggested
creating an independent committee to scrutinize the company's challenges
and pose the sort of probing questions the board wasn't used to being
asked. The idea, previously
\href{https://www.wsj.com/articles/mark-zuckerberg-asserts-control-of-facebook-pushing-aside-dissenters-11588106984}{reported}
by The Journal, was swiftly voted down by Mr. Zuckerberg and others.

Other board disagreements, specifically around political advertising and
the spread of misinformation, always ended with Mr. Zuckerberg's point
of view winning out. In March, Mr. Chenault announced he would not stand
for re-election; soon, so did another director, Jeffrey Zients, who had
also challenged some of Mr. Zuckerberg's positions.

To replace them, Mr. Zuckerberg picked Drew Houston, the chief executive
of Dropbox, who was also a longtime friend and occasional
Ping-Pong\href{https://www.usatoday.com/story/tech/talkingtech/2017/03/09/ping-pong-night-out-tech-ceos-zuckerberg-houston-and-kalanick/98985578/}{partner},
and Peggy Alford, the former chief financial officer of the Chan
Zuckerberg Initiative. Three other appointees are set to join the board
this year, including executives from McKinsey and Co. and Estée Lauder.
The remaining three board members are a friendly bunch: Mr. Thiel and
Marc Andreessen, venture capitalists who are among Facebook's earliest
and most loyal investors, and Ms. Sandberg.

With his board issues in the rearview, Mr. Zuckerberg has been able to
devote more of his attention to the coronavirus. He started following
the disease early, fielding reports from experts including Tom Frieden,
the former director of the Centers for Disease Control. Mr. Zuckerberg
was advised not to trust preliminary reports out of China that the virus
was contained, or the baseless assurances from Mr. Trump that it would
not greatly affect the United States. On March 19, well ahead of many
states' stay-at-home orders, Mr. Zuckerberg broadcast a live video chat
with Dr. Anthony Fauci, the country's top infectious disease official,
on his personal Facebook page.

Image

Mr. Zuckerberg interviews Dr. Anthony Fauci. The pandemic has given the
Facebook chief a chance to lead 50,000 employees through a crisis that,
for once, is not of their own making.

Since the pandemic began, video and audio calls on Facebook Messenger
and WhatsApp have more than doubled. Group calls in some especially
hard-hit countries, like Italy, soared by 1,000 percent. Messaging
across Instagram and Facebook is up 50 percent across many of the
busiest countries. Homebound in Palo Alto, Mr. Zuckerberg has been
pushing his employees to build new products that people can use to
connect with one another. The latest is a rival to
\href{https://www.nytimes.com/2020/04/24/technology/zoom-rivals-virus-facebook-google.html}{Zoom},
which he hopes will corner the video-calling market.

``When the world changes quickly, people have new needs, and that means
that there are more new segments to build,'' he said on a
\href{https://www.fool.com/earnings/call-transcripts/2020/04/29/facebook-inc-fb-q1-2020-earnings-call-transcript.aspx}{conference
call} with investors in April. ``I have always believed that in times of
economic downturn, the right thing to do is to keep investing in
building the future.''

It remains to be seen what an increasingly visible Mr. Zuckerberg will
do when challenged by the powerful. In March, in an
\href{https://www.nytimes.com/2020/03/15/business/media/coronavirus-facebook-twitter-social-media.html}{interview}
with The New York Times, he said Facebook would not tolerate
``misinformation that has imminent risk of danger.'' He cited as an
example ``things like `You can cure this by drinking bleach.' I mean,
that's just in a different class.''

Days later, during a White House news conference, Mr. Trump wondered
aloud about an
``\href{https://www.nytimes.com/2020/04/24/us/politics/trump-inject-disinfectant-bleach-coronavirus.html}{injection
inside}'' of disinfectant. As poison control centers were flooded with
questions and the makers of Clorox and Lysol issued statements imploring
Americans not to ingest their caustic cleaners, Facebook wilted, and
across the platform, video of the comments went
\href{https://www.nytimes.com/2020/04/30/technology/trump-coronavirus-social-media.html}{swiftly
viral}.

Advertisement

\protect\hyperlink{after-bottom}{Continue reading the main story}

\hypertarget{site-index}{%
\subsection{Site Index}\label{site-index}}

\hypertarget{site-information-navigation}{%
\subsection{Site Information
Navigation}\label{site-information-navigation}}

\begin{itemize}
\tightlist
\item
  \href{https://help.nytimes.com/hc/en-us/articles/115014792127-Copyright-notice}{©~2020~The
  New York Times Company}
\end{itemize}

\begin{itemize}
\tightlist
\item
  \href{https://www.nytco.com/}{NYTCo}
\item
  \href{https://help.nytimes.com/hc/en-us/articles/115015385887-Contact-Us}{Contact
  Us}
\item
  \href{https://www.nytco.com/careers/}{Work with us}
\item
  \href{https://nytmediakit.com/}{Advertise}
\item
  \href{http://www.tbrandstudio.com/}{T Brand Studio}
\item
  \href{https://www.nytimes.com/privacy/cookie-policy\#how-do-i-manage-trackers}{Your
  Ad Choices}
\item
  \href{https://www.nytimes.com/privacy}{Privacy}
\item
  \href{https://help.nytimes.com/hc/en-us/articles/115014893428-Terms-of-service}{Terms
  of Service}
\item
  \href{https://help.nytimes.com/hc/en-us/articles/115014893968-Terms-of-sale}{Terms
  of Sale}
\item
  \href{https://spiderbites.nytimes.com}{Site Map}
\item
  \href{https://help.nytimes.com/hc/en-us}{Help}
\item
  \href{https://www.nytimes.com/subscription?campaignId=37WXW}{Subscriptions}
\end{itemize}
