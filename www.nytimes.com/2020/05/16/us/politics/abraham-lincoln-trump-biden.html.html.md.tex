Sections

SEARCH

\protect\hyperlink{site-content}{Skip to
content}\protect\hyperlink{site-index}{Skip to site index}

\href{https://www.nytimes.com/section/politics}{Politics}

\href{https://myaccount.nytimes.com/auth/login?response_type=cookie\&client_id=vi}{}

\href{https://www.nytimes.com/section/todayspaper}{Today's Paper}

\href{/section/politics}{Politics}\textbar{}Sorry, Abe Lincoln Is Not on
the Ballot

\url{https://nyti.ms/3bHlMkN}

\begin{itemize}
\item
\item
\item
\item
\item
\end{itemize}

\begin{itemize}
\item
  \href{https://www.nytimes.com/2020/07/31/us/elections/biden-vs-trump.html?action=click\&pgtype=Article\&state=default\&region=TOP_BANNER\&context=storylines_menu}{Election
  Updates}
\item
  \href{https://www.nytimes.com/article/biden-vice-president-2020.html?action=click\&pgtype=Article\&state=default\&region=TOP_BANNER\&context=storylines_menu}{Biden's
  V.P. Search}
\item
  \href{https://www.nytimes.com/interactive/2020/07/24/us/politics/trump-biden-campaign-donors.html?action=click\&pgtype=Article\&state=default\&region=TOP_BANNER\&context=storylines_menu}{Map
  of Donations}
\item
  \href{https://www.nytimes.com/interactive/2020/us/elections/delegate-count-primary-results.html?action=click\&pgtype=Article\&state=default\&region=TOP_BANNER\&context=storylines_menu}{Delegate
  Count}
\item
  \href{https://www.nytimes.com/interactive/2019/us/politics/2020-presidential-candidates.html?action=click\&pgtype=Article\&state=default\&region=TOP_BANNER\&context=storylines_menu}{The
  Candidates}
\item
  \href{https://www.nytimes.com/newsletters/politics?action=click\&pgtype=Article\&state=default\&region=TOP_BANNER\&context=storylines_menu}{Politics
  Newsletter}
\end{itemize}

Advertisement

\protect\hyperlink{after-top}{Continue reading the main story}

Supported by

\protect\hyperlink{after-sponsor}{Continue reading the main story}

Political Memo

\hypertarget{sorry-abe-lincoln-is-not-on-the-ballot}{%
\section{Sorry, Abe Lincoln Is Not on the
Ballot}\label{sorry-abe-lincoln-is-not-on-the-ballot}}

But the 16th president is showing up quite a bit in the 2020 race. He
remains uncommitted.

\includegraphics{https://static01.nyt.com/images/2020/05/16/us/politics/16lincoln-memo/merlin_172153800_8eb4cc6e-08f7-46de-ad53-e30360d73149-articleLarge.jpg?quality=75\&auto=webp\&disable=upscale}

\href{https://www.nytimes.com/by/sarah-lyall}{\includegraphics{https://static01.nyt.com/images/2018/02/20/multimedia/author-sarah-lyall/author-sarah-lyall-thumbLarge.jpg}}\href{https://www.nytimes.com/by/matt-flegenheimer}{\includegraphics{https://static01.nyt.com/images/2018/10/02/multimedia/author-matt-flegenheimer/author-matt-flegenheimer-thumbLarge.png}}

By \href{https://www.nytimes.com/by/sarah-lyall}{Sarah Lyall} and
\href{https://www.nytimes.com/by/matt-flegenheimer}{Matt Flegenheimer}

\begin{itemize}
\item
  May 16, 2020
\item
  \begin{itemize}
  \item
  \item
  \item
  \item
  \item
  \end{itemize}
\end{itemize}

Abraham Lincoln may have died 155 years ago, but everyone still wants
his endorsement.

He is the star of a new campaign ad for Joseph R. Biden Jr., the
presumptive Democratic nominee, that juxtaposes classic Lincoln phrases
(``a house divided against itself cannot stand'') with clips of a
peevish President Trump doing things like directing his vice president
to ignore governors of virus-ravaged states ``if they don't treat you
right.''

He is the namesake for \href{https://lincolnproject.us/}{The Lincoln
Project}, a venture from a group of dismayed Republicans who are seeking
to ensure Mr. Trump's defeat in November. (``We look to Lincoln as our
guide and inspiration,'' they wrote of their effort.)

And from the other side of the argument, there was Mr. Trump himself
earlier this month at the magisterial Lincoln Memorial, that great
secular temple to American unity and hope, musing to a Fox News audience
about how he sees Lincoln in himself.

``Look, I am greeted with a hostile press the likes of which no
president has ever seen,'' Mr. Trump said, gesturing to the towering
figure looming behind him. ``The closest would be that gentleman right
up there. They always said Lincoln --- nobody got treated worse than
Lincoln. I believe I am treated worse.''

Abraham Lincoln does not approve this message. He has not approved of
anything for some time.

But if Mr. Trump's assessment is a matter of debate, especially when you
consider what ultimately happened to Lincoln, the president is also not
the first politician to promote himself as a Lincoln-esque figure,
whatever that means.

``This has been going on since Woodrow Wilson and FDR decided that both
Democrats and Republicans could fight for the mantle of Lincoln,'' said
Harold Holzer, a Lincoln scholar and the director of Roosevelt House
Public Policy Institute at Hunter College. ``He is the universal totem
for presidents.''

In an age where even facts are subject to dispute, the superiority of
Lincoln is a near-universal truth, a natural law of the land. Because he
was the first Republican president, Republicans have always considered
him a party standard-bearer. But in 1929, Franklin D. Roosevelt, then
the governor of New York, declared that the time had come for ``us
Democrats to claim Lincoln as one of our own,'' and since then virtually
every president, and many would-be presidents, have looked to the 16th
president for guidance, inspiration and cover, trying to out-Lincoln
their rivals.

\hypertarget{latest-updates-2020-election}{%
\section{\texorpdfstring{\href{https://www.nytimes.com/2020/07/31/us/elections/biden-vs-trump.html?action=click\&pgtype=Article\&state=default\&region=MAIN_CONTENT_1\&context=storylines_live_updates}{Latest
Updates: 2020
Election}}{Latest Updates: 2020 Election}}\label{latest-updates-2020-election}}

Updated 2020-08-01T01:26:45.732Z

\begin{itemize}
\tightlist
\item
  \href{https://www.nytimes.com/2020/07/31/us/elections/biden-vs-trump.html?action=click\&pgtype=Article\&state=default\&region=MAIN_CONTENT_1\&context=storylines_live_updates\#link-29fdff45}{Kamala
  Harris, a top vice-presidential contender, confronts double
  standards.}
\item
  \href{https://www.nytimes.com/2020/07/31/us/elections/biden-vs-trump.html?action=click\&pgtype=Article\&state=default\&region=MAIN_CONTENT_1\&context=storylines_live_updates\#link-13ec3d9c}{Karen
  Bass and Susan Rice are rising on Biden's vice-presidential
  shortlist.}
\item
  \href{https://www.nytimes.com/2020/07/31/us/elections/biden-vs-trump.html?action=click\&pgtype=Article\&state=default\&region=MAIN_CONTENT_1\&context=storylines_live_updates\#link-49e9a016}{Trump
  says Russian bounties to kill U.S. troops `never took place.'}
\end{itemize}

\href{https://www.nytimes.com/2020/07/31/us/elections/biden-vs-trump.html?action=click\&pgtype=Article\&state=default\&region=MAIN_CONTENT_1\&context=storylines_live_updates}{See
more updates}

``Part of it is that he is considered the greatest president ever,''
said the presidential historian Doris Kearns Goodwin, author of ``Team
of Rivals: The Political Genius of Abraham Lincoln,'' speaking to his
enduring appeal. ``And the thing about Lincoln is that, even during his
presidency, he was able to understand all sides of an issue.''

Presidents have expressed their admiration for Lincoln in various ways
over the years. Theodore Roosevelt wore to his first inauguration a ring
enclosing a lock of Lincoln's hair. Woodrow Wilson
\href{https://www.uis.edu/wepner/wp-content/uploads/sites/97/2013/04/2010skowronek.pdf}{wrote
glowingly} of Lincoln's judicious temperament, his search for common
ground, his desire to emphasize the ``community of interests'' between
people at odds with each other.

Barack Obama announced his campaign for president on the steps of the
Old State Capitol building in Springfield, Ill., where Lincoln delivered
his ``House Divided'' speech, and was sworn in as president using the
Lincoln Bible.

Bill Clinton revered Lincoln and quoted him all the time. Ronald Reagan
\href{https://www.nytimes.com/1992/08/19/news/republicans-in-houston-for-the-record-reagan-put-words-in-lincoln-s-mouth.html}{misquoted
him} at the Republican National Convention in 1992. George H.W. Bush's
\href{https://www.whitehousehistory.org/photos/george-h-w-bush}{official
portrait}shows him standing in front of
``\href{https://www.whitehousehistory.org/photos/treasures-of-the-white-house-the-peacemakers}{The
Peacemakers,''} the White House painting of Lincoln and his generals in
the waning days of the Civil War.

In recent weeks, Gov. Andrew M. Cuomo of New York has repeatedly invoked
Lincoln at his daily coronavirus briefings, suggesting that the former
president would appreciate the governor's pledge to level with his
constituents, despite the grim news. ``Lincoln, big believer in the
American people,'' Mr. Cuomo
\href{https://www.governor.ny.gov/news/video-audio-photos-rush-transcript-amid-ongoing-covid-19-pandemic-governor-cuomo-announces-21}{noted
this month}.

Mr. Cuomo's father, Mario M. Cuomo, another three-term New York
governor, was also a Lincoln fan,
\href{https://www.nytimes.com/2004/07/18/books/chapters/why-lincoln-matters.html}{producing
a book} in 2004, with Mr. Holzer's historical guidance, about his
continuing relevance (``Why Lincoln Matters: Today More Than Ever'').

``Lincoln is always available for everyone's use, and there's a long
history of that,'' said David W. Blight, a professor of history at Yale.
``Do you want a healer? There's the healer Lincoln, which is the one
Barack Obama appropriated. Or you could have Lincoln who is a military
commander in chief, who would do anything to win that war, who would
twist civil liberties inside out. There's also the ambiguous Lincoln,
part of what makes him so usable by everyone.''

Though Mr. Trump has long been criticized for having little sense of (or
interest in) history, he and his team have at times spoken admiringly of
presidents who would not necessarily figure on everyone's Top 10 list.

He has hung a
\href{https://www.nytimes.com/2017/03/15/us/politics/trump-andrew-jackson-grave.html}{portrait
of Andrew Jackson in the Oval Office}, aligning himself with Jacksonian
populism and declaring him ``an amazing figure'' in American history ---
looking beyond such episodes as the infamously brutal forced relocation
of Native Americans.

As Mr. Trump confronted an impeachment trial in January, Vice President
Mike Pence
\href{https://www.wsj.com/articles/a-partisan-impeachment-a-profile-in-courage-11579220188}{wrote}
in The Wall Street Journal about the ``partisan impeachment'' of Andrew
Johnson, widely considered one of the worst American presidents ever
but, in Mr. Pence's account, a victim of unfair political persecution.

And most recently, discussing the Russia investigation, Mr. Trump
\href{https://www.nytimes.com/2020/05/08/us/politics/trump-barr-michael-flynn.html}{suggested}
he was a student of Richard Nixon (and of history). ``I learned a lot
from Richard Nixon --- don't fire people,'' Mr. Trump said this month.
``I learned a lot. I study history.''

The president conceded ``one big difference'' between them, then offered
two: ``No. 1, he may have been guilty. And No. 2, he had tapes all over
the place.'' Mr. Trump declared himself an innocent and tapeless man.

Many supporters apparently agree with Mr. Trump's
\href{https://people.com/politics/jon-voight-argues-trump-greatest-president-since-lincoln/}{self-analysis}
that he is ``more presidential'' than anyone except for ``the late,
great Abraham Lincoln.'' At a
\href{https://www.youtube.com/watch?v=syKkvZmKbQc}{White House event} in
February to promote Mr. Trump's work with African-Americans, two guests
called him the best president since Lincoln. Last year, so did the actor
Jon Voight. ``Let us stand up for this truth,'' he said
\href{https://people.com/politics/jon-voight-argues-trump-greatest-president-since-lincoln/}{in
a video} posted on Twitter, ``that President Trump is the greatest
president since Abraham Lincoln.''

This is not a universal view. Mr. Blight, the Lincoln scholar, said he
watched Mr. Trump at the Fox event at the Lincoln Memorial with
increasing despair. The low point came, he said, when Bret Baier from
Fox asked the president how he, like Lincoln, could heal a divided
nation.

Mr. Trump didn't hesitate. ``We're winning bigger than we've ever won
before, Bret,'' he said. ``And I think that winning, ultimately, is
going to bring this country together.'' (Then he segued into a verbal
assault on the Democrats and the impeachment ``hoax.'')

Mr. Blight could not help thinking of Lincoln's Second Inaugural
Address, delivered soon before the end of the Civil War and steeped in
melancholy. In the address, Lincoln emphasized unity rather than
division and spoke of going forward with ``malice toward none'' and
``charity for all.''

``Lincoln could have stood up and gloated,'' Mr. Blight said. ``He could
have said, `Hey, we won this war, look at us, isn't it amazing.' But
there isn't a moment of boasting in it.''

Steve Schmidt, a longtime Republican strategist behind the Lincoln
Project, said he and his colleagues had chosen the name because Mr.
Trump is Mr. Lincoln's ``antithesis.''

``He's mean, he's cruel, he's vile, he lacks foresight, he lacks vision,
he lacks a capacity for forgiveness and decency,'' Mr. Schmidt said of
the sitting president.

Mr. Trump has been, well, unforgiving toward the Republican rebels.

Appraising the Lincoln Project as a team of ``LOSERS'' in a recent
post-midnight
\href{https://twitter.com/realDonaldTrump/status/1257532114666508291}{tweet},
the president insisted that he has an admirer from the beyond:

``Abe Lincoln, Republican, is all smiles!''

\hypertarget{our-2020-election-guide}{%
\section{Our 2020 Election Guide}\label{our-2020-election-guide}}

Updated July 31, 2020

\begin{itemize}
\item
  \begin{center}\rule{0.5\linewidth}{\linethickness}\end{center}

  \hypertarget{the-latest}{%
  \subsection{The Latest}\label{the-latest}}

  \begin{itemize}
  \tightlist
  \item
    President Trump's assault on the Postal Service is intersecting with
    his attacks on mail-in voting.
    \href{https://www.nytimes.com/2020/07/31/us/politics/trump-usps-mail-delays.html?action=click\&pgtype=Article\&state=default\&region=BELOW_MAIN_CONTENT\&context=storylines_guide}{Voting
    rights groups say it is a recipe for disaster.}
  \end{itemize}
\item
  \begin{center}\rule{0.5\linewidth}{\linethickness}\end{center}

  \hypertarget{bidens-vp-search}{%
  \subsection{Biden's V.P. Search}\label{bidens-vp-search}}

  \begin{itemize}
  \tightlist
  \item
    \href{https://www.nytimes.com/article/biden-vice-president-2020.html?action=click\&pgtype=Article\&state=default\&region=BELOW_MAIN_CONTENT\&context=storylines_guide}{Here
    are 13 women} who have been under consideration to be Joe Biden's
    running mate, and why each might be chosen --- and might not be.
  \end{itemize}
\item
  \begin{center}\rule{0.5\linewidth}{\linethickness}\end{center}

  \hypertarget{keep-up-with-our-coverage}{%
  \subsection{Keep Up With Our
  Coverage}\label{keep-up-with-our-coverage}}

  \begin{itemize}
  \tightlist
  \item
    Get an
    \href{https://www.nytimes.com/newsletters/politics?action=click\&pgtype=Article\&state=default\&region=BELOW_MAIN_CONTENT\&context=storylines_guide}{email}
    recapping the day's news
  \end{itemize}

  \begin{itemize}
  \tightlist
  \item
    Download our mobile app on
    \href{https://apps.apple.com/us/app/nytimes/id284862083?ls=1\&mat_click_id=5c79ae7455014fd1bd66b5610c05b8f2-20191112-16948\&referrer=mat_click_id\%3D5c79ae7455014fd1bd66b5610c05b8f2-20191112-16948\%26link_click_id\%3D722930677036718082}{iOS}
    and
    \href{http://a.localytics.com/android?id=com.nytimes.android\&referrer=utm_source\%3Dother_nyt_mobile_web\%26utm_medium\%3DWeb\%2520page\%26utm_term\%3DGeneral\%2520Mobile\%2520Page\%26utm_campaign\%3DNYT\%2520Mobile\%2520General\%2520Page}{Android}
    and turn on Breaking News and Politics alerts
  \end{itemize}
\end{itemize}

Advertisement

\protect\hyperlink{after-bottom}{Continue reading the main story}

\hypertarget{site-index}{%
\subsection{Site Index}\label{site-index}}

\hypertarget{site-information-navigation}{%
\subsection{Site Information
Navigation}\label{site-information-navigation}}

\begin{itemize}
\tightlist
\item
  \href{https://help.nytimes.com/hc/en-us/articles/115014792127-Copyright-notice}{©~2020~The
  New York Times Company}
\end{itemize}

\begin{itemize}
\tightlist
\item
  \href{https://www.nytco.com/}{NYTCo}
\item
  \href{https://help.nytimes.com/hc/en-us/articles/115015385887-Contact-Us}{Contact
  Us}
\item
  \href{https://www.nytco.com/careers/}{Work with us}
\item
  \href{https://nytmediakit.com/}{Advertise}
\item
  \href{http://www.tbrandstudio.com/}{T Brand Studio}
\item
  \href{https://www.nytimes.com/privacy/cookie-policy\#how-do-i-manage-trackers}{Your
  Ad Choices}
\item
  \href{https://www.nytimes.com/privacy}{Privacy}
\item
  \href{https://help.nytimes.com/hc/en-us/articles/115014893428-Terms-of-service}{Terms
  of Service}
\item
  \href{https://help.nytimes.com/hc/en-us/articles/115014893968-Terms-of-sale}{Terms
  of Sale}
\item
  \href{https://spiderbites.nytimes.com}{Site Map}
\item
  \href{https://help.nytimes.com/hc/en-us}{Help}
\item
  \href{https://www.nytimes.com/subscription?campaignId=37WXW}{Subscriptions}
\end{itemize}
