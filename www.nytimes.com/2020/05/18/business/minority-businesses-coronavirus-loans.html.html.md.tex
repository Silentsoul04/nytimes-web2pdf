Sections

SEARCH

\protect\hyperlink{site-content}{Skip to
content}\protect\hyperlink{site-index}{Skip to site index}

\href{https://www.nytimes.com/section/business}{Business}

\href{https://myaccount.nytimes.com/auth/login?response_type=cookie\&client_id=vi}{}

\href{https://www.nytimes.com/section/todayspaper}{Today's Paper}

\href{/section/business}{Business}\textbar{}Few Minority-Owned
Businesses Got Relief Loans They Asked For

\url{https://nyti.ms/2WJoMsZ}

\begin{itemize}
\item
\item
\item
\item
\item
\end{itemize}

\href{https://www.nytimes.com/news-event/coronavirus?action=click\&pgtype=Article\&state=default\&region=TOP_BANNER\&context=storylines_menu}{The
Coronavirus Outbreak}

\begin{itemize}
\tightlist
\item
  live\href{https://www.nytimes.com/2020/08/04/world/coronavirus-covid-19.html?action=click\&pgtype=Article\&state=default\&region=TOP_BANNER\&context=storylines_menu}{Latest
  Updates}
\item
  \href{https://www.nytimes.com/interactive/2020/us/coronavirus-us-cases.html?action=click\&pgtype=Article\&state=default\&region=TOP_BANNER\&context=storylines_menu}{Maps
  and Cases}
\item
  \href{https://www.nytimes.com/interactive/2020/science/coronavirus-vaccine-tracker.html?action=click\&pgtype=Article\&state=default\&region=TOP_BANNER\&context=storylines_menu}{Vaccine
  Tracker}
\item
  \href{https://www.nytimes.com/2020/08/02/us/covid-college-reopening.html?action=click\&pgtype=Article\&state=default\&region=TOP_BANNER\&context=storylines_menu}{College
  Reopening}
\item
  \href{https://www.nytimes.com/live/2020/08/03/business/stock-market-today-coronavirus?action=click\&pgtype=Article\&state=default\&region=TOP_BANNER\&context=storylines_menu}{Economy}
\end{itemize}

Advertisement

\protect\hyperlink{after-top}{Continue reading the main story}

Supported by

\protect\hyperlink{after-sponsor}{Continue reading the main story}

\hypertarget{few-minority-owned-businesses-got-relief-loans-they-asked-for}{%
\section{Few Minority-Owned Businesses Got Relief Loans They Asked
For}\label{few-minority-owned-businesses-got-relief-loans-they-asked-for}}

The Paycheck Protection Program and other federal efforts largely
neglected them, a survey commissioned by two equal-rights groups found.
Nearly half the respondents expect to close permanently.

\includegraphics{https://static01.nyt.com/images/2020/05/18/business/18virus-minoritybiz/merlin_172392294_3c259d1e-843f-49e1-9853-fb0f4040eb31-articleLarge.jpg?quality=75\&auto=webp\&disable=upscale}

\href{https://www.nytimes.com/by/emily-flitter}{\includegraphics{https://static01.nyt.com/images/2019/06/19/reader-center/author-emily-flitter/author-emily-flitter-thumbLarge.png}}

By \href{https://www.nytimes.com/by/emily-flitter}{Emily Flitter}

\begin{itemize}
\item
  May 18, 2020
\item
  \begin{itemize}
  \item
  \item
  \item
  \item
  \item
  \end{itemize}
\end{itemize}

Black and Latino business owners are struggling to get pandemic
assistance under the
\href{https://www.nytimes.com/2020/06/10/business/Small-business-loans-ppp.html}{Paycheck
Protection Program} and other federal aid efforts, a new survey has
found, and many say they are on the brink of closing permanently.

The survey, conducted by the Global Strategy Group for two equal-rights
organizations, \href{https://colorofchange.org/}{Color of Change} and
\href{https://www.unidosus.org/}{UnidosUS}, included interviews with 500
business owners and 1,200 workers from April 30 to last Monday. Just 12
percent of the owners who applied for aid from the Small Business
Administration --- most of them seeking loans in the \$650 billion
paycheck program --- reported receiving what they had asked for, while
26 percent said they had received only a fraction of what they had
requested. Nearly half of all owners said they anticipated having to
permanently close in the next six months.

By comparison, in a survey of small businesses by the Census Bureau from
April 26 to May 2, three-quarters said they had asked for a loan and 38
percent of them said they had received one.

Rashad Robinson, the president of Color of Change, said the new survey
showed that ``if we don't get policies to protect these communities, we
will lose a generation of black and brown businesses, which will have
deep impacts on our entire country's economy.''

Two-thirds of the respondents sought loans of under \$50,000 through the
government's aid program. Nearly half said they had to lay off at least
some employees.

The results suggest that the historically weak relationships that
minority business owners have with banks are making it harder for them
to tap into the aid program, which makes loans that become grants if
borrowers spend the money paying employees and rent and utility bills.
Many banks considered applications only from existing customers; some,
like Bank of America, even
\href{https://www.nytimes.com/2020/04/10/business/minority-business-coronavirus-loans.html}{turned
away people} who had opened credit cards through other lenders.

The program was the first time some black and Latino business owners had
ever sought a bank loan.

Equal-rights advocates and some lawmakers are pushing to get more help
for minority business owners built into the government's response to the
coronavirus pandemic, and the second round of funding for the loan
program set aside \$60 billion for small and rural banks and nonprofit
lenders, which often do more work in minority communities than large
banks do.

Mr. Robinson said his group was pushing lawmakers to come up with other
ways to transmit aid to business owners, such as direct payments to
businesses' employees through payroll processors or other means.

Advertisement

\protect\hyperlink{after-bottom}{Continue reading the main story}

\hypertarget{site-index}{%
\subsection{Site Index}\label{site-index}}

\hypertarget{site-information-navigation}{%
\subsection{Site Information
Navigation}\label{site-information-navigation}}

\begin{itemize}
\tightlist
\item
  \href{https://help.nytimes.com/hc/en-us/articles/115014792127-Copyright-notice}{©~2020~The
  New York Times Company}
\end{itemize}

\begin{itemize}
\tightlist
\item
  \href{https://www.nytco.com/}{NYTCo}
\item
  \href{https://help.nytimes.com/hc/en-us/articles/115015385887-Contact-Us}{Contact
  Us}
\item
  \href{https://www.nytco.com/careers/}{Work with us}
\item
  \href{https://nytmediakit.com/}{Advertise}
\item
  \href{http://www.tbrandstudio.com/}{T Brand Studio}
\item
  \href{https://www.nytimes.com/privacy/cookie-policy\#how-do-i-manage-trackers}{Your
  Ad Choices}
\item
  \href{https://www.nytimes.com/privacy}{Privacy}
\item
  \href{https://help.nytimes.com/hc/en-us/articles/115014893428-Terms-of-service}{Terms
  of Service}
\item
  \href{https://help.nytimes.com/hc/en-us/articles/115014893968-Terms-of-sale}{Terms
  of Sale}
\item
  \href{https://spiderbites.nytimes.com}{Site Map}
\item
  \href{https://help.nytimes.com/hc/en-us}{Help}
\item
  \href{https://www.nytimes.com/subscription?campaignId=37WXW}{Subscriptions}
\end{itemize}
