Sections

SEARCH

\protect\hyperlink{site-content}{Skip to
content}\protect\hyperlink{site-index}{Skip to site index}

\href{https://www.nytimes.com/section/politics}{Politics}

\href{https://myaccount.nytimes.com/auth/login?response_type=cookie\&client_id=vi}{}

\href{https://www.nytimes.com/section/todayspaper}{Today's Paper}

\href{/section/politics}{Politics}\textbar{}State Dept. Investigator
Fired by Trump Had Examined Weapons Sales to Saudis and Emiratis

\url{https://nyti.ms/2Tip2gj}

\begin{itemize}
\item
\item
\item
\item
\item
\item
\end{itemize}

Advertisement

\protect\hyperlink{after-top}{Continue reading the main story}

Supported by

\protect\hyperlink{after-sponsor}{Continue reading the main story}

\hypertarget{state-dept-investigator-fired-by-trump-had-examined-weapons-sales-to-saudis-and-emiratis}{%
\section{State Dept. Investigator Fired by Trump Had Examined Weapons
Sales to Saudis and
Emiratis}\label{state-dept-investigator-fired-by-trump-had-examined-weapons-sales-to-saudis-and-emiratis}}

A Democratic House committee chairman said the investigation into arms
sales might have been ``another reason'' for the firing of the inspector
general, Steve A. Linick.

\includegraphics{https://static01.nyt.com/images/2020/05/18/us/politics/18dc-pompeo1/merlin_171645006_9bb54ee2-1cff-4a13-962e-a25ac2182b81-articleLarge.jpg?quality=75\&auto=webp\&disable=upscale}

\href{https://www.nytimes.com/by/edward-wong}{\includegraphics{https://static01.nyt.com/images/2018/09/24/multimedia/author-edward-wong/author-edward-wong-thumbLarge-v5.png}}\href{https://www.nytimes.com/by/david-e-sanger}{\includegraphics{https://static01.nyt.com/images/2018/10/03/multimedia/author-david-e-sanger/author-david-e-sanger-thumbLarge.png}}

By \href{https://www.nytimes.com/by/edward-wong}{Edward Wong} and
\href{https://www.nytimes.com/by/david-e-sanger}{David E. Sanger}

\begin{itemize}
\item
  May 18, 2020
\item
  \begin{itemize}
  \item
  \item
  \item
  \item
  \item
  \item
  \end{itemize}
\end{itemize}

WASHINGTON --- The State Department inspector general
\href{https://www.nytimes.com/2020/05/16/us/politics/trump-state-dept-inspector-general.html}{fired
by President Trump} on Friday was in the final stages of an
investigation into whether the administration had unlawfully declared an
``emergency'' last year to allow the resumption of
\href{https://www.nytimes.com/2020/05/16/us/arms-deals-raytheon-yemen.html}{weapons
sales} to Saudi Arabia and the United Arab Emirates for their air war in
Yemen.

Employees from the office of the inspector general, Steve A. Linick,
presented preliminary findings to senior State Department officials in
early March, before the coronavirus forced lockdowns across the United
States. But it was not clear whether that investigation, or others that
Mr. Linick had underway, led to his dismissal.

Mr. Trump, speaking about the latest in his series of firings of
inspectors general around the government, said on Monday of Mr. Linick:
``I don't know him. Never heard of him. But I was asked by the State
Department, by Mike'' to terminate Mr. Linick. He apparently was
referring to a recommendation he received from Secretary of State Mike
Pompeo.

``I have the absolute right as president to terminate,'' Mr. Trump
added. ``I said, `Who appointed him?' and they say, `President Obama.' I
said, `Look I'll terminate him.'''

\includegraphics{https://static01.nyt.com/images/2020/05/18/us/politics/18vid-trump-pompeo-inspector-general1/18vid-trump-pompeo-inspector-general1-videoSixteenByNine3000.jpg}

The investigation into how Mr. Pompeo moved to end a congressional hold
on arms sales to the Saudis was prompted in part by demands from the
chairman of the House Foreign Affairs Committee, Representative Eliot L.
Engel of New York, who said on Monday that the subsequent investigation
might have been ``another reason'' for the firing of Mr. Linick. The
White House announced the firing Friday night under a provision that
requires 30 days' notice to Congress before removing an inspector
general.

Democratic leaders in Congress and several Republican lawmakers said on
Monday that Mr. Trump had not given sufficient justification for the
firing and that they wanted answers during the 30-day review period.

Mr. Linick's office, which has hundreds of employees assigned to look
into fraud and waste at the State Department, was also examining the
potential
\href{https://www.nytimes.com/2020/05/17/us/politics/pompeo-inspector-general-steve-linick.html}{misuse
by Mr. Pompeo of a political appointee to do personal errands} for him
and his wife, Susan Pompeo.

The inspector general's office conducts multiple, simultaneous
investigations into the activities of the State Department and its
officials.

``We don't have the full picture yet, but it's troubling that Secretary
Pompeo wanted Mr. Linick pushed out before this work could be
completed,'' Mr. Engel said of the arms sale inquiry.

The State Department did not respond to a request for comment. Mr.
Pompeo
\href{https://www.washingtonpost.com/national-security/pompeo-says-he-didnt-know-fired-inspector-general-was-investigating-him/2020/05/18/3ab08dca-9923-11ea-b60c-3be060a4f8e1_story.html}{said
in a telephone interview} with The Washington Post that he had
recommended to Mr. Trump that Mr. Linick be fired because Mr. Linick was
``undermining'' the department's mission. Mr. Pompeo did not give
details.

He also said his recommendation to fire Mr. Linick could not have been
an act of retaliation to end an investigation because he had not been
briefed on any inquiries.

However, top department officials had clearly received briefings from
Mr. Linick's office and been asked to comply with investigations.

Mr. Linick is widely seen as competent, though sometimes reluctant to
wade into the most politically charged issues.

Nonetheless, he issued a harsh report in 2016 on the use of a private
email server by Hillary Clinton, who served as Mr. Obama's secretary of
state, and played a minor role in the impeachment inquiry against Mr.
Trump last fall. He issued two reports last year that criticized
political appointees at the State Department, some of whom work closely
with Mr. Pompeo.

Mr. Trump has appointed Ambassador Stephen J. Akard, the director of the
Office of Foreign Missions, for the role of acting inspector general.
Mr. Akard, an associate of Vice President Mike Pence, failed to get
congressional support for a top State Department job under Mr. Pompeo's
predecessor but was eventually confirmed for the lesser post at the
foreign missions office.

The decision to resume lethal
\href{https://www.nytimes.com/2019/06/07/us/saudi-arabia-arms-sales-raytheon.html}{aid
to the Saudis and Emiratis} was a major initiative undertaken by Mr.
Pompeo and Mr. Trump, who often discussed the importance of the weapons
sales with officers of Raytheon, the Massachusetts-based defense
contractor that lobbied heavily to get a 2017 suspension of sales
lifted. Congress had imposed the suspension because of a political rift
among Gulf Arab nations driven by the Saudis and because of discoveries
that bomb fragments traced to Raytheon by investigators were linked to a
series of Saudi bombings that killed civilians, including children.

\includegraphics{https://static01.nyt.com/images/2020/05/18/us/politics/18dc-pompeo2/merlin_172054212_974f6070-527a-4661-b3b2-1cc78fa5f5d0-articleLarge.jpg?quality=75\&auto=webp\&disable=upscale}

Mr. Trump had pushed to resume the sales in 2018, justifying it as a
jobs issue.

``I want Boeing and I want Lockheed and I want Raytheon to take those
orders and to hire lots of people to make that incredible equipment,''
he said.

But the effort to restart the sales was delayed by the killing of Jamal
Khashoggi, the Saudi dissident, Washington Post columnist and American
resident. His death, and the suspected role of the Saudi leadership in
ordering the killing, led to calls for a full end to military aid to
Saudi Arabia and the United Arab Emirates.

Mr. Pompeo broke the logjam a year ago, declaring an ``emergency'' over
Iran's activities in the Middle East that enabled him to sidestep the
congressional ban and approve restarting the sales. That started the
resumption of more normal exchanges with the Saudi government, as the
Trump administration tried to move past Mr. Khashoggi's killing. Saudi
Arabia and Iran are archrivals in the region.

In June, after congressional hearings with State Department officials
into the rationale for declaring an emergency over Iran, Mr. Engel sent
a letter to Mr. Linick asking him to open an investigation. Mr. Engel's
office then tracked the investigation sporadically once it had begun, a
Democratic aide said. The office learned by early spring that Mr. Linick
had conveyed preliminary findings to the State Department.

This past weekend, after Mr. Trump notified Congress of the firing of
Mr. Linick, Mr. Engel's office learned more details of the circumstances
around the arms sale investigation, leading Mr. Engel to ask whether the
inquiry might have contributed to the sudden move against Mr. Linick by
Mr. Pompeo and Mr. Trump.
\href{https://www.washingtonpost.com/opinions/2020/05/18/trumps-purge-just-got-much-more-corrupt-heres-whats-coming-next/}{The
Washington Post first reported} on Mr. Engel's concerns on Monday.

The separate inquiry into the possible misuse of a political appointee
to run personal errands was still a potential factor, and there might be
other motivations for the firing that remain unknown, an aide said.

Aaron David Miller, a former American official on Middle East policy who
is now at the Carnegie Endowment for International Peace, said that a
year ago, ``there was no credible emergency nor any real urgency for
invoking an Iran emergency declaration for lethal arms sales to the
Saudis other than the administration's desire to please Saudi Arabia.''

He added that American officials ``don't want anyone digging around in
the triangular relationship between the administration, Raytheon and
Saudi because somebody crossed the line.''

Mr. Trump and Mr. Pompeo were aware of the sensitivities around trying
to bypass the congressional hold on the arms sales. Mr. Pompeo
\href{https://www.nytimes.com/2019/05/24/world/middleeast/trump-troop-increase-middle-east-iran.html}{made
the announcement} of the ``emergency'' declaration over Iran on the
Friday afternoon before Memorial Day weekend last year, a common move by
government officials to avoid immediate questions from Congress and
extensive news coverage. The administration also announced it was
sending 1,500 more troops to the Middle East.

The move was aimed at allowing American companies to
\href{https://www.nytimes.com/2019/05/23/us/politics/trump-saudi-arabia-arms-sales.html}{sell
\$8.1 billion worth of munitions in 22 pending transfers} mainly to
Saudi Arabia and the U.A.E. At the time, a person briefed on the
decision said, a part of the arrangement would involve a transfer of
munitions from the U.A.E. to Jordan that had nothing to do with Iran.

Mr. Pompeo had pushed aggressively for the sales, over the objections of
career Foreign Service officers and lawmakers.

After the announcement of the ``emergency'' on May 24, lawmakers
pointedly asked why, if there was such a crisis, Mr. Pompeo and Patrick
Shanahan, then the acting defense secretary, had not briefed them on the
situation and on the need to push through arms sales in a closed-door
discussion on Iran just three days earlier.

In June, lawmakers called top State Department officials to testify
about the decision. Some of their questions focused on the roles played
by Charles Faulkner, a former Raytheon lobbyist who worked in the State
Department's legislative affairs bureau, and Marik String, a former
deputy assistant secretary in the political-military affairs bureau who
became a top department legal adviser in late May.

In a
\href{https://www.nytimes.com/2019/06/12/us/politics/arms-sales-saudi-arabia.html}{contentious
hearing} on June 12, lawmakers pressed
\href{https://www.state.gov/assistant-secretary-of-state-for-political-military-affairs-r-clarke-cooper-travels-to-singapore-india-and-sri-lanka/}{R.
Clarke Cooper}, the assistant secretary of state in the
political-military affairs bureau, on the move. Mr. Cooper argued that a
continued hold on the sales would cede commercial advantages to Russia
and China. One lawmaker asked whether Jared Kushner, Mr. Trump's
son-in-law and a Middle East adviser with close ties to Crown Prince
Mohammed bin Salman of Saudi Arabia, had weighed in on the decision. Mr.
Cooper demurred at first, then said no.

Michael LaForgia contributed reporting.

Advertisement

\protect\hyperlink{after-bottom}{Continue reading the main story}

\hypertarget{site-index}{%
\subsection{Site Index}\label{site-index}}

\hypertarget{site-information-navigation}{%
\subsection{Site Information
Navigation}\label{site-information-navigation}}

\begin{itemize}
\tightlist
\item
  \href{https://help.nytimes.com/hc/en-us/articles/115014792127-Copyright-notice}{©~2020~The
  New York Times Company}
\end{itemize}

\begin{itemize}
\tightlist
\item
  \href{https://www.nytco.com/}{NYTCo}
\item
  \href{https://help.nytimes.com/hc/en-us/articles/115015385887-Contact-Us}{Contact
  Us}
\item
  \href{https://www.nytco.com/careers/}{Work with us}
\item
  \href{https://nytmediakit.com/}{Advertise}
\item
  \href{http://www.tbrandstudio.com/}{T Brand Studio}
\item
  \href{https://www.nytimes.com/privacy/cookie-policy\#how-do-i-manage-trackers}{Your
  Ad Choices}
\item
  \href{https://www.nytimes.com/privacy}{Privacy}
\item
  \href{https://help.nytimes.com/hc/en-us/articles/115014893428-Terms-of-service}{Terms
  of Service}
\item
  \href{https://help.nytimes.com/hc/en-us/articles/115014893968-Terms-of-sale}{Terms
  of Sale}
\item
  \href{https://spiderbites.nytimes.com}{Site Map}
\item
  \href{https://help.nytimes.com/hc/en-us}{Help}
\item
  \href{https://www.nytimes.com/subscription?campaignId=37WXW}{Subscriptions}
\end{itemize}
