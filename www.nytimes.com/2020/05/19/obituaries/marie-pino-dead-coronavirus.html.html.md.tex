Sections

SEARCH

\protect\hyperlink{site-content}{Skip to
content}\protect\hyperlink{site-index}{Skip to site index}

\href{https://www.nytimes.com/section/obituaries}{Obituaries}

\href{https://myaccount.nytimes.com/auth/login?response_type=cookie\&client_id=vi}{}

\href{https://www.nytimes.com/section/todayspaper}{Today's Paper}

\href{/section/obituaries}{Obituaries}\textbar{}Marie Pino, Navajo
Teacher Who Educated Generations, Dies at 67

\url{https://nyti.ms/3bIbZLD}

\begin{itemize}
\item
\item
\item
\item
\item
\end{itemize}

\href{https://www.nytimes.com/news-event/coronavirus?action=click\&pgtype=Article\&state=default\&region=TOP_BANNER\&context=storylines_menu}{The
Coronavirus Outbreak}

\begin{itemize}
\tightlist
\item
  live\href{https://www.nytimes.com/2020/08/03/world/coronavirus-covid-19.html?action=click\&pgtype=Article\&state=default\&region=TOP_BANNER\&context=storylines_menu}{Latest
  Updates}
\item
  \href{https://www.nytimes.com/interactive/2020/us/coronavirus-us-cases.html?action=click\&pgtype=Article\&state=default\&region=TOP_BANNER\&context=storylines_menu}{Maps
  and Cases}
\item
  \href{https://www.nytimes.com/interactive/2020/science/coronavirus-vaccine-tracker.html?action=click\&pgtype=Article\&state=default\&region=TOP_BANNER\&context=storylines_menu}{Vaccine
  Tracker}
\item
  \href{https://www.nytimes.com/2020/08/02/us/covid-college-reopening.html?action=click\&pgtype=Article\&state=default\&region=TOP_BANNER\&context=storylines_menu}{College
  Reopening}
\item
  \href{https://www.nytimes.com/live/2020/08/03/business/stock-market-today-coronavirus?action=click\&pgtype=Article\&state=default\&region=TOP_BANNER\&context=storylines_menu}{Economy}
\end{itemize}

Advertisement

\protect\hyperlink{after-top}{Continue reading the main story}

Supported by

\protect\hyperlink{after-sponsor}{Continue reading the main story}

Those We've Lost

\hypertarget{marie-pino-navajo-teacher-who-educated-generations-dies-at-67}{%
\section{Marie Pino, Navajo Teacher Who Educated Generations, Dies at
67}\label{marie-pino-navajo-teacher-who-educated-generations-dies-at-67}}

A well-known teacher, she was mourning the loss of her son to the virus
when she tested positive. Then she died as well, adding to the Navajo
Nation's toll.

\includegraphics{https://static01.nyt.com/images/2020/05/21/obituaries/18Pino1/18Pino1-articleLarge.jpg?quality=75\&auto=webp\&disable=upscale}

By \href{https://www.nytimes.com/by/simon-romero}{Simon Romero}

\begin{itemize}
\item
  May 19, 2020
\item
  \begin{itemize}
  \item
  \item
  \item
  \item
  \item
  \end{itemize}
\end{itemize}

\emph{This obituary is part of a series about people who have died in
the coronavirus pandemic. Read about others}
\href{https://www.nytimes.com/series/people-who-have-died-of-the-coronavirus}{\emph{here}}\emph{.}

Marie Pino, a teacher who educated generations of children in a remote
part of the Navajo Nation, knew how deadly Covid-19 could be. Just weeks
ago, the viral disease took the life of her son.

Marcus Pino was 42 when he
\href{https://www.abqjournal.com/1445029/alamo-navajo-hoops-coach-pino-dies-from-virus.html}{died}
in April. He was the basketball coach at the Alamo Navajo Community
School in Alamo, N.M., the same rural school where his mother taught for
years.

``She got sick at the same time she was mourning,'' said her daughter
Natalie Pino, 32, a health care worker. After testing positive for
Covid-19, Marie Pino was taken to a hospital in Albuquerque and died on
May 13, her daughter said. She was 67.

Ms. Pino's death, as well as that of her son, resonated in the Alamo
Navajo Indian Reservation, a noncontiguous outpost of the Navajo Nation
spreading over 63,000 acres of western New Mexico. About 2,000 people
live there; telephone service didn't arrive until the late 1980s.

The Navajo Nation is struggling with one of the deadliest outbreaks in
the United States, with 4,071 confirmed cases of Covid-19 and 142 deaths
as of Tuesday.

Ms. Pino was born on Nov. 9, 1952, to Luis and May Smith, and raised in
the Navajo village of Sheep Springs. Her father worked for the railroad
and herded sheep; her mother wove traditional rugs. They sent Ms. Pino
to a boarding school for Native American children in Oklahoma.

Ms. Pino attended Haskell Indian Nations University in Lawrence, Kan.,
where she met and married Ira Pino Sr. The pastor of Alamo Miracle
Church, a Pentecostal congregation, he is being treated himself for
Covid-19 at an Albuquerque hospital.

Natalie Pino said her mother devoted her life to teaching out of a
belief that Native American children should have the option of attending
public school near their home instead of boarding schools established
with the objective of assimilating Indigenous children.

Ms. Pino, who taught elementary school for much of her more than 40-year
career, would talk to her students both in English and in Diné Bizaad,
the Navajo language enduring in this part of the West.

She was still teaching middle school at the time of her death. In
addition to her husband and her daughter Natalie, Ms. Pino is survived
by her sons Ira and Anderson Pino; daughters Cheryl Ganadonegro and
Ivonne Bogg; 17 grandchildren and one great-grandchild.

Natalie Pino said her mother was also known for closely following tribal
and national politics with a well-honed sense of humor, often satirizing
political leaders.

``She loved her students and was passionate about their future,'' her
daughter said. ``She voted in every election, tribal, state or national.
My mother had that sense of duty.''

\href{https://www.nytimes.com/interactive/2020/obituaries/people-died-coronavirus-obituaries.html?action=click\&pgtype=Article\&state=default\&region=BELOW_MAIN_CONTENT\&context=covid_obits_promo}{}

\hypertarget{those-weve-lost}{%
\section{Those We've Lost}\label{those-weve-lost}}

The coronavirus pandemic has taken an incalculable death toll. This
series is designed to put names and faces to the numbers.

Read more

\includegraphics{https://static01.nyt.com/images/2020/07/30/obituaries/30Pedro/30Pedro-square640.jpg}

\hypertarget{bernaldina-josuxe9-pedro}{%
\section{Bernaldina José Pedro}\label{bernaldina-josuxe9-pedro}}

d. Boa Vista, Brazil

Leader among the Indigenous Macuxi

\includegraphics{https://static01.nyt.com/images/2020/07/31/obituaries/31Swing/merlin_175167783_8913bc90-0d64-43f3-a655-1bb1bf1601c9-square640.jpg}

\hypertarget{john-eric-swing}{%
\section{John Eric Swing}\label{john-eric-swing}}

d. Fountain Valley, Calif.

Champion of Filipino-Americans

\includegraphics{https://static01.nyt.com/images/2020/07/27/obituaries/27Victor/merlin_175001436_38b11f8e-227a-4e2c-9821-7618af9b2524-square640.jpg}

\hypertarget{victor-victor}{%
\section{Victor Victor}\label{victor-victor}}

d. Santo Domingo, Dominican Republic

Beloved musician of the Dominican Republic

\includegraphics{https://static01.nyt.com/images/2020/07/31/obituaries/31Negron/merlin_175160169_516322ae-fd23-4969-b6b2-193ced371105-square640.jpg}

\hypertarget{dr-eddie-negruxf3n}{%
\section{Dr. Eddie Negrón}\label{dr-eddie-negruxf3n}}

d. Fort Walton Beach, Fla.

Internist on Florida's Emerald Coast

\includegraphics{https://static01.nyt.com/images/2020/07/30/obituaries/30Dobson/merlin_175115928_f6b9271c-8f05-4fe1-a38a-5ca4a58f8935-square640.jpg}

\hypertarget{dobby-dobson}{%
\section{Dobby Dobson}\label{dobby-dobson}}

d. Coral Springs, Fla.

Jamaican singer and songwriter

\includegraphics{https://static01.nyt.com/images/2020/08/01/obituaries/28Gonzalez/merlin_175002771_beb57888-3951-409a-ae13-03a94b2e962e-square640.jpg}

\hypertarget{waldemar-gonzalez}{%
\section{Waldemar Gonzalez}\label{waldemar-gonzalez}}

d. White Plains, N.Y.

Teacher and social worker

Advertisement

\protect\hyperlink{after-bottom}{Continue reading the main story}

\hypertarget{site-index}{%
\subsection{Site Index}\label{site-index}}

\hypertarget{site-information-navigation}{%
\subsection{Site Information
Navigation}\label{site-information-navigation}}

\begin{itemize}
\tightlist
\item
  \href{https://help.nytimes.com/hc/en-us/articles/115014792127-Copyright-notice}{©~2020~The
  New York Times Company}
\end{itemize}

\begin{itemize}
\tightlist
\item
  \href{https://www.nytco.com/}{NYTCo}
\item
  \href{https://help.nytimes.com/hc/en-us/articles/115015385887-Contact-Us}{Contact
  Us}
\item
  \href{https://www.nytco.com/careers/}{Work with us}
\item
  \href{https://nytmediakit.com/}{Advertise}
\item
  \href{http://www.tbrandstudio.com/}{T Brand Studio}
\item
  \href{https://www.nytimes.com/privacy/cookie-policy\#how-do-i-manage-trackers}{Your
  Ad Choices}
\item
  \href{https://www.nytimes.com/privacy}{Privacy}
\item
  \href{https://help.nytimes.com/hc/en-us/articles/115014893428-Terms-of-service}{Terms
  of Service}
\item
  \href{https://help.nytimes.com/hc/en-us/articles/115014893968-Terms-of-sale}{Terms
  of Sale}
\item
  \href{https://spiderbites.nytimes.com}{Site Map}
\item
  \href{https://help.nytimes.com/hc/en-us}{Help}
\item
  \href{https://www.nytimes.com/subscription?campaignId=37WXW}{Subscriptions}
\end{itemize}
