Sections

SEARCH

\protect\hyperlink{site-content}{Skip to
content}\protect\hyperlink{site-index}{Skip to site index}

\href{https://www.nytimes.com/section/obituaries}{Obituaries}

\href{https://myaccount.nytimes.com/auth/login?response_type=cookie\&client_id=vi}{}

\href{https://www.nytimes.com/section/todayspaper}{Today's Paper}

\href{/section/obituaries}{Obituaries}\textbar{}Nita Pippins, a Mother
to AIDS Patients, Dies at 93

\url{https://nyti.ms/3cNbDVi}

\begin{itemize}
\item
\item
\item
\item
\item
\end{itemize}

\href{https://www.nytimes.com/news-event/coronavirus?action=click\&pgtype=Article\&state=default\&region=TOP_BANNER\&context=storylines_menu}{The
Coronavirus Outbreak}

\begin{itemize}
\tightlist
\item
  live\href{https://www.nytimes.com/2020/08/03/world/coronavirus-covid-19.html?action=click\&pgtype=Article\&state=default\&region=TOP_BANNER\&context=storylines_menu}{Latest
  Updates}
\item
  \href{https://www.nytimes.com/interactive/2020/us/coronavirus-us-cases.html?action=click\&pgtype=Article\&state=default\&region=TOP_BANNER\&context=storylines_menu}{Maps
  and Cases}
\item
  \href{https://www.nytimes.com/interactive/2020/science/coronavirus-vaccine-tracker.html?action=click\&pgtype=Article\&state=default\&region=TOP_BANNER\&context=storylines_menu}{Vaccine
  Tracker}
\item
  \href{https://www.nytimes.com/2020/08/02/us/covid-college-reopening.html?action=click\&pgtype=Article\&state=default\&region=TOP_BANNER\&context=storylines_menu}{College
  Reopening}
\item
  \href{https://www.nytimes.com/live/2020/08/03/business/stock-market-today-coronavirus?action=click\&pgtype=Article\&state=default\&region=TOP_BANNER\&context=storylines_menu}{Economy}
\end{itemize}

Advertisement

\protect\hyperlink{after-top}{Continue reading the main story}

Supported by

\protect\hyperlink{after-sponsor}{Continue reading the main story}

Those We've Lost

\hypertarget{nita-pippins-a-mother-to-aids-patients-dies-at-93}{%
\section{Nita Pippins, a Mother to AIDS Patients, Dies at
93}\label{nita-pippins-a-mother-to-aids-patients-dies-at-93}}

At 60, she moved to New York to care for her dying son. Instead of
returning home, she dedicated herself to AIDS patients during the worst
of the epidemic.

\includegraphics{https://static01.nyt.com/images/2020/05/21/obituaries/19pippins-virus-lost/merlin_172647987_52e17f75-5d91-4299-961f-4dfa69b3283a-articleLarge.jpg?quality=75\&auto=webp\&disable=upscale}

\href{https://www.nytimes.com/by/steven-kurutz}{\includegraphics{https://static01.nyt.com/images/2018/09/25/multimedia/author-steven-kurutz/author-steven-kurutz-thumbLarge.png}}

By \href{https://www.nytimes.com/by/steven-kurutz}{Steven Kurutz}

\begin{itemize}
\item
  Published May 19, 2020Updated May 20, 2020
\item
  \begin{itemize}
  \item
  \item
  \item
  \item
  \item
  \end{itemize}
\end{itemize}

\emph{This obituary is part of a series about people who have died in
the coronavirus pandemic. Read about others}
\href{https://www.nytimes.com/series/people-who-have-died-of-the-coronavirus}{\emph{here}}\emph{.}

One day in 1987, Nita Pippins received a call from her only child, Nick,
a 33-year-old actor in Manhattan who was dying of AIDS. ``Mom, I'm in
bed. I can't get out of bed,'' her son told her. ``It's time.''

Ms. Pippins, a retired nurse living in Pensacola, Fla., put her
belongings in a friend's house, took what she could and moved to New
York to care for her son. At age 60, she began an improbable and
remarkable second act.

Devastated and ashamed by her son's AIDS diagnosis, and troubled that he
was gay, Ms. Pippins initially kept the illness a secret from her family
and friends. And she felt out of place in the big city. On breaks while
caring for her son, with whom she had moved in, she would sit inside the
Nathan's Famous restaurant then in Times Square and repeat, ``I hate New
York. I hate New York. I hate New York.''

But Ms. Pippins nursed her son for three years as AIDS ravaged his body
and he went in and out of the hospital. She saw members of the theater
group he had founded and residents of their midtown
building,\href{https://www.nytimes.com/1989/09/08/nyregion/on-the-block-where-aids-hits-hardest-residents-rally.html}{Manhattan
Plaza}, falling sick also. And by the time her son
\href{https://www.nytimes.com/1990/07/29/obituaries/nick-pippin-35-dies-founded-aids-group.html}{died,
in 1990}, Ms. Pippins had been transformed.

She became close with her son's gay friends, and decided to stay in New
York. She dedicated herself to AIDS causes. She became a tireless
volunteer for
\href{https://www.edgemedianetwork.com/news/local/162240}{Miracle
House}, a charity that provided out-of-town families of AIDS patients
with housing and support.

For mothers working through anger, guilt and sadness, Ms. Pippins served
as a parent who had been there. For men estranged from their families,
she became a replacement mother, sometimes holding their hands as they
died.

``She didn't come here to be an activist,'' said Irwin Kroot, who met
Ms. Pippins through her son's partner, Dennis Daniel, and interviewed
her for a possible memoir. ``She was filling a void. She was usually
with young men who were dying and was, at their request, a go-between
for them and their families.''

\includegraphics{https://static01.nyt.com/images/2020/05/19/obituaries/00pippins2/00pippins2-articleLarge.jpg?quality=75\&auto=webp\&disable=upscale}

Ms. Pippins died on May 10 at Amsterdam Nursing Home in Manhattan. She
was 93. The cause was complications of the novel coronavirus, Mr. Kroot
said.

At a time when AIDS was widely misunderstood and gay men who suffered
from the disease were treated like pariahs, Ms. Pippins called on
countless families across the country to come visit their children.
Often, those parents had little sense of their sons' lives in the city
or how sick they were.

Ms. Pippins would tell them to set aside differences and be present for
their child. Some took her advice, some didn't. Ms. Pippins would meet
wary out-of-towners at the Port Authority Bus Terminal or airport, and
take them to breakfast at a midtown diner.

As a nurse, Ms. Pippins could answer their medical questions. As a
mother, and someone from a conservative Southern upbringing, she could
relate to their fears and concerns of being ostracized back home.

``At that time, you were shunned if your son died of AIDS, or you had
AIDS in your family,'' Ms. Pippins
\href{http://www.youtube.com/watch?v=JkzVku0hr3A}{told NY1} in 2010.
``And I wanted to get together and let them know there was other people
having the same problem.''

For Ms. Pippins, her work with AIDS patients was redemptive. ``I needed
to give back,'' she told Mr. Kroot. ``I needed to have something to do
that made me feel better about me.''

Jessie Juanita Pippins was born on Feb. 2, 1927, in Dothan, Ala., to
Alto Lee and Junie Roberts. Her father was a cotton farmer and wanted
his daughter to stay on the farm, Ms. Pippins told Mr. Kroot.

She was not interested. She pestered her father to let her study nursing
at Florida State University until he relented. She became a registered
nurse, and later director of nursing at a hospital in Pensacola. She
retired in 1981.

She and her first husband, Joseph Pippins, divorced. A second marriage,
in effect arranged by her son so she wouldn't be alone after he died,
was short-lived. Her survivors include a stepdaughter, Kelley Pippins
Hays, and three step-grandchildren.

In her second life as a New Yorker, Ms. Pippins attended Broadway shows
and dined out with friends like Mr. Kroot and his husband, Anthony
Catanzaro. Other friends from her son's circle took her on vacation with
them to London. The group remained bonded for decades.

Ms. Pippins, who was present at the end of life for so many during the
AIDS crisis, died alone on Mother's Day.

\href{https://www.nytimes.com/interactive/2020/obituaries/people-died-coronavirus-obituaries.html?action=click\&pgtype=Article\&state=default\&region=BELOW_MAIN_CONTENT\&context=covid_obits_promo}{}

\hypertarget{those-weve-lost}{%
\section{Those We've Lost}\label{those-weve-lost}}

The coronavirus pandemic has taken an incalculable death toll. This
series is designed to put names and faces to the numbers.

Read more

\includegraphics{https://static01.nyt.com/images/2020/07/30/obituaries/30Pedro/30Pedro-square640.jpg}

\hypertarget{bernaldina-josuxe9-pedro}{%
\section{Bernaldina José Pedro}\label{bernaldina-josuxe9-pedro}}

d. Boa Vista, Brazil

Leader among the Indigenous Macuxi

\includegraphics{https://static01.nyt.com/images/2020/07/31/obituaries/31Swing/merlin_175167783_8913bc90-0d64-43f3-a655-1bb1bf1601c9-square640.jpg}

\hypertarget{john-eric-swing}{%
\section{John Eric Swing}\label{john-eric-swing}}

d. Fountain Valley, Calif.

Champion of Filipino-Americans

\includegraphics{https://static01.nyt.com/images/2020/07/27/obituaries/27Victor/merlin_175001436_38b11f8e-227a-4e2c-9821-7618af9b2524-square640.jpg}

\hypertarget{victor-victor}{%
\section{Victor Victor}\label{victor-victor}}

d. Santo Domingo, Dominican Republic

Beloved musician of the Dominican Republic

\includegraphics{https://static01.nyt.com/images/2020/07/31/obituaries/31Negron/merlin_175160169_516322ae-fd23-4969-b6b2-193ced371105-square640.jpg}

\hypertarget{dr-eddie-negruxf3n}{%
\section{Dr. Eddie Negrón}\label{dr-eddie-negruxf3n}}

d. Fort Walton Beach, Fla.

Internist on Florida's Emerald Coast

\includegraphics{https://static01.nyt.com/images/2020/07/30/obituaries/30Dobson/merlin_175115928_f6b9271c-8f05-4fe1-a38a-5ca4a58f8935-square640.jpg}

\hypertarget{dobby-dobson}{%
\section{Dobby Dobson}\label{dobby-dobson}}

d. Coral Springs, Fla.

Jamaican singer and songwriter

\includegraphics{https://static01.nyt.com/images/2020/08/01/obituaries/28Gonzalez/merlin_175002771_beb57888-3951-409a-ae13-03a94b2e962e-square640.jpg}

\hypertarget{waldemar-gonzalez}{%
\section{Waldemar Gonzalez}\label{waldemar-gonzalez}}

d. White Plains, N.Y.

Teacher and social worker

Advertisement

\protect\hyperlink{after-bottom}{Continue reading the main story}

\hypertarget{site-index}{%
\subsection{Site Index}\label{site-index}}

\hypertarget{site-information-navigation}{%
\subsection{Site Information
Navigation}\label{site-information-navigation}}

\begin{itemize}
\tightlist
\item
  \href{https://help.nytimes.com/hc/en-us/articles/115014792127-Copyright-notice}{©~2020~The
  New York Times Company}
\end{itemize}

\begin{itemize}
\tightlist
\item
  \href{https://www.nytco.com/}{NYTCo}
\item
  \href{https://help.nytimes.com/hc/en-us/articles/115015385887-Contact-Us}{Contact
  Us}
\item
  \href{https://www.nytco.com/careers/}{Work with us}
\item
  \href{https://nytmediakit.com/}{Advertise}
\item
  \href{http://www.tbrandstudio.com/}{T Brand Studio}
\item
  \href{https://www.nytimes.com/privacy/cookie-policy\#how-do-i-manage-trackers}{Your
  Ad Choices}
\item
  \href{https://www.nytimes.com/privacy}{Privacy}
\item
  \href{https://help.nytimes.com/hc/en-us/articles/115014893428-Terms-of-service}{Terms
  of Service}
\item
  \href{https://help.nytimes.com/hc/en-us/articles/115014893968-Terms-of-sale}{Terms
  of Sale}
\item
  \href{https://spiderbites.nytimes.com}{Site Map}
\item
  \href{https://help.nytimes.com/hc/en-us}{Help}
\item
  \href{https://www.nytimes.com/subscription?campaignId=37WXW}{Subscriptions}
\end{itemize}
