Sections

SEARCH

\protect\hyperlink{site-content}{Skip to
content}\protect\hyperlink{site-index}{Skip to site index}

\href{https://www.nytimes.com/section/us}{U.S.}

\href{https://myaccount.nytimes.com/auth/login?response_type=cookie\&client_id=vi}{}

\href{https://www.nytimes.com/section/todayspaper}{Today's Paper}

\href{/section/us}{U.S.}\textbar{}Annie Glenn, Champion of Those With
Speech Disorders, Dies at 100

\url{https://nyti.ms/36g4bQc}

\begin{itemize}
\item
\item
\item
\item
\item
\end{itemize}

\href{https://www.nytimes.com/news-event/coronavirus?action=click\&pgtype=Article\&state=default\&region=TOP_BANNER\&context=storylines_menu}{The
Coronavirus Outbreak}

\begin{itemize}
\tightlist
\item
  live\href{https://www.nytimes.com/2020/08/03/world/coronavirus-covid-19.html?action=click\&pgtype=Article\&state=default\&region=TOP_BANNER\&context=storylines_menu}{Latest
  Updates}
\item
  \href{https://www.nytimes.com/interactive/2020/us/coronavirus-us-cases.html?action=click\&pgtype=Article\&state=default\&region=TOP_BANNER\&context=storylines_menu}{Maps
  and Cases}
\item
  \href{https://www.nytimes.com/interactive/2020/science/coronavirus-vaccine-tracker.html?action=click\&pgtype=Article\&state=default\&region=TOP_BANNER\&context=storylines_menu}{Vaccine
  Tracker}
\item
  \href{https://www.nytimes.com/2020/08/02/us/covid-college-reopening.html?action=click\&pgtype=Article\&state=default\&region=TOP_BANNER\&context=storylines_menu}{College
  Reopening}
\item
  \href{https://www.nytimes.com/live/2020/08/03/business/stock-market-today-coronavirus?action=click\&pgtype=Article\&state=default\&region=TOP_BANNER\&context=storylines_menu}{Economy}
\end{itemize}

Advertisement

\protect\hyperlink{after-top}{Continue reading the main story}

Supported by

\protect\hyperlink{after-sponsor}{Continue reading the main story}

those we've lost

\hypertarget{annie-glenn-champion-of-those-with-speech-disorders-dies-at-100}{%
\section{Annie Glenn, Champion of Those With Speech Disorders, Dies at
100}\label{annie-glenn-champion-of-those-with-speech-disorders-dies-at-100}}

Being an astronaut's wife thrust her into the spotlight, but a stutter
left her struggling for words until she found help.

\includegraphics{https://static01.nyt.com/images/2020/05/20/obituaries/19glenn1/merlin_172644219_d87dcd60-e605-492a-8047-08eaa0ead130-articleLarge.jpg?quality=75\&auto=webp\&disable=upscale}

\href{https://www.nytimes.com/by/neil-genzlinger}{\includegraphics{https://static01.nyt.com/images/2018/06/13/multimedia/author-neil-genzlinger/author-neil-genzlinger-thumbLarge.jpg}}

By \href{https://www.nytimes.com/by/neil-genzlinger}{Neil Genzlinger}

\begin{itemize}
\item
  May 19, 2020
\item
  \begin{itemize}
  \item
  \item
  \item
  \item
  \item
  \end{itemize}
\end{itemize}

\emph{This obituary is part of a series about people who have died in
the coronavirus pandemic. Read about others}
\href{https://www.nytimes.com/series/people-who-have-died-of-the-coronavirus}{\emph{here}}\emph{.}

Annie Glenn, who in a high-profile life as the wife of John Glenn, the
astronaut and senator, became an inspiration to many who, like her,
stuttered severely, advocating on behalf of people with communication
disorders of all kinds, died on Tuesday at a nursing home near St. Paul,
Minn. She was 100.

Hank Wilson, director of communications at the John Glenn College of
Public Affairs at the Ohio State University, said the cause was
complications of the Covid-19 virus.

Mrs. Glenn, too, was thrust into the national spotlight in 1962, when
Mr. Glenn became the first American to orbit the Earth. At the time,
though, speaking or even using the telephone was an agony for her
because of her stutter.

``I could never get through a whole sentence,'' she told The New York
Times in 1980. ``Sometimes I would open my mouth and nothing would come
out.''

But in 1973, in her 50s, she decided to address her stuttering by
participating in a fluency-shaping program developed by Dr. Ronald
Webster at Hollins College (now Hollins University) in Virginia.

``I cannot make telephone calls, so John called and enrolled me,'' she
told The Boston Globe in 1975. ``The first requirement was to do a taped
interview. That established the fact that I'm an 85 percent stutterer,
which is in the `most severe' range.''

\includegraphics{https://static01.nyt.com/images/2020/05/19/obituaries/19Glenn2/merlin_172643022_9e0ca241-6e0d-4d4a-b011-984017b90c6b-articleLarge.jpg?quality=75\&auto=webp\&disable=upscale}

She immersed herself in Dr. Webster's intensive, three-week program. By
the end of it, she said, she could do things that had been beyond her
before, like go to a mall and comfortably ask a store clerk where to
find something.

``Those three weeks, we weren't allowed at all to see our family, or to
call, or anything,'' she said.

``When I called John'' at the program's end, she added, ``he cried.''

She became a champion for people with speech disorders and an adjunct
professor in the speech pathology department at Ohio State University's
department of speech and hearing science. In 1987, the American
Speech-Language-Hearing Association created an award in her honor, known
as the Annie, presented annually to someone who demonstrates, as the
organization puts it, her ``invincible spirit in building awareness on
behalf of those with communication disorders.''

``Annie Glenn remains a hero to many of us who in various periods of our
lives couldn't get a word, a thought, or a sentiment past our lips,''
David M. Shribman, executive editor emeritus of The Pittsburgh
Post-Gazette, wrote in February in The Boston Globe on the occasion of
Mrs. Glenn's 100th birthday.

``She fought her condition, to be sure,'' Mr. Shribman, a stutterer
himself, wrote, ``but she also fought for broad public understanding of
stuttering, for the idea that stutterers weren't merely shy, weren't
unintelligent, weren't social pariahs.''

Image

Mrs. Glenn at an interview in Newport, N.H., in 1983.Credit...Associated
Press

Anna Margaret Castor was born in Columbus, Ohio, on Feb. 17, 1920, to
Homer and Margaret Castor. When she was 3 the family moved to New
Concord, Ohio, about 70 miles east of Columbus, where Mr. Glenn's family
lived. She and her future husband were childhood playmates.

Mrs. Glenn said she first became self-conscious about her stuttering in
the sixth grade, when she stood in front of her class to recite. ``I got
up to give a poem, and one of the kids laughed,'' she said in a video
interview posted on the John Glenn College website. ``And I thought,
`Uh-oh; I am not like anybody else in this room.'''

``I think I was the only stutterer in town at that time,'' she added.

She graduated from Muskingum College in 1942, majoring in music and
education. She and Mr. Glenn married in 1943, the same year that he was
commissioned in the Marine Corps.

As her husband became an American hero, Mrs. Glenn was seen but,
necessarily, not often heard.

``Our children answered questions when the media would set up at our
house,'' she recalled in a 1998 interview with The Austin
American-Statesman. ``I didn't want to be interviewed because of my
stuttering.''

A stutter, she would often explain in later years, affected aspects of
life both large and small.

Image

The couple during an interview with The Associated Press in
2015.Credit...Paul Vernon/Associated Press

``I could never tell jokes like everybody else,'' she told The Times in
1980. ``John had to order my meals at restaurants. When I asked for
something at a supermarket, clerks would snicker at me.''

The program at Hollins changed all that.

``People just couldn't believe that I could really talk like I am
talking now,'' she said in the videotape, recalling the reaction of
friends and family members. She went back for a refresher course in
1979, and shortly after made a half-hour speech in front of 300 women in
Canton, Ohio.

``Our family has shared many first experiences,'' she said toward the
end of the speech, ``but I share with all of you here today another
first that means more than I can begin to tell you. This is the first
full-length speech I have ever given in my whole life.''

She campaigned for her husband throughout his political career,
beginning with his first race for the Senate in 1974. He served 24 years
representing Ohio. When Mr. Glenn made an unsuccessful bid to be the
Democratic nominee for president in 1984, Mrs. Glenn enjoyed being a
visible part of his campaign.

``Now I can talk with people, and it is something I have never been able
to do before,'' she told The Times on the campaign trail in December
1983. ``It is like a bird being let out of a cage.''

Mrs. Glenn served on the advisory boards of numerous child-abuse and
speech and hearing organizations. Her husband
\href{https://www.nytimes.com/2016/12/08/us/john-glenn-dies.html}{died
in 2016} at 95.

Mrs. Glenn is survived by two children, John David Glenn and Carolyn Ann
Glenn, and two grandchildren.

In 1982, a reporter for The Globe asked Mr. Glenn, who was then
considering a presidential run, whether marrying someone with such a
severe stutter had given him pause.

``That never really made any difference,'' he said. ``I don't know,
maybe it was just that we grew up together with it, and I knew the
person she was and loved the person she was, and that was that.''

\href{https://www.nytimes.com/interactive/2020/obituaries/people-died-coronavirus-obituaries.html?action=click\&pgtype=Article\&state=default\&region=BELOW_MAIN_CONTENT\&context=covid_obits_promo}{}

\hypertarget{those-weve-lost}{%
\section{Those We've Lost}\label{those-weve-lost}}

The coronavirus pandemic has taken an incalculable death toll. This
series is designed to put names and faces to the numbers.

Read more

\includegraphics{https://static01.nyt.com/images/2020/07/30/obituaries/30Pedro/30Pedro-square640.jpg}

\hypertarget{bernaldina-josuxe9-pedro}{%
\section{Bernaldina José Pedro}\label{bernaldina-josuxe9-pedro}}

d. Boa Vista, Brazil

Leader among the Indigenous Macuxi

\includegraphics{https://static01.nyt.com/images/2020/07/31/obituaries/31Swing/merlin_175167783_8913bc90-0d64-43f3-a655-1bb1bf1601c9-square640.jpg}

\hypertarget{john-eric-swing}{%
\section{John Eric Swing}\label{john-eric-swing}}

d. Fountain Valley, Calif.

Champion of Filipino-Americans

\includegraphics{https://static01.nyt.com/images/2020/07/27/obituaries/27Victor/merlin_175001436_38b11f8e-227a-4e2c-9821-7618af9b2524-square640.jpg}

\hypertarget{victor-victor}{%
\section{Victor Victor}\label{victor-victor}}

d. Santo Domingo, Dominican Republic

Beloved musician of the Dominican Republic

\includegraphics{https://static01.nyt.com/images/2020/07/31/obituaries/31Negron/merlin_175160169_516322ae-fd23-4969-b6b2-193ced371105-square640.jpg}

\hypertarget{dr-eddie-negruxf3n}{%
\section{Dr. Eddie Negrón}\label{dr-eddie-negruxf3n}}

d. Fort Walton Beach, Fla.

Internist on Florida's Emerald Coast

\includegraphics{https://static01.nyt.com/images/2020/07/30/obituaries/30Dobson/merlin_175115928_f6b9271c-8f05-4fe1-a38a-5ca4a58f8935-square640.jpg}

\hypertarget{dobby-dobson}{%
\section{Dobby Dobson}\label{dobby-dobson}}

d. Coral Springs, Fla.

Jamaican singer and songwriter

\includegraphics{https://static01.nyt.com/images/2020/08/01/obituaries/28Gonzalez/merlin_175002771_beb57888-3951-409a-ae13-03a94b2e962e-square640.jpg}

\hypertarget{waldemar-gonzalez}{%
\section{Waldemar Gonzalez}\label{waldemar-gonzalez}}

d. White Plains, N.Y.

Teacher and social worker

Advertisement

\protect\hyperlink{after-bottom}{Continue reading the main story}

\hypertarget{site-index}{%
\subsection{Site Index}\label{site-index}}

\hypertarget{site-information-navigation}{%
\subsection{Site Information
Navigation}\label{site-information-navigation}}

\begin{itemize}
\tightlist
\item
  \href{https://help.nytimes.com/hc/en-us/articles/115014792127-Copyright-notice}{©~2020~The
  New York Times Company}
\end{itemize}

\begin{itemize}
\tightlist
\item
  \href{https://www.nytco.com/}{NYTCo}
\item
  \href{https://help.nytimes.com/hc/en-us/articles/115015385887-Contact-Us}{Contact
  Us}
\item
  \href{https://www.nytco.com/careers/}{Work with us}
\item
  \href{https://nytmediakit.com/}{Advertise}
\item
  \href{http://www.tbrandstudio.com/}{T Brand Studio}
\item
  \href{https://www.nytimes.com/privacy/cookie-policy\#how-do-i-manage-trackers}{Your
  Ad Choices}
\item
  \href{https://www.nytimes.com/privacy}{Privacy}
\item
  \href{https://help.nytimes.com/hc/en-us/articles/115014893428-Terms-of-service}{Terms
  of Service}
\item
  \href{https://help.nytimes.com/hc/en-us/articles/115014893968-Terms-of-sale}{Terms
  of Sale}
\item
  \href{https://spiderbites.nytimes.com}{Site Map}
\item
  \href{https://help.nytimes.com/hc/en-us}{Help}
\item
  \href{https://www.nytimes.com/subscription?campaignId=37WXW}{Subscriptions}
\end{itemize}
