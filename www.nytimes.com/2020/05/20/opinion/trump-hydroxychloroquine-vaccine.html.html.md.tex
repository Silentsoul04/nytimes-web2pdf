Sections

SEARCH

\protect\hyperlink{site-content}{Skip to
content}\protect\hyperlink{site-index}{Skip to site index}

\href{https://myaccount.nytimes.com/auth/login?response_type=cookie\&client_id=vi}{}

\href{https://www.nytimes.com/section/todayspaper}{Today's Paper}

\href{/section/opinion}{Opinion}\textbar{}What Elvis, Michael Jackson
and Trump Have in Common

\href{https://nyti.ms/2ZjA2xF}{https://nyti.ms/2ZjA2xF}

\begin{itemize}
\item
\item
\item
\item
\item
\item
\end{itemize}

Advertisement

\protect\hyperlink{after-top}{Continue reading the main story}

\href{/section/opinion}{Opinion}

Supported by

\protect\hyperlink{after-sponsor}{Continue reading the main story}

\hypertarget{what-elvis-michael-jackson-and-trump-have-in-common}{%
\section{What Elvis, Michael Jackson and Trump Have in
Common}\label{what-elvis-michael-jackson-and-trump-have-in-common}}

They were all nightmare celebrity patients.

\href{https://www.nytimes.com/by/jennifer-senior}{\includegraphics{https://static01.nyt.com/images/2018/10/26/opinion/jennifer-senior/jennifer-senior-thumbLarge.png}}

By \href{https://www.nytimes.com/by/jennifer-senior}{Jennifer Senior}

Opinion columnist

\begin{itemize}
\item
  May 20, 2020
\item
  \begin{itemize}
  \item
  \item
  \item
  \item
  \item
  \item
  \end{itemize}
\end{itemize}

\includegraphics{https://static01.nyt.com/images/2020/05/20/opinion/20seniorWeb/merlin_172649979_7450ed27-c668-46c4-a134-26a9ff80ab68-articleLarge.jpg?quality=75\&auto=webp\&disable=upscale}

Well, well. The president says he's spent
\href{https://www.nytimes.com/2020/05/18/us/politics/trump-hydroxychloroquine-covid-coronavirus.html}{the
last week and a half} enjoying his hydroxychloroquine, presumably neat.
It's impossible to say whether it's true; as doctors on Twitter were
quick to note, Sean Conley, the White House physician, said
\href{https://int.nyt.com/data/documenthelper/6959-letter-from-white-house-physic/e3e29d81b7d6339b9f56/optimized/full.pdf\#page=1}{in
a memo} that he \emph{discussed} the drug with Trump, not prescribed it,
though together he and the president concluded it was worth the risk.

But if you take the president at his word --- something I admittedly
almost never do, but let's just say --- it does make perfect sense. In
Donald Trump, you have the patient perfect storm: a science denier, a
devotee of medical quackery, and --- above all else, I cannot emphasize
this part enough --- a powerful and narcissistic celebrity. This is what
happens when your rich and famous V.I.P. client (think Michael Jackson,
but with nuclear codes) also has a nutty perspective on medicine and an
even nuttier one on facts. You get a statin-taking, extravagantly
overweight man demanding a drug that increases the risk of cardiac
arrest.

We already know a great deal about Trump's science-denialism and
fondness for snake oil. So I'd like to focus mainly on the most
under-discussed variable in this equation: the fact that Trump is rich
and powerful and very famous. People like him often seek out doctors
who'll follow their patients' egos, not science and data.

We saw this quite clearly during the presidential election, when Trump's
personal physician, Harold
Bornstein\href{https://www.npr.org/sections/thetwo-way/2018/05/02/607638733/doctor-trump-dictated-letter-attesting-to-his-extraordinary-health}{,
wrote a letter} saying Trump's lab work was ``astonishingly excellent'';
that his ``physical strength and stamina are extraordinary''; and that,
if voters chose him, Trump would be ``the healthiest individual ever
elected to the presidency.''

It then turned out Bornstein didn't write it. ``He dictated that whole
letter,''
\href{https://www.cnn.com/2018/05/01/politics/harold-bornstein-trump-letter/index.html}{the
doctor told CNN}. ``I just made it up as I went along.''

Time and time again, we've seen it: celebrities nagging their physicians
to administer questionable and risky therapies, sometimes with tragic
consequences. (Elvis being the most obvious example, but also, yes,
Michael Jackson). In 1964, A Maryland psychiatrist named Walter
Weintraub even coined a term for this problem:
\href{https://journals.lww.com/jonmd/Citation/1964/02000/_The_Vip_Syndrome___A_Clinical_Study_in_Hospital.12.aspx}{V.I.P.
Syndrome}.

``I often say that celebrities get the worst treatment,'' Richard
Friedman, a psychiatrist at Weill Cornell Medical College (and a
contributing opinion writer for The Times), told me. ``One, they often
are not properly diagnosed, because doctors don't want to ask
embarrassing questions --- about substance abuse and sexual histories,
for instance. And two, celebrities are often driven by fads, not data,
and while doctors want to do what's right, they also know that
celebrities have the power to make their lives very difficult.''

I had my reasons for phoning Friedman. He treated Philip Roth --- a fact
I never learned from Friedman, obviously, but came out after the author
died, in a memoir by a friend. It led me to conclude that Friedman has
probably cared for his share of famous patients. He demurred when I
asked but told me his own solution, when his patients are being
unreasonable, is to say yes, they are extraordinary, but that they
aren't immune to the laws of physics. ``And I tell my residents: `You
are not a 7-Eleven.'''

So now consider the case of Donald Trump. He is already a germaphobe. He
has no grasp of science, singing the praises of bleach elixirs for
covid-19. He rejects or cherry-picks his facts, at best viewing them
through a political prism: He said the data showing the hazards of
hydroxychloroquine came from his detractors --- ``it was a Trump enemy
statement'' --- when in fact
\href{https://www.vox.com/2020/5/19/21263989/trump-hydroxychloroquine-study-enemy-statement-fda}{they
came from his own Food and Drug Administration}. To bolster his case,
Trump's campaign manager, Brad Parscale, tweeted an endorsement of the
drug by the
\href{https://aapsonline.org/hcq-90-percent-chance/}{Association of
American Physicians and Surgeons}.

It sounds like an unbiased professional organization. The name is
deceptive. It is decidedly partisan. It opposes abortion. It opposes
Obamacare.
\href{https://aapsonline.org/measles-outbreak-and-federal-vaccine-mandates/}{It
opposes, of all things, mandatory vaccines}.

Until 2019, Trump himself
\href{https://twitter.com/realDonaldTrump/status/449525268529815552}{recycled
the dangerous canard} that vaccines were linked to autism,
\href{https://www.statnews.com/2019/04/26/trump-vaccinations-measles/}{only
recanting his views} after a measles outbreak.

He is now in charge of a country in a quest for a vaccine during a
plague. One shudders to think of it.

This happens against a larger backdrop still, in which radical
individualism has been extended to our health, with Americans often
deciding they know better than doctors what's best for them; our trust
in mainstream medicine has eroded right along with our trust in the
media, government, our fellow countrymen. You see it with alternative
medicine on the left (Gwyneth, sigh). You see it with the hawking of
nutritional supplements by Alex Jones and Mike Cernovich on the right.
You see it in the Oval Office.

You must pity the doctors who try to care for our president. They have
the world's most powerful patient on their hands, and very likely its
most impossible. He's not powerful enough to destroy facts. But he's
more than influential and narcissistic enough to make sure they never
get in the way.

\emph{The Times is committed to publishing}
\href{https://www.nytimes.com/2019/01/31/opinion/letters/letters-to-editor-new-york-times-women.html}{\emph{a
diversity of letters}} \emph{to the editor. We'd like to hear what you
think about this or any of our articles. Here are some}
\href{https://help.nytimes.com/hc/en-us/articles/115014925288-How-to-submit-a-letter-to-the-editor}{\emph{tips}}\emph{.
And here's our email:}
\href{mailto:letters@nytimes.com}{\emph{letters@nytimes.com}}\emph{.}

\emph{Follow The New York Times Opinion section on}
\href{https://www.facebook.com/nytopinion}{\emph{Facebook}}\emph{,}
\href{http://twitter.com/NYTOpinion}{\emph{Twitter (@NYTopinion)}}
\emph{and}
\href{https://www.instagram.com/nytopinion/}{\emph{Instagram}}\emph{.}

Advertisement

\protect\hyperlink{after-bottom}{Continue reading the main story}

\hypertarget{site-index}{%
\subsection{Site Index}\label{site-index}}

\hypertarget{site-information-navigation}{%
\subsection{Site Information
Navigation}\label{site-information-navigation}}

\begin{itemize}
\tightlist
\item
  \href{https://help.nytimes.com/hc/en-us/articles/115014792127-Copyright-notice}{©~2020~The
  New York Times Company}
\end{itemize}

\begin{itemize}
\tightlist
\item
  \href{https://www.nytco.com/}{NYTCo}
\item
  \href{https://help.nytimes.com/hc/en-us/articles/115015385887-Contact-Us}{Contact
  Us}
\item
  \href{https://www.nytco.com/careers/}{Work with us}
\item
  \href{https://nytmediakit.com/}{Advertise}
\item
  \href{http://www.tbrandstudio.com/}{T Brand Studio}
\item
  \href{https://www.nytimes.com/privacy/cookie-policy\#how-do-i-manage-trackers}{Your
  Ad Choices}
\item
  \href{https://www.nytimes.com/privacy}{Privacy}
\item
  \href{https://help.nytimes.com/hc/en-us/articles/115014893428-Terms-of-service}{Terms
  of Service}
\item
  \href{https://help.nytimes.com/hc/en-us/articles/115014893968-Terms-of-sale}{Terms
  of Sale}
\item
  \href{https://spiderbites.nytimes.com}{Site Map}
\item
  \href{https://help.nytimes.com/hc/en-us}{Help}
\item
  \href{https://www.nytimes.com/subscription?campaignId=37WXW}{Subscriptions}
\end{itemize}
