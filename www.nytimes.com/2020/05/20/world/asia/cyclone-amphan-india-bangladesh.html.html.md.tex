Sections

SEARCH

\protect\hyperlink{site-content}{Skip to
content}\protect\hyperlink{site-index}{Skip to site index}

\href{https://www.nytimes.com/section/world/asia}{Asia Pacific}

\href{https://myaccount.nytimes.com/auth/login?response_type=cookie\&client_id=vi}{}

\href{https://www.nytimes.com/section/todayspaper}{Today's Paper}

\href{/section/world/asia}{Asia Pacific}\textbar{}Cyclone Amphan Slams
India and Bangladesh

\url{https://nyti.ms/36nCMvx}

\begin{itemize}
\item
\item
\item
\item
\item
\end{itemize}

Advertisement

\protect\hyperlink{after-top}{Continue reading the main story}

Supported by

\protect\hyperlink{after-sponsor}{Continue reading the main story}

\hypertarget{cyclone-amphan-slams-india-and-bangladesh}{%
\section{Cyclone Amphan Slams India and
Bangladesh}\label{cyclone-amphan-slams-india-and-bangladesh}}

Millions fled, and several deaths have been reported, but for the
moment, at least, residents say it appears it could have been much
worse.

\includegraphics{https://static01.nyt.com/images/2020/05/20/world/20cyclone/merlin_172665615_d9272a24-7a10-4a90-95a2-eadb2d075093-articleLarge.jpg?quality=75\&auto=webp\&disable=upscale}

\href{https://www.nytimes.com/by/jeffrey-gettleman}{\includegraphics{https://static01.nyt.com/images/2018/10/10/multimedia/author-jeffrey-gettleman/author-jeffrey-gettleman-thumbLarge.png}}\href{https://www.nytimes.com/by/sameer-yasir}{\includegraphics{https://static01.nyt.com/images/2019/11/22/reader-center/author-sameer-yasir/author-sameer-yasir-thumbLarge.png}}\href{https://www.nytimes.com/by/kai-schultz}{\includegraphics{https://static01.nyt.com/images/2019/11/22/reader-center/author-kai-schultz/author-kai-schultz-thumbLarge.png}}\href{https://www.nytimes.com/by/hari-kumar}{\includegraphics{https://static01.nyt.com/images/2019/12/13/reader-center/author-hari-kumar/author-hari-kumar-thumbLarge.png}}

By \href{https://www.nytimes.com/by/jeffrey-gettleman}{Jeffrey
Gettleman}, \href{https://www.nytimes.com/by/sameer-yasir}{Sameer
Yasir}, \href{https://www.nytimes.com/by/kai-schultz}{Kai Schultz} and
\href{https://www.nytimes.com/by/hari-kumar}{Hari Kumar}

\begin{itemize}
\item
  May 20, 2020
\item
  \begin{itemize}
  \item
  \item
  \item
  \item
  \item
  \end{itemize}
\end{itemize}

NEW DELHI --- A dreaded cyclone tore through eastern India and
Bangladesh on Wednesday, knocking down trees, smashing countless shacks
and killing at least several people, but, it appeared, causing less
devastation than initially feared.

The combination of an impressive evacuation effort and the storm
weakening as it swirled onto land seems to have spared many lives.

Just a few days ago, meteorologists were calling the
\href{https://www.nytimes.com/2020/05/21/world/asia/cyclone-amphan-india-bangladesh.html}{cyclone,
named Amphan}, one of the most dangerous storms in decades. And
preparations for it were complicated by the fact that the cyclone hit in
the middle of the pandemic, with both India and Bangladesh locked down
and experiencing an
\href{https://timesofindia.indiatimes.com/india/covid-19-cases-in-india-cross-1-lakh-mark-over-3000-dead/articleshow/75816534.cms}{alarming
rise in coronavirus infections}.

Many villagers along India's coast were apprehensive about rushing into
packed emergency shelters, where they feared they would catch the virus.
Hundreds of shelters weren't even available because they had been
converted into quarantine centers two weeks ago.

Still, by Wednesday evening, more than three million people had been
whisked from their homes along the Bay of Bengal and were staying in
shelters. The Bangladeshi authorities also managed to evacuate 520,997
animals, they said, including cows, goats, buffalo, chickens and ducks.

\includegraphics{https://static01.nyt.com/images/2020/05/20/world/20cyclone2/merlin_172671894_80120db3-7afa-42e2-ad96-2a416de8df64-articleLarge.jpg?quality=75\&auto=webp\&disable=upscale}

One of the worst-hit cities was Kolkata, once the capital of British
India, which is home to many fragile buildings hundreds of years old.
The eye of Cyclone Amphan passed close to the city, bringing with it
100-mile-per-hour winds and ropes of rain.

The storm split trees into pieces, exploded transformers, tipped over
electricity poles and damaged many homes --- unusual destruction for the
city, which lies more than 50 miles inland from the Bay of Bengal and is
typically spared major cyclone damage.

``It's a pretty bad storm,'' said Jawhar Sircar, a retired government
administrator, speaking by telephone as rain lashed the windows of his
house in south Kolkata. ``Trees are falling. Flower pots are falling.
Things are flying from here to there.''

Another Kolkata resident, Manu Bandyopadhyay, a contractor, was
despondent about losing his ancestral home in a fishing village. His
grandfather was a fisherman.

``If he were alive today,'' Mr. Bandyopadhyay said, ``he would have
cried.''

As the cyclone bore down, humanitarian organizations were especially
worried about the one million Rohingya refugees stuck in muddy camps in
coastal Bangladesh, where they ended up after
\href{https://www.nytimes.com/2017/10/11/world/asia/rohingya-myanmar-atrocities.html}{fleeing
massacres in Myanmar} a few years ago. Many of the refugees live on
denuded hillsides in flimsy homes made from sticks and plastic tarps.

But the storm skirted that area, dumping it with heavy rains but not
washing away homes, as many aid workers and refugees had feared.

``We are staying inside and praying to Allah that the cyclone doesn't
affect us,'' said Enayetullah, who goes by one name and lives with his
three children in the Kutupalong refugee camp, near the town of Cox's
Bazar.

Image

Roofs were reinforced in Cox's Bazar, Bangladesh.Credit...Ro Yassin
Abdumonab, via Reuters

The Indian and Bangladeshi authorities are getting good at large-scale
coastal evacuations.

After a cyclone in 1999 killed thousands of people, both governments
built hundreds of new emergency shelters. They aren't picturesque ---
picture a bare two-story, peeling-paint, cement-block rectangular
building on stilts, almost resembling a crab. But
\href{https://www.telegraphindia.com/states/west-bengal/iit-kharagpur-professors-behind-life-saver-shelters/cid/1690016}{the
structures}, some designed by the faculty at one of India's elite
universities, the Indian Institute of Technology Kharagpur, have proved
stormworthy.

Officials have also tightened up their methods of getting the word out
--- through text messaging, television commercials and old-fashioned
door-to-door pleas to evacuate.

Last year, Indian officials
\href{https://www.nytimes.com/2019/05/03/world/asia/cyclone-fani-india-evacuations.html}{moved
more than a million people} out of harm's way when another cyclone was
bearing down, and once again, for this storm, they seemed to have done a
thorough job of evacuating villagers and pre-positioning rescue teams.

All day Tuesday and Wednesday, emergency crews in orange jumpsuits and
yellow hard hats plied the beach roads, urging people through megaphones
to leave their homes and go to the evacuation shelters as an
increasingly frothy sea pounded the sea walls and spilled into the
roads.

``Do Not Go Out In The Storm,'' said a message featured prominently on
Indian television stations.

\includegraphics{https://static01.nyt.com/images/2020/05/20/world/20cyclone-briefing6/merlin_172664145_d38dbd49-62a4-45bf-a0a4-854715bcd6a4-videoSixteenByNine3000.jpg}

The cyclone made landfall around 4 p.m. near the Indian town of Digha,
on the eastern coast, with wind speeds between 80 and 100 miles per
hour.

Though damage assessments were still sketchy on Wednesday night as
Amphan continued to churn into northeastern India, the authorities said
several people had died, including an infant boy crushed after the wall
of his mud hut crumbled and fell on him.

A Bangladeshi Red Crescent volunteer drowned after a rescue boat
capsized during a rescue operation. At least two other deaths were
reported in India media.

The cyclone washed away bridges connecting Indian islands to the
mainland and left many areas without electricity or phone service, the
West Bengal chief minister, Mamata Banerjee, told reporters Wednesday
evening. She said that while a clearer picture of the devastation would
emerge by Thursday, there had been at least seven deaths, The Associated
Press reported.

But many residents said this was better than they had expected.

On Monday, Cyclone Amphan swept over the Bay of Bengal as the strongest
cyclone ever recorded in the region. But by Tuesday a phenomenon called
vertical wind shear --- the shifting of winds with altitude --- had
disrupted the storm's rotational structure, weakening it.

Amphan initially grew powerful because the waters it passed over were
exceedingly warm, as high as 88 degrees in parts of the Indian Ocean.
Warmer water provides more energy to fuel such rotating storms.

Climate change is
\href{https://www.nytimes.com/2020/05/18/climate/climate-changes-hurricane-intensity.html}{raising
ocean temperatures}, but other factors, including natural variability,
can play a role. While it is not possible to say whether any one
specific storm like Amphan was made more powerful by climate change,
scientists have long expected that tropical storms like it will increase
in strength as the world warms.

Image

Indian emergency workers removing an uprooted tree near the border
between the eastern states of West Bengal and Odisha.Credit...Reuters

The storm drenched the Sundarbans, the world's largest mangrove forest
and a wildlife refuge, home to endangered species including Bengal
tigers.

Belinda Wright, the executive director of the Wildlife Protection
Society of India, said that some of the villages on the fringes of the
wildlife refuge had been badly hit, and that she received a panicked
call Wednesday afternoon from a man she works with in a village on a
remote island.

The man said dozens of people had holed up in a concrete shelter built
on top of a school. Outside, trees had snapped, dead livestock were
sprawled across the ground and huge waves threatened to destroy 12-foot
high dikes that protected the village of mud huts from being completely
obliterated.

If the dikes fail to hold, she said, ``They don't stand a chance.''

``He was very, very emotional,'' Ms. Wright said. ``I could hear
children crying in the background. He said to me: `This might be the
end. This might be the last time I talk to you.'''

But several hours later, Ms. Wright reached him.

``The embankment held,'' she said. ``He sounded extremely positive and
sort of triumphant that he had survived.''

Suhasini Raj contributed reporting from Lucknow, India, and Henry
Fountain from Albuquerque, N.M.

Advertisement

\protect\hyperlink{after-bottom}{Continue reading the main story}

\hypertarget{site-index}{%
\subsection{Site Index}\label{site-index}}

\hypertarget{site-information-navigation}{%
\subsection{Site Information
Navigation}\label{site-information-navigation}}

\begin{itemize}
\tightlist
\item
  \href{https://help.nytimes.com/hc/en-us/articles/115014792127-Copyright-notice}{©~2020~The
  New York Times Company}
\end{itemize}

\begin{itemize}
\tightlist
\item
  \href{https://www.nytco.com/}{NYTCo}
\item
  \href{https://help.nytimes.com/hc/en-us/articles/115015385887-Contact-Us}{Contact
  Us}
\item
  \href{https://www.nytco.com/careers/}{Work with us}
\item
  \href{https://nytmediakit.com/}{Advertise}
\item
  \href{http://www.tbrandstudio.com/}{T Brand Studio}
\item
  \href{https://www.nytimes.com/privacy/cookie-policy\#how-do-i-manage-trackers}{Your
  Ad Choices}
\item
  \href{https://www.nytimes.com/privacy}{Privacy}
\item
  \href{https://help.nytimes.com/hc/en-us/articles/115014893428-Terms-of-service}{Terms
  of Service}
\item
  \href{https://help.nytimes.com/hc/en-us/articles/115014893968-Terms-of-sale}{Terms
  of Sale}
\item
  \href{https://spiderbites.nytimes.com}{Site Map}
\item
  \href{https://help.nytimes.com/hc/en-us}{Help}
\item
  \href{https://www.nytimes.com/subscription?campaignId=37WXW}{Subscriptions}
\end{itemize}
