Sections

SEARCH

\protect\hyperlink{site-content}{Skip to
content}\protect\hyperlink{site-index}{Skip to site index}

\href{https://www.nytimes.com/section/climate}{Climate}

\href{https://myaccount.nytimes.com/auth/login?response_type=cookie\&client_id=vi}{}

\href{https://www.nytimes.com/section/todayspaper}{Today's Paper}

\href{/section/climate}{Climate}\textbar{}Scientists Predict `Busy'
Atlantic Hurricane Season Amid Virus Crisis

\url{https://nyti.ms/2XfeSON}

\begin{itemize}
\item
\item
\item
\item
\item
\end{itemize}

\href{https://www.nytimes.com/section/climate?action=click\&pgtype=Article\&state=default\&region=TOP_BANNER\&context=storylines_menu}{Climate
and Environment}

\begin{itemize}
\tightlist
\item
  \href{https://www.nytimes.com/2020/07/30/climate/sea-level-inland-floods.html?action=click\&pgtype=Article\&state=default\&region=TOP_BANNER\&context=storylines_menu}{Rising
  Seas}
\item
  \href{https://www.nytimes.com/interactive/2020/climate/trump-environment-rollbacks.html?action=click\&pgtype=Article\&state=default\&region=TOP_BANNER\&context=storylines_menu}{Trump's
  Changes}
\item
  \href{https://www.nytimes.com/interactive/2020/04/19/climate/climate-crash-course-1.html?action=click\&pgtype=Article\&state=default\&region=TOP_BANNER\&context=storylines_menu}{Climate
  101}
\item
  \href{https://www.nytimes.com/interactive/2018/08/30/climate/how-much-hotter-is-your-hometown.html?action=click\&pgtype=Article\&state=default\&region=TOP_BANNER\&context=storylines_menu}{Is
  Your Hometown Hotter?}
\item
  \href{https://www.nytimes.com/newsletters/climate-change?action=click\&pgtype=Article\&state=default\&region=TOP_BANNER\&context=storylines_menu}{Newsletter}
\end{itemize}

Advertisement

\protect\hyperlink{after-top}{Continue reading the main story}

Supported by

\protect\hyperlink{after-sponsor}{Continue reading the main story}

\hypertarget{scientists-predict-busy-atlantic-hurricane-season-amid-virus-crisis}{%
\section{Scientists Predict `Busy' Atlantic Hurricane Season Amid Virus
Crisis}\label{scientists-predict-busy-atlantic-hurricane-season-amid-virus-crisis}}

This year's season is complicated by the coronavirus pandemic, which
makes relief strategies like group shelters risky.

\includegraphics{https://static01.nyt.com/images/2020/05/21/climate/21CLI-HURRICANES1/merlin_160148385_37dfe27c-739a-457f-a07a-c4f60cafaf79-articleLarge.jpg?quality=75\&auto=webp\&disable=upscale}

\href{https://www.nytimes.com/by/john-schwartz}{\includegraphics{https://static01.nyt.com/images/2018/02/16/multimedia/author-john-schwartz/author-john-schwartz-thumbLarge.jpg}}\href{https://www.nytimes.com/by/christopher-flavelle}{\includegraphics{https://static01.nyt.com/images/2019/06/28/climate/author-chris-flavelle/author-chris-flavelle-thumbLarge-v3.png}}

By \href{https://www.nytimes.com/by/john-schwartz}{John Schwartz} and
\href{https://www.nytimes.com/by/christopher-flavelle}{Christopher
Flavelle}

\begin{itemize}
\item
  May 21, 2020
\item
  \begin{itemize}
  \item
  \item
  \item
  \item
  \item
  \end{itemize}
\end{itemize}

The coming Atlantic hurricane season is ``expected to be a busy one,''
with the likelihood of as many as 19
\href{https://www.nytimes.com/2020/06/05/us/tropical-storm-cristobal-louisiana.html}{named
storms}, including as many as six major
\href{https://www.nytimes.com/interactive/2020/07/25/us/hurricane-hanna-tracker-map.html}{hurricanes},
a federal weather scientist said Thursday. That worrisome forecast could
be further complicated by the coronavirus pandemic, which is hobbling
relief agencies and could turn evacuation shelters into disease hot
spots.

Gerry Bell, the lead
\href{https://www.nytimes.com/2020/07/09/climate/trump-hurricane-dorian-noaa.html}{hurricane}
season forecaster with the climate prediction center of the National
Oceanic and Atmospheric Administration, delivered the forecast as part
of the annual announcement of the agency's hurricane season outlook.

In the probabilistic language the agency uses to describe the season
ahead, there is a 60 percent chance of an above-normal season, and just
a 10 percent chance of a below-normal season. Agency scientists also
estimated a 70 percent chance of between 13 to 19 named storms. Of
those, NOAA predicted between three and six would be major hurricanes.

In an average hurricane season there are 12 named storms (those with
winds of 39 miles per hour or higher) and three major hurricanes (when
winds reach 111 m.p.h. or more). The Atlantic hurricane season starts
June 1 and runs through Nov. 30, though the emergence of
\href{https://www.nytimes.com/2020/05/17/us/tropical-storm-arthur-2020-path.html}{Tropical
Storm Arthur this month} made this the sixth year in a row in which a
named storm has slipped in before the official beginning of the season.

During the call with reporters to announce the forecast, Carlos J.
Castillo, acting deputy administrator of the Federal Emergency
Management Agency, said the coronavirus pandemic could add to the
challenges of the season.

In a
\href{https://www.fema.gov/media-library-data/1589997234798-adb5ce5cb98a7a89e3e1800becf0eb65/2020_Hurricane_Pandemic_Plan.pdf}{document}
issued on Wednesday, FEMA said it would ``minimize the number of
personnel deploying to disaster-impacted areas'' this hurricane season,
relying instead on what the agency called virtual forms of assistance.

FEMA advised state and local emergency managers to prepare for a range
of new challenges, including ``supporting health and medical systems
that are already stressed, with an expectation that those emergency
services will continue to be taxed into hurricane season.''

One of the challenges facing disaster officials is how to protect people
forced to leave their homes without exposing them to the coronavirus. In
previous storm seasons, local officials and nonprofit groups have relied
on what they call congregate shelters --- rows of cots in high school
gymnasiums, church basements or other crowded spaces.

The American Red Cross, which manages most of the country's shelters, is
``prioritizing individual hotel rooms over congregate shelters,''
according to Stephanie Rendon, a spokeswoman for the organization.

\href{https://www.nytimes.com/section/climate?action=click\&pgtype=Article\&state=default\&region=MAIN_CONTENT_1\&context=storylines_keepup}{}

\hypertarget{climate-and-environment-}{%
\subsubsection{Climate and Environment
›}\label{climate-and-environment-}}

\hypertarget{keep-up-on-the-latest-climate-news}{%
\paragraph{Keep Up on the Latest Climate
News}\label{keep-up-on-the-latest-climate-news}}

Updated July 30, 2020

Here's what you need to know about the latest climate change news this
week:

\begin{itemize}
\item
  \begin{itemize}
  \tightlist
  \item
    \href{https://www.nytimes.com/2020/07/30/climate/bangladesh-floods.html?action=click\&pgtype=Article\&state=default\&region=MAIN_CONTENT_1\&context=storylines_keepup}{Floods
    in}\href{https://www.nytimes.com/2020/07/30/climate/bangladesh-floods.html?action=click\&pgtype=Article\&state=default\&region=MAIN_CONTENT_1\&context=storylines_keepup}{Bangladesh}
    are punishing the people least responsible for climate change.
  \item
    As climate change raises sea levels,
    \href{https://www.nytimes.com/2020/07/30/climate/sea-level-inland-floods.html?action=click\&pgtype=Article\&state=default\&region=MAIN_CONTENT_1\&context=storylines_keepup}{storm
    surges and high tides} are likely to push farther inland.
  \item
    The E.P.A. inspector general plans to investigate whether a rollback
    of fuel efficiency standards
    \href{https://www.nytimes.com/2020/07/27/climate/trump-fuel-efficiency-rule.html?action=click\&pgtype=Article\&state=default\&region=MAIN_CONTENT_1\&context=storylines_keepup}{violated
    government rules}.
  \end{itemize}
\end{itemize}

But she said individual rooms might not be an option in large-scale
disasters like hurricanes, so the Red Cross would instead rely on
``additional safety precautions'' for group shelters, such as health
screenings, masks, additional space between cots and extra cleaning and
disinfecting.

The coronavirus has also put new strain on FEMA, which as of Thursday
was managing 103 major disasters around the country, according to agency
records.

Just 38 percent of FEMA staff members were available to be deployed to a
disaster zone; for some of the agency's specialized staff, such as field
leaders and safety experts, less than one-quarter were available.

``We have not taken our eye off the ball about handling other
disasters,'' said Peter T. Gaynor, the FEMA administrator, in a call
with reporters this month.

Factors contributing to this year's prediction of above-normal activity
include warmer-than-average sea surface temperatures in the tropical
Atlantic Ocean and Caribbean Sea, along with reduced vertical wind
shear, which can keep storms from forming or from becoming stronger.
There is also an enhanced west African monsoon.

A\href{https://www.nytimes.com/2020/05/18/climate/climate-changes-hurricane-intensity.html}{study
published on Monday} suggested that climate change has been making
hurricanes around the world stronger over the past four decades. This
makes intuitive sense, and is expected to grow worse over time, because
warmer ocean water tends to strengthens storms.

Jennifer Francis, a senior scientist at the Woods Hole Research Center,
said in a statement, ``If we want to keep these dangerous patterns from
accelerating, we need urgent action by government and private sector
leaders to shift us away from fossil fuels and toward clean energy.''

However, Dr. Bell of
\href{https://www.nytimes.com/2020/07/09/climate/trump-hurricane-dorian-noaa.html}{NOAA}
said in Thursday's call, other factors have, at least so far, had a far
greater effect on hurricane strength in the North Atlantic than climate
change.

Those include a decades-long cycle of rising and falling sea-surface
temperatures known as the Atlantic multidecadal oscillation, and the
phenomenon of El Niño and La Niña in the Pacific. El Niño tends to
suppress hurricane activity in the Atlantic; La Niña promotes storm
activity there.

The Atlantic has been in a ``high-activity era'' since 1995, Dr. Bell
said. This year, the El Nino cycle is in a neutral state, which neither
suppresses nor enhances storm activity. If La Niña should develop during
this season, then the high end of the agency's forecast becomes more
probable.

Dr. Bell added that other elements of climate change were contributing
to the destructiveness of storms, including rising sea levels and the
increased moisture content of warmer air, which can mean
\href{https://www.nytimes.com/2019/07/11/climate/hurricane-tropical-storms.html}{more
rainfall and flooding} from storms. In addition to climate issues, ``our
coastlines have built up enormously,'' he noted, which has put more
people in harm's way whenever any storm approaches.

For all of the attention that NOAA's annual announcement receives,
though, it doesn't offer a definitive verdict on the hurricane season,
said Andrew Dessler, an expert in climate change at Texas A\&M
University.

He called the forecasts ``an interesting scientific problem'' but said,
``I don't think they tell us much about how to prepare.'' They cannot
predict landfall, for example. And, even in a year with very few storms
forecast, ``it just takes one to be a true disaster.''

Therefore, he said, for people near the Gulf of Mexico or on the East
Coast, ``you should be ready for a big storm, regardless of the
forecast.''

Advertisement

\protect\hyperlink{after-bottom}{Continue reading the main story}

\hypertarget{site-index}{%
\subsection{Site Index}\label{site-index}}

\hypertarget{site-information-navigation}{%
\subsection{Site Information
Navigation}\label{site-information-navigation}}

\begin{itemize}
\tightlist
\item
  \href{https://help.nytimes.com/hc/en-us/articles/115014792127-Copyright-notice}{©~2020~The
  New York Times Company}
\end{itemize}

\begin{itemize}
\tightlist
\item
  \href{https://www.nytco.com/}{NYTCo}
\item
  \href{https://help.nytimes.com/hc/en-us/articles/115015385887-Contact-Us}{Contact
  Us}
\item
  \href{https://www.nytco.com/careers/}{Work with us}
\item
  \href{https://nytmediakit.com/}{Advertise}
\item
  \href{http://www.tbrandstudio.com/}{T Brand Studio}
\item
  \href{https://www.nytimes.com/privacy/cookie-policy\#how-do-i-manage-trackers}{Your
  Ad Choices}
\item
  \href{https://www.nytimes.com/privacy}{Privacy}
\item
  \href{https://help.nytimes.com/hc/en-us/articles/115014893428-Terms-of-service}{Terms
  of Service}
\item
  \href{https://help.nytimes.com/hc/en-us/articles/115014893968-Terms-of-sale}{Terms
  of Sale}
\item
  \href{https://spiderbites.nytimes.com}{Site Map}
\item
  \href{https://help.nytimes.com/hc/en-us}{Help}
\item
  \href{https://www.nytimes.com/subscription?campaignId=37WXW}{Subscriptions}
\end{itemize}
