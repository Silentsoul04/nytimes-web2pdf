Sections

SEARCH

\protect\hyperlink{site-content}{Skip to
content}\protect\hyperlink{site-index}{Skip to site index}

\href{https://www.nytimes.com/section/technology}{Technology}

\href{https://myaccount.nytimes.com/auth/login?response_type=cookie\&client_id=vi}{}

\href{https://www.nytimes.com/section/todayspaper}{Today's Paper}

\href{/section/technology}{Technology}\textbar{}Robots! (Don't Get Too
Excited.)

\url{https://nyti.ms/2Tpiht4}

\begin{itemize}
\item
\item
\item
\item
\item
\end{itemize}

\href{https://www.nytimes.com/news-event/coronavirus?action=click\&pgtype=Article\&state=default\&region=TOP_BANNER\&context=storylines_menu}{The
Coronavirus Outbreak}

\begin{itemize}
\tightlist
\item
  live\href{https://www.nytimes.com/2020/08/04/world/coronavirus-covid-19.html?action=click\&pgtype=Article\&state=default\&region=TOP_BANNER\&context=storylines_menu}{Latest
  Updates}
\item
  \href{https://www.nytimes.com/interactive/2020/us/coronavirus-us-cases.html?action=click\&pgtype=Article\&state=default\&region=TOP_BANNER\&context=storylines_menu}{Maps
  and Cases}
\item
  \href{https://www.nytimes.com/interactive/2020/science/coronavirus-vaccine-tracker.html?action=click\&pgtype=Article\&state=default\&region=TOP_BANNER\&context=storylines_menu}{Vaccine
  Tracker}
\item
  \href{https://www.nytimes.com/2020/08/02/us/covid-college-reopening.html?action=click\&pgtype=Article\&state=default\&region=TOP_BANNER\&context=storylines_menu}{College
  Reopening}
\item
  \href{https://www.nytimes.com/live/2020/08/03/business/stock-market-today-coronavirus?action=click\&pgtype=Article\&state=default\&region=TOP_BANNER\&context=storylines_menu}{Economy}
\end{itemize}

Advertisement

\protect\hyperlink{after-top}{Continue reading the main story}

Supported by

\protect\hyperlink{after-sponsor}{Continue reading the main story}

on tech

\hypertarget{robots-dont-get-too-excited}{%
\section{Robots! (Don't Get Too
Excited.)}\label{robots-dont-get-too-excited}}

Robots are cool. But we should be skeptical of emerging technology.

\includegraphics{https://static01.nyt.com/images/2020/05/21/business/21ontech2/21ontech2-articleLarge.jpg?quality=75\&auto=webp\&disable=upscale}

\href{https://www.nytimes.com/by/shira-ovide}{\includegraphics{https://static01.nyt.com/images/2020/03/18/reader-center/author-shira-ovide/author-shira-ovide-thumbLarge-v2.png}}

By \href{https://www.nytimes.com/by/shira-ovide}{Shira Ovide}

\begin{itemize}
\item
  May 21, 2020
\item
  \begin{itemize}
  \item
  \item
  \item
  \item
  \item
  \end{itemize}
\end{itemize}

\emph{This article is part of the On Tech newsletter. You can}
\href{https://www.nytimes.com/newsletters/signup/OT}{\emph{sign up
here}} \emph{to receive it weekdays.}

We want cool technology like jet packs and driverless cars, and we WANT
IT EVERYWHERE RIGHT NOW.

My colleague \href{https://www.nytimes.com/by/cade-metz}{Cade Metz} will
kill your dreams.

He and Erin Griffith wrote this week about one British city where
\href{https://www.nytimes.com/2020/05/20/technology/delivery-robots-coronavirus-milton-keynes.html}{sidewalk-roaming
robots} that can deliver groceries are in high demand during the
pandemic. Yay for robot helpers, right?! Optimists imagine what else
they can do
\href{https://arstechnica.com/tech-policy/2020/04/the-pandemic-is-bringing-us-closer-to-our-robot-takeout-future/}{once
the technology progresses}.

Cade has affection for a fictional killer supercomputer, which says
something about his tech optimism. He explained to me the limitations of
delivery robots, and why they'll probably never be widely available.

\textbf{SHIRA: In this one city, Milton Keynes, who is benefiting from
the robot deliveries?}

\textbf{CADE:} Before the pandemic, a resident of Milton Keynes, Liss
Page, thought these robots were fascinating but mostly pointless. On her
jogs, she'd wind up alongside a robot, and she would talk to it ---
almost tease it.

Then the pandemic happened, and she was advised not to leave her
apartment because of pre-existing health conditions. Those robots are
now vital to bring her groceries --- when the stores are in stock.

\textbf{That's very helpful right now. So why, then, are you a
robot-delivery skeptic?}

These robots can't even serve everyone in Milton Keynes, which is
ideally suited to robot deliveries because it has bike and pedestrian
paths alongside the roadways. Almost nowhere else is set up for these
deliveries on a wide scale.

You can see what these robots can do in small ways or in certain places,
but you also see the limitations when you extrapolate that out. People
\href{https://www.sfchronicle.com/business/article/Kiwibots-win-fans-at-UC-Berkeley-as-they-deliver-13895867.php?psid=9L4Fj}{vandalize
these robots} for kicks. The robots get stuck, and humans have to take
over remotely. They can't carry much. If you have a family, it's not
great to be limited to a couple of grocery bags.

\hypertarget{latest-updates-economy}{%
\section{\texorpdfstring{\href{https://www.nytimes.com/live/2020/08/03/business/stock-market-today-coronavirus?action=click\&pgtype=Article\&state=default\&region=MAIN_CONTENT_1\&context=storylines_live_updates}{Latest
Updates:
Economy}}{Latest Updates: Economy}}\label{latest-updates-economy}}

\href{https://www.nytimes.com/live/2020/08/03/business/stock-market-today-coronavirus?action=click\&pgtype=Article\&state=default\&region=MAIN_CONTENT_1\&context=storylines_live_updates\#the-chicago-fed-president-says-its-up-to-congress-to-save-the-economy}{13h
ago}

\href{https://www.nytimes.com/live/2020/08/03/business/stock-market-today-coronavirus?action=click\&pgtype=Article\&state=default\&region=MAIN_CONTENT_1\&context=storylines_live_updates\#the-chicago-fed-president-says-its-up-to-congress-to-save-the-economy}{The
Chicago Fed president says it's up to Congress to save the economy.}

\href{https://www.nytimes.com/live/2020/08/03/business/stock-market-today-coronavirus?action=click\&pgtype=Article\&state=default\&region=MAIN_CONTENT_1\&context=storylines_live_updates\#faa-says-boeing-has-effectively-mitigated-defects-in-the-737-max}{14h
ago}

\href{https://www.nytimes.com/live/2020/08/03/business/stock-market-today-coronavirus?action=click\&pgtype=Article\&state=default\&region=MAIN_CONTENT_1\&context=storylines_live_updates\#faa-says-boeing-has-effectively-mitigated-defects-in-the-737-max}{F.A.A.
says Boeing has `effectively mitigated' defects in the 737 Max.}

\href{https://www.nytimes.com/live/2020/08/03/business/stock-market-today-coronavirus?action=click\&pgtype=Article\&state=default\&region=MAIN_CONTENT_1\&context=storylines_live_updates\#small-businesses-got-emergency-loans-but-not-what-they-expected}{16h
ago}

\href{https://www.nytimes.com/live/2020/08/03/business/stock-market-today-coronavirus?action=click\&pgtype=Article\&state=default\&region=MAIN_CONTENT_1\&context=storylines_live_updates\#small-businesses-got-emergency-loans-but-not-what-they-expected}{Small
businesses got emergency loans, but not what they expected.}

\href{https://www.nytimes.com/live/2020/08/03/business/stock-market-today-coronavirus?action=click\&pgtype=Article\&state=default\&region=MAIN_CONTENT_1\&context=storylines_live_updates}{See
more updates}

More live coverage:
\href{https://www.nytimes.com/2020/08/04/world/coronavirus-covid-19.html?action=click\&pgtype=Article\&state=default\&region=MAIN_CONTENT_1\&context=storylines_live_updates}{Global}

\textbf{So robot deliveries aren't coming to my neighborhood soon?}

Probably not. Prices will come down, and autonomous technology will
improve, but there are limits to how many of these things you can put on
a sidewalk.

And delivery robots only work long term if they're cheaper than humans
doing the same thing. That's not going to happen if robots stay confined
to a tiny number of places like Milton Keynes
or\href{https://www.washingtonpost.com/technology/2019/03/25/how-gmu-students-eating-habits-changed-when-delivery-robots-invaded-their-campus/}{college
campuses}.

\textbf{You wrote earlier this month
about\href{https://www.nytimes.com/2020/05/12/technology/self-driving-cars-coronavirus.html}{problems
with driverless cars}, and now you're picking on delivery robots. Are
you a killjoy?}

Look, over the past 10 years there's been a lot of progress, but you
have to be skeptical of emerging technology. Otherwise you get an
unrealistic view of what's possible and miss where technologies go
wrong.

\textbf{OK, that's fair. Now tell us, why are people infatuated with
robots? We think they're}
\textbf{\href{https://www.nytimes.com/2020/02/26/business/robots-retail-jobs.html}{adorable}}
\textbf{or}
\textbf{\href{https://www.nytimes.com/2020/04/08/movies/ai-humans-robots-technology.html}{villainous}.}

They fascinate us and scare us. All the movies and television we've
watched for the last 60 years about robots and artificial intelligence
have been burned into our brains. It really affects the expectations we
have of technology.

\textbf{What's your favorite artificial being in pop culture?}

I'm partial to
\href{https://www.nytimes.com/2018/03/30/movies/hal-2001-a-space-odyssey-voice-douglas-rain.html}{HAL
9000} from ``2001: A Space Odyssey.'' Hal is a wonderful character ---
and a flawed one. He shows where machines can go right, and where they
can go wrong.

\emph{Get this newsletter in your inbox every
weekday;}\href{https://www.nytimes.com/newsletters/signup/OT}{\emph{please
sign up here}}\emph{.}

\begin{center}\rule{0.5\linewidth}{\linethickness}\end{center}

\hypertarget{wfh-forever-who-really-knows}{%
\subsection{WFH forever? Who really
knows}\label{wfh-forever-who-really-knows}}

There are office workers and their bosses who are itching to return to
cubicle life fast. And others who are saying
\href{https://blog.twitter.com/en_us/topics/company/2020/keeping-our-employees-and-partners-safe-during-coronavirus.html}{goodbye
forever} to toiling in an office.

And then there's Evan Spiegel, Snapchat's chief executive, who says ---
sensibly --- who the heck knows?

Snapchat's headquarters in the Los Angeles area closed in March, and
people scattered to work remotely. The company is now telling employees
they can work remotely at least through September, and it's assessing
when and how to reopen safely. The squishiness of the message doesn't
sit well with everyone.

``People want certainty, and there's a huge amount of pressure as a
leader to make definitive statements,'' Spiegel said in a conversation
Wednesday (by video chat, not Snapchat) with New York Times editors and
reporters. ``I think it's important that we remain flexible in a
situation that is changing rapidly.''

Snapchat, which has more than 3,000 employees, has been planning for a
couple months on how to reopen offices. It's keeping track of business
safety requirements issued by local authorities, and Snapchat's own. It
has assessed which teams to invite back to offices first based on job
requirements. Someone who needs access to high-end video editing
equipment available only at the office, for example, would be higher on
the list of returnees.

Spiegel and his wife, the model and skincare entrepreneur Miranda Kerr,
have two young sons. Like many parents, he said he had mixed feelings
about working remotely.

It's been challenging, he said, for two working adults and their
children to manage under the same roof 24/7. But, Spiegel said, ``I get
to spend time with my family, which has led to more fulfillment than
I've ever had in my life.''

\begin{center}\rule{0.5\linewidth}{\linethickness}\end{center}

\hypertarget{before-we-go-}{%
\subsection{Before we go \ldots{}}\label{before-we-go-}}

\begin{itemize}
\item
  \textbf{Help getting connected during the pandemic. Maybe:} Internet
  providers like Charter and Comcast promised to help low-income people
  get or stay online during the pandemic. But
  \href{https://www.nytimes.com/2020/05/20/technology/coronavirus-broadband-discounts.html}{taking
  them up on the offer hasn't always been easy}, my colleague
  \href{https://www.nytimes.com/by/david-mccabe}{David McCabe} reported.
\item
  \textbf{Everything you need to know about tracking disease, with
  humans:} ProPublica has the
  \href{https://www.propublica.org/article/you-dont-need-invasive-tech-for-successful-contact-tracing-heres-how-it-works}{best
  explanation} I've seen for how disease detectives track down people
  who may have been exposed to the coronavirus. As I've
  \href{https://www.nytimes.com/2020/04/29/technology/coronavirus-contact-tracing-technology.html}{written
  here}, this is a labor-intensive process for which smartphone location
  data may (or may not) help a little.
\item
  \textbf{Banal and utterly bizarre:} A glitch over smartphone photo
  formats is causing some high school students to fail advanced
  placement tests, The Verge
  \href{https://www.theverge.com/2020/5/20/21262302/ap-test-fail-iphone-photos-glitch-email-college-board-jpeg-heic}{reported}.
  Some test takers submit photos of their virtual test sheets, but the
  testing website doesn't support the default format on some iPhones and
  newer Android phones.
\end{itemize}

\hypertarget{hugs-to-this}{%
\subsubsection{Hugs to this}\label{hugs-to-this}}

Move over, \href{https://www.youtube.com/watch?v=Mh4f9AYRCZY}{BBC Dad}.
My newest telecast-from-home star is
\href{https://twitter.com/ratemyskyperoom/status/1261684025502113794}{cat
fight lady}.

\begin{center}\rule{0.5\linewidth}{\linethickness}\end{center}

\emph{We want to hear from you. Tell us what you think of this
newsletter and what else you'd like us to explore. You can reach us at}
\href{mailto:ontech@nytimes.com?subject=On\%20Tech\%20Feedback}{\emph{ontech@nytimes.com.}}

\emph{Get this newsletter in your inbox every
weekday;}\href{https://www.nytimes.com/newsletters/signup/OT}{\emph{please
sign up here}}\emph{.}

Advertisement

\protect\hyperlink{after-bottom}{Continue reading the main story}

\hypertarget{site-index}{%
\subsection{Site Index}\label{site-index}}

\hypertarget{site-information-navigation}{%
\subsection{Site Information
Navigation}\label{site-information-navigation}}

\begin{itemize}
\tightlist
\item
  \href{https://help.nytimes.com/hc/en-us/articles/115014792127-Copyright-notice}{©~2020~The
  New York Times Company}
\end{itemize}

\begin{itemize}
\tightlist
\item
  \href{https://www.nytco.com/}{NYTCo}
\item
  \href{https://help.nytimes.com/hc/en-us/articles/115015385887-Contact-Us}{Contact
  Us}
\item
  \href{https://www.nytco.com/careers/}{Work with us}
\item
  \href{https://nytmediakit.com/}{Advertise}
\item
  \href{http://www.tbrandstudio.com/}{T Brand Studio}
\item
  \href{https://www.nytimes.com/privacy/cookie-policy\#how-do-i-manage-trackers}{Your
  Ad Choices}
\item
  \href{https://www.nytimes.com/privacy}{Privacy}
\item
  \href{https://help.nytimes.com/hc/en-us/articles/115014893428-Terms-of-service}{Terms
  of Service}
\item
  \href{https://help.nytimes.com/hc/en-us/articles/115014893968-Terms-of-sale}{Terms
  of Sale}
\item
  \href{https://spiderbites.nytimes.com}{Site Map}
\item
  \href{https://help.nytimes.com/hc/en-us}{Help}
\item
  \href{https://www.nytimes.com/subscription?campaignId=37WXW}{Subscriptions}
\end{itemize}
