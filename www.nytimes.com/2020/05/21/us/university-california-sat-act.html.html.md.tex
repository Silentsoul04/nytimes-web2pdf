Sections

SEARCH

\protect\hyperlink{site-content}{Skip to
content}\protect\hyperlink{site-index}{Skip to site index}

\href{https://www.nytimes.com/section/us}{U.S.}

\href{https://myaccount.nytimes.com/auth/login?response_type=cookie\&client_id=vi}{}

\href{https://www.nytimes.com/section/todayspaper}{Today's Paper}

\href{/section/us}{U.S.}\textbar{}University of California Will End Use
of SAT and ACT in Admissions

\url{https://nyti.ms/2ZvTqYF}

\begin{itemize}
\item
\item
\item
\item
\item
\item
\end{itemize}

\href{https://www.nytimes.com/news-event/coronavirus?action=click\&pgtype=Article\&state=default\&region=TOP_BANNER\&context=storylines_menu}{The
Coronavirus Outbreak}

\begin{itemize}
\tightlist
\item
  live\href{https://www.nytimes.com/2020/08/02/world/coronavirus-updates.html?action=click\&pgtype=Article\&state=default\&region=TOP_BANNER\&context=storylines_menu}{Latest
  Updates}
\item
  \href{https://www.nytimes.com/interactive/2020/us/coronavirus-us-cases.html?action=click\&pgtype=Article\&state=default\&region=TOP_BANNER\&context=storylines_menu}{Maps
  and Cases}
\item
  \href{https://www.nytimes.com/interactive/2020/science/coronavirus-vaccine-tracker.html?action=click\&pgtype=Article\&state=default\&region=TOP_BANNER\&context=storylines_menu}{Vaccine
  Tracker}
\item
  \href{https://www.nytimes.com/interactive/2020/07/29/us/schools-reopening-coronavirus.html?action=click\&pgtype=Article\&state=default\&region=TOP_BANNER\&context=storylines_menu}{What
  School May Look Like}
\item
  \href{https://www.nytimes.com/live/2020/07/31/business/stock-market-today-coronavirus?action=click\&pgtype=Article\&state=default\&region=TOP_BANNER\&context=storylines_menu}{Economy}
\end{itemize}

Advertisement

\protect\hyperlink{after-top}{Continue reading the main story}

Supported by

\protect\hyperlink{after-sponsor}{Continue reading the main story}

\hypertarget{university-of-california-will-end-use-of-sat-and-act-in-admissions}{%
\section{University of California Will End Use of SAT and ACT in
Admissions}\label{university-of-california-will-end-use-of-sat-and-act-in-admissions}}

The change is expected to accelerate the momentum of American colleges
away from the tests, amid concern that they are unfair to poor, black
and Hispanic students.

\includegraphics{https://static01.nyt.com/images/2020/05/21/us/21cal-state/merlin_171166869_e4d262ed-462f-49c8-9254-4fe9b16ae80e-articleLarge.jpg?quality=75\&auto=webp\&disable=upscale}

By Shawn Hubler

\begin{itemize}
\item
  Published May 21, 2020Updated May 24, 2020
\item
  \begin{itemize}
  \item
  \item
  \item
  \item
  \item
  \item
  \end{itemize}
\end{itemize}

SACRAMENTO --- The University of California on Thursday voted to phase
out the
\href{https://www.nytimes.com/2020/06/02/us/at-home-sat-coronavirus.html}{SAT}
and ACT as requirements to apply to its system of 10 schools, which
include some of the nation's most popular campuses, in a decision with
major implications for the use of standardized tests in college
admissions.

Given the size and influence of the California system, whose marquee
schools include the University of California, Los Angeles, and the
University of California, Berkeley, the move is expected to accelerate
the momentum of American colleges away from the tests, amid charges that
they are unfair to poor, black and Hispanic students.

The school system's action, which follows many small liberal arts
colleges, comes as the ACT and the College Board, a nonprofit
organization that administers the SAT, are suffering financially from
the
\href{https://www.nytimes.com/2020/04/15/us/sat-act-test-coronavirus.html}{cancellation
of test dates} during the coronavirus pandemic. One critic of the
industry estimated that the College Board had lost \$45 million in
revenue this spring.

Although many students will likely continue to take the exams as long as
they are required by highly competitive schools like Stanford and those
in the Ivy League, California's decision will clearly be a blow to the
image of the tests, and experts said it could tip the balance for other
schools in deciding whether to eliminate them.

``The University of California is one of the best institutions in the
world, so whatever decision they make will be extraordinarily
influential,'' said Terry W. Hartle, senior vice president at the
American Council on Education, a trade group. ``Whatever U.C. does will
have ripple effects across American higher education, particularly at
leading public universities.''

Like many colleges nationwide, University of California schools had
already
\href{https://www.nytimes.com/article/sat-act-test-optional-colleges-coronavirus.html}{made
the SAT and ACT optional} for this year's applicants, after testing
dates were disrupted by the pandemic. Both companies have announced that
they will introduce an online testing option for the first time in the
fall.

On Thursday, the California system's governing board voted unanimously
to extend that optional period for another year, and then not consider
scores for two years when determining whether to accept in-state
applicants, using standardized tests only to award scholarships,
determine course placement and assess out-of-state students.

In 2025, consideration of the SAT or ACT for any student's admission, in
or out of state, would be eliminated.

\hypertarget{latest-updates-global-coronavirus-outbreak}{%
\section{\texorpdfstring{\href{https://www.nytimes.com/2020/08/01/world/coronavirus-covid-19.html?action=click\&pgtype=Article\&state=default\&region=MAIN_CONTENT_1\&context=storylines_live_updates}{Latest
Updates: Global Coronavirus
Outbreak}}{Latest Updates: Global Coronavirus Outbreak}}\label{latest-updates-global-coronavirus-outbreak}}

Updated 2020-08-02T17:52:35.962Z

\begin{itemize}
\tightlist
\item
  \href{https://www.nytimes.com/2020/08/01/world/coronavirus-covid-19.html?action=click\&pgtype=Article\&state=default\&region=MAIN_CONTENT_1\&context=storylines_live_updates\#link-34047410}{The
  U.S. reels as July cases more than double the total of any other
  month.}
\item
  \href{https://www.nytimes.com/2020/08/01/world/coronavirus-covid-19.html?action=click\&pgtype=Article\&state=default\&region=MAIN_CONTENT_1\&context=storylines_live_updates\#link-780ec966}{Top
  U.S. officials work to break an impasse over the federal jobless
  benefit.}
\item
  \href{https://www.nytimes.com/2020/08/01/world/coronavirus-covid-19.html?action=click\&pgtype=Article\&state=default\&region=MAIN_CONTENT_1\&context=storylines_live_updates\#link-2bc8948}{Its
  outbreak untamed, Melbourne goes into even greater lockdown.}
\end{itemize}

\href{https://www.nytimes.com/2020/08/01/world/coronavirus-covid-19.html?action=click\&pgtype=Article\&state=default\&region=MAIN_CONTENT_1\&context=storylines_live_updates}{See
more updates}

More live coverage:
\href{https://www.nytimes.com/live/2020/07/31/business/stock-market-today-coronavirus?action=click\&pgtype=Article\&state=default\&region=MAIN_CONTENT_1\&context=storylines_live_updates}{Markets}

``These tests are extremely flawed and very unfair,'' said Lt. Gov.
Eleni Kounalakis, a member of the board who supported the decision,
adding, ``Enough is enough.''

In the meantime, the university will do a study on the feasibility of
creating its own admissions test, perhaps in collaboration with other
California schools.

In a statement after the vote, the College Board predicted that the
governing board's decision would add to the burden of high school
students applying for college if the system creates its own exam, saying
that many students will still take the SAT or ACT to apply to other
institutions.

``Having to take multiple tests will likely cause many of these students
to limit their college options much earlier in the college search
process,'' the organization said.

Some 300,000 students attend University of California schools, and
\href{https://www.usnews.com/education/best-colleges/the-short-list-college/articles/colleges-that-received-the-most-applications}{six
of its campuses top the list} of American schools with the most
applicants, with U.C.L.A. consistently the most sought after.
Four-fifths of applicants to the system's schools take the SAT,
providing the
\href{https://www.latimes.com/california/story/2020-05-11/napolitano-says-suspend-the-sat-test-for-uc-admissions}{largest
source of customers} for the College Board, which brings in more than
\$1 billion a year in revenue.

In addition to the SAT, the organization also administers Advanced
Placement tests for high school students and other testing programs.
Experts said that despite actions like the California system's, the
testing industry is likely to survive in some form.

``Standardized testing has been declining as an element in the college
admissions process for some time,'' Mr. Hartle said. ``But the College
Board is a large and financially stable organization, and they've been
around for a long time.''

The move to do away with testing only deepened after last year's
\href{https://www.nytimes.com/2019/03/12/us/college-admissions-cheating-scandal.html}{college
admissions scandal}. More than 1,230 colleges and universities have made
the SAT and ACT optional for admission, according to FairTest, a group
that has pushed to end testing requirements --- most of them small
liberal arts colleges such as Smith, Pitzer and Sarah Lawrence.

Another 70 or so colleges and universities suspended the testing
requirement for the fall application cycle because of the coronavirus.

In California, the board acted Thursday on a
\href{https://regents.universityofcalifornia.edu/regmeet/may20/b4.pdf}{proposal}from
the university system's president, Janet Napolitano, which came after
several years of pressure. A
\href{http://www.publiccounsel.org/tools/assets/files/1250.pdf}{lawsuit
filed last year} by a largely black school district in Compton, Calif.,
and a coalition of students and advocacy groups argues that the
time-honored tests discriminate based on race and income.

The decision, however, ran counter to a recommendation from the system's
faculty senate, which voted in April to keep the SAT and ACT. A faculty
task force commissioned to study the impact of standardized tests found
that they predict college success within the University of California
system more effectively than high school grades or other measures.

\href{https://www.nytimes.com/news-event/coronavirus?action=click\&pgtype=Article\&state=default\&region=MAIN_CONTENT_3\&context=storylines_faq}{}

\hypertarget{the-coronavirus-outbreak-}{%
\subsubsection{The Coronavirus Outbreak
›}\label{the-coronavirus-outbreak-}}

\hypertarget{frequently-asked-questions}{%
\paragraph{Frequently Asked
Questions}\label{frequently-asked-questions}}

Updated July 27, 2020

\begin{itemize}
\item ~
  \hypertarget{should-i-refinance-my-mortgage}{%
  \paragraph{Should I refinance my
  mortgage?}\label{should-i-refinance-my-mortgage}}

  \begin{itemize}
  \tightlist
  \item
    \href{https://www.nytimes.com/article/coronavirus-money-unemployment.html?action=click\&pgtype=Article\&state=default\&region=MAIN_CONTENT_3\&context=storylines_faq}{It
    could be a good idea,} because mortgage rates have
    \href{https://www.nytimes.com/2020/07/16/business/mortgage-rates-below-3-percent.html?action=click\&pgtype=Article\&state=default\&region=MAIN_CONTENT_3\&context=storylines_faq}{never
    been lower.} Refinancing requests have pushed mortgage applications
    to some of the highest levels since 2008, so be prepared to get in
    line. But defaults are also up, so if you're thinking about buying a
    home, be aware that some lenders have tightened their standards.
  \end{itemize}
\item ~
  \hypertarget{what-is-school-going-to-look-like-in-september}{%
  \paragraph{What is school going to look like in
  September?}\label{what-is-school-going-to-look-like-in-september}}

  \begin{itemize}
  \tightlist
  \item
    It is unlikely that many schools will return to a normal schedule
    this fall, requiring the grind of
    \href{https://www.nytimes.com/2020/06/05/us/coronavirus-education-lost-learning.html?action=click\&pgtype=Article\&state=default\&region=MAIN_CONTENT_3\&context=storylines_faq}{online
    learning},
    \href{https://www.nytimes.com/2020/05/29/us/coronavirus-child-care-centers.html?action=click\&pgtype=Article\&state=default\&region=MAIN_CONTENT_3\&context=storylines_faq}{makeshift
    child care} and
    \href{https://www.nytimes.com/2020/06/03/business/economy/coronavirus-working-women.html?action=click\&pgtype=Article\&state=default\&region=MAIN_CONTENT_3\&context=storylines_faq}{stunted
    workdays} to continue. California's two largest public school
    districts --- Los Angeles and San Diego --- said on July 13, that
    \href{https://www.nytimes.com/2020/07/13/us/lausd-san-diego-school-reopening.html?action=click\&pgtype=Article\&state=default\&region=MAIN_CONTENT_3\&context=storylines_faq}{instruction
    will be remote-only in the fall}, citing concerns that surging
    coronavirus infections in their areas pose too dire a risk for
    students and teachers. Together, the two districts enroll some
    825,000 students. They are the largest in the country so far to
    abandon plans for even a partial physical return to classrooms when
    they reopen in August. For other districts, the solution won't be an
    all-or-nothing approach.
    \href{https://bioethics.jhu.edu/research-and-outreach/projects/eschool-initiative/school-policy-tracker/}{Many
    systems}, including the nation's largest, New York City, are
    devising
    \href{https://www.nytimes.com/2020/06/26/us/coronavirus-schools-reopen-fall.html?action=click\&pgtype=Article\&state=default\&region=MAIN_CONTENT_3\&context=storylines_faq}{hybrid
    plans} that involve spending some days in classrooms and other days
    online. There's no national policy on this yet, so check with your
    municipal school system regularly to see what is happening in your
    community.
  \end{itemize}
\item ~
  \hypertarget{is-the-coronavirus-airborne}{%
  \paragraph{Is the coronavirus
  airborne?}\label{is-the-coronavirus-airborne}}

  \begin{itemize}
  \tightlist
  \item
    The coronavirus
    \href{https://www.nytimes.com/2020/07/04/health/239-experts-with-one-big-claim-the-coronavirus-is-airborne.html?action=click\&pgtype=Article\&state=default\&region=MAIN_CONTENT_3\&context=storylines_faq}{can
    stay aloft for hours in tiny droplets in stagnant air}, infecting
    people as they inhale, mounting scientific evidence suggests. This
    risk is highest in crowded indoor spaces with poor ventilation, and
    may help explain super-spreading events reported in meatpacking
    plants, churches and restaurants.
    \href{https://www.nytimes.com/2020/07/06/health/coronavirus-airborne-aerosols.html?action=click\&pgtype=Article\&state=default\&region=MAIN_CONTENT_3\&context=storylines_faq}{It's
    unclear how often the virus is spread} via these tiny droplets, or
    aerosols, compared with larger droplets that are expelled when a
    sick person coughs or sneezes, or transmitted through contact with
    contaminated surfaces, said Linsey Marr, an aerosol expert at
    Virginia Tech. Aerosols are released even when a person without
    symptoms exhales, talks or sings, according to Dr. Marr and more
    than 200 other experts, who
    \href{https://academic.oup.com/cid/article/doi/10.1093/cid/ciaa939/5867798}{have
    outlined the evidence in an open letter to the World Health
    Organization}.
  \end{itemize}
\item ~
  \hypertarget{what-are-the-symptoms-of-coronavirus}{%
  \paragraph{What are the symptoms of
  coronavirus?}\label{what-are-the-symptoms-of-coronavirus}}

  \begin{itemize}
  \tightlist
  \item
    Common symptoms
    \href{https://www.nytimes.com/article/symptoms-coronavirus.html?action=click\&pgtype=Article\&state=default\&region=MAIN_CONTENT_3\&context=storylines_faq}{include
    fever, a dry cough, fatigue and difficulty breathing or shortness of
    breath.} Some of these symptoms overlap with those of the flu,
    making detection difficult, but runny noses and stuffy sinuses are
    less common.
    \href{https://www.nytimes.com/2020/04/27/health/coronavirus-symptoms-cdc.html?action=click\&pgtype=Article\&state=default\&region=MAIN_CONTENT_3\&context=storylines_faq}{The
    C.D.C. has also} added chills, muscle pain, sore throat, headache
    and a new loss of the sense of taste or smell as symptoms to look
    out for. Most people fall ill five to seven days after exposure, but
    symptoms may appear in as few as two days or as many as 14 days.
  \end{itemize}
\item ~
  \hypertarget{does-asymptomatic-transmission-of-covid-19-happen}{%
  \paragraph{Does asymptomatic transmission of Covid-19
  happen?}\label{does-asymptomatic-transmission-of-covid-19-happen}}

  \begin{itemize}
  \tightlist
  \item
    So far, the evidence seems to show it does. A widely cited
    \href{https://www.nature.com/articles/s41591-020-0869-5}{paper}
    published in April suggests that people are most infectious about
    two days before the onset of coronavirus symptoms and estimated that
    44 percent of new infections were a result of transmission from
    people who were not yet showing symptoms. Recently, a top expert at
    the World Health Organization stated that transmission of the
    coronavirus by people who did not have symptoms was ``very rare,''
    \href{https://www.nytimes.com/2020/06/09/world/coronavirus-updates.html?action=click\&pgtype=Article\&state=default\&region=MAIN_CONTENT_3\&context=storylines_faq\#link-1f302e21}{but
    she later walked back that statement.}
  \end{itemize}
\end{itemize}

In fact, the task force found that in many cases the tests gave a leg up
to black, Latino and low-income students by offering an additional
metric for admissions officers who might have rejected them because
their grades did not meet the university's threshold.

Robert May, a philosophy professor at the University of California,
Davis, who appointed the faculty panel, said the regents' decision would
add confusion and significant costs to the admissions process at the
mammoth system, and make admissions determinations even more subjective
in the short-term.

Marten Roorda, the chief executive of the ACT, told the regents in a
letter before the vote that dropping the testing requirements would
``further the uncertainty and anxiety of students and their families at
a time when they need all the reassurances and resources we can
provide.''

Supporters of standardized tests have argued that they provide an
important yardstick to assess students across disparate school districts
and states. And the College Board and ACT say any inequities in their
results reflect existing gaps in the American educational system, and
are not a fault of the tests themselves.

In response to criticism, the College Board proposed a new SAT grading
system last year that came to be known as the ``adversity score,'' which
would put each test taker's results into the context of that student's
school or neighborhood. But the company
\href{https://www.nytimes.com/2019/08/27/us/sat-adversity-score-college-board.html}{withdrew
that proposal} after being criticized for trying to distill complex
factors into a single score.

Testing opponents marked a major victory two years ago when the highly
ranked University of Chicago went test optional. The school reported
last year that the entering freshman class had nearly a quarter more
first-generation and low-income students and 56 percent more rural
students than the prior year, with about 10 percent opting against
submitting test scores.

In the University of California system, standardized test scores are
just one component of a complex admissions formula, which includes more
than a dozen metrics, including high school grade point averages. The
state guarantees acceptance to the top 12.5 percent of California high
school students.

But as California has struggled to maintain campus diversity since
voters passed a 1996 ban on affirmative action, pressure has grown for
the school system to take action. Its top campuses have become almost as
difficult to get into as some Ivy League schools and are demographically
dominated by white and Asian students.

For the last 20 years, black enrollment at University of California
schools has scarcely broken 4 percent, though African-Americans
represent 6.5 percent of the state's population. Nearly 40 percent of
the state is Hispanic, California's largest ethnic group, but only 22
percent of students in the school system are.

Carol Christ, now the Berkeley chancellor, was one of the first
university administrators to eliminate the SAT requirement nearly two
decades ago when she became president of Smith College. On Thursday, she
told the regents that she viewed standardized testing as ``a biased
instrument'' that would only become more skewed in the wake of the
pandemic.

Anemona Hartocollis contributed reporting from New York.

Advertisement

\protect\hyperlink{after-bottom}{Continue reading the main story}

\hypertarget{site-index}{%
\subsection{Site Index}\label{site-index}}

\hypertarget{site-information-navigation}{%
\subsection{Site Information
Navigation}\label{site-information-navigation}}

\begin{itemize}
\tightlist
\item
  \href{https://help.nytimes.com/hc/en-us/articles/115014792127-Copyright-notice}{©~2020~The
  New York Times Company}
\end{itemize}

\begin{itemize}
\tightlist
\item
  \href{https://www.nytco.com/}{NYTCo}
\item
  \href{https://help.nytimes.com/hc/en-us/articles/115015385887-Contact-Us}{Contact
  Us}
\item
  \href{https://www.nytco.com/careers/}{Work with us}
\item
  \href{https://nytmediakit.com/}{Advertise}
\item
  \href{http://www.tbrandstudio.com/}{T Brand Studio}
\item
  \href{https://www.nytimes.com/privacy/cookie-policy\#how-do-i-manage-trackers}{Your
  Ad Choices}
\item
  \href{https://www.nytimes.com/privacy}{Privacy}
\item
  \href{https://help.nytimes.com/hc/en-us/articles/115014893428-Terms-of-service}{Terms
  of Service}
\item
  \href{https://help.nytimes.com/hc/en-us/articles/115014893968-Terms-of-sale}{Terms
  of Sale}
\item
  \href{https://spiderbites.nytimes.com}{Site Map}
\item
  \href{https://help.nytimes.com/hc/en-us}{Help}
\item
  \href{https://www.nytimes.com/subscription?campaignId=37WXW}{Subscriptions}
\end{itemize}
