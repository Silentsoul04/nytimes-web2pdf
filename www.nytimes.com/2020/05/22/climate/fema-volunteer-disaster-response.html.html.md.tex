Sections

SEARCH

\protect\hyperlink{site-content}{Skip to
content}\protect\hyperlink{site-index}{Skip to site index}

\href{https://www.nytimes.com/section/climate}{Climate}

\href{https://myaccount.nytimes.com/auth/login?response_type=cookie\&client_id=vi}{}

\href{https://www.nytimes.com/section/todayspaper}{Today's Paper}

\href{/section/climate}{Climate}\textbar{}Virus Crisis Exposes Cascading
Weaknesses in U.S. Disaster Response

\url{https://nyti.ms/2Tv736k}

\begin{itemize}
\item
\item
\item
\item
\item
\item
\end{itemize}

\href{https://www.nytimes.com/section/climate?action=click\&pgtype=Article\&state=default\&region=TOP_BANNER\&context=storylines_menu}{Climate
and Environment}

\begin{itemize}
\tightlist
\item
  \href{https://www.nytimes.com/2020/07/30/climate/sea-level-inland-floods.html?action=click\&pgtype=Article\&state=default\&region=TOP_BANNER\&context=storylines_menu}{Rising
  Seas}
\item
  \href{https://www.nytimes.com/interactive/2020/climate/trump-environment-rollbacks.html?action=click\&pgtype=Article\&state=default\&region=TOP_BANNER\&context=storylines_menu}{Trump's
  Changes}
\item
  \href{https://www.nytimes.com/interactive/2020/04/19/climate/climate-crash-course-1.html?action=click\&pgtype=Article\&state=default\&region=TOP_BANNER\&context=storylines_menu}{Climate
  101}
\item
  \href{https://www.nytimes.com/interactive/2018/08/30/climate/how-much-hotter-is-your-hometown.html?action=click\&pgtype=Article\&state=default\&region=TOP_BANNER\&context=storylines_menu}{Is
  Your Hometown Hotter?}
\item
  \href{https://www.nytimes.com/newsletters/climate-change?action=click\&pgtype=Article\&state=default\&region=TOP_BANNER\&context=storylines_menu}{Newsletter}
\end{itemize}

Advertisement

\protect\hyperlink{after-top}{Continue reading the main story}

Supported by

\protect\hyperlink{after-sponsor}{Continue reading the main story}

\hypertarget{virus-crisis-exposes-cascading-weaknesses-in-us-disaster-response}{%
\section{Virus Crisis Exposes Cascading Weaknesses in U.S. Disaster
Response}\label{virus-crisis-exposes-cascading-weaknesses-in-us-disaster-response}}

\includegraphics{https://static01.nyt.com/images/2020/05/22/climate/22CLI-FEMA1/22CLI-FEMA1-articleLarge.jpg?quality=75\&auto=webp\&disable=upscale}

\href{https://www.nytimes.com/by/christopher-flavelle}{\includegraphics{https://static01.nyt.com/images/2019/06/28/climate/author-chris-flavelle/author-chris-flavelle-thumbLarge-v3.png}}

By \href{https://www.nytimes.com/by/christopher-flavelle}{Christopher
Flavelle}

\begin{itemize}
\item
  May 22, 2020
\item
  \begin{itemize}
  \item
  \item
  \item
  \item
  \item
  \item
  \end{itemize}
\end{itemize}

WASHINGTON --- For decades, the backbone of the nation's disaster
response system --- and a hallmark of American generosity --- has been
its army of volunteers who race toward danger to help shelter, feed and
counsel victims of hurricanes, wildfires and other calamities.

However, the Covid-19 pandemic has exposed a critical weakness in this
system: Most volunteers are older people at higher risk from the virus,
so this year they can't participate in person. Typically more than five
million volunteers work in disaster relief annually, said Greg
Forrester, president of National Voluntary Organizations Active in
Disasters, an association of nonprofit groups, but this year he expects
the number to decline by 50 percent.

Asked how disaster relief efforts can meet the usual demand with half as
many people, Mr. Forrester said: ``You won't.''

It is the latest in a cascading series of problems facing an already
fraying system ahead of what is expected to be an
\href{https://www.nytimes.com/2020/05/21/climate/hurricane-season-2020-noaa.html}{unusually
severe} hurricane season combined with disasters like this week's dam
collapse and flooding in Michigan, a state particularly hard hit by
Covid-19.

The Federal Emergency Management Agency is
\href{https://www.nytimes.com/2020/04/03/climate/fema-staff-shortage-coronavirus.html}{running
short} of highly trained personnel as the virus depletes its staff.
Longstanding procedures for sheltering victims in gymnasiums or other
crowded spaces
\href{https://www.nytimes.com/2020/03/21/climate/virus-fema-disaster-aid-shelter.html}{suddenly
are dangerous} because they risk worsening the pandemic. And traditional
agreements among states to help each other if crisis strikes are now
sputtering as states remain wary of exposing their own people to the
virus.

It amounts to one of the most severe tests in decades for a system
designed to respond to local or regional storms or other disasters ---
not a crisis on a national scale. Yet FEMA has been forced to take a
primary role in Covid-19, deploying more than 3,000 staff nationwide and
effectively running its first 50-state disaster response.

``A pandemic complicates every aspect of disaster planning and response
in a way that we have never experienced before,'' said Chris Currie, who
leads the team at the nonpartisan Government Accountability Office that
looks at emergency management. ``You're only as good as the weakest
link.''

FEMA says it has taken steps to prepare for hurricane season, including
expanding its coordination center in Washington, hiring staff and
working with state and local officials and nonprofits to adapt to the
pandemic. ``We have not taken our eye off the ball about handling other
disasters that may occur during this time,'' Peter Gaynor, FEMA's
administrator, said in a briefing this month.

On Wednesday, the agency said it intended to avoid, as much as possible,
sending relief staff into disaster zones this year, instead relying on
``virtual'' assistance such as talking to survivors by phone, using
photos or other documentation of storm damage to approve claims and
meeting with state and local counterparts online rather than in person.

\includegraphics{https://static01.nyt.com/images/2020/05/22/climate/22CLI-FEMA2/merlin_170845608_9276114d-fed5-432f-bdb5-5939a08e0d3f-articleLarge.jpg?quality=75\&auto=webp\&disable=upscale}

Volunteers are key to America's disaster response, distributing
supplies, clearing debris, and rebuilding homes. In interviews,
executives with the nonprofit organizations like the Salvation Army that
help organize volunteer teams said that, in normal years, they would be
training and equipping thousands of people and flying them to whichever
part of the country needs help, then housing and feeding them in close
quarters.

\href{https://www.nytimes.com/section/climate?action=click\&pgtype=Article\&state=default\&region=MAIN_CONTENT_1\&context=storylines_keepup}{}

\hypertarget{climate-and-environment-}{%
\subsubsection{Climate and Environment
›}\label{climate-and-environment-}}

\hypertarget{keep-up-on-the-latest-climate-news}{%
\paragraph{Keep Up on the Latest Climate
News}\label{keep-up-on-the-latest-climate-news}}

Updated July 30, 2020

Here's what you need to know about the latest climate change news this
week:

\begin{itemize}
\item
  \begin{itemize}
  \tightlist
  \item
    \href{https://www.nytimes.com/2020/07/30/climate/bangladesh-floods.html?action=click\&pgtype=Article\&state=default\&region=MAIN_CONTENT_1\&context=storylines_keepup}{Floods
    in}\href{https://www.nytimes.com/2020/07/30/climate/bangladesh-floods.html?action=click\&pgtype=Article\&state=default\&region=MAIN_CONTENT_1\&context=storylines_keepup}{Bangladesh}
    are punishing the people least responsible for climate change.
  \item
    As climate change raises sea levels,
    \href{https://www.nytimes.com/2020/07/30/climate/sea-level-inland-floods.html?action=click\&pgtype=Article\&state=default\&region=MAIN_CONTENT_1\&context=storylines_keepup}{storm
    surges and high tides} are likely to push farther inland.
  \item
    The E.P.A. inspector general plans to investigate whether a rollback
    of fuel efficiency standards
    \href{https://www.nytimes.com/2020/07/27/climate/trump-fuel-efficiency-rule.html?action=click\&pgtype=Article\&state=default\&region=MAIN_CONTENT_1\&context=storylines_keepup}{violated
    government rules}.
  \end{itemize}
\end{itemize}

Suddenly, none of that works.

Three-quarters of the Salvation Army's volunteers for most disasters are
65 or older, according to Jeff Jellets, the group's disaster coordinator
for the southern United States. For those people, ``We're telling them,
maybe this isn't the best time for you to deploy,'' he said, given that
older people are at particularly high risk from Covid-19.

The consequences could be enormous: The Salvation Army has more than 2.7
million volunteers annually for everything from disaster response to
after-school programs and vocational programs. Disaster volunteers
worked 3.5 million hours during the 2017 hurricane season.

The Salvation Army is considering using more paid staff and housing them
in hotels rather than dormitories. But that's expensive, Mr. Jellets
said, and the pandemic has closed many of the Salvation Army's thrift
stores, which bring in almost in \$600 million annually in sales.

Habitat for Humanity, which last year helped rebuild or repair almost
700 homes damaged by disasters in the United States, also gets many of
its volunteers from older Americans, according to Jonathan Reckford, the
chief executive officer. Given the risks of air travel combined with the
danger that volunteers inadvertently bring the disease into a community
they're trying to help, Mr. Reckford said Habitat for Humanity had hit
pause, for now, on deploying any volunteers.

Overall, the organization fielded 1.2 million volunteers last year for
all its work. It did not break out a number for disaster response.

That means its group quite likely won't be able to respond the way it
usually does if a hurricane were to strike the United States this year.
``It's our greatest fear right now,'' Mr. Reckford said.

If a disaster struck a part of the country that was under large-scale
quarantine, ``we would really have to back away from some of our
response in those areas,'' Mary Casey-Lockyer, a senior associate with
the disaster health program for the American Red Cross, said during a
webinar for nonprofits last week. The Red Cross deployed 9,000 workers
to large disasters last year; it expects to deploy half as many
volunteers as usual in person this year.

``I don't want to imagine a world where it's so bad we can't respond,''
added Cathy Earl, director of disaster response for the United Methodist
Committee on Relief, which has 10,000 volunteers around the country who
work on disaster response. She said it was hard to project how many
volunteers would be deployed this year, but called a 50 percent decrease
``a reasonable estimate.''

The volunteer shortage threatens to ripple through the nation's disaster
response system, exacerbating other problems.

One spillover effect will be financial. Under federal law, state or
local governments typically have to put up \$25 for every \$75 the
federal government provides for disaster relief. But they're allowed to
count the services of volunteers toward that amount, Mr. Forrester said.

As a result, fewer volunteers means cities, counties and states need to
come up with more of their own money to get federal aid.

But local governments are already struggling financially from the virus.
Counties alone have seen
\href{https://www.naco.org/sites/default/files/documents/NACo_COVID-19_Fiscal_Impact_Analysis-Executive_Summary_0.pdf}{\$144
billion} in lost income and increased expenditures, more than one-fifth
of their total budgets, according to the National Association of
Counties. ``Our costs are skyrocketing and our revenues are
plummeting,'' said Paul Guequierre, a spokesman for the association.

Image

Volunteers arranged cots at the Dallas Convention Center for people
displaced by Hurricane Harvey in 2017.Credit...Jim Wilson/The New York
Times

At the same time, the federal government is asking local officials to
take on new tasks.

One of the toughest challenges will be evacuating and sheltering people
\href{https://www.nytimes.com/2020/03/21/climate/virus-fema-disaster-aid-shelter.html}{without
spreading the virus}. This week following the dam collapse in Michigan,
Gov. Gretchen Whitmer
\href{https://www.nytimes.com/2020/05/19/us/michigan-dam-breach.html?action=click\&module=Top\%20Stories\&pgtype=Homepage}{acknowledged}
that social distancing in shelters would be difficult.

This week FEMA
\href{https://www.fema.gov/media-library-data/1589997234798-adb5ce5cb98a7a89e3e1800becf0eb65/2020_Hurricane_Pandemic_Plan.pdf}{advised}
state and local governments to find backup sources for supplies, find
ways to distribute them without physical contact, figure out how to stop
disaster survivors from gathering in groups, and to do all that with
``diminished support'' from volunteers.

In its new guidance, FEMA also laid out a host of new challenges facing
disaster shelters. Local officials, it said, must find more space, and
come up with a plan to shelter people with Covid-19.

FEMA even urged local officials to revise their plans for dealing with
disaster victims' pets, since spacing rules at shelters means there
might not be room for them.

When states don't have enough people to respond to a disaster, they
usually start by asking other states to send their own emergency
management teams. But with Covid-19, ``They're not sure what they might
need in their own states,'' said Joyce Flinn, Iowa's emergency
management director and head of the committee at the National Emergency
Management Association that oversees the
\href{https://www.bloomberg.com/news/articles/2017-10-06/fewer-states-sent-help-to-puerto-rico-than-to-texas-or-florida?sref=UBrhZ1ro}{mutual-aid
system}.

When those options prove inadequate, cities and states are meant to turn
to FEMA for support. However, the agency was already
\href{https://www.gao.gov/products/GAO-20-360}{stretched thin} as
climate change makes disasters more frequent and intense. The virus
crisis has stretched it
\href{https://www.nytimes.com/2020/04/03/climate/fema-staff-shortage-coronavirus.html}{further}.

Brock Long, who headed FEMA during the catastrophic hurricanes and
wildfires of 2017 and 2018, said there was only so much the agency's own
people could do. ``They're like the sixth man coming off the bench in a
basketball game, down by 20, and being told to win the game,'' Mr. Long
said. ``We win and lose together.''

Advertisement

\protect\hyperlink{after-bottom}{Continue reading the main story}

\hypertarget{site-index}{%
\subsection{Site Index}\label{site-index}}

\hypertarget{site-information-navigation}{%
\subsection{Site Information
Navigation}\label{site-information-navigation}}

\begin{itemize}
\tightlist
\item
  \href{https://help.nytimes.com/hc/en-us/articles/115014792127-Copyright-notice}{©~2020~The
  New York Times Company}
\end{itemize}

\begin{itemize}
\tightlist
\item
  \href{https://www.nytco.com/}{NYTCo}
\item
  \href{https://help.nytimes.com/hc/en-us/articles/115015385887-Contact-Us}{Contact
  Us}
\item
  \href{https://www.nytco.com/careers/}{Work with us}
\item
  \href{https://nytmediakit.com/}{Advertise}
\item
  \href{http://www.tbrandstudio.com/}{T Brand Studio}
\item
  \href{https://www.nytimes.com/privacy/cookie-policy\#how-do-i-manage-trackers}{Your
  Ad Choices}
\item
  \href{https://www.nytimes.com/privacy}{Privacy}
\item
  \href{https://help.nytimes.com/hc/en-us/articles/115014893428-Terms-of-service}{Terms
  of Service}
\item
  \href{https://help.nytimes.com/hc/en-us/articles/115014893968-Terms-of-sale}{Terms
  of Sale}
\item
  \href{https://spiderbites.nytimes.com}{Site Map}
\item
  \href{https://help.nytimes.com/hc/en-us}{Help}
\item
  \href{https://www.nytimes.com/subscription?campaignId=37WXW}{Subscriptions}
\end{itemize}
