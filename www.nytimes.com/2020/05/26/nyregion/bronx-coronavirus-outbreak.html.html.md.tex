\href{/section/nyregion}{New York}\textbar{}`It's the Death Towers': How
the Bronx Became New York's Virus Hot Spot

\url{https://nyti.ms/2LXr6GB}

\begin{itemize}
\item
\item
\item
\item
\item
\item
\end{itemize}

\href{https://www.nytimes.com/news-event/coronavirus?action=click\&pgtype=Article\&state=default\&region=TOP_BANNER\&context=storylines_menu}{The
Coronavirus Outbreak}

\begin{itemize}
\tightlist
\item
  live\href{https://www.nytimes.com/2020/08/02/world/coronavirus-updates.html?action=click\&pgtype=Article\&state=default\&region=TOP_BANNER\&context=storylines_menu}{Latest
  Updates}
\item
  \href{https://www.nytimes.com/interactive/2020/us/coronavirus-us-cases.html?action=click\&pgtype=Article\&state=default\&region=TOP_BANNER\&context=storylines_menu}{Maps
  and Cases}
\item
  \href{https://www.nytimes.com/interactive/2020/science/coronavirus-vaccine-tracker.html?action=click\&pgtype=Article\&state=default\&region=TOP_BANNER\&context=storylines_menu}{Vaccine
  Tracker}
\item
  \href{https://www.nytimes.com/interactive/2020/07/29/us/schools-reopening-coronavirus.html?action=click\&pgtype=Article\&state=default\&region=TOP_BANNER\&context=storylines_menu}{What
  School May Look Like}
\item
  \href{https://www.nytimes.com/live/2020/07/31/business/stock-market-today-coronavirus?action=click\&pgtype=Article\&state=default\&region=TOP_BANNER\&context=storylines_menu}{Economy}
\end{itemize}

\includegraphics{https://static01.nyt.com/images/2020/05/25/nyregion/00nyvirus-bronx2/00nyvirus-bronx1-articleLarge.jpg?quality=75\&auto=webp\&disable=upscale}

Sections

\protect\hyperlink{site-content}{Skip to
content}\protect\hyperlink{site-index}{Skip to site index}

\hypertarget{its-the-death-towers-how-the-bronx-became-new-yorks-virus-hot-spot}{%
\section{`It's the Death Towers': How the Bronx Became New York's Virus
Hot
Spot}\label{its-the-death-towers-how-the-bronx-became-new-yorks-virus-hot-spot}}

The borough has the city's highest rates of virus cases,
hospitalizations and deaths. Could more have been done?

``What about us in the Bronx?'' asked Vilmarie Maldonado, 42, and her
daughter, Carolina Santos, 7, outside River Park Towers, where they
live. ``We need more help.''Credit...

Supported by

\protect\hyperlink{after-sponsor}{Continue reading the main story}

By \href{https://www.nytimes.com/by/kimiko-de-freytas-tamura}{Kimiko de
Freytas-Tamura}, \href{https://www.nytimes.com/by/winnie-hu}{Winnie Hu}
and Lindsey Rogers Cook

Photographs by Gabriela Bhaskar

\begin{itemize}
\item
  May 26, 2020
\item
  \begin{itemize}
  \item
  \item
  \item
  \item
  \item
  \item
  \end{itemize}
\end{itemize}

Working on the front lines of the coronavirus pandemic can be hazardous,
but staying home isn't safe either for the emergency responders,
pharmacists, home health aides, grocery clerks and delivery men who fill
River Park Towers in the Bronx.

Even a ride down the elevator is risky. Residents often must wait up to
an hour to squeeze into small, poorly ventilated cars that break down
frequently, with people crowding the hallways like commuters trying to
push into the subway at rush hour.

There is talk that as many as 100 residents have been sickened by the
coronavirus at the two massive towers rising above the Morris Heights
neighborhood along the Harlem River. But no one knows for sure, since
the leader of the tenant association died from Covid-19 in April.

``It's the death towers, you could say that,'' said Maria Lopez, 42, a
resident with a variety of health issues, including asthma, who has
watched 10 of her neighbors being taken away by paramedics.

One of the worst health crises in a century has exploded across New York
City, and it has inflicted the
\href{https://www.nytimes.com/2020/06/24/nyregion/coronavirus-public-housing-new-york.html}{worst
toll on the Bronx}, the city's poorest borough.

\includegraphics{https://static01.nyt.com/images/2020/05/25/nyregion/00nyvirus-bronx3/00nyvirus-bronx6-articleLarge.jpg?quality=75\&auto=webp\&disable=upscale}

It has spread building by building in neighborhoods like Morris Heights
that have been unable to fight back, reflecting a legacy of
institutionalized racism, poverty, cramped housing and chronic health
problems that have put their residents at higher risk of getting sick
and dying.

The Bronx has the
\href{https://www.nytimes.com/2020/05/18/nyregion/coronavirus-deaths-nyc.html}{highest}rates\href{https://www.nytimes.com/2020/05/18/nyregion/coronavirus-deaths-nyc.html}{of
coronavirus cases, hospitalizations and deaths} in the city, while the
most well-off borough, Manhattan, has the lowest rates.

In just months, the coronavirus has threatened to wipe out more than a
decade of efforts to rebuild the Bronx with new development and
businesses and has made life even more precarious for those already
struggling to survive, including low-paid essential workers without
health insurance, paid sick time or unions to back them.

The economic fallout has shuttered stores, restaurants and businesses
across the borough and left thousands out of work, struggling to pay
rent and buy food. As in the other boroughs, unemployment claims have
surged in the Bronx --- by mid-May they had skyrocketed 2,000 percent
from a year ago. One Bronx economic development official warned that up
to half the borough's restaurants may never reopen.

The crisis has stirred frustration and resentment among Bronx residents
angered that, once again, they are the ones paying the highest price.
The Bronx has long struggled to attract attention and resources. During
the economic boom of the past decade, it lagged behind the rest of the
city in many indicators, including poverty and unemployment.

``This will happen again. This is not the last pandemic,'' said Ruben
Diaz Jr., the Bronx borough president, who counts at least three deaths
in his own apartment complex in the Soundview neighborhood. ``How do we
remedy institutionalized neglect in communities like the Bronx so in the
future we have a fighting chance?''

Image

``I think in our neighborhood we need all the attention we can get,''
said Robert Smith, 41, who has lived in River Park for about 20 years.

Image

``Put some kind of sign up,'' said Paula Givens, 60, saying that
residents in the buildings should be urged to take precautions against
the virus.

The coronavirus has been
\href{https://www1.nyc.gov/site/doh/covid/covid-19-data.page}{particularly
deadly} in the Bronx because race and income are key factors in who
survives and who does not. At least 4,400 confirmed and probable
Covid-19 deaths in the Bronx have been reported as of May 26.

\hypertarget{latest-updates-global-coronavirus-outbreak}{%
\section{\texorpdfstring{\href{https://www.nytimes.com/2020/08/01/world/coronavirus-covid-19.html?action=click\&pgtype=Article\&state=default\&region=MAIN_CONTENT_1\&context=storylines_live_updates}{Latest
Updates: Global Coronavirus
Outbreak}}{Latest Updates: Global Coronavirus Outbreak}}\label{latest-updates-global-coronavirus-outbreak}}

Updated 2020-08-02T17:52:35.962Z

\begin{itemize}
\tightlist
\item
  \href{https://www.nytimes.com/2020/08/01/world/coronavirus-covid-19.html?action=click\&pgtype=Article\&state=default\&region=MAIN_CONTENT_1\&context=storylines_live_updates\#link-34047410}{The
  U.S. reels as July cases more than double the total of any other
  month.}
\item
  \href{https://www.nytimes.com/2020/08/01/world/coronavirus-covid-19.html?action=click\&pgtype=Article\&state=default\&region=MAIN_CONTENT_1\&context=storylines_live_updates\#link-780ec966}{Top
  U.S. officials work to break an impasse over the federal jobless
  benefit.}
\item
  \href{https://www.nytimes.com/2020/08/01/world/coronavirus-covid-19.html?action=click\&pgtype=Article\&state=default\&region=MAIN_CONTENT_1\&context=storylines_live_updates\#link-2bc8948}{Its
  outbreak untamed, Melbourne goes into even greater lockdown.}
\end{itemize}

\href{https://www.nytimes.com/2020/08/01/world/coronavirus-covid-19.html?action=click\&pgtype=Article\&state=default\&region=MAIN_CONTENT_1\&context=storylines_live_updates}{See
more updates}

More live coverage:
\href{https://www.nytimes.com/live/2020/07/31/business/stock-market-today-coronavirus?action=click\&pgtype=Article\&state=default\&region=MAIN_CONTENT_1\&context=storylines_live_updates}{Markets}

Across the city, neighborhoods with large numbers of black, Latino or
poor residents have the
\href{https://www.nytimes.com/2020/05/18/nyregion/coronavirus-deaths-nyc.html}{highest
death rates}. In the Bronx, about 90 percent of the borough's 1.4
million residents are people of color, the highest concentration in the
city,
\href{https://www.census.gov/quickfacts/fact/table/newyorkcitynewyork,bronxcountybronxboroughnewyork,kingscountybrooklynboroughnewyork,newyorkcountymanhattanboroughnewyork,queenscountyqueensboroughnewyork,richmondcountystatenislandboroughnewyork/PST045218}{according
to census data}.

Many public health experts and Bronx officials say more should have been
done to protect vulnerable communities. City and state leaders, they
say, should have aggressively conducted testing to slow the spread of
the virus, deployed more services and resources and focused on
overlooked front-line workers who have kept stores open and the city
running.

``We as a state and as a city could have done better,'' said State
Assemblyman Victor M. Pichardo, whose parents were both sickened by the
virus. ``We're sort of picking up the pieces now.''

Mr. Pichardo said it was not until May --- two months after the pandemic
shut down the city --- that he received more than 4,000 masks from the
mayor's office, and 300 bottles of hand sanitizer from the governor's
office, for his district, which includes Morris Heights.

Image

Victor Pichardo, a state assemblyman from the Bronx whose parents were
both sickened by the virus, handed out food and face masks to residents
last week.

City and state officials acknowledged the challenges the Bronx faced,
but said they were constrained early on in the outbreak by limited
testing capacity and resources and focused on prioritizing health care
and emergency workers. They have since expanded testing sites in the
Bronx and opened up testing to anyone who has symptoms.

``The Covid-19 crisis has exacerbated disparities that have existed for
far too long in our city,'' said Avery Cohen, a spokeswoman for Mayor
Bill de Blasio. ``We have put equity at the forefront of our plan to
treat the virus and safely reopen the city. From opening community
testing sites across the city, to our grass-roots outreach with
community providers in the Bronx, and supporting NYC Health + Hospitals,
we remain focused on saving lives and ensuring that all New Yorkers
receive the care they deserve.''

Adults in the Bronx have the highest rates in the city of asthma,
diabetes and high blood pressure, all of which can lead to severe
complications for people who are infected with the coronavirus. Roughly
one in three Bronx adults is
\href{https://a816-health.nyc.gov/hdi/epiquery/visualizations?PageType=tsi\&PopulationSource=CHS\&Topic=1\&Subtopic=24\&Indicator=Overweight\%20and\%20Obesity\&Year=2017}{obese}
--- another factor that can make the virus worse --- and lack of ready
access to healthy foods makes it difficult for people to change their
diets.

Of those Bronx residents who died from Covid-19, 90 percent had at least
one such underlying condition.

The Bronx
\href{https://www.countyhealthrankings.org/app/new-york/2019/rankings/outcomes/overall}{ranked
last} among New York State's 62 counties in an annual survey of health
indicators. And life expectancy in the Bronx is about five years lower
than in Manhattan.

``What Covid-19 really shines a very harsh light on are the historical
inequities in socioeconomic status and structural racism that are really
driving disparities in health outcomes,'' said Nadia S. Islam, an
associate professor of population health at New York University's
Grossman School of Medicine.

Image

A public housing complex in the Morris Heights neighborhood in the
Bronx.~The Bronx has the highest rates of coronavirus cases,
hospitalizations and deaths in the city.

The city's northern borough once drew well-to-do families to its
stretches of parkland and Art Deco apartment buildings. But in the
1970s, arson fires, rampant crime and poverty drove out residents, and
turned the borough into a national symbol of urban decay.

Today, the Bronx is home to the nation's poorest congressional district.
Median household income is \$38,000, compared with \$82,000 in Manhattan
and about \$61,000 citywide.

Still, in recent years there were signs of a revival: The Bronx has
attracted retail stores, hotels, start-up companies and manufacturers as
well as an influx of newcomers priced out of Manhattan. The unemployment
rate dropped to 4.7 percent in February before rising again to 5.7
percent in March, the latest data available.

Marlene Cintron, the president of the Bronx Overall Economic Development
Corporation, which provides loans and services to businesses, said she
hears every day from stores, restaurants and businesses that may not be
able to reopen. At Hunts Point, a regional food hub, some distributors
are owed millions of dollars by restaurants and hotels that have been
unable to pay their bills, she said.

Lines at food pantries have been growing, and elected officials and
community groups have been giving out free bags of groceries.

``It's very worrisome,'' said Dr. Bola Omotosho, the chairman of the
community board that includes Morris Heights. ``People are not able to
pay rent --- whether it's for their apartments or for their
businesses.''

Morris Heights is a hilly enclave in the West Bronx, where hip-hop is
said to have been born in the 1970s. About 40 percent of the area's
residents are poor, and almost all are black or Latino.

River Park Towers, which encompasses two 44-floor buildings, was built
in the 1970s for lower-income and working families. Rents for the 1,654
apartments go up to \$1,978 for a four-bedroom. More than 70 percent of
the 5,000 tenants receive rent subsidies, a spokesman for the complex
said.

Life in the towers has always been hard, residents said, recalling
drive-by shootings in the 1980s. While violence has decreased in recent
years, gangs and illegal drugs remain a problem, and there was a
stabbing recently in a building lobby, they said.

Image

More than 70 percent of the 5,000 tenants at River Park Towers rely on
rent subsidies.~

But the coronavirus has brought new worries. Large groups walk around
without masks. One janitor complained that he did not have enough
protective gear.

``The chances of getting Covid here are greater than at work,'' said
Sandra Williams, 37, a nursing aide who commutes to a Brooklyn nursing
home.

For Margarita Brown, 48, a pharmacist technician, getting ready for work
is ``like preparing for war.'' She puts on a mask and gloves before
getting into elevators ``packed like sardines.'' She has to go around
crowds in the lobby before riding a bus to a pharmacy on the Upper West
Side of Manhattan.

``Then you get to your job, but you're not being appreciated,'' said Ms.
Brown, who has no health insurance. ``It's so stressful.''

\href{https://www.nytimes.com/news-event/coronavirus?action=click\&pgtype=Article\&state=default\&region=MAIN_CONTENT_3\&context=storylines_faq}{}

\hypertarget{the-coronavirus-outbreak-}{%
\subsubsection{The Coronavirus Outbreak
›}\label{the-coronavirus-outbreak-}}

\hypertarget{frequently-asked-questions}{%
\paragraph{Frequently Asked
Questions}\label{frequently-asked-questions}}

Updated July 27, 2020

\begin{itemize}
\item ~
  \hypertarget{should-i-refinance-my-mortgage}{%
  \paragraph{Should I refinance my
  mortgage?}\label{should-i-refinance-my-mortgage}}

  \begin{itemize}
  \tightlist
  \item
    \href{https://www.nytimes.com/article/coronavirus-money-unemployment.html?action=click\&pgtype=Article\&state=default\&region=MAIN_CONTENT_3\&context=storylines_faq}{It
    could be a good idea,} because mortgage rates have
    \href{https://www.nytimes.com/2020/07/16/business/mortgage-rates-below-3-percent.html?action=click\&pgtype=Article\&state=default\&region=MAIN_CONTENT_3\&context=storylines_faq}{never
    been lower.} Refinancing requests have pushed mortgage applications
    to some of the highest levels since 2008, so be prepared to get in
    line. But defaults are also up, so if you're thinking about buying a
    home, be aware that some lenders have tightened their standards.
  \end{itemize}
\item ~
  \hypertarget{what-is-school-going-to-look-like-in-september}{%
  \paragraph{What is school going to look like in
  September?}\label{what-is-school-going-to-look-like-in-september}}

  \begin{itemize}
  \tightlist
  \item
    It is unlikely that many schools will return to a normal schedule
    this fall, requiring the grind of
    \href{https://www.nytimes.com/2020/06/05/us/coronavirus-education-lost-learning.html?action=click\&pgtype=Article\&state=default\&region=MAIN_CONTENT_3\&context=storylines_faq}{online
    learning},
    \href{https://www.nytimes.com/2020/05/29/us/coronavirus-child-care-centers.html?action=click\&pgtype=Article\&state=default\&region=MAIN_CONTENT_3\&context=storylines_faq}{makeshift
    child care} and
    \href{https://www.nytimes.com/2020/06/03/business/economy/coronavirus-working-women.html?action=click\&pgtype=Article\&state=default\&region=MAIN_CONTENT_3\&context=storylines_faq}{stunted
    workdays} to continue. California's two largest public school
    districts --- Los Angeles and San Diego --- said on July 13, that
    \href{https://www.nytimes.com/2020/07/13/us/lausd-san-diego-school-reopening.html?action=click\&pgtype=Article\&state=default\&region=MAIN_CONTENT_3\&context=storylines_faq}{instruction
    will be remote-only in the fall}, citing concerns that surging
    coronavirus infections in their areas pose too dire a risk for
    students and teachers. Together, the two districts enroll some
    825,000 students. They are the largest in the country so far to
    abandon plans for even a partial physical return to classrooms when
    they reopen in August. For other districts, the solution won't be an
    all-or-nothing approach.
    \href{https://bioethics.jhu.edu/research-and-outreach/projects/eschool-initiative/school-policy-tracker/}{Many
    systems}, including the nation's largest, New York City, are
    devising
    \href{https://www.nytimes.com/2020/06/26/us/coronavirus-schools-reopen-fall.html?action=click\&pgtype=Article\&state=default\&region=MAIN_CONTENT_3\&context=storylines_faq}{hybrid
    plans} that involve spending some days in classrooms and other days
    online. There's no national policy on this yet, so check with your
    municipal school system regularly to see what is happening in your
    community.
  \end{itemize}
\item ~
  \hypertarget{is-the-coronavirus-airborne}{%
  \paragraph{Is the coronavirus
  airborne?}\label{is-the-coronavirus-airborne}}

  \begin{itemize}
  \tightlist
  \item
    The coronavirus
    \href{https://www.nytimes.com/2020/07/04/health/239-experts-with-one-big-claim-the-coronavirus-is-airborne.html?action=click\&pgtype=Article\&state=default\&region=MAIN_CONTENT_3\&context=storylines_faq}{can
    stay aloft for hours in tiny droplets in stagnant air}, infecting
    people as they inhale, mounting scientific evidence suggests. This
    risk is highest in crowded indoor spaces with poor ventilation, and
    may help explain super-spreading events reported in meatpacking
    plants, churches and restaurants.
    \href{https://www.nytimes.com/2020/07/06/health/coronavirus-airborne-aerosols.html?action=click\&pgtype=Article\&state=default\&region=MAIN_CONTENT_3\&context=storylines_faq}{It's
    unclear how often the virus is spread} via these tiny droplets, or
    aerosols, compared with larger droplets that are expelled when a
    sick person coughs or sneezes, or transmitted through contact with
    contaminated surfaces, said Linsey Marr, an aerosol expert at
    Virginia Tech. Aerosols are released even when a person without
    symptoms exhales, talks or sings, according to Dr. Marr and more
    than 200 other experts, who
    \href{https://academic.oup.com/cid/article/doi/10.1093/cid/ciaa939/5867798}{have
    outlined the evidence in an open letter to the World Health
    Organization}.
  \end{itemize}
\item ~
  \hypertarget{what-are-the-symptoms-of-coronavirus}{%
  \paragraph{What are the symptoms of
  coronavirus?}\label{what-are-the-symptoms-of-coronavirus}}

  \begin{itemize}
  \tightlist
  \item
    Common symptoms
    \href{https://www.nytimes.com/article/symptoms-coronavirus.html?action=click\&pgtype=Article\&state=default\&region=MAIN_CONTENT_3\&context=storylines_faq}{include
    fever, a dry cough, fatigue and difficulty breathing or shortness of
    breath.} Some of these symptoms overlap with those of the flu,
    making detection difficult, but runny noses and stuffy sinuses are
    less common.
    \href{https://www.nytimes.com/2020/04/27/health/coronavirus-symptoms-cdc.html?action=click\&pgtype=Article\&state=default\&region=MAIN_CONTENT_3\&context=storylines_faq}{The
    C.D.C. has also} added chills, muscle pain, sore throat, headache
    and a new loss of the sense of taste or smell as symptoms to look
    out for. Most people fall ill five to seven days after exposure, but
    symptoms may appear in as few as two days or as many as 14 days.
  \end{itemize}
\item ~
  \hypertarget{does-asymptomatic-transmission-of-covid-19-happen}{%
  \paragraph{Does asymptomatic transmission of Covid-19
  happen?}\label{does-asymptomatic-transmission-of-covid-19-happen}}

  \begin{itemize}
  \tightlist
  \item
    So far, the evidence seems to show it does. A widely cited
    \href{https://www.nature.com/articles/s41591-020-0869-5}{paper}
    published in April suggests that people are most infectious about
    two days before the onset of coronavirus symptoms and estimated that
    44 percent of new infections were a result of transmission from
    people who were not yet showing symptoms. Recently, a top expert at
    the World Health Organization stated that transmission of the
    coronavirus by people who did not have symptoms was ``very rare,''
    \href{https://www.nytimes.com/2020/06/09/world/coronavirus-updates.html?action=click\&pgtype=Article\&state=default\&region=MAIN_CONTENT_3\&context=storylines_faq\#link-1f302e21}{but
    she later walked back that statement.}
  \end{itemize}
\end{itemize}

Ronn Torossian, a spokesman for Reliant Realty Services and Omni New
York, which owns River Park Towers, said the complex had tried to
protect residents, including conducting ``commercial-grade cleaning'' of
the buildings multiple times a day. In addition, he said, adjustments
had been made to staff and spaces ``to accommodate proper
social-distancing protocols.''

But, he said, landlords had received no guidance from the city about
what to do to keep housing safe. ``We had to learn all about it, just
like you guys, by watching the news,'' he said.

Image

``They are not sanitizing anything,'' said June Colon, 37, who lives in
River Park Towers. ``Everything is dirty.''

Image

Elizabeth Tartt, 57, said many building residents did not know where to
buy masks. ``I had to purchase them and then the prices started going
up,'' she said.

City officials said they have held or taken part in more than 40
meetings and events in the Bronx to keep residents informed about the
virus, and worked with health care workers and community organizations.

In Morris Heights alone, they said, they have given out 75,000 face
coverings, including 10,000 specifically for River Park Towers.

But many Bronx leaders and public health experts said the city and state
should have realized sooner that the virus was especially dangerous to
higher-risk communities and taken more steps to try to stem the spread,
like expanding testing and making it more accessible.

``We could have anticipated that the Bronx and other communities of need
would have a higher infection rate,'' said Diana Hernandez, an assistant
professor of sociomedical sciences at Columbia University's Mailman
School of Public Health and a resident of the South Bronx. ``It was
really a missed opportunity to do testing, contact tracing and outreach
to high-risk populations in a more targeted fashion.''

The first state-operated testing site in the Bronx was a drive-through
center that opened March 23 at Lehman College, which made it difficult
for anyone who did not have a car --- or money for a taxi or Uber.

Bronx hospitals were rapidly overwhelmed with people trying to get
tested.

The Morris Heights Health Center set up its own temporary testing sites
in April. Nearly a third of the more than 1,000 people tested were
positive. ``We made every attempt to get ahead of it as quickly as
possible,'' said Mari G. Millet, the center's president and chief
executive, adding the center also gave out masks and introduced
telemedicine visits so older patients could stay home.

Image

At the Morris Heights Health Center, more than a third of those tested
were positive for the coronavirus.

Of the center's 58,000 regular patients, 80 have been hospitalized, and
12 have died from Covid-19 since March 1, she said.

State officials said the drive-through testing centers were a way to
provide testing safely and quickly in hard-hit areas. By mid-April they
had opened the state's first walk-in testing center in the Bronx,
followed by a second site a week later, they said.

More than 140,000 people have been tested in the Bronx, according to the
governor's office, and the Bronx has received more tests per capita than
Manhattan, Brooklyn and Queens.

The state has also given out more than 2 million pieces of personal
protective equipment in the Bronx, including masks, gloves, gowns and
face shields. Nearly 100,000 bottles of hand sanitizer have been sent to
the Bronx, starting with hospitals in mid-March.

``The Bronx has been a priority since the start of this pandemic,'' said
Robert Mujica, the state budget director, who is part of the governor's
virus task force.

City officials have also expanded testing and are working to provide
people who test positive with hotel rooms so they can remain isolated,
which is not always possible in crowded Bronx apartments.

But in Morris Heights, many residents said they felt they had been left
to fend off the virus on their own.

By the time that masks and hand sanitizer were given out this month at
Karen Adams's public housing complex, it was too late. Ms. Adams, 65,
had already gotten the virus.

``They should make things available to us, but they didn't,'' she said.
``The white community, they had everything that they needed. They had
access to things.''

Image

The manager of a deli at River Park Towers said he regularly has to
admonish younger people to wear masks.

At River Park Towers, the manager of a deli on the grounds said he has
to tell young people to put on masks because the security guards and
management do not.

``The effort was not there,'' said the manager, Nelson Pier, 34, as he
admonished a teenager from behind the counter. ``Yo, put a mask on!''

Up on the 43rd floor, Ms. Lopez, who has asthma, regularly sprays the
strip of floor between the front door of her apartment and the elevator
with disinfectant.

``I'm scared but I can't afford to get sick,'' said Ms. Lopez, showing
rashes on her hands from the disinfectant.

``We're the black sheep put in the corner,'' she said. ``The city, the
government, they have forgotten us.''

Image

The two 44-story buildings that make up River Park Towers stand above
the Harlem River.~

Gabriela Bhaskar contributed reporting.

Advertisement

\protect\hyperlink{after-bottom}{Continue reading the main story}

\hypertarget{site-index}{%
\subsection{Site Index}\label{site-index}}

\hypertarget{site-information-navigation}{%
\subsection{Site Information
Navigation}\label{site-information-navigation}}

\begin{itemize}
\tightlist
\item
  \href{https://help.nytimes.com/hc/en-us/articles/115014792127-Copyright-notice}{©~2020~The
  New York Times Company}
\end{itemize}

\begin{itemize}
\tightlist
\item
  \href{https://www.nytco.com/}{NYTCo}
\item
  \href{https://help.nytimes.com/hc/en-us/articles/115015385887-Contact-Us}{Contact
  Us}
\item
  \href{https://www.nytco.com/careers/}{Work with us}
\item
  \href{https://nytmediakit.com/}{Advertise}
\item
  \href{http://www.tbrandstudio.com/}{T Brand Studio}
\item
  \href{https://www.nytimes.com/privacy/cookie-policy\#how-do-i-manage-trackers}{Your
  Ad Choices}
\item
  \href{https://www.nytimes.com/privacy}{Privacy}
\item
  \href{https://help.nytimes.com/hc/en-us/articles/115014893428-Terms-of-service}{Terms
  of Service}
\item
  \href{https://help.nytimes.com/hc/en-us/articles/115014893968-Terms-of-sale}{Terms
  of Sale}
\item
  \href{https://spiderbites.nytimes.com}{Site Map}
\item
  \href{https://help.nytimes.com/hc/en-us}{Help}
\item
  \href{https://www.nytimes.com/subscription?campaignId=37WXW}{Subscriptions}
\end{itemize}
