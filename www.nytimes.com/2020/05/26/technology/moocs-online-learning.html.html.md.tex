Sections

SEARCH

\protect\hyperlink{site-content}{Skip to
content}\protect\hyperlink{site-index}{Skip to site index}

\href{https://www.nytimes.com/section/technology}{Technology}

\href{https://myaccount.nytimes.com/auth/login?response_type=cookie\&client_id=vi}{}

\href{https://www.nytimes.com/section/todayspaper}{Today's Paper}

\href{/section/technology}{Technology}\textbar{}Remember the MOOCs?
After Near-Death, They're Booming

\url{https://nyti.ms/3c3yKd9}

\begin{itemize}
\item
\item
\item
\item
\item
\item
\end{itemize}

\href{https://www.nytimes.com/news-event/coronavirus?action=click\&pgtype=Article\&state=default\&region=TOP_BANNER\&context=storylines_menu}{The
Coronavirus Outbreak}

\begin{itemize}
\tightlist
\item
  live\href{https://www.nytimes.com/2020/08/01/world/coronavirus-covid-19.html?action=click\&pgtype=Article\&state=default\&region=TOP_BANNER\&context=storylines_menu}{Latest
  Updates}
\item
  \href{https://www.nytimes.com/interactive/2020/us/coronavirus-us-cases.html?action=click\&pgtype=Article\&state=default\&region=TOP_BANNER\&context=storylines_menu}{Maps
  and Cases}
\item
  \href{https://www.nytimes.com/interactive/2020/science/coronavirus-vaccine-tracker.html?action=click\&pgtype=Article\&state=default\&region=TOP_BANNER\&context=storylines_menu}{Vaccine
  Tracker}
\item
  \href{https://www.nytimes.com/interactive/2020/07/29/us/schools-reopening-coronavirus.html?action=click\&pgtype=Article\&state=default\&region=TOP_BANNER\&context=storylines_menu}{What
  School May Look Like}
\item
  \href{https://www.nytimes.com/live/2020/07/31/business/stock-market-today-coronavirus?action=click\&pgtype=Article\&state=default\&region=TOP_BANNER\&context=storylines_menu}{Economy}
\end{itemize}

Advertisement

\protect\hyperlink{after-top}{Continue reading the main story}

Supported by

\protect\hyperlink{after-sponsor}{Continue reading the main story}

\hypertarget{remember-the-moocs-after-near-death-theyre-booming}{%
\section{Remember the MOOCs? After Near-Death, They're
Booming}\label{remember-the-moocs-after-near-death-theyre-booming}}

The pioneering online learning networks offer hard-earned lessons for
what works and what doesn't with online education.

\includegraphics{https://static01.nyt.com/images/2020/05/20/business/VIRUS-MOOCS1/merlin_172473702_79fc669f-247a-4260-b67c-a1e4fac274ee-articleLarge.jpg?quality=75\&auto=webp\&disable=upscale}

\href{https://www.nytimes.com/by/steve-lohr}{\includegraphics{https://static01.nyt.com/images/2018/02/20/multimedia/author-steve-lohr/author-steve-lohr-thumbLarge.jpg}}

By \href{https://www.nytimes.com/by/steve-lohr}{Steve Lohr}

\begin{itemize}
\item
  May 26, 2020
\item
  \begin{itemize}
  \item
  \item
  \item
  \item
  \item
  \item
  \end{itemize}
\end{itemize}

Sandeep Gupta, a technology manager in California, sees the economic
storm caused by the coronavirus as a time ``to try to future-proof your
working life.'' So he is taking an online course in artificial
intelligence.

Dr. Robert Davidson, an emergency-room physician in Michigan, says the
pandemic has cast ``a glaring light on the shortcomings of our public
health infrastructure.'' So he is pursuing an online master's degree in
public health.

Children and college students aren't the only ones turning to online
education during the coronavirus pandemic. Millions of adults have
signed up for online classes in the last two months, too --- a jolt that
could signal a renaissance for big online learning networks that had
struggled for years.

Coursera, in which Mr. Gupta and Dr. Davidson enrolled, added 10 million
new users from mid-March to mid-May, seven times the pace of new
sign-ups in the previous year. Enrollments at edX and Udacity, two
smaller education sites, have jumped by similar multiples.

``Crises lead to accelerations, and this is best chance ever for online
learning,'' said Sebastian Thrun, a co-founder and chairman of Udacity.

Coursera, Udacity and edX sprang up nearly a decade ago as high-profile
university experiments known as MOOCs, for massive open online courses.
They were portrayed as tech-fueled insurgents destined to disrupt the
antiquated ways of traditional higher education. But few people
completed courses, grappling with the same challenges now facing
students forced into distance learning because of the pandemic. Screen
fatigue sets in, and attention strays.

The sites even became a punchline among academics: ``Remember the
MOOCs?''

\includegraphics{https://static01.nyt.com/images/2020/05/20/business/00virus-moocs2/merlin_66789686_bb4e6734-1152-4309-aa77-221048894bd5-articleLarge.jpg?quality=75\&auto=webp\&disable=upscale}

But the online ventures adapted through trial and error, gathering
lessons that could provide a road map for schools districts and
universities pushed online. The instructional ingredients of success,
the sites found, include short videos of six minutes or less,
interspersed with interactive drills and tests; online forums where
students share problems and suggestions; and online mentoring and
tutoring.

``Active learning works, and social learning works,'' said Anant
Agarwal, founder and chief executive of edX. ``And you have to
understand that teaching online and learning online are skills of their
own.''

The proclaimed mission of the MOOCs was to ``democratize education.''
The early courses attracted hundreds of thousands of students from
around the world.

\hypertarget{latest-updates-economy}{%
\section{\texorpdfstring{\href{https://www.nytimes.com/live/2020/07/31/business/stock-market-today-coronavirus?action=click\&pgtype=Article\&state=default\&region=MAIN_CONTENT_1\&context=storylines_live_updates}{Latest
Updates:
Economy}}{Latest Updates: Economy}}\label{latest-updates-economy}}

\href{https://www.nytimes.com/live/2020/07/31/business/stock-market-today-coronavirus?action=click\&pgtype=Article\&state=default\&region=MAIN_CONTENT_1\&context=storylines_live_updates\#kodaks-chief-executive-was-given-stock-options-then-the-share-price-spiked-1000-percent}{34h
ago}

\href{https://www.nytimes.com/live/2020/07/31/business/stock-market-today-coronavirus?action=click\&pgtype=Article\&state=default\&region=MAIN_CONTENT_1\&context=storylines_live_updates\#kodaks-chief-executive-was-given-stock-options-then-the-share-price-spiked-1000-percent}{Kodak's
chief executive was given stock options. Then the share price spiked
1,000 percent.}

\href{https://www.nytimes.com/live/2020/07/31/business/stock-market-today-coronavirus?action=click\&pgtype=Article\&state=default\&region=MAIN_CONTENT_1\&context=storylines_live_updates\#fitch-ratings-downgrades-its-outlook-on-us-debt}{37h
ago}

\href{https://www.nytimes.com/live/2020/07/31/business/stock-market-today-coronavirus?action=click\&pgtype=Article\&state=default\&region=MAIN_CONTENT_1\&context=storylines_live_updates\#fitch-ratings-downgrades-its-outlook-on-us-debt}{Fitch
Ratings downgrades its outlook on U.S. debt.}

\href{https://www.nytimes.com/live/2020/07/31/business/stock-market-today-coronavirus?action=click\&pgtype=Article\&state=default\&region=MAIN_CONTENT_1\&context=storylines_live_updates\#us-sanctions-more-chinese-officials-over-human-rights-violations-as-tensions-flare}{43h
ago}

\href{https://www.nytimes.com/live/2020/07/31/business/stock-market-today-coronavirus?action=click\&pgtype=Article\&state=default\&region=MAIN_CONTENT_1\&context=storylines_live_updates\#us-sanctions-more-chinese-officials-over-human-rights-violations-as-tensions-flare}{U.S.
sanctions more Chinese officials over human rights violations as
tensions flare}

\href{https://www.nytimes.com/live/2020/07/31/business/stock-market-today-coronavirus?action=click\&pgtype=Article\&state=default\&region=MAIN_CONTENT_1\&context=storylines_live_updates}{See
more updates}

More live coverage:
\href{https://www.nytimes.com/2020/08/01/world/coronavirus-covid-19.html?action=click\&pgtype=Article\&state=default\&region=MAIN_CONTENT_1\&context=storylines_live_updates}{Global}

Udacity and Coursera were founded at Stanford University by high-profile
professors in the hot field of artificial intelligence. EdX, created by
the Massachusetts Institute of Technology and Harvard University in
2012, is a nonprofit.

Coursera and Udacity soon attracted money from Silicon Valley's leading
venture firms. The courses were all free. It was the classic internet
formula: lure a big audience, and figure out a business model later.

Executives eventually discovered that earning credentials for completing
courses and paying fees drove completion rates far higher. Typically, 10
percent or fewer students complete free courses, while the completion
rates for paid courses that grant certificates or degrees range from 40
percent to 90 percent.

A few top-tier universities, such as the University of Michigan and the
Georgia Institute of Technology, offer some full degree programs through
the online platforms. Dr. Davidson is taking Michigan's public health
course.

Image

Coursera added 10 million new users from mid-March to
mid-May.Credit...Jessica Chou for The New York Times

While those academic programs are available, the online schools have
tilted, either cautiously or wholeheartedly, toward skills-focused
courses that match student demand and hiring trends.

``Our main goal is to solve learning, not the skills problem,'' Mr.
Agarwal said. ``Though frankly, that's where the money is.''

Udacity has made the most drastic transformation toward a skills
factory. It has developed dozens of courses on its own and with
corporate collaborators including Google, Amazon and Mercedes. Its
course offerings are largely in digital skills like programming, data
science and artificial intelligence, fields where companies say they
need workers.

``Companies are better positioned to see where the jobs of tomorrow will
be and prepare people for them than universities,'' Mr. Thrun said.

Just a couple of years ago, Udacity's survival was in doubt. When Mr.
Thrun returned to oversee operations in 2018, it was a few months from
running out of cash. Over the next two years, Mr. Thrun laid off about
half the work force. ``The worst period of my life,'' he recalled.

Today, with 320 employees and 1,300 part-time project reviewers and
mentors, Udacity's fortunes have improved. It is tightly focused on its
training business, for both individual students and for corporations
that pay Udacity to upgrade the skills of their employees and to advise
them on redeploying workers in digital operations.

The Udacity courses, which it calls nanodegrees, take most students four
to six months to complete, if they put in 10 hours a week. The average
cost is \$1,200. The learning is based on projects, rapid feedback ---
including project reviews in two hours --- and online mentoring.

David Hundley has taken several Udacity courses in data science and
machine learning in the last two years. A business analyst at State
Farm, he wanted to develop tech skills for a better job and brighter
career prospects.

Today, Mr. Hundley, 30, is proficient in modern software tools like
Python and TensorFlow and has a portfolio of projects on GitHub, where
software developers display their work. In January, he landed a new job
at the insurance company as a machine-learning engineer.

State Farm paid for a couple of the Udacity courses, and he paid for the
others. ``It was a hundred-percent worth it,'' Mr. Hundley said. ``Two
years ago, I didn't know anything about coding. Now, I'm a
machine-learning engineer.''

Coursera is a hybrid, retaining much of the character of the original
MOOCs, while striving to build a sizable business.

Coursera has raised more than \$300 million in venture funding over the
years. It hosts more than 4,000 courses, created mainly by university
professors but also by companies like Google and IBM. The certificate
courses are typically priced at \$39 to \$79 a month, or a \$399 annual
fee. University master's degree programs start at \$15,000 and go up to
\$40,000.

But fewer than 10 percent of Coursera students pay for courses; they
take them free. That is part of the company's mixed model of offering
both free and paid-for learning options, said Jeff Maggioncalda, chief
executive of Coursera, noting that 60 percent of students in its degree
programs try free courses first.

Some of the most popular courses are not about writing code or making
money. The breakout hit of the pandemic season is ``The Science of
Well-Being'' by Laurie Santos, a professor of psychology at Yale
University.

Image

Jeff Maggioncalda, chief executive of Coursera.Credit...Jessica Chou for
The New York Times

Mr. Maggioncalda describes Coursera as a ``managed marketplace,''
similar in concept to Apple's app store. Coursera determines which
institutions get to publish courses on its platform, and has rules and
guidelines for format standards and price ranges. For degree courses,
universities collect 60 percent of the revenue and Coursera 40 percent.
On certificate courses, mainly in technology and business subjects, the
split is 50-50.

The millions of people flocking to take courses on Coursera in recent
weeks suggest the brand value of its learning network. About half the
company's 600 employees are product managers, engineers and data
scientists, working to improve the online learning experience and more
effectively market the university courses.

Before the pandemic hit, Coursera projected growth of 30 percent this
year, to more than \$200 million. That forecast looks decidedly
outdated, given the surge in the last two months, but how long the trend
lasts is uncertain.

The Covid-19 effect on online learning could broaden the range of
popular subjects, education experts say. But so far, training for the
tech economy is where the digital-learning money lies. With more of work
and everyday life moving online --- some of it permanently --- that will
probably not change.

``Digital-skills jobs will be where there is greatest demand,'' Mr.
Maggioncalda said, ``and those jobs will be less likely to be affected
by pandemics in the future.''

Advertisement

\protect\hyperlink{after-bottom}{Continue reading the main story}

\hypertarget{site-index}{%
\subsection{Site Index}\label{site-index}}

\hypertarget{site-information-navigation}{%
\subsection{Site Information
Navigation}\label{site-information-navigation}}

\begin{itemize}
\tightlist
\item
  \href{https://help.nytimes.com/hc/en-us/articles/115014792127-Copyright-notice}{©~2020~The
  New York Times Company}
\end{itemize}

\begin{itemize}
\tightlist
\item
  \href{https://www.nytco.com/}{NYTCo}
\item
  \href{https://help.nytimes.com/hc/en-us/articles/115015385887-Contact-Us}{Contact
  Us}
\item
  \href{https://www.nytco.com/careers/}{Work with us}
\item
  \href{https://nytmediakit.com/}{Advertise}
\item
  \href{http://www.tbrandstudio.com/}{T Brand Studio}
\item
  \href{https://www.nytimes.com/privacy/cookie-policy\#how-do-i-manage-trackers}{Your
  Ad Choices}
\item
  \href{https://www.nytimes.com/privacy}{Privacy}
\item
  \href{https://help.nytimes.com/hc/en-us/articles/115014893428-Terms-of-service}{Terms
  of Service}
\item
  \href{https://help.nytimes.com/hc/en-us/articles/115014893968-Terms-of-sale}{Terms
  of Sale}
\item
  \href{https://spiderbites.nytimes.com}{Site Map}
\item
  \href{https://help.nytimes.com/hc/en-us}{Help}
\item
  \href{https://www.nytimes.com/subscription?campaignId=37WXW}{Subscriptions}
\end{itemize}
