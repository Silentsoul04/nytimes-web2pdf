Sections

SEARCH

\protect\hyperlink{site-content}{Skip to
content}\protect\hyperlink{site-index}{Skip to site index}

\href{https://myaccount.nytimes.com/auth/login?response_type=cookie\&client_id=vi}{}

\href{https://www.nytimes.com/section/todayspaper}{Today's Paper}

\href{/section/opinion}{Opinion}\textbar{}Trump's Economic Advisers Are
Wrong

\href{https://nyti.ms/3gqRvKJ}{https://nyti.ms/3gqRvKJ}

\begin{itemize}
\item
\item
\item
\item
\item
\item
\end{itemize}

Advertisement

\protect\hyperlink{after-top}{Continue reading the main story}

\href{/section/opinion}{Opinion}

Supported by

\protect\hyperlink{after-sponsor}{Continue reading the main story}

\hypertarget{trumps-economic-advisers-are-wrong}{%
\section{Trump's Economic Advisers Are
Wrong}\label{trumps-economic-advisers-are-wrong}}

Forget a V-shaped recovery, or even a U.

\href{https://www.nytimes.com/topic/person/steven-rattner}{\includegraphics{https://static01.nyt.com/images/2013/10/04/opinion/rattner-contributor/rattner-contributor-thumbLarge-v5.png}}

By \href{https://www.nytimes.com/topic/person/steven-rattner}{Steven
Rattner}

Mr. Rattner served as counselor to the Treasury secretary in the Obama
administration.

\begin{itemize}
\item
  May 27, 2020
\item
  \begin{itemize}
  \item
  \item
  \item
  \item
  \item
  \item
  \end{itemize}
\end{itemize}

\includegraphics{https://static01.nyt.com/images/2020/05/27/opinion/27Rattner/27Rattner-articleLarge.jpg?quality=75\&auto=webp\&disable=upscale}

Like dutiful soldiers, President Trump's top economic advisers have been
ardently echoing their
\href{https://www.nytimes.com/aponline/2020/05/24/business/bc-us-election-2020-trump-2021-comeback.html}{boss's
optimism} about the likelihood of a quick economic snapback, while
downplaying the need for another rescue package.

Regrettably, that's dead wrong on both counts, as even Mr. Trump's pick
to head the Federal Reserve, Jerome Powell, has
\href{https://www.nytimes.com/2020/05/19/us/politics/mnuchin-powell-senate-hearing.html}{countered
repeatedly}.

Forget a V-shaped recovery, or even a U. Think instead about a backward
check mark: an economy that begins to spring upward as Americans return
to work but loses momentum well before reaching past levels.

That's why we urgently require more help from Washington, and
particularly, an effort to rebuild America, not just rescue America.

The prospect of large numbers of Americans returning to their workplaces
seems to recede like a mirage; a full reopening remains invisibly far in
the distance. Equally important, many
pre-\href{https://www.nytimes.com/2020/05/28/business/economy/coronavirus-stimulus-unemployment.html}{Covid-19}
jobs won't be around. They are gone, indefinitely if not for good.

Just a few examples:

Factories in states from Oregon to North Carolina are
\href{https://www.wsj.com/articles/factories-close-for-good-as-coronavirus-cuts-demand-11589122800}{closing
permanently}. Many shuttered restaurants will never reopen. A growing
number of bankruptcies, in industries ranging
\href{https://www.nytimes.com/2020/04/21/business/coronavirus-department-stores-neiman-marcus.html?searchResultPosition=3}{from
retailing} to
\href{https://www.wsj.com/articles/owner-of-cmx-cinemas-files-for-bankruptcy-protection-11587942170}{entertainment},
will mean more jobs lost forever. And chief executives tell me regularly
that they are cutting capital expenditures, another drag on economic
recovery and employment.

The rescue packages to date have shoveled trillions of dollars into the
economy to, among other things, keep consumers and small businesses
afloat, provide much needed liquidity to lending markets and bail out
special pleaders like the airlines.

That will help, but more will be needed, particularly for state and
local governments and
\href{https://www.nytimes.com/2020/05/28/business/economy/coronavirus-stimulus-unemployment.html}{unemployed}
workers, as House Democrats tried to emphasize in the
\href{https://www.nytimes.com/2020/05/15/us/politics/house-simulus-vote.html}{bill
passed} 12 days ago.

But now we also must focus urgently on longer-term initiatives that will
have deeper and more lasting positive effects on jobs and growth.

Infrastructure should top the list of priorities. During his campaign,
\href{https://www.nytimes.com/2016/08/03/us/politics/trump-clinton-infrastructure.html}{President
Trump promised us} a \$1 trillion initiative but then never made a
serious effort to achieve passage in Congress.

Moving forward on modernizing the nation's physical plant --- from roads
to buildings --- makes particular sense now and not just because of weak
economic conditions. At the moment, the government can borrow money for
30 years at record low rates of about 1.4 percent.

Meanwhile, China has just announced a five-year, \$1.4 trillion
\href{https://www.bloomberg.com/news/articles/2020-05-20/china-has-a-new-1-4-trillion-plan-to-overtake-the-u-s-in-tech?sref=qN0DZypA}{infrastructure
plan} heavily focused on technology in a bid to become the global leader
in this critical sector; why can't we even get started on rebuilding
America?

In designing a good plan, Congress must first and foremost eschew the
logrolling/favor-trading/call-it-what-you-want process of allocating
funds that is too often a part of the congressional appropriations
process. Among the effective ways of accomplishing that would be to
create an ``infrastructure bank'' that would be staffed and overseen by
private-sector individuals chosen on a bipartisan basis. Its mandate
would be to fund the projects that would add the most to economic growth
and productivity.

That doesn't mean the projects need to generate profits or even
revenues; improving roads, for example, increases the efficiency of our
economy and that in turn increases the attractiveness to companies of
locating new jobs here.

Accordingly, Congress would need to increase the bank's funding from
time to time, which is fine with me.

We also face a human capital challenge. As recent
\href{https://www.nytimes.com/aponline/2020/05/08/business/bc-us-economy-jobs-report-inequality.html}{economic
reports} have made clear, Americans near the bottom of the income scale,
who are disproportionately members of minorities, are suffering most
from the downturn. And women have been losing their jobs faster than
men.

Similarly, we're learning that technology-enabled businesses (like Zoom)
are going to be winners from this pandemic.

We should also be funding retraining and relocation costs for workers
whose jobs have been lost to the pandemic. And with college applications
\href{https://www.wsj.com/articles/fewer-students-apply-for-college-financial-aid-a-sign-coronavirus-may-disrupt-enrollment-11589284806}{already
falling}, we should be providing more financial support to jobless
Americans who have chosen to return to school to raise their skill set.

Some experts have suggested trying to restore a vibrant economy via more
robust intervention by government in the private sector. I've not been a
fan of industrial policy, and I'm still leery of Washington trying to
pick winners.

However, at this unique moment in our history, we should think big and
think creatively. To get us started and guide us, we should create a
blue-ribbon panel of individuals with strong economic credentials and
task them with developing good ideas for how government can support a
recovery in a post-Covid-19 world.

These priorities stand in stark contrast to the thin gruel from
Republicans. White House economic advisers have been teasing the idea of
still more tax cuts for business and the rich that wouldn't help the
broader economy*.*

For their part, Treasury Secretary Steven Mnuchin swiftly
\href{https://www.bloomberg.com/news/articles/2020-05-14/mnuchin-seeks-to-assuage-investors-after-powell-s-gloomy-outlook?sref=qN0DZypA}{pushed
back} on Mr. Powell's warning, then called for a 30-day pause on new
efforts. After
\href{https://www.politico.com/news/2020/05/20/mcconnell-unemployment-benefits-271661}{endorsing
a pause}, Mitch McConnell, the Senate majority leader, has begun
\href{https://www.politico.com/newsletters/morning-money/2020/05/27/mcconnell-says-another-relief-package-is-coming-787905}{edging
toward} a modest package. That could not be more wrong. Drafting
thoughtful legislation takes time; we are already late in starting to
rebuild America

\emph{The Times is committed to publishing}
\href{https://www.nytimes.com/2019/01/31/opinion/letters/letters-to-editor-new-york-times-women.html}{\emph{a
diversity of letters}} \emph{to the editor. We'd like to hear what you
think about this or any of our articles. Here are some}
\href{https://help.nytimes.com/hc/en-us/articles/115014925288-How-to-submit-a-letter-to-the-editor}{\emph{tips}}\emph{.
And here's our email:}
\href{mailto:letters@nytimes.com}{\emph{letters@nytimes.com}}\emph{.}

\emph{Follow The New York Times Opinion section on}
\href{https://www.facebook.com/nytopinion}{\emph{Facebook}}\emph{,}
\href{http://twitter.com/NYTOpinion}{\emph{Twitter (@NYTopinion)}}
\emph{and}
\href{https://www.instagram.com/nytopinion/}{\emph{Instagram}}\emph{.}

Advertisement

\protect\hyperlink{after-bottom}{Continue reading the main story}

\hypertarget{site-index}{%
\subsection{Site Index}\label{site-index}}

\hypertarget{site-information-navigation}{%
\subsection{Site Information
Navigation}\label{site-information-navigation}}

\begin{itemize}
\tightlist
\item
  \href{https://help.nytimes.com/hc/en-us/articles/115014792127-Copyright-notice}{©~2020~The
  New York Times Company}
\end{itemize}

\begin{itemize}
\tightlist
\item
  \href{https://www.nytco.com/}{NYTCo}
\item
  \href{https://help.nytimes.com/hc/en-us/articles/115015385887-Contact-Us}{Contact
  Us}
\item
  \href{https://www.nytco.com/careers/}{Work with us}
\item
  \href{https://nytmediakit.com/}{Advertise}
\item
  \href{http://www.tbrandstudio.com/}{T Brand Studio}
\item
  \href{https://www.nytimes.com/privacy/cookie-policy\#how-do-i-manage-trackers}{Your
  Ad Choices}
\item
  \href{https://www.nytimes.com/privacy}{Privacy}
\item
  \href{https://help.nytimes.com/hc/en-us/articles/115014893428-Terms-of-service}{Terms
  of Service}
\item
  \href{https://help.nytimes.com/hc/en-us/articles/115014893968-Terms-of-sale}{Terms
  of Sale}
\item
  \href{https://spiderbites.nytimes.com}{Site Map}
\item
  \href{https://help.nytimes.com/hc/en-us}{Help}
\item
  \href{https://www.nytimes.com/subscription?campaignId=37WXW}{Subscriptions}
\end{itemize}
