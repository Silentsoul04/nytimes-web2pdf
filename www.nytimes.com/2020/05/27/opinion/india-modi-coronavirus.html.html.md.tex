Sections

SEARCH

\protect\hyperlink{site-content}{Skip to
content}\protect\hyperlink{site-index}{Skip to site index}

\href{https://myaccount.nytimes.com/auth/login?response_type=cookie\&client_id=vi}{}

\href{https://www.nytimes.com/section/todayspaper}{Today's Paper}

\href{/section/opinion}{Opinion}\textbar{}How Modi Failed the Pandemic
Test

\url{https://nyti.ms/2X8XLiV}

\begin{itemize}
\item
\item
\item
\item
\item
\end{itemize}

Advertisement

\protect\hyperlink{after-top}{Continue reading the main story}

\href{/section/opinion}{Opinion}

Supported by

\protect\hyperlink{after-sponsor}{Continue reading the main story}

\hypertarget{how-modi-failed-the-pandemic-test}{%
\section{How Modi Failed the Pandemic
Test}\label{how-modi-failed-the-pandemic-test}}

The starkest failure of the Indian government's coronavirus strategy has
been the devastation it imposed on the country's workers.

By Hartosh Singh Bal

Mr. Bal is the political editor of The Caravan magazine in New Delhi.

\begin{itemize}
\item
  May 27, 2020
\item
  \begin{itemize}
  \item
  \item
  \item
  \item
  \item
  \end{itemize}
\end{itemize}

\includegraphics{https://static01.nyt.com/images/2020/05/27/opinion/27Bal/27Bal-articleLarge.jpg?quality=75\&auto=webp\&disable=upscale}

NEW DELHI --- India has been under a lockdown to stem the spread of the
coronavirus for two months. On March 25, the first day of the lockdown,
India had
\href{https://www.thehindu.com/news/national/india-coronavirus-lockdown-day-1-updates-march-25-2020/article31159466.ece}{618
confirmed cases} and 13 deaths.

As India is easing the lockdown now, it has
\href{https://www.livemint.com/news/india/coronavirus-update-covid-19-cases-in-india-surge-to-1-45-lakh-over-6-500-new-cases-in-last-24-hours-11590462824986.html}{more
than 151,000 cases} and more than 4,300 deaths --- a much smaller number
compared with the fatalities in the United States and various European
countries, with a much smaller population. The cases rose from 100 to
100,000 in the United States in 25 days, in Britain in 42 days.

In India, which had the longest and
\href{https://timesofindia.indiatimes.com/india/covid-19-cases-in-india-climbed-to-1-lakh-from-100-in-64-days/articleshow/75820314.cms}{strictest
lockdown}, the rise in cases from 100 to 100,000 took 64 days.

It may suggest the success of the Indian government's strategy, but the
almost similar trajectory of spread of the virus and fatality rates in
Bangladesh and Pakistan suggests that other factors have had a
considerable role to play.

Of the 30 countries that have registered more than 25,000 coronavirus
cases, India, Pakistan and Bangladesh are among the countries with
\href{https://www.worldometers.info/coronavirus/}{the lowest levels of
testing per million} people, which raises questions about whether
statistics on the slow spread of the pandemic in South Asia are a result
only of the lack of testing.

But the low fatality rates in South Asia seem to be real because no
evidence has surfaced of large-scale underreporting of deaths across
India, Pakistan and Bangladesh. It seems that a major factor explaining
the lower fatality rates in South Asia is demographics. The median age
in India is 29 years, 23 in Pakistan and 27 in Bangladesh, while the
median age is 38 in the United States, 40.5 in Britain and 45 in Italy.

Data from
\href{https://www1.nyc.gov/assets/doh/downloads/pdf/imm/covid-19-daily-data-summary-deaths-05132020-1.pdf}{New
York City} indicates that 73 percent of all coronavirus deaths have
occurred among patients who were age 65 or older. Only 5 percent of
\href{https://censusindia.gov.in/vital_statistics/SRS_Report/9Chap\%202\%20-\%202011.pdf}{the
Indian population} falls in this age group, as opposed to 16 percent of
the American population.

The rate of hospitalization and death has been lower than in Europe and
the United States. In Britain and Italy, the
\href{https://www.aljazeera.com/news/2020/05/curious-case-south-asia-coronavirus-deaths-200518090320358.html?fbclid=IwAR1FGYaCMqywUKsVYAht_5mStk5eMh9qqheAz30rtutRr9kkJioC8fwhqCQ}{percentage
of deaths} among coronavirus patients has been 14.4 percent and 14
percent, while it has been 3.3 percent in India and 2.2 percent in
Pakistan, where the median age is 22.8 years.

The most significant question now is whether the government has made
adequate use of the time the long lockdown bought it to prepare for the
problems that lie ahead. The number of cases is rising steadily and is
expected to continue to rise as the lockdown is now being lifted in the
face of increasing pressure on the economy.

India's health minister has claimed that India has 31,250
\href{http://www.newsonair.com/Main-News-Details.aspx?id=389635}{intensive-care
unit beds}, which is up from 9,500 at the beginning of the lockdown. The
health ministry
\href{https://www.hindustantimes.com/india-news/number-of-critical-covid-19-patients-has-halved-across-the-country-shows-data/story-kucHQKQkI1TBc0sZO0SleI.html}{said
earlier} that 4.8 percent of Covid-19 patients have required the
intensive-care unit beds. It suggests that coronavirus cases would have
to increase fivefold for India to run out of beds to treat patients
requiring critical care. But aggregate data does not take into account
the fact that the infections are not evenly spread across the country.

A staggering 60 percent of the
\href{https://timesofindia.indiatimes.com/india/80-covid-cases-from-5-states-60-from-5-cities-govt/articleshow/75907372.cms}{coronavirus
cases in India} have been reported from five cities --- Mumbai, Delhi,
Ahmedabad, Chennai and Pune. India's commercial capital, Mumbai, and
Pune are already
\href{https://indianexpress.com/article/cities/mumbai/mumbai-hospitals-run-out-of-beds-for-critical-covid-patients-6407221/}{running
out of hospital beds} for critical patients.
\href{https://www.hindustantimes.com/india-news/sikkim-reports-first-case-number-of-infections-cross-50-000-in-maharashtra-covid-19-state-tally/story-4QGIQ0kg4Jb5wleLniN8dN.html}{Mumbai
is the worst hit} and accounts for more than 20 percent of all cases,
and the rise in cases is outpacing the ability of the city to ramp up
\href{https://indianexpress.com/article/explained/mumbai-coronavirus-covid-19-cases-hospital-beds-deaths-6413025/}{its
health infrastructure}.

In the western state of Gujarat, Mr. Modi's home state, the situation is
equally grim. By mid-May public hospitals were already full and certain
privately run hospitals were trying to exploit the pandemic by charging
exorbitant fees to patients. The Gujarat High Court intervened and
remarked that
``\href{https://www.livelaw.in/news-updates/gujarat-hc-directs-pvt-hospitals-to-adopt-govt-specified-rates-lest-license-be-cancelled-read-order-156959}{an
ordinary man} will never be able to afford to avail adequate treatment
from a private hospital,'' given the fee being charged.

The high court pointed out that the most glaring problem with Mr. Modi's
badly planned lockdown was the crisis of hunger it had unleashed among
India's migrant workers and the poor. ``They are not worried about the
virus,'' the court remarked. ``They are worried about food.''

The starkest failure of Mr. Modi's coronavirus strategy has been the
devastation and misery it imposed on India's informal sector workers,
mostly people from impoverished villages, who work in Indian cities,
without a safety net.

Hundreds of thousands of migrant workers were left without wages after
the lockdown imposed with a four-hour notice closed factories and
businesses. They couldn't pay rent; they didn't have enough to eat. They
looked toward their villages, where they could find shelter and food by
relying on extended family.

With the public transport suspended, the workers set out on foot,
walking hundreds of miles in temperatures as high as 100 degrees
Fahrenheit. In May alone more than
\href{https://thejeshgn.com/projects/covid19-india/non-virus-deaths/}{150
migrant workers} walking back home have been killed in road or train
accidents.

As the lockdown is being partially eased, the migrant workers are now
making the same journey they could have made two months ago when the
cases in India numbered fewer than 1,000. Since some of the workers are
serving as
\href{https://timesofindia.indiatimes.com/india/migrants-return-bring-home-the-virus/articleshow/75544708.cms}{unwitting
carriers of the virus} to areas of low prevalence, they are greeted with
\href{https://www.hindustantimes.com/india-news/migrant-workers-battle-stigma-bias-back-home/story-0uuRSEZfoickVOrPU2agGL.html}{alarm
and apprehension} in their villages.

Demographics offered the Indian government considerable breathing room,
but India now faces two challenges at the same time --- a medical
infrastructure already under strain in the very places where cases are
rising most alarmingly and a population stretched to the limit by
economic hardship, with many facing the threat of malnutrition and
hunger.

Mr. Modi has a history of announcing sweeping measures with great
fanfare and little administrative preparation or follow-up. He and his
government will have to move beyond their
\href{https://www.narendramodi.in/prime-minister-narendra-modi-interacts-with-print-media-journalists-and-stakeholders-548937}{focus
on managing public perception}.

The lockdown was observed with ease by the middle and upper classes, who
can afford to do so. But the coronavirus cases are spreading fast in
dense urban clusters of the poor, who can't afford the luxury of social
distancing. It is India's poor who are and will be affected the most by
a rising number of infections and economic hardship.

There is an Indian phrase for such a situation, a phrase that was
already on the lips of every migrant workers leaving the pitiless cities
for their villages under a punishing sun: ``Bhagwan Bharose'' (``With
faith in God''). It is not an expression of faith in the rule of heaven,
but an expression of a lack of faith in their rulers on earth.

Hartosh Singh Bal is the political editor of The Caravan magazine in New
Delhi and the author of ``Waters Close Over Us: A Journey Along the
Narmada.''

\emph{The Times is committed to publishing}
\href{https://www.nytimes.com/2019/01/31/opinion/letters/letters-to-editor-new-york-times-women.html}{\emph{a
diversity of letters}} \emph{to the editor. We'd like to hear what you
think about this or any of our articles. Here are some}
\href{https://help.nytimes.com/hc/en-us/articles/115014925288-How-to-submit-a-letter-to-the-editor}{\emph{tips}}\emph{.
And here's our email:}
\href{mailto:letters@nytimes.com}{\emph{letters@nytimes.com}}\emph{.}

\emph{Follow The New York Times Opinion section on}
\href{https://www.facebook.com/nytopinion}{\emph{Facebook}}\emph{,}
\href{http://twitter.com/NYTOpinion}{\emph{Twitter (@NYTopinion)}}
\emph{and}
\href{https://www.instagram.com/nytopinion/}{\emph{Instagram}}\emph{.}

Advertisement

\protect\hyperlink{after-bottom}{Continue reading the main story}

\hypertarget{site-index}{%
\subsection{Site Index}\label{site-index}}

\hypertarget{site-information-navigation}{%
\subsection{Site Information
Navigation}\label{site-information-navigation}}

\begin{itemize}
\tightlist
\item
  \href{https://help.nytimes.com/hc/en-us/articles/115014792127-Copyright-notice}{©~2020~The
  New York Times Company}
\end{itemize}

\begin{itemize}
\tightlist
\item
  \href{https://www.nytco.com/}{NYTCo}
\item
  \href{https://help.nytimes.com/hc/en-us/articles/115015385887-Contact-Us}{Contact
  Us}
\item
  \href{https://www.nytco.com/careers/}{Work with us}
\item
  \href{https://nytmediakit.com/}{Advertise}
\item
  \href{http://www.tbrandstudio.com/}{T Brand Studio}
\item
  \href{https://www.nytimes.com/privacy/cookie-policy\#how-do-i-manage-trackers}{Your
  Ad Choices}
\item
  \href{https://www.nytimes.com/privacy}{Privacy}
\item
  \href{https://help.nytimes.com/hc/en-us/articles/115014893428-Terms-of-service}{Terms
  of Service}
\item
  \href{https://help.nytimes.com/hc/en-us/articles/115014893968-Terms-of-sale}{Terms
  of Sale}
\item
  \href{https://spiderbites.nytimes.com}{Site Map}
\item
  \href{https://help.nytimes.com/hc/en-us}{Help}
\item
  \href{https://www.nytimes.com/subscription?campaignId=37WXW}{Subscriptions}
\end{itemize}
