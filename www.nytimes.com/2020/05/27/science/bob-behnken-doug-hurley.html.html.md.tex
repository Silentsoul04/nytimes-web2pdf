Sections

SEARCH

\protect\hyperlink{site-content}{Skip to
content}\protect\hyperlink{site-index}{Skip to site index}

\href{https://www.nytimes.com/section/science}{Science}

\href{https://myaccount.nytimes.com/auth/login?response_type=cookie\&client_id=vi}{}

\href{https://www.nytimes.com/section/todayspaper}{Today's Paper}

\href{/section/science}{Science}\textbar{}Meet Bob Behnken and Doug
Hurley, SpaceX's First NASA Astronauts

\url{https://nyti.ms/2X9wYmB}

\begin{itemize}
\item
\item
\item
\item
\item
\item
\end{itemize}

\href{https://www.nytimes.com/2020/08/02/science/spacex-astronauts-splashdown.html?action=click\&pgtype=Article\&state=default\&region=TOP_BANNER\&context=storylines_menu}{SpaceX's
Astronaut Trip}

\begin{itemize}
\tightlist
\item
  \href{https://www.nytimes.com/2020/08/02/science/spacex-astronauts-splashdown.html?action=click\&pgtype=Article\&state=default\&region=TOP_BANNER\&context=storylines_menu}{`Thanks
  for Flying SpaceX'}
\item
  \href{https://www.nytimes.com/2020/05/26/science/spacex-launch-nasa.html?action=click\&pgtype=Article\&state=default\&region=TOP_BANNER\&context=storylines_menu}{Why
  NASA Picked SpaceX}
\item
  \href{https://www.nytimes.com/interactive/2020/05/26/science/spacex-nasa.html?action=click\&pgtype=Article\&state=default\&region=TOP_BANNER\&context=storylines_menu}{Inside
  the Capsule}
\item
  \href{https://www.nytimes.com/2020/05/27/science/bob-behnken-doug-hurley.html?action=click\&pgtype=Article\&state=default\&region=TOP_BANNER\&context=storylines_menu}{Meet
  the Astronauts}
\end{itemize}

Advertisement

\protect\hyperlink{after-top}{Continue reading the main story}

Supported by

\protect\hyperlink{after-sponsor}{Continue reading the main story}

\hypertarget{meet-bob-behnken-and-doug-hurley-spacexs-first-nasa-astronauts}{%
\section{Meet Bob Behnken and Doug Hurley, SpaceX's First NASA
Astronauts}\label{meet-bob-behnken-and-doug-hurley-spacexs-first-nasa-astronauts}}

They're best friends and veterans of the astronaut corps, and each is
married to another astronaut.

\includegraphics{https://static01.nyt.com/images/2020/05/27/science/27ASTRONAUTS1/27ASTRONAUTS1-articleLarge.jpg?quality=75\&auto=webp\&disable=upscale}

\href{https://www.nytimes.com/by/kenneth-chang}{\includegraphics{https://static01.nyt.com/images/2018/02/16/multimedia/author-kenneth-chang/author-kenneth-chang-thumbLarge.jpg}}

By \href{https://www.nytimes.com/by/kenneth-chang}{Kenneth Chang}

\begin{itemize}
\item
  Published May 27, 2020Updated Aug. 2, 2020
\item
  \begin{itemize}
  \item
  \item
  \item
  \item
  \item
  \item
  \end{itemize}
\end{itemize}

It's the Bob and Doug Show.

On Wednesday, weather permitting, two NASA astronauts, Robert L. Behnken
and Douglas G. Hurley, are to be sitting on top of
\href{https://www.nytimes.com/2020/05/26/science/spacex-launch-nasa.html}{a
SpaceX rocket headed to orbit}. But NASA and SpaceX officials more often
than not just call the pilots of this historic mission ``Bob and Doug.''

\emph{\emph{\emph{{[}}\href{https://www.nytimes.com/2020/08/02/science/spacex-nasa-return.html}{\emph{Follow
two NASA astronauts' return trip aboard SpaceX's Crew Dragon
capsule}}}.{]}}**

``I wanted to make sure everyone at SpaceX understood and knew Bob and
Doug as astronauts, as test pilots --- badass --- but also as dads and
husbands,'' said Gwynne Shotwell, president of the company that built
the Crew Dragon spacecraft that will carry the men to orbit, at a news
conference this month. ``I wanted to bring some humanity to this very
deeply technical effort as well.''

The men's trip to the space station will be the first from the United
States since the retirement of the space shuttles in 2011, and Mr.
Behnken and Mr. Hurley, friends and colleagues for two decades, have
traveled remarkably similar paths to this moment.

``One of the things that's really helpful for us as a crew is the long
relationship that Doug and I have had,'' Mr. Behnken said this month
during rounds of interviews with reporters. ``We're kind of at the point
in our experience --- whether it's flying in the T-38 or executing in a
\href{https://www.nytimes.com/2020/05/27/fashion/SpaceX-Dragon-Suits.html}{SpaceX}
simulation or approaching and docking to the International Space Station
--- where we, in addition to finishing each other's sentences, we can
predict, you know, almost by body language, what the person's opinion is
or what they're going to do, what their next action is going to be.''

The rapport and good humor between the astronauts was evident in
\href{https://www.youtube.com/watch?v=gu4o5KgnAQM}{a video created by
NASA}. Mr. Behnken said he was looking forward to the splashdown at the
end of their mission before adding, with a grin, ``I'm expecting a
little bit of vomiting, maybe, to happen in that end game. When we get
to that opportunity to do that in the water together, it's kind of a
weird thing to say, but I'm looking for that kind of celebratory
event.''

Mr. Hurley offered a more serious answer, talking about how he enjoyed
working together with a close friend.

``And yes,'' he said, ``the celebratory vomiting at the end of the
mission will be excellent.''

\includegraphics{https://static01.nyt.com/images/2020/05/27/science/27ASTRONAUTS2/merlin_172858092_fae473bb-8e46-4bb4-8b0a-0d09ccd98fff-articleLarge.jpg?quality=75\&auto=webp\&disable=upscale}

Both are former military pilots who rose to the rank of colonel --- Mr.
Behnken in the United States Air Force, Mr. Hurley in the Marines ---
before deciding they wanted to go even higher. Both joined NASA in 2000
--- two of the 17 astronauts selected by the space agency that year.

They have each flown to space twice on space shuttle missions, although
never on the same mission. Mr. Hurley flew on the final space shuttle
mission in 2011.

They both married astronauts from their class. Mr. Behnken's wife is
\href{https://www.nasa.gov/astronauts/biographies/k-megan-mcarthur/biography}{Megan
McArthur}, an oceanographer who was part of the shuttle mission that
made one last visit in 2009 to repair and upgrade the Hubble Space
Telescope. Mr. Hurley is married to
\href{https://www.nasa.gov/astronauts/biographies/karen-l-nyberg/biography}{Karen
Nyberg}, who spent nearly six months on the International Space Station
in 2013 and who retired from NASA at the end of March.

Mr. Behnken and Ms. McArthur's son, Theodore, is 6. Mr. Hurley and Ms.
Nyberg's son, Jack, is 10.

``I think it's a pretty cool looking vehicle and my 10-year-old son
certainly thinks it's a cool vehicle with a cool name, Dragon,'' Mr.
Hurley said. ``So I got the thumbs up from him and in the end, that's
all that matters.''

Mr. Hurley, 53, grew up in Apalachin, N.Y., outside Binghamton. He
graduated with a bachelor's degree in civil engineering from Tulane
University.

Mr. Behnken, 49, is a native of Saint Ann, Mo., graduating from
Washington University in St. Louis with degrees in physics and
mechanical engineering. He then completed master's and doctoral degrees
in mechanical engineering at the California Institute of Technology.

The road to the launchpad has been longer than had been expected and
planned. After a successful uncrewed test flight of a Crew Dragon to the
space station last year, it looked like Mr. Behnken and Mr. Hurley would
soon follow on their mission.

\href{https://www.nytimes.com/interactive/2020/05/26/science/spacex-nasa.html}{}

\includegraphics{https://static01.nyt.com/images/2020/05/26/us/spacex-nasa-promo-1590499638707/spacex-nasa-promo-1590499638707-articleLarge-v2.jpg}

\hypertarget{now-boarding-spacexs-new-ride-to-orbit-for-nasa-astronauts}{%
\subsection{Now Boarding: SpaceX's New Ride to Orbit for NASA
Astronauts}\label{now-boarding-spacexs-new-ride-to-orbit-for-nasa-astronauts}}

The Crew Dragon launched successfully on Saturday.

But then the spacecraft's parachutes --- essential for the safe return
of the astronauts --- failed in some tests. More disconcerting, the Crew
Dragon that had made the successful trip to space exploded on a test
stand while being fueled for a test firing of its thrusters. No one was
on board, and no one was hurt, but \href{https://youtu.be/scz-fFZT-dI}{a
video of the explosion} leaked online.

Last October, the astronauts said they still had confidence in the
spacecraft.

``Certainly, it's disappointing,'' Mr. Hurley said. ``You get questions
from your family. What happened? Do you know what happened? That kind of
thing. But the other part of it, you have to keep in mind, is this is
test, evaluation development, and it's part of the process.''

Image

Mr. Hurley, right, with Mr. Behnken and Elon Musk during a news
conference at the Kennedy Space Center in 2019.Credit...John
Raoux/Associated Press

Mr. Behnken said he and Mr. Hurley were quickly and fully informed about
the incident and the subsequent investigation as well as changes to the
design.

``Giving us insight and sharing that understanding as we go forward has
been part of what has made us comfortable with this team going
forward,'' Mr. Behnken said.

NASA recently made the decision to extend the Bob and Doug Show a bit
longer than the two weeks originally planned. The space station is
short-staffed at the moment, so Mr. Behnken and Mr. Hurley will stay
longer to pitch in with operations.

For space missions, usually planned in precise detail, this trip is
unusually open-ended. They will probably spend at least a month in
orbit, and the stay could stretch to four months. Mr. Behnken has
devoted time in
\href{https://www.nasa.gov/image-feature/neutral-buoyancy-laboratory}{the
huge pool that NASA uses to rehearse spacewalks} and Mr. Hurley has
taken refresher classes on the operation of
\href{https://www.asc-csa.gc.ca/eng/iss/canadarm2/default.asp}{the
station's Canadian-built robotic arm}.

Other astronauts who could have been in this spotlight, including Nicole
Mann --- one of two NASA astronauts assigned to a future flight on
another spacecraft, Boeing's Starliner --- do not begrudge Mr. Behnken
and Mr. Hurley.

``It feels kind of like one of your close family members having a great
lifetime achievement,'' she said, ``and really, that's what it is.''

Advertisement

\protect\hyperlink{after-bottom}{Continue reading the main story}

\hypertarget{site-index}{%
\subsection{Site Index}\label{site-index}}

\hypertarget{site-information-navigation}{%
\subsection{Site Information
Navigation}\label{site-information-navigation}}

\begin{itemize}
\tightlist
\item
  \href{https://help.nytimes.com/hc/en-us/articles/115014792127-Copyright-notice}{©~2020~The
  New York Times Company}
\end{itemize}

\begin{itemize}
\tightlist
\item
  \href{https://www.nytco.com/}{NYTCo}
\item
  \href{https://help.nytimes.com/hc/en-us/articles/115015385887-Contact-Us}{Contact
  Us}
\item
  \href{https://www.nytco.com/careers/}{Work with us}
\item
  \href{https://nytmediakit.com/}{Advertise}
\item
  \href{http://www.tbrandstudio.com/}{T Brand Studio}
\item
  \href{https://www.nytimes.com/privacy/cookie-policy\#how-do-i-manage-trackers}{Your
  Ad Choices}
\item
  \href{https://www.nytimes.com/privacy}{Privacy}
\item
  \href{https://help.nytimes.com/hc/en-us/articles/115014893428-Terms-of-service}{Terms
  of Service}
\item
  \href{https://help.nytimes.com/hc/en-us/articles/115014893968-Terms-of-sale}{Terms
  of Sale}
\item
  \href{https://spiderbites.nytimes.com}{Site Map}
\item
  \href{https://help.nytimes.com/hc/en-us}{Help}
\item
  \href{https://www.nytimes.com/subscription?campaignId=37WXW}{Subscriptions}
\end{itemize}
