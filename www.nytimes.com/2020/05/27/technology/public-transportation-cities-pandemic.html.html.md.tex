Sections

SEARCH

\protect\hyperlink{site-content}{Skip to
content}\protect\hyperlink{site-index}{Skip to site index}

\href{https://www.nytimes.com/section/technology}{Technology}

\href{https://myaccount.nytimes.com/auth/login?response_type=cookie\&client_id=vi}{}

\href{https://www.nytimes.com/section/todayspaper}{Today's Paper}

\href{/section/technology}{Technology}\textbar{}Oh No, Here Comes the
Transportation Hellscape

\url{https://nyti.ms/3c8XQr5}

\begin{itemize}
\item
\item
\item
\item
\item
\end{itemize}

Advertisement

\protect\hyperlink{after-top}{Continue reading the main story}

Supported by

\protect\hyperlink{after-sponsor}{Continue reading the main story}

on tech

\hypertarget{oh-no-here-comes-the-transportation-hellscape}{%
\section{Oh No, Here Comes the Transportation
Hellscape}\label{oh-no-here-comes-the-transportation-hellscape}}

How tech companies and cities can work together to make transit more
appealing and effective.

\includegraphics{https://static01.nyt.com/images/2020/05/27/business/27ontech-still/27ontech-still-superJumbo.png}

\href{https://www.nytimes.com/by/shira-ovide}{\includegraphics{https://static01.nyt.com/images/2020/03/18/reader-center/author-shira-ovide/author-shira-ovide-thumbLarge-v2.png}}

By \href{https://www.nytimes.com/by/shira-ovide}{Shira Ovide}

\begin{itemize}
\item
  May 27, 2020
\item
  \begin{itemize}
  \item
  \item
  \item
  \item
  \item
  \end{itemize}
\end{itemize}

\emph{This article is part of the On Tech newsletter. You can}
\href{https://www.nytimes.com/newsletters/signup/OT}{\emph{sign up
here}} \emph{to receive it weekdays.}

Is it possible the pandemic will turn the usual urban gridlock into a
hellscape?

If people continue to be wary about their safety on buses, trains and
subways, and commuters who can afford it resort to driving themselves,
navigating cities might become impossible.

Technology can't fix a problem that demands smart policy and a
rethinking of how cities work. But here are some ideas I've gleaned from
transportation experts about how tech companies and cities can work
together to make transit more appealing and effective, during the
pandemic and beyond.

\textbf{Let people reserve seats on mass transit:} Via, an on-demand
ride company that also offers software for transportation systems, says
it's working with an American city on the West Coast --- the C.E.O. said
he could not yet say which one --- to let people reserve a spot on mass
transit with a smartphone app, text or phone call. Reservation systems
that set a cap on the number of travelers per bus or train might make
commuters safer (and less anxious).

\textbf{Create more flexible bus routes on the fly:} Dozens of transit
authorities, often working with private companies like Uber or Via,
\href{https://www.viavan.com/berlin-launch/}{supplement} traditional
transit networks with on-demand
\href{https://www.theguardian.com/business/2018/jun/29/oxford-buses-turn-to-uber-style-apps-in-on-demand-experiment}{minibuses}
that pool together people who are going to roughly the same place at the
same time.

Pop-up routes are (sometimes fairly) derided as exclusionary and dumb
---
\href{https://www.shortlist.com/news/ride-sharing-company-lyft-has-invented-the-bus}{a
ride service ``accidentally invents the bus},'' one headline joked ---
and they can be
\href{https://www.wired.com/story/cities-on-demand-transit-buses/}{tricky
and expensive.} But anyone who has to choose between taking three buses
or a \$25 Uber ride knows that transit could use more experiments.

\textbf{Provide all-in-one transit and ticketing:} In some cities, Uber
and Lyft apps can help plot a route to your friend's house with a
combination of car, bus and rented bicycle. I don't think this irregular
use of the app is helpful for most people, but it might be in places
poorly served by other transit apps, particularly with the addition of
no-contact tickets.

Uber said that since the pandemic began, more U.S. cities than ever
before have inquired about getting help with adding an option to buy
public transit tickets through its app. Users can already do this in
\href{https://www.nytimes.com/2019/08/07/technology/uber-train-bus-public-transit.html}{Denver}
and
\href{https://www.sfchronicle.com/business/article/Uber-will-sell-bus-tickets-through-its-app-in-Las-14953445.php}{Las
Vegas}.

\textbf{Mesh on-demand services with public transit:} Companies that
provide on-demand cars, rental bicycles, scooters and mo-peds tend to
\href{https://www.nytimes.com/2019/08/07/technology/uber-train-bus-public-transit.html}{be
completely separate from mass transit}. This might not work anymore ---
not least because young transportation companies are mostly unprofitable
and might not make it.

What if these companies were part of the fabric of public urban
transportation, and subsidized as such? The dirty secret of
transportation, including road and private cars, is that it's almost all
subsidized by the government in some way.

This subsidizing means cities could chip in for Uber rides for people
who have disabilities or require free scooter rentals
\href{https://www.citylab.com/transportation/2019/10/pittsburgh-micromobility-collective-waze-spin-swiftmile-zipcar/599779/}{for
health care workers} and essential employees. Cities and companies could
also share data on where people are going and make sure that the data
inform their transportation decisions.

Yes, closer collaboration between public transit systems and private
companies brings a host of problems and questions. It's all made harder
because Uber and Lyft in particular have not followed through on
promises to the cities that allow them to operate. The ride-share
companies said they would reduce traffic and complement rather than
replace public transit. They've
\href{https://www.nytimes.com/interactive/2020/03/13/upshot/mystery-of-missing-bus-riders.html}{done}
\href{https://www.wsj.com/articles/the-ride-hail-utopia-that-got-stuck-in-traffic-11581742802}{neither}.

As it stands now, the status quo stinks for everyone. That means there's
an opening for tech companies and cities to take risks on how to move
people around safely, efficiently and affordably.

\begin{center}\rule{0.5\linewidth}{\linethickness}\end{center}

\hypertarget{no-decision-is-neutral}{%
\subsection{No decision is neutral}\label{no-decision-is-neutral}}

On Tuesday, the big politics news was Twitter's decision to
\href{https://www.nytimes.com/2020/05/26/technology/twitter-trump-mail-in-ballots.html}{add
context} to two of President Trump's tweets for the first time, after
Mr. Trump falsely claimed that mail-in voting ballots would mean that
the November presidential election was ``rigged.''

When people in power say things that might be misleading or even
harmful, it can be difficult and polarizing for social media companies
to decide how to react.

By adding context to the tweets on Tuesday, Twitter made it clear the
president was making unsubstantiated claims of possible voting fraud
that could erode people's trust in a central element of democracy.

It was a bold move for the company. Predictably, Mr. Trump responded by
attacking Twitter.

Nothing Twitter or Facebook might do in these circumstances is neutral.

When Twitter or Facebook let stand without comment Mr. Trump's posts
that
\href{https://www.nytimes.com/2020/05/26/us/politics/klausutis-letter-jack-dorsey.html}{dredged
up} a discredited murder conspiracy and a racist
``\href{https://www.theguardian.com/world/2020/jan/24/jair-bolsonaro-racist-comment-sparks-outrage-indigenous-groups}{joke}''
by the president of Brazil, the company made an active choice. And on
Tuesday, when Twitter added context to the president's remarks, it made
a choice.

\begin{center}\rule{0.5\linewidth}{\linethickness}\end{center}

\hypertarget{before-we-go-}{%
\subsection{Before we go \ldots{}}\label{before-we-go-}}

\begin{itemize}
\item
  \textbf{Let's revisit this in 10 years:} The boss of the new HBO Max
  streaming video service called Netflix ``the enemy'' a couple of years
  ago, my colleague \href{https://www.nytimes.com/by/edmund-lee}{Ed Lee}
  \href{https://www.nytimes.com/2020/05/26/business/media/hbo-max-netflix-streaming.html}{reported
  this week}. (The executive has disputed he ever said this.) Almost 10
  years ago, the chief executive of Time Warner said Netflix would be as
  likely as the Albanian army to one day take over the world. The
  Albanians won.
\item
  \textbf{Using internet mastery for fun:} Imagine if your favorite
  celebrity --- let's say Betty White? --- collaborated with her fan
  club to create a cult, which then became hugely popular and its
  members challenged other famous people to tug-of-war contests. My
  colleague \href{https://www.nytimes.com/by/taylor-lorenz}{Taylor
  Lorenz}
  \href{https://www.nytimes.com/2020/05/26/style/step-chickens-tiktok-cult-wars.html}{explains
  Step Chickens}, the 2020 internet equivalent of my tortured Betty
  White metaphor. It's good, harmless (I think?) fun.
\item
  \textbf{Sowing division on social media:} Facebook conducted research
  a few years ago that found that the social network made people more
  polarized, The Wall Street Journal
  \href{https://www.wsj.com/articles/facebook-knows-it-encourages-division-top-executives-nixed-solutions-11590507499}{reported}.
  One study from 2016 found that two-thirds of people who joined
  extremist Facebook groups in Germany did so because of the company's
  computerized suggestions. The company chose not to take steps its
  employees suggested to help mitigate the very divisions to which
  Facebook contributed.
\end{itemize}

\hypertarget{hugs-to-this}{%
\subsubsection{Hugs to this}\label{hugs-to-this}}

This might be the
\href{https://oceanconservancy.org/blog/2018/10/08/everything-need-know-dumbo-octopus}{cutest
octopus in the world}. (The link is from 2018, but it's a pandemic ---
what is time?)

\begin{center}\rule{0.5\linewidth}{\linethickness}\end{center}

\emph{We want to hear from you. Tell us what you think of this
newsletter and what else you'd like us to explore. You can reach us at}
\href{mailto:ontech@nytimes.com?subject=On\%20Tech\%20Feedback}{\emph{ontech@nytimes.com.}}

\emph{Get this newsletter in your inbox every
weekday;}\href{https://www.nytimes.com/newsletters/signup/OT}{\emph{please
sign up here}}\emph{.}

Advertisement

\protect\hyperlink{after-bottom}{Continue reading the main story}

\hypertarget{site-index}{%
\subsection{Site Index}\label{site-index}}

\hypertarget{site-information-navigation}{%
\subsection{Site Information
Navigation}\label{site-information-navigation}}

\begin{itemize}
\tightlist
\item
  \href{https://help.nytimes.com/hc/en-us/articles/115014792127-Copyright-notice}{©~2020~The
  New York Times Company}
\end{itemize}

\begin{itemize}
\tightlist
\item
  \href{https://www.nytco.com/}{NYTCo}
\item
  \href{https://help.nytimes.com/hc/en-us/articles/115015385887-Contact-Us}{Contact
  Us}
\item
  \href{https://www.nytco.com/careers/}{Work with us}
\item
  \href{https://nytmediakit.com/}{Advertise}
\item
  \href{http://www.tbrandstudio.com/}{T Brand Studio}
\item
  \href{https://www.nytimes.com/privacy/cookie-policy\#how-do-i-manage-trackers}{Your
  Ad Choices}
\item
  \href{https://www.nytimes.com/privacy}{Privacy}
\item
  \href{https://help.nytimes.com/hc/en-us/articles/115014893428-Terms-of-service}{Terms
  of Service}
\item
  \href{https://help.nytimes.com/hc/en-us/articles/115014893968-Terms-of-sale}{Terms
  of Sale}
\item
  \href{https://spiderbites.nytimes.com}{Site Map}
\item
  \href{https://help.nytimes.com/hc/en-us}{Help}
\item
  \href{https://www.nytimes.com/subscription?campaignId=37WXW}{Subscriptions}
\end{itemize}
