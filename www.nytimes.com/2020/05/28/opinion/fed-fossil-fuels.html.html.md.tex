Sections

SEARCH

\protect\hyperlink{site-content}{Skip to
content}\protect\hyperlink{site-index}{Skip to site index}

\href{https://myaccount.nytimes.com/auth/login?response_type=cookie\&client_id=vi}{}

\href{https://www.nytimes.com/section/todayspaper}{Today's Paper}

\href{/section/opinion}{Opinion}\textbar{}Why Is the Fed Spending So
Much Money on a Dying Industry?

\href{https://nyti.ms/3c9Ah1d}{https://nyti.ms/3c9Ah1d}

\begin{itemize}
\item
\item
\item
\item
\item
\end{itemize}

Advertisement

\protect\hyperlink{after-top}{Continue reading the main story}

\href{/section/opinion}{Opinion}

Supported by

\protect\hyperlink{after-sponsor}{Continue reading the main story}

\hypertarget{why-is-the-fed-spending-so-much-money-on-a-dying-industry}{%
\section{Why Is the Fed Spending So Much Money on a Dying
Industry?}\label{why-is-the-fed-spending-so-much-money-on-a-dying-industry}}

It should not be directing money to further entrench the carbon economy.

By Sarah Bloom Raskin

Ms. Raskin is a former member of the board of governors of the Federal
Reserve.

\begin{itemize}
\item
  May 28, 2020
\item
  \begin{itemize}
  \item
  \item
  \item
  \item
  \item
  \end{itemize}
\end{itemize}

\includegraphics{https://static01.nyt.com/images/2020/05/28/opinion/28raskin/28raskin-articleLarge.jpg?quality=75\&auto=webp\&disable=upscale}

The coronavirus pandemic has laid bare just how vulnerable the United
States is to sudden, catastrophic shocks. Climate change poses the next
big threat. Ignoring it, particularly to the benefit of fossil fuel
interests, is a risk we can't afford.

From my time as a
\href{https://www.nytimes.com/2020/06/10/business/economy/fed-june-meeting-coronavirus.html}{Federal
Reserve} governor and then deputy Treasury secretary, I've learned that
times like this not only can determine our ability to recover from a
crisis but can also help inoculate us against the next one. That's why
it is imperative that we make investments now that will increase the
resilience of our economy.

The Fed is singularly poised to seed strategic investments in future
economic stability.
\href{https://www.washingtonpost.com/news/powerpost/paloma/the-energy-202/2020/05/01/the-energy-202-oil-and-gas-companies-stand-to-gain-from-fed-loosening-coronavirus-loan-rules/5eab0bda602ff15fb0021673/}{Oil,
gas} and
\href{https://www.reuters.com/article/us-health-coronavirus-coal/u-s-coal-mining-industry-seeks-wide-ranging-coronavirus-bailout-letter-idUSKBN21701F}{coal}
companies are set or are seeking to receive billions in federal aid ---
including at least
\href{https://westernvaluesproject.org/over-one-third-of-all-oil-gas-and-mining-corporations-already-awarded-bailout-funds/}{\$3.9
billion} from the Paycheck Protection Program and at least
\href{https://www.bloomberg.com/news/articles/2020-05-15/-stealth-bailout-shovels-millions-of-dollars-to-oil-companies?sref=oUjKJw8m}{\$1.9
billion} in tax credits tucked into the CARES Act passed by Congress.
Their
\href{https://www.cruz.senate.gov/files/documents/Letters/4.24.2020\%20Oil\%20Gas\%20Fed\%20Lending\%20Facility\%20Letter.pdf}{allies}
in
\href{https://senatorkevincramer.app.box.com/s/981sfgn44nmkr8fq5hhf6xryxro6fyvp}{Congress}
and the administration have lobbied for changes to several of the Fed's
lending programs, including relaxing the
\href{https://www.bloomberg.com/news/articles/2020-05-12/energy-chief-says-fed-was-asked-to-expand-lending-for-oil-firms}{Main
Street Lending Program}. Among those eligible for government assistance
are many fossil fuel companies that were in deep financial trouble long
before the pandemic began.

\begin{center}\rule{0.5\linewidth}{\linethickness}\end{center}

These concessions to the fossil fuel industry are a risky investment in
the past. The Fed is ignoring clear warning signs about the economic
repercussions of the impending climate crisis by taking action that will
lead to increases in greenhouse gas emissions at a time when even in the
short term, fossil fuels are a
\href{https://www.eenews.net/stories/1063050681}{terrible}
\href{https://www.cnbc.com/2020/01/31/cramer-sees-oil-stocks-in-the-death-knell-phase-says-new-tobacco.html}{investment}.

Parts of the industry are awash in hundreds of billions in risky debt.
Many fossil fuel companies spent the past decade recklessly expanding
production even as they failed to turn a profit. Oil and gas companies
now hold
\href{https://ieefa.org/ieefa-commentary-federal-lending-to-the-oil-and-gas-sector-would-be-a-complete-waste-of-money/}{\$744
billion in bonds and debt}, much of it below investment grade or close
to it. Almost 83 percent of the industry's debt is now eligible for
cheap refinancing by the Fed.

For taxpayers, shouldering these liabilities is a bad deal. Buying this
bad debt is not likely to support the creation of jobs or even ensure
that existing jobs survive. Moreover, the Fed-subsidized loans come with
no strings attached regarding the
\href{https://www.bloomberg.com/news/articles/2020-05-01/fed-rebuked-over-loan-terms-that-don-t-explicitly-bar-layoffs}{retention
of jobs},
\href{https://www.washingtonpost.com/business/2020/04/28/federal-reserve-bond-corporations/}{C.E.O.
bonuses or stock buybacks}.

The decision to bring oil and gas into the Fed's investment portfolio
not only misdirects limited recovery resources but also sends a false
price signal to investors about where capital needs to be allocated. It
increases the likelihood that investors will be stuck with stranded oil
and gas assets that society no longer needs. It also forestalls the
inevitable decline of an industry that can no longer sustain itself. And
finally, it undermines urgent efforts to counter surging carbon dioxide
and methane emissions, which are bringing us closer to the catastrophe
of an unlivably hot planet.

Given the size and scope of government intervention, we should be
maximizing the public's return on our investment. The Fed's unique
independence affords it a powerful role, and its mandate includes
ensuring both the stability of the financial system and full employment.
Climate change threatens financial stability; addressing it can create
economic opportunity and more jobs. The decisions the Fed makes on our
behalf should build toward a stronger economy with more jobs in
innovative industries --- not prop up and enrich dying ones.

Last week, five economists
\href{https://www.smithschool.ox.ac.uk/publications/wpapers/workingpaper20-02.pdf}{released
a survey} of more than 200 finance ministers, central bankers and
economists from Group of 20 countries. Their recommendations were
unequivocal: The best long-term recovery plans will also be the plans
that reduce greenhouse gas emissions.

In the United States, this dynamic has already become apparent.
Renewable energy in the past few years has been shown to directly and
indirectly generate
\href{https://www.sciencedirect.com/science/article/abs/pii/S026499931630709X}{roughly
three times} as many jobs as a comparable investment in fossil fuels.
Jobs in clean energy, such as wind technicians and solar installers, are
the \href{https://www.bls.gov/ooh/fastest-growing.htm}{fastest growing}
of any industry in the country, and over the past five years, employment
in the clean energy industry has
\href{https://e2.org/wp-content/uploads/2020/04/E2-Clean-Jobs-America-2020.pdf}{grown
70 percent faster} than the economy overall.

This pandemic has illustrated the staggering costs of failing to prepare
for known risks. And the magnitude of the crisis has been made so much
worse by the flight from science and the elevation of narrow private
interests over the national common good.

But it also provides an unexpected opportunity to build an economy that
is stronger in the long term. The decisions that the Fed makes today
will go a long way to determining whether tomorrow's economy is one that
remains susceptible to more chaos and vulnerability or builds economic
security and resilience.

\href{https://law.duke.edu/fac/raskin/}{Sarah Bloom Raskin} is a former
member of the board of governors of the Federal Reserve System and a
former deputy Treasury secretary.

\emph{The Times is committed to publishing}
\href{https://www.nytimes.com/2019/01/31/opinion/letters/letters-to-editor-new-york-times-women.html}{\emph{a
diversity of letters}} \emph{to the editor. We'd like to hear what you
think about this or any of our articles. Here are some}
\href{https://help.nytimes.com/hc/en-us/articles/115014925288-How-to-submit-a-letter-to-the-editor}{\emph{tips}}\emph{.
And here's our email:}
\href{mailto:letters@nytimes.com}{\emph{letters@nytimes.com}}\emph{.}

\emph{Follow The New York Times Opinion section on}
\href{https://www.facebook.com/nytopinion}{\emph{Facebook}}\emph{,}
\href{http://twitter.com/NYTOpinion}{\emph{Twitter (@NYTopinion)}}
\emph{and}
\href{https://www.instagram.com/nytopinion/}{\emph{Instagram}}\emph{.}

Advertisement

\protect\hyperlink{after-bottom}{Continue reading the main story}

\hypertarget{site-index}{%
\subsection{Site Index}\label{site-index}}

\hypertarget{site-information-navigation}{%
\subsection{Site Information
Navigation}\label{site-information-navigation}}

\begin{itemize}
\tightlist
\item
  \href{https://help.nytimes.com/hc/en-us/articles/115014792127-Copyright-notice}{©~2020~The
  New York Times Company}
\end{itemize}

\begin{itemize}
\tightlist
\item
  \href{https://www.nytco.com/}{NYTCo}
\item
  \href{https://help.nytimes.com/hc/en-us/articles/115015385887-Contact-Us}{Contact
  Us}
\item
  \href{https://www.nytco.com/careers/}{Work with us}
\item
  \href{https://nytmediakit.com/}{Advertise}
\item
  \href{http://www.tbrandstudio.com/}{T Brand Studio}
\item
  \href{https://www.nytimes.com/privacy/cookie-policy\#how-do-i-manage-trackers}{Your
  Ad Choices}
\item
  \href{https://www.nytimes.com/privacy}{Privacy}
\item
  \href{https://help.nytimes.com/hc/en-us/articles/115014893428-Terms-of-service}{Terms
  of Service}
\item
  \href{https://help.nytimes.com/hc/en-us/articles/115014893968-Terms-of-sale}{Terms
  of Sale}
\item
  \href{https://spiderbites.nytimes.com}{Site Map}
\item
  \href{https://help.nytimes.com/hc/en-us}{Help}
\item
  \href{https://www.nytimes.com/subscription?campaignId=37WXW}{Subscriptions}
\end{itemize}
