Sections

SEARCH

\protect\hyperlink{site-content}{Skip to
content}\protect\hyperlink{site-index}{Skip to site index}

\href{https://myaccount.nytimes.com/auth/login?response_type=cookie\&client_id=vi}{}

\href{https://www.nytimes.com/section/todayspaper}{Today's Paper}

\href{/section/opinion}{Opinion}\textbar{}A Cog in the Machine of
Creation

\href{https://nyti.ms/36L9SFS}{https://nyti.ms/36L9SFS}

\begin{itemize}
\item
\item
\item
\item
\item
\end{itemize}

Advertisement

\protect\hyperlink{after-top}{Continue reading the main story}

\href{/section/opinion}{Opinion}

Supported by

\protect\hyperlink{after-sponsor}{Continue reading the main story}

THE big ideas: why does art matter?

\hypertarget{a-cog-in-the-machine-of-creation}{%
\section{A Cog in the Machine of
Creation}\label{a-cog-in-the-machine-of-creation}}

The many roles involved in producing a film rule out the notion of a
single, indispensable artist.

By Wes Studi

Mr. Studi is an actor.

\begin{itemize}
\item
  May 31, 2020
\item
  \begin{itemize}
  \item
  \item
  \item
  \item
  \item
  \end{itemize}
\end{itemize}

\includegraphics{https://static01.nyt.com/images/2020/05/31/multimedia/31bigideas-studi/31bigideas-studi-articleLarge.jpg?quality=75\&auto=webp\&disable=upscale}

\emph{This essay is part of}
\href{https://www.nytimes.com/spotlight/the-big-ideas}{\emph{The Big
Ideas}}\emph{, a special section of The Times's philosophy series,}
\href{https://www.nytimes.com/column/the-stone?action=click\&module=RelatedLinks\&pgtype=Article}{\emph{The
Stone}}\emph{, in which more than a dozen artists, writers and thinkers
answer the question, ``Why does art matter?'' The entire series can be
found}
\href{https://www.nytimes.com/spotlight/the-big-ideas}{\emph{here}}\emph{.}

\begin{center}\rule{0.5\linewidth}{\linethickness}\end{center}

As an artist, I am a creator, though referring to myself that way makes
me somewhat uncomfortable. For most of my life, the word ``creator'' has
meant the Creator of all things, especially in the native tradition, and
I would not by any means compare my meager efforts at creation to
Creation.

But art is truly an act of creation. And creation is an act of art.

When we create art, it's a product of three components: the mind, the
heart and our past experiences. **** As an actor, sculptor and author, I
have found that this interconnectedness is a large part of why art
matters.

The mind, of course, is where initial ideas are conceived, whether
through whimsy, inspiration or suggestion. What matters at the inception
of creation is deciding whether an idea is worth pursuing. The heart
then reveals to us how we feel about the idea, and we again ponder the
worthiness of the pursuit. Past experiences help us weigh the personal
significance of the creation we are considering. As artists, we ask
ourselves: Will it serve a purpose beyond our own need to create? Does
it need to?

As an actor, I have at times wondered whether acting is a true art form.
Is an actor merely an interpreter of an idea created in a screenplay,
script or play? Is performance in this context merely a craft, or is it
an act of creation itself?

In this context, the answer might seem simple: The writer is the artist,
the creator. The actor is the journeyman who interprets the plan, the
story to be told.

But the actor must also rely on many other people to do his work: makeup
artists, hair stylists, costume designers, directors. Photography and
set design are also prime considerations. The many roles involved in
producing a film rule out the notion of a sole creator or artist. The
actor becomes a cog in a larger machine of creation. And each cog is
**** an absolute necessity.

As for the question of whether an actor is a creator or an interpreter,
I lean toward considering myself an indispensable craftsman tasked with
a particular act of creation: a performance worthy of the story being
told.

While the original idea may have been created by another, it falls to
the performer to fit the situation.

\includegraphics{https://static01.nyt.com/images/2020/05/31/multimedia/31bigideas-studi-02/31bigideas-studi-02-articleLarge.jpg?quality=75\&auto=webp\&disable=upscale}

Filmmaking is an especially collaborative function, and acting
encompasses a bit of craftsmanship. ``Dances With Wolves,'' for
instance, the 1990 film in which I played a Pawnee warrior, required the
coordination of thousands of bison for a hunting scene, and battle
scenes featured dozens of riders on horseback, along with actors doing
their part to bring to life a unique story. I had to learn some of the
Pawnee language to better portray my character. In fact, in my career I
have spoken some 20 languages for various roles, including one that did
not exist: I was asked to work with linguists crafting the language of
the Na'vi, the Indigenous culture in ``Avatar,'' because I spoke
Cherokee and had studied phonetic languages, a valuable part of my past
that I brought to that film.

As a Native American actor, I've remained keenly aware that the wars and
invasions that ravaged native populations throughout the Americas have
never really ended. That knowledge, along with my own past experience,
enabled me to empathetically portray characters like the Apache warrior
Geronimo onscreen, and to better convey the deep sense of injustice that
the real Geronimo faced.

While as an actor I'm either an arty craftsman or a crafty artist,
adding and building on another's idea, as a sculptor carving stone I
find things a bit different. I remove rather than build, shaping and
carving, transforming a stone from its original organic form into
something familiar and pleasing to the eye. These carvings satisfy both
my need to create and control the process as much as the stone allows;
an added factor is the journey of creating something unique. That these
pieces of art are appreciated is, of course, additionally gratifying.

I can say the same for book writing: ``The Adventures of Billy Bean,'' a
children's book I wrote in Cherokee and English many years ago, was also
an act of creating characters in my mind and heart, and rooted in past
experiences. I hope that the stories in the book provided entertainment
and an uplifting message about life for young people becoming acquainted
with the world around them.

If sculpting makes me a sculptor, then so be it. If writing makes me an
author, then so be it. With minds, hearts and pasts, we all commit acts
of creation.

Wes Studi is an actor. In 2019 he received an honorary Academy Award,
the first Oscar awarded to a Native American actor.

\emph{\textbf{Now in print}}*:
``\emph{\href{http://bitly.com/1MW2kN3}{\emph{Modern Ethics in 77
Arguments}}},'' and ``\emph{\href{http://bitly.com/1MW2kN3}{\emph{The
Stone Reader: Modern Philosophy in 133 Arguments}}},'' with essays from
the series, edited by Peter Catapano and Simon Critchley, published by
Liveright Books.*

\emph{The Times is committed to publishing}
\href{https://www.nytimes.com/2019/01/31/opinion/letters/letters-to-editor-new-york-times-women.html}{\emph{a
diversity of letters}} \emph{to the editor. We'd like to hear what you
think about this or any of our articles. Here are some}
\href{https://help.nytimes.com/hc/en-us/articles/115014925288-How-to-submit-a-letter-to-the-editor}{\emph{tips}}\emph{.
And here's our email:}
\href{mailto:letters@nytimes.com}{\emph{letters@nytimes.com}}\emph{.}

\emph{Follow The New York Times Opinion section on}
\href{https://www.facebook.com/nytopinion}{\emph{Facebook}}\emph{,}
\href{http://twitter.com/NYTOpinion}{\emph{Twitter (@NYTopinion)}}
\emph{and}
\href{https://www.instagram.com/nytopinion/}{\emph{Instagram}}\emph{.}

Advertisement

\protect\hyperlink{after-bottom}{Continue reading the main story}

\hypertarget{site-index}{%
\subsection{Site Index}\label{site-index}}

\hypertarget{site-information-navigation}{%
\subsection{Site Information
Navigation}\label{site-information-navigation}}

\begin{itemize}
\tightlist
\item
  \href{https://help.nytimes.com/hc/en-us/articles/115014792127-Copyright-notice}{©~2020~The
  New York Times Company}
\end{itemize}

\begin{itemize}
\tightlist
\item
  \href{https://www.nytco.com/}{NYTCo}
\item
  \href{https://help.nytimes.com/hc/en-us/articles/115015385887-Contact-Us}{Contact
  Us}
\item
  \href{https://www.nytco.com/careers/}{Work with us}
\item
  \href{https://nytmediakit.com/}{Advertise}
\item
  \href{http://www.tbrandstudio.com/}{T Brand Studio}
\item
  \href{https://www.nytimes.com/privacy/cookie-policy\#how-do-i-manage-trackers}{Your
  Ad Choices}
\item
  \href{https://www.nytimes.com/privacy}{Privacy}
\item
  \href{https://help.nytimes.com/hc/en-us/articles/115014893428-Terms-of-service}{Terms
  of Service}
\item
  \href{https://help.nytimes.com/hc/en-us/articles/115014893968-Terms-of-sale}{Terms
  of Sale}
\item
  \href{https://spiderbites.nytimes.com}{Site Map}
\item
  \href{https://help.nytimes.com/hc/en-us}{Help}
\item
  \href{https://www.nytimes.com/subscription?campaignId=37WXW}{Subscriptions}
\end{itemize}
