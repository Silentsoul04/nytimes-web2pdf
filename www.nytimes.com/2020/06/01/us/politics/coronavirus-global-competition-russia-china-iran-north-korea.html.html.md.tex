Sections

SEARCH

\protect\hyperlink{site-content}{Skip to
content}\protect\hyperlink{site-index}{Skip to site index}

\href{https://www.nytimes.com/section/politics}{Politics}

\href{https://myaccount.nytimes.com/auth/login?response_type=cookie\&client_id=vi}{}

\href{https://www.nytimes.com/section/todayspaper}{Today's Paper}

\href{/section/politics}{Politics}\textbar{}As Virus Toll Preoccupies
U.S., Rivals Test Limits of American Power

\url{https://nyti.ms/2TVS7P5}

\begin{itemize}
\item
\item
\item
\item
\item
\item
\end{itemize}

\href{https://www.nytimes.com/news-event/coronavirus?action=click\&pgtype=Article\&state=default\&region=TOP_BANNER\&context=storylines_menu}{The
Coronavirus Outbreak}

\begin{itemize}
\tightlist
\item
  live\href{https://www.nytimes.com/2020/08/01/world/coronavirus-covid-19.html?action=click\&pgtype=Article\&state=default\&region=TOP_BANNER\&context=storylines_menu}{Latest
  Updates}
\item
  \href{https://www.nytimes.com/interactive/2020/us/coronavirus-us-cases.html?action=click\&pgtype=Article\&state=default\&region=TOP_BANNER\&context=storylines_menu}{Maps
  and Cases}
\item
  \href{https://www.nytimes.com/interactive/2020/science/coronavirus-vaccine-tracker.html?action=click\&pgtype=Article\&state=default\&region=TOP_BANNER\&context=storylines_menu}{Vaccine
  Tracker}
\item
  \href{https://www.nytimes.com/interactive/2020/07/29/us/schools-reopening-coronavirus.html?action=click\&pgtype=Article\&state=default\&region=TOP_BANNER\&context=storylines_menu}{What
  School May Look Like}
\item
  \href{https://www.nytimes.com/live/2020/07/31/business/stock-market-today-coronavirus?action=click\&pgtype=Article\&state=default\&region=TOP_BANNER\&context=storylines_menu}{Economy}
\end{itemize}

Advertisement

\protect\hyperlink{after-top}{Continue reading the main story}

Supported by

\protect\hyperlink{after-sponsor}{Continue reading the main story}

\hypertarget{as-virus-toll-preoccupies-us-rivals-test-limits-of-american-power}{%
\section{As Virus Toll Preoccupies U.S., Rivals Test Limits of American
Power}\label{as-virus-toll-preoccupies-us-rivals-test-limits-of-american-power}}

The coronavirus may have changed almost everything, but it didn't change
this: Global competition spins ahead --- and in many ways has
accelerated.

\includegraphics{https://static01.nyt.com/images/2020/06/02/us/politics/31dc-virus-globalcompetition-pix1/merlin_171878973_fc59eff8-86ba-42f3-9c31-16b4637ba3af-articleLarge.jpg?quality=75\&auto=webp\&disable=upscale}

\href{https://www.nytimes.com/by/david-e-sanger}{\includegraphics{https://static01.nyt.com/images/2018/10/03/multimedia/author-david-e-sanger/author-david-e-sanger-thumbLarge.png}}\href{https://www.nytimes.com/by/eric-schmitt}{\includegraphics{https://static01.nyt.com/images/2018/06/12/multimedia/author-eric-schmitt/author-eric-schmitt-thumbLarge-v2.png}}\href{https://www.nytimes.com/by/edward-wong}{\includegraphics{https://static01.nyt.com/images/2018/09/24/multimedia/author-edward-wong/author-edward-wong-thumbLarge-v5.png}}

By \href{https://www.nytimes.com/by/david-e-sanger}{David E. Sanger},
\href{https://www.nytimes.com/by/eric-schmitt}{Eric Schmitt} and
\href{https://www.nytimes.com/by/edward-wong}{Edward Wong}

\begin{itemize}
\item
  Published June 1, 2020Updated June 2, 2020
\item
  \begin{itemize}
  \item
  \item
  \item
  \item
  \item
  \item
  \end{itemize}
\end{itemize}

WASHINGTON --- With the United States preoccupied by the sobering
reality of more than
\href{https://www.nytimes.com/interactive/2020/05/24/us/us-coronavirus-deaths-100000.html}{100,000
Americans dead} from the coronavirus, China has pushed in recent weeks
to
\href{https://www.nytimes.com/2020/05/30/world/asia/india-china-border.html}{move
troops} into disputed territory with India, continue aggressive actions
in the South China Sea and rewrite the rules of
\href{https://www.nytimes.com/2020/05/24/world/asia/china-hong-kong-taiwan.html}{how
it will control Hong Kong}.

At roughly the same time, Russian fighter jets roared dangerously close
to American Navy planes over the Mediterranean Sea, while the country's
space forces conducted an antisatellite missile test clearly aimed at
sending the message that Moscow could blind U.S. spy satellites and take
down GPS and other communications systems. Russia's military cyberunits
were busy, too,
\href{https://www.nytimes.com/2020/05/28/us/politics/nsa-russian-hack.html}{the
National Security Agency reported}, with an innovative attack that may
portend accelerated planning for a strike on email systems this election
year.

The North Koreans said they were accelerating their ``nuclear
deterrent,'' moving beyond two years of vague promises of disarmament
and Kim Jong-un's warm exchanges of letters with President Trump. Iran,
Secretary of State Mike Pompeo said, is re-establishing the
infrastructure needed to make a bomb --- all a reaction, the Iranians
insist, to Mr. Trump's decision two years ago to reimpose sanctions,
reaffirmed in recent weeks as the State Department dismantled the last
elements of the Obama-era nuclear deal. Various powers are testing
American cybersecurity.

The coronavirus may have changed almost everything, but it did not
change this: Global challenges to the United States spin ahead, with
America's adversaries testing the limits and seeing what gains they can
make with minimal pushback.

It has not created a new reality as much as it has widened divisions
that existed before the pandemic. And with the United States looking
inward, preoccupied by the fear of more viral waves, unemployment
soaring over 20 percent and nationwide protests ignited by deadly police
brutality, its competitors are moving to fill the vacuum, and quickly.

In some cases, Mr. Trump has helped them along. His announcement on
Friday that the United States was
\href{https://www.nytimes.com/2020/05/29/us/politics/trump-hong-kong-china-WHO.html}{severing
ties} with the World Health Organization left the field clear for China
to broaden its influence over the organization. On Saturday, Mr. Trump
delivered a gift to President Vladimir V. Putin of Russia: Aboard Air
Force One, almost offhandedly,
\href{https://www.nytimes.com/2020/05/30/us/politics/trump-g7-russia.html}{he
said he would invite Mr. Putin} to an expanded meeting of the Group of 7
nations. Russia was banned from meetings of the world's major economic
powers after its 2014 annexation of Crimea and attacks on eastern
Ukraine.

\includegraphics{https://static01.nyt.com/images/2020/05/31/us/politics/31dc-virus-globalcompetition-pix5/merlin_173005566_16810ce9-e569-4458-b9ef-a71d7f6a47e2-articleLarge.jpg?quality=75\&auto=webp\&disable=upscale}

Most of the European allies have rejected past proposals to bring Russia
back into the fold, noting that Moscow has never loosened its hold on
Crimea, and Mr. Trump did not explain his change of policy. Apart from
\href{https://twitter.com/SecPompeo/status/1232851640698404864?s=20}{Mr.
Pompeo's declaration in February} that the United States ``does not and
will not ever recognize'' Russia's claim to the region, though, Mr.
Trump's proposal suggests the United States is moving on.

Mr. Trump
\href{https://www.nytimes.com/2018/06/19/us/politics/trump-israel-palestinians-human-rights.html}{has
also withdrawn} from various U.N. bodies and from
\href{https://www.nytimes.com/2017/06/01/climate/trump-paris-climate-agreement.html}{important
international accords}, most recently the Open Skies Treaty --- actions
that also weaken ties with allies and cede ground to China, Russia and
others.

The retreat is also happening in sub-Saharan Africa, where Defense
Secretary Mark T. Esper is weighing cuts in U.S. troop levels and aid to
French-led counterterrorism efforts in ways that analysts say could open
the door to China and Russia. Already, they are dangling deals for new
ports and railroads, arms and mercenaries, and medical supplies to help
combat Covid-19.

``The scope of medical and economic disruption that will come from
Covid-19 will leave opportunities for both nations, and others, to try
to gain advantages,'' Stanley A. McChrystal, a retired four-star
commander of the Joint Special Operations Command and American forces in
Afghanistan, said in an interview.

\hypertarget{latest-updates-global-coronavirus-outbreak}{%
\section{\texorpdfstring{\href{https://www.nytimes.com/2020/08/01/world/coronavirus-covid-19.html?action=click\&pgtype=Article\&state=default\&region=MAIN_CONTENT_1\&context=storylines_live_updates}{Latest
Updates: Global Coronavirus
Outbreak}}{Latest Updates: Global Coronavirus Outbreak}}\label{latest-updates-global-coronavirus-outbreak}}

Updated 2020-08-02T10:04:29.623Z

\begin{itemize}
\tightlist
\item
  \href{https://www.nytimes.com/2020/08/01/world/coronavirus-covid-19.html?action=click\&pgtype=Article\&state=default\&region=MAIN_CONTENT_1\&context=storylines_live_updates\#link-34047410}{The
  U.S. reels as July cases more than double the total of any other
  month.}
\item
  \href{https://www.nytimes.com/2020/08/01/world/coronavirus-covid-19.html?action=click\&pgtype=Article\&state=default\&region=MAIN_CONTENT_1\&context=storylines_live_updates\#link-780ec966}{Top
  U.S. officials work to break an impasse over the federal jobless
  benefit.}
\item
  \href{https://www.nytimes.com/2020/08/01/world/coronavirus-covid-19.html?action=click\&pgtype=Article\&state=default\&region=MAIN_CONTENT_1\&context=storylines_live_updates\#link-2bc8948}{Its
  outbreak untamed, Melbourne goes into even greater lockdown.}
\end{itemize}

\href{https://www.nytimes.com/2020/08/01/world/coronavirus-covid-19.html?action=click\&pgtype=Article\&state=default\&region=MAIN_CONTENT_1\&context=storylines_live_updates}{See
more updates}

More live coverage:
\href{https://www.nytimes.com/live/2020/07/31/business/stock-market-today-coronavirus?action=click\&pgtype=Article\&state=default\&region=MAIN_CONTENT_1\&context=storylines_live_updates}{Markets}

The United States has not stayed entirely on the sidelines, though,
creating potential arenas for new competition and possible collision.
The race for a coronavirus vaccine has come to involve both China's
People's Liberation Army and the U.S. military, which has said it would
mobilize to distribute any breakthrough discovery.

American warships have sailed into disputed waters in the South China
Sea in recent weeks to assert freedom-of-navigation rights, continuing a
standoff in a region that Beijing asserts is its territory, backed up by
the establishment of new air bases.

And the United States is speeding ahead in a renewed conventional and
nuclear arms race, though its strategic rationale --- other than to
overmatch Russia and China --- has never been fully described by this
administration. Not long after the Pentagon announced in March that it
had successfully tested an unarmed prototype of a hypersonic missile, a
weapon that could potentially overwhelm an adversary's defense systems,
Mr. Trump boasted that a ``super duper'' missile was on the way.
Presumably it is intended as an answer to Russia's introduction of
\href{https://www.nytimes.com/2019/12/27/us/politics/russia-hypersonic-weapon.html}{the
Avangard}, which made it the first country to claim it had deployed an
operable hypersonic weapon, and a range of similar weapons that China is
developing.

Mr. Trump's new arms control negotiator,
\href{https://www.nytimes.com/2020/05/21/us/politics/trump-open-skies-treaty-arms-control.html}{Marshall
Billingslea}, warned recently that Mr. Trump meant it when he vowed that
America would always have the most potent nuclear force in the world.
``We know how to win these races, and we know how to spend the adversary
into oblivion,'' he said, even as the country ran up record deficits to
avoid an economic implosion because of the virus. ``If we have to, we
will, but we sure would like to avoid it.''

\hypertarget{middle-east-power-vacuum}{%
\subsection{Middle East Power Vacuum}\label{middle-east-power-vacuum}}

It is not only China and Russia that are challenging the United States.
Across the Middle East, there is a sense that Mr. Trump's oft-expressed
desire to withdraw from the region --- along with his
\href{https://www.whitehouse.gov/wp-content/uploads/2017/12/NSS-Final-12-18-2017-0905.pdf}{National
Security Strategy}'s focus on a renewed competition among superpowers
--- offers new leeway.

Iran has bet that Mr. Trump, for all his emphasis on doubling down on
sanctions as he completes America's exit from the 2015 nuclear deal, is
not willing to risk outright confrontation. Tehran has gradually
accelerated its production of nuclear fuel and ignored requests from
international inspectors for access to suspected nuclear-related sites.
But it has not raced ahead, perhaps calculating that a slow rebuilding
of its stockpiles will not result in a strong international backlash.

Image

Several Iranian fast boats approached U.S. warships in the Persian Gulf
in mid-April.Credit...Navy Office of Information, via Agence
France-Presse --- Getty Images

And in the Persian Gulf, even after
\href{https://www.nytimes.com/2020/01/02/world/middleeast/qassem-soleimani-iraq-iran-attack.html?searchResultPosition=1}{the
U.S.-led killing in January of Qassim Suleimani}, a senior commander in
the Islamic Revolutionary Guards Corps and Iran's terrorism mastermind,
Tehran is episodically testing America's limits.

Nearly a dozen Iranian fast boats conducted what the Navy described as
``dangerous and harassing approaches'' to six American warships in the
Persian Gulf in mid-April, prompting
\href{https://www.nytimes.com/2020/04/22/world/middleeast/iran-trump-navy-persian-gulf-satellite.html}{Mr.
Trump's order} ``to shoot down and destroy any and all Iranian gunboats
if they harass our ships at sea.'' Iran backed off in the gulf --- but
then stepped up oil shipments to Venezuela, in a challenge to the
U.S.-led embargo meant to displace President Nicolás Maduro, who has
stayed in office despite a vigorous American campaign to force him out.

In mid-May, Iran's foreign minister, Mohammad Javad Zarif, said American
attempts to disrupt the course of Iranian tankers carrying fuel for
Venezuela were ``dangerous'' and ``provocative'' acts. Iran has
threatened retaliation against U.S. forces in the gulf and throughout
the Middle East if Washington interferes with Tehran's oil deliveries.

And in Iraq and Syria, the Islamic State, a year after losing its last
territorial foothold, is resurgent with a spate of roadside bombings,
ambushes and other attacks as U.S. troops in Iraq pull back from four
bases and suspend training in the country, along with other Western
allies, because of coronavirus restrictions. Mr. Trump, after initially
declaring in 2018 that the group had been defeated, has barely mentioned
its recent gains.

Russia and China are active in the region. Russia continues to support
the government of President Bashar al-Assad as he nears a brutal victory
in Syria's civil war. And China maintains a military base in Djibouti,
near an American one there. Chinese diplomats and state-owned
enterprises have increased their presence throughout the region.

``China has significantly expanded its engagement in the region,
especially in the economic and diplomatic realms,'' said Patricia M.
Kim, a China analyst at the U.S. Institute of Peace who worked on
\href{https://www.usip.org/publications/2020/04/chinas-impact-conflict-dynamics-red-sea-arena}{a
recent report on China and the Red Sea area}. ``And for the U.S. to
remain relevant --- to be able to shape norms in the region and help
states manage China's growing presence --- it needs to significantly
increase its own engagement.''

\hypertarget{from-russia-testing-boundaries}{%
\subsection{From Russia, Testing
Boundaries}\label{from-russia-testing-boundaries}}

Mr. Trump's willingness to invite Mr. Putin back into the company of the
major Western allies --- partly as an effort to counter China --- is all
the more mystifying because friction between American and Russian forces
is running high. From international territory and airspace off Alaska to
the Black Sea, combat planes and warships are pressing new boundaries
and renewing years-old brinkmanship.

On Friday, two U.S. Air Force B-1B bombers flying a long-range training
mission over the Black Sea prompted Russian fighter jets to scramble and
intercept the American warplanes. At least three times in the past two
months, Russian fighter jets intercepted Navy P-8 surveillance planes
over the Mediterranean,
\href{https://www.c6f.navy.mil/Press-Room/News/Article/2198048/unsafe-unprofessional-interception-of-a-us-navy-p-8-by-russian-su-35s-over-the/}{most
recently} on Wednesday.

Image

A Russian fighter jet intercepted a U.S. Navy P-8 surveillance plane
over the Mediterranean last week.Credit...U.S. Naval Forces
Europe-Africa, via Reuters

In an intercept in April, a Russian jet conducted a high-speed, inverted
maneuver, 25 feet in front of the P-8. ``Another unsafe \#Russian
intercept of @USNavy P-8 in international airspace above \#Mediterranean
Sea!'' the U.S. military wrote,
\href{https://www.youtube.com/watch?v=h5tk7tLbOpo}{tweeting a video} of
the encounter.

If these had been encounters with Iranian or Chinese forces, Mr. Trump
would have almost certainly protested. But amid the throes of a
pandemic, he has not been eager to ratchet up tensions with Russia. ``I
don't see it,'' Mr. Trump said when asked whether Russia was toying with
U.S. military forces. ``We had a very good relationship with Russia.''

That is not what top NATO officials and American commanders say.

The U.S. military on Tuesday
\href{https://www.nytimes.com/2020/05/26/world/middleeast/russia-warplanes-libya.html}{accused
the Kremlin of secretly sending} at least 14 fighter jets to eastern
Libya in May to support Russian mercenaries battling alongside a
beleaguered commander,
Khalifa\href{https://www.nytimes.com/2020/02/18/us/politics/hifter-torture-lawsuit-libya.html}{Hifter},
in his campaign to oust the internationally recognized government in
Tripoli, the capital.

The unusually blunt and public criticism by two top American generals
underscored the Pentagon's broader concern about Moscow's growing
influence in Libya and a looming security threat on NATO's southern
flank.

\href{https://www.nytimes.com/news-event/coronavirus?action=click\&pgtype=Article\&state=default\&region=MAIN_CONTENT_3\&context=storylines_faq}{}

\hypertarget{the-coronavirus-outbreak-}{%
\subsubsection{The Coronavirus Outbreak
›}\label{the-coronavirus-outbreak-}}

\hypertarget{frequently-asked-questions}{%
\paragraph{Frequently Asked
Questions}\label{frequently-asked-questions}}

Updated July 27, 2020

\begin{itemize}
\item ~
  \hypertarget{should-i-refinance-my-mortgage}{%
  \paragraph{Should I refinance my
  mortgage?}\label{should-i-refinance-my-mortgage}}

  \begin{itemize}
  \tightlist
  \item
    \href{https://www.nytimes.com/article/coronavirus-money-unemployment.html?action=click\&pgtype=Article\&state=default\&region=MAIN_CONTENT_3\&context=storylines_faq}{It
    could be a good idea,} because mortgage rates have
    \href{https://www.nytimes.com/2020/07/16/business/mortgage-rates-below-3-percent.html?action=click\&pgtype=Article\&state=default\&region=MAIN_CONTENT_3\&context=storylines_faq}{never
    been lower.} Refinancing requests have pushed mortgage applications
    to some of the highest levels since 2008, so be prepared to get in
    line. But defaults are also up, so if you're thinking about buying a
    home, be aware that some lenders have tightened their standards.
  \end{itemize}
\item ~
  \hypertarget{what-is-school-going-to-look-like-in-september}{%
  \paragraph{What is school going to look like in
  September?}\label{what-is-school-going-to-look-like-in-september}}

  \begin{itemize}
  \tightlist
  \item
    It is unlikely that many schools will return to a normal schedule
    this fall, requiring the grind of
    \href{https://www.nytimes.com/2020/06/05/us/coronavirus-education-lost-learning.html?action=click\&pgtype=Article\&state=default\&region=MAIN_CONTENT_3\&context=storylines_faq}{online
    learning},
    \href{https://www.nytimes.com/2020/05/29/us/coronavirus-child-care-centers.html?action=click\&pgtype=Article\&state=default\&region=MAIN_CONTENT_3\&context=storylines_faq}{makeshift
    child care} and
    \href{https://www.nytimes.com/2020/06/03/business/economy/coronavirus-working-women.html?action=click\&pgtype=Article\&state=default\&region=MAIN_CONTENT_3\&context=storylines_faq}{stunted
    workdays} to continue. California's two largest public school
    districts --- Los Angeles and San Diego --- said on July 13, that
    \href{https://www.nytimes.com/2020/07/13/us/lausd-san-diego-school-reopening.html?action=click\&pgtype=Article\&state=default\&region=MAIN_CONTENT_3\&context=storylines_faq}{instruction
    will be remote-only in the fall}, citing concerns that surging
    coronavirus infections in their areas pose too dire a risk for
    students and teachers. Together, the two districts enroll some
    825,000 students. They are the largest in the country so far to
    abandon plans for even a partial physical return to classrooms when
    they reopen in August. For other districts, the solution won't be an
    all-or-nothing approach.
    \href{https://bioethics.jhu.edu/research-and-outreach/projects/eschool-initiative/school-policy-tracker/}{Many
    systems}, including the nation's largest, New York City, are
    devising
    \href{https://www.nytimes.com/2020/06/26/us/coronavirus-schools-reopen-fall.html?action=click\&pgtype=Article\&state=default\&region=MAIN_CONTENT_3\&context=storylines_faq}{hybrid
    plans} that involve spending some days in classrooms and other days
    online. There's no national policy on this yet, so check with your
    municipal school system regularly to see what is happening in your
    community.
  \end{itemize}
\item ~
  \hypertarget{is-the-coronavirus-airborne}{%
  \paragraph{Is the coronavirus
  airborne?}\label{is-the-coronavirus-airborne}}

  \begin{itemize}
  \tightlist
  \item
    The coronavirus
    \href{https://www.nytimes.com/2020/07/04/health/239-experts-with-one-big-claim-the-coronavirus-is-airborne.html?action=click\&pgtype=Article\&state=default\&region=MAIN_CONTENT_3\&context=storylines_faq}{can
    stay aloft for hours in tiny droplets in stagnant air}, infecting
    people as they inhale, mounting scientific evidence suggests. This
    risk is highest in crowded indoor spaces with poor ventilation, and
    may help explain super-spreading events reported in meatpacking
    plants, churches and restaurants.
    \href{https://www.nytimes.com/2020/07/06/health/coronavirus-airborne-aerosols.html?action=click\&pgtype=Article\&state=default\&region=MAIN_CONTENT_3\&context=storylines_faq}{It's
    unclear how often the virus is spread} via these tiny droplets, or
    aerosols, compared with larger droplets that are expelled when a
    sick person coughs or sneezes, or transmitted through contact with
    contaminated surfaces, said Linsey Marr, an aerosol expert at
    Virginia Tech. Aerosols are released even when a person without
    symptoms exhales, talks or sings, according to Dr. Marr and more
    than 200 other experts, who
    \href{https://academic.oup.com/cid/article/doi/10.1093/cid/ciaa939/5867798}{have
    outlined the evidence in an open letter to the World Health
    Organization}.
  \end{itemize}
\item ~
  \hypertarget{what-are-the-symptoms-of-coronavirus}{%
  \paragraph{What are the symptoms of
  coronavirus?}\label{what-are-the-symptoms-of-coronavirus}}

  \begin{itemize}
  \tightlist
  \item
    Common symptoms
    \href{https://www.nytimes.com/article/symptoms-coronavirus.html?action=click\&pgtype=Article\&state=default\&region=MAIN_CONTENT_3\&context=storylines_faq}{include
    fever, a dry cough, fatigue and difficulty breathing or shortness of
    breath.} Some of these symptoms overlap with those of the flu,
    making detection difficult, but runny noses and stuffy sinuses are
    less common.
    \href{https://www.nytimes.com/2020/04/27/health/coronavirus-symptoms-cdc.html?action=click\&pgtype=Article\&state=default\&region=MAIN_CONTENT_3\&context=storylines_faq}{The
    C.D.C. has also} added chills, muscle pain, sore throat, headache
    and a new loss of the sense of taste or smell as symptoms to look
    out for. Most people fall ill five to seven days after exposure, but
    symptoms may appear in as few as two days or as many as 14 days.
  \end{itemize}
\item ~
  \hypertarget{does-asymptomatic-transmission-of-covid-19-happen}{%
  \paragraph{Does asymptomatic transmission of Covid-19
  happen?}\label{does-asymptomatic-transmission-of-covid-19-happen}}

  \begin{itemize}
  \tightlist
  \item
    So far, the evidence seems to show it does. A widely cited
    \href{https://www.nature.com/articles/s41591-020-0869-5}{paper}
    published in April suggests that people are most infectious about
    two days before the onset of coronavirus symptoms and estimated that
    44 percent of new infections were a result of transmission from
    people who were not yet showing symptoms. Recently, a top expert at
    the World Health Organization stated that transmission of the
    coronavirus by people who did not have symptoms was ``very rare,''
    \href{https://www.nytimes.com/2020/06/09/world/coronavirus-updates.html?action=click\&pgtype=Article\&state=default\&region=MAIN_CONTENT_3\&context=storylines_faq\#link-1f302e21}{but
    she later walked back that statement.}
  \end{itemize}
\end{itemize}

Closer to home, Air Force F-22 Raptor fighter jets intercepted two
Russian maritime patrol planes in April about 50 miles from Alaska's
Aleutian Islands, in an echo of the Cold War. A month earlier, a pair of
Russian reconnaissance aircraft were intercepted by U.S. and Canadian
jets 50 miles from the state's coast over the Beaufort Sea.

The North American Aerospace Defense Command, or NORAD, said the Russian
aircraft were intercepted in the Bering Sea north of the Aleutian
Islands and never entered U.S. or Canadian airspace.

In mid-March, two Russian strategic bombers flew over a U.S. submarine
that surfaced in the Arctic Ocean and were subsequently escorted by
American and Canadian fighter jets.

``What we do see is, I think, a continuous effort for them --- as they
do in the Covid-19 environment, outside the Covid-19 environment --- to
continually probe and check and see our responses,'' said Gen. Terrence
J. O'Shaughnessy, the head of the military's Northern Command, which
oversees homeland defense.

\hypertarget{china-seizes-the-moment}{%
\subsection{China Seizes the Moment}\label{china-seizes-the-moment}}

During the 2016 campaign, Mr. Trump spoke publicly,
\href{https://www.nytimes.com/2016/03/27/us/politics/donald-trump-foreign-policy.html}{in
a New York Times interview}, about leaving it to South Korea and Japan
to secure the Pacific, saying he was tired of paying so much to help
defend allies who were running big trade surpluses with the United
States. And as Mr. Trump has argued with Seoul and Tokyo, and not
significantly bolstered ties with Southeast Asia, President Xi Jinping
of China has seen his moment of opportunity.

From the waters of the Pacific and Indian Oceans to the heights of the
Himalayas, China has pressed forward on expanding its military
footprint.

Image

Road maintenance workers in the Ladakh region of India where Chinese
troops have recently made several incursions over the
border.Credit...Manish Swarup/Associated Press

``I think what Beijing is pursuing --- and it's a rational interest ---
is hegemonic authority over Asia,'' said Elbridge Colby, the former
Pentagon official who was the main writer of the Trump administration's
\href{https://dod.defense.gov/Portals/1/Documents/pubs/2018-National-Defense-Strategy-Summary.pdf}{National
Defense Strategy}, which focuses on how the American military should
reshape itself for great-power competition with Russia and particularly
China.

It is most evident in the South China Sea. Beijing has continued with
\href{https://www.washingtonpost.com/politics/2020/05/07/does-global-pandemic-open-new-south-china-sea-opportunities-beijing-not-really/}{its
yearslong strategy} of pressing maximal territorial claims. Turning
outcroppings of rock into full islands, it is forming a bulwark against
the claims of competing nations and against the findings of a 2016
international tribunal, which sought to limit China's aggressive
maritime actions.

In April, a Chinese Coast Guard vessel collided with a Vietnamese
fishing boat near a disputed archipelago, sinking the small vessel. The
same month a Chinese seismic survey ship, escorted by Chinese Coast
Guard vessels, entered waters designated as the exclusive economic zone
of Malaysia, daring the Malaysians to push back. There have been
parallel confrontations with Indonesia and the Philippines.

The Trump administration has continued President Barack Obama's policy
of not taking sides in the territorial disputes while asserting that the
United States aims to maintain freedom of navigation in the region. Mr.
Esper insists that the United States will continue naval operations ``to
send a clear message to Beijing that we will continue to protect freedom
of navigation and commerce for all nations, large and small.''

But China's leaders appear to suspect that they are empty words; Mr.
Trump has no appetite for facing off with Beijing over scarcely
populated territory half a world away.

And in
\href{https://www.wsj.com/articles/china-breaks-with-taiwan-precedent-omitting-call-for-peaceful-unification-11590151372}{an
annual policy report} last month, the Chinese government dropped the
term ``peaceful reunification'' when discussing Taiwan, the democratic,
self-governing island, breaking with a tradition of using that phrase in
the reports since 1992. Li Keqiang, the Chinese prime minister, also
omitted ``peaceful'' when he called for reunification at the opening
session of the National People's Congress on May 22.

The U.S. Navy has announced at least three instances of transits of its
warships through the Taiwan Strait this year. And last month, the State
Department notified Congress of a
\href{https://www.reuters.com/article/us-taiwan-usa-security/us-to-sell-taiwan-180-million-worth-of-torpedoes-idUSKBN22X01N}{potential
sale of advanced torpedoes} to Taiwan worth \$180 million, the latest of
several large arms sale packages to the island. But that is not enough,
some experts say.

``We need to change things on Taiwan to improve the deterrent and make
clearer where we stand,'' said Mr. Colby, who added that the United
States had to ``end any remaining ambiguity about how we'd react to the
use of force.'' Without that, China may well doubt that Mr. Trump sees
Taiwan's de facto independence as a vital American interest.

Tensions involving China extend to the roof of the world. Along a
disputed border in the Himalayas, Indian and Chinese troops engaged in
scuffles and shouting matches in recent weeks. Indian officials say the
Chinese military made at least one major incursion into Indian
territory. Both sides have amassed thousands of troops in the disputed
areas, leading to the tensest such standoff since 2017.

On Wednesday, Mr. Trump
\href{https://twitter.com/realDonaldTrump/status/1265604027678670848?s=20}{weighed
in} via Twitter. ``We have informed both India and China that the United
States is ready, willing and able to mediate or arbitrate their now
raging border dispute,'' he wrote, in an echo of an offer he made last
year on the India-Pakistan conflict over Kashmir.

Neither side seemed interested in his offer.

Advertisement

\protect\hyperlink{after-bottom}{Continue reading the main story}

\hypertarget{site-index}{%
\subsection{Site Index}\label{site-index}}

\hypertarget{site-information-navigation}{%
\subsection{Site Information
Navigation}\label{site-information-navigation}}

\begin{itemize}
\tightlist
\item
  \href{https://help.nytimes.com/hc/en-us/articles/115014792127-Copyright-notice}{©~2020~The
  New York Times Company}
\end{itemize}

\begin{itemize}
\tightlist
\item
  \href{https://www.nytco.com/}{NYTCo}
\item
  \href{https://help.nytimes.com/hc/en-us/articles/115015385887-Contact-Us}{Contact
  Us}
\item
  \href{https://www.nytco.com/careers/}{Work with us}
\item
  \href{https://nytmediakit.com/}{Advertise}
\item
  \href{http://www.tbrandstudio.com/}{T Brand Studio}
\item
  \href{https://www.nytimes.com/privacy/cookie-policy\#how-do-i-manage-trackers}{Your
  Ad Choices}
\item
  \href{https://www.nytimes.com/privacy}{Privacy}
\item
  \href{https://help.nytimes.com/hc/en-us/articles/115014893428-Terms-of-service}{Terms
  of Service}
\item
  \href{https://help.nytimes.com/hc/en-us/articles/115014893968-Terms-of-sale}{Terms
  of Sale}
\item
  \href{https://spiderbites.nytimes.com}{Site Map}
\item
  \href{https://help.nytimes.com/hc/en-us}{Help}
\item
  \href{https://www.nytimes.com/subscription?campaignId=37WXW}{Subscriptions}
\end{itemize}
