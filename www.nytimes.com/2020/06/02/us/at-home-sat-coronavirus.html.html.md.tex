Sections

SEARCH

\protect\hyperlink{site-content}{Skip to
content}\protect\hyperlink{site-index}{Skip to site index}

\href{https://www.nytimes.com/section/us}{U.S.}

\href{https://myaccount.nytimes.com/auth/login?response_type=cookie\&client_id=vi}{}

\href{https://www.nytimes.com/section/todayspaper}{Today's Paper}

\href{/section/us}{U.S.}\textbar{}College Board Scraps Plans for SAT at
Home

\url{https://nyti.ms/3gP2vC0}

\begin{itemize}
\item
\item
\item
\item
\item
\end{itemize}

\href{https://www.nytimes.com/news-event/coronavirus?action=click\&pgtype=Article\&state=default\&region=TOP_BANNER\&context=storylines_menu}{The
Coronavirus Outbreak}

\begin{itemize}
\tightlist
\item
  live\href{https://www.nytimes.com/2020/08/02/world/coronavirus-updates.html?action=click\&pgtype=Article\&state=default\&region=TOP_BANNER\&context=storylines_menu}{Latest
  Updates}
\item
  \href{https://www.nytimes.com/interactive/2020/us/coronavirus-us-cases.html?action=click\&pgtype=Article\&state=default\&region=TOP_BANNER\&context=storylines_menu}{Maps
  and Cases}
\item
  \href{https://www.nytimes.com/interactive/2020/science/coronavirus-vaccine-tracker.html?action=click\&pgtype=Article\&state=default\&region=TOP_BANNER\&context=storylines_menu}{Vaccine
  Tracker}
\item
  \href{https://www.nytimes.com/interactive/2020/07/29/us/schools-reopening-coronavirus.html?action=click\&pgtype=Article\&state=default\&region=TOP_BANNER\&context=storylines_menu}{What
  School May Look Like}
\item
  \href{https://www.nytimes.com/live/2020/07/31/business/stock-market-today-coronavirus?action=click\&pgtype=Article\&state=default\&region=TOP_BANNER\&context=storylines_menu}{Economy}
\end{itemize}

Advertisement

\protect\hyperlink{after-top}{Continue reading the main story}

Supported by

\protect\hyperlink{after-sponsor}{Continue reading the main story}

\hypertarget{college-board-scraps-plans-for-sat-at-home}{%
\section{College Board Scraps Plans for SAT at
Home}\label{college-board-scraps-plans-for-sat-at-home}}

The organization that oversees the standardized test used for college
admissions said the technology requirements for taking it remotely would
be too great for some students.

\includegraphics{https://static01.nyt.com/images/2020/06/02/us/02VIRUS-SAT/02VIRUS-SAT-articleLarge.jpg?quality=75\&auto=webp\&disable=upscale}

\href{https://www.nytimes.com/by/anemona-hartocollis}{\includegraphics{https://static01.nyt.com/images/2018/06/13/multimedia/author-anemona-hartocollis/author-anemona-hartocollis-thumbLarge-v3.jpg}}

By \href{https://www.nytimes.com/by/anemona-hartocollis}{Anemona
Hartocollis}

\begin{itemize}
\item
  Published June 2, 2020Updated June 5, 2020
\item
  \begin{itemize}
  \item
  \item
  \item
  \item
  \item
  \end{itemize}
\end{itemize}

The College Board said on Tuesday that it would postpone plans to offer
an online version of the SAT for high school students to take at home
this year, further muddying a ritual of the college application process
that had already been thrown into chaos by the coronavirus.

After canceling test dates this spring, the board announced in mid-April
that it was
\href{https://www.nytimes.com/2020/04/15/us/sat-act-test-coronavirus.html}{developing
a digital version of the SAT} to be introduced if the pandemic continued
to require social distancing in the fall, which would make it hard for
the nonprofit organization to provide enough testing dates and centers.

But in its latest statement, the board said the technological challenges
of developing an online test that all students could take had led to the
decision to drop it. Some 2.2 million students took the SAT last year,
the College Board said.

``Taking it would require three hours of uninterrupted, video-quality
internet for each student, which can't be guaranteed for all,'' the
board said, acknowledging the technology gap facing lower-income
students, which could further exacerbate inequities in access to higher
education.

The organization added that it would continue to deliver an online
version of the SAT at some schools, but would not ``introduce the stress
that could result from extended at-home testing in an already disrupted
admissions season.''

Bob Schaeffer, the head of FairTest, which is opposed to the use of
standardized tests in college admissions, said the College Board was
``simply conceding the inevitable.''

Its decision came after the organization had a rocky experience last
month introducing
\href{https://www.nytimes.com/2020/05/16/us/AP-exams-test-glitch-virus.html}{a
digital version of the Advanced Placement exams}, which it also
oversees. Many students complained that they were not able to submit
their answer sheets electronically, and their tests were disqualified.

Mr. Schaeffer's group and several students and parents have filed a
class-action lawsuit seeking to force the College Board to score the
rejected answer sheets. The College Board said less than 1 percent of
students who had taken the test were affected.

The College Board asked colleges and universities on Tuesday to ``show
flexibility'' to the millions of students who were not able to take the
SAT this spring because of cancellations. It asked colleges to extend
deadlines for receiving test scores, and to give equal consideration to
students who were unable to take the test because of the pandemic.

The SAT's rival exam, the ACT, said on Tuesday that it still planned to
offer a remote option in the fall.

Over the last few years, the College Board has been fighting a movement
to persuade colleges to drop the SAT and the ACT entirely, with
opponents arguing that they are biased along racial and income lines.

That movement has accelerated during the pandemic, with
\href{https://www.nytimes.com/article/sat-act-test-optional-colleges-coronavirus.html}{more
than 1,200 schools now making the tests optional}. In the biggest recent
blow, the 300,000-student
\href{https://www.nytimes.com/2020/05/21/us/university-california-sat-act.html}{University
of California system decided last month} to phase out the SAT and the
ACT at its 10 schools, which include some of the most sought-after in
the country.

Advertisement

\protect\hyperlink{after-bottom}{Continue reading the main story}

\hypertarget{site-index}{%
\subsection{Site Index}\label{site-index}}

\hypertarget{site-information-navigation}{%
\subsection{Site Information
Navigation}\label{site-information-navigation}}

\begin{itemize}
\tightlist
\item
  \href{https://help.nytimes.com/hc/en-us/articles/115014792127-Copyright-notice}{©~2020~The
  New York Times Company}
\end{itemize}

\begin{itemize}
\tightlist
\item
  \href{https://www.nytco.com/}{NYTCo}
\item
  \href{https://help.nytimes.com/hc/en-us/articles/115015385887-Contact-Us}{Contact
  Us}
\item
  \href{https://www.nytco.com/careers/}{Work with us}
\item
  \href{https://nytmediakit.com/}{Advertise}
\item
  \href{http://www.tbrandstudio.com/}{T Brand Studio}
\item
  \href{https://www.nytimes.com/privacy/cookie-policy\#how-do-i-manage-trackers}{Your
  Ad Choices}
\item
  \href{https://www.nytimes.com/privacy}{Privacy}
\item
  \href{https://help.nytimes.com/hc/en-us/articles/115014893428-Terms-of-service}{Terms
  of Service}
\item
  \href{https://help.nytimes.com/hc/en-us/articles/115014893968-Terms-of-sale}{Terms
  of Sale}
\item
  \href{https://spiderbites.nytimes.com}{Site Map}
\item
  \href{https://help.nytimes.com/hc/en-us}{Help}
\item
  \href{https://www.nytimes.com/subscription?campaignId=37WXW}{Subscriptions}
\end{itemize}
