Sections

SEARCH

\protect\hyperlink{site-content}{Skip to
content}\protect\hyperlink{site-index}{Skip to site index}

\href{https://www.nytimes.com/section/us}{U.S.}

\href{https://myaccount.nytimes.com/auth/login?response_type=cookie\&client_id=vi}{}

\href{https://www.nytimes.com/section/todayspaper}{Today's Paper}

\href{/section/us}{U.S.}\textbar{}`Did I Miss Anything?': A Man Emerges
From a 75-Day Silent Retreat

\url{https://nyti.ms/2XS7s4l}

\begin{itemize}
\item
\item
\item
\item
\item
\end{itemize}

\href{https://www.nytimes.com/news-event/coronavirus?action=click\&pgtype=Article\&state=default\&region=TOP_BANNER\&context=storylines_menu}{The
Coronavirus Outbreak}

\begin{itemize}
\tightlist
\item
  live\href{https://www.nytimes.com/2020/08/01/world/coronavirus-covid-19.html?action=click\&pgtype=Article\&state=default\&region=TOP_BANNER\&context=storylines_menu}{Latest
  Updates}
\item
  \href{https://www.nytimes.com/interactive/2020/us/coronavirus-us-cases.html?action=click\&pgtype=Article\&state=default\&region=TOP_BANNER\&context=storylines_menu}{Maps
  and Cases}
\item
  \href{https://www.nytimes.com/interactive/2020/science/coronavirus-vaccine-tracker.html?action=click\&pgtype=Article\&state=default\&region=TOP_BANNER\&context=storylines_menu}{Vaccine
  Tracker}
\item
  \href{https://www.nytimes.com/interactive/2020/07/29/us/schools-reopening-coronavirus.html?action=click\&pgtype=Article\&state=default\&region=TOP_BANNER\&context=storylines_menu}{What
  School May Look Like}
\item
  \href{https://www.nytimes.com/live/2020/07/31/business/stock-market-today-coronavirus?action=click\&pgtype=Article\&state=default\&region=TOP_BANNER\&context=storylines_menu}{Economy}
\end{itemize}

Advertisement

\protect\hyperlink{after-top}{Continue reading the main story}

Supported by

\protect\hyperlink{after-sponsor}{Continue reading the main story}

\hypertarget{did-i-miss-anything-a-man-emerges-from-a-75-day-silent-retreat}{%
\section{`Did I Miss Anything?': A Man Emerges From a 75-Day Silent
Retreat}\label{did-i-miss-anything-a-man-emerges-from-a-75-day-silent-retreat}}

Daniel Thorson went into a silent retreat in mid-March, meditating
through 75 coronavirus news cycles, Boris Johnson's hospitalization,
social distancing and sourdough starter. Now he's catching up.

\includegraphics{https://static01.nyt.com/images/2020/06/02/us/00virus-ripvanwinkle/merlin_172961856_0b8e6857-dba3-4b9a-824b-369756d52712-articleLarge.jpg?quality=75\&auto=webp\&disable=upscale}

\href{https://www.nytimes.com/by/ellen-barry}{\includegraphics{https://static01.nyt.com/images/2018/10/08/multimedia/author-ellen-barry/author-ellen-barry-thumbLarge.png}}

By \href{https://www.nytimes.com/by/ellen-barry}{Ellen Barry}

\begin{itemize}
\item
  Published June 2, 2020Updated June 4, 2020
\item
  \begin{itemize}
  \item
  \item
  \item
  \item
  \item
  \end{itemize}
\end{itemize}

\hypertarget{listen-to-this-article}{%
\subsubsection{Listen to This Article}\label{listen-to-this-article}}

Audio Recording by Audm

\emph{To hear more audio stories from publishers like The New York
Times,
download}\href{https://www.audm.com/?utm_source=nytmag\&utm_medium=embed\&utm_campaign=left_behind_draper}{**}\href{https://www.audm.com/?utm_source=nyt\&utm_medium=embed\&utm_campaign=rip_van_winkle}{\emph{Audm
for iPhone or Android}}\emph{.}

On the morning of May 23, Daniel Thorson rejoined society after an
absence of two and a half months.

He had spent that time in silent meditation in a cabin in remote
northwestern Vermont, where he is part of a **** Buddhist monastic
community. During his 75 days in isolation, his hair had grown out. The
last snow of winter had melted, and the trees had budded. Frogs had come
out of hibernation and begun peeping.

Mr. Thorson, a
\href{https://podcasts.apple.com/us/podcast/emerge-making-sense-of-whats-next/id1057220344}{podcaster}
and enthusiastic online philosopher, had also missed 75 news cycles. And
so, less than two **** hours after ending his silent retreat, Mr.
Thorson logged back onto Twitter.

\href{https://twitter.com/dthorson/status/1264231987025457157}{``Did I
miss anything?''} he wrote.

The last week was a strange one for Mr. Thorson, 33, a staff member at
the \href{https://www.monasticacademy.com/}{Monastic Academy}, as he
tried to catch up with the changes that had taken place during his
absence.

He learned of Boris Johnson's hospitalization --- and his recovery. He
learned that meatpacking plants had emerged as pockets of infection and
death. He learned that his cousin had met her new love interest on a
social-distance dating website. And that there is now such a thing as a
Zoom channel devoted to ecstatic dance.

Re-engaging --- with his mom, with the supermarket, with the internet
--- was at times intensely pleasurable. Other times it was just intense.
He had trouble sleeping.

People wanted to talk to him. They compared him to Rip Van Winkle, the
fictional character who falls asleep in the Catskills and wakes up 20
years later to discover that his beard is a foot long and the United
States is no longer ruled by the British Crown.

\hypertarget{latest-updates-global-coronavirus-outbreak}{%
\section{\texorpdfstring{\href{https://www.nytimes.com/2020/08/01/world/coronavirus-covid-19.html?action=click\&pgtype=Article\&state=default\&region=MAIN_CONTENT_1\&context=storylines_live_updates}{Latest
Updates: Global Coronavirus
Outbreak}}{Latest Updates: Global Coronavirus Outbreak}}\label{latest-updates-global-coronavirus-outbreak}}

Updated 2020-08-02T07:42:09.613Z

\begin{itemize}
\tightlist
\item
  \href{https://www.nytimes.com/2020/08/01/world/coronavirus-covid-19.html?action=click\&pgtype=Article\&state=default\&region=MAIN_CONTENT_1\&context=storylines_live_updates\#link-34047410}{The
  U.S. reels as July cases more than double the total of any other
  month.}
\item
  \href{https://www.nytimes.com/2020/08/01/world/coronavirus-covid-19.html?action=click\&pgtype=Article\&state=default\&region=MAIN_CONTENT_1\&context=storylines_live_updates\#link-780ec966}{Top
  U.S. officials work to break an impasse over the federal jobless
  benefit.}
\item
  \href{https://www.nytimes.com/2020/08/01/world/coronavirus-covid-19.html?action=click\&pgtype=Article\&state=default\&region=MAIN_CONTENT_1\&context=storylines_live_updates\#link-2bc8948}{Its
  outbreak untamed, Melbourne goes into even greater lockdown.}
\end{itemize}

\href{https://www.nytimes.com/2020/08/01/world/coronavirus-covid-19.html?action=click\&pgtype=Article\&state=default\&region=MAIN_CONTENT_1\&context=storylines_live_updates}{See
more updates}

More live coverage:
\href{https://www.nytimes.com/live/2020/07/31/business/stock-market-today-coronavirus?action=click\&pgtype=Article\&state=default\&region=MAIN_CONTENT_1\&context=storylines_live_updates}{Markets}

It stunned him to discover that the many and various topics that
interested him --- global warming, electoral politics, the health care
system --- had been subsumed by a single topic of conversation, the
coronavirus. That feeling of confusion deepened when, during his first
week back, American cities erupted in protests over the death of George
Floyd.

``While I was on retreat, there was a collective traumatic emotional
experience that I was not a part of,'' he said, on the second day. ``To
what degree do I have to piece it back together?''

Mr. Thorson is not the kind of Buddhist to shy from current events.

After graduating from college, he was an organizer for Occupy Wall
Street, camping in Zuccotti Park in Lower Manhattan and engaging with
pedestrians. He logged a few years with the Buddhist Geeks movement,
promoting the use of online technology for enlightenment seeking. His
podcast, ``Emerge,''
\href{https://podcasts.apple.com/us/podcast/emerge-making-sense-of-whats-next/id1057220344}{seeks
to explore}``the next phase of the human experiment.''

So he was eager, after ending his 75 days of silence, to see what was
going on in the world.

``I was thinking, is it going to be `Mad Max' out there, like are we the
last survivors?'' he said. ``How is humanity doing?''

After leaving the meditation center, the first evidence he saw was a gas
station, and people coming in and out wearing shorts, a scene so
characteristic of northern Vermont that he was deeply reassured.

``It's Vermont,'' he said. ``Somebody's getting gas.''

But a new set of impressions followed. He ventured into a Shaw's
supermarket eager for human contact, and what he found instead was
anxiety. When he passed people, their eyes darted around, as if they
were scanning for threats. One thing that seemed to scare them was Mr.
Thorson, who had not gotten the hang of social distancing.

``I would turn a corner in the grocery store, and someone would be
there, and they would recoil,'' he said. ``I haven't installed the Covid
operating system. At first, I was, like, `Whoa, what did I do?'''

He had looked forward to plunging back into his online world, a setting
he had always found ``nourishing.''

But when he reviewed two and a half months of posts from people he
admires, he found, to his shock, that they were only talking about one
thing. ``Everything else is gone,'' he said. ``There's nothing about the
election! It's amazing! The Australian wildfires, what happened there?
Didn't Brexit happen?''

But there was nothing close to a consensus.

``Everybody has extremely strongly held, very different opinions about
everything: how dangerous it is, what the response should have been, how
it's going, whether or not we need to isolate, how to treat it if you
get it,'' he said. ``There is one consensus proposition that, it seems
to me, everybody holds. It's that whatever happened in the last three
months is one of the most significant events in modern history.''

Talking through the preceding months, he often felt he had stumbled into
something painful, conflicts that dated back to March or April.

\href{https://www.nytimes.com/news-event/coronavirus?action=click\&pgtype=Article\&state=default\&region=MAIN_CONTENT_3\&context=storylines_faq}{}

\hypertarget{the-coronavirus-outbreak-}{%
\subsubsection{The Coronavirus Outbreak
›}\label{the-coronavirus-outbreak-}}

\hypertarget{frequently-asked-questions}{%
\paragraph{Frequently Asked
Questions}\label{frequently-asked-questions}}

Updated July 27, 2020

\begin{itemize}
\item ~
  \hypertarget{should-i-refinance-my-mortgage}{%
  \paragraph{Should I refinance my
  mortgage?}\label{should-i-refinance-my-mortgage}}

  \begin{itemize}
  \tightlist
  \item
    \href{https://www.nytimes.com/article/coronavirus-money-unemployment.html?action=click\&pgtype=Article\&state=default\&region=MAIN_CONTENT_3\&context=storylines_faq}{It
    could be a good idea,} because mortgage rates have
    \href{https://www.nytimes.com/2020/07/16/business/mortgage-rates-below-3-percent.html?action=click\&pgtype=Article\&state=default\&region=MAIN_CONTENT_3\&context=storylines_faq}{never
    been lower.} Refinancing requests have pushed mortgage applications
    to some of the highest levels since 2008, so be prepared to get in
    line. But defaults are also up, so if you're thinking about buying a
    home, be aware that some lenders have tightened their standards.
  \end{itemize}
\item ~
  \hypertarget{what-is-school-going-to-look-like-in-september}{%
  \paragraph{What is school going to look like in
  September?}\label{what-is-school-going-to-look-like-in-september}}

  \begin{itemize}
  \tightlist
  \item
    It is unlikely that many schools will return to a normal schedule
    this fall, requiring the grind of
    \href{https://www.nytimes.com/2020/06/05/us/coronavirus-education-lost-learning.html?action=click\&pgtype=Article\&state=default\&region=MAIN_CONTENT_3\&context=storylines_faq}{online
    learning},
    \href{https://www.nytimes.com/2020/05/29/us/coronavirus-child-care-centers.html?action=click\&pgtype=Article\&state=default\&region=MAIN_CONTENT_3\&context=storylines_faq}{makeshift
    child care} and
    \href{https://www.nytimes.com/2020/06/03/business/economy/coronavirus-working-women.html?action=click\&pgtype=Article\&state=default\&region=MAIN_CONTENT_3\&context=storylines_faq}{stunted
    workdays} to continue. California's two largest public school
    districts --- Los Angeles and San Diego --- said on July 13, that
    \href{https://www.nytimes.com/2020/07/13/us/lausd-san-diego-school-reopening.html?action=click\&pgtype=Article\&state=default\&region=MAIN_CONTENT_3\&context=storylines_faq}{instruction
    will be remote-only in the fall}, citing concerns that surging
    coronavirus infections in their areas pose too dire a risk for
    students and teachers. Together, the two districts enroll some
    825,000 students. They are the largest in the country so far to
    abandon plans for even a partial physical return to classrooms when
    they reopen in August. For other districts, the solution won't be an
    all-or-nothing approach.
    \href{https://bioethics.jhu.edu/research-and-outreach/projects/eschool-initiative/school-policy-tracker/}{Many
    systems}, including the nation's largest, New York City, are
    devising
    \href{https://www.nytimes.com/2020/06/26/us/coronavirus-schools-reopen-fall.html?action=click\&pgtype=Article\&state=default\&region=MAIN_CONTENT_3\&context=storylines_faq}{hybrid
    plans} that involve spending some days in classrooms and other days
    online. There's no national policy on this yet, so check with your
    municipal school system regularly to see what is happening in your
    community.
  \end{itemize}
\item ~
  \hypertarget{is-the-coronavirus-airborne}{%
  \paragraph{Is the coronavirus
  airborne?}\label{is-the-coronavirus-airborne}}

  \begin{itemize}
  \tightlist
  \item
    The coronavirus
    \href{https://www.nytimes.com/2020/07/04/health/239-experts-with-one-big-claim-the-coronavirus-is-airborne.html?action=click\&pgtype=Article\&state=default\&region=MAIN_CONTENT_3\&context=storylines_faq}{can
    stay aloft for hours in tiny droplets in stagnant air}, infecting
    people as they inhale, mounting scientific evidence suggests. This
    risk is highest in crowded indoor spaces with poor ventilation, and
    may help explain super-spreading events reported in meatpacking
    plants, churches and restaurants.
    \href{https://www.nytimes.com/2020/07/06/health/coronavirus-airborne-aerosols.html?action=click\&pgtype=Article\&state=default\&region=MAIN_CONTENT_3\&context=storylines_faq}{It's
    unclear how often the virus is spread} via these tiny droplets, or
    aerosols, compared with larger droplets that are expelled when a
    sick person coughs or sneezes, or transmitted through contact with
    contaminated surfaces, said Linsey Marr, an aerosol expert at
    Virginia Tech. Aerosols are released even when a person without
    symptoms exhales, talks or sings, according to Dr. Marr and more
    than 200 other experts, who
    \href{https://academic.oup.com/cid/article/doi/10.1093/cid/ciaa939/5867798}{have
    outlined the evidence in an open letter to the World Health
    Organization}.
  \end{itemize}
\item ~
  \hypertarget{what-are-the-symptoms-of-coronavirus}{%
  \paragraph{What are the symptoms of
  coronavirus?}\label{what-are-the-symptoms-of-coronavirus}}

  \begin{itemize}
  \tightlist
  \item
    Common symptoms
    \href{https://www.nytimes.com/article/symptoms-coronavirus.html?action=click\&pgtype=Article\&state=default\&region=MAIN_CONTENT_3\&context=storylines_faq}{include
    fever, a dry cough, fatigue and difficulty breathing or shortness of
    breath.} Some of these symptoms overlap with those of the flu,
    making detection difficult, but runny noses and stuffy sinuses are
    less common.
    \href{https://www.nytimes.com/2020/04/27/health/coronavirus-symptoms-cdc.html?action=click\&pgtype=Article\&state=default\&region=MAIN_CONTENT_3\&context=storylines_faq}{The
    C.D.C. has also} added chills, muscle pain, sore throat, headache
    and a new loss of the sense of taste or smell as symptoms to look
    out for. Most people fall ill five to seven days after exposure, but
    symptoms may appear in as few as two days or as many as 14 days.
  \end{itemize}
\item ~
  \hypertarget{does-asymptomatic-transmission-of-covid-19-happen}{%
  \paragraph{Does asymptomatic transmission of Covid-19
  happen?}\label{does-asymptomatic-transmission-of-covid-19-happen}}

  \begin{itemize}
  \tightlist
  \item
    So far, the evidence seems to show it does. A widely cited
    \href{https://www.nature.com/articles/s41591-020-0869-5}{paper}
    published in April suggests that people are most infectious about
    two days before the onset of coronavirus symptoms and estimated that
    44 percent of new infections were a result of transmission from
    people who were not yet showing symptoms. Recently, a top expert at
    the World Health Organization stated that transmission of the
    coronavirus by people who did not have symptoms was ``very rare,''
    \href{https://www.nytimes.com/2020/06/09/world/coronavirus-updates.html?action=click\&pgtype=Article\&state=default\&region=MAIN_CONTENT_3\&context=storylines_faq\#link-1f302e21}{but
    she later walked back that statement.}
  \end{itemize}
\end{itemize}

``People are so desperate to make sense of it,'' he said.

And it was true, he had missed a lot of friction, even in the
ideological bubble of a Buddhist monastic community in Vermont. In
mid-March, Soryu Forall, the group's head teacher, had just begun a
weeklong silent retreat with a larger group of students. They had just
ended communication with their families and the internet when state
governments began banning large gatherings and advising people to stay
home.

He began to get emails and phone calls from his students' families,
insisting that he end the silent retreat. ``Everyone wanted their
children to come home immediately,'' Mr. Forall said.

But he refused, saying they should be allowed to finish their week of
silence.

``It was painful for the parents, painful for me,'' he said. ``It was a
very strenuous time.''

He said he valued Mr. Thorson's perspective precisely because he had not
lived through it.

``His clarity is just what the world needs now,'' he said. ``He's been
hit by all of it in one wave.''

And it was true: In his first days out, Mr. Thorson found himself in
demand, the subject of intense curiosity.

``I feel like an oddity, I feel like a curiosity,'' he said. ``I don't
know what they expect me to say.''

Part of him wonders whether he needs to catch up on the clamor and
dispute of the last months at all. And so he has taken a few small steps
back, particularly from the internet. He has begun to regard his phone
use, he said on \href{https://www.thestoa.ca/}{``The Stoa,'' a
philosophy podcast}, with fear.

``This whole thing is a hell of a drug,'' he said. ``It really, really,
really has an impact on my nervous system.''

On Day 3 after he returned to the modern world, Mr. Thorson restored
color to the screen of his mobile phone, which had been locked in gray
scale throughout his retreat. But he found that the colors now hurt his
eyes. ``The red on the phone is nothing like the red of a flower,'' he
said. ``It was a kind of super-stimulating thing.''

And so, on Day 4, he set it back to gray scale, and that is where it has
remained.

Advertisement

\protect\hyperlink{after-bottom}{Continue reading the main story}

\hypertarget{site-index}{%
\subsection{Site Index}\label{site-index}}

\hypertarget{site-information-navigation}{%
\subsection{Site Information
Navigation}\label{site-information-navigation}}

\begin{itemize}
\tightlist
\item
  \href{https://help.nytimes.com/hc/en-us/articles/115014792127-Copyright-notice}{©~2020~The
  New York Times Company}
\end{itemize}

\begin{itemize}
\tightlist
\item
  \href{https://www.nytco.com/}{NYTCo}
\item
  \href{https://help.nytimes.com/hc/en-us/articles/115015385887-Contact-Us}{Contact
  Us}
\item
  \href{https://www.nytco.com/careers/}{Work with us}
\item
  \href{https://nytmediakit.com/}{Advertise}
\item
  \href{http://www.tbrandstudio.com/}{T Brand Studio}
\item
  \href{https://www.nytimes.com/privacy/cookie-policy\#how-do-i-manage-trackers}{Your
  Ad Choices}
\item
  \href{https://www.nytimes.com/privacy}{Privacy}
\item
  \href{https://help.nytimes.com/hc/en-us/articles/115014893428-Terms-of-service}{Terms
  of Service}
\item
  \href{https://help.nytimes.com/hc/en-us/articles/115014893968-Terms-of-sale}{Terms
  of Sale}
\item
  \href{https://spiderbites.nytimes.com}{Site Map}
\item
  \href{https://help.nytimes.com/hc/en-us}{Help}
\item
  \href{https://www.nytimes.com/subscription?campaignId=37WXW}{Subscriptions}
\end{itemize}
