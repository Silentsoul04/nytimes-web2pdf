Sections

SEARCH

\protect\hyperlink{site-content}{Skip to
content}\protect\hyperlink{site-index}{Skip to site index}

\href{https://www.nytimes.com/section/politics}{Politics}

\href{https://myaccount.nytimes.com/auth/login?response_type=cookie\&client_id=vi}{}

\href{https://www.nytimes.com/section/todayspaper}{Today's Paper}

\href{/section/politics}{Politics}\textbar{}What Democracy Scholars
Thought of Trump's Bible Photo Op

\url{https://nyti.ms/2U38KbC}

\begin{itemize}
\item
\item
\item
\item
\item
\end{itemize}

\href{https://www.nytimes.com/news-event/george-floyd-protests-minneapolis-new-york-los-angeles?action=click\&pgtype=Article\&state=default\&region=TOP_BANNER\&context=storylines_menu}{Race
and America}

\begin{itemize}
\tightlist
\item
  \href{https://www.nytimes.com/2020/07/26/us/protests-portland-seattle-trump.html?action=click\&pgtype=Article\&state=default\&region=TOP_BANNER\&context=storylines_menu}{Protesters
  Return to Other Cities}
\item
  \href{https://www.nytimes.com/2020/07/24/us/portland-oregon-protests-white-race.html?action=click\&pgtype=Article\&state=default\&region=TOP_BANNER\&context=storylines_menu}{Portland
  at the Center}
\item
  \href{https://www.nytimes.com/2020/07/23/podcasts/the-daily/portland-protests.html?action=click\&pgtype=Article\&state=default\&region=TOP_BANNER\&context=storylines_menu}{Podcast:
  Showdown in Portland}
\item
  \href{https://www.nytimes.com/interactive/2020/07/16/us/black-lives-matter-protests-louisville-breonna-taylor.html?action=click\&pgtype=Article\&state=default\&region=TOP_BANNER\&context=storylines_menu}{45
  Days in Louisville}
\end{itemize}

Advertisement

\protect\hyperlink{after-top}{Continue reading the main story}

Supported by

\protect\hyperlink{after-sponsor}{Continue reading the main story}

Political Memo

\hypertarget{what-democracy-scholars-thought-of-trumps-bible-photo-op}{%
\section{What Democracy Scholars Thought of Trump's Bible Photo
Op}\label{what-democracy-scholars-thought-of-trumps-bible-photo-op}}

The president's true believers saw a message to appreciate. Many others
saw something more alarming.

\includegraphics{https://static01.nyt.com/images/2020/06/02/us/politics/02trump-message1/merlin_173085330_9e87b8cc-bf17-451b-bfeb-6c00f96fcc1a-articleLarge.jpg?quality=75\&auto=webp\&disable=upscale}

\href{https://www.nytimes.com/by/matt-flegenheimer}{\includegraphics{https://static01.nyt.com/images/2018/10/02/multimedia/author-matt-flegenheimer/author-matt-flegenheimer-thumbLarge.png}}

By \href{https://www.nytimes.com/by/matt-flegenheimer}{Matt
Flegenheimer}

\begin{itemize}
\item
  Published June 2, 2020Updated June 4, 2020
\item
  \begin{itemize}
  \item
  \item
  \item
  \item
  \item
  \end{itemize}
\end{itemize}

\href{https://www.nytimes.com/es/2020/06/04/espanol/mundo/trump-biblia.html}{Leer
en español}

If another leader of another nation stood in another simmering capital
and instructed police and law enforcement to ``dominate the streets''
against protesters, then walked through a park where government officers
had forcibly cleared demonstrators from his path, then arrived outside a
church to hold a Bible aloft like a championship trophy for the cameras
--- well, what would America think of that?

``If we were seeing this in another country,'' said Kori Schake, a
former Pentagon official and Republican policy adviser, ``we would be
deeply concerned and talking about the foreign policy consequences of
states behaving this way.''

It is time, some opponents and academics agree, to have the
conversation.

From the earliest days of this norm-smashing administration, fretful
critics, scholars and foreign policy experts have kept watch for signals
of President Trump's anti-democratic streak. This has not always
required an exhaustive search.

But the White House response to the gushing national traumas of this
moment appears to have registered on another plane, producing the kinds
of scenes and sound bites that some doomsayers had long prophesied and
adding to the mounting social and public health crises a festering
concern about the state of American democracy itself.

Mr. Trump's defense secretary, Mark T. Esper,
\href{https://www.nytimes.com/2020/06/01/us/politics/police-military-gear.html}{told
governors} to ``dominate the battle space'' against protesters. A Black
Hawk helicopter flew low enough above the city's Chinatown district to
snap tree limbs and tear signs from the sides of buildings, a
\href{https://www.nytimes.com/2020/06/02/us/politics/military-national-guard-trump-protests.html}{show-of-force
maneuver} often seen in combat zones to scare off insurgents.

And presiding over it all was the man who had threatened to send the
American military to states where governors could not restore calm,
labeling demonstrators who have used violence to draw attention to
police brutality against black people as ``organizers'' of terror.

If the episode has generally been processed, thus far, along typical
ideological lines, the reactions have also been laced with more urgent
passions to match the times.

Many of Mr. Trump's admirers have encouraged his vows to curb chaos,
cheering the religious imagery he reached for, quite literally, in
service of a photo opportunity.

``Every believer I talked to certainly appreciates what the president
did and the message he was sending,'' said Robert Jeffress, the pastor
of First Baptist Dallas and a stalwart evangelical Trump supporter. ``I
think it will be one of those historic moments in his presidency,
especially when set against the backdrop of nights of violence
throughout our country.''

\includegraphics{https://static01.nyt.com/images/2020/06/02/us/politics/02trump-message2/merlin_173090928_e945f848-0738-4d73-90e5-286991ee47c9-articleLarge.jpg?quality=75\&auto=webp\&disable=upscale}

All the while, some Democrats are deploying a term that they have turned
to occasionally in these three and a half years, but perhaps never with
such frequency and conviction.

``The words of a dictator,'' Senator Kamala Harris of California said.

``He behaves like a dictator,'' Senator Ed Markey of Massachusetts
tweeted.

``For us to just shut our eyes and somehow believe he won't go that far
--- he just ordered the federal government to fire at innocent
protesters,'' Representative Ruben Gallego of Arizona said in an
interview. ``We need to accept the fact that this president, if given
the opportunity, will try to be a dictator.''

Mr. Gallego, a veteran of the Iraq War, predicted that military leaders
would find themselves at a decision point soon: ``They're going to have
to say no to the president and not follow illegal orders.''

Adm. Mike Mullen, the former chairman of the Joint Chiefs of Staff,
seemed to echo this anxiety on Tuesday in
\href{https://www.theatlantic.com/ideas/archive/2020/06/american-cities-are-not-battlespaces/612553/}{an
article for The Atlantic}. While he was confident that uniformed
officers would obey lawful orders, he wrote, he had less faith ``in the
soundness of the orders they will be given by this commander in chief.''

Experts on democratic systems have been careful to distinguish certain
conspicuous traits and data points --- Mr. Trump's boundary-pushing
instincts, his inveterate bluster, his fondness for some phrases
associated with strongmen --- from the most legitimate challenges to the
country's institutions and ideals.

They note that recent events are broadly consistent with the spirit of
Mr. Trump's tenure to date, much of which they have found troubling:
Here is a president who had already fired an F.B.I. director leading an
investigation into his campaign; who urged a foreign power to
investigate a political rival; who purged inspectors general tasked with
oversight of his administration; who led a public crusade for his own
Justice Department to drop charges against his first national security
adviser, who had already pleaded guilty.

Yascha Mounk, an associate professor at Johns Hopkins University who has
written extensively about threats to liberal democracy, said that Mr.
Trump was best understood as ``an authoritarian populist.'' In Mr.
Trump's conception of authority, Mr. Mounk said, ``what that means is
that he and he alone truly represents the people. And anybody who
disagrees with them, anybody who criticizes him, by virtue of that fact
is an enemy of the people.''

Projecting military might as personal political power was of a piece,
Mr. Mounk suggested.

``I don't believe Donald Trump, when he took his oath of office,
thought, `I want to be a dictator.' I don't think that today that he
wants to be a dictator,'' he said. ``But I don't think it's outlandish
to worry that should he be re-elected, the democratic system in the
United States would be in serious danger.''

Mr. Trump's invocation of religion in the context of law enforcement
muscle struck several scholars as especially notable.

Katherine Stewart, an author who has focused often on the Christian
right, said that the church visit on Monday called to mind political
leaders like Prime Minister Viktor Orban of Hungary and President Recep
Tayyip Erdogan of Turkey.

``Trump doesn't quote anything from the Bible. He really just uses it as
a pure symbol of partisan identity,'' she said, adding:
``Authoritarianism frequently comes veiled in religion.''

Ms. Schake, the director of foreign and defense policy studies at the
American Enterprise Institute, sounded a touch more hopeful. Warnings
about authoritarian backslide were not quite alarmist, she said, ``but I
don't share that concern just yet.''

``I remain optimistic,'' she said, ``that the Congress, including
Republicans in Congress, will see that we have given the chief executive
of this country too wide a latitude.''

There is little indication of that to date --- and little political
incentive, it seems, for party leaders to condemn a figure who remains
widely popular with their base (and whose rampaging conduct has been
well-known since before his election).

Most Republican lawmakers have declined to air any misgivings about Mr.
Trump this week, though a handful have publicly taken issue with his
comportment.

Senator Ben Sasse of Nebraska on Tuesday declared himself ``against
clearing out a peaceful protest for a photo op that treats the Word of
God as a political prop.'' Gov. Charlie Baker of Massachusetts, often a
willing Trump critic,
\href{https://www.cnn.com/2020/06/01/politics/charlie-baker-donald-trump-governor-call/index.html}{has
lamented} the president's ``incendiary words.'' And Senator Tim Scott of
South Carolina, the capital's most prominent black Republican, spoke
disapprovingly of the decision to violently clear protesters from the
area for a presidential photograph.

So far, Mr. Trump appears plainly unbowed. He spent much of Tuesday
morning tweeting about the disorder in New York, instructing local
leaders to ``CALL UP THE NATIONAL GUARD,'' and insisting that a ``SILENT
MAJORITY'' remained on his side.

And he framed the actions in Washington on Monday evening as a success
worth emulating.

``D.C. had no problems last night,'' the president wrote. ``Many
arrests. Great job done by all. Overwhelming force. Domination.''

Advertisement

\protect\hyperlink{after-bottom}{Continue reading the main story}

\hypertarget{site-index}{%
\subsection{Site Index}\label{site-index}}

\hypertarget{site-information-navigation}{%
\subsection{Site Information
Navigation}\label{site-information-navigation}}

\begin{itemize}
\tightlist
\item
  \href{https://help.nytimes.com/hc/en-us/articles/115014792127-Copyright-notice}{©~2020~The
  New York Times Company}
\end{itemize}

\begin{itemize}
\tightlist
\item
  \href{https://www.nytco.com/}{NYTCo}
\item
  \href{https://help.nytimes.com/hc/en-us/articles/115015385887-Contact-Us}{Contact
  Us}
\item
  \href{https://www.nytco.com/careers/}{Work with us}
\item
  \href{https://nytmediakit.com/}{Advertise}
\item
  \href{http://www.tbrandstudio.com/}{T Brand Studio}
\item
  \href{https://www.nytimes.com/privacy/cookie-policy\#how-do-i-manage-trackers}{Your
  Ad Choices}
\item
  \href{https://www.nytimes.com/privacy}{Privacy}
\item
  \href{https://help.nytimes.com/hc/en-us/articles/115014893428-Terms-of-service}{Terms
  of Service}
\item
  \href{https://help.nytimes.com/hc/en-us/articles/115014893968-Terms-of-sale}{Terms
  of Sale}
\item
  \href{https://spiderbites.nytimes.com}{Site Map}
\item
  \href{https://help.nytimes.com/hc/en-us}{Help}
\item
  \href{https://www.nytimes.com/subscription?campaignId=37WXW}{Subscriptions}
\end{itemize}
