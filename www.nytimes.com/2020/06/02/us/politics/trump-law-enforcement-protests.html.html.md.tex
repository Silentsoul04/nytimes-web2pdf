Sections

SEARCH

\protect\hyperlink{site-content}{Skip to
content}\protect\hyperlink{site-index}{Skip to site index}

\href{https://www.nytimes.com/section/politics}{Politics}

\href{https://myaccount.nytimes.com/auth/login?response_type=cookie\&client_id=vi}{}

\href{https://www.nytimes.com/section/todayspaper}{Today's Paper}

\href{/section/politics}{Politics}\textbar{}Trump Deploys the Full Might
of Federal Law Enforcement to Crush Protests

\url{https://nyti.ms/3eKBOfP}

\begin{itemize}
\item
\item
\item
\item
\item
\item
\end{itemize}

\href{https://www.nytimes.com/news-event/george-floyd-protests-minneapolis-new-york-los-angeles?action=click\&pgtype=Article\&state=default\&region=TOP_BANNER\&context=storylines_menu}{Race
and America}

\begin{itemize}
\tightlist
\item
  \href{https://www.nytimes.com/2020/07/26/us/protests-portland-seattle-trump.html?action=click\&pgtype=Article\&state=default\&region=TOP_BANNER\&context=storylines_menu}{Protesters
  Return to Other Cities}
\item
  \href{https://www.nytimes.com/2020/07/24/us/portland-oregon-protests-white-race.html?action=click\&pgtype=Article\&state=default\&region=TOP_BANNER\&context=storylines_menu}{Portland
  at the Center}
\item
  \href{https://www.nytimes.com/2020/07/23/podcasts/the-daily/portland-protests.html?action=click\&pgtype=Article\&state=default\&region=TOP_BANNER\&context=storylines_menu}{Podcast:
  Showdown in Portland}
\item
  \href{https://www.nytimes.com/interactive/2020/07/16/us/black-lives-matter-protests-louisville-breonna-taylor.html?action=click\&pgtype=Article\&state=default\&region=TOP_BANNER\&context=storylines_menu}{45
  Days in Louisville}
\end{itemize}

Advertisement

\protect\hyperlink{after-top}{Continue reading the main story}

Supported by

\protect\hyperlink{after-sponsor}{Continue reading the main story}

\hypertarget{trump-deploys-the-full-might-of-federal-law-enforcement-to-crush-protests}{%
\section{Trump Deploys the Full Might of Federal Law Enforcement to
Crush
Protests}\label{trump-deploys-the-full-might-of-federal-law-enforcement-to-crush-protests}}

Nearly a dozen federal agencies --- even the Transportation Security
Administration --- were sent to Washington and other cities after the
president vowed to ``dominate'' protesters.

\includegraphics{https://static01.nyt.com/images/2020/06/02/us/politics/02dc-unrest-feds-1/merlin_173091027_dda9523a-ad2c-49be-a8cc-fa488f9f37d3-articleLarge.jpg?quality=75\&auto=webp\&disable=upscale}

\href{https://www.nytimes.com/by/zolan-kanno-youngs}{\includegraphics{https://static01.nyt.com/images/2019/12/13/reader-center/author-zolan-kanno-youngs/author-zolan-kanno-youngs-thumbLarge.png}}\href{https://www.nytimes.com/by/katie-benner}{\includegraphics{https://static01.nyt.com/images/2018/02/16/multimedia/author-katie-benner/author-katie-benner-thumbLarge-v2.png}}

By \href{https://www.nytimes.com/by/zolan-kanno-youngs}{Zolan
Kanno-Youngs} and \href{https://www.nytimes.com/by/katie-benner}{Katie
Benner}

\begin{itemize}
\item
  Published June 2, 2020Updated June 12, 2020
\item
  \begin{itemize}
  \item
  \item
  \item
  \item
  \item
  \item
  \end{itemize}
\end{itemize}

WASHINGTON ---
\href{https://www.nytimes.com/2020/06/12/us/politics/trump-police-chokeholds.html}{President
Trump's} vow to ``dominate'' demonstrators protesting
\href{https://www.nytimes.com/2020/06/12/us/politics/trump-police-chokeholds.html}{police}
brutality has mobilized the full might of federal law enforcement, from
border agencies and the Drug Enforcement Administration to F.B.I.
hostage rescue teams, working alongside local law enforcement, the
military police and the National Guard.

The extraordinary deployments have reached the streets of San Diego,
Buffalo and Las Vegas.

But nowhere is the show of force as strong as in Washington, where Mr.
Trump is seeking to demonstrate his might by flooding the city's
downtown with agents from the F.B.I., the Bureau of Prisons, the U.S.
Marshals, the Bureau of Alcohol, Tobacco, Firearms and Explosives,
Homeland Security Investigations, Customs and Border Protection and the
Defense Department, turning the nation's capital into a heavily armed
federal fortress. Even Transportation Security Administration officers
have been called out of the airports to help protect federal property in
the ``national capital region.''

``D.H.S. and its partners will not allow anarchists, disrupters and
opportunists to exploit the ongoing civil unrest to loot and destroy our
communities,'' said Chad Wolf, the acting secretary of the Department of
Homeland Security. ``While the department respects every American's
right to protest peacefully, violence and civil unrest will not be
tolerated. We will control the situation and protect the American people
and the homeland at any cost.''

In all, nearly a dozen federal agencies and components have joined in
Mr. Trump's effort to quell protests incited by the killing of George
Floyd in Minneapolis and ostensibly to put an end to rioting and looting
--- and determine whether anarchists and other extremist groups had
infiltrated the protests.

But local officials say the federal response has gone beyond acceptable,
verging on overkill. The mayor of the District of Columbia called it
``shameful.'' A Virginia county pulled its officers out of Washington
rather than deploy alongside federal agents. The governor of Illinois
said the federal presence had actually undercut efforts to restore law
and order, while Texas said it needed no help from the U.S. military.

Regardless, Attorney General William P. Barr promised to deploy ``even
greater law enforcement resources'' in Washington on Tuesday evening.

``I can't remember the last time this number of federal agencies were
brought together to try and deal with a large number of demonstrators,''
said Chuck Wexler, the executive director of the Police Executive
Research Forum, a police research and policy organization. He said the
rush of multiple federal forces into the city could be a recipe for
``chaos.''

The tension is more evidence of a schism that has opened between federal
and local governments --- first over the pandemic and now over how to
respond to protests provoked by a spate of killings of black people.
Neither side has coordinated clearly with the other, and neither has
been willing to take responsibility for some of the ugliest episodes
between protesters and officers.

While mayors and governors express sympathy for the demonstrations, the
Department of Homeland Security and the F.B.I. have monitored the
protests for domestic terrorist activity. The department said in a
bulletin to law enforcement agencies that militia extremists and
anarchists could use the protests to cause violence and mayhem,
according to an official in possession of the document, who asked for
anonymity because he was not authorized to speak on the memo.

The Justice Department said that it would deploy all of its forces,
including hostage rescue teams and riots squads, and that it had given
agents at the Drug Enforcement Administration the power to make arrests.

\includegraphics{https://static01.nyt.com/images/2020/06/02/us/politics/02dc-unrest-feds-2/02dc-unrest-feds-2-articleLarge.jpg?quality=75\&auto=webp\&disable=upscale}

Customs and Border Protection said agents had arrest authority,
including those on elite tactical teams that are preparing to deploy to
cities and states.

Overnight Sunday, after protesters defaced the Treasury Department
building and a part of St. John's Church had caught on fire,
administration officials decided it was essential to clear Lafayette
Square and expand the amount of territory near the White House that was
controlled by officers.

``President Trump directed Attorney General Barr to lead federal law
enforcement efforts to assist in the restoration of order to the
District of Columbia,'' Kerri Kupec, a Justice Department spokeswoman,
said on Monday.

To maintain control of a protest, the local police typically employ
lines of officers to separate crowds and encourage uniformed officers to
use discretion to de-escalate encounters with tense crowds, law
enforcement experts said. But in Mr. Trump's rush to assert dominance
over the demonstrations, the local police on Monday night were joined by
federal authorities who donned ballistic gear and carried riot shields.
Military helicopters flew overhead.

At the request of the Justice Department, Customs and Border Protection
dispatched border agents and tactical officers to cities throughout the
country to assist the local police with the protests. The border
agency's air and marine operations branch, which uses aircrafts and
drones, was directed to provide surveillance of the protests, including
demonstrations in Detroit.

Mark Morgan, the acting commissioner of Customs and Border Protection,
said
\href{https://twitter.com/CBPMarkMorgan/status/1267845664156901380?s=20}{in
a tweet} on Tuesday that the aircraft specialists were used in Buffalo
to track people who hit officers with a vehicle.

Around 600 homeland security officials were deployed to the Washington
area, including agents from Immigration and Customs Enforcement who
found out about the assignment from an alert shortly before noon telling
them to prepare to aid the local police.

The agency responsible for arresting and deporting undocumented
immigrants deployed ``specialized teams'' to major cities to help
contract security officers in the Federal Protective Service, another
homeland security agency, provide security at federal buildings.

The Secret Service was also directed to bolster protective shifts of
uniformed officers at the White House.

But communication issues between the state and federal authorities
seemed to come as quickly as each deployment. Washington's mayor, Muriel
E. Bowser, learned from neighboring local leaders that National Guard
support was headed for Washington, and she resisted.

Hours after protesters in Lafayette Square were pepper-sprayed and
rammed by federal authorities armed with riot shields, the Arlington
County Board in suburban Northern Virginia ordered its police force ---
which had been helping to patrol the protests in Washington --- to
return, saying their mutual aid agreement had been ``abused.''

When multiple federal agencies are rushed out to assist local law
enforcement, ``you can have chaos,'' Mr. Wexler said. Each agency has
different use-of-force policies and different de-escalation training.

Eugene O'Donnell, a professor at the John Jay College of Criminal
Justice, agreed. ``It's like sending the firefighters in for a police
call,'' he said. ``A lot of them are investigators. They're not urban
street cops.''

On Monday, video footage captured U.S. Park Police officers in riot gear
delivering multiple blows to a news camera crew in Lafayette Square.
Federal officials were shown pushing demonstrators over to make way for
Mr. Trump's photo opportunity in front of St. John's.

For the next few hours, the protest remained largely peaceful as the
local police allowed the demonstrators to march through the streets.
Teams from the F.B.I. and the D.E.A. lined side streets. But shortly
before 10 p.m., military helicopters lowered to rooftop level in the
Chinatown area of Washington. Around the same time, some storefronts
were shattered by protesters.

The helicopters sent gusts of dust into the air, causing one tree to
split and nearly hit bystanders. Demonstrators fled a couple of blocks
where they were met by authorities who began to toss what appeared to be
smoke canisters.

Community ``police like to go out with a soft approach,'' Mr. Wexler
said.

``They wear a regular uniform. They don't put on the heavy gear, the
riot gear. They keep that in reserve,'' he continued. ``What they
recognized is if you bring out people in riot gear to begin with, you're
basically sending this message.''

That message, he said, is: ``We expect trouble or we don't trust you.''

That was not the approach of federal law enforcement officers who
surrounded dozens of protesters late Monday on a residential street and
fired chemicals at them, a moment that was photographed and shared by
residents. A stranger who heard their cries for help let them shelter in
his home until the city's curfew lifted, while law enforcement waited
outside to arrest them.

To Mr. Trump and Mr. Barr, the raucous night was a success.

Mr. Barr called Monday evening ``a more peaceful night in the District
of Columbia.''

Representative Bennie Thompson, Democrat of Mississippi and the chairman
of the House Homeland Security Committee, sent a letter to the Secret
Service director, James Murray, criticizing the agency for using what he
described as tear gas.

``I write to you stunned, disturbed and furious at the sight of federal
authorities tear-gassing peaceful protesters in Lafayette Park, outside
the White House, last night, in order to clear the way for the president
to walk over and hold a Bible in front of St. John's Episcopal Church,''
Mr. Thompson wrote. ``It is shameful.''

When asked who decided to use rubber bullets and chemicals to clear
clergy members from the patio of St. John's and peaceful protesters from
the park, all of the federal agencies contacted declined to answer.

Advertisement

\protect\hyperlink{after-bottom}{Continue reading the main story}

\hypertarget{site-index}{%
\subsection{Site Index}\label{site-index}}

\hypertarget{site-information-navigation}{%
\subsection{Site Information
Navigation}\label{site-information-navigation}}

\begin{itemize}
\tightlist
\item
  \href{https://help.nytimes.com/hc/en-us/articles/115014792127-Copyright-notice}{©~2020~The
  New York Times Company}
\end{itemize}

\begin{itemize}
\tightlist
\item
  \href{https://www.nytco.com/}{NYTCo}
\item
  \href{https://help.nytimes.com/hc/en-us/articles/115015385887-Contact-Us}{Contact
  Us}
\item
  \href{https://www.nytco.com/careers/}{Work with us}
\item
  \href{https://nytmediakit.com/}{Advertise}
\item
  \href{http://www.tbrandstudio.com/}{T Brand Studio}
\item
  \href{https://www.nytimes.com/privacy/cookie-policy\#how-do-i-manage-trackers}{Your
  Ad Choices}
\item
  \href{https://www.nytimes.com/privacy}{Privacy}
\item
  \href{https://help.nytimes.com/hc/en-us/articles/115014893428-Terms-of-service}{Terms
  of Service}
\item
  \href{https://help.nytimes.com/hc/en-us/articles/115014893968-Terms-of-sale}{Terms
  of Sale}
\item
  \href{https://spiderbites.nytimes.com}{Site Map}
\item
  \href{https://help.nytimes.com/hc/en-us}{Help}
\item
  \href{https://www.nytimes.com/subscription?campaignId=37WXW}{Subscriptions}
\end{itemize}
