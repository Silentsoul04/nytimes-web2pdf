Sections

SEARCH

\protect\hyperlink{site-content}{Skip to
content}\protect\hyperlink{site-index}{Skip to site index}

\href{https://www.nytimes.com/section/obituaries}{Obituaries}

\href{https://myaccount.nytimes.com/auth/login?response_type=cookie\&client_id=vi}{}

\href{https://www.nytimes.com/section/todayspaper}{Today's Paper}

\href{/section/obituaries}{Obituaries}\textbar{}Celine Fariala Mangaza,
Congolese Heroine of Disabled People, Dies at 52

\url{https://nyti.ms/3gQHp6e}

\begin{itemize}
\item
\item
\item
\item
\item
\end{itemize}

\href{https://www.nytimes.com/news-event/coronavirus?action=click\&pgtype=Article\&state=default\&region=TOP_BANNER\&context=storylines_menu}{The
Coronavirus Outbreak}

\begin{itemize}
\tightlist
\item
  live\href{https://www.nytimes.com/2020/08/03/world/coronavirus-covid-19.html?action=click\&pgtype=Article\&state=default\&region=TOP_BANNER\&context=storylines_menu}{Latest
  Updates}
\item
  \href{https://www.nytimes.com/interactive/2020/us/coronavirus-us-cases.html?action=click\&pgtype=Article\&state=default\&region=TOP_BANNER\&context=storylines_menu}{Maps
  and Cases}
\item
  \href{https://www.nytimes.com/interactive/2020/science/coronavirus-vaccine-tracker.html?action=click\&pgtype=Article\&state=default\&region=TOP_BANNER\&context=storylines_menu}{Vaccine
  Tracker}
\item
  \href{https://www.nytimes.com/2020/08/02/us/covid-college-reopening.html?action=click\&pgtype=Article\&state=default\&region=TOP_BANNER\&context=storylines_menu}{College
  Reopening}
\item
  \href{https://www.nytimes.com/live/2020/08/03/business/stock-market-today-coronavirus?action=click\&pgtype=Article\&state=default\&region=TOP_BANNER\&context=storylines_menu}{Economy}
\end{itemize}

Advertisement

\protect\hyperlink{after-top}{Continue reading the main story}

Supported by

\protect\hyperlink{after-sponsor}{Continue reading the main story}

Those We've Lost

\hypertarget{celine-fariala-mangaza-congolese-heroine-of-disabled-people-dies-at-52}{%
\section{Celine Fariala Mangaza, Congolese Heroine of Disabled People,
Dies at
52}\label{celine-fariala-mangaza-congolese-heroine-of-disabled-people-dies-at-52}}

A survivor of polio, she created a place for disabled women to earn a
little money, and find solidarity. Doctors believe she died because of
Covid-19.

\includegraphics{https://static01.nyt.com/images/2020/06/06/obituaries/03Mangaza/merlin_173129259_17803371-7320-41e3-a7f1-06a3dd8e91bc-articleLarge.jpg?quality=75\&auto=webp\&disable=upscale}

By Lauren Wolfe

\begin{itemize}
\item
  Published June 3, 2020Updated June 5, 2020
\item
  \begin{itemize}
  \item
  \item
  \item
  \item
  \item
  \end{itemize}
\end{itemize}

\emph{This obituary is part of a series about people who have died in
the coronavirus pandemic. Read about others}
\href{https://www.nytimes.com/interactive/2020/obituaries/people-died-coronavirus-obituaries.html}{\emph{here}}\emph{.}

A low murmur seeped from a packed room at the back of a school in the
Democratic Republic of Congo one rain-soaked January afternoon.
Hand-cranked Singer sewing machines thrummed while more than a dozen
women worked, chatting quietly, the wood or metal crutches they relied
on within easy reach. Children wove between them.

Mama Leki, a woman with a tremendous gaptoothed smile who easily broke
into raucous laughter, sat in the middle of the group, one she had
created in 2006 for women disabled mainly as a result of polio or
meningitis.

Mama Leki had been tired of the stories of women who were sexually
assaulted or beaten while home alone or forced to beg on the street,
made all the more vulnerable because of their condition. Isolation,
poverty and loneliness were part of their everyday existence until she
brought them together to earn a little money by sewing brightly colored
dolls, bags and dresses.

While the women had to make their way to the school and back on cheap
crutches through mud streets that were more pothole than road, they came
because they finally had a place to put aside their troubles for a few
hours.

``We have so many problems, we can't even name them all,'' Mama Leki
said in an interview that day in January 2016.

She died on May 28 of what doctors believed to be the novel coronavirus,
according to her close friend and colleague Neema Namadamu. Mama Leki
had gone to Bukavu General Hospital --- which has no ventilators ---
with a severe cough, even though many others with symptoms were staying
home for fear of being ostracized. Diagnostic tests for about 40
suspected cases, Mama Leki's among them, were sent off to Kinshasa for
analysis. No one knew when they might come back. She died the next day.

The activist who became widely known as Mama Leki was born Celine
Fariala Mangaza on Aug. 27, 1967, in Bukavu, in the eastern part of the
country near the Rwanda border. She contracted polio when she was 3, her
family said, and though girls rarely attended school then, let alone
girls with disabilities, she started school in 1974.

She stayed through sixth grade and went on to learn to be a tailor,
eventually opening her own training center in Bukavu for people with
disabilities. The name of her sewing group can be translated as the
Association for the Wellness of Handicapped Women.

Mama Leki was vice president of Safeco, an advocacy organization in
Bukavu, led by Ms. Namadamu, that teaches Congolese women digital
skills.

The name she acquired was a sign of respect: ``Leki'' means ``aunt'' in
the Lingala language spoken in parts of Congo.

Mama Leki married Fidel Batumike in 1994. Early on she ran into trouble
with his family --- they didn't like that she was disabled. But, she
said, ``Love doesn't have eyes.''

The marriage was happy, and the couple had four children. And
eventually, she said, she was a ``10 out of 10'' with her in-laws.

\href{https://www.nytimes.com/interactive/2020/obituaries/people-died-coronavirus-obituaries.html?action=click\&pgtype=Article\&state=default\&region=BELOW_MAIN_CONTENT\&context=covid_obits_promo}{}

\hypertarget{those-weve-lost}{%
\section{Those We've Lost}\label{those-weve-lost}}

The coronavirus pandemic has taken an incalculable death toll. This
series is designed to put names and faces to the numbers.

Read more

\includegraphics{https://static01.nyt.com/images/2020/07/30/obituaries/30Pedro/30Pedro-square640.jpg}

\hypertarget{bernaldina-josuxe9-pedro}{%
\section{Bernaldina José Pedro}\label{bernaldina-josuxe9-pedro}}

d. Boa Vista, Brazil

Leader among the Indigenous Macuxi

\includegraphics{https://static01.nyt.com/images/2020/07/31/obituaries/31Swing/merlin_175167783_8913bc90-0d64-43f3-a655-1bb1bf1601c9-square640.jpg}

\hypertarget{john-eric-swing}{%
\section{John Eric Swing}\label{john-eric-swing}}

d. Fountain Valley, Calif.

Champion of Filipino-Americans

\includegraphics{https://static01.nyt.com/images/2020/07/27/obituaries/27Victor/merlin_175001436_38b11f8e-227a-4e2c-9821-7618af9b2524-square640.jpg}

\hypertarget{victor-victor}{%
\section{Victor Victor}\label{victor-victor}}

d. Santo Domingo, Dominican Republic

Beloved musician of the Dominican Republic

\includegraphics{https://static01.nyt.com/images/2020/07/31/obituaries/31Negron/merlin_175160169_516322ae-fd23-4969-b6b2-193ced371105-square640.jpg}

\hypertarget{dr-eddie-negruxf3n}{%
\section{Dr. Eddie Negrón}\label{dr-eddie-negruxf3n}}

d. Fort Walton Beach, Fla.

Internist on Florida's Emerald Coast

\includegraphics{https://static01.nyt.com/images/2020/07/30/obituaries/30Dobson/merlin_175115928_f6b9271c-8f05-4fe1-a38a-5ca4a58f8935-square640.jpg}

\hypertarget{dobby-dobson}{%
\section{Dobby Dobson}\label{dobby-dobson}}

d. Coral Springs, Fla.

Jamaican singer and songwriter

\includegraphics{https://static01.nyt.com/images/2020/08/01/obituaries/28Gonzalez/merlin_175002771_beb57888-3951-409a-ae13-03a94b2e962e-square640.jpg}

\hypertarget{waldemar-gonzalez}{%
\section{Waldemar Gonzalez}\label{waldemar-gonzalez}}

d. White Plains, N.Y.

Teacher and social worker

Advertisement

\protect\hyperlink{after-bottom}{Continue reading the main story}

\hypertarget{site-index}{%
\subsection{Site Index}\label{site-index}}

\hypertarget{site-information-navigation}{%
\subsection{Site Information
Navigation}\label{site-information-navigation}}

\begin{itemize}
\tightlist
\item
  \href{https://help.nytimes.com/hc/en-us/articles/115014792127-Copyright-notice}{©~2020~The
  New York Times Company}
\end{itemize}

\begin{itemize}
\tightlist
\item
  \href{https://www.nytco.com/}{NYTCo}
\item
  \href{https://help.nytimes.com/hc/en-us/articles/115015385887-Contact-Us}{Contact
  Us}
\item
  \href{https://www.nytco.com/careers/}{Work with us}
\item
  \href{https://nytmediakit.com/}{Advertise}
\item
  \href{http://www.tbrandstudio.com/}{T Brand Studio}
\item
  \href{https://www.nytimes.com/privacy/cookie-policy\#how-do-i-manage-trackers}{Your
  Ad Choices}
\item
  \href{https://www.nytimes.com/privacy}{Privacy}
\item
  \href{https://help.nytimes.com/hc/en-us/articles/115014893428-Terms-of-service}{Terms
  of Service}
\item
  \href{https://help.nytimes.com/hc/en-us/articles/115014893968-Terms-of-sale}{Terms
  of Sale}
\item
  \href{https://spiderbites.nytimes.com}{Site Map}
\item
  \href{https://help.nytimes.com/hc/en-us}{Help}
\item
  \href{https://www.nytimes.com/subscription?campaignId=37WXW}{Subscriptions}
\end{itemize}
