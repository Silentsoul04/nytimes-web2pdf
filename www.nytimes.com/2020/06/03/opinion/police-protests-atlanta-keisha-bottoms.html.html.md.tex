Sections

SEARCH

\protect\hyperlink{site-content}{Skip to
content}\protect\hyperlink{site-index}{Skip to site index}

\href{/section/opinion}{Opinion}\textbar{}The Police Report to Me, but I
Knew I Couldn't Protect My Son

\url{https://nyti.ms/2U2nQOj}

\begin{itemize}
\item
\item
\item
\item
\item
\item
\end{itemize}

\includegraphics{https://static01.nyt.com/images/2020/06/02/opinion/03Bottoms/03Bottoms-articleLarge.jpg?quality=75\&auto=webp\&disable=upscale}

\href{/section/opinion}{Opinion}

\hypertarget{the-police-report-to-me-but-i-knew-i-couldnt-protect-my-son}{%
\section{The Police Report to Me, but I Knew I Couldn't Protect My
Son}\label{the-police-report-to-me-but-i-knew-i-couldnt-protect-my-son}}

Even as mayor, I know what every other parent to a black boy in America
knows: My son is simply who he is, a young black man.

A young man in Atlanta this week protesting the killing of George Floyd
by police in Minneapolis.Credit...Ben Gray/Atlanta Journal-Constitution,
via Associated Press

Supported by

\protect\hyperlink{after-sponsor}{Continue reading the main story}

By Keisha Lance Bottoms

Ms. Bottoms is the mayor of Atlanta.

\begin{itemize}
\item
  June 3, 2020
\item
  \begin{itemize}
  \item
  \item
  \item
  \item
  \item
  \item
  \end{itemize}
\end{itemize}

ATLANTA --- I frantically screamed into the phone to my teenage son:
``Lance, WHERE ARE YOU?!''

Social media posts were swirling that protests were being planned in
Atlanta in response to the death of George Floyd, a black Minnesotan,
while a police officer knelt on his neck.

Although as mayor, the chief of police reports to me, in that moment, I
knew what every other parent to a black child in America knows: I could
not protect my son. To anyone who saw him, he was simply who he is, a
black man-child in the promised land that we all know as America.

I know that as a mayor of one of the largest cities in our country, I
should now be offering solutions. But the only comforting words I have
to offer so far are those that I know to be most true: that we are
better than this; that we as a country are better than the barbaric
actions that we are forced to keep watching play out on our screens like
a grotesque horror movie stuck on repeat. We are better than the hatred
and anger that consumes so many of us. We are better than this
deplorable disease called racism that remains so rampant.

With each passing second separating me from the peace of mind a mother
feels having secured the safety of her children, I could not waste
minutes articulating all of those things to my son. All I could say was,
``Baby, please come home --- now! It's not safe for black boys to be out
today.''

I thought of his adoption process, when my husband and I were told there
was no wait for black boys.

I wondered then if this country's fear --- and too frequent hatred ---
of black men began, even subconsciously, at their birth. The harsh
reality is that if we examine the historical conditions of living while
black in America, then we'll realize that there has never been a day
when it was truly safe for black boys to be out, to be free, to just be.

America has a long and unreconciled history of tearing black boys and
men from their homes, their families and their communities --- and of
throwing them into the unrelenting grip of death, more often than many
Americans may like to admit. From being captured and assailed on African
shores, subjected to mass incarceration and being cuffed and asphyxiated
in American streets, black men have always had an inverse relationship
with life, liberty, and the pursuit of happiness.

Reflecting on the current state of affairs, my mother said to me, ``This
doesn't feel like we've gone back to 1965; this feels like before 1965
in America.''

\includegraphics{https://static01.nyt.com/images/2020/06/03/opinion/03Bottoms1/03Bottoms1-articleLarge.jpg?quality=75\&auto=webp\&disable=upscale}

To hear her say that was heartbreaking. To think that her generation
made so many sacrifices and that despite it all today's climate hearkens
back to feelings that predate the reforms they fought so hard for is
scary and sobering. But recognizing the truth within it is also
necessary.

During the Civil Rights Movement we saw people of all races and all
walks of life coming together to say: \emph{This is not right and we are
going to stand up for the goodness of America.} That same spirit must
rise and prevail today. Such a pursuit is not partisan. It's American. I
cannot guarantee that I will pass freedom down to my children, but I can
and will continue to fight for it and teach them how to fight for it
every single day. One of the best ways that we can fight for it is by
fighting to ensure that our governing bodies are led by people who value
the freedom, equality and humanity of all of mankind. Now, more than
ever, elections matter; leadership matters. That's why November 2020
matters.

So as Atlanta's mayor, I would like to offer one salient solution to the
atrocities we are faced with today. Let us each commit to exercise our
right to vote this November. Let us vote against state-sanctioned
violence, vitriolic discourse and the violation of human rights. In
memory of George Floyd and all the other innocent black lives that have
been taken in the recent and distant past, let us commit to registering
black people, especially black men, to vote.

Think of what could be possible if each of us allied in favor of justice
spent more than nine minutes getting people registered in preparation to
make change at the federal, state and local levels this fall. That would
be the most effective response, the deepest payback, for each minute
that passed when that Minneapolis policeman pressed into Mr. Floyd's
innocent body.

Join me in getting ready for the polls. Together our generation of
Americans can declare --- without equivocation --- that freedom will not
face extinction and that progress will not be paralyzed.

Keisha Lance Bottoms is the mayor of Atlanta.

\emph{The Times is committed to publishing}
\href{https://www.nytimes.com/2019/01/31/opinion/letters/letters-to-editor-new-york-times-women.html}{\emph{a
diversity of letters}} \emph{to the editor. We'd like to hear what you
think about this or any of our articles. Here are some}
\href{https://help.nytimes.com/hc/en-us/articles/115014925288-How-to-submit-a-letter-to-the-editor}{\emph{tips}}\emph{.
And here's our email:}
\href{mailto:letters@nytimes.com}{\emph{letters@nytimes.com}}\emph{.}

\emph{Follow The New York Times Opinion section on}
\href{https://www.facebook.com/nytopinion}{\emph{Facebook}}\emph{,}
\href{http://twitter.com/NYTOpinion}{\emph{Twitter (@NYTopinion)}}
\emph{and}
\href{https://www.instagram.com/nytopinion/}{\emph{Instagram}}\emph{.}

Advertisement

\protect\hyperlink{after-bottom}{Continue reading the main story}

\hypertarget{site-index}{%
\subsection{Site Index}\label{site-index}}

\hypertarget{site-information-navigation}{%
\subsection{Site Information
Navigation}\label{site-information-navigation}}

\begin{itemize}
\tightlist
\item
  \href{https://help.nytimes.com/hc/en-us/articles/115014792127-Copyright-notice}{©~2020~The
  New York Times Company}
\end{itemize}

\begin{itemize}
\tightlist
\item
  \href{https://www.nytco.com/}{NYTCo}
\item
  \href{https://help.nytimes.com/hc/en-us/articles/115015385887-Contact-Us}{Contact
  Us}
\item
  \href{https://www.nytco.com/careers/}{Work with us}
\item
  \href{https://nytmediakit.com/}{Advertise}
\item
  \href{http://www.tbrandstudio.com/}{T Brand Studio}
\item
  \href{https://www.nytimes.com/privacy/cookie-policy\#how-do-i-manage-trackers}{Your
  Ad Choices}
\item
  \href{https://www.nytimes.com/privacy}{Privacy}
\item
  \href{https://help.nytimes.com/hc/en-us/articles/115014893428-Terms-of-service}{Terms
  of Service}
\item
  \href{https://help.nytimes.com/hc/en-us/articles/115014893968-Terms-of-sale}{Terms
  of Sale}
\item
  \href{https://spiderbites.nytimes.com}{Site Map}
\item
  \href{https://help.nytimes.com/hc/en-us}{Help}
\item
  \href{https://www.nytimes.com/subscription?campaignId=37WXW}{Subscriptions}
\end{itemize}
