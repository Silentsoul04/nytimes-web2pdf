Sections

SEARCH

\protect\hyperlink{site-content}{Skip to
content}\protect\hyperlink{site-index}{Skip to site index}

\href{https://myaccount.nytimes.com/auth/login?response_type=cookie\&client_id=vi}{}

\href{https://www.nytimes.com/section/todayspaper}{Today's Paper}

\href{/section/opinion}{Opinion}\textbar{}Trump's Magic Word

\href{https://nyti.ms/2Uau7I5}{https://nyti.ms/2Uau7I5}

\begin{itemize}
\item
\item
\item
\item
\item
\item
\end{itemize}

Advertisement

\protect\hyperlink{after-top}{Continue reading the main story}

\href{/section/opinion}{Opinion}

Supported by

\protect\hyperlink{after-sponsor}{Continue reading the main story}

\hypertarget{trumps-magic-word}{%
\section{Trump's Magic Word}\label{trumps-magic-word}}

What we have here is a failure to dominate.

\href{https://www.nytimes.com/by/gail-collins}{\includegraphics{https://static01.nyt.com/images/2018/04/03/opinion/gail-collins/gail-collins-thumbLarge.png}}

By \href{https://www.nytimes.com/by/gail-collins}{Gail Collins}

Opinion Columnist

\begin{itemize}
\item
  June 3, 2020
\item
  \begin{itemize}
  \item
  \item
  \item
  \item
  \item
  \item
  \end{itemize}
\end{itemize}

\includegraphics{https://static01.nyt.com/images/2020/06/03/opinion/03collins1/03collins1-articleLarge-v2.jpg?quality=75\&auto=webp\&disable=upscale}

Have you noticed how almost every other word out of Donald Trump's mouth
lately seems to be some variation on ``dominate?''

``If you don't dominate, you're wasting your time,'' he told America's
governors. ``They're going to run all over you. You'll look like a bunch
of jerks.''

This, of course, was in that
\href{https://www.cnn.com/2020/06/01/politics/wh-governors-call-protests/index.html}{telephone
rant about protesters}. There is something about crowds of people
willing to take to the streets to denounce racism that seems to make the
president feel, um, unmanly.

``I will not allow angry mobs to dominate,'' he told the country during
his visit to the space launch.

Minneapolis authorities, he contended, were ``weak and pathetic'' until
events spiraled out of control and the National Guard moved in.
(``Domination \ldots{} it's a beautiful thing to watch.'')

Tweeting on the same subject, Trump reported: ``Great job done by all.
Overwhelming force. Domination. Likewise, Minneapolis was great. (thank
you President Trump!)''

Which pretty much sums up his week. And, I guess, his id.

The president's most famous response to the protests was that assault on
demonstrators outside the White House, in which federal
\href{https://www.nytimes.com/2020/06/01/us/politics/trump-st-johns-church-bible.html}{troops
cleared the area} so that Trump could accomplish his important mission
of standing in front of a church and holding up a Bible. Critics felt
that in the pictures Trump looked as if he had never touched a Bible
before in his life. True cynics felt that it looked as if he had never
touched a book, period.

Anyhow, everything has been going great, by Trump's interpretation.
``And we had no problem at all last night,''
\href{https://radio.foxnews.com/2020/06/03/president-trump-on-the-brian-kilmeade-show/}{he
told Fox Radio} on Wednesday. ``We had substantial dominant force and it
--- we have to have a dominant force. Maybe it doesn't sound good to say
it but you have to have a dominant force.''

Dependency on the d-word seems to be catching.

``We need to dominate the battlespace,'' Defense Secretary Mark Esper
told the governors during that same rather lively presidential phone
call. Despite their mutual affection for the concept of domination,
Esper has been drifting away from the administration line. Lately he's
been trying to reassure the nation that the president isn't going to
take it upon himself to
\href{https://www.washingtonpost.com/history/2020/06/03/insurrection-act-trump-history/}{send
federal troops into cities uninvited}, even as Trump himself seems to
feel it's a pretty cool idea.

Esper has been serving as defense secretary for almost a year, which
makes him, in Trump terms, a long-running cabinet veteran. Given the way
the president treasures cabinet members who aren't afraid to speak their
mind, insiders expressed confidence that he might well remain in the
administration for quite a few more \ldots{} hours.

``As of right now, Secretary Esper is still Secretary Esper,'' said the
White House press secretary, Kayleigh McEnany, on Wednesday.

It's certainly been a tough time for the cabineteers. Attorney General
William P. Barr is
\href{https://www.washingtonpost.com/politics/barr-personally-ordered-removal-of-protesters-near-white-house-leading-to-use-of-force-against-largely-peaceful-crowd/2020/06/02/0ca2417c-a4d5-11ea-b473-04905b1af82b_story.html}{getting
the blame} for all the messiest aspects of that Trump trip to the
church. And Secretary of State Mike Pompeo has been
\href{https://www.nytimes.com/2020/05/17/us/politics/pompeo-inspector-general-steve-linick.html}{fighting
off investigations} into whether he gets government employees to run his
personal errands.

We are only bringing that last matter up because it provides a chance to
revisit
\href{https://thehill.com/homenews/administration/498403-trump-on-pompeo-id-rather-have-him-working-than-doing-dishes-because}{Trump's
defense of Pompeo} --- that it's better to have him use federal funds to
buy a home helper than forcing him to ``wash dishes because maybe his
wife isn't there.'' After all, if Pompeo wasn't in the kitchen he might
otherwise be ``on the phone with some world leader.''

Yeah, and one thing we do not have to worry about is Donald Trump doing
housework when Melania and all the help are out of town. What could be
more un-dominant?

The president's super-favorite word came up in
\href{https://www.whitehouse.gov/briefings-statements/statement-by-the-president-39/}{his
speech to the nation} this week, when he urged deployment of the
national guard ``in sufficient numbers that we dominate the streets.''
In passing, he also assured Americans that they had no need to worry
about ``your Second Amendment rights.'' Have you noticed how often Trump
throws the right to bear arms into these conversations? It's as if he's
worried that any time he tells the country things are OK, he has to
reassure them that won't mean less armaments.

Just recently, while he was unveiling a program to help support
agriculture during the coronavirus crisis, Trump assured visiting
Virginia farmers: ``We're going after Virginia with your crazy governor.
They want to take your Second Amendment away. You know that, right?
You'll have nobody guarding your potatoes.''

Signing a proclamation in honor of National Nurses Day, he bragged to
his guests about ``saving your Second Amendment, which is under siege,
by the way.''

And on it goes. Trump has also fit the Second Amendment into Coronavirus
Task Force news conferences, a signing ceremony for a bill on veterans'
education, the celebration of a new trade agreement with Mexico and
Canada, and of course, the rallies back in the happy days when rallies
were his way of life.

Now, some people believe that when men go overboard with weaponry issues
it may be linked to insecurity about their sexuality. Certainly isn't
always true, but here you've got a guy who talks compulsively both about
the Second Amendment and his need to dominate.

This could be a great protest theme. Fill the street with banners
saying, ``Mr. President, we're not really questioning your
masculinity.'' Very positive message that'll drive him completely nuts.

\emph{The Times is committed to publishing}
\href{https://www.nytimes.com/2019/01/31/opinion/letters/letters-to-editor-new-york-times-women.html}{\emph{a
diversity of letters}} \emph{to the editor. We'd like to hear what you
think about this or any of our articles. Here are some}
\href{https://help.nytimes.com/hc/en-us/articles/115014925288-How-to-submit-a-letter-to-the-editor}{\emph{tips}}\emph{.
And here's our email:}
\href{mailto:letters@nytimes.com}{\emph{letters@nytimes.com}}\emph{.}

\emph{Follow The New York Times Opinion section on}
\href{https://www.facebook.com/nytopinion}{\emph{Facebook}}\emph{,}
\href{http://twitter.com/NYTOpinion}{\emph{Twitter (@NYTopinion)}}
\emph{and}
\href{https://www.instagram.com/nytopinion/}{\emph{Instagram}}\emph{.}

Advertisement

\protect\hyperlink{after-bottom}{Continue reading the main story}

\hypertarget{site-index}{%
\subsection{Site Index}\label{site-index}}

\hypertarget{site-information-navigation}{%
\subsection{Site Information
Navigation}\label{site-information-navigation}}

\begin{itemize}
\tightlist
\item
  \href{https://help.nytimes.com/hc/en-us/articles/115014792127-Copyright-notice}{©~2020~The
  New York Times Company}
\end{itemize}

\begin{itemize}
\tightlist
\item
  \href{https://www.nytco.com/}{NYTCo}
\item
  \href{https://help.nytimes.com/hc/en-us/articles/115015385887-Contact-Us}{Contact
  Us}
\item
  \href{https://www.nytco.com/careers/}{Work with us}
\item
  \href{https://nytmediakit.com/}{Advertise}
\item
  \href{http://www.tbrandstudio.com/}{T Brand Studio}
\item
  \href{https://www.nytimes.com/privacy/cookie-policy\#how-do-i-manage-trackers}{Your
  Ad Choices}
\item
  \href{https://www.nytimes.com/privacy}{Privacy}
\item
  \href{https://help.nytimes.com/hc/en-us/articles/115014893428-Terms-of-service}{Terms
  of Service}
\item
  \href{https://help.nytimes.com/hc/en-us/articles/115014893968-Terms-of-sale}{Terms
  of Sale}
\item
  \href{https://spiderbites.nytimes.com}{Site Map}
\item
  \href{https://help.nytimes.com/hc/en-us}{Help}
\item
  \href{https://www.nytimes.com/subscription?campaignId=37WXW}{Subscriptions}
\end{itemize}
