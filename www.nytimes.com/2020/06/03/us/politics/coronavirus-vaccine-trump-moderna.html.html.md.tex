Sections

SEARCH

\protect\hyperlink{site-content}{Skip to
content}\protect\hyperlink{site-index}{Skip to site index}

\href{https://www.nytimes.com/section/politics}{Politics}

\href{https://myaccount.nytimes.com/auth/login?response_type=cookie\&client_id=vi}{}

\href{https://www.nytimes.com/section/todayspaper}{Today's Paper}

\href{/section/politics}{Politics}\textbar{}Trump Administration Selects
Five Coronavirus Vaccine Candidates as Finalists

\url{https://nyti.ms/3gS1Saz}

\begin{itemize}
\item
\item
\item
\item
\item
\end{itemize}

\href{https://www.nytimes.com/news-event/coronavirus?action=click\&pgtype=Article\&state=default\&region=TOP_BANNER\&context=storylines_menu}{The
Coronavirus Outbreak}

\begin{itemize}
\tightlist
\item
  live\href{https://www.nytimes.com/2020/08/02/world/coronavirus-updates.html?action=click\&pgtype=Article\&state=default\&region=TOP_BANNER\&context=storylines_menu}{Latest
  Updates}
\item
  \href{https://www.nytimes.com/interactive/2020/us/coronavirus-us-cases.html?action=click\&pgtype=Article\&state=default\&region=TOP_BANNER\&context=storylines_menu}{Maps
  and Cases}
\item
  \href{https://www.nytimes.com/interactive/2020/science/coronavirus-vaccine-tracker.html?action=click\&pgtype=Article\&state=default\&region=TOP_BANNER\&context=storylines_menu}{Vaccine
  Tracker}
\item
  \href{https://www.nytimes.com/interactive/2020/07/29/us/schools-reopening-coronavirus.html?action=click\&pgtype=Article\&state=default\&region=TOP_BANNER\&context=storylines_menu}{What
  School May Look Like}
\item
  \href{https://www.nytimes.com/live/2020/07/31/business/stock-market-today-coronavirus?action=click\&pgtype=Article\&state=default\&region=TOP_BANNER\&context=storylines_menu}{Economy}
\end{itemize}

Advertisement

\protect\hyperlink{after-top}{Continue reading the main story}

Supported by

\protect\hyperlink{after-sponsor}{Continue reading the main story}

\hypertarget{trump-administration-selects-five-coronavirus-vaccine-candidates-as-finalists}{%
\section{Trump Administration Selects Five Coronavirus Vaccine
Candidates as
Finalists}\label{trump-administration-selects-five-coronavirus-vaccine-candidates-as-finalists}}

The White House is eager to project progress, but the public-private
partnership it has created still faces scientific hurdles, internal
tensions and questions from Congress.

\includegraphics{https://static01.nyt.com/images/2020/06/03/us/politics/03dc-virus-vaccine1/merlin_171408228_5309746a-b867-4f7e-a307-6dd809656b06-articleLarge.jpg?quality=75\&auto=webp\&disable=upscale}

\href{https://www.nytimes.com/by/noah-weiland}{\includegraphics{https://static01.nyt.com/images/2019/07/23/reader-center/author-noah-weiland/author-noah-weiland-thumbLarge.png}}\href{https://www.nytimes.com/by/david-e-sanger}{\includegraphics{https://static01.nyt.com/images/2018/10/03/multimedia/author-david-e-sanger/author-david-e-sanger-thumbLarge.png}}

By \href{https://www.nytimes.com/by/noah-weiland}{Noah Weiland} and
\href{https://www.nytimes.com/by/david-e-sanger}{David E. Sanger}

\begin{itemize}
\item
  Published June 3, 2020Updated July 27, 2020
\item
  \begin{itemize}
  \item
  \item
  \item
  \item
  \item
  \end{itemize}
\end{itemize}

WASHINGTON --- The Trump administration has selected five companies as
the most likely candidates to produce a
\href{https://www.nytimes.com/2020/07/27/health/moderna-vaccine-covid.html}{vaccine
for the coronavirus}, senior officials said, a critical step in the
White House's effort to deliver on its
\href{https://www.nytimes.com/2020/05/15/us/politics/coronavirus-vaccine-timeline.html}{promise}
of being able to start widespread inoculation of Americans by the end of
the year.

By winnowing the field in a matter of weeks from a pool of around a
dozen companies, the federal government is betting that it can identify
the most promising
\href{https://www.nytimes.com/interactive/2020/06/09/magazine/covid-vaccine.html}{vaccine}
projects at an early stage, speed along the process of determining which
will work and ensure that the winner or winners can be quickly
manufactured in huge quantities and distributed across the country.

The announcement of the decision will be made at the White House in the
next few weeks, government officials said. Dr. Anthony S. Fauci, the
federal government's top epidemiologist and director of the National
Institute of Allergy and Infectious Diseases, hinted at the coming
action on Tuesday when he told a medical seminar that ``by the beginning
of 2021 we hope to have a couple of hundred million doses.''

The five companies are
\href{https://www.nytimes.com/2020/07/27/health/moderna-vaccine-covid.html}{Moderna},
a Massachusetts-based biotechnology firm, which Dr. Fauci said he
expected would enter into the final phase of clinical trials next month;
the combination of
\href{https://www.nytimes.com/2020/04/27/world/europe/coronavirus-vaccine-update-oxford.html}{Oxford
University} and AstraZeneca, on a similar schedule; and three large
pharmaceutical companies: Johnson \& Johnson, Merck and
\href{https://www.nytimes.com/2020/07/27/health/moderna-vaccine-covid.html}{Pfizer}.
Each is taking a somewhat different approach.

President Trump has been eager to show rapid progress as the nation
slowly emerges from lockdown, and as he faces the growing challenge of
winning re-election in the midst of national upheaval:
\href{https://www.nytimes.com/interactive/2020/us/coronavirus-us-cases.html}{more
than 106,000 Americans dead} from the virus, unemployment at record
levels and now discord and violence in the streets.

Despite
\href{https://www.nytimes.com/2020/05/20/health/coronavirus-vaccines.html}{promising
early results} and the administration's strong interest in nurturing a
government-industry partnership, substantial hurdles remain, and many
scientists consider Mr. Trump's goal of having a vaccine widely
available by early next year to be optimistic, if not unrealistic.
Vaccine development is notoriously difficult and time-consuming; the
record is four years, and a decade is not unusual.

Moderna, Johnson \& Johnson and the Oxford-AstraZeneca group
\href{https://www.nytimes.com/2020/05/21/health/coronavirus-vaccine-astrazeneca.html}{have
already received a total of \$2.2 billion in federal funding} to support
their vaccine programs. Their selection as finalists, along with Merck
and Pfizer, will give all five companies access to additional government
money, help in running clinical trials and financial and logistical
support for a manufacturing base that is being built even before it is
clear which if any of the vaccines in development will work.

More funding is likely to be announced soon, officials said. This week,
the Department of Health and Human Services added \$628 million to a
contract with Emergent BioSolutions, a Maryland firm, to expand
development of vaccine manufacturing capacity.

Dr. Fauci, who had been sounding cautionary notes, now sounds more
optimistic: Among his concerns, he said during the session run by The
Journal of the American Medical Association, is how long immunity
triggered by a vaccine might last.

\includegraphics{https://static01.nyt.com/images/2020/06/03/us/politics/03dc-virus-vaccine2/merlin_172064049_f0ef3d61-d256-47e8-8e01-1681842f2405-articleLarge.jpg?quality=75\&auto=webp\&disable=upscale}

``Vaccines are coming along really well,'' Mr. Trump
\href{https://twitter.com/realDonaldTrump/status/1267826457796984832}{wrote
on Twitter} on Tuesday, hours before he was scheduled to meet with Alex
M. Azar II, the health and human services secretary. ``Moving faster
than anticipated. Good news ahead.''

\hypertarget{latest-updates-global-coronavirus-outbreak}{%
\section{\texorpdfstring{\href{https://www.nytimes.com/2020/08/01/world/coronavirus-covid-19.html?action=click\&pgtype=Article\&state=default\&region=MAIN_CONTENT_1\&context=storylines_live_updates}{Latest
Updates: Global Coronavirus
Outbreak}}{Latest Updates: Global Coronavirus Outbreak}}\label{latest-updates-global-coronavirus-outbreak}}

Updated 2020-08-02T17:52:35.962Z

\begin{itemize}
\tightlist
\item
  \href{https://www.nytimes.com/2020/08/01/world/coronavirus-covid-19.html?action=click\&pgtype=Article\&state=default\&region=MAIN_CONTENT_1\&context=storylines_live_updates\#link-34047410}{The
  U.S. reels as July cases more than double the total of any other
  month.}
\item
  \href{https://www.nytimes.com/2020/08/01/world/coronavirus-covid-19.html?action=click\&pgtype=Article\&state=default\&region=MAIN_CONTENT_1\&context=storylines_live_updates\#link-780ec966}{Top
  U.S. officials work to break an impasse over the federal jobless
  benefit.}
\item
  \href{https://www.nytimes.com/2020/08/01/world/coronavirus-covid-19.html?action=click\&pgtype=Article\&state=default\&region=MAIN_CONTENT_1\&context=storylines_live_updates\#link-2bc8948}{Its
  outbreak untamed, Melbourne goes into even greater lockdown.}
\end{itemize}

\href{https://www.nytimes.com/2020/08/01/world/coronavirus-covid-19.html?action=click\&pgtype=Article\&state=default\&region=MAIN_CONTENT_1\&context=storylines_live_updates}{See
more updates}

More live coverage:
\href{https://www.nytimes.com/live/2020/07/31/business/stock-market-today-coronavirus?action=click\&pgtype=Article\&state=default\&region=MAIN_CONTENT_1\&context=storylines_live_updates}{Markets}

The project --- called Operation Warp Speed --- amounts to a sprawling,
on-the-fly experiment in industrial policy by a Republican
administration that has been otherwise dedicated to giving private
industry a free hand.

Democrats in Congress are already seeking details about the contracts
with the companies, many of which are still wrapped in secrecy. They are
asking how much Americans will have to pay to be vaccinated and whether
the firms, or American taxpayers, will retain the profits and
intellectual property.

Other countries, including China, are also rushing their own efforts to
produce a vaccine, raising concerns that nationalism rather than need
could drive decisions about who first gets inoculated.

Two of the vaccine candidates selected by the Trump administration ---
developed by Moderna and scientists at Oxford --- are already in Phase
II trials, meaning their effectiveness is being tested on scores of
human subjects.

They will likely shift to large-scale human trials, called Phase III, as
early as July, two senior administration officials said.

While Johnson \& Johnson has said it would begin Phase I trials by
September at the latest, that now appears likely to be sped up
considerably, officials said. Phase I focuses on testing for safety, a
particularly important factor for vaccines since they are administered
widely to healthy people.

Several of the companies said that they did not want to speak ahead of
any announcement by the White House, and the others did not respond to
requests for a comment. Along with Moderna, Merck, Pfizer and Johnson \&
Johnson are based in the United States. AstraZeneca is based in Britain.

Under the administration plan, according to officials, around 30,000
people will take part in Phase III trials for each vaccine when they
reach that stage. If all five companies reach Phase III trials, around
150,000 people, mostly Americans, would ultimately become the test
subjects for a vaccine.

All age groups will be covered, including older people and those with
underlying health conditions.

It is possible, officials and corporate executives in several of the
firms said, that some of the Phase III trials will be conducted outside
the United States, and may be focused on coronavirus hot spots, where a
greater possibility of infection could speed the process of determining
the effectiveness of a potential vaccine. The other alternative
---~deliberately exposing inoculated volunteers to the disease ---~ is
fraught with ethical issues and officials seem reluctant to take that
route, even if it might speed results.

The plans are being assembled in an office suite on the seventh floor of
the Department of Health and Human Services's headquarters, where two
newly appointed leaders of the project, Dr. Moncef Slaoui and Gen.
Gustave F. Perna, have set up temporary offices.

Dr. Slaoui comes from the pharmaceutical and venture capital worlds.
General Perna heads the Army Matériel Command and is an expert in
complex logistics but not medicine.

Their work is monitored by Mr. Azar, Defense Secretary Mark T. Esper and
Jared Kushner, the president's son-in-law and senior adviser. They are
coordinating with the senior infectious disease experts on the White
House's coronavirus task force, Dr. Fauci and Dr. Deborah L. Birx, who
is overseeing the task force's day-to-day operations.

Weeks ago, Mr. Trump compared the Warp Speed effort to the Manhattan
Project, the government-led program during World War II to develop the
atomic bomb. There are superficial similarities: Lives at stake,
crushing deadlines, and a combination of civilian and military
leadership. (The Manhattan Project was headed by J. Robert Oppenheimer,
a theoretical physicist, and Gen. Leslie R. Groves, who oversaw the
project to deliver the bomb to its targets in Japan.)

But even some of the president's top aides say the analogy goes only so
far: This effort is an amalgamation of private-industry vaccine
projects, with an overlay of military coordination.

\href{https://www.nytimes.com/news-event/coronavirus?action=click\&pgtype=Article\&state=default\&region=MAIN_CONTENT_3\&context=storylines_faq}{}

\hypertarget{the-coronavirus-outbreak-}{%
\subsubsection{The Coronavirus Outbreak
›}\label{the-coronavirus-outbreak-}}

\hypertarget{frequently-asked-questions}{%
\paragraph{Frequently Asked
Questions}\label{frequently-asked-questions}}

Updated July 27, 2020

\begin{itemize}
\item ~
  \hypertarget{should-i-refinance-my-mortgage}{%
  \paragraph{Should I refinance my
  mortgage?}\label{should-i-refinance-my-mortgage}}

  \begin{itemize}
  \tightlist
  \item
    \href{https://www.nytimes.com/article/coronavirus-money-unemployment.html?action=click\&pgtype=Article\&state=default\&region=MAIN_CONTENT_3\&context=storylines_faq}{It
    could be a good idea,} because mortgage rates have
    \href{https://www.nytimes.com/2020/07/16/business/mortgage-rates-below-3-percent.html?action=click\&pgtype=Article\&state=default\&region=MAIN_CONTENT_3\&context=storylines_faq}{never
    been lower.} Refinancing requests have pushed mortgage applications
    to some of the highest levels since 2008, so be prepared to get in
    line. But defaults are also up, so if you're thinking about buying a
    home, be aware that some lenders have tightened their standards.
  \end{itemize}
\item ~
  \hypertarget{what-is-school-going-to-look-like-in-september}{%
  \paragraph{What is school going to look like in
  September?}\label{what-is-school-going-to-look-like-in-september}}

  \begin{itemize}
  \tightlist
  \item
    It is unlikely that many schools will return to a normal schedule
    this fall, requiring the grind of
    \href{https://www.nytimes.com/2020/06/05/us/coronavirus-education-lost-learning.html?action=click\&pgtype=Article\&state=default\&region=MAIN_CONTENT_3\&context=storylines_faq}{online
    learning},
    \href{https://www.nytimes.com/2020/05/29/us/coronavirus-child-care-centers.html?action=click\&pgtype=Article\&state=default\&region=MAIN_CONTENT_3\&context=storylines_faq}{makeshift
    child care} and
    \href{https://www.nytimes.com/2020/06/03/business/economy/coronavirus-working-women.html?action=click\&pgtype=Article\&state=default\&region=MAIN_CONTENT_3\&context=storylines_faq}{stunted
    workdays} to continue. California's two largest public school
    districts --- Los Angeles and San Diego --- said on July 13, that
    \href{https://www.nytimes.com/2020/07/13/us/lausd-san-diego-school-reopening.html?action=click\&pgtype=Article\&state=default\&region=MAIN_CONTENT_3\&context=storylines_faq}{instruction
    will be remote-only in the fall}, citing concerns that surging
    coronavirus infections in their areas pose too dire a risk for
    students and teachers. Together, the two districts enroll some
    825,000 students. They are the largest in the country so far to
    abandon plans for even a partial physical return to classrooms when
    they reopen in August. For other districts, the solution won't be an
    all-or-nothing approach.
    \href{https://bioethics.jhu.edu/research-and-outreach/projects/eschool-initiative/school-policy-tracker/}{Many
    systems}, including the nation's largest, New York City, are
    devising
    \href{https://www.nytimes.com/2020/06/26/us/coronavirus-schools-reopen-fall.html?action=click\&pgtype=Article\&state=default\&region=MAIN_CONTENT_3\&context=storylines_faq}{hybrid
    plans} that involve spending some days in classrooms and other days
    online. There's no national policy on this yet, so check with your
    municipal school system regularly to see what is happening in your
    community.
  \end{itemize}
\item ~
  \hypertarget{is-the-coronavirus-airborne}{%
  \paragraph{Is the coronavirus
  airborne?}\label{is-the-coronavirus-airborne}}

  \begin{itemize}
  \tightlist
  \item
    The coronavirus
    \href{https://www.nytimes.com/2020/07/04/health/239-experts-with-one-big-claim-the-coronavirus-is-airborne.html?action=click\&pgtype=Article\&state=default\&region=MAIN_CONTENT_3\&context=storylines_faq}{can
    stay aloft for hours in tiny droplets in stagnant air}, infecting
    people as they inhale, mounting scientific evidence suggests. This
    risk is highest in crowded indoor spaces with poor ventilation, and
    may help explain super-spreading events reported in meatpacking
    plants, churches and restaurants.
    \href{https://www.nytimes.com/2020/07/06/health/coronavirus-airborne-aerosols.html?action=click\&pgtype=Article\&state=default\&region=MAIN_CONTENT_3\&context=storylines_faq}{It's
    unclear how often the virus is spread} via these tiny droplets, or
    aerosols, compared with larger droplets that are expelled when a
    sick person coughs or sneezes, or transmitted through contact with
    contaminated surfaces, said Linsey Marr, an aerosol expert at
    Virginia Tech. Aerosols are released even when a person without
    symptoms exhales, talks or sings, according to Dr. Marr and more
    than 200 other experts, who
    \href{https://academic.oup.com/cid/article/doi/10.1093/cid/ciaa939/5867798}{have
    outlined the evidence in an open letter to the World Health
    Organization}.
  \end{itemize}
\item ~
  \hypertarget{what-are-the-symptoms-of-coronavirus}{%
  \paragraph{What are the symptoms of
  coronavirus?}\label{what-are-the-symptoms-of-coronavirus}}

  \begin{itemize}
  \tightlist
  \item
    Common symptoms
    \href{https://www.nytimes.com/article/symptoms-coronavirus.html?action=click\&pgtype=Article\&state=default\&region=MAIN_CONTENT_3\&context=storylines_faq}{include
    fever, a dry cough, fatigue and difficulty breathing or shortness of
    breath.} Some of these symptoms overlap with those of the flu,
    making detection difficult, but runny noses and stuffy sinuses are
    less common.
    \href{https://www.nytimes.com/2020/04/27/health/coronavirus-symptoms-cdc.html?action=click\&pgtype=Article\&state=default\&region=MAIN_CONTENT_3\&context=storylines_faq}{The
    C.D.C. has also} added chills, muscle pain, sore throat, headache
    and a new loss of the sense of taste or smell as symptoms to look
    out for. Most people fall ill five to seven days after exposure, but
    symptoms may appear in as few as two days or as many as 14 days.
  \end{itemize}
\item ~
  \hypertarget{does-asymptomatic-transmission-of-covid-19-happen}{%
  \paragraph{Does asymptomatic transmission of Covid-19
  happen?}\label{does-asymptomatic-transmission-of-covid-19-happen}}

  \begin{itemize}
  \tightlist
  \item
    So far, the evidence seems to show it does. A widely cited
    \href{https://www.nature.com/articles/s41591-020-0869-5}{paper}
    published in April suggests that people are most infectious about
    two days before the onset of coronavirus symptoms and estimated that
    44 percent of new infections were a result of transmission from
    people who were not yet showing symptoms. Recently, a top expert at
    the World Health Organization stated that transmission of the
    coronavirus by people who did not have symptoms was ``very rare,''
    \href{https://www.nytimes.com/2020/06/09/world/coronavirus-updates.html?action=click\&pgtype=Article\&state=default\&region=MAIN_CONTENT_3\&context=storylines_faq\#link-1f302e21}{but
    she later walked back that statement.}
  \end{itemize}
\end{itemize}

One senior administration official said the more appropriate comparison
would be Lockheed Martin's
\href{https://www.nytimes.com/1990/12/05/business/business-people-lockheeds-skunk-works-appoints-engineermanager.html}{``Skunk
Works''} program in California, where the company's most sensitive
aircraft projects have been developed and built. Many never left the
design stage.

Much of the work at the Warp Speed project involves making sure no
surprises slow development.

But Dr. Amesh Adalja, an infectious disease physician and senior scholar
at the Johns Hopkins University Center for Health Security, said that
the administration should ``be prepared for things to slow down once we
get further along.''

``All of the timelines are optimistic,'' he said. ``Vaccine development
doesn't always go as predicted. There are a lot of hiccups in the
production process.''

Image

A hospital worker in Brooklyn last month. New York has had one of the
highest infection rates in the world.Credit...Dave Sanders for The New
York Times

Democratic lawmakers on Tuesday
\href{https://oversight.house.gov/sites/democrats.oversight.house.gov/files/2020-06-02.Clyburn\%20CBM\%20to\%20HHS\%20re\%20Vaccine\%20and\%20Treatment\%20Contracts.pdf}{wrote
to Mr. Azar} with concerns about how his department was awarding
contracts to the pharmaceutical companies.

Representative James E. Clyburn, Democrat of South Carolina and the
chairman of the House's select committee on the coronavirus, and
Representative Carolyn B. Maloney, Democrat of New York and the
chairwoman of the Committee on Oversight and Reform, said that they were
``seeking to determine whether these contracts include provisions to
ensure affordability and prevent profiteering.''

Agreements have included promises from pharmaceutical companies related
to intellectual property, the number of doses that will be produced if a
candidate is successful and the price of a vaccine, one senior
administration official said. But few details have been made public.

Contracts are being awarded through the department's Biomedical Advanced
Research and Development Authority. Congress allocated billions of
dollars for vaccine development in various components of the \$2
trillion coronavirus relief package.

Senior administration officials said that between new congressional
funds and money that can be drawn from appropriations for the federal
health agencies, the project will have plenty of funding.

Behind the scenes, the project has undergone upheaval in its leadership.

Dr. Peter Marks, the federal scientist who devised and initially oversaw
the project at the Food and Drug Administration, stepped aside from his
role as its lead vaccine specialist, in part because he believed Dr.
Slaoui had potential conflicts of interest, according to senior
administration officials.

Dr. Slaoui, a venture capitalist and a former executive at the
pharmaceutical firm GlaxoSmithKline, sat on the board of Moderna before
accepting his current role last month. The value of his stock holdings
in Moderna
\href{https://www.nytimes.com/2020/05/20/health/coronavirus-vaccine-czar.html}{jumped
significantly} when the company released preliminary data from an early
phase of its candidate vaccine trial. He sold his \$12 million in shares
afterward, and the administration said he would donate the increased
value to cancer research.

Dr. Slaoui also joined the project on a contract rather than as a
government employee,
\href{https://www.nytimes.com/2020/05/20/health/coronavirus-vaccine-czar.html}{leaving
him exempt} from federal disclosure rules that would require him to list
his outside positions, stock holdings and other potential conflicts. The
arrangement is not subject to the same conflict-of-interest laws and
regulations that executive branch employees must follow.

In a May 20 meeting, according to one official, Dr. Marks, the director
of the F.D.A.'s Center for Biologics Evaluation and Research, informed
Dr. Stephen M. Hahn, the agency's commissioner, that he wanted to exit
the vaccine program.

He also left the White House's coronavirus task force, a group he had
been named to five days earlier.

Two senior officials said that
\href{https://ghss.georgetown.edu/people/hepburn/}{Col. Matthew
Hepburn}, who is also a physician, has stepped in for Dr. Marks on the
vaccine program.

Advertisement

\protect\hyperlink{after-bottom}{Continue reading the main story}

\hypertarget{site-index}{%
\subsection{Site Index}\label{site-index}}

\hypertarget{site-information-navigation}{%
\subsection{Site Information
Navigation}\label{site-information-navigation}}

\begin{itemize}
\tightlist
\item
  \href{https://help.nytimes.com/hc/en-us/articles/115014792127-Copyright-notice}{©~2020~The
  New York Times Company}
\end{itemize}

\begin{itemize}
\tightlist
\item
  \href{https://www.nytco.com/}{NYTCo}
\item
  \href{https://help.nytimes.com/hc/en-us/articles/115015385887-Contact-Us}{Contact
  Us}
\item
  \href{https://www.nytco.com/careers/}{Work with us}
\item
  \href{https://nytmediakit.com/}{Advertise}
\item
  \href{http://www.tbrandstudio.com/}{T Brand Studio}
\item
  \href{https://www.nytimes.com/privacy/cookie-policy\#how-do-i-manage-trackers}{Your
  Ad Choices}
\item
  \href{https://www.nytimes.com/privacy}{Privacy}
\item
  \href{https://help.nytimes.com/hc/en-us/articles/115014893428-Terms-of-service}{Terms
  of Service}
\item
  \href{https://help.nytimes.com/hc/en-us/articles/115014893968-Terms-of-sale}{Terms
  of Sale}
\item
  \href{https://spiderbites.nytimes.com}{Site Map}
\item
  \href{https://help.nytimes.com/hc/en-us}{Help}
\item
  \href{https://www.nytimes.com/subscription?campaignId=37WXW}{Subscriptions}
\end{itemize}
