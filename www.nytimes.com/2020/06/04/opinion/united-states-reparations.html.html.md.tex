Sections

SEARCH

\protect\hyperlink{site-content}{Skip to
content}\protect\hyperlink{site-index}{Skip to site index}

\href{https://myaccount.nytimes.com/auth/login?response_type=cookie\&client_id=vi}{}

\href{https://www.nytimes.com/section/todayspaper}{Today's Paper}

\href{/section/opinion}{Opinion}\textbar{}How to Do Reparations Right

\href{https://nyti.ms/2UbFBeu}{https://nyti.ms/2UbFBeu}

\begin{itemize}
\item
\item
\item
\item
\item
\item
\end{itemize}

Advertisement

\protect\hyperlink{after-top}{Continue reading the main story}

\href{/section/opinion}{Opinion}

Supported by

\protect\hyperlink{after-sponsor}{Continue reading the main story}

\hypertarget{how-to-do-reparations-right}{%
\section{How to Do Reparations
Right}\label{how-to-do-reparations-right}}

It's time to tackle racial disparities.

\href{https://www.nytimes.com/by/david-brooks}{\includegraphics{https://static01.nyt.com/images/2018/04/03/opinion/david-brooks/david-brooks-thumbLarge-v2.png}}

By \href{https://www.nytimes.com/by/david-brooks}{David Brooks}

Opinion Columnist

\begin{itemize}
\item
  June 4, 2020
\item
  \begin{itemize}
  \item
  \item
  \item
  \item
  \item
  \item
  \end{itemize}
\end{itemize}

\includegraphics{https://static01.nyt.com/images/2020/06/04/opinion/04brooks1/04brooks1-articleLarge.jpg?quality=75\&auto=webp\&disable=upscale}

This moment is about police brutality, but it's not only about police
brutality. The word I keep hearing is ``exhausted.''

People are exhausted by and fed up with the enduring
\href{https://www.brookings.edu/blog/up-front/2020/02/27/examining-the-black-white-wealth-gap/}{wealth
disparities} between white and black, with the health disparities that
leave black people more vulnerable to Covid-19, with the centuries-long
disparities in violence and the threat of violence, with daily
indignities of African-Americans and stains that linger on our nation
decade after decade.

The killing of George Floyd happened in a context --- and that context
is racial disparity.

Racial disparity doesn't make for gripping YouTube videos. It doesn't
spark mass protests because it's not an event; it's just the daily
condition of our lives.

It's just a condition that people in affluent Manhattan live in one
universe and people a few miles away in the Bronx live in a different
universe. It's just a condition that many black families send their kids
to struggling inner-city schools while white families move to the
suburbs and put on black T-shirts every few years to protest racial
injustice.

The response to this moment will be inadequate if it's just police
reforms. There has to be a greater effort to tackle the wider
disparities.

Reparations and integration are the way to do that. Reparations would
involve an official apology for centuries of slavery and discrimination,
and spending money to reduce their effects.

There's a wrong way to spend that money: trying to find the descendants
of slaves and sending them a check. That would launch a politically
ruinous argument over who qualifies for the money, and at the end of the
day people might be left with a \$1,000 check that would produce no
lasting change.

Giving reparations money to neighborhoods is the way to go.

A lot of the segregation in this country is geographic. In
\href{https://www.washingtonpost.com/business/2020/05/30/minneapolis-racial-inequality/}{Minneapolis},
where Floyd was killed, early-20th-century whites-only housing covenants
pushed blacks into smaller and smaller patches of the city. Highways
were built through black neighborhoods, ripping their fabric and
crippling their economic vitality.

Today, Minneapolis is as progressive as the day is long, but the city
gradually gave up on aggressive desegregation. And so you have these
long-suffering black neighborhoods. The homeownership rate for blacks in
Minneapolis is one-third the white rate. The typical black family earns
less than half as much as the typical white family.

To really change things, you have to lift up and integrate whole
communities. That's because it takes a whole community to raise a child,
to support an adult, to have a bustling local economy and a vibrant
civic life. The neighborhood is the unit of change.

Who has the expertise to lift up whole neighborhoods? \emph{It's the
people who live in the neighborhoods themselves.} No outsider with a
foundation grant or a government contract really knows what's going on
in any neighborhood or would be trusted to make change. The people who
live in the neighborhoods know what to do. They just need the resources
to do it.

A few weeks before the lockdown I was in and around South Los Angeles.
In Watts I interviewed Keisha Daniels from
\href{https://sistersofwatts.org/}{Sisters of Watts}, which helps kids
and homeless people in a variety of ways. I interviewed Barak and Sara
Bomani of \href{https://www.unearthandempower.org/}{Unearth and Empower
Communities}, which helps educate and nurture young people in nearby
Compton.

Daniels and the Bomanis are experts in how to lift up their
neighborhoods. If we got them money and support they would figure out
what to do.

How can government focus money on formerly redlined neighborhoods and
other communities?

National service programs
\href{https://www.nytimes.com/2020/05/07/opinion/national-service-americorps-coronavirus.html}{would
pay young people} to work for these organizations. A National Endowment
for Civic Architecture, modeled on, say, the National Endowment for the
Arts, could support neighborhood groups around the country. A Social
Innovation Fund would be a private/public partnership to fund such
organizations. Moving to Opportunity grants and K-12 education savings
accounts would help minorities to move to integrated schools.
\href{https://www.nytimes.com/2018/10/08/opinion/collective-impact-community-civic-architecture.html}{Collective
impact structures} could coordinate
\href{https://www.nytimes.com/2019/04/04/opinion/canada-poverty-record.html}{local
action} and use data to find what works.

In the progressive era, governments built libraries across the country,
which remain vital centers of neighborhood life. We're about to have a
lot of empty retail space. Why can't we build Opportunity Centers where
all the groups moving children from cradle to career could work and
collaborate?

It's true this has sort of been tried before. The Great Society had a
``Community Action'' project that professed to redistribute power to
neighborhoods. But it did it in the worst possible ways. A lot of what
it did involved sending disruptive agitators to stir up conflict between
local activists and local elected officials. The result was rancor and
gridlock.

This tumultuous moment offers a chance to launch a new chapter in our
history, and reparations are part of that launch. They offer a chance to
build vibrant neighborhoods where diverse people want to live together,
where the atmosphere is kids playing on the sidewalks and not a knee in
the back of the neck.

\emph{The Times is committed to publishing}
\href{https://www.nytimes.com/2019/01/31/opinion/letters/letters-to-editor-new-york-times-women.html}{\emph{a
diversity of letters}} \emph{to the editor. We'd like to hear what you
think about this or any of our articles. Here are some}
\href{https://help.nytimes.com/hc/en-us/articles/115014925288-How-to-submit-a-letter-to-the-editor}{\emph{tips}}\emph{.
And here's our email:}
\href{mailto:letters@nytimes.com}{\emph{letters@nytimes.com}}\emph{.}

\emph{Follow The New York Times Opinion section on}
\href{https://www.facebook.com/nytopinion}{\emph{Facebook}}\emph{,}
\href{http://twitter.com/NYTOpinion}{\emph{Twitter (@NYTopinion)}}
\emph{and}
\href{https://www.instagram.com/nytopinion/}{\emph{Instagram}}\emph{.}

Advertisement

\protect\hyperlink{after-bottom}{Continue reading the main story}

\hypertarget{site-index}{%
\subsection{Site Index}\label{site-index}}

\hypertarget{site-information-navigation}{%
\subsection{Site Information
Navigation}\label{site-information-navigation}}

\begin{itemize}
\tightlist
\item
  \href{https://help.nytimes.com/hc/en-us/articles/115014792127-Copyright-notice}{©~2020~The
  New York Times Company}
\end{itemize}

\begin{itemize}
\tightlist
\item
  \href{https://www.nytco.com/}{NYTCo}
\item
  \href{https://help.nytimes.com/hc/en-us/articles/115015385887-Contact-Us}{Contact
  Us}
\item
  \href{https://www.nytco.com/careers/}{Work with us}
\item
  \href{https://nytmediakit.com/}{Advertise}
\item
  \href{http://www.tbrandstudio.com/}{T Brand Studio}
\item
  \href{https://www.nytimes.com/privacy/cookie-policy\#how-do-i-manage-trackers}{Your
  Ad Choices}
\item
  \href{https://www.nytimes.com/privacy}{Privacy}
\item
  \href{https://help.nytimes.com/hc/en-us/articles/115014893428-Terms-of-service}{Terms
  of Service}
\item
  \href{https://help.nytimes.com/hc/en-us/articles/115014893968-Terms-of-sale}{Terms
  of Sale}
\item
  \href{https://spiderbites.nytimes.com}{Site Map}
\item
  \href{https://help.nytimes.com/hc/en-us}{Help}
\item
  \href{https://www.nytimes.com/subscription?campaignId=37WXW}{Subscriptions}
\end{itemize}
