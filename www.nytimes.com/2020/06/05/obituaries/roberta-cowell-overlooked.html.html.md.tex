Sections

SEARCH

\protect\hyperlink{site-content}{Skip to
content}\protect\hyperlink{site-index}{Skip to site index}

\href{https://www.nytimes.com/section/obituaries}{Obituaries}

\href{https://myaccount.nytimes.com/auth/login?response_type=cookie\&client_id=vi}{}

\href{https://www.nytimes.com/section/todayspaper}{Today's Paper}

\href{/section/obituaries}{Obituaries}\textbar{}Overlooked No More:
Roberta Cowell, Trans Trailblazer, Pilot and Auto Racer

\url{https://nyti.ms/3eWhyrw}

\begin{itemize}
\item
\item
\item
\item
\item
\item
\end{itemize}

Advertisement

\protect\hyperlink{after-top}{Continue reading the main story}

Supported by

\protect\hyperlink{after-sponsor}{Continue reading the main story}

The Great Read

\hypertarget{overlooked-no-more-roberta-cowell-trans-trailblazer-pilot-and-auto-racer}{%
\section{Overlooked No More: Roberta Cowell, Trans Trailblazer, Pilot
and Auto
Racer}\label{overlooked-no-more-roberta-cowell-trans-trailblazer-pilot-and-auto-racer}}

Cowell is the first woman known to undergo sex reassignment surgery in
Britain. But after a splash in the 1950s, she withdrew from public life
and died in obscurity.

\includegraphics{https://static01.nyt.com/images/2020/06/08/obituaries/08overlooked-cowell-1/00overlooked-cowell-1-articleLarge.jpg?quality=75\&auto=webp\&disable=upscale}

By \href{https://www.nytimes.com/by/alan-cowell}{Alan Cowell}

\begin{itemize}
\item
  June 5, 2020
\item
  \begin{itemize}
  \item
  \item
  \item
  \item
  \item
  \item
  \end{itemize}
\end{itemize}

\emph{Overlooked is a series of obituaries about remarkable people whose
deaths, beginning in 1851, went unreported in The Times. This month
we're adding the stories of important L.G.B.T. figures.}

At the height of Roberta Cowell's celebrity status, in 1954, her face
adorned \href{https://www.listal.com/roberta-cowell}{the cover of
Britain's popular Picture Post magazine}. When her story appeared in a
newspaper, ``I received 400 proposals. Some of them of marriage,'' she
said in an interview for
\href{http://www.lizhodgkinson.com/lh/journalismArticle/interview_with_transsexual_roberta_betty_cowell}{The
Sunday Times of London} in 1972. ``I could have had titles, money, the
lot.''

She achieved this fame when she became the first person in her country
known to have her gender reassigned from male to female. Her transition
--- and all of the yearnings and hopes that came with it --- involved
hormone treatments and surgeries despite what some regarded in
strait-laced 1950s Britain as flouting contemporary laws.

``Since May 18th, 1951, I have been Roberta Cowell, female,'' she
pronounced in her autobiography. ``I have become woman physically,
psychologically, glandularly and legally.''

Yet by the time Cowell died in 2011 at 93, her voyage across the lines
of gender and social norms had faded into obscurity.

Americans were perhaps more familiar with Christine Jorgensen, a former
U.S. Army clerk who transitioned in Denmark just months after Cowell.
When Jorgensen died of cancer in 1989 at 62, the event was recorded
\href{https://www.nytimes.com/1989/05/04/obituaries/christine-jorgensen-62-is-dead-was-first-to-have-a-sex-change.html}{in
an obituary in The New York Times}.

Cowell's death, by contrast, went all but unremarked upon, even in
Britain. Her body was found on Oct. 11, 2011, in her small apartment in
southwest London by the building superintendent. A handful of friends
attended her funeral, but, apparently at her request, there was no
fanfare for the woman who had helped pioneer gender reassignment at a
time when it was virtually taboo.

Only in 2013 --- two years after her death --- was her passing reported,
by the British newspaper
\href{https://www.independent.co.uk/news/people/profiles/its-easier-to-change-a-body-than-to-change-a-mind-the-extraordinary-life-and-lonely-death-of-roberta-8899823.html}{The
Independent on Sunday}.

``So complete was her withdrawal from public life that even her own
children did not know she had died,'' the article said.

The disclosure of her death inspired a brief resurgence of media
interest in her story, focusing partly on what was broadly depicted as
the severing of all ties with her two daughters and on the idiosyncratic
circumstances of her transition.

\includegraphics{https://static01.nyt.com/images/2020/06/08/obituaries/08overlooked-cowell-3/00overlooked-cowell-3-articleLarge.jpg?quality=75\&auto=webp\&disable=upscale}

After World War II, she developed an interest in the idea of a
combination of hormone therapy and surgery to more closely align her
body with her gender identity. This had been reinforced by a book called
``\href{https://books.google.com/books/about/Self.html?id=-oVtNAAACAAJ}{Self:
A Study in Ethics and Endocrinology}'' (1946) by Michael Dillon, a
medical student whom she sought out in 1950.

Cowell wrote in her autobiography
``\href{http://library.transgenderzone.com/?page_id=3058}{Roberta
Cowell's Story},'' that during their meeting, over lunch, Dillon
revealed that he had himself changed his gender identity through doses
of testosterone and gender-affirming surgery.

Together they agreed that he would help her transition by performing a
procedure that was prohibited under so-called ``mayhem'' laws,
forbidding the intentional ``disfiguring'' of men who would otherwise
qualify to serve in the military. If discovered, Dillon would almost
certainly have been prevented from completing his studies to become a
physician. The operation, thus, was conducted in great secrecy, and its
success enabled Cowell to seek medical affidavits to have her birth
gender formally re-registered as female.

Soon afterward, Cowell became a patient of Harold Gillies, a pioneer of
plastic surgery who had performed gender-affirming surgery on Dillon,
according to the book ``The First Man-Made Man: The Story of Two Sex
Changes, One Love Affair, and a Twentieth-Century Medical Revolution''
(2006).

``If it gives real happiness,'' Gillies wrote of his procedures, ``that
is the most that any surgeon or medicine can give.''

By several accounts, Dillon fell deeply in love with Cowell, but she
ultimately rejected his proposal of marriage.

Roberta Elizabeth Marshall Cowell (no relation to the author of this
article) was born on April 8, 1918, in Croydon, south of London, one of
three children born to Dorothy Elizabeth Miller and a high-ranking
military surgeon, Maj. Gen. Ernest Marshall Cowell, who had served as a
physician in both world wars and, in 1944, was appointed honorary
surgeon to King George VI. In the social order of the time, it was
guaranteed that Roberta Cowell would be educated in tuition-based,
single-sex schools. She developed an abiding interest in cars and
racing. ``It was the be-all and nearly the end-all of my existence,''
she said in her autobiography.

From an early age, she wrote, she felt conflicted about her gender,
compensating for feminine ``characteristics'' with an ``aggressively
masculine manner'' that persuaded gay men to take her ``for one of
themselves.''

Physically, she was sensitive about being overweight, displaying what
she called ``feminoidal fat distribution.'' In her teenage years, other
pupils nicknamed her ``Circumference'' and ``Bottom.'' She left school
at 16 to work briefly as an apprentice engineer until she joined the
Royal Air Force in 1935. Her ambition was to become a fighter pilot, but
she was found to suffer from acute airsickness and was deemed
``permanently unfit for further flying duties with the R.A.F.''

From then until the start of World War II in 1939, she studied
engineering at University College London and entered a series of
automobile races including the Antwerp Grand Prix in Belgium. She
enlisted in the Army in 1940.

In 1941 she married Diana Margaret Zelma Carpenter, a fellow engineer
and racecar driver whom she had met in college. They had two daughters,
Anne and Diana. They separated in 1948 and divorced in 1952.

Despite her earlier dismissal from flying duties, Cowell was allowed to
return to the R.A.F. in 1942, flying combat and aerial reconnaissance
missions in Spitfires and other aircraft. After the Allied D-Day
landings in Normandy in June 1944, she flew out of a Belgian air base in
a Hawker Typhoon airplane that was shot down by ground fire over Germany
on a low-level attack east of the Rhine River. The flight, she said, had
been scheduled as the ``very last trip of my second tour of
operations.'' In fact it was her last flight of the war.

She crash-landed the stricken warplane and was taken prisoner.

Fearing that her captors would treat her harshly, she twice sought to
escape and twice she failed. She was transferred to Stalag Luft I, a
prison camp for Allied aircrews in north Germany near the Baltic Sea
between Lübeck and Rostock.

In her autobiography, she described the surreal elements of wartime
life, relating perilous adventures with ironic detachment. She spoke of
blacking out at 40,000 feet when her oxygen supply malfunctioned but
somehow reviving after her plane plummeted almost to the ground. And, on
another occasion, she recounted making an emergency landing atop a cliff
on the English coastline just as her plane ran out of fuel.

In the early days of her captivity, she said, an Allied air raid on
Frankfurt forced her and her captors into a bomb shelter where angry
German civilians realized that she was an enemy pilot. She persuaded
them ``in my halting German'' that she was not a bomber pilot and told
them the untruth that her mother and father had been killed in a German
raid on London. ``It seemed to do the trick and the angry growling died
down,'' she wrote in her autobiography. ``I wonder what would have
happened to a Luftwaffe pilot discovered in an air-raid shelter during
the blitz.''

Conditions at Stalag Luft I worsened as the end of the war approached,
with Soviet Red Army troops advancing across Germany toward Berlin. Food
supplies were so meager, she wrote, that inmates ate stray cats raw and
she lost 49 pounds. In May 1945, as German forces surrendered, their
captors abandoned the facility, leaving it unguarded until Soviet troops
liberated it. Within days, Cowell and other British captives had been
flown home aboard American Flying Fortress bombers.

The immediate postwar years confronted Cowell with the practical
problems of earning a living, variously building and racing cars and
renovating houses to sell at a profit. But she also detected a mounting
sense of ``restlessness and unhappiness,'' she wrote in her
autobiography, and resolved to undergo Freudian psychoanalysis. ``It
became quite obvious that the feminine side of my nature, which all my
life I had known of and severely repressed, was very much more
fundamental and deep-rooted than I had supposed.''

Image

Cowell participating in the women's race car competition in Sussex,
England, when she was 39. After the war, she earned a living by building
and racing cars.Credit...PA Images/Alamy Stock Photo

She began to live a double life, taking hormone treatments to enhance
her femininity while still living as a man.

People, she wrote, would speculate openly on her gender. ``I preferred
to steer clear of children and elderly ladies; they were too observant
or at least too outspoken in their remarks.''

Then came the turning point when she met Dillon. The encounter was ``so
shattering that the scene will be crystal-clear in my memory for the
rest of my life,'' she wrote.

After three years of therapy and surgery, Cowell seemed to find
emotional contentment that was matched only intermittently by material
security. Two business ventures, in experimental car engineering and
women's clothing, did not survive.

The publication of her story in Picture Post in 1954 and her
autobiography earned her the equivalent of several hundred thousand
dollars. In 1957 she won a noted hill climb auto race and bought a
wartime Mosquito fighter-bomber in which she planned to break the speed
record for a flight across the South Atlantic. But the attempt never
came about. In 1958 she appeared in bankruptcy court where she said she
had no assets and significant debts, owed mainly to her father.

Cowell's name has been summoned as a trailblazer in the years since her
death, her transition having preceded by decades the public discourse
over gender identity and L.G.B.T.Q. rights.

Perhaps because she was one of the first to transition medically, she
didn't recommend it easily to others, saying, ``Many of those people
will regret the operation later. There have been attempted suicides.''

Whether her views would have changed over time will never be known; in
1972 she said she was writing a second autobiography, but it was never
published.

Advertisement

\protect\hyperlink{after-bottom}{Continue reading the main story}

\hypertarget{site-index}{%
\subsection{Site Index}\label{site-index}}

\hypertarget{site-information-navigation}{%
\subsection{Site Information
Navigation}\label{site-information-navigation}}

\begin{itemize}
\tightlist
\item
  \href{https://help.nytimes.com/hc/en-us/articles/115014792127-Copyright-notice}{©~2020~The
  New York Times Company}
\end{itemize}

\begin{itemize}
\tightlist
\item
  \href{https://www.nytco.com/}{NYTCo}
\item
  \href{https://help.nytimes.com/hc/en-us/articles/115015385887-Contact-Us}{Contact
  Us}
\item
  \href{https://www.nytco.com/careers/}{Work with us}
\item
  \href{https://nytmediakit.com/}{Advertise}
\item
  \href{http://www.tbrandstudio.com/}{T Brand Studio}
\item
  \href{https://www.nytimes.com/privacy/cookie-policy\#how-do-i-manage-trackers}{Your
  Ad Choices}
\item
  \href{https://www.nytimes.com/privacy}{Privacy}
\item
  \href{https://help.nytimes.com/hc/en-us/articles/115014893428-Terms-of-service}{Terms
  of Service}
\item
  \href{https://help.nytimes.com/hc/en-us/articles/115014893968-Terms-of-sale}{Terms
  of Sale}
\item
  \href{https://spiderbites.nytimes.com}{Site Map}
\item
  \href{https://help.nytimes.com/hc/en-us}{Help}
\item
  \href{https://www.nytimes.com/subscription?campaignId=37WXW}{Subscriptions}
\end{itemize}
