\href{/section/sports/football}{Pro Football}\textbar{}Kneeling,
Fiercely Debated in the N.F.L., Resonates in Protests

\url{https://nyti.ms/2MAo2At}

\begin{itemize}
\item
\item
\item
\item
\item
\item
\end{itemize}

\href{https://www.nytimes.com/news-event/george-floyd-protests-minneapolis-new-york-los-angeles?action=click\&pgtype=Article\&state=default\&region=TOP_BANNER\&context=storylines_menu}{Race
and America}

\begin{itemize}
\tightlist
\item
  \href{https://www.nytimes.com/2020/07/26/us/protests-portland-seattle-trump.html?action=click\&pgtype=Article\&state=default\&region=TOP_BANNER\&context=storylines_menu}{Protesters
  Return to Other Cities}
\item
  \href{https://www.nytimes.com/2020/07/24/us/portland-oregon-protests-white-race.html?action=click\&pgtype=Article\&state=default\&region=TOP_BANNER\&context=storylines_menu}{Portland
  at the Center}
\item
  \href{https://www.nytimes.com/2020/07/23/podcasts/the-daily/portland-protests.html?action=click\&pgtype=Article\&state=default\&region=TOP_BANNER\&context=storylines_menu}{Podcast:
  Showdown in Portland}
\item
  \href{https://www.nytimes.com/interactive/2020/07/16/us/black-lives-matter-protests-louisville-breonna-taylor.html?action=click\&pgtype=Article\&state=default\&region=TOP_BANNER\&context=storylines_menu}{45
  Days in Louisville}
\end{itemize}

\includegraphics{https://static01.nyt.com/images/2020/06/05/sports/05unrest-kneeling-1/merlin_144253599_9c2c509e-3e32-45af-a31c-1bf1535b42dd-articleLarge.jpg?quality=75\&auto=webp\&disable=upscale}

Sections

\protect\hyperlink{site-content}{Skip to
content}\protect\hyperlink{site-index}{Skip to site index}

\hypertarget{kneeling-fiercely-debated-in-the-nfl-resonates-in-protests}{%
\section{Kneeling, Fiercely Debated in the N.F.L., Resonates in
Protests}\label{kneeling-fiercely-debated-in-the-nfl-resonates-in-protests}}

Some demonstrators, and in some cases the police, have paused to kneel,
recalling the manner of George Floyd's death and the gesture by Colin
Kaepernick.

Colin Kaepernick in 2016.Credit...Ted S. Warren/Associated Press

Supported by

\protect\hyperlink{after-sponsor}{Continue reading the main story}

\href{https://www.nytimes.com/by/kurt-streeter}{\includegraphics{https://static01.nyt.com/images/2018/11/26/multimedia/author-kurt-streeter/author-kurt-streeter-thumbLarge.png}}

By \href{https://www.nytimes.com/by/kurt-streeter}{Kurt Streeter}

\begin{itemize}
\item
  Published June 5, 2020Updated Aug. 3, 2020
\item
  \begin{itemize}
  \item
  \item
  \item
  \item
  \item
  \item
  \end{itemize}
\end{itemize}

\hypertarget{listen-to-this-article}{%
\subsubsection{Listen to This Article}\label{listen-to-this-article}}

Audio Recording by Audm

\emph{To hear more audio stories from publishers like The New York
Times,
download}\href{https://www.audm.com/?utm_source=nytmag\&utm_medium=embed\&utm_campaign=left_behind_draper}{**}\href{https://www.audm.com/?utm_source=nyt\&utm_medium=embed\&utm_campaign=kneeling_nfl_protests}{\emph{Audm
for iPhone or Android}}\emph{.}

It is a simple gesture, swaddled in outrage and long-endured grief, that
gained powerful currency through the protest against police brutality
and racial injustice led by quarterback
\href{https://www.nytimes.com/2020/08/03/us/navy-seal-museum-kaepernick.html}{Colin
Kaepernick} on the fields of the National Football League.

Taking a knee.

Across the nation these last hard, uncertain days, demonstrators have
turned to the gesture on city streets. At a nighttime march in
Minneapolis on Wednesday, a crowd of 400 knelt for nearly five somber
minutes. On the same day, George Floyd's son, Quincy Mason, walked
through a crowd at the site where a white police officer had pinned his
father to the ground by a knee to the neck. There, before a makeshift
memorial, Mason dropped to a knee.

The gesture has even been made sporadically by law enforcement officers,
members of the National Guard and by prominent politicians as an act of
solidarity or effort to pacify.

In New York, an N.Y.P.D. commander knelt with activists outside
Washington Square Park. In Portland, Ore., police in riot gear knelt
before cheering demonstrators, some of whom responded by walking toward
the officers to shake their hands. Mayor Eric Garcetti of Los Angeles
walked amid a demonstration and knelt. And the presumptive Democratic
presidential nominee, Joseph R. Biden Jr., took a knee at a campaign
visit to a black church in Delaware.

Kaepernick has not played in the N.F.L. since Jan. 1, 2017, his career
cut short when no team would sign him following a season of player
protest he led with the help of a teammate, Eric Reid.

But his kneeling objection during the playing of the national anthem has
boomeranged through the choppy slipstream of the American consciousness,
and is again at the center of a turbulent moment with newfound force,
and for the N.F.L., renewed debate.

``It's a powerful, peaceful way to say you're not OK with what's been
happening,'' said Hibes Galeano, 32, a Latina who attended a protest in
Minneapolis this week. Others who knelt spoke of Kaepernick with
reverence. ``He did what a lot of other athletes wouldn't have done,''
said Dorien Harris, a black, 19-year-old marcher who wore a face mask
inscribed with the words ``I Can't Breathe'' as he knelt.

``It took a lot of guts for him to do that, a lot of heart,'' he added.
``He knows what the community needs. It needs that strength. He was
saying to stand up for what you believe in, no matter your position.''

\includegraphics{https://static01.nyt.com/images/2020/06/07/sports/07jpunrest-kneeling3-print/merlin_173115486_24c245e8-04b5-4e9f-a803-2ae12d3f4175-articleLarge.jpg?quality=75\&auto=webp\&disable=upscale}

While some demonstrators say they have had Kaepernick and his campaign
in mind when kneeling, the gesture is also --- intended or not --- an
echo to the manner of Floyd's death.

``Kneeling is both an act of defiance and resistance, but also of
reverence, of mourning, but also honoring lives lost,'' said Chad
Williams, the chairman of the Department of African and Afro-American
Studies at Brandeis University. ``It is also simple and clear. Its
simplicity gave it symbolic power, and as we see now, its power
persists.''

So does the controversy surrounding it.

Starting in 2016, despite Kaepernick's explanation that his kneeling
during the national anthem was a call to end racial injustice and police
brutality toward people of color, a backlash fomented, spurred largely
by President Trump, who tried to recast Kaepernick and the predominantly
African-American group of players who followed his lead as unpatriotic.
That viewpoint persists.

New Orleans Saints quarterback Drew Brees, when asked on Wednesday about
kneeling during the anthem, told an interviewer, ``I will never agree
with anybody disrespecting the flag of the United States of America or
our country.'' (Brees did kneel on the sideline with teammates before a
game in 2017 as an act of solidarity opposing Trump's criticism of
players, but he rose when the anthem was played.)

He then issued two apologies for his comments after an unusual wave of
criticism from players, some of them his own teammates, at a time when
athletes across the sports world have responded to the civil unrest by
participating in marches and expressing support for combating racial
injustice.

Many applauded Brees's contrition but on Friday,
\href{https://twitter.com/realDonaldTrump/status/1268998143733051394?s=20}{Trump
tweeted} that Brees should not have shifted his stance, saying in all
caps that there should be ``NO KNEELING!" during a display of
patriotism. Hours later,
\href{https://twitter.com/NFL/status/1269034074552721408}{the league's
commissioner, Roger Goodell, apologized to his players} for not
listening to the concerns of African-American players earlier. He said
he supported athletes in protesting peacefully, though he notably did
not name Kaepernick directly.

Not long after that, Brees
\href{https://www.instagram.com/p/CBE4y_9Hj2S/?igshid=7ejunv7ktpcn}{addressed
an Instagram post}to Trump that forcefully repudiated his original
remark about disrespect, saying: ``We can no longer use the flag to turn
people away or distract them from the real issues that face our black
communities.''

Taking a knee might be a simple gesture, but the fraught, contentious
opinions about it are a mirror into the complexity of race in America.

Consider its N.F.L. origin story.

Kaepernick and Reid came up with the idea after consulting a former
Green Beret, Nate Boyer, who had served in Iraq and Afghanistan before
playing college football at Texas and then getting a tryout with the
Seattle Seahawks. ``Colin straight up asked me what I thought he should
do,'' said Boyer, speaking recently over the phone from Oregon.

Boyer said he did some research and came across a photograph of Martin
Luther King Jr. kneeling in prayer and protest in Selma, Ala., during
the 1960s. Boyer also remembered taking a knee at Arlington National
Cemetery, in reverence of fallen friends.

Image

Nate Boyer, a former Green Beret, advised Kaepernick and Eric Reid to
kneel after seeing an image of the Rev. Dr. Martin Luther King Jr. and
others in a civil rights protest in 1965.Credit...BH/Associated Press

``If you're not going to stand,'' Boyer told Kaepernick and Reid, as
they sat in a hotel lobby hours before the 49ers' final preseason game,
against the San Diego Chargers. ``I'd say your only other option is to
take a knee.''

Boyer said he would never do such a thing during the anthem. But he had
fought for the right of free expression, and though he said he was
apolitical, he empathized with the drive to end racism and police
brutality.

At the game that evening, he stood next to Kaepernick as he knelt, and
felt the sting of an angry, booing crowd rain onto the field. ``Maybe
that was my little taste of what it is like to be black. It helped me
understand,'' he said.

The players' kneeling reached a peak in the 2017 season --- when Trump
demanded that team owners ``Get that son of a bitch off the field right
now!'' for kneeling --- but has since largely petered out.

In early 2019, the N.F.L. handed over a payout believed to be roughly
\$6 million to settle a legal fight with Kaepernick and Reid, who argued
they had been denied jobs because of their actions during the national
anthem.

The league agreed to donate millions of dollars to community groups and
causes chosen by players. It joined with Jay-Z, the hip-hop impresario,
to consult on entertainment and contribute to the league's activism
campaign, Inspire Change. It also updated a policy, so far not enforced,
requiring players to stand for the national anthem or remain in the
locker room while it is played.

Within a week of Floyd's death, kneeling became a common gesture. And
its complexity carries on.

The way it has been adopted by members of law enforcement and
politicians, for example, is best viewed with an eye that is both
skeptical and hopeful, said Mark Anthony Neal, chairman of the African
and African-American Studies Department at Duke.

``It's an important gesture, showing maybe they get it now,'' he said.
``But if those same officers and politicians are not willing to hold
their own accountable going forward, or look at their own actions and
examine them closely, this is at best empty rhetoric.''

Kaepernick has remained publicly silent aside from recent postings about
the protest on social media.

Among his latest posts on Twitter? A retweet that jabs at Brees
and\href{https://twitter.com/kylekuzma/status/1268366393457602561?s=20}{shows
a 2017 photo of the Saints quarterback} taking a knee alongside
protesting teammates.

Reporting was contributed by Kim Barker, Dionne Searcey, John Eligon,
Ken Belson and Matt Furber. Jack Begg contributed research.

Advertisement

\protect\hyperlink{after-bottom}{Continue reading the main story}

\hypertarget{site-index}{%
\subsection{Site Index}\label{site-index}}

\hypertarget{site-information-navigation}{%
\subsection{Site Information
Navigation}\label{site-information-navigation}}

\begin{itemize}
\tightlist
\item
  \href{https://help.nytimes.com/hc/en-us/articles/115014792127-Copyright-notice}{©~2020~The
  New York Times Company}
\end{itemize}

\begin{itemize}
\tightlist
\item
  \href{https://www.nytco.com/}{NYTCo}
\item
  \href{https://help.nytimes.com/hc/en-us/articles/115015385887-Contact-Us}{Contact
  Us}
\item
  \href{https://www.nytco.com/careers/}{Work with us}
\item
  \href{https://nytmediakit.com/}{Advertise}
\item
  \href{http://www.tbrandstudio.com/}{T Brand Studio}
\item
  \href{https://www.nytimes.com/privacy/cookie-policy\#how-do-i-manage-trackers}{Your
  Ad Choices}
\item
  \href{https://www.nytimes.com/privacy}{Privacy}
\item
  \href{https://help.nytimes.com/hc/en-us/articles/115014893428-Terms-of-service}{Terms
  of Service}
\item
  \href{https://help.nytimes.com/hc/en-us/articles/115014893968-Terms-of-sale}{Terms
  of Sale}
\item
  \href{https://spiderbites.nytimes.com}{Site Map}
\item
  \href{https://help.nytimes.com/hc/en-us}{Help}
\item
  \href{https://www.nytimes.com/subscription?campaignId=37WXW}{Subscriptions}
\end{itemize}
