Sections

SEARCH

\protect\hyperlink{site-content}{Skip to
content}\protect\hyperlink{site-index}{Skip to site index}

\href{https://www.nytimes.com/section/us}{U.S.}

\href{https://myaccount.nytimes.com/auth/login?response_type=cookie\&client_id=vi}{}

\href{https://www.nytimes.com/section/todayspaper}{Today's Paper}

\href{/section/us}{U.S.}\textbar{}U.S. Marine Corps Issues Ban on
Confederate Battle Flags

\href{https://nyti.ms/2YcAsDS}{https://nyti.ms/2YcAsDS}

\begin{itemize}
\item
\item
\item
\item
\item
\end{itemize}

\href{https://www.nytimes.com/news-event/george-floyd-protests-minneapolis-new-york-los-angeles?action=click\&pgtype=Article\&state=default\&region=TOP_BANNER\&context=storylines_menu}{Race
and America}

\begin{itemize}
\tightlist
\item
  \href{https://www.nytimes.com/interactive/2020/07/03/us/george-floyd-protests-crowd-size.html?action=click\&pgtype=Article\&state=default\&region=TOP_BANNER\&context=storylines_menu}{Black
  Lives Matter Movement}
\item
  \href{https://www.nytimes.com/interactive/2020/06/28/us/i-cant-breathe-police-arrest.html?action=click\&pgtype=Article\&state=default\&region=TOP_BANNER\&context=storylines_menu}{History
  of `I Can't Breathe'}
\item
  \href{https://www.nytimes.com/interactive/2020/06/10/upshot/black-lives-matter-attitudes.html?action=click\&pgtype=Article\&state=default\&region=TOP_BANNER\&context=storylines_menu}{How
  Public Opinion Shifted}
\item
  \href{https://www.nytimes.com/interactive/2020/07/16/us/black-lives-matter-protests-louisville-breonna-taylor.html?action=click\&pgtype=Article\&state=default\&region=TOP_BANNER\&context=storylines_menu}{45
  Days in Louisville}
\end{itemize}

Advertisement

\protect\hyperlink{after-top}{Continue reading the main story}

Supported by

\protect\hyperlink{after-sponsor}{Continue reading the main story}

\hypertarget{us-marine-corps-issues-ban-on-confederate-battle-flags}{%
\section{U.S. Marine Corps Issues Ban on Confederate Battle
Flags}\label{us-marine-corps-issues-ban-on-confederate-battle-flags}}

The directive, which was announced on Friday, details what is prohibited
in Marine installations, office buildings and work spaces.

\includegraphics{https://static01.nyt.com/images/2020/06/06/multimedia/06xp-marines-pix/06xp-marines-pix-articleLarge.jpg?quality=75\&auto=webp\&disable=upscale}

\href{https://www.nytimes.com/by/jenny-gross}{\includegraphics{https://static01.nyt.com/images/2020/03/06/reader-center/author-jenny-gross/author-jenny-gross-thumbLarge.png}}

By \href{https://www.nytimes.com/by/jenny-gross}{Jenny Gross}

\begin{itemize}
\item
  Published June 6, 2020Updated June 10, 2020
\item
  \begin{itemize}
  \item
  \item
  \item
  \item
  \item
  \end{itemize}
\end{itemize}

The U.S. Marine Corps on Friday issued detailed directives about
removing and banning
\href{https://www.nytimes.com/news-event/confederate-flags-monuments-statues}{public
displays of the Confederate battle flag} at Marine installations --- an
order that extended to such items as
\href{https://twitter.com/USMC/status/1269075089078784001?s=20}{mugs,
posters and bumper stickers}.

``Current events are a stark reminder that it is not enough for us to
remove symbols that cause division --- rather, we also must strive to
eliminate division itself,'' the commandant of the Marine Corps, Gen.
David H. Berger,
\href{https://www.marines.mil/News/Press-Releases/Press-Release-Display/Article/2207572/message-from-the-commandant-of-the-marine-corps-and-the-sergeant-major-of-the-m/fbclid/IwAR12-WqkUEslW8WmtMiCk6CwitGjlg6xOg__tU5z7nUZH-2KD19eG4jCvBY/}{said
in a statement on Wednesday.}

As protests across the United States have
\href{https://www.nytimes.com/interactive/2020/05/30/us/george-floyd-protest-photos.html}{erupted
over police brutality}, pressure has grown on officials to remove
monuments and flags seen as symbols of racism.

Dozens of statues were removed after a white nationalist rally in
Charlottesville, Va., in 2017, and protesters demonstrating over the
death of
\href{https://www.nytimes.com/2020/05/29/us/minneapolis-police-george-floyd.html}{George
Floyd, a black man who died in police custody} in Minneapolis, are
targeting some that remain.

In several states, anger has given way to the
\href{https://www.nytimes.com/2020/06/03/us/confederate-statues-george-floyd.html}{damaging
or defacing of more than a dozen symbols} of the Confederacy.

The mayor of Birmingham, Ala., this week
\href{https://www.nytimes.com/2020/06/02/us/george-floyd-birmingham-confederate-statue.html}{ordered
the removal of a contentious Confederate statue} from a public park a
day after dozens demonstrated against it.

Gov. Ralph Northam of Virginia said this week he planned to order the
Robert E. Lee statue in Richmond to be removed. And the
\href{https://www.nytimes.com/2020/06/03/us/frank-rizzo-statue-removal.html}{city
of Philadelphia this week removed the statue of its former mayor, Frank
Rizzo}, who took a confrontational approach to black and gay people as
police commissioner in the 1960s and '70s.

\href{https://twitter.com/USMC/status/1269075089078784001/photo/1}{The
Marine Corps said in a statement on Twitter} that the
\href{https://www.nytimes.com/2020/07/06/us/politics/trump-bubba-wallace-nascar.html}{Confederate
battle flag} had ``all too often been co-opted by violent extremists and
racist groups whose divisive beliefs have no place in our Corps.''

``This presents a threat to our core values, unit cohesion, security,
and good order and discipline,'' the statement said. ``This must be
addressed.''

The move came after
\href{https://www.nytimes.com/2020/04/23/us/marine-corps-confederate-flag.html}{an
announcement in April} by General Berger that the ban would be put in
effect. At the time, however, it was not clear how it would be applied
and whether it would extend to clothing and cars owned by Marines, for
instance.

``I am mindful that many people believe that flag to be a symbol of
heritage and regional pride,'' General Berger
\href{https://mca-marines.org/wp-content/uploads/CMC-Letter-R1.pdf}{said
in a letter} in April to his fellow Marines. ``But I am also mindful of
the feelings of pain and rejection of those who inherited the cultural
memory and present effects of the scourge of slavery in our country.''

The rule announced on Friday for the first time articulated in detail
what sorts of displays would be prohibited at office buildings, naval
vessels, hangars, ready rooms, conference rooms, individual offices,
cubicles, tool and equipment rooms, workshops, as well as other areas.

Among other items, the ban includes posters and flags depicting the
Confederate battle flag. The order allows for inspections to take place
and directs that any nonconforming displays be removed.

It was not immediately clear when the directive would be carried out. A
Marine Corps representative could not be reached on Saturday night.

The directive said that displays in which the Confederate battle flag
was depicted, but not the main focus of the display, were exempted from
the ban. This could apply to a presentation of the flag in a work of art
or an educational or historical display depicting a Civil War battle,
for instance.

Lecia Brooks, chief workplace transformation officer at the Southern
Poverty Law Center, said on Saturday that the Confederate battle flag
was a symbol of white supremacy and the enslavement of black people.

``We urge the other branches of the U.S. military to follow the U.S.
Marine Corps' example, including the National Guard, state and local law
enforcement agencies, and other law enforcement branches of the federal
government and agencies governed by the Department of Homeland Security,
and remove all symbols of the Confederacy,'' she said.

Advertisement

\protect\hyperlink{after-bottom}{Continue reading the main story}

\hypertarget{site-index}{%
\subsection{Site Index}\label{site-index}}

\hypertarget{site-information-navigation}{%
\subsection{Site Information
Navigation}\label{site-information-navigation}}

\begin{itemize}
\tightlist
\item
  \href{https://help.nytimes.com/hc/en-us/articles/115014792127-Copyright-notice}{©~2020~The
  New York Times Company}
\end{itemize}

\begin{itemize}
\tightlist
\item
  \href{https://www.nytco.com/}{NYTCo}
\item
  \href{https://help.nytimes.com/hc/en-us/articles/115015385887-Contact-Us}{Contact
  Us}
\item
  \href{https://www.nytco.com/careers/}{Work with us}
\item
  \href{https://nytmediakit.com/}{Advertise}
\item
  \href{http://www.tbrandstudio.com/}{T Brand Studio}
\item
  \href{https://www.nytimes.com/privacy/cookie-policy\#how-do-i-manage-trackers}{Your
  Ad Choices}
\item
  \href{https://www.nytimes.com/privacy}{Privacy}
\item
  \href{https://help.nytimes.com/hc/en-us/articles/115014893428-Terms-of-service}{Terms
  of Service}
\item
  \href{https://help.nytimes.com/hc/en-us/articles/115014893968-Terms-of-sale}{Terms
  of Sale}
\item
  \href{https://spiderbites.nytimes.com}{Site Map}
\item
  \href{https://help.nytimes.com/hc/en-us}{Help}
\item
  \href{https://www.nytimes.com/subscription?campaignId=37WXW}{Subscriptions}
\end{itemize}
