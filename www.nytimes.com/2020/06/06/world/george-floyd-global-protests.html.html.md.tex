Sections

SEARCH

\protect\hyperlink{site-content}{Skip to
content}\protect\hyperlink{site-index}{Skip to site index}

\href{https://www.nytimes.com/section/world}{World}

\href{https://myaccount.nytimes.com/auth/login?response_type=cookie\&client_id=vi}{}

\href{https://www.nytimes.com/section/todayspaper}{Today's Paper}

\href{/section/world}{World}\textbar{}Huge Crowds Around the Globe March
in Solidarity Against Police Brutality

\url{https://nyti.ms/3dzgfi6}

\begin{itemize}
\item
\item
\item
\item
\item
\end{itemize}

\href{https://www.nytimes.com/news-event/george-floyd-protests-minneapolis-new-york-los-angeles?action=click\&pgtype=Article\&state=default\&region=TOP_BANNER\&context=storylines_menu}{Race
and America}

\begin{itemize}
\tightlist
\item
  \href{https://www.nytimes.com/2020/07/26/us/protests-portland-seattle-trump.html?action=click\&pgtype=Article\&state=default\&region=TOP_BANNER\&context=storylines_menu}{Protesters
  Return to Other Cities}
\item
  \href{https://www.nytimes.com/2020/07/24/us/portland-oregon-protests-white-race.html?action=click\&pgtype=Article\&state=default\&region=TOP_BANNER\&context=storylines_menu}{Portland
  at the Center}
\item
  \href{https://www.nytimes.com/2020/07/23/podcasts/the-daily/portland-protests.html?action=click\&pgtype=Article\&state=default\&region=TOP_BANNER\&context=storylines_menu}{Podcast:
  Showdown in Portland}
\item
  \href{https://www.nytimes.com/interactive/2020/07/16/us/black-lives-matter-protests-louisville-breonna-taylor.html?action=click\&pgtype=Article\&state=default\&region=TOP_BANNER\&context=storylines_menu}{45
  Days in Louisville}
\end{itemize}

Advertisement

\protect\hyperlink{after-top}{Continue reading the main story}

Supported by

\protect\hyperlink{after-sponsor}{Continue reading the main story}

\hypertarget{huge-crowds-around-the-globe-march-in-solidarity-against-police-brutality}{%
\section{Huge Crowds Around the Globe March in Solidarity Against Police
Brutality}\label{huge-crowds-around-the-globe-march-in-solidarity-against-police-brutality}}

Tens of thousands turned out in Australia, Britain, France, Germany and
other nations in support of U.S. protests against the death of George
Floyd, while denouncing racism in their own countries.

\includegraphics{https://static01.nyt.com/images/2020/06/06/world/06unrest-global01sub/merlin_173255814_9331f2e7-bcb4-482c-b23f-616772e5b691-articleLarge.jpg?quality=75\&auto=webp\&disable=upscale}

\href{https://www.nytimes.com/by/damien-cave}{\includegraphics{https://static01.nyt.com/images/2018/10/08/multimedia/author-damien-cave/author-damien-cave-thumbLarge.png}}\href{https://www.nytimes.com/by/livia-albeck-ripka}{\includegraphics{https://static01.nyt.com/images/2018/06/12/multimedia/author-livia-albeck-ripka/author-livia-albeck-ripka-thumbLarge.png}}\href{https://www.nytimes.com/by/iliana-magra}{\includegraphics{https://static01.nyt.com/images/2019/09/17/reader-center/author-iliana-magra/author-iliana-magra-thumbLarge-v2.png}}

By \href{https://www.nytimes.com/by/damien-cave}{Damien Cave},
\href{https://www.nytimes.com/by/livia-albeck-ripka}{Livia Albeck-Ripka}
and \href{https://www.nytimes.com/by/iliana-magra}{Iliana Magra}

\begin{itemize}
\item
  Published June 6, 2020Updated June 9, 2020
\item
  \begin{itemize}
  \item
  \item
  \item
  \item
  \item
  \end{itemize}
\end{itemize}

SYDNEY, Australia --- They were warned by Prime Minister Scott Morrison
of Australia against attending Black Lives Matter marches on Saturday
because of the coronavirus risk, but tens of thousands would not be
deterred.

The health minister in Britain pleaded with residents not to gather for
similar demonstrations in cities like London, Manchester and Birmingham
to stop the virus's spread. But throngs showed up anyway --- despite the
cold weather, the spitting rain and warnings by the police that mass
gatherings would violate the rule that only six people from different
households could gather outside during the pandemic.

From Paris to Berlin --- as in demonstrations this past week in Japan,
Sweden and Zimbabwe --- people around the world once again turned out in
solidarity with Americans protesters calling for justice in the death of
an African-American man, George Floyd, at the hands of the police in
Minneapolis.

They showed up in circumstances that made it almost impossible to adhere
to social distancing regulations. Tens of thousands flowed to Parliament
Square in London on Saturday afternoon, shouting anti-racist slogans and
carrying signs paying homage to Mr. Floyd, 46, who died after
\href{https://www.nytimes.com/2020/05/27/us/george-floyd-minneapolis-death.html?searchResultPosition=4}{a
white police officer held his knee to Mr. Floyd's neck} in Minneapolis
on May 25.

Though most people were wearing masks, their collective chants could be
heard loud and clear: ``George Floyd,'' ``Black lives matter,'' ``No
justice, no peace,'' they said. Footage showed hundreds streaming toward
the U.S. Embassy on foot and by car,
\href{https://twitter.com/emmabowds/status/1269294466042658817}{hooting
and honking horns.}

\includegraphics{https://static01.nyt.com/images/2020/06/06/world/06unrest-global02/merlin_173259147_ac333ffb-5446-499d-8b0a-7a08de4dd2e4-articleLarge.jpg?quality=75\&auto=webp\&disable=upscale}

Silence fell among the crowds for about one minute when the protesters
all took a knee on the wet ground, and many raised their fists in the
air.

\href{https://www.nytimes.com/2020/06/03/world/americas/global-protests-george-floyd.html?searchResultPosition=2}{The
world has been transfixed} by the unrest in the United States amid video
footage of brutal clashes between the police and protesters, along with
episodes of looting and destruction --- though many cities held peaceful
marches and vigils in Mr. Floyd's memory. The global demonstrations,
continuing for a week now, were inspired by the demonstrations in the
United States to call for an end to racism and police brutality in their
own countries.

On Tuesday, almost three dozen Kenyans and Americans gathered outside
the U.S. Embassy in Nairobi in support of demonstrations in the United
States. The protesters wore masks, chanted slogans like ``Down with
police brutality'' and carried posters that read, ``Silence is
violence.''

African leaders, including Ghana's president,
\href{https://twitter.com/NAkufoAddo/status/1267537316391849984}{Nana
Akufo-Addo, tweeted} that he hoped the ``tragic death of George Floyd
will inspire a lasting change in how America confronts head on the
problems of hate and racism.''

Stephany Zoo, an American living in Kenya who took part in the protests
outside the embassy in Nairobi, said the marches were important in order
to agitate for change. ``I expect better from the U.S.,'' she said,
adding that the United States was supposed to be ``this shining
lighthouse of democracy and equality.''

Large crowds gathered on Saturday in cities and towns
\href{https://www.nytimes.com/2020/06/06/us/george-floyd-protests.html}{for
the 11th straight day in the United States}, denouncing police brutality
and seeking reforms after a long line of deaths of African-Americans at
the hands of law enforcement, including Mr. Floyd in Minneapolis,
\href{https://www.nytimes.com/2020/05/21/us/fbi-louisville-shooting.html?searchResultPosition=2}{Breonna
Taylor in Louisville, Ky}., and others. In the nation's capital,
peaceful rallies took place near the White House, the U.S. Capitol, the
Lincoln Memorial and other iconic locations.

In Paris, the authorities barred people from gathering in front of the
United States Embassy, but thousands protested there anyway in the late
afternoon, as well as near the Eiffel Tower, echoing a protest this past
week that drew nearly 20,000 people in memory of Adama Traoré, a
Frenchman who died in police custody in 2016.

In Australia, even as Mr. Morrison, the prime minister, advised against
attending the Black Lives Matter marches on Saturday for fear of new
outbreaks in a country that has managed to beat back the virus, huge
crowds turned out in cities like Sydney and Melbourne, calling for an
end to systemic racism and Aboriginal deaths in police custody. Anger
has grown for years over
\href{https://www.theguardian.com/australia-news/2020/jun/06/aboriginal-deaths-in-custody-434-have-died-since-1991-new-data-shows}{the
deaths}: There have been more than 400 such fatalities since 1991,
without a single officer having been convicted.

Despite warnings that they could be fined for defying coronavirus
restrictions, protesters showed up wearing masks, holding signs with
slogans like ``Australia is not innocent'' and shouting, ``I can't
breathe'' --- echoing Mr. Floyd's plea.

In Melbourne, many held Indigenous flags, signs and clap sticks, which
they struck in solidarity, chanting, ``I can't breathe'' --- also the
final plea of an Aboriginal man, David Dungay, who died at the hands of
the Australian police in 2015.

Image

In Melbourne.Credit...William West/Agence France-Presse --- Getty Images

Police officers surrounded many of the Australian protests but did not
engage with the demonstrators, at least initially. In many cases, the
protests grew larger than organizers had expected.

The intensity, scale and scope of the protests seemed to dwarf anything
Australia has seen in terms of mobilization around the issue of race
since at least 2000,
\href{https://www.indigenous.gov.au/news-and-media/stories/20-years-road-reconciliation-continues}{when
250,000 people marched for reconciliation} over Australia's treatment of
its Aboriginal people.

In Sydney, the protests on Saturday came together under a cloud of
tension and uncertainty. After a court ruled late Friday that the
marches could not be held, citing the need for social distancing in
light of the coronavirus pandemic, organizers appealed. And a higher
court's last-minute decision on Saturday let the demonstration proceed
--- just minutes before it was to start.

Image

Protesting in Sydney.Credit...James Gourley/EPA, via Shutterstock

Among the crowds, anger was mixed with resolve.

``We will not be silenced,'' an organizer shouted to the crowd of
thousands, as helicopters buzzed overhead. ``We will be coming to your
streets until you get it right.''

Many of the Sydney rally's supporters suggested that the attempt to
cancel Saturday's event had been an example of racism. They noted that
gatherings of mostly white Australians, such as at farmers' markets,
seemed to have continued without interruption.

Within minutes of the rally's start, however, the focus shifted to the
subject of deaths at the hands of police in Australia.

``No justice, no peace, no racist police,'' the crowd shouted. People
then marched through the usually bustling center of the city.

``I've never seen so many emotions expressed by so many people in my
whole lifetime of protesting,'' said Margret Campbell, 70, an Aboriginal
elder whose heritage is Dunghutti, who watched from the steps where
organizers spoke. But she added, ``What really matters is what happens
when people have to make decisions --- how will they vote, how will they
keep it up?''

Indigenous activists spoke in somber but impassioned tones to the
heaving crowd in Melbourne, where protesters held signs with the names
and photographs of people who had died in police custody.

``You're on our land,'' said Kaya Nicholson, a 17-year-old Indigenous
organizer. She told protesters that while she appreciated their support
in this moment of global unrest around race, it was crucial that
Australians continue to speak up for Indigenous people.

``Don't just support Black Lives Matter because it's trending,'' she
said.

Ron Baird, an African-American living and teaching in Australia, drew
parallels between Australia's troubles and the crisis in the United
States, disputing the prime minister's suggestion this past week that
\href{https://www.google.com.au/amp/s/amp.theguardian.com/australia-news/2020/jun/04/morrison-says-australia-should-not-import-black-lives-matter-protests-after-deaths-in-custody-rally}{Australians
were ``importing'' problems} that had not existed in the country.

``No Mr. Morrison, Australia is not the United States, but Australia
does have its own long, dark, brutal past of oppression,'' Dr. Baird
said.

In Germany, calls went out on social media for protesters to take to the
streets to honor Mr. Floyd, after a week of demonstrations in cities
like Hamburg and Frankfurt.
\href{https://twitter.com/dwnews/status/1269120077510500353}{In an
interview with the German public broadcaster DW News}, Chancellor Angela
Merkel called the death of Mr. Floyd ``a murder.''

Image

On Alexanderplatz in Berlin.Credit...Fabrizio Bensch/Reuters

``It is racist,'' she said, adding, ``But I trust in the power of
democracy in the United States, that they are able to come through this
difficult situation.''

In Britain, the health minister, Mr. Hancock, cited Covid-19 on Friday
in warning protesters not to gather in groups of more than six people
this weekend. ``Like so many I'm appalled by the death of George
Floyd,'' he said at a news briefing. ``But we are still facing a health
crisis.''

His warning came as the infection rate increased in the northwest and
southwest of England, health officials said, with the R number rising to
1 or above it.

Laurence Taylor, the deputy assistant commissioner of the Metropolitan
Police, the main force in London, told the BBC that the planned
demonstrations across the country were simply ``unlawful.''

But social media users noted on Twitter that the government was asking
people not to protest while opening up Parliament, asking some to go to
work and urging others to take public transport and send their children
to school. Still others added that Downing Street had lost its
credibility when it
\href{https://www.nytimes.com/2020/05/26/world/europe/boris-johnson-dominic-cummings-uk.html}{rallied
behind Dominic Cummings}, a top aide to Prime Minister Boris Johnson who
was accused of flouting social distance rules during the lockdown.

Rahma Mohammad, a 37-year-old history teacher, said things needed to
change systemically. ``It's been discussed historically, but it's never
been resolved,'' she said.

Standing next to her, Victoria Weakerly, 42, held a placard that read:
``I'm social distancing from my white privilege.'' She said that being
at the protest and supporting the Black Lives Matter movement was more
important than the coronavirus.

``I feel safe here among these people,'' she said.

Damien Cave reported from Sydney; Livia Albeck-Ripka from Melbourne,
Australia; and Iliana Magra from London. Elian Peltier contributed
reporting from Paris; Abdi Latif Dahir from Nairobi, Kenya; and Yonette
Joseph from London.

Advertisement

\protect\hyperlink{after-bottom}{Continue reading the main story}

\hypertarget{site-index}{%
\subsection{Site Index}\label{site-index}}

\hypertarget{site-information-navigation}{%
\subsection{Site Information
Navigation}\label{site-information-navigation}}

\begin{itemize}
\tightlist
\item
  \href{https://help.nytimes.com/hc/en-us/articles/115014792127-Copyright-notice}{©~2020~The
  New York Times Company}
\end{itemize}

\begin{itemize}
\tightlist
\item
  \href{https://www.nytco.com/}{NYTCo}
\item
  \href{https://help.nytimes.com/hc/en-us/articles/115015385887-Contact-Us}{Contact
  Us}
\item
  \href{https://www.nytco.com/careers/}{Work with us}
\item
  \href{https://nytmediakit.com/}{Advertise}
\item
  \href{http://www.tbrandstudio.com/}{T Brand Studio}
\item
  \href{https://www.nytimes.com/privacy/cookie-policy\#how-do-i-manage-trackers}{Your
  Ad Choices}
\item
  \href{https://www.nytimes.com/privacy}{Privacy}
\item
  \href{https://help.nytimes.com/hc/en-us/articles/115014893428-Terms-of-service}{Terms
  of Service}
\item
  \href{https://help.nytimes.com/hc/en-us/articles/115014893968-Terms-of-sale}{Terms
  of Sale}
\item
  \href{https://spiderbites.nytimes.com}{Site Map}
\item
  \href{https://help.nytimes.com/hc/en-us}{Help}
\item
  \href{https://www.nytimes.com/subscription?campaignId=37WXW}{Subscriptions}
\end{itemize}
