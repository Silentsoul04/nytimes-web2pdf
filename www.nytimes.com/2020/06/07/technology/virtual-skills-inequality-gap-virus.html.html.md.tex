Sections

SEARCH

\protect\hyperlink{site-content}{Skip to
content}\protect\hyperlink{site-index}{Skip to site index}

\href{https://www.nytimes.com/section/technology}{Technology}

\href{https://myaccount.nytimes.com/auth/login?response_type=cookie\&client_id=vi}{}

\href{https://www.nytimes.com/section/todayspaper}{Today's Paper}

\href{/section/technology}{Technology}\textbar{}Gaining Skills Virtually
to Close the Inequality Gap

\url{https://nyti.ms/2UkGEbV}

\begin{itemize}
\item
\item
\item
\item
\item
\end{itemize}

\href{https://www.nytimes.com/spotlight/at-home?action=click\&pgtype=Article\&state=default\&region=TOP_BANNER\&context=at_home_menu}{At
Home}

\begin{itemize}
\tightlist
\item
  \href{https://www.nytimes.com/2020/07/28/books/time-for-a-literary-road-trip.html?action=click\&pgtype=Article\&state=default\&region=TOP_BANNER\&context=at_home_menu}{Take:
  A Literary Road Trip}
\item
  \href{https://www.nytimes.com/2020/07/29/magazine/bored-with-your-home-cooking-some-smoky-eggplant-will-fix-that.html?action=click\&pgtype=Article\&state=default\&region=TOP_BANNER\&context=at_home_menu}{Cook:
  Smoky Eggplant}
\item
  \href{https://www.nytimes.com/2020/07/27/travel/moose-michigan-isle-royale.html?action=click\&pgtype=Article\&state=default\&region=TOP_BANNER\&context=at_home_menu}{Look
  Out: For Moose}
\item
  \href{https://www.nytimes.com/interactive/2020/at-home/even-more-reporters-editors-diaries-lists-recommendations.html?action=click\&pgtype=Article\&state=default\&region=TOP_BANNER\&context=at_home_menu}{Explore:
  Reporters' Obsessions}
\end{itemize}

Advertisement

\protect\hyperlink{after-top}{Continue reading the main story}

Supported by

\protect\hyperlink{after-sponsor}{Continue reading the main story}

\hypertarget{gaining-skills-virtually-to-close-the-inequality-gap}{%
\section{Gaining Skills Virtually to Close the Inequality
Gap}\label{gaining-skills-virtually-to-close-the-inequality-gap}}

Successful job-training programs for low-income young people have long
been held in person. Can a virtual ladder still be a path to the middle
class?

\includegraphics{https://static01.nyt.com/images/2020/06/08/business/00JPvirus-skills1-print/merlin_172778121_63b158c5-ad5f-48e9-8ea3-7ab9a08df082-articleLarge.jpg?quality=75\&auto=webp\&disable=upscale}

\href{https://www.nytimes.com/by/steve-lohr}{\includegraphics{https://static01.nyt.com/images/2018/02/20/multimedia/author-steve-lohr/author-steve-lohr-thumbLarge.jpg}}

By \href{https://www.nytimes.com/by/steve-lohr}{Steve Lohr}

\begin{itemize}
\item
  June 7, 2020
\item
  \begin{itemize}
  \item
  \item
  \item
  \item
  \item
  \end{itemize}
\end{itemize}

When the word came in early March, Ashley Russell recalled his first
reaction as ``sheer astonishment.'' Within a week, Year Up, a nonprofit
job-training program in cities across the country, would go entirely
online after being held entirely in person.

The promise of \href{https://www.yearup.org/}{Year Up} is that an
intensive regimen of technical and professional training can be an
on-ramp to a middle-class career. ``You can change your life,'' said Mr.
Russell, an instructor at Year Up in Chicago.

Trying to translate life-changing experiences to computer screens and
video classes is the lockdown-induced experiment now being conducted by
Year Up and other programs designed for disadvantaged Americans.

The future of these programs is in doubt at a time when they would seem
to be needed more than ever.
\href{https://www.nytimes.com/2020/06/04/business/economy/coronavirus-unemployment-claims.html}{Tens
of millions of Americans have lost their jobs} in the last few months
because of the coronavirus pandemic, while the recent
\href{https://www.nytimes.com/news-event/george-floyd-protests-minneapolis-new-york-los-angeles?action=click\&pgtype=Article\&state=default\&module=styln-george-floyd\&variant=show\&region=TOP_BANNER\&context=storylines_menu}{unrest
over the death of George Floyd}, an African-American man killed in
police custody in Minneapolis, has been intensified by
\href{https://www.nytimes.com/interactive/2020/06/11/multimedia/coronavirus-new-york-inequality.html}{persistent
income inequality} and the lack of opportunity for many.

Pointing to those issues, Gerald Chertavian, founder and chief executive
of Year Up, asked, ``As we rebuild and recover, will it be in a way that
is more economically inclusive --- that brings more Americans along?''

Mr. Chertavian and the leaders of other programs, which operate in
dozens of American cities, from Seattle to Miami, said they saw
opportunity beyond their immediate challenges. The forced march online,
they said, has triggered a drastic rethinking across the
education-to-employment field and will most likely bring lasting change
--- and perhaps open the door to significant expansion.

Program directors spoke of a post-pandemic model, in a year or so, in
which half or even three-quarters of instruction and coaching would be
done virtually, and the remainder face-to-face.

``The way our kind of work force development is done has changed
permanently,'' said Plinio Ayala, chief executive of
\href{https://perscholas.org/}{Per Scholas}, a skills nonprofit based in
the South Bronx.

The long-held view was that hands-on personal attention was necessary to
lift up students who have to fill gaps in their education, overcome life
obstacles and then make their way in the corporate world.

But Year Up and others say they have found that much more of their
training can be done effectively online than they expected. While the
attrition rates for students are higher, they are only slightly higher,
they said.

The few dozen nonprofit, upward-mobility programs share certain
characteristics. They cater mainly to people in their 20s and 30s. They
have forged close ties with local employers and focus on skills that are
in demand by companies, particularly in technology but also in health
care, finance and advanced manufacturing.

The programs rely on charitable, corporate and some government funding.
Some have a national reach, including Year Up, Per Scholas,
\href{https://www.npower.org/}{NPower} and
\href{https://usa.generation.org/}{Generation}, and some are local, like
\href{https://questsa.org/}{Project Quest} in San Antonio and
\href{https://www.nytimes.com/2019/03/15/business/pursuit-tech-jobs-training.html}{Pursuit}
in Queens.

But most remain small. Year Up, one of the largest, had 2,900 graduates
last year.

\includegraphics{https://static01.nyt.com/images/2020/06/08/business/00JPvirus-skills2-print/merlin_168350493_3ededf62-803c-4d54-8a19-6998307424d7-articleLarge.jpg?quality=75\&auto=webp\&disable=upscale}

Moving a large share of training online would remove barriers to
expansion by bringing down costs, requiring less classroom space and
reaching more students, program leaders said.

``It could accelerate the growth and increase the importance of this
whole category of programs,'' said Norman Atkins, who is leading a
research project on education-to-employment initiatives for
\href{http://americaachieves.org/}{America Achieves}, a nonprofit that
advises foundations on education policy.

Year Up, founded two decades ago, is a full-year program with six months
of course training and a six-month apprenticeship at a company.

The program stands out for the size of the jump in income it has
delivered for its graduates, results that have been verified by
independent assessments.

Before Year Up, its students' annual earnings ranged from \$9,000 to
\$15,000, depending on where they lived in the country. The graduates
typically land jobs that pay from \$35,000 to \$55,000, with the
national average \$42,000. Companies that have consistently hired from
Year Up include Accenture, JPMorgan Chase, Salesforce, LinkedIn, Bank of
America and American Express.

Typically, 75 percent of the graduates are employed within four months.
Job placements have slowed this year but by less than 10 percent so far,
the program said.

Year Up conducted some online experiments before, but tentative digital
steps became a survival sprint in March. It's unclear how much
coursework will eventually be done remotely, though Mr. Chertavian
estimated it would be half or more. ``And there's a real opportunity for
us to scale up and reach more people,'' he said.

The coronavirus shock to the economy has hit many Year Up students. They
receive modest biweekly stipends, but most depend on the support of
family members or friends or income from side jobs while they are in the
full-time program.

Estefan Salgado, a Year Up intern at JP Morgan Chase, lives in the South
Bronx with his wife, Carmen, and their two young children. After his
wife was laid off in March, Mr. Salgado got \$150 from Year Up's
Covid-19 impact fund to buy groceries and pay bills.

``It really helped me stay in the program,'' said Mr. Salgado, 26, whose
wife recently got a new job as a home health care worker.

For Mr. Russell, a veteran Year Up instructor in Chicago, the move to
online classes had some ``train wreck moments'' getting students set up
with laptops, internet service and video software. But he teaches a
computer-support course, and he said he used the problems encountered by
his 40 students as learning opportunities.

When teaching, Mr. Russell sometimes found that students' interest
strayed as screen fatigue set in. So he shortened his
lecture-and-demonstration sessions to a maximum of 30 minutes, compared
with up to two hours before. He also used the interactive features in
\href{https://www.nytimes.com/2020/04/24/technology/zoom-rivals-virus-facebook-google.html}{Zoom's
video software} to pepper students with frequent questions to monitor
whether his lessons were being absorbed.

With less class time, Mr. Russell is assigning his students short
projects, which they do in teams of five or six. He conducts virtual
``office hours'' for one-on-one mentoring. And he holds open sessions,
where students can ask him any questions they have.

It has gone surprisingly well, Mr. Russell said, but he has misgivings
about what is lost without interacting in person, like the informal
conversations in hallways and over lunch, often about students' personal
lives and challenges.

``We don't teach a subject,'' he said. ``We teach people.''

Image

``They're trying to build you into the best professional person you can
be,'' said Marianna Torres, a student of Year Up.Credit...Lyndon French
for The New York Times

Marianna Torres, 20, went through the in-person coursework at Year Up
last year. The technical training was rigorous and difficult, she said,
but there was another side to the program focused on ``soft skills.''
That curriculum included speaking in public, networking, working in
teams, even how to sit and dress. She was taught to wear neutral colors,
avoid patterns, skirts no shorter than one inch above the knee, and
heels no higher than two inches.

``It was strict but also very supportive,'' said Ms. Torres, who in
January began a six-month internship in digital marketing at Salesforce
in Chicago. ``They're trying to build you into the best professional
person you can be.''

A new nonprofit, \href{https://www.meritamerica.org/}{Merit America},
may be a glimpse of the hybrid future of training programs for the
disadvantaged. It is run by its co-chief executives, Rebecca Taber
Staehelin and Connor Diemand-Yauman, who both previously worked at
Coursera, a large online learning network. The duo said they were
convinced of the potential for online learning but also realized that
online instruction alone didn't really work for the underserved
community.

To overcome that, Merit America combines online training, in-person
small-group meetings and one-to-one coaching. The split is roughly 75
percent online and 25 percent in person. Evening and weekend sessions
are available, so students can hold onto their current jobs and incomes
while completing the program.

The courses began in the fall of 2018, and their early results ---
before lockdowns forced Merit America to move entirely online --- have
been encouraging. Courses in technology support and computer programming
range from eight weeks to five months, and income gains for graduates in
Dallas and Washington have averaged \$18,000.

``It's a combination of tech and touch,'' Mr. Diemand-Yauman said.

Advertisement

\protect\hyperlink{after-bottom}{Continue reading the main story}

\hypertarget{site-index}{%
\subsection{Site Index}\label{site-index}}

\hypertarget{site-information-navigation}{%
\subsection{Site Information
Navigation}\label{site-information-navigation}}

\begin{itemize}
\tightlist
\item
  \href{https://help.nytimes.com/hc/en-us/articles/115014792127-Copyright-notice}{©~2020~The
  New York Times Company}
\end{itemize}

\begin{itemize}
\tightlist
\item
  \href{https://www.nytco.com/}{NYTCo}
\item
  \href{https://help.nytimes.com/hc/en-us/articles/115015385887-Contact-Us}{Contact
  Us}
\item
  \href{https://www.nytco.com/careers/}{Work with us}
\item
  \href{https://nytmediakit.com/}{Advertise}
\item
  \href{http://www.tbrandstudio.com/}{T Brand Studio}
\item
  \href{https://www.nytimes.com/privacy/cookie-policy\#how-do-i-manage-trackers}{Your
  Ad Choices}
\item
  \href{https://www.nytimes.com/privacy}{Privacy}
\item
  \href{https://help.nytimes.com/hc/en-us/articles/115014893428-Terms-of-service}{Terms
  of Service}
\item
  \href{https://help.nytimes.com/hc/en-us/articles/115014893968-Terms-of-sale}{Terms
  of Sale}
\item
  \href{https://spiderbites.nytimes.com}{Site Map}
\item
  \href{https://help.nytimes.com/hc/en-us}{Help}
\item
  \href{https://www.nytimes.com/subscription?campaignId=37WXW}{Subscriptions}
\end{itemize}
