Sections

SEARCH

\protect\hyperlink{site-content}{Skip to
content}\protect\hyperlink{site-index}{Skip to site index}

\href{https://www.nytimes.com/section/books}{Books}

\href{https://myaccount.nytimes.com/auth/login?response_type=cookie\&client_id=vi}{}

\href{https://www.nytimes.com/section/todayspaper}{Today's Paper}

\href{/section/books}{Books}\textbar{}A Teenager Plays With Fire and
Family Secrets in `The Margot Affair'

\url{https://nyti.ms/37dOedG}

\begin{itemize}
\item
\item
\item
\item
\item
\end{itemize}

\href{https://www.nytimes.com/spotlight/at-home?action=click\&pgtype=Article\&state=default\&region=TOP_BANNER\&context=at_home_menu}{At
Home}

\begin{itemize}
\tightlist
\item
  \href{https://www.nytimes.com/2020/07/28/books/time-for-a-literary-road-trip.html?action=click\&pgtype=Article\&state=default\&region=TOP_BANNER\&context=at_home_menu}{Take:
  A Literary Road Trip}
\item
  \href{https://www.nytimes.com/2020/07/29/magazine/bored-with-your-home-cooking-some-smoky-eggplant-will-fix-that.html?action=click\&pgtype=Article\&state=default\&region=TOP_BANNER\&context=at_home_menu}{Cook:
  Smoky Eggplant}
\item
  \href{https://www.nytimes.com/2020/07/27/travel/moose-michigan-isle-royale.html?action=click\&pgtype=Article\&state=default\&region=TOP_BANNER\&context=at_home_menu}{Look
  Out: For Moose}
\item
  \href{https://www.nytimes.com/interactive/2020/at-home/even-more-reporters-editors-diaries-lists-recommendations.html?action=click\&pgtype=Article\&state=default\&region=TOP_BANNER\&context=at_home_menu}{Explore:
  Reporters' Obsessions}
\end{itemize}

Advertisement

\protect\hyperlink{after-top}{Continue reading the main story}

Supported by

\protect\hyperlink{after-sponsor}{Continue reading the main story}

\href{/column/books-of-the-times}{Books of The Times}

\hypertarget{a-teenager-plays-with-fire-and-family-secrets-in-the-margot-affair}{%
\section{A Teenager Plays With Fire and Family Secrets in `The Margot
Affair'}\label{a-teenager-plays-with-fire-and-family-secrets-in-the-margot-affair}}

By \href{https://www.nytimes.com/by/sarah-lyall}{Sarah Lyall}

\begin{itemize}
\item
  June 9, 2020
\item
  \begin{itemize}
  \item
  \item
  \item
  \item
  \item
  \end{itemize}
\end{itemize}

\includegraphics{https://static01.nyt.com/images/2020/06/10/books/09BOOKLEMOINE1/09BOOKLEMOINE1-articleLarge.png?quality=75\&auto=webp\&disable=upscale}

Buy Book ▾

\begin{itemize}
\tightlist
\item
  \href{https://www.amazon.com/gp/search?index=books\&tag=NYTBSREV-20\&field-keywords=The+Margot+Affair+Sana\%C3\%AB+Lemoine}{Amazon}
\item
  \href{https://du-gae-books-dot-nyt-du-prd.appspot.com/buy?title=The+Margot+Affair\&author=Sana\%C3\%AB+Lemoine}{Apple
  Books}
\item
  \href{https://www.anrdoezrs.net/click-7990613-11819508?url=https\%3A\%2F\%2Fwww.barnesandnoble.com\%2Fw\%2F\%3Fean\%3D9781984854438}{Barnes
  and Noble}
\item
  \href{https://www.anrdoezrs.net/click-7990613-35140?url=https\%3A\%2F\%2Fwww.booksamillion.com\%2Fp\%2FThe\%2BMargot\%2BAffair\%2FSana\%25C3\%25AB\%2BLemoine\%2F9781984854438}{Books-A-Million}
\item
  \href{https://bookshop.org/a/3546/9781984854438}{Bookshop}
\item
  \href{https://www.indiebound.org/book/9781984854438?aff=NYT}{Indiebound}
\end{itemize}

When you purchase an independently reviewed book through our site, we
earn an affiliate commission.

One of the worst things about being 17 is the gulf between how much you
feel and how little you know. You mistake your nascent power --- sexual
power, disruptive power --- for real power. You are angry, passionate,
misunderstood, restless for some amorphous thing you cannot define.

The adults in your life either ignore you or, in the case of Margot
Louve, the narrator of Sanaë Lemoine's gorgeous debut novel, ``The
Margot Affair,'' make grand diagnostic pronouncements that amount to
misdirection.

``Your world is still contained, small and intense, and every change to
the status quo feels like a rug is being pulled from under your feet,''
a woman in her late 30s tells Margot. ``When you're older, you'll
experience these things with more distance and forgiveness.''

Like many teenagers, Margot is furious at her parents. Her mother,
Anouk, is a celebrated stage actress with an unsentimental,
unconventional approach to motherhood. (In her most famous role, she
played a woman who kills her three children in the bath.) Margot's
father, Bertrand Lapierre, is the French culture minister, but he is
married to someone else and has never publicly acknowledged the
existence of Margot or of Anouk, his mistress for two decades.

This all sounds very French. It also brings to mind the story of
François Mitterrand, the former French president who had a daughter,
Mazarine, with his own mistress while remaining married to his wife.

But the president's arrangement was more democratic, as Margot points
out in this novel: ``Mitterrand had split his holidays between both
families, the women and children stood together at his funeral.'' Not so
with her own father, she says, whose worlds ``existed on parallel
planes, never intersecting.''

The situation becomes insupportable when Anouk and Margot spy Bertrand's
wife walking in the Luxembourg Gardens, wearing a pair of expensive
Roger Vivier pumps (the kind, Margot notes, popularized by Catherine
Deneuve in ``Belle de Jour''). Anouk is deeply unsettled; Margot has
never thought of Madame Lapierre as a real person before. Desperate to
do something --- to jolt her father into action, maybe even prod him to
leave his wife --- Margot reveals her parentage to a sympathetic
journalist she meets at a party. ``I wanted to crack open our family and
force him into the limelight,'' she says of her father.

Anyone who has read Ian McEwan's ``Atonement,'' in which the misguided
intervention of an overly imaginative teenager ends up ruining two
people's lives, knows that Margot is playing with fire; that the future
she tries to orchestrate could collapse under the weight of too many
unknown variables. But what happens next --- the damage to Margot and
others --- is as startling to the reader as it is to her.

As the novel progresses, Margot falls increasingly in thrall to the only
people who seem interested in her inner life. One is David Perrin, that
friendly journalist she meets and for whom she feels a frisson of
illicit attraction. (He is older and married.)

Image

Sanaë Lemoine, whose debut novel is ``The Margot
Affair.''Credit...Gieves Anderson

The other is David's wife, Brigitte --- beautiful, preternaturally
empathetic and strangely intimate --- who persuades Margot that the
larger truth about her family needs to be told in a memoir that she,
Brigitte, will ghostwrite based on Margot's memories and
interpretations.

``Already then, Brigitte had a capacity to pluck sentences from my mind,
though she was able to echo my ideas in a more sophisticated and assured
manner,'' Margot thinks. (Uh-oh, we think.) ``How not to trust a woman
who articulated my thoughts as if they were her own, without judgment,
who for once sided with me?''

Like the narrator in a Rachel Cusk novel, Margot finds herself captive
to the long, discursive stories from the past that this overly attentive
couple recounts --- stories the reader should pay attention to, because
they help us understand the forces at work here, even if Margot doesn't.

It's no coincidence that Brigitte brings up ``Bonjour Tristesse,''
Françoise Sagan's classic novel of sexual awakening and a young person's
meddling gone awry. ``The Margot Affair'' invites explicit comparisons
to Sagan's book.

``Like her, I thought I understood grown-ups and their games --- I
looked at them from above --- and yet I was often disappointed,''
Brigitte tells Margot, comparing her own younger self to the heroine
(also 17) of ``Bonjour Tristesse.'' ``Once their games had played out, I
inevitably found myself discarded, not having anticipated how my actions
could have unpredictable consequences.''

As she lectures Margot about the ways of the world, Brigitte also draws
the younger woman in, doling out intimate details, eroding appropriate
boundaries, dropping mean offhand remarks. Soon she begins to seem like
one of those toxic men who seduce and destabilize women by alternating
praise and criticism.

And she lulls Margot into a false sense of her own power.

``Something dark grew inside me, spreading like mold,'' Margot says. ``I
was duplicitous, like my parents, and I loved the feeling of controlling
what others knew and didn't know about me.''

Because you can't get away from the subject of appearance when you are a
young woman (maybe especially when you are also French), there is a lot
of talk about it in this novel, and a particular obsession with women's
relationship to food. The adults eye Margot up and down, noting whether
she has gained or lost weight. Brigitte relates how her own mother
bullied her into thinness, giving her ``cotton balls soaked in warm
water to swallow to quell her pangs of hunger,'' telling her that ``the
gnawing feeling of your organs eating themselves from starvation was a
good sensation.''

Even when Margot is at her most misguided, the reader aches for her.
Lemoine, who was born in Paris and lives in New York City, writes in
lush, lyrical prose that perfectly captures the heightened emotion and
confusion of being a young woman with a bruised heart and limited
experience.

Though the book seems to be about an absent father, it's more about a
tricky mother, and about motherhood in general. It asks the ultimate
question about this most complicated of relationships: What will a
mother do for her child?

``You went through a phase of asking me the same question over and over
again,'' Anouk says, recalling Margot's childhood. ``You wanted to know
if I would throw myself in front of a bus to save you. You asked if all
mothers are willing to sacrifice their lives for their children. No
matter what I said, you were never satisfied with my answer.''

Advertisement

\protect\hyperlink{after-bottom}{Continue reading the main story}

\hypertarget{site-index}{%
\subsection{Site Index}\label{site-index}}

\hypertarget{site-information-navigation}{%
\subsection{Site Information
Navigation}\label{site-information-navigation}}

\begin{itemize}
\tightlist
\item
  \href{https://help.nytimes.com/hc/en-us/articles/115014792127-Copyright-notice}{©~2020~The
  New York Times Company}
\end{itemize}

\begin{itemize}
\tightlist
\item
  \href{https://www.nytco.com/}{NYTCo}
\item
  \href{https://help.nytimes.com/hc/en-us/articles/115015385887-Contact-Us}{Contact
  Us}
\item
  \href{https://www.nytco.com/careers/}{Work with us}
\item
  \href{https://nytmediakit.com/}{Advertise}
\item
  \href{http://www.tbrandstudio.com/}{T Brand Studio}
\item
  \href{https://www.nytimes.com/privacy/cookie-policy\#how-do-i-manage-trackers}{Your
  Ad Choices}
\item
  \href{https://www.nytimes.com/privacy}{Privacy}
\item
  \href{https://help.nytimes.com/hc/en-us/articles/115014893428-Terms-of-service}{Terms
  of Service}
\item
  \href{https://help.nytimes.com/hc/en-us/articles/115014893968-Terms-of-sale}{Terms
  of Sale}
\item
  \href{https://spiderbites.nytimes.com}{Site Map}
\item
  \href{https://help.nytimes.com/hc/en-us}{Help}
\item
  \href{https://www.nytimes.com/subscription?campaignId=37WXW}{Subscriptions}
\end{itemize}
