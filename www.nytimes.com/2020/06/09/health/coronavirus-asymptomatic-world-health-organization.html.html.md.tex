Sections

SEARCH

\protect\hyperlink{site-content}{Skip to
content}\protect\hyperlink{site-index}{Skip to site index}

\href{https://www.nytimes.com/section/health}{Health}

\href{https://myaccount.nytimes.com/auth/login?response_type=cookie\&client_id=vi}{}

\href{https://www.nytimes.com/section/todayspaper}{Today's Paper}

\href{/section/health}{Health}\textbar{}In the W.H.O.'s Coronavirus
Stumbles, Some Scientists See a Pattern

\url{https://nyti.ms/3hc5SDc}

\begin{itemize}
\item
\item
\item
\item
\item
\end{itemize}

\href{https://www.nytimes.com/news-event/coronavirus?action=click\&pgtype=Article\&state=default\&region=TOP_BANNER\&context=storylines_menu}{The
Coronavirus Outbreak}

\begin{itemize}
\tightlist
\item
  live\href{https://www.nytimes.com/2020/08/01/world/coronavirus-covid-19.html?action=click\&pgtype=Article\&state=default\&region=TOP_BANNER\&context=storylines_menu}{Latest
  Updates}
\item
  \href{https://www.nytimes.com/interactive/2020/us/coronavirus-us-cases.html?action=click\&pgtype=Article\&state=default\&region=TOP_BANNER\&context=storylines_menu}{Maps
  and Cases}
\item
  \href{https://www.nytimes.com/interactive/2020/science/coronavirus-vaccine-tracker.html?action=click\&pgtype=Article\&state=default\&region=TOP_BANNER\&context=storylines_menu}{Vaccine
  Tracker}
\item
  \href{https://www.nytimes.com/interactive/2020/07/29/us/schools-reopening-coronavirus.html?action=click\&pgtype=Article\&state=default\&region=TOP_BANNER\&context=storylines_menu}{What
  School May Look Like}
\item
  \href{https://www.nytimes.com/live/2020/07/31/business/stock-market-today-coronavirus?action=click\&pgtype=Article\&state=default\&region=TOP_BANNER\&context=storylines_menu}{Economy}
\end{itemize}

Advertisement

\protect\hyperlink{after-top}{Continue reading the main story}

Supported by

\protect\hyperlink{after-sponsor}{Continue reading the main story}

\hypertarget{in-the-whos-coronavirus-stumbles-some-scientists-see-a-pattern}{%
\section{In the W.H.O.'s Coronavirus Stumbles, Some Scientists See a
Pattern}\label{in-the-whos-coronavirus-stumbles-some-scientists-see-a-pattern}}

The agency's advice sometimes lags behind rapidly evolving research into
the coronavirus, experts contend.

\includegraphics{https://static01.nyt.com/images/2020/06/09/science/09VIRUS-WHO1/merlin_170598384_dc7b024d-4b3f-4063-8351-c45dc12f3bc4-articleLarge.jpg?quality=75\&auto=webp\&disable=upscale}

By \href{https://www.nytimes.com/by/apoorva-mandavilli}{Apoorva
Mandavilli}

\begin{itemize}
\item
  Published June 9, 2020Updated July 9, 2020
\item
  \begin{itemize}
  \item
  \item
  \item
  \item
  \item
  \end{itemize}
\end{itemize}

Even as the World Health Organization leads the worldwide response to
the
\href{https://www.nytimes.com/2020/07/09/podcasts/the-daily/asymptomatic-coronavirus-spread.html}{coronavirus}
pandemic, the agency is failing to take stock of rapidly evolving
research findings and to communicate clearly about them, several
scientists warned on Tuesday.

In a news briefing on Monday, a W.H.O. official asserted that
transmission of the coronavirus by people without symptoms is ``very
rare.'' Following concerted pushback from researchers, officials on
Tuesday walked back the claim, saying it was a ``misunderstanding.''

But it is not the first time the W.H.O.'s assessment has seemed to lag
behind scientific opinion.

The agency
\href{https://www.nytimes.com/2020/06/05/health/coronavirus-masks-who.html}{delayed
endorsing masks for the general public until Friday}, claiming there was
too little evidence that they prevented transmission of the virus.
Virtually all scientists and governments have been recommending masks
for months.

The W.H.O. has said repeatedly that small airborne droplets, or
aerosols, are not a significant factor in the pandemic's spread,
although a growing body of evidence suggests that they may be.

``The W.H.O. has been out of step with most of the world on the issue of
droplets and aerosols,'' said Michael Osterholm, an infectious disease
expert at the University of Minnesota.

These scientific disagreements have wide policy implications. Many
countries, including the United States, adopted lockdown strategies
because they recognized that isolating only people who were sick might
not be enough to contain the epidemic.

If the virus is transmitted by small airborne droplets, people will need
to continue to avoid congregating in poorly ventilated spaces, even if
they practice rigorous hand hygiene.

The W.H.O. traditionally has taken a cautious approach to evaluating
scientific evidence. But the pace of research has changed: Now
scientists are rushing to publish preliminary research, even before
their results can be thoroughly vetted by other experts.

The avalanche of findings may bring advances --- like a vaccine --- in
record time. But the onslaught also has led to confusion, even
\href{https://www.nytimes.com/2020/06/04/health/coronavirus-hydroxychloroquine.html}{retractions
of high-profile results}.

``On the one hand, I do want to cut the W.H.O. some slack, because it is
hard to do this in an evolving pandemic,'' said Dr. Ashish Jha, director
of the Harvard Global Health Institute. ``At the same time, we do rely
on the W.H.O. to give us the best scientific data and evidence.''

\hypertarget{latest-updates-global-coronavirus-outbreak}{%
\section{\texorpdfstring{\href{https://www.nytimes.com/2020/08/01/world/coronavirus-covid-19.html?action=click\&pgtype=Article\&state=default\&region=MAIN_CONTENT_1\&context=storylines_live_updates}{Latest
Updates: Global Coronavirus
Outbreak}}{Latest Updates: Global Coronavirus Outbreak}}\label{latest-updates-global-coronavirus-outbreak}}

Updated 2020-08-01T22:08:05.443Z

\begin{itemize}
\tightlist
\item
  \href{https://www.nytimes.com/2020/08/01/world/coronavirus-covid-19.html?action=click\&pgtype=Article\&state=default\&region=MAIN_CONTENT_1\&context=storylines_live_updates\#link-34047410}{The
  U.S. reels as July cases more than double the total of any other
  month.}
\item
  \href{https://www.nytimes.com/2020/08/01/world/coronavirus-covid-19.html?action=click\&pgtype=Article\&state=default\&region=MAIN_CONTENT_1\&context=storylines_live_updates\#link-3ac56579}{Top
  officials work to break impasse over jobless benefit.}
\item
  \href{https://www.nytimes.com/2020/08/01/world/coronavirus-covid-19.html?action=click\&pgtype=Article\&state=default\&region=MAIN_CONTENT_1\&context=storylines_live_updates\#link-25930521}{Thousands
  in Berlin protest Germany's coronavirus measures.}
\end{itemize}

\href{https://www.nytimes.com/2020/08/01/world/coronavirus-covid-19.html?action=click\&pgtype=Article\&state=default\&region=MAIN_CONTENT_1\&context=storylines_live_updates}{See
more updates}

More live coverage:
\href{https://www.nytimes.com/live/2020/07/31/business/stock-market-today-coronavirus?action=click\&pgtype=Article\&state=default\&region=MAIN_CONTENT_1\&context=storylines_live_updates}{Markets}

The W.H.O.'s thinking on asymptomatic transmission does not appear to
have changed much since February, when the W.H.O. China Joint Mission
\href{https://www.who.int/docs/default-source/coronaviruse/who-china-joint-mission-on-covid-19-final-report.pdf}{reported}
that ``the proportion of truly asymptomatic infections is unclear, but
appears to be relatively rare and does not appear to be a major driver
of transmission.''

Studies later estimated this number could be
\href{https://www.nytimes.com/2020/03/31/health/coronavirus-asymptomatic-transmission.html}{as
high as 40 percent}; the current best estimate from the Centers for
Disease Control and Prevention is 35 percent. The research prompted many
countries, including the United States, to endorse use of masks by
everyone.

But on Monday, Dr. Maria Van Kerkhove, the W.H.O.'s technical lead for
coronavirus response, said that ``it still seems to be rare that an
\href{https://twitter.com/mvankerkhove/status/1270081494552281094}{asymptomatic
person actually transmits} onward to a secondary individual.''

Her statement provoked an immediate backlash from scientists, who noted
that study after study had shown transmission of the virus from people
before they ever felt symptoms.

\includegraphics{https://static01.nyt.com/images/2017/01/29/podcasts/the-daily-album-art/the-daily-album-art-articleInline-v2.jpg?quality=75\&auto=webp\&disable=upscale}

\hypertarget{listen-to-the-daily-a-missed-warning-about-silent-coronavirus-infections}{%
\subsubsection{Listen to `The Daily': A Missed Warning About Silent
Coronavirus
Infections}\label{listen-to-the-daily-a-missed-warning-about-silent-coronavirus-infections}}

Why an early scientific report of symptom-free cases went unheeded.

transcript

Back to The Daily

bars

0:00/31:44

-31:44

transcript

\hypertarget{listen-to-the-daily-a-missed-warning-about-silent-coronavirus-infections-1}{%
\subsection{Listen to `The Daily': A Missed Warning About Silent
Coronavirus
Infections}\label{listen-to-the-daily-a-missed-warning-about-silent-coronavirus-infections-1}}

\hypertarget{hosted-by-michael-barbaro-produced-by-asthaa-chaturvedi-and-alexandra-leigh-young-with-help-from-rachelle-bonja-and-edited-by-lisa-tobin}{%
\subsubsection{Hosted by Michael Barbaro; produced by Asthaa Chaturvedi
and Alexandra Leigh Young, with help from Rachelle Bonja; and edited by
Lisa
Tobin}\label{hosted-by-michael-barbaro-produced-by-asthaa-chaturvedi-and-alexandra-leigh-young-with-help-from-rachelle-bonja-and-edited-by-lisa-tobin}}

\hypertarget{why-an-early-scientific-report-of-symptom-free-cases-went-unheeded}{%
\paragraph{Why an early scientific report of symptom-free cases went
unheeded.}\label{why-an-early-scientific-report-of-symptom-free-cases-went-unheeded}}

\begin{itemize}
\item
  michael barbaro\\
  From The New York Times, I'm Michael Barbaro. This is ``The Daily.''
\item
  {[}music{]}\\
  Today: Long before the world understood that seemingly healthy people
  could spread the coronavirus, a doctor in Germany tried to sound the
  alarm. Matt Apuzzo on why that warning was so unwelcome.

  It's Thursday, July 9.
\item
  michael barbaro\\
  Good. OK, so we're going to get started.
\item
  camilla rothe\\
  Very good, OK.
\item
  michael barbaro\\
  So you're recording, right?
\item
  camilla rothe\\
  Yep.
\item
  michael barbaro\\
  You are recording, and I'm recording. So I think we can ---
\end{itemize}

michael barbaro

Matt, tell me about this doctor in Germany, Dr. Camilla Rothe.

matt apuzzo

Yeah, she is a tropical medicine specialist, basically an infectious
disease specialist. She's at the Munich University Hospital. She's at
the infectious disease clinic there. And she's part of this network of
doctors around the world that serve as kind of like an early detection
system.

\begin{itemize}
\tightlist
\item
  camilla rothe\\
  We mainly work with returned travelers, as well as with migrants from
  tropical destinations and who may import even novel pathogens.
\end{itemize}

matt apuzzo

They kind of report back to major health organizations, like, hey, I'm
seeing a weird virus over here, or Ebola case pops up over here, or
here's a weird thing.

\begin{itemize}
\tightlist
\item
  camilla rothe\\
  In Germany, we are responsible for anything exotic.
\end{itemize}

michael barbaro

And so what prompted you to begin talking to her?

matt apuzzo

So January 27, this patient in Munich --- 33-year-old employee from an
auto parts company --- walks into her clinic. And right away it's pretty
clear something weird is going on.

\begin{itemize}
\item
  camilla rothe\\
  He informed us that when he came to work in the morning, he'd been
  told by his boss that a business partner who'd visited the company the
  week before, coming from China, had just phoned. And she'd said that
  on the weekend back home in China, she'd been diagnosed with the novel
  coronavirus infection. And he'd actually been ill over the weekend
  himself. And he'd asked us whether he could be checked for this new
  virus at our institution.
\item
  michael barbaro\\
  And this was January 27, so had the coronavirus been detected in
  Germany?
\item
  camilla rothe\\
  No, not yet.
\item
  michael barbaro\\
  So what were you thinking when this man came into your office?
\item
  camilla rothe\\
  Well, it was January. It was a time of year when there are lots of
  respiratory infections circulating. In fact, it's the peak of the
  influenza season. And he'd been very unspectacularly ill over the
  weekend, so it could have been anything. In contrast, the pictures
  that we received from China by then were pictures of a very serious
  disease --- people being on ventilators, et cetera. So I thought,
  well, I mean, interesting story, but this could be anything. And we
  took a swab from his nasal pharynx and sent it to the lab. And a few
  hours later, I was phoned by the lab and informed that, in fact, the
  test tested positive for the novel virus.
\item
  michael barbaro\\
  Huh. So at this point, you have just been told that you have the first
  confirmed case of the coronavirus in Germany. And based on what you
  knew about the contact that had brought him to you in the first place,
  what were you thinking about that?
\item
  camilla rothe\\
  That was quite a puzzle, because I'd obviously grilled him on the fact
  whether the Chinese business colleague appeared in any way ill. Had
  she coughed, or did she have a runny nose, or did she look ill in any
  way? And actually, he said she had held quite intense business
  workshops and meetings without showing any obvious signs of illness.
  And then on day two, which was the Tuesday, more employees of the
  company came to our clinic. And another three, all of them with very
  minor mild symptoms, were tested positive. That was the point when I
  thought, we need to spread the news to get this out to the world. And
  we contacted The New England Journal of Medicine, and they were
  interested. And it was very rapidly accepted and put online for people
  to read it.
\end{itemize}

michael barbaro

Matt, this paper that Dr. Rothe publishes, what does it find?

matt apuzzo

On the surface, it's a really simple straightforward paper. It just
says, we had this weird case with this guy who tested positive for the
coronavirus, and the person he caught it from didn't appear to have any
symptoms. And that's kind of weird because that's not what we think is
supposed to happen with this disease. So we thought it was important to
just put it out there, and this could have serious ramifications. Just
telling the world. In a nutshell, that's all it really says.

michael barbaro

Well, help me understand that. For people who don't work in the world of
infectious diseases, what is the significance of this man's diagnosis?

matt apuzzo

It's funny. You look back now at the end of January, and it's sort of
like you're looking at another time, another world. We're still trying
to find out what this disease is. And so the assumption kind of was,
well, this is probably going to behave like SARS, because their genetic
cousins. So good chance it's going to spread like that. And the thing
about SARS is, you don't spread it until you are sick. Until you have
symptoms, you are not really contagious.

\begin{itemize}
\item
  camilla rothe\\
  So if a disease behaves like that, it is much more easy to control.
  It's easy to define what a case is, who is a suspect case, someone who
  has symptoms. And if you ask these people to stay at home, you have
  already a good means to contain the virus. Now, if you have a virus
  that behaves differently, like what we had observed, which spreads
  before it even causes symptoms, this is much more difficult to control
  because people would never go for a test. They are not aware they are
  infected. They are mixing with people in the same way that they
  normally do --- with colleagues, with friends, with loved ones. So
  it's far more difficult to contain an infection like that.
\item
  michael barbaro\\
  So this just flew in the face of the common understanding of the virus
  at that point.
\item
  camilla rothe\\
  Absolutely. And then something strange happened that I personally
  don't fully understand until today.
\end{itemize}

matt apuzzo

What Dr. Rothe didn't know was that around 20 minutes away, in this sort
of suburb of Munich, in the regional health office, they were starting
up a command center basically to do all the tracking and the tracing and
all of the stuff that needed to be done. And the doctors there were
working on their own paper that they were going to get published in a
different journal. And so now you've got two separate groups of
scientists writing on the same case for different journals.

michael barbaro

And what did this group 20 minutes away, what did they find in their
paper?

matt apuzzo

It's really, really similar. So whereas Dr. Rothe says, this woman is
not symptomatic --- and she says that because this woman is leading two
days of business meetings and she's not sneezing, she's not coughing.
She's not showing any signs of being fatigued or feverish or in any way
sick. The government doctors, after extensive interviews, they come back
and say, yeah, but we don't think the Chinese patient had no symptoms.
We think she was probably experiencing some symptoms that were so mild
that even she didn't recognize them. And so this dispute became, does
she have no symptoms, or does she have such mild symptoms that even she
doesn't recognize that she's sick?

michael barbaro

OK, so Dr. Rothe has published a paper saying that the patient had no
symptoms. This government agency has now published a paper saying that
she had early, essentially undetectable symptoms. So what's the
significance of that distinction?

matt apuzzo

So this has been the story of my life for the past however many months.
The amount of time I've spent in conversations about the word
``asymptomatic'' or ``pre-symptomatic'' or ``oligosymptomatic'' or any
of these words, right? What does it mean? If you are somebody who
studies diseases, and you are somebody who really wants to understand
the characteristics of this new virus, well, then obviously you want to
know what exactly needs to happen before you can become contagious. Can
you be just a passive carrier? Can you just walk around spreading this
thing and you'll never get sick? Is that a real thing that happens? Or
does it only spread after you get symptoms? Or does it only spread when
you have mild symptoms? That is a real distinction in the scientific
world.

michael barbaro

Right.

matt apuzzo

However, from a practical standpoint of what you're going to tell the
public and how you're going to control this disease, if you wake up in
the morning and you're like, aw man, my neck is kind of stiff, I
probably just slept wrong. And then you go into work and you infect
people, what does it matter whether you are pre-symptomatic, whether
that neck ache was actually an early sign that you were getting sick and
you just didn't recognize it? If your strategy is, if you are sick, stay
home, that all falls apart if you can spread this disease before you
even know you're sick.

michael barbaro

In other words, any version of not feeling sick is a huge danger when it
comes to this virus.

matt apuzzo

Absolutely. Because it means, I don't recognize that I'm a danger to
you. And you don't recognize that you're a danger to me. And I don't
recognize that you're a danger to me. And we all walk around and we
don't know that we can make each other sick.

\begin{itemize}
\tightlist
\item
  camilla rothe\\
  Whether this person is ever going to be symptomatic or not doesn't
  really matter. What the key message is, you can infect other people
  without knowing that you're infected. I think the somehow sad thing is
  that this semantic debate --- which is OK between scientists and so
  on, but it's slightly splitting hairs --- this debate somehow obscured
  the message we wanted to send out. And was somehow misleading, because
  it's led us away from the core message to say, guys, keep your eyes
  open. This virus may spread without people knowing.
\end{itemize}

matt apuzzo

So this all would have been a kind of academic discussion between two
groups of doctors in Germany. But at the beginning of February, a couple
of days after Dr. Rothe's paper came out, this thing completely
escalated.

{[}music{]}

And what happened was, Science Magazine, a very respected journal, wrote
a story in which the German national health official said, Dr. Rothe's
paper is flawed. She never interviewed the woman. We don't think she was
asymptomatic. We do think she had symptoms. And now suddenly, this issue
that might have otherwise been a very academic debate is now front and
center in the national discussion over what exactly are the
characteristics of this new virus.

\begin{itemize}
\tightlist
\item
  camilla rothe\\
  We were accused of, how can you claim someone is asymptomatic when you
  haven't talked to him? So if you like, this is formally a correct
  accusation, so the correct title could have been pre-symptomatic
  because the patient then developed symptoms at some point. But that
  was a slightly misguiding debate we were somehow sucked into then.
\end{itemize}

michael barbaro

And Matt, as best you can tell, is there validity to this critique from
the government scientists of Dr. Rothe's paper and her findings?

matt apuzzo

Yeah, I mean, I think there's definitely a fair critique that she should
have interviewed the Chinese patient before asserting that she had no
symptoms.

\begin{itemize}
\tightlist
\item
  camilla rothe\\
  Which, by the way, wouldn't have been our role because we are
  physicians. It's none of our business if anyone should have spoken to
  her, then the public health authorities of Bavaria. And they did. And
  they kind of summarized their phone call to say, she didn't have any
  symptoms while she was in Munich. All she had was what she already
  knew as a feeling of jet lag and, well, the way you feel after a long
  distance flight. And she herself had not noticed anything abnormal to
  the situation until the Thursday when she'd returned to Shanghai and
  she fell ill, when she had chills and fever and cough and all that.
\end{itemize}

matt apuzzo

So as if this debate couldn't get any bigger, it's now going to go
totally global because the world's leading health organization, the
W.H.O., is about to weigh in.

{[}music{]}

michael barbaro

We'll be right back.

matt apuzzo

So the morning of February 4, Dr. Sylvie Briand from the World Health
Organization tweets the science article, and she says, ``It is important
to differentiate asymptomatic from pre-symptomatic transition. 2019-nCoV
study claiming new coronavirus can be transmitted by people without
symptoms was flawed.'' And so now everybody who's on the frontlines of
this discussion is now basically saying Dr. Rothe got it wrong.

\begin{itemize}
\tightlist
\item
  camilla rothe\\
  And that, of course, was disappointing in a way to see that even the
  highest-ranking somehow health authority didn't get a very simple
  clinical message, but also got lost in semantics.
\end{itemize}

michael barbaro

Matt, why did the W.H.O. take the step of publicly disputing and
criticizing Dr. Rothe's finding?

matt apuzzo

I think there's a couple ways to look at it. And one is that if you are
the World Health Organization, and you jump in with two feet into this
idea that this disease can spread without symptoms, it is a seismic
change in the way we think about Covid-19, and has massive ramifications
for public health policy in every country in the world. So of course,
they need to be cautious. They can't just go, oh my god, here's an
observation by one doctor with one patient, and we're going to change
the world's policies based on that. That's unrealistic. But what's
really confusing about all this is it didn't take very long until it
wasn't just Dr. Rothe in Munich. Because the Bavarian health authorities
get genetic information back, and they find genetic proof that it did
spread before symptom onset in two other patients. And so now it's not
just Dr. Rothe saying, hey, I saw something weird. Now we have mounting
evidence from this cluster saying, it's pretty darn clear this is
happening. And so now you really do wonder, why was the response from
the World Health Organization, we don't think this is a big deal? And
not, boy, the evidence is growing, we're not there yet, but we're taking
this really seriously. And we should maybe be start thinking about how
we would adapt our policies if this really catches on.

\begin{itemize}
\tightlist
\item
  camilla rothe\\
  I would have expected a very neutral and curious way and an open way,
  and to at least take into consideration that this virus might behave
  different than the other SARS virus that we knew. And that somehow
  didn't happen. I don't understand why it wasn't. I still don't
  understand. Maybe one day someone will be able to explain to me. I
  don't know.
\end{itemize}

matt apuzzo

I talked to a lot of doctors about this. And there are many who say that
you have to look at this from kind of a stark public policy standpoint.
This is early to mid February. If you tell the world that this thing can
spread before people are symptomatic, before they even know they're
sick, then the next question is, OK, so what do we do about it? We don't
have enough testing. We don't have a contact tracing capability to
handle this, and we don't have P.P.E. for everybody. What do we do?
We've talked to public health officials in other countries who said,
yeah, looking back, we probably could have said, this is looking more
and more like a possibility. But that's a scary place to be if you don't
have an answer for what you're supposed to do next, and that's kind of
where we found ourselves.

michael barbaro

I mean, I just want to wrap my head around this. Because what you seem
to be saying is that there's a possibility that embracing this finding
is just too frightening for the people of the W.H.O., because of what it
would mean for policymakers in every country of the world. But isn't
that the job of the World Health Organization to sometimes scare the
crap out of people, even if there's no logical solution to the scare,
because they need to know?

matt apuzzo

So the W.H.O. says they definitely did not do that. That is not what
happened. But this issue of should the World Health Organization or
other public health officials be scaring the crap out of people, I mean,
I get that. But I think most people would tell you no. Because there's a
huge danger in telling people, this is the big one, this is it. Because
the vast majority of alerts aren't the big one. You need people to take
their advice seriously and rationally and not feel like, oh, here comes
another alert. And so it's like they got to constantly straddle this
line between, I need you to hear me and take this seriously, but I can't
also get crazy and say, oh my god, oh my god, oh my god, this is the
one, tbjs is the one. Because if it's not the one --- and most aren't
--- then the next time, you're just not going to listen.

{[}music{]}

michael barbaro

Matt, what is the implication on the ground of organizations like the
W.H.O. resisting this idea that there is symptomless spread? What does
that actually mean throughout the world?

matt apuzzo

Well, I mean, so I'm in Belgium. And here's a practical example. Belgium
locked down nursing homes and said, you can't visit if you're sick. And
thousands of people in nursing homes died. And they think that
symptomless visitors and symptomless care workers brought the disease
in, and they just had no idea that was even a possibility. We had the
Diamond Princess cruise off the coast of Japan, where one of the reasons
that people were allowed to mix and mingle and go to the buffet, even
after a former passenger tested positive, was because, well, we don't
think he was symptomatic when he was on board. And then February 29, we
get a tweet from the U.S. surgeon general, all caps: ``Seriously,
people, stop buying masks. They are not effective in preventing the
general public from catching coronavirus.'' And it's hard to imagine the
surgeon general weighing in like that if there was kind of a growing
acceptance in the medical community that, boy, this might actually be
spreading before symptoms.

michael barbaro

And of course, now we know that symptomless spread can be curbed, and a
primary way to curb it is masks.

matt apuzzo

Yeah, and now good luck messaging that when you've been telling the
public, in all caps, masks don't help. As you look at these moments, it
just cost us time. And that's kind of the story of Covid right now. We
lost time.

{[}music{]}

michael barbaro

So, Matt, where does this debate stand at the moment? I mean, is there a
settled understanding of whether or not, and how frequently, someone
without symptoms can spread the coronavirus?

matt apuzzo

I think the best science now is people without symptoms are contributing
to the spread of this pandemic. It's significant. We don't know exactly
how significant it is.

michael barbaro

Mm-hmm. But it is clearly something that happens. And because it's
symptomless, it represents a special danger in this pandemic.

matt apuzzo

Exactly right.

michael barbaro

So that's the public health consensus. But given everything that you
have just told us, do you think that the public has reached that same
conclusion? Has that message convincingly reached the world?

matt apuzzo

Well, the message is still a mess, right? I mean, we saw in early June,
the W.H.O. came out and said, oh, symptomless spreading is really rare.
And then they walked it back the next day. And part of what the W.H.O.
is still doing is trying to draw this distinction between asymptomatic
and pre-symptomatic, and it feels like we're right back in February.

michael barbaro

Right. We're making distinctions that don't mean all that much to people
who are trying to decide whether to go to work, whether to go to a
restaurant, whether to see friends.

matt apuzzo

Yeah, and those are life and death situations right now. If I wake up in
the morning and I believe that I'm not sick, and if the whole policy
comes down to me understanding the difference between asymptomatic and
oligosymptomatic and pre-symptomatic transmission, then the important
message is lost. I'm putting other people's lives in danger with my
decisions.

michael barbaro

Matt, thank you very much.

matt apuzzo

Good to be with you.

{[}music{]}

\begin{itemize}
\item
  michael barbaro\\
  Doctor, if German authorities and European health officials and the
  W.H.O. had taken your findings seriously back in January, despite the
  fact that it was a single patient, despite the fact that there was a
  semantic debate around the title of the paper, how do you think it
  would have made a difference in the state of the pandemic today?
\item
  camilla rothe\\
  Ha. That is very difficult to tell. It would be too easy, even though
  I would like to say that that could have saved hundred thousands of
  lives. Had authorities been stricter at an earlier point in time,
  well, would have people accepted it? This may sound strange, but maybe
  we needed the drastic pictures that we saw in Italy, when the military
  had to basically bury the coffins because nobody else could, or the
  dramatic pictures from New York City. Maybe that we needed, all of us
  needed that shock to take it seriously and really to pull up our socks
  to fight the virus. So it's very difficult to tell what would have
  happened had we taken this onboard early on.
\item
  matt apuzzo\\
  Has this experience changed how you see the global health community,
  your colleagues essentially?
\item
  camilla rothe\\
  Oh, yes, definitely. Definitely. It was a very sobering experience. I
  still don't know what to make of it. What I really hope is that
  someone is going to somehow work this up in a, again, in a scientific
  way to say, what happened? What happened in the heads of people? Why
  was this unwelcome news? Why was this dismissed? Can we learn from
  this? Is this, if you like, a cognitive error on the side of decision
  makers? And what can we do to prevent this from happening again? And I
  was, to be honest, deeply disappointed by it. But more so, I really
  wish to understand what was behind it. I'm really hoping that one day
  someone will come and explain to me what the issue really was.
\end{itemize}

{[}music{]}

\begin{itemize}
\item
  michael barbaro\\
  Well, Doctor, we really appreciate your time, and we wish you the best
  of luck.
\item
  camilla rothe\\
  Thank you so much. Thank you.
\end{itemize}

michael barbaro

We'll be right back. Here's what else you need to know today. Amid
intense pressure from President Trump to reopen schools, the U.S.
Centers for Disease Control and Prevention said it would issue new
guidelines to local school districts. In tweets on Wednesday morning,
Trump described the original C.D.C. guidelines, which call for masks,
social distancing, staggered arrival times and no meals in cafeterias,
as, quote, ``tough, expensive and impractical.''

\begin{itemize}
\tightlist
\item
  archived recording (robert redfield)\\
  But I want to make it very clear that what is not the intent of
  C.D.C.`s guidelines is to be used as a rationale to keep schools
  closed.
\end{itemize}

michael barbaro

A few hours later during a briefing at the White House, the C.D.C.`s
director emphasized that those guidelines are suggestions, not
requirements, and said that he did not want the guidelines to prevent
schools from reopening. And in a major ruling on Wednesday, the Supreme
Court upheld a regulation from the Trump administration that lets
companies with religious or moral objections to birth control limit
coverage of them under the Affordable Care Act. The 7-to-2 ruling could
result in as many as 126,000 women losing coverage for contraceptives
from their employers. It was the latest case involving the relationship
between church and state, in which the court's majority has sided with
religious groups.

{[}music{]}

That's it for ``The Daily.'' I'm Michael Barbaro. See you tomorrow.

The reaction prompted the W.H.O. to clarify its position in a live
session hosted on Facebook and Twitter. Dr. Van Kerkhove said her
comment had been based on only two or three studies.

``I was just responding to a question, I wasn't stating a policy of
W.H.O. or anything like that,'' she said. Dr. Van Kerkhove said her
statement was also based on unpublished evidence that some countries
have shared with the W.H.O.

But critics, including its own officials, said the organization should
be transparent about its sources. ``W.H.O.'s first and foremost
responsibility is to be the science leader,'' said Lawrence Gostin,
director of the W.H.O. Collaborating Center on National and Global
Health Law.

``And when they come out with things that are clearly contradicted by
the scientific establishment without any justification or citing
studies, it significantly reduces their credibility.''

A key point of confusion is the difference between people who are
``pre-symptomatic'' and will go on to develop symptoms, and those who
are ``asymptomatic'' and never feel sick. Dr. Van Kerkhove suggested
that her comments were about people who are truly asymptomatic.

A widely cited
\href{https://www.nature.com/articles/s41591-020-0869-5}{paper}
published in April suggested that people are most infectious up to two
days before the onset of symptoms, and estimated that 44 percent of new
infections were a result of transmission from people who were not yet
showing symptoms.

W.H.O. refers to such people as pre-symptomatic. ``OK, technically
fine,'' Dr. Jha said. ``But for all intents and purposes, they are
asymptomatic --- they are without symptoms.''

Dr. Van Kerkhove said that by using the two terms, W.H.O. officials are
in fact trying to be very clear about the group of people they are
referring to.

``Unfortunately, that's not how everybody uses it,'' she said. ``I
didn't intend that to make things more complicated.''

\href{https://www.nytimes.com/news-event/coronavirus?action=click\&pgtype=Article\&state=default\&region=MAIN_CONTENT_3\&context=storylines_faq}{}

\hypertarget{the-coronavirus-outbreak-}{%
\subsubsection{The Coronavirus Outbreak
›}\label{the-coronavirus-outbreak-}}

\hypertarget{frequently-asked-questions}{%
\paragraph{Frequently Asked
Questions}\label{frequently-asked-questions}}

Updated July 27, 2020

\begin{itemize}
\item ~
  \hypertarget{should-i-refinance-my-mortgage}{%
  \paragraph{Should I refinance my
  mortgage?}\label{should-i-refinance-my-mortgage}}

  \begin{itemize}
  \tightlist
  \item
    \href{https://www.nytimes.com/article/coronavirus-money-unemployment.html?action=click\&pgtype=Article\&state=default\&region=MAIN_CONTENT_3\&context=storylines_faq}{It
    could be a good idea,} because mortgage rates have
    \href{https://www.nytimes.com/2020/07/16/business/mortgage-rates-below-3-percent.html?action=click\&pgtype=Article\&state=default\&region=MAIN_CONTENT_3\&context=storylines_faq}{never
    been lower.} Refinancing requests have pushed mortgage applications
    to some of the highest levels since 2008, so be prepared to get in
    line. But defaults are also up, so if you're thinking about buying a
    home, be aware that some lenders have tightened their standards.
  \end{itemize}
\item ~
  \hypertarget{what-is-school-going-to-look-like-in-september}{%
  \paragraph{What is school going to look like in
  September?}\label{what-is-school-going-to-look-like-in-september}}

  \begin{itemize}
  \tightlist
  \item
    It is unlikely that many schools will return to a normal schedule
    this fall, requiring the grind of
    \href{https://www.nytimes.com/2020/06/05/us/coronavirus-education-lost-learning.html?action=click\&pgtype=Article\&state=default\&region=MAIN_CONTENT_3\&context=storylines_faq}{online
    learning},
    \href{https://www.nytimes.com/2020/05/29/us/coronavirus-child-care-centers.html?action=click\&pgtype=Article\&state=default\&region=MAIN_CONTENT_3\&context=storylines_faq}{makeshift
    child care} and
    \href{https://www.nytimes.com/2020/06/03/business/economy/coronavirus-working-women.html?action=click\&pgtype=Article\&state=default\&region=MAIN_CONTENT_3\&context=storylines_faq}{stunted
    workdays} to continue. California's two largest public school
    districts --- Los Angeles and San Diego --- said on July 13, that
    \href{https://www.nytimes.com/2020/07/13/us/lausd-san-diego-school-reopening.html?action=click\&pgtype=Article\&state=default\&region=MAIN_CONTENT_3\&context=storylines_faq}{instruction
    will be remote-only in the fall}, citing concerns that surging
    coronavirus infections in their areas pose too dire a risk for
    students and teachers. Together, the two districts enroll some
    825,000 students. They are the largest in the country so far to
    abandon plans for even a partial physical return to classrooms when
    they reopen in August. For other districts, the solution won't be an
    all-or-nothing approach.
    \href{https://bioethics.jhu.edu/research-and-outreach/projects/eschool-initiative/school-policy-tracker/}{Many
    systems}, including the nation's largest, New York City, are
    devising
    \href{https://www.nytimes.com/2020/06/26/us/coronavirus-schools-reopen-fall.html?action=click\&pgtype=Article\&state=default\&region=MAIN_CONTENT_3\&context=storylines_faq}{hybrid
    plans} that involve spending some days in classrooms and other days
    online. There's no national policy on this yet, so check with your
    municipal school system regularly to see what is happening in your
    community.
  \end{itemize}
\item ~
  \hypertarget{is-the-coronavirus-airborne}{%
  \paragraph{Is the coronavirus
  airborne?}\label{is-the-coronavirus-airborne}}

  \begin{itemize}
  \tightlist
  \item
    The coronavirus
    \href{https://www.nytimes.com/2020/07/04/health/239-experts-with-one-big-claim-the-coronavirus-is-airborne.html?action=click\&pgtype=Article\&state=default\&region=MAIN_CONTENT_3\&context=storylines_faq}{can
    stay aloft for hours in tiny droplets in stagnant air}, infecting
    people as they inhale, mounting scientific evidence suggests. This
    risk is highest in crowded indoor spaces with poor ventilation, and
    may help explain super-spreading events reported in meatpacking
    plants, churches and restaurants.
    \href{https://www.nytimes.com/2020/07/06/health/coronavirus-airborne-aerosols.html?action=click\&pgtype=Article\&state=default\&region=MAIN_CONTENT_3\&context=storylines_faq}{It's
    unclear how often the virus is spread} via these tiny droplets, or
    aerosols, compared with larger droplets that are expelled when a
    sick person coughs or sneezes, or transmitted through contact with
    contaminated surfaces, said Linsey Marr, an aerosol expert at
    Virginia Tech. Aerosols are released even when a person without
    symptoms exhales, talks or sings, according to Dr. Marr and more
    than 200 other experts, who
    \href{https://academic.oup.com/cid/article/doi/10.1093/cid/ciaa939/5867798}{have
    outlined the evidence in an open letter to the World Health
    Organization}.
  \end{itemize}
\item ~
  \hypertarget{what-are-the-symptoms-of-coronavirus}{%
  \paragraph{What are the symptoms of
  coronavirus?}\label{what-are-the-symptoms-of-coronavirus}}

  \begin{itemize}
  \tightlist
  \item
    Common symptoms
    \href{https://www.nytimes.com/article/symptoms-coronavirus.html?action=click\&pgtype=Article\&state=default\&region=MAIN_CONTENT_3\&context=storylines_faq}{include
    fever, a dry cough, fatigue and difficulty breathing or shortness of
    breath.} Some of these symptoms overlap with those of the flu,
    making detection difficult, but runny noses and stuffy sinuses are
    less common.
    \href{https://www.nytimes.com/2020/04/27/health/coronavirus-symptoms-cdc.html?action=click\&pgtype=Article\&state=default\&region=MAIN_CONTENT_3\&context=storylines_faq}{The
    C.D.C. has also} added chills, muscle pain, sore throat, headache
    and a new loss of the sense of taste or smell as symptoms to look
    out for. Most people fall ill five to seven days after exposure, but
    symptoms may appear in as few as two days or as many as 14 days.
  \end{itemize}
\item ~
  \hypertarget{does-asymptomatic-transmission-of-covid-19-happen}{%
  \paragraph{Does asymptomatic transmission of Covid-19
  happen?}\label{does-asymptomatic-transmission-of-covid-19-happen}}

  \begin{itemize}
  \tightlist
  \item
    So far, the evidence seems to show it does. A widely cited
    \href{https://www.nature.com/articles/s41591-020-0869-5}{paper}
    published in April suggests that people are most infectious about
    two days before the onset of coronavirus symptoms and estimated that
    44 percent of new infections were a result of transmission from
    people who were not yet showing symptoms. Recently, a top expert at
    the World Health Organization stated that transmission of the
    coronavirus by people who did not have symptoms was ``very rare,''
    \href{https://www.nytimes.com/2020/06/09/world/coronavirus-updates.html?action=click\&pgtype=Article\&state=default\&region=MAIN_CONTENT_3\&context=storylines_faq\#link-1f302e21}{but
    she later walked back that statement.}
  \end{itemize}
\end{itemize}

The W.H.O. continues to maintain that large respiratory droplets
expelled by sneezing or coughing are the main route of transmission and
to downplay a possible role for aerosols, smaller particles that may
linger in the air.

But evidence is piling up that aerosols may be an important route.

``What they haven't recognized is that activities like coughing and
talking, even breathing in some cases, are also aerosol-producing
procedures,'' said Linsey Marr, who studies airborne transmission of
viruses at Virginia Tech.

W.H.O. officials said they were aware that breathing and talking might
result in aerosols, but questioned their importance in spreading the
virus.

``To date, there has been no demonstration of transmission by this type
of aerosol route,'' said Dr. Benedetta Allegranzi, the W.H.O.'s
technical lead on the coronavirus.

But the W.H.O. defines airborne transmission too narrowly, some
scientists said. Airborne transmission also includes the possibility
that the virus is aloft for shorter distances, then inhaled.

``They have a very early 20th century, very unsophisticated view of what
aerosols and airborne transmission are,'' said Dr. Don Milton, an expert
on public health aerobiology at the University of Maryland.

Up until the 1950s, Dr. Milton said, tuberculosis was thought to be
spread by prolonged close contact. ``We now know that it's only
transmitted by aerosols,'' he said.

Some scientists are suspicious that W.H.O.'s stance on masks and
aerosols may stem less from scientific research than from a concern over
supplies of personal protective equipment for medical workers.

The organization currently recommends respirator masks that would block
aerosols only for health care workers doing medical procedures that
produce aerosols.

Dr. Van Kerkhove said that the W.H.O.'s guidance was based only on
science and not on any considerations of supply. While a shortage of
P.P.E. is a problem, she said, ``it doesn't change what we recommend.''

All of the experts said it was not that the W.H.O. is wrong on all
counts, but that given the implications of its statements, it should be
more cautious in concluding that transmission by air or by people
without symptoms is not significant.

``We don't know,'' Dr. Milton said. ``But they also don't know.''

Some experts said that when the W.H.O. uses the phrase ``there is no
evidence'' to indicate uncertainty, it is in fact conveying certainty
about the absence of a phenomenon.

Dr. Van Kerkhove conceded that point.

``That's a fair statement,'' she said. ``There's a lot of research that
needs to be done to really understand this, and we are open to the fact
that there is new research every day.''

Advertisement

\protect\hyperlink{after-bottom}{Continue reading the main story}

\hypertarget{site-index}{%
\subsection{Site Index}\label{site-index}}

\hypertarget{site-information-navigation}{%
\subsection{Site Information
Navigation}\label{site-information-navigation}}

\begin{itemize}
\tightlist
\item
  \href{https://help.nytimes.com/hc/en-us/articles/115014792127-Copyright-notice}{©~2020~The
  New York Times Company}
\end{itemize}

\begin{itemize}
\tightlist
\item
  \href{https://www.nytco.com/}{NYTCo}
\item
  \href{https://help.nytimes.com/hc/en-us/articles/115015385887-Contact-Us}{Contact
  Us}
\item
  \href{https://www.nytco.com/careers/}{Work with us}
\item
  \href{https://nytmediakit.com/}{Advertise}
\item
  \href{http://www.tbrandstudio.com/}{T Brand Studio}
\item
  \href{https://www.nytimes.com/privacy/cookie-policy\#how-do-i-manage-trackers}{Your
  Ad Choices}
\item
  \href{https://www.nytimes.com/privacy}{Privacy}
\item
  \href{https://help.nytimes.com/hc/en-us/articles/115014893428-Terms-of-service}{Terms
  of Service}
\item
  \href{https://help.nytimes.com/hc/en-us/articles/115014893968-Terms-of-sale}{Terms
  of Sale}
\item
  \href{https://spiderbites.nytimes.com}{Site Map}
\item
  \href{https://help.nytimes.com/hc/en-us}{Help}
\item
  \href{https://www.nytimes.com/subscription?campaignId=37WXW}{Subscriptions}
\end{itemize}
