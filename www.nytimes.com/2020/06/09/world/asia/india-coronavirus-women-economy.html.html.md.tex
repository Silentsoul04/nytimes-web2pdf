Sections

SEARCH

\protect\hyperlink{site-content}{Skip to
content}\protect\hyperlink{site-index}{Skip to site index}

\href{https://www.nytimes.com/section/world/asia}{Asia Pacific}

\href{https://myaccount.nytimes.com/auth/login?response_type=cookie\&client_id=vi}{}

\href{https://www.nytimes.com/section/todayspaper}{Today's Paper}

\href{/section/world/asia}{Asia Pacific}\textbar{}For Indian Women, the
Coronavirus Economy Is a Devastating Setback

\url{https://nyti.ms/2YbHatK}

\begin{itemize}
\item
\item
\item
\item
\item
\item
\end{itemize}

\href{https://www.nytimes.com/news-event/coronavirus?action=click\&pgtype=Article\&state=default\&region=TOP_BANNER\&context=storylines_menu}{The
Coronavirus Outbreak}

\begin{itemize}
\tightlist
\item
  live\href{https://www.nytimes.com/2020/08/02/world/coronavirus-updates.html?action=click\&pgtype=Article\&state=default\&region=TOP_BANNER\&context=storylines_menu}{Latest
  Updates}
\item
  \href{https://www.nytimes.com/interactive/2020/us/coronavirus-us-cases.html?action=click\&pgtype=Article\&state=default\&region=TOP_BANNER\&context=storylines_menu}{Maps
  and Cases}
\item
  \href{https://www.nytimes.com/interactive/2020/science/coronavirus-vaccine-tracker.html?action=click\&pgtype=Article\&state=default\&region=TOP_BANNER\&context=storylines_menu}{Vaccine
  Tracker}
\item
  \href{https://www.nytimes.com/interactive/2020/07/29/us/schools-reopening-coronavirus.html?action=click\&pgtype=Article\&state=default\&region=TOP_BANNER\&context=storylines_menu}{What
  School May Look Like}
\item
  \href{https://www.nytimes.com/live/2020/07/31/business/stock-market-today-coronavirus?action=click\&pgtype=Article\&state=default\&region=TOP_BANNER\&context=storylines_menu}{Economy}
\end{itemize}

Advertisement

\protect\hyperlink{after-top}{Continue reading the main story}

Supported by

\protect\hyperlink{after-sponsor}{Continue reading the main story}

\hypertarget{for-indian-women-the-coronavirus-economy-is-a-devastating-setback}{%
\section{For Indian Women, the Coronavirus Economy Is a Devastating
Setback}\label{for-indian-women-the-coronavirus-economy-is-a-devastating-setback}}

India's women were already dropping out of the labor force. Coronavirus
restrictions --- and one of the worst economic slumps in decades ---
threaten even more losses for them.

\includegraphics{https://static01.nyt.com/images/2020/06/08/world/00virus-india-women4/merlin_172294047_96ba0378-3535-49e5-84be-91c3d5ccf79f-articleLarge.jpg?quality=75\&auto=webp\&disable=upscale}

\href{https://www.nytimes.com/by/kai-schultz}{\includegraphics{https://static01.nyt.com/images/2019/11/22/reader-center/author-kai-schultz/author-kai-schultz-thumbLarge.png}}\href{https://www.nytimes.com/by/suhasini-raj}{\includegraphics{https://static01.nyt.com/images/2019/11/22/reader-center/author-Suhasini-Raj/author-Suhasini-Raj-thumbLarge.png}}

By \href{https://www.nytimes.com/by/kai-schultz}{Kai Schultz} and
\href{https://www.nytimes.com/by/suhasini-raj}{Suhasini Raj}

\begin{itemize}
\item
  Published June 9, 2020Updated July 15, 2020
\item
  \begin{itemize}
  \item
  \item
  \item
  \item
  \item
  \item
  \end{itemize}
\end{itemize}

NEW DELHI --- Over and over, Seema Munda kept refusing her parents'
pleas to get married. She wanted to be a nurse, not a housewife --- and
why was employment all right for her brother but not her?

So last summer, Ms. Munda lied about where she was going and slipped out
of her conservative village in northern
\href{https://www.nytimes.com/2020/07/02/world/asia/india-coronavirus-wedding-groom.html}{India}.
She traveled 1,000 miles south, to the city of Bengaluru, where she
found work stitching shirts at a factory.

``This job liberated me,'' she said.

But when the coronavirus pandemic hit, Ms. Munda's life of independence
shattered. In March, India instituted one of the strictest lockdowns in
the world. In April, more than 120 million Indians lost their jobs,
including Ms. Munda, 21.

As the world takes stock of staggering losses from the
\href{https://www.nytimes.com/2020/07/15/business/economy/economic-recovery-coronavirus-resurgence.html}{coronavirus},
economists predict especially dire setbacks for women in the work force.
The United Nations
\href{https://www.unwomen.org/-/media/headquarters/attachments/sections/library/publications/2020/policy-brief-the-impact-of-covid-19-on-women-en.pdf?la=en\&vs=1406}{warned
in a recent report} that the pandemic has not only exacerbated
\href{https://www.nytimes.com/2020/05/09/us/unemployment-coronavirus-women.html}{inequalities
between the sexes}, but threatened to undo decades of gains in the
workplace.

The International Labour Organization
\href{https://www.ilo.org/wcmsp5/groups/public/---dgreports/---gender/documents/publication/wcms_744685.pdf}{found}
that 41 percent of women were employed in sectors at high risk for job
or working-hour losses from the pandemic, compared with 35 percent of
men.

The global slowdown could have especially stark consequences in
developing economies, where around 70 percent of working women are
employed in the informal
\href{https://www.nytimes.com/2020/07/15/business/economy/economic-recovery-coronavirus-resurgence.html}{economy}
with few protections.

After Ebola quarantine measures were lifted in West Africa, for
instance,
\href{https://www.un.org/sites/un2.un.org/files/policy_brief_on_covid_impact_on_women_9_apr_2020_updated.pdf}{women
were slower than men to recover their livelihoods} and had a harder time
securing loans to rebuild businesses.

In India, a nation of 1.3 billion, the coronavirus lockdown, which was
imposed in late March, has only added to the setbacks for women, who
were already being shaken out of the work force in greater numbers in
recent years.

One national employment study conducted in May
\href{https://www.outlookindia.com/newsscroll/women-people-in-semiurban-areas-bear-the-brunt-of-job-losses/1836606}{found
that a higher proportion of women reported losing their jobs than men}.
Among Indians who remained employed, women were more likely to report
anxiety about their futures.

\includegraphics{https://static01.nyt.com/images/2020/06/08/world/00virus-india-women2/merlin_171990804_3c417e80-8e73-4fb2-9987-1f3f0e286ea4-articleLarge.jpg?quality=75\&auto=webp\&disable=upscale}

Out of the economic wreckage, arranged marriages may also increase,
experts say, with families seeing these unions as a way to secure their
daughters' futures. Since the lockdown went into effect, India's leading
matrimony websites have
\href{https://www.thehindu.com/business/online-matchmaker-sign-ups-increase-30/article31434245.ece}{reported
30 percent surges in new registrations.}

\hypertarget{latest-updates-global-coronavirus-outbreak}{%
\section{\texorpdfstring{\href{https://www.nytimes.com/2020/08/01/world/coronavirus-covid-19.html?action=click\&pgtype=Article\&state=default\&region=MAIN_CONTENT_1\&context=storylines_live_updates}{Latest
Updates: Global Coronavirus
Outbreak}}{Latest Updates: Global Coronavirus Outbreak}}\label{latest-updates-global-coronavirus-outbreak}}

Updated 2020-08-02T17:52:35.962Z

\begin{itemize}
\tightlist
\item
  \href{https://www.nytimes.com/2020/08/01/world/coronavirus-covid-19.html?action=click\&pgtype=Article\&state=default\&region=MAIN_CONTENT_1\&context=storylines_live_updates\#link-34047410}{The
  U.S. reels as July cases more than double the total of any other
  month.}
\item
  \href{https://www.nytimes.com/2020/08/01/world/coronavirus-covid-19.html?action=click\&pgtype=Article\&state=default\&region=MAIN_CONTENT_1\&context=storylines_live_updates\#link-780ec966}{Top
  U.S. officials work to break an impasse over the federal jobless
  benefit.}
\item
  \href{https://www.nytimes.com/2020/08/01/world/coronavirus-covid-19.html?action=click\&pgtype=Article\&state=default\&region=MAIN_CONTENT_1\&context=storylines_live_updates\#link-2bc8948}{Its
  outbreak untamed, Melbourne goes into even greater lockdown.}
\end{itemize}

\href{https://www.nytimes.com/2020/08/01/world/coronavirus-covid-19.html?action=click\&pgtype=Article\&state=default\&region=MAIN_CONTENT_1\&context=storylines_live_updates}{See
more updates}

More live coverage:
\href{https://www.nytimes.com/live/2020/07/31/business/stock-market-today-coronavirus?action=click\&pgtype=Article\&state=default\&region=MAIN_CONTENT_1\&context=storylines_live_updates}{Markets}

In India, marriage does not necessarily translate into a loss of ****
employment. But it often constrains women's autonomy, making it
difficult for them to leave secluded villages where policing of their
choices is common, patriarchal values are ironclad and job opportunities
are scarce.

Rohini Pande, an economics professor at Yale who researches women's
employment patterns in India, said female migrant workers could face
steep challenges recovering work. Many women
\href{https://www.nytimes.com/2016/09/25/world/asia/bangalore-india-women-factories.html}{struggle
to persuade their parents to let them defer marriage} and leave their
villages for jobs.

``The pipeline was already extremely leaky,'' said Ms. Pande, who
directs the Economic Growth Center at Yale. ``It's just going to get
leakier.''

Employment figures for India's women have been a cause of concern for
years.

From 2005 to 2018, female labor participation in India
\href{https://data.worldbank.org/indicator/SL.TLF.CACT.FE.ZS?locations=IN}{declined
to 21 percent from about 32 percent}, one of the lowest rates in the
world. The rate for men also fell --- India is experiencing a youth boom
and has not been able to create enough new jobs to keep up --- but not
nearly as far as for women.

Economists have offered several explanations for the slide, including a
cultural one: As India's economy expanded, families that could afford to
keep women at home did so, thinking it afforded them a degree of social
status.

Domestic duties cut into the time women can search for jobs. In India,
\href{https://www.ilo.org/wcmsp5/groups/public/---dgreports/---dcomm/---publ/documents/publication/wcms_633135.pdf}{women
perform 9.6 times more unpaid care work than men}, about three times the
global average. The pandemic has increased that burden for many women,
according to the International Labour Organization.

Swarna Rajagopalan, a political scientist and founder of Prajnya Trust,
an organization focusing on gender equality in India, said job scarcity
could make it harder for women to enter or re-enter the work force ---
at least in the short term.

India's economy may contract by five percent this year,
\href{https://www.crisil.com/en/home/our-analysis/reports/2020/05/minus-five.html}{according
to some estimates}, representing perhaps the worst slump since the
country became independent from the British.

``I really worry about this,'' Ms. Rajagopalan said. ``We still think of
men as being the primary breadwinners of our families, and if we have to
make choices about letting people go, women will lose their jobs. It
doesn't matter how desperately they need them, or how hard they work.''

Many of the hardest-hit industries have been those with a high
proportion of female workers, including hospitality and manufacturing,
where women are often employed without contracts, making it easier to
let them go.

Although India recently lifted most of its lockdown measures in an
effort to ease pressure on the economy, many women fear that even a
limited degree of economic freedom will be difficult to regain.

Image

Women working at a textile factory in the southern Indian city of
Bengaluru in 2016.Credit...Andrea Bruce for The New York Times

Seema Munda said she had found ``the key to unlocking my dreams'' when
she moved to Bengaluru last July to work at Pearl Global, a garment
factory that employs women from poorer states like Odisha and Jharkhand,
where she is from.

Her parents were initially furious when they realized she had left their
village, Laujoda, for a city at the other end of India. Ms. Munda calmed
them down when she offered to send some of her earnings home.

\href{https://www.nytimes.com/news-event/coronavirus?action=click\&pgtype=Article\&state=default\&region=MAIN_CONTENT_3\&context=storylines_faq}{}

\hypertarget{the-coronavirus-outbreak-}{%
\subsubsection{The Coronavirus Outbreak
›}\label{the-coronavirus-outbreak-}}

\hypertarget{frequently-asked-questions}{%
\paragraph{Frequently Asked
Questions}\label{frequently-asked-questions}}

Updated July 27, 2020

\begin{itemize}
\item ~
  \hypertarget{should-i-refinance-my-mortgage}{%
  \paragraph{Should I refinance my
  mortgage?}\label{should-i-refinance-my-mortgage}}

  \begin{itemize}
  \tightlist
  \item
    \href{https://www.nytimes.com/article/coronavirus-money-unemployment.html?action=click\&pgtype=Article\&state=default\&region=MAIN_CONTENT_3\&context=storylines_faq}{It
    could be a good idea,} because mortgage rates have
    \href{https://www.nytimes.com/2020/07/16/business/mortgage-rates-below-3-percent.html?action=click\&pgtype=Article\&state=default\&region=MAIN_CONTENT_3\&context=storylines_faq}{never
    been lower.} Refinancing requests have pushed mortgage applications
    to some of the highest levels since 2008, so be prepared to get in
    line. But defaults are also up, so if you're thinking about buying a
    home, be aware that some lenders have tightened their standards.
  \end{itemize}
\item ~
  \hypertarget{what-is-school-going-to-look-like-in-september}{%
  \paragraph{What is school going to look like in
  September?}\label{what-is-school-going-to-look-like-in-september}}

  \begin{itemize}
  \tightlist
  \item
    It is unlikely that many schools will return to a normal schedule
    this fall, requiring the grind of
    \href{https://www.nytimes.com/2020/06/05/us/coronavirus-education-lost-learning.html?action=click\&pgtype=Article\&state=default\&region=MAIN_CONTENT_3\&context=storylines_faq}{online
    learning},
    \href{https://www.nytimes.com/2020/05/29/us/coronavirus-child-care-centers.html?action=click\&pgtype=Article\&state=default\&region=MAIN_CONTENT_3\&context=storylines_faq}{makeshift
    child care} and
    \href{https://www.nytimes.com/2020/06/03/business/economy/coronavirus-working-women.html?action=click\&pgtype=Article\&state=default\&region=MAIN_CONTENT_3\&context=storylines_faq}{stunted
    workdays} to continue. California's two largest public school
    districts --- Los Angeles and San Diego --- said on July 13, that
    \href{https://www.nytimes.com/2020/07/13/us/lausd-san-diego-school-reopening.html?action=click\&pgtype=Article\&state=default\&region=MAIN_CONTENT_3\&context=storylines_faq}{instruction
    will be remote-only in the fall}, citing concerns that surging
    coronavirus infections in their areas pose too dire a risk for
    students and teachers. Together, the two districts enroll some
    825,000 students. They are the largest in the country so far to
    abandon plans for even a partial physical return to classrooms when
    they reopen in August. For other districts, the solution won't be an
    all-or-nothing approach.
    \href{https://bioethics.jhu.edu/research-and-outreach/projects/eschool-initiative/school-policy-tracker/}{Many
    systems}, including the nation's largest, New York City, are
    devising
    \href{https://www.nytimes.com/2020/06/26/us/coronavirus-schools-reopen-fall.html?action=click\&pgtype=Article\&state=default\&region=MAIN_CONTENT_3\&context=storylines_faq}{hybrid
    plans} that involve spending some days in classrooms and other days
    online. There's no national policy on this yet, so check with your
    municipal school system regularly to see what is happening in your
    community.
  \end{itemize}
\item ~
  \hypertarget{is-the-coronavirus-airborne}{%
  \paragraph{Is the coronavirus
  airborne?}\label{is-the-coronavirus-airborne}}

  \begin{itemize}
  \tightlist
  \item
    The coronavirus
    \href{https://www.nytimes.com/2020/07/04/health/239-experts-with-one-big-claim-the-coronavirus-is-airborne.html?action=click\&pgtype=Article\&state=default\&region=MAIN_CONTENT_3\&context=storylines_faq}{can
    stay aloft for hours in tiny droplets in stagnant air}, infecting
    people as they inhale, mounting scientific evidence suggests. This
    risk is highest in crowded indoor spaces with poor ventilation, and
    may help explain super-spreading events reported in meatpacking
    plants, churches and restaurants.
    \href{https://www.nytimes.com/2020/07/06/health/coronavirus-airborne-aerosols.html?action=click\&pgtype=Article\&state=default\&region=MAIN_CONTENT_3\&context=storylines_faq}{It's
    unclear how often the virus is spread} via these tiny droplets, or
    aerosols, compared with larger droplets that are expelled when a
    sick person coughs or sneezes, or transmitted through contact with
    contaminated surfaces, said Linsey Marr, an aerosol expert at
    Virginia Tech. Aerosols are released even when a person without
    symptoms exhales, talks or sings, according to Dr. Marr and more
    than 200 other experts, who
    \href{https://academic.oup.com/cid/article/doi/10.1093/cid/ciaa939/5867798}{have
    outlined the evidence in an open letter to the World Health
    Organization}.
  \end{itemize}
\item ~
  \hypertarget{what-are-the-symptoms-of-coronavirus}{%
  \paragraph{What are the symptoms of
  coronavirus?}\label{what-are-the-symptoms-of-coronavirus}}

  \begin{itemize}
  \tightlist
  \item
    Common symptoms
    \href{https://www.nytimes.com/article/symptoms-coronavirus.html?action=click\&pgtype=Article\&state=default\&region=MAIN_CONTENT_3\&context=storylines_faq}{include
    fever, a dry cough, fatigue and difficulty breathing or shortness of
    breath.} Some of these symptoms overlap with those of the flu,
    making detection difficult, but runny noses and stuffy sinuses are
    less common.
    \href{https://www.nytimes.com/2020/04/27/health/coronavirus-symptoms-cdc.html?action=click\&pgtype=Article\&state=default\&region=MAIN_CONTENT_3\&context=storylines_faq}{The
    C.D.C. has also} added chills, muscle pain, sore throat, headache
    and a new loss of the sense of taste or smell as symptoms to look
    out for. Most people fall ill five to seven days after exposure, but
    symptoms may appear in as few as two days or as many as 14 days.
  \end{itemize}
\item ~
  \hypertarget{does-asymptomatic-transmission-of-covid-19-happen}{%
  \paragraph{Does asymptomatic transmission of Covid-19
  happen?}\label{does-asymptomatic-transmission-of-covid-19-happen}}

  \begin{itemize}
  \tightlist
  \item
    So far, the evidence seems to show it does. A widely cited
    \href{https://www.nature.com/articles/s41591-020-0869-5}{paper}
    published in April suggests that people are most infectious about
    two days before the onset of coronavirus symptoms and estimated that
    44 percent of new infections were a result of transmission from
    people who were not yet showing symptoms. Recently, a top expert at
    the World Health Organization stated that transmission of the
    coronavirus by people who did not have symptoms was ``very rare,''
    \href{https://www.nytimes.com/2020/06/09/world/coronavirus-updates.html?action=click\&pgtype=Article\&state=default\&region=MAIN_CONTENT_3\&context=storylines_faq\#link-1f302e21}{but
    she later walked back that statement.}
  \end{itemize}
\end{itemize}

``I said, `Whenever I ask you for money for my studies, you always turn
me down,''' she recalled. ```So if you let me go and work, not only will
I have money for myself but also to spare for you.'''

Ms. Munda started fresh. She moved into a hostel with dozens of other
young women from the factory. They slept on straw mats on the ground.

When Ms. Munda received her first paycheck, about \$112, she went to a
clothing store with a handful of crisp notes.

``I bought my favorite dress,'' she said. ``It was exhilarating.''

But when India went under lockdown, the factory closed and the women
found themselves in a precarious situation. Around the country,
businesses shut down. Trains and buses suspended their services,
stranding millions of migrant workers in cities.

Within a few weeks, Ms. Munda said Pearl Global stopped paying her. She
was forced to leave the hostel and take shelter in a school.

By the end of May, as India's travel restrictions eased further, Ms.
Munda made a wrenching decision and joined others in boarding trains
home. With little money left, she said she had no other choice but to
return to Jharkhand.

``My family will never let me come back now,'' she said by telephone.
``I don't want to get married.''

Ms. Munda stopped answering calls from a reporter. Friends from
Bengaluru were also unable to reach her. They worried that her parents
had taken her phone.

One factory worker who traveled with Ms. Munda said she hid her face
with a stole and sobbed during the three-day trip to Jharkhand.

Hemant Kumar, the chairman of Ashankura Trust, a nongovernmental
organization that helps migrant workers in Bengaluru, said only a few
women had picked up his calls after reaching their villages.

Image

A woman who makes a living as a garment worker on the roof of her
apartment building in Bengaluru last year.Credit...Rebecca Conway for
The New York Times

Some told him that their families had forbidden them from venturing back
out for jobs.

In one of her last conversations with a reporter, Ms. Munda expressed
anger that ``parents value sons more than daughters.'' She said
returning home could mean ``the end of my economic activity and hence my
life.''

``I dread to think of that possibility,'' she said. ``Our future is in
darkness.''

Surabhi Singh contributed reporting from Raipur, India.

Advertisement

\protect\hyperlink{after-bottom}{Continue reading the main story}

\hypertarget{site-index}{%
\subsection{Site Index}\label{site-index}}

\hypertarget{site-information-navigation}{%
\subsection{Site Information
Navigation}\label{site-information-navigation}}

\begin{itemize}
\tightlist
\item
  \href{https://help.nytimes.com/hc/en-us/articles/115014792127-Copyright-notice}{©~2020~The
  New York Times Company}
\end{itemize}

\begin{itemize}
\tightlist
\item
  \href{https://www.nytco.com/}{NYTCo}
\item
  \href{https://help.nytimes.com/hc/en-us/articles/115015385887-Contact-Us}{Contact
  Us}
\item
  \href{https://www.nytco.com/careers/}{Work with us}
\item
  \href{https://nytmediakit.com/}{Advertise}
\item
  \href{http://www.tbrandstudio.com/}{T Brand Studio}
\item
  \href{https://www.nytimes.com/privacy/cookie-policy\#how-do-i-manage-trackers}{Your
  Ad Choices}
\item
  \href{https://www.nytimes.com/privacy}{Privacy}
\item
  \href{https://help.nytimes.com/hc/en-us/articles/115014893428-Terms-of-service}{Terms
  of Service}
\item
  \href{https://help.nytimes.com/hc/en-us/articles/115014893968-Terms-of-sale}{Terms
  of Sale}
\item
  \href{https://spiderbites.nytimes.com}{Site Map}
\item
  \href{https://help.nytimes.com/hc/en-us}{Help}
\item
  \href{https://www.nytimes.com/subscription?campaignId=37WXW}{Subscriptions}
\end{itemize}
