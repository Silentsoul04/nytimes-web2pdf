Sections

SEARCH

\protect\hyperlink{site-content}{Skip to
content}\protect\hyperlink{site-index}{Skip to site index}

\href{https://www.nytimes.com/section/your-money}{Your Money}

\href{https://myaccount.nytimes.com/auth/login?response_type=cookie\&client_id=vi}{}

\href{https://www.nytimes.com/section/todayspaper}{Today's Paper}

\href{/section/your-money}{Your Money}\textbar{}The Stark Racial
Inequity of Personal Finances in America

\url{https://nyti.ms/2MHOZSC}

\begin{itemize}
\item
\item
\item
\item
\item
\end{itemize}

Advertisement

\protect\hyperlink{after-top}{Continue reading the main story}

Supported by

\protect\hyperlink{after-sponsor}{Continue reading the main story}

Your Money

\hypertarget{the-stark-racial-inequity-of-personal-finances-in-america}{%
\section{The Stark Racial Inequity of Personal Finances in
America}\label{the-stark-racial-inequity-of-personal-finances-in-america}}

Economic equality is crucial to racial equality. But at nearly every
stage of their lives, black Americans have less than whites.

\includegraphics{https://static01.nyt.com/images/2020/06/09/business/09money/00money-articleLarge.jpg?quality=75\&auto=webp\&disable=upscale}

\href{https://www.nytimes.com/by/ron-lieber}{\includegraphics{https://static01.nyt.com/images/2018/10/22/multimedia/author-ron-lieber/author-ron-lieber-thumbLarge.png}}\href{https://www.nytimes.com/by/tara-siegel-bernard}{\includegraphics{https://static01.nyt.com/images/2019/01/18/multimedia/author-tara-siegel-bernard/author-tara-siegel-bernard-thumbLarge.png}}

By \href{https://www.nytimes.com/by/ron-lieber}{Ron Lieber} and
\href{https://www.nytimes.com/by/tara-siegel-bernard}{Tara Siegel
Bernard}

\begin{itemize}
\item
  Published June 9, 2020Updated June 16, 2020
\item
  \begin{itemize}
  \item
  \item
  \item
  \item
  \item
  \end{itemize}
\end{itemize}

We cannot quantify the injustice of a white policeman holding his knee
on the neck of a handcuffed, dying black man. And mere numbers cannot
fully express the power imbalance involved in the deaths of George Floyd
and too many others like him.

But we can measure the
\href{https://www.nytimes.com/interactive/2020/06/11/multimedia/coronavirus-new-york-inequality.html}{economic
inequity} that serves as their backdrop.

Dollars are like air --- crucial to vitality. And when it comes to
wealth, black Americans have less at nearly every juncture of life, from
birth to death.

Perversely, having less can cost more. Black students borrow more to go
to college, don't finish as often and more frequently default on their
student loans. They earn less, and generally have lower credit scores
--- so they pay higher interest rates. It's harder for them to save for
retirement, and they leave less to the next generation when they die.

An imbalance of societal power cannot be separated from
\href{https://www.nytimes.com/interactive/2018/03/19/upshot/race-class-white-and-black-men.html}{cradle-to-grave
economic inequality}. This is what that looks like.

\hypertarget{young-black-families-earn-far-less-than-similar-white-families}{%
\subsection{Young black families earn far less than similar white
families}\label{young-black-families-earn-far-less-than-similar-white-families}}

From board books for toddlers to quality care, it can be costly to get a
child started in life. And black families typically have fewer financial
resources to draw on.

Black families with a new baby have a median household income of
\$36,300, according to an analysis of 2018 census data by the Center on
Poverty \& Social Policy. For white families, it was more than twice as
much: \$80,000.

Black families were behind other groups, too. For Asian-Americans and
Pacific Islanders, the median income was \$105,600. Among multiracial
families, the figure was \$64,000. Hispanic families had \$48,400 in
income, and Native American and Alaskan Native families had \$41,000.

Starting with less makes many things in the future that much harder. For
example, every unspent dollar of earnings can potentially be saved for
higher education.

\hypertarget{a-graduation-gap-in-college}{%
\subsection{A graduation gap in
college}\label{a-graduation-gap-in-college}}

Once a child enrolls in college, graduating with a bachelor's degree
isn't a given. But here, too, blacks
\href{https://www.nytimes.com/2018/03/25/opinion/college-graduation-gap.html}{have
it worse} than nearly any other group.

Their six-year completion rate through June 2017 for students starting
at a four-year institution was 38.9 percent, according to data from the
National Center for Education Statistics. For whites, it was 64.8
percent --- even though both groups graduate from high school
\href{https://nces.ed.gov/programs/dropout/ind_03.asp}{at roughly the
same rate}.

Asian-Americans (72.1 percent), mixed-race students (54.5 percent) and
Hispanics (50.5 percent) were also ahead of blacks. Only Native
Americans and Alaskan Natives finished at a lower rate: Just 26.3
percent within six years.

Starting but not finishing is often the worst of both worlds: Large
numbers of these students end up with debt, but they don't get the
degree and the earnings boost that usually come with it.

\hypertarget{more-student-loan-debt-and-more-defaults}{%
\subsection{More student loan debt, and more
defaults}\label{more-student-loan-debt-and-more-defaults}}

A college education is supposed to help pave a path to financial
security. For many black students, that's far from guaranteed: They tend
to borrow significantly more than their white peers, and they're more
likely to default on their loans.

Twenty-one percent of black graduates with bachelor's degrees default.
That's more than five times the rate of their white peers (4 percent).
Even white dropouts (18 percent) are less likely to default, according
to
\href{https://www.brookings.edu/research/the-looming-student-loan-default-crisis-is-worse-than-we-thought/}{a
2018 analysis} by Judith Scott-Clayton, an associate professor of
economics and education in the Economics and Education program at
Teachers College, Columbia University.

She looked at data for people who started school for the first time in
the 2003-4 academic year and analyzed their experiences over the next
dozen years. Only 1.4 percent of Asian bachelor's degree graduates
defaulted during that period, with Hispanic graduates defaulting 8.6
percent of the time.

Black students who earned bachelor's degrees also accumulated more debt
than whites. They borrowed \$21,149 on average, nearly twice as much as
whites, by the time they left school. (This includes students who didn't
borrow at all.) But it got worse after that: By the end of the 12-year
period that Dr. Scott-Clayton examined, blacks owed \$64,142 --- three
times as much as whites. That's because black degree-holders had both
higher levels of graduate school borrowing and lower rates of repayment.

Even with a college degree, black Americans can't count on getting a
paycheck of the same size.

\hypertarget{734-cents-on-the-dollar}{%
\subsection{73.4 cents on the dollar}\label{734-cents-on-the-dollar}}

The black/white wage gap was significantly wider in 2019 than at the
start of the century --- even as Hispanic workers have slightly narrowed
their own gap with white workers, according to
\href{https://www.epi.org/publication/swa-wages-2019/}{research from the
Economic Policy Institute}.

But the gap isn't a function of differences in education levels. Even
among those who attain advanced degrees, blacks were paid 82.4 cents for
every dollar earned by their white peers. Hispanics do better, at 90.1
cents on the dollar.

And the gender pay gap expands the racial gap into a chasm: Black women,
on average, earn 64 cents for every dollar a white man earns, according
to \href{https://www.epi.org/publication/black-workers-covid/}{another
report from the institute}.

\hypertarget{the-widest-homeownership-gap-in-50-years}{%
\subsection{The widest homeownership gap in 50
years}\label{the-widest-homeownership-gap-in-50-years}}

The home is the largest asset for many American families, which may help
build wealth over time. Paying down a mortgage often serves as a forced
savings plan, enabling families to build equity that they can tap in
retirement or leave to their heirs.

Black families have long been behind their white peers in homeownership,
but that gap is the largest it has been in a half-century,
\href{https://www.urban.org/sites/default/files/publication/101160/explaining_the_black-white_homeownership_gap_a_closer_look_at_disparities_across_local_markets.pdf}{according
to the Urban Institute}.

In 2018, about 72 percent of white households owned homes, compared with
nearly 41.7 percent of blacks, 47.5 percent of Hispanics and 59.5
percent of Asians, according to the institute, using the 2018 American
Community Survey. In 1960, nearly 65 percent of whites owned homes,
compared with 38.1 percent of blacks, 45.2 percent of Hispanics and 42.8
percent of Asians, according to an analysis of census data.

``The gap in the homeownership rate between black and white families in
the U.S. is bigger today than it was when it was legal to refuse to sell
someone a home because of the color of their skin,'' the Urban Institute
\href{https://www.urban.org/policy-centers/housing-finance-policy-center/projects/reducing-racial-homeownership-gap}{wrote}.

\hypertarget{fewer-retirement-accounts-with-less-in-them}{%
\subsection{Fewer retirement accounts, with less in
them}\label{fewer-retirement-accounts-with-less-in-them}}

The science of measuring retirement assets is imperfect, because older
Americans can draw on any number of resources if they have them,
including home equity, a pension and Social Security.

Those assets aren't as flexible, however, as a workplace savings plan
like a 401(k) or an individual retirement account. Blacks are less
likely to have such accounts, and tend to have less in them when they
do.

Sixty percent of white families have at least one retirement account,
while just 34 percent of black families do, according to the most recent
Federal Reserve Survey of Consumer Finances, which drew on
\href{https://www.federalreserve.gov/econres/notes/feds-notes/recent-trends-in-wealth-holding-by-race-and-ethnicity-evidence-from-the-survey-of-consumer-finances-20170927.htm}{data}
from 2016. Hispanic families have even fewer, at 30 percent. (The survey
does not break other groups into distinct categories.)

Families with white heads of household have balances that dwarf the
holdings of families headed by blacks, according to the Employee Benefit
Research Institute, which looked at the same Federal Reserve data and
measured families with family heads between the ages of 55 and 64.

The median balance was \$151,000 for whites and \$46,100 for blacks.
Hispanics had the lowest numbers here, too, with a median of \$43,000.

\hypertarget{less-left-over-for-the-next-generation}{%
\subsection{Less left over for the next
generation}\label{less-left-over-for-the-next-generation}}

The imbalance in homeownership and retirement accounts makes it
unsurprising that white households are more likely to receive an
inheritance than black ones. In fact, they are about two and a half
times as likely to do so, according to
\href{https://www.federalreserve.gov/econres/feds/files/2015076r1pap.pdf}{research
from two Fed economists}, Jeffrey P. Thompson and Gustavo A. Suarez.

They looked at households headed by people ages 30 to 59 in 2013 and
2016. Twenty-three percent of white families reported having received an
inheritance. Just 9 percent of black families answered affirmatively,
and only 5 percent of Hispanic families did so.

Whites received more, too: The median inheritance in white families was
\$56,217, while blacks received \$38,224 and Hispanics were just behind
at \$37,124.

So even if white families had fallen behind in the first part of
adulthood, they had a better chance of catching up with a single boost.
And those who are already doing better widen the gap further when a
relative dies.

At that point, the process begins anew for their kids. And their kids'
kids.

And here we are.

Advertisement

\protect\hyperlink{after-bottom}{Continue reading the main story}

\hypertarget{site-index}{%
\subsection{Site Index}\label{site-index}}

\hypertarget{site-information-navigation}{%
\subsection{Site Information
Navigation}\label{site-information-navigation}}

\begin{itemize}
\tightlist
\item
  \href{https://help.nytimes.com/hc/en-us/articles/115014792127-Copyright-notice}{©~2020~The
  New York Times Company}
\end{itemize}

\begin{itemize}
\tightlist
\item
  \href{https://www.nytco.com/}{NYTCo}
\item
  \href{https://help.nytimes.com/hc/en-us/articles/115015385887-Contact-Us}{Contact
  Us}
\item
  \href{https://www.nytco.com/careers/}{Work with us}
\item
  \href{https://nytmediakit.com/}{Advertise}
\item
  \href{http://www.tbrandstudio.com/}{T Brand Studio}
\item
  \href{https://www.nytimes.com/privacy/cookie-policy\#how-do-i-manage-trackers}{Your
  Ad Choices}
\item
  \href{https://www.nytimes.com/privacy}{Privacy}
\item
  \href{https://help.nytimes.com/hc/en-us/articles/115014893428-Terms-of-service}{Terms
  of Service}
\item
  \href{https://help.nytimes.com/hc/en-us/articles/115014893968-Terms-of-sale}{Terms
  of Sale}
\item
  \href{https://spiderbites.nytimes.com}{Site Map}
\item
  \href{https://help.nytimes.com/hc/en-us}{Help}
\item
  \href{https://www.nytimes.com/subscription?campaignId=37WXW}{Subscriptions}
\end{itemize}
