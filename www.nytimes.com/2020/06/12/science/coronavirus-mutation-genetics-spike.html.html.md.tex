Sections

SEARCH

\protect\hyperlink{site-content}{Skip to
content}\protect\hyperlink{site-index}{Skip to site index}

\href{https://www.nytimes.com/section/science}{Science}

\href{https://myaccount.nytimes.com/auth/login?response_type=cookie\&client_id=vi}{}

\href{https://www.nytimes.com/section/todayspaper}{Today's Paper}

\href{/section/science}{Science}\textbar{}Mutation Allows Coronavirus to
Infect More Cells, Study Finds. Scientists Urge Caution.

\url{https://nyti.ms/2AvJ6Wb}

\begin{itemize}
\item
\item
\item
\item
\item
\item
\end{itemize}

\href{https://www.nytimes.com/news-event/coronavirus?action=click\&pgtype=Article\&state=default\&region=TOP_BANNER\&context=storylines_menu}{The
Coronavirus Outbreak}

\begin{itemize}
\tightlist
\item
  live\href{https://www.nytimes.com/2020/08/01/world/coronavirus-covid-19.html?action=click\&pgtype=Article\&state=default\&region=TOP_BANNER\&context=storylines_menu}{Latest
  Updates}
\item
  \href{https://www.nytimes.com/interactive/2020/us/coronavirus-us-cases.html?action=click\&pgtype=Article\&state=default\&region=TOP_BANNER\&context=storylines_menu}{Maps
  and Cases}
\item
  \href{https://www.nytimes.com/interactive/2020/science/coronavirus-vaccine-tracker.html?action=click\&pgtype=Article\&state=default\&region=TOP_BANNER\&context=storylines_menu}{Vaccine
  Tracker}
\item
  \href{https://www.nytimes.com/interactive/2020/07/29/us/schools-reopening-coronavirus.html?action=click\&pgtype=Article\&state=default\&region=TOP_BANNER\&context=storylines_menu}{What
  School May Look Like}
\item
  \href{https://www.nytimes.com/live/2020/07/31/business/stock-market-today-coronavirus?action=click\&pgtype=Article\&state=default\&region=TOP_BANNER\&context=storylines_menu}{Economy}
\end{itemize}

Advertisement

\protect\hyperlink{after-top}{Continue reading the main story}

Supported by

\protect\hyperlink{after-sponsor}{Continue reading the main story}

\hypertarget{mutation-allows-coronavirus-to-infect-more-cells-study-finds-scientists-urge-caution}{%
\section{Mutation Allows Coronavirus to Infect More Cells, Study Finds.
Scientists Urge
Caution.}\label{mutation-allows-coronavirus-to-infect-more-cells-study-finds-scientists-urge-caution}}

Geneticists said more evidence is needed to determine if a common
genetic variation of the virus spreads more easily between people.

\includegraphics{https://static01.nyt.com/images/2020/06/12/science/12VIRUS-MUTATION1/12VIRUS-MUTATION1-articleLarge.jpg?quality=75\&auto=webp\&disable=upscale}

By \href{https://www.nytimes.com/by/benedict-carey}{Benedict Carey} and
\href{https://www.nytimes.com/by/james-glanz}{James Glanz}

\begin{itemize}
\item
  June 12, 2020
\item
  \begin{itemize}
  \item
  \item
  \item
  \item
  \item
  \item
  \end{itemize}
\end{itemize}

For months, scientists have debated why one genetic variation of the
coronavirus became dominant in many parts of the world.

Many scientists argue that the variation spread widely by chance,
multiplying outward from explosive outbreaks in Europe. Others have
proposed the possibility that a
\href{https://www.nytimes.com/2020/07/02/health/coronavirus-korber-mutation.html}{mutation}
gave it some kind of biological edge and have been urgently
investigating the effect of that mutation.

Now, scientists have shown ---~at least in the tightly controlled
environment of a laboratory cell culture --- that viruses carrying that
particular mutation infect more cells and are more resilient than those
without it.

Geneticists cautioned against drawing conclusions about whether the
variant, which has been circulating widely since February, spreads more
easily in humans. There is no evidence that it is more deadly or
harmful, and differences seen in a cell culture do not necessarily mean
it is more contagious, they said.

But
\href{https://www.scripps.edu/news-and-events/press-room/2020/20200612-choe-farzan-coronavirus-spike-mutation.html}{the
new study}, which has not yet been peer reviewed, does show that this
mutation appears to change the biological function of the virus, experts
said. The insight could be a crucial first step in understanding how the
mutation behaves at a biomolecular level.

Researchers at Scripps Research, Florida, found that the mutation, known
as D614G, stabilized the virus's spike proteins, which protrude from the
viral surface and give the coronavirus its name. The number of
functional and intact spikes on each viral particle was about five times
higher because of this mutation, they found.

These spike proteins must attach to a cell for a virus to infect it. As
a result, the viruses with D614G were far more likely to infect a cell
than viruses without that mutation, according to the scientists who led
the study, Hyeryun Choe and Michael Farzan.

\hypertarget{the-d614g-mutation}{%
\subsection{The D614G Mutation}\label{the-d614g-mutation}}

A tiny mutation in the coronavirus genome may stabilize the spike
proteins that protrude from the virus and allow it to infect more cells
--- at least in laboratory experiments.

The SARS-CoV-2

coronavirus

◀ Spike protein

D614G

mutation

◀ Areas affected by the mutation

ORF1a protein

ORF1b

Spike

E

M

N

The 614th amino acid in the spike protein mutated from D to

G

Spike

protein

D614G

mutation

◀ Affected area

ORF1a protein

ORF1b

Spike

E

M

N

The 614th amino acid in the

spike protein mutated from D to

G

Spike

protein

D614G

mutation

Affected

area

ORF1a protein

ORF1b

Spike

N

The 614th amino acid in the

spike protein mutated from D to

G

By Jonathan Corum \textbar{} Source: Lizhou Zhang et al., Scripps
Research

``Viruses with more functional spikes on the surface would be more
infectious,'' Dr. Farzan said. ``And there are very clear differences
between the two viruses in the experiment.'' He added: ``Those
differences just popped out.''

Dr. Choe, the senior author on the paper, said that the virus spikes
with the mutation were ``nearly 10 times more infectious in the cell
culture system that we used'' than those without that same mutation.

\hypertarget{latest-updates-global-coronavirus-outbreak}{%
\section{\texorpdfstring{\href{https://www.nytimes.com/2020/08/01/world/coronavirus-covid-19.html?action=click\&pgtype=Article\&state=default\&region=MAIN_CONTENT_1\&context=storylines_live_updates}{Latest
Updates: Global Coronavirus
Outbreak}}{Latest Updates: Global Coronavirus Outbreak}}\label{latest-updates-global-coronavirus-outbreak}}

Updated 2020-08-02T07:42:09.613Z

\begin{itemize}
\tightlist
\item
  \href{https://www.nytimes.com/2020/08/01/world/coronavirus-covid-19.html?action=click\&pgtype=Article\&state=default\&region=MAIN_CONTENT_1\&context=storylines_live_updates\#link-34047410}{The
  U.S. reels as July cases more than double the total of any other
  month.}
\item
  \href{https://www.nytimes.com/2020/08/01/world/coronavirus-covid-19.html?action=click\&pgtype=Article\&state=default\&region=MAIN_CONTENT_1\&context=storylines_live_updates\#link-780ec966}{Top
  U.S. officials work to break an impasse over the federal jobless
  benefit.}
\item
  \href{https://www.nytimes.com/2020/08/01/world/coronavirus-covid-19.html?action=click\&pgtype=Article\&state=default\&region=MAIN_CONTENT_1\&context=storylines_live_updates\#link-2bc8948}{Its
  outbreak untamed, Melbourne goes into even greater lockdown.}
\end{itemize}

\href{https://www.nytimes.com/2020/08/01/world/coronavirus-covid-19.html?action=click\&pgtype=Article\&state=default\&region=MAIN_CONTENT_1\&context=storylines_live_updates}{See
more updates}

More live coverage:
\href{https://www.nytimes.com/live/2020/07/31/business/stock-market-today-coronavirus?action=click\&pgtype=Article\&state=default\&region=MAIN_CONTENT_1\&context=storylines_live_updates}{Markets}

Mutations are tiny, random changes to viral genetic material that occur
as it is copied. The vast majority do not affect the virus's function,
one way or another.

Virologists shown the study said that the
\href{https://www.scripps.edu/news-and-events/press-room/2020/20200612-choe-farzan-coronavirus-spike-mutation.html}{Scripps
research} was a strong demonstration that this specific mutation does
indeed cause a significant change in how the virus behaves,
biologically.

``This is a powerful experimental study and the best evidence yet that
the D614G mutation increases the infectivity of SARS-CoV-2,'' said Eddie
Holmes, a professor at the University of Sydney and a specialist in
viral evolution.

The mutation the researchers studied has predominated in Europe and in
much of the United States, especially in the Northeast. They compared it
to viruses without that mutation, like those found at the beginning of
the pandemic in Wuhan, China.

Dr. Choe said that the results do suggest that biological factors played
a role in the rapid spread of the D614G virus.

``This mutation may explain the predominance of viruses carrying it,''
Dr. Choe said.

But other scientists cautioned that it would take significantly more
research to determine if differences in the virus were a factor in
shaping the course of the outbreak. Other factors clearly played a role
in the spread, including the timing of lockdowns, travel patterns and
luck, scientists argue.

And luck alone may still be the best explanation for why viruses with
the mutation have become so widespread, they said.

Kristian Andersen, a geneticist at Scripps Research, La Jolla, said that
analyses of D614G and other variants in Washington and California had so
far found no difference in how quickly or widely one variant spread over
another.

``That's the main reason that I'm so hesitant at the moment,'' Dr.
Andersen said. ``Because if one really was able to spread significantly
better than the other, then we would expect to see a difference here,
and we don't.''

Tests of the Ebola virus, which spread in West Africa starting in 2013,
indicated that a common mutation
\href{https://jvi.asm.org/content/91/2/e01913-16}{infected more cells}
in cell cultures than its competitors, potentially suggesting that the
mutated virus was more contagious. But the difference
\href{https://www.cell.com/cell-reports/fulltext/S2211-1247(18)30569-2?_returnURL=https\%3A\%2F\%2Flinkinghub.elsevier.com\%2Fretrieve\%2Fpii\%2FS2211124718305692\%3Fshowall\%3Dtrue}{did
not hold up} when later tested in animals.

Scientists' attention had begun to focus on the D614G mutation by May,
when Bette Korber, a researcher at Los Alamos National Laboratory,
posted
\href{https://www.biorxiv.org/content/10.1101/2020.04.29.069054v2}{a
paper} arguing that ``when introduced to new regions it rapidly becomes
the dominant form.''

\href{https://www.nytimes.com/news-event/coronavirus?action=click\&pgtype=Article\&state=default\&region=MAIN_CONTENT_3\&context=storylines_faq}{}

\hypertarget{the-coronavirus-outbreak-}{%
\subsubsection{The Coronavirus Outbreak
›}\label{the-coronavirus-outbreak-}}

\hypertarget{frequently-asked-questions}{%
\paragraph{Frequently Asked
Questions}\label{frequently-asked-questions}}

Updated July 27, 2020

\begin{itemize}
\item ~
  \hypertarget{should-i-refinance-my-mortgage}{%
  \paragraph{Should I refinance my
  mortgage?}\label{should-i-refinance-my-mortgage}}

  \begin{itemize}
  \tightlist
  \item
    \href{https://www.nytimes.com/article/coronavirus-money-unemployment.html?action=click\&pgtype=Article\&state=default\&region=MAIN_CONTENT_3\&context=storylines_faq}{It
    could be a good idea,} because mortgage rates have
    \href{https://www.nytimes.com/2020/07/16/business/mortgage-rates-below-3-percent.html?action=click\&pgtype=Article\&state=default\&region=MAIN_CONTENT_3\&context=storylines_faq}{never
    been lower.} Refinancing requests have pushed mortgage applications
    to some of the highest levels since 2008, so be prepared to get in
    line. But defaults are also up, so if you're thinking about buying a
    home, be aware that some lenders have tightened their standards.
  \end{itemize}
\item ~
  \hypertarget{what-is-school-going-to-look-like-in-september}{%
  \paragraph{What is school going to look like in
  September?}\label{what-is-school-going-to-look-like-in-september}}

  \begin{itemize}
  \tightlist
  \item
    It is unlikely that many schools will return to a normal schedule
    this fall, requiring the grind of
    \href{https://www.nytimes.com/2020/06/05/us/coronavirus-education-lost-learning.html?action=click\&pgtype=Article\&state=default\&region=MAIN_CONTENT_3\&context=storylines_faq}{online
    learning},
    \href{https://www.nytimes.com/2020/05/29/us/coronavirus-child-care-centers.html?action=click\&pgtype=Article\&state=default\&region=MAIN_CONTENT_3\&context=storylines_faq}{makeshift
    child care} and
    \href{https://www.nytimes.com/2020/06/03/business/economy/coronavirus-working-women.html?action=click\&pgtype=Article\&state=default\&region=MAIN_CONTENT_3\&context=storylines_faq}{stunted
    workdays} to continue. California's two largest public school
    districts --- Los Angeles and San Diego --- said on July 13, that
    \href{https://www.nytimes.com/2020/07/13/us/lausd-san-diego-school-reopening.html?action=click\&pgtype=Article\&state=default\&region=MAIN_CONTENT_3\&context=storylines_faq}{instruction
    will be remote-only in the fall}, citing concerns that surging
    coronavirus infections in their areas pose too dire a risk for
    students and teachers. Together, the two districts enroll some
    825,000 students. They are the largest in the country so far to
    abandon plans for even a partial physical return to classrooms when
    they reopen in August. For other districts, the solution won't be an
    all-or-nothing approach.
    \href{https://bioethics.jhu.edu/research-and-outreach/projects/eschool-initiative/school-policy-tracker/}{Many
    systems}, including the nation's largest, New York City, are
    devising
    \href{https://www.nytimes.com/2020/06/26/us/coronavirus-schools-reopen-fall.html?action=click\&pgtype=Article\&state=default\&region=MAIN_CONTENT_3\&context=storylines_faq}{hybrid
    plans} that involve spending some days in classrooms and other days
    online. There's no national policy on this yet, so check with your
    municipal school system regularly to see what is happening in your
    community.
  \end{itemize}
\item ~
  \hypertarget{is-the-coronavirus-airborne}{%
  \paragraph{Is the coronavirus
  airborne?}\label{is-the-coronavirus-airborne}}

  \begin{itemize}
  \tightlist
  \item
    The coronavirus
    \href{https://www.nytimes.com/2020/07/04/health/239-experts-with-one-big-claim-the-coronavirus-is-airborne.html?action=click\&pgtype=Article\&state=default\&region=MAIN_CONTENT_3\&context=storylines_faq}{can
    stay aloft for hours in tiny droplets in stagnant air}, infecting
    people as they inhale, mounting scientific evidence suggests. This
    risk is highest in crowded indoor spaces with poor ventilation, and
    may help explain super-spreading events reported in meatpacking
    plants, churches and restaurants.
    \href{https://www.nytimes.com/2020/07/06/health/coronavirus-airborne-aerosols.html?action=click\&pgtype=Article\&state=default\&region=MAIN_CONTENT_3\&context=storylines_faq}{It's
    unclear how often the virus is spread} via these tiny droplets, or
    aerosols, compared with larger droplets that are expelled when a
    sick person coughs or sneezes, or transmitted through contact with
    contaminated surfaces, said Linsey Marr, an aerosol expert at
    Virginia Tech. Aerosols are released even when a person without
    symptoms exhales, talks or sings, according to Dr. Marr and more
    than 200 other experts, who
    \href{https://academic.oup.com/cid/article/doi/10.1093/cid/ciaa939/5867798}{have
    outlined the evidence in an open letter to the World Health
    Organization}.
  \end{itemize}
\item ~
  \hypertarget{what-are-the-symptoms-of-coronavirus}{%
  \paragraph{What are the symptoms of
  coronavirus?}\label{what-are-the-symptoms-of-coronavirus}}

  \begin{itemize}
  \tightlist
  \item
    Common symptoms
    \href{https://www.nytimes.com/article/symptoms-coronavirus.html?action=click\&pgtype=Article\&state=default\&region=MAIN_CONTENT_3\&context=storylines_faq}{include
    fever, a dry cough, fatigue and difficulty breathing or shortness of
    breath.} Some of these symptoms overlap with those of the flu,
    making detection difficult, but runny noses and stuffy sinuses are
    less common.
    \href{https://www.nytimes.com/2020/04/27/health/coronavirus-symptoms-cdc.html?action=click\&pgtype=Article\&state=default\&region=MAIN_CONTENT_3\&context=storylines_faq}{The
    C.D.C. has also} added chills, muscle pain, sore throat, headache
    and a new loss of the sense of taste or smell as symptoms to look
    out for. Most people fall ill five to seven days after exposure, but
    symptoms may appear in as few as two days or as many as 14 days.
  \end{itemize}
\item ~
  \hypertarget{does-asymptomatic-transmission-of-covid-19-happen}{%
  \paragraph{Does asymptomatic transmission of Covid-19
  happen?}\label{does-asymptomatic-transmission-of-covid-19-happen}}

  \begin{itemize}
  \tightlist
  \item
    So far, the evidence seems to show it does. A widely cited
    \href{https://www.nature.com/articles/s41591-020-0869-5}{paper}
    published in April suggests that people are most infectious about
    two days before the onset of coronavirus symptoms and estimated that
    44 percent of new infections were a result of transmission from
    people who were not yet showing symptoms. Recently, a top expert at
    the World Health Organization stated that transmission of the
    coronavirus by people who did not have symptoms was ``very rare,''
    \href{https://www.nytimes.com/2020/06/09/world/coronavirus-updates.html?action=click\&pgtype=Article\&state=default\&region=MAIN_CONTENT_3\&context=storylines_faq\#link-1f302e21}{but
    she later walked back that statement.}
  \end{itemize}
\end{itemize}

Many scientists criticized the study, saying that its analysis was not
sufficient to conclude that the virus with that mutation was more
transmissible in humans. The analysis did not adequately account for the
role of luck, they said: When a mutation seeds a new outbreak, it can
build an advantage by pure chance.

David Montefiori, a virologist at Duke University, said that he was
involved in a new analysis, led by Dr. Korber, addressing those
concerns. As part of that work, he said, his team at Duke had found lab
results that were very similar to those of the Scripps Research
scientists: Viruses with the D614G mutation infected cells more
efficiently than those without it. He said the full paper has been
submitted to a journal.

In the new research, the team led by Dr. Choe and Dr. Farzan performed
two experiments. In one, they created harmless proxy viruses using
standard tools, including retroviruses and so-called virus-like
particles. Each was engineered to have the signature corona ``spike''
proteins that allow it to fasten to the surface of cells like Velcro.

The scientists found that viruses with the D614G mutation were more
resilient, preserving about five times more functional spike proteins to
infect cells than viruses without that mutation.

In another experiment, they found that viruses carrying the D614G
mutation infected tissue cells far more efficiently than viruses without
the mutation. Dr. Farzan said that the difference probably stems from a
biological property of the mutation that confers more flexibility to the
spike protein and stabilizes it.

Coronavirus vaccines, once developed, should work as well against the
D614G variant as they do against others, the researchers said.

Outside experts said the new study, while impressive, left more work to
be done. Michael Letko, an assistant professor in the Laboratory of
Functional Viromics at Washington State University, said that other
biological factors could also influence the spread of the virus in the
real world. ``We focus on the part of the virus we know the best, the
spike, but we don't know as much about how other parts work,'' he said.

Still, Dr. Letko said, the new research was convincing in demonstrating
that viruses with the D614G mutation were more infectious in the lab.

Experts said the next step in determining if there are differences in
real-world transmission is to test different variants in animals.

``That's the incredible thing about viruses,'' Dr. Letko said. ``They're
called Darwinian machines, and these small changes can amplify quite
dramatically. These small gains can be just enough to allow a virus to
outcompete another virus that doesn't have these things.''

Advertisement

\protect\hyperlink{after-bottom}{Continue reading the main story}

\hypertarget{site-index}{%
\subsection{Site Index}\label{site-index}}

\hypertarget{site-information-navigation}{%
\subsection{Site Information
Navigation}\label{site-information-navigation}}

\begin{itemize}
\tightlist
\item
  \href{https://help.nytimes.com/hc/en-us/articles/115014792127-Copyright-notice}{©~2020~The
  New York Times Company}
\end{itemize}

\begin{itemize}
\tightlist
\item
  \href{https://www.nytco.com/}{NYTCo}
\item
  \href{https://help.nytimes.com/hc/en-us/articles/115015385887-Contact-Us}{Contact
  Us}
\item
  \href{https://www.nytco.com/careers/}{Work with us}
\item
  \href{https://nytmediakit.com/}{Advertise}
\item
  \href{http://www.tbrandstudio.com/}{T Brand Studio}
\item
  \href{https://www.nytimes.com/privacy/cookie-policy\#how-do-i-manage-trackers}{Your
  Ad Choices}
\item
  \href{https://www.nytimes.com/privacy}{Privacy}
\item
  \href{https://help.nytimes.com/hc/en-us/articles/115014893428-Terms-of-service}{Terms
  of Service}
\item
  \href{https://help.nytimes.com/hc/en-us/articles/115014893968-Terms-of-sale}{Terms
  of Sale}
\item
  \href{https://spiderbites.nytimes.com}{Site Map}
\item
  \href{https://help.nytimes.com/hc/en-us}{Help}
\item
  \href{https://www.nytimes.com/subscription?campaignId=37WXW}{Subscriptions}
\end{itemize}
