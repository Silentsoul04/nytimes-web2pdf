Sections

SEARCH

\protect\hyperlink{site-content}{Skip to
content}\protect\hyperlink{site-index}{Skip to site index}

\href{https://www.nytimes.com/section/us}{U.S.}

\href{https://myaccount.nytimes.com/auth/login?response_type=cookie\&client_id=vi}{}

\href{https://www.nytimes.com/section/todayspaper}{Today's Paper}

\href{/section/us}{U.S.}\textbar{}Do Police Officers Make Schools Safer
or More Dangerous?

\url{https://nyti.ms/2AYI0lO}

\begin{itemize}
\item
\item
\item
\item
\item
\end{itemize}

\href{https://www.nytimes.com/news-event/george-floyd-protests-minneapolis-new-york-los-angeles?action=click\&pgtype=Article\&state=default\&region=TOP_BANNER\&context=storylines_menu}{Race
and America}

\begin{itemize}
\tightlist
\item
  \href{https://www.nytimes.com/interactive/2020/07/03/us/george-floyd-protests-crowd-size.html?action=click\&pgtype=Article\&state=default\&region=TOP_BANNER\&context=storylines_menu}{Black
  Lives Matter Movement}
\item
  \href{https://www.nytimes.com/interactive/2020/06/28/us/i-cant-breathe-police-arrest.html?action=click\&pgtype=Article\&state=default\&region=TOP_BANNER\&context=storylines_menu}{History
  of `I Can't Breathe'}
\item
  \href{https://www.nytimes.com/interactive/2020/06/10/upshot/black-lives-matter-attitudes.html?action=click\&pgtype=Article\&state=default\&region=TOP_BANNER\&context=storylines_menu}{How
  Public Opinion Shifted}
\item
  \href{https://www.nytimes.com/interactive/2020/07/16/us/black-lives-matter-protests-louisville-breonna-taylor.html?action=click\&pgtype=Article\&state=default\&region=TOP_BANNER\&context=storylines_menu}{45
  Days in Louisville}
\end{itemize}

Advertisement

\protect\hyperlink{after-top}{Continue reading the main story}

Supported by

\protect\hyperlink{after-sponsor}{Continue reading the main story}

\hypertarget{do-police-officers-make-schools-safer-or-more-dangerous}{%
\section{Do Police Officers Make Schools Safer or More
Dangerous?}\label{do-police-officers-make-schools-safer-or-more-dangerous}}

School resource officers were supposed to prevent mass shootings and
juvenile crime. But some schools are eliminating them amid a clamor from
students after George Floyd's death.

\includegraphics{https://static01.nyt.com/images/2020/06/12/us/12UNREST-SCHOOLPOLICE/merlin_173446230_ec2c6edc-cdc5-4c56-b27d-a8b50eafb1b3-articleLarge.jpg?quality=75\&auto=webp\&disable=upscale}

\href{https://www.nytimes.com/by/dana-goldstein}{\includegraphics{https://static01.nyt.com/images/2018/06/12/multimedia/author-dana-goldstein/author-dana-goldstein-thumbLarge.png}}

By \href{https://www.nytimes.com/by/dana-goldstein}{Dana Goldstein}

\begin{itemize}
\item
  June 12, 2020
\item
  \begin{itemize}
  \item
  \item
  \item
  \item
  \item
  \end{itemize}
\end{itemize}

The
\href{https://www.nytimes.com/news-event/george-floyd-protests-minneapolis-new-york-los-angeles}{national
reckoning over police violence} has spread to schools, with several
districts choosing in recent days to sever their relationships with
local police departments out of concern that the officers patrolling
their hallways represent more of a threat than a form of protection.

School districts in
\href{https://www.twincities.com/2020/06/03/minneapolis-ends-school-resource-officer-program-will-st-paul-be-next/}{Minneapolis},
\href{https://www.seattletimes.com/seattle-news/education/seattle-schools-chief-announces-one-year-suspension-of-partnership-with-seattle-police-department/?utm_source=twitter\&utm_medium=social\&utm_campaign=article_inset_1.1}{Seattle}
and
\href{https://www.oregonlive.com/education/2020/06/portland-superintendent-says-hes-discontinuing-school-resource-officer-program.html}{Portland,
Ore.}, have all promised to remove officers, with the Seattle
superintendent saying the presence of armed police officers ``prohibits
many students and staff from feeling fully safe.'' In Oakland, Calif.,
leaders
\href{https://www.sfchronicle.com/bayarea/article/Oakland-school-board-and-superintendent-back-15331729.php}{expressed
support} on Wednesday for eliminating the district's internal police
force, while the Denver Board of Education
\href{https://twitter.com/jennifermeckles/status/1271284860439556096}{voted
unanimously} on Thursday to
\href{https://co.chalkbeat.org/2020/6/5/21281902/majority-denver-school-board-support-removing-police-from-schools}{end
its police contract}.

In Los Angeles and Chicago, two of the country's three largest school
districts, teachers' unions are pushing to get the police out, showing a
willingness to confront another politically powerful, heavily unionized
profession.

Some teachers and students, African-Americans in particular, say they
consider officers on campus a danger, rather than a bulwark against
everything from fights to drug use to mass shootings.

There has been no shortage of episodes to back up their concerns. In
Orange County, Fla., in November, a school resource officer
\href{https://www.orlandosentinel.com/news/crime/os-ne-orange-school-resource-officer-removed-after-inappropriate-force-20191108-efwod23yu5az5i4xl2lt6dbk4y-story.html}{was
fired} after a video showed him grasping a middle school student's hair
and yanking her head back during an arrest after students fought near
school grounds. A few weeks later, an officer assigned to a school in
Vance County, N.C., lost his job after he
\href{https://www.buzzfeednews.com/article/juliareinstein/school-resource-officer-slammed-middle-school-north-vance}{repeatedly
slammed} an 11-year-old boy to the ground.

Nadera Powell, 17, said seeing officers in the hallways at Venice High
School in Los Angeles sent a clear message to black students like her:
``Don't get too comfortable, regardless of whether this school is your
second home. We have you on watch. We are able to take legal or even
physical action against you.''

During student walkouts to protest gun violence and push for climate
action over the past two years, some officers blocked students from
leaving school grounds or clashed verbally with protesters, she
recalled. At Fremont High School in another part of Los Angeles, where
the student body is
\href{https://nces.ed.gov/ccd/schoolsearch/school_detail.asp?Search=1\&DistrictID=0622710\&SchoolPageNum=33\&ID=062271003023}{about
90 percent Latino}, the police
\href{https://www.latimes.com/california/story/2019-11-12/fremont-high-school-fight-lapd-pepper-spray}{used
pepper spray} in November to break up a fight.

``All people who are of color here are looked at as a threat,'' Ms.
Powell said.

For years, activists have called on districts to rein in campus police.
They
\href{http://www.justicepolicy.org/uploads/justicepolicy/documents/educationunderarrest_fullreport.pdf}{cite
data} showing that mass shootings like those in Parkland, Fla., or
Newtown, Conn.,
\href{https://www.nytimes.com/2018/05/22/us/safe-school-shootings.html}{are
rare}, and that crime on school grounds has
\href{https://www.nytimes.com/2019/04/20/us/columbine-anniversary-school-violence-statistics.html}{generally
declined} in recent years.

The presence of officers in hallways has a profound impact on students
of color and those with disabilities, who, according to several analyses
and studies, are
\href{https://www.edweek.org/ew/projects/2017/policing-americas-schools/student-arrests.html\#/overview}{more
likely to be harshly punished} for ordinary misbehavior.

Still, efforts to remove school resource officers face the same pushback
as a
\href{https://www.nytimes.com/2020/06/08/us/unrest-defund-police.html}{broader
national effort} to reduce funding for police departments: resistance
from the police themselves, who are often politically powerful, and
concern from some parents and school officials that removing officers
could leave schools and students vulnerable.

In Oakland, Jumoke Hinton Hodge, a school board member, said that
although she strongly supported the Black Lives Matter movement, she
opposed the effort to eliminate district police officers. Those officers
are better equipped to work with teenagers than are the city police, who
could be called to schools more often if the district no longer had its
own force, she said.

The district's officers train to prevent school shootings, Ms. Hinton
Hodge said, and they respond to students who have reported sexual abuse
or are at risk of suicide. The proposal to eliminate the force felt
rushed, she said, and would leave the district without an adequate
safety plan.

``Are you here for the long haul, about a movement?'' she asked. ``Or
are you in a moment?''

In New York City last weekend,
\href{https://twitter.com/madinatoure/status/1269278624009551874}{hundreds
of teachers and students marched} in a protest calling for the police to
be removed from schools and replaced by a new crop of guidance
counselors and social workers. Mayor Bill de Blasio
\href{https://www.nytimes.com/2020/06/07/nyregion/deblasio-nypd-funding.html}{committed
to diverting some of the Police Department's funding} to social services
for children, but has so far not shown a willingness to significantly
reduce police presence in hallways.

Mayor Lori Lightfoot of Chicago has
\href{https://news.wttw.com/2020/06/05/lightfoot-rules-out-removing-police-officers-chicago-schools\#:~:text=Mayor\%20Lori\%20Lightfoot\%20will\%20not,can\%20negatively\%20impact\%20their\%20learning.}{rejected
calls} from the teachers' union and others to remove officers from
schools, saying they are needed to provide security.

Both mayors control their city's school systems. It is districts with
elected school boards, which are more independent from other local
government agencies, that are currently driving the wave of change.

Mo Canady, executive director of the National Association of School
Resource Officers, said he was disappointed by attempts to link school
policing to the killing of
\href{https://www.nytimes.com/article/george-floyd-who-is.html}{George
Floyd} in Minneapolis. He called Mr. Floyd's death during an arrest
``the most horrific police abuse situation I've seen in my career.''

Well-trained school resource officers operate more like counselors and
educators, Mr. Canady said, working with students to defuse peer
conflict and address issues such as drug and alcohol use. He suggested
that
\href{http://www.justicepolicy.org/uploads/justicepolicy/documents/educationunderarrest_fullreport.pdf}{disproportionate
discipline and arrest rates} for students of color and those with
disabilities could be driven by the actions of police officers coming
off the street to respond to one-off calls from schools, or by campus
officers who lack adequate training in concepts such as implicit bias.

``The message to the districts has to be, `Don't throw the baby out with
the bath water,''' Mr. Canady said.

But as schools face
\href{https://www.nytimes.com/2020/06/10/us/politics/virus-schools-funding-budget.html}{significant
budget cuts} brought about by
\href{https://www.nytimes.com/news-event/coronavirus}{the coronavirus
pandemic}, some students, educators and policymakers say it would be
wiser to hire psychologists to provide counseling and nurses to advise
students on drugs and alcohol, instead of training police officers to do
such tasks.

In Prince George's County, Md., outside of Washington, Joshua Omolola,
18, has marched to protest the killing of Mr. Floyd. Now, as the student
member of the Board of Education, he is supporting a proposal to remove
police officers from the county's schools, whose students are
predominantly black and Hispanic.

The millions the county spends annually on school policing should be
reallocated to mental health services, Mr. Omolola argued, to treat the
root causes of student behavioral problems.

\includegraphics{https://static01.nyt.com/images/2020/06/12/us/12UNREST-SCHOOLS-omolola/merlin_173474373_00898edf-3e59-4237-a3f9-6ecd6ed445d5-articleLarge.jpg?quality=75\&auto=webp\&disable=upscale}

Police departments have typically responded to calls from school
employees, but the everyday presence of officers in hallways did not
become widespread until the 1990s. That was when concern over mass
shootings, drug abuse and juvenile crime led federal and state officials
to offer local districts money to hire officers and purchase law
enforcement equipment, such as metal detectors.

By the 2013-14 school year, two-thirds of high school students, 45
percent of middle schoolers and 19 percent of elementary school students
attended a school with a police officer, according to a
\href{https://www.urban.org/urban-wire/prevalence-police-officers-us-schools\#:~:text=In\%20every\%20state\%2C\%20high\%20school,school\%20with\%20a\%20police\%20officer.}{2018
report from the Urban Institute}. Majority black and Hispanic schools
are more likely to have officers on campus than majority white schools.

But when the Congressional Research Service
\href{https://fas.org/sgp/crs/misc/R43126.pdf}{reported} on the
effectiveness of school resource officers in 2013, it concluded that
there was little rigorous research showing a connection between the
presence of police officers in schools and changes in crime or student
discipline rates.

Activists who have worked for years to remove officers from hallways
said they were shocked at the speed with which school districts were
promising significant change after Mr. Floyd's death. The coming weeks
may equal the impact of a decade's worth of incremental reforms,
according to Jasmine Dellafosse, an organizer in Stockton, Calif., east
of San Francisco, with the Gathering for Justice, a nonprofit group.

After the A.C.L.U. Foundation of Northern California and the state
Department of Justice investigated
\href{https://www.aclunc.org/our-work/legal-docket/aclu-northern-california-v-stockton-unified-school-district-school-prison}{harsh
discipline practices} in Stockton schools, the district police force
agreed last year to establish new restrictions on the use of force and
on when to arrest students.

Now the school board plans to consider, later this month, a resolution
to remove police officers entirely from schools and to reallocate their
budget to programs such as
\href{https://www.nytimes.com/2019/08/15/us/california-ethnic-studies.html}{ethnic
studies}, counseling and
\href{https://www.nytimes.com/2013/04/04/education/restorative-justice-programs-take-root-in-schools.html}{restorative
justice}.

``There won't be real change,'' Ms. Dellafosse said, ``until police are
out of the schools.''

Eliza Shapiro and Erica L. Green contributed reporting.

Advertisement

\protect\hyperlink{after-bottom}{Continue reading the main story}

\hypertarget{site-index}{%
\subsection{Site Index}\label{site-index}}

\hypertarget{site-information-navigation}{%
\subsection{Site Information
Navigation}\label{site-information-navigation}}

\begin{itemize}
\tightlist
\item
  \href{https://help.nytimes.com/hc/en-us/articles/115014792127-Copyright-notice}{©~2020~The
  New York Times Company}
\end{itemize}

\begin{itemize}
\tightlist
\item
  \href{https://www.nytco.com/}{NYTCo}
\item
  \href{https://help.nytimes.com/hc/en-us/articles/115015385887-Contact-Us}{Contact
  Us}
\item
  \href{https://www.nytco.com/careers/}{Work with us}
\item
  \href{https://nytmediakit.com/}{Advertise}
\item
  \href{http://www.tbrandstudio.com/}{T Brand Studio}
\item
  \href{https://www.nytimes.com/privacy/cookie-policy\#how-do-i-manage-trackers}{Your
  Ad Choices}
\item
  \href{https://www.nytimes.com/privacy}{Privacy}
\item
  \href{https://help.nytimes.com/hc/en-us/articles/115014893428-Terms-of-service}{Terms
  of Service}
\item
  \href{https://help.nytimes.com/hc/en-us/articles/115014893968-Terms-of-sale}{Terms
  of Sale}
\item
  \href{https://spiderbites.nytimes.com}{Site Map}
\item
  \href{https://help.nytimes.com/hc/en-us}{Help}
\item
  \href{https://www.nytimes.com/subscription?campaignId=37WXW}{Subscriptions}
\end{itemize}
