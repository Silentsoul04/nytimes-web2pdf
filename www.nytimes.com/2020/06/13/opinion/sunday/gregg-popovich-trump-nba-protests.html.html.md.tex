Sections

SEARCH

\protect\hyperlink{site-content}{Skip to
content}\protect\hyperlink{site-index}{Skip to site index}

\href{https://www.nytimes.com/section/opinion/sunday}{Sunday Review}

\href{https://myaccount.nytimes.com/auth/login?response_type=cookie\&client_id=vi}{}

\href{https://www.nytimes.com/section/todayspaper}{Today's Paper}

\href{/section/opinion/sunday}{Sunday Review}\textbar{}An Anti-Trump
Slam Dunk

\href{https://nyti.ms/37rTf2a}{https://nyti.ms/37rTf2a}

\begin{itemize}
\item
\item
\item
\item
\item
\item
\end{itemize}

Advertisement

\protect\hyperlink{after-top}{Continue reading the main story}

\href{/section/opinion}{Opinion}

Supported by

\protect\hyperlink{after-sponsor}{Continue reading the main story}

\hypertarget{an-anti-trump-slam-dunk}{%
\section{An Anti-Trump Slam Dunk}\label{an-anti-trump-slam-dunk}}

On the issue of race, America's Coach boxes out America's Cretin.

\href{https://www.nytimes.com/by/maureen-dowd}{\includegraphics{https://static01.nyt.com/images/2018/04/02/opinion/maureen-dowd/maureen-dowd-thumbLarge.png}}

By \href{https://www.nytimes.com/by/maureen-dowd}{Maureen Dowd}

Opinion Columnist

\begin{itemize}
\item
  June 13, 2020
\item
  \begin{itemize}
  \item
  \item
  \item
  \item
  \item
  \item
  \end{itemize}
\end{itemize}

\includegraphics{https://static01.nyt.com/images/2020/06/14/opinion/sunday/14Dowd/14Dowd-articleLarge.jpg?quality=75\&auto=webp\&disable=upscale}

WASHINGTON --- ``Shut up and dribble.''

Those four words sum up the attitude of Donald Trump and his acolytes
toward athletes who speak out when the president uses sports to foment
racial animosity and rile up his base.

LeBron James,
\href{https://www.nytimes.com/2020/06/10/us/politics/lebron-james-voting-rights.html}{who
has a new group} with other sports stars designed to protect and inspire
the black vote, dunked on Laura Ingraham the other week.
\href{https://twitter.com/KingJames/status/1268616817544531969}{He
tweeted}: ``If you still haven't figured out why the protesting is going
on. Why we're acting as we are,'' it's because of the utter fatigue with
disparities such as this: Back when King James told ESPN in 2018 that
Trump did not care about the people, comparing him to a bad coach,
Ingraham commanded him to ``Shut up and dribble.''

But Ingraham reacted quite differently to Drew Brees's recent comment
(\href{https://www.instagram.com/p/CBE4y_9Hj2S/}{since rescinded}) that
he would ``never agree with anybody disrespecting the flag,'' when asked
whether players should kneel this season. ``He's allowed to have his
view about what kneeling and the flag means to him,'' the Fox anchor
said.

The
\href{https://classic.esquire.com/article/1968/4/1/muhammad-ali}{classic
1968 Esquire cover} of Muhammad Ali shot through with arrows comes to
mind as we watch the dynamic between sports and politics become more
torrid in this season of racial pain and introspection. The two
indelible images of this American chapter are a quarterback kneeling on
the turf to protest police brutality and a policeman kneeling on a man's
neck in a rancid display of that brutality. (Trump's new campaign ad
mocks Joe Biden for kneeling.)

I've been trying for three years to talk to Gregg Popovich, the coach of
the San Antonio Spurs and the U.S.A. Olympics basketball team. At 71,
he's an N.B.A. legend who has long called race ``the elephant in the
room'' and argued that we are all just an ``accident of birth.'' He's a
passionate Trump critic thriving in a red state.

He graduated from the Air Force Academy with a degree in Soviet studies
and a yellow Corvette and toyed with the idea of a career in military
intelligence. He's a celebrated curmudgeon with sports reporters and an
oenophile.

Raised by a steelworker and a secretary at the Inland plant in Gary,
Ind., Popovich is as open-minded, principled and curious as Trump is
narrow-minded, unprincipled and incurious.

``Pop,'' as he's known, is very private, but he finally agreed to pop
off on a phone call. He wouldn't pose for a picture, however, explaining
that he should not be the focus.

He has spent 25 years in a dialogue about race with his teams. He took
players to see ``Hamilton'' on Broadway, Ava DuVernay in L.A., the
African-American Museum in D.C. and the National Civil Rights Museum in
Memphis. He gave players copies of ``Between the World and Me,'' by
Ta-Nehisi Coates.

``Especially if you're a white coach and you're coaching a group that's
largely black, you'd better gain their trust, you'd better be genuine,
you'd better understand their situation,'' he tells me. ``You'd better
understand where they grew up. Maybe there's a black kid from a prep
school. Maybe there's another black kid who saw his first murder when he
was 7 years old.''

But in recent calls with the Spurs' players and staff he has been amazed
at the level of hurt.

``It would bring you to tears,'' he says, his voice cracking. ``It's
even deeper than you thought, and that's what really made me start to
think: You're a privileged son of a bitch and you still don't \emph{get}
it as much as you think you do. You gotta work harder. You gotta be more
aware. You gotta be pushed and embarrassed. You've gotta call it out.''

He tells of a recent Zoom town hall with Spurs employees. ``A black
mother said, `My son is angry with me.' I said, `Why?' and she said,
`Well, because he's 16 and I'm basically lying to him and dragging my
feet and giving him excuses because I don't want to take him down to the
D.M.V. to get his driver's license because I don't want him in a car.'
So her own son is angry with her for that but doesn't realize that she's
scared to death for him.''

I wonder if the former Air Force officer thinks the law-and-order
militaristic approach can work for Cadet Bone Spurs in the campaign.

``I honestly do,'' he says. ``I feel badly for the military around Trump
because they're dealing with the guy who is the poster boy for the
aggrieved wannabe. And he's taking it out on the world and it's ruining
our country.''

About Trump's refusal to consider renaming military installations named
for Confederate leaders, Pop says of U.S. soldiers, ``They didn't go to
war for General Bragg; they went to war for our country.''

About Roger Goodell's mea culpa that the N.F.L. was wrong for not
listening sooner to players who wanted to speak out and protest --- an
apologia he made without mentioning Colin Kaepernick's name --- Popovich
is skeptical.

``A smart man is running the N.F.L. and he didn't understand the
difference between the flag and what makes the country great --- all the
people who fought to allow Kaepernick to have the right to kneel for
justice,'' he says. ``The flag is irrelevant. It's just a symbol that
people glom onto for political reasons, just like Cheney back in the
Iraq war.''

He continues about Goodell: ``He got intimidated when Trump jumped on
the kneeling'' and ``he folded.'' Popovich says it is analogous to
Republican lawmakers who support Trump out of fear ``that they'll get
tweeted out of their office and not get elected the next go-round.''
Don't Tom Cotton and Lindsey Graham have people at home they are
embarrassed to look in the eye, he ponders.

What does he think about the fact that seven N.F.L. owners, including
Jerry Jones and Robert Kraft, each gave a million to Trump's Inaugural
Committee?

``It's just hypocritical,'' he replies. ``It's incongruent. It doesn't
make sense. People aren't blind. Do you go to your staff and your
players and talk about injustices and democracy and how to protest? I
don't get it. I think they put themselves in a position that's
untenable.''

When he trashed Trump soon after the election, the suits at the Spurs
told him people were turning in their season tickets.

``I just said: `I don't care. If they don't come, I don't care. That's
the way it is,''' he recalls. ``From ownership, not one phone call, not
one look, about dialing it back.''

Is he worried about starting to play again on
\href{https://www.nytimes.com/2020/06/02/sports/basketball/disney-world-nba-sports-complex.html}{July
30 at Disney World}, with that other plague still on the loose?

He passes the ball to Adam Silver, the smooth N.B.A. commish.

``Ah, the Covid,'' the coach murmurs. ``I'm just counting on Adam to
make sure we're all safe.''

\emph{The Times is committed to publishing}
\href{https://www.nytimes.com/2019/01/31/opinion/letters/letters-to-editor-new-york-times-women.html}{\emph{a
diversity of letters}} \emph{to the editor. We'd like to hear what you
think about this or any of our articles. Here are some}
\href{https://help.nytimes.com/hc/en-us/articles/115014925288-How-to-submit-a-letter-to-the-editor}{\emph{tips}}\emph{.
And here's our email:}
\href{mailto:letters@nytimes.com}{\emph{letters@nytimes.com}}\emph{.}

\emph{Follow The New York Times Opinion section on}
\href{https://www.facebook.com/nytopinion}{\emph{Facebook}}\emph{,}
\href{http://twitter.com/NYTOpinion}{\emph{Twitter (@NYTopinion)}}
\emph{and}
\href{https://www.instagram.com/nytopinion/}{\emph{Instagram}}\emph{.}

Advertisement

\protect\hyperlink{after-bottom}{Continue reading the main story}

\hypertarget{site-index}{%
\subsection{Site Index}\label{site-index}}

\hypertarget{site-information-navigation}{%
\subsection{Site Information
Navigation}\label{site-information-navigation}}

\begin{itemize}
\tightlist
\item
  \href{https://help.nytimes.com/hc/en-us/articles/115014792127-Copyright-notice}{©~2020~The
  New York Times Company}
\end{itemize}

\begin{itemize}
\tightlist
\item
  \href{https://www.nytco.com/}{NYTCo}
\item
  \href{https://help.nytimes.com/hc/en-us/articles/115015385887-Contact-Us}{Contact
  Us}
\item
  \href{https://www.nytco.com/careers/}{Work with us}
\item
  \href{https://nytmediakit.com/}{Advertise}
\item
  \href{http://www.tbrandstudio.com/}{T Brand Studio}
\item
  \href{https://www.nytimes.com/privacy/cookie-policy\#how-do-i-manage-trackers}{Your
  Ad Choices}
\item
  \href{https://www.nytimes.com/privacy}{Privacy}
\item
  \href{https://help.nytimes.com/hc/en-us/articles/115014893428-Terms-of-service}{Terms
  of Service}
\item
  \href{https://help.nytimes.com/hc/en-us/articles/115014893968-Terms-of-sale}{Terms
  of Sale}
\item
  \href{https://spiderbites.nytimes.com}{Site Map}
\item
  \href{https://help.nytimes.com/hc/en-us}{Help}
\item
  \href{https://www.nytimes.com/subscription?campaignId=37WXW}{Subscriptions}
\end{itemize}
