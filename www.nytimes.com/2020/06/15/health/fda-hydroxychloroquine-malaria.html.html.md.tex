Sections

SEARCH

\protect\hyperlink{site-content}{Skip to
content}\protect\hyperlink{site-index}{Skip to site index}

\href{https://www.nytimes.com/section/health}{Health}

\href{https://myaccount.nytimes.com/auth/login?response_type=cookie\&client_id=vi}{}

\href{https://www.nytimes.com/section/todayspaper}{Today's Paper}

\href{/section/health}{Health}\textbar{}F.D.A. Revokes Emergency
Approval of Malaria Drugs Promoted by Trump

\url{https://nyti.ms/2Au654f}

\begin{itemize}
\item
\item
\item
\item
\item
\end{itemize}

\href{https://www.nytimes.com/news-event/coronavirus?action=click\&pgtype=Article\&state=default\&region=TOP_BANNER\&context=storylines_menu}{The
Coronavirus Outbreak}

\begin{itemize}
\tightlist
\item
  live\href{https://www.nytimes.com/2020/08/02/world/coronavirus-updates.html?action=click\&pgtype=Article\&state=default\&region=TOP_BANNER\&context=storylines_menu}{Latest
  Updates}
\item
  \href{https://www.nytimes.com/interactive/2020/us/coronavirus-us-cases.html?action=click\&pgtype=Article\&state=default\&region=TOP_BANNER\&context=storylines_menu}{Maps
  and Cases}
\item
  \href{https://www.nytimes.com/interactive/2020/science/coronavirus-vaccine-tracker.html?action=click\&pgtype=Article\&state=default\&region=TOP_BANNER\&context=storylines_menu}{Vaccine
  Tracker}
\item
  \href{https://www.nytimes.com/interactive/2020/07/29/us/schools-reopening-coronavirus.html?action=click\&pgtype=Article\&state=default\&region=TOP_BANNER\&context=storylines_menu}{What
  School May Look Like}
\item
  \href{https://www.nytimes.com/live/2020/07/31/business/stock-market-today-coronavirus?action=click\&pgtype=Article\&state=default\&region=TOP_BANNER\&context=storylines_menu}{Economy}
\end{itemize}

Advertisement

\protect\hyperlink{after-top}{Continue reading the main story}

Supported by

\protect\hyperlink{after-sponsor}{Continue reading the main story}

\hypertarget{fda-revokes-emergency-approval-of-malaria-drugs-promoted-by-trump}{%
\section{F.D.A. Revokes Emergency Approval of Malaria Drugs Promoted by
Trump}\label{fda-revokes-emergency-approval-of-malaria-drugs-promoted-by-trump}}

The agency said that a review of some studies showed that the drugs'
potential benefits in treating Covid-19 did not outweigh the risks.

\includegraphics{https://static01.nyt.com/images/2020/06/15/science/15VIRUS-HCQ/15VIRUS-HCQ-articleLarge.jpg?quality=75\&auto=webp\&disable=upscale}

By \href{https://www.nytimes.com/by/katie-thomas}{Katie Thomas}

\begin{itemize}
\item
  June 15, 2020
\item
  \begin{itemize}
  \item
  \item
  \item
  \item
  \item
  \end{itemize}
\end{itemize}

The Food and Drug Administration said on Monday that it was
\href{https://www.fda.gov/media/138945/download}{revoking emergency
authorization} of two malaria drugs to treat Covid-19 in hospitalized
patients, saying that they are ``unlikely to be effective'' and could
carry potential risks.

The drugs,
\href{https://www.nytimes.com/2020/06/20/health/hydroxychloroquine-coronavirus-trial.html}{hydroxychloroquine
and chloroquine}, were heavily promoted by President Trump after a
handful of small, poorly controlled studies suggested that they could
work against the disease caused by the coronavirus. Mr. Trump said he
\href{https://www.nytimes.com/2020/06/03/us/politics/trump-physical-hydroxychloroquine.html}{took
hydroxychloroquine} after he had been exposed to two people who tested
positive for the coronavirus.

The agency said that after reviewing some data, it determined that the
drugs, particularly hydroxychloroquine, did not demonstrate potential
benefits that outweighed their risks. Earlier this spring, the F.D.A.
had also issued a
\href{https://www.nytimes.com/2020/04/24/health/fda-hydroxychloroquine-coronavirus.html}{warning
that the drugs could cause dangerous heart arrhythmias in Covid-19
patients.}

The review that led to the revocation found more than 100 cases of
serious heart disorders in Covid-19 patients taking the drugs, including
25 that were fatal. Other problems were linked to the drugs as well.

Lawmakers and some public health experts have criticized the Trump
administration for politicizing the government's medical and science
arms during the pandemic, and of pressuring agencies like the F.D.A. to
relax its standards for drugs and medical devices, to get them on the
market faster.

``The F.D.A. withdrew an emergency use authorization that never should
have been issued in the first place,'' Senator Ron Wyden, Democrat of
Oregon, said in a statement. ``By ignoring science and caving to
political pressure from the White House, the F.D.A. stoked false hope
and put American lives in danger, while damaging the agency's reputation
in the process.''

On Monday, Mr. Trump stood by his support of the drugs, saying at a
White House round table, ``All I know is that we've had some tremendous
reports.'' He added: ``It certainly didn't hurt me. I feel good.''

Dr. Peter Lurie, the president of the Center for Science in the Public
Interest, said the F.D.A.'s move showed ``how, in the end, science can
triumph over celebrity and unscientific pronouncements from the White
House. In the end, the truth comes out.''

Dr. Lurie said that while some clinical trials of hydroxychloroquine
were still underway, so far the evidence ``keeps going in the same
direction'' --- that the drug is not effective to treat Covid-19.

In March, the F.D.A. authorized hospitals to use stockpiles of the
drugs, which pharmaceutical companies had donated, to treat patients
with the virus. Doctors have always been able to prescribe the drugs to
individual patients as they see fit.

\hypertarget{latest-updates-global-coronavirus-outbreak}{%
\section{\texorpdfstring{\href{https://www.nytimes.com/2020/08/01/world/coronavirus-covid-19.html?action=click\&pgtype=Article\&state=default\&region=MAIN_CONTENT_1\&context=storylines_live_updates}{Latest
Updates: Global Coronavirus
Outbreak}}{Latest Updates: Global Coronavirus Outbreak}}\label{latest-updates-global-coronavirus-outbreak}}

Updated 2020-08-02T17:52:35.962Z

\begin{itemize}
\tightlist
\item
  \href{https://www.nytimes.com/2020/08/01/world/coronavirus-covid-19.html?action=click\&pgtype=Article\&state=default\&region=MAIN_CONTENT_1\&context=storylines_live_updates\#link-34047410}{The
  U.S. reels as July cases more than double the total of any other
  month.}
\item
  \href{https://www.nytimes.com/2020/08/01/world/coronavirus-covid-19.html?action=click\&pgtype=Article\&state=default\&region=MAIN_CONTENT_1\&context=storylines_live_updates\#link-780ec966}{Top
  U.S. officials work to break an impasse over the federal jobless
  benefit.}
\item
  \href{https://www.nytimes.com/2020/08/01/world/coronavirus-covid-19.html?action=click\&pgtype=Article\&state=default\&region=MAIN_CONTENT_1\&context=storylines_live_updates\#link-2bc8948}{Its
  outbreak untamed, Melbourne goes into even greater lockdown.}
\end{itemize}

\href{https://www.nytimes.com/2020/08/01/world/coronavirus-covid-19.html?action=click\&pgtype=Article\&state=default\&region=MAIN_CONTENT_1\&context=storylines_live_updates}{See
more updates}

More live coverage:
\href{https://www.nytimes.com/live/2020/07/31/business/stock-market-today-coronavirus?action=click\&pgtype=Article\&state=default\&region=MAIN_CONTENT_1\&context=storylines_live_updates}{Markets}

But in a letter on Monday revoking the authorization, the agency said
that further studies had shown that the two drugs were unlikely to be
effective in stopping the virus, and that national treatment guidelines
didn't recommend using them outside of clinical trials.

According to the letter, written by Denise M. Hinton, the F.D.A.'s chief
scientist, the request to revoke the authorization came from the
Biomedical Advanced Research and Development Authority, the unit of the
Department of Health and Human Services that had initially asked for the
authorization.

In April, the head of that unit, Dr. Rick Bright,
\href{https://www.nytimes.com/2020/04/22/us/politics/rick-bright-trump-hydroxychloroquine-coronavirus.html}{said
he was removed from his post} after he pushed for rigorous vetting of
hydroxychloroquine, even as Mr. Trump and his allies were
enthusiastically promoting it.

The use of hydroxychloroquine
\href{https://www.nytimes.com/2020/04/25/us/coronavirus-trump-chloroquine-hydroxychloroquine.html}{spiked
after Mr. Trump} continuously praised its potential, calling it a
possible ``game changer'' and
\href{https://www.nytimes.com/2020/03/20/health/coronavirus-chloroquine-trump.html}{saying,
``What the hell do you have to lose?}'' His repeated promotions during
daily briefings at the White House prompted runs on pharmacies,
threatening supplies for the drugs, which are also taken by people with
rheumatoid arthritis and lupus.

Alex M. Azar II, the secretary of health and human services, said at the
round table Monday that the F.D.A.'s action only ended the authorization
for hospitals to use federal stockpiles of the drugs on hospitalized
patients and noted that doctors could still prescribe the drugs to
patients.

``In fact the F.D.A. removal of the emergency use authorization takes
away what had been a significant misunderstanding by many that had made
people think it could only be used in a hospital setting,'' he said.

Interest in hydroxychloroquine has waned in recent weeks as further
studies showed that the drug did not appear to be effective in treating
or preventing Covid-19. Earlier this month,
\href{https://www.nytimes.com/2020/06/03/health/hydroxychloroquine-coronavirus-trump.html}{a
study of 821 people} who had been exposed to patients infected with the
virus showed that the drug did not prevent infection.

In May, an article in The Lancet about another study concluded that
hydroxychloroquine and chloroquine did not help patients and might have
harmed them ---
\href{https://www.nytimes.com/2020/06/04/health/coronavirus-hydroxychloroquine.html}{but
that study was later retracted} after the authors could not verify the
database of medical records on which the article was based.

As of May 6, there were 347 adverse events in Covid-19 patients taking
hydroxychloroquine, and 38 in those taking chloroquine (which is used
less often), the F.D.A. said, based on a search of its own database and
reports to poison-control centers. The majority of cases, 69 percent,
involved men with a median age in the early 60s.

The total included 109 serious heart problems, including 80 cases of a
serious heart rhythm disorder called QT prolongation. Other patients had
different rhythm abnormalities. Over all, 25 of the 109 died. Many who
had cardiac effects had been given other drugs at the same time, like
the antibiotic azithromycin, that can also cause QT prolongation.

There were also serious adverse events not affecting the heart in 113
cases, including liver abnormalities, which are listed on the drugs'
labeling as a possible problem. Some patients had severe kidney
problems, but renal disease has been linked to the coronavirus itself.

\href{https://www.nytimes.com/news-event/coronavirus?action=click\&pgtype=Article\&state=default\&region=MAIN_CONTENT_3\&context=storylines_faq}{}

\hypertarget{the-coronavirus-outbreak-}{%
\subsubsection{The Coronavirus Outbreak
›}\label{the-coronavirus-outbreak-}}

\hypertarget{frequently-asked-questions}{%
\paragraph{Frequently Asked
Questions}\label{frequently-asked-questions}}

Updated July 27, 2020

\begin{itemize}
\item ~
  \hypertarget{should-i-refinance-my-mortgage}{%
  \paragraph{Should I refinance my
  mortgage?}\label{should-i-refinance-my-mortgage}}

  \begin{itemize}
  \tightlist
  \item
    \href{https://www.nytimes.com/article/coronavirus-money-unemployment.html?action=click\&pgtype=Article\&state=default\&region=MAIN_CONTENT_3\&context=storylines_faq}{It
    could be a good idea,} because mortgage rates have
    \href{https://www.nytimes.com/2020/07/16/business/mortgage-rates-below-3-percent.html?action=click\&pgtype=Article\&state=default\&region=MAIN_CONTENT_3\&context=storylines_faq}{never
    been lower.} Refinancing requests have pushed mortgage applications
    to some of the highest levels since 2008, so be prepared to get in
    line. But defaults are also up, so if you're thinking about buying a
    home, be aware that some lenders have tightened their standards.
  \end{itemize}
\item ~
  \hypertarget{what-is-school-going-to-look-like-in-september}{%
  \paragraph{What is school going to look like in
  September?}\label{what-is-school-going-to-look-like-in-september}}

  \begin{itemize}
  \tightlist
  \item
    It is unlikely that many schools will return to a normal schedule
    this fall, requiring the grind of
    \href{https://www.nytimes.com/2020/06/05/us/coronavirus-education-lost-learning.html?action=click\&pgtype=Article\&state=default\&region=MAIN_CONTENT_3\&context=storylines_faq}{online
    learning},
    \href{https://www.nytimes.com/2020/05/29/us/coronavirus-child-care-centers.html?action=click\&pgtype=Article\&state=default\&region=MAIN_CONTENT_3\&context=storylines_faq}{makeshift
    child care} and
    \href{https://www.nytimes.com/2020/06/03/business/economy/coronavirus-working-women.html?action=click\&pgtype=Article\&state=default\&region=MAIN_CONTENT_3\&context=storylines_faq}{stunted
    workdays} to continue. California's two largest public school
    districts --- Los Angeles and San Diego --- said on July 13, that
    \href{https://www.nytimes.com/2020/07/13/us/lausd-san-diego-school-reopening.html?action=click\&pgtype=Article\&state=default\&region=MAIN_CONTENT_3\&context=storylines_faq}{instruction
    will be remote-only in the fall}, citing concerns that surging
    coronavirus infections in their areas pose too dire a risk for
    students and teachers. Together, the two districts enroll some
    825,000 students. They are the largest in the country so far to
    abandon plans for even a partial physical return to classrooms when
    they reopen in August. For other districts, the solution won't be an
    all-or-nothing approach.
    \href{https://bioethics.jhu.edu/research-and-outreach/projects/eschool-initiative/school-policy-tracker/}{Many
    systems}, including the nation's largest, New York City, are
    devising
    \href{https://www.nytimes.com/2020/06/26/us/coronavirus-schools-reopen-fall.html?action=click\&pgtype=Article\&state=default\&region=MAIN_CONTENT_3\&context=storylines_faq}{hybrid
    plans} that involve spending some days in classrooms and other days
    online. There's no national policy on this yet, so check with your
    municipal school system regularly to see what is happening in your
    community.
  \end{itemize}
\item ~
  \hypertarget{is-the-coronavirus-airborne}{%
  \paragraph{Is the coronavirus
  airborne?}\label{is-the-coronavirus-airborne}}

  \begin{itemize}
  \tightlist
  \item
    The coronavirus
    \href{https://www.nytimes.com/2020/07/04/health/239-experts-with-one-big-claim-the-coronavirus-is-airborne.html?action=click\&pgtype=Article\&state=default\&region=MAIN_CONTENT_3\&context=storylines_faq}{can
    stay aloft for hours in tiny droplets in stagnant air}, infecting
    people as they inhale, mounting scientific evidence suggests. This
    risk is highest in crowded indoor spaces with poor ventilation, and
    may help explain super-spreading events reported in meatpacking
    plants, churches and restaurants.
    \href{https://www.nytimes.com/2020/07/06/health/coronavirus-airborne-aerosols.html?action=click\&pgtype=Article\&state=default\&region=MAIN_CONTENT_3\&context=storylines_faq}{It's
    unclear how often the virus is spread} via these tiny droplets, or
    aerosols, compared with larger droplets that are expelled when a
    sick person coughs or sneezes, or transmitted through contact with
    contaminated surfaces, said Linsey Marr, an aerosol expert at
    Virginia Tech. Aerosols are released even when a person without
    symptoms exhales, talks or sings, according to Dr. Marr and more
    than 200 other experts, who
    \href{https://academic.oup.com/cid/article/doi/10.1093/cid/ciaa939/5867798}{have
    outlined the evidence in an open letter to the World Health
    Organization}.
  \end{itemize}
\item ~
  \hypertarget{what-are-the-symptoms-of-coronavirus}{%
  \paragraph{What are the symptoms of
  coronavirus?}\label{what-are-the-symptoms-of-coronavirus}}

  \begin{itemize}
  \tightlist
  \item
    Common symptoms
    \href{https://www.nytimes.com/article/symptoms-coronavirus.html?action=click\&pgtype=Article\&state=default\&region=MAIN_CONTENT_3\&context=storylines_faq}{include
    fever, a dry cough, fatigue and difficulty breathing or shortness of
    breath.} Some of these symptoms overlap with those of the flu,
    making detection difficult, but runny noses and stuffy sinuses are
    less common.
    \href{https://www.nytimes.com/2020/04/27/health/coronavirus-symptoms-cdc.html?action=click\&pgtype=Article\&state=default\&region=MAIN_CONTENT_3\&context=storylines_faq}{The
    C.D.C. has also} added chills, muscle pain, sore throat, headache
    and a new loss of the sense of taste or smell as symptoms to look
    out for. Most people fall ill five to seven days after exposure, but
    symptoms may appear in as few as two days or as many as 14 days.
  \end{itemize}
\item ~
  \hypertarget{does-asymptomatic-transmission-of-covid-19-happen}{%
  \paragraph{Does asymptomatic transmission of Covid-19
  happen?}\label{does-asymptomatic-transmission-of-covid-19-happen}}

  \begin{itemize}
  \tightlist
  \item
    So far, the evidence seems to show it does. A widely cited
    \href{https://www.nature.com/articles/s41591-020-0869-5}{paper}
    published in April suggests that people are most infectious about
    two days before the onset of coronavirus symptoms and estimated that
    44 percent of new infections were a result of transmission from
    people who were not yet showing symptoms. Recently, a top expert at
    the World Health Organization stated that transmission of the
    coronavirus by people who did not have symptoms was ``very rare,''
    \href{https://www.nytimes.com/2020/06/09/world/coronavirus-updates.html?action=click\&pgtype=Article\&state=default\&region=MAIN_CONTENT_3\&context=storylines_faq\#link-1f302e21}{but
    she later walked back that statement.}
  \end{itemize}
\end{itemize}

Four patients developed a blood disorder called methemoglobinemia, and
two died. That condition is not mentioned in the drugs' labeling, but
had been known as a rare side effect of some medications.

It was not possible to calculate rates of the adverse events, because
the total number of patients given the drugs was not known, the F.D.A.
said.

The agency
\href{https://www.fda.gov/media/137566/download?utm_campaign=FDA\%20Warns\%20of\%20Newly\%20Discovered\%20Potential\%20Drug\%20Interaction\%20That\%20May\%20Reduce\%20Effectiveness\&utm_medium=email\&utm_source=Eloqua}{also
issued a warning Monday} about combining hydroxychloroquine or
chloroquine with remdesivir, a recently authorized treatment for
patients with Covid-19. The F.DA. said the malaria drugs could interfere
with remdesivir's ability to fight the virus.

Several trials of hydroxychloroquine are still underway, including
additional studies of whether it can be used to prevent coronavirus
infection. The World Health Organization resumed a study of the drug
after briefly halting it in the wake of the Lancet article.

Two arms of the National Institutes of Health --- the National Heart,
Lung and Blood Institute and the National Institute for Allergy and
Infectious Diseases --- are conducting clinical trials of
hydroxychloroquine. Dr. Francis Collins, the N.I.H. director, said those
studies would continue.

``I think that would be unfortunate not to,'' Dr. Collins said. ``What's
been missing here are really well-designed, randomized
placebo-controlled trials for hospitalized patients.''

Hydroxychloroquine is still being embraced elsewhere,
\href{https://www.nytimes.com/2020/06/13/world/americas/virus-brazil-bolsonaro-chloroquine.html}{including
in Brazil,} which is battling an explosive outbreak.

Members of Congress
\href{https://www.warren.senate.gov/imo/media/doc/2020.05.06\%20Letter\%20to\%20FDA\%20re\%20data\%20tracking.pdf}{have
questioned} increases in the F.D.A.'s granting of emergency use
authorizations during the pandemic for certain drugs as potential
treatments. They have also questioned authorizations for antibody and
diagnostic tests whose data had not been thoroughly vetted before
approval, and for certain types of masks and other devices.

Some Democratic lawmakers have criticized the Trump administration for
pressuring the agency into issuing too many emergency approvals.

In some cases, the F.D.A. has recently rescinded emergency
approvals\href{https://www.nytimes.com/2020/06/07/science/masks-china-coronavirus.html}{for
use or reuse of some masks} and told companies that
\href{https://www.nytimes.com/2020/05/04/health/fda-antibody-tests-coronavirus.html}{did
not meet a deadline for submitting data} on antibody tests that they
should not be selling them. The Government Accountability Office
recently testified that it planned to look into the F.D.A.'s emergency
authorizations.

On Monday, Democrats framed the news as further evidence that Mr. Trump
cannot be relied upon in the coronavirus pandemic. Senator Chuck
Schumer, the minority leader from New York, said on Twitter, ``On
medical issues like on so much else, he doesn't know what he is talking
about.''

Denise Grady, Sheryl Gay Stolberg and Michael Crowley contributed
reporting.

Advertisement

\protect\hyperlink{after-bottom}{Continue reading the main story}

\hypertarget{site-index}{%
\subsection{Site Index}\label{site-index}}

\hypertarget{site-information-navigation}{%
\subsection{Site Information
Navigation}\label{site-information-navigation}}

\begin{itemize}
\tightlist
\item
  \href{https://help.nytimes.com/hc/en-us/articles/115014792127-Copyright-notice}{©~2020~The
  New York Times Company}
\end{itemize}

\begin{itemize}
\tightlist
\item
  \href{https://www.nytco.com/}{NYTCo}
\item
  \href{https://help.nytimes.com/hc/en-us/articles/115015385887-Contact-Us}{Contact
  Us}
\item
  \href{https://www.nytco.com/careers/}{Work with us}
\item
  \href{https://nytmediakit.com/}{Advertise}
\item
  \href{http://www.tbrandstudio.com/}{T Brand Studio}
\item
  \href{https://www.nytimes.com/privacy/cookie-policy\#how-do-i-manage-trackers}{Your
  Ad Choices}
\item
  \href{https://www.nytimes.com/privacy}{Privacy}
\item
  \href{https://help.nytimes.com/hc/en-us/articles/115014893428-Terms-of-service}{Terms
  of Service}
\item
  \href{https://help.nytimes.com/hc/en-us/articles/115014893968-Terms-of-sale}{Terms
  of Sale}
\item
  \href{https://spiderbites.nytimes.com}{Site Map}
\item
  \href{https://help.nytimes.com/hc/en-us}{Help}
\item
  \href{https://www.nytimes.com/subscription?campaignId=37WXW}{Subscriptions}
\end{itemize}
