Sections

SEARCH

\protect\hyperlink{site-content}{Skip to
content}\protect\hyperlink{site-index}{Skip to site index}

\href{https://myaccount.nytimes.com/auth/login?response_type=cookie\&client_id=vi}{}

\href{https://www.nytimes.com/section/todayspaper}{Today's Paper}

\href{/section/opinion}{Opinion}\textbar{}Market Madness in the Pandemic

\href{https://nyti.ms/2YH9LaE}{https://nyti.ms/2YH9LaE}

\begin{itemize}
\item
\item
\item
\item
\item
\item
\end{itemize}

Advertisement

\protect\hyperlink{after-top}{Continue reading the main story}

\href{/section/opinion}{Opinion}

Supported by

\protect\hyperlink{after-sponsor}{Continue reading the main story}

\hypertarget{market-madness-in-the-pandemic}{%
\section{Market Madness in the
Pandemic}\label{market-madness-in-the-pandemic}}

Why are investors rushing to buy junk?

\href{https://www.nytimes.com/by/paul-krugman}{\includegraphics{https://static01.nyt.com/images/2018/04/02/opinion/paul-krugman/paul-krugman-thumbLarge.png}}

By \href{https://www.nytimes.com/by/paul-krugman}{Paul Krugman}

Opinion Columnist

\begin{itemize}
\item
  June 15, 2020
\item
  \begin{itemize}
  \item
  \item
  \item
  \item
  \item
  \item
  \end{itemize}
\end{itemize}

\includegraphics{https://static01.nyt.com/images/2020/06/15/opinion/15krugman1/merlin_147997977_1ccca318-b522-42e5-8f1c-9a68d1d0d9b0-articleLarge.jpg?quality=75\&auto=webp\&disable=upscale}

After all these years, Hertz is No. 1 again. Not in market share: The
car-rental company is a
\href{https://www.statista.com/statistics/1022011/car-rental-companies-market-share-united-states/}{distant
second} to Enterprise. But Hertz has become Exhibit \#1 of the madness
that has been sweeping the stock market in these times of Covid-19 --- a
madness that may do considerable harm, not because stock prices
themselves matter all that much, but because Donald Trump and his
minions treat the stock market as a measure of their success.

About Hertz: Last month the company, which is deeply in debt and has
seen its business plunge amid the pandemic, filed for Chapter 11
protection. This is a form of bankruptcy that keeps a company operating
by restructuring its debts.

But while companies that enter Chapter 11 often survive, their
stockholders are normally wiped out. So Hertz stock should have become
more or less worthless.

Sure enough, Hertz's \href{https://finance.yahoo.com/quote/HTZ/}{stock
price} fell from more than \$20 in February to less than \$1 in early
June. But then a funny thing happened: Investors suddenly piled into the
stock, driving it up by more than 500 percent. And Hertz --- in
bankruptcy! --- announced plans to raise money by
\href{https://www.marketwatch.com/story/hertz-seeks-bankruptcy-court-approval-to-offer-1-billion-in-stock-but-experts-expect-equity-to-be-wiped-out-2020-06-12}{selling
more stock}.

The Hertz story was just one example of a broader phenomenon. The run-up
\href{https://fred.stlouisfed.org/graph/fredgraph.png?g=rDaj}{in stock
prices} that took place between mid-May and Thursday's sudden plummet
was driven, to an important extent, by investors rushing into very
dubious companies --- what one observer called a
``\href{https://t.co/18VTGtR2rj?amp=1}{flight to crap}.''

Stock markets never bear much relationship to the real economy, but
these days they don't seem to have much to do with reality in general.

So what is going on in the market? Think of it as a play in three acts
(so far).

The first act was the huge decline that markets experienced as the
threat from Covid-19 became clear. This decline reflected justified
concerns about future profits, but it also reflected a developing
financial crisis: For a few weeks credit markets were seizing up pretty
much the same way they did in 2008.

The Federal Reserve, however, has been there and done that. It moved
quickly, buying bonds, establishing special lending facilities, and
essentially doing whatever it took to lubricate markets and keep money
flowing freely.

The result was the second act of the play, a stock rebound that made up
about half of the losses from the initial plunge.

Up to that point the behavior of stock prices generally made sense. But
then came the third act, a surge in prices that eliminated most of the
previous losses and drove the Nasdaq to a new high. And this surge bore
all the usual signs of a bubble.

Robert Shiller, the world's leading expert on such things, has
\href{https://www.jstor.org/stable/3216841?seq=1\#metadata_info_tab_contents}{pointed
out} that asset bubbles are, in effect, naturally occurring Ponzi
schemes. Early investors see big gains because later investors drive
prices up, inducing more people to buy in, and so on; the party
continues until something cuts off the flow of new money, and suddenly
everything crashes.

So it was with the recent stock surge. Encouraged by the Fed-induced
recovery of stocks from their March lows, some investors began buying.
Their optimism became a self-fulfilling prophecy, as initial gains led
more cautious investors to join in, driven by FOMO --- fear of missing
out. It looked a lot like the dot-com bubble of the 1990s, except on a
vastly accelerated timetable.

Although there is
\href{https://www.marketwatch.com/story/its-like-the-wild-west-in-the-stock-market-with-the-get-rich-crowd-vs-wall-st-pros-but-its-too-easy-to-blame-retail-investors-for-rampant-speculation-2020-06-13}{some
dispute} about how important they were, most of the evidence suggests
that a major role in this apparent bubble was played by small investors
--- ``retail bros'' --- pursuing get-rich-quick dreams. Some of these
exuberant investors were people who
\href{https://www.nytimes.com/2020/06/14/business/sports-gamblers-stocks-virus.html}{normally
bet on sports} and were looking for an alternative source of excitement.
And as the Hertz example shows, they didn't care much about quality.

Why didn't large investors offset this apparent irrational exuberance by
selling stocks? As John Maynard Keynes
\href{http://gutenberg.net.au/ebooks03/0300071h/chap12.html}{argued long
ago}, staid investors who usually stabilize the market tend to abdicate
judgment in ``abnormal times.'' We are, you might say, in a time when
the smart money lacks all conviction, while the dumb money is filled
with a passionate intensity.

And now the bubble may --- may --- be bursting. But does any of this
matter?

In a direct sense, not much. Stock prices surely have some impact on
business investment and consumer spending, but these effects are
probably small.

But the Trump team sees stock prices as the ultimate measure of policy
success. Back in 2007 --- on the eve of the Great Recession --- Larry
Kudlow, who is now Trump's top economist,
\href{https://www.realclearpolitics.com/articles/2007/07/a_stock_market_vote_of_confide.html}{declared}
that things were going great, because the market was up, and stock
prices are ``the best barometer of the health, wealth and security of a
nation.''

So the Trumpists took the rising market as validation for everything
they were doing --- their push for early reopening even though the
coronavirus was by no means contained, their opposition to further
relief for unemployed workers. In other words, the irrational exuberance
of the retail bros may have enabled the irresponsibility of an
administration that didn't want to deal with reality in the first place.

And while falling stocks may provoke a reconsideration, a lot of damage
has already been done.

\emph{The Times is committed to publishing}
\href{https://www.nytimes.com/2019/01/31/opinion/letters/letters-to-editor-new-york-times-women.html}{\emph{a
diversity of letters}} \emph{to the editor. We'd like to hear what you
think about this or any of our articles. Here are some}
\href{https://help.nytimes.com/hc/en-us/articles/115014925288-How-to-submit-a-letter-to-the-editor}{\emph{tips}}\emph{.
And here's our email:}
\href{mailto:letters@nytimes.com}{\emph{letters@nytimes.com}}\emph{.}

\emph{Follow The New York Times Opinion section on}
\href{https://www.facebook.com/nytopinion}{\emph{Facebook}}\emph{,}
\href{http://twitter.com/NYTOpinion}{\emph{Twitter (@NYTopinion)}}
\emph{and}
\href{https://www.instagram.com/nytopinion/}{\emph{Instagram}}\emph{.}

Advertisement

\protect\hyperlink{after-bottom}{Continue reading the main story}

\hypertarget{site-index}{%
\subsection{Site Index}\label{site-index}}

\hypertarget{site-information-navigation}{%
\subsection{Site Information
Navigation}\label{site-information-navigation}}

\begin{itemize}
\tightlist
\item
  \href{https://help.nytimes.com/hc/en-us/articles/115014792127-Copyright-notice}{©~2020~The
  New York Times Company}
\end{itemize}

\begin{itemize}
\tightlist
\item
  \href{https://www.nytco.com/}{NYTCo}
\item
  \href{https://help.nytimes.com/hc/en-us/articles/115015385887-Contact-Us}{Contact
  Us}
\item
  \href{https://www.nytco.com/careers/}{Work with us}
\item
  \href{https://nytmediakit.com/}{Advertise}
\item
  \href{http://www.tbrandstudio.com/}{T Brand Studio}
\item
  \href{https://www.nytimes.com/privacy/cookie-policy\#how-do-i-manage-trackers}{Your
  Ad Choices}
\item
  \href{https://www.nytimes.com/privacy}{Privacy}
\item
  \href{https://help.nytimes.com/hc/en-us/articles/115014893428-Terms-of-service}{Terms
  of Service}
\item
  \href{https://help.nytimes.com/hc/en-us/articles/115014893968-Terms-of-sale}{Terms
  of Sale}
\item
  \href{https://spiderbites.nytimes.com}{Site Map}
\item
  \href{https://help.nytimes.com/hc/en-us}{Help}
\item
  \href{https://www.nytimes.com/subscription?campaignId=37WXW}{Subscriptions}
\end{itemize}
