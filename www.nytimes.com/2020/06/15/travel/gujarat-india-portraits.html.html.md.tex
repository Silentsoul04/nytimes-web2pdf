Sections

SEARCH

\protect\hyperlink{site-content}{Skip to
content}\protect\hyperlink{site-index}{Skip to site index}

\href{/section/travel}{Travel}\textbar{}Portraits of Everyday Life in
the Indian State of Gujarat

\url{https://nyti.ms/3d0EjcM}

\begin{itemize}
\item
\item
\item
\item
\item
\item
\end{itemize}

\href{https://www.nytimes.com/news-event/coronavirus?action=click\&pgtype=Article\&state=default\&region=TOP_BANNER\&context=storylines_menu}{The
Coronavirus Outbreak}

\begin{itemize}
\tightlist
\item
  live\href{https://www.nytimes.com/2020/08/04/world/coronavirus-covid-19.html?action=click\&pgtype=Article\&state=default\&region=TOP_BANNER\&context=storylines_menu}{Latest
  Updates}
\item
  \href{https://www.nytimes.com/interactive/2020/us/coronavirus-us-cases.html?action=click\&pgtype=Article\&state=default\&region=TOP_BANNER\&context=storylines_menu}{Maps
  and Cases}
\item
  \href{https://www.nytimes.com/interactive/2020/science/coronavirus-vaccine-tracker.html?action=click\&pgtype=Article\&state=default\&region=TOP_BANNER\&context=storylines_menu}{Vaccine
  Tracker}
\item
  \href{https://www.nytimes.com/2020/08/02/us/covid-college-reopening.html?action=click\&pgtype=Article\&state=default\&region=TOP_BANNER\&context=storylines_menu}{College
  Reopening}
\item
  \href{https://www.nytimes.com/live/2020/08/03/business/stock-market-today-coronavirus?action=click\&pgtype=Article\&state=default\&region=TOP_BANNER\&context=storylines_menu}{Economy}
\end{itemize}

\includegraphics{https://static01.nyt.com/images/2020/06/20/travel/20travel-india5/merlin_173198832_0e7aae21-6dd8-43c5-b53f-7943fd89d082-articleLarge.jpg?quality=75\&auto=webp\&disable=upscale}

The World Through a Lens

\hypertarget{portraits-of-everyday-life-in-the-indian-state-of-gujarat}{%
\section{Portraits of Everyday Life in the Indian State of
Gujarat}\label{portraits-of-everyday-life-in-the-indian-state-of-gujarat}}

In India's westernmost state, philosophies of asceticism live
side-by-side with those of raw capitalism.

A barbershop in Rajkot.Credit...

Supported by

\protect\hyperlink{after-sponsor}{Continue reading the main story}

Photographs and Text by Michael Benanav

\begin{itemize}
\item
  Published June 15, 2020Updated June 18, 2020
\item
  \begin{itemize}
  \item
  \item
  \item
  \item
  \item
  \item
  \end{itemize}
\end{itemize}

\emph{With travel restrictions in place worldwide, we've launched a new
series,}
\href{https://www.nytimes.com/column/the-world-through-a-lens}{\emph{The
World Through a Lens}}\emph{, in which photojournalists help transport
you, virtually, to some of our planet's most beautiful and intriguing
places. This week, Michael Benanav shares a collection of portraits from
Gujarat, a state in western India.}

\begin{center}\rule{0.5\linewidth}{\linethickness}\end{center}

With around 10,000 cases reported daily, India ranks
\href{https://www.nytimes.com/2020/06/10/world/asia/reopening-before-coronavirus-ends.html}{third
in the world in new coronavirus infections}, behind the United States
and Brazil. The city of Ahmedabad ---~the largest metropolis in Gujarat,
one of the country's
\href{https://www.bbc.com/news/world-asia-india-53009560}{hardest-hit
states} --- lags only Mumbai in the total number of
\href{https://timesofindia.indiatimes.com/india/ahmedabad-has-indias-highest-corona-deaths/m-population/articleshow/76226143.cms}{Covid-related
deaths}.

PAKISTAN

New Delhi

RAJASTHAN

INDIA

Little Rann of Kutch

MADHYA

PRADESH

Mandvi

Ahmedabad

GUJARAT

Bhavnagar

Arabian

Sea

Mumbai

200 miles

CHINA

PAK.

NEPAL

New

Delhi

Detail

area

INDIA

Bay of

Bengal

Arabian

Sea

400 miles

By The New York Times

Though my first glimpses of India were in Delhi and Rajasthan, my
experiences of the country were largely superficial until I got to
Gujarat --- the country's westernmost state, which sticks like an elbow
into the Arabian Sea. It's a place that defies easy generalizations,
inextricably linked to the contradictory legacies of Gandhi's nonviolent
\href{http://blogs.law.columbia.edu/uprising1313/bernard-e-harcourt-introduction-to-satyagraha/}{satyagraha}
movement, which was launched there in 1930, and the brutal
\href{https://www.nytimes.com/interactive/2014/04/06/world/asia/modi-gujarat-riots-timeline.html}{Godhra
Riots of 2002}, during which more than 1,000 people --- most of whom
were Muslim --- were killed by raging mobs in one of India's worst
explosions of communal upheaval since
\href{https://www.newyorker.com/magazine/2015/06/29/the-great-divide-books-dalrymple}{Partition}.

\includegraphics{https://static01.nyt.com/images/2020/06/16/travel/16travel-india-01/merlin_173199267_7ba0bd38-d07b-46c5-afdf-db08d6de7ae2-articleLarge.jpg?quality=75\&auto=webp\&disable=upscale}

Image

A girl dressed for a festival celebrating Muharram, the first month of
the Islamic calendar.

Here, philosophies of asceticism live side-by-side with those of raw
capitalism. In the city of Ahmedabad, with an estimated population of 8
million, exquisite examples of centuries-old architecture stand near
gleaming modern structures and tarp-covered slums. The state is rightly
renowned for its exceptional textile arts, its food and as the last home
of the endangered Asiatic lion (though there are plans to move some to
neighboring Madhya Pradesh).

Image

A shepherd in western Gujarat.

Image

A vegetable seller.

I first went to Gujarat in 2006 to work on a project about the nomadic
Maldhari tribes that roam the countryside herding cows, buffaloes,
camels, goats and sheep. A local NGO helped open their world to me and,
before long, I found myself in remote villages, drinking chai from
saucers, listening to the stories of people's lives, attending weddings
and stopping in random places to talk to groups who were camped in the
bush while migrating.

Image

A woman from Gujarat's Maldhari community makes chai.~

Image

Men wash before prayers at the Jama Masjid, an impressive 15th-century
mosque in Ahmedabad's old city.

Aside from learning about the struggles that these communities face in a
rapidly changing country, I met people proud of their way of life who
have a deep connection to their animals. The Maldhari are famed in
Gujarat for the quality of their dairy products, and they know their
herds so intimately, I was told, that they can tell from one sip of milk
exactly which buffalo or cow it came from. Through many conversations
covering a wide range of subjects, the complexities of Indian life and
politics and caste and religion slowly came into focus.

Image

A salt farmer in Gujarat's Little Rann of Kutch takes a tea break in the
rudimentary shelter he shares with his family.

I also became unexpectedly close to the couple that ran the
Ahmedabad-based \href{http://www.marag.org/}{Maldhari Rural Action
Group} and members of their family --- particularly two journalism
professors. Aside from everything I learned about India that can only be
gleaned through time spent with families, I left with lifelong friends.

Image

A man dressed as Vasudeva --- the father of the god Krishna --- carries
the baby deity to safety in a re-enactment celebrating Krishna's birth
in a Janmashtami parade.

Image

A Maldhari cattle herder relaxes with a bidi.

In between projects and assignments that took me to other parts of the
country, I continued to return to Gujarat over the years, delving
further into Maldhari culture and visiting all corners of the state
while updating the Lonely Planet India guidebook. More than any
historical sights or natural wonders about which the tourism department
may boast, my encounters with Gujarat's people have always been the
highlights of my experiences there.

Image

A girl from the Maldhari community poses for a photograph.

There was Lavuben Rozia, who had one of the most brilliant smiles I've
ever seen, or photographed. Along with a handful of family members, she
was migrating with a herd of cattle, seeking whatever patches of ground
they could find that hadn't been desiccated by drought.

Image

Lavuben Rozia, a Rabari cattle herder. Severe drought forced her to
migrate during the season when rains usually allow her to stay in one
place with her family.

She explained that they normally move for about eight months of the
year, then return to their village for four --- but that year, 2012, the
monsoon rains never came. Typically, she said, the whole family travels
together, which is why her children can't go to school. But she had now
left her older three children in their village with her mother-in-law,
as this ``special migration,'' she said, was particularly tough. Her
sister-in-law, Puriben Rozia, put it succinctly: ``Every day a new fire
hearth; every day a new well.''

There was Prabhubhai Kalar, 19 years old, from the Rabari tribe, who was
to be married on the day I met him. Dressed in a robe and crown
embroidered with purple sequins, with heavy lines of kohl below his
eyes, he was more ambivalent than his attire implied.

Image

Prabhubhai Kalar on the day of his wedding.

He would meet his wife for the very first time at the wedding, he
explained --- but he wouldn't see her face until later that night. The
union had been arranged by his and the bride's family. One of the main
qualities that the Rabari look for when they make matches for their
children is called najar, meaning foresight, or the ability to think
ahead and plan for the future. Prabhubhai was said to have lots of
najar, because he was in school, aiming for a career in a medical field
rather than the fields where his ancestors grazed their livestock.

Image

A camel herder at dusk.

There was Dilip Asher, who began chatting with me on the street in the
coastal city of Mandvi. The 65-year-old chess coach, with black glasses,
a gray mustache and a few missing teeth, invited me to his home: a
mansion long past its glory days, where he lived with his blind sister.
Many of the walls were covered with Italian and Portuguese tiles,
ceilings were painted with faded murals, and old portraits of his
ancestors and Kachchhi royalty from the Raj era ringed one room. His was
once the richest merchant family in town, he said --- then showed me the
decaying 1932 Chevrolet that sat in his garage, which had belonged to
his father and was said to be the first car in Mandvi.

Image

Dilip Asher, a chess aficionado and Beatles fan, inside the mansion
that's been in his family for generations. He died in 2015.

For some reason I can't recall, while standing by the car, he launched
wholeheartedly into a scene from ``My Fair Lady,'' reciting it verbatim,
using different voices for different characters. Later, while sitting
behind his chess board, he sang a rendition of The Beatles' ``Ob-La-Di,
Ob-La-Da'' to illustrate a point he was making. Full of stories and
brimming with laughter, he ranks among the most charming souls I've ever
met.

Image

A woman from Gujarat's Maldhari community, with a customary tattoo on
her cheek.

Amid the lethal challenges confronting Gujarat today, some of my friends
there --- Gandhians and activists at heart --- have been working hard to
help protect the lives and livelihoods of some of the state's most
vulnerable people. Good friends in other parts of the country are doing
the same. Their resolve and idealism reminds me of what first drew me to
them --- and India --- and why I've kept going back over the years.

\begin{center}\rule{0.5\linewidth}{\linethickness}\end{center}

\href{https://michaelbenanav.com/}{\emph{Michael Benanav}} \emph{is a
writer and photographer whose most recent book,}
\href{https://himalayabound.com/}{\emph{Himalaya Bound: One Family's
Quest to Save Their Animals and an Ancient Way of Life}}\emph{, was
published in 2018.}

\emph{\textbf{Follow New York Times Travel}} \emph{on}
\href{https://www.instagram.com/nytimestravel/}{\emph{Instagram}}\emph{,}
\href{https://twitter.com/nytimestravel}{\emph{Twitter}} \emph{and}
\href{https://www.facebook.com/nytimestravel/}{\emph{Facebook}}\emph{.
And}
\href{https://www.nytimes.com/newsletters/traveldispatch}{\emph{sign up
for our weekly Travel Dispatch newsletter}} \emph{to receive expert tips
on traveling smarter and inspiration for your next vacation.}

Advertisement

\protect\hyperlink{after-bottom}{Continue reading the main story}

\hypertarget{site-index}{%
\subsection{Site Index}\label{site-index}}

\hypertarget{site-information-navigation}{%
\subsection{Site Information
Navigation}\label{site-information-navigation}}

\begin{itemize}
\tightlist
\item
  \href{https://help.nytimes.com/hc/en-us/articles/115014792127-Copyright-notice}{©~2020~The
  New York Times Company}
\end{itemize}

\begin{itemize}
\tightlist
\item
  \href{https://www.nytco.com/}{NYTCo}
\item
  \href{https://help.nytimes.com/hc/en-us/articles/115015385887-Contact-Us}{Contact
  Us}
\item
  \href{https://www.nytco.com/careers/}{Work with us}
\item
  \href{https://nytmediakit.com/}{Advertise}
\item
  \href{http://www.tbrandstudio.com/}{T Brand Studio}
\item
  \href{https://www.nytimes.com/privacy/cookie-policy\#how-do-i-manage-trackers}{Your
  Ad Choices}
\item
  \href{https://www.nytimes.com/privacy}{Privacy}
\item
  \href{https://help.nytimes.com/hc/en-us/articles/115014893428-Terms-of-service}{Terms
  of Service}
\item
  \href{https://help.nytimes.com/hc/en-us/articles/115014893968-Terms-of-sale}{Terms
  of Sale}
\item
  \href{https://spiderbites.nytimes.com}{Site Map}
\item
  \href{https://help.nytimes.com/hc/en-us}{Help}
\item
  \href{https://www.nytimes.com/subscription?campaignId=37WXW}{Subscriptions}
\end{itemize}
