Sections

SEARCH

\protect\hyperlink{site-content}{Skip to
content}\protect\hyperlink{site-index}{Skip to site index}

\href{https://www.nytimes.com/section/food}{Food}

\href{https://myaccount.nytimes.com/auth/login?response_type=cookie\&client_id=vi}{}

\href{https://www.nytimes.com/section/todayspaper}{Today's Paper}

\href{/section/food}{Food}\textbar{}A Top Chicago Restaurant Messaged
Its Virtue. Then Workers Spoke Up.

\url{https://nyti.ms/2Y5IrUw}

\begin{itemize}
\item
\item
\item
\item
\item
\item
\end{itemize}

\href{https://www.nytimes.com/news-event/george-floyd-protests-minneapolis-new-york-los-angeles?action=click\&pgtype=Article\&state=default\&region=TOP_BANNER\&context=storylines_menu}{Race
and America}

\begin{itemize}
\tightlist
\item
  \href{https://www.nytimes.com/interactive/2020/07/03/us/george-floyd-protests-crowd-size.html?action=click\&pgtype=Article\&state=default\&region=TOP_BANNER\&context=storylines_menu}{Black
  Lives Matter Movement}
\item
  \href{https://www.nytimes.com/interactive/2020/06/28/us/i-cant-breathe-police-arrest.html?action=click\&pgtype=Article\&state=default\&region=TOP_BANNER\&context=storylines_menu}{History
  of `I Can't Breathe'}
\item
  \href{https://www.nytimes.com/interactive/2020/06/10/upshot/black-lives-matter-attitudes.html?action=click\&pgtype=Article\&state=default\&region=TOP_BANNER\&context=storylines_menu}{How
  Public Opinion Shifted}
\item
  \href{https://www.nytimes.com/interactive/2020/07/16/us/black-lives-matter-protests-louisville-breonna-taylor.html?action=click\&pgtype=Article\&state=default\&region=TOP_BANNER\&context=storylines_menu}{45
  Days in Louisville}
\end{itemize}

Advertisement

\protect\hyperlink{after-top}{Continue reading the main story}

Supported by

\protect\hyperlink{after-sponsor}{Continue reading the main story}

\hypertarget{a-top-chicago-restaurant-messaged-its-virtue-then-workers-spoke-up}{%
\section{A Top Chicago Restaurant Messaged Its Virtue. Then Workers
Spoke
Up.}\label{a-top-chicago-restaurant-messaged-its-virtue-then-workers-spoke-up}}

Since Fat Rice proclaimed its support for justice, former employees have
come forward with complaints that its chef created a hostile work
environment.

\includegraphics{https://static01.nyt.com/images/2020/06/17/dining/17FatRice1/00FatRice1-articleLarge.jpg?quality=75\&auto=webp\&disable=upscale}

By \href{https://www.nytimes.com/by/brett-anderson}{Brett Anderson}

\begin{itemize}
\item
  Published June 16, 2020Updated June 19, 2020
\item
  \begin{itemize}
  \item
  \item
  \item
  \item
  \item
  \item
  \end{itemize}
\end{itemize}

Two weeks ago, Abe Conlon and Adrienne Lo decided to declare their
solidarity with the fight for racial justice. They did so with two posts
on the Instagram account of Fat Rice, their award-winning restaurant in
Chicago: one a plain black square, the other a photo of the words
``Stand for change'' spray-painted inside a heart.

The posts did not have their intended effect. They were immediately
condemned on social media as shallow acts of self-aggrandizement,
particularly by former Fat Rice employees, who took to the internet with
a barrage of complaints about a culture of verbal abuse, rage and racial
insensitivity they said had flourished at the restaurant.

Nearly all of the 20 former Fat Rice employees who spoke to The New York
Times in recent days described Mr. Conlon, 39, as an extreme example of
a restaurant-business archetype: a tantrum-prone chef who rules by fear
and bullying. He ended one staff meeting, they said, by dumping a can of
garbage onto the floor, and flew into fits of anger so severe onlookers
feared they would lead to violence.

``Working there was pretty much a nightmare when Abe was around,'' said
Molly Pachay, 27, a former Fat Rice bar manager.

At a moment when restaurants across the country have made efforts to
align themselves with protests over the killing of
\href{https://www.nytimes.com/article/george-floyd-who-is.html}{George
Floyd} by a police officer in Minneapolis, the furor surrounding Fat
Rice shows
\href{https://www.nytimes.com/2020/06/11/dining/food-brands-black-lives-matter-social-media.html}{there
are dangers for businesses that try to turn their names into symbols of
virtue}. Last week, the owners of
\href{https://ny.eater.com/2020/6/8/21284326/mission-chinese-race-danny-bowien}{Mission
Chinese Food} in New York and of the California restaurant chain
\href{https://www.sfgate.com/food/article/Boba-Guys-fires-manager-for-racist-comments-15330946.php}{Boba
Guys} issued apologies for the racist behavior of staff members ---
accounts that had emerged earlier but resurfaced after the restaurants
declared public support for the Black Lives Matter movement.

The misbehavior that the former Fat Rice employees have described does
not include allegations of sexual harassment, which many chefs and
restaurateurs have faced in the \#MeToo era. But the uproar reveals a
growing intolerance for a type of verbal mistreatment that has long been
accepted as routine in the industry --- one in which blacks and other
minorities do much of the hardest work.

Mr. Conlon posted an apology to Instagram on June 6. ``I have reinforced
a culture of hostility and oppression due to my own insecurities,'' he
wrote in part. ``I have much unlearning ahead of me.'' (The Instagram
account has since been deleted.)

Last Wednesday, he and Ms. Lo, 36, the restaurant's co-owner, said that
in response to the criticism, they had closed Fat Rice, which they had
\href{https://www.nytimes.com/2020/04/28/dining/super-fat-rice-mart-chicago-coronavirus.html}{converted
to a general store} focusing on meal kits during the coronavirus
shutdown. ``We've stopped all orders of business in support of the
movement and to take time to reflect,'' Mr. Conlon said. Their roughly
70 employees were laid off in March after a shutdown order took effect
in Illinois.

\includegraphics{https://static01.nyt.com/images/2020/06/17/dining/17FatRice2/merlin_171984108_dd543c0f-25a9-4c9f-b924-01e41f43f215-articleLarge.jpg?quality=75\&auto=webp\&disable=upscale}

Fat Rice, which opened in 2012, specialized in the food of Macau, a
former Portuguese colony in China. In 2018, Mr. Conlon won the
\href{https://www.jamesbeard.org/chef/abraham-conlon}{James Beard award}
for Best Chef in the Great Lakes Region, and Chicago magazine
\href{https://www.chicagomag.com/dining-drinking/July-2018/The-50-Best-Restaurants-in-Chicago/Fat-Rice/}{proclaimed}
that ``Fat Rice may be the most universally beloved restaurant in
Chicago.''

But Alex Szabo, 29, who worked as a chef there from 2015 to 2016, said
Mr. Conlon's brute management style --- ``when Abe was on the line, it
was as bad as any kitchen I've seen,'' he said --- took an emotional
toll on the kitchen staff. He recalls breaking down in front of Mr.
Conlon one night.

``I told him my life right now makes me want to kill myself, and I do
not know what to do,'' Mr. Szabo recalled. ``His response to that was I
should just work more.''

Mr. Conlon said he did not remember the incident, and in a lengthy
interview with The Times, he and Ms. Lo took exception to some former
employees' description of Fat Rice as an unsafe place to work.

``I don't think that's a fair conclusion for you to make,'' Ms. Lo said.
Many former employees criticized her for not intervening on their behalf
more often.

Both Ms. Lo and Mr. Conlon attributed his behavior in part to his own
past ``traumas,'' including drinking, drug abuse and the way he himself
was treated as a younger chef. In a text message after the interview,
Mr. Conlon reiterated his belief that he is an example of an
industrywide problem that he alone should not have to answer for.

``I am acknowledging that my harsh behaviors, poor leadership and
temperament are indicative of the greater problems in the restaurant
world in which I have learned them,'' he wrote. ``I am complicit in my
participation in the `that's just how it is, because that's how it has
been' culture.''

Mr. Conlon's former employees said his outbursts were routine and often
alarming. One such incident occurred at the 2018 Pitchfork Music
Festival in Chicago, where Fat Rice had a stall. Mr. Conlon became so
enraged with an employee who accidentally threw out his breakfast that
someone called security.

``I remember security was like: `Hey, you need to take it easy. You
can't really talk to people like that,' '' recalled Mr. Conlon, who said
he was fatigued, having just returned from an overseas trip to film an
episode of ``Top Chef.'' ``I was wrongfully upset.''

Image

Joey Pham, a former Fat Rice chef, has been critical of the restaurant's
owners on social media.Credit...Joey Pham

Many former Fat Rice employees who voiced complaints about the
restaurant on social media say the current push for racial equality
should include a reckoning for chefs like Mr. Conlon. They point out
that abuse in restaurants falls disproportionately on people of color,
who often occupy low-level positions that keep them from fighting back.

``Black people have experienced this on a greater scale, but it doesn't
minimize what's happening under our noses,'' said Joey Pham, 32, a
former Fat Rice chef who has accused the owners
\href{https://www.instagram.com/flavorsupreme/}{on social media} of
creating a hostile workplace. ``When is it a good time to come out about
abuse?''

Fat Rice has been under fire before. Last year, the restaurant made
local
\href{https://chicago.eater.com/2019/2/14/18202640/chicago-restaurants-music-hip-hop-n-word-playlist-explicit-lyrics-fat-rice-furious-spoon-no-bones}{news}
when an African-American guest complained that it played loud hip-hop
music that contained racial slurs. The owners responded by posting a
warning in the restaurant that diners could be exposed to explicit
lyrics.

In 2017, Ryan Zeh, a chef, said he heard Mr. Conlon talking angrily
about a female server, and became so concerned that he followed Mr.
Conlon as he approached the employee.

``I thought I needed to be there,'' said Mr. Zeh, 27. ``He's thrown
plates at cooks.''

Ms. Pachay, the bar manager, said Mr. Conlon was furious with the
employee for improperly coursing the meal of a single diner. ``Abe
verbally berated and abused her for it in front of a crowded bar,'' Ms.
Pachay said.

Mr. Conlon never laid hands on the server. He said he doesn't remember
making the threat, and while he admitted to throwing plates, he denied
ever having thrown one at an employee.

He nonetheless said he regretted losing his temper, and confirmed that
it led to his taking a leave of absence from the restaurant, after
several employees, including Ms. Pachay and Mr. Zeh, urged Ms. Lo to
tell Mr. Conlon to seek professional help.

``It wasn't my incident alone that caused the sabbatical,'' said the
server who was the target of Mr. Conlon's anger. ``It was the tipping
point for it.'' She asked not to be named because she didn't want the
controversy to hurt her new career.

Mr. Conlon spent three weeks in his native Massachusetts, where he took
some time to reflect at a yoga and meditation retreat. ``Why am I acting
this way? What are the behaviors that I need to change?'' he recalled
asking himself. ``It was an important time for me. But it obviously
wasn't enough.''

Image

Sukainah Jallow, a former Fat Rice server, said Mr. Conlon continued to
verbally abuse employees, even after he took a leave of absence to deal
with the problem.Credit...Taylor Glascock for The New York Times

Sukainah Jallow, 32, said she started working at Fat Rice as a server
during Mr. Conlon's leave. ``People were very open, telling me, `You
should be happy he's not here,''' she said.

Still, when Mr. Conlon returned, he did not appear to have changed, she
said. ``I was never personally a target, but people around me were
emotionally broken by him,'' Ms. Jallow said. ``He's been doing this
stuff for years. He's not understanding.''

Ms. Jallow is one of several African-American former staff members who
accused Mr. Conlon and Ms. Lo of treating black employees differently
than white employees.

Some of these differences were subtle, like the way Ms. Jallow said Mr.
Conlon ``changed his tone of voice with people who are black,'' adopting
black slang.

Anisa McGowan, 22, a former server, said Ms. Lo once ordered her to
cover her head during service when she came to work with an Afro.

``She tried to frame it as a hygiene thing,'' Ms. McGowan said, ``but I
don't believe that, because there were a lot of white people who had
shoulder-length hair and didn't wear it back or anything like that.''

Ms. Lo said that all employees, regardless of race, were instructed to
wear their hair up, and that she was ``not picking on'' Ms. McGowan.
``There had been problems with hair in the food,'' she said.

But Ms. McGowan said she interpreted the attention paid to her Afro as
``anti-black,'' in part because of the other ways she saw race play out
at Fat Rice.

She was particularly troubled watching a young black food runner whom
she believed Mr. Conlon had singled out for abuse. The former employee,
who asked not to be identified for fear of being blackballed in the
industry, said he worked at the restaurant for less than a year, in 2017
and 2018, and that he was 20 when hired.

Mr. Conlon, he said, ``had a lot of animosity toward me personally. I
don't know if it was because I was new or because I was black or because
I was young, or because of all three of those things.''

He said Mr. Conlon once showed him a photograph on his phone of a
``black banjo guy with overalls and a hat and a straw in his mouth.'' He
said the chef encouraged him to dress up like the man for the
restaurant's New Orleans-themed Halloween party.

``I was like, `Why would I do that? I look like this guy?' '' the former
food runner recalled. ``He said, `Sure, you do.' ''

Mr. Conlon said he remembers the food runner, but not the incident he
described. ``I know that we would have brainstorming sessions when we
had events,'' he said.

The young man recalled the night Mr. Conlon fired him. ``He kept pulling
me aside so he could yell at me,'' he said. ``I got really mad,'' having
taken several months of such treatment. ``The last time he pulled me
aside into a corner really hard. I walked away from him and he grabbed
me again. Then I yelled at him and told him, `Don't touch me.' ''

Mr. Conlon and Ms. Lo said the food runner often moved too quickly
through the restaurant's cramped space, sometimes colliding with other
people. ``He was moving headfirst,'' said Mr. Conlon, who said he told
the young man: ``You make me nervous. I'm afraid you're going to crash
into somebody again.''

``I reacted and I grabbed him by the shoulder because I thought he was
going to crash into somebody,'' said Mr. Conlon, who added that his
reaction ``wasn't based on who he was as a person. It wasn't based on
his race.''

Both men say they moved into the alley to continue the argument. The
young man said he asked why Mr. Conlon kept pulling his arm. ``He was
like, `I thought you were going to attack a guest.' In, like, five
seconds, he changed the narrative into, I was going to attack a guest
because I was walking away from him angrily.''

He said Mr. Conlon then told him, ``I could have called the police on
you.'' Mr. Conlon said doesn't recall saying that. ``I think I was
trying to explain what my concerns were to him,'' he said, ``and why I
don't think it was being heard, and things obviously got heated.''

He fired the employee later that evening. Ms. McGowan said the young man
told her later that night about the incident.

``That's a really hard thing to manage,'' she said, ``to work with
people who damn well may call the cops on you.''

\emph{Follow} \href{https://twitter.com/nytfood}{\emph{NYT Food on
Twitter}} \emph{and}
\href{https://www.instagram.com/nytcooking/}{\emph{NYT Cooking on
Instagram}}\emph{,}
\href{https://www.facebook.com/nytcooking/}{\emph{Facebook}}\emph{,}
\href{https://www.youtube.com/nytcooking}{\emph{YouTube}} \emph{and}
\href{https://www.pinterest.com/nytcooking/}{\emph{Pinterest}}\emph{.}
\href{https://www.nytimes.com/newsletters/cooking}{\emph{Get regular
updates from NYT Cooking, with recipe suggestions, cooking tips and
shopping advice}}\emph{.}

Advertisement

\protect\hyperlink{after-bottom}{Continue reading the main story}

\hypertarget{site-index}{%
\subsection{Site Index}\label{site-index}}

\hypertarget{site-information-navigation}{%
\subsection{Site Information
Navigation}\label{site-information-navigation}}

\begin{itemize}
\tightlist
\item
  \href{https://help.nytimes.com/hc/en-us/articles/115014792127-Copyright-notice}{©~2020~The
  New York Times Company}
\end{itemize}

\begin{itemize}
\tightlist
\item
  \href{https://www.nytco.com/}{NYTCo}
\item
  \href{https://help.nytimes.com/hc/en-us/articles/115015385887-Contact-Us}{Contact
  Us}
\item
  \href{https://www.nytco.com/careers/}{Work with us}
\item
  \href{https://nytmediakit.com/}{Advertise}
\item
  \href{http://www.tbrandstudio.com/}{T Brand Studio}
\item
  \href{https://www.nytimes.com/privacy/cookie-policy\#how-do-i-manage-trackers}{Your
  Ad Choices}
\item
  \href{https://www.nytimes.com/privacy}{Privacy}
\item
  \href{https://help.nytimes.com/hc/en-us/articles/115014893428-Terms-of-service}{Terms
  of Service}
\item
  \href{https://help.nytimes.com/hc/en-us/articles/115014893968-Terms-of-sale}{Terms
  of Sale}
\item
  \href{https://spiderbites.nytimes.com}{Site Map}
\item
  \href{https://help.nytimes.com/hc/en-us}{Help}
\item
  \href{https://www.nytimes.com/subscription?campaignId=37WXW}{Subscriptions}
\end{itemize}
