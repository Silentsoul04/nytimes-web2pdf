Sections

SEARCH

\protect\hyperlink{site-content}{Skip to
content}\protect\hyperlink{site-index}{Skip to site index}

\href{https://myaccount.nytimes.com/auth/login?response_type=cookie\&client_id=vi}{}

\href{https://www.nytimes.com/section/todayspaper}{Today's Paper}

\href{/section/opinion}{Opinion}\textbar{}Is Trump Trying to Spread
Covid-19?

\href{https://nyti.ms/3fwK2se}{https://nyti.ms/3fwK2se}

\begin{itemize}
\item
\item
\item
\item
\item
\item
\end{itemize}

Advertisement

\protect\hyperlink{after-top}{Continue reading the main story}

\href{/section/opinion}{Opinion}

Supported by

\protect\hyperlink{after-sponsor}{Continue reading the main story}

\hypertarget{is-trump-trying-to-spread-covid-19}{%
\section{Is Trump Trying to Spread
Covid-19?}\label{is-trump-trying-to-spread-covid-19}}

Does he start each day wondering what expert advice to ignore next?

\href{https://www.nytimes.com/by/thomas-l-friedman}{\includegraphics{https://static01.nyt.com/images/2018/04/02/opinion/thomas-l-friedman/thomas-l-friedman-thumbLarge.png}}

By \href{https://www.nytimes.com/by/thomas-l-friedman}{Thomas L.
Friedman}

Opinion Columnist

\begin{itemize}
\item
  June 16, 2020
\item
  \begin{itemize}
  \item
  \item
  \item
  \item
  \item
  \item
  \end{itemize}
\end{itemize}

\includegraphics{https://static01.nyt.com/images/2020/06/16/opinion/16friedman1/merlin_173565756_7549f509-9663-44b8-bc18-dc8f5b912fca-articleLarge.jpg?quality=75\&auto=webp\&disable=upscale}

When the full record of the coronavirus in America is written,
historians may argue that President Trump's biggest mistake was not what
he failed to do in early 2020, when the right strategy for combating the
virus was widely debated, unproven and hard. No, they will point to what
Trump failed to do in June 2020, when the right strategy was clear,
proven and relatively easy.

No doubt, this virus is inscrutable. It pops up, it disappears, it
reappears, some people are symptomatic, some asymptomatic, some seem to
have natural immunities to it that we don't understand, and once it
infects people it hits in radically different ways: It comes in the
equivalents of decaf, regular and double macchiato --- and you never
know if you're going to get the mild or the extra-strength version.

But there is \emph{so much that we do know now} that could make this
post-lockdown phase so much less dangerous and so much more economically
viable than it is.

We know that countries where everyone wears a mask outside the home
sharply reduce the spread and that people who practice strict social
distancing infect fewer people and are infected less often. And we know
that people who avoid ``superspreading'' events --- large, prolonged
social gatherings, religious services and crammed nightclubs and
workplaces, where one highly contagious person can quickly spew the
virus to many others ---
\href{https://arstechnica.com/science/2020/06/just-10-20-of-covid-19-cases-behind-80-of-transmission-studies-suggest/}{are
less likely to get infected}.

Top government expert Dr. Anthony Fauci has pointed out that taking just
these relatively easy steps, plus testing, tracing chains of
transmission and quarantining the infected, would tamp down what appears
to be a brewing, post-lockdown resurgence and limit the number of people
needing hospitalization as we await a vaccine.

And yet we have a president who, instead of wearing a mask, turns
defiance of mask-wearing into a heroic act of defiance against liberals;
who forces 1,100 West Point cadets to travel back to campus, and
quarantine for two weeks, so he can get a photo op addressing their
graduation; who is planning a mass rally in Tulsa, Okla., on Saturday
--- where the most notable precaution is that you sign a legal
disclaimer that you ``voluntarily assume all risks related to exposure
to Covid-19 and agree not to hold Donald J. Trump for President Inc.''
liable --- and who hails governors who open bars and restaurants for
people to crowd together.

It is absolutely devilish --- like Trump wakes up every morning and asks
himself: What health expert's advice can I defy today? **** What simple
gesture to reduce the odds that the coronavirus continues to surge,
post-lockdowns, can I ignore today? What quack remedy can I promote
today?

I've argued from the onset of this pandemic that our goal had to be a
sustainable strategy that maximizes saving lives and livelihoods, and
I've been stunned by the criticism that anyone talking about saving
lives and jobs in the same breath is an unfeeling capitalist. That's
crazy. We now have 40 million Americans unemployed. The physical and
mental health consequences of that number, if it continues for six more
months, will be devastating.

But Trump wants as many Americans back to work now, and the stock market
to rise now, without asking Americans to take even easy precautions.

That's not just cynical, it's incredibly stupid --- if you're Trump.
Because people are not going to go back to work or out to dinner if they
see lots of family, co-workers and friends getting sick and dying, no
matter what he says.

You would think Trump had learned by now that Mother Nature is calling
the shots and she asks only three questions about your personal or
communal adaptation strategy toward her virus.

First, are you humble --- do you respect my virus? Because if you don't,
it will hurt you or someone you love. Second, is your response
coordinated? Because Mother Nature has evolved her viruses over
millenniums to find any crack in your personal or communal immune
system. And third, is your strategy for maximizing lives and livelihoods
based on chemistry, biology and physics and not politics, ideology and
election dates? Because Mother Nature is only chemistry, biology and
physics and responds to nothing else.

Oh, and lockdowns are meaningless to her. Her viruses go away only if
you can develop a vaccine or enough people develop herd immunity by
acquiring the infection and building natural antibodies to it.

Trump, alas, does not respect the virus. He is not coordinating a
coherent public health response, and the response he is coordinating is
based not on chemistry, biology and physics but on his own political
needs.

If a nationwide resurgence of Covid-19 hospitalizations meets crowded,
intense social protests against police killings --- particularly by
black and brown Americans who have also been disproportionately harmed
by the coronavirus --- meets stubborn mass unemployment, meets an
exhausted nation being ordered into a second round of lockdowns, watch
out.

What would a real president be urging governors to do today? Prepare
detailed plans to get people back to work on a risk-stratified basis
with proper protections,
\href{https://medium.com/@drdarrialonganddrdavidkatz/as-cities-move-toward-reopening-how-to-manage-risks-1834a264f9d1}{along
the lines recently proposed} by public health experts Dr. Darria Long
and Dr. David Katz.

``The data are now overwhelming, from here in the U.S. and all around
the world, that this infection is a grave threat to the elderly and
chronically ill, but generally mild for younger, generally healthy
people,'' said Katz in an interview.

It's also clear that ``many of the worried projections about social
determinants of health and the consequences of mass unemployment are
confirmed. We have, indeed, seen rising rates of
\href{https://www.mlive.com/public-interest/2020/05/michigans-coronavirus-crisis-creates-epidemic-of-mental-health-issues.html}{addiction,
domestic violence and mental duress.}''

We also know much more now, Katz continued, ``about the risks of
exposure. This virus is not transmitted all that easily. \ldots{} Many
people with transient, ordinary exposures don't get infected because of
low exposure dose, partial resistance to this pathogen, or both.''

All of this provides actionable intelligence, Katz argued. We can and
must do a far better job of protecting the frail and elderly, especially
in nursing homes, and all of those with serious chronic disease, he
said. ``Then the rest of us can go about our business, but with policies
in place to regulate any interactions we might have with higher-risk
people, so we protect them, and with reasonable precautions for our own
sakes, like wearing masks, practicing social distancing and avoiding
crowded indoor settings, that limit exposure to high doses of
coronavirus and our ability to pass it along.''

We also can see now --- with cases spiking in locations around the
country that did not experience an early wave of infection and are now
opening up haphazardly --- ``how right it was to warn about the dangers
of just flattening the curve without a risk-stratification strategy,''
added Katz. ``A flattened curve delays cases, it does not prevent them,
because no immunity has been developed.''

To get back to normalcy requires widespread immunity to the coronavirus,
which happens in only two ways.

One is a vaccine that is safe, effective, mass produced and universally
distributed. That would be the best solution, and God willing, a vaccine
will come in the fall **** and everyone can get back to work safely in
subsequent months. But it may not, and we can't just keep the economy on
hold.

``The other,'' said Katz, ``is natural herd immunity, achieved by those
of us at low risk for severe infection, who can most safely go back to
work and school and life as we knew it, while taking the right,
reasonable protections. Meanwhile, we should guard those most vulnerable
until we can sound the all-clear. Only this kind of thoughtful,
risk-stratified approach can allow for herd immunity with maximal safety
and minimal total harm from infection and the consequences of prolonged
lockdown alike.''

Our current haphazard approach is just begging for trouble.

\emph{The Times is committed to publishing}
\href{https://www.nytimes.com/2019/01/31/opinion/letters/letters-to-editor-new-york-times-women.html}{\emph{a
diversity of letters}} \emph{to the editor. We'd like to hear what you
think about this or any of our articles. Here are some}
\href{https://help.nytimes.com/hc/en-us/articles/115014925288-How-to-submit-a-letter-to-the-editor}{\emph{tips}}\emph{.
And here's our email:}
\href{mailto:letters@nytimes.com}{\emph{letters@nytimes.com}}\emph{.}

\emph{Follow The New York Times Opinion section on}
\href{https://www.facebook.com/nytopinion}{\emph{Facebook}}\emph{,}
\href{http://twitter.com/NYTOpinion}{\emph{Twitter (@NYTopinion)}}
\emph{and}
\href{https://www.instagram.com/nytopinion/}{\emph{Instagram}}\emph{.}

Advertisement

\protect\hyperlink{after-bottom}{Continue reading the main story}

\hypertarget{site-index}{%
\subsection{Site Index}\label{site-index}}

\hypertarget{site-information-navigation}{%
\subsection{Site Information
Navigation}\label{site-information-navigation}}

\begin{itemize}
\tightlist
\item
  \href{https://help.nytimes.com/hc/en-us/articles/115014792127-Copyright-notice}{©~2020~The
  New York Times Company}
\end{itemize}

\begin{itemize}
\tightlist
\item
  \href{https://www.nytco.com/}{NYTCo}
\item
  \href{https://help.nytimes.com/hc/en-us/articles/115015385887-Contact-Us}{Contact
  Us}
\item
  \href{https://www.nytco.com/careers/}{Work with us}
\item
  \href{https://nytmediakit.com/}{Advertise}
\item
  \href{http://www.tbrandstudio.com/}{T Brand Studio}
\item
  \href{https://www.nytimes.com/privacy/cookie-policy\#how-do-i-manage-trackers}{Your
  Ad Choices}
\item
  \href{https://www.nytimes.com/privacy}{Privacy}
\item
  \href{https://help.nytimes.com/hc/en-us/articles/115014893428-Terms-of-service}{Terms
  of Service}
\item
  \href{https://help.nytimes.com/hc/en-us/articles/115014893968-Terms-of-sale}{Terms
  of Sale}
\item
  \href{https://spiderbites.nytimes.com}{Site Map}
\item
  \href{https://help.nytimes.com/hc/en-us}{Help}
\item
  \href{https://www.nytimes.com/subscription?campaignId=37WXW}{Subscriptions}
\end{itemize}
