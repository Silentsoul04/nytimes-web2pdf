Sections

SEARCH

\protect\hyperlink{site-content}{Skip to
content}\protect\hyperlink{site-index}{Skip to site index}

\href{https://www.nytimes.com/section/world/asia}{Asia Pacific}

\href{https://myaccount.nytimes.com/auth/login?response_type=cookie\&client_id=vi}{}

\href{https://www.nytimes.com/section/todayspaper}{Today's Paper}

\href{/section/world/asia}{Asia Pacific}\textbar{}In China-India Clash,
Two Nationalist Leaders With Little Room to Give

\url{https://nyti.ms/2N8gbKy}

\begin{itemize}
\item
\item
\item
\item
\item
\item
\end{itemize}

Advertisement

\protect\hyperlink{after-top}{Continue reading the main story}

Supported by

\protect\hyperlink{after-sponsor}{Continue reading the main story}

\hypertarget{in-china-india-clash-two-nationalist-leaders-with-little-room-to-give}{%
\section{In China-India Clash, Two Nationalist Leaders With Little Room
to
Give}\label{in-china-india-clash-two-nationalist-leaders-with-little-room-to-give}}

Xi Jinping and Narendra Modi have sought to project a muscular global
profile despite their countries' problems. Backing down could hurt their
efforts.

\includegraphics{https://static01.nyt.com/images/2020/06/17/world/17china-india002sub/merlin_162534615_a4140789-ef22-42af-9eef-b3d8d1bcb4e1-articleLarge.jpg?quality=75\&auto=webp\&disable=upscale}

\href{https://www.nytimes.com/by/steven-lee-myers}{\includegraphics{https://static01.nyt.com/images/2018/10/15/multimedia/author-steven-lee-myers/author-steven-lee-myers-thumbLarge.png}}\href{https://www.nytimes.com/by/maria-abi-habib}{\includegraphics{https://static01.nyt.com/images/2018/10/08/multimedia/author-maria-abi-habib/author-maria-abi-habib-thumbLarge.png}}\href{https://www.nytimes.com/by/jeffrey-gettleman}{\includegraphics{https://static01.nyt.com/images/2018/10/10/multimedia/author-jeffrey-gettleman/author-jeffrey-gettleman-thumbLarge.png}}

By \href{https://www.nytimes.com/by/steven-lee-myers}{Steven Lee Myers},
\href{https://www.nytimes.com/by/maria-abi-habib}{Maria Abi-Habib} and
\href{https://www.nytimes.com/by/jeffrey-gettleman}{Jeffrey Gettleman}

\begin{itemize}
\item
  Published June 17, 2020Updated June 29, 2020
\item
  \begin{itemize}
  \item
  \item
  \item
  \item
  \item
  \item
  \end{itemize}
\end{itemize}

They are both ambitious, nationalist leaders, eager to assert greater
roles for their countries in a turbulent world. With major challenges at
home, neither wants to risk losing face, even in a dispute over
mountainous territory that is all but desolate.

Xi Jinping of
\href{https://www.nytimes.com/2020/06/29/world/asia/tik-tok-banned-india-china.html}{China}
and Narendra Modi of
\href{https://www.nytimes.com/2020/06/29/world/asia/tik-tok-banned-india-china.html}{India}
probably did not intend to ignite
\href{https://www.nytimes.com/2020/06/16/world/asia/indian-china-border-clash.html}{a
clash} on their
\href{https://www.nytimes.com/2020/06/18/world/asia/india-china-border.html}{border}
on Monday, high in the Himalayas, that killed 20 Indian troops and may
have resulted in Chinese casualties, too. Yet the leaders of the two
nuclear-equipped countries now confront a military crisis that could
spin dangerously out of control.

``The sovereignty and integrity of India is supreme, and nobody can stop
us in defending that,'' Mr. Modi said on Wednesday in a short televised
speech, breaking his public silence over the incident. He vowed that
``the sacrifice of our soldiers will not be in vain.''

``India wants peace,'' he went on, ``but if provoked India is capable of
giving a befitting reply.''

The clash, the worst violence between them in 45 years, resulted from
policies both leaders have pushed to bolster forces along their
2,100-mile border and to project a muscular image at home and abroad.

\includegraphics{https://static01.nyt.com/images/2020/06/17/world/17china-india10/merlin_173608881_337b6864-1aea-40f3-a1c3-56fe95ed0403-articleLarge.jpg?quality=75\&auto=webp\&disable=upscale}

``This kind of provocation is aggressive, and we have no choice but to
contain it,'' Yue Gang, a retired colonel in the People's Liberation
Army, said on Wednesday, blaming border tensions, which flared up in
May, on India's actions and Mr. Modi's political ambitions.

``China is willing to go head-to-head with it and isn't afraid of deaths
or even of opening fire,'' he added.

China pledged on Wednesday to avoid a broader conflict, but its foreign
minister, Wang Yi, scolded his Indian counterpart in a telephone
conversation. He accused India of provoking the clash on Monday night,
despite an earlier agreement to withdraw forces from the Galwan Valley,
a remote area straddling the disputed frontier that has been the focus
of fighting before, including a war between India and China in 1962.

``The Indian side must not misjudge the current situation and must not
underestimate China's firm will to safeguard territorial sovereignty,''
Mr. Wang told India's Minister of External Affairs, Subrahmanyam
Jaishankar, according to a statement released in Beijing.

Hours later, India's ministry posted its own version of the
conversation, saying that Mr. Jaishankar blamed China for ``a
premeditated and planned action that was directly responsible for the
resulting violence and casualties.''

Both Mr. Xi and Mr. Modi are in uncharted territory. They are dealing
with the coronavirus, which is still spreading in India while China
tries to contain a new outbreak in Beijing. The economies of both
countries are weakened and vulnerable.

The deaths along the border only increase their anxieties, making a
diplomatic resolution all the more difficult.

Line of Actual

Control

CHINA

PAKISTAN

Diamer Bhasha dam

CHINA

NEPAL

GILGIT-

BALTISTAN

Controlled by Pakistan

Area of detail

Disputed borders

or cease-fire lines

CHINA

INDIA

Galwan Valley

Controlled by India

Bay of

Bengal

Pangong Tso

Arabian

Sea

JAMMU

AND KASHMIR

LADAKH

CHINA

CHINA

PAKISTAN

Lipulekh

Pass

INDIA

Naku La

NEPAL

BHUTAN

100 miles

BANGLADESH

Line of Actual

Control

Diamer Bhasha dam

CHINA

CHINA

PAKISTAN

GILGIT-

BALTISTAN

Controlled by Pakistan

NEPAL

Disputed borders

or cease-fire lines

CHINA

Area of detail

Galwan Valley

Controlled by India

INDIA

Pangong Tso

JAMMU

AND KASHMIR

LADAKH

CHINA

Bay of

Bengal

CHINA

Arabian

Sea

PAKISTAN

Lipulekh

Pass

INDIA

Naku La

NEPAL

BHUTAN

100 miles

BANGLADESH

Line of Actual

Control

Diamer

Bhasha dam

CHINA

CHINA

GILGIT-

BALTISTAN

Controlled by Pakistan

PAKISTAN

Disputed borders

or cease-fire lines

NEPAL

CHINA

Area of detail

Galwan Valley

Controlled by India

INDIA

LADAKH

JAMMU

AND KASHMIR

Pangong Tso

Bay of

Bengal

CHINA

CHINA

Arabian

Sea

PAKISTAN

Lipulekh

Pass

INDIA

Naku La

NEPAL

100 miles

Line of Actual

Control

CHINA

CHINA

GILGIT-

BALTISTAN

Controlled by Pakistan

1

Disputed borders

or cease-fire lines

Area of detail

CHINA

2

INDIA

Controlled by India

LADAKH

3

JAMMU

AND KASHMIR

Bay of

Bengal

Arabian

Sea

CHINA

CHINA

PAKISTAN

4

INDIA

5

NEPAL

100 miles

1

2

3

4

5

Diamer Bhasha dam

Galwan Valley

Pangong Tso

Lipulekh Pass

Naku La

CHINA

Disputed borders

or cease-fire lines

1

Ctrl. by Pakistan

CHINA

2

3

Ctrl. by India

CHINA

CHINA

PAKISTAN

4

INDIA

NEPAL

5

100 miles

1

2

3

4

5

Diamer Bhasha dam

Galwan Valley

Pangong Tso

Lipulekh Pass

Naku La

CHINA

PAKISTAN

NEPAL

Area of detail

INDIA

Bay of

Bengal

Arabian

Sea

Source: Satellite image via Microsoft Corporation Earthstar Geographics

By Jugal K. Patel

``For years India and China have said there have been no fatalities''
along the border, said Tanvi Madan, director of the India Project at the
Brookings Institution in Washington. ``That they have crossed that
threshold in and of itself makes this a significant incident.''

So far, Mr. Modi appears to be projecting toughness while trying to
avoid a deeper conflict. Photos in Indian media showed military convoys
on the winding roads approaching the disputed region, and local
residents described heavier than usual troop movement. Even so, military
analysts said that forces had not been put on full alert.

Mr. Modi, who met with his defense and foreign ministers Tuesday night
at his home, already faced criticism from opposition leaders that he was
feckless, despite his nationalist appeals and
\href{https://www.nytimes.com/2019/02/25/world/asia/india-pakistan-kashmir-jets.html}{his
sharp response last year to an attack linked to a militant group based
in Pakistan}.

``Why is he hiding?''
\href{https://twitter.com/RahulGandhi/status/1273094280307867648}{Rahul
Gandhi}, one of India's most prominent opposition leaders, wrote on
Twitter. ``Enough is enough. We need to know what has happened. How dare
China kill our soldiers? How dare they take our land?''

Mr. Modi's media allies could help him weather the storm. On Wednesday,
media friendly to his party circulated unsubstantiated reports that
Indian troops had won the skirmish. If people believe the reports, true
or not, that could lessen the pressure to retaliate.

``A whole bunch of myths have been created,'' said Ajai Shukla, a
retired army colonel turned popular commentator. ``And many of Modi's
followers will back him no matter what.''

Image

Activists from Mr. Modi's Bharatiya Janata Party at an anti-China
protest in Siliguri, India on Wednesday.Credit...Diptendu Dutta/Agence
France-Presse --- Getty Images

China is carefully minding its own message. Unlike India, which has a
freewheeling media and vibrant if weakened opposition parties, Mr. Xi's
seasoned propaganda apparatus is able to control the flow of
information. China's state media have offered only sparse reports on the
clash, rather than stirring nationalist fervor, though some still
spilled out.

It is still not clear whether China suffered fatalities. In four
decades, the People's Liberation Army had lost just three soldiers to
fighting abroad --- troops who were killed in United Nations
peacekeeping operations in Mali and South Sudan in 2016.

``I just don't understand,'' one person
\href{https://m.weibo.cn/2615417307/4516740824521878}{wrote}on Weibo,
the Chinese social media site. ``Why didn't we announce the casualties
of our side? What do we have to hide and why?''

A spokesman for China's foreign ministry, Zhao Lijian, four times on
Wednesday sidestepped questions about media reports in India that China
had suffered 43 casualties on Monday, including some deaths, saying he
had no information to offer. A spokesman for China's Western Theater
Command, which oversees the region, referred to casualties but offered
\href{http://eng.mod.gov.cn/news/2020-06/16/content_4866818.htm}{no
details}.

Officials in India have blamed China for starting
\href{https://www.nytimes.com/2020/05/30/world/asia/india-china-border.html}{the
confrontation along the border} in May by sending troops into the Galwan
Valley, among other places. Several brawls broke out between the
opposing forces, causing serious injuries.

Mr. Modi ordered India's military to reinforce the border, even as
commanders on both sides tried quietly to resolve tensions through a
meeting on June 6.

While China remained tight-lipped about the latest violence, new details
emerged from India. A number of Indian troops were captured, two Indian
military officials said in interviews, speaking on condition of
anonymity to discuss military operations. Their fate remains unclear,
and presumably is the subject of intense maneuvering behind the scenes.

Image

An Indian military convoy driving along a highway leading to Ladakh, an
area near the Chinese border, on Wednesday.Credit...Danish
Ismail/Reuters

The fighting began after Indian troops on Monday set fire to tents
erected by Chinese soldiers, the officials said. The Indian force later
encountered a much larger patrol of more than 100 Chinese soldiers, they
said, wielding fence posts and clubs wrapped in barbed wire or studded
with nails.

The two sides reportedly had no firearms, according to longstanding
protocols for the two militaries along
\href{https://www.nytimes.com/2020/06/16/world/asia/india-china-border.html}{the
Line of Actual Control}, the boundary drawn after the 1962 war to keep
them apart.

Among those killed was the Indian commander, Col. Bikkumalla Santosh
Babu, whose death appeared to spark a larger fight along a steep ridge,
which continued into Tuesday morning.

The officials said several Indians died after falling from the ridge in
the dark or intentionally jumping into the Galwan River. India initially
announced the deaths of only three soldiers but then the toll rose by
17, whose deaths were attributed to ``environmental conditions'' at the
high altitude.

Hu Xijin, the editor of The Global Times, a Chinese newspaper controlled
by the Communist Party,
\href{https://twitter.com/HuXijin_GT/status/1272973497766051840}{taunted
India on Twitter}, saying 17 injured soldiers died because India's
military could not evacuate them quickly enough for medical treatment.
``This is not an army with real modern combat capabilities,'' he wrote.

For all of Mr. Modi's aggressive rhetoric, the clash has underscored how
far India has fallen behind its neighbor, militarily and economically.

``India has no options, or limited ones,'' said Rahul Bedi, a military
analyst in Delhi. ``Appeasement is a risk,'' he added, but one that
``may be more digestible for now.''

Others in India say that the danger of inaction is greater. China has
chipped away at territory along the disputed frontier --- seizing
\href{https://theprint.in/opinion/china-believes-india-wants-aksai-chin-back-thats-why-it-has-crossed-lac-in-ladakh/430899/}{about
23 square miles in the past two months}.

Image

Pangong Tso lake in Ladakh, India, in 2017. Clashes took place in the
area last month.Credit...Manish Swarup/Associated Press

Ms. Madan of Brookings said that many Indians viewed this as ``salami
slicing'' similar to China's establishment of
\href{https://www.nytimes.com/2018/09/20/world/asia/south-china-sea-navy.html}{military
bases on contested islets} in the South China Sea, moves that have
frustrated neighboring countries and the United States.

``It's not lost on folks that this is the fourth major boundary incident
since 2012,'' she said, referring to the year Mr. Xi rose to power.
``What they seem to be trying to do is the territorial version of what
they have done in the South China Sea.''

Samir Saran, the president of the Observer Research Foundation, a Delhi
think tank, who is seen as close to Mr. Modi's government, said that if
India does not stand up to China, Beijing may double down and continue
to claim more contested terrain.

``You don't surrender to a powerful enemy,'' he said.

``There have to be costs attached to Chinese behavior,'' Mr. Saran
added. ``India may have to be prepared for series of limited skirmishes
to occasional conflicts. Maybe that is the new normal in our region.''

Reporting and research were contributed by Claire Fu, Hari Kumar, Amber
Wang and Sameer Yasir.

Advertisement

\protect\hyperlink{after-bottom}{Continue reading the main story}

\hypertarget{site-index}{%
\subsection{Site Index}\label{site-index}}

\hypertarget{site-information-navigation}{%
\subsection{Site Information
Navigation}\label{site-information-navigation}}

\begin{itemize}
\tightlist
\item
  \href{https://help.nytimes.com/hc/en-us/articles/115014792127-Copyright-notice}{©~2020~The
  New York Times Company}
\end{itemize}

\begin{itemize}
\tightlist
\item
  \href{https://www.nytco.com/}{NYTCo}
\item
  \href{https://help.nytimes.com/hc/en-us/articles/115015385887-Contact-Us}{Contact
  Us}
\item
  \href{https://www.nytco.com/careers/}{Work with us}
\item
  \href{https://nytmediakit.com/}{Advertise}
\item
  \href{http://www.tbrandstudio.com/}{T Brand Studio}
\item
  \href{https://www.nytimes.com/privacy/cookie-policy\#how-do-i-manage-trackers}{Your
  Ad Choices}
\item
  \href{https://www.nytimes.com/privacy}{Privacy}
\item
  \href{https://help.nytimes.com/hc/en-us/articles/115014893428-Terms-of-service}{Terms
  of Service}
\item
  \href{https://help.nytimes.com/hc/en-us/articles/115014893968-Terms-of-sale}{Terms
  of Sale}
\item
  \href{https://spiderbites.nytimes.com}{Site Map}
\item
  \href{https://help.nytimes.com/hc/en-us}{Help}
\item
  \href{https://www.nytimes.com/subscription?campaignId=37WXW}{Subscriptions}
\end{itemize}
