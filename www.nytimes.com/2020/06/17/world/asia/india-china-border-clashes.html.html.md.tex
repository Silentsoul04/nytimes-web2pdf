Sections

SEARCH

\protect\hyperlink{site-content}{Skip to
content}\protect\hyperlink{site-index}{Skip to site index}

\href{https://www.nytimes.com/section/world/asia}{Asia Pacific}

\href{https://myaccount.nytimes.com/auth/login?response_type=cookie\&client_id=vi}{}

\href{https://www.nytimes.com/section/todayspaper}{Today's Paper}

\href{/section/world/asia}{Asia Pacific}\textbar{}India-China Border
Dispute: A Conflict Explained

\url{https://nyti.ms/37NtxFT}

\begin{itemize}
\item
\item
\item
\item
\item
\end{itemize}

Advertisement

\protect\hyperlink{after-top}{Continue reading the main story}

Supported by

\protect\hyperlink{after-sponsor}{Continue reading the main story}

\hypertarget{india-china-border-dispute-a-conflict-explained}{%
\section{India-China Border Dispute: A Conflict
Explained}\label{india-china-border-dispute-a-conflict-explained}}

The deaths of 20 Indian soldiers in a brawl with Chinese troops was the
deadliest clash between the two nuclear-armed nations in decades, but
hardly the first.

\includegraphics{https://static01.nyt.com/images/2020/06/17/world/17china-india-clashes-1/merlin_173606910_d096a3b1-723a-4e95-ad40-4c539e697809-articleLarge.jpg?quality=75\&auto=webp\&disable=upscale}

\href{https://www.nytimes.com/by/russell-goldman}{\includegraphics{https://static01.nyt.com/images/2018/10/15/multimedia/author-russell-goldman/author-russell-goldman-thumbLarge.png}}

By \href{https://www.nytimes.com/by/russell-goldman}{Russell Goldman}

\begin{itemize}
\item
  Published June 17, 2020Updated June 29, 2020
\item
  \begin{itemize}
  \item
  \item
  \item
  \item
  \item
  \end{itemize}
\end{itemize}

\href{https://cn.nytimes.com/world/20200618/india-china-border-clashes/}{阅读简体中文版}\href{https://cn.nytimes.com/world/20200618/india-china-border-clashes/zh-hant/}{閱讀繁體中文版}

The most violent encounter in decades between the
\href{https://www.nytimes.com/2020/06/29/world/asia/tik-tok-banned-india-china.html}{Chinese}
and
\href{https://www.nytimes.com/2020/06/29/world/asia/tik-tok-banned-india-china.html}{Indian}
armies
\href{https://www.nytimes.com/2020/06/16/world/asia/india-china-border.html}{arrayed
along a disputed border} high in the Himalayas did not involve any
exchange of gunfire.

Instead, soldiers from the two nuclear-armed nations fashioned weapons
from what they could find in the desolate landscape, some 14,000 feet
above sea level.

Wielding fence posts and clubs wrapped in barbed wire, they squared off
under a moonlit sky along jagged cliffs soaring high above the Galwan
Valley, fighting for hours in pitched hand-to-hand battles.

Some Indian soldiers died after tumbling into the river in the valley
below. Others were beaten to death. By the next day, 20 Indian troops
were dead. It remains unclear if there were Chinese casualties.

The two countries' soldiers are not allowed to carry guns in the area, a
reflection of the depth of the bad blood that courses through the ranks
of the military forces on both sides in the
\href{https://timesmachine.nytimes.com/timesmachine/1962/11/21/87047861.pdf?pdf_redirect=true\&ip=0}{disputed
territory}.

The clash on Monday night, fought in one of the most forbidding
landscapes on the planet, was a startling culmination of months of
mounting tension and years of dispute.

And it comes at
\href{https://www.nytimes.com/2020/06/17/world/asia/china-india-border.html}{a
fraught moment}, with the world focused on battling the coronavirus and
with the nationalist leaders of both nations eager to flex their
muscles.

Here's a look at how both nations arrived at this juncture, the battles
that came before, and how The New York Times covered the conflict.

1914

\hypertarget{a-border-china-never-agreed-to}{%
\subsection{A Border China Never Agreed
To}\label{a-border-china-never-agreed-to}}

\includegraphics{https://static01.nyt.com/images/2020/06/17/world/17china-india-clashes-border/17china-india-clashes-border-articleLarge.jpg?quality=75\&auto=webp\&disable=upscale}

The conflict stretches back to at least 1914, when representatives from
Britain, the Republic of China and Tibet gathered in Simla, in what is
now India, to negotiate a treaty that would determine the status of
Tibet and effectively settle the borders between China and British
India.

The Chinese, balking at proposed terms that would have allowed Tibet to
be autonomous and remain under Chinese control, refused to sign the
deal. But Britain and Tibet signed a treaty establishing what would be
called the McMahon Line, named after a British colonial official, Henry
McMahon, who proposed the border.

India maintains that the McMahon Line, a 550-mile frontier that extends
through the Himalayas, is the official legal border between China and
India.

But China has never accepted it.

1962

\hypertarget{india-and-china-go-to-war}{%
\subsection{India and China Go to War}\label{india-and-china-go-to-war}}

Image

Crowds lining the streets to watch as Indian troops drive through the
Ladakh region during border clashes between India and China in
1962.Credit...Radloff/Hulton Archive, via Getty Images

In 1947, India declared its independence from Britain. Two years later,
the Chinese revolutionary Mao Zedong proclaimed an end to his country's
Communist Revolution and founded the People's Republic of China.

Almost immediately, the two countries --- now the world's most populous
--- found themselves at odds over the border. Tensions rose throughout
the 1950s. The Chinese insisted that Tibet was never independent and
could not have signed a treaty creating an international border. There
were several failed attempts at peaceful negotiation.

China sought to control critical roadways near its western frontier in
Xinjiang, while India and its Western allies saw any attempts at Chinese
incursion as part of a wider plot to export Maoist-style Communism
across the region.

By 1962, war had broken out.

Chinese troops crossed the McMahon Line and took up positions deep in
Indian territory, capturing mountain passes and towns. The war lasted
one month but resulted in more than 1,000 Indian deaths and over 3,000
Indians taken as prisoners. The Chinese military suffered fewer than 800
deaths.

By November, Premier Zhou Enlai of China declared a cease-fire,
unofficially redrawing the border near where Chinese troops had
conquered territory. It was the so-called
\href{https://www.nytimes.com/2020/06/16/world/asia/india-china-border.html}{Line
of Actual Control.}

\begin{center}\rule{0.5\linewidth}{\linethickness}\end{center}

1967

\hypertarget{india-pushes-china-back}{%
\subsection{India Pushes China Back}\label{india-pushes-china-back}}

Image

Chinese soldiers guarding the border on the Nathu La mountain pass
connecting India and China's Tibet Autonomous Region in
1967.Credit...Express/Hulton Archive, via Getty Images

Tensions came to a head again in 1967 along two mountain passes, Nathu
La and Cho La, that connected Sikkim --- then a kingdom and a
protectorate of India --- and China's Tibet Autonomous Region.

A scuffle broke out when Indian troops began laying barbed wire along
what they recognized as the border. The scuffles soon escalated when a
Chinese military unit began firing artillery shells at the Indians. In
the ensuing conflict, more than 150 Indians and 340 Chinese were killed.

The clashes in September and October 1967 in those passes would later be
considered the second all-out war between China and India.

But India prevailed, destroying Chinese fortifications in Nathu La and
pushing them farther back into their territory near Cho La. The change
in positions, however, meant that China and India each had different and
conflicting ideas about the location of the Line of Actual Control.

The fighting was the last time that troops on either side would be
killed --- until the skirmishes in the Galwan Valley on Tuesday. Indian
news outlets reported that Chinese soldiers had also been killed, but
Beijing was tight-lipped.

1987

\hypertarget{bloodless-clashes}{%
\subsection{Bloodless Clashes}\label{bloodless-clashes}}

Image

Chinese fighter jets at Gonggar Airport in the Tibet Autonomous Region
in June 1987.~Credit...John Giannini/Agence France-Presse --- Getty
Images

It would be 20 more years before India and China clashed again at the
disputed border.

In 1987, the Indian military was conducting a training operation to see
how fast it could move troops to the border. The large number of troops
and material arriving next to Chinese outposts surprised Chinese
commanders --- who responded by advancing toward what they considered
the Line of Actual Control.

Realizing the potential to inadvertently start a war, both India and
China de-escalated, and a crisis was averted.

2013

\hypertarget{push-comes-to-shove-in-daulat-beg-oldi}{%
\subsection{Push Comes to Shove in Daulat Beg
Oldi}\label{push-comes-to-shove-in-daulat-beg-oldi}}

Image

Chinese troops in Ladakh holding a banner in 2013 that reads, ``You've
crossed the border, please go back.'' Credit...Associated Press

Cat-and-mouse tactics unfolded on both sides.

After decades of patrolling the border, a Chinese platoon pitched a camp
near Daulat Beg Oldi in April 2013. The Indians soon followed, setting
up their own base fewer than 1,000 feet away.

The camps were later fortified by troops and heavy equipment.

By May, the sides had agreed to dismantle both encampments, but disputes
about the location of the Line of Actual Control persisted.

2017

\hypertarget{bhutan-gets-caught-in-the-middle}{%
\subsection{Bhutan Gets Caught in the
Middle}\label{bhutan-gets-caught-in-the-middle}}

Image

An Indian Army base in 2017 in Haa, Bhutan, close to a disputed border
with China.Credit...Gilles Sabrié for The New York Times

In June 2017, the Chinese set to work building a road in the Doklam
Plateau, an area of the Himalayas controlled not by India, but by its
ally Bhutan.

The plateau lies on the border of Bhutan and China, but India sees it as
a buffer zone that is close to other disputed areas with China.

Indian troops carrying weapons and operating bulldozers confronted the
Chinese with the intention of destroying the road. A standoff ensued,
soldiers threw rocks at each other, and troops from both sides suffered
injuries.

In August, the countries agreed to withdraw from the area, and China
stopped construction on the road.

2020

\hypertarget{brawls-break-out}{%
\subsection{Brawls Break Out}\label{brawls-break-out}}

Image

Indian soldiers carrying the body of their colleague, who was killed in
a border clash with Chinese troops, to an autopsy center at the Sonam
Norboo Memorial Hospital in Leh on Wednesday.Credit...Reuters

In May, melees broke out several times. In one clash at the glacial lake
Pangong Tso, Indian troops were badly inured and had to be evacuated by
helicopter. Indian analysts said Chinese troops were injured as well.

China bolstered its forces with dump trucks, excavators, troop carriers,
artillery and armored vehicles, Indian experts said.

President Trump
\href{https://twitter.com/realDonaldTrump/status/1265604027678670848}{offered
on Twitter to mediate} what he called ``a raging border dispute.''

What was clear was that it was the most serious series of clashes
between the two sides since 2017 --- and a harbinger of the deadly
confrontation to come.

Marc Santora contributed reporting.

Advertisement

\protect\hyperlink{after-bottom}{Continue reading the main story}

\hypertarget{site-index}{%
\subsection{Site Index}\label{site-index}}

\hypertarget{site-information-navigation}{%
\subsection{Site Information
Navigation}\label{site-information-navigation}}

\begin{itemize}
\tightlist
\item
  \href{https://help.nytimes.com/hc/en-us/articles/115014792127-Copyright-notice}{©~2020~The
  New York Times Company}
\end{itemize}

\begin{itemize}
\tightlist
\item
  \href{https://www.nytco.com/}{NYTCo}
\item
  \href{https://help.nytimes.com/hc/en-us/articles/115015385887-Contact-Us}{Contact
  Us}
\item
  \href{https://www.nytco.com/careers/}{Work with us}
\item
  \href{https://nytmediakit.com/}{Advertise}
\item
  \href{http://www.tbrandstudio.com/}{T Brand Studio}
\item
  \href{https://www.nytimes.com/privacy/cookie-policy\#how-do-i-manage-trackers}{Your
  Ad Choices}
\item
  \href{https://www.nytimes.com/privacy}{Privacy}
\item
  \href{https://help.nytimes.com/hc/en-us/articles/115014893428-Terms-of-service}{Terms
  of Service}
\item
  \href{https://help.nytimes.com/hc/en-us/articles/115014893968-Terms-of-sale}{Terms
  of Sale}
\item
  \href{https://spiderbites.nytimes.com}{Site Map}
\item
  \href{https://help.nytimes.com/hc/en-us}{Help}
\item
  \href{https://www.nytimes.com/subscription?campaignId=37WXW}{Subscriptions}
\end{itemize}
