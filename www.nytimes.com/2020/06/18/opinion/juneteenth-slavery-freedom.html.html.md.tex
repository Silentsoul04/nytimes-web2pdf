Sections

SEARCH

\protect\hyperlink{site-content}{Skip to
content}\protect\hyperlink{site-index}{Skip to site index}

\href{https://myaccount.nytimes.com/auth/login?response_type=cookie\&client_id=vi}{}

\href{https://www.nytimes.com/section/todayspaper}{Today's Paper}

\href{/section/opinion}{Opinion}\textbar{}Why Juneteenth Matters

\href{https://nyti.ms/37FE0D0}{https://nyti.ms/37FE0D0}

\begin{itemize}
\item
\item
\item
\item
\item
\item
\end{itemize}

\href{https://www.nytimes.com/interactive/2020/06/18/style/juneteenth-celebration.html?action=click\&pgtype=Article\&state=default\&region=TOP_BANNER\&context=storylines_menu}{Juneteenth}

\begin{itemize}
\tightlist
\item
  \href{https://www.nytimes.com/2020/06/19/us/juneteenth-2020-in-photos.html?action=click\&pgtype=Article\&state=default\&region=TOP_BANNER\&context=storylines_menu}{America
  Photographed}
\item
  \href{https://www.nytimes.com/2020/06/18/us/politics/reparations-slavery.html?action=click\&pgtype=Article\&state=default\&region=TOP_BANNER\&context=storylines_menu}{Will
  Reparations Come?}
\item
  \href{https://www.nytimes.com/2020/06/18/opinion/juneteenth-slavery-freedom.html?action=click\&pgtype=Article\&state=default\&region=TOP_BANNER\&context=storylines_menu}{``Emancipation
  Wasn't a Gift''}
\item
  \href{https://www.nytimes.com/2020/06/19/us/politics/juneteenth-tulsa-trump-rally.html?action=click\&pgtype=Article\&state=default\&region=TOP_BANNER\&context=storylines_menu}{In
  Tulsa}
\item
  \href{https://www.nytimes.com/2020/06/18/style/self-care/healing-trauma-racism-wellness.html?action=click\&pgtype=Article\&state=default\&region=TOP_BANNER\&context=storylines_menu}{Trauma
  and Self-Care}
\end{itemize}

Advertisement

\protect\hyperlink{after-top}{Continue reading the main story}

\href{/section/opinion}{Opinion}

Supported by

\protect\hyperlink{after-sponsor}{Continue reading the main story}

\hypertarget{why-juneteenth-matters}{%
\section{Why Juneteenth Matters}\label{why-juneteenth-matters}}

It was black Americans who delivered on Lincoln's promise of ``a new
birth of freedom.''

\href{https://www.nytimes.com/column/jamelle-bouie}{\includegraphics{https://static01.nyt.com/images/2019/01/24/opinion/jamelle-bouie/jamelle-bouie-thumbLarge-v3.png}}

By \href{https://www.nytimes.com/column/jamelle-bouie}{Jamelle Bouie}

Opinion Columnist

\begin{itemize}
\item
  June 18, 2020
\item
  \begin{itemize}
  \item
  \item
  \item
  \item
  \item
  \item
  \end{itemize}
\end{itemize}

\includegraphics{https://static01.nyt.com/images/2020/06/19/opinion/18bouie3/merlin_173682720_47c71b35-543d-42e2-b4cf-31553edd2062-articleLarge.jpg?quality=75\&auto=webp\&disable=upscale}

Neither Abraham Lincoln nor the Republican Party freed the slaves. They
helped set freedom in motion and eventually codified it into law with
the 13th Amendment, but they were not themselves responsible for the end
of slavery. They were not the ones who brought about its final
destruction.

Who freed the slaves? The slaves freed the slaves.

``Slave resistance,'' as the historian Manisha Sinha points out in
``\href{https://yalebooks.yale.edu/book/9780300227116/slaves-cause}{The
Slave's Cause}: A History of Abolition,'' ``lay at the heart of the
abolition movement.''

``Prominent slave revolts marked the turn toward immediate abolition,''
Sinha writes, and ``fugitive slaves united all factions of the movement
and led the abolitionists to justify revolutionary resistance to
slavery.''

When secession turned to war, it was enslaved people who turned a narrow
conflict over union into a revolutionary war for freedom. ``From the
first guns at Sumter, the strongest advocates of emancipation were the
slaves themselves,'' the historian Ira Berlin
\href{https://www.washingtonpost.com/archive/opinions/1992/12/27/how-the-slaves-freed-themselves/7d58b82c-3446-4f96-a07d-52fc868eb960/}{wrote}
in 1992. ``Lacking political standing or public voice, forbidden access
to the weapons of war, slaves tossed aside the grand pronouncements of
Lincoln and other Union leaders that the sectional conflict was only a
war for national unity and moved directly to put their own freedom ---
and that of their posterity --- atop the national agenda.''

All of this is apropos of
\href{https://www.nytimes.com/interactive/2020/06/18/style/juneteenth-celebration.html}{Juneteenth},
which commemorates June 19, 1865, when Gen. Gordon Granger entered
Galveston, Texas, to lead the Union occupation force and delivered the
news of the Emancipation Proclamation to enslaved people in the region.
This holiday, which only became a nationwide celebration (among black
Americans) in the 20th century, has grown in stature over the last
decade as a result of key anniversaries (2011 to 2015 was the
sesquicentennial of the Civil War), trends in public opinion (the
\href{https://www.vox.com/2019/3/22/18259865/great-awokening-white-liberals-race-polling-trump-2020}{growing
racial liberalism} of left-leaning whites), and the rise of the Black
Lives Matter movement.

Over the last week, as Americans continued to protest police brutality,
institutional racism and structural disadvantage in cities and towns
across the country, elected officials in New York and Virginia have
announced plans to make Juneteenth a paid holiday, as have a number of
prominent businesses like Nike, Twitter and the NFL.

There's obviously a certain opportunism here, an attempt to respond to
the moment and win favorable coverage, with as little sacrifice as
possible. (Paid holidays, while nice, are a grossly inadequate response
to calls for justice and equality.) But if Americans are going to mark
and celebrate Juneteenth, then they should do so with the knowledge and
awareness of the agency of enslaved people.

\includegraphics{https://static01.nyt.com/images/2020/06/18/opinion/18bouie1/merlin_173671071_07cb133d-1875-42a4-b181-3dc6cd7f04d5-articleLarge.jpg?quality=75\&auto=webp\&disable=upscale}

Emancipation wasn't a gift bestowed on the slaves; it was something they
took for themselves, the culmination of their long struggle for freedom,
which began as soon as chattel slavery was established in the 17th
century, and gained even greater steam with the Revolution and the birth
of a country committed, at least rhetorically, to freedom and equality.
In fighting that struggle, black Americans would open up new vistas of
\href{https://www.nytimes.com/interactive/2019/08/14/magazine/black-history-american-democracy.html}{democratic
possibility} for the entire country.

To return to Ira Berlin --- who tackled this subject in
``\href{https://www.hup.harvard.edu/catalog.php?isbn=9780674286085}{The
Long Emancipation}: The Demise of Slavery in the United States'' --- it
is useful to look at the end of slavery as ``a near-century-long
process'' rather than ``the work of a moment, even if that moment was a
great civil war.'' Those in bondage were part of this process at every
step of the way, from resistance and rebellion to escape, which gave
them the chance, as free blacks, to weigh directly on the politics of
slavery. ``They gave the slaves' oppositional activities a political
form,'' Berlin writes, ``denying the masters' claim that malingering and
tool breaking were reflections of African idiocy and indolence, that
sabotage represented the mindless thrashings of a primitive people, and
that outsiders were the ones who always inspired conspiracies and
insurrections.''

By pushing the question of emancipation into public view, black
Americans raised the issue of their ``status in freedom'' and therefore
``the question of citizenship and its attributes.'' And as the historian
Martha Jones details in
``\href{https://marthasjones.com/birthright-citizens/}{Birthright
Citizens}: A History of Race and Rights in Antebellum America,'' it is
black advocacy that ultimately shapes the nation's understanding of what
it means to be an American citizen. ``Never just objects of judicial,
legislative, or antislavery thought,'' black Americans ``drove lawmakers
to refine their thinking about citizenship. On the necessity of debating
birthright citizenship, black Americans forced the issue.''

After the Civil War, black Americans --- free and freed --- would work
to realize the promise of emancipation, and to make the South a true
democracy. They abolished property qualifications for voting and
officeholding, instituted universal manhood suffrage, opened the
region's first public schools and made them available to all children.
They stood against racial distinctions and discrimination in public life
and sought assistance for the poor and disadvantaged. Just a few years
removed from degradation and social death, these millions, wrote W.E.B.
Du Bois in ``\href{http://www.webdubois.org/wdb-BlackReconst.html}{Black
Reconstruction in America}, ``took decisive and encouraging steps toward
the widening and strengthening of human democracy.''

Juneteenth may mark just one moment in the struggle for emancipation,
but the holiday gives us an occasion to reflect on the profound
contributions of enslaved black Americans to the cause of human freedom.
It gives us another way to recognize the central place of slavery and
its demise in our national story. And it gives us an opportunity to
remember that American democracy has more authors than the shrewd
lawyers and erudite farmer-philosophers of the Revolution, that our
experiment in liberty owes as much to the men and women who toiled in
bondage as it does to anyone else in this nation's history.

\begin{center}\rule{0.5\linewidth}{\linethickness}\end{center}

\emph{The Times is committed to publishing}
\href{https://www.nytimes.com/2019/01/31/opinion/letters/letters-to-editor-new-york-times-women.html}{\emph{a
diversity of letters}} \emph{to the editor. We'd like to hear what you
think about this or any of our articles. Here are some}
\href{https://help.nytimes.com/hc/en-us/articles/115014925288-How-to-submit-a-letter-to-the-editor}{\emph{tips}}\emph{.
And here's our email:}
\href{mailto:letters@nytimes.com}{\emph{letters@nytimes.com}}\emph{.}

\emph{Follow The New York Times Opinion section on}
\href{https://www.facebook.com/nytopinion}{\emph{Facebook}}\emph{,}
\href{http://twitter.com/NYTOpinion}{\emph{Twitter (@NYTopinion)}}
\emph{and}
\href{https://www.instagram.com/nytopinion/}{\emph{Instagram}}\emph{.}

Advertisement

\protect\hyperlink{after-bottom}{Continue reading the main story}

\hypertarget{site-index}{%
\subsection{Site Index}\label{site-index}}

\hypertarget{site-information-navigation}{%
\subsection{Site Information
Navigation}\label{site-information-navigation}}

\begin{itemize}
\tightlist
\item
  \href{https://help.nytimes.com/hc/en-us/articles/115014792127-Copyright-notice}{©~2020~The
  New York Times Company}
\end{itemize}

\begin{itemize}
\tightlist
\item
  \href{https://www.nytco.com/}{NYTCo}
\item
  \href{https://help.nytimes.com/hc/en-us/articles/115015385887-Contact-Us}{Contact
  Us}
\item
  \href{https://www.nytco.com/careers/}{Work with us}
\item
  \href{https://nytmediakit.com/}{Advertise}
\item
  \href{http://www.tbrandstudio.com/}{T Brand Studio}
\item
  \href{https://www.nytimes.com/privacy/cookie-policy\#how-do-i-manage-trackers}{Your
  Ad Choices}
\item
  \href{https://www.nytimes.com/privacy}{Privacy}
\item
  \href{https://help.nytimes.com/hc/en-us/articles/115014893428-Terms-of-service}{Terms
  of Service}
\item
  \href{https://help.nytimes.com/hc/en-us/articles/115014893968-Terms-of-sale}{Terms
  of Sale}
\item
  \href{https://spiderbites.nytimes.com}{Site Map}
\item
  \href{https://help.nytimes.com/hc/en-us}{Help}
\item
  \href{https://www.nytimes.com/subscription?campaignId=37WXW}{Subscriptions}
\end{itemize}
