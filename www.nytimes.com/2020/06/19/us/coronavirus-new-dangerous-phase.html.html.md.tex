Sections

SEARCH

\protect\hyperlink{site-content}{Skip to
content}\protect\hyperlink{site-index}{Skip to site index}

\href{https://www.nytimes.com/section/us}{U.S.}

\href{https://myaccount.nytimes.com/auth/login?response_type=cookie\&client_id=vi}{}

\href{https://www.nytimes.com/section/todayspaper}{Today's Paper}

\href{/section/us}{U.S.}\textbar{}W.H.O. Warns of `Dangerous Phase' of
Pandemic as Outbreaks Widen

\url{https://nyti.ms/2BsAkZ9}

\begin{itemize}
\item
\item
\item
\item
\item
\end{itemize}

\href{https://www.nytimes.com/news-event/coronavirus?action=click\&pgtype=Article\&state=default\&region=TOP_BANNER\&context=storylines_menu}{The
Coronavirus Outbreak}

\begin{itemize}
\tightlist
\item
  live\href{https://www.nytimes.com/2020/08/01/world/coronavirus-covid-19.html?action=click\&pgtype=Article\&state=default\&region=TOP_BANNER\&context=storylines_menu}{Latest
  Updates}
\item
  \href{https://www.nytimes.com/interactive/2020/us/coronavirus-us-cases.html?action=click\&pgtype=Article\&state=default\&region=TOP_BANNER\&context=storylines_menu}{Maps
  and Cases}
\item
  \href{https://www.nytimes.com/interactive/2020/science/coronavirus-vaccine-tracker.html?action=click\&pgtype=Article\&state=default\&region=TOP_BANNER\&context=storylines_menu}{Vaccine
  Tracker}
\item
  \href{https://www.nytimes.com/interactive/2020/07/29/us/schools-reopening-coronavirus.html?action=click\&pgtype=Article\&state=default\&region=TOP_BANNER\&context=storylines_menu}{What
  School May Look Like}
\item
  \href{https://www.nytimes.com/live/2020/07/31/business/stock-market-today-coronavirus?action=click\&pgtype=Article\&state=default\&region=TOP_BANNER\&context=storylines_menu}{Economy}
\end{itemize}

Advertisement

\protect\hyperlink{after-top}{Continue reading the main story}

Supported by

\protect\hyperlink{after-sponsor}{Continue reading the main story}

\hypertarget{who-warns-of-dangerous-phase-of-pandemic-as-outbreaks-widen-}{%
\section{W.H.O. Warns of `Dangerous Phase' of Pandemic as Outbreaks
Widen
}\label{who-warns-of-dangerous-phase-of-pandemic-as-outbreaks-widen-}}

Beijing and Seoul have had a recent surge in coronavirus cases, and
businesses are recoiling in America as infections sharply increase in
Southern and Western states.

\includegraphics{https://static01.nyt.com/images/2020/06/19/us/19VIRUS-NEWPHASE-delray/merlin_173646633_383e4270-e386-4fe1-8b16-882fb5ca965c-articleLarge.jpg?quality=75\&auto=webp\&disable=upscale}

\href{https://www.nytimes.com/by/julie-bosman}{\includegraphics{https://static01.nyt.com/images/2018/11/09/multimedia/author-julie-bosman/author-julie-bosman-thumbLarge.png}}

By \href{https://www.nytimes.com/by/julie-bosman}{Julie Bosman}

\begin{itemize}
\item
  Published June 19, 2020Updated June 23, 2020
\item
  \begin{itemize}
  \item
  \item
  \item
  \item
  \item
  \end{itemize}
\end{itemize}

CHICAGO --- The world has entered a ``new and dangerous phase'' of the
coronavirus pandemic, a top official from the World Health Organization
said on Friday, a stark warning that came as the United States struggled
to control spiraling outbreaks and as business leaders signaled growing
unease with the country's ability to effectively contend with the virus.

Coronavirus cases spiked sharply across the American South and West,
particularly in states that loosened restrictions on businesses several
weeks ago.

In Florida, Oklahoma, South Carolina and Arizona, daily counts of new
coronavirus cases reached their highest levels of the pandemic this
week. Texas, which has seen known cases double in the past month, became
the sixth state to surpass 100,000 cases,
\href{https://www.nytimes.com/interactive/2020/us/texas-coronavirus-cases.html}{according
to a New York Times database of cases in the United States}.

Around the country, there were indications that major companies and
sports teams were changing their own plans as the new surges emerged.

Apple said it was temporarily closing 11 retail stores across four
states amid an uptick in cases.
\href{https://www.nytimes.com/2020/06/18/business/AMC-theaters-masks-coronavirus.html}{AMC
Entertainment reversed course on its mask policy} on Friday, saying it
will now require patrons to wear face coverings when movie theaters
reopen next month.

Two Major League Baseball clubs, the Philadelphia Phillies and the
Toronto Blue Jays, and a professional hockey team, the Tampa Bay
Lightning, abruptly shut down training facilities in Florida over
concerns that the virus was threatening players' safety.

Across the globe, the outlook for containing the coronavirus worsened. A
pandemic that had been defined early on by a series of shifting
epicenters --- including Wuhan, China; Iran; northern Italy; Spain; and
New York --- was now distinguished by a wide and expanding scope.
Eighty-one nations
\href{https://www.nytimes.com/interactive/2020/world/coronavirus-maps.html}{have
seen a growth in new cases} over the past two weeks. Only 36 have seen
declines.

``Many people are understandably fed up with being at home,'' Dr. Tedros
Adhanom Ghebreyesus, the director general of the W.H.O., said in a news
conference in which he described the new phase of the virus. ``Countries
are understandably eager to open up their societies and their economies.
But the virus is still spreading fast. It is still deadly and most
people are still susceptible.''

A sobering lesson in the virus's tenacity came in China, where officials
had recently proclaimed that they had vanquished the virus --- only to
see it surge back in Beijing, the capital. That metropolis, of 21
million people, is facing new restrictions on travel and renewed school
closures. Seoul, South Korea, also reported a new surge in cases on
Friday.

\hypertarget{latest-updates-global-coronavirus-outbreak}{%
\section{\texorpdfstring{\href{https://www.nytimes.com/2020/08/01/world/coronavirus-covid-19.html?action=click\&pgtype=Article\&state=default\&region=MAIN_CONTENT_1\&context=storylines_live_updates}{Latest
Updates: Global Coronavirus
Outbreak}}{Latest Updates: Global Coronavirus Outbreak}}\label{latest-updates-global-coronavirus-outbreak}}

Updated 2020-08-02T07:42:09.613Z

\begin{itemize}
\tightlist
\item
  \href{https://www.nytimes.com/2020/08/01/world/coronavirus-covid-19.html?action=click\&pgtype=Article\&state=default\&region=MAIN_CONTENT_1\&context=storylines_live_updates\#link-34047410}{The
  U.S. reels as July cases more than double the total of any other
  month.}
\item
  \href{https://www.nytimes.com/2020/08/01/world/coronavirus-covid-19.html?action=click\&pgtype=Article\&state=default\&region=MAIN_CONTENT_1\&context=storylines_live_updates\#link-780ec966}{Top
  U.S. officials work to break an impasse over the federal jobless
  benefit.}
\item
  \href{https://www.nytimes.com/2020/08/01/world/coronavirus-covid-19.html?action=click\&pgtype=Article\&state=default\&region=MAIN_CONTENT_1\&context=storylines_live_updates\#link-2bc8948}{Its
  outbreak untamed, Melbourne goes into even greater lockdown.}
\end{itemize}

\href{https://www.nytimes.com/2020/08/01/world/coronavirus-covid-19.html?action=click\&pgtype=Article\&state=default\&region=MAIN_CONTENT_1\&context=storylines_live_updates}{See
more updates}

More live coverage:
\href{https://www.nytimes.com/live/2020/07/31/business/stock-market-today-coronavirus?action=click\&pgtype=Article\&state=default\&region=MAIN_CONTENT_1\&context=storylines_live_updates}{Markets}

For weeks, the United States has been slowly but determinedly returning
to its pre-pandemic existence amid economic turmoil. Businesses
reopened, summer camps started and retail workers returned to stores.

But efforts to boost the economy by bringing more people back to work
may be happening too soon, experts said, warning that the economic
outlook in the United States remains wildly uncertain.

\includegraphics{https://static01.nyt.com/images/2020/06/19/us/19VIRUS-TEXAS-ladybird/merlin_173638956_7a4f3057-1941-40f4-8c63-ec3ecc7ac339-articleLarge.jpg?quality=75\&auto=webp\&disable=upscale}

Eric Rosengren, the president of the Federal Reserve Bank of Boston and
an influential policy maker within the central bank system, cited rising
caseloads in South Carolina and Florida as he cautioned of the economic
impact of states reopening before the virus was under control. The
tension between a tumbling economy and a global pandemic remained stark.

``I expect the economic rebound in the second half of the year to be
less than was hoped for at the outset of the pandemic,'' Mr. Rosengren
said, citing the virus's continuing spread and the acceleration of new
cases in many states.

Around the world, risks are multiplying as nations reopen their
economies.

In India, which placed all 1.3 billion of its citizens under a lockdown
--- then
\href{https://www.nytimes.com/2020/06/10/world/asia/reopening-before-coronavirus-ends.html}{moved
to reopen} even with its strained public health system near the breaking
point --- officials reported a record number of new cases this week. And
the virus is now spreading rapidly in Pakistan and Bangladesh.

Some countries where caseloads had appeared to taper --- including
Israel, Sweden and Costa Rica --- are now watching them rise.

Cases have continued to surge across much of the United States, with new
single-day infection records reported in nine states. More widespread
testing is no doubt playing some role in the increase in the number of
known cases. But growing hospitalizations and rising rates of positive
tests compared with total tests in many of those states make clear that
the virus is raging uncontrolled across much of the Sun Belt.

In Arizona and Texas, more people with the coronavirus are hospitalized
now than at any previous point in the pandemic. In Utah, the percent of
positive tests compared with total tests reached the highest levels yet
this month. In Nevada, the percent of positive tests recently began
increasing again after more than a month of sustained declines.

Image

Gov. Doug Ducey of Arizona, where more people are hospitalized with the
virus than at any earlier point in the pandemic. Credit...Michael
Chow/POOL, The Arizona Republic, via Associated Press

Dr. Jeff Duchin, the health officer for Seattle and King County, Wash.,
said in a statement on Friday that the area had seen a spike in cases in
the last week, as it proceeded to loosen restrictions. ``As we move into
Phase 2 and for the foreseeable future, our risk will be increasing, not
decreasing,'' Dr. Duchin said. ``Covid-19 has not gone away and we must
take the ongoing risk very seriously.''

\href{https://www.nytimes.com/news-event/coronavirus?action=click\&pgtype=Article\&state=default\&region=MAIN_CONTENT_3\&context=storylines_faq}{}

\hypertarget{the-coronavirus-outbreak-}{%
\subsubsection{The Coronavirus Outbreak
›}\label{the-coronavirus-outbreak-}}

\hypertarget{frequently-asked-questions}{%
\paragraph{Frequently Asked
Questions}\label{frequently-asked-questions}}

Updated July 27, 2020

\begin{itemize}
\item ~
  \hypertarget{should-i-refinance-my-mortgage}{%
  \paragraph{Should I refinance my
  mortgage?}\label{should-i-refinance-my-mortgage}}

  \begin{itemize}
  \tightlist
  \item
    \href{https://www.nytimes.com/article/coronavirus-money-unemployment.html?action=click\&pgtype=Article\&state=default\&region=MAIN_CONTENT_3\&context=storylines_faq}{It
    could be a good idea,} because mortgage rates have
    \href{https://www.nytimes.com/2020/07/16/business/mortgage-rates-below-3-percent.html?action=click\&pgtype=Article\&state=default\&region=MAIN_CONTENT_3\&context=storylines_faq}{never
    been lower.} Refinancing requests have pushed mortgage applications
    to some of the highest levels since 2008, so be prepared to get in
    line. But defaults are also up, so if you're thinking about buying a
    home, be aware that some lenders have tightened their standards.
  \end{itemize}
\item ~
  \hypertarget{what-is-school-going-to-look-like-in-september}{%
  \paragraph{What is school going to look like in
  September?}\label{what-is-school-going-to-look-like-in-september}}

  \begin{itemize}
  \tightlist
  \item
    It is unlikely that many schools will return to a normal schedule
    this fall, requiring the grind of
    \href{https://www.nytimes.com/2020/06/05/us/coronavirus-education-lost-learning.html?action=click\&pgtype=Article\&state=default\&region=MAIN_CONTENT_3\&context=storylines_faq}{online
    learning},
    \href{https://www.nytimes.com/2020/05/29/us/coronavirus-child-care-centers.html?action=click\&pgtype=Article\&state=default\&region=MAIN_CONTENT_3\&context=storylines_faq}{makeshift
    child care} and
    \href{https://www.nytimes.com/2020/06/03/business/economy/coronavirus-working-women.html?action=click\&pgtype=Article\&state=default\&region=MAIN_CONTENT_3\&context=storylines_faq}{stunted
    workdays} to continue. California's two largest public school
    districts --- Los Angeles and San Diego --- said on July 13, that
    \href{https://www.nytimes.com/2020/07/13/us/lausd-san-diego-school-reopening.html?action=click\&pgtype=Article\&state=default\&region=MAIN_CONTENT_3\&context=storylines_faq}{instruction
    will be remote-only in the fall}, citing concerns that surging
    coronavirus infections in their areas pose too dire a risk for
    students and teachers. Together, the two districts enroll some
    825,000 students. They are the largest in the country so far to
    abandon plans for even a partial physical return to classrooms when
    they reopen in August. For other districts, the solution won't be an
    all-or-nothing approach.
    \href{https://bioethics.jhu.edu/research-and-outreach/projects/eschool-initiative/school-policy-tracker/}{Many
    systems}, including the nation's largest, New York City, are
    devising
    \href{https://www.nytimes.com/2020/06/26/us/coronavirus-schools-reopen-fall.html?action=click\&pgtype=Article\&state=default\&region=MAIN_CONTENT_3\&context=storylines_faq}{hybrid
    plans} that involve spending some days in classrooms and other days
    online. There's no national policy on this yet, so check with your
    municipal school system regularly to see what is happening in your
    community.
  \end{itemize}
\item ~
  \hypertarget{is-the-coronavirus-airborne}{%
  \paragraph{Is the coronavirus
  airborne?}\label{is-the-coronavirus-airborne}}

  \begin{itemize}
  \tightlist
  \item
    The coronavirus
    \href{https://www.nytimes.com/2020/07/04/health/239-experts-with-one-big-claim-the-coronavirus-is-airborne.html?action=click\&pgtype=Article\&state=default\&region=MAIN_CONTENT_3\&context=storylines_faq}{can
    stay aloft for hours in tiny droplets in stagnant air}, infecting
    people as they inhale, mounting scientific evidence suggests. This
    risk is highest in crowded indoor spaces with poor ventilation, and
    may help explain super-spreading events reported in meatpacking
    plants, churches and restaurants.
    \href{https://www.nytimes.com/2020/07/06/health/coronavirus-airborne-aerosols.html?action=click\&pgtype=Article\&state=default\&region=MAIN_CONTENT_3\&context=storylines_faq}{It's
    unclear how often the virus is spread} via these tiny droplets, or
    aerosols, compared with larger droplets that are expelled when a
    sick person coughs or sneezes, or transmitted through contact with
    contaminated surfaces, said Linsey Marr, an aerosol expert at
    Virginia Tech. Aerosols are released even when a person without
    symptoms exhales, talks or sings, according to Dr. Marr and more
    than 200 other experts, who
    \href{https://academic.oup.com/cid/article/doi/10.1093/cid/ciaa939/5867798}{have
    outlined the evidence in an open letter to the World Health
    Organization}.
  \end{itemize}
\item ~
  \hypertarget{what-are-the-symptoms-of-coronavirus}{%
  \paragraph{What are the symptoms of
  coronavirus?}\label{what-are-the-symptoms-of-coronavirus}}

  \begin{itemize}
  \tightlist
  \item
    Common symptoms
    \href{https://www.nytimes.com/article/symptoms-coronavirus.html?action=click\&pgtype=Article\&state=default\&region=MAIN_CONTENT_3\&context=storylines_faq}{include
    fever, a dry cough, fatigue and difficulty breathing or shortness of
    breath.} Some of these symptoms overlap with those of the flu,
    making detection difficult, but runny noses and stuffy sinuses are
    less common.
    \href{https://www.nytimes.com/2020/04/27/health/coronavirus-symptoms-cdc.html?action=click\&pgtype=Article\&state=default\&region=MAIN_CONTENT_3\&context=storylines_faq}{The
    C.D.C. has also} added chills, muscle pain, sore throat, headache
    and a new loss of the sense of taste or smell as symptoms to look
    out for. Most people fall ill five to seven days after exposure, but
    symptoms may appear in as few as two days or as many as 14 days.
  \end{itemize}
\item ~
  \hypertarget{does-asymptomatic-transmission-of-covid-19-happen}{%
  \paragraph{Does asymptomatic transmission of Covid-19
  happen?}\label{does-asymptomatic-transmission-of-covid-19-happen}}

  \begin{itemize}
  \tightlist
  \item
    So far, the evidence seems to show it does. A widely cited
    \href{https://www.nature.com/articles/s41591-020-0869-5}{paper}
    published in April suggests that people are most infectious about
    two days before the onset of coronavirus symptoms and estimated that
    44 percent of new infections were a result of transmission from
    people who were not yet showing symptoms. Recently, a top expert at
    the World Health Organization stated that transmission of the
    coronavirus by people who did not have symptoms was ``very rare,''
    \href{https://www.nytimes.com/2020/06/09/world/coronavirus-updates.html?action=click\&pgtype=Article\&state=default\&region=MAIN_CONTENT_3\&context=storylines_faq\#link-1f302e21}{but
    she later walked back that statement.}
  \end{itemize}
\end{itemize}

Movie theaters, shuttered for months because of the coronavirus, have
struggled to find a balance between making money and ensuring public
safety in the midst of a pandemic. Some companies have followed the
guidance of scientists and required patrons to wear masks or face
coverings for entry, but they have encountered resistance from customers
who see mask-wearing as an infringement of personal liberty.

Alamo Drafthouse Cinema, a company with 41 theaters in 10 states, said
on Friday that it would require face masks in its theaters ``except when
eating or drinking,'' saying the safety of patrons and workers could not
be compromised. ``This is not political,'' the theater chain
\href{https://twitter.com/alamodrafthouse/status/1274016499821404161}{said
in a tweet}.

Regal Cinemas joined AMC and Alamo on Friday afternoon in stating that
all movie theater employees and patrons would be required to wear masks.
The chain, which had previously said it would require masks only in
cities that mandated them, said that disposable masks would be made
available to customers who needed them.

But a rival chain, Cinemark, began reopening some theaters in Texas on
Friday without requiring face masks. ``It's a big country out there,''
Mark Zoradi, Cinemark's chief executive, told the entertainment news
site Deadline on Wednesday. ``There are places that may require it.
California may be one. If it's required in California, we'll abide by
it. There are other places like Texas where it's not required. In those
cases, we'll highly recommend, but not require it.''

All over, businesses were grappling with those sorts of decisions, as
state rules have loosened and cases have risen. The possibility of
repeated openings and closings was emerging.

In Arizona, Gila River Hotels and Casinos
\href{https://playatgila.com/wp-content/uploads/2020/06/GRGE-Closure-final.pdf}{shut
its doors again} this week after reopening in mid-May with new safety
procedures in place. The company said it would close for two weeks ``to
see whether the recent rise in Arizona Covid cases subsides and to
re-examine every aspect of its operation.''

Even as coronavirus cases increase in many states, there was reason for
some optimism about the national picture over all. Coronavirus deaths in
the United States have fallen to roughly 700 a day from a peak of more
than 2,000 a day, and some of the country's hardest-hit regions have
showed sustained improvement. New case reports continue to plummet
across most of the Northeast and much of the Midwest. The Chicago,
Boston, Milwaukee, Detroit and New York areas all continue to improve.

But there were also worrisome signs in those same regions. Case numbers
have started trending upward again in Kansas after weeks of falling. The
La Crosse, Wis., area is experiencing a period of explosive case growth.
And new trouble spots have cropped up in parts of Missouri, Iowa and
Pennsylvania.

As more Americans return to ordinary routines, clusters are emerging in
places that had been largely closed. More than 230 cases were tied to a
Pentecostal church in Oregon, and far smaller clusters were reported
recently at churches in Alabama, Wisconsin and West Virginia. Outbreaks
have also been reported recently at a Panda Express restaurant in
California, an Advance Auto Parts store in Colorado and among athletes
at Kansas State University and the University of Texas.

South Carolina's state epidemiologist pleaded with residents to wear
masks and practice social distancing as that state identified more than
990 new cases of the virus on Thursday. It was the sixth time in 10 days
that the state broke its single-day case record.

``We understand that what we're continuing to ask of everyone is not
easy and that many are tired of hearing the same warnings and of taking
the same daily precautions,'' Dr. Linda Bell, the epidemiologist, said
in a statement. ``Every day that we don't all do our part, we are
extending the duration of illnesses, missed work, hospitalizations and
deaths in our state.''

Reporting was contributed by Rick Gladstone from Eastham, Mass.; David
Gelles from Putnam Valley, N.Y.; Gillian Friedman from Salt Lake City;
Gillian R. Brassil from Andover, Mass.; Jeanna Smialek from Washington;
Mitch Smith from Chicago; and David Waldstein from New York.

Advertisement

\protect\hyperlink{after-bottom}{Continue reading the main story}

\hypertarget{site-index}{%
\subsection{Site Index}\label{site-index}}

\hypertarget{site-information-navigation}{%
\subsection{Site Information
Navigation}\label{site-information-navigation}}

\begin{itemize}
\tightlist
\item
  \href{https://help.nytimes.com/hc/en-us/articles/115014792127-Copyright-notice}{©~2020~The
  New York Times Company}
\end{itemize}

\begin{itemize}
\tightlist
\item
  \href{https://www.nytco.com/}{NYTCo}
\item
  \href{https://help.nytimes.com/hc/en-us/articles/115015385887-Contact-Us}{Contact
  Us}
\item
  \href{https://www.nytco.com/careers/}{Work with us}
\item
  \href{https://nytmediakit.com/}{Advertise}
\item
  \href{http://www.tbrandstudio.com/}{T Brand Studio}
\item
  \href{https://www.nytimes.com/privacy/cookie-policy\#how-do-i-manage-trackers}{Your
  Ad Choices}
\item
  \href{https://www.nytimes.com/privacy}{Privacy}
\item
  \href{https://help.nytimes.com/hc/en-us/articles/115014893428-Terms-of-service}{Terms
  of Service}
\item
  \href{https://help.nytimes.com/hc/en-us/articles/115014893968-Terms-of-sale}{Terms
  of Sale}
\item
  \href{https://spiderbites.nytimes.com}{Site Map}
\item
  \href{https://help.nytimes.com/hc/en-us}{Help}
\item
  \href{https://www.nytimes.com/subscription?campaignId=37WXW}{Subscriptions}
\end{itemize}
