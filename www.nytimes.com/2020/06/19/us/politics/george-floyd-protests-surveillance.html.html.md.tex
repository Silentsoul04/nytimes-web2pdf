Sections

SEARCH

\protect\hyperlink{site-content}{Skip to
content}\protect\hyperlink{site-index}{Skip to site index}

\href{https://www.nytimes.com/section/politics}{Politics}

\href{https://myaccount.nytimes.com/auth/login?response_type=cookie\&client_id=vi}{}

\href{https://www.nytimes.com/section/todayspaper}{Today's Paper}

\href{/section/politics}{Politics}\textbar{}U.S. Watched George Floyd
Protests in 15 Cities Using Aerial Surveillance

\url{https://nyti.ms/3dcED89}

\begin{itemize}
\item
\item
\item
\item
\item
\end{itemize}

\href{https://www.nytimes.com/news-event/george-floyd-protests-minneapolis-new-york-los-angeles?action=click\&pgtype=Article\&state=default\&region=TOP_BANNER\&context=storylines_menu}{Race
and America}

\begin{itemize}
\tightlist
\item
  \href{https://www.nytimes.com/2020/07/26/us/protests-portland-seattle-trump.html?action=click\&pgtype=Article\&state=default\&region=TOP_BANNER\&context=storylines_menu}{Protesters
  Return to Other Cities}
\item
  \href{https://www.nytimes.com/2020/07/24/us/portland-oregon-protests-white-race.html?action=click\&pgtype=Article\&state=default\&region=TOP_BANNER\&context=storylines_menu}{Portland
  at the Center}
\item
  \href{https://www.nytimes.com/2020/07/23/podcasts/the-daily/portland-protests.html?action=click\&pgtype=Article\&state=default\&region=TOP_BANNER\&context=storylines_menu}{Podcast:
  Showdown in Portland}
\item
  \href{https://www.nytimes.com/interactive/2020/07/16/us/black-lives-matter-protests-louisville-breonna-taylor.html?action=click\&pgtype=Article\&state=default\&region=TOP_BANNER\&context=storylines_menu}{45
  Days in Louisville}
\end{itemize}

Advertisement

\protect\hyperlink{after-top}{Continue reading the main story}

Supported by

\protect\hyperlink{after-sponsor}{Continue reading the main story}

\hypertarget{us-watched-george-floyd-protests-in-15-cities-using-aerial-surveillance}{%
\section{U.S. Watched George Floyd Protests in 15 Cities Using Aerial
Surveillance}\label{us-watched-george-floyd-protests-in-15-cities-using-aerial-surveillance}}

From Minneapolis to Buffalo, Homeland Security officials dispatched
drones, helicopters and airplanes to monitor Black Lives Matter
protests.

\includegraphics{https://static01.nyt.com/images/2020/06/19/us/politics/19dc-unrest-surveillance1/merlin_173138961_d4fa84e5-33d4-4dff-8e2e-5651148162a6-articleLarge.jpg?quality=75\&auto=webp\&disable=upscale}

\href{https://www.nytimes.com/by/zolan-kanno-youngs}{\includegraphics{https://static01.nyt.com/images/2019/12/13/reader-center/author-zolan-kanno-youngs/author-zolan-kanno-youngs-thumbLarge.png}}

By \href{https://www.nytimes.com/by/zolan-kanno-youngs}{Zolan
Kanno-Youngs}

\begin{itemize}
\item
  June 19, 2020
\item
  \begin{itemize}
  \item
  \item
  \item
  \item
  \item
  \end{itemize}
\end{itemize}

GRAND FORKS, N.D. --- The Department of Homeland Security deployed
helicopters, airplanes and drones over 15 cities where demonstrators
gathered to protest the death of
\href{https://www.nytimes.com/2020/07/28/us/umbrella-man-identified-minneapolis.html}{George
Floyd}, logging at least 270 hours of surveillance, far more than
previously revealed, according to Customs and Border Protection data.

The department's dispatching of unmanned aircraft over protests in
Minneapolis last month sparked a congressional inquiry and widespread
accusations that the federal agency had infringed on the privacy rights
of demonstrators.

But that was just one piece of a nationwide operation that deployed
resources usually used to patrol the U.S. border for smugglers and
illegal crossings. Aircraft filmed demonstrations in Dayton, Ohio; New
York City; Buffalo and Philadelphia, among other cities, sending video
footage in real time to control centers managed by Air and Marine
Operations, a branch of Customs and Border Protection.

The footage was then fed into a digital network managed by the Homeland
Security Department, called ``Big Pipe,'' which can be accessed by other
federal agencies and local police departments for use in future
investigations, according to senior officials with Air and Marine
Operations.

The revelations come amid a fierce national debate over police tactics
and the role that federal law enforcement and military forces should
play in controlling or monitoring demonstrations. The clearing of
demonstrators from Lafayette Park in Washington for a presidential photo
op is still under scrutiny. The Air Force inspector general is
\href{https://www.nytimes.com/2020/06/18/us/politics/investigation-military-surveillance-planes-george-floyd-protests.html}{investigating}
whether the military improperly used a reconnaissance plane to monitor
peaceful protesters in Washington and Minneapolis this month.

And the National Guard in the District of Columbia has already reached a
preliminary conclusion that a lack of clarity in commands led to one of
its medical evacuation helicopters swooping low on protesters in the
nation's capital. Renewed calls to demilitarize police work have not
only come from criminal justice advocates but
\href{https://www.nytimes.com/2020/06/17/us/politics/trump-protesters.html}{also
former Republican Homeland Security officials such as Michael Chertoff
and Tom Ridge}, the first two leaders of the Homeland Security
Department, which was created after the Sept. 11, 2001, attacks.

Officials at the Customs and Border Protection base here rejected any
notion that their fleet of aircraft had been misused, either to violate
privacy rights or intimidate protesters.

``The worst part for me is when we're made out to be storm troopers,''
said David Fulcher, the deputy director for air operations at the
National Air Security Operations Center in Grand Forks. ``We believe in
peaceful protests.''

The aircraft, they said, were used to provide an eagle-eyed view of
violent acts and arson. The Predator drone deployed to Minneapolis, like
eight other unmanned aircraft owned by Air and Marine Operations, was
neither armed nor equipped with facial recognition technology and flew
at a height that made it impossible to identify individuals or license
plates, according to senior officials here.

``The legend of the Predator --- the all-seeing, all-knowing,
hover-outside-your-window Predator --- it's just not accurate,'' Mr.
Fulcher said. ``The technology is not there.''

But House Democrats and privacy advocates still worry over the potential
dissemination of the footage and the chilling effect that militarized
aircrafts could have on peaceful protests.

Earlier this month, Democrats with the House Oversight Committee,
including Representatives Carolyn B. Maloney and Alexandria
Ocasio-Cortez of New York, Jamie Raskin of Maryland, and Stephen F.
Lynch and Ayanna Pressley, both of Massachusetts, protested to Chad
Wolf, the acting secretary of homeland security.

``This administration has undermined the First Amendment freedoms of
Americans of all races who are rightfully protesting George Floyd's
killing,'' the Democrats said in
\href{https://oversight.house.gov/sites/democrats.oversight.house.gov/files/2020-06-05.CBM\%20et.\%20al\%20to\%20Wolf-\%20DHS\%20re\%20Peaceful\%20Protestors_0.pdf}{a
letter to Mr. Wolf.}``The deployment of drones and officers to surveil
protests is a gross abuse of authority and is particularly chilling when
used against Americans who are protesting law enforcement brutality.''

But Democrats apparently were unaware of the breadth of the agency's
actions. Most of surveillance was done with planes and helicopters. Air
and Marine Operations did dispatch drones to two demonstrations --- in
Minneapolis and in Del Rio, Texas.

\includegraphics{https://static01.nyt.com/images/2020/06/19/us/politics/19dc-unrest-surveillance2/merlin_173160402_2294710b-d999-4375-89cd-7eab4e2fd509-articleLarge.jpg?quality=75\&auto=webp\&disable=upscale}

The agency's AS350 helicopters conducted more than 168 hours of
surveillance of protests in 13 different cities, the longest stretch
being 58 hours over Detroit, according to data provided by Air and
Marine Operations. The agency also deployed a Blackhawk helicopter for
nearly 13 hours, assisting other federal agencies with surveillance in
Washington, D.C. Kris Grogan, a spokesman for Customs and Border
Protection, said the agency's Blackhawk was not one of the helicopters
that flew low over the demonstrators and caused panic.

A Cessna single-engine plane conducted nearly 58 hours of surveillance,
more than 38 of them over Buffalo. Mark Morgan, the acting commissioner
of Customs and Border Protection, said in a tweet this month that the
officers manning that plane helped track down suspects who used an
S.U.V. to hit local police on the ground.

Most of the requests did not come from local police departments. In
Minneapolis, the call came from an agent in Homeland Security
Investigations, the branch of Immigration and Customs Enforcement that
conducts longer-term investigations into terrorists, weapons trafficking
and drug smuggling.

The agent, who was on the ground in Minneapolis and works with Air and
Marine Operations regularly, requested the help on May 28 after reports
of arson and violence in the area. Air and Marine Operations, which also
dispatches drones from Sierra Vista, Ariz., and Corpus Christi, Texas,
was not able to send the aircraft until the next day. After about two
hours of surveilling, the agent and other law enforcement agencies said
it was no longer needed.

``It's discretionary, but there's a huge degree of accountability as far
as who can say yes or no to deploying these assets,'' said Jonathan
Miller, the executive director of the National Air Security Operations
at Customs and Border Protection.

Air and Marine Operations officials said agency protocol prevents
infringement on the right to protest. The drones, which can stay in the
air from 12 to roughly 24 hours depending on how much radar equipment is
attached, are directed to fly no lower than 19,000 feet. From that
height, the ``electrical optical-infrared ball'' on the drones wouldn't
allow the operators to see faces, eyes or hair color, according to the
Department of Homeland Security's privacy impact assessment for the
aircrafts.

But operators can track movements of protesters or looters, direct law
enforcement on the ground and see if someone is wearing a backpack or
rifle. And stored footage could be accessed later to corroborate
investigative findings**,** such as a witness account that a fire was
set at a given time by a small group or the escape route of a suspect.

Image

Air and Marine Operations dispatched drones to two demonstrations,
including in Minneapolis.Credit...Victor J. Blue for The New York Times

A live feed of the footage is sent to a mobile operations center, where
a group of agents monitor television screens while moving the drone with
joysticks. Other federal agents that request a view from the sky can
also see the footage on their phones, Mr. Fulcher said.

Mr. Fulcher said the surveillance footage, stored on the aircraft and in
control rooms, is overwritten after an average of 30 days by new feeds.
But video feeds and radar images sent to ``Big Pipe'' can also be
analyzed by Homeland Security Department intelligence officers. That
data may be stored for ``up to five years,'' according to Homeland
Security's Privacy Impact Assessment. If federal agencies or police
departments can prove they need the footage for a criminal
investigation, the video can be provided, according to the document and
Mr. Fulcher.

The Department of Homeland Security did not say whether any law
enforcement agencies had requested footage of the demonstrations.

Jay Stanley, a senior policy analyst at the American Civil Liberties
Union, said the aircraft could discourage people from protesting. The
concern is not only what the border agency is doing with the aircraft
and footage but how future operations could adapt to quickly advancing
technology.

``You see an aircraft, you have no idea currently what technologies that
aircraft is carrying,'' Mr. Stanley said. ``There is something
militaristic and dominating about a militarized police aircraft hovering
over you when you're out there protesting police abuse.''

Air and Marine Operations recorded more than 92,800 hours of flight time
in the fiscal year that ended in September, most of that spent
patrolling the border. But the helicopters, planes and drones spent more
than 8,000 hours helping law enforcement agencies with search and rescue
missions and other criminal investigations.

Advertisement

\protect\hyperlink{after-bottom}{Continue reading the main story}

\hypertarget{site-index}{%
\subsection{Site Index}\label{site-index}}

\hypertarget{site-information-navigation}{%
\subsection{Site Information
Navigation}\label{site-information-navigation}}

\begin{itemize}
\tightlist
\item
  \href{https://help.nytimes.com/hc/en-us/articles/115014792127-Copyright-notice}{©~2020~The
  New York Times Company}
\end{itemize}

\begin{itemize}
\tightlist
\item
  \href{https://www.nytco.com/}{NYTCo}
\item
  \href{https://help.nytimes.com/hc/en-us/articles/115015385887-Contact-Us}{Contact
  Us}
\item
  \href{https://www.nytco.com/careers/}{Work with us}
\item
  \href{https://nytmediakit.com/}{Advertise}
\item
  \href{http://www.tbrandstudio.com/}{T Brand Studio}
\item
  \href{https://www.nytimes.com/privacy/cookie-policy\#how-do-i-manage-trackers}{Your
  Ad Choices}
\item
  \href{https://www.nytimes.com/privacy}{Privacy}
\item
  \href{https://help.nytimes.com/hc/en-us/articles/115014893428-Terms-of-service}{Terms
  of Service}
\item
  \href{https://help.nytimes.com/hc/en-us/articles/115014893968-Terms-of-sale}{Terms
  of Sale}
\item
  \href{https://spiderbites.nytimes.com}{Site Map}
\item
  \href{https://help.nytimes.com/hc/en-us}{Help}
\item
  \href{https://www.nytimes.com/subscription?campaignId=37WXW}{Subscriptions}
\end{itemize}
