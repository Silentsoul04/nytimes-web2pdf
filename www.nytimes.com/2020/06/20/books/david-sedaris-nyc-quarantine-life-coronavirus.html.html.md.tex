Sections

SEARCH

\protect\hyperlink{site-content}{Skip to
content}\protect\hyperlink{site-index}{Skip to site index}

\href{https://www.nytimes.com/section/books}{Books}

\href{https://myaccount.nytimes.com/auth/login?response_type=cookie\&client_id=vi}{}

\href{https://www.nytimes.com/section/todayspaper}{Today's Paper}

\href{/section/books}{Books}\textbar{}David Sedaris, Dressed Up With
Nowhere to Go

\url{https://nyti.ms/3hNe4Kv}

\begin{itemize}
\item
\item
\item
\item
\item
\end{itemize}

\href{https://www.nytimes.com/news-event/coronavirus?action=click\&pgtype=Article\&state=default\&region=TOP_BANNER\&context=storylines_menu}{The
Coronavirus Outbreak}

\begin{itemize}
\tightlist
\item
  live\href{https://www.nytimes.com/2020/08/01/world/coronavirus-covid-19.html?action=click\&pgtype=Article\&state=default\&region=TOP_BANNER\&context=storylines_menu}{Latest
  Updates}
\item
  \href{https://www.nytimes.com/interactive/2020/us/coronavirus-us-cases.html?action=click\&pgtype=Article\&state=default\&region=TOP_BANNER\&context=storylines_menu}{Maps
  and Cases}
\item
  \href{https://www.nytimes.com/interactive/2020/science/coronavirus-vaccine-tracker.html?action=click\&pgtype=Article\&state=default\&region=TOP_BANNER\&context=storylines_menu}{Vaccine
  Tracker}
\item
  \href{https://www.nytimes.com/interactive/2020/07/29/us/schools-reopening-coronavirus.html?action=click\&pgtype=Article\&state=default\&region=TOP_BANNER\&context=storylines_menu}{What
  School May Look Like}
\item
  \href{https://www.nytimes.com/live/2020/07/31/business/stock-market-today-coronavirus?action=click\&pgtype=Article\&state=default\&region=TOP_BANNER\&context=storylines_menu}{Economy}
\end{itemize}

Advertisement

\protect\hyperlink{after-top}{Continue reading the main story}

Supported by

\protect\hyperlink{after-sponsor}{Continue reading the main story}

\hypertarget{david-sedaris-dressed-up-with-nowhere-to-go}{%
\section{David Sedaris, Dressed Up With Nowhere to
Go}\label{david-sedaris-dressed-up-with-nowhere-to-go}}

With two books in the works but all plans on hold, the writer is pacing
New York City and destroying his Fitbit friends.

\includegraphics{https://static01.nyt.com/images/2020/06/18/books/18Sedaris1/merlin_173453745_85f7633a-11d3-43f6-9e6e-4997702a110b-articleLarge.jpg?quality=75\&auto=webp\&disable=upscale}

\href{https://www.nytimes.com/by/sarah-lyall}{\includegraphics{https://static01.nyt.com/images/2018/02/20/multimedia/author-sarah-lyall/author-sarah-lyall-thumbLarge.jpg}}

By \href{https://www.nytimes.com/by/sarah-lyall}{Sarah Lyall}

\begin{itemize}
\item
  June 20, 2020
\item
  \begin{itemize}
  \item
  \item
  \item
  \item
  \item
  \end{itemize}
\end{itemize}

When New York went into lockdown, David Sedaris settled into his
apartment on the Upper East Side and canceled his 45-city book tour.

``I had bought all these outfits, and I was so looking forward to
wearing them,'' he said, mentioning with particular wistfulness a
lavishly ruffled black Comme de Garçons jacket --- ``a cross between
when Mammy was in mourning after the baby died in
`\href{https://www.nytimes.com/2020/06/14/movies/gone-with-the-wind-battle.html}{Gone
With the Wind},' and something that P.T. Barnum would wear'' --- now
hanging in his closet, an artifact from an alternative reality.

But Sedaris's realization that it's no fun dressing up in semi-satirical
garments when there is no one to see you is of course not the only thing
he has had to contend with. The author of 10 books of autobiographical
essays and short fictional pieces, Sedaris, 63, is a keen anatomist of
the skewed intricacies of human behavior, and there has been a lot of
behavior to sort through at the moment.

First, his own. He has two books coming out: ``The Best of Me,'' a
collection of his favorite essays, in the fall, and ``Carnival of
Snackeries,'' a second volume of diaries, tentatively scheduled for next
year. But his life, like everyone else's, is more or less on hold.

``I figured out early on that there's absolutely nothing I can do about
this,'' he said. ``That should be obvious, and for some reason it
wasn't. I kept thinking, `I should be able to fix this or control it.'
Whenever I feel sorry for myself, I think, `Everyone in the world is
going through this.' That makes it much easier.''

As he spoke, Sedaris sounded short of breath, a worrisome symptom in the
current climate. In fact, he said, it was because he has not let the
pandemic thwart his efforts to
\href{https://www.newyorker.com/magazine/2014/06/30/stepping-out-3}{rack
up miles on Fitbit}, the physical-activity-recording device.

``I'm walking in my apartment,'' he said into the phone. ``Right now.''

He considers it a competitive sport.

\hypertarget{latest-updates-global-coronavirus-outbreak}{%
\section{\texorpdfstring{\href{https://www.nytimes.com/2020/08/01/world/coronavirus-covid-19.html?action=click\&pgtype=Article\&state=default\&region=MAIN_CONTENT_1\&context=storylines_live_updates}{Latest
Updates: Global Coronavirus
Outbreak}}{Latest Updates: Global Coronavirus Outbreak}}\label{latest-updates-global-coronavirus-outbreak}}

Updated 2020-08-02T10:04:29.623Z

\begin{itemize}
\tightlist
\item
  \href{https://www.nytimes.com/2020/08/01/world/coronavirus-covid-19.html?action=click\&pgtype=Article\&state=default\&region=MAIN_CONTENT_1\&context=storylines_live_updates\#link-34047410}{The
  U.S. reels as July cases more than double the total of any other
  month.}
\item
  \href{https://www.nytimes.com/2020/08/01/world/coronavirus-covid-19.html?action=click\&pgtype=Article\&state=default\&region=MAIN_CONTENT_1\&context=storylines_live_updates\#link-780ec966}{Top
  U.S. officials work to break an impasse over the federal jobless
  benefit.}
\item
  \href{https://www.nytimes.com/2020/08/01/world/coronavirus-covid-19.html?action=click\&pgtype=Article\&state=default\&region=MAIN_CONTENT_1\&context=storylines_live_updates\#link-2bc8948}{Its
  outbreak untamed, Melbourne goes into even greater lockdown.}
\end{itemize}

\href{https://www.nytimes.com/2020/08/01/world/coronavirus-covid-19.html?action=click\&pgtype=Article\&state=default\&region=MAIN_CONTENT_1\&context=storylines_live_updates}{See
more updates}

More live coverage:
\href{https://www.nytimes.com/live/2020/07/31/business/stock-market-today-coronavirus?action=click\&pgtype=Article\&state=default\&region=MAIN_CONTENT_1\&context=storylines_live_updates}{Markets}

``I destroy everyone I'm a Fitbit friend of,'' Sedaris said. ``Like, I
might be walking 130 miles a week, and they're walking 30 miles a
week.'' But recently he has made a new Fitbit friend, someone whose
determination to see and raise him mile for mile has forced Sedaris to
increase his own efforts. Some days he walks nearly 20 miles.

At home, this involves pacing the floor like Gus, the neurotic polar
bear who compulsively trudged back and forth in his enclosure at the
Central Park Zoo. But throughout the pandemic Sedaris has also been
walking, masked, to the far ends of New York City.

\includegraphics{https://static01.nyt.com/images/2020/06/18/books/18Sedaris2-sub/merlin_173453634_eddba048-4dcb-4531-8a0a-1b703200c68f-articleLarge.jpg?quality=75\&auto=webp\&disable=upscale}

``The other week I walked all the way to Astoria,'' he said.
``Everywhere I go it smells the same, and it smells like my breath.'' He
generally has two outdoor shifts, the second after midnight, so that he
(or Fitbit) can apply those miles to the next day's tally.

``I like to start the next day with six miles under my belt,'' Sedaris
said. Although he is a compulsive collector of trash in the English
countryside, where he lives much of the time, he has resisted the
temptation to clean up the streets of New York. ``I'm not against it,''
he said, ``but everything changes once you start doing that --- you
can't stop.''

These excursions have showed him the city at its best. He is constantly
amazed, he said, at the high caliber of New Yorkers' discourse.

``You'll be in the park, and suddenly you'll hear some very articulate
person talking about what a horrible person Donald Trump is,'' Sedaris
said. ``They're so articulate and thoughtful, and they're not
regurgitating what they've already heard. Usually people who come up
with that stuff are writing for newspapers, or they're on TV.''

He has also seen the city at its most vulnerable, its late-night streets
dotted with the homeless and destitute; and occasionally at its
weirdest.

``I was at Times Square at 1:30 in the morning and there was a guy in a
wheelchair who was pushing himself along and he said, `Look at that
clown,''' Sedaris related. ``I thought he was talking about me. But then
I followed his eyes and there was a clown, with purple hair and a red
nose.''

More recently, he has walked city streets crowded with people, finding
camaraderie and shared humanity in the Black Lives Matter protests.

``The people are kind and thoughtful --- always distributing snacks and
water,'' Sedaris said. ```Do you need sunblock? Hand sanitizer? It's
nice to be part of a group, and I like walking down the center of the
street. Over time I came to think of the marches the way I think of
buses and subways. `I'll just take this BLM down to 23rd,' I'd tell
myself. Later I'd maybe get a crosstown BLM to Second Avenue, then walk
home from there.''

Those who follow Sedaris's autobiographical writing, which has softened
and become more emotional and self-reflective in recent years, will
recall that the author and his father have long had a contentious
relationship. They made a kind of peace last year, Sedaris
\href{https://www.newyorker.com/magazine/2020/03/02/unbuttoned}{wrote in
March}, as his father lay dying in a hospice.

In a quintessentially Sedaris move, though, his father did not die. He
rallied, left the hospice and is now in an assisted-living facility, in
good health considering that he is 97 and a global pandemic is underway.

Image

``Whenever I feel sorry for myself, I think, `Everyone in the world is
going through this,''' Sedaris said. ``That makes it much
easier.''Credit...Vincent Tullo for The New York Times

``I'm pretty sure my father wants a crowd at his funeral,'' Sedaris
said, of his father's ability to hang on until crowded funerals are
possible again. ``In a lot of ways I feel fortunate to have had him. I
wouldn't have changed anything, because I needed somebody to sort of
push against.''

\href{https://www.nytimes.com/news-event/coronavirus?action=click\&pgtype=Article\&state=default\&region=MAIN_CONTENT_3\&context=storylines_faq}{}

\hypertarget{the-coronavirus-outbreak-}{%
\subsubsection{The Coronavirus Outbreak
›}\label{the-coronavirus-outbreak-}}

\hypertarget{frequently-asked-questions}{%
\paragraph{Frequently Asked
Questions}\label{frequently-asked-questions}}

Updated July 27, 2020

\begin{itemize}
\item ~
  \hypertarget{should-i-refinance-my-mortgage}{%
  \paragraph{Should I refinance my
  mortgage?}\label{should-i-refinance-my-mortgage}}

  \begin{itemize}
  \tightlist
  \item
    \href{https://www.nytimes.com/article/coronavirus-money-unemployment.html?action=click\&pgtype=Article\&state=default\&region=MAIN_CONTENT_3\&context=storylines_faq}{It
    could be a good idea,} because mortgage rates have
    \href{https://www.nytimes.com/2020/07/16/business/mortgage-rates-below-3-percent.html?action=click\&pgtype=Article\&state=default\&region=MAIN_CONTENT_3\&context=storylines_faq}{never
    been lower.} Refinancing requests have pushed mortgage applications
    to some of the highest levels since 2008, so be prepared to get in
    line. But defaults are also up, so if you're thinking about buying a
    home, be aware that some lenders have tightened their standards.
  \end{itemize}
\item ~
  \hypertarget{what-is-school-going-to-look-like-in-september}{%
  \paragraph{What is school going to look like in
  September?}\label{what-is-school-going-to-look-like-in-september}}

  \begin{itemize}
  \tightlist
  \item
    It is unlikely that many schools will return to a normal schedule
    this fall, requiring the grind of
    \href{https://www.nytimes.com/2020/06/05/us/coronavirus-education-lost-learning.html?action=click\&pgtype=Article\&state=default\&region=MAIN_CONTENT_3\&context=storylines_faq}{online
    learning},
    \href{https://www.nytimes.com/2020/05/29/us/coronavirus-child-care-centers.html?action=click\&pgtype=Article\&state=default\&region=MAIN_CONTENT_3\&context=storylines_faq}{makeshift
    child care} and
    \href{https://www.nytimes.com/2020/06/03/business/economy/coronavirus-working-women.html?action=click\&pgtype=Article\&state=default\&region=MAIN_CONTENT_3\&context=storylines_faq}{stunted
    workdays} to continue. California's two largest public school
    districts --- Los Angeles and San Diego --- said on July 13, that
    \href{https://www.nytimes.com/2020/07/13/us/lausd-san-diego-school-reopening.html?action=click\&pgtype=Article\&state=default\&region=MAIN_CONTENT_3\&context=storylines_faq}{instruction
    will be remote-only in the fall}, citing concerns that surging
    coronavirus infections in their areas pose too dire a risk for
    students and teachers. Together, the two districts enroll some
    825,000 students. They are the largest in the country so far to
    abandon plans for even a partial physical return to classrooms when
    they reopen in August. For other districts, the solution won't be an
    all-or-nothing approach.
    \href{https://bioethics.jhu.edu/research-and-outreach/projects/eschool-initiative/school-policy-tracker/}{Many
    systems}, including the nation's largest, New York City, are
    devising
    \href{https://www.nytimes.com/2020/06/26/us/coronavirus-schools-reopen-fall.html?action=click\&pgtype=Article\&state=default\&region=MAIN_CONTENT_3\&context=storylines_faq}{hybrid
    plans} that involve spending some days in classrooms and other days
    online. There's no national policy on this yet, so check with your
    municipal school system regularly to see what is happening in your
    community.
  \end{itemize}
\item ~
  \hypertarget{is-the-coronavirus-airborne}{%
  \paragraph{Is the coronavirus
  airborne?}\label{is-the-coronavirus-airborne}}

  \begin{itemize}
  \tightlist
  \item
    The coronavirus
    \href{https://www.nytimes.com/2020/07/04/health/239-experts-with-one-big-claim-the-coronavirus-is-airborne.html?action=click\&pgtype=Article\&state=default\&region=MAIN_CONTENT_3\&context=storylines_faq}{can
    stay aloft for hours in tiny droplets in stagnant air}, infecting
    people as they inhale, mounting scientific evidence suggests. This
    risk is highest in crowded indoor spaces with poor ventilation, and
    may help explain super-spreading events reported in meatpacking
    plants, churches and restaurants.
    \href{https://www.nytimes.com/2020/07/06/health/coronavirus-airborne-aerosols.html?action=click\&pgtype=Article\&state=default\&region=MAIN_CONTENT_3\&context=storylines_faq}{It's
    unclear how often the virus is spread} via these tiny droplets, or
    aerosols, compared with larger droplets that are expelled when a
    sick person coughs or sneezes, or transmitted through contact with
    contaminated surfaces, said Linsey Marr, an aerosol expert at
    Virginia Tech. Aerosols are released even when a person without
    symptoms exhales, talks or sings, according to Dr. Marr and more
    than 200 other experts, who
    \href{https://academic.oup.com/cid/article/doi/10.1093/cid/ciaa939/5867798}{have
    outlined the evidence in an open letter to the World Health
    Organization}.
  \end{itemize}
\item ~
  \hypertarget{what-are-the-symptoms-of-coronavirus}{%
  \paragraph{What are the symptoms of
  coronavirus?}\label{what-are-the-symptoms-of-coronavirus}}

  \begin{itemize}
  \tightlist
  \item
    Common symptoms
    \href{https://www.nytimes.com/article/symptoms-coronavirus.html?action=click\&pgtype=Article\&state=default\&region=MAIN_CONTENT_3\&context=storylines_faq}{include
    fever, a dry cough, fatigue and difficulty breathing or shortness of
    breath.} Some of these symptoms overlap with those of the flu,
    making detection difficult, but runny noses and stuffy sinuses are
    less common.
    \href{https://www.nytimes.com/2020/04/27/health/coronavirus-symptoms-cdc.html?action=click\&pgtype=Article\&state=default\&region=MAIN_CONTENT_3\&context=storylines_faq}{The
    C.D.C. has also} added chills, muscle pain, sore throat, headache
    and a new loss of the sense of taste or smell as symptoms to look
    out for. Most people fall ill five to seven days after exposure, but
    symptoms may appear in as few as two days or as many as 14 days.
  \end{itemize}
\item ~
  \hypertarget{does-asymptomatic-transmission-of-covid-19-happen}{%
  \paragraph{Does asymptomatic transmission of Covid-19
  happen?}\label{does-asymptomatic-transmission-of-covid-19-happen}}

  \begin{itemize}
  \tightlist
  \item
    So far, the evidence seems to show it does. A widely cited
    \href{https://www.nature.com/articles/s41591-020-0869-5}{paper}
    published in April suggests that people are most infectious about
    two days before the onset of coronavirus symptoms and estimated that
    44 percent of new infections were a result of transmission from
    people who were not yet showing symptoms. Recently, a top expert at
    the World Health Organization stated that transmission of the
    coronavirus by people who did not have symptoms was ``very rare,''
    \href{https://www.nytimes.com/2020/06/09/world/coronavirus-updates.html?action=click\&pgtype=Article\&state=default\&region=MAIN_CONTENT_3\&context=storylines_faq\#link-1f302e21}{but
    she later walked back that statement.}
  \end{itemize}
\end{itemize}

Many authors have taken this opportunity to connect to audiences
virtually. But don't look for Sedaris online anytime soon.

``My goal is to get through this without ever going on Zoom or FaceTime
or Skype,'' he said. ``People are like, Can you record a message of hope
for all the people who were going to come to your show?' and I'm like,
`No, because it's not like there aren't things to watch already.'''

Sedaris himself subscribed to Netflix in January. ``I was the last
person on Earth to get it,'' he said. ``Literally the last person. I
thought we'd spend a lot of time watching things, but Hugh'' --- that
would be his boyfriend, Hugh Hamrick, an artist and a familiar character
in the Sedaris oeuvre --- ``falls asleep, so you can't watch anything
with him.''

In normal times, Sedaris travels so frequently that the two are rarely
in one place together for long.

``For the past 20 years I've been gone every fall and every spring, and
people said, `It must be horrible to be away from Hugh for so long,' and
I've always thought, `No, it's actually kind of great,''' Sedaris said.
``You've been with someone for 30 years, and it's great not to see them
for a few months.''

But lockdown a deux has been a revelation.

``The thing is,'' Sedaris added, ``I mean, I've talked to people who
said, `We've been home trapped together and we're at each other's
throats.' But in our case, we've never gotten along better. How am I
supposed to write about that? I said to him the other day, `I hope you
die of coronavirus, so I can write about it.'''

(He was kidding. In any case, both he and Hamrick fell ill with and then
recovered from Covid-like symptoms early in the spring, though they have
not been tested for the virus.)

``It's been fantastic, it really has,'' Sedaris went on, in an
unexpected burst of straight-up emotional enthusiasm. ``I was really
afraid he'd get tired of me. Like this morning, I got up at 10 and at
10:30 Hugh said to me, `I'm tired of you already.' So I said, `OK, can
we start over?' And we just started the day again.''

\emph{Follow New York Times Books on}
\href{https://www.facebook.com/nytbooks/}{\emph{Facebook}}\emph{,}
\href{https://twitter.com/nytimesbooks}{\emph{Twitter}} \emph{and}
\href{https://www.instagram.com/nytbooks/}{\emph{Instagram}}\emph{, sign
up for}
\href{https://www.nytimes.com/newsletters/books-review}{\emph{our
newsletter}} \emph{or}
\href{https://www.nytimes.com/interactive/2017/books/books-calendar.html}{\emph{our
literary calendar}}\emph{. And listen to us on the}
\href{https://www.nytimes.com/column/book-review-podcast}{\emph{Book
Review podcast}}\emph{.}

Advertisement

\protect\hyperlink{after-bottom}{Continue reading the main story}

\hypertarget{site-index}{%
\subsection{Site Index}\label{site-index}}

\hypertarget{site-information-navigation}{%
\subsection{Site Information
Navigation}\label{site-information-navigation}}

\begin{itemize}
\tightlist
\item
  \href{https://help.nytimes.com/hc/en-us/articles/115014792127-Copyright-notice}{©~2020~The
  New York Times Company}
\end{itemize}

\begin{itemize}
\tightlist
\item
  \href{https://www.nytco.com/}{NYTCo}
\item
  \href{https://help.nytimes.com/hc/en-us/articles/115015385887-Contact-Us}{Contact
  Us}
\item
  \href{https://www.nytco.com/careers/}{Work with us}
\item
  \href{https://nytmediakit.com/}{Advertise}
\item
  \href{http://www.tbrandstudio.com/}{T Brand Studio}
\item
  \href{https://www.nytimes.com/privacy/cookie-policy\#how-do-i-manage-trackers}{Your
  Ad Choices}
\item
  \href{https://www.nytimes.com/privacy}{Privacy}
\item
  \href{https://help.nytimes.com/hc/en-us/articles/115014893428-Terms-of-service}{Terms
  of Service}
\item
  \href{https://help.nytimes.com/hc/en-us/articles/115014893968-Terms-of-sale}{Terms
  of Sale}
\item
  \href{https://spiderbites.nytimes.com}{Site Map}
\item
  \href{https://help.nytimes.com/hc/en-us}{Help}
\item
  \href{https://www.nytimes.com/subscription?campaignId=37WXW}{Subscriptions}
\end{itemize}
