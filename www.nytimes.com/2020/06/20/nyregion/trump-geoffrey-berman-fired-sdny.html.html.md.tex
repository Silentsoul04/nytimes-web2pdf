Sections

SEARCH

\protect\hyperlink{site-content}{Skip to
content}\protect\hyperlink{site-index}{Skip to site index}

\href{https://www.nytimes.com/section/nyregion}{New York}

\href{https://myaccount.nytimes.com/auth/login?response_type=cookie\&client_id=vi}{}

\href{https://www.nytimes.com/section/todayspaper}{Today's Paper}

\href{/section/nyregion}{New York}\textbar{}Trump Fires U.S. Attorney in
New York Who Investigated His Inner Circle

\url{https://nyti.ms/2CkSTis}

\begin{itemize}
\item
\item
\item
\item
\item
\item
\end{itemize}

Advertisement

\protect\hyperlink{after-top}{Continue reading the main story}

Supported by

\protect\hyperlink{after-sponsor}{Continue reading the main story}

\hypertarget{trump-fires-us-attorney-in-new-york-who-investigated-his-inner-circle}{%
\section{Trump Fires U.S. Attorney in New York Who Investigated His
Inner
Circle}\label{trump-fires-us-attorney-in-new-york-who-investigated-his-inner-circle}}

The president's move heightened criticism that he was purging his
administration of officials whose independence could be a threat to his
re-election.

\includegraphics{https://static01.nyt.com/images/2020/06/20/nyregion/20nyberman-1/merlin_173742474_9c6aa14f-e895-46b2-882f-a60e547cb7fb-articleLarge.jpg?quality=75\&auto=webp\&disable=upscale}

By \href{https://www.nytimes.com/by/alan-feuer}{Alan Feuer},
\href{https://www.nytimes.com/by/katie-benner}{Katie Benner},
\href{https://www.nytimes.com/by/ben-protess}{Ben Protess},
\href{https://www.nytimes.com/by/maggie-haberman}{Maggie Haberman},
\href{https://www.nytimes.com/by/william-k-rashbaum}{William K.
Rashbaum}, \href{https://www.nytimes.com/by/nicole-hong}{Nicole Hong}
and \href{https://www.nytimes.com/by/benjamin-weiser}{Benjamin Weiser}

\begin{itemize}
\item
  June 20, 2020
\item
  \begin{itemize}
  \item
  \item
  \item
  \item
  \item
  \item
  \end{itemize}
\end{itemize}

President Trump on Saturday fired the federal prosecutor whose office
put his former personal lawyer in prison and is investigating his
current one, heightening criticism that the president was carrying out
an extraordinary purge to rid his administration of officials whose
independence could be a threat to his re-election campaign.

Mr. Trump's dismissal of the prosecutor,
\href{https://www.nytimes.com/2020/07/09/us/politics/top-manhattan-prosecutor-ousted-by-trump-details-firing.html}{Geoffrey
S. Berman}, the United States attorney in Manhattan, whose office has
pursued one case after another that have rankled Mr. Trump, led to
political blowback and an unexpected result: By the end of the day, Mr.
Berman's handpicked deputy, not the administration's favored
replacement, was chosen to succeed him for now.

The abrupt ouster of Mr. Berman came as Mr. Trump sought to reinvigorate
his campaign with its first public rally in months and days after new
allegations by his former national security adviser that he had engaged
in ``obstruction of justice as a way of life.''

It was the latest move in a broader purge of administration officials
that has intensified in the months since the Republican-led Senate
acquitted Mr. Trump at an impeachment trial.

Since the beginning of the year, the president has fired or forced out
inspectors general with independent oversight over executive branch
agencies and other key figures from the trial.

Mr. Berman, who has been in office since 2018,
\href{https://www.nytimes.com/2020/06/19/nyregion/us-attorney-manhattan-trump.html}{had
declined to leave his post after Attorney General William P. Barr
announced late on Friday night} that Mr. Berman would be replaced by Jay
Clayton, the chairman of the Securities and Exchange Commission.

Mr. Clayton is friendly with Mr. Trump and had golfed with the president
at his club in Bedminster, N.J., as recently as last weekend, according
to two people familiar with the matter.

But on Saturday, facing a standoff with Mr. Berman, Mr. Barr shifted
course. In a letter released by the Justice Department, Mr. Barr told
Mr. Berman that Mr. Trump had fired him and that he would be replaced
temporarily with the prosecutor's own chief deputy, Audrey Strauss.

The choice of Ms. Strauss appeared to mollify Mr. Berman, who then
issued a statement saying he would step down in light of the reversal.

In the statement, Mr. Berman said that under Ms. Strauss, the Southern
District of New York, as the prosecutors' office in Manhattan is
formally known, ``will continue to safeguard'' its ``enduring tradition
of integrity and independence.''

The swirl of events on Saturday, which changed by the hour, was the
culmination of longstanding tensions between the White House and Mr.
Berman's office, which in the past three years has brought a series of
highly sensitive cases that have troubled and angered Mr. Trump and
others in his inner circle.

First, there was the
\href{https://www.nytimes.com/2018/08/21/nyregion/michael-cohen-plea-deal-trump.html}{arrest
and prosecution in 2018 of Michael D. Cohen}, Mr. Trump's longtime legal
fixer. Then, there was
\href{https://www.nytimes.com/2019/10/16/us/politics/halkbank-trump-turkey.html}{the
indictment last year of a state-owned bank in Turkey} with political
connections that had drawn the president's attention. More recently, Mr.
Berman began an
\href{https://www.nytimes.com/2019/10/11/us/politics/rudy-giuliani-investigation.html}{inquiry
into Rudolph W. Giuliani, Mr. Trump's personal lawyer} and one of his
most ardent supporters.

Speaking briefly to reporters outside the White House before heading to
a campaign rally in Tulsa, Okla., Mr. Trump tried to distance himself
from the firing. He insisted he was ``not involved'' in the decision to
remove Mr. Berman despite what Mr. Barr said in his letter.

Mr. Clayton had recently signaled to his friends and the president that
he wanted to return to home in New York City and was interested in Mr.
Berman's job, according to people familiar with the matter. Mr. Barr had
said New Jersey's top federal prosecutor, Craig Carpenito, would hold
the seat until the Senate could confirm Mr. Clayton.

By Saturday afternoon, the plan began to unravel, as the president and
his senior aides scrambled to secure support for Mr. Clayton's
confirmation in the Senate, according to people familiar with the
events.

The refusal of Republicans to defend Mr. Trump was palpable, and some
people close to the president expressed concern that lawmakers in his
own party would feel compelled to distance themselves from Mr. Trump's
decision.

The most prominent critic of the move was Senator Lindsey Graham,
Republican of South Carolina and a close ally of the president's.

Mr. Graham, chairman of the Senate Judiciary Committee, suggested in a
statement that he would allow New York's two Democratic senators to
thwart the nomination through a procedural maneuver. He complimented Mr.
Clayton but noted that he had not heard from the administration about
formal plans to name him.

Given the number of sore spots between Mr. Trump's Justice Department
and the Southern District, its most prominent prosecutors' office, it
was not clear what prompted Mr. Trump and Mr. Barr to fire Mr. Berman.

At least two of the politically sensitive investigations that the New
York prosecutors have pursued --- those involving the Turkish bank and
Mr. Giuliani --- are continuing.

Throughout the day on Saturday, many current and former employees of the
Southern District marveled at just how sour relations with their
colleagues in Washington had gotten. Some worried openly that the move
threatened the independence of federal prosecutors.

\includegraphics{https://static01.nyt.com/images/2020/06/20/nyregion/20nyberman-3/merlin_173591163_60d3f989-ca0f-4119-a895-c287fd841ce9-articleLarge.jpg?quality=75\&auto=webp\&disable=upscale}

``While there have always been turf battles between the Southern
District and the Justice Department in Washington, and occasionally
sharp elbows, to take someone out suddenly while they're investigating
the president's lawyer, it is just unprecedented in modern times,'' said
David Massey, a defense lawyer, who served as a Southern District
prosecutor for nearly a decade.

The decision to remove Mr. Berman unfolded with particularly dizzying
speed and seemed to take even several of the participants aback.

On Friday, Mr. Barr came to New York to meet with senior New York Police
Department officials and, after nearly a month of public protests, to
talk with them about ``policing issues that have been at the forefront
of national conversation and debate,'' according to a Justice Department
news release.

When he later met with Mr. Berman, according to two people familiar with
the conversation, Mr. Barr suggested that Mr. Berman could take over
\href{https://www.nytimes.com/2020/06/16/us/politics/justice-department-jody-hunt.html}{the
civil division of the Justice Department} or become chairman of the
S.E.C. if he agreed to leave his position in Manhattan.

But Mr. Berman declined, and Mr. Barr quickly moved to fire him,
announcing his decision in a highly unusual late-night
\href{https://www.justice.gov/opa/pr/attorney-general-william-p-barr-nomination-jay-clayton-serve-us-attorney-southern-district}{Justice
Department news release}. Hours later, Mr. Berman issued a
counterstatement denying he was leaving.

``I have not resigned, and have no intention of resigning, my
position,'' Mr. Berman's statement said. He added that he had learned of
Mr. Barr's actions only from the news release.

On Saturday, the pressure reached a breaking point. Mr. Barr told Mr.
Berman in his letter that he had persuaded Mr. Trump to fire Mr. Berman
because he had chosen ``public spectacle over public service'' by not
voluntarily quitting the day before.

``Because you have declared that you have no intention of resigning, I
have asked the president to remove you as of today, and he has done
so,'' the letter read.

In one sign that Mr. Barr's move to oust Mr. Berman may have been
hastily arranged, even Mr. Clayton, the man who had been poised to take
Mr. Berman's place, appeared to be caught off guard.

Mr. Clayton had sent an email to his staff on Thursday saying that he
looked forward to seeing them in person, once work-at-home restrictions
that had been put in place because of the coronavirus could be lifted.
The email offered no indication that Mr. Clayton was planning to leave
the S.E.C., according to a person briefed on it.

Just after midnight on Saturday, Mr. Clayton sent another email to his
employees, telling them about his new position. ``Pending
confirmation,'' he wrote, ``I will remain fully committed to the work of
the commission and the supportive community we have built,'' according
to a copy reviewed by The New York Times.

Mr. Clayton could not be reached for comment.

On Saturday, Representative Jerrold Nadler, a New York Democrat who
heads the House Judiciary Committee, said the committee would
investigate the firing of Mr. Berman as part of a larger inquiry into
what he said was undue political interference at the Justice Department.

``The whole thing smacks of corruption and incompetence,'' Mr. Nadler
said of Mr. Berman's dismissal.

Under Mr. Trump, the Justice Department has long believed that the
Southern District was out of control. In no small part that was because
the department believed that prosecutors in New York delayed in warning
them that they were naming Mr. Trump --- as ``Individual-1'' --- in
court documents in the Cohen prosecution.

When Mr. Barr became attorney general, officials in the deputy attorney
general's office, which oversees regional prosecutors, asked him to rein
in Mr. Berman, who they believed was exacerbating the Southern
District's propensity for autonomy. The office has embraced its nickname
the ``Sovereign District'' of New York because of its tradition of
independence.

One particular point of contention was the question of how Mr. Berman
and his staff should investigate Halkbank, a Turkish state-owned bank
that the office indicted last year, according to three people familiar
with the investigation.

In a new book, John Bolton, Mr. Trump's former national security
adviser, wrote that Mr. Trump had promised the Turkish president, Recep
Tayyip Erdogan, in 2018 that he would intervene in the investigation of
the bank, which had been accused of violating sanctions against Iran.

Then there was the inquiry into Mr. Giuliani, which has focused on
whether he violated laws on lobbying for foreign entities in his efforts
to dig up dirt in Ukraine on the president's political rivals. That
investigation began after Mr. Berman's office
\href{https://www.nytimes.com/2019/10/10/us/politics/lev-parnas-igor-fruman-arrested-giuliani.html}{brought
indictments against two of Mr. Giuliani's close associates}.

Mr. Trump has told advisers he was pleased with the move to dismiss Mr.
Berman, and a person close to the president described it as a long time
coming.

Mr. Trump has been dissatisfied with Mr. Berman, despite choosing him
for the post himself, going back to 2018. That year, he told the acting
attorney general at the time, Matthew G. Whitaker, that he was
frustrated that
\href{https://www.nytimes.com/2019/02/19/us/politics/trump-investigations.html}{Mr.
Berman had been recused from the case against Mr. Cohen} and wanted him
to somehow undo it.

A Republican who contributed to the president's campaign and worked at
the same law firm as Mr. Giuliani, Mr. Berman had suggested that the
Justice Department cannot fire him because of the way he came into his
job.

Image

Manhattan prosecutors have been investigating Rudolph W. Giuliani,
center, President Trump's personal lawyer and one of his strongest
supporters.Credit...Anna Moneymaker/The New York Times

In 2018, the attorney general at the time, Jeff Sessions, appointed Mr.
Berman as interim United States attorney in Manhattan. But Mr. Trump
never formally sent Mr. Berman's nomination to the Senate, as is normal
protocol. After 120 days, Mr. Berman's official appointment to the post
was made by the judges of the United States District Court.

\href{https://www.nytimes.com/2020/06/20/us/politics/geoff-berman-who-can-fire.html}{Mr.
Berman suggested that only those judges could dismiss him from his
position, although that was far from a settled legal matter.} A 1979
Justice Department memo holds the position that the president could fire
a prosecutor in Mr. Berman's position.

Last year, Mr. Barr considered replacing Mr. Berman with Edward
O'Callaghan, a top Justice Department official and a former Southern
District prosecutor, according to people familiar with the matter. The
plan fell through, however, in part because of the complex legal issues
around how Mr. Berman was appointed.

Ms. Strauss, who currently serves as Mr. Berman's deputy, was one of his
early hires after he became U.S. attorney in 2018. She had previously
worked in the office as a rank-and-file prosecutor and later spent many
years as a lawyer in private practice.

As deputy, Ms. Strauss had also assumed responsibility for several of
the prominent investigations from which Mr. Berman had recused himself.
Among them was the prosecution of Mr. Cohen.

Deborah Solomon contributed reporting.

Advertisement

\protect\hyperlink{after-bottom}{Continue reading the main story}

\hypertarget{site-index}{%
\subsection{Site Index}\label{site-index}}

\hypertarget{site-information-navigation}{%
\subsection{Site Information
Navigation}\label{site-information-navigation}}

\begin{itemize}
\tightlist
\item
  \href{https://help.nytimes.com/hc/en-us/articles/115014792127-Copyright-notice}{©~2020~The
  New York Times Company}
\end{itemize}

\begin{itemize}
\tightlist
\item
  \href{https://www.nytco.com/}{NYTCo}
\item
  \href{https://help.nytimes.com/hc/en-us/articles/115015385887-Contact-Us}{Contact
  Us}
\item
  \href{https://www.nytco.com/careers/}{Work with us}
\item
  \href{https://nytmediakit.com/}{Advertise}
\item
  \href{http://www.tbrandstudio.com/}{T Brand Studio}
\item
  \href{https://www.nytimes.com/privacy/cookie-policy\#how-do-i-manage-trackers}{Your
  Ad Choices}
\item
  \href{https://www.nytimes.com/privacy}{Privacy}
\item
  \href{https://help.nytimes.com/hc/en-us/articles/115014893428-Terms-of-service}{Terms
  of Service}
\item
  \href{https://help.nytimes.com/hc/en-us/articles/115014893968-Terms-of-sale}{Terms
  of Sale}
\item
  \href{https://spiderbites.nytimes.com}{Site Map}
\item
  \href{https://help.nytimes.com/hc/en-us}{Help}
\item
  \href{https://www.nytimes.com/subscription?campaignId=37WXW}{Subscriptions}
\end{itemize}
