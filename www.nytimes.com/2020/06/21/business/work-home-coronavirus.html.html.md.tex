Sections

SEARCH

\protect\hyperlink{site-content}{Skip to
content}\protect\hyperlink{site-index}{Skip to site index}

\href{https://www.nytimes.com/section/business}{Business}

\href{https://myaccount.nytimes.com/auth/login?response_type=cookie\&client_id=vi}{}

\href{https://www.nytimes.com/section/todayspaper}{Today's Paper}

\href{/section/business}{Business}\textbar{}How to Succeed in Your
Office Job When There Is No Office

\url{https://nyti.ms/3hQqpxe}

\begin{itemize}
\item
\item
\item
\item
\item
\end{itemize}

\href{https://www.nytimes.com/spotlight/at-home?action=click\&pgtype=Article\&state=default\&region=TOP_BANNER\&context=at_home_menu}{At
Home}

\begin{itemize}
\tightlist
\item
  \href{https://www.nytimes.com/2020/08/03/well/family/the-benefits-of-talking-to-strangers.html?action=click\&pgtype=Article\&state=default\&region=TOP_BANNER\&context=at_home_menu}{Talk:
  To Strangers}
\item
  \href{https://www.nytimes.com/2020/08/01/at-home/coronavirus-make-pizza-on-a-grill.html?action=click\&pgtype=Article\&state=default\&region=TOP_BANNER\&context=at_home_menu}{Make:
  Grilled Pizza}
\item
  \href{https://www.nytimes.com/2020/07/31/arts/television/goldbergs-abc-stream.html?action=click\&pgtype=Article\&state=default\&region=TOP_BANNER\&context=at_home_menu}{Watch:
  'The Goldbergs'}
\item
  \href{https://www.nytimes.com/interactive/2020/at-home/even-more-reporters-editors-diaries-lists-recommendations.html?action=click\&pgtype=Article\&state=default\&region=TOP_BANNER\&context=at_home_menu}{Explore:
  Reporters' Google Docs}
\end{itemize}

Advertisement

\protect\hyperlink{after-top}{Continue reading the main story}

Supported by

\protect\hyperlink{after-sponsor}{Continue reading the main story}

\hypertarget{how-to-succeed-in-your-office-job-when-there-is-no-office}{%
\section{How to Succeed in Your Office Job When There Is No
Office}\label{how-to-succeed-in-your-office-job-when-there-is-no-office}}

The coronavirus has many of us trying to be productive at home. Here are
some tips for all of us, including our bosses.

\includegraphics{https://static01.nyt.com/images/2020/06/22/business/22wfh-success/22wfh-success-articleLarge.jpg?quality=75\&auto=webp\&disable=upscale}

By Julie Weed

\begin{itemize}
\item
  June 21, 2020
\item
  \begin{itemize}
  \item
  \item
  \item
  \item
  \item
  \end{itemize}
\end{itemize}

The challenges of working from home are myriad. There are children to
tutor, dogs to walk, shows to binge on. Hallway hellos and brainstorming
at the whiteboard have given way to the stilted cadence of Zoom
meetings. But three months into the work-from-home era, some best
practices are emerging.

\hypertarget{for-everyone}{%
\subsubsection{\texorpdfstring{\textbf{For
Everyone}}{For Everyone}}\label{for-everyone}}

\begin{itemize}
\item
  \textbf{Shift your mind-set.} More than ever you will be measured on
  output, not how many hours you sat at your desk. ``It's a different
  way to approach work'' and translates to more freedom to design your
  day, said Ann Herrmann-Nehdi, the chief executive of
  \href{https://www.thinkherrmann.com/}{Herrmann}, a multinational
  company that creates tools to help employees communicate better. While
  there are still unavoidable meetings, creating chunks of time to turn
  off notifications and focus deeply on your own projects, called
  ``time-boxing,'' can lift the quality of your output, she said.
\item
  \textbf{Take the initiative.} Don't expect your higher-ups to have it
  all figured out. Almost every aspect of work is being reconsidered, so
  jump in with suggestions, big or small. Even figuring out new Zoom or
  Teams features and giving a quick lesson can be useful. Ms.
  Herrmann-Nehdi said a colleague recently began creating infographics
  that better explained her work findings to her remote teammates than
  the usual slide deck.
\item
  \textbf{Speak up quickly if something isn't working}. ``Raise a flag
  if something looks off,'' said Lauren Kaplowitz, a customer success
  manager at Lively, a small company that helps people manage health
  savings accounts. The business is based in San Francisco but she is
  working from the Seattle area. Check in with others to see if they are
  experiencing the same problems, Ms. Kaplowitz said. It's harder now
  for managers to see that you are spinning your wheels and aren't
  making progress, so let them know.
\item
  \textbf{Re-create ``in person.''} If you do your best collaborating in
  the same room with a work partner, use technology and block a few
  hours to ``share a room,'' Ms. Herrmann-Nehdi suggested. That way, you
  and your teammate can see each other, view each other's screens and
  share a virtual white board for ideas.
\item
  \textbf{Find office allies.} Brainstorm, review work together before
  submitting it or just check in. Crossing paths in the break room is a
  thing of the past, so Ms. Kaplowitz schedules short ``coffee chats''
  via videoconference to catch up with colleagues on work, or just to
  talk.
\item
  \textbf{Remove distractions.} Without the boss periodically peeking
  over your shoulder, it's easy to take a quick break and realize an
  hour later you're still on that unending Twitter or Instagram scroll.
  Take social media off your work machine. Leave your phone in another
  room.
\item
  \textbf{Use what worked before}. Take home with you the best habits
  you formed at the office. Setting priorities and communicating, for
  example, are still essential to effective work. Iyobosa Bello-Asemota,
  an investment banking analyst at Morgan Stanley, creates financial
  analyses for a number of different teams. She makes sure to keep them
  all apprised of the tasks she is juggling and how she is organizing
  her time. ``It helps set expectations and is something I've always
  done,'' she said, but it's even more important now that the teams are
  physically separated.
\item
  \textbf{Don't forget career advancement.} Keep thinking and talking
  about the areas you want to improve, the parts of the company you want
  to explore and how you may get there. While it's not as easy as poking
  her head in an office, Ms. Bello-Asemota carves out time to connect
  with her mentor, a vice president at the firm, to get feedback on
  topics like work style and the level of responsibility she is being
  given.
\end{itemize}

\hypertarget{for-managers}{%
\subsubsection{\texorpdfstring{\textbf{For
Managers}}{For Managers}}\label{for-managers}}

\begin{itemize}
\item
  \textbf{Overcommunicate.} Provide additional context. Explain the
  ``whys'' of decisions and their possible effects to replace the
  information picked up organically in the office. Make sure to clarify
  goals, identify resources and explain where to find information, Ms.
  Herrmann-Nehdi said. If you can, share how the company is doing
  financially, what's going on with reorganizations, layoffs, raises and
  when staff can expect to be back in the office.
\item
  \textbf{Make consistency a priority}. Updates should come at
  predictable times and days. ``It's important to have a consistent
  cadence to communication,'' especially when so many other things are
  uncertain, said Tracey Armstrong, chief executive of Copyright
  Clearance Center, a 500-employee organization that helps companies
  license copyrighted materials. Managers should also regularly check
  communication channels like email, text, Slack and Teams to make sure
  they are not creating a roadblock.
\item
  \textbf{Meet differently.} In phone calls and videoconferences, take
  extra time to encourage questions and engage those who haven't chimed
  in, Ms. Herrmann-Nehdi suggested. Afterward, reiterate shared
  information, confirm understanding and distribute decisions, actions
  and key discussion points in writing. Frequent short meetings like a
  daily ``Stand Up,'' where team members each say what they are working
  on that day, resources they need or challenges they face, can keep
  them from moving in the wrong direction.
\item
  \textbf{Rotate responsibilities.} New ways of doing things offer new
  experiences**.** Brooke O'Berry, vice president for digital customer
  experiences at Starbucks, helps team members feel comfortable in
  videoconferences by rotating roles so everyone gets a chance to lead
  the meeting or act as a moderator, managing the questions and
  suggestions in the comment stream or chat window.
\item
  \textbf{Repurpose your teams}. Find new ways to meet old objectives.
  Ms. Armstrong's trade-show team, whose goals included increasing
  awareness of the company's services, now meets that objective by
  raising the company's presence on LinkedIn, working with employees to
  fill out their profiles and post content from the company.
\item
  \textbf{Keep Experimenting.} Jean-Claude Saghbini, chief technology
  officer at Wolters Kluwer Health, a global provider of health care
  information technology, said at the beginning of the pandemic his team
  was working ``beyond full capacity'' in creating ways to get things
  done to keep their systems updated for front-line health care workers.
  Now that the situation has changed from a ``sprint to a marathon,''
  Mr. Saghbini said, managers must keep checking in with employees to
  hear what's working and what isn't, and to keep making adjustments.
  Along with feedback on new processes and technologies put in place for
  remote work, ``managers need to ask employees about their individual
  constraints like child-care hours,'' he said, ``and design around each
  person's constraints.''
\item
  \textbf{Find ways to highlight your teams' great work} \textbf{with
  higher-ups}. Upper management may be physically out of sight, Ms.
  O'Berry said, but ``it's important for the team to still feel
  recognized.''
\item
  \textbf{Remember the extra stress.} Employees aren't just figuring out
  how to work from home. They are managing newly full households, and
  worrying about a potentially deadly virus as well as economic fallout
  and social justice issues. Part of understanding that is accepting
  messiness, and showing your team that you are not immune to it, Mr.
  Saghbini said. ``If your kids interrupt your video call, if you have
  to get off the phone because the plumber is there, if you just don't
  want to turn on your video that day, as a manager, do it, and show
  that's OK.''
\end{itemize}

Advertisement

\protect\hyperlink{after-bottom}{Continue reading the main story}

\hypertarget{site-index}{%
\subsection{Site Index}\label{site-index}}

\hypertarget{site-information-navigation}{%
\subsection{Site Information
Navigation}\label{site-information-navigation}}

\begin{itemize}
\tightlist
\item
  \href{https://help.nytimes.com/hc/en-us/articles/115014792127-Copyright-notice}{©~2020~The
  New York Times Company}
\end{itemize}

\begin{itemize}
\tightlist
\item
  \href{https://www.nytco.com/}{NYTCo}
\item
  \href{https://help.nytimes.com/hc/en-us/articles/115015385887-Contact-Us}{Contact
  Us}
\item
  \href{https://www.nytco.com/careers/}{Work with us}
\item
  \href{https://nytmediakit.com/}{Advertise}
\item
  \href{http://www.tbrandstudio.com/}{T Brand Studio}
\item
  \href{https://www.nytimes.com/privacy/cookie-policy\#how-do-i-manage-trackers}{Your
  Ad Choices}
\item
  \href{https://www.nytimes.com/privacy}{Privacy}
\item
  \href{https://help.nytimes.com/hc/en-us/articles/115014893428-Terms-of-service}{Terms
  of Service}
\item
  \href{https://help.nytimes.com/hc/en-us/articles/115014893968-Terms-of-sale}{Terms
  of Sale}
\item
  \href{https://spiderbites.nytimes.com}{Site Map}
\item
  \href{https://help.nytimes.com/hc/en-us}{Help}
\item
  \href{https://www.nytimes.com/subscription?campaignId=37WXW}{Subscriptions}
\end{itemize}
