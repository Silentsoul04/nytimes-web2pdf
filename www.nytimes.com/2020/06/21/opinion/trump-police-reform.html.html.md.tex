Sections

SEARCH

\protect\hyperlink{site-content}{Skip to
content}\protect\hyperlink{site-index}{Skip to site index}

\href{https://myaccount.nytimes.com/auth/login?response_type=cookie\&client_id=vi}{}

\href{https://www.nytimes.com/section/todayspaper}{Today's Paper}

\href{/section/opinion}{Opinion}\textbar{}`Law and Order' for `Blacks
and Hippies'

\href{https://nyti.ms/2YjtYo7}{https://nyti.ms/2YjtYo7}

\begin{itemize}
\item
\item
\item
\item
\item
\item
\end{itemize}

Advertisement

\protect\hyperlink{after-top}{Continue reading the main story}

\href{/section/opinion}{Opinion}

Supported by

\protect\hyperlink{after-sponsor}{Continue reading the main story}

\hypertarget{law-and-order-for-blacks-and-hippies}{%
\section{`Law and Order' for `Blacks and
Hippies'}\label{law-and-order-for-blacks-and-hippies}}

Trump's tough talk doesn't seek to address the rage that inequity has
bred, but rather to contain it.

\href{https://www.nytimes.com/by/charles-m-blow}{\includegraphics{https://static01.nyt.com/images/2018/04/02/opinion/charles-m-blow/charles-m-blow-thumbLarge.png}}

By \href{https://www.nytimes.com/by/charles-m-blow}{Charles M. Blow}

Opinion Columnist

\begin{itemize}
\item
  June 21, 2020
\item
  \begin{itemize}
  \item
  \item
  \item
  \item
  \item
  \item
  \end{itemize}
\end{itemize}

\includegraphics{https://static01.nyt.com/images/2020/06/21/opinion/21blow/merlin_173763309_5d03fb71-9dc2-4700-8825-ee4c78597673-articleLarge.jpg?quality=75\&auto=webp\&disable=upscale}

Last week, Donald Trump stood in the White House Rose Garden and
announced an executive order on police reform --- a list of minor,
unfunded actions that incentivized some changes but mandated none.

This was his response to the anti-racism, anti-police brutality Black
Lives Matter protests sweeping the country. I don't think it was an
action he wanted to take, but one that he had to take at this moment
when his poll numbers are dipping and people are demanding change.

Not once in his speech did he say the words ``protests'' or
``protesters.''

Instead, it was a whiplash speech that swung from acknowledging the pain
of families who've lost loved ones to police violence and promising ``to
fight for justice for all of our people,'' to more law-and-order talk
and condemnation of riots, looting and arson.

Those lawless acts occurred in some cities in the beginning, but the
protests have moved well beyond that now.

Trump knows that, but that is an inconvenient truth.

Trump is a full-blown, unrepentant racist and white supremacist, and
many people don his MAGA hats as a form of racist regalia.

Trump has no taste or tolerance for a movement for black lives, only for
the instruments to control them and quiet them. And he knows that many
of his supporters share this view.

That's why he paints black protesters as criminals and their white
allies as leftist radicals and even
\href{https://www.nytimes.com/article/what-antifa-trump.html}{antifa}.

This is the blacks-and-hippies duo that racists on the right have long
targeted. Author Dan Baum wrote in Harper's Magazine in 2016 that John
Ehrlichman, aide to Richard Nixon and Watergate co-conspirator,
\href{https://harpers.org/archive/2016/04/legalize-it-all/}{told him
about} the birth of the war on drugs:

``The Nixon campaign in 1968, and the Nixon White House after that, had
two enemies: the antiwar left and black people. You understand what I'm
saying? We knew we couldn't make it illegal to be either against the war
or black, but by getting the public to associate the hippies with
marijuana and blacks with heroin, and then criminalizing both heavily,
we could disrupt those communities. We could arrest their leaders, raid
their homes, break up their meetings, and vilify them night after night
on the evening news. Did we know we were lying about the drugs? Of
course we did.''

Trump, too, is trying to vilify what he must see as a nightmare
alliance. At his Tulsa rally,
\href{https://www.rev.com/blog/transcripts/donald-trump-tulsa-oklahoma-rally-speech-transcript}{he
warned}, ``If the Democrats gain power, then the rioters will be in
charge and no one will be safe and no one will have control.''

At that rally he told his supporters:

``The unhinged left-wing mob is trying to vandalize our history,
desecrate our monuments, our beautiful monuments. Tear down our statues
and punish, cancel and persecute anyone who does not conform to their
demands for absolute and total control. We're not conforming, that's why
we're here, actually. This cruel campaign of censorship and exclusion
violates everything we hold dear as Americans. They want to demolish our
heritage so they can impose their new oppressive regime in its place.''

Make no mistake, the ``our'' in that passage is ``white people's.''
This, for Trump, and Trump culture, is about white heritage, white power
and the possibility of displacement. That's what it has been about from
the beginning.

Blacks and hippies must be controlled and the police are the instruments
of that control.
\href{https://www.rev.com/blog/transcripts/donald-trump-press-conference-transcript-on-policing-june-16}{He
said in the Rose Garden,} ``Nobody needs a strong, trustworthy police
force more than those who live in distressed areas.''

But distress is an issue of resources that cannot be solved by police
repression. As the
\href{https://ellabakercenter.org/blog/2011/08/how-to-turn-a-single-day-of-service-into-long-lasting-impact}{Ella
Baker Center for Human Rights} has put it:

``The safest neighborhoods aren't the ones with the most prisons and the
most police --- they're the ones with the best schools, the cleanest
environment, and the most opportunities for young people and working
people.''

Trump's law-and-order talk doesn't seek to address the rage this
inequity has bred, but rather to contain it, to return society to
slumber, to have the oppressed suffer in silence so that the oppressors
can revel in the void.

As Geoff Nunberg
\href{https://www.npr.org/2016/07/28/487560886/is-trumps-call-for-law-and-order-a-coded-racial-message}{wrote
for NPR} about the racial coding of Trump's ``law and order'' push,
``Trump's single-handed effort to revive the slogan `law and order' is
the key to creating the perception of a new crisis of crime and
violence.''

The people protesting want justice and equality, an end to racism and a
dawning of a new egalitarianism. For a white supremacist, that is
heretical. Trump and his supporters see the protests as a crisis, a form
of chaos that threatens the order.

Trump said in the Rose Garden, ``Americans want law and order, they
demand law and order. They may not say it, they may not be talking about
it, but that's what they want. Some of them don't even know that's what
they want but that's what they want.''

That America he is talking to is white America, his portion of it, and
he is signaling that ``law and order'' is the only way to ensure the
continuity of their control.

\emph{The Times is committed to publishing}
\href{https://www.nytimes.com/2019/01/31/opinion/letters/letters-to-editor-new-york-times-women.html}{\emph{a
diversity of letters}} \emph{to the editor. We'd like to hear what you
think about this or any of our articles. Here are some}
\href{https://help.nytimes.com/hc/en-us/articles/115014925288-How-to-submit-a-letter-to-the-editor}{\emph{tips}}\emph{.
And here's our email:}
\href{mailto:letters@nytimes.com}{\emph{letters@nytimes.com}}\emph{.}

\emph{Follow The New York Times Opinion section on}
\href{https://www.facebook.com/nytopinion}{\emph{Facebook}} \emph{and}
\href{http://twitter.com/NYTOpinion}{\emph{Twitter
(@NYTopinion)}}\emph{, and}
\href{https://www.instagram.com/nytopinion/}{\emph{Instagram}}\emph{.}

Advertisement

\protect\hyperlink{after-bottom}{Continue reading the main story}

\hypertarget{site-index}{%
\subsection{Site Index}\label{site-index}}

\hypertarget{site-information-navigation}{%
\subsection{Site Information
Navigation}\label{site-information-navigation}}

\begin{itemize}
\tightlist
\item
  \href{https://help.nytimes.com/hc/en-us/articles/115014792127-Copyright-notice}{©~2020~The
  New York Times Company}
\end{itemize}

\begin{itemize}
\tightlist
\item
  \href{https://www.nytco.com/}{NYTCo}
\item
  \href{https://help.nytimes.com/hc/en-us/articles/115015385887-Contact-Us}{Contact
  Us}
\item
  \href{https://www.nytco.com/careers/}{Work with us}
\item
  \href{https://nytmediakit.com/}{Advertise}
\item
  \href{http://www.tbrandstudio.com/}{T Brand Studio}
\item
  \href{https://www.nytimes.com/privacy/cookie-policy\#how-do-i-manage-trackers}{Your
  Ad Choices}
\item
  \href{https://www.nytimes.com/privacy}{Privacy}
\item
  \href{https://help.nytimes.com/hc/en-us/articles/115014893428-Terms-of-service}{Terms
  of Service}
\item
  \href{https://help.nytimes.com/hc/en-us/articles/115014893968-Terms-of-sale}{Terms
  of Sale}
\item
  \href{https://spiderbites.nytimes.com}{Site Map}
\item
  \href{https://help.nytimes.com/hc/en-us}{Help}
\item
  \href{https://www.nytimes.com/subscription?campaignId=37WXW}{Subscriptions}
\end{itemize}
