Sections

SEARCH

\protect\hyperlink{site-content}{Skip to
content}\protect\hyperlink{site-index}{Skip to site index}

\href{https://www.nytimes.com/section/style}{Style}

\href{https://myaccount.nytimes.com/auth/login?response_type=cookie\&client_id=vi}{}

\href{https://www.nytimes.com/section/todayspaper}{Today's Paper}

\href{/section/style}{Style}\textbar{}TikTok Teens and K-Pop Stans Say
They Sank Trump Rally

\url{https://nyti.ms/37O0txZ}

\begin{itemize}
\item
\item
\item
\item
\item
\end{itemize}

\begin{itemize}
\item
  \href{https://www.nytimes.com/interactive/2020/08/04/us/elections/results-arizona-kansas-michigan-missouri-primaries.html?action=click\&pgtype=Article\&state=default\&region=TOP_BANNER\&context=storylines_menu}{Latest
  Results}
\item
  \href{https://www.nytimes.com/article/biden-vice-president-2020.html?action=click\&pgtype=Article\&state=default\&region=TOP_BANNER\&context=storylines_menu}{Biden's
  V.P. Search}
\item
  \href{https://www.nytimes.com/interactive/2020/07/24/us/politics/trump-biden-campaign-donors.html?action=click\&pgtype=Article\&state=default\&region=TOP_BANNER\&context=storylines_menu}{Map
  of Donations}
\item
  \href{https://www.nytimes.com/interactive/2020/us/elections/delegate-count-primary-results.html?action=click\&pgtype=Article\&state=default\&region=TOP_BANNER\&context=storylines_menu}{Delegate
  Count}
\item
  \href{https://www.nytimes.com/interactive/2019/us/politics/2020-presidential-candidates.html?action=click\&pgtype=Article\&state=default\&region=TOP_BANNER\&context=storylines_menu}{The
  Candidates}
\item
  \href{https://www.nytimes.com/newsletters/politics?action=click\&pgtype=Article\&state=default\&region=TOP_BANNER\&context=storylines_menu}{Politics
  Newsletter}
\end{itemize}

Advertisement

\protect\hyperlink{after-top}{Continue reading the main story}

Supported by

\protect\hyperlink{after-sponsor}{Continue reading the main story}

\hypertarget{tiktok-teens-and-k-pop-stans-say-they-sank-trump-rally}{%
\section{TikTok Teens and K-Pop Stans Say They Sank Trump
Rally}\label{tiktok-teens-and-k-pop-stans-say-they-sank-trump-rally}}

Did a successful prank inflate attendance expectations for President
Trump's rally in Tulsa, Okla.?

By \href{https://www.nytimes.com/by/taylor-lorenz}{Taylor Lorenz},
\href{https://www.nytimes.com/by/kellen-browning}{Kellen Browning} and
\href{https://www.nytimes.com/by/sheera-frenkel}{Sheera Frenkel}

\begin{itemize}
\item
  Published June 21, 2020Updated July 11, 2020
\item
  \begin{itemize}
  \item
  \item
  \item
  \item
  \item
  \end{itemize}
\end{itemize}

\includegraphics{https://static01.nyt.com/images/2020/07/20/world/20TikTok-spansub/merlin_173759871_a4db2097-8ece-4797-9524-800ea901b9b6-articleLarge.jpg?quality=75\&auto=webp\&disable=upscale}

\href{https://www.nytimes.com/2020/06/22/style/trump-tulsa-tie.html}{President
Trump's} campaign promised huge crowds at his
\href{https://www.nytimes.com/2020/07/08/us/politics/trump-rally-portsmouth-new-hampshire.html}{rally}
in Tulsa, Okla., on Saturday, but it failed to deliver. Hundreds of
teenage
\href{https://www.nytimes.com/2020/08/03/technology/trump-tiktok-microsoft.html}{TikTok}
users and K-pop fans say they're at least partially responsible.

Brad Parscale, the chairman of Mr. Trump's re-election campaign, posted
on Twitter on Monday that the campaign had fielded
\href{https://twitter.com/parscale/status/1272543199647666176?s=20}{more
than a million} ticket requests, but
\href{https://twitter.com/AsteadWesley/status/1274465912951844866?s=20}{reporters
at the event noted} the attendance was lower than expected. The campaign
also canceled planned events outside the rally for an anticipated
overflow crowd that did not materialize.

Tim Murtaugh, a spokesman for the Trump campaign,
\href{https://twitter.com/cmsub/status/1274473814211125249/photo/1}{said
protesters} stopped supporters from entering the rally, held at the BOK
Center, which has a 19,000-seat capacity.

But reporters present said there
\href{https://www.nytimes.com/2020/06/20/us/trump-rally-tulsa.html\#link-60a18f83}{were
few protests}. According to a spokesman for the Tulsa Fire Department on
Sunday, the fire marshal counted 6,200 scanned tickets of attendees.
(That number would not include staff, media or those in box suites.)

TikTok users and
\href{https://www.nytimes.com/2018/12/11/smarter-living/the-edit-k-pop.html}{fans
of Korean pop music groups} claimed to have registered potentially
hundreds of thousands of tickets for Mr. Trump's campaign rally as a
prank. After the Trump campaign's official account @TeamTrump posted a
\href{https://twitter.com/TeamTrump/status/1271205174611259393}{tweet}
asking supporters to register for free tickets using their phones on
June 11, K-pop fan accounts began sharing the information with
followers, encouraging them to register for the
\href{https://www.nytimes.com/2020/07/08/us/politics/trump-rally-portsmouth-new-hampshire.html}{rally}
--- and then not show.

\begin{center}\rule{0.5\linewidth}{\linethickness}\end{center}

Some of the latest from
\href{https://www.nytimes.com/by/taylor-lorenz}{Taylor Lorenz}:

\begin{itemize}
\item
  \href{https://www.nytimes.com/2020/07/10/style/tiktok-ban-us-users-influencers-taylor-lorenz.html}{TikTok
  Users React to Threat to Ban App in U.S.}
\item
  \href{https://www.nytimes.com/2020/02/13/style/the-original-renegade.html}{Meet
  the Original Renegade Dance Creator}
\item
  \href{https://www.nytimes.com/2020/07/09/style/tiktok-stars-race-to-land-reality-shows.html}{TikTok
  Stars Race to Land Reality TV Shows}
\item
  \href{https://www.nytimes.com/2020/07/16/style/taylor-lorenz-internet-culture-reporting.html}{How
  We Report on Internet Culture and the Teens Who Rule It}
\end{itemize}

\begin{center}\rule{0.5\linewidth}{\linethickness}\end{center}

The trend quickly spread on TikTok, where videos with millions of views
instructed viewers to do the same,
\href{https://www.cnn.com/2020/06/16/politics/tiktok-trump-tulsa-rally-trnd/index.html}{as
CNN reported} on Tuesday. ``Oh no, I signed up for a Trump rally, and I
can't go,'' one woman joked, along with a fake cough,
\href{https://www.tiktok.com/@proloser12245/video/6838621598229056773}{in
a TikTok} posted on June 15.

Thousands of other users posted similar tweets and videos to TikTok that
racked up millions of views. Representatives for TikTok did not
immediately respond to requests for comment.

``It spread mostly
\href{https://www.nytimes.com/2020/06/10/style/elite-tiktok.html}{through
Alt TikTok} --- we kept it on the quiet side where people do pranks and
a lot of activism,'' said the YouTuber Elijah Daniel, 26, who
participated in the social media campaign. ``K-pop Twitter and Alt
TikTok have a good alliance where they spread information amongst each
other very quickly. They all know the algorithms and how they can boost
videos to get where they want.''

Many users deleted their posts after 24 to 48 hours in order to conceal
their plan and keep it from spreading into the mainstream internet.
``The majority of people who made them deleted them after the first day
because we didn't want the Trump campaign to catch wind,'' Mr. Daniel
said. ``These kids are smart and they thought of everything.''

Twitter users on Saturday night were quick to declare the social media
campaign's victory. ``Actually you just got ROCKED by teens on TikTok,''
Representative Alexandria Ocasio-Cortez of New York
\href{https://twitter.com/AOC/status/1274499021625794565?s=20}{tweeted}
in response to Mr. Parscale, who had tweeted that ``radical protestors''
had ``interfered'' with attendance.

\includegraphics{https://static01.nyt.com/images/2020/06/20/us/politics/20Tulsa-memo/20Tulsa-memo-videoSixteenByNine3000.jpg}

Steve Schmidt, a longtime Republican strategist,
\href{https://twitter.com/SteveSchmidtSES/status/1274486428160811009}{added},
``The teens of America have struck a savage blow against
@realDonaldTrump.''

``Leftists and online trolls doing a victory lap, thinking they somehow
impacted rally attendance, don't know what they're talking about or how
our rallies work,'' Mr. Parscale said in a statement on Sunday.
``Registering for a rally means you've RSVPed with a cellphone number
and we constantly weed out bogus numbers, as we did with tens of
thousands at the Tulsa rally, in calculating our possible attendee
pool.''

Mary Jo Laupp, a 51-year-old from Fort Dodge, Iowa,
\href{https://www.dailydot.com/debug/tiktok-challenge-trump-rally/}{said
she had been watching}black TikTok users express their frustration about
Mr. Trump's hosting his rally on
\href{https://www.nytimes.com/interactive/2020/06/18/style/juneteenth-celebration.html}{Juneteenth},
the holiday on June 19. (The rally was later moved to June 20.) She
``vented'' her own anger in a
\href{https://www.tiktok.com/@maryjolaupp/video/6837311838640803078}{late-night
TikTok video} on June 11 --- and provided a call to action.

\hypertarget{latest-updates-2020-election}{%
\section{\texorpdfstring{\href{https://www.nytimes.com/2020/08/04/us/elections/primary-election-michigan-arizona-kansas.html?action=click\&pgtype=Article\&state=default\&region=MAIN_CONTENT_1\&context=storylines_live_updates}{Latest
Updates: 2020
Election}}{Latest Updates: 2020 Election}}\label{latest-updates-2020-election}}

Updated 2020-08-05T03:23:56.561Z

\begin{itemize}
\tightlist
\item
  \href{https://www.nytimes.com/2020/08/04/us/elections/primary-election-michigan-arizona-kansas.html?action=click\&pgtype=Article\&state=default\&region=MAIN_CONTENT_1\&context=storylines_live_updates\#link-3924dd44}{Two
  G.O.P. Senate primaries offer --- what else? --- a test of loyalty to
  Trump.}
\item
  \href{https://www.nytimes.com/2020/08/04/us/elections/primary-election-michigan-arizona-kansas.html?action=click\&pgtype=Article\&state=default\&region=MAIN_CONTENT_1\&context=storylines_live_updates\#link-62a8e06b}{The
  military-style uniforms of federal agents who responded to the unrest
  in Portland will be replaced.}
\item
  \href{https://www.nytimes.com/2020/08/04/us/elections/primary-election-michigan-arizona-kansas.html?action=click\&pgtype=Article\&state=default\&region=MAIN_CONTENT_1\&context=storylines_live_updates\#link-32b39e33}{President
  Trump is suddenly a big supporter of mail-in voting --- in Florida.}
\end{itemize}

\href{https://www.nytimes.com/2020/08/04/us/elections/primary-election-michigan-arizona-kansas.html?action=click\&pgtype=Article\&state=default\&region=MAIN_CONTENT_1\&context=storylines_live_updates}{See
more updates}

``I recommend all of those of us that want to see this 19,000-seat
auditorium barely filled or completely empty go reserve tickets now, and
leave him standing there alone on the stage,'' Ms. Laupp said in the
video.

When she checked her phone the next morning, Ms. Laupp said, the video
was starting to go viral. It has more than 700,000 likes, she added, and
more than two million views.

Image

Many of the arena's 19,000 seats remained empty as Mr. Trump
spoke.Credit...Doug Mills/The New York Times

She said she believed that at least 17,000 tickets were accounted for
based on comments she received on her TikTok videos, but added that
people reaching out to her said tens of thousands more had been
reserved.

Ms. Laupp said she was ``overwhelmed'' and ``stunned'' by the
possibility that she and the effort
\href{https://www.tiktok.com/@maryjolaupp/video/6840619115585998085}{she
helped inspire} might have contributed to the low rally attendance.

``There are teenagers in this country who participated in this little
no-show protest, who believe that they can have an impact in their
country in the political system even though they're not old enough to
vote right now,'' she said.

The effort to deprive Mr. Trump of a large crowd spread from Twitter and
TikTok across multiple social media platforms, including Instagram and
Snapchat.

Erin Hoffman, an 18-year-old from upstate New York, said she heard from
a friend on Instagram about the social media campaign. She then spread
it herself via her Snapchat story, and said friends who saw her post
told her they were reserving tickets.

``Trump has been actively trying to disenfranchise millions of Americans
in so many ways, and to me, this was the protest I was able to
perform,'' said Ms. Hoffman, who reserved two tickets herself and
persuaded one of her parents to nab two more. ``He doesn't deserve the
platform he has been given.''

Ms. Laupp said that many of the people who shared her video added
commentary encouraging people to procure the tickets with fake names and
phone numbers. In the comment section under her own video, TikTok users
exchanged advice on how to acquire a Google Voice number or another
internet-connected phone line.

``We all know the Trump campaign feeds on data, they are constantly
mining these rallies for data,'' said Ms. Laupp, who worked on several
rallies for Pete Buttigieg's campaign for the Democratic nomination for
president. ``Feeding them false data was a bonus. The data they think
they have, the data they are collecting from this rally, isn't
accurate.''

Campaign officials on Sunday said that many people who had signed up
were not supporters, but online tricksters. One campaign adviser claimed
that ``troll data'' was still usable, claiming it would help the
campaign avoid the same pitfall in the future. The adviser said that the
data could be put into the system to ``tighten up the formula used to
determine projected attendance for rallies.''

Ms. Laupp added that several people who took part in her campaign
complained that once they signed up for the rally with their real phone
numbers, they couldn't get the Trump campaign to stop texting them and
sending them messages.

Mary Garcia, a 19-year-old student from California, said that she used a
Google Voice number to sign up for the rally, but that two of her
friends who also signed up used their real numbers and had been
inundated with texts from the Trump campaign.

Ms. Garcia said she decided to sign up on a whim after seeing Ms.
Laupp's video, but after she saw the Trump campaign boasting about its
record-setting ticket numbers she regretted what she had done.

``I feel like it doesn't even matter if the rally is full or not,'' Ms.
Garcia said. ``They are going to boast about a million tickets being
registered, and then they'll just lie or whatever about how big the
audience was.''

K-pop stans have been getting increasingly involved in American politics
in recent months. After the Trump campaign
\href{https://twitter.com/TeamTrump/status/1270127968736677888}{solicited
messages} for the president's birthday on June 8, K-pop stans submitted
a stream of prank messages. And earlier in June, when the Dallas Police
Department asked citizens to submit videos of suspicious or illegal
activity through a dedicated app, K-pop Twitter claimed credit for
\href{https://melmagazine.com/en-us/story/what-we-can-learn-from-k-pop-stans-about-fighting-fascism}{crashing
the app} by uploading thousands of ``fancam'' videos.

They also reclaimed the \#WhiteLivesMatter hashtag in May, by spamming
it with endless K-pop videos, in hopes to make it harder for white
supremacists and sympathizers to find one another and communicate their
messaging.

Whether or not the prank to call in false tickets was the reason for the
empty upper rafters at Mr. Trump's rally, teenagers online celebrated.
On Twitter,
\href{https://twitter.com/sophiadelsol/status/1274145891490959360?s=20}{several}
\href{https://twitter.com/cbjeffreys/status/1274514747241750529?s=20}{accounts}
\href{https://twitter.com/s87788255/status/1274536326528856064?s=20}{tweeted},
``best senior prank ever.''

Annie Karni contributed reporting.

\begin{center}\rule{0.5\linewidth}{\linethickness}\end{center}

\hypertarget{our-2020-election-guide}{%
\section{Our 2020 Election Guide}\label{our-2020-election-guide}}

Updated Aug. 4, 2020

\begin{itemize}
\item
  \begin{center}\rule{0.5\linewidth}{\linethickness}\end{center}

  \hypertarget{the-latest}{%
  \subsection{The Latest}\label{the-latest}}

  \begin{itemize}
  \tightlist
  \item
    Kris Kobach, a polarizing figure in Kansas politics,
    \href{https://www.nytimes.com/2020/08/04/us/politics/kobach-tlaib.html?action=click\&pgtype=Article\&state=default\&region=BELOW_MAIN_CONTENT\&context=storylines_guide}{lost
    the Senate primary there}, relieving G.O.P. officials who feared he
    could jeopardize a safe seat.
  \end{itemize}
\item
  \begin{center}\rule{0.5\linewidth}{\linethickness}\end{center}

  \hypertarget{bidens-vp-search}{%
  \subsection{Biden's V.P. Search}\label{bidens-vp-search}}

  \begin{itemize}
  \tightlist
  \item
    \href{https://www.nytimes.com/article/biden-vice-president-2020.html?action=click\&pgtype=Article\&state=default\&region=BELOW_MAIN_CONTENT\&context=storylines_guide}{Here
    are 13 women} who have been under consideration to be Joe Biden's
    running mate, and why each might be chosen --- and might not be.
  \end{itemize}
\item
  \begin{center}\rule{0.5\linewidth}{\linethickness}\end{center}

  \hypertarget{keep-up-with-our-coverage}{%
  \subsection{Keep Up With Our
  Coverage}\label{keep-up-with-our-coverage}}

  \begin{itemize}
  \tightlist
  \item
    Get an
    \href{https://www.nytimes.com/newsletters/politics?action=click\&pgtype=Article\&state=default\&region=BELOW_MAIN_CONTENT\&context=storylines_guide}{email}
    recapping the day's news
  \end{itemize}

  \begin{itemize}
  \tightlist
  \item
    Download our mobile app on
    \href{https://apps.apple.com/us/app/nytimes/id284862083?ls=1\&mat_click_id=5c79ae7455014fd1bd66b5610c05b8f2-20191112-16948\&referrer=mat_click_id\%3D5c79ae7455014fd1bd66b5610c05b8f2-20191112-16948\%26link_click_id\%3D722930677036718082}{iOS}
    and
    \href{http://a.localytics.com/android?id=com.nytimes.android\&referrer=utm_source\%3Dother_nyt_mobile_web\%26utm_medium\%3DWeb\%2520page\%26utm_term\%3DGeneral\%2520Mobile\%2520Page\%26utm_campaign\%3DNYT\%2520Mobile\%2520General\%2520Page}{Android}
    and turn on Breaking News and Politics alerts
  \end{itemize}
\end{itemize}

Advertisement

\protect\hyperlink{after-bottom}{Continue reading the main story}

\hypertarget{site-index}{%
\subsection{Site Index}\label{site-index}}

\hypertarget{site-information-navigation}{%
\subsection{Site Information
Navigation}\label{site-information-navigation}}

\begin{itemize}
\tightlist
\item
  \href{https://help.nytimes.com/hc/en-us/articles/115014792127-Copyright-notice}{©~2020~The
  New York Times Company}
\end{itemize}

\begin{itemize}
\tightlist
\item
  \href{https://www.nytco.com/}{NYTCo}
\item
  \href{https://help.nytimes.com/hc/en-us/articles/115015385887-Contact-Us}{Contact
  Us}
\item
  \href{https://www.nytco.com/careers/}{Work with us}
\item
  \href{https://nytmediakit.com/}{Advertise}
\item
  \href{http://www.tbrandstudio.com/}{T Brand Studio}
\item
  \href{https://www.nytimes.com/privacy/cookie-policy\#how-do-i-manage-trackers}{Your
  Ad Choices}
\item
  \href{https://www.nytimes.com/privacy}{Privacy}
\item
  \href{https://help.nytimes.com/hc/en-us/articles/115014893428-Terms-of-service}{Terms
  of Service}
\item
  \href{https://help.nytimes.com/hc/en-us/articles/115014893968-Terms-of-sale}{Terms
  of Sale}
\item
  \href{https://spiderbites.nytimes.com}{Site Map}
\item
  \href{https://help.nytimes.com/hc/en-us}{Help}
\item
  \href{https://www.nytimes.com/subscription?campaignId=37WXW}{Subscriptions}
\end{itemize}
