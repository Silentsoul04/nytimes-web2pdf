Sections

SEARCH

\protect\hyperlink{site-content}{Skip to
content}\protect\hyperlink{site-index}{Skip to site index}

\href{https://www.nytimes.com/section/us}{U.S.}

\href{https://myaccount.nytimes.com/auth/login?response_type=cookie\&client_id=vi}{}

\href{https://www.nytimes.com/section/todayspaper}{Today's Paper}

\href{/section/us}{U.S.}\textbar{}Health Officials Had to Face a
Pandemic. Then Came the Death Threats.

\url{https://nyti.ms/3152Cnk}

\begin{itemize}
\item
\item
\item
\item
\item
\item
\end{itemize}

\href{https://www.nytimes.com/news-event/coronavirus?action=click\&pgtype=Article\&state=default\&region=TOP_BANNER\&context=storylines_menu}{The
Coronavirus Outbreak}

\begin{itemize}
\tightlist
\item
  live\href{https://www.nytimes.com/2020/08/01/world/coronavirus-covid-19.html?action=click\&pgtype=Article\&state=default\&region=TOP_BANNER\&context=storylines_menu}{Latest
  Updates}
\item
  \href{https://www.nytimes.com/interactive/2020/us/coronavirus-us-cases.html?action=click\&pgtype=Article\&state=default\&region=TOP_BANNER\&context=storylines_menu}{Maps
  and Cases}
\item
  \href{https://www.nytimes.com/interactive/2020/science/coronavirus-vaccine-tracker.html?action=click\&pgtype=Article\&state=default\&region=TOP_BANNER\&context=storylines_menu}{Vaccine
  Tracker}
\item
  \href{https://www.nytimes.com/interactive/2020/07/29/us/schools-reopening-coronavirus.html?action=click\&pgtype=Article\&state=default\&region=TOP_BANNER\&context=storylines_menu}{What
  School May Look Like}
\item
  \href{https://www.nytimes.com/live/2020/07/31/business/stock-market-today-coronavirus?action=click\&pgtype=Article\&state=default\&region=TOP_BANNER\&context=storylines_menu}{Economy}
\end{itemize}

Advertisement

\protect\hyperlink{after-top}{Continue reading the main story}

Supported by

\protect\hyperlink{after-sponsor}{Continue reading the main story}

\hypertarget{health-officials-had-to-face-a-pandemic-then-came-the-death-threats}{%
\section{Health Officials Had to Face a Pandemic. Then Came the Death
Threats.}\label{health-officials-had-to-face-a-pandemic-then-came-the-death-threats}}

State and local health officials have found themselves at the center of
regular news briefings amid the coronavirus outbreak, making them
targets for harassment and threats.

\includegraphics{https://static01.nyt.com/images/2020/06/22/us/22VIRUS-HEALTHCZARS-ferrer/merlin_170127687_2a28ca12-e225-43d9-981b-6a05954e01d2-articleLarge.jpg?quality=75\&auto=webp\&disable=upscale}

\href{https://www.nytimes.com/by/julie-bosman}{\includegraphics{https://static01.nyt.com/images/2018/11/09/multimedia/author-julie-bosman/author-julie-bosman-thumbLarge.png}}

By \href{https://www.nytimes.com/by/julie-bosman}{Julie Bosman}

\begin{itemize}
\item
  Published June 22, 2020Updated June 24, 2020
\item
  \begin{itemize}
  \item
  \item
  \item
  \item
  \item
  \item
  \end{itemize}
\end{itemize}

Leaders of local and state health departments have been subject to
harassment, personal insults and death threats in recent weeks, a
response from a vocal and angry minority of the public who say that mask
requirements and restrictions on businesses have gone too far.

One top health official, Dr. Barbara Ferrer, the director of the Los
Angeles County Department of Public Health, issued a statement on Monday
condemning attacks on public health directors and disclosing that she
faced repeated threats to her safety.

``The death threats started last month, during a Covid-19 Facebook Live
public briefing when someone very casually suggested that I should be
shot,'' Dr. Ferrer said in a statement. ``I didn't immediately see the
message, but my husband did, my children did, and so did my
colleagues.''

``It is deeply worrisome,'' she added, ``to imagine that our hardworking
infectious disease physicians, nurses, epidemiologists and environmental
health specialists or any of our other team members would have to face
this level of hatred.''

Across the country, many public health officials
\href{https://www.nytimes.com/2020/03/14/us/coronavirus-health-departments.html}{entered
the coronavirus pandemic with bare-bones staffs and strained budgets},
leaving them ill-prepared to handle a mounting crisis. Before the
pandemic, they had focused on illness prevention, contact tracing for
communicable diseases, vaccinations and campaigns against smoking and
vaping.

Now some of them, suddenly facing the public with regular television
briefings about efforts to fight the coronavirus, are choosing to leave
their positions entirely.

\hypertarget{latest-updates-global-coronavirus-outbreak}{%
\section{\texorpdfstring{\href{https://www.nytimes.com/2020/08/01/world/coronavirus-covid-19.html?action=click\&pgtype=Article\&state=default\&region=MAIN_CONTENT_1\&context=storylines_live_updates}{Latest
Updates: Global Coronavirus
Outbreak}}{Latest Updates: Global Coronavirus Outbreak}}\label{latest-updates-global-coronavirus-outbreak}}

Updated 2020-08-02T00:50:37.907Z

\begin{itemize}
\tightlist
\item
  \href{https://www.nytimes.com/2020/08/01/world/coronavirus-covid-19.html?action=click\&pgtype=Article\&state=default\&region=MAIN_CONTENT_1\&context=storylines_live_updates\#link-34047410}{The
  U.S. reels as July cases more than double the total of any other
  month.}
\item
  \href{https://www.nytimes.com/2020/08/01/world/coronavirus-covid-19.html?action=click\&pgtype=Article\&state=default\&region=MAIN_CONTENT_1\&context=storylines_live_updates\#link-3ac56579}{Top
  officials work to break impasse over jobless benefit.}
\item
  \href{https://www.nytimes.com/2020/08/01/world/coronavirus-covid-19.html?action=click\&pgtype=Article\&state=default\&region=MAIN_CONTENT_1\&context=storylines_live_updates\#link-25930521}{Thousands
  in Berlin protest Germany's coronavirus measures.}
\end{itemize}

\href{https://www.nytimes.com/2020/08/01/world/coronavirus-covid-19.html?action=click\&pgtype=Article\&state=default\&region=MAIN_CONTENT_1\&context=storylines_live_updates}{See
more updates}

More live coverage:
\href{https://www.nytimes.com/live/2020/07/31/business/stock-market-today-coronavirus?action=click\&pgtype=Article\&state=default\&region=MAIN_CONTENT_1\&context=storylines_live_updates}{Markets}

Lori Tremmel Freeman, the chief executive of the National Association of
County and City Health Officials, said last week that dozens of top
health officials have resigned or been fired since the pandemic began.
At least four state health directors have resigned from their posts; Dr.
Amy Acton, the state health director of Ohio,
\href{https://www.dispatch.com/news/20200611/dr-amy-acton-resigns-as-state-health-director-democrats-cite-criticism-from-gop-lawmakers}{stepped
down} this month after enduring anti-Semitic attacks and demonstrations
by armed protesters on her front lawn.

Dr. Umair A. Shah, the executive director of the public health
department in Harris County, Texas, which includes Houston, described a
tense new role. ``Now that we're quite visible and we're part of very
difficult decision-making, naturally those decisions are having an
incredible impact on community members in a very specific way,'' Dr.
Shah said. ``That's where the problem comes in.''

Not all of the officials have said why they are leaving, and some have
cited personal reasons or planned retirements, but Ms. Freeman said she
had heard many accounts of harassment.

``There's a big red target on their backs,'' Ms. Freeman said. ``They're
becoming villainized for their guidance. In normal times, they're very
trusted members of their community.''

Some critics of the public health directors have said that they believe
that allowing businesses to operate is worth the risk of spreading the
coronavirus, and that health directors are too cautious about
reopenings. Others have cited conspiracy theories that claim that the
coronavirus is a hoax; that the development of a vaccine is part of a
massive effort to track citizens and monitor their movements; and that
wearing a mask or cloth face covering is a practice that impedes
personal freedom.

In Washington State, where rural counties are
\href{https://www.nytimes.com/2020/06/22/us/new-coronavirus-phase.html}{struggling
with new outbreaks} and trying to warn residents to take basic
precautions to stem the spread of the virus, pleas from local health
officials have often been answered with hostility and threats.

In Yakima County, which has more than six times as many cases per capita
as the county that includes Seattle, hospitals have reached capacity and
patients were being taken elsewhere for medical care. Gov. Jay Inslee
warned over the weekend that ``we are frankly at the breaking point,''
and has said he would require Yakima residents to wear face coverings in
an effort to slow the virus's spread.

\href{https://www.nytimes.com/news-event/coronavirus?action=click\&pgtype=Article\&state=default\&region=MAIN_CONTENT_3\&context=storylines_faq}{}

\hypertarget{the-coronavirus-outbreak-}{%
\subsubsection{The Coronavirus Outbreak
›}\label{the-coronavirus-outbreak-}}

\hypertarget{frequently-asked-questions}{%
\paragraph{Frequently Asked
Questions}\label{frequently-asked-questions}}

Updated July 27, 2020

\begin{itemize}
\item ~
  \hypertarget{should-i-refinance-my-mortgage}{%
  \paragraph{Should I refinance my
  mortgage?}\label{should-i-refinance-my-mortgage}}

  \begin{itemize}
  \tightlist
  \item
    \href{https://www.nytimes.com/article/coronavirus-money-unemployment.html?action=click\&pgtype=Article\&state=default\&region=MAIN_CONTENT_3\&context=storylines_faq}{It
    could be a good idea,} because mortgage rates have
    \href{https://www.nytimes.com/2020/07/16/business/mortgage-rates-below-3-percent.html?action=click\&pgtype=Article\&state=default\&region=MAIN_CONTENT_3\&context=storylines_faq}{never
    been lower.} Refinancing requests have pushed mortgage applications
    to some of the highest levels since 2008, so be prepared to get in
    line. But defaults are also up, so if you're thinking about buying a
    home, be aware that some lenders have tightened their standards.
  \end{itemize}
\item ~
  \hypertarget{what-is-school-going-to-look-like-in-september}{%
  \paragraph{What is school going to look like in
  September?}\label{what-is-school-going-to-look-like-in-september}}

  \begin{itemize}
  \tightlist
  \item
    It is unlikely that many schools will return to a normal schedule
    this fall, requiring the grind of
    \href{https://www.nytimes.com/2020/06/05/us/coronavirus-education-lost-learning.html?action=click\&pgtype=Article\&state=default\&region=MAIN_CONTENT_3\&context=storylines_faq}{online
    learning},
    \href{https://www.nytimes.com/2020/05/29/us/coronavirus-child-care-centers.html?action=click\&pgtype=Article\&state=default\&region=MAIN_CONTENT_3\&context=storylines_faq}{makeshift
    child care} and
    \href{https://www.nytimes.com/2020/06/03/business/economy/coronavirus-working-women.html?action=click\&pgtype=Article\&state=default\&region=MAIN_CONTENT_3\&context=storylines_faq}{stunted
    workdays} to continue. California's two largest public school
    districts --- Los Angeles and San Diego --- said on July 13, that
    \href{https://www.nytimes.com/2020/07/13/us/lausd-san-diego-school-reopening.html?action=click\&pgtype=Article\&state=default\&region=MAIN_CONTENT_3\&context=storylines_faq}{instruction
    will be remote-only in the fall}, citing concerns that surging
    coronavirus infections in their areas pose too dire a risk for
    students and teachers. Together, the two districts enroll some
    825,000 students. They are the largest in the country so far to
    abandon plans for even a partial physical return to classrooms when
    they reopen in August. For other districts, the solution won't be an
    all-or-nothing approach.
    \href{https://bioethics.jhu.edu/research-and-outreach/projects/eschool-initiative/school-policy-tracker/}{Many
    systems}, including the nation's largest, New York City, are
    devising
    \href{https://www.nytimes.com/2020/06/26/us/coronavirus-schools-reopen-fall.html?action=click\&pgtype=Article\&state=default\&region=MAIN_CONTENT_3\&context=storylines_faq}{hybrid
    plans} that involve spending some days in classrooms and other days
    online. There's no national policy on this yet, so check with your
    municipal school system regularly to see what is happening in your
    community.
  \end{itemize}
\item ~
  \hypertarget{is-the-coronavirus-airborne}{%
  \paragraph{Is the coronavirus
  airborne?}\label{is-the-coronavirus-airborne}}

  \begin{itemize}
  \tightlist
  \item
    The coronavirus
    \href{https://www.nytimes.com/2020/07/04/health/239-experts-with-one-big-claim-the-coronavirus-is-airborne.html?action=click\&pgtype=Article\&state=default\&region=MAIN_CONTENT_3\&context=storylines_faq}{can
    stay aloft for hours in tiny droplets in stagnant air}, infecting
    people as they inhale, mounting scientific evidence suggests. This
    risk is highest in crowded indoor spaces with poor ventilation, and
    may help explain super-spreading events reported in meatpacking
    plants, churches and restaurants.
    \href{https://www.nytimes.com/2020/07/06/health/coronavirus-airborne-aerosols.html?action=click\&pgtype=Article\&state=default\&region=MAIN_CONTENT_3\&context=storylines_faq}{It's
    unclear how often the virus is spread} via these tiny droplets, or
    aerosols, compared with larger droplets that are expelled when a
    sick person coughs or sneezes, or transmitted through contact with
    contaminated surfaces, said Linsey Marr, an aerosol expert at
    Virginia Tech. Aerosols are released even when a person without
    symptoms exhales, talks or sings, according to Dr. Marr and more
    than 200 other experts, who
    \href{https://academic.oup.com/cid/article/doi/10.1093/cid/ciaa939/5867798}{have
    outlined the evidence in an open letter to the World Health
    Organization}.
  \end{itemize}
\item ~
  \hypertarget{what-are-the-symptoms-of-coronavirus}{%
  \paragraph{What are the symptoms of
  coronavirus?}\label{what-are-the-symptoms-of-coronavirus}}

  \begin{itemize}
  \tightlist
  \item
    Common symptoms
    \href{https://www.nytimes.com/article/symptoms-coronavirus.html?action=click\&pgtype=Article\&state=default\&region=MAIN_CONTENT_3\&context=storylines_faq}{include
    fever, a dry cough, fatigue and difficulty breathing or shortness of
    breath.} Some of these symptoms overlap with those of the flu,
    making detection difficult, but runny noses and stuffy sinuses are
    less common.
    \href{https://www.nytimes.com/2020/04/27/health/coronavirus-symptoms-cdc.html?action=click\&pgtype=Article\&state=default\&region=MAIN_CONTENT_3\&context=storylines_faq}{The
    C.D.C. has also} added chills, muscle pain, sore throat, headache
    and a new loss of the sense of taste or smell as symptoms to look
    out for. Most people fall ill five to seven days after exposure, but
    symptoms may appear in as few as two days or as many as 14 days.
  \end{itemize}
\item ~
  \hypertarget{does-asymptomatic-transmission-of-covid-19-happen}{%
  \paragraph{Does asymptomatic transmission of Covid-19
  happen?}\label{does-asymptomatic-transmission-of-covid-19-happen}}

  \begin{itemize}
  \tightlist
  \item
    So far, the evidence seems to show it does. A widely cited
    \href{https://www.nature.com/articles/s41591-020-0869-5}{paper}
    published in April suggests that people are most infectious about
    two days before the onset of coronavirus symptoms and estimated that
    44 percent of new infections were a result of transmission from
    people who were not yet showing symptoms. Recently, a top expert at
    the World Health Organization stated that transmission of the
    coronavirus by people who did not have symptoms was ``very rare,''
    \href{https://www.nytimes.com/2020/06/09/world/coronavirus-updates.html?action=click\&pgtype=Article\&state=default\&region=MAIN_CONTENT_3\&context=storylines_faq\#link-1f302e21}{but
    she later walked back that statement.}
  \end{itemize}
\end{itemize}

``I've been called a Nazi numerous times,'' said Andre Fresco, the
executive director of the Yakima Health District. ``I've been told not
to show up at certain businesses. I've been called a Communist and
Gestapo. I've been cursed at and generally treated in a very
unprofessional way. It's very difficult.''

In California, angry protesters have tracked down addresses of public
health officers and gathered outside their homes, chanting and holding
signs. Last week, a group called the Freedom Angels did just that in
Contra Costa County, Calif., filming themselves and posting the videos
on Facebook.

``We came today to protest in front of our county public health
officer's house, and some people might have issues with that, that we
took it to their house,'' one woman said in a video. ``But I have to
tell you guys, they're coming to our houses. Their agenda is contact
tracing, testing, mandatory masks and ultimately an injection that has
not been tested,'' she said, apparently referring to a vaccine even
though none have been approved.

Dr. Nichole Quick, the chief health officer for Orange County, Calif.,
resigned as protests and harassment intensified after
\href{https://www.ocregister.com/2020/06/08/o-c-chief-health-officer-dr-nichole-quick-resigns-amid-mask-controversy-threats/}{an
order to require face masks in certain businesses}, including grocery
stores and pharmacies. Emily Brown, director of the Rio Grande County
Public Health Department in rural Colorado, was
\href{https://www.npr.org/2020/06/14/876714176/public-health-workers-face-threats-unemployment-while-fighting-virus}{fired
when she encountered community resistance} to the stricter rules she had
encouraged.

The departures across the country have prompted industry officials to
ask whether a dearth of leadership is ahead for health departments, even
as they combat the pandemic.

``We've never seen this level of vitriol before,'' said Kat DeBurgh, the
executive director of the Health Officers Association of California.
``I'm worried not just about the present but about the future. When
they're subject to such harassment, who is going to step into these
jobs?''

Mitch Smith contributed reporting.

Advertisement

\protect\hyperlink{after-bottom}{Continue reading the main story}

\hypertarget{site-index}{%
\subsection{Site Index}\label{site-index}}

\hypertarget{site-information-navigation}{%
\subsection{Site Information
Navigation}\label{site-information-navigation}}

\begin{itemize}
\tightlist
\item
  \href{https://help.nytimes.com/hc/en-us/articles/115014792127-Copyright-notice}{©~2020~The
  New York Times Company}
\end{itemize}

\begin{itemize}
\tightlist
\item
  \href{https://www.nytco.com/}{NYTCo}
\item
  \href{https://help.nytimes.com/hc/en-us/articles/115015385887-Contact-Us}{Contact
  Us}
\item
  \href{https://www.nytco.com/careers/}{Work with us}
\item
  \href{https://nytmediakit.com/}{Advertise}
\item
  \href{http://www.tbrandstudio.com/}{T Brand Studio}
\item
  \href{https://www.nytimes.com/privacy/cookie-policy\#how-do-i-manage-trackers}{Your
  Ad Choices}
\item
  \href{https://www.nytimes.com/privacy}{Privacy}
\item
  \href{https://help.nytimes.com/hc/en-us/articles/115014893428-Terms-of-service}{Terms
  of Service}
\item
  \href{https://help.nytimes.com/hc/en-us/articles/115014893968-Terms-of-sale}{Terms
  of Sale}
\item
  \href{https://spiderbites.nytimes.com}{Site Map}
\item
  \href{https://help.nytimes.com/hc/en-us}{Help}
\item
  \href{https://www.nytimes.com/subscription?campaignId=37WXW}{Subscriptions}
\end{itemize}
