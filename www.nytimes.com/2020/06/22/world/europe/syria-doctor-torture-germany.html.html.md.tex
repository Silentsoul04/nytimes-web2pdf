Sections

SEARCH

\protect\hyperlink{site-content}{Skip to
content}\protect\hyperlink{site-index}{Skip to site index}

\href{https://www.nytimes.com/section/world/europe}{Europe}

\href{https://myaccount.nytimes.com/auth/login?response_type=cookie\&client_id=vi}{}

\href{https://www.nytimes.com/section/todayspaper}{Today's Paper}

\href{/section/world/europe}{Europe}\textbar{}Syrian Doctor Accused of
Torture Is Arrested in Germany

\url{https://nyti.ms/3dnBfHU}

\begin{itemize}
\item
\item
\item
\item
\item
\end{itemize}

Advertisement

\protect\hyperlink{after-top}{Continue reading the main story}

Supported by

\protect\hyperlink{after-sponsor}{Continue reading the main story}

\hypertarget{syrian-doctor-accused-of-torture-is-arrested-in-germany}{%
\section{Syrian Doctor Accused of Torture Is Arrested in
Germany}\label{syrian-doctor-accused-of-torture-is-arrested-in-germany}}

Alaa Mousa, who worked in a prison, is expected to be charged with
crimes against humanity. He is the third former Syrian official to be
detained in Germany.

\includegraphics{https://static01.nyt.com/images/2020/06/22/world/22germany-syria/22germany-syria-articleLarge.jpg?quality=75\&auto=webp\&disable=upscale}

\href{https://www.nytimes.com/by/christopher-f-schuetze}{\includegraphics{https://static01.nyt.com/images/2019/12/13/reader-center/author-christopher-f-schuetze/author-christopher-f-schuetze-thumbLarge.png}}\href{https://www.nytimes.com/by/ben-hubbard}{\includegraphics{https://static01.nyt.com/images/2018/10/10/multimedia/author-ben-hubbard/author-ben-hubbard-thumbLarge.png}}

By \href{https://www.nytimes.com/by/christopher-f-schuetze}{Christopher
F. Schuetze} and \href{https://www.nytimes.com/by/ben-hubbard}{Ben
Hubbard}

\begin{itemize}
\item
  June 22, 2020
\item
  \begin{itemize}
  \item
  \item
  \item
  \item
  \item
  \end{itemize}
\end{itemize}

BERLIN --- A Syrian doctor living in Germany has been arrested on
accusations that he tortured a detainee in a secret military prison in
his home country, the latest example of efforts to hold accountable
former Syrian officials who entered Germany as refugees.

The doctor, Alaa Mousa, is expected to face charges of crimes against
humanity and causing grievous bodily harm in a prison where he worked in
2011, the federal prosecutor said. He is the third former Syrian
official to be arrested in Germany on such charges, the prosecutor
added. The other two
\href{https://www.nytimes.com/2020/04/23/world/middleeast/syria-germany-war-crimes-trial.html}{went
on trial} in April.

The German authorities have been criticized for insufficiently vetting
the more than
\href{https://www.nytimes.com/2016/04/28/world/europe/germany-migrants-struggles-to-integrate.html}{one
million migrants who entered the country} during the refugee crisis of
2015 and 2016, many of them from Syria. While most of the Syrians were
fleeing the government of President Bashar al-Assad, others had served
in his military and security services.

Dr. Mousa, who was arrested on Friday, arrived in Germany in 2015 and
appears to have passed the country's
\href{https://www.nytimes.com/2018/09/08/world/europe/germany-refugees-doctors.html}{strict
re-certification procedure} to be allowed to practice medicine.

Other Syrian refugees found him working in a clinic near the central
city of Kassel and informed the German authorities of his background,
said Anwar al-Bunni, a Syrian human rights lawyer in Germany who helped
identify witnesses in the case.

Dr. Mousa, who was still practicing when he was arrested, is accused of
torturing a man detained during an antigovernment demonstration in
October 2011 in the Syrian city of Homs, according to
\href{https://www.generalbundesanwalt.de/SharedDocs/Pressemitteilungen/DE/aktuelle/Pressemitteilung-vom-22-06-2020.html;jsessionid=BA8A3B5E3F218468ECCE68602B83CC80.intranet251}{a
statement from the German federal prosecutor's office}.

After being tortured by prison officials, the man had an epileptic
attack, the statement said. Dr. Mousa was called to the scene and,
according to the statement, proceeded to beat the man to the ground with
a plastic pipe and then kick him. Dr. Mousa was called to treat the man
the next day, but instead beat him until he lost consciousness, the
statement said.

The prisoner later died in custody.

\includegraphics{https://static01.nyt.com/images/2020/06/22/world/22germany-syria2/merlin_159390768_c40f6381-c655-48d4-8c2f-61fd0f8e2896-articleLarge.jpg?quality=75\&auto=webp\&disable=upscale}

In Germany, the federal prosecutor's office and the federal police
compile witness testimonies and other evidence to create a broad picture
of
\href{https://www.nytimes.com/2016/08/27/world/middleeast/syria-civil-war-why-get-worse.html}{the
civil war in Syria}, which has killed more than half a million people
since it began in 2011.

Patrick Kroker, a human rights lawyer with the Berlin-based European
Center for Constitutional and Human Rights, an independent nonprofit,
said the inquiries had helped identify the roles of lower-level
operatives in Syria.

``The result of the strong and systematic effort by German authorities
to investigate structural crimes committed by the Syrian state is that
they have the kind of overview that allows cases like this to be
recognized and prosecuted,'' Mr. Kroker said.

Two other Syrians --- Anwar Raslan and Eyad al-Gharib, who
\href{https://www.nytimes.com/2020/04/23/world/middleeast/syria-germany-war-crimes-trial.html?searchResultPosition=2}{are
said to have worked in a secret prison in Syria's capital, Damascus} ---
were arrested last year and are on trial in Germany for crimes against
humanity. That trial was the world's first to deal with state-sponsored
torture in Syria, and Mr. Raslan --- a former colonel with Syria's
secret police --- is the highest-ranking former Syrian official to stand
trial outside Syria.

Mr. al-Bunni, the Syrian human rights lawyer, said of Dr. Mousa's
arrest, ``It is important to send a message that no one can hide from
justice, from the one who commits the smallest crime against humanity to
the largest.''

His nonprofit organization, the Syrian Center for Legal Studies and
Research, also helped track down witnesses for the trial of the two
other men and is investigating other cases in hopes that they will be
addressed by the German authorities.

Mr. Kroker, the German human rights lawyer, said he hoped the arrest
would lead to cases against higher-ranking senior Syrian officials.

``My hope,'' he said, ``is that it will eventually lead to the
investigation of the people actually responsible.''

Christopher F. Schuetze reported from Berlin, and Ben Hubbard from
Beirut, Lebanon.

Advertisement

\protect\hyperlink{after-bottom}{Continue reading the main story}

\hypertarget{site-index}{%
\subsection{Site Index}\label{site-index}}

\hypertarget{site-information-navigation}{%
\subsection{Site Information
Navigation}\label{site-information-navigation}}

\begin{itemize}
\tightlist
\item
  \href{https://help.nytimes.com/hc/en-us/articles/115014792127-Copyright-notice}{©~2020~The
  New York Times Company}
\end{itemize}

\begin{itemize}
\tightlist
\item
  \href{https://www.nytco.com/}{NYTCo}
\item
  \href{https://help.nytimes.com/hc/en-us/articles/115015385887-Contact-Us}{Contact
  Us}
\item
  \href{https://www.nytco.com/careers/}{Work with us}
\item
  \href{https://nytmediakit.com/}{Advertise}
\item
  \href{http://www.tbrandstudio.com/}{T Brand Studio}
\item
  \href{https://www.nytimes.com/privacy/cookie-policy\#how-do-i-manage-trackers}{Your
  Ad Choices}
\item
  \href{https://www.nytimes.com/privacy}{Privacy}
\item
  \href{https://help.nytimes.com/hc/en-us/articles/115014893428-Terms-of-service}{Terms
  of Service}
\item
  \href{https://help.nytimes.com/hc/en-us/articles/115014893968-Terms-of-sale}{Terms
  of Sale}
\item
  \href{https://spiderbites.nytimes.com}{Site Map}
\item
  \href{https://help.nytimes.com/hc/en-us}{Help}
\item
  \href{https://www.nytimes.com/subscription?campaignId=37WXW}{Subscriptions}
\end{itemize}
