Sections

SEARCH

\protect\hyperlink{site-content}{Skip to
content}\protect\hyperlink{site-index}{Skip to site index}

\href{https://myaccount.nytimes.com/auth/login?response_type=cookie\&client_id=vi}{}

\href{https://www.nytimes.com/section/todayspaper}{Today's Paper}

\href{/section/opinion}{Opinion}\textbar{}China and America Are Heading
Toward Divorce

\href{https://nyti.ms/2NqZVV7}{https://nyti.ms/2NqZVV7}

\begin{itemize}
\item
\item
\item
\item
\item
\item
\end{itemize}

Advertisement

\protect\hyperlink{after-top}{Continue reading the main story}

\href{/section/opinion}{Opinion}

Supported by

\protect\hyperlink{after-sponsor}{Continue reading the main story}

\hypertarget{china-and-america-are-heading-toward-divorce}{%
\section{China and America Are Heading Toward
Divorce}\label{china-and-america-are-heading-toward-divorce}}

For 40 years the two countries had an unconscious economic coupling.

\href{https://www.nytimes.com/by/thomas-l-friedman}{\includegraphics{https://static01.nyt.com/images/2018/04/02/opinion/thomas-l-friedman/thomas-l-friedman-thumbLarge.png}}

By \href{https://www.nytimes.com/by/thomas-l-friedman}{Thomas L.
Friedman}

Opinion Columnist

\begin{itemize}
\item
  June 23, 2020
\item
  \begin{itemize}
  \item
  \item
  \item
  \item
  \item
  \item
  \end{itemize}
\end{itemize}

\includegraphics{https://static01.nyt.com/images/2020/06/25/opinion/23friedmanWeb/merlin_159622032_2fad83a3-043b-4163-a64b-28bfaa376bea-articleLarge.jpg?quality=75\&auto=webp\&disable=upscale}

\href{https://cn.nytimes.com/opinion/20200624/china-united-states-trump/}{阅读简体中文版}\href{https://cn.nytimes.com/opinion/20200624/china-united-states-trump/zh-hant/}{閱讀繁體中文版}

My favorite story in John Bolton's book about the Trump Fun House ---
sorry, White House --- was that President Trump appealed to China's
leader to buy more U.S. agricultural products to boost Trump's farm vote
and his re-election.

Donald: Stop begging. Both Xi Jinping and Vladimir Putin have decided to
vote for you. \emph{Don't worry!}

They know that as long as you're president, America will be in turmoil.
For Xi, that means we're a less formidable economic rival, and for Vlad,
that means we're a less attractive democratic model for his people. They
also both know that as long as you're president the U.S. will never be
able to galvanize a global coalition of allies against them, which is
what China fears most on trade, human rights and Covid-19 and Russia on
Ukraine and Syria.

Don't take it from me. Here's what Zhou Xiaoming, a former Chinese trade
negotiator and deputy representative in Geneva,
\href{https://www.bloomberg.com/news/articles/2020-06-15/china-warms-to-idea-of-four-more-years-of-trump-presidency}{told
Bloomberg's Peter Martin}: ``If Biden is elected, I think this could be
more dangerous for China, because he will work with allies to target
China, whereas Trump is destroying U.S. alliances.''

Chinese officials, Martin reported, see a unified front on trade or
human rights by the U.S. and its allies as ``Washington's greatest asset
for checking China's widening influence,'' and Trump's behavior ensures
that will never come about.

But while China may think it has nothing to fear and much to gain from a
Trump victory over Joe Biden, the real U.S.-China story should be cause
for alarm in Beijing.

The real story is that China's standing in America today is lower than
at any time since Tiananmen Square in 1989. The real story is that if
China was to buy a few more beans and Boeings from America, that would
not fix Beijing's problems here. The real question the Chinese should be
asking themselves is not who will be America's next president, but
rather: ``Who in China lost America?''

Because the real story is that the U.S. and China are heading for a
divorce.

The divorce papers will just say the cause was ``irreconcilable
differences.'' But Mom and Dad know better. They are getting divorced,
after 40 years of being one couple, two systems, because China
\emph{badly overreached} and America \emph{badly underperformed.}

Love it or hate it, the U.S.-China partnership forged between 1979 and
2019 delivered a lot of prosperity to a lot of people and a lot of
relative peace to the world --- and, baby, we will miss it when it's
gone.

It was a period of unconscious economic coupling. **** Steadily over
this era, and then rapidly after China joined the World Trade
Organization in 2001, any America entrepreneur could wake up and say,
``I want to purchase from this Chinese company'' or ``I want to move
this supply chain to China.'' Any U.S. university could say, ``I want to
open a campus in China,'' and any U.S. tech company could say, ``I want
to open a research lab in China or hire a Chinese scientist.''

And any Chinese student could say, ``I want to study in America,'' and
any Chinese company that qualified could say, ``I want to list on the
New York Stock Exchange'' or ``invest in or buy an American company.''

These four decades of unconscious coupling hurt some workers, benefited
many others and especially benefited consumers; it also took the edge
off the natural rivalry between the world's most powerful country and
the most important rising power and enabled them to collaborate on
global problems, like climate change and the post-2008 economic crisis.

This 40 years of unconscious coupling is over. We will still trade,
still engage diplomatically; tourists will still come and go; U.S.
businesses will still look to operate in the giant China market, because
they must to survive.

But the unconscious coupling is over. Henceforth, it will be more
hedged, opportunities will be more restricted and the relationship will
be full of a lot more conscious suspicion, pressures for
self-sufficiency and fear that a rupture could happen at any time.

Compared with the last 40 years, \emph{it will feel like a divorce.}

``Both sides are saying, `We've had enough of you,'' remarked Jim
McGregor, chairman of APCO Worldwide for Greater China. And as Trump
himself put it in a tweet last week, the U.S. has the option ``of a
complete decoupling from China.''

But both sides are not equally to blame. The Xi era in U.S.-China
relations, which began in 2012, has led the relationship steadily
downhill. China went too far on a broad range of issues.

Start with business. For many years U.S. companies thought they had
enough market share inside China that they would tolerate the stealing
of intellectual property and other trade abuses China engaged in. But in
the last decade, China started to overreach, and the American Chamber of
Commerce in China began to complain louder and louder. Gradually, many
in the U.S. business community, which was a key buffer in the
relationship, began to endorse Donald Trump's hard-line approach
(although they don't like paying tariffs).

Since Xi took power and made himself effectively president for life and
tightened the Communist Party's control over all matters, U.S.
journalists working in China have had their access sharply curtailed;
China has become more aggressive in projecting its power into the South
China Sea; it's become more fixated on subsidizing its high-tech
start-ups to dominate key industries by 2025; it is imposing a new
national security law to curtail longstanding freedoms in Hong Kong;
it's stepped up its bullying of Taiwan, taken a very aggressive approach
toward India and intensified its internment of Uighur Muslims in
Xinjiang; it's
\href{https://www.theglobeandmail.com/canada/article-something-has-to-change-michael-kovrigs-letters-detail-life-in-a/}{jailed
two innocent Canadians} to swap for a detained Chinese businesswoman;
and it even hammered countries that dared to ask for an independent
inquiry into how the coronavirus emerged in Wuhan.

After Australia's prime minister called for such an investigation in
April, China's ambassador to Australia brazenly threatened economic
retaliation, and a few weeks later China cut off beef and barley imports
from Australian companies, citing bogus health and trade violations.

That is the kind of crude bullying that has helped to strip China of
virtually every ally it had in Washington --- allies for a policy that
basically said, ``We have different systems, but let's build bridges
with China where possible, engage where it is mutually beneficial and
draw redlines where necessary.''

That balanced policy approach always had to contain serious tensions,
ugliness and disagreements on issues --- but in the end it delivered
enough mutual benefit to be sustained for 40 years. That balance is now
off as far as many Americans are concerned. I am one of them.

As Orville Schell, one of the most sensible advocates of this balanced
approach,
\href{https://www.thewirechina.com/2020/06/07/the-birth-life-and-death-of-engagement/}{wrote
in an essay} a few weeks ago on TheWireChina.com: ``Today, as the U.S.
faces its most adversarial state with the People's Republic of China in
years, the always fragile policy framework of engagement feels like a
burnt-out case. \ldots{} A
\href{https://www.pewresearch.org/global/2020/04/21/u-s-views-of-china-increasingly-negative-amid-coronavirus-outbreak/}{recent
Pew poll} shows that only 26 percent of Americans view China favorably,
the lowest percentage since its surveys began in 2005.''

But if China has increasingly overreached, America has increasingly
underperformed.

It is not just that China reportedly has fewer than 5,000 Covid-19
deaths and America has over 120,000 --- and the virus started there. It
is not just that it takes about 22 hours on Amtrak to go from New York
to Chicago, while it takes 4.5 hours to take the bullet train from
Beijing to Shanghai, slightly farther apart. It's not just that the
pandemic has accelerated China's transformation to a cashless, digital
society.

It's that we have reduced investments in the true sources of our
strength --- infrastructure, education, government-funded scientific
research, immigration and the right rules to incentivize productive
investment and prevent excessive risk-taking. And we have stopped
leveraging our greatest advantage over China --- that we have allies who
share our values and China only has customers who fear its wrath.

If we got together with our allies, we could collectively influence
China to accept new rules on trade and Covid-19 and a range of other
issues. But Trump refused to do so, making everything a bilateral deal
or a fight with Xi. So now China is offering sweetheart deals to U.S.
and other foreign companies to come into or stay in China, and its
market is now so big, few companies can resist.

Summing up the relationship today, McGregor, of APCO Worldwide, noted:
``I don't know if the Chinese are taking America seriously anymore. They
are happy to just let us keep damaging ourselves. We have to wake up and
grow up'' --- and get our own act and allies together. China respects
one thing only: leverage. Today, we have too little and China has too
much.

\emph{The Times is committed to publishing}
\href{https://www.nytimes.com/2019/01/31/opinion/letters/letters-to-editor-new-york-times-women.html}{\emph{a
diversity of letters}} \emph{to the editor. We'd like to hear what you
think about this or any of our articles. Here are some}
\href{https://help.nytimes.com/hc/en-us/articles/115014925288-How-to-submit-a-letter-to-the-editor}{\emph{tips}}\emph{.
And here's our email:}
\href{mailto:letters@nytimes.com}{\emph{letters@nytimes.com}}\emph{.}

\emph{Follow The New York Times Opinion section on}
\href{https://www.facebook.com/nytopinion}{\emph{Facebook}}\emph{,}
\href{http://twitter.com/NYTOpinion}{\emph{Twitter (@NYTopinion)}}
\emph{and}
\href{https://www.instagram.com/nytopinion/}{\emph{Instagram}}\emph{.}

Advertisement

\protect\hyperlink{after-bottom}{Continue reading the main story}

\hypertarget{site-index}{%
\subsection{Site Index}\label{site-index}}

\hypertarget{site-information-navigation}{%
\subsection{Site Information
Navigation}\label{site-information-navigation}}

\begin{itemize}
\tightlist
\item
  \href{https://help.nytimes.com/hc/en-us/articles/115014792127-Copyright-notice}{©~2020~The
  New York Times Company}
\end{itemize}

\begin{itemize}
\tightlist
\item
  \href{https://www.nytco.com/}{NYTCo}
\item
  \href{https://help.nytimes.com/hc/en-us/articles/115015385887-Contact-Us}{Contact
  Us}
\item
  \href{https://www.nytco.com/careers/}{Work with us}
\item
  \href{https://nytmediakit.com/}{Advertise}
\item
  \href{http://www.tbrandstudio.com/}{T Brand Studio}
\item
  \href{https://www.nytimes.com/privacy/cookie-policy\#how-do-i-manage-trackers}{Your
  Ad Choices}
\item
  \href{https://www.nytimes.com/privacy}{Privacy}
\item
  \href{https://help.nytimes.com/hc/en-us/articles/115014893428-Terms-of-service}{Terms
  of Service}
\item
  \href{https://help.nytimes.com/hc/en-us/articles/115014893968-Terms-of-sale}{Terms
  of Sale}
\item
  \href{https://spiderbites.nytimes.com}{Site Map}
\item
  \href{https://help.nytimes.com/hc/en-us}{Help}
\item
  \href{https://www.nytimes.com/subscription?campaignId=37WXW}{Subscriptions}
\end{itemize}
