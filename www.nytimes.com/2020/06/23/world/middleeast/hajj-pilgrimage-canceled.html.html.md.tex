Sections

SEARCH

\protect\hyperlink{site-content}{Skip to
content}\protect\hyperlink{site-index}{Skip to site index}

\href{https://www.nytimes.com/section/world/middleeast}{Middle East}

\href{https://myaccount.nytimes.com/auth/login?response_type=cookie\&client_id=vi}{}

\href{https://www.nytimes.com/section/todayspaper}{Today's Paper}

\href{/section/world/middleeast}{Middle East}\textbar{}The Hajj
Pilgrimage Is Canceled, and Grief Rocks the Muslim World

\url{https://nyti.ms/2NqCI5I}

\begin{itemize}
\item
\item
\item
\item
\item
\end{itemize}

\href{https://www.nytimes.com/news-event/coronavirus?action=click\&pgtype=Article\&state=default\&region=TOP_BANNER\&context=storylines_menu}{The
Coronavirus Outbreak}

\begin{itemize}
\tightlist
\item
  live\href{https://www.nytimes.com/2020/08/04/world/coronavirus-cases.html?action=click\&pgtype=Article\&state=default\&region=TOP_BANNER\&context=storylines_menu}{Latest
  Updates}
\item
  \href{https://www.nytimes.com/interactive/2020/us/coronavirus-us-cases.html?action=click\&pgtype=Article\&state=default\&region=TOP_BANNER\&context=storylines_menu}{Maps
  and Cases}
\item
  \href{https://www.nytimes.com/interactive/2020/science/coronavirus-vaccine-tracker.html?action=click\&pgtype=Article\&state=default\&region=TOP_BANNER\&context=storylines_menu}{Vaccine
  Tracker}
\item
  \href{https://www.nytimes.com/2020/08/02/us/covid-college-reopening.html?action=click\&pgtype=Article\&state=default\&region=TOP_BANNER\&context=storylines_menu}{College
  Reopening}
\item
  \href{https://www.nytimes.com/live/2020/08/04/business/stock-market-today-coronavirus?action=click\&pgtype=Article\&state=default\&region=TOP_BANNER\&context=storylines_menu}{Economy}
\end{itemize}

Advertisement

\protect\hyperlink{after-top}{Continue reading the main story}

Supported by

\protect\hyperlink{after-sponsor}{Continue reading the main story}

\hypertarget{the-hajj-pilgrimage-is-canceled-and-grief-rocks-the-muslim-world}{%
\section{The Hajj Pilgrimage Is Canceled, and Grief Rocks the Muslim
World}\label{the-hajj-pilgrimage-is-canceled-and-grief-rocks-the-muslim-world}}

The coronavirus pandemic upended the plans of millions of Muslims, for
whom the once-in-a-lifetime trip is a sacred milestone.

\includegraphics{https://static01.nyt.com/images/2020/04/08/world/23hajj/merlin_171251070_51c3b7df-fb13-432c-b298-ff6b6c476584-articleLarge.jpg?quality=75\&auto=webp\&disable=upscale}

\href{https://www.nytimes.com/by/ben-hubbard}{\includegraphics{https://static01.nyt.com/images/2018/10/10/multimedia/author-ben-hubbard/author-ben-hubbard-thumbLarge.png}}\href{https://www.nytimes.com/by/declan-walsh}{\includegraphics{https://static01.nyt.com/images/2018/10/15/multimedia/author-declan-walsh/author-declan-walsh-thumbLarge-v3.png}}

By \href{https://www.nytimes.com/by/ben-hubbard}{Ben Hubbard} and
\href{https://www.nytimes.com/by/declan-walsh}{Declan Walsh}

\begin{itemize}
\item
  June 23, 2020
\item
  \begin{itemize}
  \item
  \item
  \item
  \item
  \item
  \end{itemize}
\end{itemize}

BEIRUT, Lebanon --- For much of his life, Abdul-Halim al-Akoum stashed
away cash in hopes of one day traveling from his Lebanese mountain
village to perform the hajj, the pilgrimage to Mecca that all Muslims
who can are obliged to make once in their lives.

He was all set to go this year until the coronavirus pandemic forced
Saudi Arabia to effectively cancel the hajj for what some scholars say
may be the first time in history.

``It is the dream of every Muslim believer to visit Mecca and do the
hajj,'' said Mr. al-Akoum, 61, a village official. ``But the pandemic
came with no warning and took away that dream.''

The Saudi announcement sent shock waves of sadness and disappointment
across the Muslim world, upending the plans of millions of believers to
make a trip that many look forward to their whole lives and which, for
many, marks a profound spiritual awakening.

A 72-year-old retired port worker in Pakistan will stay home, despite
his six children having pooled their money to finance his trip. A mother
in Kenya will forgo visiting sites she has long dreamed of seeing. An
Egyptian school administrator named Zeinab Ibrahim burst into tears.

``It was my only wish,'' Ms. Ibrahim said. ``To cancel it completely is
such a shame. May God relieve us of this burden.''

Performing the pilgrimage at least once for those who are physically and
financially able is one of the five pillars of Islam. Making the trip is
such a sacred milestone for the world's 1.8 billion Muslims that in
parts of the Arab world families of returned pilgrims paint murals on
their homes to alert their neighbors to the pilgrim in their midst.

\includegraphics{https://static01.nyt.com/images/2020/06/23/world/23hajj2/merlin_142565523_ce5ca7be-4b4f-4a6a-956d-4e0d39d34ea9-articleLarge.jpg?quality=75\&auto=webp\&disable=upscale}

Many people save up their entire lives to make the hajj and, before
modern transportation, spent months getting there.

The pilgrimage conveys such religious status that many Muslims add the
honorific ``al-Hajj'' or ``Hajji'' to their names on their business
cards.

\hypertarget{latest-updates-global-coronavirus-outbreak}{%
\section{\texorpdfstring{\href{https://www.nytimes.com/2020/08/04/world/coronavirus-cases.html?action=click\&pgtype=Article\&state=default\&region=MAIN_CONTENT_1\&context=storylines_live_updates}{Latest
Updates: Global Coronavirus
Outbreak}}{Latest Updates: Global Coronavirus Outbreak}}\label{latest-updates-global-coronavirus-outbreak}}

Updated 2020-08-04T20:42:41.838Z

\begin{itemize}
\tightlist
\item
  \href{https://www.nytimes.com/2020/08/04/world/coronavirus-cases.html?action=click\&pgtype=Article\&state=default\&region=MAIN_CONTENT_1\&context=storylines_live_updates\#link-1228a480}{Novavax
  sees encouraging results from two studies of its experimental
  vaccine.}
\item
  \href{https://www.nytimes.com/2020/08/04/world/coronavirus-cases.html?action=click\&pgtype=Article\&state=default\&region=MAIN_CONTENT_1\&context=storylines_live_updates\#link-4825b93}{Public
  and private schools in Maryland and elsewhere are divided over
  in-person instruction.}
\item
  \href{https://www.nytimes.com/2020/08/04/world/coronavirus-cases.html?action=click\&pgtype=Article\&state=default\&region=MAIN_CONTENT_1\&context=storylines_live_updates\#link-50f7386d}{The
  United Nations calls on policymakers to `plan thoroughly for school
  reopenings.'}
\end{itemize}

\href{https://www.nytimes.com/2020/08/04/world/coronavirus-cases.html?action=click\&pgtype=Article\&state=default\&region=MAIN_CONTENT_1\&context=storylines_live_updates}{See
more updates}

More live coverage:
\href{https://www.nytimes.com/live/2020/08/04/business/stock-market-today-coronavirus?action=click\&pgtype=Article\&state=default\&region=MAIN_CONTENT_1\&context=storylines_live_updates}{Markets}

``The hajj is a transformative, emotional and spiritually moving
experience --- the spiritual pinnacle of a devout Muslim's life,'' said
Yasir Qadhi, dean of the Islamic Seminary of America, who was supposed
to lead a group of 250 pilgrims to Mecca this year.

Since the Saudi announcement, he added, ``There's a sense of deflation
and spiritual loss, and a great sadness.''

The hajj is also big business. The hajj, a five- or six-day pilgrimage
that starts this year at the end of July, and the umrah, a lesser
pilgrimage that can be performed at any time of the year, earn Saudi
Arabia billions of dollars each year, and Muslim communities from Texas
to Tajikistan have travel agencies specializing in getting pilgrims to
and from the holy sites and providing accommodation along the way.

``It is a catastrophe on all levels --- economic, social and
religious,'' said Tariq Kalach, who runs a Beirut travel agency that was
planning to take 400 pilgrims to Mecca this year.

Pilgrimage packages cost from \$3,000 to \$10,000, he said. He also
provides services to a number of Islamic associations that pay for
groups of poor Muslims to make the trip each year.

Image

Pilgrims touching the side of the Kaaba, Islam's holiest shrine, during
the hajj last year.Credit...Fethi Belaid/Agence France-Presse --- Getty
Images

He said the cancellation was devastating, but that it was the right
thing to do.

``It is a very dangerous virus and it will spread like a brush fire,''
he said. ``May the almighty make things easy for the Muslims.''

The Saudi government, for which the hajj is a major source of prestige
and tourism,
\href{https://www.nytimes.com/2020/06/22/world/middleeast/saudi-arabia-hajj-mecca-pilgrims.html}{announced
Monday} that no pilgrims from outside the kingdom could perform the hajj
this year in order to prevent contagion.

On Tuesday, Saudi officials narrowed the order, saying that only about
1,000 pilgrims would be permitted this year --- a tiny fraction of the
2.5 million who came last year.

The pilgrimage has been interrupted or curtailed many times because of
wars and disease, but has faced no significant limits on attendance
since the mid-1800s, when outbreaks of cholera and plague kept pilgrims
away for a number of years.

Saudi Arabia, whose king bears the title ``the custodian of the two holy
mosques,'' a reference to holy sites in Mecca and Medina, has never
canceled the hajj since the modern kingdom was founded in 1932.

``This is the first time in the global phenomenon of the hajj that it
has been canceled in such a manner,'' said Dr. Qadhi, the scholar. ``The
dynamics have changed. Five hundred years ago you couldn't ban it. There
were no passports, no visas.''

The Mongol invasion of the Levant in the 13th century, for example,
prevented pilgrims from reaching Mecca, he said, ``but even then, the
locals did it.''

Few criticized the decision to limit the event since Saudi Arabia is
suffering from one of the largest coronavirus outbreaks in the Middle
East, with 161,000 declared infections and more than 1,300 deaths.
Epidemiologists have warned that mass gatherings --- from concerts to
sporting matches --- can become so-called super-spreader events.

Khalid Almaeena, a Saudi political and media analyst who has attended
the hajj many times, said that much of the pilgrimage's importance comes
from the way it mixes Muslims from different countries, races and social
classes who might not otherwise cross paths.

``This is the religious, social, cultural aspect of the hajj,'' he said.
``It is not just the ritual, but the meeting places, the many great
friendships and bonds that are established and built there year after
year.''

Image

Disinfecting the floor near the Kaaba in the Grand Mosque in
March.Credit...Reuters

In Egypt, the economic hardship of recent years has turned the hajj into
an elusive dream for many, which only sharpened the blow of the
cancellation.

Ms. Ibrahim, the school administrator, applied four years in a row to a
government lottery that offers free trips to the hajj, failing every
time. But this year, she scraped together the cost from her own funds.
``I wanted to go while my health is still good,'' said Ms. Ibrahim, 58,
who earns about \$175 a month. ``I didn't care about the cost.''

\href{https://www.nytimes.com/news-event/coronavirus?action=click\&pgtype=Article\&state=default\&region=MAIN_CONTENT_3\&context=storylines_faq}{}

\hypertarget{the-coronavirus-outbreak-}{%
\subsubsection{The Coronavirus Outbreak
›}\label{the-coronavirus-outbreak-}}

\hypertarget{frequently-asked-questions}{%
\paragraph{Frequently Asked
Questions}\label{frequently-asked-questions}}

Updated August 4, 2020

\begin{itemize}
\item ~
  \hypertarget{i-have-antibodies-am-i-now-immune}{%
  \paragraph{I have antibodies. Am I now
  immune?}\label{i-have-antibodies-am-i-now-immune}}

  \begin{itemize}
  \tightlist
  \item
    As of right
    now,\href{https://www.nytimes.com/2020/07/22/health/covid-antibodies-herd-immunity.html?action=click\&pgtype=Article\&state=default\&region=MAIN_CONTENT_3\&context=storylines_faq}{that
    seems likely, for at least several months.} There have been
    frightening accounts of people suffering what seems to be a second
    bout of Covid-19. But experts say these patients may have a
    drawn-out course of infection, with the virus taking a slow toll
    weeks to months after initial exposure. People infected with the
    coronavirus typically
    \href{https://www.nature.com/articles/s41586-020-2456-9}{produce}
    immune molecules called antibodies, which are
    \href{https://www.nytimes.com/2020/05/07/health/coronavirus-antibody-prevalence.html?action=click\&pgtype=Article\&state=default\&region=MAIN_CONTENT_3\&context=storylines_faq}{protective
    proteins made in response to an
    infection}\href{https://www.nytimes.com/2020/05/07/health/coronavirus-antibody-prevalence.html?action=click\&pgtype=Article\&state=default\&region=MAIN_CONTENT_3\&context=storylines_faq}{.
    These antibodies may} last in the body
    \href{https://www.nature.com/articles/s41591-020-0965-6}{only two to
    three months}, which may seem worrisome, but that's perfectly normal
    after an acute infection subsides, said Dr. Michael Mina, an
    immunologist at Harvard University. It may be possible to get the
    coronavirus again, but it's highly unlikely that it would be
    possible in a short window of time from initial infection or make
    people sicker the second time.
  \end{itemize}
\item ~
  \hypertarget{im-a-small-business-owner-can-i-get-relief}{%
  \paragraph{I'm a small-business owner. Can I get
  relief?}\label{im-a-small-business-owner-can-i-get-relief}}

  \begin{itemize}
  \tightlist
  \item
    The
    \href{https://www.nytimes.com/article/small-business-loans-stimulus-grants-freelancers-coronavirus.html?action=click\&pgtype=Article\&state=default\&region=MAIN_CONTENT_3\&context=storylines_faq}{stimulus
    bills enacted in March} offer help for the millions of American
    small businesses. Those eligible for aid are businesses and
    nonprofit organizations with fewer than 500 workers, including sole
    proprietorships, independent contractors and freelancers. Some
    larger companies in some industries are also eligible. The help
    being offered, which is being managed by the Small Business
    Administration, includes the Paycheck Protection Program and the
    Economic Injury Disaster Loan program. But lots of folks have
    \href{https://www.nytimes.com/interactive/2020/05/07/business/small-business-loans-coronavirus.html?action=click\&pgtype=Article\&state=default\&region=MAIN_CONTENT_3\&context=storylines_faq}{not
    yet seen payouts.} Even those who have received help are confused:
    The rules are draconian, and some are stuck sitting on
    \href{https://www.nytimes.com/2020/05/02/business/economy/loans-coronavirus-small-business.html?action=click\&pgtype=Article\&state=default\&region=MAIN_CONTENT_3\&context=storylines_faq}{money
    they don't know how to use.} Many small-business owners are getting
    less than they expected or
    \href{https://www.nytimes.com/2020/06/10/business/Small-business-loans-ppp.html?action=click\&pgtype=Article\&state=default\&region=MAIN_CONTENT_3\&context=storylines_faq}{not
    hearing anything at all.}
  \end{itemize}
\item ~
  \hypertarget{what-are-my-rights-if-i-am-worried-about-going-back-to-work}{%
  \paragraph{What are my rights if I am worried about going back to
  work?}\label{what-are-my-rights-if-i-am-worried-about-going-back-to-work}}

  \begin{itemize}
  \tightlist
  \item
    Employers have to provide
    \href{https://www.osha.gov/SLTC/covid-19/standards.html}{a safe
    workplace} with policies that protect everyone equally.
    \href{https://www.nytimes.com/article/coronavirus-money-unemployment.html?action=click\&pgtype=Article\&state=default\&region=MAIN_CONTENT_3\&context=storylines_faq}{And
    if one of your co-workers tests positive for the coronavirus, the
    C.D.C.} has said that
    \href{https://www.cdc.gov/coronavirus/2019-ncov/community/guidance-business-response.html}{employers
    should tell their employees} -\/- without giving you the sick
    employee's name -\/- that they may have been exposed to the virus.
  \end{itemize}
\item ~
  \hypertarget{should-i-refinance-my-mortgage}{%
  \paragraph{Should I refinance my
  mortgage?}\label{should-i-refinance-my-mortgage}}

  \begin{itemize}
  \tightlist
  \item
    \href{https://www.nytimes.com/article/coronavirus-money-unemployment.html?action=click\&pgtype=Article\&state=default\&region=MAIN_CONTENT_3\&context=storylines_faq}{It
    could be a good idea,} because mortgage rates have
    \href{https://www.nytimes.com/2020/07/16/business/mortgage-rates-below-3-percent.html?action=click\&pgtype=Article\&state=default\&region=MAIN_CONTENT_3\&context=storylines_faq}{never
    been lower.} Refinancing requests have pushed mortgage applications
    to some of the highest levels since 2008, so be prepared to get in
    line. But defaults are also up, so if you're thinking about buying a
    home, be aware that some lenders have tightened their standards.
  \end{itemize}
\item ~
  \hypertarget{what-is-school-going-to-look-like-in-september}{%
  \paragraph{What is school going to look like in
  September?}\label{what-is-school-going-to-look-like-in-september}}

  \begin{itemize}
  \tightlist
  \item
    It is unlikely that many schools will return to a normal schedule
    this fall, requiring the grind of
    \href{https://www.nytimes.com/2020/06/05/us/coronavirus-education-lost-learning.html?action=click\&pgtype=Article\&state=default\&region=MAIN_CONTENT_3\&context=storylines_faq}{online
    learning},
    \href{https://www.nytimes.com/2020/05/29/us/coronavirus-child-care-centers.html?action=click\&pgtype=Article\&state=default\&region=MAIN_CONTENT_3\&context=storylines_faq}{makeshift
    child care} and
    \href{https://www.nytimes.com/2020/06/03/business/economy/coronavirus-working-women.html?action=click\&pgtype=Article\&state=default\&region=MAIN_CONTENT_3\&context=storylines_faq}{stunted
    workdays} to continue. California's two largest public school
    districts --- Los Angeles and San Diego --- said on July 13, that
    \href{https://www.nytimes.com/2020/07/13/us/lausd-san-diego-school-reopening.html?action=click\&pgtype=Article\&state=default\&region=MAIN_CONTENT_3\&context=storylines_faq}{instruction
    will be remote-only in the fall}, citing concerns that surging
    coronavirus infections in their areas pose too dire a risk for
    students and teachers. Together, the two districts enroll some
    825,000 students. They are the largest in the country so far to
    abandon plans for even a partial physical return to classrooms when
    they reopen in August. For other districts, the solution won't be an
    all-or-nothing approach.
    \href{https://bioethics.jhu.edu/research-and-outreach/projects/eschool-initiative/school-policy-tracker/}{Many
    systems}, including the nation's largest, New York City, are
    devising
    \href{https://www.nytimes.com/2020/06/26/us/coronavirus-schools-reopen-fall.html?action=click\&pgtype=Article\&state=default\&region=MAIN_CONTENT_3\&context=storylines_faq}{hybrid
    plans} that involve spending some days in classrooms and other days
    online. There's no national policy on this yet, so check with your
    municipal school system regularly to see what is happening in your
    community.
  \end{itemize}
\end{itemize}

In many countries, even those who can muster the expense often wait
years to be included in their country's quota of pilgrims, which are set
by Saudi Arabia with the aim of equalizing the opportunity across the
Muslim world.

Imam Mokhi Turk, 45, said that 15 people from his embattled farming
village in Kunduz Province, Afghanistan, had been waiting for their turn
to do the hajj and that some of his neighbors had sold land to pay for
it.

Mr. Turk and four of his relatives registered for the pilgrimage four
years ago, but only he made the list this year.

``This makes me very sad, because every Muslim hopes to go to hajj once
in his whole life, and when it was my turn, it was canceled,'' Mr. Turk
said. ``I'm very upset because I'm not sure if I'll be alive in the next
few days, let alone next year.''

Image

Praying near Al Safa mountain in the Grand Mosque during the 2019
hajj.Credit...Amr Nabil/Associated Press

Since the first hajj in 632, Muslims have traveled to Mecca in the face
of hardship, adversity and disasters, gradually transforming the
pilgrimage from an elite pursuit limited to small numbers of people into
one of the world's largest Muslim gatherings.

For centuries, it was a feat just to make it to Mecca in one piece.

Under the Ottoman Empire, camel-riding pilgrims crossed the vast deserts
of Arabia in giant caravans that set out from Cairo or Damascus in a
journey often taking six weeks and vulnerable to attacks by Bedouin
bandits.

Others came by sea, braving storms, disease outbreaks in crowded ships,
and other threats. In 1502, the Portuguese explorer Vasco da Gama,
battling for control of trade routes, captured a ship filled with
pilgrims as it returned from Mecca, set it on fire and killed several
hundred people. In the 19th century, periodic cholera epidemics killed
thousands of pilgrims.

The Suez Canal shortened the sea voyage for many after it opened in
1869, and the advent of motor vehicles eased the land voyage starting in
the 1920s. Even then, numbers remained low: The hajj of 1929 registered
66,000 pilgrims.

The numbers started soaring in the 1970s, as mass air travel became more
affordable, and Saudi rulers recognized that the pilgrimage brought not
just religious prestige but also income. The hajj currently earns the
kingdom billions of dollars a year.

Image

Pilgrims arrive in Mecca for the hajj in 1968.Credit...Terry
Fincher/Daily Express, via Getty Images

Since the 1990s, the pilgrimage has been marred by stampedes, giant tent
fires and worries about outbreaks of diseases such as SARS or, more
recently, MERS. The deadliest stampede occurred in 2015 when
\href{https://www.nytimes.com/2015/12/11/world/middleeast/death-toll-from-hajj-stampede.html}{more
than 2,200 people died}.

Despite the periodic tragedies, the Saudi authorities never canceled it.

The cancellation weighs particularly heavily on older Muslims who have
been waiting for years to go in hopes that they can fulfill their
religious obligation before death.

``I have been dreaming about it for 20 years and I hoped to do it before
I got this old,'' said Firiyan al-Masri, 68, a woman from Beirut.

Finally this year, she got her name on the list of a Lebanese Islamic
association that finances trips for those in need, only to see her
chances dashed by the pandemic.

``If God wills it, I will do the pilgrimage next year,'' she said. ``If
I am still alive.''

Ben Hubbard reported from Beirut, and Declan Walsh from Cairo. Reporting
was contributed by Abdi Latif Dahir from Nairobi, Kenya; Nada Rashwan
from Cairo; Nahim Rahim and Fahim Abed from Kabul, Afghanistan; Zia ur
Rehman from Karachi, Pakistan; and Ismail Khan in Peshawar, Pakistan.

Advertisement

\protect\hyperlink{after-bottom}{Continue reading the main story}

\hypertarget{site-index}{%
\subsection{Site Index}\label{site-index}}

\hypertarget{site-information-navigation}{%
\subsection{Site Information
Navigation}\label{site-information-navigation}}

\begin{itemize}
\tightlist
\item
  \href{https://help.nytimes.com/hc/en-us/articles/115014792127-Copyright-notice}{©~2020~The
  New York Times Company}
\end{itemize}

\begin{itemize}
\tightlist
\item
  \href{https://www.nytco.com/}{NYTCo}
\item
  \href{https://help.nytimes.com/hc/en-us/articles/115015385887-Contact-Us}{Contact
  Us}
\item
  \href{https://www.nytco.com/careers/}{Work with us}
\item
  \href{https://nytmediakit.com/}{Advertise}
\item
  \href{http://www.tbrandstudio.com/}{T Brand Studio}
\item
  \href{https://www.nytimes.com/privacy/cookie-policy\#how-do-i-manage-trackers}{Your
  Ad Choices}
\item
  \href{https://www.nytimes.com/privacy}{Privacy}
\item
  \href{https://help.nytimes.com/hc/en-us/articles/115014893428-Terms-of-service}{Terms
  of Service}
\item
  \href{https://help.nytimes.com/hc/en-us/articles/115014893968-Terms-of-sale}{Terms
  of Sale}
\item
  \href{https://spiderbites.nytimes.com}{Site Map}
\item
  \href{https://help.nytimes.com/hc/en-us}{Help}
\item
  \href{https://www.nytimes.com/subscription?campaignId=37WXW}{Subscriptions}
\end{itemize}
