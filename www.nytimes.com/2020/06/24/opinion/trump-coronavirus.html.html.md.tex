Sections

SEARCH

\protect\hyperlink{site-content}{Skip to
content}\protect\hyperlink{site-index}{Skip to site index}

\href{https://myaccount.nytimes.com/auth/login?response_type=cookie\&client_id=vi}{}

\href{https://www.nytimes.com/section/todayspaper}{Today's Paper}

\href{/section/opinion}{Opinion}\textbar{}Trump Is Feeding America's
Coronavirus Nightmare

\href{https://nyti.ms/2ZcOnuq}{https://nyti.ms/2ZcOnuq}

\begin{itemize}
\item
\item
\item
\item
\item
\item
\end{itemize}

Advertisement

\protect\hyperlink{after-top}{Continue reading the main story}

\href{/section/opinion}{Opinion}

Supported by

\protect\hyperlink{after-sponsor}{Continue reading the main story}

\hypertarget{trump-is-feeding-americas-coronavirus-nightmare}{%
\section{Trump Is Feeding America's Coronavirus
Nightmare}\label{trump-is-feeding-americas-coronavirus-nightmare}}

``It's going away'' only in the president's delusion.

\href{https://www.nytimes.com/column/nicholas-kristof}{\includegraphics{https://static01.nyt.com/images/2018/04/03/opinion/nicholas-kristof/nicholas-kristof-thumbLarge-v2.png}}

By \href{https://www.nytimes.com/column/nicholas-kristof}{Nicholas
Kristof}

Opinion Columnist

\begin{itemize}
\item
  June 24, 2020
\item
  \begin{itemize}
  \item
  \item
  \item
  \item
  \item
  \item
  \end{itemize}
\end{itemize}

\hypertarget{listen-to-this-op-ed}{%
\subsubsection{Listen to This Op-Ed}\label{listen-to-this-op-ed}}

Audio Recording by Audm

\emph{To hear more audio stories from publishers like The New York
Times, download}
\href{https://www.audm.com/?utm_source=nytmag\&utm_medium=embed\&utm_campaign=left_behind_draper}{**}
\href{https://www.audm.com/?utm_source=nytopinion\&utm_medium=embed\&utm_campaign=trump_coronavirus_nightmare}{\emph{Audm
for iPhone or Android}}\emph{.}

President Trump says the coronavirus is
``\href{https://time.com/5855541/trump-coronavirus-fade-away/}{fading
away}'' and pats himself on the back for
``\href{https://twitter.com/realDonaldTrump/status/1275438921552257026}{a
great job} on CoronaVirus'' that saved ``millions of U.S. lives.''

``It's going away,'' Trump
\href{https://www.politico.com/news/2020/06/23/trump-rally-arizona-336565}{said}
Tuesday at a packed megachurch in Phoenix where few people wore masks.

That's what delusion sounds like. We need a Churchill to lead our nation
against a deadly challenge; instead, we have a president who helps an
enemy virus infiltrate our churches and homes. Churchill and Roosevelt
worked to deceive the enemy; Trump is trying to deceive us.

For a reality check, look at this map by my colleague Nathaniel Lash
showing how much of America is trending in the wrong direction.

\hypertarget{covid-19-cases-rose-during-june-in-most-states}{%
\subsection{Covid-19 cases rose during June in most
states}\label{covid-19-cases-rose-during-june-in-most-states}}

The hardest-hit states in the Northeast and Midwest saw declines, but
new cases in the rest of the country were on the rise in June.

In 11 states, new cases more than doubled.

Change in new cases

-75

-50

-25

0

+25

+50

+75\%

Change in new cases

The hardest-hit states in the Northeast and Midwest saw declines, but
new cases in the

rest of the country

were on the rise in June.

-75

-50

-25

0

+25

+50

+75\%

In 11 states, new cases more than doubled.

The hardest-hit states in the Northeast and Midwest saw declines, but
new cases in the rest of the country were on the rise in June.

In 11 states, new cases more than doubled.

Change in new cases

-75

-50

-25

0

+25

+50

+75\%

Compares the average of new cases for the 14 days ending June 8 with the
14 days ending June 22.

New York Times collection of data from state and local health agencies
and hospitals

By Nathaniel Lash \textbar{} New York Times

A few glimpses of the challenge:

\begin{itemize}
\item
  Texas, California, Arizona and four other states reported record
  numbers of cases this week.
\item
  Some 27 states, by the count of the
  \href{https://www.nytimes.com/interactive/2020/us/coronavirus-us-cases.html}{Times
  tracker}, are reporting increasing numbers of new cases. Ten states
  and Washington, D.C., are reporting declining numbers, with the rest
  holding steady.
\item
  Arizona, where Trump held his rally, now has the
  \href{https://www.covidexitstrategy.org/}{highest number of new cases
  per day} per million population, and the highest share of positive
  test results.
\end{itemize}

Black Lives Matter protests do not seem to have spread the virus much,
perhaps because they were held outside and many participants wore masks.
The virus is spreading most quickly in Trump Country in the South and
Southwest and in both red and blue states in the West.

``The next couple of weeks are going to be critical in our ability to
address those surges that we're seeing in Florida, in Texas, in Arizona,
and other states,'' Dr. Anthony Fauci
\href{https://www.nytimes.com/2020/06/23/us/politics/fauci-congress-coronavirus.html}{told}
a congressional hearing on Tuesday.

The rest of the world is watching aghast.

``What's happened in the U.S. is utterly tragic, and seems like a
consequence of appalling leadership and incompetent government,'' said
Devi Sridhar, an American who is a professor of global health at the
University of Edinburgh. ``Those of us abroad are watching in horror,
disbelief and pity.''

``This is a warning to other countries of the dangers of the virus going
out of control,'' she said.

The European Union is even
\href{https://www.nytimes.com/2020/06/23/world/europe/coronavirus-EU-American-travel-ban.html}{preparing
to bar} American visitors because of the United States' failure to
manage the coronavirus properly. Visitors from countries that have
controlled the virus better, like Vietnam, Cuba and Uganda, will be
welcome.

That's humiliating for the United States, but it should be a wake-up
call as well. Europe is right to fear American visitors. The United
States hasn't brought down case numbers the way European countries have,
and seems to simply accept a vast continuing toll of deaths.

Look at this graph of new Covid-19 cases in the European Union versus
the United States, with Canada and Australia thrown in for good measure:

\hypertarget{united-states-failing-to-flatten-the-curve}{%
\subsection{United States failing to flatten the
curve}\label{united-states-failing-to-flatten-the-curve}}

120 new cases per million population

100

80

United States

European Union

60

Canada

40

20

Australia

Mar 15

Mar 22

Mar 29

Apr 05

Apr 12

Apr 19

Apr 26

May 03

May 10

May 17

May 24

May 31

Jun 07

Jun 14

Jun 21

100 new cases per million population

80

United

States

European

Union

60

40

Canada

20

Australia

Mar 15

Jun 14

Jun 24

Shows 7-day rolling average of newly reported cases

Our World in Data

By Nathaniel Lash \textbar{} New York Times

The United States is now reporting new cases at
\href{https://ourworldindata.org/coronavirus}{nine times} the rate of
Europe, per million people.

In the New York region, memories are fresh, people are scared and the
virus is under control. But in much of the rest of the country, the
virus initially seemed remote, and people relaxed in ways that are now
leading to a crisis.

I passed through Phoenix twice last month
\href{https://www.nytimes.com/2020/05/30/opinion/sunday/coronavirus-native-americans.html}{to
report on Covid-19 cases in the Navajo Nation}, and I was horrified then
by how few Arizonans wore masks. Now we see the consequences.

Deaths are still below their peaks because for now it's
disproportionately younger people getting sick. That may change.

``I wonder how many fathers got a Father's Day present from their kids
--- this virus,'' reflected Michael T. Osterholm, a University of
Minnesota epidemiologist.

While some epidemiologists expect a second wave to arrive this fall,
Osterholm foresees more of a relentless toll of sickness and death. He
anticipates spikes in this city or that --- he fears Houston may become
the next New York --- but not much of a reprieve.

``I think it's going to keep going on,'' he told me. But he also
emphasizes that even the experts don't really understand the virus or
know what to anticipate.

His advice: Be humble and be bold, and make rigorous preparations.

We don't know for sure, but the post-peak experience from New York and
Europe as well as from street protests, offers some guidance: If people
wear masks, distance as much as possible and avoid mixing indoors, it
just might be possible to keep the virus in check.

Instead, our president refuses to wear a mask and brings people together
indoors to cheer his newest proposed strategy, which in his words is
``slow the testing down.'' After aides rushed to say he was joking,
Trump denied that, saying,
\href{https://www.cbsnews.com/news/trump-slow-down-testing-coronavirus-i-dont-kid/}{``I
don't kid.''} He amplified in
\href{https://twitter.com/realDonaldTrump/status/1275381670561095682}{a
tweet}: ``With smaller testing, we would show fewer cases!''

Yes, and by ending cancer screenings, we would reduce cancer rates. By
locking hospital doors, we would reduce hospitalizations. And if we
stopped issuing death certificates, Americans would achieve immortality!

That's the kind of strategizing that has led the United States, with 4
percent of the world's population, to experience one-quarter of the
deaths worldwide from the coronavirus --- and instead of ``fading
away,'' it's surging.

\emph{The Times is committed to publishing}
\href{https://www.nytimes.com/2019/01/31/opinion/letters/letters-to-editor-new-york-times-women.html}{\emph{a
diversity of letters}} \emph{to the editor. We'd like to hear what you
think about this or any of our articles. Here are some}
\href{https://help.nytimes.com/hc/en-us/articles/115014925288-How-to-submit-a-letter-to-the-editor}{\emph{tips}}\emph{.
And here's our email:}
\href{mailto:letters@nytimes.com}{\emph{letters@nytimes.com}}\emph{.}

Advertisement

\protect\hyperlink{after-bottom}{Continue reading the main story}

\hypertarget{site-index}{%
\subsection{Site Index}\label{site-index}}

\hypertarget{site-information-navigation}{%
\subsection{Site Information
Navigation}\label{site-information-navigation}}

\begin{itemize}
\tightlist
\item
  \href{https://help.nytimes.com/hc/en-us/articles/115014792127-Copyright-notice}{©~2020~The
  New York Times Company}
\end{itemize}

\begin{itemize}
\tightlist
\item
  \href{https://www.nytco.com/}{NYTCo}
\item
  \href{https://help.nytimes.com/hc/en-us/articles/115015385887-Contact-Us}{Contact
  Us}
\item
  \href{https://www.nytco.com/careers/}{Work with us}
\item
  \href{https://nytmediakit.com/}{Advertise}
\item
  \href{http://www.tbrandstudio.com/}{T Brand Studio}
\item
  \href{https://www.nytimes.com/privacy/cookie-policy\#how-do-i-manage-trackers}{Your
  Ad Choices}
\item
  \href{https://www.nytimes.com/privacy}{Privacy}
\item
  \href{https://help.nytimes.com/hc/en-us/articles/115014893428-Terms-of-service}{Terms
  of Service}
\item
  \href{https://help.nytimes.com/hc/en-us/articles/115014893968-Terms-of-sale}{Terms
  of Sale}
\item
  \href{https://spiderbites.nytimes.com}{Site Map}
\item
  \href{https://help.nytimes.com/hc/en-us}{Help}
\item
  \href{https://www.nytimes.com/subscription?campaignId=37WXW}{Subscriptions}
\end{itemize}
