Sections

SEARCH

\protect\hyperlink{site-content}{Skip to
content}\protect\hyperlink{site-index}{Skip to site index}

\href{https://www.nytimes.com/section/politics}{Politics}

\href{https://myaccount.nytimes.com/auth/login?response_type=cookie\&client_id=vi}{}

\href{https://www.nytimes.com/section/todayspaper}{Today's Paper}

\href{/section/politics}{Politics}\textbar{}Senate Democrats Block
G.O.P. Police Bill, Calling It Inadequate

\url{https://nyti.ms/2VidScy}

\begin{itemize}
\item
\item
\item
\item
\item
\end{itemize}

\href{https://www.nytimes.com/news-event/george-floyd-protests-minneapolis-new-york-los-angeles?action=click\&pgtype=Article\&state=default\&region=TOP_BANNER\&context=storylines_menu}{Race
and America}

\begin{itemize}
\tightlist
\item
  \href{https://www.nytimes.com/2020/07/26/us/protests-portland-seattle-trump.html?action=click\&pgtype=Article\&state=default\&region=TOP_BANNER\&context=storylines_menu}{Protesters
  Return to Other Cities}
\item
  \href{https://www.nytimes.com/2020/07/24/us/portland-oregon-protests-white-race.html?action=click\&pgtype=Article\&state=default\&region=TOP_BANNER\&context=storylines_menu}{Portland
  at the Center}
\item
  \href{https://www.nytimes.com/2020/07/23/podcasts/the-daily/portland-protests.html?action=click\&pgtype=Article\&state=default\&region=TOP_BANNER\&context=storylines_menu}{Podcast:
  Showdown in Portland}
\item
  \href{https://www.nytimes.com/interactive/2020/07/16/us/black-lives-matter-protests-louisville-breonna-taylor.html?action=click\&pgtype=Article\&state=default\&region=TOP_BANNER\&context=storylines_menu}{45
  Days in Louisville}
\end{itemize}

Advertisement

\protect\hyperlink{after-top}{Continue reading the main story}

Supported by

\protect\hyperlink{after-sponsor}{Continue reading the main story}

\hypertarget{senate-democrats-block-gop-police-bill-calling-it-inadequate}{%
\section{Senate Democrats Block G.O.P. Police Bill, Calling It
Inadequate}\label{senate-democrats-block-gop-police-bill-calling-it-inadequate}}

The vote reflected the waning likelihood that Congress would be able to
reach an election-year compromise on legislation to address racial bias
in policing.

\includegraphics{https://static01.nyt.com/images/2020/06/24/us/politics/24dc-unrest-cong/merlin_173849244_33adba9e-4c97-44b5-a14a-b95efc14771c-articleLarge.jpg?quality=75\&auto=webp\&disable=upscale}

\href{https://www.nytimes.com/by/catie-edmondson}{\includegraphics{https://static01.nyt.com/images/2019/11/20/us/politics/catie-edmonson-twitter-chatblog/catie-edmonson-twitter-chatblog-thumbLarge.png}}

By \href{https://www.nytimes.com/by/catie-edmondson}{Catie Edmondson}

\begin{itemize}
\item
  June 24, 2020
\item
  \begin{itemize}
  \item
  \item
  \item
  \item
  \item
  \end{itemize}
\end{itemize}

WASHINGTON --- Senate Democrats on Wednesday blocked a narrow Republican
bill to incentivize police departments to change their tactics, refusing
even to open debate on a measure they denounced as an insufficient and
irredeemably flawed answer to the problem of
\href{https://www.nytimes.com/2020/05/30/us/derek-chauvin-george-floyd.html}{systemic
racism in law enforcement}.

The vote, 55 to 45, was a setback in the effort to pass legislation this
year to address excessive use of force and racial discrimination by the
police, amid a
\href{https://www.nytimes.com/interactive/2020/05/30/us/george-floyd-protest-photos.html}{groundswell
of public sentiment} in favor of overhauling law enforcement. The
Democratic-led House is set on Thursday to pass its own sprawling
legislation, but Senate Republican leaders have said they will not take
up that measure, setting the stage for a bitter stalemate on the issue.

Expressing their deep opposition to the bill, Democrats demanded on
Tuesday that Republicans negotiate a more expansive package that both
parties could support, citing the opposition of dozens of civil rights
groups to the measure as drafted and arguing that it was an unacceptable
starting point for discussion.

Senator Kamala Harris, Democrat of California, told reporters that
Democrats' decision to block the bill was an effort ``to not take crumbs
on the table when there is a hunger that America has for real solutions
to a very real problem.''

``This movement will not accept anything less than real, real
substantial, substantive solutions, which are the solutions we have
offered,'' Ms. Harris said.

Republicans were livid at Democrats' refusal to even allow the measure
to reach the floor for a debate and accused them of deliberately sinking
the bill for political purposes. It would have needed 60 votes to
advance in the Senate, where a three-fifths supermajority is necessary
for most major action. But only two Democrats, Senators Doug Jones of
Alabama and Joe Manchin III of West Virginia, as well as Senator Angus
King, independent of Maine, joined Republicans in supporting moving it
forward.

``If you don't think we're right, make it better. Don't walk away,''
Senator Tim Scott of South Carolina, who
\href{https://www.nytimes.com/2020/06/16/us/politics/tim-scott-police-protests.html}{spearheaded
the legislation}, said before the vote. He urged Democrats to support
advancing the bill ``so that we have an opportunity to deal with this
very real threat to the America that is civil, that is balanced.''

``This is an opportunity to say yes,'' he said.

After the measure failed, a visibly frustrated Mr. Scott returned and
delivered extended remarks, saying that he had offered to give Democrats
as many as 20 votes on proposed modifications to his bill that they were
demanding, but that they had refused to accept. Privately, Democrats
noted that revising the bill would have also required the approval of 60
senators, a threshold they feared they would not be able to meet.

``Instead of going forward and getting what you want now, they've
decided to punt this ball until the election,'' Mr. Scott said of
Democrats. ``You know why? Because they believe the polls reflect a
15-point deficit on our side, therefore they can get the bill they want
in November.''

``The actual problem is not what is being offered, it is who is offering
it,'' he continued.

As Mr. Scott left the floor, senators who had gathered there to hear him
speak stood to applaud him. One of them, Senator Tim Kaine, Democrat of
Virginia, who said he had come to listen to Mr. Scott in appreciation of
his work, launched into his own impassioned speech, denying that the
outcome had been politically driven.

``That is a stiff charge,'' Mr. Kaine said. ``I voted not on the `what'
and not on the `who.' I voted on the `how.' We tried it the wrong way.
Let's try it the right way.''

The Republican bill would encourage state and local police departments
to change their practices, including penalizing departments that do not
require the use of body cameras and limiting the use of chokeholds. It
would not alter the
\href{https://www.nytimes.com/2020/06/23/us/politics/qualified-immunity.html}{qualified
immunity doctrine} that shields officers from lawsuits or place new
federal restrictions on the use of lethal force.

The measure that the House will consider on Thursday,
\href{https://www.nytimes.com/2020/06/08/us/politics/democrats-police-misconduct-bill-protests.html}{the
most aggressive intervention into policing} that lawmakers have proposed
in recent memory, would in effect eliminate qualified immunity, make it
easier to track and prosecute police misconduct, restrict the use of
lethal force and aim to force departments to eliminate the use of
chokeholds.

Wednesday's vote did not foreclose the possibility of reviving the
policing measure. Senator Mitch McConnell, Republican of Kentucky and
the majority leader, used a procedural maneuver that would allow him to
bring it up again in the future, changing his vote from ``yes'' to
``no'' so he could later call for its reconsideration. But a flurry of
private bipartisan talks to strike a deal on the issue had not borne
fruit.

Advertisement

\protect\hyperlink{after-bottom}{Continue reading the main story}

\hypertarget{site-index}{%
\subsection{Site Index}\label{site-index}}

\hypertarget{site-information-navigation}{%
\subsection{Site Information
Navigation}\label{site-information-navigation}}

\begin{itemize}
\tightlist
\item
  \href{https://help.nytimes.com/hc/en-us/articles/115014792127-Copyright-notice}{©~2020~The
  New York Times Company}
\end{itemize}

\begin{itemize}
\tightlist
\item
  \href{https://www.nytco.com/}{NYTCo}
\item
  \href{https://help.nytimes.com/hc/en-us/articles/115015385887-Contact-Us}{Contact
  Us}
\item
  \href{https://www.nytco.com/careers/}{Work with us}
\item
  \href{https://nytmediakit.com/}{Advertise}
\item
  \href{http://www.tbrandstudio.com/}{T Brand Studio}
\item
  \href{https://www.nytimes.com/privacy/cookie-policy\#how-do-i-manage-trackers}{Your
  Ad Choices}
\item
  \href{https://www.nytimes.com/privacy}{Privacy}
\item
  \href{https://help.nytimes.com/hc/en-us/articles/115014893428-Terms-of-service}{Terms
  of Service}
\item
  \href{https://help.nytimes.com/hc/en-us/articles/115014893968-Terms-of-sale}{Terms
  of Sale}
\item
  \href{https://spiderbites.nytimes.com}{Site Map}
\item
  \href{https://help.nytimes.com/hc/en-us}{Help}
\item
  \href{https://www.nytimes.com/subscription?campaignId=37WXW}{Subscriptions}
\end{itemize}
