Sections

SEARCH

\protect\hyperlink{site-content}{Skip to
content}\protect\hyperlink{site-index}{Skip to site index}

\href{https://www.nytimes.com/section/obituaries}{Obituaries}

\href{https://myaccount.nytimes.com/auth/login?response_type=cookie\&client_id=vi}{}

\href{https://www.nytimes.com/section/todayspaper}{Today's Paper}

\href{/section/obituaries}{Obituaries}\textbar{}Akbar Nurid-Din Shabazz,
Texas Prison Chaplain, Dies at 70

\url{https://nyti.ms/2CzGKpY}

\begin{itemize}
\item
\item
\item
\item
\item
\end{itemize}

\href{https://www.nytimes.com/news-event/coronavirus?action=click\&pgtype=Article\&state=default\&region=TOP_BANNER\&context=storylines_menu}{The
Coronavirus Outbreak}

\begin{itemize}
\tightlist
\item
  live\href{https://www.nytimes.com/2020/08/03/world/coronavirus-covid-19.html?action=click\&pgtype=Article\&state=default\&region=TOP_BANNER\&context=storylines_menu}{Latest
  Updates}
\item
  \href{https://www.nytimes.com/interactive/2020/us/coronavirus-us-cases.html?action=click\&pgtype=Article\&state=default\&region=TOP_BANNER\&context=storylines_menu}{Maps
  and Cases}
\item
  \href{https://www.nytimes.com/interactive/2020/science/coronavirus-vaccine-tracker.html?action=click\&pgtype=Article\&state=default\&region=TOP_BANNER\&context=storylines_menu}{Vaccine
  Tracker}
\item
  \href{https://www.nytimes.com/2020/08/02/us/covid-college-reopening.html?action=click\&pgtype=Article\&state=default\&region=TOP_BANNER\&context=storylines_menu}{College
  Reopening}
\item
  \href{https://www.nytimes.com/live/2020/08/03/business/stock-market-today-coronavirus?action=click\&pgtype=Article\&state=default\&region=TOP_BANNER\&context=storylines_menu}{Economy}
\end{itemize}

Advertisement

\protect\hyperlink{after-top}{Continue reading the main story}

Supported by

\protect\hyperlink{after-sponsor}{Continue reading the main story}

Those We've Lost

\hypertarget{akbar-nurid-din-shabazz-texas-prison-chaplain-dies-at-70}{%
\section{Akbar Nurid-Din Shabazz, Texas Prison Chaplain, Dies at
70}\label{akbar-nurid-din-shabazz-texas-prison-chaplain-dies-at-70}}

Mr. Shabazz spent 40 years providing support to inmates in Texas prisons
and was credited with expanding the practice of Islam there. He died of
Covid-19.

\includegraphics{https://static01.nyt.com/images/2020/06/27/obituaries/22Shabazz/merlin_173672781_e09da0ae-8961-40c6-803c-fb6d68153c49-articleLarge.jpg?quality=75\&auto=webp\&disable=upscale}

By David Montgomery

\begin{itemize}
\item
  Published June 25, 2020Updated June 26, 2020
\item
  \begin{itemize}
  \item
  \item
  \item
  \item
  \item
  \end{itemize}
\end{itemize}

\emph{This obituary is part of a series about people who have died in
the coronavirus pandemic. Read about others}
\href{https://www.nytimes.com/interactive/2020/obituaries/people-died-coronavirus-obituaries.html}{\emph{here}}\emph{.}

Throughout his four decades as the first Muslim chaplain in the Texas
prison system, Akbar Nurid-Din Shabazz somehow found a way to instill
hope in inmates who were often without it.

In 1995, Charlesetta Myers of Dallas was, in her words, ``spiraling out
of control.'' After violating parole, she was sent back to prison for a
third time. She picked up a copy of the Quran and decided to go to a
Muslim service, where she met Mr. Shabazz. She soon converted to Islam
and took the name Rashidah Muhammad.

Mr. Shabazz ``just had a way of encouraging people,'' Ms. Muhammad said,
crediting him with turning her life around. Twenty years after leaving
prison, she is a small-business owner and an active figure in Dallas
Muslim circles who travels to prisons each month as a volunteer teacher.

Mr. Shabazz died on April 23 at a hospital in The Woodlands, near
Houston. He was 70. His daughter, Rabiah Shabazz, said the cause was
Covid-19.

The disease, caused by the new coronavirus, has taken a particularly
heavy toll on prison populations. As of June 25, Mr. Shabazz was among
eight staff members and 72 offenders in the Texas prison system to
succumb to the virus, according to
the\href{https://www.tdcj.texas.gov/}{Texas Department of Criminal
Justice}.

Mr. Shabazz, who lived in Huntsville, Texas, was one of more than 100
chaplains in the prison system, including five Muslims.~He provided
pastoral care, led prayer services and worked at about 25 prisons, often
counseling staff members as well as inmates. He was credited with
expanding the practice of Islam in Texas prisons and with cementing
Muslim traditions like Friday prayers and the observance of Ramadan, the
Muslim holy month.

Mr. Shabazz was well regarded among inmates of all faiths, said Bryan
Collier, the executive director of the criminal justice department.
``Anybody who had been here for a while knew who Chaplain Shabazz was,''
he said.

Several former prisoners said Mr. Shabazz had remained a close friend
long after they left prison and helped guide former inmates toward
receiving college degrees or starting businesses.

He had another skill inside the prisons: ``He knew how to de-escalate
conflicts,'' said his daughter, a mental health manager for offenders
and a former correctional officer. ``He had so much patience and calm.''

Mr. Shabazz, whose story was recently told by The Houston Chronicle,
first served as a volunteer chaplain and began working full time in
1979. In 1982, he was present as a chaplain at the nation's
\href{https://www.nytimes.com/1982/12/07/us/technician-executes-murderer-in-texas-by-lethal-injection.html}{first
execution by lethal injection} and was one of two chaplains who escorted
the condemned killer, Charles Brooks Jr., into the death chamber,
according to Texas Monthly magazine, which chronicled Mr. Brooks's final
hours.

Akbar Nurid-Din Shabazz was born Robert Lynn Williams on March 20, 1950,
in Monroe, La., the second oldest of nine children of Robert and Matheal
Williams. His father owned several small businesses and was a truck
driver; his mother was a nurse. The family converted to Islam after
moving to Flint, Mich., in 1951, and adopted Muslim names in the
mid-1970s; his parents became Omar and Matheal Shabazz.

The family moved to Texas in 1959, and Robert grew up in Dallas. He
attended El Centro College there before becoming a research employee at
the University of Texas Southwestern Medical Center.

Mr. Shabazz married Mary Smith in 1972; they divorced in 1981. In
addition to his daughter, he is survived by his second wife, Janice
Shabazz; a son, Akbar Nurid-Din Shabazz II; and five stepchildren, Ira
Mosley, Janice O'Guin, Shirley Kamau, Vickey Shaffer and Janiece Burns.

\href{https://www.nytimes.com/interactive/2020/obituaries/people-died-coronavirus-obituaries.html?action=click\&pgtype=Article\&state=default\&region=BELOW_MAIN_CONTENT\&context=covid_obits_promo}{}

\hypertarget{those-weve-lost}{%
\section{Those We've Lost}\label{those-weve-lost}}

The coronavirus pandemic has taken an incalculable death toll. This
series is designed to put names and faces to the numbers.

Read more

\includegraphics{https://static01.nyt.com/images/2020/07/30/obituaries/30Pedro/30Pedro-square640.jpg}

\hypertarget{bernaldina-josuxe9-pedro}{%
\section{Bernaldina José Pedro}\label{bernaldina-josuxe9-pedro}}

d. Boa Vista, Brazil

Leader among the Indigenous Macuxi

\includegraphics{https://static01.nyt.com/images/2020/07/31/obituaries/31Swing/merlin_175167783_8913bc90-0d64-43f3-a655-1bb1bf1601c9-square640.jpg}

\hypertarget{john-eric-swing}{%
\section{John Eric Swing}\label{john-eric-swing}}

d. Fountain Valley, Calif.

Champion of Filipino-Americans

\includegraphics{https://static01.nyt.com/images/2020/07/27/obituaries/27Victor/merlin_175001436_38b11f8e-227a-4e2c-9821-7618af9b2524-square640.jpg}

\hypertarget{victor-victor}{%
\section{Victor Victor}\label{victor-victor}}

d. Santo Domingo, Dominican Republic

Beloved musician of the Dominican Republic

\includegraphics{https://static01.nyt.com/images/2020/07/31/obituaries/31Negron/merlin_175160169_516322ae-fd23-4969-b6b2-193ced371105-square640.jpg}

\hypertarget{dr-eddie-negruxf3n}{%
\section{Dr. Eddie Negrón}\label{dr-eddie-negruxf3n}}

d. Fort Walton Beach, Fla.

Internist on Florida's Emerald Coast

\includegraphics{https://static01.nyt.com/images/2020/07/30/obituaries/30Dobson/merlin_175115928_f6b9271c-8f05-4fe1-a38a-5ca4a58f8935-square640.jpg}

\hypertarget{dobby-dobson}{%
\section{Dobby Dobson}\label{dobby-dobson}}

d. Coral Springs, Fla.

Jamaican singer and songwriter

\includegraphics{https://static01.nyt.com/images/2020/08/01/obituaries/28Gonzalez/merlin_175002771_beb57888-3951-409a-ae13-03a94b2e962e-square640.jpg}

\hypertarget{waldemar-gonzalez}{%
\section{Waldemar Gonzalez}\label{waldemar-gonzalez}}

d. White Plains, N.Y.

Teacher and social worker

Advertisement

\protect\hyperlink{after-bottom}{Continue reading the main story}

\hypertarget{site-index}{%
\subsection{Site Index}\label{site-index}}

\hypertarget{site-information-navigation}{%
\subsection{Site Information
Navigation}\label{site-information-navigation}}

\begin{itemize}
\tightlist
\item
  \href{https://help.nytimes.com/hc/en-us/articles/115014792127-Copyright-notice}{©~2020~The
  New York Times Company}
\end{itemize}

\begin{itemize}
\tightlist
\item
  \href{https://www.nytco.com/}{NYTCo}
\item
  \href{https://help.nytimes.com/hc/en-us/articles/115015385887-Contact-Us}{Contact
  Us}
\item
  \href{https://www.nytco.com/careers/}{Work with us}
\item
  \href{https://nytmediakit.com/}{Advertise}
\item
  \href{http://www.tbrandstudio.com/}{T Brand Studio}
\item
  \href{https://www.nytimes.com/privacy/cookie-policy\#how-do-i-manage-trackers}{Your
  Ad Choices}
\item
  \href{https://www.nytimes.com/privacy}{Privacy}
\item
  \href{https://help.nytimes.com/hc/en-us/articles/115014893428-Terms-of-service}{Terms
  of Service}
\item
  \href{https://help.nytimes.com/hc/en-us/articles/115014893968-Terms-of-sale}{Terms
  of Sale}
\item
  \href{https://spiderbites.nytimes.com}{Site Map}
\item
  \href{https://help.nytimes.com/hc/en-us}{Help}
\item
  \href{https://www.nytimes.com/subscription?campaignId=37WXW}{Subscriptions}
\end{itemize}
