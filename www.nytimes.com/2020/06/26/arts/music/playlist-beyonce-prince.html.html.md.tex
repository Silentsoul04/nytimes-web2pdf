Sections

SEARCH

\protect\hyperlink{site-content}{Skip to
content}\protect\hyperlink{site-index}{Skip to site index}

\href{https://www.nytimes.com/section/arts/music}{Music}

\href{https://myaccount.nytimes.com/auth/login?response_type=cookie\&client_id=vi}{}

\href{https://www.nytimes.com/section/todayspaper}{Today's Paper}

\href{/section/arts/music}{Music}\textbar{}Beyoncé's Surprise Juneteenth
Anthem, and 12 More New Songs

\url{https://nyti.ms/3dBhyME}

\begin{itemize}
\item
\item
\item
\item
\item
\end{itemize}

\href{https://www.nytimes.com/spotlight/at-home?action=click\&pgtype=Article\&state=default\&region=TOP_BANNER\&context=at_home_menu}{At
Home}

\begin{itemize}
\tightlist
\item
  \href{https://www.nytimes.com/2020/07/28/books/time-for-a-literary-road-trip.html?action=click\&pgtype=Article\&state=default\&region=TOP_BANNER\&context=at_home_menu}{Take:
  A Literary Road Trip}
\item
  \href{https://www.nytimes.com/2020/07/29/magazine/bored-with-your-home-cooking-some-smoky-eggplant-will-fix-that.html?action=click\&pgtype=Article\&state=default\&region=TOP_BANNER\&context=at_home_menu}{Cook:
  Smoky Eggplant}
\item
  \href{https://www.nytimes.com/2020/07/27/travel/moose-michigan-isle-royale.html?action=click\&pgtype=Article\&state=default\&region=TOP_BANNER\&context=at_home_menu}{Look
  Out: For Moose}
\item
  \href{https://www.nytimes.com/interactive/2020/at-home/even-more-reporters-editors-diaries-lists-recommendations.html?action=click\&pgtype=Article\&state=default\&region=TOP_BANNER\&context=at_home_menu}{Explore:
  Reporters' Obsessions}
\end{itemize}

Advertisement

\protect\hyperlink{after-top}{Continue reading the main story}

Supported by

\protect\hyperlink{after-sponsor}{Continue reading the main story}

The Playlist

\hypertarget{beyoncuxe9s-surprise-juneteenth-anthem-and-12-more-new-songs}{%
\section{Beyoncé's Surprise Juneteenth Anthem, and 12 More New
Songs}\label{beyoncuxe9s-surprise-juneteenth-anthem-and-12-more-new-songs}}

Hear tracks by Prince, the Chicks, Sun Ra Arkestra and others.

\includegraphics{https://static01.nyt.com/images/2020/06/30/arts/26playlist/26playlist-articleLarge.jpg?quality=75\&auto=webp\&disable=upscale}

By \href{https://www.nytimes.com/by/jon-pareles}{Jon Pareles},
\href{https://www.nytimes.com/by/jon-caramanica}{Jon Caramanica} and
\href{https://www.nytimes.com/by/giovanni-russonello}{Giovanni
Russonello}

\begin{itemize}
\item
  Published June 26, 2020Updated Aug. 2, 2020
\item
  \begin{itemize}
  \item
  \item
  \item
  \item
  \item
  \end{itemize}
\end{itemize}

\emph{Every Friday, pop critics for The New York Times weigh in on the
week's most notable new songs and videos. Just want the music?}
\href{https://open.spotify.com/playlist/0PKt3BNsgOe4kR3dwI2gXp?si=oXEt9ajMT02NhFe0Xn4R5Q}{\emph{Listen
to the Playlist on Spotify here}} \emph{(or find our profile: nytimes).
Like what you hear? Let us know at}
\href{mailto:theplaylist@nytimes.com}{\emph{theplaylist@nytimes.com}}
\emph{and}
\href{https://www.nytimes.com/newsletters/louder?module=inline}{\emph{sign
up for our Louder newsletter}}\emph{, a once-a-week blast of our pop
music coverage.}

\hypertarget{beyoncuxe9-black-parade}{%
\subsection{Beyoncé, `Black Parade'}\label{beyoncuxe9-black-parade}}

\href{https://www.nytimes.com/2020/07/31/arts/music/beyonce-black-is-king.html}{Beyoncé}
released ``Black Parade'' on Juneteenth, and it makes ambitious,
far-reaching connections. The lyrics allude to black American
achievement, culture and struggle, to African history and deities, to
the power of women, to Beyoncé's own success and to this month's
\href{https://www.nytimes.com/news-event/george-floyd-protests-minneapolis-new-york-los-angeles}{protests}:
``Rubber bullets bouncin' off me/Made a picket sign off your picket
fence/Take it as a warning.'' The music pulls its own connections --- to
trap electronics, African songs, brass bands, gospel choirs --- while
Beyoncé flaunts new melody ideas in each verse. Voices gather around
her, as her solo strut turns into a parade, or a more purposeful march:
``Put your fists up in the air/Show black love,'' she insists. JON
PARELES

\hypertarget{prince-witness-4-the-prosecution-version-1}{%
\subsection{Prince, `Witness 4 the Prosecution (Version
1)'}\label{prince-witness-4-the-prosecution-version-1}}

``Witness 4 the Prosecution (Version 1)'' is the first previously
unreleased song from what will be a vastly expanded reissue of Prince's
1987 double album ``Sign `o' the Times,'' due Sept. 25. It's meaty
funk-rock that sounds like it was recorded live: heavy on the backbeat,
with sassy horns, thumb-popping bass, a gospelly backup choir (shouting
``witness!'') and biting, distorted lead guitar, all stoking Prince as
he testifies in a case of obsessive love. PARELES

\hypertarget{dinner-party-featuring-phoelix-freeze-tag}{%
\subsection{Dinner Party featuring Phoelix, `Freeze
Tag'}\label{dinner-party-featuring-phoelix-freeze-tag}}

Dinner Party is the alliance of the producers and musicians 9th Wonder,
Terrace Martin, Kamasi Washington and Robert Glasper. Phoelix joins them
to sing ``Freeze Tag,'' about an all-too-common scenario: ``They told me
put my hands up behind my head/I think they got the wrong one,'' he
recounts in a high, gentle croon. ``Then they told me if I move, they
gon' shoot me dead.'' The music is quiet-storm R\&B, complete with wind
chimes, but as the chord progression circles and Phoelix sings the verse
again and again, the fraught, frozen moment grows harrowing. PARELES

\hypertarget{sun-ra-arkestra-seductive-fantasy}{%
\subsection{Sun Ra Arkestra, `Seductive
Fantasy'}\label{sun-ra-arkestra-seductive-fantasy}}

More than 25 years after Sun Ra's death, the Afrofuturist pioneer's
ensemble continues to uphold his legacy in performances around the
world, but it hasn't released a studio album of new music in two
decades. That will change later this year. The first single from the
Arkestra's forthcoming LP is ``Seductive Fantasy,'' a slow-moving,
blood-pumping vamp that first appeared on
\href{https://www.youtube.com/watch?v=fyTdUr1u9ok}{the 1979 album ``On
Jupiter.''} On the new version, the first sound you hear is the steady
baritone saxophone line of
\href{https://www.nytimes.com/2020/03/20/arts/music/danny-ray-thompson-dead.html}{Danny
Ray Thompson}, who played on the original too; he died just months after
this newer recording was made. Across a quick four minutes, saxophones
carry a simple melody, then join up with the reeds to make a messy
gouache of harmonies while Marshall Allen's alto saxophone nearly flies
off the handle, squealing its way toward liftoff. GIOVANNI RUSSONELLO

\hypertarget{charlie-puth-girlfriend}{%
\subsection{Charlie Puth, `Girlfriend'}\label{charlie-puth-girlfriend}}

Some pleasant falsetto funk from
\href{https://www.nytimes.com/2018/05/16/arts/music/charlie-puth-voicenotes-interview.html}{Charlie
Puth}, a formalist with a lithe voice and a cloying demeanor. ``Baby
would you ever want to be my girlfriend?'' he coos. ``I don't want to
play no games/this is more than just a phase.'' It's effective, but
perhaps not quite as catchy as his
\href{https://www.youtube.com/watch?v=JD61F6SNkTo}{recent commercials
for Subway}. JON CARAMANICA

\hypertarget{the-chicks-march-march}{%
\subsection{The Chicks, `March March'}\label{the-chicks-march-march}}

The Chicks ---
\href{https://www.nytimes.com/2020/06/25/arts/music/dixie-chicks-change-name.html}{they
have dropped Dixie} in this moment of rejecting references to the Civil
War-era South --- are a long way from traditional country in ``March
March'' from their coming album, ``Gaslighter.'' The initial beat is an
electronic thump and blip, and the lyrics are topical and sometimes
profane, praising the teenage activists who are demanding gun control
and environmental action: ``Watching our youth have to solve our
problems/I'm standing with them, who's coming with me?'' sings Natalie
Maines. Fiddle and banjo do arrive --- and the tune has modal
Appalachian echoes --- but the song sends a message for right now.
PARELES

\hypertarget{becca-mancari-lonely-boy}{%
\subsection{Becca Mancari, `Lonely
Boy'}\label{becca-mancari-lonely-boy}}

Becca Mancari and her producer, Zac Farro from Paramore, build a
Minimalist latticework of plucked strings, syncopated drums and pealing
guitars --- Stereolab gone to Nashville --- as she sings ``Are you a
lonely boy?'' in an upturned phrase like an encouraging smile. ``Lonely
Boy'' is from her new album, ``The Greatest Part''; she has said she
\href{https://consequenceofsound.net/2020/06/becca-mancari-origins-lonely-boy-stream/}{wrote
it about her dog}, but its affection extends further. PARELES

\hypertarget{nadine-shah-trad}{%
\subsection{Nadine Shah, `Trad'}\label{nadine-shah-trad}}

``Take me to the ceremony/Make me holy matrimony,'' Nadine Shah sings,
layering on vocal harmonies as a guitar line buzzes around her like a
persistent mosquito. It's from her new album, ``Kitchen Sink,'' which is
by turns sardonic and haunted. The verses of ``Trad'' are dryly
skeptical about the institution of marriage, while that chorus makes it
sound like an ominous, decisive ritual. PARELES

\hypertarget{flock-of-dimes-like-so-much-desire}{%
\subsection{Flock of Dimes, `Like So Much
Desire'}\label{flock-of-dimes-like-so-much-desire}}

``Like So Much Desire'' is the title song from an absorbing new EP by
Flock of Dimes, Jenn Wasner's solo project when she's not fronting Wye
Oak. It's a gorgeous, lofty waltz, with synthesizers billowing around
acoustic guitars and vocal harmonies, as Wasner sings an enigmatic
reverie about ``losing the old for the new, like so much desire.''
PARELES

\hypertarget{boyband-tattoo}{%
\subsection{boyband, `Tattoo'}\label{boyband-tattoo}}

First came the first-wave protean
\href{https://www.nytimes.com/2017/06/22/arts/music/soundcloud-rap-lil-pump-smokepurrp-xxxtentacion.html}{SoundCloud
rappers}, followed by the auteur era of
\href{https://www.nytimes.com/2018/10/31/arts/music/lil-peep-archives-come-over-when-youre-sober.html}{Lil
Peep} and his acolytes. This song, the fourth single from boyband --- a
producer affiliated with the Internet Money collective --- perhaps
suggests the third wave of the emo-rap revival is here. The sentiment is
satisfyingly direct, if delivered a little awkwardly in places. But at
the hook, boyband summons an image that's tough and soft all at once:
``Might tattoo your name so I can watch it fade away.'' CARAMANICA

\hypertarget{jack-harlow-featuring-dababy-tory-lanez-and-lil-wayne-whats-poppin-remix}{%
\subsection{Jack Harlow featuring DaBaby, Tory Lanez and Lil Wayne,
`Whats Poppin'
(Remix)}\label{jack-harlow-featuring-dababy-tory-lanez-and-lil-wayne-whats-poppin-remix}}

Wafting up from TikTok ubiquity, ``Whats Poppin'' has been a surprising
breakthrough for the brash and buoyant rapper Jack Harlow, with a jiggly
beat that leavens him just enough. Now, six months after the song's
initial release, it's hovering in the Top 20 of the Hot 100 and has
earned a posse-cut remix of heavyweights: DaBaby, pugnacious and
indignant; Tory Lanez, thirsty and salacious; and Lil Wayne, as limber
as he's sounded in quite some time. CARAMANICA

\hypertarget{arca-mequetrefe}{%
\subsection{Arca, `Mequetrefe'}\label{arca-mequetrefe}}

Arca's music has often been a barrage, summoning peak impact from sounds
that are cranked up, maxed out and then suddenly truncated, as if
they've hurtled directly into a brick wall. Her new album, ``KiCk i,''
mingles that attack with flashes of quasi-pop. The slapping, sputtering
percussion in ``Mequetrefe'' (a derogatory Spanish slang term for
certain men) is joined by a pretty instrumental line, and Arca's vocal
chants are patterned like song verses and choruses; still, impact
prevails. PARELES

\hypertarget{nicole-mitchell-and-lisa-e-harris-purify-me-with-the-power-to-transform}{%
\subsection{Nicole Mitchell and Lisa E. Harris, `Purify Me With the
Power to
Transform'}\label{nicole-mitchell-and-lisa-e-harris-purify-me-with-the-power-to-transform}}

The flutist
\href{https://www.nytimes.com/2018/01/10/arts/music/nicole-mitchell-black-earth-ensemble-mandorla-awakening.html}{Nicole
Mitchell} and the vocalist and multi-instrumentalist Lisa E. Harris both
use composition and sound to open up listeners to bold new ways of
thinking about the world, and to encourage them to see past its
present-day limitations. Both have long taken inspiration from the
allegorical science-fiction writings of Octavia Butler; for their first
collaboration, Mitchell and Harris wrote ``EarthSeed,'' an album-length
suite of compositions, based on a
\href{https://www.goodreads.com/series/57804-earthseed}{Butlerian idea}.
``Purify Me With the Power to Transform'' is the closing piece, a drift
of voices and strings and ambient tones that sounds like it's bidding
goodbye to a broken present. RUSSONELLO

Advertisement

\protect\hyperlink{after-bottom}{Continue reading the main story}

\hypertarget{site-index}{%
\subsection{Site Index}\label{site-index}}

\hypertarget{site-information-navigation}{%
\subsection{Site Information
Navigation}\label{site-information-navigation}}

\begin{itemize}
\tightlist
\item
  \href{https://help.nytimes.com/hc/en-us/articles/115014792127-Copyright-notice}{©~2020~The
  New York Times Company}
\end{itemize}

\begin{itemize}
\tightlist
\item
  \href{https://www.nytco.com/}{NYTCo}
\item
  \href{https://help.nytimes.com/hc/en-us/articles/115015385887-Contact-Us}{Contact
  Us}
\item
  \href{https://www.nytco.com/careers/}{Work with us}
\item
  \href{https://nytmediakit.com/}{Advertise}
\item
  \href{http://www.tbrandstudio.com/}{T Brand Studio}
\item
  \href{https://www.nytimes.com/privacy/cookie-policy\#how-do-i-manage-trackers}{Your
  Ad Choices}
\item
  \href{https://www.nytimes.com/privacy}{Privacy}
\item
  \href{https://help.nytimes.com/hc/en-us/articles/115014893428-Terms-of-service}{Terms
  of Service}
\item
  \href{https://help.nytimes.com/hc/en-us/articles/115014893968-Terms-of-sale}{Terms
  of Sale}
\item
  \href{https://spiderbites.nytimes.com}{Site Map}
\item
  \href{https://help.nytimes.com/hc/en-us}{Help}
\item
  \href{https://www.nytimes.com/subscription?campaignId=37WXW}{Subscriptions}
\end{itemize}
