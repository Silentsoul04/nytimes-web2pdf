Sections

SEARCH

\protect\hyperlink{site-content}{Skip to
content}\protect\hyperlink{site-index}{Skip to site index}

\href{/section/opinion}{Opinion}\textbar{}`Let Freedom Ring' From
Georgia

\href{https://nyti.ms/3ezQoHu}{https://nyti.ms/3ezQoHu}

\begin{itemize}
\item
\item
\item
\item
\item
\item
\end{itemize}

\includegraphics{https://static01.nyt.com/images/2020/06/28/opinion/26cohen1a/merlin_173570394_aa1fdc57-f92b-4914-8f57-1aabc131e0f3-articleLarge.jpg?quality=75\&auto=webp\&disable=upscale}

\href{/section/opinion}{Opinion}

\hypertarget{let-freedom-ring-from-georgia}{%
\section{`Let Freedom Ring' From
Georgia}\label{let-freedom-ring-from-georgia}}

``The fact I am here means I am descended from people who, even
enslaved, did not give up hope.''

A group of demonstrators locked arms at the intersection of Metropolitan
Parkway and Cleveland Ave. in southwest Atlanta during a protest earlier
this month.Credit...Joshua Rashaad McFadden for The New York Times

Supported by

\protect\hyperlink{after-sponsor}{Continue reading the main story}

\href{https://www.nytimes.com/by/roger-cohen}{\includegraphics{https://static01.nyt.com/images/2014/11/01/opinion/cohen-circular/cohen-circular-thumbLarge-v6.png}}

By \href{https://www.nytimes.com/by/roger-cohen}{Roger Cohen}

Opinion Columnist

\begin{itemize}
\item
  June 26, 2020
\item
  \begin{itemize}
  \item
  \item
  \item
  \item
  \item
  \item
  \end{itemize}
\end{itemize}

ATLANTA --- This, an old saying goes, is ``the city too busy to hate,''
one of the few places in America where enlightened leaders, black and
white, chose prosperity over prejudice and a large black middle class
emerged decades ago. Birthplace of Martin Luther King Jr., cradle of the
civil rights movement, Atlanta, with its gleaming towers and porch
swings, was an American exception. The city managed racial conflict
through compromise. It was the black mecca. Or so the story went.

James Forman Jr., a Pulitzer-prize winning professor at Yale Law School,
and the son of the prominent civil rights activist James Forman Sr.,
recalled how, at age 12, he moved from New York to Atlanta because ``my
mother, as a divorced white woman raising black children, wanted us
surrounded by black success. She wanted my brother and me to open the
paper every day and see black people making decisions.'' That was the
1970s. Every Atlanta mayor since 1974 has been black.

Yet now the city is an epicenter of America's double meltdown: over
racial injustice and over the coronavirus that has hit marginalized
African-Americans particularly hard. This is the home of the Centers for
Disease Control and Prevention, which went AWOL on the virus. This is
where a young black man, Rayshard Brooks, was killed on June 12 by a
white police officer.

Over the course of a two-week stay, I encountered swirling fury over the
Brooks killing; a primary election debacle that, by design or Republican
dereliction, included hourslong waits in polling stations in
predominantly black counties; and a protest march on the State Capitol
where a banner saying ``Legalize being Black'' conveyed the rampant ire.

\includegraphics{https://static01.nyt.com/images/2020/06/26/opinion/26cohenWeb/merlin_173657106_1c94a5af-6488-40cd-931c-2b24521b818b-articleLarge.jpg?quality=75\&auto=webp\&disable=upscale}

What became of the dream of Atlanta? It was always a progressive enclave
surrounded by reactionary forces. If City Hall was the nexus of racial
cooperation, the State Capitol was the nexus of segregation now and
forever. Perhaps things were never quite as good as they seemed.
African-Americans remained disproportionately poor and vulnerable. When
Atlanta hosted the 1996 Olympics, Georgia's flag was still,
\href{https://www.latimes.com/archives/la-xpm-1996-07-19-mn-25755-story.html}{in
essence, the Confederate flag}.

\textbf{Progress on race issues is not resolution of race issues.} Not
in Atlanta, not anywhere, as Derek Chauvin's white knee on George
Floyd's black neck demonstrated. Police brutality, mass black
incarceration, poor education, redlining of neighborhoods all told a
story so routine as to be invisible: A black life is worth less than a
white life in America. That idea is woven into the psyches even of
people loath to admit it.

The Floyd detonation was long in the making. With its large
African-American population, about a third of the electorate, Georgia
was bound to feel the reverberations. Democrats have not won Georgia,
with its 16 electoral votes, since 1992, and Donald Trump had a clear
victory here in 2016. Now several polls suggest Joe Biden is
\href{https://projects.fivethirtyeight.com/polls/president-general/georgia/}{leading
by a small margin} (and is considering driving home his ascendancy here
by choosing either Stacey Abrams or the Atlanta mayor, Keisha Lance
Bottoms, as his running mate). This is the Covid-Floyd election, and
Georgia has become a bellwether.

The narrow 2018 defeat of Abrams, campaigning to become the nation's
first black female governor, showed how demographic shifts have changed
Georgia. The metropolitan-rural political and cultural chasm, evident
across the nation, is particularly acute here. Fast-growing Metro
Atlanta, with its diverse Democratic-leaning population, faces a
hinterland where, for many white rural Georgians, Trump is still the
tough, straight-talking dude the country needs. The vote will be close.
If Trump loses Georgia to Biden, he likely loses everything. But that's
still a big ``if.''

The bungled June 9 primary has
\href{https://www.nytimes.com/2020/06/09/us/politics/atlanta-voting-georgia-primary.html}{sharpened
fears of voter suppression} in a state where the governor, Brian Kemp,
is Republican and the House and Senate Republican-controlled. ``We never
thought we'd be talking about voting rights a half-century on from the
civil rights movement,'' Andrea Young, the executive director of the
A.C.L.U. of Georgia, told me. ``The dysfunction is distressing ahead of
what will be a highly contested general election, the most important of
our lifetimes. We believed in America's promise, not a George Wallace
presidency.''

Image

Voters line up at Christian City, an assisted living home, to cast their
ballots in Union City, Ga., on June 9.Credit...Dustin Chambers/Reuters

That promise has generally proved illusory when it comes to race.
Throughout American history white cruelty in keeping blacks down has
been matched only by white ingenuity in finding new ways to do so. Trump
is part of that tradition. He has doubled down of late on the same
images of lawless blacks that sustained Jim Crow.

Forman ``toggles back and forth,'' as he put it, on the question of how
much has changed between the time his father was arrested, beaten and
held incommunicado by the L.A. police in the 1950s and his 11-year-old
son insisting, today, on joining the countrywide uprising against racial
injustice.

``I have never seen anything like this in my lifetime,'' Forman told me.
``I have many white friends with whom I have tried to raise issues of
racial inequality and injustice. But it was never front and center in
their lives. Now they bring it up nonstop. Perhaps it's like when people
saw the images of police attack dogs being set on black children in
Birmingham in 1963. You know, `I can't believe that!' Maybe this is how
that felt.''

``Like Emmett Till in the casket, the Floyd image made clear no black
person is safe,'' Carol Anderson, a professor here at Emory University
and author of ``White Rage,'' told me.

The question of course is whether this awakening can achieve what even
the Civil Rights Movement could not: the full \emph{humanization} of
black Americans. ``It has been said that the opposite of criminalization
is humanization,'' Jonathan Rapping, an Atlanta defense attorney who has
focused on providing equal justice for marginalized communities, said.

In other words, when will America awaken to the fact that Rayshard
Brooks was a human being, in full, who should not have ended up dead
because he dozed off in his car in the drive-thru lane of an Atlanta
Wendy's?

\textbf{\href{https://www.nytimes.com/2020/06/14/us/videos-rayshard-brooks-shooting-atlanta-police.html}{I
have watched the video too often}.} Brooks groggy in his parked car on
June 12. The initially amiable 41-minute encounter between Brooks and
officers, including Garrett Rolfe. Brooks's reasonable offer to lock his
car and walk to his sister's place. The tussle when Rolfe abruptly moves
to make a DUI arrest and handcuff Brooks. A Taser grabbed by Brooks from
an officer. Brooks running. Turning and firing the Taser toward Rolfe,
who responds with two bullets into Brooks's back.

``What I see is a shooting that was unnecessary,'' Sam Starks, a black
Atlanta lawyer, told me. ``Park the car. Lock it. Take that person home.
Brooks was on probation. He is terrified. He knows the cage he's headed
for.''

Unarmed, Brooks was no threat to anyone. His car was stationary. He
would not be dead if he was white. He would be at his sister's place.

Having served a one-year sentence for credit card fraud, Brooks was in
the maw of a system that condemns young black lives long after the cell.
A poor black man's chances of finding work on probation resemble a
snowball's chances of surviving hell.

In an interview
\href{https://www.cnn.com/2020/06/17/us/rayshard-brooks-interview-reconnect-life-after-incarceration/index.html}{in
February with Reconnect}, a company that works to combat mass
incarceration and recidivism, Brooks, 27, said: ``I just feel like some
of the system could look at us like individuals. We do have lives. It's
just a mistake we made.'' A mistake is not a reason to be treated ``as
if we are animals.''

Image

Protesters in front of the Wendy's where the police killed Rayshard
Brooks.Credit...Joshua Rashaad McFadden for The New York Times

Image

Demonstrators raise their fists at a parade of passing motorcyclists
riding in honor of Ahmaud Arbery on May 9 in Brunswick, Ga.Credit...Sean
Rayford/Getty Images

Ahmaud Arbery, 25, another young black man killed in Georgia this year,
was hunted down like an animal on Feb. 23 as he jogged through Satilla
Shores, near Brunswick, a coastal neighborhood of pleasant bungalows
beneath live oaks garlanded with Spanish moss.

Gregory McMichael, 64, and his son Travis McMichael, 34, both white,
grabbed a revolver and a shotgun, piled into their pick-up truck and
pursued Arbery --- convinced, they told the police, that he looked like
a suspect in recent break-ins. In a video that took months to emerge,
Travis is seen shooting Arbery dead at point-blank range as they tussle
over his shotgun in the bright sunlight.

\includegraphics{https://static01.nyt.com/images/2020/06/09/autossell/HateCrime_Thumb1/HateCrime_Thumb1-videoSixteenByNineJumbo1600.jpg}

No arrest was made at the time. The McMichaels had acted in accordance
with
\href{https://www.nytimes.com/article/ahmaud-arbery-citizen-arrest-law-georgia.html}{Georgia's
citizen's arrest statute}! Travis McMichael had fired in self-defense!
This was the initial police view.

So, on the one hand, a dead black man, Arbery, and two white men with
guns who walk away. On the other, a young black man, Brooks, dozing in a
car, and police try to \emph{arrest him,} and he ends up dead.

A growing outcry --- driven by social media, a
\href{https://www.nytimes.com/2020/04/26/us/ahmed-arbery-shooting-georgia.html}{groundbreaking
article in April} by my colleague Richard Fausset, and at last the
release in early May of the incriminating video --- led to the
McMichaels' arrest on May 7. It took 74 days. The video had been in the
possession of the police from Day 1.

Three days after the Brooks killing, on an unseasonably cool Georgia
morning, I joined a protest in downtown Atlanta. ``We are done dyin',''
a banner proclaimed. A large crowd, mostly young, of every hue, milled
around. I fell into conversation with Justin Brock, a white professional
skateboarder, who had brought along his 7-year-old son, Jasper.

``We need education reforms,'' he told me. ``We need to teach the
terrible things we did to make this country. They are known and hidden
at the same time.''

Brock looked hard at me. ``I want to show my son the world and
\emph{what actually goes on}.''

Jamal Harrison Bryant, a pastor, grabbed a microphone. ``This is not a
moment, it's a \emph{movement},'' he said. Cheers echoed around the
still-ghostly pandemic-hit city.

``We're sick and tired of every week having a different hashtag for
innocent black lives,''
\href{https://www.facebook.com/jamalbryant/videos/2020-naacp-georgia-march-to-state-capital/2797160607062525/}{he
continued}. ``We're sick and tired of them finding money for Georgia
Tech but finding no money for Morehouse and Spelman.''

Catherine Quashie, a black woman, was standing next to me. Bryant is her
pastor. She told me it took her two hours and 47 minutes to vote in
Stonecrest, a city southeast of Atlanta. I heard stories of seven-hour
waits in Fulton County. In upscale Buckhead voters were in and out in 10
minutes. ``The encouraging thing,'' Quashie said, ``is nobody left the
line.''

Most of her family is in Europe. ``They keep asking me: `WHAT IS GOING
ON IN AMERICA?'''

\textbf{When I leave the demonstration,} I drive southeast out of
Atlanta toward Arbery's hometown, Brunswick, five hours away on the
Atlantic Ocean, across God's country, where nobody wears a mask.

Roger Johnson runs a fruit stand near McRae, in an area famous for its
sweet Vidalia onions and, of course, Georgia peaches. His daughter,
Taylor, helps out. ``This is the Bible Belt,'' Johnson tells me.
``Twelve churches between here and the Interstate.'' He's a stocky,
friendly guy with a mustache, a belly and narrow, shrewd eyes. A sign
outside says TOMATOES and ONIONS in red and blue letters, with TRUMP's
name at the top.

So why, I ask, do you like the president? ``Because he doesn't take any
crap. Because he cannot be bought by other pols. Because he's not a
career politician. He might stretch the truth a little but don't we all?
And it's the news that stretches it a lot.''

Those knowing eyes look me over. Watermelons, Johnson advises, are a
little mushy if they give a dull thud when tapped. ``Should be like
knocking on a door,'' he says. Noted. ``People work hard for what they
got,'' he continues. ``They should not face looting.''

I like this man. I disagree with him on just about everything. I was a
foreign correspondent much of my life. This, for a New Yorker, is
foreign soil. It's interesting, if unfashionable, to consider everything
from a different angle, to imagine your way into a stranger's life, to
have conversations that involve more than the quest for the wittiest
expression of agreement on Trump's perfidy.

What about the killing of George Floyd? ``They arrested and charged the
officer who did that, and the other three standing there like dummies
also need to be prosecuted,'' Johnson says. ``But that's no reason to
tear up stores.''

Image

Roger Johnson at his produce stand.Credit...Audra Melton for The New
York Times

Image

Jerome Wilson, bottom right, is a veteran and a friend and customer of
Mr. Johnson's.Credit...Audra Melton for The New York Times

Jerome Wilson, a black vet, strides in, wearing a 25th Infantry Division
red cap. He's from Jesup, 70 miles down the road and likes the fruit
here enough to make the journey. He tells me about being in military
uniform, about to be deployed to Vietnam, and having to enter the bus
taking him to Fort Benning through the back door.

``I was going to fight for my country, maybe die, and I was only good
enough for the back doors,'' he says.

It's not true that nothing has changed. Many things have, for the
better, in the great fight for racial justice. It's just the
\emph{essence} that has not changed. Wilson and Johnson stand there, arm
in arm, a black man and a white man, friends. That, too, is America,
perhaps especially the South, ever ready to surprise you when you write
it off.

Morris Selph, Johnson's father-in-law, put up the Trump sign. Selph
tells me he's ``had more brag on that sign than people condemning it.''
Seated on a plastic chair at roadside, red faced and bearded, he says he
likes Trump a lot.

``Business went up. Toughest president I've ever seen. He's the
Energizer Bunny. Ain't nobody going to knock him down.''

Trump's lies are viewed here as straight talk. His detention of child
migrants in camps at the border is a stand for law and order. His
toughness is a remedy for moral decay. ``In schools here they still
paddle,'' Selph says approvingly.

``He's a redneck,'' Taylor, 20, says with a smile.

Her mother, Elsie Johnson, trained as an accountant. ``What Trump sees
is not people, but numbers. He can't see people at all. China, pay more!
But people, no. Maybe that helps him make the tough decisions. He toots
his own horn, but I think he's looking out for America.''

Image

A campaign sign in Baxley, Ga.Credit...Audra Melton for The New York
Times

Image

South Brunswick Street in Jesup, Ga.Credit...Audra Melton for The New
York Times

Away into the distance, green and undulating, America unfurls. Loggers
haul timber. Stores advertise guns and ammo. Pawn shops abound. Outside
a church a sign proclaims: ``Hell is real. Hell is hot. Jesus is coming.
Ready or not.''

This is Trump country, even if Trump doesn't know
\href{https://www.facebook.com/watch/?v=10153577166571880}{which way is
up in the Bible}. Georgia was flattened by the Union Army in the Civil
War, much of Atlanta burned to the ground. This humiliation has never
been entirely digested by many white Georgians. Defiance simmers below
the surface of Southern gentility. The lost cause of the Confederacy has
a tenacious hold; and that cause comes down to white dominion, Trump's
leitmotif.

\textbf{There's a small shrine} at the corner of Holmes Road and Satilla
Drive where Ahmaud Arbery was killed. Flowers half-cover a plaque that
reads: ``It's hard to forget someone who gave us so much to remember.''
Yet Arbery was nearly forgotten, just another black man cut down by
white men in a tranquil subdivision.

Arbery was quiet, polite and unassuming, friends and family told me. He
was killed a couple of miles from his home, a white bungalow with blue
shutters that now has a ``For Sale'' sign outside. In the house
opposite, Jenifer Bolin fumes. ``Citizen's arrest, my ass! They were
racists.''

If Arbery was not forgotten, if the McMichaels were indicted this week
by a grand jury for felony murder, if \#IRunWithMaud has become a global
hashtag signifying the fight against racism, it is thanks in part to
Jason Vaughn, a force of nature who as a football coach at Brunswick
High School coached Arbery.

Image

Jason Vaughn, an assistant coach at Brunwick High School, made sure
Ahmaud Arbery was not forgotten.Credit...Malcolm Jackson for The New
York Times

I met Vaughn at a Mexican joint. The case, long dead in the water, had
troubled him from the outset. The whole thing was a fiasco: white
connections and impunity denying justice in the good old way of the Deep
South. With the help of his brother, a lawyer, Vaughn pressed to get the
police report and also helped start a Facebook page to coordinate
pressure.

``The wheels on the bus of justice turn slowly,'' Vaughn told me. ``But
this bus had \emph{no wheels} until we got engaged. A football coach
should not have to study law and policing to bring this about.''

Image

Wanda Cooper-Jones stands near the spot where her son Ahmaud Arbery was
fatally shot while jogging. She says her son ran every day to clear his
mind.Credit...Sarah Blake Morgan/Associated Press

Back in Atlanta, I met Wanda Cooper-Jones, Arbery's mother. She is a
woman of great poise and fierce dignity. I asked her what she would say
to the McMichaels.

``To the father I would say, as mother to father, our job as parents is
to train our children in the way to go. I think you failed Travis in
that. How can you love and teach them hate? To Travis I would say, I
don't really know, but my heart goes out to him because he was deprived
of love.''

She thought for a moment. ``People who are hurt hurt other people.
People who are loved love other people.''

\textbf{James Baldwin wrote:} ``It demands great spiritual resilience
not to hate the hater whose foot is on your neck, and an even greater
miracle of perception and charity not to teach your child to hate.''

Four years ago,
\href{https://www.nytimes.com/2016/09/11/opinion/sunday/we-need-somebody-spectacular-views-from-trump-country.html}{I
traveled to Kentucky} and came away with the clear impression a Trump
victory was likely. It was in the air, a heady excitement. Today the
Trump balloon feels deflated, his old race-baiting, anti-elite,
anti-science lines tired. He still has a hard core of support. The
biggest mistake for Democrats would be to think he cannot win. Still, I
came away from Georgia thinking the energy is with the people who want
Trump out, and his defeat is more likely than not.

The response to the killing of Arbery and Brooks has been remarkable.
The Georgia Legislature this week passed a hate crime bill that Governor
Kemp says he intends to sign into law. Georgia was one of only four
states holding out against such legislation.

Arbery's mother, Cooper-Jones,
\href{https://www.nytimes.com/2020/06/09/opinion/hate-crime-bill-ahmaud-arbery.html}{was
a vigorous proponent of the law.} She has emerged as a national figure.
This month, she met Trump at the White House before he signed an
executive order banning police use of chokeholds ``unless an officer's
life is at risk,'' as he put it, and encouraging the adoption of less
lethal weapons.

Image

President Trump signed the Executive Order on Safe Policing for Safe
Communities in the Rose Garden on June 16.Credit...Doug Mills/The New
York Times

From abandonment as her son was killed to the White House in just four
months is quite a journey. How did she feel, I asked, about meeting with
Trump? ``I respect him as the president. He is a man and a human
being,'' Cooper-Jones told me. ``I was criticized, but he gave time to
listen to a mother in pain and that is what mattered.''

Cooper-Jones did the right thing, setting an example of brave cordiality
in an age of facile declamation. America could use more listening across
its lines of violent fracture. Confronting racial injustice involves
recognition and reconciliation, however painful. That was Mandela's
message. My parents were South African. I know that.

In Atlanta, recent months have shown that for all its black
professionals and power, the city is as much in need of reform as any
other. ``As a public defender, you would not know white people are
breaking any laws,'' Rapping, the defense attorney, told me. ``Like
every city, Atlanta has been shaped by a 400-year-old narrative that
says black or brown people don't matter.''

The system that turns black kids into case numbers, that holds young
black men in cells for months pretrial because they cannot put up money
bonds, that
\href{https://www.aclu.org/report/tale-two-countries-racially-targeted-arrests-era-marijuana-reform}{prosecutes
for smoking marijuana}, has to change. It's a form of violence, and it
breeds violence. ``Law and order'' is no answer.

\textbf{Every weekend, Georgians in their ever-growing diversity} ---
interracial couples, people in hijabs, gay couples --- swarm over Stone
Mountain, whose North Face is carved with bas-reliefs of Confederate
generals. It's as if a new Georgia, defying its racist past, is heeding
King's 1963 ``I Have a Dream'' speech in which he said, ``Let Freedom
Ring from Stone Mountain of Georgia!''

One day, I went to Decatur, a city in the Metro Atlanta sprawl, to see a
Confederate monument, a 30-foot obelisk engraved with tributes to the
``loyalty and truth'' of men ``who held fast to the faith as it was
given by the fathers of the Republic.'' Graffiti --- ``No justice, No
Peace''; ``Black Lives Matter'' --- had been scrawled all over it. A few
days later, on a judge's order, it was gone, hoisted out by a crane.
This is not the election, or the country, it was before Ahmaud Arbery
and George Floyd and Rayshard Brooks.

Image

A Confederate monument in Decatur, Ga.~ It was removed on June
18.Credit...Johnathon Kelso for The New York Times

Andrea Young, the A.C.L.U. director, is the daughter of Andrew Young, an
Atlanta mayor, United Nations ambassador and civil rights icon. I asked
her if there was reason to hope that this moment could accomplish what
that movement could not.

``Nobody has believed more in the promise and mythology of America than
blacks,'' she told me. ``We have believed all people were created equal,
fought over generations for the truth of the statement. The fact I am
here means I am descended from people who, even enslaved, did not give
up hope. To do so now would be a betrayal.''

\emph{The Times is committed to publishing}
\href{https://www.nytimes.com/2019/01/31/opinion/letters/letters-to-editor-new-york-times-women.html}{\emph{a
diversity of letters}} \emph{to the editor. We'd like to hear what you
think about this or any of our articles. Here are some}
\href{https://help.nytimes.com/hc/en-us/articles/115014925288-How-to-submit-a-letter-to-the-editor}{\emph{tips}}\emph{.
And here's our email:}
\href{mailto:letters@nytimes.com}{\emph{letters@nytimes.com}}\emph{.}

\emph{Follow The New York Times Opinion section on}
\href{https://www.facebook.com/nytopinion}{\emph{Facebook}}\emph{,}
\href{http://twitter.com/NYTOpinion}{\emph{Twitter (@NYTopinion)}}
\emph{and}
\href{https://www.instagram.com/nytopinion/}{\emph{Instagram}}\emph{.}

Advertisement

\protect\hyperlink{after-bottom}{Continue reading the main story}

\hypertarget{site-index}{%
\subsection{Site Index}\label{site-index}}

\hypertarget{site-information-navigation}{%
\subsection{Site Information
Navigation}\label{site-information-navigation}}

\begin{itemize}
\tightlist
\item
  \href{https://help.nytimes.com/hc/en-us/articles/115014792127-Copyright-notice}{©~2020~The
  New York Times Company}
\end{itemize}

\begin{itemize}
\tightlist
\item
  \href{https://www.nytco.com/}{NYTCo}
\item
  \href{https://help.nytimes.com/hc/en-us/articles/115015385887-Contact-Us}{Contact
  Us}
\item
  \href{https://www.nytco.com/careers/}{Work with us}
\item
  \href{https://nytmediakit.com/}{Advertise}
\item
  \href{http://www.tbrandstudio.com/}{T Brand Studio}
\item
  \href{https://www.nytimes.com/privacy/cookie-policy\#how-do-i-manage-trackers}{Your
  Ad Choices}
\item
  \href{https://www.nytimes.com/privacy}{Privacy}
\item
  \href{https://help.nytimes.com/hc/en-us/articles/115014893428-Terms-of-service}{Terms
  of Service}
\item
  \href{https://help.nytimes.com/hc/en-us/articles/115014893968-Terms-of-sale}{Terms
  of Sale}
\item
  \href{https://spiderbites.nytimes.com}{Site Map}
\item
  \href{https://help.nytimes.com/hc/en-us}{Help}
\item
  \href{https://www.nytimes.com/subscription?campaignId=37WXW}{Subscriptions}
\end{itemize}
