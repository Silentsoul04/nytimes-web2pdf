Sections

SEARCH

\protect\hyperlink{site-content}{Skip to
content}\protect\hyperlink{site-index}{Skip to site index}

\href{https://www.nytimes.com/section/us}{U.S.}

\href{https://myaccount.nytimes.com/auth/login?response_type=cookie\&client_id=vi}{}

\href{https://www.nytimes.com/section/todayspaper}{Today's Paper}

\href{/section/us}{U.S.}\textbar{}Many Students Will Be in Classrooms
Only Part of the Week This Fall

\url{https://nyti.ms/2NxYQLp}

\begin{itemize}
\item
\item
\item
\item
\item
\end{itemize}

\href{https://www.nytimes.com/news-event/coronavirus?action=click\&pgtype=Article\&state=default\&region=TOP_BANNER\&context=storylines_menu}{The
Coronavirus Outbreak}

\begin{itemize}
\tightlist
\item
  live\href{https://www.nytimes.com/2020/08/04/world/coronavirus-cases.html?action=click\&pgtype=Article\&state=default\&region=TOP_BANNER\&context=storylines_menu}{Latest
  Updates}
\item
  \href{https://www.nytimes.com/interactive/2020/us/coronavirus-us-cases.html?action=click\&pgtype=Article\&state=default\&region=TOP_BANNER\&context=storylines_menu}{Maps
  and Cases}
\item
  \href{https://www.nytimes.com/interactive/2020/science/coronavirus-vaccine-tracker.html?action=click\&pgtype=Article\&state=default\&region=TOP_BANNER\&context=storylines_menu}{Vaccine
  Tracker}
\item
  \href{https://www.nytimes.com/2020/08/02/us/covid-college-reopening.html?action=click\&pgtype=Article\&state=default\&region=TOP_BANNER\&context=storylines_menu}{College
  Reopening}
\item
  \href{https://www.nytimes.com/live/2020/08/04/business/stock-market-today-coronavirus?action=click\&pgtype=Article\&state=default\&region=TOP_BANNER\&context=storylines_menu}{Economy}
\end{itemize}

Advertisement

\protect\hyperlink{after-top}{Continue reading the main story}

Supported by

\protect\hyperlink{after-sponsor}{Continue reading the main story}

\hypertarget{many-students-will-be-in-classrooms-only-part-of-the-week-this-fall}{%
\section{Many Students Will Be in Classrooms Only Part of the Week This
Fall}\label{many-students-will-be-in-classrooms-only-part-of-the-week-this-fall}}

Some American school districts are beginning to announce hybrid
schedules that include a mix of online and in-school learning,
presenting a difficult challenge for working parents.

\includegraphics{https://static01.nyt.com/images/2020/06/28/us/26virus-schools01/26virus-schools01-articleLarge-v3.jpg?quality=75\&auto=webp\&disable=upscale}

\href{https://www.nytimes.com/by/dana-goldstein}{\includegraphics{https://static01.nyt.com/images/2018/06/12/multimedia/author-dana-goldstein/author-dana-goldstein-thumbLarge.png}}\href{https://www.nytimes.com/by/eliza-shapiro}{\includegraphics{https://static01.nyt.com/images/2018/12/28/multimedia/author-eliza-shapiro/author-eliza-shapiro-thumbLarge.png}}

By \href{https://www.nytimes.com/by/dana-goldstein}{Dana Goldstein} and
\href{https://www.nytimes.com/by/eliza-shapiro}{Eliza Shapiro}

\begin{itemize}
\item
  Published June 26, 2020Updated July 8, 2020
\item
  \begin{itemize}
  \item
  \item
  \item
  \item
  \item
  \end{itemize}
\end{itemize}

As school districts across the country began to reveal reopening plans
this week, parents and students were forced to grapple with a difficult
reality: It is unlikely that many schools will return to a normal
schedule this fall, requiring the grind of
\href{https://www.nytimes.com/2020/06/05/us/coronavirus-education-lost-learning.html}{online
learning},
\href{https://www.nytimes.com/2020/05/29/us/coronavirus-child-care-centers.html}{makeshift
child care} and
\href{https://www.nytimes.com/2020/06/03/business/economy/coronavirus-working-women.html}{stunted
workdays} to continue.

Students in Seattle are likely to go to school in person only once or
twice a week, officials said. Half of
\href{https://www.omaha.com/news/ops-outlines-plan-for-fall-school-reopening-with-only-half-of-students-in-buldings-at/article_ca1b2f72-9cdd-5cfe-a77a-2a5d0821dfa6.html}{Omaha's
students} will attend Monday and Tuesday, and the other half Thursday
and Friday, rotating Wednesdays. And Fairfax County, Va., outside
Washington, said students would spend at least two days a week in class,
with the rest online.

The governors of Connecticut and New Jersey announced guidance that they
said would allow students to return to school, but left the details up
to districts, with Gov. Phil Murphy of New Jersey acknowledging on
Friday that some schools would likely need to adopt a hybrid model and
restrict daily attendance.

Many of the nation's largest districts have yet to announce plans,
although Mayor Bill de Blasio of New York suggested on Friday that all
of the city's 1.1 million students are unlikely to return full-time
during the pandemic. ``One day there's going to be a vaccine,'' he said,
``and I think that's the day when you're going to see things go back 100
percent --- to every kid in the classroom --- as normal.''

Reopening decisions are likely to vary greatly based on the size and
density of districts, as well as region. In the South and West, where
political leaders have been more eager to reopen their economies despite
a recent upswing in coronavirus cases, school systems may be more likely
to return to a full-time schedule, albeit with distance restrictions,
new sanitation procedures and mask recommendations.

But for districts that hope to closely follow guidance from the Centers
for Disease Control and Prevention, it would be nearly impossible to
maintain sanitation and social distancing with normal school attendance.
Class sizes in many schools would need to be cut by more than half. And
while health experts generally recommend face coverings for both staff
and students, some educators and parents say that is not realistic,
especially for the youngest children.

\hypertarget{latest-updates-global-coronavirus-outbreak}{%
\section{\texorpdfstring{\href{https://www.nytimes.com/2020/08/04/world/coronavirus-cases.html?action=click\&pgtype=Article\&state=default\&region=MAIN_CONTENT_1\&context=storylines_live_updates}{Latest
Updates: Global Coronavirus
Outbreak}}{Latest Updates: Global Coronavirus Outbreak}}\label{latest-updates-global-coronavirus-outbreak}}

Updated 2020-08-04T20:30:24.650Z

\begin{itemize}
\tightlist
\item
  \href{https://www.nytimes.com/2020/08/04/world/coronavirus-cases.html?action=click\&pgtype=Article\&state=default\&region=MAIN_CONTENT_1\&context=storylines_live_updates\#link-1228a480}{Novavax
  sees encouraging results from two studies of its experimental
  vaccine.}
\item
  \href{https://www.nytimes.com/2020/08/04/world/coronavirus-cases.html?action=click\&pgtype=Article\&state=default\&region=MAIN_CONTENT_1\&context=storylines_live_updates\#link-4825b93}{Public
  and private schools in Maryland and elsewhere are divided over
  in-person instruction.}
\item
  \href{https://www.nytimes.com/2020/08/04/world/coronavirus-cases.html?action=click\&pgtype=Article\&state=default\&region=MAIN_CONTENT_1\&context=storylines_live_updates\#link-50f7386d}{The
  United Nations calls on policymakers to `plan thoroughly for school
  reopenings.'}
\end{itemize}

\href{https://www.nytimes.com/2020/08/04/world/coronavirus-cases.html?action=click\&pgtype=Article\&state=default\&region=MAIN_CONTENT_1\&context=storylines_live_updates}{See
more updates}

More live coverage:
\href{https://www.nytimes.com/live/2020/08/04/business/stock-market-today-coronavirus?action=click\&pgtype=Article\&state=default\&region=MAIN_CONTENT_1\&context=storylines_live_updates}{Markets}

Those complications are likely to prompt many districts --- where
administrators must decide how to implement the broad guidelines from
federal and state health officials --- to adopt a hybrid model in which
students will spend some time in the classroom but a significant portion
of the week at home.

Although that reality has been apparent to many educators for weeks, it
is just beginning to confront parents. Some are finding out this week
that their scramble to balance their own jobs with their children's
education and daily care will continue for many months, if not all of
next school year.

``Everyone, including myself, wants to go back every day,'' said Naomi
Peña, a mother of three New York City public school students. ``I've
come to terms with the fact that that's wishful thinking.''

Ms. Peña is hopeful that her children will be able to report to school
every other day for the fall semester. ``These kids need some sense of
normalcy back,'' she said, adding, ``Every single parent I know, their
whole routine they have cherished and worked so hard to preserve is
completely out the window, and in the trash can.''

The enormous strain caused by remote learning and limited child-care
options has been particularly hard on working mothers, said Julie
Kashen, the director for women's economic justice at the Century
Foundation, a public policy research group. ``We've worked so hard to
have choices for women in the work force, and it doesn't feel like we
have real choices right now,'' she said.

Experts on working families are concerned that employers who were
flexible in the spring may lose their patience come fall, even if
schools do not fully reopen. ``Families have used up any slack we've
created in the system,'' said Brigid Schulte, who runs the
\href{https://www.newamerica.org/better-life-lab/}{Better Life Lab} at
New America, a think tank.

Most parents,
\href{https://www.nytimes.com/2020/07/11/us/virus-teachers-classrooms.html}{teachers}
and school leaders acknowledge that remote learning did not work as well
as it should have during the spring semester, and that it will need to
improve rapidly for the coming year. The average American student is
expected to return to school
\href{https://www.nytimes.com/2020/06/05/us/coronavirus-education-lost-learning.html}{significantly
behind academically}, with low-income, black and Hispanic students
experiencing the greatest learning losses.

\includegraphics{https://static01.nyt.com/images/2020/06/26/us/26virus-schools02/merlin_173956410_e5be296e-63a6-4f0a-ac6d-f4057fbb8db1-articleLarge.jpg?quality=75\&auto=webp\&disable=upscale}

Many schools are providing professional development to teachers this
summer and also reconsidering how they use technology. Still, no matter
how much online learning improves, keeping school buildings shuttered
will have a profound impact on children, especially the most vulnerable.

Those students have endured economic strife, parental unemployment and
the burden of caring for younger siblings while being isolated from
their own friends. In cities like New York, some are struggling with the
trauma of seeing their family members,
\href{https://www.nytimes.com/2020/04/07/obituaries/sandra-santos-vizcaino-dead-coronavirus.html}{teachers
and principals} die of the virus. Schools offer a support system that is
impossible to replicate online.

As they make reopening plans, many districts are taking into account the
special needs of some students. In Seattle, where the schools
\href{https://www.seattleschools.org/district/calendars/news/what_s_new/coronavirus_update}{announced
this week} that their goal was to provide at least two days per week of
in-person instruction to elementary students and one day to middle and
high school students, officials said children with disabilities, those
learning English and those living in poverty would be given priority for
additional in-school support.

Educators crafting reopening plans face a daunting set of challenges
this summer, from how to procure enough masks and cleaning supplies, to
how to reduce class sizes and redesign lesson plans to adhere to social
distancing guidelines.

\href{https://www.nytimes.com/news-event/coronavirus?action=click\&pgtype=Article\&state=default\&region=MAIN_CONTENT_3\&context=storylines_faq}{}

\hypertarget{the-coronavirus-outbreak-}{%
\subsubsection{The Coronavirus Outbreak
›}\label{the-coronavirus-outbreak-}}

\hypertarget{frequently-asked-questions}{%
\paragraph{Frequently Asked
Questions}\label{frequently-asked-questions}}

Updated August 4, 2020

\begin{itemize}
\item ~
  \hypertarget{i-have-antibodies-am-i-now-immune}{%
  \paragraph{I have antibodies. Am I now
  immune?}\label{i-have-antibodies-am-i-now-immune}}

  \begin{itemize}
  \tightlist
  \item
    As of right
    now,\href{https://www.nytimes.com/2020/07/22/health/covid-antibodies-herd-immunity.html?action=click\&pgtype=Article\&state=default\&region=MAIN_CONTENT_3\&context=storylines_faq}{that
    seems likely, for at least several months.} There have been
    frightening accounts of people suffering what seems to be a second
    bout of Covid-19. But experts say these patients may have a
    drawn-out course of infection, with the virus taking a slow toll
    weeks to months after initial exposure. People infected with the
    coronavirus typically
    \href{https://www.nature.com/articles/s41586-020-2456-9}{produce}
    immune molecules called antibodies, which are
    \href{https://www.nytimes.com/2020/05/07/health/coronavirus-antibody-prevalence.html?action=click\&pgtype=Article\&state=default\&region=MAIN_CONTENT_3\&context=storylines_faq}{protective
    proteins made in response to an
    infection}\href{https://www.nytimes.com/2020/05/07/health/coronavirus-antibody-prevalence.html?action=click\&pgtype=Article\&state=default\&region=MAIN_CONTENT_3\&context=storylines_faq}{.
    These antibodies may} last in the body
    \href{https://www.nature.com/articles/s41591-020-0965-6}{only two to
    three months}, which may seem worrisome, but that's perfectly normal
    after an acute infection subsides, said Dr. Michael Mina, an
    immunologist at Harvard University. It may be possible to get the
    coronavirus again, but it's highly unlikely that it would be
    possible in a short window of time from initial infection or make
    people sicker the second time.
  \end{itemize}
\item ~
  \hypertarget{im-a-small-business-owner-can-i-get-relief}{%
  \paragraph{I'm a small-business owner. Can I get
  relief?}\label{im-a-small-business-owner-can-i-get-relief}}

  \begin{itemize}
  \tightlist
  \item
    The
    \href{https://www.nytimes.com/article/small-business-loans-stimulus-grants-freelancers-coronavirus.html?action=click\&pgtype=Article\&state=default\&region=MAIN_CONTENT_3\&context=storylines_faq}{stimulus
    bills enacted in March} offer help for the millions of American
    small businesses. Those eligible for aid are businesses and
    nonprofit organizations with fewer than 500 workers, including sole
    proprietorships, independent contractors and freelancers. Some
    larger companies in some industries are also eligible. The help
    being offered, which is being managed by the Small Business
    Administration, includes the Paycheck Protection Program and the
    Economic Injury Disaster Loan program. But lots of folks have
    \href{https://www.nytimes.com/interactive/2020/05/07/business/small-business-loans-coronavirus.html?action=click\&pgtype=Article\&state=default\&region=MAIN_CONTENT_3\&context=storylines_faq}{not
    yet seen payouts.} Even those who have received help are confused:
    The rules are draconian, and some are stuck sitting on
    \href{https://www.nytimes.com/2020/05/02/business/economy/loans-coronavirus-small-business.html?action=click\&pgtype=Article\&state=default\&region=MAIN_CONTENT_3\&context=storylines_faq}{money
    they don't know how to use.} Many small-business owners are getting
    less than they expected or
    \href{https://www.nytimes.com/2020/06/10/business/Small-business-loans-ppp.html?action=click\&pgtype=Article\&state=default\&region=MAIN_CONTENT_3\&context=storylines_faq}{not
    hearing anything at all.}
  \end{itemize}
\item ~
  \hypertarget{what-are-my-rights-if-i-am-worried-about-going-back-to-work}{%
  \paragraph{What are my rights if I am worried about going back to
  work?}\label{what-are-my-rights-if-i-am-worried-about-going-back-to-work}}

  \begin{itemize}
  \tightlist
  \item
    Employers have to provide
    \href{https://www.osha.gov/SLTC/covid-19/standards.html}{a safe
    workplace} with policies that protect everyone equally.
    \href{https://www.nytimes.com/article/coronavirus-money-unemployment.html?action=click\&pgtype=Article\&state=default\&region=MAIN_CONTENT_3\&context=storylines_faq}{And
    if one of your co-workers tests positive for the coronavirus, the
    C.D.C.} has said that
    \href{https://www.cdc.gov/coronavirus/2019-ncov/community/guidance-business-response.html}{employers
    should tell their employees} -\/- without giving you the sick
    employee's name -\/- that they may have been exposed to the virus.
  \end{itemize}
\item ~
  \hypertarget{should-i-refinance-my-mortgage}{%
  \paragraph{Should I refinance my
  mortgage?}\label{should-i-refinance-my-mortgage}}

  \begin{itemize}
  \tightlist
  \item
    \href{https://www.nytimes.com/article/coronavirus-money-unemployment.html?action=click\&pgtype=Article\&state=default\&region=MAIN_CONTENT_3\&context=storylines_faq}{It
    could be a good idea,} because mortgage rates have
    \href{https://www.nytimes.com/2020/07/16/business/mortgage-rates-below-3-percent.html?action=click\&pgtype=Article\&state=default\&region=MAIN_CONTENT_3\&context=storylines_faq}{never
    been lower.} Refinancing requests have pushed mortgage applications
    to some of the highest levels since 2008, so be prepared to get in
    line. But defaults are also up, so if you're thinking about buying a
    home, be aware that some lenders have tightened their standards.
  \end{itemize}
\item ~
  \hypertarget{what-is-school-going-to-look-like-in-september}{%
  \paragraph{What is school going to look like in
  September?}\label{what-is-school-going-to-look-like-in-september}}

  \begin{itemize}
  \tightlist
  \item
    It is unlikely that many schools will return to a normal schedule
    this fall, requiring the grind of
    \href{https://www.nytimes.com/2020/06/05/us/coronavirus-education-lost-learning.html?action=click\&pgtype=Article\&state=default\&region=MAIN_CONTENT_3\&context=storylines_faq}{online
    learning},
    \href{https://www.nytimes.com/2020/05/29/us/coronavirus-child-care-centers.html?action=click\&pgtype=Article\&state=default\&region=MAIN_CONTENT_3\&context=storylines_faq}{makeshift
    child care} and
    \href{https://www.nytimes.com/2020/06/03/business/economy/coronavirus-working-women.html?action=click\&pgtype=Article\&state=default\&region=MAIN_CONTENT_3\&context=storylines_faq}{stunted
    workdays} to continue. California's two largest public school
    districts --- Los Angeles and San Diego --- said on July 13, that
    \href{https://www.nytimes.com/2020/07/13/us/lausd-san-diego-school-reopening.html?action=click\&pgtype=Article\&state=default\&region=MAIN_CONTENT_3\&context=storylines_faq}{instruction
    will be remote-only in the fall}, citing concerns that surging
    coronavirus infections in their areas pose too dire a risk for
    students and teachers. Together, the two districts enroll some
    825,000 students. They are the largest in the country so far to
    abandon plans for even a partial physical return to classrooms when
    they reopen in August. For other districts, the solution won't be an
    all-or-nothing approach.
    \href{https://bioethics.jhu.edu/research-and-outreach/projects/eschool-initiative/school-policy-tracker/}{Many
    systems}, including the nation's largest, New York City, are
    devising
    \href{https://www.nytimes.com/2020/06/26/us/coronavirus-schools-reopen-fall.html?action=click\&pgtype=Article\&state=default\&region=MAIN_CONTENT_3\&context=storylines_faq}{hybrid
    plans} that involve spending some days in classrooms and other days
    online. There's no national policy on this yet, so check with your
    municipal school system regularly to see what is happening in your
    community.
  \end{itemize}
\end{itemize}

Instead of clustering around tables for group projects, teenagers will
likely receive more individual assignments, with the students seated at
desks facing forward. Younger children won't be able to pile onto a soft
rug for story time; instead, they will be required to sit in clearly
marked spaces, six feet apart.

Many districts are surveying parents to better understand their comfort
level with reopening school buildings. They are finding a significant
minority --- up to a third of parents in some large districts --- do not
want to send their children into classrooms, according to Mike Magee,
chief executive of Chiefs for Change, a coalition of district and state
education leaders.

Most districts are expected to give parents the option of keeping their
children home. Schools in
\href{https://www.tennessean.com/story/news/education/2020/06/26/metro-nashville-parents-have-choose-between-person-remote-learning-fall/3264076001/}{Nashville}
and \href{https://www.marietta-city.org/reopening}{Marietta, Ga.}, said
this week that families would be given a choice between in-person
schooling and full-time online instruction.

But the hybrid approach, with only limited classroom time, could become
the norm in states that have experienced heavy coronavirus caseloads and
have chosen to take a slower approach to reopening the economy. Those
states, mostly controlled by Democrats, also tend to have powerful
teachers' unions,
\href{https://www.nydailynews.com/opinion/ny-oped-teachers-will-return-in-the-fall-if-20200626-bu5nvsmx6zdqjfeppetepjnk2i-story.html}{which
have repeatedly raised a red flag} about the health risks of reopening
schools --- even as they have
\href{https://www.nytimes.com/2020/04/21/us/coronavirus-teachers-unions-school-home.html}{pushed
for limitations} on the expectations placed on teachers working from
home.

The American Federation of Teachers, a national union, has estimated
that in order to safely and effectively reopen, the nation's schools
will need an additional \$116 billion to cover costs such as reducing
class sizes, increasing cleaning staff, and hiring counselors and
educators to help students recover from the emotional and academic
impact of the pandemic.

Despite the challenges, some education and health experts have
\href{https://jamanetwork.com/journals/jamapediatrics/fullarticle/2766113}{called
for fully reopening schools} before the development of a Covid-19
vaccine, given the central role that American schools play in both the
lives of children and the ability of parents to work outside the home.

The experts point to
\href{https://www.npr.org/2020/06/24/882316641/what-parents-can-learn-from-child-care-centers-that-stayed-open-during-lockdowns}{hopeful
evidence} from child care centers that have remained open to serve the
children of essential workers: Widespread outbreaks of the virus there
appear to have been rare or even nonexistent.
\href{https://www.thelancet.com/journals/lanchi/article/PIIS2352-4642(20)30177-2/fulltext}{Research
suggests} that children are much less likely than adults to die from the
virus or to suffer a severe health consequence; children may also be
\href{http://ncirs.org.au/sites/default/files/2020-04/NCIRS\%20NSW\%20Schools\%20COVID_Summary_FINAL\%20public_26\%20April\%202020.pdf?fbclid=IwAR3oiRE3aeW_iDVBO8ilXa7VdkBMYLcV89pknobZx-cvtlumLQ-FSRwLEDg}{less
likely} than adults to transmit the illness.

But the science on the virus has shifted rapidly in recent months, and
much is still unknown. In addition, the pandemic has become politicized
in many parts of the country, leaving school leaders wondering whether
they can effectively enforce mask wearing and other risk-mitigating
behaviors among students and staff.

That is just one of many concerns for Scott Muri, the superintendent of
the Ector County schools in Odessa, Texas. His days are filled with
considerations of both logistics and teaching strategies.

For students to return to school, he will have to reduce the number of
children on buses at any given time. He is also weighing whether to
reassign teachers so that only the most skilled at online instruction
are creating video lessons, while others are focused more on small-group
tutoring and counseling, which could be provided either online or in
person.

Dr. Muri has yet to announce a reopening plan, but expects a hybrid
model.

``I will not welcome back any child or staff member if I do not feel
it's the safest environment we could possibly create,'' he said. ``Until
we get past the dangerous period, right now, everything is on the table
for consideration.''

Daniel E. Slotnik contributed reporting.

Advertisement

\protect\hyperlink{after-bottom}{Continue reading the main story}

\hypertarget{site-index}{%
\subsection{Site Index}\label{site-index}}

\hypertarget{site-information-navigation}{%
\subsection{Site Information
Navigation}\label{site-information-navigation}}

\begin{itemize}
\tightlist
\item
  \href{https://help.nytimes.com/hc/en-us/articles/115014792127-Copyright-notice}{©~2020~The
  New York Times Company}
\end{itemize}

\begin{itemize}
\tightlist
\item
  \href{https://www.nytco.com/}{NYTCo}
\item
  \href{https://help.nytimes.com/hc/en-us/articles/115015385887-Contact-Us}{Contact
  Us}
\item
  \href{https://www.nytco.com/careers/}{Work with us}
\item
  \href{https://nytmediakit.com/}{Advertise}
\item
  \href{http://www.tbrandstudio.com/}{T Brand Studio}
\item
  \href{https://www.nytimes.com/privacy/cookie-policy\#how-do-i-manage-trackers}{Your
  Ad Choices}
\item
  \href{https://www.nytimes.com/privacy}{Privacy}
\item
  \href{https://help.nytimes.com/hc/en-us/articles/115014893428-Terms-of-service}{Terms
  of Service}
\item
  \href{https://help.nytimes.com/hc/en-us/articles/115014893968-Terms-of-sale}{Terms
  of Sale}
\item
  \href{https://spiderbites.nytimes.com}{Site Map}
\item
  \href{https://help.nytimes.com/hc/en-us}{Help}
\item
  \href{https://www.nytimes.com/subscription?campaignId=37WXW}{Subscriptions}
\end{itemize}
