Sections

SEARCH

\protect\hyperlink{site-content}{Skip to
content}\protect\hyperlink{site-index}{Skip to site index}

\href{https://www.nytimes.com/section/books}{Books}

\href{https://myaccount.nytimes.com/auth/login?response_type=cookie\&client_id=vi}{}

\href{https://www.nytimes.com/section/todayspaper}{Today's Paper}

\href{/section/books}{Books}\textbar{}Assaulted at 15, a Writer Looks
Back and Comes Forward

\url{https://nyti.ms/3eF6TlB}

\begin{itemize}
\item
\item
\item
\item
\item
\end{itemize}

Advertisement

\protect\hyperlink{after-top}{Continue reading the main story}

Supported by

\protect\hyperlink{after-sponsor}{Continue reading the main story}

\hypertarget{assaulted-at-15-a-writer-looks-back-and-comes-forward}{%
\section{Assaulted at 15, a Writer Looks Back and Comes
Forward}\label{assaulted-at-15-a-writer-looks-back-and-comes-forward}}

``I'm done with shame,'' says Lacy Crawford, the author of the memoir
``Notes on a Silencing.''

\includegraphics{https://static01.nyt.com/images/2020/06/27/books/27Crawford1/merlin_173789985_db4f264f-1163-4856-bbe1-e644995d5fb4-articleLarge.jpg?quality=75\&auto=webp\&disable=upscale}

\href{https://www.nytimes.com/by/sarah-lyall}{\includegraphics{https://static01.nyt.com/images/2018/02/20/multimedia/author-sarah-lyall/author-sarah-lyall-thumbLarge.jpg}}

By \href{https://www.nytimes.com/by/sarah-lyall}{Sarah Lyall}

\begin{itemize}
\item
  Published June 27, 2020Updated July 7, 2020
\item
  \begin{itemize}
  \item
  \item
  \item
  \item
  \item
  \end{itemize}
\end{itemize}

There are so many upsetting things about the assault Lacy Crawford
suffered in 1990, when she was 15 and a junior at St. Paul's School in
New Hampshire, but one of the most upsetting is how commonplace she
believes it was.

``This may sound disingenuous, but I don't think my assault is
particularly interesting,'' she said in an interview earlier this month.
She speaks deliberately, calmly, as if observing her feelings at a
remove. ``There are so many stories of abuse and assault. Mine is one of
just a dime a dozen.''

Crawford has had 30 years to grapple with what happened that day. But
her memoir, ``Notes on a Silencing,'' out next Tuesday from Little,
Brown, focuses much more on what came afterward. For her, St. Paul's
response only compounded the attack, piling a second trauma on top of
the first.

``The way they came to their own conclusions about what happened,''
Crawford, now 45, said by phone from Southern California, where she
lives with her husband and three sons, ``that was breathtaking to me and
remains breathtaking to me.''

She added: ``The edge has not come off that.''

Image

Lacy Crawford's book ``Notes on a Silencing'' comes out on July 7.

The details are horrible to repeat and horrible to read: how two senior
boys pinned Crawford down, grabbed her breasts, unzipped her jeans and
penetrated her with their fingers; how they jammed their penises deep
into her mouth; how they bragged about it afterward. The attack left her
feverish and with a chronically raw, bleeding throat that made talking
and eating painful, she writes. It turned out that she had been infected
with herpes.

The school's story --- at least the story officials told Crawford's
devastated parents --- was that the encounter was consensual, that their
daughter brought it on herself, that she was promiscuous and hardly a
victim. They never asked Lacy for her own account, she writes, and they
made it clear that unless she dropped the matter she would not be able
to return to school.

``Trust me,'' one official told her father. ``She's not a good girl.''

As far as St. Paul's was concerned, the issue was then closed. The boys
who attacked her weren't accused of wrongdoing, and Crawford doesn't
name them in her book. She returned to school, graduating in 1992, but
she felt like a ghost of a person, shrouded in private misery, rendered
voiceless even as her throat healed. Her friends drifted away; other
students whispered about her; someone threw things at her from a dorm
window as she walked by. She made herself ``as silent and slender as I
could,'' she writes. ``I was diseased; I was disgraced; I was alone.''

\href{https://www.nytimes.com/2018/05/03/us/stpauls-boarding-school-abuse.html}{St.
Paul's}, like a number of other private boarding schools, including
\href{https://www.nytimes.com/2017/07/31/nyregion/second-phillips-andover-sex-abuse-report-includes-teacher-named-by-choate.html}{Phillips
Academy},
\href{https://www.nytimes.com/2017/04/13/nyregion/sexual-abuse-choate-connecticut-school.html}{Choate
Rosemary Hall},
\href{https://www.nytimes.com/2016/04/18/us/prep-schools-wrestle-with-sex-abuse-accusations-against-teachers.html}{Phillips
Exeter Academy}, the
\href{https://www.nytimes.com/2018/08/17/nyregion/hotchkiss-school-sexual-misconduct.html}{Hotchkiss
School} and
\href{https://www.nytimes.com/2016/01/06/us/40-alumni-assert-sexual-abuse-at-a-rhode-island-prep-school.html}{St.
George's School}, has in recent years had to face up to and answer for
decades of sexual abuse and misconduct. In 2017, a report commissioned
by St. Paul's found substantiated abuse reports by faculty members of
students
\href{https://www.nytimes.com/2017/05/22/us/st-pauls-school-acknowledges-decades-of-sexual-misconduct.html}{stretching
back as far as 1948}.

In 2018, the school reached a settlement with the New Hampshire attorney
general that put the institution under the state's oversight for five
years --- and installed a compliance officer on campus --- to ensure
that it followed basic protocols about protecting students and
investigating complaints. Crawford was interviewed as part of the
attorney general's investigation. A spokesman for the attorney general's
office said her participation had helped provide the basis for ``what we
continue to believe is an unprecedented settlement with the school.''

Kathy Giles, who last year became St. Paul's rector, as the principal is
called, said in an interview that the school did not dispute Crawford's
account.

``We respect Lacy's courage and we admire her voice,'' Giles said.
``There's a truth to her experience that's powerful and important.''

The school is committed to doing better by its students, she added.
``Who would want that experience for anyone --- for Lacy, for her
family, for her friends?'' she said. ``If there's anything we've
learned, it's that we have to receive the stories and respect the
experience and then take what steps we need to address the hurt and
pain.''

\emph{{[}}
\href{https://www.nytimes.com/2020/07/07/books/review/notes-on-a-silencing-lacy-crawford.html}{\emph{Read
our review of ``Notes on a Silencing.''}} \emph{{]}}

After Giles read an advance copy of the book, she requested a meeting
with Crawford. The two had lunch in California last winter.

``She apologized six different ways to Wednesday,'' Crawford said. ``I
said thank you, and I thought that that and my Starbucks card could get
me a latte. But so far, the quality of our discourse makes me hopeful.''

For years, Crawford kept her story private. She went to Princeton,
volunteered as a rape crisis counselor, wrote her master's thesis on the
use of rape testimony in legal cases, and worked as a high school
teacher and environmental campaigner, among other jobs. Before ``Notes
on a Silencing,'' she wrote ``Early Decision,'' a satirical novel about
the college admissions process based on her work as a private admissions
counselor.

She never planned to write a book about her attack. It was only after
she learned about Chessy Prout, a St. Paul's student who was sexually
assaulted in 2014, when she was 15, and who waived her anonymity to
\href{https://www.nytimes.com/2016/08/31/us/chessy-prout-sexual-assault-victim-of-owen-labrie-at-new-hampshire-school-speaks-out.html}{discuss
the case publicly}, that Crawford began to think afresh about telling
her story. (Owen Labrie, the St. Paul's student convicted of Prout's
assault,
\href{https://www.nytimes.com/2019/07/05/us/rich-privilege-courts.html}{served
six months in jail}.)

Crawford started writing her book in the fall of 2017, working while her
three young sons were at school, and on weekends, when her husband took
them on daylong outings. By March 2018, she had most of a first draft.

``The thing about assault is that it devastates so many people,'' she
said. ``Not just the victim, but also the people who are told and who
share the pain; and the people who are told and don't know how to
respond; and the people who aren't told but feel that something really
bad has entered the room but can't put it into words.''

Therapy, a loving marriage, raising her children and writing the book
all helped Crawford come through to the other side of her experience ---
to redress the balance of power between herself and the boys who
assaulted her, between herself and the school that betrayed her. The
thing about a book is that you get to have the last word.

``I felt utterly exposed and shamed,'' she said of her younger self.
``It was a very small community, and it was all I had at that age. But
for me now, I'm not going to hold on to it any more. I'm done with
shame.''

\emph{Follow New York Times Books on}
\href{https://www.facebook.com/nytbooks/}{\emph{Facebook}}\emph{,}
\href{https://twitter.com/nytimesbooks}{\emph{Twitter}} \emph{and}
\href{https://www.instagram.com/nytbooks/}{\emph{Instagram}}\emph{, sign
up for}
\href{https://www.nytimes.com/newsletters/books-review}{\emph{our
newsletter}} \emph{or}
\href{https://www.nytimes.com/interactive/2017/books/books-calendar.html}{\emph{our
literary calendar}}\emph{. And listen to us on the}
\href{https://www.nytimes.com/column/book-review-podcast}{\emph{Book
Review podcast}}\emph{.}

Advertisement

\protect\hyperlink{after-bottom}{Continue reading the main story}

\hypertarget{site-index}{%
\subsection{Site Index}\label{site-index}}

\hypertarget{site-information-navigation}{%
\subsection{Site Information
Navigation}\label{site-information-navigation}}

\begin{itemize}
\tightlist
\item
  \href{https://help.nytimes.com/hc/en-us/articles/115014792127-Copyright-notice}{©~2020~The
  New York Times Company}
\end{itemize}

\begin{itemize}
\tightlist
\item
  \href{https://www.nytco.com/}{NYTCo}
\item
  \href{https://help.nytimes.com/hc/en-us/articles/115015385887-Contact-Us}{Contact
  Us}
\item
  \href{https://www.nytco.com/careers/}{Work with us}
\item
  \href{https://nytmediakit.com/}{Advertise}
\item
  \href{http://www.tbrandstudio.com/}{T Brand Studio}
\item
  \href{https://www.nytimes.com/privacy/cookie-policy\#how-do-i-manage-trackers}{Your
  Ad Choices}
\item
  \href{https://www.nytimes.com/privacy}{Privacy}
\item
  \href{https://help.nytimes.com/hc/en-us/articles/115014893428-Terms-of-service}{Terms
  of Service}
\item
  \href{https://help.nytimes.com/hc/en-us/articles/115014893968-Terms-of-sale}{Terms
  of Sale}
\item
  \href{https://spiderbites.nytimes.com}{Site Map}
\item
  \href{https://help.nytimes.com/hc/en-us}{Help}
\item
  \href{https://www.nytimes.com/subscription?campaignId=37WXW}{Subscriptions}
\end{itemize}
