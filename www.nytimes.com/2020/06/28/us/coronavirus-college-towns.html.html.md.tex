Sections

SEARCH

\protect\hyperlink{site-content}{Skip to
content}\protect\hyperlink{site-index}{Skip to site index}

\href{/section/us}{U.S.}\textbar{}`We Could Be Feeling This for the Next
Decade': Virus Hits College Towns

\begin{itemize}
\item
\item
\item
\item
\item
\end{itemize}

\href{https://www.nytimes.com/news-event/coronavirus?action=click\&pgtype=Article\&state=default\&region=TOP_BANNER\&context=storylines_menu}{The
Coronavirus Outbreak}

\begin{itemize}
\tightlist
\item
  live\href{https://www.nytimes.com/2020/08/02/world/coronavirus-updates.html?action=click\&pgtype=Article\&state=default\&region=TOP_BANNER\&context=storylines_menu}{Latest
  Updates}
\item
  \href{https://www.nytimes.com/interactive/2020/us/coronavirus-us-cases.html?action=click\&pgtype=Article\&state=default\&region=TOP_BANNER\&context=storylines_menu}{Maps
  and Cases}
\item
  \href{https://www.nytimes.com/interactive/2020/science/coronavirus-vaccine-tracker.html?action=click\&pgtype=Article\&state=default\&region=TOP_BANNER\&context=storylines_menu}{Vaccine
  Tracker}
\item
  \href{https://www.nytimes.com/interactive/2020/07/29/us/schools-reopening-coronavirus.html?action=click\&pgtype=Article\&state=default\&region=TOP_BANNER\&context=storylines_menu}{What
  School May Look Like}
\item
  \href{https://www.nytimes.com/live/2020/07/31/business/stock-market-today-coronavirus?action=click\&pgtype=Article\&state=default\&region=TOP_BANNER\&context=storylines_menu}{Economy}
\end{itemize}

\includegraphics{https://static01.nyt.com/images/2020/06/24/us/00virus-collegetowns01/merlin_173814348_5bb71923-a644-48f4-9aa7-7d8af2bf6b62-articleLarge.jpg?quality=75\&auto=webp\&disable=upscale}

\hypertarget{we-could-be-feeling-this-for-the-next-decade-virus-hits-college-towns}{%
\section{`We Could Be Feeling This for the Next Decade': Virus Hits
College
Towns}\label{we-could-be-feeling-this-for-the-next-decade-virus-hits-college-towns}}

Opening bars and bringing back football teams have led to new outbreaks.
Communities that evolved around campuses face potentially existential
losses in population, jobs and revenue.

The nearly deserted campus of the University of California, Davis, this
month.Credit...Tommy Ly for The New York Times

Supported by

\protect\hyperlink{after-sponsor}{Continue reading the main story}

\href{https://www.nytimes.com/by/shawn-hubler}{\includegraphics{https://static01.nyt.com/images/2020/06/05/reader-center/author-shawn-hubler/author-shawn-hubler-thumbLarge.png}}

By \href{https://www.nytimes.com/by/shawn-hubler}{Shawn Hubler}

\begin{itemize}
\item
  Published June 28, 2020Updated June 30, 2020
\item
  \begin{itemize}
  \item
  \item
  \item
  \item
  \item
  \end{itemize}
\end{itemize}

DAVIS, Calif. --- The community around the University of California,
Davis, used to have a population of 70,000 and a thriving economy.
Rentals were tight. Downtown was jammed. Hotels were booked months in
advance for commencement. Students swarmed to the town's bar crawl,
sampling the trio of signature cocktails known on campus as ``the Davis
Trinity.''

Then came the coronavirus. When the campus closed in March, an estimated
20,000 students and faculty left town.

With them went about a third of the demand for goods and services, from
books to bikes to brunches. City officials are expecting most of that
demand to stay gone even as the economy reopens.

Fall classes will be mostly remote, the university announced last week,
with ``reduced density'' in dorms. Davis's incoming vice mayor, Lucas
Frerichs, said the city was anticipating ``a huge impact'' with a
majority of the university's 39,000-plus students still dispersed in
September.

For ``townies,'' rules require congregation to remain limited, too, as
confirmed coronavirus cases continue to climb in California. One of the
Davis Trinity bars has closed, with no plan to reopen. On a recent
Sunday, downtown was filled with ``takeout only'' signs and half-empty,
far-flung cafe tables. Outside the closed theater, a lone busker stood
on a corner playing ``Swan Lake'' on a violin to virtually no one.

Efforts to stem the pandemic have squeezed local economies across the
nation, but the threat is starting to look existential in college towns.

Reliant on institutions that once seemed impervious to recession, ``town
and gown'' communities that have evolved around rural campuses ---
Cornell, Amherst College, Penn State --- are confronting not only
Covid-19 but also major losses in population, revenue and jobs.

Where business as usual has been tried, punishment has followed: This
week, Iowa health authorities
\href{https://ktiv.com/2020/06/23/virus-cases-spike-among-young-adults-in-iowa-college-towns-2/}{reported
case spikes} among young adults in its two largest college towns, Ames
and Iowa City, after the governor allowed bars to reopen. And on
campuses across the country,
\href{https://www.nytimes.com/2020/06/25/sports/ncaafootball/college-football-coronavirus-cases.html?action=click\&module=Top\%20Stories\&pgtype=Homepage}{attempts
to bring back football teams} for preseason practice have resulted in
outbreaks.

More than 130 coronavirus cases have been linked to athletic departments
at 28 Division I universities. At Clemson, at least 23 football players
and two coaches have been infected. At Arkansas State University, seven
athletes across three teams tested positive. And at the University of
Houston, the athletic department stopped off-season workouts after an
outbreak was discovered.

\hypertarget{latest-updates-global-coronavirus-outbreak}{%
\section{\texorpdfstring{\href{https://www.nytimes.com/2020/08/01/world/coronavirus-covid-19.html?action=click\&pgtype=Article\&state=default\&region=MAIN_CONTENT_1\&context=storylines_live_updates}{Latest
Updates: Global Coronavirus
Outbreak}}{Latest Updates: Global Coronavirus Outbreak}}\label{latest-updates-global-coronavirus-outbreak}}

Updated 2020-08-02T17:52:35.962Z

\begin{itemize}
\tightlist
\item
  \href{https://www.nytimes.com/2020/08/01/world/coronavirus-covid-19.html?action=click\&pgtype=Article\&state=default\&region=MAIN_CONTENT_1\&context=storylines_live_updates\#link-34047410}{The
  U.S. reels as July cases more than double the total of any other
  month.}
\item
  \href{https://www.nytimes.com/2020/08/01/world/coronavirus-covid-19.html?action=click\&pgtype=Article\&state=default\&region=MAIN_CONTENT_1\&context=storylines_live_updates\#link-780ec966}{Top
  U.S. officials work to break an impasse over the federal jobless
  benefit.}
\item
  \href{https://www.nytimes.com/2020/08/01/world/coronavirus-covid-19.html?action=click\&pgtype=Article\&state=default\&region=MAIN_CONTENT_1\&context=storylines_live_updates\#link-2bc8948}{Its
  outbreak untamed, Melbourne goes into even greater lockdown.}
\end{itemize}

\href{https://www.nytimes.com/2020/08/01/world/coronavirus-covid-19.html?action=click\&pgtype=Article\&state=default\&region=MAIN_CONTENT_1\&context=storylines_live_updates}{See
more updates}

More live coverage:
\href{https://www.nytimes.com/live/2020/07/31/business/stock-market-today-coronavirus?action=click\&pgtype=Article\&state=default\&region=MAIN_CONTENT_1\&context=storylines_live_updates}{Markets}

Sports are not the only source of outbreaks in college towns.
Mississippi officials tied several cases to fraternity rush parties that
apparently flouted social distancing rules. In Baton Rouge, La., at
least 100 cases were linked to bars in the Tigerland nightlife district
near Louisiana State's campus. And in Manhattan, Kan., home to Kansas
State, officials said Wednesday that there had been two recent
outbreaks: one on the football team, and another in the Aggieville
entertainment district just off campus.

For the cities involved, the prognosis is also daunting. In most college
towns, university students, faculty and staff are a primary market.
Local economies depend on their numbers and dollars, from sales taxes to
football weekends to federal funds determined by the U.S. census.

Students at Ohio University represent three-quarters of the usual
population of Athens, Ohio. In Ithaca, N.Y., every other person in town
is --- or used to be --- connected to Cornell or Ithaca College.

\includegraphics{https://static01.nyt.com/images/2020/06/24/us/00virus-collegetowns02/00virus-collegetowns02-articleLarge.jpg?quality=75\&auto=webp\&disable=upscale}

Image

The deserted Ohio University campus in Athens.Credit...Maddie McGarvey
for The New York Times

Image

Students at Ohio University represent three-quarters of the population
of the town.Credit...Maddie McGarvey for The New York Times

The local economy in Ann Arbor, Mich., takes in
\href{https://communityrelations.umich.edu/facts-figures/}{nearly \$95
million a year} in discretionary spending from the University of
Michigan's 45,000-plus students. Ari Weinzweig, cofounding partner of
Zingerman's, a landmark bakery and deli, said sales have been down 50
percent, and the company has had to furlough nearly 300 of its 700
employees since the pandemic.

The town's Literati Bookstore launched a
\href{https://www.gofundme.com/f/support-literati-bookstore-amp-booksellers}{GoFundMe
campaign} to keep from going out of business, and created a virtual site
for its famed ``\href{https://www.publictypewriter.com/}{public
typewriter}'' so customers could keep leaving anonymous typed messages,
a company tradition. (``Oh how I wish for a coffee not made by my own
hands,'' someone typed online in May.)

In State College, Pa.,
\href{http://www.statecollegepa.us/DocumentCenter/View/9291/State-College-Neighborhood-Plan-Borough-Wide-Conditions?bidId=}{an
estimated 65 percent} of the community is made up of students at Penn
State's main campus, a local juggernaut that enrolls 46,000 students,
\href{https://factbook.psu.edu/factbook/HrDynamic/EmployeesbyClassificationSummaryPSULaw.aspx?YearCode=2019humors\&FBPlusIndc=N}{employs}
more than 17,000 nonstudents and injects
\href{https://www.psu.edu/ur/newsdocuments/Penn-State_Economic-Contribution-Study_February-2019.pdf}{about
\$128 million a year} into rural Centre County.

The university has announced plans to reopen with double-occupancy dorm
rooms but many classes will be remote, and it is still not known how
many students will return. Also in question is the future of Penn State
football, a local economic linchpin that generated \$100 million in
2018-19
\href{https://www.scribd.com/document/448272170/Penn-State-2018-19-fiscal-year-NCAA-Report-Final\#from_embed?campaign=SkimbitLtd\&ad_group=126006X1587341X3675db3cbb273c0b115aafcc767f8ca5\&keyword=660149026\&source=hp_affiliate\&medium=affiliate}{for
the university alone}.

Local governments are bracing, too. Amherst, Mass., is scheduled to vote
this week on a proposal to increase annual water and sewer fees by an
average of \$100 per household, a result of a
\href{https://www.amherstma.gov/DocumentCenter/View/51625/8b-FY21-Water-and-Sewer-Rate-Memo---61220-FINAL}{precipitous
drop in water use} as students have abandoned Hampshire College, Amherst
College and the University of Massachusetts in that New England college
town.

Ithaca's mayor, Svante Myrick, said his city was preparing to cut its
\$70 million budget by about \$14 million, and has furloughed a quarter
of its employees, including his assistant. He personally has taken a 10
percent pay cut. A
\href{https://www.ithacatu.org/cancelrent}{resolution} passed earlier
this month asked the state to let him authorize blanket rent forgiveness
for three months.

Unemployment in the Ithaca metropolitan area has soared to 10 percent
from 3 percent before the pandemic. Sales tax receipts have tanked as
about \$4 million per week in student spending has disappeared along
with Cornell's students, Mr. Myrick said. About two-thirds of the land
in his jurisdiction is university-owned, he said, and therefore exempt
from property tax.

``We're going to be looking at Hoovervilles --- or maybe Trump Towns ---
all over the country,'' said the mayor, a Democrat who clashes
frequently with his upstate area's Republican congressional delegation.
``It's bad. It's really bad.''

Compounding the concern is the 2020 census. Conducted every 10 years,
the national head count determines the distribution of federal funding
for a vast number of local and state programs, including transit, public
safety and Medicaid.

\href{https://www.nytimes.com/news-event/coronavirus?action=click\&pgtype=Article\&state=default\&region=MAIN_CONTENT_3\&context=storylines_faq}{}

\hypertarget{the-coronavirus-outbreak-}{%
\subsubsection{The Coronavirus Outbreak
›}\label{the-coronavirus-outbreak-}}

\hypertarget{frequently-asked-questions}{%
\paragraph{Frequently Asked
Questions}\label{frequently-asked-questions}}

Updated July 27, 2020

\begin{itemize}
\item ~
  \hypertarget{should-i-refinance-my-mortgage}{%
  \paragraph{Should I refinance my
  mortgage?}\label{should-i-refinance-my-mortgage}}

  \begin{itemize}
  \tightlist
  \item
    \href{https://www.nytimes.com/article/coronavirus-money-unemployment.html?action=click\&pgtype=Article\&state=default\&region=MAIN_CONTENT_3\&context=storylines_faq}{It
    could be a good idea,} because mortgage rates have
    \href{https://www.nytimes.com/2020/07/16/business/mortgage-rates-below-3-percent.html?action=click\&pgtype=Article\&state=default\&region=MAIN_CONTENT_3\&context=storylines_faq}{never
    been lower.} Refinancing requests have pushed mortgage applications
    to some of the highest levels since 2008, so be prepared to get in
    line. But defaults are also up, so if you're thinking about buying a
    home, be aware that some lenders have tightened their standards.
  \end{itemize}
\item ~
  \hypertarget{what-is-school-going-to-look-like-in-september}{%
  \paragraph{What is school going to look like in
  September?}\label{what-is-school-going-to-look-like-in-september}}

  \begin{itemize}
  \tightlist
  \item
    It is unlikely that many schools will return to a normal schedule
    this fall, requiring the grind of
    \href{https://www.nytimes.com/2020/06/05/us/coronavirus-education-lost-learning.html?action=click\&pgtype=Article\&state=default\&region=MAIN_CONTENT_3\&context=storylines_faq}{online
    learning},
    \href{https://www.nytimes.com/2020/05/29/us/coronavirus-child-care-centers.html?action=click\&pgtype=Article\&state=default\&region=MAIN_CONTENT_3\&context=storylines_faq}{makeshift
    child care} and
    \href{https://www.nytimes.com/2020/06/03/business/economy/coronavirus-working-women.html?action=click\&pgtype=Article\&state=default\&region=MAIN_CONTENT_3\&context=storylines_faq}{stunted
    workdays} to continue. California's two largest public school
    districts --- Los Angeles and San Diego --- said on July 13, that
    \href{https://www.nytimes.com/2020/07/13/us/lausd-san-diego-school-reopening.html?action=click\&pgtype=Article\&state=default\&region=MAIN_CONTENT_3\&context=storylines_faq}{instruction
    will be remote-only in the fall}, citing concerns that surging
    coronavirus infections in their areas pose too dire a risk for
    students and teachers. Together, the two districts enroll some
    825,000 students. They are the largest in the country so far to
    abandon plans for even a partial physical return to classrooms when
    they reopen in August. For other districts, the solution won't be an
    all-or-nothing approach.
    \href{https://bioethics.jhu.edu/research-and-outreach/projects/eschool-initiative/school-policy-tracker/}{Many
    systems}, including the nation's largest, New York City, are
    devising
    \href{https://www.nytimes.com/2020/06/26/us/coronavirus-schools-reopen-fall.html?action=click\&pgtype=Article\&state=default\&region=MAIN_CONTENT_3\&context=storylines_faq}{hybrid
    plans} that involve spending some days in classrooms and other days
    online. There's no national policy on this yet, so check with your
    municipal school system regularly to see what is happening in your
    community.
  \end{itemize}
\item ~
  \hypertarget{is-the-coronavirus-airborne}{%
  \paragraph{Is the coronavirus
  airborne?}\label{is-the-coronavirus-airborne}}

  \begin{itemize}
  \tightlist
  \item
    The coronavirus
    \href{https://www.nytimes.com/2020/07/04/health/239-experts-with-one-big-claim-the-coronavirus-is-airborne.html?action=click\&pgtype=Article\&state=default\&region=MAIN_CONTENT_3\&context=storylines_faq}{can
    stay aloft for hours in tiny droplets in stagnant air}, infecting
    people as they inhale, mounting scientific evidence suggests. This
    risk is highest in crowded indoor spaces with poor ventilation, and
    may help explain super-spreading events reported in meatpacking
    plants, churches and restaurants.
    \href{https://www.nytimes.com/2020/07/06/health/coronavirus-airborne-aerosols.html?action=click\&pgtype=Article\&state=default\&region=MAIN_CONTENT_3\&context=storylines_faq}{It's
    unclear how often the virus is spread} via these tiny droplets, or
    aerosols, compared with larger droplets that are expelled when a
    sick person coughs or sneezes, or transmitted through contact with
    contaminated surfaces, said Linsey Marr, an aerosol expert at
    Virginia Tech. Aerosols are released even when a person without
    symptoms exhales, talks or sings, according to Dr. Marr and more
    than 200 other experts, who
    \href{https://academic.oup.com/cid/article/doi/10.1093/cid/ciaa939/5867798}{have
    outlined the evidence in an open letter to the World Health
    Organization}.
  \end{itemize}
\item ~
  \hypertarget{what-are-the-symptoms-of-coronavirus}{%
  \paragraph{What are the symptoms of
  coronavirus?}\label{what-are-the-symptoms-of-coronavirus}}

  \begin{itemize}
  \tightlist
  \item
    Common symptoms
    \href{https://www.nytimes.com/article/symptoms-coronavirus.html?action=click\&pgtype=Article\&state=default\&region=MAIN_CONTENT_3\&context=storylines_faq}{include
    fever, a dry cough, fatigue and difficulty breathing or shortness of
    breath.} Some of these symptoms overlap with those of the flu,
    making detection difficult, but runny noses and stuffy sinuses are
    less common.
    \href{https://www.nytimes.com/2020/04/27/health/coronavirus-symptoms-cdc.html?action=click\&pgtype=Article\&state=default\&region=MAIN_CONTENT_3\&context=storylines_faq}{The
    C.D.C. has also} added chills, muscle pain, sore throat, headache
    and a new loss of the sense of taste or smell as symptoms to look
    out for. Most people fall ill five to seven days after exposure, but
    symptoms may appear in as few as two days or as many as 14 days.
  \end{itemize}
\item ~
  \hypertarget{does-asymptomatic-transmission-of-covid-19-happen}{%
  \paragraph{Does asymptomatic transmission of Covid-19
  happen?}\label{does-asymptomatic-transmission-of-covid-19-happen}}

  \begin{itemize}
  \tightlist
  \item
    So far, the evidence seems to show it does. A widely cited
    \href{https://www.nature.com/articles/s41591-020-0869-5}{paper}
    published in April suggests that people are most infectious about
    two days before the onset of coronavirus symptoms and estimated that
    44 percent of new infections were a result of transmission from
    people who were not yet showing symptoms. Recently, a top expert at
    the World Health Organization stated that transmission of the
    coronavirus by people who did not have symptoms was ``very rare,''
    \href{https://www.nytimes.com/2020/06/09/world/coronavirus-updates.html?action=click\&pgtype=Article\&state=default\&region=MAIN_CONTENT_3\&context=storylines_faq\#link-1f302e21}{but
    she later walked back that statement.}
  \end{itemize}
\end{itemize}

Because the window for responses has coincided with campus shutdowns,
college towns are reporting significant undercounts of students living
off-campus, with dire financial implications.

A census without Ohio University students could knock the official
population of Athens from 24,000 down to as few as 6,000 people. With an
Oct. 31 deadline approaching, responses in student neighborhoods are
currently running some 20 percentage points lower than in 2010, with
\href{https://2020census.gov/en/response-rates.html}{response rates} in
some tracts of less than 31 percent.

Mayor Steve Patterson of Athens estimates an undercount could cost his
small city up to \$40 million over the next 10 years ``for things like
community development block grants, jobs and family services and senior
services that rely on a strong census count to get a full funding.''

``We could be feeling this for the next decade,'' Mr. Patterson said.

In California, where Democrats have prioritized the census, the city of
Davis and its surrounding county partnered long before the pandemic with
the university to maximize its response rate, which is now higher than
the state average. But the exodus of students has cut sales tax revenues
by 50 percent, Mr. Frerichs said.

Image

When the University of California, Davis, campus closed in March, some
20,000 students and faculty left town.Credit...Tommy Ly for The New York
Times

Image

Empty outdoor seating in downtown Davis last weekend.Credit...Tommy Ly
for The New York Times

Image

The shuttered box office at the Mondavi Center for the Performing Arts
on the campus in Davis.Credit...Tommy Ly for The New York Times

Virtual graduation in June slashed hotel occupancy from 90 percent to 10
percent during the local hospitality industry's usual peak season.
Bookings have since rebounded slightly, Mr. Frerichs said, but only to
about 25 percent, substantially denting hotel occupancy tax revenues.

Transit ridership has dropped so precipitously, he said, that local
authorities have been using the buses to transport supplies to and from
food banks. The city has begun reaching out to unions and identifying
budget cuts in case the economy does not quickly bounce back.

Already, Mr. Frerichs said, the council has opted to leave three open
positions for police officers vacant. ``That's three sets of eyes and
ears on the street,'' he said, ``but this is a legitimate concern. Long
term, this could be on par with the great recession for us.''

Or maybe worse than the recession, he added, because in 2008 at least
the town could still gather.

Now the bike traffic is scant, the farmers market socially distanced,
and the baristas working reduced hours at coffee shops ask customers to
alert them when they leave so maintenance can disinfect their tables.
The virus even canceled Davis's annual town-and-gown party, Picnic Day.

``Part of me is enjoying reclaiming the community,'' said Mr. Frerichs,
who attended the university and has lived for 24 years in Davis. ``But
one of the things that makes a college town so wonderful is the vibrant
young population.''

``They're the lifeblood, and without them --- well, the squirrels are
having a field day,'' he said. ``But for the rest of us, it's just so
quiet.''

Mitch Smith contributed reporting from Chicago, and Lauryn Higgins from
Lincoln, Neb.

Advertisement

\protect\hyperlink{after-bottom}{Continue reading the main story}

\hypertarget{site-index}{%
\subsection{Site Index}\label{site-index}}

\hypertarget{site-information-navigation}{%
\subsection{Site Information
Navigation}\label{site-information-navigation}}

\begin{itemize}
\tightlist
\item
  \href{https://help.nytimes.com/hc/en-us/articles/115014792127-Copyright-notice}{©~2020~The
  New York Times Company}
\end{itemize}

\begin{itemize}
\tightlist
\item
  \href{https://www.nytco.com/}{NYTCo}
\item
  \href{https://help.nytimes.com/hc/en-us/articles/115015385887-Contact-Us}{Contact
  Us}
\item
  \href{https://www.nytco.com/careers/}{Work with us}
\item
  \href{https://nytmediakit.com/}{Advertise}
\item
  \href{http://www.tbrandstudio.com/}{T Brand Studio}
\item
  \href{https://www.nytimes.com/privacy/cookie-policy\#how-do-i-manage-trackers}{Your
  Ad Choices}
\item
  \href{https://www.nytimes.com/privacy}{Privacy}
\item
  \href{https://help.nytimes.com/hc/en-us/articles/115014893428-Terms-of-service}{Terms
  of Service}
\item
  \href{https://help.nytimes.com/hc/en-us/articles/115014893968-Terms-of-sale}{Terms
  of Sale}
\item
  \href{https://spiderbites.nytimes.com}{Site Map}
\item
  \href{https://help.nytimes.com/hc/en-us}{Help}
\item
  \href{https://www.nytimes.com/subscription?campaignId=37WXW}{Subscriptions}
\end{itemize}
