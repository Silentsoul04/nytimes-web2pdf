Sections

SEARCH

\protect\hyperlink{site-content}{Skip to
content}\protect\hyperlink{site-index}{Skip to site index}

\href{https://www.nytimes.com/section/world/asia}{Asia Pacific}

\href{https://myaccount.nytimes.com/auth/login?response_type=cookie\&client_id=vi}{}

\href{https://www.nytimes.com/section/todayspaper}{Today's Paper}

\href{/section/world/asia}{Asia Pacific}\textbar{}India Debates
Skin-Tone Bias as Beauty Companies Alter Ads

\url{https://nyti.ms/3dG2lds}

\begin{itemize}
\item
\item
\item
\item
\item
\end{itemize}

Advertisement

\protect\hyperlink{after-top}{Continue reading the main story}

Supported by

\protect\hyperlink{after-sponsor}{Continue reading the main story}

\hypertarget{india-debates-skin-tone-bias-as-beauty-companies-alter-ads}{%
\section{India Debates Skin-Tone Bias as Beauty Companies Alter
Ads}\label{india-debates-skin-tone-bias-as-beauty-companies-alter-ads}}

America's intense conversation on race has focused attention on a type
of discrimination that has long vexed India.

\includegraphics{https://static01.nyt.com/images/2020/06/26/world/26india-skin01/merlin_139123014_0abbf32e-546e-44ff-9a69-e122d64eb975-articleLarge.jpg?quality=75\&auto=webp\&disable=upscale}

\href{https://www.nytimes.com/by/sameer-yasir}{\includegraphics{https://static01.nyt.com/images/2019/11/22/reader-center/author-sameer-yasir/author-sameer-yasir-thumbLarge.png}}\href{https://www.nytimes.com/by/jeffrey-gettleman}{\includegraphics{https://static01.nyt.com/images/2018/10/10/multimedia/author-jeffrey-gettleman/author-jeffrey-gettleman-thumbLarge.png}}

By \href{https://www.nytimes.com/by/sameer-yasir}{Sameer Yasir} and
\href{https://www.nytimes.com/by/jeffrey-gettleman}{Jeffrey Gettleman}

\begin{itemize}
\item
  June 28, 2020
\item
  \begin{itemize}
  \item
  \item
  \item
  \item
  \item
  \end{itemize}
\end{itemize}

NEW DELHI --- Throughout the years she was growing up in southern India,
Christy Jennifer, a producer with a media house in the city of Chennai,
was traumatized by episodes of prejudice.

As she walked through school corridors, classmates pointed at her darker
skin and teased her, she said. Even friends and family members told her
never to wear black. She said she was constantly advised on which skin
lightening cream to use, as if the remedy to this deep-seated social
bias lay in a plastic bottle.

``Every day, my dignity and self-esteem were reduced to the color of my
skin,'' she said. ``I felt a worthless piece of flesh.''

Colorism, the bias against people of darker skin tones, has vexed India
for a long time. It is partly a product of colonial prejudices, and it
has been exacerbated by caste, regional differences and Bollywood, the
nation's film industry, which has long promoted lighter-skinned heroes.

But America's intense discussion of race, in
\href{https://www.nytimes.com/news-event/george-floyd-protests-minneapolis-new-york-los-angeles?action=click\&pgtype=Article\&state=default\&module=styln-george-floyd\&variant=show\&region=TOP_BANNER\&context=storylines_menu}{the
wake of George Floyd's death}, seems to be having some impact here.

This past week, Unilever and other major international consumer brands,
\href{https://www.nytimes.com/2020/06/13/us/george-floyd-racism-america.html}{facing
accusations that they were promoting racist attitudes}, said they would
remove labels such as
\href{https://www.nytimes.com/2020/06/25/business/unilever-jj-skin-care-lightening.html}{``fair''
``white'' and ``light''} from their products, including the
skin-lightening creams that are wildly popular in India.

At the same time, a big Indian matchmaking website, Shaadi.com, decided
to
\href{https://www.fox5ny.com/news/dating-site-removes-filter-that-allowed-users-to-sort-matches-by-skin-tone?utm_campaign=snd-autopilot}{remove}a
filter that allowed people to select partners based on skin tone after
facing a backlash from users that began in North America.

Ms. Jennifer and several other Indians said these were moves in the
right direction.

``This is a fantastic news --- a stepping stone toward ending
colorism,'' Ms. Jennifer said. ``Now young people won't feel ashamed of
how they look while growing up with dark-tone skin.''

Image

Christy Jennifer, who was teased as a child for her skin color, praised
an Indian matchmaking site's decision to remove a skin-tone filter amid
objections from users.Credit...Christy Jennifer

For centuries, discrimination over skin tones has been a feature of
Indian society. Some historians say it was greatly intensified by
colonialism and a practice by the British rulers of favoring
light-skinned Indians for government jobs.

Preferences for light-toned skin over dark --- **** when it comes to
marriages and some jobs --- are still upending the lives of hundreds of
thousands of Indians.

In some families, daughters-in-law with darker skin are called
derogatory names, sometimes branded with the same words used for
thieves. Students with dark-toned skin are more frequently bullied in
schools.

Such attitudes have spawned a huge demand in India for whiteners and
bleaching products. Shop shelves are crammed with creams, oils, soaps
and serums promising to lighten skin, and some are manufactured by the
world's biggest cosmetic companies. The king of the market is Unilever's
Fair \& Lovely cream, a fixture in many Indian households for decades.

But even before this past week, the culture had been changing.

Earlier this year,
\href{https://main.mohfw.gov.in/sites/default/files/Draft\%20of\%20the\%20Drugs\%20and\%20Magic\%20Remedies.pdf}{India's
government proposed a law} that would make it illegal to market products
that make false health claims, including those that promise to lighten
skin.

Kavitha Emmanuel, the director of Women of Worth, an organization in
Chennai, started a campaign in 2019 called ``Dark Is Beautiful.'' Many
young men and women, she said, have complained to her that their skin
tone is an impediment to social mobility.

She welcomed the moves by Unilever and the matchmaking website
\href{https://www.shaadi.com/}{Shaadi.com}, but said India was still
slow in confronting such discrimination.

\includegraphics{https://static01.nyt.com/images/2020/06/26/world/26india-skin03/merlin_173910024_9c57d414-a287-47d1-a505-8a57c8dad8d4-articleLarge.jpg?quality=75\&auto=webp\&disable=upscale}

``While movements like Black Lives Matter have had a profound impact in
the West, in South Asian countries it is still a long-drawn battle,''
she said.

Ms. Emmanuel said that skin-tone biases had the greatest impact on
marriage and social issues but that in some fields, including
entertainment, hospitality and modeling, ``the qualification goes
without saying that you need to be fair-skinned, particularly for
women.''

Other commentators, though, insisted that colorism in India is different
from racism in the West.

``The preference for lighter skin is largely aesthetic and does not have
structural economic or power consequences,'' said Dipankar Gupta, a
well-known sociologist.

``It is not as if a policeman would routinely harass darker-skinned
people,'' Mr. Gupta added. ``Indians can recognize class and status
through a number of markers, but skin color is not one of them.''

Still, across India, there is great social pressure for people to seek
light-skinned spouses.

Mohinder Verma, a businessman, defended placing an ad in a newspaper in
which he sought for his son a ``tall, good-looking'' bride with ``fair
skin'' who has a university degree but prefers to be a stay-at-home
wife.

Mr. Verma, 72, said parents in India felt pressure within their social
circles to find brides for their sons who look ``gori,'' or fair,
although he agreed that ``this thinking needs to change.''

``It's somehow ingrained in our minds,'' said Mr. Verma, who lives in
the northern Indian state of Punjab. ``When you have a dark-skinned
daughter-in-law, people talk behind your back. They ask what wrong had
we committed in our previous life.''

A \href{https://www.ncbi.nlm.nih.gov/pmc/articles/PMC5787082/\#B2}{2017
study} of 1,992 Indians found that more than half said they were
influenced by TV advertisements to appear lighter-skinned.

``Indian preference for lighter skins is a reflection of the successful
branding of white skin as superior,'' said Deepa Narayan, a commentator
on gender issues in India.

Ms. Narayan, who recently published
\href{https://www.vogue.in/content/deepa-narayan-the-shocking-secrets-i-learned-about-indian-women/}{a
book on how women are treated in India}, said Bollywood had contributed
to these prejudices.

``Every heroine and now heroes, too, are whitewashed,'' she said. ``And
the villains are dark.''

For decades,
\href{https://www.livemint.com/Opinion/KZecVebUGLmPnnbd9uobTP/Matrimonial-ads-reflect-prejudices-Indians-wear-on-their-sle.html}{matrimonial
ads in Indian newspapers} displayed a preference for lighter skin,
reinforcing entrenched beliefs partly rooted in India's stubborn caste
system. Many Indians believe that lower caste people are darker.

Social scientists say that there is
\href{https://openscholarship.wustl.edu/cgi/viewcontent.cgi?article=1553\&context=law_globalstudies}{no
direct relationship between caste and skin color}, but that this
perception might have been perpetuated by a long history of lower-caste
people being relegated to menial jobs, often performed under the sun,
that made their skin darker.

Image

For centuries, Indians with darker skin have faced bias, though there is
no direct link between caste and complexion.Credit...Harish Tyagi/EPA,
via Shutterstock

The debate over skin color bias flared into the open after a few women
of Indian descent started a petition drive against Shaadi.com, the
matchmaking service, which claims to have delivered ``millions of happy
stories.''

Meghan Nagpal, who lives in Vancouver, British Columbia, said that four
days after the killing of Mr. Floyd in Minneapolis, she visited
Shaadi.com and was struck by the filter that categorized prospects based
on skin tone.

She flagged the issue with the website but got no response. She then
posted about it on Facebook, which led Hetal Lakhani, a Dallas resident,
to open an online petition, which quickly garnered more than 1,500
signatures. The company then removed the filter.

Ms. Nagpal said that when she was younger, she used skin-lightening
products.

``It was like buying jeans of size 8 when you want 10,'' said Ms.
Nagpal, 28, a graduate student who was born in Canada. ``You are never
comfortable with it.''

Ms. Jennifer, the media producer in Chennai, has spent her adult life
struggling with concepts of beauty and the role of skin color.

When she lived in Tirunelveli, in southern India, people gave her
unsolicited advice all the time on how to look lighter-skinned. She then
moved to Chennai, one of India's biggest cities, and met a man on a
dating website who, she said, was a little lighter skinned than she is.
They chatted online for months, and she eventually traveled to meet him
in person.

They went to dinner, they talked future plans, they shared laughs, and a
warmth grew between them, she said.

But the next day, right before she headed home, the man turned to her
and asked: ``Can you do something about your dark skin?''

Ms. Jennifer, now 42, immediately ended the relationship.

``A family wants their daughter-in-law to be fair-skinned, even if their
son might not be?'' she said. ``I would rather die than marry someone
who judges me by my looks.''

Advertisement

\protect\hyperlink{after-bottom}{Continue reading the main story}

\hypertarget{site-index}{%
\subsection{Site Index}\label{site-index}}

\hypertarget{site-information-navigation}{%
\subsection{Site Information
Navigation}\label{site-information-navigation}}

\begin{itemize}
\tightlist
\item
  \href{https://help.nytimes.com/hc/en-us/articles/115014792127-Copyright-notice}{©~2020~The
  New York Times Company}
\end{itemize}

\begin{itemize}
\tightlist
\item
  \href{https://www.nytco.com/}{NYTCo}
\item
  \href{https://help.nytimes.com/hc/en-us/articles/115015385887-Contact-Us}{Contact
  Us}
\item
  \href{https://www.nytco.com/careers/}{Work with us}
\item
  \href{https://nytmediakit.com/}{Advertise}
\item
  \href{http://www.tbrandstudio.com/}{T Brand Studio}
\item
  \href{https://www.nytimes.com/privacy/cookie-policy\#how-do-i-manage-trackers}{Your
  Ad Choices}
\item
  \href{https://www.nytimes.com/privacy}{Privacy}
\item
  \href{https://help.nytimes.com/hc/en-us/articles/115014893428-Terms-of-service}{Terms
  of Service}
\item
  \href{https://help.nytimes.com/hc/en-us/articles/115014893968-Terms-of-sale}{Terms
  of Sale}
\item
  \href{https://spiderbites.nytimes.com}{Site Map}
\item
  \href{https://help.nytimes.com/hc/en-us}{Help}
\item
  \href{https://www.nytimes.com/subscription?campaignId=37WXW}{Subscriptions}
\end{itemize}
