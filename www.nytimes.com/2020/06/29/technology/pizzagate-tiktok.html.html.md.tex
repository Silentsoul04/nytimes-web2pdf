Sections

SEARCH

\protect\hyperlink{site-content}{Skip to
content}\protect\hyperlink{site-index}{Skip to site index}

\href{https://www.nytimes.com/section/technology}{Technology}

\href{https://myaccount.nytimes.com/auth/login?response_type=cookie\&client_id=vi}{}

\href{https://www.nytimes.com/section/todayspaper}{Today's Paper}

\href{/section/technology}{Technology}\textbar{}A TikTok Twist on
`PizzaGate'

\url{https://nyti.ms/2NDV1Et}

\begin{itemize}
\item
\item
\item
\item
\item
\end{itemize}

Advertisement

\protect\hyperlink{after-top}{Continue reading the main story}

Supported by

\protect\hyperlink{after-sponsor}{Continue reading the main story}

on tech

\hypertarget{a-tiktok-twist-on-pizzagate}{%
\section{A TikTok Twist on
`PizzaGate'}\label{a-tiktok-twist-on-pizzagate}}

Young people have tweaked the conspiracy, and today's internet sites
help spread such false ideas.

\includegraphics{https://static01.nyt.com/images/2020/07/07/business/29ontech/29ontech-articleLarge-v5.gif?quality=75\&auto=webp\&disable=upscale}

\href{https://www.nytimes.com/by/shira-ovide}{\includegraphics{https://static01.nyt.com/images/2020/03/18/reader-center/author-shira-ovide/author-shira-ovide-thumbLarge-v2.png}}

By \href{https://www.nytimes.com/by/shira-ovide}{Shira Ovide}

\begin{itemize}
\item
  June 29, 2020
\item
  \begin{itemize}
  \item
  \item
  \item
  \item
  \item
  \end{itemize}
\end{itemize}

\emph{This article is part of the On Tech newsletter. You can}
\href{https://www.nytimes.com/newsletters/signup/OT}{\emph{sign up
here}} \emph{to receive it weekdays.}

One of social media's early conspiracy theories is back, but remade in
creatively horrible ways.

``\href{https://www.nytimes.com/interactive/2016/12/10/business/media/pizzagate.html}{PizzaGate,}''
a baseless notion that a Washington pizza parlor was the center of a
child sex abuse ring,
\href{https://www.nytimes.com/2016/12/05/business/media/comet-ping-pong-pizza-shooting-fake-news-consequences.html}{leading
to a shooting} in 2016, is
\href{https://www.nytimes.com/2020/06/27/technology/pizzagate-justin-bieber-qanon-tiktok.html}{catching
on again with younger people} on TikTok and other online hangouts, my
colleagues \href{https://www.nytimes.com/by/cecilia-kang}{Cecilia Kang}
and \href{https://www.nytimes.com/by/sheera-frenkel}{Sheera Frenkel}
wrote.

I talked to Sheera about how young people have tweaked this conspiracy
and how internet sites help spread false ideas. (And, yes, our names are
pronounced the same but spelled differently.)

\textbf{Shira: How has this false conspiracy changed in four years?}

\textbf{Sheera:} Younger people on TikTok have made PizzaGate more
relatable for them. So a conspiracy that centered on Hillary Clinton and
other politicians a few years ago now instead ropes in celebrities like
Justin Bieber. Everyone is at home, bored and online more than usual.
When I talked to teens who were spreading these conspiracy videos, many
of them said it seemed like fun.

\textbf{If it's for ``fun,'' is this version of the PizzaGate conspiracy
harmless?}

It's not. We've seen over and over that some people can get
\href{https://www.nytimes.com/2020/04/23/podcasts/rabbit-hole-internet-youtube-virus.html}{so
far into conspiracies} that they take them seriously and
\href{https://www.nytimes.com/2020/04/10/technology/coronavirus-5g-uk.html}{commit}\href{https://www.nytimes.com/2020/02/09/us/politics/qanon-trump-conspiracy-theory.html}{real-world
harm}. And for people who are survivors of sexual abuse, it can be
painful to see people talking about it all over social media.

\textbf{Have the internet companies gotten better at stopping false
conspiracies like this?}

They have, but people who want to spread conspiracies are figuring out
workarounds. Facebook banned the PizzaGate hashtag, for example, but the
hashtag is not banned on Instagram, even though it's owned by Facebook.
People also migrated to private groups where Facebook has less
visibility into what's going on.

Tech companies'
\href{https://www.nytimes.com/2020/04/20/technology/youtube-conspiracy-theories.html}{automated
recommendation systems} also can suck people further into false ideas. I
recently tried to join
\href{https://www.nytimes.com/2018/08/01/us/politics/what-is-qanon.html}{Facebook
QAnon conspiracy} groups, and Facebook immediately recommended I join
PizzaGate groups, too. On TikTok, what you see is largely decided by
computer recommendations. So I watched one video about PizzaGate, and
the next videos I saw in the app were all about PizzaGate.

\textbf{TikTok is a relatively new place where conspiracies can spread.
What is it doing to address this?}

TikTok is not proactively going out and looking for videos with
potentially false and dangerous ideas and removing them. There were more
than 80 million views of TikTok videos with PizzaGate-related hashtags.

The New York Times reached out to TikTok about the videos, pointing out
their spike. After we sent our email, TikTok removed many of the videos
and seemed to limit their spread. Facebook and Twitter often do this,
too --- they frequently remove content only after journalists reach out
and point it out.

\textbf{Do you worry that writing about baseless conspiracies gives them
more oxygen?}

We worry about that all the time, and spend as much time debating
whether to write about false conspiracies and misinformation as we do
writing about them.

We watch for ones that reach a critical mass; we don't want to be the
place where people first find out about conspiracies. When a major news
organization writes about a conspiracy --- even to debunk it --- people
who want to believe it will twist it to appear to validate their views.

But to ignore them completely could also be dangerous.

\begin{center}\rule{0.5\linewidth}{\linethickness}\end{center}

\hypertarget{tip-of-the-week}{%
\subsubsection{Tip of the Week}\label{tip-of-the-week}}

\hypertarget{use-a-password-manager-just-do-it}{%
\subsection{Use a Password Manager. Just Do
It.}\label{use-a-password-manager-just-do-it}}

\href{https://www.nytimes.com/by/brian-x-chen}{\emph{Brian X.
Chen}}\emph{, a consumer technology writer at The Times, has annoying
but necessary medicine to better protect ourselves online.}

Everybody needs to use a password manager, a piece of software that
helps you come up with complex passwords for all of your internet
accounts and secures them in a digital vault.

I'm not exaggerating. I mean everyone --- from our grandparents to the
most tech savvy gearheads.

You know that you shouldn't reuse the same password across multiple
accounts. If you do and just one company gets hacked, all of your other
accounts are also vulnerable.

But let's be real: Everybody reuses passwords, or just changes a number
or two at the end. An On Tech reader, Rob Brennan in Montreal, recently
wrote in asking for help because he's now visiting more and more sites
that require passwords. Another reader said that he creates complex
passwords and then has a hard time remembering them. Yes, of course.

It's not your fault. Blame the tech companies and the digital security
community for not coming up with something better than the antiquated,
hopelessly broken password as a way to prove who we are online.

Until there's a better way, the best thing you can do is download a
password management app like 1Password or LastPass onto your mobile
devices and computer. (People who have Apple phones and computers also
have the option of iCloud Keychain. It works only on Apple devices, but
it's better than nothing.)

These password management apps simplify the process by generating a
complex, unique password for you and saving your login details. To
unlock your vault of passwords, you just need to enter one master
password. Even better, you can set up the password management app to
quickly unlock your vault with biometrics like the iPhone's face scanner
or the fingerprint scanner on Android phones.

To guide you through this process, both The Times's product
recommendation site
\href{https://www.nytimes.com/wirecutter/blog/why-you-need-a-password-manager-yes-you/}{Wirecutter}
and I have written
\href{https://www.nytimes.com/2016/01/21/technology/personaltech/apps-to-manage-passwords-so-they-are-harder-to-crack-than-password.html}{articles
on getting started with password management apps}. I'm warning you: This
is a huge chore. For those of us with dozens of internet accounts, you
will have to log in to every one of them and manually change each
password.

These password management apps, however, simplify that process by
generating a complex, unique password for you and saving your login
details.

Just take on this task once, and your digital life will be much safer
for years to come.

\begin{center}\rule{0.5\linewidth}{\linethickness}\end{center}

\hypertarget{before-we-go-}{%
\subsection{Before we go \ldots{}}\label{before-we-go-}}

\begin{itemize}
\item
  \textbf{Resist overly simplistic utopian or dystopian explanations:}
  My colleague \href{https://www.nytimes.com/by/john-herrman}{John
  Herrman} digs into research on
  \href{https://www.nytimes.com/2020/06/28/style/tiktok-teen-politics-gen-z.html}{young
  people's use of TikTok for political expression and activism}. They're
  neither all geniuses who cracked new kinds of political activism
  online, nor all cynics who are drowning in PizzaGate misinformation.
  People are complicated!
  (\href{https://www.nytimes.com/by/charlie-warzel}{Charlie Warzel}, a
  Times Opinion writer,
  \href{https://www.nytimes.com/2020/06/22/opinion/trump-protest-gen-z.html}{had
  a similar take} recently.)
\item
  \textbf{Maybe this time is different? (Or not.)} The Times tech
  reporter \href{https://www.nytimes.com/by/david-streitfeld}{David
  Streitfeld} has
  \href{https://www.nytimes.com/2020/06/29/technology/working-from-home-failure.html}{a
  historical reality check} to predictions that the pandemic will
  permanently result in more remote office work. ``The history of
  telecommuting has been strewn with failure,'' David writes. One
  challenge is that remote work should be paired with fundamental
  changes in work culture, but it usually isn't.
\item
  \textbf{Tracing the history of Facebook's relationship with President
  Trump:} The Washington Post has a
  \href{https://www.washingtonpost.com/people/elizabeth-dwoskin/}{worthy
  read} of Facebook executives' struggles with handling political
  figures, including Mr. Trump, who say inflammatory things online. The
  Post suggested that Facebook employees recommended ways for the
  president to revise one of his recent provocative posts so that it
  pushed right up to the edge of Facebook's rules without breaking them.
\end{itemize}

\hypertarget{hugs-to-this}{%
\subsubsection{Hugs to this}\label{hugs-to-this}}

Sometimes, TikTok is just a place to show your
\href{https://www.tiktok.com/@mzav7/video/6842879004727430406}{wacky cat
stuck in a chandelier}.

\begin{center}\rule{0.5\linewidth}{\linethickness}\end{center}

\emph{We want to hear from you. Tell us what you think of this
newsletter and what else you'd like us to explore. You can reach us at}
\href{mailto:ontech@nytimes.com?subject=On\%20Tech\%20Feedback}{\emph{ontech@nytimes.com.}}
**

\emph{If you don't already get this newsletter in your inbox,}
\href{https://www.nytimes.com/newsletters/signup/OT}{\emph{please sign
up here}}\emph{.}

Advertisement

\protect\hyperlink{after-bottom}{Continue reading the main story}

\hypertarget{site-index}{%
\subsection{Site Index}\label{site-index}}

\hypertarget{site-information-navigation}{%
\subsection{Site Information
Navigation}\label{site-information-navigation}}

\begin{itemize}
\tightlist
\item
  \href{https://help.nytimes.com/hc/en-us/articles/115014792127-Copyright-notice}{©~2020~The
  New York Times Company}
\end{itemize}

\begin{itemize}
\tightlist
\item
  \href{https://www.nytco.com/}{NYTCo}
\item
  \href{https://help.nytimes.com/hc/en-us/articles/115015385887-Contact-Us}{Contact
  Us}
\item
  \href{https://www.nytco.com/careers/}{Work with us}
\item
  \href{https://nytmediakit.com/}{Advertise}
\item
  \href{http://www.tbrandstudio.com/}{T Brand Studio}
\item
  \href{https://www.nytimes.com/privacy/cookie-policy\#how-do-i-manage-trackers}{Your
  Ad Choices}
\item
  \href{https://www.nytimes.com/privacy}{Privacy}
\item
  \href{https://help.nytimes.com/hc/en-us/articles/115014893428-Terms-of-service}{Terms
  of Service}
\item
  \href{https://help.nytimes.com/hc/en-us/articles/115014893968-Terms-of-sale}{Terms
  of Sale}
\item
  \href{https://spiderbites.nytimes.com}{Site Map}
\item
  \href{https://help.nytimes.com/hc/en-us}{Help}
\item
  \href{https://www.nytimes.com/subscription?campaignId=37WXW}{Subscriptions}
\end{itemize}
