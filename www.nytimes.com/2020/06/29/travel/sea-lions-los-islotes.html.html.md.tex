Sections

SEARCH

\protect\hyperlink{site-content}{Skip to
content}\protect\hyperlink{site-index}{Skip to site index}

\href{/section/travel}{Travel}\textbar{}Swimming With the Sea Lions of
Los Islotes

\url{https://nyti.ms/2ZiOgNN}

\begin{itemize}
\item
\item
\item
\item
\item
\item
\end{itemize}

\href{https://www.nytimes.com/spotlight/at-home?action=click\&pgtype=Article\&state=default\&region=TOP_BANNER\&context=at_home_menu}{At
Home}

\begin{itemize}
\tightlist
\item
  \href{https://www.nytimes.com/2020/07/28/books/time-for-a-literary-road-trip.html?action=click\&pgtype=Article\&state=default\&region=TOP_BANNER\&context=at_home_menu}{Take:
  A Literary Road Trip}
\item
  \href{https://www.nytimes.com/2020/07/29/magazine/bored-with-your-home-cooking-some-smoky-eggplant-will-fix-that.html?action=click\&pgtype=Article\&state=default\&region=TOP_BANNER\&context=at_home_menu}{Cook:
  Smoky Eggplant}
\item
  \href{https://www.nytimes.com/2020/07/27/travel/moose-michigan-isle-royale.html?action=click\&pgtype=Article\&state=default\&region=TOP_BANNER\&context=at_home_menu}{Look
  Out: For Moose}
\item
  \href{https://www.nytimes.com/interactive/2020/at-home/even-more-reporters-editors-diaries-lists-recommendations.html?action=click\&pgtype=Article\&state=default\&region=TOP_BANNER\&context=at_home_menu}{Explore:
  Reporters' Obsessions}
\end{itemize}

\includegraphics{https://static01.nyt.com/images/2020/06/30/travel/29travel-sealions-01/merlin_171489507_3e63f5d9-44d5-493b-a1f2-be8d3cc3dac3-articleLarge.jpg?quality=75\&auto=webp\&disable=upscale}

The World Through a Lens

\hypertarget{swimming-with-the-sea-lions-of-los-islotes}{%
\section{Swimming With the Sea Lions of Los
Islotes}\label{swimming-with-the-sea-lions-of-los-islotes}}

Sea lions are often referred to as ``dogs of the sea.'' On a small
island off the Baja coast, where the playful animals populate every
rocky outcropping, they live up to their nickname.

Credit...

Supported by

\protect\hyperlink{after-sponsor}{Continue reading the main story}

Photographs and Text by Benjamin Lowy

\begin{itemize}
\item
  Published June 29, 2020Updated July 5, 2020
\item
  \begin{itemize}
  \item
  \item
  \item
  \item
  \item
  \item
  \end{itemize}
\end{itemize}

\emph{At the onset of the coronavirus pandemic, with travel restrictions
in place worldwide, we launched a new series ---}
\href{https://www.nytimes.com/column/the-world-through-a-lens}{\emph{The
World Through a Lens}} \emph{--- in which photojournalists help
transport you, virtually, to some of our planet's most beautiful and
intriguing places. This week, Benjamin Lowy shares a collection of
photos from an underwater shoot off the coast of Baja.}

\begin{center}\rule{0.5\linewidth}{\linethickness}\end{center}

Nestled off the coast of Baja California Sur, near La Paz, lies a string
of islands in the Sea of Cortez --- including Isla Espíritu Santo and
Isla Partida. Part of a
\href{http://www.unesco.org/new/en/natural-sciences/environment/ecological-sciences/biosphere-reserves/latin-america-and-the-caribbean/mexico/islas-del-golfo-de-california/}{UNESCO
Biosphere Reserve}, and one of Mexico's leading eco-tourism
destinations, these islands, along with their surrounding reefs and
outcrops, are home to countless forms of marine life.

At the northern tip of the chain, just beyond Isla Partida, is a small
craggy outcrop --- roughly a quarter mile long --- called
\href{https://goo.gl/maps/Fb7rYDoTs2rKT8sw6}{Los Islotes}, or Isla
Lobos. Here, where few people have access, lives a large colony of sea
lions.

\includegraphics{https://static01.nyt.com/images/2020/06/29/travel/29travel-sealions-08/merlin_171479976_8dc243c8-d453-4a0a-9770-5b2edbeb2f85-articleLarge.jpg?quality=75\&auto=webp\&disable=upscale}

I visited the area on one of my first underwater assignments, after
years spent covering war, politics and sports.~Diving there was a
transformational experience. Alone, floating in the open water, I found
peace among these playful animals, sometimes referred to as ``dogs of
the sea.''

Image

Image

Sea lions really \emph{do} seem like dogs: They play fetch with rocks,
starfish and the occasional bone, and they often seem enamored by the
few humans who swim with them. They constantly nibbled at my fins, or
stared at their reflection in the dome of my underwater camera housing.

Image

Their evolutionary origin can be traced back to ancient bears and the
\href{https://www.britannica.com/animal/mustelid}{mustelids} (otters,
weasels, ferrets) that, in a reversal of terrestrial mammal evolution,
\href{https://www.nationalgeographic.com/science/phenomena/2009/04/22/puijila-the-walking-seal-a-beautiful-transitional-fossil/}{returned
to the sea and developed fins}.

Image

Swimming around the circumference of Los Islotes is possible, though the
water in some areas is quite shallow, and access is often restricted by
tour guides. Young adolescent sea lions can be seen playing in the surf
and sunbathing on rocks. The bark of dominant males echoes above and
below the water as they patrol the sea for threats to the colony and to
their rule. Above the water, the animals populate every rocky
outcropping, lying prostrate in the sun.

Image

Each side of Los Islotes offers a different look at the sea lions'
habits. Massive rocks jut from the seabed on the northern side, having
eroded off the island long ago. The sea lions frolic back and forth in
these underwater obstacles, chasing each other through the rocky
crevices.

Image

The southern side of the island is where the males sit and sunbathe.
Territorial and ferocious, these gigantic males can be dangerous and
should be given a wide berth.

Near the eastern tip of the island is a giant arch that rises from the
sea. There, divers can drop down to 50 feet, stare up and watch the
silhouettes of sea lions as they swim from one side of the island to the
other.

Image

Perhaps most charming of all: Hidden on the southern side of the
islands, within a small yet easily accessible underwater cave, is a sea
lion ``nursery.'' Small juveniles dart back and forth under the watchful
eye of older females.

Unfortunately, even these protected sea lions can't avoid the
encroachment of humans; they can occasionally be seen with fishing line
wrapped around their necks, which can lead to infection and death. And
therein lies part of the challenge for local marine park rangers and
scientists: how to strike the right balance between conservation,
education and eco-tourism.

Image

In a way, it's similar to the balance I try to strike myself: My mission
as a photographer is to communicate what I see and experience to an
audience. My mission as a father is to
\href{https://vimeo.com/258546007}{educate my children about the world
around them}, and about the conservation of the environment.

Many of these photographs were made with my wife and boys by my side.
(These waters, after all, are where my son learned how to scuba dive
under the tutelage of a lifelong friend.) And I have yet to find a
photographic subject that has brought me more peace and tranquillity
than swimming with the sea lions of Los Islotes.

\begin{center}\rule{0.5\linewidth}{\linethickness}\end{center}

\emph{Benjamin Lowy is an American photojournalist based in New York
City. You can follow his work on}
\href{https://www.instagram.com/benlowy/}{\emph{Instagram}} \emph{and}
\href{https://twitter.com/benlowy}{\emph{Twitter}}\emph{.}

\emph{\textbf{Follow New York Times Travel}} \emph{on}
\href{https://www.instagram.com/nytimestravel/}{\emph{Instagram}}\emph{,}
\href{https://twitter.com/nytimestravel}{\emph{Twitter}} \emph{and}
\href{https://www.facebook.com/nytimestravel/}{\emph{Facebook}}\emph{.
And}
\href{https://www.nytimes.com/newsletters/traveldispatch}{\emph{sign up
for our weekly Travel Dispatch newsletter}} \emph{to receive expert tips
on traveling smarter and inspiration for your next vacation.}

Advertisement

\protect\hyperlink{after-bottom}{Continue reading the main story}

\hypertarget{site-index}{%
\subsection{Site Index}\label{site-index}}

\hypertarget{site-information-navigation}{%
\subsection{Site Information
Navigation}\label{site-information-navigation}}

\begin{itemize}
\tightlist
\item
  \href{https://help.nytimes.com/hc/en-us/articles/115014792127-Copyright-notice}{©~2020~The
  New York Times Company}
\end{itemize}

\begin{itemize}
\tightlist
\item
  \href{https://www.nytco.com/}{NYTCo}
\item
  \href{https://help.nytimes.com/hc/en-us/articles/115015385887-Contact-Us}{Contact
  Us}
\item
  \href{https://www.nytco.com/careers/}{Work with us}
\item
  \href{https://nytmediakit.com/}{Advertise}
\item
  \href{http://www.tbrandstudio.com/}{T Brand Studio}
\item
  \href{https://www.nytimes.com/privacy/cookie-policy\#how-do-i-manage-trackers}{Your
  Ad Choices}
\item
  \href{https://www.nytimes.com/privacy}{Privacy}
\item
  \href{https://help.nytimes.com/hc/en-us/articles/115014893428-Terms-of-service}{Terms
  of Service}
\item
  \href{https://help.nytimes.com/hc/en-us/articles/115014893968-Terms-of-sale}{Terms
  of Sale}
\item
  \href{https://spiderbites.nytimes.com}{Site Map}
\item
  \href{https://help.nytimes.com/hc/en-us}{Help}
\item
  \href{https://www.nytimes.com/subscription?campaignId=37WXW}{Subscriptions}
\end{itemize}
