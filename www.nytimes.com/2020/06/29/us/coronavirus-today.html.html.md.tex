Sections

SEARCH

\protect\hyperlink{site-content}{Skip to
content}\protect\hyperlink{site-index}{Skip to site index}

\href{https://www.nytimes.com/section/us}{U.S.}

\href{https://myaccount.nytimes.com/auth/login?response_type=cookie\&client_id=vi}{}

\href{https://www.nytimes.com/section/todayspaper}{Today's Paper}

\href{/section/us}{U.S.}\textbar{}Coronavirus Briefing: What Happened
Today

\url{https://nyti.ms/2Zpqgc1}

\begin{itemize}
\item
\item
\item
\item
\item
\end{itemize}

\href{https://www.nytimes.com/news-event/coronavirus?action=click\&pgtype=Article\&state=default\&region=TOP_BANNER\&context=storylines_menu}{The
Coronavirus Outbreak}

\begin{itemize}
\tightlist
\item
  live\href{https://www.nytimes.com/2020/08/04/world/coronavirus-cases.html?action=click\&pgtype=Article\&state=default\&region=TOP_BANNER\&context=storylines_menu}{Latest
  Updates}
\item
  \href{https://www.nytimes.com/interactive/2020/us/coronavirus-us-cases.html?action=click\&pgtype=Article\&state=default\&region=TOP_BANNER\&context=storylines_menu}{Maps
  and Cases}
\item
  \href{https://www.nytimes.com/interactive/2020/science/coronavirus-vaccine-tracker.html?action=click\&pgtype=Article\&state=default\&region=TOP_BANNER\&context=storylines_menu}{Vaccine
  Tracker}
\item
  \href{https://www.nytimes.com/2020/08/02/us/covid-college-reopening.html?action=click\&pgtype=Article\&state=default\&region=TOP_BANNER\&context=storylines_menu}{College
  Reopening}
\item
  \href{https://www.nytimes.com/live/2020/08/04/business/stock-market-today-coronavirus?action=click\&pgtype=Article\&state=default\&region=TOP_BANNER\&context=storylines_menu}{Economy}
\end{itemize}

Advertisement

\protect\hyperlink{after-top}{Continue reading the main story}

Supported by

\protect\hyperlink{after-sponsor}{Continue reading the main story}

\hypertarget{coronavirus-briefing-what-happened-today}{%
\section{Coronavirus Briefing: What Happened
Today}\label{coronavirus-briefing-what-happened-today}}

Californians are asking themselves, what went wrong?

\href{https://www.nytimes.com/by/jonathan-wolfe}{\includegraphics{https://static01.nyt.com/images/2018/08/24/multimedia/author-jonathan-wolfe/author-jonathan-wolfe-thumbLarge.png}}\href{https://www.nytimes.com/by/lara-takenaga}{\includegraphics{https://static01.nyt.com/images/2019/02/20/multimedia/author-lara-takenaga/author-lara-takenaga-thumbLarge.png}}

By \href{https://www.nytimes.com/by/jonathan-wolfe}{Jonathan Wolfe} and
\href{https://www.nytimes.com/by/lara-takenaga}{Lara Takenaga}

\begin{itemize}
\item
  June 29, 2020
\item
  \begin{itemize}
  \item
  \item
  \item
  \item
  \item
  \end{itemize}
\end{itemize}

This is the Coronavirus Briefing, an informed guide to the global
outbreak.
\href{https://www.nytimes.com/newsletters/coronavirus-briefing}{Sign up
here to get the briefing by email}.

\includegraphics{https://static01.nyt.com/images/2020/06/29/us/oakImage-1593461886425/oakImage-1593461886425-articleLarge.png?quality=75\&auto=webp\&disable=upscale}

\begin{itemize}
\item
  The global death toll
  \href{https://www.nytimes.com/2020/06/28/world/coronavirus-updates.html\#link-31665a4b}{has
  passed 500,000}, with more than 10 million confirmed coronavirus
  cases.
\item
  Vice President Mike Pence made a point of
  \href{https://www.nytimes.com/2020/06/29/world/coronavirus-updates.html\#link-56ea353b}{wearing
  a mask} during public events over the weekend.
\item
  The World Health Organization is sending a team of experts to China
  \href{https://www.nytimes.com/2020/06/29/world/coronavirus-live-updates.html\#link-9c42599}{to
  investigate the origins of the virus}.
\item
  Get the \href{https://www.nytimes.com/news-event/coronavirus}{latest
  updates here}, plus
  \href{https://www.nytimes.com/interactive/2020/world/asia/china-wuhan-coronavirus-maps.html?action=click\&pgtype=Article\&state=default\&module=styln-coronavirus\&variant=show\&region=TOP_BANNER\&context=storyline_menu}{maps}
  and a
  \href{https://www.nytimes.com/interactive/2020/04/03/upshot/coronavirus-metro-area-tracker.html}{tracker
  for U.S. metro areas}.
\end{itemize}

\begin{center}\rule{0.5\linewidth}{\linethickness}\end{center}

\hypertarget{california-backslides}{%
\subsection{California backslides}\label{california-backslides}}

Early in the outbreak, California emerged as a leader in fighting the
spread of the coronavirus. It was the first state to impose a
stay-at-home order, and its swift response is thought to have
\href{https://www.sfchronicle.com/bayarea/article/Study-Shelter-in-place-helped-avert-4-8-million-15325645.php}{prevented
1.7 million coronavirus cases in the state}.

For months it seemed that California --- considered especially
vulnerable to the virus because of its large, globe-trotting population
--- was weathering the storm relatively well.

But over the last week, things have changed. The
\href{https://www.nytimes.com/interactive/2020/us/california-coronavirus-cases.html}{state's
case count} has exploded, reaching 200,000 infections. Gov. Gavin Newsom
has rolled back reopening plans. And officials in Los Angeles County are
\href{https://www.latimes.com/california/story/2020-06-29/l-a-county-issues-dire-warning-amid-alarming-increases-in-coronavirus}{projecting
that they may run out of hospital beds} in two to three weeks.

Now Californians are asking themselves, what went wrong?

The turn, some say, may have come Memorial Day weekend, when cooped-up
residents responded to the state's reopening by getting out and
socializing. According to
\href{https://www.latimes.com/california/story/2020-06-29/california-coronavirus-hot-spot-memorial-day}{an
analysis by The Los Angeles Times}, coronavirus hospitalizations in the
state began accelerating around June 15 --- which, given the incubation
period of the virus, points to holiday barbecues, beach trips and
graduation parties as potential culprits.

The Los Angeles Times also
\href{https://www.latimes.com/california/story/2020-06-29/california-coronavirus-hot-spot-memorial-day}{noted}
that Californians, unlike New Yorkers, had not yet felt the trauma of
having their hospitals overloaded with patients, so many saw reopening
as a license to return to life as it was before the pandemic.

Our colleague Jill Cowan, who
\href{https://www.nytimes.com/column/california-today}{writes the
California Today newsletter}, told us that a complex patchwork of rules
that can change from county to county was also to blame.

``State and local officials would say that, recently, people have been
letting their guard down, and they've been gathering with family members
and they've been going to bars and restaurants,'' Jill told us. ``But a
lot of people are then saying, `Well, why did you let restaurants and
bars reopen?'''

\hypertarget{latest-updates-global-coronavirus-outbreak}{%
\section{\texorpdfstring{\href{https://www.nytimes.com/2020/08/04/world/coronavirus-cases.html?action=click\&pgtype=Article\&state=default\&region=MAIN_CONTENT_1\&context=storylines_live_updates}{Latest
Updates: Global Coronavirus
Outbreak}}{Latest Updates: Global Coronavirus Outbreak}}\label{latest-updates-global-coronavirus-outbreak}}

Updated 2020-08-04T20:57:54.346Z

\begin{itemize}
\tightlist
\item
  \href{https://www.nytimes.com/2020/08/04/world/coronavirus-cases.html?action=click\&pgtype=Article\&state=default\&region=MAIN_CONTENT_1\&context=storylines_live_updates\#link-1228a480}{Novavax
  sees encouraging results from two studies of its experimental
  vaccine.}
\item
  \href{https://www.nytimes.com/2020/08/04/world/coronavirus-cases.html?action=click\&pgtype=Article\&state=default\&region=MAIN_CONTENT_1\&context=storylines_live_updates\#link-4825b93}{Public
  and private schools in Maryland and elsewhere are divided over
  in-person instruction.}
\item
  \href{https://www.nytimes.com/2020/08/04/world/coronavirus-cases.html?action=click\&pgtype=Article\&state=default\&region=MAIN_CONTENT_1\&context=storylines_live_updates\#link-50f7386d}{The
  United Nations calls on policymakers to `plan thoroughly for school
  reopenings.'}
\end{itemize}

\href{https://www.nytimes.com/2020/08/04/world/coronavirus-cases.html?action=click\&pgtype=Article\&state=default\&region=MAIN_CONTENT_1\&context=storylines_live_updates}{See
more updates}

More live coverage:
\href{https://www.nytimes.com/live/2020/08/04/business/stock-market-today-coronavirus?action=click\&pgtype=Article\&state=default\&region=MAIN_CONTENT_1\&context=storylines_live_updates}{Markets}

``It's very confusing and it's very complicated,'' she added. ``And so
for regular Californians, I think it's been tough to navigate what you
shouldn't do --- even if you can.''

\begin{center}\rule{0.5\linewidth}{\linethickness}\end{center}

\hypertarget{a-nightmare-side-effect}{%
\subsection{A nightmare side effect}\label{a-nightmare-side-effect}}

Many severely ill patients have developed a terrifying condition
\href{https://www.nytimes.com/2020/06/28/health/coronavirus-delirium-hallucinations.html}{that
causes nightmarish visions and can have long-lasting consequences}.
Known as hospital delirium, the phenomenon, which was observed mostly in
older people before the pandemic, has struck Covid-19 patients of all
ages.

Reports suggest that about two-thirds to three-quarters of virus
patients who end up in intensive-care units, even for relatively short
stays, have experienced the condition. Their hospitalization often
provides the perfect combination of elements: long stints on
ventilators, heavy sedatives, poor sleep, minimal social interaction.

Delirium takes two forms --- hyperactive, which leads to paranoid
hallucinations and agitation, and hypoactive, which causes internalized
visions and confusion. Some people experience both.

Recovered virus patients have described thinking they were being
abducted or burned alive. Even after their visions go away, the
condition can slow the healing process and increase the risk of
depression or post-traumatic stress. Older patients can also develop
dementia sooner than they otherwise would have, and even die earlier.

\textbf{Another troubling development:} Immunologists believe the
coronavirus may be responsible
\href{https://www.nytimes.com/2020/06/26/health/coronavirus-immune-system.html}{for
depleting disease-fighting T cells}, similar to how H.I.V. operates. If
that's the case, protection against the virus could be fleeting, and a
cocktail of antiviral drugs may be needed to control it.

\href{https://www.nytimes.com/news-event/coronavirus?action=click\&pgtype=Article\&state=default\&region=MAIN_CONTENT_3\&context=storylines_faq}{}

\hypertarget{the-coronavirus-outbreak-}{%
\subsubsection{The Coronavirus Outbreak
›}\label{the-coronavirus-outbreak-}}

\hypertarget{frequently-asked-questions}{%
\paragraph{Frequently Asked
Questions}\label{frequently-asked-questions}}

Updated August 4, 2020

\begin{itemize}
\item ~
  \hypertarget{i-have-antibodies-am-i-now-immune}{%
  \paragraph{I have antibodies. Am I now
  immune?}\label{i-have-antibodies-am-i-now-immune}}

  \begin{itemize}
  \tightlist
  \item
    As of right
    now,\href{https://www.nytimes.com/2020/07/22/health/covid-antibodies-herd-immunity.html?action=click\&pgtype=Article\&state=default\&region=MAIN_CONTENT_3\&context=storylines_faq}{that
    seems likely, for at least several months.} There have been
    frightening accounts of people suffering what seems to be a second
    bout of Covid-19. But experts say these patients may have a
    drawn-out course of infection, with the virus taking a slow toll
    weeks to months after initial exposure. People infected with the
    coronavirus typically
    \href{https://www.nature.com/articles/s41586-020-2456-9}{produce}
    immune molecules called antibodies, which are
    \href{https://www.nytimes.com/2020/05/07/health/coronavirus-antibody-prevalence.html?action=click\&pgtype=Article\&state=default\&region=MAIN_CONTENT_3\&context=storylines_faq}{protective
    proteins made in response to an
    infection}\href{https://www.nytimes.com/2020/05/07/health/coronavirus-antibody-prevalence.html?action=click\&pgtype=Article\&state=default\&region=MAIN_CONTENT_3\&context=storylines_faq}{.
    These antibodies may} last in the body
    \href{https://www.nature.com/articles/s41591-020-0965-6}{only two to
    three months}, which may seem worrisome, but that's perfectly normal
    after an acute infection subsides, said Dr. Michael Mina, an
    immunologist at Harvard University. It may be possible to get the
    coronavirus again, but it's highly unlikely that it would be
    possible in a short window of time from initial infection or make
    people sicker the second time.
  \end{itemize}
\item ~
  \hypertarget{im-a-small-business-owner-can-i-get-relief}{%
  \paragraph{I'm a small-business owner. Can I get
  relief?}\label{im-a-small-business-owner-can-i-get-relief}}

  \begin{itemize}
  \tightlist
  \item
    The
    \href{https://www.nytimes.com/article/small-business-loans-stimulus-grants-freelancers-coronavirus.html?action=click\&pgtype=Article\&state=default\&region=MAIN_CONTENT_3\&context=storylines_faq}{stimulus
    bills enacted in March} offer help for the millions of American
    small businesses. Those eligible for aid are businesses and
    nonprofit organizations with fewer than 500 workers, including sole
    proprietorships, independent contractors and freelancers. Some
    larger companies in some industries are also eligible. The help
    being offered, which is being managed by the Small Business
    Administration, includes the Paycheck Protection Program and the
    Economic Injury Disaster Loan program. But lots of folks have
    \href{https://www.nytimes.com/interactive/2020/05/07/business/small-business-loans-coronavirus.html?action=click\&pgtype=Article\&state=default\&region=MAIN_CONTENT_3\&context=storylines_faq}{not
    yet seen payouts.} Even those who have received help are confused:
    The rules are draconian, and some are stuck sitting on
    \href{https://www.nytimes.com/2020/05/02/business/economy/loans-coronavirus-small-business.html?action=click\&pgtype=Article\&state=default\&region=MAIN_CONTENT_3\&context=storylines_faq}{money
    they don't know how to use.} Many small-business owners are getting
    less than they expected or
    \href{https://www.nytimes.com/2020/06/10/business/Small-business-loans-ppp.html?action=click\&pgtype=Article\&state=default\&region=MAIN_CONTENT_3\&context=storylines_faq}{not
    hearing anything at all.}
  \end{itemize}
\item ~
  \hypertarget{what-are-my-rights-if-i-am-worried-about-going-back-to-work}{%
  \paragraph{What are my rights if I am worried about going back to
  work?}\label{what-are-my-rights-if-i-am-worried-about-going-back-to-work}}

  \begin{itemize}
  \tightlist
  \item
    Employers have to provide
    \href{https://www.osha.gov/SLTC/covid-19/standards.html}{a safe
    workplace} with policies that protect everyone equally.
    \href{https://www.nytimes.com/article/coronavirus-money-unemployment.html?action=click\&pgtype=Article\&state=default\&region=MAIN_CONTENT_3\&context=storylines_faq}{And
    if one of your co-workers tests positive for the coronavirus, the
    C.D.C.} has said that
    \href{https://www.cdc.gov/coronavirus/2019-ncov/community/guidance-business-response.html}{employers
    should tell their employees} -\/- without giving you the sick
    employee's name -\/- that they may have been exposed to the virus.
  \end{itemize}
\item ~
  \hypertarget{should-i-refinance-my-mortgage}{%
  \paragraph{Should I refinance my
  mortgage?}\label{should-i-refinance-my-mortgage}}

  \begin{itemize}
  \tightlist
  \item
    \href{https://www.nytimes.com/article/coronavirus-money-unemployment.html?action=click\&pgtype=Article\&state=default\&region=MAIN_CONTENT_3\&context=storylines_faq}{It
    could be a good idea,} because mortgage rates have
    \href{https://www.nytimes.com/2020/07/16/business/mortgage-rates-below-3-percent.html?action=click\&pgtype=Article\&state=default\&region=MAIN_CONTENT_3\&context=storylines_faq}{never
    been lower.} Refinancing requests have pushed mortgage applications
    to some of the highest levels since 2008, so be prepared to get in
    line. But defaults are also up, so if you're thinking about buying a
    home, be aware that some lenders have tightened their standards.
  \end{itemize}
\item ~
  \hypertarget{what-is-school-going-to-look-like-in-september}{%
  \paragraph{What is school going to look like in
  September?}\label{what-is-school-going-to-look-like-in-september}}

  \begin{itemize}
  \tightlist
  \item
    It is unlikely that many schools will return to a normal schedule
    this fall, requiring the grind of
    \href{https://www.nytimes.com/2020/06/05/us/coronavirus-education-lost-learning.html?action=click\&pgtype=Article\&state=default\&region=MAIN_CONTENT_3\&context=storylines_faq}{online
    learning},
    \href{https://www.nytimes.com/2020/05/29/us/coronavirus-child-care-centers.html?action=click\&pgtype=Article\&state=default\&region=MAIN_CONTENT_3\&context=storylines_faq}{makeshift
    child care} and
    \href{https://www.nytimes.com/2020/06/03/business/economy/coronavirus-working-women.html?action=click\&pgtype=Article\&state=default\&region=MAIN_CONTENT_3\&context=storylines_faq}{stunted
    workdays} to continue. California's two largest public school
    districts --- Los Angeles and San Diego --- said on July 13, that
    \href{https://www.nytimes.com/2020/07/13/us/lausd-san-diego-school-reopening.html?action=click\&pgtype=Article\&state=default\&region=MAIN_CONTENT_3\&context=storylines_faq}{instruction
    will be remote-only in the fall}, citing concerns that surging
    coronavirus infections in their areas pose too dire a risk for
    students and teachers. Together, the two districts enroll some
    825,000 students. They are the largest in the country so far to
    abandon plans for even a partial physical return to classrooms when
    they reopen in August. For other districts, the solution won't be an
    all-or-nothing approach.
    \href{https://bioethics.jhu.edu/research-and-outreach/projects/eschool-initiative/school-policy-tracker/}{Many
    systems}, including the nation's largest, New York City, are
    devising
    \href{https://www.nytimes.com/2020/06/26/us/coronavirus-schools-reopen-fall.html?action=click\&pgtype=Article\&state=default\&region=MAIN_CONTENT_3\&context=storylines_faq}{hybrid
    plans} that involve spending some days in classrooms and other days
    online. There's no national policy on this yet, so check with your
    municipal school system regularly to see what is happening in your
    community.
  \end{itemize}
\end{itemize}

\begin{center}\rule{0.5\linewidth}{\linethickness}\end{center}

\hypertarget{the-things-they-left-behind}{%
\subsection{The things they left
behind}\label{the-things-they-left-behind}}

Image

Credit...Christopher Gregory for The New York Times

Cellphones, chargers, canes, hearing aids, glasses, clothing, shoes,
wallets, Bibles, jewelry: Across New York, hospital workers have had to
figure out what to do with the
\href{https://www.nytimes.com/2020/06/29/nyregion/coronavirus-hospitals-patients-belongings.html}{thousands
of items left behind by patients who have died from the coronavirus.}

\begin{center}\rule{0.5\linewidth}{\linethickness}\end{center}

\hypertarget{resurgences}{%
\subsection{Resurgences}\label{resurgences}}

\begin{itemize}
\item
  Authorities in \textbf{China} have
  \href{https://www.nytimes.com/2020/06/29/world/coronavirus-updates.html\#link-4f49ac1}{imposed
  a strict lockdown} on nearly half a million people near Beijing, after
  a small but stubborn second wave of infections.
\item
  In a reversal, officials in \textbf{Jacksonville, Fla.,} say
  \href{https://www.nytimes.com/2020/06/29/world/coronavirus-live-updates.html\#link-102ec095}{masks
  will be required} in indoor public spaces where social distancing
  isn't possible.
\item
  Even as the virus surges across far-flung regions of \textbf{Russia},
  the Kremlin
  \href{https://www.nytimes.com/2020/06/29/world/europe/coronavirus-russia-putin-referendum.html}{is
  pushing ahead with a referendum} that could keep President Vladimir
  Putin in power until 2036.
\item
  After an outbreak around \textbf{Leicester}, in central England, the
  city
  \href{https://www.nytimes.com/2020/06/29/world/coronavirus-updates.html\#link-4f49ac1}{might
  not follow the rest of the country} as it reopens pubs, restaurants
  and hotels this weekend.
\item
  \textbf{Broadway}
  \href{https://www.nytimes.com/2020/06/29/world/coronavirus-live-updates.html\#link-27ecd755}{will
  remain closed} for at least the rest of this year. Many shows are
  signaling that they do not expect a return until late winter or early
  spring.
\end{itemize}

\href{https://www.nytimes.com/interactive/2020/us/states-reopen-map-coronavirus.html}{\emph{Here's
a roundup of restrictions in all 50 states}}\emph{.}

\begin{center}\rule{0.5\linewidth}{\linethickness}\end{center}

\hypertarget{what-else-were-following}{%
\subsection{What else we're following}\label{what-else-were-following}}

\begin{itemize}
\item
  For the last decade, a growing middle class in Africa has helped spur
  educational, political and economic development. But the pandemic is
  threatening to
  \href{https://www.nytimes.com/2020/06/29/world/africa/Africa-middle-class-coronavirus.html}{throw
  millions of people into poverty}.
\item
  More than
  \href{https://www.nytimes.com/interactive/2020/us/coronavirus-nursing-homes.html}{43
  percent of virus deaths} in the U.S. have been tied to nursing homes
  and long-term care facilities, according to a New York Times database.
\item
  Gilead Sciences announced that it would charge
  \href{https://www.nytimes.com/2020/06/29/world/coronavirus-updates.html\#link-3789537d}{at
  least \$2,340 per treatment of remdesivir}, a drug that has shown
  modest benefits in some patients.
\item
  Health care prices can be unpredictable in the U.S. Two friends in
  Texas were tested for the coronavirus at the same location, and
  \href{https://www.nytimes.com/2020/06/29/upshot/coronavirus-tests-unpredictable-prices.html?action=click\&module=Top\%20Stories\&pgtype=Homepage}{one
  bill was \$6,000 more than the other}.
\item
  Outbreaks
  \href{https://www.nytimes.com/2020/06/28/world/coronavirus-updates.html\#link-212c33ba}{from
  restaurants are growing} as more U.S. states permit dining indoors.
\item
  A Times correspondent
  \href{https://www.nytimes.com/2020/06/28/us/coronavirus-hospital-houston.html}{visited
  a hospital in Houston} to see how it is coping with a surge in virus
  cases and preparing for its peak.
\item
  Some companies are now convinced that remote work has a bright future.
  But decades of
  \href{https://www.nytimes.com/2020/06/29/technology/working-from-home-failure.html}{failed
  telecommuting ventures say otherwise}.
\item
  Buyer beware: Cards for sale that
  \href{https://www.nytimes.com/2020/06/28/world/coronavirus-updates.html\#link-6f2f6d9a}{claim
  to exempt people from wearing masks} are fake.
\end{itemize}

\begin{center}\rule{0.5\linewidth}{\linethickness}\end{center}

\hypertarget{what-youre-doing}{%
\subsection{What you're doing}\label{what-youre-doing}}

\begin{quote}
I live in a really small, rural logging town. While we don't have Covid
cases here, we have mostly all been staying home and following the
guidelines. Several of my neighbors and I have started a ``Porch Ninja''
game, where we sneak over and leave homemade goodies, wine or fun
surprises on each other's porches.

--- Carrie Bredy, Morton, Wash.
\end{quote}

Let us know how you're dealing with the outbreak.
\href{https://www.nytimes.com/2020/03/02/reader-center/coronavirus-preparation.html}{Send
us a response here}, and we may feature it in an upcoming newsletter.

\href{https://www.nytimes.com/newsletters/coronavirus-briefing}{Sign up
here to get the briefing by email}.

Advertisement

\protect\hyperlink{after-bottom}{Continue reading the main story}

\hypertarget{site-index}{%
\subsection{Site Index}\label{site-index}}

\hypertarget{site-information-navigation}{%
\subsection{Site Information
Navigation}\label{site-information-navigation}}

\begin{itemize}
\tightlist
\item
  \href{https://help.nytimes.com/hc/en-us/articles/115014792127-Copyright-notice}{©~2020~The
  New York Times Company}
\end{itemize}

\begin{itemize}
\tightlist
\item
  \href{https://www.nytco.com/}{NYTCo}
\item
  \href{https://help.nytimes.com/hc/en-us/articles/115015385887-Contact-Us}{Contact
  Us}
\item
  \href{https://www.nytco.com/careers/}{Work with us}
\item
  \href{https://nytmediakit.com/}{Advertise}
\item
  \href{http://www.tbrandstudio.com/}{T Brand Studio}
\item
  \href{https://www.nytimes.com/privacy/cookie-policy\#how-do-i-manage-trackers}{Your
  Ad Choices}
\item
  \href{https://www.nytimes.com/privacy}{Privacy}
\item
  \href{https://help.nytimes.com/hc/en-us/articles/115014893428-Terms-of-service}{Terms
  of Service}
\item
  \href{https://help.nytimes.com/hc/en-us/articles/115014893968-Terms-of-sale}{Terms
  of Sale}
\item
  \href{https://spiderbites.nytimes.com}{Site Map}
\item
  \href{https://help.nytimes.com/hc/en-us}{Help}
\item
  \href{https://www.nytimes.com/subscription?campaignId=37WXW}{Subscriptions}
\end{itemize}
