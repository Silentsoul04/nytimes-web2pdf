Sections

SEARCH

\protect\hyperlink{site-content}{Skip to
content}\protect\hyperlink{site-index}{Skip to site index}

\href{https://www.nytimes.com/section/world/asia}{Asia Pacific}

\href{https://myaccount.nytimes.com/auth/login?response_type=cookie\&client_id=vi}{}

\href{https://www.nytimes.com/section/todayspaper}{Today's Paper}

\href{/section/world/asia}{Asia Pacific}\textbar{}India Bans Nearly 60
Chinese Apps, Including TikTok and WeChat

\url{https://nyti.ms/31qy53w}

\begin{itemize}
\item
\item
\item
\item
\item
\end{itemize}

Advertisement

\protect\hyperlink{after-top}{Continue reading the main story}

Supported by

\protect\hyperlink{after-sponsor}{Continue reading the main story}

\hypertarget{india-bans-nearly-60-chinese-apps-including-tiktok-and-wechat}{%
\section{India Bans Nearly 60 Chinese Apps, Including TikTok and
WeChat}\label{india-bans-nearly-60-chinese-apps-including-tiktok-and-wechat}}

The move is part of the tit-for-tat retaliation after the Indian and
Chinese militaries clashed earlier this month.

\includegraphics{https://static01.nyt.com/images/2020/06/29/world/29india-chinese-apps/29india-chinese-apps-articleLarge-v2.jpg?quality=75\&auto=webp\&disable=upscale}

\href{https://www.nytimes.com/by/maria-abi-habib}{\includegraphics{https://static01.nyt.com/images/2018/10/08/multimedia/author-maria-abi-habib/author-maria-abi-habib-thumbLarge.png}}

By \href{https://www.nytimes.com/by/maria-abi-habib}{Maria Abi-Habib}

\begin{itemize}
\item
  Published June 29, 2020Updated June 30, 2020
\item
  \begin{itemize}
  \item
  \item
  \item
  \item
  \item
  \end{itemize}
\end{itemize}

\href{https://cn.nytimes.com/world/20200630/tik-tok-banned-india-china/}{阅读简体中文版}\href{https://cn.nytimes.com/world/20200630/tik-tok-banned-india-china/zh-hant/}{閱讀繁體中文版}

\href{https://www.nytimes.com/2020/06/30/technology/india-china-tiktok.html}{India's}
government banned nearly 60 Chinese mobile apps on Monday, including
\href{https://www.nytimes.com/2020/06/30/technology/india-china-tiktok.html}{TikTok},
citing national security concerns, after a deadly clash between their
militaries this month raised tensions between the two countries to the
highest level in decades.

The fighting two weeks ago, along the disputed border between the
world's two most populous countries,
\href{https://www.nytimes.com/2020/06/16/world/asia/indian-china-border-clash.html}{left
20 Indian soldiers dead} and an unknown number of Chinese casualties.

While India has vowed to retaliate, it lags far behind
\href{https://www.nytimes.com/2020/06/30/technology/india-china-tiktok.html}{China}
in military and economic power,
\href{https://www.nytimes.com/2020/06/19/world/asia/india-china-border.html}{leaving
it with few options}. But Chinese telecommunication and social
networking companies have long eyed India's giant market and its
enormous potential. About 50 percent of India's 1.3 billion citizens are
online.

In addition to TikTok, the popular video-sharing social networking
platform, the banned apps include WeChat, UC Browser, Shareit and Baidu
Map.

Analysts say up to a third of TikTok's global users are based in India.

The Chinese apps were ``stealing and surreptitiously transmitting users'
data in an unauthorized manner to servers which have locations outside
India,'' India's Ministry of Electronics and Information Technology said
in a statement Monday.

``The compilation of these data, its mining and profiling by elements
hostile to national security and defense of India, which ultimately
impinges upon the sovereignty and integrity of India, is a matter of
very deep and immediate concern which requires emergency measures,'' the
statement added.

The Chinese government did not immediately comment on the move, which
was announced late at night in Beijing, nor did TikTok.

This month's border brawl was the worst violence between the two
nuclear-armed countries in more than 50 years. India blamed China for
provoking the clash by intruding into territory it claims high in the
Himalayan mountain range. China said the incident happened on its side
of the border, and that Indian troops intruded.

The situation remains tense, with troop buildups on both sides of the
border.

\includegraphics{https://static01.nyt.com/images/2020/06/29/world/29india-chinese-apps2/merlin_174041985_6b8befd9-3c76-434d-8d76-d0ab380753a6-articleLarge.jpg?quality=75\&auto=webp\&disable=upscale}

Cybersecurity analysts have warned in the past about the risks Chinese
apps and telecom companies may pose, citing the country's
\href{https://www.lawfareblog.com/beijings-new-national-intelligence-law-defense-offense}{National
Intelligence Law}. The law holds Chinese companies legally responsible
for providing access, cooperation or support for Chinese intelligence
gathering.

The same argument has been at the center of a campaign to persuade
Western countries not to allow Chinese companies to build their
next-generation 5G wireless networks.

``India's concerns aren't overblown, they are valid,'' said Christopher
Ahlberg, the chief executive of Recorded Future, a cybersecurity company
in Massachusetts that analyzes and collects threat intelligence.

``China would not be above using these apps for large scale data
collection,'' Mr. Ahlberg added. ``I don't expect that the government is
running all these apps, but they may make an agreement with the
companies that they have to cooperate once in a while. And it's easy
under Chinese law to require them to do so.''

After
\href{https://www.nytimes.com/2017/07/26/world/asia/dolam-plateau-china-india-bhutan.html}{a
2017 clash} between India's and China's militaries over another border
dispute, Indian troops
were\href{https://theprint.in/defence/troops-told-to-format-smartphones-delete-42-apps-after-chinese-spyware-threat/19042/}{forced
to delete dozens of Chinese apps} from their phones over national
security concerns. Some of the apps Indian troops were ordered to delete
then --- such as Weibo, UC Browser and Shareit --- are among those that
are now banned for the entire country.

``Chinese mobile app firms and other tech firms are beholden to the CCP
under Chinese law,'' Brahma Chellaney, a former adviser to India's
National Security Council,
\href{https://twitter.com/Chellaney/status/1277646337803038721}{tweeted
on Monday}, referring to the country's ruling Communist Party. ``As
extensions of the Chinese state, they pose a national security risk.''

Monday's move comes shortly after India's government quietly told two
state-run telecommunication firms to stop using Chinese equipment and
instead use local providers,
\href{https://economictimes.indiatimes.com/industry/telecom/telecom-news/india-to-bar-bsnl-from-sourcing-gear-from-huawei-zte-may-also-bar-pvt-telcos-from-using-chinese-gear/articleshow/76431605.cms?utm_source=contentofinterest\&utm_medium=text\&utm_campaign=cppst}{according
to Reuters}. And in April,
\href{https://www.reuters.com/article/us-health-coronavirus-india-investments/india-toughens-rules-on-investments-from-neighbours-seen-aimed-at-china-idUSKBN2200LQ}{the
government passed legislation} requiring government approval for any
investments from Chinese entities.

``Techno-nationalism has been in vogue in India for a while, which views
data as a national asset,'' said Nikhil Pahwa, the founder of MediaNama,
an organization that advocates a free and open internet.

Image

A protester calling for the boycott of Chinese products in Ahmedabad,
India, last week.Credit...Ajit Solanki/Associated Press

Mr. Pahwa added that while the Indian government has had longstanding
worries that Chinese companies are dominating local markets and are
beating out Indian app developers, it also has national security
concerns about what China does with the data it collects.

``The government views Indian citizens' data as sovereign. India has a
fear of digital colonization, and has avoided signing data sharing
agreements in the past,'' Mr. Pahwa said.

India's announcement highlights how technology companies are
increasingly becoming entangled in broader geopolitical disputes. In
China, American platforms like Facebook, Google, Twitter, Instagram,
Wikipedia and many others have long been banned.

The Chinese telecommunications giant Huawei has been the subject of some
of the greatest scrutiny, as American authorities push allies to ban the
company from building wireless networks over claims it aids the Chinese
state in cyberespionage. Huawei has denied the accusations.

Since the deadly border clash earlier this month,
\href{https://www.nytimes.com/2020/06/19/world/asia/india-china-border.html}{diplomats
have expected India to prevent Huawei} entry into the Indian market to
build a 5G network.

The U.S.
\href{https://www.nytimes.com/2019/11/01/technology/tiktok-national-security-review.html}{government
is investigating}TikTok, which is owned by the Chinese company
ByteDance, over national security concerns. In Europe, authorities are
probing the company's data-collection practices.

In India, a remaining question is how the government will go about
enforcing the ban announced on Monday. One option would be to pressure
operators of app stores, like Apple and Google, to no longer make the
services available for download. That could set off further retaliatory
actions by Chinese authorities.

Internet researchers have long
\href{https://www.nytimes.com/2017/09/17/technology/facebook-government-regulations.html}{warned}
that competing national interests could lead to a more fractured
internet, with people's access to certain information and services
limited by their governments.

Adam Satariano contributed reporting from London and Vindu Goel
contributed from San Francisco.

Advertisement

\protect\hyperlink{after-bottom}{Continue reading the main story}

\hypertarget{site-index}{%
\subsection{Site Index}\label{site-index}}

\hypertarget{site-information-navigation}{%
\subsection{Site Information
Navigation}\label{site-information-navigation}}

\begin{itemize}
\tightlist
\item
  \href{https://help.nytimes.com/hc/en-us/articles/115014792127-Copyright-notice}{©~2020~The
  New York Times Company}
\end{itemize}

\begin{itemize}
\tightlist
\item
  \href{https://www.nytco.com/}{NYTCo}
\item
  \href{https://help.nytimes.com/hc/en-us/articles/115015385887-Contact-Us}{Contact
  Us}
\item
  \href{https://www.nytco.com/careers/}{Work with us}
\item
  \href{https://nytmediakit.com/}{Advertise}
\item
  \href{http://www.tbrandstudio.com/}{T Brand Studio}
\item
  \href{https://www.nytimes.com/privacy/cookie-policy\#how-do-i-manage-trackers}{Your
  Ad Choices}
\item
  \href{https://www.nytimes.com/privacy}{Privacy}
\item
  \href{https://help.nytimes.com/hc/en-us/articles/115014893428-Terms-of-service}{Terms
  of Service}
\item
  \href{https://help.nytimes.com/hc/en-us/articles/115014893968-Terms-of-sale}{Terms
  of Sale}
\item
  \href{https://spiderbites.nytimes.com}{Site Map}
\item
  \href{https://help.nytimes.com/hc/en-us}{Help}
\item
  \href{https://www.nytimes.com/subscription?campaignId=37WXW}{Subscriptions}
\end{itemize}
