Sections

SEARCH

\protect\hyperlink{site-content}{Skip to
content}\protect\hyperlink{site-index}{Skip to site index}

\href{https://myaccount.nytimes.com/auth/login?response_type=cookie\&client_id=vi}{}

\href{https://www.nytimes.com/section/todayspaper}{Today's Paper}

\href{/section/opinion}{Opinion}\textbar{}Clean Up Your Act, Facebook,
or We're Leaving

\href{https://nyti.ms/2ZlSB2R}{https://nyti.ms/2ZlSB2R}

\begin{itemize}
\item
\item
\item
\item
\item
\item
\end{itemize}

Advertisement

\protect\hyperlink{after-top}{Continue reading the main story}

\href{/section/opinion}{Opinion}

Supported by

\protect\hyperlink{after-sponsor}{Continue reading the main story}

\hypertarget{clean-up-your-act-facebook-or-were-leaving}{%
\section{Clean Up Your Act, Facebook, or We're
Leaving}\label{clean-up-your-act-facebook-or-were-leaving}}

The social media company has taken steps toward reining in Trump. It's
too little, too late.

\includegraphics{https://static01.nyt.com/images/2018/08/02/opinion/02swisher/02swisher-thumbLarge.png}

By Kara Swisher

Ms. Swisher covers technology and is a contributing Opinion writer.

\begin{itemize}
\item
  June 30, 2020
\item
  \begin{itemize}
  \item
  \item
  \item
  \item
  \item
  \item
  \end{itemize}
\end{itemize}

\includegraphics{https://static01.nyt.com/images/2020/06/29/opinion/29Swisher/29Swisher-articleLarge.jpg?quality=75\&auto=webp\&disable=upscale}

``I put the dishes in the dishwasher,'' my son said to me recently, as
if it was a favor rather than something he should do \emph{just
because}.

This prompted me to write to you, Mark Zuckerberg, the
\href{https://www.nytimes.com/2020/06/30/us/politics/brad-parscale-trump.html}{Facebook}
chief executive, with all the irritation of a mother whose last nerves
were worked a long time ago when it comes to the abuses that thrive on
your platform. I'd like to let you know: You get zero claps for doing a
tiny right thing after doing the wrong thing for far too long.

Last week, you announced that you are
\href{https://www.nytimes.com/2020/06/26/technology/facebook-labels-advertisers.html}{finally
labeling} the most egregious dreck that is broadcast on Facebook by
\href{https://www.nytimes.com/2020/06/30/us/politics/brad-parscale-trump.html}{President
Trump}, after years of his escalating behavior. But it's too little, too
late.

And it's too calculating. You and other Facebook executives keep hauling
out the tired line, ``We know we have more work to do.'' It's irksome.
And you won't like me when I am irked.

You seem to be shifting toward labeling --- after
\href{https://www.nytimes.com/2020/06/02/technology/zuckerberg-defends-facebook-trump-posts.html}{insisting
recently} to your employees that you would not budge on this --- in
reaction to a campaign to persuade advertisers to boycott your company,
a movement known as \href{https://www.stophateforprofit.org/}{Stop the
Hate for Profit}. After years of other forms of pressure that apparently
failed, those who are seeking to force you to change are finally getting
traction by focusing on your wallet, knocking billions off your net
worth in recent days, as your stock price goes down.

This fast-moving campaign --- organized by the N.A.A.C.P., the ADL,
Color of Change and others --- is aimed directly at Facebook and has
been joined by companies like Patagonia, REI and, most significantly,
the consumer goods giant Unilever.

Still, other companies, like Starbucks and Coca-Cola, are not joining
Stop the Hate, but instead are declaring that they will cut off
marketing on \emph{all} of social media. As if all social media
companies are equal. They are not.

Allowing Facebook to get cover in a group will only end up hurting
smaller companies like Snap and Twitter, both of which have tried to
deal with this problem more actively. It's not fair to lump them in with
you; they have fewer resources to withstand a marketing drought. Since
Facebook and Google are the overwhelmingly dominant players in the game
of digital advertising, the problem of hate and misinformation flowing
on social media is yours to own.

Your stranglehold on the ad business is undeniably the source of your
power. I talked recently with some people running businesses that rely
on Facebook, all of whom are scared to speak out publicly against your
platform. Many compared your service to a bad relationship.

``I really cannot stand them at all,'' said a leader of a medium-size
company that gets a lot of its leads on Facebook. But while he worries
about the damage Facebook is doing to society, he added, ``I am going to
keep marketing there because I have no choice.''

No choice --- that's certainly why Starbucks did not get rid of its page
on Facebook, where it posts content to close to 36 million followers.
(Today, for example, you can ``start off your day with Cold Brew!'') I
don't blame Starbucks or Coca-Cola or anyone with a business to run for
not bailing on Facebook totally; all marketers need Facebook (and its
Instagram unit) to operate in today's media environment.

But I don't need you, since I am pretty sure that being on Facebook has
never helped me at all.

So, it is time to go. After years of inertia, and not much use of
Facebook, this week I finally took the first big steps toward leaving,
deactivating my personal page and unpublishing my brand page. This was a
many-click process in which my decision was questioned by Facebook's
pop-ups a lot more than I wanted my decision to be questioned (ARE YOU
SURE? ARE YOU \emph{REALLY} SURE?). I'm likely soon enough to delete the
pages altogether, along with my Instagram account, once I figure out
what to do with the material living there like boxes in a digital attic.

As I deactivated, I was asked by Facebook why I was doing it, and I
picked ``other'' from a long menu of reasons, many of which I would have
clicked if I could have chosen more than one, including: I have a
privacy concern; I don't feel safe on Facebook; I don't find Facebook
useful.

This column --- and the deactivation of my account --- is my way of
cleaning up my world. But to say I am confident that you, Mark
Zuckerberg, will do your part to clean up the rest of the world would be
something of an overstatement. Facebook's still high stock price and
your complete control over the company means you can and will continue
to do as you please.

And since you are not my kid --- yes, I know, lucky you! --- there is
little I can do about it. Yet I do hope for progress, however painful it
is for Facebook, its advertisers and the rest of us.

Already this week, other big platforms have started to make long overdue
changes in content policy, including banning an out-of-control Trump
community (Reddit) and temporarily suspending Mr. Trump's account due to
``hateful content'' (Twitch).

One of the hallmarks of adulthood is the ability to evolve. My son is
about to turn 16 (as Facebook did this year), and he's also learning how
to do that, which has put me in the mind of something James Baldwin
said: ``People pay for what they do, and still more for what they have
allowed themselves to become. They pay for it very simply; by the lives
they lead.''

By which I mean to say, Mark, you need to put your dirty dishes in the
dishwasher without my asking, just because.

\emph{The Times is committed to publishing}
\href{https://www.nytimes.com/2019/01/31/opinion/letters/letters-to-editor-new-york-times-women.html}{\emph{a
diversity of letters}} \emph{to the editor. We'd like to hear what you
think about this or any of our articles. Here are some}
\href{https://help.nytimes.com/hc/en-us/articles/115014925288-How-to-submit-a-letter-to-the-editor}{\emph{tips}}\emph{.
And here's our email:}
\href{mailto:letters@nytimes.com}{\emph{letters@nytimes.com}}\emph{.}

\emph{Follow The New York Times Opinion section on}
\href{https://www.facebook.com/nytopinion}{\emph{Facebook}}\emph{,}
\href{http://twitter.com/NYTOpinion}{\emph{Twitter (@NYTopinion)}}
\emph{and}
\href{https://www.instagram.com/nytopinion/}{\emph{Instagram}}\emph{,
and sign up for the}
\href{http://www.nytimes.com/newsletters/opiniontoday/}{\emph{Opinion
Today newsletter}}\emph{.}

Advertisement

\protect\hyperlink{after-bottom}{Continue reading the main story}

\hypertarget{site-index}{%
\subsection{Site Index}\label{site-index}}

\hypertarget{site-information-navigation}{%
\subsection{Site Information
Navigation}\label{site-information-navigation}}

\begin{itemize}
\tightlist
\item
  \href{https://help.nytimes.com/hc/en-us/articles/115014792127-Copyright-notice}{©~2020~The
  New York Times Company}
\end{itemize}

\begin{itemize}
\tightlist
\item
  \href{https://www.nytco.com/}{NYTCo}
\item
  \href{https://help.nytimes.com/hc/en-us/articles/115015385887-Contact-Us}{Contact
  Us}
\item
  \href{https://www.nytco.com/careers/}{Work with us}
\item
  \href{https://nytmediakit.com/}{Advertise}
\item
  \href{http://www.tbrandstudio.com/}{T Brand Studio}
\item
  \href{https://www.nytimes.com/privacy/cookie-policy\#how-do-i-manage-trackers}{Your
  Ad Choices}
\item
  \href{https://www.nytimes.com/privacy}{Privacy}
\item
  \href{https://help.nytimes.com/hc/en-us/articles/115014893428-Terms-of-service}{Terms
  of Service}
\item
  \href{https://help.nytimes.com/hc/en-us/articles/115014893968-Terms-of-sale}{Terms
  of Sale}
\item
  \href{https://spiderbites.nytimes.com}{Site Map}
\item
  \href{https://help.nytimes.com/hc/en-us}{Help}
\item
  \href{https://www.nytimes.com/subscription?campaignId=37WXW}{Subscriptions}
\end{itemize}
