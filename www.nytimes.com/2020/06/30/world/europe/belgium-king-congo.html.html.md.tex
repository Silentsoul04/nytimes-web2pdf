Sections

SEARCH

\protect\hyperlink{site-content}{Skip to
content}\protect\hyperlink{site-index}{Skip to site index}

\href{https://www.nytimes.com/section/world/europe}{Europe}

\href{https://myaccount.nytimes.com/auth/login?response_type=cookie\&client_id=vi}{}

\href{https://www.nytimes.com/section/todayspaper}{Today's Paper}

\href{/section/world/europe}{Europe}\textbar{}Belgium's King Sends
Letter of Regret Over Colonial Past in Congo

\url{https://nyti.ms/2AfPL6Q}

\begin{itemize}
\item
\item
\item
\item
\item
\end{itemize}

\href{https://www.nytimes.com/news-event/george-floyd-protests-minneapolis-new-york-los-angeles?action=click\&pgtype=Article\&state=default\&region=TOP_BANNER\&context=storylines_menu}{Race
and America}

\begin{itemize}
\tightlist
\item
  \href{https://www.nytimes.com/2020/07/26/us/protests-portland-seattle-trump.html?action=click\&pgtype=Article\&state=default\&region=TOP_BANNER\&context=storylines_menu}{Protesters
  Return to Other Cities}
\item
  \href{https://www.nytimes.com/2020/07/24/us/portland-oregon-protests-white-race.html?action=click\&pgtype=Article\&state=default\&region=TOP_BANNER\&context=storylines_menu}{Portland
  at the Center}
\item
  \href{https://www.nytimes.com/2020/07/23/podcasts/the-daily/portland-protests.html?action=click\&pgtype=Article\&state=default\&region=TOP_BANNER\&context=storylines_menu}{Podcast:
  Showdown in Portland}
\item
  \href{https://www.nytimes.com/interactive/2020/07/16/us/black-lives-matter-protests-louisville-breonna-taylor.html?action=click\&pgtype=Article\&state=default\&region=TOP_BANNER\&context=storylines_menu}{45
  Days in Louisville}
\end{itemize}

Advertisement

\protect\hyperlink{after-top}{Continue reading the main story}

Supported by

\protect\hyperlink{after-sponsor}{Continue reading the main story}

\hypertarget{belgiums-king-sends-letter-of-regret-over-colonial-past-in-congo}{%
\section{Belgium's King Sends Letter of Regret Over Colonial Past in
Congo}\label{belgiums-king-sends-letter-of-regret-over-colonial-past-in-congo}}

The first acknowledgment by the royal family of the European country's
brutal actions in central Africa stopped short of an outright apology.
It comes as Belgium begins to publicly reckon with the abuses.

\includegraphics{https://static01.nyt.com/images/2020/06/30/world/30belgium1/merlin_174064239_4ddf9f49-15ec-487d-8394-6813baf771c9-articleLarge.jpg?quality=75\&auto=webp\&disable=upscale}

By Monika Pronczuk and
\href{https://www.nytimes.com/by/megan-specia}{Megan Specia}

\begin{itemize}
\item
  June 30, 2020
\item
  \begin{itemize}
  \item
  \item
  \item
  \item
  \item
  \end{itemize}
\end{itemize}

BRUSSELS --- King Philippe of Belgium on Tuesday expressed his ``deepest
regrets'' for his country's brutal past in a letter to the president of
the Democratic Republic of Congo, the first public acknowledgment from a
member of the Belgian royal family of the devastating human and
financial toll during eight decades of colonization.

The king's letter, issued on the 60th anniversary of Congo's
independence, acknowledged the historical legacy and pointed out
continuing issues of racism and discrimination, though it stopped short
of the apology that some, including the United Nations, had asked for.

``I want to express my deepest regrets for the wounds of the past, the
pain of which is revived today by discriminations that are still too
present in our societies,'' the king wrote in the letter sent to
President Félix Tshisekedi of the Democratic Republic of Congo. The king
would, he added, ``continue to fight against all forms of racism.''

The letter, which was followed by a statement from Prime Minister Sophie
Wilmès of Belgium urging her country to ``look its past in the face,''
is part of the European nation's newfound willingness to address its
vicious colonial past.

\includegraphics{https://static01.nyt.com/images/2020/06/30/world/30belgium3/merlin_174062592_096378ce-5ce0-4e96-b0c7-57c28fb712fb-articleLarge.jpg?quality=75\&auto=webp\&disable=upscale}

In an address on Monday, Mr. Tshisekedi said that King Philippe had
planned to be at the Independence Day celebrations in Kinshasa, the
capital of the Democratic Republic of Congo, but that the coronavirus
pandemic had intervened.

Mr. Tshisekedi said he was trying to foster a strong relationship with
the European country. ``I consider it necessary that our common history
with Belgium and its people be told to our children in the Democratic
Republic of Congo as well as in Belgium on the basis of scientific work
carried out by historians of the two countries,'' he said.

``But the most important thing for the future is to build harmonious
relations with Belgium,'' he added, ``because beyond the stigmas of
history, the two peoples have been able to build a strong
relationship.''

Belgium has long grappled with its legacy in Africa, and protests in the
United States against the death of George Floyd at the hands of the
police have spurred a global conversation about racism that has given a
new intensity to the issue.

In addition to the remarks from the king and prime minister, statues of
King Leopold II, known for his violent personal rule of what was then
the Congo Free State, have been
\href{https://www.nytimes.com/2020/06/09/world/europe/king-leopold-statue-antwerp.html}{removed
from city squares and government buildings across Belgium}. On Tuesday,
the city of Ghent removed a bust of the former king from public display.

Leopold, an ancestor of King Philippe, extracted wealth from the
resource-rich territory in central Africa while inflicting immense harm
that led to the deaths of millions.

Jean-Luc Crucke, the finance minister for Wallonia, one of Belgium's
three regions, said on Tuesday that a parliamentary commission would
begin work in September to scrutinize the country's colonial past. The
panel would allow Belgium to ``continue this path'' laid out by the
king's letter, which he called ``heavy with meaning and more than
symbolic.''

Ms. Wilmès, speaking at a commemoration event in Brussels later in the
day, acknowledged the troubled history with the Democratic Republic of
Congo, ``a past imprinted with inequalities and violence against the
Congolese.''

Some activists said that the king's letter did not go far enough because
it did not contain an apology and, because he is not a member of the
government, it did not formally reflect the views of the Belgian state,
which took control of the vast land after King Leopold II and continued
colonial exploitation.

Jean Omasombo, a political scientist at the University of Kinshasa and a
researcher at the Africa Museum in Tervuren, Belgium, said that the
Belgian state had never recognized its responsibility for colonial
atrocities.

``This letter is a first step,'' Mr. Omasombo said. ``But it is not
sufficient.'' Mr. Omasombo added that he welcomed the idea of the
parliamentary commission but that it should not be ``a distraction''
from accountability.

Image

An event in Brussels on Tuesday to mark the 60th anniversary of Congo's
independence from Belgium.Credit...Pool photo by Stephanie Lecocq

Until 1908, Leopold ran the Congo Free State as a venture for personal
profit. With an army that included Congolese orphans, the king and his
agents drained the land of resources, and then forcibly moved, separated
and enslaved families, before being forced to turn control of the area
back over to the Belgian state. Congo achieved independence from Belgium
in 1960, but the following decades were
\href{https://www.nytimes.com/2005/07/03/magazine/the-congo-case.html}{scarred
by civil war}.

Almost 10,000 people demonstrated in Brussels against racism this month
in the wake of the killing of Mr. Floyd. Some protesters climbed on a
statue of King Leopold II and flew a giant flag of the Democratic
Republic of Congo, chanting ``murderer'' and ``reparations,'' repeating
a demand for the Belgian state to pay damages to the Democratic Republic
of Congo.

Belgium's grappling with its colonial heritage has long been fraught.
For decades, Belgians were taught that the country had brought
``civilization'' to the African continent, and some have defended King
Leopold II as a foundational figure. Streets and parks are named after
him, and statues of the king can be found throughout the country.

As in so many European nations, racial discrimination is an ongoing
issue in Belgium. Recently, a black member of the European Parliament
said she had been mistreated by the police in Brussels.

The member, Pierrette Herzberger-Fofana, a 71-year-old Green party
representative from Germany, filed a legal complaint this month against
Belgian officers who she said had pushed her against a wall and taken
away her purse and her mobile phone as she was trying to film what she
described as the police ``harassing'' young black men at a Brussels
train station.

According to Ms. Herzberger-Fofana, police officers did not believe her
when she said she was a member of the Parliament, despite her
identification and a diplomatic passport.

``I consider this as a racist and discriminatory act,'' she said in a
recent speech at the European Parliament. ``We can't ignore this police
violence.''

The police claim she insulted officers and have filed their own
complaint. The public prosecutor is investigating the episode.

Monika Pronczuk reported from Brussels, and Megan Specia from London.
Ruth Maclean contributed reporting from Dakar, Senegal.

Advertisement

\protect\hyperlink{after-bottom}{Continue reading the main story}

\hypertarget{site-index}{%
\subsection{Site Index}\label{site-index}}

\hypertarget{site-information-navigation}{%
\subsection{Site Information
Navigation}\label{site-information-navigation}}

\begin{itemize}
\tightlist
\item
  \href{https://help.nytimes.com/hc/en-us/articles/115014792127-Copyright-notice}{©~2020~The
  New York Times Company}
\end{itemize}

\begin{itemize}
\tightlist
\item
  \href{https://www.nytco.com/}{NYTCo}
\item
  \href{https://help.nytimes.com/hc/en-us/articles/115015385887-Contact-Us}{Contact
  Us}
\item
  \href{https://www.nytco.com/careers/}{Work with us}
\item
  \href{https://nytmediakit.com/}{Advertise}
\item
  \href{http://www.tbrandstudio.com/}{T Brand Studio}
\item
  \href{https://www.nytimes.com/privacy/cookie-policy\#how-do-i-manage-trackers}{Your
  Ad Choices}
\item
  \href{https://www.nytimes.com/privacy}{Privacy}
\item
  \href{https://help.nytimes.com/hc/en-us/articles/115014893428-Terms-of-service}{Terms
  of Service}
\item
  \href{https://help.nytimes.com/hc/en-us/articles/115014893968-Terms-of-sale}{Terms
  of Sale}
\item
  \href{https://spiderbites.nytimes.com}{Site Map}
\item
  \href{https://help.nytimes.com/hc/en-us}{Help}
\item
  \href{https://www.nytimes.com/subscription?campaignId=37WXW}{Subscriptions}
\end{itemize}
