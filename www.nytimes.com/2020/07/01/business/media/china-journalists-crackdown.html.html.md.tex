Sections

SEARCH

\protect\hyperlink{site-content}{Skip to
content}\protect\hyperlink{site-index}{Skip to site index}

\href{https://www.nytimes.com/section/business/media}{Media}

\href{https://myaccount.nytimes.com/auth/login?response_type=cookie\&client_id=vi}{}

\href{https://www.nytimes.com/section/todayspaper}{Today's Paper}

\href{/section/business/media}{Media}\textbar{}China Announces New
Retaliation Against U.S. News Outlets

\url{https://nyti.ms/2VC8j8N}

\begin{itemize}
\item
\item
\item
\item
\item
\end{itemize}

Advertisement

\protect\hyperlink{after-top}{Continue reading the main story}

Supported by

\protect\hyperlink{after-sponsor}{Continue reading the main story}

\hypertarget{china-announces-new-retaliation-against-us-news-outlets}{%
\section{China Announces New Retaliation Against U.S. News
Outlets}\label{china-announces-new-retaliation-against-us-news-outlets}}

Beijing demanded that four organizations submit information about their
operations in China, adding to tensions with the U.S. over the treatment
of journalists.

\includegraphics{https://static01.nyt.com/images/2020/07/01/world/01china-media-1/merlin_170660466_42a9f41a-a88a-4544-8313-012e1ec4e394-articleLarge.jpg?quality=75\&auto=webp\&disable=upscale}

\href{https://www.nytimes.com/by/raymond-zhong}{\includegraphics{https://static01.nyt.com/images/2018/10/15/multimedia/author-raymond-zhong/author-raymond-zhong-thumbLarge.png}}

By \href{https://www.nytimes.com/by/raymond-zhong}{Raymond Zhong}

\begin{itemize}
\item
  July 1, 2020
\item
  \begin{itemize}
  \item
  \item
  \item
  \item
  \item
  \end{itemize}
\end{itemize}

China demanded on Wednesday that four American news organizations
provide the government with information about their staffs, finances and
real estate holdings inside the country, in what the Ministry of Foreign
Affairs said was retaliation for the Trump administration's
\href{https://www.nytimes.com/2020/06/22/us/politics/us-china-news-organizations.html}{recent
actions against Chinese news outlets}in the United States.

The Chinese government stopped short, however, of announcing the
expulsions of journalists at any of the four American organizations: The
Associated Press, CBS News, National Public Radio and United Press
International.

The action is the latest in a series of tit-for-tat clashes over the
treatment of journalists, part of an intensifying
\href{https://www.nytimes.com/2020/07/23/world/asia/us-china-consulate.html}{rivalry
between the two powers}.

In March, China required five other American media organizations to
submit information about their operations. It also
\href{https://www.nytimes.com/2020/03/17/business/media/china-expels-american-journalists.html}{expelled
almost all of the American journalists} working for three of them: The
New York Times, The Wall Street Journal and The Washington Post.

The expulsions followed a decision by the Trump administration in
February to designate China's five pre-eminent state-run news
organizations as
\href{https://www.nytimes.com/2020/02/18/world/asia/china-media-trump.html}{foreign
government functionaries}, subject to rules similar to those that apply
to diplomatic missions. The administration in March also reduced the
number of Chinese state-media employees permitted to work in the United
States from 160 to 100.

Then, in June, the administration listed
\href{https://www.nytimes.com/2020/06/22/us/politics/us-china-news-organizations.html}{four
additional Chinese news agencies} as foreign missions.

Wednesday's move came as China began to enforce a new national security
law in Hong Kong that limits free expression in the semiautonomous
territory, raising doubts about reporters' ability to effectively cover
China from anywhere in the country.

Zhao Lijian, a spokesman for the Foreign Ministry, called Wednesday's
request for information a ``necessary countermeasure'' against last
month's American action, which he said constituted ``unreasonable
suppression'' of Chinese news outlets in the United States.

Representatives for the Associated Press and National Public Radio said
they were reviewing the Chinese authorities' request. CBS News and
United Press International did not immediately reply when approached for
comment.

American and other foreign correspondents in China say the working
environment there has worsened considerably in recent years. The police
have harassed journalists and
\href{https://www.nytimes.com/2020/04/16/business/china-coronavirus-censorship.html}{their
interview subjects}, and some reporters have received limited work
permits as punishment for coverage that is critical of the government.

But the latest cycle of escalation between Washington and Beijing
started in earnest when the State Department said in February that it
would begin
\href{https://www.nytimes.com/2020/02/18/world/asia/china-media-trump.html}{treating
five Chinese state-controlled news outlets} as foreign missions.

One day after the department announced its plans to reclassify the
Chinese state-media employees working in the United States as foreign
government workers, China said it was
\href{https://www.nytimes.com/2020/02/19/business/media/china-wall-street-journal.html}{expelling
three Wall Street Journal reporters}.

China said the move was in retaliation for a headline on
\href{https://www.wsj.com/articles/china-is-the-real-sick-man-of-asia-11580773677}{an
opinion article}, which the expelled reporters were not involved with.
The headline, ``China Is the Real Sick Man of Asia,'' used a term
freighted with associations to China's weakness in the late 19th and
early 20th centuries.

Claire Fu contributed research.

Advertisement

\protect\hyperlink{after-bottom}{Continue reading the main story}

\hypertarget{site-index}{%
\subsection{Site Index}\label{site-index}}

\hypertarget{site-information-navigation}{%
\subsection{Site Information
Navigation}\label{site-information-navigation}}

\begin{itemize}
\tightlist
\item
  \href{https://help.nytimes.com/hc/en-us/articles/115014792127-Copyright-notice}{©~2020~The
  New York Times Company}
\end{itemize}

\begin{itemize}
\tightlist
\item
  \href{https://www.nytco.com/}{NYTCo}
\item
  \href{https://help.nytimes.com/hc/en-us/articles/115015385887-Contact-Us}{Contact
  Us}
\item
  \href{https://www.nytco.com/careers/}{Work with us}
\item
  \href{https://nytmediakit.com/}{Advertise}
\item
  \href{http://www.tbrandstudio.com/}{T Brand Studio}
\item
  \href{https://www.nytimes.com/privacy/cookie-policy\#how-do-i-manage-trackers}{Your
  Ad Choices}
\item
  \href{https://www.nytimes.com/privacy}{Privacy}
\item
  \href{https://help.nytimes.com/hc/en-us/articles/115014893428-Terms-of-service}{Terms
  of Service}
\item
  \href{https://help.nytimes.com/hc/en-us/articles/115014893968-Terms-of-sale}{Terms
  of Sale}
\item
  \href{https://spiderbites.nytimes.com}{Site Map}
\item
  \href{https://help.nytimes.com/hc/en-us}{Help}
\item
  \href{https://www.nytimes.com/subscription?campaignId=37WXW}{Subscriptions}
\end{itemize}
