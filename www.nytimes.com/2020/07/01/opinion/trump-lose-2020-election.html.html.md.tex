Sections

SEARCH

\protect\hyperlink{site-content}{Skip to
content}\protect\hyperlink{site-index}{Skip to site index}

\href{https://myaccount.nytimes.com/auth/login?response_type=cookie\&client_id=vi}{}

\href{https://www.nytimes.com/section/todayspaper}{Today's Paper}

\href{/section/opinion}{Opinion}\textbar{}Is Trump Toast?

\href{https://nyti.ms/2ZnEES4}{https://nyti.ms/2ZnEES4}

\begin{itemize}
\item
\item
\item
\item
\item
\item
\end{itemize}

Advertisement

\protect\hyperlink{after-top}{Continue reading the main story}

\href{/section/opinion}{Opinion}

Supported by

\protect\hyperlink{after-sponsor}{Continue reading the main story}

\hypertarget{is-trump-toast}{%
\section{Is Trump Toast?}\label{is-trump-toast}}

There's a persuasive argument that the 2020 election is already over.

\href{https://www.nytimes.com/by/frank-bruni}{\includegraphics{https://static01.nyt.com/images/2018/04/03/opinion/frank-bruni/frank-bruni-thumbLarge.png}}

By \href{https://www.nytimes.com/by/frank-bruni}{Frank Bruni}

Opinion Columnist

\begin{itemize}
\item
  July 1, 2020
\item
  \begin{itemize}
  \item
  \item
  \item
  \item
  \item
  \item
  \end{itemize}
\end{itemize}

\includegraphics{https://static01.nyt.com/images/2020/07/01/opinion/01Bruni/01Bruni-articleLarge.jpg?quality=75\&auto=webp\&disable=upscale}

Only two of the past six presidents before
\href{https://www.nytimes.com/2020/07/01/us/politics/trump-fundraising-2020.html}{Donald
Trump} lost their bids for re-election. That's good news for him.

But their stories are bad news for him, too.

In their final years in office, both of those presidents, Jimmy Carter
and George H.W. Bush, experienced a noticeable slide in popularity right
around the time --- early May through late June --- that Trump hit his
current ugly patch.

According to
\href{https://news.gallup.com/interactives/185273/presidential-job-approval-center.aspx}{Gallup's
ongoing tracking} of the percentage of Americans who approve of a
president's job performance, Carter's and Bush's numbers sank below 40
percent during this period and pretty much stayed there through Election
Day. It's as if they both met their fates on the cusp of summer.

And the cusp of summer has been a mean season for Trump, who has never
flailed more pathetically or lashed out more desperately and who just
experienced the Carter-Bush dip.
\href{https://news.gallup.com/poll/203198/presidential-approval-ratings-donald-trump.aspx}{According
to Gallup}, his approval rating fell to 39 percent in early June from 49
a month earlier. So if Carter and Bush are harbingers, Trump is toast.

He's toast by other measures as well. Two much-discussed polls by The
Times and Siena College that were published last week suggested that
\href{https://www.nytimes.com/2020/06/25/upshot/poll-2020-biden-battlegrounds.html}{in
key swing states},
\href{https://www.nytimes.com/2020/06/24/us/politics/trump-biden-poll-nyt-upshot-siena-college.html}{as
well as nationally}, he's the limping dead, trailing Joe Biden by double
digits. That assessment is mostly consistent with other modeling and
projections since the economy turned on Trump.

According to some abstruse algorithm that The Economist regularly
updates, he has only
\href{https://projects.economist.com/us-2020-forecast/president}{a one
in 10 chance} of winning the Electoral College and thus the presidency.
According to
\href{https://fivethirtyeight.com/features/biden-has-a-historically-large-lead-over-trump-but-it-could-disappear/?cid=taboola_rcc_r}{a
historical averaging} of election-year polls by the website
FiveThirtyEight, Biden's lead over Trump right now is the biggest at
this stage of the contest since Bill Clinton's over Bob Dole in 1996,
when Clinton won his second term.

Trump's response? To set himself on fire.

His gratuitously touted instincts are nowhere to be found, supplanted by
self-defeating provocations, kamikaze tantrums and an itchy Twitter
finger. There's a culture war for him to exploit, but instead of simply
pillorying monument destroyers, he created his own living monuments: a
\href{https://www.nytimes.com/2020/06/28/us/politics/trump-white-power-video-racism.html}{white
supremacist astride a golf cart} in a Florida retirement community and
\href{https://www.nytimes.com/2020/06/29/us/politics/trump-white-couple-protesters.html}{a
pistol-toting Karen} shouting at peaceful Black protesters from the
stoop of her St. Louis manse. As a statement of values, it's grotesque.
As a re-election strategy, it's deranged.

``Trump is in a deep hole and his reaction is to keep digging,'' David
Axelrod told me. ``What he's doing is shrinking his vote to excite his
base.'' But that base is almost certainly not big enough to carry him to
victory.

Of course, November is still plenty distant. ``Nobody could have
predicted what these last four months would bring,'' Axelrod said. ``We
can't predict what the \emph{next} four months will bring.''

And Trump has at times seemed to live beyond the laws of political
gravity, untethered by precedents and unanswerable to pundits. For
instance, his approval rating since his inauguration has been
consistently --- and unusually --- low, lingering between 35 percent and
45 percent,
\href{https://news.gallup.com/poll/203207/trump-job-approval-weekly.aspx}{according
to Gallup.}

But his situation appears to be dire --- direr than Democrats allow
themselves to admit. They remember how they counted their chickens last
time around and got totally plucked.

``Every Democrat rightly has 2016 PTSD,'' Lis Smith, a communications
strategist who has advised Pete Buttigieg and Andrew Cuomo, told me.
``But right now? You can't imagine normal suburban people voting for
Trump anymore. He has really, really alienated everyone but the MAGA
true believers.''

Additionally, 2016 is a possibly irrelevant point of reference, for
reasons that become clearer all the time. I wouldn't be entirely shocked
if Biden stages a rout in November --- or at least as much of a rout as
this era of hyperpartisanship permits --- and the commentary afterward
casts Trump's reign not as some profound wake-up call but as a freak
accident made possible by a perfect storm of circumstances.

In fact that commentary has started. In The Washington Post last week,
Matt Bai
\href{https://www.washingtonpost.com/opinions/2020/06/22/2016-what-democrats-get-wrong-about-that-election/}{astutely
observed} that even as Trump won the presidency, most Americans rejected
the core tenets of his campaign and viewed him darkly. His margin of
victory ``came from reluctant voters who almost certainly thought they
were voting for the losing candidate, and who felt confident he'd make a
terrible president,'' Bai wrote.

``It was mostly about the intense emotions triggered by his opponent,''
he added, referring to Hillary Clinton. ``In the only national
referendum on Trumpism since 2016 --- the midterm cycle two years later
--- the president's party was resoundingly rejected.''

There are many ways in which the last presidential election doesn't
apply to this one, when Trump faces a much tougher challenge. In 2016,
an unusually high percentage of voters, especially in such pivotal
states as Pennsylvania, Michigan and Wisconsin, told pollsters that
they'd
\href{https://www.washingtonpost.com/news/the-fix/wp/2016/11/17/how-america-decided-at-the-very-last-moment-to-elect-donald-trump/}{decided
whom to vote for in the final week}. And these late deciders favored
Trump.

That could mean that many of them didn't have an entirely fixed opinion
of him. But just about every American does now. He has dominated the
media like none of his recent predecessors, with flamboyant behavior
that repels ambivalence.

His luck with late deciders in 2016 could also speak to Clinton
aversion. But there's no comparable Biden aversion. If many voters can't
bring themselves to adore him, they also can't bring themselves to abhor
him.

And Trump and his minions know it. That's why, instead of simply
portraying Biden as some lefty nightmare, they're
\href{https://www.nytimes.com/2020/05/17/opinion/trump-biden-age.html}{claiming}
that he's so mentally diminished that he'll be the puppet of progressive
extremists.

``Biden is just not scary enough for Trump,'' Axelrod said. ``He's
culturally inconvenient.''

And because of the coronavirus pandemic, Trump has less time and fewer
ways to change the dynamics of the presidential race than he would have
had in some other year.

The party conventions, for example, may have less impact than ever:
They're not rival shows but rival coronavirus narratives, with the
Democrats planning a
\href{https://www.wispolitics.com/2020/2020-dem-national-convention-moving-to-largely-virtual-format-smaller-venue/}{largely
virtual event}. Also,
\href{https://www.theatlantic.com/politics/archive/2020/04/voting-mail-2020-race-between-biden-and-trump/609799/}{more
Americans than usual} are certain to vote early, by mail, possibly
casting ballots even before the expected Trump-Biden debates.

``If somebody were asking me for advice on an October surprise, I'd tell
them to do it in September,'' Doug Sosnik, a longtime Democratic
strategist, told me.

Meantime we've had other surprises, all cutting \emph{against} Trump.
There was the early June surprise of
\href{https://www.nytimes.com/2020/06/01/us/politics/trump-st-johns-church-bible.html}{tear
gas being used} on peaceful protesters so that he could walk across
Lafayette Square for a photo op; the mid-June surprise of
\href{https://www.nytimes.com/2020/06/17/books/review-room-where-it-happened-john-bolton-memoir.html}{John
Bolton's book}; the late June surprise of The Times's scoop that Trump
\href{https://www.nytimes.com/2020/06/29/us/politics/russian-bounty-trump.html}{was
informed about Russian bounties} on American soldiers but didn't pay
attention or care.

The surprises will no doubt keep coming for an administration as steeped
in incompetence and corruption as Trump's. That's the other thing about
chickens: They come home to roost.

\emph{I invite you to sign up for my free}
\href{https://www.nytimes.com/newsletters/frank-bruni}{\emph{weekly
email newsletter}}\emph{. You can follow me on Twitter
(}\href{https://twitter.com/FrankBruni}{\emph{@FrankBruni}}\emph{).}

\emph{Listen to}
\href{https://www.nytimes.com/column/the-argument}{\emph{``The
Argument'' podcast}} \emph{every Thursday morning, with Ross Douthat,
Michelle Goldberg and me.}

Advertisement

\protect\hyperlink{after-bottom}{Continue reading the main story}

\hypertarget{site-index}{%
\subsection{Site Index}\label{site-index}}

\hypertarget{site-information-navigation}{%
\subsection{Site Information
Navigation}\label{site-information-navigation}}

\begin{itemize}
\tightlist
\item
  \href{https://help.nytimes.com/hc/en-us/articles/115014792127-Copyright-notice}{©~2020~The
  New York Times Company}
\end{itemize}

\begin{itemize}
\tightlist
\item
  \href{https://www.nytco.com/}{NYTCo}
\item
  \href{https://help.nytimes.com/hc/en-us/articles/115015385887-Contact-Us}{Contact
  Us}
\item
  \href{https://www.nytco.com/careers/}{Work with us}
\item
  \href{https://nytmediakit.com/}{Advertise}
\item
  \href{http://www.tbrandstudio.com/}{T Brand Studio}
\item
  \href{https://www.nytimes.com/privacy/cookie-policy\#how-do-i-manage-trackers}{Your
  Ad Choices}
\item
  \href{https://www.nytimes.com/privacy}{Privacy}
\item
  \href{https://help.nytimes.com/hc/en-us/articles/115014893428-Terms-of-service}{Terms
  of Service}
\item
  \href{https://help.nytimes.com/hc/en-us/articles/115014893968-Terms-of-sale}{Terms
  of Sale}
\item
  \href{https://spiderbites.nytimes.com}{Site Map}
\item
  \href{https://help.nytimes.com/hc/en-us}{Help}
\item
  \href{https://www.nytimes.com/subscription?campaignId=37WXW}{Subscriptions}
\end{itemize}
