Sections

SEARCH

\protect\hyperlink{site-content}{Skip to
content}\protect\hyperlink{site-index}{Skip to site index}

\href{https://myaccount.nytimes.com/auth/login?response_type=cookie\&client_id=vi}{}

\href{https://www.nytimes.com/section/todayspaper}{Today's Paper}

\href{/section/opinion}{Opinion}\textbar{}The World Builds a Wall to
Keep America Out

\href{https://nyti.ms/2Bs21ll}{https://nyti.ms/2Bs21ll}

\begin{itemize}
\item
\item
\item
\item
\item
\item
\end{itemize}

Advertisement

\protect\hyperlink{after-top}{Continue reading the main story}

\href{/section/opinion}{Opinion}

Supported by

\protect\hyperlink{after-sponsor}{Continue reading the main story}

\hypertarget{the-world-builds-a-wall-to-keep-america-out}{%
\section{The World Builds a Wall to Keep America
Out}\label{the-world-builds-a-wall-to-keep-america-out}}

America has no monopoly on success.

\href{https://www.nytimes.com/by/farhad-manjoo}{\includegraphics{https://static01.nyt.com/images/2019/01/08/opinion/farhad-manjoo-opinion/farhad-manjoo-opinion-thumbLarge.png}}

By \href{https://www.nytimes.com/by/farhad-manjoo}{Farhad Manjoo}

Opinion Columnist

\begin{itemize}
\item
  July 1, 2020
\item
  \begin{itemize}
  \item
  \item
  \item
  \item
  \item
  \item
  \end{itemize}
\end{itemize}

\includegraphics{https://static01.nyt.com/images/2020/07/02/opinion/02manjoo1/merlin_173849757_1ff802dc-ae38-43af-bf96-1f9e58fa81e9-articleLarge.jpg?quality=75\&auto=webp\&disable=upscale}

\href{https://www.nytimes.com/es/2020/07/03/espanol/opinion/prohibicion-viajar-europa.html}{Leer
en español}

\hypertarget{listen-to-this-opinion-column}{%
\subsubsection{Listen to This Opinion
Column}\label{listen-to-this-opinion-column}}

Audio Recording by Audm

\emph{To hear more audio stories from publishers like The New York
Times, download}
\href{https://www.audm.com/?utm_source=nytopinion\&utm_medium=embed\&utm_campaign=world_america_out}{\emph{Audm
for iPhone or Android}}\emph{.}

You might call it poetic, if it weren't so painful. Donald Trump won the
White House largely on a campaign of shutting America's borders to
pretty much everyone other than people of European descent. ``Why are we
having all these people from shithole countries come here?''
\href{https://www.washingtonpost.com/politics/trump-attacks-protections-for-immigrants-from-shithole-countries-in-oval-office-meeting/2018/01/11/bfc0725c-f711-11e7-91af-31ac729add94_story.html}{he
once asked}, about Haitians, Salvadorans and Africans. ``We should have
more people from places like Norway.''

So what should one conclude about America's own proximity to Trump's
global latrine now that ``places like Norway'' have decided to keep
\emph{their} borders indefinitely closed to us?

Among
\href{https://www.consilium.europa.eu/en/press/press-releases/2020/06/30/council-agrees-to-start-lifting-travel-restrictions-for-residents-of-some-third-countries/}{the
list of nations} to which Norway and the rest of Europe will soon reopen
for travel are three from the continent that Trump flushed down the
toilet:
\href{https://www.nytimes.com/2020/06/30/world/europe/eu-reopening-blocks-us-travelers.html}{Algeria,
Morocco and Rwanda}. Canada is also on the list. So is China, assuming
it reciprocates.

But Trump's America is not, because we are nowhere close to meeting
Europe's criteria for reducing the spread of the coronavirus. How
successfully a society can fight a pandemic is as objective a measure of
national capacity, not to mention ``greatness,'' as one is likely to
find --- and on this, like so much else these days, America ranks near
the bottom.

I have lived in the United States for more than 30 years, and I can't
think of any national failure as naked and complete as this one. When I
look at the graphs showing American infections soaring while the virus
abates
\href{https://www.nytimes.com/2020/06/29/briefing/coronavirus-mississippi-new-england-patriots-your-monday-briefing.html}{in
nearly every other affluent country}, I feel the sting of defeat, misery
and embarrassment.

As an immigrant from South Africa, I find it hard to resist seeing
Europe's travel dis as the ultimate comeuppance of Trump's xenophobia.
Like a lot of Americans, I sometimes find myself assuming
\href{https://theweek.com/articles/654508/what-exactly-american-exceptionalism}{American
exceptionalism} --- the idea that America's founding ideals make us
morally superior to ``ordinary'' nations and confer on us special
credibility and insight when dealing with global crises.

But America's pandemic failure demolishes the notion that our country is
better off without people and ideas from beyond our borders. The last
few months should stick a fork in the absurd proposition that the United
States enjoys some kind of monopoly on brilliance. Clearly, we do not.
Rather than close ourselves off from the planet, we should be inviting
others to join the urgent project of rebuilding America.

I bang this drum often. As I've argued before, I am in favor of
\href{https://www.nytimes.com/2019/01/16/opinion/open-borders-immigration.html}{throwing
America's borders wide open} to much of the world. My primary reasons
are moral --- I don't think a country founded on the idea that everyone
is equal should seal itself off to the ambitious billions who live
beyond our shores.

There are also powerful economic and strategic arguments for openness;
American exceptionalism is impossible without immigration.
\href{http://paulgraham.com/95.html}{The only way} that a country with
less than 5 percent of the world's population can maintain the long-term
economic and cultural superiority to which many Americans feel entitled
is to collectively produce much more than 5 percent of the world's best
ideas.

The only way to do that is to invite in the other 95 percent. I spent
much of my career covering Silicon Valley. Some of the most innovative
companies in the world --- from Google to Intel to Instagram to Stripe
--- were founded by immigrants, and many in the industry say the whole
place
\href{https://www.nytimes.com/2017/02/08/technology/personaltech/why-silicon-valley-wouldnt-work-without-immigrants.html}{would
not work without immigration}.

I am not one of those lefties who believe that Trump bears all of the
blame for our flawed response to the virus. The breakdown here was so
total that it
\href{https://www.theatlantic.com/magazine/archive/2020/06/underlying-conditions/610261/}{lays
bare larger and more persistent ailments}: our creaking health care
system, the ruthlessness of our economy, our Swiss-cheese safety net,
and political polarization that poisons effective action but excels at
whipping up nonsensical culture wars.

The totality of our failure is precisely why we should look to the
outside for success --- yet Trump has used the virus as an excuse to
\href{https://www.nytimes.com/2020/06/12/us/politics/coronavirus-trump-immigration-policies.html}{accelerate
his restrictions on immigration}.

Last week, Trump
\href{https://www.nytimes.com/2020/06/22/us/politics/trump-h1b-work-visas.html}{suspended
the issuance of work visas} for hundreds of thousands of foreigners,
from tech workers to seasonal workers in the hospitality industry to au
pairs and students.

Another group the restriction affects is doctors.
\href{https://www.nbcnews.com/news/asian-america/fear-deportation-heightened-immigrant-doctors-h-1b-visas-amid-pandemic-n1204791}{About
127,000 doctors}, nearly a quarter of the physicians in the United
States, are immigrants. Many of them are now
\href{https://www.motherjones.com/coronavirus-updates/2020/06/immigrant-h1b-doctors-coronavirus-green-card/}{caring
for coronavirus patients} in communities without enough health care
professionals. All the while, immigrant doctors have had to worry not
only that they might die of the virus while taking care of Americans,
but also that if they do, their families could be deported.

This is madness. More than that: If we keep shutting foreigners out,
what justifies our arrogant assumption that the world's best and
brightest will keep wanting to come here?

Consider, for instance, Rwanda, one of the countries that did make
Europe's list. In 1994, it suffered a genocide in which the United
States and the United Nations infamously refused to intervene. Almost a
million people were killed. In the 26 years since, Rwanda has
\href{https://www.nytimes.com/2019/04/06/world/africa/rwanda-genocide-25-years.html}{rebuilt
itself}, and now it boasts one of the most
\href{https://www.atlanticcouncil.org/blogs/africasource/rwandas-successes-and-challenges-in-response-to-covid-19/}{capable
medical systems in Africa}. Rwanda's 13 million people have nearly
universal health care coverage; the country uses drones to carry blood
and other supplies to far-flung hospitals.

And when the coronavirus came,
\href{https://www.newyorker.com/news/news-desk/what-african-nations-are-teaching-the-west-about-fighting-the-coronavirus}{Rwanda}
set up contact tracing to quickly halt the spread of the virus, making
it
\href{https://www.newyorker.com/news/news-desk/what-african-nations-are-teaching-the-west-about-fighting-the-coronavirus}{one
of several African countries} to squash it. To date, only two Rwandans
are known to have died of Covid-19.

I truly hope that Rwandans and others witnessing America's dysfunction
are not tempted to celebrate our fall. The United States' coronavirus
failure is a loss for the world, which has long depended on American
leadership to combat global crises.

The lesson here is obvious: We are all in this together. It's time to
stop pretending that America, and Americans, have all the answers. We
need all the help we can get.

\hypertarget{office-hours-with-farhad-manjoo}{%
\subsection{Office Hours With Farhad
Manjoo}\label{office-hours-with-farhad-manjoo}}

\emph{Farhad wants to}
\href{https://www.nytimes.com/2019/05/16/opinion/farhad-office-hours.html?module=inline}{\emph{chat
with readers on the phone}}\emph{. If you're interested in talking to a
New York Times columnist about anything that's on your mind, please fill
out this form. Farhad will select a few readers to call.}

\emph{The Times is committed to publishing}
\href{https://www.nytimes.com/2019/01/31/opinion/letters/letters-to-editor-new-york-times-women.html}{\emph{a
diversity of letters}} \emph{to the editor. We'd like to hear what you
think about this or any of our articles. Here are some}
\href{https://help.nytimes.com/hc/en-us/articles/115014925288-How-to-submit-a-letter-to-the-editor}{\emph{tips}}\emph{.
And here's our email:}
\href{mailto:letters@nytimes.com}{\emph{letters@nytimes.com}}\emph{.}

\emph{Follow The New York Times Opinion section on}
\href{https://www.facebook.com/nytopinion}{\emph{Facebook}}\emph{,}
\href{http://twitter.com/NYTOpinion}{\emph{Twitter (@NYTopinion)}}
\emph{and}
\href{https://www.instagram.com/nytopinion/}{\emph{Instagram}}\emph{.}

Advertisement

\protect\hyperlink{after-bottom}{Continue reading the main story}

\hypertarget{site-index}{%
\subsection{Site Index}\label{site-index}}

\hypertarget{site-information-navigation}{%
\subsection{Site Information
Navigation}\label{site-information-navigation}}

\begin{itemize}
\tightlist
\item
  \href{https://help.nytimes.com/hc/en-us/articles/115014792127-Copyright-notice}{©~2020~The
  New York Times Company}
\end{itemize}

\begin{itemize}
\tightlist
\item
  \href{https://www.nytco.com/}{NYTCo}
\item
  \href{https://help.nytimes.com/hc/en-us/articles/115015385887-Contact-Us}{Contact
  Us}
\item
  \href{https://www.nytco.com/careers/}{Work with us}
\item
  \href{https://nytmediakit.com/}{Advertise}
\item
  \href{http://www.tbrandstudio.com/}{T Brand Studio}
\item
  \href{https://www.nytimes.com/privacy/cookie-policy\#how-do-i-manage-trackers}{Your
  Ad Choices}
\item
  \href{https://www.nytimes.com/privacy}{Privacy}
\item
  \href{https://help.nytimes.com/hc/en-us/articles/115014893428-Terms-of-service}{Terms
  of Service}
\item
  \href{https://help.nytimes.com/hc/en-us/articles/115014893968-Terms-of-sale}{Terms
  of Sale}
\item
  \href{https://spiderbites.nytimes.com}{Site Map}
\item
  \href{https://help.nytimes.com/hc/en-us}{Help}
\item
  \href{https://www.nytimes.com/subscription?campaignId=37WXW}{Subscriptions}
\end{itemize}
