Sections

SEARCH

\protect\hyperlink{site-content}{Skip to
content}\protect\hyperlink{site-index}{Skip to site index}

\href{https://www.nytimes.com/section/politics}{Politics}

\href{https://myaccount.nytimes.com/auth/login?response_type=cookie\&client_id=vi}{}

\href{https://www.nytimes.com/section/todayspaper}{Today's Paper}

\href{/section/politics}{Politics}\textbar{}Trump Attacks a Suburban
Housing Program. Critics See a Play for White Votes.

\url{https://nyti.ms/3dLRChu}

\begin{itemize}
\item
\item
\item
\item
\item
\item
\end{itemize}

Advertisement

\protect\hyperlink{after-top}{Continue reading the main story}

Supported by

\protect\hyperlink{after-sponsor}{Continue reading the main story}

\hypertarget{trump-attacks-a-suburban-housing-program-critics-see-a-play-for-white-votes}{%
\section{Trump Attacks a Suburban Housing Program. Critics See a Play
for White
Votes.}\label{trump-attacks-a-suburban-housing-program-critics-see-a-play-for-white-votes}}

Proponents of the policy saw the move as an attempt to shore up the
president's sagging support among white suburban voters by stoking
racial division.

\includegraphics{https://static01.nyt.com/images/2020/07/01/us/politics/01dc-housingtweet/merlin_162748548_d2723ab6-d854-4ce6-8d74-abae828a5d89-articleLarge.jpg?quality=75\&auto=webp\&disable=upscale}

By \href{https://www.nytimes.com/by/glenn-thrush}{Glenn Thrush}

\begin{itemize}
\item
  Published July 1, 2020Updated July 10, 2020
\item
  \begin{itemize}
  \item
  \item
  \item
  \item
  \item
  \item
  \end{itemize}
\end{itemize}

WASHINGTON --- President Trump has taken aim at an Obama-era program
intended to eliminate racial housing disparities in the
\href{https://www.nytimes.com/2020/07/10/us/politics/trump-white-voters-in-suburbs.html}{suburbs},
a move proponents of the policy see as an attempt to shore up his
sagging support among white suburban voters by stoking racial division.

In a Twitter post late Tuesday, Mr. Trump announced that he was
considering the elimination of a 2015 initiative known as
\href{https://www.hudexchange.info/programs/affh/}{Affirmatively
Furthering Fair Housing}, which requires localities to identify and
address patterns of racial segregation outlawed under the Fair Housing
Act of 1968 by creating detailed corrective plans.

``At the request of many great Americans who live in the
\href{https://www.nytimes.com/2020/07/10/us/politics/trump-white-voters-in-suburbs.html}{Suburbs},
and others, I am studying the AFFH housing regulation that is having a
devastating impact on these once thriving Suburban areas,''
\href{https://twitter.com/realDonaldTrump/status/1278136326647406593}{he
wrote}, adding, ``Not fair to homeowners, I may END!''

The tweet came just a day after Mr. Trump
\href{https://www.nytimes.com/2020/06/29/us/politics/trump-white-couple-protesters.html}{posted
a video} showing a pair of angry white homeowners pointing an assault
rifle and pistol in the direction of chanting protesters walking past
their house in St. Louis --- part of a barrage of footage on social
media capturing white homeowners confronting Black bystanders, often
their neighbors, for simply standing on their streets.

Officials at the Department of Housing and Urban Development were
puzzled by the timing of Mr. Trump's tweet; the housing rule has been in
limbo since Mr. Obama left office, tied up by litigation, lengthy
rule-making and exemption requests by local officials.

In January, HUD
\href{https://www.federalregister.gov/documents/2020/01/14/2020-00234/affirmatively-furthering-fair-housing}{posted
a notice} saying it was considering weakening the original regulation by
factoring in ``the unique needs and difficulties faced by individual
jurisdictions'' in complying with a 92-item questionnaire that must be
completed to obtain funding from the department.

Mr. Trump's claim that he might end the housing initiative was a
reference to the January proposal, not the announcement of a new policy,
a White House spokesman said.

The process may not have changed, but there has been a significant shift
in political sentiment prompted by the president's response to the
coronavirus pandemic.

Mr. Trump and his campaign team, already concerned about his weakness in
battleground states, have become increasingly alarmed by internal
polling showing a softening of support among suburban voters, especially
women without college degrees, according to two Republican officials
close to the campaign.

They said Mr. Trump was also seeking to highlight a set of proposals,
put forward by the campaign of his presumptive Democratic rival, Joseph
R. Biden Jr., to reassess zoning laws relating to single-family homes, a
move some critics argue could turn suburbs into cities. ``Corrupt Joe
Biden wants to make them MUCH WORSE,''
\href{https://twitter.com/realDonaldTrump/status/1278136326647406593}{Mr.
Trump tweeted}.

Hours after Mr. Trump tweeted, his son Donald Trump Jr.,
\href{https://twitter.com/DonaldJTrumpJr/status/1278313243996377095}{posted
a link} to a National Review article titled,
\href{https://www.nationalreview.com/corner/biden-and-dems-are-set-to-abolish-the-suburbs/}{``Joe
Biden and Democrats Are Set to Abolish the Suburbs.''}

Shaun Donovan, who as secretary of Housing and Urban Development created
the Affirmatively Furthering Fair Housing policy after pursuing fair
housing actions against Yonkers, N.Y., and other localities, said the
messages had little to do with housing policy, but were part of a
campaign to stoke long-simmering racial animosities for political
purposes.

``Trump's tweet is racist and wrong,'' said Mr. Donovan, who established
the rule after months of consultation with civil rights groups. ``He
would turn back the clock to the days when the federal government
perpetuated the lie that Black families' moving to suburban
neighborhoods brings down property values.''

The core of the fair housing policy was ``a recognition that outlawing
intentional discrimination is not enough for people of color to overcome
the consequences of centuries of oppression,'' added Mr. Donovan, a
former New York City housing commissioner who
\href{https://www.nytimes.com/2020/02/03/nyregion/shaun-donovan-mayor-nyc.html}{is
now running for mayor}.

Whatever the motive, the tweet was a pointed reminder of just how firmly
Mr. Trump's slashing style is rooted in the racially polarized conflicts
of his early days in Queens.

The president, fighting for survival at 74, is choosing to make a
political stand on a similar racial and political battlefield he first
stormed in 1973, when, at 27, he
\href{https://www.nytimes.com/2016/08/28/us/politics/donald-trump-housing-race.html}{vehemently
fought a federal fair housing lawsuit} accusing his father Fred Trump's
rental developments in boroughs outside Manhattan of discriminating
against Black applicants.

Diane Yentel, the president of the National Low Income Housing
Coalition, an advocacy group based in Washington, questioned the timing
of the tweet --- especially since Mr. Trump seldom weighs in on such
specific policy matters.

``It's especially abhorrent for the president to threaten further
entrenchment of segregated communities now, during a time of reckoning
on racial injustices in our country,'' Ms. Yentel said. ``A direct line
connects America's history of racist housing policies to today's
overpolicing of Black and brown communities.''

It is unclear how the pandemic, economic swoon and local moratoriums on
rent payments will affect the proposed rule changes. Even before the
current crisis,
\href{https://www.nytimes.com/2019/12/20/us/politics/homelessness-trump-california.html}{homelessness
rates were on the rise,} especially on the West Coast. At the same time,
Black homeownership rates
\href{https://www.urban.org/urban-wire/five-point-strategy-reducing-black-homeownership-gap}{have
dropped} to levels not seen since the 1960s.

The Affirmatively Furthering Fair Housing policy was meant to replace
oversight of federal spending on housing that was widely seen as
ineffective, especially when it came to discrimination based on race,
disability, gender, age and sexual orientation.

The new rules, in theory, created stricter benchmarks for communities
receiving federal funding, but compliance proved difficult, and HUD was
still working on a tool kit that would have made it easier for
localities to file the necessary reports when Mr. Trump was elected, Mr.
Donovan said.

Opponents, including some local officials, viewed the new system as
onerous --- and found a receptive audience for their complaints when Ben
Carson, a brain surgeon with no housing experience, was confirmed as Mr.
Trump's housing secretary in 2017.

By 2018, Mr. Carson, a free-market conservative and the only Black
person in Mr. Trump's cabinet,
delayed\href{https://www.nytimes.com/2018/01/04/upshot/trump-delays-hud-fair-housing-obama-rule.html}{enactment
of the regulation} and signaled his intention to eliminate it
altogether,
\href{https://www.nytimes.com/2018/03/28/us/ben-carson-hud-fair-housing-discrimination.html}{part
of a larger strategy} of slow-walking fair housing investigations and
marginalizing department officials who aggressively pursued cases.

Mr. Carson's moves were part of a larger push to reduce housing
regulation led by the Heritage Foundation and other conservative think
tanks that worked closely with department officials and the White House
Domestic Policy Council.

Mr. Carson has also tried to roll back the Obama administration's
attempt to more closely monitor the use of computer algorithms and other
methods that have historically been used to exclude minority applicants
from receiving housing loans.

In announcing the proposed rule changes, Mr. Carson
\href{https://www.nytimes.com/2020/01/03/us/politics/trump-housing-segregation.html}{claimed
the Obama-era initiative} was ``actually suffocating investment in some
of our most distressed neighborhoods that need our investment the
most.''

Advertisement

\protect\hyperlink{after-bottom}{Continue reading the main story}

\hypertarget{site-index}{%
\subsection{Site Index}\label{site-index}}

\hypertarget{site-information-navigation}{%
\subsection{Site Information
Navigation}\label{site-information-navigation}}

\begin{itemize}
\tightlist
\item
  \href{https://help.nytimes.com/hc/en-us/articles/115014792127-Copyright-notice}{©~2020~The
  New York Times Company}
\end{itemize}

\begin{itemize}
\tightlist
\item
  \href{https://www.nytco.com/}{NYTCo}
\item
  \href{https://help.nytimes.com/hc/en-us/articles/115015385887-Contact-Us}{Contact
  Us}
\item
  \href{https://www.nytco.com/careers/}{Work with us}
\item
  \href{https://nytmediakit.com/}{Advertise}
\item
  \href{http://www.tbrandstudio.com/}{T Brand Studio}
\item
  \href{https://www.nytimes.com/privacy/cookie-policy\#how-do-i-manage-trackers}{Your
  Ad Choices}
\item
  \href{https://www.nytimes.com/privacy}{Privacy}
\item
  \href{https://help.nytimes.com/hc/en-us/articles/115014893428-Terms-of-service}{Terms
  of Service}
\item
  \href{https://help.nytimes.com/hc/en-us/articles/115014893968-Terms-of-sale}{Terms
  of Sale}
\item
  \href{https://spiderbites.nytimes.com}{Site Map}
\item
  \href{https://help.nytimes.com/hc/en-us}{Help}
\item
  \href{https://www.nytimes.com/subscription?campaignId=37WXW}{Subscriptions}
\end{itemize}
