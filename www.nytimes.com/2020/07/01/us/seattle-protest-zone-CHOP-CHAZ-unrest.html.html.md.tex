Sections

SEARCH

\protect\hyperlink{site-content}{Skip to
content}\protect\hyperlink{site-index}{Skip to site index}

\href{https://www.nytimes.com/section/us}{U.S.}

\href{https://myaccount.nytimes.com/auth/login?response_type=cookie\&client_id=vi}{}

\href{https://www.nytimes.com/section/todayspaper}{Today's Paper}

\href{/section/us}{U.S.}\textbar{}Police Clear Seattle's Protest
`Autonomous Zone'

\url{https://nyti.ms/2CQbSBJ}

\begin{itemize}
\item
\item
\item
\item
\item
\item
\end{itemize}

\href{https://www.nytimes.com/news-event/george-floyd-protests-minneapolis-new-york-los-angeles?action=click\&pgtype=Article\&state=default\&region=TOP_BANNER\&context=storylines_menu}{Race
and America}

\begin{itemize}
\tightlist
\item
  \href{https://www.nytimes.com/2020/07/26/us/protests-portland-seattle-trump.html?action=click\&pgtype=Article\&state=default\&region=TOP_BANNER\&context=storylines_menu}{Protesters
  Return to Other Cities}
\item
  \href{https://www.nytimes.com/2020/07/24/us/portland-oregon-protests-white-race.html?action=click\&pgtype=Article\&state=default\&region=TOP_BANNER\&context=storylines_menu}{Portland
  at the Center}
\item
  \href{https://www.nytimes.com/2020/07/23/podcasts/the-daily/portland-protests.html?action=click\&pgtype=Article\&state=default\&region=TOP_BANNER\&context=storylines_menu}{Podcast:
  Showdown in Portland}
\item
  \href{https://www.nytimes.com/interactive/2020/07/16/us/black-lives-matter-protests-louisville-breonna-taylor.html?action=click\&pgtype=Article\&state=default\&region=TOP_BANNER\&context=storylines_menu}{45
  Days in Louisville}
\end{itemize}

Advertisement

\protect\hyperlink{after-top}{Continue reading the main story}

Supported by

\protect\hyperlink{after-sponsor}{Continue reading the main story}

\hypertarget{police-clear-seattles-protest-autonomous-zone}{%
\section{Police Clear Seattle's Protest `Autonomous
Zone'}\label{police-clear-seattles-protest-autonomous-zone}}

The so-called Capitol Hill Organized Protest area was taken over by
protesters after the death of George Floyd in Minneapolis. It was the
site of at least four shootings last month.

\includegraphics{https://static01.nyt.com/images/2020/07/01/us/01unrest-seattle01sub/01unrest-seattle01sub-videoSixteenByNine3000.jpg}

\href{https://www.nytimes.com/by/rachel-abrams}{\includegraphics{https://static01.nyt.com/images/2018/02/16/multimedia/author-rachel-abrams/author-rachel-abrams-thumbLarge.jpg}}

By \href{https://www.nytimes.com/by/rachel-abrams}{Rachel Abrams}

\begin{itemize}
\item
  Published July 1, 2020Updated July 23, 2020
\item
  \begin{itemize}
  \item
  \item
  \item
  \item
  \item
  \item
  \end{itemize}
\end{itemize}

SEATTLE --- For weeks, officials in
\href{https://www.nytimes.com/2020/07/23/us/seattle-protests-feds.html}{Seattle}
have been grappling with what to do about a protest zone set up near
downtown that became the scene of several shootings and a war of words
between city leaders and President Trump: Support it as an exercise in
populist democracy? Or shut it down?

On Wednesday, a resolution took the form of squads of riot police and
several pieces of heavy machinery. The police swept through the
so-called Capitol Hill Organized
\href{https://www.nytimes.com/2020/07/23/us/seattle-protests-feds.html}{Protest}
area in the early-morning hours with little opposition, pulling aside
barricades, arresting protesters and retaking the police station they
had abandoned several weeks earlier.

``Our job is to support peaceful demonstrations,'' Carmen Best, the
city's police chief, said as police officers re-entered the East
Precinct station and set up formidable lines outside. ``What has
happened here on these streets over the last two weeks --- few weeks,
that is --- is lawless, and it's brutal, and bottom line, it is simply
unacceptable,'' she said.

At a news conference on Wednesday, Mayor Jenny Durkan said she was
urging the police to avoid criminal charges against anyone arrested in
the zone for failure to disperse or other misdemeanors.

But she said the city had been forced to act because of the repeated
episodes of violence.

Seattle's largely progressive leadership had sought mightily to find
common ground with the protesters who were demanding an end to the
disparate and sometimes violent treatment of African-Americans by the
police, in part because of the city's own recent history. The city
committed to sweeping police reforms after the Department of Justice
accused it of biased policing and excessive force in 2012. Ms. Best, its
first Black female police chief, was appointed in 2018.

But the encampment known as the Capitol Hill Organized Protest began
drawing homeless people from elsewhere in the city who showed no
inclination to leave anytime soon, and the happy communal vibe during
the day was often turning darker at night.

The outbreak of violence over the past week left many in the
neighborhood --- an area of artists and students and also some of the
city's grandest old homes --- demanding an end to the chaos.

\includegraphics{https://static01.nyt.com/images/2020/07/01/us/01unrest-seattle01alt/merlin_174113700_3936f8ac-cc76-4793-9853-fb347e3c8ab0-articleLarge.jpg?quality=75\&auto=webp\&disable=upscale}

``The deteriorating conditions and repeated gun violence required us to
immediately address public safety concerns,'' Ms. Durkan said. ``It was
clear that many individuals would not leave, and that the impacts to the
community could not be reduced, and public safety could not be improved,
until they did leave.''

Chief Best said it even more succinctly earlier in the day: ``Enough is
enough.''

The move to clear the area came as protests against police brutality
around the country, sparked by the death of George Floyd in police
custody in Minneapolis in May, have begun to wane. Many cities,
including Seattle, have committed to new police reforms after Mr.
Floyd's death, and some of the officers involved in the most recent
shooting deaths of Black people, including Mr. Floyd, have been fired
and charged with crimes.

There are continuing moves to
\href{https://www.nytimes.com/2020/06/05/us/defund-police-floyd-protests.html}{redefine
the mission} of police departments around the country, and to arrest
other officers involved in deadly shootings of Black people, including
\href{https://www.nytimes.com/article/breonna-taylor-police.html}{Breonna
Taylor}, a 26-year-old emergency medical technician who was shot by
Louisville, Ky., police officers at her home.

But millions of Americans are out of work because of the coronavirus,
and some of those whose grievances go well beyond the latest cases of
police violence have remained in the streets to demand further change.
Perhaps taking a cue from Seattle, demonstrators in Portland,
\href{https://itsgoingdown.org/hahnemann-hospital-report/}{Philadelphia},
\href{https://www.nytimes.com/2020/06/23/style/statue-richmond-lee.html}{Richmond,
Va}., and elsewhere have tried to set up protest sites of their own.

In New York, what started as
\href{https://www.nytimes.com/2020/06/28/nyregion/occupy-city-hall-nyc.html}{a
small group of protesters} and a few square feet morphed into a group
that took over the large part of a park, and attracted extensive
attention on social media. Police
\href{https://www.nydailynews.com/news/politics/ny-occupy-city-hall-nypd-clash-20200701-wxq7mxffqrgt5mqxodvqpqrgka-story.html}{tried
to clear} the area on Wednesday morning, removing barricades and making
several arrests.

The Seattle protest zone included tents, a ``Decolonization Conversation
Cafe'' and even a medic station over six blocks, establishing what
protesters called a ``no-cop'' zone after the police agreed to board up
their precinct station and withdraw outside the barricades. Part street
fair, part commune, the so-called CHOP became an experiment in
maintaining order with no police in sight.

But leadership in the zone was unclear, and community organizers said it
was hard to figure out who was in charge. Some activists called for a
few specific demands including defunding the police department, while
others focused on broader issues such as economic inequality.

``There was no leadership because there were different factions,'' said
Andre Taylor, a local community organizer who had struggled to broker a
meeting between representatives in the zone and the mayor. ``When you're
dealing with a very volatile situation and there's no cohesive voice,
it's really hard to deal with.''

Some residents and local business owners had initially been supportive
of the enclave. Matt Mitgang, who lives across from the abandoned police
station, said he had joined protests after Mr. Floyd's death, and had
been tear-gassed in his apartment when the police initially clashed with
protesters.

Image

Bystanders watched the Seattle police dismantle the encampment on
Wednesday.Credit...Ruth Fremson/The New York Times

``None of us were happy with the police department,'' Mr. Mitgang said.
``I think we were all looking at the evolving situation on our street
with cautious optimism.''

But the recent violence worried him.

``We had kind of reached the point as residents where it felt like the
message was getting lost,'' he said.

A few days ago, Mr. Mitgang and his neighbors called the fire department
to assist a man who appeared to be in medical distress, he said. No one
came.

``Their station is less than a block away, but they never came,'' he
said, ``and I think at that point a lot of us got truly shaken. Seeing
that they wouldn't even come in half a block I think made a lot of us
really worried about what would happen if something else were to
occur.''

A 19-year-old man died and a 33-year-old man was injured in the first
shooting that took place at CHOP on June 20. A 17-year-old man was
injured in a second shooting the following day. And on Monday, the
police said they were investigating
\href{https://www.nytimes.com/2020/06/29/us/seattle-protests-CHOP-CHAZ-autonomous-zone.html}{a
third shooting} that had left a 16-year-old dead and a 14-year-old
seriously injured.

The tension over how to handle the zone had drawn national scrutiny,
including from President Trump, who blasted Democratic officials in
Seattle and Washington State for failing to clear the area earlier.

``If they don't do the job, I'll do the job,'' the president
\href{https://www.youtube.com/watch?v=tWZkptr0FOI}{said last month}.

City officials had responded irritably, with Mayor Durkan saying on
Twitter: ``Make us all safe. Go back to your bunker.''

On Wednesday, the White House spokeswoman, Kayleigh McEnany, called the
protest zone ``a failed four-week Democrat experiment by the radical
left.''

``The results are in,'' she added. ``Anarchy is anti-American. Law and
order is essential. Peace in our streets will be secured.''

Seattle officials had initially announced their intention to shut down
the protest zone over the weekend, but it was not until Wednesday
morning that a crowd of police officers pushed through the area just
after 5 a.m.,
\href{https://twitter.com/MichaelReports/status/1278307571799977985?s=20}{some
wearing helmets and carrying batons.} Officials said the equipment was
``not meant to be a pre-emptive show of force'' but was necessary
because people gathered in the area were known to be armed.

``I woke up to everybody screaming and running, saying, `The cops, the
cops, they're here,''' said Derrek Allen Jones II, who said he had been
staying at the zone for several weeks.

Officers lined up on the edge of the area as a helicopter whirred
overhead. Protesters milled around the intersection, some shouting at
the police. A couple of officers engaged in dialogue directly with
protesters as others led a man away in handcuffs. One man said he had
been hit with pepper spray as officers pushed protesters back to 12th
and Pike Street. A woman raised her fist in the air and chanted, ``These
are our streets.''

Image

A pile of leftover belongings from the encampment.Credit...Ruth
Fremson/The New York Times

Thirty-one people were arrested on charges of failure to disperse,
obstruction, resisting arrest and assault, the police department
\href{https://twitter.com/SeattlePD/status/1278342077311406081}{said on
Twitter}, including a 29-year-old man who had a large metal pipe and a
kitchen knife.

Protesters had been issued warnings to disperse when the police arrived,
according to Detective Mark Jamieson.

``There were people that wanted to be arrested,'' he said. ``We gave
multiple orders to disperse and then either people leave or they
don't.''

The police also cleared protesters from nearby Cal Anderson Park, he
said.

After the arrests, traffic began moving through the streets once again,
and city workers in yellow and orange vests hauled out spray-painted
barricades and artwork. Left behind were some tents, and a few signs:
``R.I.P. E. Precinct,'' one of them read. ``All lives don't matter until
Black lives matter,'' said another.

Sarah Mervosh contributed reporting.

Advertisement

\protect\hyperlink{after-bottom}{Continue reading the main story}

\hypertarget{site-index}{%
\subsection{Site Index}\label{site-index}}

\hypertarget{site-information-navigation}{%
\subsection{Site Information
Navigation}\label{site-information-navigation}}

\begin{itemize}
\tightlist
\item
  \href{https://help.nytimes.com/hc/en-us/articles/115014792127-Copyright-notice}{©~2020~The
  New York Times Company}
\end{itemize}

\begin{itemize}
\tightlist
\item
  \href{https://www.nytco.com/}{NYTCo}
\item
  \href{https://help.nytimes.com/hc/en-us/articles/115015385887-Contact-Us}{Contact
  Us}
\item
  \href{https://www.nytco.com/careers/}{Work with us}
\item
  \href{https://nytmediakit.com/}{Advertise}
\item
  \href{http://www.tbrandstudio.com/}{T Brand Studio}
\item
  \href{https://www.nytimes.com/privacy/cookie-policy\#how-do-i-manage-trackers}{Your
  Ad Choices}
\item
  \href{https://www.nytimes.com/privacy}{Privacy}
\item
  \href{https://help.nytimes.com/hc/en-us/articles/115014893428-Terms-of-service}{Terms
  of Service}
\item
  \href{https://help.nytimes.com/hc/en-us/articles/115014893968-Terms-of-sale}{Terms
  of Sale}
\item
  \href{https://spiderbites.nytimes.com}{Site Map}
\item
  \href{https://help.nytimes.com/hc/en-us}{Help}
\item
  \href{https://www.nytimes.com/subscription?campaignId=37WXW}{Subscriptions}
\end{itemize}
