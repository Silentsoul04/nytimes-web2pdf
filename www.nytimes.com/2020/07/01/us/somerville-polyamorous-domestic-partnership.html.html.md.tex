Sections

SEARCH

\protect\hyperlink{site-content}{Skip to
content}\protect\hyperlink{site-index}{Skip to site index}

\href{https://www.nytimes.com/section/us}{U.S.}

\href{https://myaccount.nytimes.com/auth/login?response_type=cookie\&client_id=vi}{}

\href{https://www.nytimes.com/section/todayspaper}{Today's Paper}

\href{/section/us}{U.S.}\textbar{}A Massachusetts City Decides to
Recognize Polyamorous Relationships

\url{https://nyti.ms/31AmSgZ}

\begin{itemize}
\item
\item
\item
\item
\item
\end{itemize}

Advertisement

\protect\hyperlink{after-top}{Continue reading the main story}

Supported by

\protect\hyperlink{after-sponsor}{Continue reading the main story}

\hypertarget{a-massachusetts-city-decides-to-recognize-polyamorous-relationships}{%
\section{A Massachusetts City Decides to Recognize Polyamorous
Relationships}\label{a-massachusetts-city-decides-to-recognize-polyamorous-relationships}}

The city of Somerville has broadened the definition of domestic
partnership to include relationships between three or more adults,
expanding access to health care.

\includegraphics{https://static01.nyt.com/images/2020/07/01/us/01POLYAMOROUS/01POLYAMOROUS-articleLarge.jpg?quality=75\&auto=webp\&disable=upscale}

\href{https://www.nytimes.com/by/ellen-barry}{\includegraphics{https://static01.nyt.com/images/2018/10/08/multimedia/author-ellen-barry/author-ellen-barry-thumbLarge.png}}

By \href{https://www.nytimes.com/by/ellen-barry}{Ellen Barry}

\begin{itemize}
\item
  July 1, 2020
\item
  \begin{itemize}
  \item
  \item
  \item
  \item
  \item
  \end{itemize}
\end{itemize}

SOMERVILLE, Mass. --- At the tail end of a City Council meeting last
week, so quickly and quietly that you could have easily missed it, a
left-leaning Massachusetts city expanded its notion of family to include
people who are polyamorous, or maintaining consenting relationships with
multiple partners.

Under its new domestic partnership ordinance, the city of Somerville now
\href{https://www.wickedlocal.com/news/20200701/somerville-recognizes-polyamorous-domestic-partnerships}{grants
polyamorous groups} the rights held by spouses in marriage, such as the
right to confer health insurance benefits or make hospital visits.

J.T. Scott, a city councilor who supported the move, said he believed it
was the first such municipal ordinance in the country.

``People have been living in families that include more than two adults
forever,'' Mr. Scott said. ``Here in Somerville, families sometimes look
like one man and one woman, but sometimes it looks like two people
everyone on the block thinks are sisters because they've lived together
forever, or sometimes it's an aunt and an uncle, or an aunt and two
uncles, raising two kids.''

He said he knew of at least two dozen polyamorous households in
Somerville, which has a population of about 80,000.

``This is simply allowing that change, allowing people to say, `This is
my partner and this is my other partner,''' he said. ``It has a legal
bearing, so when one of them is sick, they can both go to the
hospital.''

Until last month, Somerville had no domestic partnership ordinance,
unlike neighboring cities like Cambridge and Boston. It had become an
urgent need with the spread of the coronavirus because residents found
themselves unable to access their partners' health insurance, said
Matthew McLaughlin, the City Council's president. He said expanding
access to health care was his pressing concern.

As the Council prepared language for an ordinance last week, Mr. Scott
raised the issue that it excluded Somerville's polyamorous residents by
specifying that domestic partnerships were ``an entity formed by two
persons.''

The councilor drafting the ordinance, Lance Davis, rewrote it to allow
for multiple partners. It passed unanimously.

\includegraphics{https://static01.nyt.com/images/2020/07/05/us/05polyamorous-02/01polyamorous-02-articleLarge.jpg?quality=75\&auto=webp\&disable=upscale}

``I don't think it's the place of the government to tell people what is
or is not a family,'' Mr. Davis, who is a lawyer, said at a meeting last
week. ``Defining families is something that historically we've gotten
quite wrong as a society, and we ought not to continue to try and
undertake to do so.''

Under the new ordinance, city employees in polyamorous relationships
would be able to extend health benefits to multiple partners. But it is
not clear, Mr. Davis said, whether private employers will follow the
city's lead.

``Based on the conversations I've had,'' he said, ``the most important
aspect is that the city is legally recognizing and validating people's
existence. That's the first time this is happening.''

He said he had considered the possibility that a large number of people
--- say, 20 --- would approach the city and ask to be registered as
domestic partners.

``I say, well what if they do?'' Mr. Davis said. ``I see no reason to
think that is more of an issue than two people.''

Nancy Polikoff, a professor at American University Washington College of
Law and a widely published scholar of family law, said she was not aware
of any other city that has extended such protections to polyamorous
families.

Andy Izenson, a lawyer with the
\href{https://chosenfamilylawcenter.org/}{Chosen Family Law Center}, a
nonprofit organization that provides legal services to polyamorous and
other nontraditional families, said the ordinance could be put to a
judicial test if health insurance companies reject the city's more
expansive definition of domestic partnership. It could also run into
resistance from conservatives,
\href{https://www.nytimes.com/2015/06/27/us/supreme-court-same-sex-marriage.html}{as
same-sex marriage did in 2015}.

Or it could, as he put it, ``fly under the radar.''

``When one area does it, and it serves as a test case, and legislators
see that the town or county has not had a culture war implosion,'' he
said, ``that's how things spread.''

Mr. Scott, the councilman, said he had been inundated by calls and
messages all day, including from lawyers interested in pursuing a
similar measure at the state or federal level.

Under the ordinance, domestic partners, whether in groupings of two or
more, would not necessarily be romantic partners.

Miles Bratton, 47, said she would consider forming a domestic
partnership with Anne-Marie Taylor, 43, whom she called her ``platonic
lifemate.''

The status would allow them to buy a house together and share benefits,
like health insurance, but also to have outside romantic partners, or
add a third ``nesting partner'' if they wished. Ms. Taylor said they had
long held back from registering as domestic partners because the
language her workplace used seemed to require that they be romantic
partners.

``That has not felt right, so we haven't done it,'' she said.
``Somerville is coming out and saying, `Hey, family can be a lot of
other things, other than just two people.'''

Advertisement

\protect\hyperlink{after-bottom}{Continue reading the main story}

\hypertarget{site-index}{%
\subsection{Site Index}\label{site-index}}

\hypertarget{site-information-navigation}{%
\subsection{Site Information
Navigation}\label{site-information-navigation}}

\begin{itemize}
\tightlist
\item
  \href{https://help.nytimes.com/hc/en-us/articles/115014792127-Copyright-notice}{©~2020~The
  New York Times Company}
\end{itemize}

\begin{itemize}
\tightlist
\item
  \href{https://www.nytco.com/}{NYTCo}
\item
  \href{https://help.nytimes.com/hc/en-us/articles/115015385887-Contact-Us}{Contact
  Us}
\item
  \href{https://www.nytco.com/careers/}{Work with us}
\item
  \href{https://nytmediakit.com/}{Advertise}
\item
  \href{http://www.tbrandstudio.com/}{T Brand Studio}
\item
  \href{https://www.nytimes.com/privacy/cookie-policy\#how-do-i-manage-trackers}{Your
  Ad Choices}
\item
  \href{https://www.nytimes.com/privacy}{Privacy}
\item
  \href{https://help.nytimes.com/hc/en-us/articles/115014893428-Terms-of-service}{Terms
  of Service}
\item
  \href{https://help.nytimes.com/hc/en-us/articles/115014893968-Terms-of-sale}{Terms
  of Sale}
\item
  \href{https://spiderbites.nytimes.com}{Site Map}
\item
  \href{https://help.nytimes.com/hc/en-us}{Help}
\item
  \href{https://www.nytimes.com/subscription?campaignId=37WXW}{Subscriptions}
\end{itemize}
