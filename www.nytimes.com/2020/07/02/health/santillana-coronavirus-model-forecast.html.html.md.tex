Sections

SEARCH

\protect\hyperlink{site-content}{Skip to
content}\protect\hyperlink{site-index}{Skip to site index}

\href{https://www.nytimes.com/section/health}{Health}

\href{https://myaccount.nytimes.com/auth/login?response_type=cookie\&client_id=vi}{}

\href{https://www.nytimes.com/section/todayspaper}{Today's Paper}

\href{/section/health}{Health}\textbar{}Can an Algorithm Predict the
Pandemic's Next Moves?

\url{https://nyti.ms/31B48Ox}

\begin{itemize}
\item
\item
\item
\item
\item
\end{itemize}

\href{https://www.nytimes.com/news-event/coronavirus?action=click\&pgtype=Article\&state=default\&region=TOP_BANNER\&context=storylines_menu}{The
Coronavirus Outbreak}

\begin{itemize}
\tightlist
\item
  live\href{https://www.nytimes.com/2020/08/01/world/coronavirus-covid-19.html?action=click\&pgtype=Article\&state=default\&region=TOP_BANNER\&context=storylines_menu}{Latest
  Updates}
\item
  \href{https://www.nytimes.com/interactive/2020/us/coronavirus-us-cases.html?action=click\&pgtype=Article\&state=default\&region=TOP_BANNER\&context=storylines_menu}{Maps
  and Cases}
\item
  \href{https://www.nytimes.com/interactive/2020/science/coronavirus-vaccine-tracker.html?action=click\&pgtype=Article\&state=default\&region=TOP_BANNER\&context=storylines_menu}{Vaccine
  Tracker}
\item
  \href{https://www.nytimes.com/interactive/2020/07/29/us/schools-reopening-coronavirus.html?action=click\&pgtype=Article\&state=default\&region=TOP_BANNER\&context=storylines_menu}{What
  School May Look Like}
\item
  \href{https://www.nytimes.com/live/2020/07/31/business/stock-market-today-coronavirus?action=click\&pgtype=Article\&state=default\&region=TOP_BANNER\&context=storylines_menu}{Economy}
\end{itemize}

Advertisement

\protect\hyperlink{after-top}{Continue reading the main story}

Supported by

\protect\hyperlink{after-sponsor}{Continue reading the main story}

\hypertarget{can-an-algorithm-predict-the-pandemics-next-moves}{%
\section{Can an Algorithm Predict the Pandemic's Next
Moves?}\label{can-an-algorithm-predict-the-pandemics-next-moves}}

Researchers have developed a model that uses social-media and search
data to forecast outbreaks of Covid-19 well before they occur.

\includegraphics{https://static01.nyt.com/images/2020/07/07/science/02SCI-VIRUS-PREDICT-1/merlin_174125376_c2aa3d47-abb1-4e6b-99c2-2752dca96f70-articleLarge.jpg?quality=75\&auto=webp\&disable=upscale}

\href{https://www.nytimes.com/by/benedict-carey}{\includegraphics{https://static01.nyt.com/images/2018/02/16/multimedia/author-benedict-carey/author-benedict-carey-thumbLarge.jpg}}

By \href{https://www.nytimes.com/by/benedict-carey}{Benedict Carey}

\begin{itemize}
\item
  July 2, 2020
\item
  \begin{itemize}
  \item
  \item
  \item
  \item
  \item
  \end{itemize}
\end{itemize}

Judging when to tighten, or loosen, the local economy has become the
world's most consequential guessing game, and each policymaker has his
or her own instincts and benchmarks. The point when hospitals reach 70
percent capacity is a red flag, for instance; so are upticks in
coronavirus case counts and deaths.

But as the governors of states like Florida, California and Texas have
learned in recent days, such benchmarks make for a poor alarm system.
Once the coronavirus finds an opening in the population, it gains a
two-week head start on health officials, circulating and multiplying
swiftly before its re-emergence becomes apparent at hospitals, testing
clinics and elsewhere.

Now, an international team of scientists has developed a model --- or,
at minimum, the template for a model --- that could predict outbreaks
about two weeks before they occur, in time to put effective containment
measures in place.

In \href{https://arxiv.org/abs/2007.00756}{a paper} posted on Thursday
on arXiv.org, the team, led by Mauricio Santillana and Nicole Kogan of
Harvard, presented an algorithm that registered danger 14 days or more
before case counts begin to increase. The system uses real-time
monitoring of Twitter, Google searches and mobility data from
smartphones, among other data streams.

The algorithm, the researchers write, could function ``as a thermostat,
in a cooling or heating system, to guide intermittent activation or
relaxation of public health interventions'' --- that is, a smoother,
safer reopening.

``In most infectious-disease modeling, you project different scenarios
based on assumptions made up front,'' said Dr. Santillana, director of
the Machine Intelligence Lab at Boston Children's Hospital and an
assistant professor of pediatrics and epidemiology at Harvard. ``What
we're doing here is observing, without making assumptions. The
difference is that our methods are responsive to immediate changes in
behavior and we can incorporate those.''

Outside experts who were shown the new analysis, which has not yet been
peer reviewed, said it demonstrated the increasing value of real-time
data, like social media, in improving existing models.

\hypertarget{latest-updates-global-coronavirus-outbreak}{%
\section{\texorpdfstring{\href{https://www.nytimes.com/2020/08/01/world/coronavirus-covid-19.html?action=click\&pgtype=Article\&state=default\&region=MAIN_CONTENT_1\&context=storylines_live_updates}{Latest
Updates: Global Coronavirus
Outbreak}}{Latest Updates: Global Coronavirus Outbreak}}\label{latest-updates-global-coronavirus-outbreak}}

Updated 2020-08-02T05:48:45.291Z

\begin{itemize}
\tightlist
\item
  \href{https://www.nytimes.com/2020/08/01/world/coronavirus-covid-19.html?action=click\&pgtype=Article\&state=default\&region=MAIN_CONTENT_1\&context=storylines_live_updates\#link-34047410}{The
  U.S. reels as July cases more than double the total of any other
  month.}
\item
  \href{https://www.nytimes.com/2020/08/01/world/coronavirus-covid-19.html?action=click\&pgtype=Article\&state=default\&region=MAIN_CONTENT_1\&context=storylines_live_updates\#link-780ec966}{Top
  U.S. officials work to break an impasse over the federal jobless
  benefit.}
\item
  \href{https://www.nytimes.com/2020/08/01/world/coronavirus-covid-19.html?action=click\&pgtype=Article\&state=default\&region=MAIN_CONTENT_1\&context=storylines_live_updates\#link-25930521}{Thousands
  in Berlin protest Germany's coronavirus measures.}
\end{itemize}

\href{https://www.nytimes.com/2020/08/01/world/coronavirus-covid-19.html?action=click\&pgtype=Article\&state=default\&region=MAIN_CONTENT_1\&context=storylines_live_updates}{See
more updates}

More live coverage:
\href{https://www.nytimes.com/live/2020/07/31/business/stock-market-today-coronavirus?action=click\&pgtype=Article\&state=default\&region=MAIN_CONTENT_1\&context=storylines_live_updates}{Markets}

The study shows ``that alternative, next-gen data sources may provide
early signals of rising Covid-19 prevalence,'' said Lauren Ancel Meyers,
a biologist and statistician at the University of Texas, Austin.
``Particularly if confirmed case counts are lagged by delays in seeking
treatment and obtaining test results.''

The use of real-time data analysis to gauge disease progression goes
back at least to 2008, when engineers at Google began estimating doctor
visits for the flu by tracking search trends for words like ``feeling
exhausted,'' ``joints aching,'' ``Tamiflu dosage'' and many others.

The Google Flu Trends algorithm, as it is known, performed poorly. For
instance, it continually overestimated doctor visits, later evaluations
found, because of limitations of the data and the influence of outside
factors such as media attention, which can
\href{https://dash.harvard.edu/bitstream/handle/1/12016837/ssrn-id2408560_2.pdf?sequence=1\&isAllowed=y}{drive
up searches} that are unrelated to actual illness.

Since then, researchers have made multiple adjustments to this approach,
combining Google searches with other kinds of data. Teams at
Carnegie-Mellon University, University College London and the University
of Texas, among others, have models incorporating some real-time data
analysis.

``We know that no single data stream is useful in isolation,'' said
Madhav Marathe, a computer scientist at the University of Virginia.
``The contribution of this new paper is that they have a good, wide
variety of streams.''

In the new paper, the team analyzed real-time data from four sources, in
addition to Google: Covid-related Twitter posts, geotagged for location;
doctors' searches on a physician platform called UpToDate; anonymous
mobility data from smartphones; and readings from the Kinsa Smart
Thermometer, which uploads to an app. It integrated those data streams
with a sophisticated prediction model developed at Northeastern
University, based on how people move and interact in communities.

The team tested the predictive value of trends in the data stream by
looking at how each correlated with case counts and deaths over March
and April, in each state.

In New York, for instance, a sharp uptrend in Covid-related Twitter
posts began more than a week before case counts exploded in mid-March;
relevant Google searches and Kinsa measures spiked several days
beforehand.

The team combined all its data sources, in effect weighting each
according to how strongly it was correlated to a coming increase in
cases. This ``harmonized'' algorithm anticipated outbreaks by 21 days,
on average, the researchers found.

Looking ahead, it predicts that Nebraska and New Hampshire are likely to
see cases increase in the coming weeks if no further measures are taken,
despite case counts being currently flat.

\href{https://www.nytimes.com/news-event/coronavirus?action=click\&pgtype=Article\&state=default\&region=MAIN_CONTENT_3\&context=storylines_faq}{}

\hypertarget{the-coronavirus-outbreak-}{%
\subsubsection{The Coronavirus Outbreak
›}\label{the-coronavirus-outbreak-}}

\hypertarget{frequently-asked-questions}{%
\paragraph{Frequently Asked
Questions}\label{frequently-asked-questions}}

Updated July 27, 2020

\begin{itemize}
\item ~
  \hypertarget{should-i-refinance-my-mortgage}{%
  \paragraph{Should I refinance my
  mortgage?}\label{should-i-refinance-my-mortgage}}

  \begin{itemize}
  \tightlist
  \item
    \href{https://www.nytimes.com/article/coronavirus-money-unemployment.html?action=click\&pgtype=Article\&state=default\&region=MAIN_CONTENT_3\&context=storylines_faq}{It
    could be a good idea,} because mortgage rates have
    \href{https://www.nytimes.com/2020/07/16/business/mortgage-rates-below-3-percent.html?action=click\&pgtype=Article\&state=default\&region=MAIN_CONTENT_3\&context=storylines_faq}{never
    been lower.} Refinancing requests have pushed mortgage applications
    to some of the highest levels since 2008, so be prepared to get in
    line. But defaults are also up, so if you're thinking about buying a
    home, be aware that some lenders have tightened their standards.
  \end{itemize}
\item ~
  \hypertarget{what-is-school-going-to-look-like-in-september}{%
  \paragraph{What is school going to look like in
  September?}\label{what-is-school-going-to-look-like-in-september}}

  \begin{itemize}
  \tightlist
  \item
    It is unlikely that many schools will return to a normal schedule
    this fall, requiring the grind of
    \href{https://www.nytimes.com/2020/06/05/us/coronavirus-education-lost-learning.html?action=click\&pgtype=Article\&state=default\&region=MAIN_CONTENT_3\&context=storylines_faq}{online
    learning},
    \href{https://www.nytimes.com/2020/05/29/us/coronavirus-child-care-centers.html?action=click\&pgtype=Article\&state=default\&region=MAIN_CONTENT_3\&context=storylines_faq}{makeshift
    child care} and
    \href{https://www.nytimes.com/2020/06/03/business/economy/coronavirus-working-women.html?action=click\&pgtype=Article\&state=default\&region=MAIN_CONTENT_3\&context=storylines_faq}{stunted
    workdays} to continue. California's two largest public school
    districts --- Los Angeles and San Diego --- said on July 13, that
    \href{https://www.nytimes.com/2020/07/13/us/lausd-san-diego-school-reopening.html?action=click\&pgtype=Article\&state=default\&region=MAIN_CONTENT_3\&context=storylines_faq}{instruction
    will be remote-only in the fall}, citing concerns that surging
    coronavirus infections in their areas pose too dire a risk for
    students and teachers. Together, the two districts enroll some
    825,000 students. They are the largest in the country so far to
    abandon plans for even a partial physical return to classrooms when
    they reopen in August. For other districts, the solution won't be an
    all-or-nothing approach.
    \href{https://bioethics.jhu.edu/research-and-outreach/projects/eschool-initiative/school-policy-tracker/}{Many
    systems}, including the nation's largest, New York City, are
    devising
    \href{https://www.nytimes.com/2020/06/26/us/coronavirus-schools-reopen-fall.html?action=click\&pgtype=Article\&state=default\&region=MAIN_CONTENT_3\&context=storylines_faq}{hybrid
    plans} that involve spending some days in classrooms and other days
    online. There's no national policy on this yet, so check with your
    municipal school system regularly to see what is happening in your
    community.
  \end{itemize}
\item ~
  \hypertarget{is-the-coronavirus-airborne}{%
  \paragraph{Is the coronavirus
  airborne?}\label{is-the-coronavirus-airborne}}

  \begin{itemize}
  \tightlist
  \item
    The coronavirus
    \href{https://www.nytimes.com/2020/07/04/health/239-experts-with-one-big-claim-the-coronavirus-is-airborne.html?action=click\&pgtype=Article\&state=default\&region=MAIN_CONTENT_3\&context=storylines_faq}{can
    stay aloft for hours in tiny droplets in stagnant air}, infecting
    people as they inhale, mounting scientific evidence suggests. This
    risk is highest in crowded indoor spaces with poor ventilation, and
    may help explain super-spreading events reported in meatpacking
    plants, churches and restaurants.
    \href{https://www.nytimes.com/2020/07/06/health/coronavirus-airborne-aerosols.html?action=click\&pgtype=Article\&state=default\&region=MAIN_CONTENT_3\&context=storylines_faq}{It's
    unclear how often the virus is spread} via these tiny droplets, or
    aerosols, compared with larger droplets that are expelled when a
    sick person coughs or sneezes, or transmitted through contact with
    contaminated surfaces, said Linsey Marr, an aerosol expert at
    Virginia Tech. Aerosols are released even when a person without
    symptoms exhales, talks or sings, according to Dr. Marr and more
    than 200 other experts, who
    \href{https://academic.oup.com/cid/article/doi/10.1093/cid/ciaa939/5867798}{have
    outlined the evidence in an open letter to the World Health
    Organization}.
  \end{itemize}
\item ~
  \hypertarget{what-are-the-symptoms-of-coronavirus}{%
  \paragraph{What are the symptoms of
  coronavirus?}\label{what-are-the-symptoms-of-coronavirus}}

  \begin{itemize}
  \tightlist
  \item
    Common symptoms
    \href{https://www.nytimes.com/article/symptoms-coronavirus.html?action=click\&pgtype=Article\&state=default\&region=MAIN_CONTENT_3\&context=storylines_faq}{include
    fever, a dry cough, fatigue and difficulty breathing or shortness of
    breath.} Some of these symptoms overlap with those of the flu,
    making detection difficult, but runny noses and stuffy sinuses are
    less common.
    \href{https://www.nytimes.com/2020/04/27/health/coronavirus-symptoms-cdc.html?action=click\&pgtype=Article\&state=default\&region=MAIN_CONTENT_3\&context=storylines_faq}{The
    C.D.C. has also} added chills, muscle pain, sore throat, headache
    and a new loss of the sense of taste or smell as symptoms to look
    out for. Most people fall ill five to seven days after exposure, but
    symptoms may appear in as few as two days or as many as 14 days.
  \end{itemize}
\item ~
  \hypertarget{does-asymptomatic-transmission-of-covid-19-happen}{%
  \paragraph{Does asymptomatic transmission of Covid-19
  happen?}\label{does-asymptomatic-transmission-of-covid-19-happen}}

  \begin{itemize}
  \tightlist
  \item
    So far, the evidence seems to show it does. A widely cited
    \href{https://www.nature.com/articles/s41591-020-0869-5}{paper}
    published in April suggests that people are most infectious about
    two days before the onset of coronavirus symptoms and estimated that
    44 percent of new infections were a result of transmission from
    people who were not yet showing symptoms. Recently, a top expert at
    the World Health Organization stated that transmission of the
    coronavirus by people who did not have symptoms was ``very rare,''
    \href{https://www.nytimes.com/2020/06/09/world/coronavirus-updates.html?action=click\&pgtype=Article\&state=default\&region=MAIN_CONTENT_3\&context=storylines_faq\#link-1f302e21}{but
    she later walked back that statement.}
  \end{itemize}
\end{itemize}

``I think we can expect to see at least a week or more of advanced
warning, conservatively, taking into account that the epidemic is
continually changing,'' Dr. Santillana said. His co-authors included
scientists from the University of Maryland, Baltimore County; Stanford
University; and the University of Salzburg, as well as Northeastern.

He added: ``And we don't see this data as replacing traditional
surveillance but confirming it. It's the kind of information that can
enable decision makers to say, `Let's not wait one more week, let's act
now.'''

For all its appeal, big-data analytics cannot anticipate sudden changes
in mass behavior any better than other, traditional models can, experts
said. There is no algorithm that might have predicted the nationwide
protests in the wake of George Floyd's killing, for instance --- mass
gatherings that may have seeded new outbreaks, despite precautions taken
by protesters.

Social media and search engines also can become less sensitive with
time; the more familiar with a pathogen people become, the less they
will search with selected key words.

Public health agencies like the Centers for Disease Control and
Prevention, which also consults real-time data from social media and
other sources, have not made such algorithms central to their forecasts.

``This is extremely valuable data for us to have,'' said Shweta Bansal,
a biologist at Georgetown University. ``But I wouldn't want to go into
the forecasting business on this; the harm that can be done is quite
severe. We need to see such models verified and validated over time.''

Given the persistent and repeating challenges of the coronavirus and the
inadequacy of the current public health infrastructure, that seems
likely to happen, most experts said. There is an urgent need, and there
is no lack of data.

``What we've looked at is what we think are the best available data
streams,'' Dr. Santillana said. ``We'd be eager to see what Amazon could
give us, or Netflix.''

\textbf{\emph{{[}}\href{http://on.fb.me/1paTQ1h}{\emph{Like the Science
Times page on Facebook.}}} ****** \emph{\textbar{} Sign up for the}
\textbf{\href{http://nyti.ms/1MbHaRU}{\emph{Science Times
newsletter.}}\emph{{]}}}

Advertisement

\protect\hyperlink{after-bottom}{Continue reading the main story}

\hypertarget{site-index}{%
\subsection{Site Index}\label{site-index}}

\hypertarget{site-information-navigation}{%
\subsection{Site Information
Navigation}\label{site-information-navigation}}

\begin{itemize}
\tightlist
\item
  \href{https://help.nytimes.com/hc/en-us/articles/115014792127-Copyright-notice}{©~2020~The
  New York Times Company}
\end{itemize}

\begin{itemize}
\tightlist
\item
  \href{https://www.nytco.com/}{NYTCo}
\item
  \href{https://help.nytimes.com/hc/en-us/articles/115015385887-Contact-Us}{Contact
  Us}
\item
  \href{https://www.nytco.com/careers/}{Work with us}
\item
  \href{https://nytmediakit.com/}{Advertise}
\item
  \href{http://www.tbrandstudio.com/}{T Brand Studio}
\item
  \href{https://www.nytimes.com/privacy/cookie-policy\#how-do-i-manage-trackers}{Your
  Ad Choices}
\item
  \href{https://www.nytimes.com/privacy}{Privacy}
\item
  \href{https://help.nytimes.com/hc/en-us/articles/115014893428-Terms-of-service}{Terms
  of Service}
\item
  \href{https://help.nytimes.com/hc/en-us/articles/115014893968-Terms-of-sale}{Terms
  of Sale}
\item
  \href{https://spiderbites.nytimes.com}{Site Map}
\item
  \href{https://help.nytimes.com/hc/en-us}{Help}
\item
  \href{https://www.nytimes.com/subscription?campaignId=37WXW}{Subscriptions}
\end{itemize}
