Sections

SEARCH

\protect\hyperlink{site-content}{Skip to
content}\protect\hyperlink{site-index}{Skip to site index}

\href{https://myaccount.nytimes.com/auth/login?response_type=cookie\&client_id=vi}{}

\href{https://www.nytimes.com/section/todayspaper}{Today's Paper}

\href{/section/opinion}{Opinion}\textbar{}A Difficult Independence Day

\href{https://nyti.ms/2ZLux9V}{https://nyti.ms/2ZLux9V}

\begin{itemize}
\item
\item
\item
\item
\item
\item
\end{itemize}

Advertisement

\protect\hyperlink{after-top}{Continue reading the main story}

\href{/section/opinion}{Opinion}

Supported by

\protect\hyperlink{after-sponsor}{Continue reading the main story}

\hypertarget{a-difficult-independence-day}{%
\section{A Difficult Independence
Day}\label{a-difficult-independence-day}}

The virus has feasted on a compromised body.

\href{https://www.nytimes.com/by/roger-cohen}{\includegraphics{https://static01.nyt.com/images/2014/11/01/opinion/cohen-circular/cohen-circular-thumbLarge-v6.png}}

By \href{https://www.nytimes.com/by/roger-cohen}{Roger Cohen}

Opinion Columnist

\begin{itemize}
\item
  July 2, 2020
\item
  \begin{itemize}
  \item
  \item
  \item
  \item
  \item
  \item
  \end{itemize}
\end{itemize}

\includegraphics{https://static01.nyt.com/images/2020/07/03/opinion/03cohen1/merlin_173889483_75600184-175c-497c-a14e-44dde0adf6c9-articleLarge.jpg?quality=75\&auto=webp\&disable=upscale}

MONTROSE, Colo. --- Around here, on Colorado's Western Slope, the virus
is known as ``the Rona.'' As in, ``the Rona'' is back. Montrose voted
for Donald Trump in 2016, and he still has enough support to make masks
rare. Up the way, in liberal Telluride, everyone wears them. Even
disease is tribal now.

It's hard to know how to celebrate this July 4. The nation's
unemployment rate is 11.1 percent. The coronavirus spreads unabated. In
Europe, Americans are the new pariahs.

Once saviors, long allies, Americans are now lepers from the Land of the
Crazed. To Europeans,
\href{https://www.nytimes.com/2020/06/26/world/europe/europe-us-travel-ban.html}{they
are a banned infection risk}.

This is the 244th anniversary of the Declaration of Independence by the
founding fathers. ``We hold these truths to be self-evident, that all
men are created equal, that they are endowed by their Creator with
certain unalienable Rights, that among these are Life, Liberty and the
pursuit of Happiness.''

What a beautiful sentence and how it has moved humankind and how many
people have been freed by the inspiration of the revolutionary American
idea. It changed and shaped my life.

But all men were not created equal in the nascent United States; Black
men were worth less, and of course there's no mention of women. This was
\emph{self-evident} to these men formed by their experience in 1776.

Some of the founders are now under attack for owning slaves. When George
Washington and Thomas Jefferson fall from grace, you have to wonder.
Union generals, including Ulysses Grant, who fought to defeat the
Confederacy and slavery, were not good enough. They were imperfect, the
human condition.

Moral absolutism has its giddy day. The guillotine falls. This is
madness. Be careful what you say. It is the hour of the new judges; the
judged are scared; and judgment of the judges may be decades or even
centuries off.

The pendulum never swings the right amount. It goes too far. What other
happy news can we chew on with our burgers?

\href{https://www.nytimes.com/2020/06/26/opinion/let-freedom-ring-from-georgia.html}{I
have been traveling around the country.} I can report that it's a
free-for-all. At Newark Liberty International Airport, circles on the
floors indicate where to stand six feet apart. Then United Airlines
crams you in like the canned sardines of old.

The
\href{https://www.nytimes.com/2020/06/30/style/mask-america-freedom-coronavirus.html}{masked
eye the unmasked}. The act of greeting has become an awkward contortion.
Distance is prized over closeness. At six feet, through a plastic panel,
a masked person is inaudible. Words are muffled ghosts. We can no longer
hear one another. Even children are boxed.

The United States has disaggregated. It cannot find in itself coherent
resolve. Its democracy is dysfunctional. This unraveling is the
fundamental story of the decades since the Cold War. The virus has
gorged itself on an already compromised body.

President Trump, who makes King Lear on the cliffs seem sane, spirals
down, bowing to Vladimir Putin even when American lives are lost,
yearning for the late Roger Ailes and his Fox News monument to misogyny,
mouthing nonsense about the virus, doing his race-baiting thing, playing
the strongman as only cowards can, denying truth.

But truth is as inexorable as the plague.

We elected this man, knowing who he is. If consistency is a virtue, it's
as close as Trump gets to having one. For Americans, there is no
possible plea of ignorance.

In its 300th issue of January 1991, Mad magazine published a feature
called
``\href{https://www.comicartfans.com/gallerypiece.asp?piece=1306080}{The
Wizard of Odds.}'' The wizard lives in the Palace of Glitz in an America
where greed has become God. The wizard is the Donald.

He tosses dollar bills into the air with stubby hands, delighting in ``a
world full of schmucks'' whom he loves because he needs them as he's
``piling up the bucks.'' He eyes up young Dorothy, who believes the
Trump-wizard can deliver her from materialistic hell back to her
down-to-earth world of Kansas in 1939. He offers instead to put her up
in a penthouse.

``In a couple of years, after you fill out, you could be my steady
bimbo,'' the Trump character says in the magazine.

Even three decades ago, we knew precisely who this man was.

So back to those burgers. We can begin by celebrating that Trump is not
Putin or China's president, Xi Jinping, and we still have the chance to
kick the bum out by voting in November. We can celebrate that the
Supreme Court is not yet his. We can celebrate the early-morning can-do
smile of Americans. We can celebrate the vastness of American space, its
enduring possibility. We can celebrate the beauty of the American idea,
as expressed by the founders, and its capacity for reinvention through
the potential attainment, at last, of racial equity. We can celebrate
our history without hiding from its stains.

I stand on Main Street, U.S.A. It's not beautiful, but it contains
beauty through its embodiment of the fact that America is a nation of
plain-spoken strivers drawn from every corner of the earth. They will
not desist until America's promise is reclaimed. Chew on that.

\emph{The Times is committed to publishing}
\href{https://www.nytimes.com/2019/01/31/opinion/letters/letters-to-editor-new-york-times-women.html}{\emph{a
diversity of letters}} \emph{to the editor. We'd like to hear what you
think about this or any of our articles. Here are some}
\href{https://help.nytimes.com/hc/en-us/articles/115014925288-How-to-submit-a-letter-to-the-editor}{\emph{tips}}\emph{.
And here's our email:}
\href{mailto:letters@nytimes.com}{\emph{letters@nytimes.com}}\emph{.}

\emph{Follow The New York Times Opinion section on}
\href{https://www.facebook.com/nytopinion}{\emph{Facebook}}\emph{,}
\href{http://twitter.com/NYTOpinion}{\emph{Twitter (@NYTopinion)}}
\emph{and}
\href{https://www.instagram.com/nytopinion/}{\emph{Instagram}}\emph{.}

Advertisement

\protect\hyperlink{after-bottom}{Continue reading the main story}

\hypertarget{site-index}{%
\subsection{Site Index}\label{site-index}}

\hypertarget{site-information-navigation}{%
\subsection{Site Information
Navigation}\label{site-information-navigation}}

\begin{itemize}
\tightlist
\item
  \href{https://help.nytimes.com/hc/en-us/articles/115014792127-Copyright-notice}{©~2020~The
  New York Times Company}
\end{itemize}

\begin{itemize}
\tightlist
\item
  \href{https://www.nytco.com/}{NYTCo}
\item
  \href{https://help.nytimes.com/hc/en-us/articles/115015385887-Contact-Us}{Contact
  Us}
\item
  \href{https://www.nytco.com/careers/}{Work with us}
\item
  \href{https://nytmediakit.com/}{Advertise}
\item
  \href{http://www.tbrandstudio.com/}{T Brand Studio}
\item
  \href{https://www.nytimes.com/privacy/cookie-policy\#how-do-i-manage-trackers}{Your
  Ad Choices}
\item
  \href{https://www.nytimes.com/privacy}{Privacy}
\item
  \href{https://help.nytimes.com/hc/en-us/articles/115014893428-Terms-of-service}{Terms
  of Service}
\item
  \href{https://help.nytimes.com/hc/en-us/articles/115014893968-Terms-of-sale}{Terms
  of Sale}
\item
  \href{https://spiderbites.nytimes.com}{Site Map}
\item
  \href{https://help.nytimes.com/hc/en-us}{Help}
\item
  \href{https://www.nytimes.com/subscription?campaignId=37WXW}{Subscriptions}
\end{itemize}
