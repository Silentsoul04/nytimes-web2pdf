Sections

SEARCH

\protect\hyperlink{site-content}{Skip to
content}\protect\hyperlink{site-index}{Skip to site index}

\href{https://www.nytimes.com/section/politics}{Politics}

\href{https://myaccount.nytimes.com/auth/login?response_type=cookie\&client_id=vi}{}

\href{https://www.nytimes.com/section/todayspaper}{Today's Paper}

\href{/section/politics}{Politics}\textbar{}The View From Waukesha

\url{https://nyti.ms/3eWWCkD}

\begin{itemize}
\item
\item
\item
\item
\item
\end{itemize}

\begin{itemize}
\item
  \href{https://www.nytimes.com/2020/07/31/us/elections/biden-vs-trump.html?action=click\&pgtype=Article\&state=default\&region=TOP_BANNER\&context=storylines_menu}{Election
  Updates}
\item
  \href{https://www.nytimes.com/article/biden-vice-president-2020.html?action=click\&pgtype=Article\&state=default\&region=TOP_BANNER\&context=storylines_menu}{Biden's
  V.P. Search}
\item
  \href{https://www.nytimes.com/interactive/2020/07/24/us/politics/trump-biden-campaign-donors.html?action=click\&pgtype=Article\&state=default\&region=TOP_BANNER\&context=storylines_menu}{Map
  of Donations}
\item
  \href{https://www.nytimes.com/interactive/2020/us/elections/delegate-count-primary-results.html?action=click\&pgtype=Article\&state=default\&region=TOP_BANNER\&context=storylines_menu}{Delegate
  Count}
\item
  \href{https://www.nytimes.com/interactive/2019/us/politics/2020-presidential-candidates.html?action=click\&pgtype=Article\&state=default\&region=TOP_BANNER\&context=storylines_menu}{The
  Candidates}
\item
  \href{https://www.nytimes.com/newsletters/politics?action=click\&pgtype=Article\&state=default\&region=TOP_BANNER\&context=storylines_menu}{Politics
  Newsletter}
\end{itemize}

Advertisement

\protect\hyperlink{after-top}{Continue reading the main story}

Supported by

\protect\hyperlink{after-sponsor}{Continue reading the main story}

On Politics

\hypertarget{the-view-from-waukesha}{%
\section{The View From Waukesha}\label{the-view-from-waukesha}}

The 29-year-old Democratic Party chairman in this key Wisconsin county
has a plan for Joe Biden.

\href{https://www.nytimes.com/by/astead-w-herndon}{\includegraphics{https://static01.nyt.com/images/2018/09/14/us/author-head-astead/author-head-astead-thumbLarge-v2.png}}

By \href{https://www.nytimes.com/by/astead-w-herndon}{Astead W. Herndon}

\begin{itemize}
\item
  July 2, 2020
\item
  \begin{itemize}
  \item
  \item
  \item
  \item
  \item
  \end{itemize}
\end{itemize}

\emph{Hi. Welcome to}
\href{https://www.nytimes.com/spotlight/on-politics}{\emph{On
Politics}}\emph{, your guide to the day in national politics. I'm Astead
Herndon, reporting on the road in Wisconsin while Lisa Lerer is away.}

\href{https://www.nytimes.com/newsletters/politics?module=inline}{\emph{Sign
up here}} \emph{to get On Politics in your inbox every weekday.}

\includegraphics{https://static01.nyt.com/images/2020/07/02/us/politics/oakImage-1593728999796/oakImage-1593728999796-articleLarge.jpg?quality=75\&auto=webp\&disable=upscale}

WAUKESHA, Wis. --- If Wisconsin is the bellwether state for November's
general election, then Waukesha County might be the most important area
of the most important state. Just outside the Democratic stronghold of
Milwaukee, it is a safe Republican area, but with many of the kinds of
voters
\href{https://www.nytimes.com/2020/06/25/upshot/poll-2020-biden-battlegrounds.html}{who
have been souring on President Trump}: college-educated white moderates.

Hillary Clinton's performance here --- around a third of the vote ---
was disastrous for her chances in the state. However, Senator Tammy
Baldwin, a progressive Democrat, picked up nearly 40 percent of the
Waukesha vote while cruising to re-election in 2018. Former Vice
President Joe Biden will need to be more like Ms. Baldwin if he wants to
win Wisconsin.

Matt Lowe, the 29-year-old chairman of the Waukesha Democratic Party, is
working to pull that off. He says his target number for Mr. Biden in the
county is 40 percent, which would help erase the small margin Mrs.
Clinton lost the state by.

In an interview today at the county party's office, Mr. Lowe explained
how he thought Mr. Trump's response to the coronavirus pandemic and
protests against racial injustice would hurt the president in the
all-important Midwestern suburbs. (As always, our conversation has been
edited and condensed.)

\textbf{What have you seen since 2016 that has changed in this
community?}

It's night and day. Whether it's party organization, energy, enthusiasm,
financially --- I mean, really almost any way you measure, we're in a
better place now than we were then. When I took over, our membership was
in the low hundreds. We had maybe 10 or 15 people coming out for monthly
events. It was a very old crowd, not very active or organized. We relied
a lot on the state party and the national party to kind of drive
everything.

Today we have over 500 dues-paying members. Our last meeting before the
pandemic we had over 300 people show up. We have the youngest executive
board in the state of Wisconsin. We have the largest high school chapter
in the state of Wisconsin. The organization and the enthusiasm have been
radically different.

\textbf{Give me your diagnosis: What happened in this state, and more
specifically in this region, in 2016?}

I think everybody was complacent. We were just kind of like, ``Trump is
this insane Republican outlier that no one's going to back, and come
Election Day everyone is going to vote rational.'' And come Election Day
we were dramatically surprised.

We probably knocked less doors. We put in a little less effort. People
who volunteered, they'd say instead of doing those five extra shifts,
I'm just going to take it easy and not worry about it. And we got
surprised.

\hypertarget{latest-updates-2020-election}{%
\section{\texorpdfstring{\href{https://www.nytimes.com/2020/07/31/us/elections/biden-vs-trump.html?action=click\&pgtype=Article\&state=default\&region=MAIN_CONTENT_1\&context=storylines_live_updates}{Latest
Updates: 2020
Election}}{Latest Updates: 2020 Election}}\label{latest-updates-2020-election}}

Updated 2020-08-01T01:26:45.732Z

\begin{itemize}
\tightlist
\item
  \href{https://www.nytimes.com/2020/07/31/us/elections/biden-vs-trump.html?action=click\&pgtype=Article\&state=default\&region=MAIN_CONTENT_1\&context=storylines_live_updates\#link-29fdff45}{Kamala
  Harris, a top vice-presidential contender, confronts double
  standards.}
\item
  \href{https://www.nytimes.com/2020/07/31/us/elections/biden-vs-trump.html?action=click\&pgtype=Article\&state=default\&region=MAIN_CONTENT_1\&context=storylines_live_updates\#link-13ec3d9c}{Karen
  Bass and Susan Rice are rising on Biden's vice-presidential
  shortlist.}
\item
  \href{https://www.nytimes.com/2020/07/31/us/elections/biden-vs-trump.html?action=click\&pgtype=Article\&state=default\&region=MAIN_CONTENT_1\&context=storylines_live_updates\#link-49e9a016}{Trump
  says Russian bounties to kill U.S. troops `never took place.'}
\end{itemize}

\href{https://www.nytimes.com/2020/07/31/us/elections/biden-vs-trump.html?action=click\&pgtype=Article\&state=default\&region=MAIN_CONTENT_1\&context=storylines_live_updates}{See
more updates}

\textbf{In terms of his appeal here, is there something about Joe Biden
that makes your job easier?}

I think Biden comes with this statesman appeal. Joe Biden is a guy who's
devoted his entire life to serving this country. And he's always spoken
his mind --- for good or bad. He's always been very true with who he is,
and you never really question his motives or where he's coming from.

I think the biggest question most voters ask themselves is, ``Who cares
about me?'' And it's stark: Who is going to put people like them at the
front of their mind, Joe Biden or Donald Trump?

\textbf{But there was so much talk in the primary, among your
generation, of structural, systemic change and how a nominee needs to
embrace that to excite the base. The Biden campaign says that he does
the structural stuff, but also that he has a different coalition that
might not be thrilling the young folks but can lead to success in places
like Wisconsin. Is that true?}

Conor Lamb {[}a Democratic congressman from Pennsylvania{]} did an
interview recently talking about western Pennsylvania, a very
conservative part of a very Wisconsin-like state. And a lot of what he
said resonated with me here in Waukesha. You need somebody who's
community first, who is very honest with their stances. And when they
disagree with you, they're telling you why but making sure that you know
they're putting their community first even in that disagreement.

I think Conor Lamb is a great model for what the Democratic Party is.
Because the Democratic Party is a big-tent coalition. We have the Conor
Lambs and the
\href{https://www.nytimes.com/2020/06/23/us/politics/aoc-facebook-ads.html}{A.O.C.s}
{[}Representative Alexandria Ocasio-Cortez of New York{]} in the same
party, fighting the same fights with radically different ideas. And we
welcome that debate.

\textbf{You're 29. How does that view fit with your generation, which
has many progressives who have been more rigid about what constitutes a
Democrat?}

I oftentimes have a hard time being this idealistic 29-year-old and a
pragmatic county party chair. In 2012, I was a paid Obama staffer and I
got my ballot and I wasn't sure if I was going to vote for him because I
was so upset with seeing him govern as a moderate Democrat. And I talked
to my mentor at the time, and he told me that I can stand outside the
system and {[}complain{]}, or I can jump in and make the party reflect
what I want it to.

I made my choice. I voted for Obama, I dove into the party. So the way I
talk to a lot of my friends is that if we want to continue seeing
progress, we can try to burn the system down and see what happens, or we
can take our steps and get the progressive progression we want. And I
think the best thing that we can do, and the best thing that I've sold a
lot of my younger folks on, is that we elect Joe Biden and we hold him
accountable to govern as progressive as we possibly want him to be. That
we continue to take safe Democratic seats where there's maybe a moderate
and we put an A.O.C. in there. And we govern with the biggest coalition
that pushes our country as far progressive as we can.

\textbf{The Trump campaign tells reporters like me that the protests are
going to scare the Waukeshas of the world, that images of rioting and
looting are going to help them win people back. Is that possible?}

People are going to see the very limited amounts of rioting and the very
limited amounts of looting and weigh them against the thousands of
images of peaceful demonstrations. Even in our own county, we have had
dozens of peaceful, large demonstrations, over 100 people demonstrating
for Black Lives Matter. Yes, they're going to see these images of fire
and looting, but they're also going to remember driving through their
streets and seeing 100 people with masks and signs being super peaceful
and calm. More and more as we go along, the ads are becoming a little
less effective.

When you do six months of fearmongering, it's not going to work.

\begin{center}\rule{0.5\linewidth}{\linethickness}\end{center}

\textbf{Drop us a line!}

Image

\emph{We want to hear from our readers. Have a question? We'll try to
answer it. Have a comment? We're all ears. Email us at}
\href{mailto:onpolitics@nytimes.com}{\emph{onpolitics@nytimes.com}}\emph{.}

\begin{center}\rule{0.5\linewidth}{\linethickness}\end{center}

\hypertarget{from-opinion-americas-unkept-promises}{%
\subsection{From Opinion: America's unkept
promises}\label{from-opinion-americas-unkept-promises}}

Image

This morning, the New York Times editorial board commemorated the July 4
weekend with
\href{https://www.nytimes.com/2020/07/02/opinion/income-inequality-solutions.html}{a
rollout of the latest chapter} in
``\href{https://www.nytimes.com/2020/04/09/opinion/sunday/coronavirus-inequality-america.html}{The
America We Need}'' series, a Times Opinion project on economic
inequality. Throughout this weekend, the board wrote, ``we celebrate the
creation of the United States, though that project remains substantially
incomplete.''

``This year of crises has underscored the distance between the lofty
rhetoric of our founding documents and the persistent inequalities of
American life,'' the editorial says. ``This nation began as a set of
promises that it has yet to keep.''

The board recommends ``reversing the economic segregation of residential
life,'' which compounds economic and social inequality, and ``funding
schools based on the needs of the students, rather than the value of
parents' homes.'' It calls for measures like ``reducing, and in some
cases eliminating, occupational licensing requirements'' --- and much
more.

The essay acknowledges structural barriers that might prevent bold
reforms. Among them is the fact that the ``connection between government
and the governed is being strained by the growing divide between the
distribution of the population and the distribution of senators.''

Still,
\href{https://www.nytimes.com/2020/07/02/opinion/income-inequality-solutions.html}{the
editorial concludes}, the country must recommit itself to the
``difficult but essential work of ensuring all Americans have the
freedom to enjoy life and liberty, and to pursue happiness.''

\emph{--- Talmon Joseph Smith}

\begin{center}\rule{0.5\linewidth}{\linethickness}\end{center}

\hypertarget{-seriously}{%
\subsection{\ldots{} Seriously}\label{-seriously}}

A deep thought from yours truly:

\begin{center}\rule{0.5\linewidth}{\linethickness}\end{center}

\emph{Thanks for reading. On Politics is your guide to the political
news cycle, delivering clarity from the chaos.}

\emph{On Politics is also available as a newsletter.}
\href{https://www.nytimes.com/newsletters/politics}{\emph{Sign up here}}
\emph{to get it delivered to your inbox.}

\emph{Is there anything you think we're missing? Anything you want to
see more of? We'd love to hear from you. Email us at}
\href{mailto:onpolitics@nytimes.com}{\emph{onpolitics@nytimes.com}}\emph{.}

\hypertarget{our-2020-election-guide}{%
\section{Our 2020 Election Guide}\label{our-2020-election-guide}}

Updated July 31, 2020

\begin{itemize}
\item
  \begin{center}\rule{0.5\linewidth}{\linethickness}\end{center}

  \hypertarget{the-latest}{%
  \subsection{The Latest}\label{the-latest}}

  \begin{itemize}
  \tightlist
  \item
    President Trump's assault on the Postal Service is intersecting with
    his attacks on mail-in voting.
    \href{https://www.nytimes.com/2020/07/31/us/politics/trump-usps-mail-delays.html?action=click\&pgtype=Article\&state=default\&region=BELOW_MAIN_CONTENT\&context=storylines_guide}{Voting
    rights groups say it is a recipe for disaster.}
  \end{itemize}
\item
  \begin{center}\rule{0.5\linewidth}{\linethickness}\end{center}

  \hypertarget{bidens-vp-search}{%
  \subsection{Biden's V.P. Search}\label{bidens-vp-search}}

  \begin{itemize}
  \tightlist
  \item
    \href{https://www.nytimes.com/article/biden-vice-president-2020.html?action=click\&pgtype=Article\&state=default\&region=BELOW_MAIN_CONTENT\&context=storylines_guide}{Here
    are 13 women} who have been under consideration to be Joe Biden's
    running mate, and why each might be chosen --- and might not be.
  \end{itemize}
\item
  \begin{center}\rule{0.5\linewidth}{\linethickness}\end{center}

  \hypertarget{keep-up-with-our-coverage}{%
  \subsection{Keep Up With Our
  Coverage}\label{keep-up-with-our-coverage}}

  \begin{itemize}
  \tightlist
  \item
    Get an
    \href{https://www.nytimes.com/newsletters/politics?action=click\&pgtype=Article\&state=default\&region=BELOW_MAIN_CONTENT\&context=storylines_guide}{email}
    recapping the day's news
  \end{itemize}

  \begin{itemize}
  \tightlist
  \item
    Download our mobile app on
    \href{https://apps.apple.com/us/app/nytimes/id284862083?ls=1\&mat_click_id=5c79ae7455014fd1bd66b5610c05b8f2-20191112-16948\&referrer=mat_click_id\%3D5c79ae7455014fd1bd66b5610c05b8f2-20191112-16948\%26link_click_id\%3D722930677036718082}{iOS}
    and
    \href{http://a.localytics.com/android?id=com.nytimes.android\&referrer=utm_source\%3Dother_nyt_mobile_web\%26utm_medium\%3DWeb\%2520page\%26utm_term\%3DGeneral\%2520Mobile\%2520Page\%26utm_campaign\%3DNYT\%2520Mobile\%2520General\%2520Page}{Android}
    and turn on Breaking News and Politics alerts
  \end{itemize}
\end{itemize}

Advertisement

\protect\hyperlink{after-bottom}{Continue reading the main story}

\hypertarget{site-index}{%
\subsection{Site Index}\label{site-index}}

\hypertarget{site-information-navigation}{%
\subsection{Site Information
Navigation}\label{site-information-navigation}}

\begin{itemize}
\tightlist
\item
  \href{https://help.nytimes.com/hc/en-us/articles/115014792127-Copyright-notice}{©~2020~The
  New York Times Company}
\end{itemize}

\begin{itemize}
\tightlist
\item
  \href{https://www.nytco.com/}{NYTCo}
\item
  \href{https://help.nytimes.com/hc/en-us/articles/115015385887-Contact-Us}{Contact
  Us}
\item
  \href{https://www.nytco.com/careers/}{Work with us}
\item
  \href{https://nytmediakit.com/}{Advertise}
\item
  \href{http://www.tbrandstudio.com/}{T Brand Studio}
\item
  \href{https://www.nytimes.com/privacy/cookie-policy\#how-do-i-manage-trackers}{Your
  Ad Choices}
\item
  \href{https://www.nytimes.com/privacy}{Privacy}
\item
  \href{https://help.nytimes.com/hc/en-us/articles/115014893428-Terms-of-service}{Terms
  of Service}
\item
  \href{https://help.nytimes.com/hc/en-us/articles/115014893968-Terms-of-sale}{Terms
  of Sale}
\item
  \href{https://spiderbites.nytimes.com}{Site Map}
\item
  \href{https://help.nytimes.com/hc/en-us}{Help}
\item
  \href{https://www.nytimes.com/subscription?campaignId=37WXW}{Subscriptions}
\end{itemize}
