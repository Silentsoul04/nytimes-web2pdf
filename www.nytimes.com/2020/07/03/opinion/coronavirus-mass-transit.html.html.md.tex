Sections

SEARCH

\protect\hyperlink{site-content}{Skip to
content}\protect\hyperlink{site-index}{Skip to site index}

\href{https://myaccount.nytimes.com/auth/login?response_type=cookie\&client_id=vi}{}

\href{https://www.nytimes.com/section/todayspaper}{Today's Paper}

\href{/section/opinion}{Opinion}\textbar{}Mass Transit, and Cities,
Could Grind to a Halt Without Federal Aid

\href{https://nyti.ms/2BwYKBa}{https://nyti.ms/2BwYKBa}

\begin{itemize}
\item
\item
\item
\item
\item
\end{itemize}

Advertisement

\protect\hyperlink{after-top}{Continue reading the main story}

\href{/section/opinion}{Opinion}

Supported by

\protect\hyperlink{after-sponsor}{Continue reading the main story}

\hypertarget{mass-transit-and-cities-could-grind-to-a-halt-without-federal-aid}{%
\section{Mass Transit, and Cities, Could Grind to a Halt Without Federal
Aid}\label{mass-transit-and-cities-could-grind-to-a-halt-without-federal-aid}}

Reviving subway and bus services helped bring back cities before. It can
do so again.

By Nicole Gelinas

Ms. Gelinas is a contributing editor to the Manhattan Institute's City
Journal.

\begin{itemize}
\item
  July 3, 2020
\item
  \begin{itemize}
  \item
  \item
  \item
  \item
  \item
  \end{itemize}
\end{itemize}

\includegraphics{https://static01.nyt.com/images/2020/07/03/opinion/03Gelinas2/merlin_172341372_bbf3f9ef-8311-4939-90eb-02d0a2d1743c-articleLarge.jpg?quality=75\&auto=webp\&disable=upscale}

As Congress plans another round of economic rescue, it will have to take
a step that lawmakers from both parties have found distasteful for four
decades: federal operating aid for mass transit.

The pandemic is an existential crisis for transit. Patrick Foye,
chairman of New York's Metropolitan Transportation Authority, the
nation's largest provider, said the entity's fiscal situation was a
``four-alarm fire.'' The threat is far greater than after Sept. 11 and
the 2008 recession.

Even as ridership has plummeted by double-digit percentages, transit
agencies like the Washington Metropolitan Area Transit Authority and
Boston's Massachusetts Bay Transportation Authority must resume full
service as economies reopen. Otherwise, they risk overcrowded trains and
buses that do not allow for minimal social distancing.

Transit agencies have never before faced a situation where they must pay
to run full service with a fraction of revenue. This is devastating to
budgets. The New York M.T.A. faces a shortfall of \$14.3 billion over
two years on a \$34.5 billion budget. Washington's and Boston's transit
authorities and San Francisco's Bay Area Rapid Transit face commensurate
shortfalls, adjusted for size.

In April, Congress provided \$25 billion to transit agencies through the
CARES Act, including \$4 billion for the M.T.A. But this aid was
targeted to a monthslong shortfall, not a yearslong recovery.

If Congress doesn't provide more aid, the M.T.A. risks a downward
spiral. As transit agencies cut back service or raise fares,
white-collar workers and their employers will remain reluctant to come
back. Service workers with no choice but to use transit --- including
lower-wage Black and Hispanic people --- will face less reliable service
at a higher cost, shouldering delays and overcrowding.

Congress should save transit not for transit's sake, but to save cities.
Subways, buses and commuter rail make up the physical infrastructure
that enables urban life.

Before the coronavirus pandemic, of the 3.8 million commuters and
visitors who descended on Manhattan every day, three-quarters took a
subway, bus, train or ferry.

In Washington, Boston and San Francisco, more than a third of commuters
take public transit to work. In Chicago; Newark; Arlington, Va.; and
Philadelphia, it's more than a quarter. Transit riders, by staying off
the streets, enable others to drive to work without creating impassible
traffic jams.

We have seen this crisis before, after World War II, with the mass
marketing of the automobile. Cities saw a population exodus as
middle-class residents embraced the car and suburban life, and elected
officials began to neglect transit.

It turned out, though, that a big part of what makes cities attractive
is street life, which is incompatible with suburban-style car
dependency. In the early '80s, cities, thanks in large part to federal
funds, began rebuilding their transit systems and started regaining
population.

Covid-19 may pose just as grave a challenge for cities as did post-World
War II suburbanization, just in a compressed time frame. New York, with
less than 3 percent of the U.S. population, has 19 percent of the
fatalities.

Wealthier people are fleeing to suburbs, where they can have outdoor
space. Executives in industries from banking to tech have no idea when
or whether to require workers to come back to central-city offices five
days a week. But for Congress to give up on cities would be catastrophic
for America's creativity and productivity.

Covid-19 has hit cities so hard because of what they do so well: bring
people closely together for fun and profit. New York, for example,
generates more than 7 percent of the G.D.P., again outweighing
population share.

To save transit and save cities, Congress should do something it hasn't
liked to do since the mid-'80s: provide continuing aid for transit
operations.

\includegraphics{https://static01.nyt.com/images/2020/07/03/opinion/03Gelinas/03Gelinas-articleLarge.jpg?quality=75\&auto=webp\&disable=upscale}

Starting in the Reagan administration and continuing through the Clinton
years, lawmakers and the White House gradually eliminated '70s-era
crisis aid for larger transit systems (in cities with more than 200,000
residents). The theory was that riders and taxpayers should pay for
day-to-day service and that operating aid would only avoid the need for
fare hikes. At the time, the decision was sound. Cities such as New York
and Boston were rebounding from decades of economic decline, and their
growing tax and ridership bases could support fares and subsidies.

Congress and successive presidents reserved federal aid for
transformative infrastructure projects, such as new rail lines to
increase ridership.

Now these cities once again face population loss and deep deficits.
There is no point in investing in new infrastructure if agencies can't
afford to run service; indeed, the M.T.A. has already frozen its
infrastructure investments (and will need more aid there, too).

Riders --- particularly lower-paid essential workers --- cannot bear
large fare hikes.

Congress should solve the problem now, rather than ask transit systems
to keep coming back for more. Lawmakers should create a formula for
operating aid for the next half-decade, one tied to pre-pandemic
ridership as well as to post-pandemic recovery. As riders returned,
federal aid would automatically decline.

Finally, Congress can tie aid to cost reform, creating a commission to
assess why European and developed Asian cities can provide
higher-quality transit at a lower cost.

Congress and the White House should remember: In the two decades leading
up to 1970s, urban decline presaged a fragile national economy that grew
increasingly dependent on mortgage debt and environmental degradation.
Let's not make the same mistake for a different reason.

Nicole Gelinas is a contributing editor to the Manhattan Institute's
City Journal. She is writing a book about how over the past 40 years
transit saved New York City.

\emph{The Times is committed to publishing}
\href{https://www.nytimes.com/2019/01/31/opinion/letters/letters-to-editor-new-york-times-women.html}{\emph{a
diversity of letters}} \emph{to the editor. We'd like to hear what you
think about this or any of our articles. Here are some}
\href{https://help.nytimes.com/hc/en-us/articles/115014925288-How-to-submit-a-letter-to-the-editor}{\emph{tips}}\emph{.
And here's our email:}
\href{mailto:letters@nytimes.com}{\emph{letters@nytimes.com}}\emph{.}

\emph{Follow The New York Times Opinion section on}
\href{https://www.facebook.com/nytopinion}{\emph{Facebook}}\emph{,}
\href{http://twitter.com/NYTOpinion}{\emph{Twitter (@NYTopinion)}}
\emph{and}
\href{https://www.instagram.com/nytopinion/}{\emph{Instagram}}\emph{.}

Advertisement

\protect\hyperlink{after-bottom}{Continue reading the main story}

\hypertarget{site-index}{%
\subsection{Site Index}\label{site-index}}

\hypertarget{site-information-navigation}{%
\subsection{Site Information
Navigation}\label{site-information-navigation}}

\begin{itemize}
\tightlist
\item
  \href{https://help.nytimes.com/hc/en-us/articles/115014792127-Copyright-notice}{©~2020~The
  New York Times Company}
\end{itemize}

\begin{itemize}
\tightlist
\item
  \href{https://www.nytco.com/}{NYTCo}
\item
  \href{https://help.nytimes.com/hc/en-us/articles/115015385887-Contact-Us}{Contact
  Us}
\item
  \href{https://www.nytco.com/careers/}{Work with us}
\item
  \href{https://nytmediakit.com/}{Advertise}
\item
  \href{http://www.tbrandstudio.com/}{T Brand Studio}
\item
  \href{https://www.nytimes.com/privacy/cookie-policy\#how-do-i-manage-trackers}{Your
  Ad Choices}
\item
  \href{https://www.nytimes.com/privacy}{Privacy}
\item
  \href{https://help.nytimes.com/hc/en-us/articles/115014893428-Terms-of-service}{Terms
  of Service}
\item
  \href{https://help.nytimes.com/hc/en-us/articles/115014893968-Terms-of-sale}{Terms
  of Sale}
\item
  \href{https://spiderbites.nytimes.com}{Site Map}
\item
  \href{https://help.nytimes.com/hc/en-us}{Help}
\item
  \href{https://www.nytimes.com/subscription?campaignId=37WXW}{Subscriptions}
\end{itemize}
