Sections

SEARCH

\protect\hyperlink{site-content}{Skip to
content}\protect\hyperlink{site-index}{Skip to site index}

\href{https://myaccount.nytimes.com/auth/login?response_type=cookie\&client_id=vi}{}

\href{https://www.nytimes.com/section/todayspaper}{Today's Paper}

\href{/section/opinion}{Opinion}\textbar{}The Mystery of High Stock
Prices

\href{https://nyti.ms/31DTy9g}{https://nyti.ms/31DTy9g}

\begin{itemize}
\item
\item
\item
\item
\item
\end{itemize}

Advertisement

\protect\hyperlink{after-top}{Continue reading the main story}

\href{/section/opinion}{Opinion}

Supported by

\protect\hyperlink{after-sponsor}{Continue reading the main story}

\hypertarget{the-mystery-of-high-stock-prices}{%
\section{The Mystery of High Stock
Prices}\label{the-mystery-of-high-stock-prices}}

Why is the market doing so well when the economy is doing so poorly?

\href{https://www.nytimes.com/topic/person/steven-rattner}{\includegraphics{https://static01.nyt.com/images/2013/10/04/opinion/rattner-contributor/rattner-contributor-thumbLarge-v5.png}}

By \href{https://www.nytimes.com/topic/person/steven-rattner}{Steven
Rattner}

Mr. Rattner served as counselor to the Treasury secretary in the Obama
administration.

\begin{itemize}
\item
  July 3, 2020
\item
  \begin{itemize}
  \item
  \item
  \item
  \item
  \item
  \end{itemize}
\end{itemize}

\includegraphics{https://static01.nyt.com/images/2020/07/03/opinion/03Rattner2/merlin_173964405_ebba17f0-08f3-44ee-a6e1-b9797c871427-articleLarge.jpg?quality=75\&auto=webp\&disable=upscale}

From the Department of Curiosities: On Tuesday, as the number of new
coronavirus cases continued to spike to record levels, the stock market
closed out its strongest quarter in more than two decades.

That was just one stop for the equity markets on a spring roller coaster
ride, three months that saw the fastest 30 percent
\href{https://www.cnbc.com/2020/03/23/this-was-the-fastest-30percent-stock-market-decline-ever.html}{decline}
in stock prices in history, followed by the fastest 50-day
\href{https://www.cnbc.com/2020/06/03/this-is-the-greatest-50-day-rally-in-the-history-of-the-sp-500.html}{increase}
on record.

This year's volatility may be extreme, but it's only the latest of many
seeming disconnects between stocks and the economy. In March 2009, for
example, while reported monthly job losses were topping 700,000, share
prices abruptly ended their 17-month decline and began a recovery that
essentially lasted until the virus arrived.

\hypertarget{in-2009-stocks-turned-up-even-as-economy-was-bleeding-jobs}{%
\subsection{In 2009, Stocks Turned Up Even as Economy Was Bleeding
Jobs}\label{in-2009-stocks-turned-up-even-as-economy-was-bleeding-jobs}}

Monthly change in

nonfarm payrolls, in

thousands

S\&P 500

1.6k

+400

1.4k

0

1.2k

While jobs

were

declining ...

... stocks

were

climbing.

-400

1k

800

-800

2008

`10

`12

2008

`10

`12

Monthly change in nonfarm payrolls,

in thousands

S\&P 500

1.6k

+400

1.4k

0

1.2k

-400

1k

While jobs

were declining ...

... stocks were

recovering.

800

-800

2008

2010

2012

2008

2010

2012

Source: Bureau of Labor Statistics, Standard \& Poor's

By The New York Times

What gives? Why has an economy that has experienced the biggest collapse
since the Great Depression not --- at least to date --- inflicted any
lasting damage on a market that is often expected to reflect the state
of the economy or, at least, of corporate profits?

Some hold the view that the economy's troubles will be short-lived; a
V-shaped recovery will soon unfold and the stock market is merely
looking ahead. Others cite the upsurge in buying by small individual
investors.

My vote for the most significant driver of stock prices is the huge
amount of liquidity that the Federal Reserve has injected into the
financial system, in an effort to counteract the depressive economic
impact of the virus.

That has pushed interest rates to record lows, turning money market
funds, bonds and other fixed-income instruments into low-returning
investments. The Standard \& Poor's index of 500 stocks, for example,
currently has a dividend yield of 1.9 percent, compared with 0.7 percent
for 10-year Treasury notes.

Unusually, an investor can now make more in current income from stocks
than from high-quality fixed-income securities while participating in
any future appreciation in share prices. (Yes, while stocks can also go
down, over the long term, they have always appreciated.)

Coincidence or not, the day the
\href{https://www.nytimes.com/2020/03/23/business/economy/coronavirus-fed-bond-buying.html}{Fed
announced} a massive injection of liquidity, the plunge in the market
abated and the extraordinary recovery in stocks began.

``Don't fight the Fed'' has been a mantra for investors for decades.
During the tenure of Alan Greenspan as Fed chairman, the notion that the
Fed would provide a fire hose of liquidity whenever a crisis threatened
became known as the ``Greenspan put.''

In fairness, the Fed is not the only factor influencing the market.
Individual investors, known for their often poor timing of entry and
exit points, have been trading actively, aided by commissions that major
online brokers have
\href{https://www.nytimes.com/2019/10/01/your-money/charles-schwab-free-trades.html}{dropped
to zero}.

That has created some weird anomalies: After pandemic losses
\href{https://www.ft.com/content/b592847a-2061-4460-8aa5-3b22a2153210}{drove
Hertz shares} below \$1 and the company filed for bankruptcy, small
investors piled in and sent the stock briefly above \$5, even though
shareholders rarely receive material proceeds from a bankruptcy.

However, as a whole, data on fund flows do not show --- at least yet ---
enough new retail money coming into the market to materially account for
its quick and strong recovery.

And the overall strong performance of stocks masks the fact that the
market has recognized that profits of fast-growing technology companies
have not been significantly hurt by the pandemic while more cyclical
companies in manufacturing, retail and the like are suffering mightily.

Since the market peaked on Feb. 19, the tech-heavy Nasdaq index is up
four percent essentially unchanged while the Dow Jones average --- more
oriented toward cyclical companies --- has fallen by 12 percent.

Nonetheless, many
\href{https://www.bloomberg.com/news/articles/2020-05-13/wall-street-heavyweights-are-sounding-alarm-about-stock-prices?sref=qN0DZypA}{legendary
investors} --- from Stanley Druckenmiller to Paul Tudor Jones --- remain
deeply concerned about the gap between share prices and economic
fundamentals. Warren Buffett, who made billions for his company,
Berkshire Hathaway, by investing heavily during the financial crisis,
appears to have mostly stayed on the sidelines.

In recent days, the market has seemed sympathetic to their view. As
virus cases have begun spiking, stock prices have shuddered, as they did
on Friday, June 26. But so far, at least, they have quickly stabilized,
albeit below the highs of early June.

So at the moment, as much as President Trump would like to think
otherwise, lofty stock prices are not a sign of a strong economy.

And in the long run, the view of professional investors that share
prices must eventually align with economic fundamentals will prevail.

To me, those fundamentals look scary. The new climb in virus cases
threatens to force shutdowns and delay reopenings. At best, the recovery
is likely to be lengthy, particularly for industries including travel
and hospitality.

Without a robust economy, corporate profits are unlikely to recover
fully, eventually pulling down stocks. To know when that recovery might
occur, keep a close eye on the path of the virus.

But also, don't forget to watch the Fed.

Advertisement

\protect\hyperlink{after-bottom}{Continue reading the main story}

\hypertarget{site-index}{%
\subsection{Site Index}\label{site-index}}

\hypertarget{site-information-navigation}{%
\subsection{Site Information
Navigation}\label{site-information-navigation}}

\begin{itemize}
\tightlist
\item
  \href{https://help.nytimes.com/hc/en-us/articles/115014792127-Copyright-notice}{©~2020~The
  New York Times Company}
\end{itemize}

\begin{itemize}
\tightlist
\item
  \href{https://www.nytco.com/}{NYTCo}
\item
  \href{https://help.nytimes.com/hc/en-us/articles/115015385887-Contact-Us}{Contact
  Us}
\item
  \href{https://www.nytco.com/careers/}{Work with us}
\item
  \href{https://nytmediakit.com/}{Advertise}
\item
  \href{http://www.tbrandstudio.com/}{T Brand Studio}
\item
  \href{https://www.nytimes.com/privacy/cookie-policy\#how-do-i-manage-trackers}{Your
  Ad Choices}
\item
  \href{https://www.nytimes.com/privacy}{Privacy}
\item
  \href{https://help.nytimes.com/hc/en-us/articles/115014893428-Terms-of-service}{Terms
  of Service}
\item
  \href{https://help.nytimes.com/hc/en-us/articles/115014893968-Terms-of-sale}{Terms
  of Sale}
\item
  \href{https://spiderbites.nytimes.com}{Site Map}
\item
  \href{https://help.nytimes.com/hc/en-us}{Help}
\item
  \href{https://www.nytimes.com/subscription?campaignId=37WXW}{Subscriptions}
\end{itemize}
