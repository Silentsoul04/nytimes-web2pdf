Sections

SEARCH

\protect\hyperlink{site-content}{Skip to
content}\protect\hyperlink{site-index}{Skip to site index}

\href{https://www.nytimes.com/section/sports/baseball}{Baseball}

\href{https://myaccount.nytimes.com/auth/login?response_type=cookie\&client_id=vi}{}

\href{https://www.nytimes.com/section/todayspaper}{Today's Paper}

\href{/section/sports/baseball}{Baseball}\textbar{}Cleveland Indians Say
They Will `Determine Best Path Forward' on Name

\url{https://nyti.ms/38spfUz}

\begin{itemize}
\item
\item
\item
\item
\item
\end{itemize}

Advertisement

\protect\hyperlink{after-top}{Continue reading the main story}

Supported by

\protect\hyperlink{after-sponsor}{Continue reading the main story}

\hypertarget{cleveland-indians-say-they-will-determine-best-path-forward-on-name}{%
\section{Cleveland Indians Say They Will `Determine Best Path Forward'
on
Name}\label{cleveland-indians-say-they-will-determine-best-path-forward-on-name}}

The announcement came hours after the Washington Redskins of the N.F.L.
made a similar statement. Both teams have been the subject of protests
and criticism from Native Americans and others.

\includegraphics{https://static01.nyt.com/images/2020/07/03/sports/03indians2/merlin_170431323_329cb49b-bccd-4190-a371-883456e69133-articleLarge.jpg?quality=75\&auto=webp\&disable=upscale}

\href{https://www.nytimes.com/by/david-waldstein}{\includegraphics{https://static01.nyt.com/images/2018/02/20/multimedia/author-david-waldstein/author-david-waldstein-thumbLarge.jpg}}

By \href{https://www.nytimes.com/by/david-waldstein}{David Waldstein}

\begin{itemize}
\item
  July 3, 2020
\item
  \begin{itemize}
  \item
  \item
  \item
  \item
  \item
  \end{itemize}
\end{itemize}

After decades of resisting calls to change their team name, the
Cleveland Indians announced on Friday that they were willing to engage
in discussions about whether the name is appropriate in the wake of
national calls for social justice and reform.

``We are committed to engaging our community and appropriate
stakeholders to determine the best path forward with regard to our team
name,'' the team said in a statement released Friday night.

The announcement came hours after the Washington Redskins of the
National Football League
\href{https://www.nytimes.com/2020/07/03/sports/football/washington-redskins-nickname-nfl.html}{made
a similar announcement}, vowing to undertake a ``thorough review'' of
their team name, which many consider to be a racial slur against Native
Americans.

Both team names are considered offensive by many Native Americans, who
oppose their heritage's being used as an identity for sports teams and
their mascots.

The news that Cleveland is willing to discuss changing the name was
greeted with surprise and enthusiasm from Philip Yenyo, the executive
director of the American Indian Movement of Ohio. Mr. Yenyo has been
protesting the name since 1991, often in the face of withering abuse
from fans as they enter the gates of the team's stadium on game days.

``Wow,'' Mr. Yenyo said on Friday in a telephone interview. ``This is a
good step.''

Cleveland has used the same name since 1915, often accompanied by a
caricature of a Native American, known as Chief Wahoo. The team
\href{https://www.nytimes.com/2018/04/09/sports/baseball/cleveland-indians-chief-wahoo-protests.html}{phased
out the Chief Wahoo logo last year}, removing it from their uniforms and
from walls and banners in the stadium.

The logo was still used on items for sale at the team store last year,
however, and Mr. Yenyo called on the club to cease manufacturing such
items and selling them in perpetuity.

``We've been out there on the street chanting, `Change the name, change
the logo,' for years,'' he said. ``They didn't seem to hear the first
part of that chant, but maybe now they are listening. There are a lot of
issues we are fighting, but the name is the main one.''

\includegraphics{https://static01.nyt.com/images/2020/07/03/sports/03cleveland-1/merlin_157509669_93b81f67-b926-4df0-a43e-fb54d38c5d43-articleLarge.jpg?quality=75\&auto=webp\&disable=upscale}

The Indians have said that the name was originally intended to honor a
former Native American player, Louis Sockalexis, who played for the
Cleveland Spiders, a major league club, in the 19th century. Some have
suggested that Cleveland adopt Spiders as a replacement.

In their statement on Friday, the team cited the ``recent social unrest
in our community and in our country'' --- a reference to the nationwide
protests in the wake of the killing of George Floyd in police custody in
Minneapolis --- as spurring its revisiting of the name.

``Our organization fully recognizes our team name is among the most
visible ways in which we connect with the community,'' the statement
said. ``We have had ongoing discussions organizationally on these
issues.''

The team did not name any specific individuals or groups as the
``appropriate stakeholders'' it planned to engage, but it could include
sponsors and financial partners of the team.

The Washington Redskins' announcement came a day after FedEx, the
company that holds the naming rights to their stadium, asked the team
owner Daniel Snyder to change the name.

Many Cleveland fans are emotionally attached to the name, and some will
most likely be consulted in the process. A name change is not definite.

Mr. Yenyo said he had been calling team offices for weeks hoping for a
meeting, and would like to be included in any discussions.

``We want to be there,'' he said. ``We have to strike while the iron is
hot.''

Advertisement

\protect\hyperlink{after-bottom}{Continue reading the main story}

\hypertarget{site-index}{%
\subsection{Site Index}\label{site-index}}

\hypertarget{site-information-navigation}{%
\subsection{Site Information
Navigation}\label{site-information-navigation}}

\begin{itemize}
\tightlist
\item
  \href{https://help.nytimes.com/hc/en-us/articles/115014792127-Copyright-notice}{©~2020~The
  New York Times Company}
\end{itemize}

\begin{itemize}
\tightlist
\item
  \href{https://www.nytco.com/}{NYTCo}
\item
  \href{https://help.nytimes.com/hc/en-us/articles/115015385887-Contact-Us}{Contact
  Us}
\item
  \href{https://www.nytco.com/careers/}{Work with us}
\item
  \href{https://nytmediakit.com/}{Advertise}
\item
  \href{http://www.tbrandstudio.com/}{T Brand Studio}
\item
  \href{https://www.nytimes.com/privacy/cookie-policy\#how-do-i-manage-trackers}{Your
  Ad Choices}
\item
  \href{https://www.nytimes.com/privacy}{Privacy}
\item
  \href{https://help.nytimes.com/hc/en-us/articles/115014893428-Terms-of-service}{Terms
  of Service}
\item
  \href{https://help.nytimes.com/hc/en-us/articles/115014893968-Terms-of-sale}{Terms
  of Sale}
\item
  \href{https://spiderbites.nytimes.com}{Site Map}
\item
  \href{https://help.nytimes.com/hc/en-us}{Help}
\item
  \href{https://www.nytimes.com/subscription?campaignId=37WXW}{Subscriptions}
\end{itemize}
