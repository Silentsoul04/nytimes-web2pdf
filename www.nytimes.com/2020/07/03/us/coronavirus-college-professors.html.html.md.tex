Sections

SEARCH

\protect\hyperlink{site-content}{Skip to
content}\protect\hyperlink{site-index}{Skip to site index}

\href{https://www.nytimes.com/section/us}{U.S.}

\href{https://myaccount.nytimes.com/auth/login?response_type=cookie\&client_id=vi}{}

\href{https://www.nytimes.com/section/todayspaper}{Today's Paper}

\href{/section/us}{U.S.}\textbar{}Colleges Face Rising Revolt by
Professors

\url{https://nyti.ms/3eVjAZr}

\begin{itemize}
\item
\item
\item
\item
\item
\item
\end{itemize}

\href{https://www.nytimes.com/news-event/coronavirus?action=click\&pgtype=Article\&state=default\&region=TOP_BANNER\&context=storylines_menu}{The
Coronavirus Outbreak}

\begin{itemize}
\tightlist
\item
  live\href{https://www.nytimes.com/2020/08/02/world/coronavirus-updates.html?action=click\&pgtype=Article\&state=default\&region=TOP_BANNER\&context=storylines_menu}{Latest
  Updates}
\item
  \href{https://www.nytimes.com/interactive/2020/us/coronavirus-us-cases.html?action=click\&pgtype=Article\&state=default\&region=TOP_BANNER\&context=storylines_menu}{Maps
  and Cases}
\item
  \href{https://www.nytimes.com/interactive/2020/science/coronavirus-vaccine-tracker.html?action=click\&pgtype=Article\&state=default\&region=TOP_BANNER\&context=storylines_menu}{Vaccine
  Tracker}
\item
  \href{https://www.nytimes.com/interactive/2020/07/29/us/schools-reopening-coronavirus.html?action=click\&pgtype=Article\&state=default\&region=TOP_BANNER\&context=storylines_menu}{What
  School May Look Like}
\item
  \href{https://www.nytimes.com/live/2020/07/31/business/stock-market-today-coronavirus?action=click\&pgtype=Article\&state=default\&region=TOP_BANNER\&context=storylines_menu}{Economy}
\end{itemize}

Advertisement

\protect\hyperlink{after-top}{Continue reading the main story}

Supported by

\protect\hyperlink{after-sponsor}{Continue reading the main story}

\hypertarget{colleges-face-rising-revolt-by-professors}{%
\section{Colleges Face Rising Revolt by
Professors}\label{colleges-face-rising-revolt-by-professors}}

Most universities plan to bring students back to campus. But many of
their teachers are concerned about joining them.

\includegraphics{https://static01.nyt.com/images/2020/07/02/us/00virus-professors-01/merlin_174135804_9b07b1cd-40eb-4f90-9278-9e1893cfc34b-articleLarge.jpg?quality=75\&auto=webp\&disable=upscale}

\href{https://www.nytimes.com/by/anemona-hartocollis}{\includegraphics{https://static01.nyt.com/images/2018/06/13/multimedia/author-anemona-hartocollis/author-anemona-hartocollis-thumbLarge-v3.jpg}}

By \href{https://www.nytimes.com/by/anemona-hartocollis}{Anemona
Hartocollis}

\begin{itemize}
\item
  Published July 3, 2020Updated July 4, 2020
\item
  \begin{itemize}
  \item
  \item
  \item
  \item
  \item
  \item
  \end{itemize}
\end{itemize}

\href{https://www.nytimes.com/2020/07/03/your-money/students-unemployment-insurance-coronavirus.html}{College
students} across the country have been warned that campus life will look
drastically different in the fall, with temperature checks at academic
buildings, masks in half-empty lecture halls and maybe no football
games.

What they might not expect: a lack of professors in the classroom.

Thousands of instructors at American colleges and universities have told
administrators in recent days that they are unwilling to resume
in-person classes because of
\href{https://www.nytimes.com/news-event/coronavirus}{the pandemic}.

\href{https://www.chronicle.com/article/Here-s-a-List-of-Colleges-/248626}{More
than three-quarters of colleges and universities} have decided students
can return to campus this fall. But they face a growing faculty revolt.

``Until there's a vaccine, I'm not setting foot on campus,'' said Dana
Ward, 70, an emeritus professor of political studies at Pitzer College
in Claremont, Calif., who teaches a class in anarchist history and
thought. ``Going into the classroom is like playing Russian roulette.''

This comes as major outbreaks have hit college towns this summer, spread
by partying students and practicing athletes.

In an indication of how fluid the situation is, the University of
Southern California said late Wednesday that
\href{https://www.provost.usc.edu/letter-on-student-housing-and-course-schedules/}{``an
alarming spike in coronavirus cases''} had prompted it to reverse an
earlier decision to encourage attending classes in person.

With more than a month before schools start reopening, it is hard to
predict how many professors will refuse to teach face to face in the
fall. But schools and professors are planning ahead.

A Cornell University survey of its faculty found that about one-third
were ``not interested in teaching classes in person,'' one-third were
``open to doing it if conditions were deemed to be safe,'' and about
one-third were ``willing and anxious to teach in person,'' said Michael
Kotlikoff, Cornell's provost.

Faculty members at institutions including Penn State, the University of
Illinois, Notre Dame and the State University of New York have signed
petitions complaining that they are not being consulted and are being
pushed back into classrooms too fast.

\hypertarget{latest-updates-global-coronavirus-outbreak}{%
\section{\texorpdfstring{\href{https://www.nytimes.com/2020/08/01/world/coronavirus-covid-19.html?action=click\&pgtype=Article\&state=default\&region=MAIN_CONTENT_1\&context=storylines_live_updates}{Latest
Updates: Global Coronavirus
Outbreak}}{Latest Updates: Global Coronavirus Outbreak}}\label{latest-updates-global-coronavirus-outbreak}}

Updated 2020-08-02T17:52:35.962Z

\begin{itemize}
\tightlist
\item
  \href{https://www.nytimes.com/2020/08/01/world/coronavirus-covid-19.html?action=click\&pgtype=Article\&state=default\&region=MAIN_CONTENT_1\&context=storylines_live_updates\#link-34047410}{The
  U.S. reels as July cases more than double the total of any other
  month.}
\item
  \href{https://www.nytimes.com/2020/08/01/world/coronavirus-covid-19.html?action=click\&pgtype=Article\&state=default\&region=MAIN_CONTENT_1\&context=storylines_live_updates\#link-780ec966}{Top
  U.S. officials work to break an impasse over the federal jobless
  benefit.}
\item
  \href{https://www.nytimes.com/2020/08/01/world/coronavirus-covid-19.html?action=click\&pgtype=Article\&state=default\&region=MAIN_CONTENT_1\&context=storylines_live_updates\#link-2bc8948}{Its
  outbreak untamed, Melbourne goes into even greater lockdown.}
\end{itemize}

\href{https://www.nytimes.com/2020/08/01/world/coronavirus-covid-19.html?action=click\&pgtype=Article\&state=default\&region=MAIN_CONTENT_1\&context=storylines_live_updates}{See
more updates}

More live coverage:
\href{https://www.nytimes.com/live/2020/07/31/business/stock-market-today-coronavirus?action=click\&pgtype=Article\&state=default\&region=MAIN_CONTENT_1\&context=storylines_live_updates}{Markets}

The University of Illinois at Urbana-Champaign campus is known for its
lively social scene, says a faculty petition. To expect more than 50,000
students to behave according to public health guidelines, it goes on,
``would be to ignore reality.''

At Penn State, an open letter signed by more than 1,000 faculty members
demands that the university ``affirm the autonomy of instructors in
deciding whether to teach classes, attend meetings and hold office hours
remotely, in person or in some hybrid mode.'' The letter also asks for
faculty members to be able to change their mode of teaching at any time,
and not to be obligated to disclose personal health information as a
condition of teaching online.

``I shudder at the prospect of teaching in a room filled with
asymptomatic superspreaders,'' wrote Paul M. Kellermann, 62, an English
professor at Penn State,
\href{https://www.esquire.com/news-politics/a32973676/penn-state-university-covid-19-petition-professors/}{in
an essay for Esquire magazine}, proclaiming that ``1,000 of my
colleagues agree.'' Those colleagues have demanded that the university
give them a choice of doing their jobs online or in person.

University officials say they are taking all the right precautions, and
that the bottom line is that face-to-face classes are what students and
their families --- and even most faculty members --- want. Rachel Pell,
a spokeswoman for Penn State, said the petition signers there
represented only about 12 percent of the 9,000-member full- and
part-time faculty. ``Our expectation is that faculty who are able to
teach will return to the classroom as part of a flexible approach,'' she
said, noting that those who are at high risk or live with someone at
risk can request adjustments.

\includegraphics{https://static01.nyt.com/images/2020/07/02/us/00virus-professors-03/merlin_174131970_3a01ec6e-04a2-439d-ba2b-01d67e29ef8a-articleLarge.jpg?quality=75\&auto=webp\&disable=upscale}

Driving some of the concern is the fact that tenure-track professors
skew significantly older than the wider U.S. labor force ---
\href{https://www.insidehighered.com/quicktakes/2020/01/27/aging-faculty}{37
percent are 55 or older}, compared with 23 percent of workers in general
--- and they are more than twice as likely as other workers to stay on
the job past 65, when they would be at increased risk of adverse health
effects from the virus.

Many younger professors have concerns as well, including about
underlying health conditions, taking care of children
\href{https://www.nytimes.com/2020/06/26/us/coronavirus-schools-reopen-fall.html}{who
might not be in school full-time this fall}, and not wanting to become a
danger to their older relatives. Some are angry that their schools are
making a return to classrooms the default option. And those who are not
tenured said they felt especially vulnerable if they asked for
accommodations.

Many professors are calling for a sweeping no-questions-asked policy for
those who want to teach remotely, saying that anything less is a
violation of their privacy and their family's privacy. But many
universities are turning to their human resources departments to make
decisions case by case.

Anna Curtis, an associate professor of criminology at the State
University of New York, Cortland, asked to be allowed to teach remotely
from home so she could care for her 4-year-old son. She said she was
worried about what she would do if he were sent home from day care for
ordinary things like sniffles and a fever that could be seen as possible
signs of Covid-19, and she did not want to constantly be scrambling to
find child care during a pandemic. Her request was denied, she said.

The university's human resources department, she said, told her that
caring for a child did not qualify as a reason to stay home under the
federal Americans with Disabilities Act, and that she would have to take
family leave.

``But that doesn't happen until the sickness happens,'' she said. Going
in and out of virtual mode will be disruptive to both her and her
students, she said, adding, ``It's a parent penalty, and most of the
time it's the women doing the primary care.''

Image

Stephanie Silvera, a professor at Montclair State University, said that
many of the school's students were health care workers at heightened
risk of contracting the coronavirus.Credit...Tony Cenicola/The New York
Times

Stephanie Silvera, 45, an epidemiology professor at Montclair State
University in New Jersey, said she withdrew from a planning committee in
frustration after she could not get the other members, who were
administrators, to focus on deciding which classes needed to be taught
in person and which ones could be done online.

Many students at the university are commuters, and work in the health
care industry, Dr. Silvera said, heightening the risks of their
contracting the virus and passing it to the faculty.

Joseph Brennan, a spokesman for Montclair State, said that another group
was looking at pedagogical issues, and that the university was making
getting back to in-person classes a high priority.

``Our students generally feel that they learn better in person,'' Mr.
Brennan said. ``We do not want to be a 100 percent online university.''

\href{https://www.nytimes.com/news-event/coronavirus?action=click\&pgtype=Article\&state=default\&region=MAIN_CONTENT_3\&context=storylines_faq}{}

\hypertarget{the-coronavirus-outbreak-}{%
\subsubsection{The Coronavirus Outbreak
›}\label{the-coronavirus-outbreak-}}

\hypertarget{frequently-asked-questions}{%
\paragraph{Frequently Asked
Questions}\label{frequently-asked-questions}}

Updated July 27, 2020

\begin{itemize}
\item ~
  \hypertarget{should-i-refinance-my-mortgage}{%
  \paragraph{Should I refinance my
  mortgage?}\label{should-i-refinance-my-mortgage}}

  \begin{itemize}
  \tightlist
  \item
    \href{https://www.nytimes.com/article/coronavirus-money-unemployment.html?action=click\&pgtype=Article\&state=default\&region=MAIN_CONTENT_3\&context=storylines_faq}{It
    could be a good idea,} because mortgage rates have
    \href{https://www.nytimes.com/2020/07/16/business/mortgage-rates-below-3-percent.html?action=click\&pgtype=Article\&state=default\&region=MAIN_CONTENT_3\&context=storylines_faq}{never
    been lower.} Refinancing requests have pushed mortgage applications
    to some of the highest levels since 2008, so be prepared to get in
    line. But defaults are also up, so if you're thinking about buying a
    home, be aware that some lenders have tightened their standards.
  \end{itemize}
\item ~
  \hypertarget{what-is-school-going-to-look-like-in-september}{%
  \paragraph{What is school going to look like in
  September?}\label{what-is-school-going-to-look-like-in-september}}

  \begin{itemize}
  \tightlist
  \item
    It is unlikely that many schools will return to a normal schedule
    this fall, requiring the grind of
    \href{https://www.nytimes.com/2020/06/05/us/coronavirus-education-lost-learning.html?action=click\&pgtype=Article\&state=default\&region=MAIN_CONTENT_3\&context=storylines_faq}{online
    learning},
    \href{https://www.nytimes.com/2020/05/29/us/coronavirus-child-care-centers.html?action=click\&pgtype=Article\&state=default\&region=MAIN_CONTENT_3\&context=storylines_faq}{makeshift
    child care} and
    \href{https://www.nytimes.com/2020/06/03/business/economy/coronavirus-working-women.html?action=click\&pgtype=Article\&state=default\&region=MAIN_CONTENT_3\&context=storylines_faq}{stunted
    workdays} to continue. California's two largest public school
    districts --- Los Angeles and San Diego --- said on July 13, that
    \href{https://www.nytimes.com/2020/07/13/us/lausd-san-diego-school-reopening.html?action=click\&pgtype=Article\&state=default\&region=MAIN_CONTENT_3\&context=storylines_faq}{instruction
    will be remote-only in the fall}, citing concerns that surging
    coronavirus infections in their areas pose too dire a risk for
    students and teachers. Together, the two districts enroll some
    825,000 students. They are the largest in the country so far to
    abandon plans for even a partial physical return to classrooms when
    they reopen in August. For other districts, the solution won't be an
    all-or-nothing approach.
    \href{https://bioethics.jhu.edu/research-and-outreach/projects/eschool-initiative/school-policy-tracker/}{Many
    systems}, including the nation's largest, New York City, are
    devising
    \href{https://www.nytimes.com/2020/06/26/us/coronavirus-schools-reopen-fall.html?action=click\&pgtype=Article\&state=default\&region=MAIN_CONTENT_3\&context=storylines_faq}{hybrid
    plans} that involve spending some days in classrooms and other days
    online. There's no national policy on this yet, so check with your
    municipal school system regularly to see what is happening in your
    community.
  \end{itemize}
\item ~
  \hypertarget{is-the-coronavirus-airborne}{%
  \paragraph{Is the coronavirus
  airborne?}\label{is-the-coronavirus-airborne}}

  \begin{itemize}
  \tightlist
  \item
    The coronavirus
    \href{https://www.nytimes.com/2020/07/04/health/239-experts-with-one-big-claim-the-coronavirus-is-airborne.html?action=click\&pgtype=Article\&state=default\&region=MAIN_CONTENT_3\&context=storylines_faq}{can
    stay aloft for hours in tiny droplets in stagnant air}, infecting
    people as they inhale, mounting scientific evidence suggests. This
    risk is highest in crowded indoor spaces with poor ventilation, and
    may help explain super-spreading events reported in meatpacking
    plants, churches and restaurants.
    \href{https://www.nytimes.com/2020/07/06/health/coronavirus-airborne-aerosols.html?action=click\&pgtype=Article\&state=default\&region=MAIN_CONTENT_3\&context=storylines_faq}{It's
    unclear how often the virus is spread} via these tiny droplets, or
    aerosols, compared with larger droplets that are expelled when a
    sick person coughs or sneezes, or transmitted through contact with
    contaminated surfaces, said Linsey Marr, an aerosol expert at
    Virginia Tech. Aerosols are released even when a person without
    symptoms exhales, talks or sings, according to Dr. Marr and more
    than 200 other experts, who
    \href{https://academic.oup.com/cid/article/doi/10.1093/cid/ciaa939/5867798}{have
    outlined the evidence in an open letter to the World Health
    Organization}.
  \end{itemize}
\item ~
  \hypertarget{what-are-the-symptoms-of-coronavirus}{%
  \paragraph{What are the symptoms of
  coronavirus?}\label{what-are-the-symptoms-of-coronavirus}}

  \begin{itemize}
  \tightlist
  \item
    Common symptoms
    \href{https://www.nytimes.com/article/symptoms-coronavirus.html?action=click\&pgtype=Article\&state=default\&region=MAIN_CONTENT_3\&context=storylines_faq}{include
    fever, a dry cough, fatigue and difficulty breathing or shortness of
    breath.} Some of these symptoms overlap with those of the flu,
    making detection difficult, but runny noses and stuffy sinuses are
    less common.
    \href{https://www.nytimes.com/2020/04/27/health/coronavirus-symptoms-cdc.html?action=click\&pgtype=Article\&state=default\&region=MAIN_CONTENT_3\&context=storylines_faq}{The
    C.D.C. has also} added chills, muscle pain, sore throat, headache
    and a new loss of the sense of taste or smell as symptoms to look
    out for. Most people fall ill five to seven days after exposure, but
    symptoms may appear in as few as two days or as many as 14 days.
  \end{itemize}
\item ~
  \hypertarget{does-asymptomatic-transmission-of-covid-19-happen}{%
  \paragraph{Does asymptomatic transmission of Covid-19
  happen?}\label{does-asymptomatic-transmission-of-covid-19-happen}}

  \begin{itemize}
  \tightlist
  \item
    So far, the evidence seems to show it does. A widely cited
    \href{https://www.nature.com/articles/s41591-020-0869-5}{paper}
    published in April suggests that people are most infectious about
    two days before the onset of coronavirus symptoms and estimated that
    44 percent of new infections were a result of transmission from
    people who were not yet showing symptoms. Recently, a top expert at
    the World Health Organization stated that transmission of the
    coronavirus by people who did not have symptoms was ``very rare,''
    \href{https://www.nytimes.com/2020/06/09/world/coronavirus-updates.html?action=click\&pgtype=Article\&state=default\&region=MAIN_CONTENT_3\&context=storylines_faq\#link-1f302e21}{but
    she later walked back that statement.}
  \end{itemize}
\end{itemize}

Instructors at Georgia Tech said they were told last week that they
would either have to be 65 or older or have one of seven specific health
conditions, like diabetes or chronic lung disease, to qualify to teach
remotely.

Professors at the university are being urged to help ``achieve our
objective to have a fall term that approximates normal residential
instruction and is cognizant of public health requirements,'' according
to a PowerPoint presentation circulated among the faculty.

Image

``I don't feel safe, personally, going onto campus to teach,'' said
Alexandra Edwards, an instructor at Georgia Tech, who is worried that
the school will not give her clearance to teach remotely.Credit...Audra
Melton for The New York Times

Alexandra Edwards, who teaches first-year writing at Georgia Tech, had
planned to teach from home, and thought her request to do so would be
``just a formality.'' Now Ms. Edwards, 35, who says she has a disability
that is not on the exemption list, is concerned that she will not
qualify to teach remotely. ``I don't feel safe, personally, going onto
campus to teach,'' she said.

Joshua Stewart, a spokesman for Georgia Tech, said the university's
high-risk categories were based on guidance from the federal government
and the Georgia Department of Public Health. ``If that guidance evolves,
our plan will evolve along with it,'' he said.

Other universities have been more open to letting professors decide for
themselves what to do. ``Due to these extraordinary circumstances, the
university is temporarily suspending the normal requirement that
teaching be done in person,'' the University of Chicago said in a
message to instructors on June 26.

Yale said on Wednesday that it would bring only a portion of its
students back to campus for each semester: freshmen, juniors and seniors
in the fall, and sophomores, juniors and seniors in the spring. ``Nearly
all'' college courses will be taught remotely, the university said, so
that all students can enroll in them.

Cornell plans to make clear to students before each semester begins
which classes will be offered in person and which will be online, so
they are not surprised, said Mr. Kotlikoff, the provost. He said the
university environment would be safer than the outside world because
students would be tested even when they did not have symptoms.

Still, campuses are not fortresses, and professors in states that have
seen recent spikes in coronavirus infections are particularly worried.
Hundreds of cases
\href{https://www.nytimes.com/2020/06/28/us/coronavirus-college-towns.html}{have
been linked to universities in Southern states} in recent days,
including clusters among the football teams at Clemson, Auburn and Texas
Tech, and outbreaks tied to fraternity rush parties in Mississippi and
to the Tigerland nightlife district near the Louisiana State campus.

``We're all holding our breath to see what the policies will be,'' said
Terrence Peterson, an assistant professor of history at Florida
International University in Miami. Professor Peterson, 35, said he had
respiratory ailments and a 6-month-old daughter at home.

Joshua Wede, 40, a psychology professor at Penn State, argued that it
was not possible to maintain a meaningful level of human interaction
when students are wearing masks, sitting at least six feet apart and
facing straight ahead.

``The value that you have in the classroom is totally lost,'' he said.
``My style of teaching, I'm walking all over the room. I wouldn't be
able to do that.''

Professor Wede said a survey of his department found that one out of
five faculty members would not be comfortable teaching face to face. But
people fear speaking out, he said: ``If the university knows they are
high-risk, and they have to go remote, are they not going to renew their
contracts?''

At Pitzer College, Professor Ward said that whether to go back into the
classroom to teach is a hot topic among the faculty.

``Nine out of 10 are worried,'' he said, especially with the recent rise
in cases in California. He is not scheduled to teach until spring, he
said, but he expects to sit out that course for health reasons and on
principle, because he does not think it is fair to promise students
something they will not get.

``It's not possible to replicate an in-class experience,'' he said.
``It's a kind of bait and switch.''

Advertisement

\protect\hyperlink{after-bottom}{Continue reading the main story}

\hypertarget{site-index}{%
\subsection{Site Index}\label{site-index}}

\hypertarget{site-information-navigation}{%
\subsection{Site Information
Navigation}\label{site-information-navigation}}

\begin{itemize}
\tightlist
\item
  \href{https://help.nytimes.com/hc/en-us/articles/115014792127-Copyright-notice}{©~2020~The
  New York Times Company}
\end{itemize}

\begin{itemize}
\tightlist
\item
  \href{https://www.nytco.com/}{NYTCo}
\item
  \href{https://help.nytimes.com/hc/en-us/articles/115015385887-Contact-Us}{Contact
  Us}
\item
  \href{https://www.nytco.com/careers/}{Work with us}
\item
  \href{https://nytmediakit.com/}{Advertise}
\item
  \href{http://www.tbrandstudio.com/}{T Brand Studio}
\item
  \href{https://www.nytimes.com/privacy/cookie-policy\#how-do-i-manage-trackers}{Your
  Ad Choices}
\item
  \href{https://www.nytimes.com/privacy}{Privacy}
\item
  \href{https://help.nytimes.com/hc/en-us/articles/115014893428-Terms-of-service}{Terms
  of Service}
\item
  \href{https://help.nytimes.com/hc/en-us/articles/115014893968-Terms-of-sale}{Terms
  of Sale}
\item
  \href{https://spiderbites.nytimes.com}{Site Map}
\item
  \href{https://help.nytimes.com/hc/en-us}{Help}
\item
  \href{https://www.nytimes.com/subscription?campaignId=37WXW}{Subscriptions}
\end{itemize}
