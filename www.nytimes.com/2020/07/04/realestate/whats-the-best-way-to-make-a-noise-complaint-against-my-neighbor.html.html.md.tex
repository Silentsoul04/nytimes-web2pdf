Sections

SEARCH

\protect\hyperlink{site-content}{Skip to
content}\protect\hyperlink{site-index}{Skip to site index}

\href{https://www.nytimes.com/section/realestate}{Real Estate}

\href{https://myaccount.nytimes.com/auth/login?response_type=cookie\&client_id=vi}{}

\href{https://www.nytimes.com/section/todayspaper}{Today's Paper}

\href{/section/realestate}{Real Estate}\textbar{}What's the Best Way to
Make a Noise Complaint Against My Neighbor?

\url{https://nyti.ms/2AziQdO}

\begin{itemize}
\item
\item
\item
\item
\item
\item
\end{itemize}

Advertisement

\protect\hyperlink{after-top}{Continue reading the main story}

Supported by

\protect\hyperlink{after-sponsor}{Continue reading the main story}

Ask Real Estate

\hypertarget{whats-the-best-way-to-make-a-noise-complaint-against-my-neighbor}{%
\section{What's the Best Way to Make a Noise Complaint Against My
Neighbor?}\label{whats-the-best-way-to-make-a-noise-complaint-against-my-neighbor}}

Noise complaints are often difficult to win, but there are some avenues
to pursue that can work.

\includegraphics{https://static01.nyt.com/images/2020/07/05/realestate/04Ask/04Ask-articleLarge.jpg?quality=75\&auto=webp\&disable=upscale}

\href{https://www.nytimes.com/by/ronda-kaysen}{\includegraphics{https://static01.nyt.com/images/2018/07/16/multimedia/author-ronda-kaysen/author-ronda-kaysen-thumbLarge-v2.png}}

By \href{https://www.nytimes.com/by/ronda-kaysen}{Ronda Kaysen}

\begin{itemize}
\item
  July 4, 2020
\item
  \begin{itemize}
  \item
  \item
  \item
  \item
  \item
  \item
  \end{itemize}
\end{itemize}

\textbf{Q: I live in a Harlem condo. The owner of the neighboring unit
rents out her apartment to a tenant who runs her air-conditioner all day
and night. The unit is directly outside my bedroom window. During the
day I don't mind the hum, but at night the noise is unbearable if I open
my window. I reached out to the owner of the apartment, who claimed the
tenants agreed to put the air-conditioner on a low setting at night. But
they haven't done so. What should I do?}

\textbf{A:} Air-conditioners are notoriously noisy, but your neighbor's
might have bigger problems than a typical rattly window unit. ``It could
be broken, a bad design, bad installation or bad geometry,'' said Alan
Fierstein, the owner of \href{https://www.acoustilog.com/}{Acoustilog},
a Manhattan noise consultant.

While noise complaints can be difficult to win, you do have some avenues
to pursue that could work. Start with your condo association. Review the
condo's governing documents to see if an air-conditioner is a common
element of the building. If so, the condo is responsible for maintaining
it and you should write a letter, sent by certified mail, insisting that
the board repair or replace the unit.

But even if the board isn't responsible for maintaining the
air-conditioner, it has other obligations to you. Check the house rules
to see what language exists about unreasonable noise or nuisance, then
write the board a letter insisting that it compel your neighbor to fix
or replace the unit. The board could hire a noise consultant to check
the noise levels in your apartment.

\begin{center}\rule{0.5\linewidth}{\linethickness}\end{center}

\begin{center}\rule{0.5\linewidth}{\linethickness}\end{center}

You could also file a complaint with the city. For that, call 311 and
request that an inspector come to measure the decibels. Despite the
pandemic, city inspectors are still out in the field. ``For a complaint
concerning noise created by an air-conditioner, inspectors have been
taking noise readings from outside of dwellings, typically within a few
days of the report coming in,'' said Ted Timbers, a Department of
Environmental Protection spokesman. If the levels violate the city noise
code, the unit owner could get a ticket.

Throughout this process, continue a dialogue with the unit owner.
Encourage her to repair or replace the air-conditioner. Record the sound
at night so she can understand how loud it is. Tell her if she doesn't
resolve this problem, you will escalate your efforts and get the city or
board to act on your behalf.

For weekly email updates on residential real estate news,
\href{http://www.nytimes.com/newsletters/realestate/}{sign up here}.
Follow us on Twitter:
\href{https://twitter.com/nytrealestate}{@nytrealestate}.

Advertisement

\protect\hyperlink{after-bottom}{Continue reading the main story}

\hypertarget{site-index}{%
\subsection{Site Index}\label{site-index}}

\hypertarget{site-information-navigation}{%
\subsection{Site Information
Navigation}\label{site-information-navigation}}

\begin{itemize}
\tightlist
\item
  \href{https://help.nytimes.com/hc/en-us/articles/115014792127-Copyright-notice}{©~2020~The
  New York Times Company}
\end{itemize}

\begin{itemize}
\tightlist
\item
  \href{https://www.nytco.com/}{NYTCo}
\item
  \href{https://help.nytimes.com/hc/en-us/articles/115015385887-Contact-Us}{Contact
  Us}
\item
  \href{https://www.nytco.com/careers/}{Work with us}
\item
  \href{https://nytmediakit.com/}{Advertise}
\item
  \href{http://www.tbrandstudio.com/}{T Brand Studio}
\item
  \href{https://www.nytimes.com/privacy/cookie-policy\#how-do-i-manage-trackers}{Your
  Ad Choices}
\item
  \href{https://www.nytimes.com/privacy}{Privacy}
\item
  \href{https://help.nytimes.com/hc/en-us/articles/115014893428-Terms-of-service}{Terms
  of Service}
\item
  \href{https://help.nytimes.com/hc/en-us/articles/115014893968-Terms-of-sale}{Terms
  of Sale}
\item
  \href{https://spiderbites.nytimes.com}{Site Map}
\item
  \href{https://help.nytimes.com/hc/en-us}{Help}
\item
  \href{https://www.nytimes.com/subscription?campaignId=37WXW}{Subscriptions}
\end{itemize}
