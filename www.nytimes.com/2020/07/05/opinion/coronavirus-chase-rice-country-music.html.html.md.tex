Sections

SEARCH

\protect\hyperlink{site-content}{Skip to
content}\protect\hyperlink{site-index}{Skip to site index}

\href{https://myaccount.nytimes.com/auth/login?response_type=cookie\&client_id=vi}{}

\href{https://www.nytimes.com/section/todayspaper}{Today's Paper}

\href{/section/opinion}{Opinion}\textbar{}How to Keep Music (and One
Another) Alive

\href{https://nyti.ms/38v8wzT}{https://nyti.ms/38v8wzT}

\begin{itemize}
\item
\item
\item
\item
\item
\item
\end{itemize}

\href{https://www.nytimes.com/spotlight/at-home?action=click\&pgtype=Article\&state=default\&region=TOP_BANNER\&context=at_home_menu}{At
Home}

\begin{itemize}
\tightlist
\item
  \href{https://www.nytimes.com/2020/08/04/arts/television/sam-jay-netflix-special.html?action=click\&pgtype=Article\&state=default\&region=TOP_BANNER\&context=at_home_menu}{Watch:
  Sam Jay}
\item
  \href{https://www.nytimes.com/interactive/2020/at-home/even-more-reporters-editors-diaries-lists-recommendations.html?action=click\&pgtype=Article\&state=default\&region=TOP_BANNER\&context=at_home_menu}{Peruse:
  Reporters' Google Docs}
\item
  \href{https://www.nytimes.com/2020/08/04/dining/colombian-empanadas-carlos-gaviria.html?action=click\&pgtype=Article\&state=default\&region=TOP_BANNER\&context=at_home_menu}{Make:
  Empanadas}
\item
  \href{https://www.nytimes.com/2020/08/06/arts/design/street-art-nyc-george-floyd.html?action=click\&pgtype=Article\&state=default\&region=TOP_BANNER\&context=at_home_menu}{Explore:
  N.Y.C. Street Art}
\end{itemize}

Advertisement

\protect\hyperlink{after-top}{Continue reading the main story}

\href{/section/opinion}{Opinion}

Supported by

\protect\hyperlink{after-sponsor}{Continue reading the main story}

\hypertarget{how-to-keep-music-and-one-another-alive}{%
\section{How to Keep Music (and One Another)
Alive}\label{how-to-keep-music-and-one-another-alive}}

If we want to experience concerts again, we need to support the artists
we love until the pandemic passes. Here's how.

\href{https://www.nytimes.com/by/margaret-renkl}{\includegraphics{https://static01.nyt.com/images/2017/04/08/opinion/margaret-renkl/margaret-renkl-thumbLarge-v2.png}}

By \href{https://www.nytimes.com/by/margaret-renkl}{Margaret Renkl}

Contributing Opinion Writer

\begin{itemize}
\item
  July 5, 2020
\item
  \begin{itemize}
  \item
  \item
  \item
  \item
  \item
  \item
  \end{itemize}
\end{itemize}

\includegraphics{https://static01.nyt.com/images/2020/07/06/opinion/06renkl1/merlin_174161520_518b73e4-9f19-47d0-ba47-fecdd0d3351d-articleLarge.jpg?quality=75\&auto=webp\&disable=upscale}

NASHVILLE --- On June 27, as Covid-19 cases were
\href{https://wpln.org/post/following-high-case-counts-tennessees-pandemic-tracking-site-goes-down/?mc_cid=a2c6e0f2a3\&mc_eid=e952ad88f8}{rising
to a crisis level in Tennessee}, the country music artist
\href{https://www.tennessean.com/story/entertainment/music/2020/06/29/chase-rice-chris-janson-concerts-roil-nashvilles-ravaged-music-industry/3280563001/}{Chase
Rice held a concert} outside the Historic Brushy Mountain State
Penitentiary in Petros, Tenn. Packed tightly together and wearing no
masks --- at least none that were visible in the video Mr. Rice posted
on Instagram --- fans seemed unconcerned that a deadly pandemic was
unfolding around them. And probably among them, too.

The video has since expired, but the backlash against Mr. Rice --- and
also Chris Janson, who played to a similar crowd in Filer, Idaho --- was
fierce. Coming on the heels of an image that circulated on social media
\href{https://twitter.com/MissMandyHale/status/1272231947708891144}{of a
packed bar} at Kid Rock's Big Ass Honkytonk and Rock `n' Roll Steakhouse
in Nashville's tourist district, it seemed emblematic, a giant middle
finger to the pandemic itself.

I'm as outraged as anyone at the sight of people making choices that
will inevitably cost lives and prolong the pandemic, but I can
understand, at least a little bit, why Mr. Rice held that concert and
why his fans showed up.

Bands once supported themselves primarily through record sales. But then
music went digital, and Napster, the early file-sharing platform,
\href{https://www.nytimes.com/2000/07/15/opinion/freedom-one-song-at-a-time.html?searchResultPosition=2}{changed
the industry almost overnight}. I haven't thought of Napster in years,
but the shuttering of clubs and concert halls during this pandemic has
reminded me of something the musician Rich Brotherton, my husband's
lifelong friend, heard Loudon Wainwright say at a concert some 20 years
ago: ``You're file-sharing the food right out of my mouth.''

Today's streaming services,
\href{https://www.rollingstone.com/pro/features/how-musicians-make-money-or-dont-at-all-in-2018-706745/}{which
pay out a fraction of a cent per play}, are marginally fairer than
outright piracy, but their chief value to a performer is the opportunity
to reach new listeners: If someone falls in love with a song, that new
fan is apt to buy a concert ticket when the artist performs nearby.

Musicians can't pay their bills if they can't perform, but it's not like
Chase Rice had no options for waiting out the pandemic safely. Other
musicians have found
\href{https://www.vulture.com/2020/05/all-musicians-streaming-live-concerts.html}{new
ways to reach their old audiences}.
``\href{http://www.livenation.com/drivein/}{Live From the Drive-In},''
with performances by country, rock and rap artists, and the
``\href{https://www.driveintheatertour.com/}{Drive-In Theater Tour},''
featuring Christian artists, offer concerts with social distancing baked
in: Fans bring their own refreshments and stay with their cars,
tailgate-style, for the whole show. At the country artist Keith Urban's
\href{https://www.tennessean.com/story/entertainment/music/2020/05/15/keith-urban-drive-concert-stardust-vanderbilt-medical-workers-tennessee/5196898002/}{pop-up
drive-in performance} for front line medical workers in May, the
audience ``clapped'' with their headlights.

Far more common are
\href{https://www.vulture.com/2020/05/all-musicians-streaming-live-concerts.html}{online
performances} streamed live and archived for those who missed the show.
Last month Sturgill Simpson
\href{https://www.nashvillescene.com/music/spin/article/21136475/sturgill-simpson-lets-the-grass-grow-at-the-ryman}{live
streamed a benefit concert} from the stage of Nashville's Ryman
Auditorium to an empty hall. Every week she's in town, Marshall Chapman
streams a
``\href{https://www.facebook.com/pg/marshallchapmanmusic/events/}{Saturdays
at Springwater}'' show from \href{https://www.thespringwater.com/}{the
oldest continuously operated bar} in Tennessee. Mary Gauthier streams
both a concert series,
``\href{https://www.marygauthier.com/tour}{Sundays With Mary},'' and
master classes \href{https://www.marygauthier.com/masterclass}{in
songwriting}. Mr. Brotherton's Irish band, Úlla, can no longer keep its
weekly date at an Austin club, so the musicians have
\href{https://www.facebook.com/pg/ullairishmusic/posts/?ref=page_internal}{moved
their Sunday evening shows online}, each performing from their own
homes.

But for all its better-than-nothing virtues, a digital concert doesn't
have even remotely the power of a live performance. And the experience
of seeing an artist in the flesh --- or on a jumbotron --- is only part
of the draw.

The real beauty of an in-person concert is the relationship between the
audience and the performer, and among members of the audience. It's the
feeling of being a part of something huge and beautiful and fleeting. A
live musical performance, whether it's in a stadium or in a storied
concert hall or in the shabbiest dive bar on the loneliest back street,
is a shared experience of transcendence. As Rosanne Cash wrote in
\href{https://www.theatlantic.com/ideas/archive/2020/05/what-pandemic-has-clarified-me-about-life-road/612076/}{a
recent essay for The Atlantic}, a live performance is an irreplaceable
act of reciprocity: ``They needed something from me, and giving it to
them gave something back to me. I loved them. They knew it.''

These may seem like frivolous things in the context of a global health
emergency, but they are not at all frivolous in the context of fear and
isolation. It's an awful lot to ask of performers to give up performing,
and it is an awful lot to ask of fans to skip their shows.

Expecting people to do the right thing when the right thing flies in the
face of human nature is never a good bet. Until it's safe to sing along
in public again, the only answer is for leaders to show some backbone
and
\href{https://www.nytimes.com/aponline/2020/07/02/health/bc-us-med-virus-outbreak-bars.html?searchResultPosition=2}{lock
down the concert halls and the bars}. Last week Nashville's mayor John
Cooper
\href{https://wpln.org/post/nashville-backtracks-on-reopening-plan-reverting-to-phase-2/}{did
just that}.

If we ever hope to experience the transcendence of live performance
again, we're going to have to support the artists we love until the
pandemic passes. We're going to have to put some money in the tip jar at
virtual concerts. Buy the T-shirts and the ball caps with the band logos
on them. Above all, we're going to have to start buying records again.

``The LPs and CDs that musicians would have on their merchandise tables
at shows across the country are there to be had on their website stores
right now,'' the Nashville music journalist Craig Havighurst told me in
a recent email. Buying the merch ``is the most potent way fans can help
artists survive this crisis.''

If people can get in the habit of buying records again, it would go a
long way toward helping musicians and songwriters survive the pandemic
and beyond. ``This is the best possible time to rethink our consumption
habits as fans for the short and long term,'' Mr. Havighurst pointed
out. ``We should strive for an ethos where we stream to discover and
purchase what we love.''

Margaret Renkl is a contributing opinion writer who covers flora, fauna,
politics and culture in the American South. She is the author of the
book ``\href{https://milkweed.org/book/late-migrations}{Late Migrations:
A Natural History of Love and Loss}.''

\emph{The Times is committed to publishing}
\href{https://www.nytimes.com/2019/01/31/opinion/letters/letters-to-editor-new-york-times-women.html}{\emph{a
diversity of letters}} \emph{to the editor. We'd like to hear what you
think about this or any of our articles. Here are some}
\href{https://help.nytimes.com/hc/en-us/articles/115014925288-How-to-submit-a-letter-to-the-editor}{\emph{tips}}\emph{.
And here's our email:}
\href{mailto:letters@nytimes.com}{\emph{letters@nytimes.com}}\emph{.}

\emph{Follow The New York Times Opinion section on}
\href{https://www.facebook.com/nytopinion}{\emph{Facebook}}\emph{,}
\href{http://twitter.com/NYTOpinion}{\emph{Twitter (@NYTopinion)}}
\emph{and}
\href{https://www.instagram.com/nytopinion/}{\emph{Instagram}}\emph{.}

Advertisement

\protect\hyperlink{after-bottom}{Continue reading the main story}

\hypertarget{site-index}{%
\subsection{Site Index}\label{site-index}}

\hypertarget{site-information-navigation}{%
\subsection{Site Information
Navigation}\label{site-information-navigation}}

\begin{itemize}
\tightlist
\item
  \href{https://help.nytimes.com/hc/en-us/articles/115014792127-Copyright-notice}{©~2020~The
  New York Times Company}
\end{itemize}

\begin{itemize}
\tightlist
\item
  \href{https://www.nytco.com/}{NYTCo}
\item
  \href{https://help.nytimes.com/hc/en-us/articles/115015385887-Contact-Us}{Contact
  Us}
\item
  \href{https://www.nytco.com/careers/}{Work with us}
\item
  \href{https://nytmediakit.com/}{Advertise}
\item
  \href{http://www.tbrandstudio.com/}{T Brand Studio}
\item
  \href{https://www.nytimes.com/privacy/cookie-policy\#how-do-i-manage-trackers}{Your
  Ad Choices}
\item
  \href{https://www.nytimes.com/privacy}{Privacy}
\item
  \href{https://help.nytimes.com/hc/en-us/articles/115014893428-Terms-of-service}{Terms
  of Service}
\item
  \href{https://help.nytimes.com/hc/en-us/articles/115014893968-Terms-of-sale}{Terms
  of Sale}
\item
  \href{https://spiderbites.nytimes.com}{Site Map}
\item
  \href{https://help.nytimes.com/hc/en-us}{Help}
\item
  \href{https://www.nytimes.com/subscription?campaignId=37WXW}{Subscriptions}
\end{itemize}
