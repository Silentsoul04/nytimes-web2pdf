Sections

SEARCH

\protect\hyperlink{site-content}{Skip to
content}\protect\hyperlink{site-index}{Skip to site index}

\href{https://myaccount.nytimes.com/auth/login?response_type=cookie\&client_id=vi}{}

\href{https://www.nytimes.com/section/todayspaper}{Today's Paper}

\href{/section/opinion}{Opinion}\textbar{}`Tell the Truth and Shame the
Devil'

\href{https://nyti.ms/38s5Pza}{https://nyti.ms/38s5Pza}

\begin{itemize}
\item
\item
\item
\item
\item
\item
\end{itemize}

Advertisement

\protect\hyperlink{after-top}{Continue reading the main story}

\href{/section/opinion}{Opinion}

Supported by

\protect\hyperlink{after-sponsor}{Continue reading the main story}

\hypertarget{tell-the-truth-and-shame-the-devil}{%
\section{`Tell the Truth and Shame the
Devil'}\label{tell-the-truth-and-shame-the-devil}}

For Trump, the truth about patriarchal white supremacy defiles the
American heroes who practiced it.

\href{https://www.nytimes.com/by/charles-m-blow}{\includegraphics{https://static01.nyt.com/images/2018/04/02/opinion/charles-m-blow/charles-m-blow-thumbLarge.png}}

By \href{https://www.nytimes.com/by/charles-m-blow}{Charles M. Blow}

Opinion Columnist

\begin{itemize}
\item
  July 5, 2020
\item
  \begin{itemize}
  \item
  \item
  \item
  \item
  \item
  \item
  \end{itemize}
\end{itemize}

\includegraphics{https://static01.nyt.com/images/2020/07/06/opinion/06blow_print/merlin_174228909_9433c178-68ca-4c33-a3d0-303dc5802a97-articleLarge.jpg?quality=75\&auto=webp\&disable=upscale}

As Donald Trump gave his race-baiting speeches over the Fourth of July
weekend, hoping to rile his base and jump-start his flagging campaign
for re-election, I was forced to recall the ranting of a Columbia
University sophomore that caught the nation's attention in 2018.

In the video, a student named
\href{https://www.columbiaspectator.com/news/2018/12/10/students-of-color-harassed-outside-butler-by-columbia-sophomore-spewing-racist-white-supremacist-rhetoric/}{Julian
von Abele} exclaims, ``We built the modern world!'' When someone asks
who, he responds, ``Europeans.''

\href{https://www.washingtonpost.com/nation/2018/12/11/were-white-men-we-did-everything-columbia-condemns-students-tirade-targeting-minorities/}{Von
Abele goes on}:

``We invented science and industry, and you want to tell us to stop
because \emph{oh my God, we're so baaad}. We invented the modern world.
We saved billions of people from starvation. We built modern
civilization. White people are the best thing that ever happened to the
world. We are so amazing! I love myself! And I love white people!''

He concludes: ``I don't hate other people. I just love white men.''

Von Abele later
\href{https://www.thedailybeast.com/columbia-student-from-viral-video-speaks-i-am-not-racist}{apologized}
for ``going over the top,'' saying, ``I emphasize that my reaction was
not one of hate'' and arguing that his remarks were taken ``out of
context.'' But the sentiments like the one this young man expressed ---
that white men must be venerated, regardless of their sins, in spite of
their sins, because they used maps, Bibles and guns to change the world,
and thereby lifted it and saved it ---~aren't limited to one college
student's regrettable video. They are at the root of patriarchal white
supremacist ideology.

To people who believe in this, white men are the heroes in the history
of the world. They conquered those who could be conquered. They enslaved
those who could be enslaved. And their religion and philosophy, and
sometimes even their pseudoscience, provided the rationale for their
actions.

It was hard not to hear the voice of von Abele when Trump stood at the
base of Mount Rushmore
\href{https://www.whitehouse.gov/briefings-statements/remarks-president-trump-south-dakotas-2020-mount-rushmore-fireworks-celebration-keystone-south-dakota/}{and
said}, ``Seventeen seventy-six represented the culmination of thousands
of years of Western civilization and the triumph not only of spirit, but
of wisdom, philosophy and reason.'' He continued later, ``Our nation is
witnessing a merciless campaign to wipe out our history, defame our
heroes, erase our values and indoctrinate our children.''

To be clear, the ``our'' in that passage is white people, specifically
white men. Trump is telling white men that they are their ancestors, and
that they're now being attacked for that which they should be thanked.

The ingratitude of it all.

How dare historically oppressed minorities in this country recall the
transgressions of their oppressors? How dare they demand that the whole
truth be told? How dare they withhold their adoration of the abominable?

At another point, Trump said of recent protests:

``This left-wing cultural revolution is designed to overthrow the
American Revolution. In so doing, they would destroy the very
civilization that rescued billions from poverty, disease, violence and
hunger, and that lifted humanity to new heights of achievement,
discovery and progress.''

In fact, many of the protesters are simply pointing out the hypocrisy of
these men, including many of the founders, who fought for freedom and
liberty from the British while simultaneously enslaving Africans and
slaughtering the Indigenous.

But, Trump, like white supremacy itself, rejects the inclusion of this
context. As Trump put it:

``Against every law of society and nature, our children are taught in
school to hate their own country, and to believe that the men and women
who built it were not heroes, but that were villains. The radical view
of American history is a web of lies --- all perspective is removed,
every virtue is obscured, every motive is twisted, every fact is
distorted, and every flaw is magnified until the history is purged and
the record is disfigured beyond all recognition.''

In fact, the record is not being disfigured but corrected.

\href{https://www.whitehouse.gov/briefings-statements/remarks-president-trump-south-dakotas-2020-mount-rushmore-fireworks-celebration-keystone-south-dakota/}{According
to Trump}: ``This movement is openly attacking the legacies of every
person on Mount Rushmore. They defile the memory of Washington,
Jefferson, Lincoln and Roosevelt.''

Is it a defilement to point out that
\href{https://www.nytimes.com/2020/06/28/opinion/george-washington-confederate-statues.html}{George
Washington} was an enslaver who signed a fugitive slave act and only
freed his slaves in his will, after he was dead and no longer had
earthly use for them?

Is it a defilement to point out that Thomas Jefferson enslaved over 600
human beings during his life, many when he wrote the Declaration of
Independence, and that he had sex with a child whom he enslaved --- I
call it rape --- and even enslaved the children she bore for him?

Is it a defilement to recall that during the Lincoln-Douglas debates
\href{https://www.stjoe.k12.in.us/ourpages/auto/2011/11/14/53458274/Lincoln-Douglas\%20Debates_\%20Excerpts.pdf}{Abraham
Lincoln said}:

``I have no purpose to introduce political and social equality between
the white and the Black races. There is a physical difference between
the two, which in my judgment will probably forever forbid their living
together upon the footing of perfect equality, and inasmuch as it
becomes a necessity that there must be a difference, I, as well as Judge
Douglas, am in favor of the race to which I belong, having the superior
position.''

Is it defilement to recall that Theodore Roosevelt was a white
supremacist, supporter of eugenics and an imperialist? As Gary Gerstle,
a professor of American history at the University of Cambridge,
\href{https://www.wbur.org/hereandnow/2019/03/21/teddy-roosevelt-legacy-100-years}{once
put it}, ``He would have had no patience with the Indigenous and
original inhabitants of a sacred American space interfering with his
conception of the American sublime.''

It is not a defilement, but deprogramming. It is a telling of the truth,
and the time for it is long overdue.

As the old folks used to put it, ``Tell the truth and shame the devil.''

\hypertarget{what-new-monuments-would-you-like-to-see}{%
\subsection{What new monuments would you like to
see?}\label{what-new-monuments-would-you-like-to-see}}

\emph{The Times is committed to publishing}
\href{https://www.nytimes.com/2019/01/31/opinion/letters/letters-to-editor-new-york-times-women.html}{\emph{a
diversity of letters}} \emph{to the editor. We'd like to hear what you
think about this or any of our articles. Here are some}
\href{https://help.nytimes.com/hc/en-us/articles/115014925288-How-to-submit-a-letter-to-the-editor}{\emph{tips}}\emph{.
And here's our email:}
\href{mailto:letters@nytimes.com}{\emph{letters@nytimes.com}}\emph{.}

\emph{Follow The New York Times Opinion section on}
\href{https://www.facebook.com/nytopinion}{\emph{Facebook}} \emph{and}
\href{http://twitter.com/NYTOpinion}{\emph{Twitter
(@NYTopinion)}}\emph{, and}
\href{https://www.instagram.com/nytopinion/}{\emph{Instagram}}\emph{.}

Advertisement

\protect\hyperlink{after-bottom}{Continue reading the main story}

\hypertarget{site-index}{%
\subsection{Site Index}\label{site-index}}

\hypertarget{site-information-navigation}{%
\subsection{Site Information
Navigation}\label{site-information-navigation}}

\begin{itemize}
\tightlist
\item
  \href{https://help.nytimes.com/hc/en-us/articles/115014792127-Copyright-notice}{©~2020~The
  New York Times Company}
\end{itemize}

\begin{itemize}
\tightlist
\item
  \href{https://www.nytco.com/}{NYTCo}
\item
  \href{https://help.nytimes.com/hc/en-us/articles/115015385887-Contact-Us}{Contact
  Us}
\item
  \href{https://www.nytco.com/careers/}{Work with us}
\item
  \href{https://nytmediakit.com/}{Advertise}
\item
  \href{http://www.tbrandstudio.com/}{T Brand Studio}
\item
  \href{https://www.nytimes.com/privacy/cookie-policy\#how-do-i-manage-trackers}{Your
  Ad Choices}
\item
  \href{https://www.nytimes.com/privacy}{Privacy}
\item
  \href{https://help.nytimes.com/hc/en-us/articles/115014893428-Terms-of-service}{Terms
  of Service}
\item
  \href{https://help.nytimes.com/hc/en-us/articles/115014893968-Terms-of-sale}{Terms
  of Sale}
\item
  \href{https://spiderbites.nytimes.com}{Site Map}
\item
  \href{https://help.nytimes.com/hc/en-us}{Help}
\item
  \href{https://www.nytimes.com/subscription?campaignId=37WXW}{Subscriptions}
\end{itemize}
