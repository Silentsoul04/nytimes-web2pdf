\href{/section/nyregion}{New York}\textbar{}How a Brooklyn Artist Is
Making Black Women Her Focus

\url{https://nyti.ms/2Z4RWUq}

\begin{itemize}
\item
\item
\item
\item
\item
\item
\end{itemize}

\includegraphics{https://static01.nyt.com/images/2020/07/03/nyregion/00xp-virus-artists1/merlin_174172602_d06683e7-bac2-475b-b8cb-a3b4d6829490-articleLarge.jpg?quality=75\&auto=webp\&disable=upscale}

Sections

\protect\hyperlink{site-content}{Skip to
content}\protect\hyperlink{site-index}{Skip to site index}

\hypertarget{how-a-brooklyn-artist-is-making-black-women-her-focus}{%
\section{How a Brooklyn Artist Is Making Black Women Her
Focus}\label{how-a-brooklyn-artist-is-making-black-women-her-focus}}

Through her Essential Workers series, Aya Brown, 24, has shined a
spotlight on the Black women in New York who work in hospitals, schools
and retail.

Aya Brown, left, an artist in Brooklyn, with Brittany Robles, one of the
subjects of her Essential Workers series.Credit...Naima Green for The
New York Times

Supported by

\protect\hyperlink{after-sponsor}{Continue reading the main story}

\href{https://www.nytimes.com/by/sandra-e-garcia}{\includegraphics{https://static01.nyt.com/images/2020/07/10/reader-center/author-sandra-e-garcia/author-sandra-e-garcia-thumbLarge.png}}

By \href{https://www.nytimes.com/by/sandra-e-garcia}{Sandra E. Garcia}

\begin{itemize}
\item
  Published July 6, 2020Updated July 14, 2020
\item
  \begin{itemize}
  \item
  \item
  \item
  \item
  \item
  \item
  \end{itemize}
\end{itemize}

The faces of the women in her portraits are often partly covered by a
mask tied behind their heads, tugging at braids, low buns or tufts of
curls. They are dressed in uniforms that show their
\href{https://www.nytimes.com/2020/07/14/business/coronavirus-essential-workers-pay-raises.html}{essential
jobs}, but their style and charisma shine through their everyday armor.

They are Black women who work in jobs that the
\href{https://www.nytimes.com/2020/07/14/business/coronavirus-essential-workers-pay-raises.html}{coronavirus}
pandemic quickly revealed as essential to the functioning of New York
City. And they were all drawn by Aya Brown, 24, a Brooklyn artist. They
are women who took care of Ms. Brown during a hospital or a supermarket
visit. They include janitors, M.T.A. workers, mail carriers and security
guards.

The drawings --- made with color pencils on brown paper --- comprise Ms.
Brown's \href{http://ayabrown.com/series/Essential\%20Workers}{Essential
Worker series}, a collection drawn with an intimacy that makes the
viewers feel as if they too know the subject. It's not just their jobs
that are depicted through the lines and colors, but their panache.

``My goal is to uplift Black women who look like me and inspire me ---
to give them a space to be seen and to bring awareness to them,'' Ms.
Brown said.

\includegraphics{https://static01.nyt.com/images/2020/07/03/nyregion/00xp-virus-artists-1/00xp-virus-artists-1-articleLarge.jpg?quality=75\&auto=webp\&disable=upscale}

Women have been the heroes of the pandemic. They are in the emergency
rooms, on the streets delivering packages, in nursing homes, on
construction sites, and many are still teaching their students who have
been attending school from home.

One in three of the jobs held by women is essential, according to a
\href{https://www.nytimes.com/2020/04/18/us/coronavirus-women-essential-workers.html}{New
York Times analysis} of census data crossed with the federal
government's essential worker guidelines. Most of the women who have
essential jobs are women of color.

``I guess when you think about essential workers, you don't really think
of yourself,'' said Aja Brown, 26, Ms. Brown's sister and a subject of
one of her portraits.

Image

Aja Brown, left, with her sister, Aya Brown, the artist, in Prospect
Heights, outside Geido, their favorite restaurant.Credit...Naima Green
for The New York Times

Aja is a paraprofessional educator, a role similar to a teacher's aide,
and works with fifth graders in Brooklyn. She has been working from home
since the city closed schools in March. She never considered herself an
essential worker until she saw her sister's portrait of her on
Instagram. The portrait made her cry, she said.

``I don't know if I needed that space,'' Aja said. ``I just want my kids
to get where they need to be emotionally and academically. I kind of
don't really think about myself.''

Ms. Brown aims to change that thinking, to help Black women see
themselves as essential by putting them at the center of her artwork and
bringing the viewer into her universe.

Image

``AJA'' EDU WORKER, COVID-19, 2020.Credit...Aya Brown

``It's very clear how close she is to \emph{her} mainstream, how
unfiltered her perspective is and how much she loves her people and her
village,'' said Tamara P. Carter, a writer and director of the upcoming
TV show ``Freshwater.''

After being furloughed by her employer,
\href{https://gavinbrown.biz/}{Gavin Brown Enterprises}, where she
organized events, Ms. Brown has used her free time to delve into her
art, which focuses on showing Black queer women fully: their sexuality,
strength, style, bodies, joy and edge. Even the materials she uses are
intentional: She draws on brown paper, she said, because ``Black bodies
do not need to start from white.''

Occasionally, she hosts parties that are meant to provide a safe space
for Black lesbians, like herself. It is the kind of support Ms. Brown
was entrenched in growing up in Brooklyn, and a foundation that was
notably missing when she attended Cooper Union, a private college in
Manhattan. She said her experience there was traumatic, that she did not
feel as if her blackness was accepted. After three years, she dropped
out in 2017.

Image

Credit...Naima Green for The New York Times

``They made me feel like I didn't deserve to be there,'' Ms. Brown said.

She began her Essential Worker series in April, after a trip to the
emergency room. There she noticed that her nurse, a Black, West Indian
woman, took care of her while her doctor stopped by intermittently.

``I noticed that nurses in the E.R. are usually Black women,'' Ms. Brown
said. ``I am thinking about these Black women on the front lines. It
just bothered me because no one is noticing this.''

A few months later, out of work because of the pandemic and with not
much to do, she began to develop her Essential Workers series.

Image

``KEYANNA'' EMT, COVID-19, 2020.Credit...Aya Brown

Brittany Tabor, 29, one of Ms. Brown's subjects, has been a store
director at a Target in Brooklyn for six years.

``You never knew you were essential until Covid hit,'' Ms. Tabor said,
``and it's like, I have to stand up for the community now. I didn't
realize all that we do.''

Like countless Black women around the country, Ms. Tabor had to be a
counselor for her staff during the pandemic. When someone lost a family
member or a neighbor, she tried to put them at ease.

``I needed them to know, `I am in it with you, and let's get through
this together,' '' Ms. Tabor said. ``But I was freaking out, too. I was
human with everyone else. I was just able to put on a different hat.''

Black women are also underrepresented in the worlds of art and media,
and Black queer women are nearly nonexistent in museums, according to
\href{https://www.nytimes.com/2019/07/30/arts/design/basquiat-defacement-guggenheim-curator.html}{Chaédria
LaBouvier, the curator} of
``\href{https://www.guggenheim.org/exhibition/basquiats-defacement-the-untold-story}{Basquiat's
``Defacement'': The Untold Story,}'' at the Guggenheim Museum.

Image

Ms. Brown and her friend Keyanna Louis, an emergency medical worker, who
was a subject for one of her Essential Worker drawings.Credit...Naima
Green for The New York Times

``It is disgusting in a really violent and indifferent way,'' Ms.
LaBouvier said. ``There is no excuse, and even Black curators can be
complicit in perpetuating that.''

Ms. LaBouvier said Ms. Brown's work is not about being left out of the
white, heterosexual, patriarchal art world, but about the Black working
class saying, ``I am already the center, and there is a lot of beauty
here.''

Ms. Brown's work ``looks at what liberation actually could be,'' Ms.
LaBouvier said. ``You're in a moment where queer women are saying, `It
is so much bigger than fitting into the system; let's abolish the
system.'''

According to Ms. Carter, when we look back on this moment in history and
wonder who saved New York City from the coronavirus pandemic, Ms.
Brown's portraits will provide the answer.

``Who she's making the art for seems to be just as important as the art
itself,'' Ms. Carter said. ``Art made with that kind of love and rigor
is self-evident and can't be co-opted.''

Advertisement

\protect\hyperlink{after-bottom}{Continue reading the main story}

\hypertarget{site-index}{%
\subsection{Site Index}\label{site-index}}

\hypertarget{site-information-navigation}{%
\subsection{Site Information
Navigation}\label{site-information-navigation}}

\begin{itemize}
\tightlist
\item
  \href{https://help.nytimes.com/hc/en-us/articles/115014792127-Copyright-notice}{©~2020~The
  New York Times Company}
\end{itemize}

\begin{itemize}
\tightlist
\item
  \href{https://www.nytco.com/}{NYTCo}
\item
  \href{https://help.nytimes.com/hc/en-us/articles/115015385887-Contact-Us}{Contact
  Us}
\item
  \href{https://www.nytco.com/careers/}{Work with us}
\item
  \href{https://nytmediakit.com/}{Advertise}
\item
  \href{http://www.tbrandstudio.com/}{T Brand Studio}
\item
  \href{https://www.nytimes.com/privacy/cookie-policy\#how-do-i-manage-trackers}{Your
  Ad Choices}
\item
  \href{https://www.nytimes.com/privacy}{Privacy}
\item
  \href{https://help.nytimes.com/hc/en-us/articles/115014893428-Terms-of-service}{Terms
  of Service}
\item
  \href{https://help.nytimes.com/hc/en-us/articles/115014893968-Terms-of-sale}{Terms
  of Sale}
\item
  \href{https://spiderbites.nytimes.com}{Site Map}
\item
  \href{https://help.nytimes.com/hc/en-us}{Help}
\item
  \href{https://www.nytimes.com/subscription?campaignId=37WXW}{Subscriptions}
\end{itemize}
