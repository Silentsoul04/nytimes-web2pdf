Sections

SEARCH

\protect\hyperlink{site-content}{Skip to
content}\protect\hyperlink{site-index}{Skip to site index}

\href{/section/travel}{Travel}\textbar{}Commuting, and Confronting
History, on a Remote Canadian Railway

\url{https://nyti.ms/38ycIPe}

\begin{itemize}
\item
\item
\item
\item
\item
\item
\end{itemize}

\href{https://www.nytimes.com/spotlight/at-home?action=click\&pgtype=Article\&state=default\&region=TOP_BANNER\&context=at_home_menu}{At
Home}

\begin{itemize}
\tightlist
\item
  \href{https://www.nytimes.com/2020/07/28/books/time-for-a-literary-road-trip.html?action=click\&pgtype=Article\&state=default\&region=TOP_BANNER\&context=at_home_menu}{Take:
  A Literary Road Trip}
\item
  \href{https://www.nytimes.com/2020/07/29/magazine/bored-with-your-home-cooking-some-smoky-eggplant-will-fix-that.html?action=click\&pgtype=Article\&state=default\&region=TOP_BANNER\&context=at_home_menu}{Cook:
  Smoky Eggplant}
\item
  \href{https://www.nytimes.com/2020/07/27/travel/moose-michigan-isle-royale.html?action=click\&pgtype=Article\&state=default\&region=TOP_BANNER\&context=at_home_menu}{Look
  Out: For Moose}
\item
  \href{https://www.nytimes.com/interactive/2020/at-home/even-more-reporters-editors-diaries-lists-recommendations.html?action=click\&pgtype=Article\&state=default\&region=TOP_BANNER\&context=at_home_menu}{Explore:
  Reporters' Obsessions}
\end{itemize}

\includegraphics{https://static01.nyt.com/images/2020/07/07/travel/07a2_photography/06travel-canada-17-articleLarge.jpg?quality=75\&auto=webp\&disable=upscale}

The World Through a Lens

\hypertarget{commuting-and-confronting-history-on-a-remote-canadian-railway}{%
\section{Commuting, and Confronting History, on a Remote Canadian
Railway}\label{commuting-and-confronting-history-on-a-remote-canadian-railway}}

The Tshiuetin line, the first railroad in North America owned and
operated by First Nations people, is a symbol of reclamation and
defiance for the communities it serves.

The Tshiuetin railway sustains the lives of hundreds of people whose
connections to the region far outlasted the life of an iron-ore
mine.Credit...

Supported by

\protect\hyperlink{after-sponsor}{Continue reading the main story}

Photographs and Text by Chloë Ellingson

\begin{itemize}
\item
  Published July 6, 2020Updated July 17, 2020
\item
  \begin{itemize}
  \item
  \item
  \item
  \item
  \item
  \item
  \end{itemize}
\end{itemize}

\emph{At the onset of the coronavirus pandemic, with travel restrictions
in place worldwide, we launched a new series ---}
\href{https://www.nytimes.com/column/the-world-through-a-lens}{\emph{The
World Through a Lens}} \emph{--- in which photojournalists help
transport you, virtually, to some of our planet's most beautiful and
intriguing places. This week, Chloë Ellingson shares a collection of
photos from a remote railway in Canada.}

\begin{center}\rule{0.5\linewidth}{\linethickness}\end{center}

In 2015, as I drove through my hometown, Toronto, a radio documentary
came on the air describing a remote railway, the Tshiuetin line, that
runs through rural Quebec.

Named after the \href{https://www.innu.ca/}{Innu} word for ``wind of the
north,'' Tshiuetin is the first railway in North America owned and
operated by First Nations people. Its southern terminus, I would soon
learn, is about 15 hours east of Toronto by car.

\includegraphics{https://static01.nyt.com/images/2020/07/06/travel/06travel-canada-01/06travel-canada-articleLarge.jpg?quality=75\&auto=webp\&disable=upscale}

Canada was built by rail. The country's early railway system was a vital
tool for economic growth, but it also abetted Canada's colonial mission.
In addition to carrying goods and services, trains in Canada
disseminated disease among the Indigenous communities on whose land this
country was built. And while the country's railroads offered the
possibility of expansion to some, for others they were harbingers of
forced relocation.

Image

Elizabeth Rossignol waits for a lift to the train station in
Schefferville.

Image

A passenger waves as a group of students heads off for an educational
camping trip.

The fraught history of Canada's railways have made them a setting for
political demonstrations, most recently against plans for a
\href{https://www.nytimes.com/2020/02/18/world/canada/trudeau-rail-blockade.html}{pipeline
to be built through Wet'suwet'en territory} in northern British
Columbia.

Tshiuetin has a different history. The company operates on the 360-mile
line between Sept-Îles, a city on the northern shore of the Saint
Lawrence River, and Schefferville, a remote town on the verge of the
tundra in Quebec. (Tshiuetin owns 132.5 miles of that track --- the
stretch between Schefferville and Emeril Junction in Labrador --- and
manages passenger services for the entire line.)

\includegraphics{https://static01.nyt.com/images/2020/07/06/travel/06travel-canada-25/06travel-canada-25-mobileMasterAt3x.jpg}\includegraphics{https://static01.nyt.com/images/2020/07/06/travel/06travel-canada-26/06travel-canada-26-mobileMasterAt3x.jpg}

The rail line in autumn ...

... and after nightfall.

Schefferville was built by the Iron Ore Company of Canada to facilitate
mining in the 1950s. After the closure of the area's I.O.C. mines in the
1980s, the company had no use for the northern portion of its railway.
Since 2005, it has been run by the three First Nations that it connects:
the Innu nations of Uashat Mak Mani-utenam and Matimekush-Lac John, and
the Naskapi nation of Kawawachikamach. With a mandated 85 percent
Indigenous work force, Tshiuetin is now a symbol of reclamation and
defiance for those it serves.

Image

Cynthia Pien travels on the southbound Tshiuetin train.

Image

A passenger walks to his cabin after disembarking the Tshiuetin train.
Without regularly scheduled stops between the railway's northern and
southern termini, passengers notify train staff in advance of the mile
at which they wish to disembark.

Andy-Greg Jérôme, an Innu employee who joined Tshiuetin in 2012 at the
age of 22, said the company gave him leadership experience. ``Best thing
I did in my life,'' he said of his job. ``I am proud to work for a local
company. I hope that in the future we can have more employees who come
from the three communities.''

Image

Families often bring blankets and sheets to cover their seats --- along
with food, drinks, phones, tablets and even the occasional monitor for
group movie viewings or video games.

The combined population of Schefferville and Matimekush-Lac John is
around 800, and the Tshiuetin line plays a central role in the
community. The passenger train normally completes its round-trip journey
twice a week, with its estimated arrival times announced on the local
Schefferville radio. (Since the town lies outside the country's
provincial road network, a costly plane trip is the only alternative way
in or out.)

During the coronavirus pandemic, the frequency of trips has been
limited, and restrictions have been placed on passengers' reasons for
travel.

Image

Schefferville was established in the 1950s by the Iron Ore Company of
Canada to support nearby mining projects. The town had 4,500 residents
then, as well as its own movie theatre, community centre, hospital, and
Hudson's Bay department store.

Image

Bla Bla is a one of the few restaurants in Schefferville, and is
frequented both by residents of Schefferville and nearby
Kawawachikamach.

Image

After the Schefferville I.O.C. mines closed in 1982 (due to falling
iron-ore prices), workers moved elsewhere, and the town's population
plummeted. The I.O.C. dismantled much of what it had built, leaving vast
vacant lots where buildings used to be.

On any given trip on the Tshiuetin train, most passengers are regulars.
Some are heading to hunting grounds --- like Stéphane Lessard, whom I
met en route to his friend's cabin, which he has been frequenting for 17
years.

Image

Stéphane Lessard disembarks the Tshiuetin train en route to a friend's
hunting cabin.

Image

The Jourdain family waits to board the Tshiuetin train at the side of
the tracks after spending time at a cabin.

Others are residents of Kawawachikamach or Matimekush-Lac John, near
Schefferville, traveling south to appointments, or to shop in Sept-Îles,
where goods are less expensive than they are up north.

Image

The Tshiuetin train winds its way through the boreal forest.

Passengers might also be setting off on a holiday, with Sept-Îles as a
point of departure --- like Elayna Vollant-Einish and her family, whom I
saw several times on the train, including after a celebratory trip to
Québec City to mark her brother Shane's high school graduation.

Image

Elayna Vollant-Einish, right, travels home from a family trip to
celebrate her brother's high school graduation.

On my many trips aboard the Tshiuetin train, I have met passengers like
Gary Einish and Cynthia Pien, traveling with young children and equipped
for the long day ahead. They bring whatever comforts they need to
transform their seats into their own temporary living rooms.

\includegraphics{https://static01.nyt.com/images/2020/07/06/travel/06travel-canada-13/06travel-canada-13-mobileMasterAt3x-v4.jpg}\includegraphics{https://static01.nyt.com/images/2020/07/06/travel/06travel-canada-15/06travel-canada-15-mobileMasterAt3x-v2.jpg}\includegraphics{https://static01.nyt.com/images/2020/07/06/travel/06travel-canada-16/06travel-canada-16-mobileMasterAt3x-v2.jpg}

Gary Einish holds his cousin up to the window on a southbound trip to
Sept-Îles, Quebec.

The scenery draws a passenger's eyes away from the screen of her
e-device.

A young child gazes out the window on a foggy summer evening as the
Tshiuetin train nears Sept-Îles.

There aren't exactly formal stops between Schefferville and Sept-Îles;
the only passengers who board or depart en route are either railway
workers or those who have been at their hunting cabins, and who catch
the train at the sides of the tracks. Along the route, the train becomes
a sea of patterned sheets and blankets; passengers know the surrounding
seats will likely be theirs for the duration of the journey.

Image

On a northbound journey, toward Schefferville.

``I feel like Kawawachikamach wouldn't be where it is today without the
train,'' says Shane Vollant-Einish. ``It's the lifeblood, the main
artery for here.''

Shane will soon set off on a cross-country trip to study in British
Columbia, on Canada's West Coast. ``The train won't be as important for
me as it used to be. Instead of waiting every Thursday for fresh fruits
and vegetables, they'll be available readily, and instead of waiting
over two weeks for mail, it'll be next day,'' he said.

Image

The Tshiuetin railway tracks seen from Schefferville, across Knob Lake.

``I guess the train will symbolize how willing we are to live in
ancestral lands and walk on the same hills and ice covered lakes as our
forefathers did,'' he added. ``The train always will mean home for me.''

\begin{center}\rule{0.5\linewidth}{\linethickness}\end{center}

\emph{Chloë Ellingson is a documentary photographer who lives in
Toronto. You can follow her work on}
\href{https://www.instagram.com/chloeellingson/}{\emph{Instagram}}
\emph{and}
\href{https://twitter.com/chloeellingson}{\emph{Twitter}}\emph{.}

\emph{\textbf{Follow New York Times Travel}} \emph{on}
\href{https://www.instagram.com/nytimestravel/}{\emph{Instagram}}\emph{,}
\href{https://twitter.com/nytimestravel}{\emph{Twitter}} \emph{and}
\href{https://www.facebook.com/nytimestravel/}{\emph{Facebook}}\emph{.
And}
\href{https://www.nytimes.com/newsletters/traveldispatch}{\emph{sign up
for our weekly Travel Dispatch newsletter}} \emph{to receive expert tips
on traveling smarter and inspiration for your next vacation.}

Advertisement

\protect\hyperlink{after-bottom}{Continue reading the main story}

\hypertarget{site-index}{%
\subsection{Site Index}\label{site-index}}

\hypertarget{site-information-navigation}{%
\subsection{Site Information
Navigation}\label{site-information-navigation}}

\begin{itemize}
\tightlist
\item
  \href{https://help.nytimes.com/hc/en-us/articles/115014792127-Copyright-notice}{©~2020~The
  New York Times Company}
\end{itemize}

\begin{itemize}
\tightlist
\item
  \href{https://www.nytco.com/}{NYTCo}
\item
  \href{https://help.nytimes.com/hc/en-us/articles/115015385887-Contact-Us}{Contact
  Us}
\item
  \href{https://www.nytco.com/careers/}{Work with us}
\item
  \href{https://nytmediakit.com/}{Advertise}
\item
  \href{http://www.tbrandstudio.com/}{T Brand Studio}
\item
  \href{https://www.nytimes.com/privacy/cookie-policy\#how-do-i-manage-trackers}{Your
  Ad Choices}
\item
  \href{https://www.nytimes.com/privacy}{Privacy}
\item
  \href{https://help.nytimes.com/hc/en-us/articles/115014893428-Terms-of-service}{Terms
  of Service}
\item
  \href{https://help.nytimes.com/hc/en-us/articles/115014893968-Terms-of-sale}{Terms
  of Sale}
\item
  \href{https://spiderbites.nytimes.com}{Site Map}
\item
  \href{https://help.nytimes.com/hc/en-us}{Help}
\item
  \href{https://www.nytimes.com/subscription?campaignId=37WXW}{Subscriptions}
\end{itemize}
