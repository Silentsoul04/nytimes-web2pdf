Sections

SEARCH

\protect\hyperlink{site-content}{Skip to
content}\protect\hyperlink{site-index}{Skip to site index}

\href{https://www.nytimes.com/section/us}{U.S.}

\href{https://myaccount.nytimes.com/auth/login?response_type=cookie\&client_id=vi}{}

\href{https://www.nytimes.com/section/todayspaper}{Today's Paper}

\href{/section/us}{U.S.}\textbar{}As the Virus Surged, Florida Partied.
Tracking the Revelers Has Been Tough.

\url{https://nyti.ms/2Z3QEc8}

\begin{itemize}
\item
\item
\item
\item
\item
\item
\end{itemize}

\href{https://www.nytimes.com/news-event/coronavirus?action=click\&pgtype=Article\&state=default\&region=TOP_BANNER\&context=storylines_menu}{The
Coronavirus Outbreak}

\begin{itemize}
\tightlist
\item
  live\href{https://www.nytimes.com/2020/08/04/world/coronavirus-cases.html?action=click\&pgtype=Article\&state=default\&region=TOP_BANNER\&context=storylines_menu}{Latest
  Updates}
\item
  \href{https://www.nytimes.com/interactive/2020/us/coronavirus-us-cases.html?action=click\&pgtype=Article\&state=default\&region=TOP_BANNER\&context=storylines_menu}{Maps
  and Cases}
\item
  \href{https://www.nytimes.com/interactive/2020/science/coronavirus-vaccine-tracker.html?action=click\&pgtype=Article\&state=default\&region=TOP_BANNER\&context=storylines_menu}{Vaccine
  Tracker}
\item
  \href{https://www.nytimes.com/2020/08/02/us/covid-college-reopening.html?action=click\&pgtype=Article\&state=default\&region=TOP_BANNER\&context=storylines_menu}{College
  Reopening}
\item
  \href{https://www.nytimes.com/live/2020/08/04/business/stock-market-today-coronavirus?action=click\&pgtype=Article\&state=default\&region=TOP_BANNER\&context=storylines_menu}{Economy}
\end{itemize}

Advertisement

\protect\hyperlink{after-top}{Continue reading the main story}

Supported by

\protect\hyperlink{after-sponsor}{Continue reading the main story}

\hypertarget{as-the-virus-surged-florida-partied-tracking-the-revelers-has-been-tough}{%
\section{As the Virus Surged, Florida Partied. Tracking the Revelers Has
Been
Tough.}\label{as-the-virus-surged-florida-partied-tracking-the-revelers-has-been-tough}}

With the coronavirus exploding, trying to trace the contacts of every
positive case has become unrealistic in Florida, especially among
elusive partygoers.

\includegraphics{https://static01.nyt.com/images/2020/07/05/us/00virus-florida01/00virus-florida01-articleLarge.jpg?quality=75\&auto=webp\&disable=upscale}

\href{https://www.nytimes.com/by/patricia-mazzei}{\includegraphics{https://static01.nyt.com/images/2018/11/28/multimedia/author-patricia-mazzei/author-patricia-mazzei-thumbLarge.png}}

By \href{https://www.nytimes.com/by/patricia-mazzei}{Patricia Mazzei}

\begin{itemize}
\item
  Published July 6, 2020Updated July 20, 2020
\item
  \begin{itemize}
  \item
  \item
  \item
  \item
  \item
  \item
  \end{itemize}
\end{itemize}

MIAMI --- Miami's flashy nightclubs closed in March, but the parties
have raged on in the waterfront manse tucked in the lush residential
neighborhood of Belle Meade Island. Revelers arrive in sports cars and
ride-shares several nights a week, say neighbors who have spied
professional bouncers at the door and bought earplugs to try to sleep
through the thumping dance beats.

They are the sort of parties --- drawing throngs of maskless strangers
to rave until sunrise --- that local health officials say have been a
notable contributing factor to the
\href{https://www.nytimes.com/2020/06/28/us/coronavirus-florida-miami.html}{soaring
number of coronavirus cases} in Florida, one of the most troubling
infection spots in the country.

Just how many parties have been linked to
\href{https://www.nytimes.com/2020/07/20/us/coronavirus-florida-elderly.html}{Covid-19}
is unclear because
\href{https://www.nytimes.com/2020/07/20/us/coronavirus-florida-elderly.html}{Florida}
does not make public information about confirmed disease clusters. On
Belle Meade Island, neighbors fear the large numbers of people going in
and out of the house parties are precisely what public health officials
have warned them about.

``We have hundreds of people coming onto this island,'' said Jeri
Klemme-Zaiac, a nurse practitioner who has lived in the neighborhood for
25 years. ``This is how this is spreading: People have no regard for
anyone else.''

The city of Miami and the Miami-Dade Police Department shut down a party
at the house just before midnight on Wednesday, a spokesman for the
department said. Officers kicked out perhaps a hundred people, estimated
Rita Lagace, who lives next door and saw the attendees reluctantly
depart. She predicted the festivities would soon return: Targeting loud
parties has always been a game of whack-a-mole in Miami, a city famous
for its dazzling nightlife.

But the quest to end parties and other social gatherings has gained new
urgency because of the
\href{https://www.nytimes.com/2020/06/26/us/coronavirus-florida-texas-bars-closing.html}{exploding
coronavirus in Florida}, which reported more than 10,000 new cases on
Sunday. The state's contact tracers, already overwhelmed by the surging
number of new cases, have found it especially difficult to track how the
virus jumped from one party guest to the next because some infected
people refused to divulge whom they went out with or had over to their
house.

``We are starting to encounter a fair amount of pushback from younger
folks when you call them up and say, `We want to know everyone who was
at your party,''' said Dr. J. Glenn Morris Jr., director of the Emerging
Pathogens Institute at the University of Florida in Gainesville, a
college town where local officials have begged students to stop
partying. ``There's very much a sense of, `That's none of your
business.'''

On Monday, Miami-Dade County's mayor, Carlos Gimenez, announced that he
would close restaurants, other than takeout and delivery services, along
with ballrooms, banquet facilities, party venues, gyms and fitness
centers. The short-term rentals that often turn into party venues were
also included in the order.

Later on Monday, the mayor relented and said he would allow outdoor
dining at tables of no more than four people, ``with music played at a
level that does not require shouting'' in order to minimize the emission
of airborne droplets.

\hypertarget{latest-updates-global-coronavirus-outbreak}{%
\section{\texorpdfstring{\href{https://www.nytimes.com/2020/08/04/world/coronavirus-cases.html?action=click\&pgtype=Article\&state=default\&region=MAIN_CONTENT_1\&context=storylines_live_updates}{Latest
Updates: Global Coronavirus
Outbreak}}{Latest Updates: Global Coronavirus Outbreak}}\label{latest-updates-global-coronavirus-outbreak}}

Updated 2020-08-04T20:57:54.346Z

\begin{itemize}
\tightlist
\item
  \href{https://www.nytimes.com/2020/08/04/world/coronavirus-cases.html?action=click\&pgtype=Article\&state=default\&region=MAIN_CONTENT_1\&context=storylines_live_updates\#link-1228a480}{Novavax
  sees encouraging results from two studies of its experimental
  vaccine.}
\item
  \href{https://www.nytimes.com/2020/08/04/world/coronavirus-cases.html?action=click\&pgtype=Article\&state=default\&region=MAIN_CONTENT_1\&context=storylines_live_updates\#link-4825b93}{Public
  and private schools in Maryland and elsewhere are divided over
  in-person instruction.}
\item
  \href{https://www.nytimes.com/2020/08/04/world/coronavirus-cases.html?action=click\&pgtype=Article\&state=default\&region=MAIN_CONTENT_1\&context=storylines_live_updates\#link-50f7386d}{The
  United Nations calls on policymakers to `plan thoroughly for school
  reopenings.'}
\end{itemize}

\href{https://www.nytimes.com/2020/08/04/world/coronavirus-cases.html?action=click\&pgtype=Article\&state=default\&region=MAIN_CONTENT_1\&context=storylines_live_updates}{See
more updates}

More live coverage:
\href{https://www.nytimes.com/live/2020/08/04/business/stock-market-today-coronavirus?action=click\&pgtype=Article\&state=default\&region=MAIN_CONTENT_1\&context=storylines_live_updates}{Markets}

The mayor cited a spike in cases among people aged 18 to 34 that
appeared to result, in part, from gatherings without proper social
distancing or masks. ``Contributing to the positives in that age group,
the doctors have told me, were graduation parties, gatherings at
restaurants that turned into packed parties in violation of the rules
and street protests where people could not maintain social distancing
and where not everyone was wearing facial coverings,'' he said in a
statement.

The party problem is not limited to Florida. In New York, officials in
Rockland County
\href{https://www.nytimes.com/2020/07/01/nyregion/rockland-coronavirus-party.html}{issued
subpoenas} to eight partygoers, all in their 20s, who had refused to
answer even basic questions about a party they attended, hosted by a
person who was sick. The subpoenas threatened a daily fine of up to
\$2,000. The eight people quickly complied.

In Miami, the city filed an injunction against the owner of the Belle
Meade Island party house on Monday, citing multiple zoning and building
code violations, including what the city says is unlawful use of the
property as a venue space.

Florida's cases began climbing in June, about a month after the start of
the state's economic reopening. The surge came after Memorial Day and
several weeks of protests against police brutality, though public health
officials had not publicly tied any outbreaks directly to
\href{https://www.nytimes.com/2020/04/30/us/newsom-beaches-california-coronavirus.html}{the
beaches} or the demonstrations. Instead, they said people resuming their
normal jaunts to bars, restaurants and parties had spread the virus.

Governments can force restaurants and bars to scale back or close, but
it is harder to tackle house parties --- or even define them in a way
that would grant officials jurisdiction.

``What's a house party?'' Mayor Gimenez of Miami-Dade County said last
week. ``It's very hard to control,'' he said, unless unlawful commercial
activities in residences, like cover charges, are involved.

\includegraphics{https://static01.nyt.com/images/2020/07/05/us/00virus-florida02/merlin_174006723_485b4aba-240b-4ab1-8d66-13d754693544-articleLarge.jpg?quality=75\&auto=webp\&disable=upscale}

Some of the parties have involved inviting friends of friends --- and
even random people --- on social media, making attendees challenging to
trace.

The contact tracing effort was intended to be comprehensive. Human
nature, however, has made it frustratingly narrow, its limitations
amplified in Florida by the state's failure to hire sufficient contact
tracers, test everyone who has shared close quarters with infected
people and isolate all of those who test positive, experts say.

``Contact tracing and testing is a tool for action, and that's not the
way we've been using it in the United States, for the most part,'' said
Dr. Aileen M. Marty, an infectious disease professor at Florida
International University. ``When you do it right, testing and contact
tracing can eliminate the virus from the community.''

``We failed to act,'' she said.

The socializing that followed
\href{https://www.nytimes.com/2020/06/26/nyregion/florida-coronavirus-ny.html}{Florida's
rapid economic reopening} has left the state reeling from the virus. The
Department of Health reported more than 11,400 infections on Saturday, a
record. Florida cases made up 20 percent of
\href{https://www.nytimes.com/2020/07/02/world/coronavirus-us.html?action=click\&module=Top\%20Stories\&pgtype=Homepage}{all
U.S. cases on Thursday}. Patients with Covid-19 have begun to fill up
Florida hospital wards, forcing some hospitals to
\href{https://www.nytimes.com/2020/07/01/world/coronavirus-updates.html}{scrap
elective surgeries}, as they did early on in the pandemic. More than
3,600 people have died, including
\href{https://www.miamiherald.com/news/coronavirus/article243959612.html}{an
11-year-old boy}.

Desperate local officials have adopted local mask requirements and
\href{https://www.nytimes.com/2020/07/02/us/coronavirus-fourth-of-july.html}{closed
the beaches} over the long holiday weekend. Some communities were
deploying teams to go door-to-door in the hardest hit neighborhoods,
distributing masks, bottles of hand sanitizer and fliers with
information on coronavirus symptoms and testing.

Gov. Ron DeSantis, a Republican, insisted there would be no new
shutdown, but a piecemeal rollback is still underway: The state banned
drinking at bars. Miami-Dade County ordered entertainment venues to
close again and imposed a curfew.

``If everyone is enjoying life but doing it responsibly, we're going to
be fine,'' Mr. DeSantis said on Thursday in Tampa after a visit from
Vice President Mike Pence.

The Florida Department of Health has about 1,600 students,
epidemiologists and other staff doing contact tracing, and it has hired
a contractor to bring on 600 more people, for a total of 2,200. That is
about a third of the roughly 6,400 tracers that will be needed to meet
the target of 30 tracers per 100,000 people
\href{https://www.naccho.org/uploads/full-width-images/Contact-Tracing-Statement-4-16-2020.pdf}{recommended}
by the National Association of County and City Health Officials.

\href{https://www.nytimes.com/news-event/coronavirus?action=click\&pgtype=Article\&state=default\&region=MAIN_CONTENT_3\&context=storylines_faq}{}

\hypertarget{the-coronavirus-outbreak-}{%
\subsubsection{The Coronavirus Outbreak
›}\label{the-coronavirus-outbreak-}}

\hypertarget{frequently-asked-questions}{%
\paragraph{Frequently Asked
Questions}\label{frequently-asked-questions}}

Updated August 4, 2020

\begin{itemize}
\item ~
  \hypertarget{i-have-antibodies-am-i-now-immune}{%
  \paragraph{I have antibodies. Am I now
  immune?}\label{i-have-antibodies-am-i-now-immune}}

  \begin{itemize}
  \tightlist
  \item
    As of right
    now,\href{https://www.nytimes.com/2020/07/22/health/covid-antibodies-herd-immunity.html?action=click\&pgtype=Article\&state=default\&region=MAIN_CONTENT_3\&context=storylines_faq}{that
    seems likely, for at least several months.} There have been
    frightening accounts of people suffering what seems to be a second
    bout of Covid-19. But experts say these patients may have a
    drawn-out course of infection, with the virus taking a slow toll
    weeks to months after initial exposure. People infected with the
    coronavirus typically
    \href{https://www.nature.com/articles/s41586-020-2456-9}{produce}
    immune molecules called antibodies, which are
    \href{https://www.nytimes.com/2020/05/07/health/coronavirus-antibody-prevalence.html?action=click\&pgtype=Article\&state=default\&region=MAIN_CONTENT_3\&context=storylines_faq}{protective
    proteins made in response to an
    infection}\href{https://www.nytimes.com/2020/05/07/health/coronavirus-antibody-prevalence.html?action=click\&pgtype=Article\&state=default\&region=MAIN_CONTENT_3\&context=storylines_faq}{.
    These antibodies may} last in the body
    \href{https://www.nature.com/articles/s41591-020-0965-6}{only two to
    three months}, which may seem worrisome, but that's perfectly normal
    after an acute infection subsides, said Dr. Michael Mina, an
    immunologist at Harvard University. It may be possible to get the
    coronavirus again, but it's highly unlikely that it would be
    possible in a short window of time from initial infection or make
    people sicker the second time.
  \end{itemize}
\item ~
  \hypertarget{im-a-small-business-owner-can-i-get-relief}{%
  \paragraph{I'm a small-business owner. Can I get
  relief?}\label{im-a-small-business-owner-can-i-get-relief}}

  \begin{itemize}
  \tightlist
  \item
    The
    \href{https://www.nytimes.com/article/small-business-loans-stimulus-grants-freelancers-coronavirus.html?action=click\&pgtype=Article\&state=default\&region=MAIN_CONTENT_3\&context=storylines_faq}{stimulus
    bills enacted in March} offer help for the millions of American
    small businesses. Those eligible for aid are businesses and
    nonprofit organizations with fewer than 500 workers, including sole
    proprietorships, independent contractors and freelancers. Some
    larger companies in some industries are also eligible. The help
    being offered, which is being managed by the Small Business
    Administration, includes the Paycheck Protection Program and the
    Economic Injury Disaster Loan program. But lots of folks have
    \href{https://www.nytimes.com/interactive/2020/05/07/business/small-business-loans-coronavirus.html?action=click\&pgtype=Article\&state=default\&region=MAIN_CONTENT_3\&context=storylines_faq}{not
    yet seen payouts.} Even those who have received help are confused:
    The rules are draconian, and some are stuck sitting on
    \href{https://www.nytimes.com/2020/05/02/business/economy/loans-coronavirus-small-business.html?action=click\&pgtype=Article\&state=default\&region=MAIN_CONTENT_3\&context=storylines_faq}{money
    they don't know how to use.} Many small-business owners are getting
    less than they expected or
    \href{https://www.nytimes.com/2020/06/10/business/Small-business-loans-ppp.html?action=click\&pgtype=Article\&state=default\&region=MAIN_CONTENT_3\&context=storylines_faq}{not
    hearing anything at all.}
  \end{itemize}
\item ~
  \hypertarget{what-are-my-rights-if-i-am-worried-about-going-back-to-work}{%
  \paragraph{What are my rights if I am worried about going back to
  work?}\label{what-are-my-rights-if-i-am-worried-about-going-back-to-work}}

  \begin{itemize}
  \tightlist
  \item
    Employers have to provide
    \href{https://www.osha.gov/SLTC/covid-19/standards.html}{a safe
    workplace} with policies that protect everyone equally.
    \href{https://www.nytimes.com/article/coronavirus-money-unemployment.html?action=click\&pgtype=Article\&state=default\&region=MAIN_CONTENT_3\&context=storylines_faq}{And
    if one of your co-workers tests positive for the coronavirus, the
    C.D.C.} has said that
    \href{https://www.cdc.gov/coronavirus/2019-ncov/community/guidance-business-response.html}{employers
    should tell their employees} -\/- without giving you the sick
    employee's name -\/- that they may have been exposed to the virus.
  \end{itemize}
\item ~
  \hypertarget{should-i-refinance-my-mortgage}{%
  \paragraph{Should I refinance my
  mortgage?}\label{should-i-refinance-my-mortgage}}

  \begin{itemize}
  \tightlist
  \item
    \href{https://www.nytimes.com/article/coronavirus-money-unemployment.html?action=click\&pgtype=Article\&state=default\&region=MAIN_CONTENT_3\&context=storylines_faq}{It
    could be a good idea,} because mortgage rates have
    \href{https://www.nytimes.com/2020/07/16/business/mortgage-rates-below-3-percent.html?action=click\&pgtype=Article\&state=default\&region=MAIN_CONTENT_3\&context=storylines_faq}{never
    been lower.} Refinancing requests have pushed mortgage applications
    to some of the highest levels since 2008, so be prepared to get in
    line. But defaults are also up, so if you're thinking about buying a
    home, be aware that some lenders have tightened their standards.
  \end{itemize}
\item ~
  \hypertarget{what-is-school-going-to-look-like-in-september}{%
  \paragraph{What is school going to look like in
  September?}\label{what-is-school-going-to-look-like-in-september}}

  \begin{itemize}
  \tightlist
  \item
    It is unlikely that many schools will return to a normal schedule
    this fall, requiring the grind of
    \href{https://www.nytimes.com/2020/06/05/us/coronavirus-education-lost-learning.html?action=click\&pgtype=Article\&state=default\&region=MAIN_CONTENT_3\&context=storylines_faq}{online
    learning},
    \href{https://www.nytimes.com/2020/05/29/us/coronavirus-child-care-centers.html?action=click\&pgtype=Article\&state=default\&region=MAIN_CONTENT_3\&context=storylines_faq}{makeshift
    child care} and
    \href{https://www.nytimes.com/2020/06/03/business/economy/coronavirus-working-women.html?action=click\&pgtype=Article\&state=default\&region=MAIN_CONTENT_3\&context=storylines_faq}{stunted
    workdays} to continue. California's two largest public school
    districts --- Los Angeles and San Diego --- said on July 13, that
    \href{https://www.nytimes.com/2020/07/13/us/lausd-san-diego-school-reopening.html?action=click\&pgtype=Article\&state=default\&region=MAIN_CONTENT_3\&context=storylines_faq}{instruction
    will be remote-only in the fall}, citing concerns that surging
    coronavirus infections in their areas pose too dire a risk for
    students and teachers. Together, the two districts enroll some
    825,000 students. They are the largest in the country so far to
    abandon plans for even a partial physical return to classrooms when
    they reopen in August. For other districts, the solution won't be an
    all-or-nothing approach.
    \href{https://bioethics.jhu.edu/research-and-outreach/projects/eschool-initiative/school-policy-tracker/}{Many
    systems}, including the nation's largest, New York City, are
    devising
    \href{https://www.nytimes.com/2020/06/26/us/coronavirus-schools-reopen-fall.html?action=click\&pgtype=Article\&state=default\&region=MAIN_CONTENT_3\&context=storylines_faq}{hybrid
    plans} that involve spending some days in classrooms and other days
    online. There's no national policy on this yet, so check with your
    municipal school system regularly to see what is happening in your
    community.
  \end{itemize}
\end{itemize}

With so much community spread, trying to trace the contacts of every
positive case becomes unrealistic, several public health officials said.

``We may have to change the priorities on tracing as the numbers
continue to increase, because at some point it is like drinking out of a
fire hose,'' said Dr. Raul Pino, the health department officer in
Orlando.

He has traced some contacts himself and found that many of the cases
emerged from people going out to dinner and parties.

Image

Crews of county employees fanned out to neighborhoods with the highest
concentration of cases, bringing blue tote bags containing a reusable
mask, bottles of hand sanitizer and informational
pamphlets.Credit...Saul Martinez for The New York Times

``What we have found is young individuals who went out in a group,'' he
said. ``They later learned that someone in that group was positive.''

Dalton Price, a recent college graduate who has been hired to do contact
tracing in Daytona Beach, said he and the other tracers used to have a
handful of new cases to call each day. Now they have more than a
hundred, which has shortened each interview to perhaps 10 to 20 minutes
from 30 to 45 minutes, he said. They must also spend time calling people
being monitored for their exposure, who get asked periodically whether
they have developed symptoms.

``It's kind of overwhelming,'' said Mr. Price, 22. ``We've started doing
a more expedited case investigation so if the person's symptoms aren't
super severe, we will just get a general idea of where they've been, but
we won't necessarily monitor all their contacts. We just say to them if
they could call them and tell them to self-quarantine.''

Officials in Miami-Dade, which recorded more than 1,600 cases on
Thursday, wanted to pay for additional contact tracers to work locally.
But because they must be hired by the state, the county has been unable
to grow the contact tracing force on its own.

Instead, the county assembled teams of county employees and sent them to
neighborhoods with the highest concentration of cases: less affluent
communities full of essential workers living in small, often
multigenerational homes. They carried blue tote bags each containing a
reusable mask, bottles of hand sanitizer and informational pamphlets.

Wearing masks, gloves and face shields, crews of workers fanned across
Miami's Allapattah neighborhood on a recent scorching morning and began
knocking on doors.

``After three months?'' one man said, chiding the team for not having
come by earlier in the pandemic. ``I'd be dead by now!''

Most people, however, were grateful. Narcisa Jirón, 67, hustled from her
second-floor apartment into the courtyard to get her bag and later asked
for a second mask. ``I need this like I need water,'' she said.

``They have to lock everything down now,'' said Tomás Trujillo, 47.
``Because this is just too much.''

Frances Robles contributed reporting from Key West, Fla. Sheelagh
McNeill contributed research.

Advertisement

\protect\hyperlink{after-bottom}{Continue reading the main story}

\hypertarget{site-index}{%
\subsection{Site Index}\label{site-index}}

\hypertarget{site-information-navigation}{%
\subsection{Site Information
Navigation}\label{site-information-navigation}}

\begin{itemize}
\tightlist
\item
  \href{https://help.nytimes.com/hc/en-us/articles/115014792127-Copyright-notice}{©~2020~The
  New York Times Company}
\end{itemize}

\begin{itemize}
\tightlist
\item
  \href{https://www.nytco.com/}{NYTCo}
\item
  \href{https://help.nytimes.com/hc/en-us/articles/115015385887-Contact-Us}{Contact
  Us}
\item
  \href{https://www.nytco.com/careers/}{Work with us}
\item
  \href{https://nytmediakit.com/}{Advertise}
\item
  \href{http://www.tbrandstudio.com/}{T Brand Studio}
\item
  \href{https://www.nytimes.com/privacy/cookie-policy\#how-do-i-manage-trackers}{Your
  Ad Choices}
\item
  \href{https://www.nytimes.com/privacy}{Privacy}
\item
  \href{https://help.nytimes.com/hc/en-us/articles/115014893428-Terms-of-service}{Terms
  of Service}
\item
  \href{https://help.nytimes.com/hc/en-us/articles/115014893968-Terms-of-sale}{Terms
  of Sale}
\item
  \href{https://spiderbites.nytimes.com}{Site Map}
\item
  \href{https://help.nytimes.com/hc/en-us}{Help}
\item
  \href{https://www.nytimes.com/subscription?campaignId=37WXW}{Subscriptions}
\end{itemize}
