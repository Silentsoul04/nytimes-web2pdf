Sections

SEARCH

\protect\hyperlink{site-content}{Skip to
content}\protect\hyperlink{site-index}{Skip to site index}

\href{https://www.nytimes.com/section/us}{U.S.}

\href{https://myaccount.nytimes.com/auth/login?response_type=cookie\&client_id=vi}{}

\href{https://www.nytimes.com/section/todayspaper}{Today's Paper}

\href{/section/us}{U.S.}\textbar{}Colleges Plan to Reopen Campuses, but
for Just Some Students at a Time

\url{https://nyti.ms/38xTpFN}

\begin{itemize}
\item
\item
\item
\item
\item
\item
\end{itemize}

\href{https://www.nytimes.com/news-event/coronavirus?action=click\&pgtype=Article\&state=default\&region=TOP_BANNER\&context=storylines_menu}{The
Coronavirus Outbreak}

\begin{itemize}
\tightlist
\item
  live\href{https://www.nytimes.com/2020/08/02/world/coronavirus-updates.html?action=click\&pgtype=Article\&state=default\&region=TOP_BANNER\&context=storylines_menu}{Latest
  Updates}
\item
  \href{https://www.nytimes.com/interactive/2020/us/coronavirus-us-cases.html?action=click\&pgtype=Article\&state=default\&region=TOP_BANNER\&context=storylines_menu}{Maps
  and Cases}
\item
  \href{https://www.nytimes.com/interactive/2020/science/coronavirus-vaccine-tracker.html?action=click\&pgtype=Article\&state=default\&region=TOP_BANNER\&context=storylines_menu}{Vaccine
  Tracker}
\item
  \href{https://www.nytimes.com/interactive/2020/07/29/us/schools-reopening-coronavirus.html?action=click\&pgtype=Article\&state=default\&region=TOP_BANNER\&context=storylines_menu}{What
  School May Look Like}
\item
  \href{https://www.nytimes.com/live/2020/07/31/business/stock-market-today-coronavirus?action=click\&pgtype=Article\&state=default\&region=TOP_BANNER\&context=storylines_menu}{Economy}
\end{itemize}

Advertisement

\protect\hyperlink{after-top}{Continue reading the main story}

Supported by

\protect\hyperlink{after-sponsor}{Continue reading the main story}

\hypertarget{colleges-plan-to-reopen-campuses-but-for-just-some-students-at-a-time}{%
\section{Colleges Plan to Reopen Campuses, but for Just Some Students at
a
Time}\label{colleges-plan-to-reopen-campuses-but-for-just-some-students-at-a-time}}

To provide some semblance of the campus experience during a pandemic,
colleges say large chunks of the student body will have to stay away and
study remotely for all or part of the year.

\includegraphics{https://static01.nyt.com/images/2020/07/06/us/06VIRUS-UNIVERSITIES/merlin_172952781_3cb0a600-f181-4bfe-8f35-b6f95d3a7f79-articleLarge.jpg?quality=75\&auto=webp\&disable=upscale}

\href{https://www.nytimes.com/by/anemona-hartocollis}{\includegraphics{https://static01.nyt.com/images/2018/06/13/multimedia/author-anemona-hartocollis/author-anemona-hartocollis-thumbLarge-v3.jpg}}

By \href{https://www.nytimes.com/by/anemona-hartocollis}{Anemona
Hartocollis}

\begin{itemize}
\item
  Published July 6, 2020Updated July 8, 2020
\item
  \begin{itemize}
  \item
  \item
  \item
  \item
  \item
  \item
  \end{itemize}
\end{itemize}

With the
\href{https://www.nytimes.com/2020/07/08/sports/coronavirus-stanford-cuts.html}{coronavirus}
still raging and the fall semester approaching, colleges and
universities are telling large segments of their student populations to
stay home. Those who are allowed on campus, they say, will be living in
a world where parties are banned, where everyone is frequently tested
for the coronavirus and --- perhaps most draconian of all --- where
students attend many if not all their courses remotely, from their dorm
rooms.

In order to achieve social distancing, many colleges are saying they
will allow only 40 to 60 percent of their students to return to campus
and live in the college residence halls at any one time, often divided
by class year.

\href{https://www.nytimes.com/2020/07/08/sports/coronavirus-stanford-cuts.html}{Stanford}
has said freshmen and sophomores will be on campus when classes start in
the fall, while juniors and seniors study remotely from home. Harvard
announced on Monday that it will mainly be first-year students and some
students in special circumstances who will be there in the fall; in the
spring, freshmen will leave and it will be seniors' turn.

At the same time, very few colleges are offering tuition discounts, even
for those students being forced to take classes from home.

Professors, students and parents all seem to be conflicted over how
these plans will work out.

Pascale Bradley, a senior studying English and French literature at
Yale, is just looking forward to seeing some classmates again. Yale is
allowing first-year students, juniors and seniors on campus in the fall,
but nearly all classes will be taught remotely.

``It won't be the same social life,'' she said. ``Not that students are
upset there might not be big parties. People are just looking forward to
daily small interactions, being able to sit and study with someone and
have a meal with someone.''

\hypertarget{latest-updates-global-coronavirus-outbreak}{%
\section{\texorpdfstring{\href{https://www.nytimes.com/2020/08/01/world/coronavirus-covid-19.html?action=click\&pgtype=Article\&state=default\&region=MAIN_CONTENT_1\&context=storylines_live_updates}{Latest
Updates: Global Coronavirus
Outbreak}}{Latest Updates: Global Coronavirus Outbreak}}\label{latest-updates-global-coronavirus-outbreak}}

Updated 2020-08-02T17:52:35.962Z

\begin{itemize}
\tightlist
\item
  \href{https://www.nytimes.com/2020/08/01/world/coronavirus-covid-19.html?action=click\&pgtype=Article\&state=default\&region=MAIN_CONTENT_1\&context=storylines_live_updates\#link-34047410}{The
  U.S. reels as July cases more than double the total of any other
  month.}
\item
  \href{https://www.nytimes.com/2020/08/01/world/coronavirus-covid-19.html?action=click\&pgtype=Article\&state=default\&region=MAIN_CONTENT_1\&context=storylines_live_updates\#link-780ec966}{Top
  U.S. officials work to break an impasse over the federal jobless
  benefit.}
\item
  \href{https://www.nytimes.com/2020/08/01/world/coronavirus-covid-19.html?action=click\&pgtype=Article\&state=default\&region=MAIN_CONTENT_1\&context=storylines_live_updates\#link-2bc8948}{Its
  outbreak untamed, Melbourne goes into even greater lockdown.}
\end{itemize}

\href{https://www.nytimes.com/2020/08/01/world/coronavirus-covid-19.html?action=click\&pgtype=Article\&state=default\&region=MAIN_CONTENT_1\&context=storylines_live_updates}{See
more updates}

More live coverage:
\href{https://www.nytimes.com/live/2020/07/31/business/stock-market-today-coronavirus?action=click\&pgtype=Article\&state=default\&region=MAIN_CONTENT_1\&context=storylines_live_updates}{Markets}

Her father, Kirby Bradley, is less forgiving. ``This just seems to be
the worst of all worlds,'' said Mr. Bradley, who owns a video production
company. ``They are exposing the kids to increased virus risk, something
that is arguably justifiable in exchange for in-person learning, which
everyone agrees is better than online. But no, the kids will do remote
learning, from campus! At full tuition!''

College administrators say they are in a bind and doing the best they
can to bring students back to campus to get at least some of the social
and academic benefits of being surrounded by their peers.

``This pandemic is among the worst crises ever to hit Princeton, or
college education more broadly,'' Christopher L. Eisgruber, president of
Princeton, said in his reopening announcement. ``Princeton's preferred
model of education emphasizes in-person engagement, but in-person
engagement is what spreads this terrible virus.''

Princeton is one of the few universities that has said it would offer a
tuition discount this fall because of the limitations. Students, whether
on campus or learning remotely, will be charged 10 percent less
---~\$48,501 for the coming year, instead of \$53,890, according to a
spokesman, Ben Chang. It was unclear how students receiving financial
aid ---~who account for more than 60 percent of undergraduates~--- would
be affected.

Princeton said it was instituting the policy because most undergraduates
would be on campus only half the year ---~freshmen and juniors in the
fall, sophomores and seniors in the spring.

Harvard University announced on Monday that no more than 40 percent of
its undergraduates would be allowed on campus at a time during the next
academic year, but that tuition would remain the same. All first-year
students would be allowed in the fall semester, along with some students
in other years whose home environments are not conducive to learning;
the freshmen would leave in the spring to make space for seniors to
finish and graduate on campus.

At Harvard, all classes will be held online, even for students living on
campus. While it is not discounting its tuition and fees ---~about
\$54,000 for the year ---~the university said it would offer a summer
term next year of two tuition-free courses for all students who had to
study away from campus for the full academic year.

Cutting down the number of students on campus will allow many colleges
to offer everyone a single or double dormitory bedroom. Students are
being told they will have to eat takeout meals from dining halls in
their rooms, or perhaps make a reservation to eat.

Many universities are requiring behavioral contracts in which students
agree to wear face masks in public, to be tested regularly for the
coronavirus, and to limit travel and socializing. If they break the
rules, they can be disciplined.

Universities say they are keeping a tight rein on students because the
trajectory of the virus is still uncertain. Several universities cited
the recent surge in virus cases in some states as justification for
keeping classes virtual, even for students living on campus.

\href{https://www.nytimes.com/2020/07/03/us/coronavirus-college-professors.html}{Faculty
members are also worried}. More than 850 members of the Georgia Tech
faculty have signed a letter opposing the school's reopening plans for
the fall, which say that wearing face masks on campus would not be
mandatory, just ``strongly encouraged.''

\href{https://www.nytimes.com/news-event/coronavirus?action=click\&pgtype=Article\&state=default\&region=MAIN_CONTENT_3\&context=storylines_faq}{}

\hypertarget{the-coronavirus-outbreak-}{%
\subsubsection{The Coronavirus Outbreak
›}\label{the-coronavirus-outbreak-}}

\hypertarget{frequently-asked-questions}{%
\paragraph{Frequently Asked
Questions}\label{frequently-asked-questions}}

Updated July 27, 2020

\begin{itemize}
\item ~
  \hypertarget{should-i-refinance-my-mortgage}{%
  \paragraph{Should I refinance my
  mortgage?}\label{should-i-refinance-my-mortgage}}

  \begin{itemize}
  \tightlist
  \item
    \href{https://www.nytimes.com/article/coronavirus-money-unemployment.html?action=click\&pgtype=Article\&state=default\&region=MAIN_CONTENT_3\&context=storylines_faq}{It
    could be a good idea,} because mortgage rates have
    \href{https://www.nytimes.com/2020/07/16/business/mortgage-rates-below-3-percent.html?action=click\&pgtype=Article\&state=default\&region=MAIN_CONTENT_3\&context=storylines_faq}{never
    been lower.} Refinancing requests have pushed mortgage applications
    to some of the highest levels since 2008, so be prepared to get in
    line. But defaults are also up, so if you're thinking about buying a
    home, be aware that some lenders have tightened their standards.
  \end{itemize}
\item ~
  \hypertarget{what-is-school-going-to-look-like-in-september}{%
  \paragraph{What is school going to look like in
  September?}\label{what-is-school-going-to-look-like-in-september}}

  \begin{itemize}
  \tightlist
  \item
    It is unlikely that many schools will return to a normal schedule
    this fall, requiring the grind of
    \href{https://www.nytimes.com/2020/06/05/us/coronavirus-education-lost-learning.html?action=click\&pgtype=Article\&state=default\&region=MAIN_CONTENT_3\&context=storylines_faq}{online
    learning},
    \href{https://www.nytimes.com/2020/05/29/us/coronavirus-child-care-centers.html?action=click\&pgtype=Article\&state=default\&region=MAIN_CONTENT_3\&context=storylines_faq}{makeshift
    child care} and
    \href{https://www.nytimes.com/2020/06/03/business/economy/coronavirus-working-women.html?action=click\&pgtype=Article\&state=default\&region=MAIN_CONTENT_3\&context=storylines_faq}{stunted
    workdays} to continue. California's two largest public school
    districts --- Los Angeles and San Diego --- said on July 13, that
    \href{https://www.nytimes.com/2020/07/13/us/lausd-san-diego-school-reopening.html?action=click\&pgtype=Article\&state=default\&region=MAIN_CONTENT_3\&context=storylines_faq}{instruction
    will be remote-only in the fall}, citing concerns that surging
    coronavirus infections in their areas pose too dire a risk for
    students and teachers. Together, the two districts enroll some
    825,000 students. They are the largest in the country so far to
    abandon plans for even a partial physical return to classrooms when
    they reopen in August. For other districts, the solution won't be an
    all-or-nothing approach.
    \href{https://bioethics.jhu.edu/research-and-outreach/projects/eschool-initiative/school-policy-tracker/}{Many
    systems}, including the nation's largest, New York City, are
    devising
    \href{https://www.nytimes.com/2020/06/26/us/coronavirus-schools-reopen-fall.html?action=click\&pgtype=Article\&state=default\&region=MAIN_CONTENT_3\&context=storylines_faq}{hybrid
    plans} that involve spending some days in classrooms and other days
    online. There's no national policy on this yet, so check with your
    municipal school system regularly to see what is happening in your
    community.
  \end{itemize}
\item ~
  \hypertarget{is-the-coronavirus-airborne}{%
  \paragraph{Is the coronavirus
  airborne?}\label{is-the-coronavirus-airborne}}

  \begin{itemize}
  \tightlist
  \item
    The coronavirus
    \href{https://www.nytimes.com/2020/07/04/health/239-experts-with-one-big-claim-the-coronavirus-is-airborne.html?action=click\&pgtype=Article\&state=default\&region=MAIN_CONTENT_3\&context=storylines_faq}{can
    stay aloft for hours in tiny droplets in stagnant air}, infecting
    people as they inhale, mounting scientific evidence suggests. This
    risk is highest in crowded indoor spaces with poor ventilation, and
    may help explain super-spreading events reported in meatpacking
    plants, churches and restaurants.
    \href{https://www.nytimes.com/2020/07/06/health/coronavirus-airborne-aerosols.html?action=click\&pgtype=Article\&state=default\&region=MAIN_CONTENT_3\&context=storylines_faq}{It's
    unclear how often the virus is spread} via these tiny droplets, or
    aerosols, compared with larger droplets that are expelled when a
    sick person coughs or sneezes, or transmitted through contact with
    contaminated surfaces, said Linsey Marr, an aerosol expert at
    Virginia Tech. Aerosols are released even when a person without
    symptoms exhales, talks or sings, according to Dr. Marr and more
    than 200 other experts, who
    \href{https://academic.oup.com/cid/article/doi/10.1093/cid/ciaa939/5867798}{have
    outlined the evidence in an open letter to the World Health
    Organization}.
  \end{itemize}
\item ~
  \hypertarget{what-are-the-symptoms-of-coronavirus}{%
  \paragraph{What are the symptoms of
  coronavirus?}\label{what-are-the-symptoms-of-coronavirus}}

  \begin{itemize}
  \tightlist
  \item
    Common symptoms
    \href{https://www.nytimes.com/article/symptoms-coronavirus.html?action=click\&pgtype=Article\&state=default\&region=MAIN_CONTENT_3\&context=storylines_faq}{include
    fever, a dry cough, fatigue and difficulty breathing or shortness of
    breath.} Some of these symptoms overlap with those of the flu,
    making detection difficult, but runny noses and stuffy sinuses are
    less common.
    \href{https://www.nytimes.com/2020/04/27/health/coronavirus-symptoms-cdc.html?action=click\&pgtype=Article\&state=default\&region=MAIN_CONTENT_3\&context=storylines_faq}{The
    C.D.C. has also} added chills, muscle pain, sore throat, headache
    and a new loss of the sense of taste or smell as symptoms to look
    out for. Most people fall ill five to seven days after exposure, but
    symptoms may appear in as few as two days or as many as 14 days.
  \end{itemize}
\item ~
  \hypertarget{does-asymptomatic-transmission-of-covid-19-happen}{%
  \paragraph{Does asymptomatic transmission of Covid-19
  happen?}\label{does-asymptomatic-transmission-of-covid-19-happen}}

  \begin{itemize}
  \tightlist
  \item
    So far, the evidence seems to show it does. A widely cited
    \href{https://www.nature.com/articles/s41591-020-0869-5}{paper}
    published in April suggests that people are most infectious about
    two days before the onset of coronavirus symptoms and estimated that
    44 percent of new infections were a result of transmission from
    people who were not yet showing symptoms. Recently, a top expert at
    the World Health Organization stated that transmission of the
    coronavirus by people who did not have symptoms was ``very rare,''
    \href{https://www.nytimes.com/2020/06/09/world/coronavirus-updates.html?action=click\&pgtype=Article\&state=default\&region=MAIN_CONTENT_3\&context=storylines_faq\#link-1f302e21}{but
    she later walked back that statement.}
  \end{itemize}
\end{itemize}

The Montana University System is also facing pushback from the faculty
over its mask policy.

International students may have the hardest time of all. Many have
returned to their home countries and will not be able to re-enter the
United States because of travel and visa restrictions.

U.S. Immigration and Customs Enforcement said in a statement on Monday
that student visas would not be issued to people enrolled in schools or
programs that are fully online for the fall semester. Students in such
programs will not be permitted to enter the country, and those already
in the United States ``must depart the country or take other measures,
such as transferring to a school with in-person instruction, to remain
in lawful status.''

Under pressure both to return to normal
---~\href{https://twitter.com/realDonaldTrump/status/1280209946085339136?s=20}{President
Trump wrote on Twitter} on Monday that ``Schools must open in the
fall!!!'' ---~and to keep students and faculty safe, universities are
fighting back against the perception that virtual classes are inferior.

\href{https://vimeo.com/johnshopkins/review/433786130/1ed867a018}{A
video touts the sophistication} of a studio system of creating and
recording virtual lecture classes at Johns Hopkins, where first- and
second-year students are being invited to apply for housing.

The video shows professors delivering their lectures in a large bright
studio, almost as if they were actors onstage, while the eager faces of
students look back at them from oversize video screens.

Many universities are bracing for the possibility that upperclassmen
will request leaves of absence until things return to normal. The
schools are warning students that if they do, there may not be dormitory
housing for them in a year or two when they come back.

Cornell University is bucking the trend and allowing all its students
back to campus, with a mixture of in-person and online instruction.
Cornell said it based its decision on an
\href{https://people.orie.cornell.edu/pfrazier/COVID_19_Modeling_Jun15.pdf}{analysis}
that found that conducting a semester entirely remotely could result in
far more students becoming infected --- up to 10 times as many ---
compared with reopening the campus. That is because of the likelihood
that even if classes were conducted remotely, many Cornell students
would return to off-campus housing in Ithaca, N.Y., and the university
would not be able to enforce virus testing requirements or restrictions
on their behavior.

Rutgers University said on Monday that its fall semester would combine a
majority of remotely delivered courses with a limited number of
in-person classes.

``We have wanted very fervently to be able to resume some version of a
normal semester,'' Jonathan Holloway, the university's president, said
in a
\href{http://link.mediaoutreach.meltwater.com/ls/click?upn=B7qRmy9MCgTfwcglnVfMK-2BAqVl42XzZJ6qWfrpNErMSbHn5Lc2TGqnqiYYEklxHMPmyRbN2vlvYM-2Ba4MJ05tkw-3D-3DByuE_hL7TmydlaMnATh3nUjJggvIFksBL8YdYIW6OdNBg50LtPVULmxBnuqXyyhYlYbSAfyXTCD3A7tmCA0yK8Sc2xQ1ELtNEQ-2FAeZKeeT-2FIVbyaVOW2-2FrzGsXCh8jBDZ4nJlbA7akhdGAu92-2F6x3g2iYXSIFoo-2Bf-2BYpfP399Su8omi3TYwAf3DUtypSNMAVPURTwCu2WEK3ej70cx-2F3euc8hcssOKCmXa2e1K5hFMsU3DXneL8DhASpYjb7lMr53rteDHYpfkK5SoFsB93XaAS0K7ecGaEu6fLUheBJNFjcuGT62Cw7jtQFdC-2FMPC0xoeqCgla-2F0-2BNDm2k-2Fk1E-2B2bzDFjwJ5hXPBPI9weC4Dsihb33qx6xXPOX55uldfzM7GbX0ymSuBOG6DXLCRPiqN-2BD7uxg-3D-3D}{message
to the Rutgers community}.

The University of Pennsylvania is also pursuing a ``hybrid model'' in
the fall, with classroom instruction ``where it can be provided safely
and when it is essential to the academic needs of the course
curriculum,'' while other courses like large lecture classes are held
online.

Many schools
\href{https://www.nytimes.com/2020/03/28/us/coronavirus-college-pass-fail.html}{suspended
their usual grading policies} amid the chaos of the spring term,
substituting a pass/fail system instead. Now they say they are planning
to restore normal grading policies.

Lucy Tompkins and Caitlin Dickerson contributed reporting.

Advertisement

\protect\hyperlink{after-bottom}{Continue reading the main story}

\hypertarget{site-index}{%
\subsection{Site Index}\label{site-index}}

\hypertarget{site-information-navigation}{%
\subsection{Site Information
Navigation}\label{site-information-navigation}}

\begin{itemize}
\tightlist
\item
  \href{https://help.nytimes.com/hc/en-us/articles/115014792127-Copyright-notice}{©~2020~The
  New York Times Company}
\end{itemize}

\begin{itemize}
\tightlist
\item
  \href{https://www.nytco.com/}{NYTCo}
\item
  \href{https://help.nytimes.com/hc/en-us/articles/115015385887-Contact-Us}{Contact
  Us}
\item
  \href{https://www.nytco.com/careers/}{Work with us}
\item
  \href{https://nytmediakit.com/}{Advertise}
\item
  \href{http://www.tbrandstudio.com/}{T Brand Studio}
\item
  \href{https://www.nytimes.com/privacy/cookie-policy\#how-do-i-manage-trackers}{Your
  Ad Choices}
\item
  \href{https://www.nytimes.com/privacy}{Privacy}
\item
  \href{https://help.nytimes.com/hc/en-us/articles/115014893428-Terms-of-service}{Terms
  of Service}
\item
  \href{https://help.nytimes.com/hc/en-us/articles/115014893968-Terms-of-sale}{Terms
  of Sale}
\item
  \href{https://spiderbites.nytimes.com}{Site Map}
\item
  \href{https://help.nytimes.com/hc/en-us}{Help}
\item
  \href{https://www.nytimes.com/subscription?campaignId=37WXW}{Subscriptions}
\end{itemize}
