Sections

SEARCH

\protect\hyperlink{site-content}{Skip to
content}\protect\hyperlink{site-index}{Skip to site index}

\href{https://www.nytimes.com/section/us}{U.S.}

\href{https://myaccount.nytimes.com/auth/login?response_type=cookie\&client_id=vi}{}

\href{https://www.nytimes.com/section/todayspaper}{Today's Paper}

\href{/section/us}{U.S.}\textbar{}Lobbyists, Law Firms and Trade Groups
Took Small-Business Loans

\url{https://nyti.ms/31OWIHj}

\begin{itemize}
\item
\item
\item
\item
\item
\item
\end{itemize}

Advertisement

\protect\hyperlink{after-top}{Continue reading the main story}

Supported by

\protect\hyperlink{after-sponsor}{Continue reading the main story}

\hypertarget{lobbyists-law-firms-and-trade-groups-took-small-business-loans}{%
\section{Lobbyists, Law Firms and Trade Groups Took Small-Business
Loans}\label{lobbyists-law-firms-and-trade-groups-took-small-business-loans}}

The Trump administration began releasing details of which businesses
received Paycheck Protection Program aid.

\includegraphics{https://static01.nyt.com/images/2020/07/03/business/06dc-virus-ppp-01/merlin_174150666_98a0fe57-e474-4bb8-b4fa-0f7fc9667f4c-articleLarge.jpg?quality=75\&auto=webp\&disable=upscale}

By \href{https://www.nytimes.com/by/jeanna-smialek}{Jeanna Smialek},
\href{https://www.nytimes.com/by/jim-tankersley}{Jim Tankersley} and
\href{https://www.nytimes.com/by/luke-broadwater}{Luke Broadwater}

\begin{itemize}
\item
  Published July 6, 2020Updated July 7, 2020
\item
  \begin{itemize}
  \item
  \item
  \item
  \item
  \item
  \item
  \end{itemize}
\end{itemize}

WASHINGTON --- The Trump administration,
\href{https://www.nytimes.com/2020/06/15/us/politics/coronavirus-ppp-trump-congress.html}{under
pressure} to reveal which companies received loans from a \$660 billion
program intended to keep small businesses afloat, on Monday
\href{https://home.treasury.gov/news/press-releases/sm1052}{released
data} showing that restaurants, medical offices and car dealerships
ranked high among the top loan recipients.

The detailed information from the Paycheck Protection Program was
confined to companies that received loans of more than \$150,000. The
administration said 86.5 percent of the loans were for less than that
amount, so the snapshot captured only one sliver of businesses that
tapped funds. So far, banks have made about 4.9 million loans through
the program, with an average size of \$107,000.

Nearly 5,000 businesses received individual loans between \$5 million
and \$10 million, according to the data. The administration included
ranges for the amounts, not specific figures.

And the figures did not include details on the roughly \$30 billion in
loans that were returned as companies
\href{https://www.nytimes.com/2020/06/10/business/Small-business-loans-ppp.html}{realized
that they were not eligible} for the program, worried that they couldn't
meet program requirements or reacted to a
\href{https://www.nytimes.com/2020/04/28/us/politics/coronavirus-treasury-payment-protection-program.html}{public
outcry} about big firms getting funds.

Restaurants, medical offices and car dealerships were the top recipients
of large loans from the program. More than 40,000 full- or
limited-service restaurants received loans worth as much as \$32
billion, according to the ranges provided by the government.

Sprinkled among the beneficiaries were businesses that are likely to
attract scrutiny, including a fancy sushi restaurant at the Trump
International Hotel in Washington; Kanye West's company, Yeezy; and
President Trump's longtime personal lawyer.

Washington lobbying shops, high-priced law firms and special-interest
groups also received big loans, according to the administration, the
latest indication of how the government's centerpiece effort to shore up
mom-and-pop shops set off a race by organizations far afield from Main
Street to secure federal money. The disclosure could
\href{https://www.nytimes.com/2020/04/28/us/politics/coronavirus-treasury-payment-protection-program.html}{further
fuel outrage} toward the program, which has been complicated by
revelations that large, publicly traded companies were taking big loans
and concerns that it might leave borrowers saddled with debt.

``My 1,000-foot takeaway is that the government was handing out free
money and the line went around the corner,'' said Aaron Klein, a fellow
in economic studies at the Brookings Institution in Washington. ``This
is not your mom-and-pop shop on Main Street.''

The administration said the Paycheck Protection Program had helped to
support more than 50 million jobs. The share of overall small-business
payroll supported per state ranged from 72 percent in Virginia to 96
percent in Florida, according to the Treasury Department.

The program provides forgivable loans to companies that have 500 or
fewer employees and meet certain requirements, such as using the bulk of
the money to keep workers on the payroll. Lenders are responsible for
reviewing recipients' forgiveness applications to verify that they
complied with the program's rules.

More than 100 law firms received loans ranging from \$1 million to \$10
million, the data showed. The list included well-known names like Boies
Schiller Flexner, the high-priced law firm run by David Boies, which
received between \$5 million and \$10 million. ``We don't comment on our
financials,'' the firm said.

Kasowitz Benson Torres, founded and run by Mr. Trump's longtime personal
lawyer, Marc E. Kasowitz, received a loan for between \$5 million and
\$10 million. The firm represented Mr. Trump for over a decade before he
was elected president, both in his business dealings and in other
matters, such as helping him keep
\href{https://www.kasowitz.com/evidence/media/kasowitz-successfully-defends-donald-trump-keeping-divorce-records-sealed}{divorce
records sealed}. Mr. Kasowitz and the firm also
\href{https://www.nytimes.com/2018/03/25/us/politics/trump-lawyers-digenova.html}{represented}
Mr. Trump during Robert S. Mueller III's investigation into Russian
interference in the 2016 presidential election.

Mr. Trump later
\href{https://www.nytimes.com/2018/09/18/us/politics/trump-legal-team-lawyers.html}{diminished
the role of Mr. Kasowitz} in his dealings with Mr. Mueller's
investigators. A spokeswoman for Kasowitz Benson Torres said the loan,
along with cost-cutting, ``enabled us to preserve the jobs of our
hundreds of employees at full salary and benefits without
interruption.''

The president also appears to have benefited from government support, at
least indirectly. While the Trump Organization did not apply for loans
under the program, the data showed that dozens of tenants at buildings
owned by Mr. Trump or managed by his companies received funds.

One reported recipient was a hair salon in the president's hotel in
Chicago. More than 20 businesses listed at 40 Wall Street, an office
building near the New York Stock Exchange that Mr. Trump has owned since
the mid-1990s, also reportedly received government loans totaling at
least \$20 million. Among the recipients were law offices, financial
service firms and nonprofit organizations.

Sushi Nakazawa, a restaurant at the Trump International Hotel in
Washington, received a loan of between \$150,000 and \$350,000. The
company did not respond to a request for comment about how it planned to
use the funds.

Some loan recipients are connected to the president's son-in-law and
senior adviser, Jared Kushner. The data show that a loan of between
\$350,000 and \$1 million was made to Esplanade Livingston, a Kushner
family entity that owns the land in Livingston, N.J., where the family's
Westminster Hotel is. In 2018, Mr. Kushner divested his stake in the
entity, from which he once derived income generated by that hotel.

Princeton Forrestal, a real estate entity owned by various members of
the Kushner family not including Mr. Kushner, received a loan of between
\$1 million and \$2 million.

``Several of our hotels have applied for federal loans, in accordance
with all guidelines, with a vast majority of funds going to furloughed
employees,'' said Pete Febo, Kushner Companies' chief operating officer.

The Paycheck Protection Program, which was included in the \$2 trillion
stimulus bill passed by lawmakers in March, also benefited several
members of Congress.

Car dealerships connected to Representative Mike Kelly, Republican of
Pennsylvania, received three loans, each between \$150,000 and
\$350,000. Mr. Kelly, a multimillionaire, owns Mike Kelly Automotive
Group, Mike Kelly Automotive L.P.; and Mike Kelly Hyundai, all of which
accepted loans.

Andrew Eisenberger, spokesman for Mr. Kelly, said that the congressman
had properly followed the rules of the disaster aid program and that the
money would be used ``to sustain the income of workers who would
otherwise have been without pay or employment at no fault of their own
during the coronavirus pandemic.''

Businesses associated with other members of Congress or their relatives,
both Democrats and Republicans, received similar disaster aid. They
included a farm and other businesses owned by Representative Vicky
Harzler, Republican of Missouri, and her husband. Ms. Harzler
\href{https://hartzler.house.gov/media-center/press-releases/hartzler-small-business-among-47000-missouri-companies-utilize-paycheck}{cited
the need} ``to ensure the continued ability to maintain the employment
of all team members during this time.''

A New York shipping business owned by the family of Transportation
Secretary Elaine Chao, the wife of the Senate majority leader, Mitch
McConnell, received at least \$350,000, according to the data. A person
familiar with the company,
\href{https://www.nytimes.com/2019/09/16/us/politics/elaine-chao-house-investigation.html}{Foremost
Group}, said that the loan was for less than \$500,000 and that no
employees had been laid off during the pandemic. Ms. Chao has no formal
affiliation or stake in the business. In 2008, she and Mr. McConnell
received millions of dollars in gifts from her father, James, who ran
the company until 2018.

Many of the biggest and most influential lobbying and political
consulting firms received money --- despite prohibitions intended to
restrict access --- most likely qualifying by highlighting lines of
business that fell outside the restrictions.

Wiley Rein, which has a large lobbying practice focusing on trade
issues, received between \$5 million and \$10 million, according to the
data. Van Ness Feldman and Beveridge \& Diamond, two law firms that
focus on helping energy industry clients push their agendas in
Washington, received loans between \$2 million and \$5 million,
according to the administration.

A firm that raises money for Mr. Trump's re-election campaign and the
Republican National Committee received a loan of more than \$1 million,
according to the data set, while a company that produces Mr. Trump's
political advertisements received between \$350,000 and \$1 million. So
did a consulting firm started by President Barack Obama's former
campaign manager Jim Messina and one that Hillary Clinton's 2008
campaign paid for communications consulting.

Several firms that advise companies on how to deal with the government,
but are not officially registered to lobby, were also said to have
received loans. They include companies run by former Secretary of State
Madeleine Albright, who served in the Clinton administration.

The administration listed loans worth between \$350,000 and \$1 million
to a consulting firm started by former Senator William S. Cohen, a Maine
Republican who also served in the Clinton administration as the
secretary of defense, and one run by a homeland security secretary in
the Bush administration, Michael Chertoff. And DCI Group AZ, a prominent
political and corporate consulting firm, collected as much as \$5
million.

An affiliate of Americans for Tax Reform, the influential conservative
group that has been a vocal critic of government spending, received
between \$150,000 and \$350,000, according to the government's data. In
a statement, the group said the foundation ``was badly hurt by the
government shutdown'' and ``does not engage in lobbying.''

A number of prominent private schools were listed as loan recipients,
despite
\href{https://www.nytimes.com/2020/04/29/us/prep-schools-coronavirus-loans.html}{the
controversy} over whether such institutions should take the money.

In New York City, St. Ann's School took a loan valued between \$5
million and \$10 million. Kent Place School, a private school in New
Jersey, was reported to have received a loan worth between \$1 million
and \$2 million.

Schools with political ties in the Washington area also received loans.
Sidwell Friends, which has educated the children of presidents, received
a loan worth between \$5 million and \$10 million, based on the data.
Georgetown Preparatory School, which the Supreme Court justices Brett
Kavanaugh and Neil Gorsuch attended, received a loan worth between \$2
million and \$5 million.

Georgetown Preparatory's president, the Rev. James R. Van Dyke, said,
``We remain committed to doing all that we can to retain our immensely
talented faculty and staff,'' adding that many of them accept salaries
and wages ``lower than what they might earn in the for-profit sector or
the public school system.''

St. Andrew's Episcopal School, where Mr. Trump's youngest son is a
student, also received a loan.

Touring companies for rock bands also turned to the government for help
as concert venues around the country went dark to prevent the spread of
the virus.

A limited liability company called Eagles Touring Company II Foreign
received a loan between \$350,000 and \$1 million. Documents filed in
California show that the entity shares an agent with a similarly named
company whose president is Don Henley, a founding member of the Eagles,
the rock band that had to postpone its tour this spring.

Lil' Jon Touring and
\href{https://slack-redir.net/link?url=https\%3A\%2F\%2Fbusinesssearch.sos.ca.gov\%2FDocument\%2FRetrievePDF\%3FId\%3D02813188-11505402}{Nickelback
Touring 2}, among other acts, received loans between \$150,000 and
\$350,000. Lil Jon's publicist, Tamar Juda, said that the artist had
canceled over 75 shows since the beginning of the pandemic and that the
funds allowed his core touring staff to remain employed.

Yeezy, which California
\href{https://businesssearch.sos.ca.gov/Document/RetrievePDF?Id=201613910148-27294586}{business
filings} show is a holding company registered to Mr. West, received
between \$2 million and \$5 million to support 106 jobs, based on the
disclosures. The holding company appears to be linked to Mr. West's
apparel brand, having
\href{https://tsdr.uspto.gov/documentviewer?caseId=sn90023673\&docId=APP20200630084733\#docIndex=1\&page=1}{recently
filed} to trademark the phrase ``West Day Ever'' for use on clothing.
Public relations firms that have worked for Yeezy did not respond to
emailed requests for comment.

There was no apparent link between the amount of economic damage
suffered by states and how successful the small businesses in them were
at getting the loans from the program.

North Dakota, South Dakota, Nebraska and Kansas all saw loan approvals
of at least 90 percent of their eligible small-business payroll, even
though they rank among the
\href{https://taxfoundation.org/unemployment-insurance-claims/}{least-affected
states} in terms of unemployment claims during the crisis. Two of the
hardest-hit states for claims, New York and California, saw loan
approvals equal to about three-quarters of their eligible payrolls; by
that measure, California companies would have received billions more
from the program if they had seen approvals at the same rate as the
Plains states.

Reporting was contributed by Kenneth P. Vogel, Eric Lipton, David
McCabe, Steve Eder, Ben Protess, Stacy Cowley and Noam Scheiber.

Advertisement

\protect\hyperlink{after-bottom}{Continue reading the main story}

\hypertarget{site-index}{%
\subsection{Site Index}\label{site-index}}

\hypertarget{site-information-navigation}{%
\subsection{Site Information
Navigation}\label{site-information-navigation}}

\begin{itemize}
\tightlist
\item
  \href{https://help.nytimes.com/hc/en-us/articles/115014792127-Copyright-notice}{©~2020~The
  New York Times Company}
\end{itemize}

\begin{itemize}
\tightlist
\item
  \href{https://www.nytco.com/}{NYTCo}
\item
  \href{https://help.nytimes.com/hc/en-us/articles/115015385887-Contact-Us}{Contact
  Us}
\item
  \href{https://www.nytco.com/careers/}{Work with us}
\item
  \href{https://nytmediakit.com/}{Advertise}
\item
  \href{http://www.tbrandstudio.com/}{T Brand Studio}
\item
  \href{https://www.nytimes.com/privacy/cookie-policy\#how-do-i-manage-trackers}{Your
  Ad Choices}
\item
  \href{https://www.nytimes.com/privacy}{Privacy}
\item
  \href{https://help.nytimes.com/hc/en-us/articles/115014893428-Terms-of-service}{Terms
  of Service}
\item
  \href{https://help.nytimes.com/hc/en-us/articles/115014893968-Terms-of-sale}{Terms
  of Sale}
\item
  \href{https://spiderbites.nytimes.com}{Site Map}
\item
  \href{https://help.nytimes.com/hc/en-us}{Help}
\item
  \href{https://www.nytimes.com/subscription?campaignId=37WXW}{Subscriptions}
\end{itemize}
