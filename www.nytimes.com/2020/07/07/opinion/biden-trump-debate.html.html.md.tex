Sections

SEARCH

\protect\hyperlink{site-content}{Skip to
content}\protect\hyperlink{site-index}{Skip to site index}

\href{https://myaccount.nytimes.com/auth/login?response_type=cookie\&client_id=vi}{}

\href{https://www.nytimes.com/section/todayspaper}{Today's Paper}

\href{/section/opinion}{Opinion}\textbar{}Biden Should Not Debate Trump
Unless \ldots{}

\href{https://nyti.ms/2ZKZIBX}{https://nyti.ms/2ZKZIBX}

\begin{itemize}
\item
\item
\item
\item
\item
\item
\end{itemize}

Advertisement

\protect\hyperlink{after-top}{Continue reading the main story}

\href{/section/opinion}{Opinion}

Supported by

\protect\hyperlink{after-sponsor}{Continue reading the main story}

\hypertarget{biden-should-not-debate-trump-unless-}{%
\section{Biden Should Not Debate Trump Unless
\ldots{}}\label{biden-should-not-debate-trump-unless-}}

Here are two conditions the Democrat should set.

\href{https://www.nytimes.com/by/thomas-l-friedman}{\includegraphics{https://static01.nyt.com/images/2018/04/02/opinion/thomas-l-friedman/thomas-l-friedman-thumbLarge.png}}

By \href{https://www.nytimes.com/by/thomas-l-friedman}{Thomas L.
Friedman}

Opinion Columnist

\begin{itemize}
\item
  July 7, 2020
\item
  \begin{itemize}
  \item
  \item
  \item
  \item
  \item
  \item
  \end{itemize}
\end{itemize}

\includegraphics{https://static01.nyt.com/images/2020/07/07/opinion/07FRIEDMAN-COMBO/07FRIEDMAN-COMBO-articleLarge.jpg?quality=75\&auto=webp\&disable=upscale}

I worry about
\href{https://www.nytimes.com/2020/08/06/us/politics/presidential-debates-trump-biden.html}{Joe
Biden debating Donald Trump}. He should do it only under two conditions.
Otherwise, he's giving Trump unfair advantages.

First, Biden should declare that he will take part in a debate only if
Trump releases his tax returns for 2016 through 2018. Biden has already
done so, and they are
\href{https://joebiden.com/financial-disclosure/}{on his website}. Trump
must, too. No more gifting Trump something he can attack while hiding
his own questionable finances.

And second, Biden should insist that a real-time fact-checking team
approved by both candidates be hired by the nonpartisan Commission on
Presidential Debates --- and that 10 minutes before the scheduled
conclusion of the debate this team report on any misleading statements,
phony numbers or outright lies either candidate had uttered. That way no
one in that massive television audience can go away easily misled.

Debates always have ground rules. Why can't telling the truth and equal
transparency on taxes be conditions for this one?

Yes, the fact that we have to make truth-telling an explicit condition
is an incredibly sad statement about our time; normally such things are
unspoken and understood. But if the past teaches us anything, Trump
might very well lie and mislead for the entire debate, forcing Biden to
have to spend a majority of his time correcting Trump before making his
own points.

That is not a good way for Biden to reintroduce himself to the American
people. And, let's not kid ourselves, these debates \emph{will}
\emph{be} his reintroduction to most Americans, who have neither seen
nor heard from him for months if not years.

Because of Covid-19, Biden has been sticking close to home, wearing a
mask and social distancing. And with the coronavirus now spreading
further, and Biden being a responsible individual and role model, it's
likely that he won't be able to engage with any large groups of voters
before Election Day. Therefore, the three scheduled televised debates,
which will garner huge audiences, will carry more weight for him than
ever.

He should not go into such a high-stakes moment ceding any advantages to
Trump. Trump is badly trailing in the polls, and he needs these debates
much more than Biden does to win over undecided voters. So Biden needs
to make Trump pay for them in the currency of transparency and
fact-checking --- universal principles that will level the playing field
for him and illuminate and enrich the debates for all citizens.

Of course, Trump will stomp and protest and say, ``No way.'' Fine. Let
Trump cancel. Let Trump look American voters in the eye and say: ``There
will be no debate, because I should be able to continue hiding my tax
returns from you all, even though I promised that I wouldn't and even
though Biden has shown you his. And there will be no debate, because I
should be able to make any statement I want without any independent
fact-checking.''

If Trump says that, Biden can retort: ``Well, that's not a debate then,
that's a circus. If that's what you want, why don't we just arm wrestle
or flip a coin to see who wins?''

I get why Republican senators and Fox News don't press Trump on his
taxes or call out his lies. They're afraid of him and his base and
unconcerned about the truth. But why should Biden, or the rest of us,
play along?

After all, these issues around taxes and truth are more vital than ever
for voters to make an informed choice.

Trump, you will recall, never sold his Trump Organization holdings or
put them into a blind trust --- as past presidents did with their
investments --- to avoid any conflicts of interest. Rather, his assets
are in a revocable trust, whose trustees are his eldest son, Donald Jr.,
and Allen Weisselberg, the Trump Organization's chief financial officer.
\href{https://www.npr.org/2017/04/03/522511211/change-to-president-trumps-trust-lets-him-tap-business-profits}{Which
is a joke.}

Trump promised during the last campaign to release his tax returns after
an I.R.S. ``audit'' was finished. **** Which turned out to have been
another joke.

Once elected, Trump claimed that the American people were not interested
in seeing his tax returns. Actually, we are now more interested than
ever --- and not just because it's utterly unfair that Biden go into the
debate with all his income exposed (he and his wife, Jill,
\href{https://www.cnbc.com/2019/07/09/joe-biden-releases-tax-returns-during-2020-democratic-primary.html}{earned
more than \$15 million} in the two years after they left the Obama
administration, largely from speaking engagements and books) while Trump
doesn't have to do the same.

There must be something in those tax returns that Trump really does not
want the American public to see. It may be just silly --- that he's
actually not all that rich. It may have to do with the fact that foreign
delegations and domestic lobbyists, who want to curry favor with him,
stay in his hotel in Washington or use it for corporate entertaining.

Or, more ominously, it may be related to Trump's incomprehensible
willingness to give Russian President Vladimir Putin the benefit of
every doubt for the last three-plus years. Virtually every time there
has been a major public dispute between Putin and U.S. intelligence
agencies alleging Russian misdeeds --- including, of late, that the
Kremlin offered bounties for the killing of U.S. soldiers in Afghanistan
--- Trump has sided with Putin.

The notion that Putin may have leverage over him is not crazy, given
little previous hints by his sons.

As Michael Hirsh recalled in a 2018 article in
\href{https://foreignpolicy.com/2018/12/21/how-russian-money-helped-save-trumps-business/}{Foreign
Policy} about how Russian money helped to save the Trump empire from
bankruptcy: ``In September 2008, at the `Bridging U.S. and Emerging
Markets Real Estate' conference in New York, the president's eldest son,
Donald Jr., said: `In terms of high-end product influx into the United
States, Russians make up a pretty disproportionate cross-section of a
lot of our assets. Say, in Dubai, and certainly with our project in
SoHo, and anywhere in New York. We see a lot of money pouring in from
Russia.'''

The American people need to know if Trump is in debt in any way to
Russian banks and financiers who might be close to Putin. Because if
Trump is re-elected, and unconstrained from needing to run again, he
will most likely act even more slavishly toward Putin, and that is a
national security threat.

At the same time, debating Trump is unlike debating any other human
being. Trump literally lies as he breathes, and because he has
absolutely no shame**,** there are no guardrails. **** According to the
Fact Checker team at The Washington Post, between Trump's inauguration
on Jan. 20, 2017, and May 29, 2020, he made
\href{https://www.washingtonpost.com/politics/2020/06/01/president-trump-made-19127-false-or-misleading-claims-1226-days/}{19,127
false or misleading claims}.

Biden has been dogged by bone-headed issues of plagiarism in his career,
but nothing compared to Trump's daily fire hose of dishonesty, which has
no rival in U.S. presidential history. That's why it's so important to
insist that the nonpartisan Commission on Presidential Debates hire
independent fact-checkers who, after the two candidates give their
closing arguments --- but before the debate goes off the air --- would
present a rundown of any statements that were false or only partly true.

Only if leading into the debate, American voters have a clear picture of
Trump's tax returns alongside Biden's, and only if, coming out of the
debate, they have a clear picture of who was telling the truth and who
was not, will they be able to make a fair judgment between the two
candidates.

\emph{That kind of debate} \emph{and only that kind of debate} would be
worthy of voters' consideration and Biden's participation.

Otherwise, Joe, stay in your basement.

\emph{The Times is committed to publishing}
\href{https://www.nytimes.com/2019/01/31/opinion/letters/letters-to-editor-new-york-times-women.html}{\emph{a
diversity of letters}} \emph{to the editor. We'd like to hear what you
think about this or any of our articles. Here are some}
\href{https://help.nytimes.com/hc/en-us/articles/115014925288-How-to-submit-a-letter-to-the-editor}{\emph{tips}}\emph{.
And here's our email:}
\href{mailto:letters@nytimes.com}{\emph{letters@nytimes.com}}\emph{.}

\emph{Follow The New York Times Opinion section on}
\href{https://www.facebook.com/nytopinion}{\emph{Facebook}}\emph{,}
\href{http://twitter.com/NYTOpinion}{\emph{Twitter (@NYTopinion)}}
\emph{and}
\href{https://www.instagram.com/nytopinion/}{\emph{Instagram}}\emph{.}

Advertisement

\protect\hyperlink{after-bottom}{Continue reading the main story}

\hypertarget{site-index}{%
\subsection{Site Index}\label{site-index}}

\hypertarget{site-information-navigation}{%
\subsection{Site Information
Navigation}\label{site-information-navigation}}

\begin{itemize}
\tightlist
\item
  \href{https://help.nytimes.com/hc/en-us/articles/115014792127-Copyright-notice}{©~2020~The
  New York Times Company}
\end{itemize}

\begin{itemize}
\tightlist
\item
  \href{https://www.nytco.com/}{NYTCo}
\item
  \href{https://help.nytimes.com/hc/en-us/articles/115015385887-Contact-Us}{Contact
  Us}
\item
  \href{https://www.nytco.com/careers/}{Work with us}
\item
  \href{https://nytmediakit.com/}{Advertise}
\item
  \href{http://www.tbrandstudio.com/}{T Brand Studio}
\item
  \href{https://www.nytimes.com/privacy/cookie-policy\#how-do-i-manage-trackers}{Your
  Ad Choices}
\item
  \href{https://www.nytimes.com/privacy}{Privacy}
\item
  \href{https://help.nytimes.com/hc/en-us/articles/115014893428-Terms-of-service}{Terms
  of Service}
\item
  \href{https://help.nytimes.com/hc/en-us/articles/115014893968-Terms-of-sale}{Terms
  of Sale}
\item
  \href{https://spiderbites.nytimes.com}{Site Map}
\item
  \href{https://help.nytimes.com/hc/en-us}{Help}
\item
  \href{https://www.nytimes.com/subscription?campaignId=37WXW}{Subscriptions}
\end{itemize}
