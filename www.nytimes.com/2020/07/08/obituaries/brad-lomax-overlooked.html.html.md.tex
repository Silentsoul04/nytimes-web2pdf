Sections

SEARCH

\protect\hyperlink{site-content}{Skip to
content}\protect\hyperlink{site-index}{Skip to site index}

\href{https://www.nytimes.com/section/obituaries}{Obituaries}

\href{https://myaccount.nytimes.com/auth/login?response_type=cookie\&client_id=vi}{}

\href{https://www.nytimes.com/section/todayspaper}{Today's Paper}

\href{/section/obituaries}{Obituaries}\textbar{}Overlooked No More: Brad
Lomax, a Bridge Between Civil Rights Movements

\url{https://nyti.ms/2AHwPhJ}

\begin{itemize}
\item
\item
\item
\item
\item
\item
\end{itemize}

Advertisement

\protect\hyperlink{after-top}{Continue reading the main story}

Supported by

\protect\hyperlink{after-sponsor}{Continue reading the main story}

\hypertarget{overlooked-no-more-brad-lomax-a-bridge-between-civil-rights-movements}{%
\section{Overlooked No More: Brad Lomax, a Bridge Between Civil Rights
Movements}\label{overlooked-no-more-brad-lomax-a-bridge-between-civil-rights-movements}}

A member of the Black Panthers, he helped lead a historic, and
successful, sit-in in San Francisco as part of a nationwide
anti-discrimination campaign on behalf of people with disabilities.

\includegraphics{https://static01.nyt.com/images/2020/07/26/obituaries/00overlooked-lomax-01/00overlooked-lomax-01-articleLarge.jpg?quality=75\&auto=webp\&disable=upscale}

By Eileen AJ Connelly

\begin{itemize}
\item
  Published July 8, 2020Updated July 20, 2020
\item
  \begin{itemize}
  \item
  \item
  \item
  \item
  \item
  \item
  \end{itemize}
\end{itemize}

\emph{Overlooked is a series of obituaries about remarkable people whose
deaths, beginning in 1851, went unreported in The Times. This latest
installment is part of a series exploring how the Americans With
Disabilities Act has shaped modern life for disabled people.}
\href{https://www.nytimes.com/2020/07/10/reader-center/disability-america-questions.html}{\emph{Share
your stories}} \emph{or email us at}
\href{mailto:ada@nytimes.com}{\emph{ada@nytimes.com}}\emph{.}

\hypertarget{listen-to-this-article}{%
\subsubsection{Listen to This Article}\label{listen-to-this-article}}

Computer-generated audio recording.

When Brad Lomax joined the Black Panther Party in the late 1960s, he
hoped to be part of a revolution that would provide a better life for
Black Americans, free of inequality, poverty and police brutality. And
to a large extent he succeeded, making important contributions to the
Panthers on both coasts. But it was in an entirely different civil
rights movement --- one for people with disabilities --- that he would
make his most indelible mark.

Lomax had already learned that he had multiple sclerosis when he helped
found the Black Panther Party's Washington chapter in 1969. He went on
to help organize, in 1972, the first African Liberation Day
demonstration, which drew tens of thousands of marchers to the National
Mall to proclaim solidarity with African nations in their efforts to win
independence.

He moved to Oakland, Calif., the next year --- a time when the Bay Area
was a whirl of social action. The disabilities rights movement had
gained a foothold there, as had the gay rights and Native American
rights movements.

In Oakland, Lomax struggled to navigate its transit system. To board a
bus, his brother, Glenn, would have to lift him out of his wheelchair,
carry him up the steps and place him in a seat, then go back to retrieve
the wheelchair.

Such indignities galvanized Lomax to consider the plight of people with
disabilities in a world that didn't make it easy for them. ``He could
see all the obstacles they had to face,'' Glenn Lomax said in a phone
interview.

\includegraphics{https://static01.nyt.com/images/2020/06/30/multimedia/00overlooked-lomax-04/00overlooked-lomax-04-articleLarge.jpg?quality=75\&auto=webp\&disable=upscale}

Lomax became a key figure in the disability rights movement when he
joined more than 100 people in 1977 in occupying the fourth-floor
offices of the Department of Health, Education and Welfare in the
\href{https://www.gsa.gov/historic-buildings/federal-building-san-francisco-ca}{San
Francisco Federal Building}. Their goal was to persuade the government
to enforce a long-ignored section of
\href{https://www2.ed.gov/policy/speced/reg/narrative.html}{the
Rehabilitation Act of 1973}.

Tucked in the act was Section 504, which, modeled on the Civil Rights
Act of 1964, had the potential to change the lives of people living with
disabilities by prohibiting recipients of federal aid from
discriminating against any ``otherwise qualified individuals with a
disability.''

The demonstration, known as the 504 Sit-in, would last for almost a
month, making it the longest peaceful occupation of a federal building
in the nation's history. Lomax, accompanied by an attendant, Chuck
Jackson, not only helped lead the protest; he also gained the support of
their fellow Black Panthers, a group that had advocated armed
self-defense while also providing social services to the Black
community; the Panthers agreed to bring hot meals and other provisions
to the building daily.

``Without the presence of Brad Lomax and Chuck Jackson, the Black
Panthers would not have fed the 504 participants occupying the H.E.W.
building,'' Corbett O'Toole, who took part in the demonstration, wrote
in an unpublished memoir. ``Without that food, the sit-in would have
collapsed.''

Bradford Clyde Lomax, the eldest of three siblings, was born on Sept.
13, 1950, in Philadelphia to Katie Lee (Bell) Lomax and Joseph Randolph
Lomax, a World War II veteran and electrician.

Image

{[}Image description: Lomax tilting his head back and smiling while
posing for a portrait with friends and family members who are standing
and sitting around him.{]} Lomax celebrating his birthday with friends
and family members at his mother's house in San Francisco in
1981.Credit...Glenn Lomax

Lomax's childhood was filled with football, drama clubs and Boy Scouts,
all set against the charged backdrop of the civil rights movement. A
trip to visit his mother's family in Alabama when he was 13 formed an
indelible impression, his brother recalled.

Alabama in 1963 was an epicenter of the civil rights movement, with
lunch counter sit-ins, protest marches and other actions aimed at
dismantling state-sponsored segregation. There, for the first time, Brad
encountered signs designating some public spaces for white people and
some for Black people.

After graduating from Benjamin Franklin High School in Philadelphia in
1968, Lomax considered joining the military, but as the war in Vietnam
raged, with Black soldiers bearing a disproportionate share of the
burden, he decided instead to attend Howard University in Washington.

That year --- inexplicably, it seemed --- Lomax began falling as he
walked. Then he learned that he had multiple sclerosis. As the disease
progressed, he began using a wheelchair --- and it opened his eyes to
another form of discrimination. Wheelchairs that were supposed to
provide independence, he found, were of little use in gaining entrance
to public buildings without ramps. He saw that people with disabilities
were regularly denied an education, and that there were few services to
help them find housing or jobs, especially if they were Black.

After moving to Oakland, he learned about the Center for Independent
Living, an organization started by people with disabilities that was
instrumental in getting curb cuts for wheelchairs at street corners in
San Francisco and nearby Berkeley. In 1975, he approached the director,
Ed Roberts, and proposed that the center combine efforts with the Black
Panthers to offer assistance to disabled people in East Oakland's mostly
Black community.

The relationships he built across the two communities would prove vital
during the 504 Sit-in two years later.

The Department of Health, Education and Welfare had been charged with
writing the regulations for implementing Section 504, which was to be a
model for other federal agencies. But the regulations were never enacted
under President Gerald R. Ford because of resistance from business and
government interests, which were wary of the costs of providing
accessibility to education, employment, health care and public
buildings.

In his campaign for president in 1976, Jimmy Carter promised to move the
regulations forward, but after he took office, his new H.E.W. secretary,
Joseph A. Califano Jr., said the rules would have to be overhauled
before he would sign them.

To the activists, it seemed as if Califano was stalling in an attempt to
water down the regulations.

After years of calling for Section 504 to be enacted, disabilities
rights activists set April 5, 1977, as a deadline for action.

The demonstration in San Francisco that day began outside of H.E.W.'s
offices, at 50 United Nations Plaza, with speeches and chants. A group
of protesters then went inside demanding to know when Califano would
sign the regulations.

One of the demonstrators, Dennis Billups, said in a phone interview that
many of the protesters had no idea that sit-ins were happening
simultaneously at H.E.W. offices in Washington, Denver, Los Angeles and
elsewhere around the country, or that the sit-in would last so long.
Most had no toothbrushes, no change of clothing and no food. But even as
the sit-ins in the other cities fizzled, and even as the government
switched off the water supply, cut off telephone lines and restricted
movement in and out of the federal building, the San Francisco
contingent held strong.

Image

{[}Image description: Lomax appearing pensive as he sits with others,
many in wheelchairs.{]} Lomax with other demonstrators inside the San
Francisco Federal Building in April 1977.Credit...Glenn Lomax

After Lomax sent word to the Black Panthers that the protesters intended
to stay put, the group mobilized, delivering ribs, fried chicken and a
message of solidarity.

``Brad was able to get the Black Panther Party to see that this was
critical to the work that they were doing,'' said Judy Heumann, a
demonstration leader. ``He was the linchpin for that.''

Though the sit-in received only sporadic coverage by the mainstream news
media, the Black Panther's newspaper followed it closely, with Lomax
acting as a go-between for reporters.

``Even though disability rights was not high on the agenda of the Black
Panther Party per se, this was,'' Michael Fultz, who was editor of the
paper, said in an interview.

After three weeks, Lomax and Jackson were among 25 demonstrators chosen
to travel to Washington to pressure Califano to sign the regulations.
The Black Panthers paid their way. The trip was successful: Califano
signed the regulations on April 28, 1977, and the contingent returned to
San Francisco to lead the demonstrators out of the federal building
singing, ``We Have Overcome.''

Lomax continued to work with the Panthers for a few more years. He died
of complications of multiple sclerosis in Sacramento on Aug. 28, 1984.
He was 33.

``I don't think that all of his aspirations were fulfilled, even after
the demonstration,'' Billups said. ``He really wanted more.''

While the 504 regulations pushed the government toward accessibility,
they applied only to federally funded programs. But their enactment laid
the groundwork for the Americans With Disabilities Act, which was passed
in 1990.

Advertisement

\protect\hyperlink{after-bottom}{Continue reading the main story}

\hypertarget{site-index}{%
\subsection{Site Index}\label{site-index}}

\hypertarget{site-information-navigation}{%
\subsection{Site Information
Navigation}\label{site-information-navigation}}

\begin{itemize}
\tightlist
\item
  \href{https://help.nytimes.com/hc/en-us/articles/115014792127-Copyright-notice}{©~2020~The
  New York Times Company}
\end{itemize}

\begin{itemize}
\tightlist
\item
  \href{https://www.nytco.com/}{NYTCo}
\item
  \href{https://help.nytimes.com/hc/en-us/articles/115015385887-Contact-Us}{Contact
  Us}
\item
  \href{https://www.nytco.com/careers/}{Work with us}
\item
  \href{https://nytmediakit.com/}{Advertise}
\item
  \href{http://www.tbrandstudio.com/}{T Brand Studio}
\item
  \href{https://www.nytimes.com/privacy/cookie-policy\#how-do-i-manage-trackers}{Your
  Ad Choices}
\item
  \href{https://www.nytimes.com/privacy}{Privacy}
\item
  \href{https://help.nytimes.com/hc/en-us/articles/115014893428-Terms-of-service}{Terms
  of Service}
\item
  \href{https://help.nytimes.com/hc/en-us/articles/115014893968-Terms-of-sale}{Terms
  of Sale}
\item
  \href{https://spiderbites.nytimes.com}{Site Map}
\item
  \href{https://help.nytimes.com/hc/en-us}{Help}
\item
  \href{https://www.nytimes.com/subscription?campaignId=37WXW}{Subscriptions}
\end{itemize}
