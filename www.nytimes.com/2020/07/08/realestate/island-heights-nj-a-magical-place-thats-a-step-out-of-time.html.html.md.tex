Sections

SEARCH

\protect\hyperlink{site-content}{Skip to
content}\protect\hyperlink{site-index}{Skip to site index}

\href{https://www.nytimes.com/section/realestate}{Real Estate}

\href{https://myaccount.nytimes.com/auth/login?response_type=cookie\&client_id=vi}{}

\href{https://www.nytimes.com/section/todayspaper}{Today's Paper}

\href{/section/realestate}{Real Estate}\textbar{}Island Heights, N.J.: A
`Magical Place' That's a Step Out of Time

\url{https://nyti.ms/3iIk2g8}

\begin{itemize}
\item
\item
\item
\item
\item
\item
\end{itemize}

Advertisement

\protect\hyperlink{after-top}{Continue reading the main story}

Supported by

\protect\hyperlink{after-sponsor}{Continue reading the main story}

Living in

\hypertarget{island-heights-nj-a-magical-place-thats-a-step-out-of-time}{%
\section{Island Heights, N.J.: A `Magical Place' That's a Step Out of
Time}\label{island-heights-nj-a-magical-place-thats-a-step-out-of-time}}

Founded in the 19th century as a Methodist camp meeting site, this Ocean
County borough retains an old-fashioned feeling, even as newcomers move
in.

\href{https://www.nytimes.com/slideshow/2020/07/08/realestate/living-in-island-heights-nj.html}{}

\hypertarget{living-in--island-heights-nj}{%
\subsection{Living In ... Island Heights,
N.J.}\label{living-in--island-heights-nj}}

13 Photos

View Slide Show ›

\includegraphics{https://static01.nyt.com/images/2020/07/08/realestate/08LIVING-ISLANDHEIGHTS-slide-H1IQ/08LIVING-ISLANDHEIGHTS-slide-H1IQ-articleLarge.jpg?quality=75\&auto=webp\&disable=upscale}

Tony Cenicola/The New York Times

By Jill P. Capuzzo

\begin{itemize}
\item
  July 8, 2020
\item
  \begin{itemize}
  \item
  \item
  \item
  \item
  \item
  \item
  \end{itemize}
\end{itemize}

Residents and visitors are back on the boardwalk for sunset strolls, the
pickleball courts are open and white sails have been dotting the
shoreline of \href{http://islandheightsboro.com/}{Island Heights, N.J.},
for some time. But it wasn't until this week, when the Island Heights
Yacht Club's junior sailing program started up again, that something of
a sense of normalcy would return to this Ocean County borough on the
banks of Toms River.

``I look forward to this every summer,'' said Brian Hull, 44, a high
school social studies teacher who has taught sailing here for the last
28 years. ``All over town, you see kids riding their bikes, wearing
their helmets and life jackets, heading down the hill to the yacht
club.''

Despite the late start, Mr. Hull expects to squeeze most of the
eight-week sailing lesson season in before Labor Day, although many of
the boat races, along with other large public events, have been
canceled. Still, for many of the borough's 1,673 residents, life has not
been dramatically altered throughout the months of a state-mandated
coronavirus shutdown.

New JERSEY

Toms River

Island Heights

Seaside

Heights

Dillon's Creek

37

John F. Peto Studio Museum

Island

Heights

Yacht Club

Toms River

Seaside

Park

OCEAN

COUNTY

N.Y.

New

York

City

N.J.

Barnegat

Bay

Island

Heights

PA.

island beach

state park

OCEAN

COUNTY

1 mile

ATLANTIC

By The New York Times

``When I'm in Island Heights, I feel like I'm in a different part of the
country,'' said Michael DellaRocca, 67, a broker with Crossroads Realty
who moved to Island Heights three years ago. ``We're not being stupid,
but people have been running and biking, families have been getting
together. If you didn't leave town, you wouldn't know anything was going
on.''

With no commercial district, limited highway access and a public school
system that serves just over 120 students, Island Heights has long
seemed a step out of time, particularly in comparison with bustling
areas like Seaside Heights, a short causeway ride away. Social
activities center on the water in this 0.9-square-mile borough (a third
of which is in the water). The yacht club and three other marinas are
home to many boats from across Barnegat Bay, where numerous sailing
regattas are normally held throughout the summer.

\includegraphics{https://static01.nyt.com/images/2020/07/08/realestate/08LIVING-ISLANDHEIGHTS-slide-3J0V/08LIVING-ISLANDHEIGHTS-slide-3J0V-articleLarge.jpg?quality=75\&auto=webp\&disable=upscale}

An abbreviated racing season will begin in early August, said Mayor
Steve Doyle, commodore of the Island Heights Yacht Club. Other popular
Island Heights events --- including the fire department's Summerbrew
fund-raiser in late June and the Rotary Club's Sailfest in September ---
were canceled because of a 250-person cap on public gatherings imposed
by Gov. Philip D. Murphy of New Jersey.

``We're adapting to the new normal,'' said Mr. Doyle, 60, noting that
borough council meetings are now being held outdoors. ``Some families
were hard hit by the Covid shutdown, dealing with job loss or financial
issues. But everybody is supporting everybody else in town.'' (As of
late June, Island Heights had 14 Covid-19 positive cases and one
Covid-related death, the mayor said.)

Mr. Doyle, who has summered here each year since he was a boy, moved to
his family's Victorian house full time upon retiring in 2016, a
transition many of the borough's residents have made.

Founded in 1878 as a Methodist camp meeting site, Island Heights has
long served as a summer retreat, especially for those from the
Philadelphia area. And while full-timers now outnumber summer-only
residents, occasional rifts arise among the various constituent groups,
said Harry Bower, who bought an 1890s farmhouse here for \$175,000 in
1987.

``You have the townies, who want no change, and the yachties, who are
just here for the summer, so they don't care,'' said Mr. Bower, 68, an
art teacher and curator of the John F. Peto Studio Museum. ``But in the
last 10 years, we've seen new people, younger families moving here, who
really appreciate the town's charm.''

One of those appreciative newcomers is Therese Heimbold, 52, a
commercial director at the US Pharmaceutical Corporation, who remembered
her father talking about ``this magical place'' where he used to spend
his summers. In 2017, she sold her house in Haddonfield, N.J., and
bought one of Island Heights' original camp meeting houses, a
two-bedroom cottage, for \$225,000.

Returning from her job in Philadelphia in the evening, Ms. Heimbold
said, she finds her stress dissipates upon arrival: ``I drive into town
along River Road, and I see all the sailboats, and it's pure serenity.''

Image

138 CAMP MEETING AVENUE \textbar{} A four-bedroom,
three-and-a-half-bathroom house, built in 1985 on 0.53 acres overlooking
Toms River, listed for \$1.275 million. 732-267-3688Credit...Tony
Cenicola/The New York Times

\hypertarget{what-youll-find}{%
\subsection{What You'll Find}\label{what-youll-find}}

Surrounded by water on three sides, Island Heights offers many homes
with water views, nearly all a short walk from Toms River or Dillon's
Creek. While it technically sits on a peninsula rather than an island,
the borough does have one of the highest shore points along the Eastern
Seaboard, with a bluff that rises 60 feet above sea level. Atop this
bluff is a mix of newer waterfront homes with long staircases down to
private river docks and smaller houses, including a handful of
turn-of-the-last-century cottages surrounding the camp meeting site
grounds on West Camp Walk. About a dozen newer homes have been built
throughout the town.

At the lower altitude, River Avenue hugs the curve of the river, where
some of the borough's grand Victorian and Queen Anne homes sit facing
the water, complete with bright colors, ornate trim and the occasional
widow's walk. (Island Heights was once home to a number of ship
captains.) About a third of the borough's houses are within the Island
Heights Historic District, listed on the National Register of Historic
Places, although the designation doesn't restrict renovations.

``It was enacted without teeth,'' Mr. Doyle said. ``But most people
follow the rules in keeping up with the style. You're buying into that
culture.''

There are two small riverfront beaches in town, but many opt to cross
the Route 37 causeway to the ocean beaches along Barnegat Peninsula,
like Seaside Heights, Seaside Park and Island Beach State Park.

Image

178 OCEAN AVENUE \textbar{} A seven-bedroom, four-and-a-half-bathroom
house, built around the turn of the last century and updated in 2003, on
0.23 acres, listed for \$650,000. 732-513-1302Credit...Tony Cenicola/The
New York Times

\hypertarget{what-youll-pay}{%
\subsection{What You'll Pay}\label{what-youll-pay}}

As of early July, there were 23 properties on the market in Island
Heights, Mr. DellaRocca said, including nine vacant developer lots along
Dillon's Creek. The most expensive house was a 1985 four-bedroom
waterfront home with a 200-foot dock, listed for \$1.275 million; the
least expensive was a 1990 two-bedroom, two-bathroom house for
\$262,000. The development lots range from \$650,000 to \$1.1 million.
Rental properties are almost nonexistent.

The average price of the 13 houses sold in the first six months of this
year was \$480,000; during the same time period in 2019, 28 homes sold
at an average price of \$411,000, according to the Monmouth Ocean
Regional Multiple Listing Service. Elizabeth Hull, an agent with Re/Max
and the wife of Mr. Hull, the sailing instructor, said the waterfront
Victorians rarely come on the market, but when they do, they are priced
at \$600,000 or more, depending on the shape they are in.

Image

128 EAST CAMP WALK \textbar{} A two-bedroom, two-bathroom riverfront
cottage, built in 1900 on 0.14 acres, listed for \$599,000.
732-278-9070Credit...Tony Cenicola/The New York Times

\hypertarget{the-vibe}{%
\subsection{The Vibe}\label{the-vibe}}

The Hulls live in a rented house on ``the bluff,'' where Ms. Hull said
she makes sure to keep cheese, wine and beer in her refrigerator,
because ``people are always dropping by to hang out.'' More organized
gatherings take place at the yacht club, which holds regular Friday
night B.Y.O.B. catered dinners that are open to members and nonmembers.
(Island Heights is a dry borough.)

Beyond the boating scene, Island Heights supports an active arts
community, housing three art institutions (currently closed) within its
borders: a cultural heritage museum, the Ocean County Artists' Guild and
the John F. Peto Studio Museum, in a building where Mr. Peto, a
world-renowned trompe l'oeil artist, lived and painted in his final
years.

While many events have been canceled, the Island Heights fire department
is hoping to hold its Labor Day Races, a more than century-old tradition
that concludes with squad members shoveling a dump truck full of peanuts
onto the field for everyone to scramble and collect.

Image

Cozy Cove Marina is one of three private marinas in Island Heights.
Along with the yacht club, it provides docking space for hundreds of
boats from throughout the region.Credit...Tony Cenicola/The New York
Times

\hypertarget{the-schools}{%
\subsection{The Schools}\label{the-schools}}

The \href{http://www.islandheights.k12.nj.us/ihsd/}{Island Heights
School District}, one of the smallest in the state, serves around 120
students from kindergarten through sixth grade. The elementary school
provides Chromebooks to all its students and offers extracurricular
clubs for chess, robotics, art, music and broadcasting.

Beyond sixth grade, students attend Central Regional Middle School and
Central Regional High School, along with students from Berkeley
Township, Ocean Gate, Seaside Heights and Seaside Park. The regional
high school has about 1,400 students. In 2018-19, average SAT scores
were 529 in English and 535 in math, compared with state averages of 539
and 541.

Among the private school choices are St. Joseph Grade School for
prekindergarten through eighth grade and Donovan Catholic High School,
both in neighboring Toms River.

\hypertarget{the-commute}{%
\subsection{The Commute}\label{the-commute}}

Island Heights is about 75 miles south of New York City and is served by
buses only, out of Toms River. New Jersey Transit Bus No. 137 travels
from the Toms River park-and-ride station (about a seven-minute drive
from Island Heights) to Port Authority in Manhattan, in a little over an
hour and a half. Tickets are \$21.25 one way or \$496 for a monthly
pass. Academy Bus Lines runs a Parkway Express bus from the Toms River
park-and-ride station to the Wall Street area; the trip takes about an
hour and 45 minutes and costs \$21 one way or \$490 a month.

Image

One of the 30 original cabins still stands on the east side of the
former Methodist camp meeting grounds. The borough was founded in 1878
as a camp revival meeting destination and summer resort for Methodists.
~Credit...Tony Cenicola/The New York Times

\hypertarget{the-history}{%
\subsection{The History}\label{the-history}}

In the early 1900s, the department store magnate John Wanamaker
established Camp Wanamaker, a quasi-military encampment in the
northeastern part of Island Heights, where young employees were sent for
two weeks during the summer to take part in drills, drum and bugle corps
training, field competition and swimming. The site was taken over by the
U.S. Army during World War II and is where the borough's municipal
buildings and post office are now.

For weekly email updates on residential real estate news,
\href{http://www.nytimes.com/newsletters/realestate/}{sign up here}.
Follow us on Twitter:
\href{https://twitter.com/nytrealestate}{@nytrealestate}.

Advertisement

\protect\hyperlink{after-bottom}{Continue reading the main story}

\hypertarget{site-index}{%
\subsection{Site Index}\label{site-index}}

\hypertarget{site-information-navigation}{%
\subsection{Site Information
Navigation}\label{site-information-navigation}}

\begin{itemize}
\tightlist
\item
  \href{https://help.nytimes.com/hc/en-us/articles/115014792127-Copyright-notice}{©~2020~The
  New York Times Company}
\end{itemize}

\begin{itemize}
\tightlist
\item
  \href{https://www.nytco.com/}{NYTCo}
\item
  \href{https://help.nytimes.com/hc/en-us/articles/115015385887-Contact-Us}{Contact
  Us}
\item
  \href{https://www.nytco.com/careers/}{Work with us}
\item
  \href{https://nytmediakit.com/}{Advertise}
\item
  \href{http://www.tbrandstudio.com/}{T Brand Studio}
\item
  \href{https://www.nytimes.com/privacy/cookie-policy\#how-do-i-manage-trackers}{Your
  Ad Choices}
\item
  \href{https://www.nytimes.com/privacy}{Privacy}
\item
  \href{https://help.nytimes.com/hc/en-us/articles/115014893428-Terms-of-service}{Terms
  of Service}
\item
  \href{https://help.nytimes.com/hc/en-us/articles/115014893968-Terms-of-sale}{Terms
  of Sale}
\item
  \href{https://spiderbites.nytimes.com}{Site Map}
\item
  \href{https://help.nytimes.com/hc/en-us}{Help}
\item
  \href{https://www.nytimes.com/subscription?campaignId=37WXW}{Subscriptions}
\end{itemize}
