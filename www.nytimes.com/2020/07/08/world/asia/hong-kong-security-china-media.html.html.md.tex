Sections

SEARCH

\protect\hyperlink{site-content}{Skip to
content}\protect\hyperlink{site-index}{Skip to site index}

\href{https://www.nytimes.com/section/world/asia}{Asia Pacific}

\href{https://myaccount.nytimes.com/auth/login?response_type=cookie\&client_id=vi}{}

\href{https://www.nytimes.com/section/todayspaper}{Today's Paper}

\href{/section/world/asia}{Asia Pacific}\textbar{}China's Leash on Hong
Kong Tightens, Choking a Broadcaster

\url{https://nyti.ms/3iAxMt8}

\begin{itemize}
\item
\item
\item
\item
\item
\end{itemize}

Advertisement

\protect\hyperlink{after-top}{Continue reading the main story}

Supported by

\protect\hyperlink{after-sponsor}{Continue reading the main story}

\hypertarget{chinas-leash-on-hong-kong-tightens-choking-a-broadcaster}{%
\section{China's Leash on Hong Kong Tightens, Choking a
Broadcaster}\label{chinas-leash-on-hong-kong-tightens-choking-a-broadcaster}}

RTHK, a government-funded news organization, has a fierce independent
streak that has long angered the authorities.

\includegraphics{https://static01.nyt.com/images/2020/07/08/world/08hk-media1/merlin_173650671_d58ed9e7-a4c4-446b-bd5f-770eb891865b-articleLarge.jpg?quality=75\&auto=webp\&disable=upscale}

By \href{https://www.nytimes.com/by/austin-ramzy}{Austin Ramzy} and Ezra
Cheung

\begin{itemize}
\item
  Published July 8, 2020Updated July 13, 2020
\item
  \begin{itemize}
  \item
  \item
  \item
  \item
  \item
  \end{itemize}
\end{itemize}

\href{https://cn.nytimes.com/china/20200709/hong-kong-security-china-media/}{阅读简体中文版}\href{https://cn.nytimes.com/china/20200709/hong-kong-security-china-media/zh-hant/}{閱讀繁體中文版}

HONG KONG ---
\href{https://www.nytimes.com/2020/07/13/world/asia/hong-kong-elections-security.html}{Hong
Kong's} public broadcaster has long been a rare example of a
government-funded news organization operating on Chinese soil that
fearlessly attempts to hold officials accountable.

The broadcaster, Radio Television Hong Kong, dug into security footage
last year to show how
\href{https://tvfilm.newyorkfestivals.com/Winners/WinnerDetailsNew/ae1a0cfa-80bf-4449-bddf-47ddee103758}{the
police failed to respond} when
\href{https://www.nytimes.com/2019/07/22/world/asia/hong-kong-protest-mob-attack-yuen-long.html}{a
mob attacked protesters in a train station}, leading to widespread
criticism of the authorities. The broadcaster also produced a
\href{https://tvfilm.newyorkfestivals.com/Winners/WinnerDetailsNew/e8b03ef2-7a8a-422f-b42c-cc44f802f7f6}{three-part
documentary on China's crackdown on Muslims in Xinjiang}. One RTHK
journalist, Nabela Qoser,
\href{https://twitter.com/tomgrundy/status/1153515102043197442}{became
famous} in Hong Kong for her persistent questioning of top officials.

Now, RTHK's journalists and hard-hitting investigations appear
vulnerable to
\href{https://www.nytimes.com/2020/07/13/world/asia/hong-kong-elections-security.html}{China's
new national security law}, which takes aim at dissent and could rein in
the city's largely freewheeling news organizations. The broadcaster,
modeled on the British Broadcasting Corporation, has already been
feeling pressure.

RTHK has drawn fire in recent months from the police, establishment
lawmakers and pro-Beijing activists. Its critics have filed thousands of
complaints accusing the broadcaster of bias against the government and
regularly protest outside its studios.

``If you want to enjoy freedom, you have obligations to follow,'' said
Innes Tang, the chairman of Politihk Social Strategic, a nonprofit
pro-Beijing group that has organized protests and petitions against
RTHK. ``You cannot use fake news to attack people. That is not part of
freedom of expression.''

\includegraphics{https://static01.nyt.com/images/2020/07/08/world/08hk-media2/merlin_173650587_5136eeac-c105-4cf5-aeb7-ae33e7d6c571-articleLarge.jpg?quality=75\&auto=webp\&disable=upscale}

As the objections mounted, RTHK was forced to suspend a satirical
program that made fun of the police. It was criticized by the Hong Kong
government for asking the World Health Organization if Taiwan could join
the global health body from which Beijing has shut it out. The
broadcaster faces a formal government review into its operations
starting next week.

The sweeping national security law China imposed last week on Hong Kong
is directed at quelling
\href{https://www.nytimes.com/2020/06/09/world/asia/hong-kong-protests-one-year-later.html}{the
pro-democracy protest movement} that roiled the territory last year, but
it also calls for tougher regulation of the media. The worry is that the
law would be used to muzzle outlets by requiring publishers and
broadcasters to avoid content and discussions that could be seen by the
authorities as subversive. The worst-case fear is that RTHK, as a
government department, could be forced to become an organ of state
propaganda.

The city's news outlets have faced an onslaught. Reporters covering
protests have been pepper-sprayed and detained by the police.
\href{https://www.nytimes.com/2019/08/23/world/asia/jimmy-lai-hong-kong-protests.html}{Jimmy
Lai}, the publisher of the Apple Daily, a pro-democracy newspaper, was
\href{https://www.nytimes.com/2020/02/28/world/asia/jimmy-lai-hong-kong-arrested.html}{one
of several opposition figures arrested} early this year, and state media
have accused him of fomenting unrest.

Pro-Beijing lawmakers have urged the government to register journalists.
The new security law also calls for a group of government bodies,
including the national security office, to oversee foreign journalists,
raising concerns about the erosion of press freedoms.

A reporter asked Carrie Lam, the city's leader, at a briefing on Tuesday
if she would guarantee that journalists in the city would be free to
report with the new law in place. Mrs. Lam responded that if ``all
reporters in Hong Kong can give me a 100 percent guarantee that they
will not commit any offenses under this piece of national legislation,
then I can do the same.''

Yuen Chan, a senior lecturer of journalism at City, University of London
who worked for RTHK in the late 1990s and early 2000s, said the
broadcaster was in an ``extremely perilous situation'' because its
status as a government department made it easier for Beijing to exert
control.

Image

Jimmy Lai, the publisher of Apple Daily, a pro-democracy newspaper, has
been accused of fomenting unrest.Credit...Lam Yik Fei for The New York
Times

The news organization appears to be taking pre-emptive steps to avoid
falling afoul of the security law. In recent weeks, several RTHK
journalists say, editors have told reporters not to emphasize
pro-independence slogans in their news reports.

An RTHK spokeswoman, Amen Ng, said that RTHK journalists ``have been
doing their job professionally'' but added that the broadcaster was not
a ``platform to promote Hong Kong independence.''

But there were already signs in RTHK's newsroom that a chill was setting
in.

Kirindi Chan, a top RTHK executive, announced unexpectedly in June that
she would resign, citing health reasons. Days later, she met with RTHK
reporters who pressed her if she was being forced out over their
coverage of the antigovernment demonstrations. Ms. Chan denied being
ousted, but she sought to deliver some solemn advice.

Ms. Chan reminded the reporters and producers of their role as civil
servants, and urged them to comply with the government's code of
conduct, according to two people who attended the meeting and spoke on
condition of anonymity to discuss an internal matter.

She did not go into details, but the civil service code calls for
impartiality and loyalty to the government, values the authorities have
stressed to discourage government employees from joining the protests.

Over an RTHK career of nearly three decades, Ms. Chan earned the respect
of her staff for being a staunch defender of the organization's
editorial independence. At the end of the somber half-hour meeting, the
reporters gave Ms. Chan a bouquet of red and yellow tulips, but an
employees' union said her departure was an ominous sign.

``We worry that Ms. Chan's resignation would set the scene for further
attacks on RTHK,'' the union said in a statement.

RTHK has also found itself caught in geopolitical wrangling between
China and Taiwan, the self-governing island that Beijing claims as part
of its territory.

Image

Filming a scene for ``Headliner'' last month. The show's suspension has
caused some alarm within RTHK.Credit...Lam Yik Fei for The New York
Times

In April, the government criticized RTHK over an interview the
broadcaster ran with a World Health Organization official, Dr. Bruce
Aylward, who was asked whether Taiwan should be allowed to participate
in the health body. Taiwan had been shut out by Beijing in recent years.

In an awkward exchange that highlighted the sensitivity of the topic,
Dr. Aylward first said he did not hear the question, then asked to move
on. When the reporter repeated it, the line went dead; minutes later,
asked again, Dr. Aylward replied, ``We've already talked about China.''
The interaction gave further ammunition to critics who say the health
body is unduly beholden to Beijing.

Edward Yau, the Hong Kong secretary for commerce and economic
development, which supervises RTHK, accused the broadcaster of having
breached China's official stance toward Taiwan. Such a rebuke now
carries more significance against the backdrop of the security law,
which focuses heavily on perceived threats to China's sovereignty.

If RTHK were forced to adopt a new role as a broadcaster that serves as
the voice of the government, it would be the culmination of a
decades-long campaign by its pro-Beijing critics.

RTHK was founded as a government radio station in 1928, when Hong Kong
was a British colony, and broadcast official bulletins for half a
century before it set up its own newsroom in 1973. Not long after the
territory returned to Chinese rule in 1997, pro-Beijing politicians
started urging RTHK to fall in line with the central government.

Image

Mr. Ng Chi-sum, the ``Headliner'' host,~ posed for a selfie with a crew
member on the day of the final shoot.Credit...Lam Yik Fei for The New
York Times

Editorial independence is enshrined in RTHK's charter. But unlike the
United States or Britain, where public broadcasting is given greater
autonomy from the government through nonprofit corporations, RTHK is a
government department, which makes it far more vulnerable to official
intervention.

The government flexed its grip over RTHK most overtly in May when it
complained about ``Headliner,'' a satirical program that had taken
pointed jabs at the police. That prompted the broadcaster to apologize
and suspend the show, a decision that caused some alarm within the
organization.

``If those who are in power cannot tolerate `Headliner,' then their
intolerance will extend to other current affairs programs,'' said Gladys
Chiu, the chairwoman of RTHK's labor union.

Image

Before ``Headliner'' was suspended, Mr. Ng~portrayed Hong Kong's leader
as the out-of-touch empress who led China during the decline of the Qing
dynasty.~Credit...Lam Yik Fei for The New York Times

On a recent Wednesday, the staff of ``Headliner'' gathered in RTHK's
aging studio for a final shoot. Ng Chi-sum, a longtime host of the show,
portrayed Carrie Lam, Hong Kong's leader, as Cixi, the out-of-touch
empress dowager during the final decline of the Qing dynasty, donning a
gaudy headdress, a fake pearl necklace and a gown.

The hosts kept up a light banter between takes, but off camera, Mr. Ng,
61, spoke gloomily of the show's prospects, and those of the city
itself.

``The worst is yet to come,'' Mr. Ng said. ``The overall trend nowadays
is an exhaustive takeover of Hong Kong.''

Advertisement

\protect\hyperlink{after-bottom}{Continue reading the main story}

\hypertarget{site-index}{%
\subsection{Site Index}\label{site-index}}

\hypertarget{site-information-navigation}{%
\subsection{Site Information
Navigation}\label{site-information-navigation}}

\begin{itemize}
\tightlist
\item
  \href{https://help.nytimes.com/hc/en-us/articles/115014792127-Copyright-notice}{©~2020~The
  New York Times Company}
\end{itemize}

\begin{itemize}
\tightlist
\item
  \href{https://www.nytco.com/}{NYTCo}
\item
  \href{https://help.nytimes.com/hc/en-us/articles/115015385887-Contact-Us}{Contact
  Us}
\item
  \href{https://www.nytco.com/careers/}{Work with us}
\item
  \href{https://nytmediakit.com/}{Advertise}
\item
  \href{http://www.tbrandstudio.com/}{T Brand Studio}
\item
  \href{https://www.nytimes.com/privacy/cookie-policy\#how-do-i-manage-trackers}{Your
  Ad Choices}
\item
  \href{https://www.nytimes.com/privacy}{Privacy}
\item
  \href{https://help.nytimes.com/hc/en-us/articles/115014893428-Terms-of-service}{Terms
  of Service}
\item
  \href{https://help.nytimes.com/hc/en-us/articles/115014893968-Terms-of-sale}{Terms
  of Sale}
\item
  \href{https://spiderbites.nytimes.com}{Site Map}
\item
  \href{https://help.nytimes.com/hc/en-us}{Help}
\item
  \href{https://www.nytimes.com/subscription?campaignId=37WXW}{Subscriptions}
\end{itemize}
