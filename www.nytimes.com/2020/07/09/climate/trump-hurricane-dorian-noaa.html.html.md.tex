Sections

SEARCH

\protect\hyperlink{site-content}{Skip to
content}\protect\hyperlink{site-index}{Skip to site index}

\href{https://www.nytimes.com/section/climate}{Climate}

\href{https://myaccount.nytimes.com/auth/login?response_type=cookie\&client_id=vi}{}

\href{https://www.nytimes.com/section/todayspaper}{Today's Paper}

\href{/section/climate}{Climate}\textbar{}NOAA Officials Feared Firings
After Trump's Hurricane Claims, Inspector General Says

\url{https://nyti.ms/2W45GNl}

\begin{itemize}
\item
\item
\item
\item
\item
\end{itemize}

\href{https://www.nytimes.com/section/climate?action=click\&pgtype=Article\&state=default\&region=TOP_BANNER\&context=storylines_menu}{Climate
and Environment}

\begin{itemize}
\tightlist
\item
  \href{https://www.nytimes.com/2020/07/30/climate/sea-level-inland-floods.html?action=click\&pgtype=Article\&state=default\&region=TOP_BANNER\&context=storylines_menu}{Rising
  Seas}
\item
  \href{https://www.nytimes.com/interactive/2020/climate/trump-environment-rollbacks.html?action=click\&pgtype=Article\&state=default\&region=TOP_BANNER\&context=storylines_menu}{Trump's
  Changes}
\item
  \href{https://www.nytimes.com/interactive/2020/04/19/climate/climate-crash-course-1.html?action=click\&pgtype=Article\&state=default\&region=TOP_BANNER\&context=storylines_menu}{Climate
  101}
\item
  \href{https://www.nytimes.com/interactive/2018/08/30/climate/how-much-hotter-is-your-hometown.html?action=click\&pgtype=Article\&state=default\&region=TOP_BANNER\&context=storylines_menu}{Is
  Your Hometown Hotter?}
\item
  \href{https://www.nytimes.com/newsletters/climate-change?action=click\&pgtype=Article\&state=default\&region=TOP_BANNER\&context=storylines_menu}{Newsletter}
\end{itemize}

Advertisement

\protect\hyperlink{after-top}{Continue reading the main story}

Supported by

\protect\hyperlink{after-sponsor}{Continue reading the main story}

\hypertarget{noaa-officials-feared-firings-after-trumps-hurricane-claims-inspector-general-says}{%
\section{NOAA Officials Feared Firings After Trump's Hurricane Claims,
Inspector General
Says}\label{noaa-officials-feared-firings-after-trumps-hurricane-claims-inspector-general-says}}

The report found White House pressure led to NOAA's rebuke of
forecasters who contradicted Mr. Trump's inaccurate claim that Hurricane
Dorian would hit Alabama.

\includegraphics{https://static01.nyt.com/images/2020/07/09/climate/00cli-NOAA/00cli-NOAA-articleLarge.jpg?quality=75\&auto=webp\&disable=upscale}

\href{https://www.nytimes.com/by/christopher-flavelle}{\includegraphics{https://static01.nyt.com/images/2019/06/28/climate/author-chris-flavelle/author-chris-flavelle-thumbLarge-v3.png}}\href{https://www.nytimes.com/by/lisa-friedman}{\includegraphics{https://static01.nyt.com/images/2018/07/18/multimedia/author-lisa-friedman/author-lisa-friedman-thumbLarge.png}}

By \href{https://www.nytimes.com/by/christopher-flavelle}{Christopher
Flavelle} and \href{https://www.nytimes.com/by/lisa-friedman}{Lisa
Friedman}

\begin{itemize}
\item
  July 9, 2020
\item
  \begin{itemize}
  \item
  \item
  \item
  \item
  \item
  \end{itemize}
\end{itemize}

WASHINGTON --- The head of the National Oceanic and Atmospheric
Administration felt that his job and the jobs of others would be in
jeopardy if the agency did not rebuke forecasters who contradicted
President Trump's
\href{https://www.nytimes.com/2019/09/09/climate/hurricane-dorian-trump-tweet.html}{inaccurate
claim last year} about the path of Hurricane Dorian, a government report
found.

The inspector general's report examined the aftermath of Mr. Trump's
insistence that Hurricane Dorian was headed toward Alabama, which
National Weather Service forecasters in Alabama contradicted. It found a
politicized process that investigators described as having ``significant
flaws'' in which late-night demands from White House led to urgent
intercontinental telephone calls, text messages and emails that
culminated in a controversial NOAA statement criticizing the
forecasters.

The inspector general, Peggy E. Gustafson, placed blame largely with top
aides to Secretary of Commerce Wilbur L. Ross Jr., whose agency oversees
NOAA, and who were tasked with coordinating the Sept. 6 unsigned
statement suggesting that the president was right, and that Alabama
forecasters had acted improperly by suggesting otherwise.

She called that statement ``contrary to the apolitical mission'' of the
science agency and described it as ``the end result of events triggered
by an external demand placed on Secretary Ross --- specifically, a
request from the White House to, in Secretary Ross's words, `close the
gap' between President Trump's statement and the
\href{https://www.nytimes.com/2019/11/07/climate/trump-alabama-sharpie-hurricane.html}{NWS
Birmingham tweet}.''

She did not find ``credible evidence'' that top Commerce Department
officials explicitly threatened to fire Neil Jacobs, then the acting
administrator of NOAA. But Dr. Jacobs told investigators that he
``definitely felt like our jobs were on the line'' if he refused to
counter his own weather forecasters.

\href{https://www.nytimes.com/section/climate?action=click\&pgtype=Article\&state=default\&region=MAIN_CONTENT_1\&context=storylines_keepup}{}

\hypertarget{climate-and-environment-}{%
\subsubsection{Climate and Environment
›}\label{climate-and-environment-}}

\hypertarget{keep-up-on-the-latest-climate-news}{%
\paragraph{Keep Up on the Latest Climate
News}\label{keep-up-on-the-latest-climate-news}}

Updated July 30, 2020

Here's what you need to know about the latest climate change news this
week:

\begin{itemize}
\item
  \begin{itemize}
  \tightlist
  \item
    \href{https://www.nytimes.com/2020/07/30/climate/bangladesh-floods.html?action=click\&pgtype=Article\&state=default\&region=MAIN_CONTENT_1\&context=storylines_keepup}{Floods
    in}\href{https://www.nytimes.com/2020/07/30/climate/bangladesh-floods.html?action=click\&pgtype=Article\&state=default\&region=MAIN_CONTENT_1\&context=storylines_keepup}{Bangladesh}
    are punishing the people least responsible for climate change.
  \item
    As climate change raises sea levels,
    \href{https://www.nytimes.com/2020/07/30/climate/sea-level-inland-floods.html?action=click\&pgtype=Article\&state=default\&region=MAIN_CONTENT_1\&context=storylines_keepup}{storm
    surges and high tides} are likely to push farther inland.
  \item
    The E.P.A. inspector general plans to investigate whether a rollback
    of fuel efficiency standards
    \href{https://www.nytimes.com/2020/07/27/climate/trump-fuel-efficiency-rule.html?action=click\&pgtype=Article\&state=default\&region=MAIN_CONTENT_1\&context=storylines_keepup}{violated
    government rules}.
  \end{itemize}
\end{itemize}

``At a minimum, miscommunication or a lack of clarity surrounded the key
issues of whether anyone's job was at risk,'' the report found.

On Sept. 1, Mr. Trump wrote on Twitter that Dorian, which was then
approaching the East Coast of the United States, would hit states,
including Alabama,
``\href{https://twitter.com/realDonaldTrump/status/1168174613827899393}{harder
than anticipated}.'' Forecasters in the Birmingham, Ala., office of the
National Weather Service then contradicted him by assuring the public
they were safe. ``Alabama will NOT see any impacts from Dorian,'' they
wrote.

On Sept. 4 Mr. Trump appeared in the Oval Office
\href{https://www.nytimes.com/2019/09/04/us/politics/trump-hurricane-alabama-sharpie.html}{with
an altered map} of Hurricane Dorian's path, increasing scrutiny of the
president's insistence that Alabama was in danger and lending the
moniker ``Sharpiegate'' to the episode.

The pressure on Dr. Jacobs and his staff originated with Mr. Trump's
acting chief of staff, Mick Mulvaney, who emailed Secretary Ross while
in Greece on agency travel the morning of Sept. 5 asking him to look
into the discrepancy. Mr. Mulvaney then followed up with an email.

Mr. Trump, Mr. Mulvaney said, ``wants either a correction or an
explanation or both'' for the forecasters' statement, according to the
report.

On Sept. 6 NOAA issued an
\href{https://www.nytimes.com/2019/09/06/us/politics/trump-alabama-dorian.html}{unsigned
statement} calling the Birmingham office's Twitter posting
``inconsistent with probabilities from the best forecast products
available at the time.''

In a report last month, NOAA concluded that the statement from Dr.
Jacobs's office
\href{https://www.nytimes.com/2020/06/15/climate/noaa-sharpiegate-ethics-violation.html}{violated
the agency's code of conduct}. That report did not address the actions
of Secretary Ross or other officials at the Commerce Department.

In a series of text message exchanges from Sept. 6 that were included in
the report, Michael Walsh, the chief of staff at the Commerce
Department, suggested a way to portray the president's statements about
Alabama in a more favorable light.

An earlier forecast, which was out of date by the time of Mr. Trump's
post on Twitter, had shown a small chance that Alabama would experience
moderate winds from Dorian. ``I wonder whether we build a narrative that
validates the early Alabama forecast,'' Mr. Walsh wrote to Dr. Jacobs
and Julie Roberts, then a senior political staffer at NOAA.

Mr. Walsh proposed that Dr. Jacobs issue a statement, in which Dr.
Jacobs would say that he had told Mr. Trump during a briefing on the
previous Sunday that ``there was a strong possibility that the hurricane
would punch through Florida and hit the panhandle including Alabama,''
in Mr. Walsh's proposed language.

Ms. Roberts responded to Mr. Walsh: ``We did not tell him Alabama was in
play on Sunday.''

In a response included in the report, Mr. Walsh called the report's
conclusions ``completely unsupported by any of the evidence.''

``The Inspector General instead selectively quotes from interviews,
takes facts out of context, portrays events as related to one another
without any evidence establishing a connection, and ignores basic
governance structures at the Department of Commerce,'' Mr. Walsh wrote.

In a separate response, Sean B. Brebbia, the department's acting deputy
general counsel for the Office of Special Projects, said the report's
lack of formal recommendations ``shows that there were no major flaws in
the Department's handling of this situation."

``The Department views this matter as closed,'' Mr. Brebbia concluded.

Advertisement

\protect\hyperlink{after-bottom}{Continue reading the main story}

\hypertarget{site-index}{%
\subsection{Site Index}\label{site-index}}

\hypertarget{site-information-navigation}{%
\subsection{Site Information
Navigation}\label{site-information-navigation}}

\begin{itemize}
\tightlist
\item
  \href{https://help.nytimes.com/hc/en-us/articles/115014792127-Copyright-notice}{©~2020~The
  New York Times Company}
\end{itemize}

\begin{itemize}
\tightlist
\item
  \href{https://www.nytco.com/}{NYTCo}
\item
  \href{https://help.nytimes.com/hc/en-us/articles/115015385887-Contact-Us}{Contact
  Us}
\item
  \href{https://www.nytco.com/careers/}{Work with us}
\item
  \href{https://nytmediakit.com/}{Advertise}
\item
  \href{http://www.tbrandstudio.com/}{T Brand Studio}
\item
  \href{https://www.nytimes.com/privacy/cookie-policy\#how-do-i-manage-trackers}{Your
  Ad Choices}
\item
  \href{https://www.nytimes.com/privacy}{Privacy}
\item
  \href{https://help.nytimes.com/hc/en-us/articles/115014893428-Terms-of-service}{Terms
  of Service}
\item
  \href{https://help.nytimes.com/hc/en-us/articles/115014893968-Terms-of-sale}{Terms
  of Sale}
\item
  \href{https://spiderbites.nytimes.com}{Site Map}
\item
  \href{https://help.nytimes.com/hc/en-us}{Help}
\item
  \href{https://www.nytimes.com/subscription?campaignId=37WXW}{Subscriptions}
\end{itemize}
