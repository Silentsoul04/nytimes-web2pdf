Sections

SEARCH

\protect\hyperlink{site-content}{Skip to
content}\protect\hyperlink{site-index}{Skip to site index}

\href{https://www.nytimes.com/section/health}{Health}

\href{https://myaccount.nytimes.com/auth/login?response_type=cookie\&client_id=vi}{}

\href{https://www.nytimes.com/section/todayspaper}{Today's Paper}

\href{/section/health}{Health}\textbar{}The Coronavirus Can Be Airborne
Indoors, W.H.O. Says

\url{https://nyti.ms/2W3eGlK}

\begin{itemize}
\item
\item
\item
\item
\item
\item
\end{itemize}

\href{https://www.nytimes.com/news-event/coronavirus?action=click\&pgtype=Article\&state=default\&region=TOP_BANNER\&context=storylines_menu}{The
Coronavirus Outbreak}

\begin{itemize}
\tightlist
\item
  live\href{https://www.nytimes.com/2020/08/01/world/coronavirus-covid-19.html?action=click\&pgtype=Article\&state=default\&region=TOP_BANNER\&context=storylines_menu}{Latest
  Updates}
\item
  \href{https://www.nytimes.com/interactive/2020/us/coronavirus-us-cases.html?action=click\&pgtype=Article\&state=default\&region=TOP_BANNER\&context=storylines_menu}{Maps
  and Cases}
\item
  \href{https://www.nytimes.com/interactive/2020/science/coronavirus-vaccine-tracker.html?action=click\&pgtype=Article\&state=default\&region=TOP_BANNER\&context=storylines_menu}{Vaccine
  Tracker}
\item
  \href{https://www.nytimes.com/interactive/2020/07/29/us/schools-reopening-coronavirus.html?action=click\&pgtype=Article\&state=default\&region=TOP_BANNER\&context=storylines_menu}{What
  School May Look Like}
\item
  \href{https://www.nytimes.com/live/2020/07/31/business/stock-market-today-coronavirus?action=click\&pgtype=Article\&state=default\&region=TOP_BANNER\&context=storylines_menu}{Economy}
\end{itemize}

Advertisement

\protect\hyperlink{after-top}{Continue reading the main story}

Supported by

\protect\hyperlink{after-sponsor}{Continue reading the main story}

\hypertarget{the-coronavirus-can-be-airborne-indoors-who-says}{%
\section{The Coronavirus Can Be Airborne Indoors, W.H.O.
Says}\label{the-coronavirus-can-be-airborne-indoors-who-says}}

The agency also explained more directly that people without symptoms may
spread the virus. The acknowledgments should have come sooner, some
experts said.

\includegraphics{https://static01.nyt.com/images/2020/07/12/science/09virus-aerosols-print/09virus-aerosols-articleLarge.jpg?quality=75\&auto=webp\&disable=upscale}

By \href{https://www.nytimes.com/by/apoorva-mandavilli}{Apoorva
Mandavilli}

\begin{itemize}
\item
  July 9, 2020
\item
  \begin{itemize}
  \item
  \item
  \item
  \item
  \item
  \item
  \end{itemize}
\end{itemize}

The
\href{https://www.nytimes.com/2020/07/06/health/coronavirus-airborne-aerosols.html}{coronavirus
may linger in the air} in crowded indoor spaces, spreading from one
person to the next, the World Health Organization acknowledged on
Thursday.

The W.H.O. had described this form of transmission as doubtful and a
problem mostly in medical procedures. But growing scientific and
anecdotal evidence suggest this route may be important in spreading the
virus, and this week
\href{https://www.nytimes.com/2020/07/04/health/239-experts-with-one-big-claim-the-coronavirus-is-airborne.html}{more
than 200 scientists urged the agency to revisit the research and revise
its position}.

In an updated scientific brief, the agency also asserted more directly
than it had in the past that the virus may be spread by people who do
not have symptoms: ``Infected people can transmit the virus both when
they have symptoms and when they don't have symptoms,'' the agency said.

The W.H.O. previously said asymptomatic transmission, while it may
occur, was
\href{https://www.nytimes.com/2020/06/09/health/coronavirus-asymptomatic-world-health-organization.html}{probably
``very rare.''}

Some experts said both revisions were long overdue, and not as extensive
as they had hoped.

``It is refreshing to see that W.H.O. is now acknowledging that
\href{https://www.nytimes.com/2020/07/06/health/coronavirus-airborne-aerosols.html}{airborne
transmission} may occur, although it is clear that the evidence must
clear a higher bar for this route compared to others,'' Linsey Marr, an
aerosol expert at Virginia Tech, said in an email.

An aerosol is a respiratory droplet so small it may linger in the air.
In its latest description of
\href{https://www.who.int/news-room/q-a-detail/q-a-how-is-covid-19-transmitted}{how
the virus is spread}, the agency said transmission of the virus by
aerosols may have been responsible for ``outbreaks of Covid-19 reported
in some closed settings, such as restaurants, nightclubs, places of
worship or places of work where people may be shouting, talking or
singing.''

The W.H.O. had maintained that airborne spread is only a concern when
health care workers are engaged in certain medical procedures that
produce aerosols. But mounting evidence has suggested that in crowded
indoor spaces, the virus can stay aloft for hours and infect others, and
may even seed so-called superspreader events.

\hypertarget{latest-updates-global-coronavirus-outbreak}{%
\section{\texorpdfstring{\href{https://www.nytimes.com/2020/08/01/world/coronavirus-covid-19.html?action=click\&pgtype=Article\&state=default\&region=MAIN_CONTENT_1\&context=storylines_live_updates}{Latest
Updates: Global Coronavirus
Outbreak}}{Latest Updates: Global Coronavirus Outbreak}}\label{latest-updates-global-coronavirus-outbreak}}

Updated 2020-08-02T00:50:37.907Z

\begin{itemize}
\tightlist
\item
  \href{https://www.nytimes.com/2020/08/01/world/coronavirus-covid-19.html?action=click\&pgtype=Article\&state=default\&region=MAIN_CONTENT_1\&context=storylines_live_updates\#link-34047410}{The
  U.S. reels as July cases more than double the total of any other
  month.}
\item
  \href{https://www.nytimes.com/2020/08/01/world/coronavirus-covid-19.html?action=click\&pgtype=Article\&state=default\&region=MAIN_CONTENT_1\&context=storylines_live_updates\#link-3ac56579}{Top
  officials work to break impasse over jobless benefit.}
\item
  \href{https://www.nytimes.com/2020/08/01/world/coronavirus-covid-19.html?action=click\&pgtype=Article\&state=default\&region=MAIN_CONTENT_1\&context=storylines_live_updates\#link-25930521}{Thousands
  in Berlin protest Germany's coronavirus measures.}
\end{itemize}

\href{https://www.nytimes.com/2020/08/01/world/coronavirus-covid-19.html?action=click\&pgtype=Article\&state=default\&region=MAIN_CONTENT_1\&context=storylines_live_updates}{See
more updates}

More live coverage:
\href{https://www.nytimes.com/live/2020/07/31/business/stock-market-today-coronavirus?action=click\&pgtype=Article\&state=default\&region=MAIN_CONTENT_1\&context=storylines_live_updates}{Markets}

The agency still largely emphasizes the role played by larger droplets
that are coughed or inhaled, or by contact with a contaminated surface,
also called a fomite. And in a
\href{https://www.who.int/publications/i/item/modes-of-transmission-of-virus-causing-covid-19-implications-for-ipc-precaution-recommendations}{longer
document detailing scientific evidence}, the W.H.O. still maintained
that ``detailed investigations of these clusters suggest that droplet
and fomite transmission could also explain human-to-human
transmission.''

In addition to avoiding close contact with infected people and washing
hands, people should ``avoid crowded places, close-contact settings, and
confined and enclosed spaces with poor ventilation,'' the agency said,
and homes and offices should ensure good ventilation.

These recommendations are ``what is needed to help slow transmission in
communities,'' Dr. Marr said.

There is debate about the relative contribution of airborne spread,
compared with transmission by droplets and surfaces. The new brief still
skirts that question.

``I interpret this as saying, `While it is reasonable to think it
\emph{can} happen, there's not consistent evidence that it is happening
often,''' Bill Hanage, an epidemiologist at the Harvard T.H. Chan School
of Public Health, said in an email.

After The Times reported that an
\href{https://www.nytimes.com/2020/07/04/health/239-experts-with-one-big-claim-the-coronavirus-is-airborne.html}{international
group of 239 experts} planned to call on the W.H.O. to review the
research, Dr. Benedetta Allegranzi, head of the agency's infection
prevention and control committee, said on Tuesday that the possibility
of airborne spread in ``crowded, closed, poorly ventilated settings''
\href{https://www.nytimes.com/2020/07/07/health/coronavirus-aerosols-who.html}{could
not be ruled out}.

Outdoors, any virus in small or large droplets may be diluted too
quickly in the air to pose a risk. But even a small possibility of
airborne spread indoors has enormous implications for
\href{https://www.nytimes.com/2020/07/06/health/coronavirus-airborne-aerosols.html}{how
people should protect themselves}.

People may need to minimize time indoors with others from outside the
household, in addition to maintaining a safe distance and wearing cloth
face coverings. Businesses, schools and nursing homes may need to invest
in new ventilation systems or ultraviolet lights that destroy the virus.

Some experts have criticized the W.H.O. for being slow to acknowledge
the possibility of airborne spread while emphasizing hand washing as the
primary preventive strategy. Even in the new brief, it's clear that
members of the committee interpreted the evidence differently, said Dr.
Trish Greenhalgh, a professor of primary health care at the University
of Oxford.

\href{https://www.nytimes.com/news-event/coronavirus?action=click\&pgtype=Article\&state=default\&region=MAIN_CONTENT_3\&context=storylines_faq}{}

\hypertarget{the-coronavirus-outbreak-}{%
\subsubsection{The Coronavirus Outbreak
›}\label{the-coronavirus-outbreak-}}

\hypertarget{frequently-asked-questions}{%
\paragraph{Frequently Asked
Questions}\label{frequently-asked-questions}}

Updated July 27, 2020

\begin{itemize}
\item ~
  \hypertarget{should-i-refinance-my-mortgage}{%
  \paragraph{Should I refinance my
  mortgage?}\label{should-i-refinance-my-mortgage}}

  \begin{itemize}
  \tightlist
  \item
    \href{https://www.nytimes.com/article/coronavirus-money-unemployment.html?action=click\&pgtype=Article\&state=default\&region=MAIN_CONTENT_3\&context=storylines_faq}{It
    could be a good idea,} because mortgage rates have
    \href{https://www.nytimes.com/2020/07/16/business/mortgage-rates-below-3-percent.html?action=click\&pgtype=Article\&state=default\&region=MAIN_CONTENT_3\&context=storylines_faq}{never
    been lower.} Refinancing requests have pushed mortgage applications
    to some of the highest levels since 2008, so be prepared to get in
    line. But defaults are also up, so if you're thinking about buying a
    home, be aware that some lenders have tightened their standards.
  \end{itemize}
\item ~
  \hypertarget{what-is-school-going-to-look-like-in-september}{%
  \paragraph{What is school going to look like in
  September?}\label{what-is-school-going-to-look-like-in-september}}

  \begin{itemize}
  \tightlist
  \item
    It is unlikely that many schools will return to a normal schedule
    this fall, requiring the grind of
    \href{https://www.nytimes.com/2020/06/05/us/coronavirus-education-lost-learning.html?action=click\&pgtype=Article\&state=default\&region=MAIN_CONTENT_3\&context=storylines_faq}{online
    learning},
    \href{https://www.nytimes.com/2020/05/29/us/coronavirus-child-care-centers.html?action=click\&pgtype=Article\&state=default\&region=MAIN_CONTENT_3\&context=storylines_faq}{makeshift
    child care} and
    \href{https://www.nytimes.com/2020/06/03/business/economy/coronavirus-working-women.html?action=click\&pgtype=Article\&state=default\&region=MAIN_CONTENT_3\&context=storylines_faq}{stunted
    workdays} to continue. California's two largest public school
    districts --- Los Angeles and San Diego --- said on July 13, that
    \href{https://www.nytimes.com/2020/07/13/us/lausd-san-diego-school-reopening.html?action=click\&pgtype=Article\&state=default\&region=MAIN_CONTENT_3\&context=storylines_faq}{instruction
    will be remote-only in the fall}, citing concerns that surging
    coronavirus infections in their areas pose too dire a risk for
    students and teachers. Together, the two districts enroll some
    825,000 students. They are the largest in the country so far to
    abandon plans for even a partial physical return to classrooms when
    they reopen in August. For other districts, the solution won't be an
    all-or-nothing approach.
    \href{https://bioethics.jhu.edu/research-and-outreach/projects/eschool-initiative/school-policy-tracker/}{Many
    systems}, including the nation's largest, New York City, are
    devising
    \href{https://www.nytimes.com/2020/06/26/us/coronavirus-schools-reopen-fall.html?action=click\&pgtype=Article\&state=default\&region=MAIN_CONTENT_3\&context=storylines_faq}{hybrid
    plans} that involve spending some days in classrooms and other days
    online. There's no national policy on this yet, so check with your
    municipal school system regularly to see what is happening in your
    community.
  \end{itemize}
\item ~
  \hypertarget{is-the-coronavirus-airborne}{%
  \paragraph{Is the coronavirus
  airborne?}\label{is-the-coronavirus-airborne}}

  \begin{itemize}
  \tightlist
  \item
    The coronavirus
    \href{https://www.nytimes.com/2020/07/04/health/239-experts-with-one-big-claim-the-coronavirus-is-airborne.html?action=click\&pgtype=Article\&state=default\&region=MAIN_CONTENT_3\&context=storylines_faq}{can
    stay aloft for hours in tiny droplets in stagnant air}, infecting
    people as they inhale, mounting scientific evidence suggests. This
    risk is highest in crowded indoor spaces with poor ventilation, and
    may help explain super-spreading events reported in meatpacking
    plants, churches and restaurants.
    \href{https://www.nytimes.com/2020/07/06/health/coronavirus-airborne-aerosols.html?action=click\&pgtype=Article\&state=default\&region=MAIN_CONTENT_3\&context=storylines_faq}{It's
    unclear how often the virus is spread} via these tiny droplets, or
    aerosols, compared with larger droplets that are expelled when a
    sick person coughs or sneezes, or transmitted through contact with
    contaminated surfaces, said Linsey Marr, an aerosol expert at
    Virginia Tech. Aerosols are released even when a person without
    symptoms exhales, talks or sings, according to Dr. Marr and more
    than 200 other experts, who
    \href{https://academic.oup.com/cid/article/doi/10.1093/cid/ciaa939/5867798}{have
    outlined the evidence in an open letter to the World Health
    Organization}.
  \end{itemize}
\item ~
  \hypertarget{what-are-the-symptoms-of-coronavirus}{%
  \paragraph{What are the symptoms of
  coronavirus?}\label{what-are-the-symptoms-of-coronavirus}}

  \begin{itemize}
  \tightlist
  \item
    Common symptoms
    \href{https://www.nytimes.com/article/symptoms-coronavirus.html?action=click\&pgtype=Article\&state=default\&region=MAIN_CONTENT_3\&context=storylines_faq}{include
    fever, a dry cough, fatigue and difficulty breathing or shortness of
    breath.} Some of these symptoms overlap with those of the flu,
    making detection difficult, but runny noses and stuffy sinuses are
    less common.
    \href{https://www.nytimes.com/2020/04/27/health/coronavirus-symptoms-cdc.html?action=click\&pgtype=Article\&state=default\&region=MAIN_CONTENT_3\&context=storylines_faq}{The
    C.D.C. has also} added chills, muscle pain, sore throat, headache
    and a new loss of the sense of taste or smell as symptoms to look
    out for. Most people fall ill five to seven days after exposure, but
    symptoms may appear in as few as two days or as many as 14 days.
  \end{itemize}
\item ~
  \hypertarget{does-asymptomatic-transmission-of-covid-19-happen}{%
  \paragraph{Does asymptomatic transmission of Covid-19
  happen?}\label{does-asymptomatic-transmission-of-covid-19-happen}}

  \begin{itemize}
  \tightlist
  \item
    So far, the evidence seems to show it does. A widely cited
    \href{https://www.nature.com/articles/s41591-020-0869-5}{paper}
    published in April suggests that people are most infectious about
    two days before the onset of coronavirus symptoms and estimated that
    44 percent of new infections were a result of transmission from
    people who were not yet showing symptoms. Recently, a top expert at
    the World Health Organization stated that transmission of the
    coronavirus by people who did not have symptoms was ``very rare,''
    \href{https://www.nytimes.com/2020/06/09/world/coronavirus-updates.html?action=click\&pgtype=Article\&state=default\&region=MAIN_CONTENT_3\&context=storylines_faq\#link-1f302e21}{but
    she later walked back that statement.}
  \end{itemize}
\end{itemize}

``The push-pull of that committee is palpable,'' she said. ``As everyone
knows, if you ask a committee to design a horse, you get a camel.''

Airborne transmission is the most likely explanation for several
clusters of infection, including a choir in Washington State and a
restaurant in China, according to some scientists.

But W.H.O. staff members have yet to accept the importance of these case
studies and instead have ``dreamed up an alternative story'' in which an
infected person spat on his hands, wiped it on something and
``magically'' infected numerous other people, Dr. Greenhalgh said.

\textbf{\emph{{[}}\href{http://on.fb.me/1paTQ1h}{\emph{Like the Science
Times page on Facebook.}}} ****** \emph{\textbar{} Sign up for the}
\textbf{\href{http://nyti.ms/1MbHaRU}{\emph{Science Times
newsletter.}}\emph{{]}}}

The agency's staff and nearly 30 volunteer experts have spent weeks
reviewing evidence on the possible modes of transmission: by exhalation
of large and small droplets, for example, by contact with a contaminated
surface, or from a mother to her baby.

The W.H.O. easily accepts that droplet and fomite transmission occur,
but seems to want more definitive proof of spread by aerosols, some
experts said. The agency has noted that the virus has not been cultured
from air samples, for example, but the same was true of influenza for
many years until two groups of scientists figured out how to do it,
noted Don Milton, an aerosol expert at the University of Maryland.

W.H.O. staff members are reluctant to make statements when they do not
have irrefutable proof of certain phenomena, and are slow to update
their hypotheses, scientists have charged. ``They are still challenged
by the absence of evidence, and the difficulty of proving a negative,''
Dr. Hanage said.

``The W.H.O. is being overly cautious and shortsighted unnecessarily,''
Dr. Julian W. Tang, honorary professor of respiratory sciences at the
University of Leicester in the United Kingdom, said in an email.

``By recognizing aerosol transmission of SARS-CoV-2 and recommending
improved ventilation facilities to be upgraded or installed, you can
improve the health of people'' by eliminating a variety of hazards,
including indoor pollutants and allergens, he added.

``Isn't that what the W.H.O. stands for --- the improvement of human
health from all angles?''

Advertisement

\protect\hyperlink{after-bottom}{Continue reading the main story}

\hypertarget{site-index}{%
\subsection{Site Index}\label{site-index}}

\hypertarget{site-information-navigation}{%
\subsection{Site Information
Navigation}\label{site-information-navigation}}

\begin{itemize}
\tightlist
\item
  \href{https://help.nytimes.com/hc/en-us/articles/115014792127-Copyright-notice}{©~2020~The
  New York Times Company}
\end{itemize}

\begin{itemize}
\tightlist
\item
  \href{https://www.nytco.com/}{NYTCo}
\item
  \href{https://help.nytimes.com/hc/en-us/articles/115015385887-Contact-Us}{Contact
  Us}
\item
  \href{https://www.nytco.com/careers/}{Work with us}
\item
  \href{https://nytmediakit.com/}{Advertise}
\item
  \href{http://www.tbrandstudio.com/}{T Brand Studio}
\item
  \href{https://www.nytimes.com/privacy/cookie-policy\#how-do-i-manage-trackers}{Your
  Ad Choices}
\item
  \href{https://www.nytimes.com/privacy}{Privacy}
\item
  \href{https://help.nytimes.com/hc/en-us/articles/115014893428-Terms-of-service}{Terms
  of Service}
\item
  \href{https://help.nytimes.com/hc/en-us/articles/115014893968-Terms-of-sale}{Terms
  of Sale}
\item
  \href{https://spiderbites.nytimes.com}{Site Map}
\item
  \href{https://help.nytimes.com/hc/en-us}{Help}
\item
  \href{https://www.nytimes.com/subscription?campaignId=37WXW}{Subscriptions}
\end{itemize}
