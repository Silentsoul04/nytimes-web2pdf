Sections

SEARCH

\protect\hyperlink{site-content}{Skip to
content}\protect\hyperlink{site-index}{Skip to site index}

\href{https://myaccount.nytimes.com/auth/login?response_type=cookie\&client_id=vi}{}

\href{https://www.nytimes.com/section/todayspaper}{Today's Paper}

\href{/section/opinion}{Opinion}\textbar{}The Deadly Delusions of Mad
King Donald

\href{https://nyti.ms/320K6Nu}{https://nyti.ms/320K6Nu}

\begin{itemize}
\item
\item
\item
\item
\item
\item
\end{itemize}

Advertisement

\protect\hyperlink{after-top}{Continue reading the main story}

\href{/section/opinion}{Opinion}

Supported by

\protect\hyperlink{after-sponsor}{Continue reading the main story}

\hypertarget{the-deadly-delusions-of-mad-king-donald}{%
\section{The Deadly Delusions of Mad King
Donald}\label{the-deadly-delusions-of-mad-king-donald}}

He won't give up on a failing pandemic strategy.

\href{https://www.nytimes.com/by/paul-krugman}{\includegraphics{https://static01.nyt.com/images/2018/04/02/opinion/paul-krugman/paul-krugman-thumbLarge.png}}

By \href{https://www.nytimes.com/by/paul-krugman}{Paul Krugman}

Opinion Columnist

\begin{itemize}
\item
  July 9, 2020
\item
  \begin{itemize}
  \item
  \item
  \item
  \item
  \item
  \item
  \end{itemize}
\end{itemize}

\includegraphics{https://static01.nyt.com/images/2020/07/09/opinion/09krugman1/merlin_173860392_59b3040f-fc6a-42e1-980c-759ddbaeabe8-articleLarge.jpg?quality=75\&auto=webp\&disable=upscale}

I don't know about you, but I'm feeling more and more as if we're all
trapped on the Titanic --- except that this time around the captain is a
madman who insists on steering straight for the iceberg. And his crew is
too cowardly to contradict him, let alone mutiny to save the passengers.

A month ago it was still possible to hope that the push by Donald Trump
and the Trumpist governors of Sunbelt states to relax social distancing
and reopen businesses like restaurants and bars --- even though we met
none of the criteria for doing so safely --- wouldn't have completely
catastrophic results.

At this point, however, it's clear that everything the experts warned
was likely to happen, is happening. Daily new cases of Covid-19 are
running
\href{https://www.nytimes.com/interactive/2020/us/coronavirus-us-cases.html?action=click\&module=Top\%20Stories\&pgtype=Homepage}{two
and a half times as high} as in early June, and rising fast. Hospitals
in early-reopening states are under
\href{https://www.washingtonpost.com/politics/surge-in-virus-hospitalizations-strains-hospitals-in-several-states/2020/07/08/12855e5e-c135-11ea-864a-0dd31b9d6917_story.html}{terrible
pressure}. National death totals are still declining thanks to falling
fatalities in the Northeast, but they're
\href{https://twitter.com/COVID19Tracking/status/1281011411901177858}{rising}
in the Sunbelt, and the worst is surely yet to come.

A normal president and a normal political party would be horrified by
this turn of events. They would realize that they made a bad call and
that it was time for a major course correction; they would start taking
warnings from health experts seriously.

But Trump, who began his presidency with a lurid, fact-challenged rant
about
``\href{https://www.nytimes.com/interactive/projects/cp/opinion/presidential-inauguration-2017/trump-gives-us-american-carnage}{American
carnage},'' seems completely untroubled by the toll from a pandemic that
seems certain to kill more Americans than were murdered over the
\href{http://www.disastercenter.com/crime/uscrime.htm}{whole of the past
decade}. And he's doubling down on his rejection of expertise, this week
demanding
\href{https://www.nytimes.com/2020/07/08/us/politics/trump-schools-reopening.html?action=click\&module=Well\&pgtype=Homepage\&section=Politics}{full
reopening} of schools in defiance of existing guidelines.

Oh, and he still won't call on Americans to protect one another by
wearing masks, or set an example by wearing one himself.

How can we make sense of Trump's pathologically inept response to the
coronavirus? There's an underlying core of utter cynicism: Clearly,
Trump and those around him don't care very much how many Americans die
or suffer lasting damage from Covid-19, as long as the politics work in
their favor. But this cynicism is wrapped in multiple layers of
delusion.

On one side, it's clear that the Trumpists still can't accept that this
is really happening.

Until early 2020, Trump led a charmed political life. All his recent
predecessors had to deal with some kind of external challenge during
their first three years. Barack Obama inherited an economy wracked by a
financial crisis. Whatever you think of his response, George W. Bush
faced 9/11. Bill Clinton faced stubbornly high
\href{https://fred.stlouisfed.org/series/UNRATE}{unemployment}. But
Trump inherited a nation at peace and in the middle of a long economic
expansion that continued, with no visible change in the
\href{https://fred.stlouisfed.org/series/PAYEMS}{trend}, after he took
office.

Then came Covid-19. Another president might have seen the pandemic as a
crisis to be dealt with. But that thought never seems to have crossed
Trump's mind. Instead, he has spent the past five months trying to will
us back to where we were in February, when he was sitting on top of a
moving train and pretending that he was driving it.

This helps explain his otherwise bizarre aversion to masks: They remind
people that we're in the midst of a pandemic, which is something he
wants everyone to forget. Unfortunately for him --- and for the rest of
us --- positive thinking won't make a virus go away.

That, however, is where the second layer of delusion comes in. By now
it's clear that the cynical decision to sacrifice American lives in
pursuit of political advantage is failing even on its own terms. The
rush to reopen did produce big job gains in May and early June, but
voters were distinctly unimpressed; his polling just kept
\href{https://projects.fivethirtyeight.com/polls/president-general/national/}{getting}
\href{https://projects.fivethirtyeight.com/trump-approval-ratings/?cid=rrpromo}{worse}.
This year, it's not the economy, stupid --- it's the
\href{https://www.cnn.com/2020/07/09/politics/2020-election-issues-coronavirus-trump/index.html}{virus}.

And now the surge in infections may be causing the economic recovery to
\href{https://www.wsj.com/articles/new-coronavirus-surges-stall-economic-recovery-11594209321}{stall}.

In other words, the strategy of ``damn the experts, full speed ahead''
is looking foolish as well as immoral. But Trump, far from
reconsidering, is digging the hole he's in ever deeper --- much the same
way that he keeps turning up the dial on racism despite the fact that
it's not working for him politically. Incredibly, even as
hospitalizations climb he's still insisting that the rise in reported
cases is just an illusion created by
\href{https://twitter.com/GarrettHaake/status/1281209237281046529}{more
testing}.

So what can we do? Trump has another six months in office (if he's still
there after Jan. 20, God help us all). And it's now clear that he won't
change course, no matter how bad the pandemic gets. As I said, we're all
passengers at the mercy of a mad captain determined to wreck his ship.

It's true that federalism is our friend. Trump doesn't actually have any
direct authority over things like school openings. And many though not
all states have rational governors who are trying to contain the damage,
although it's hard to keep the lid on in New Jersey or Michigan when the
coronavirus is running wild in Florida.

But a lot more Americans are going to die. And if Joe Biden becomes
president, he, like Obama 12 years ago, is going to take the helm of a
nation in a deep crisis.

\emph{The Times is committed to publishing}
\href{https://www.nytimes.com/2019/01/31/opinion/letters/letters-to-editor-new-york-times-women.html}{\emph{a
diversity of letters}} \emph{to the editor. We'd like to hear what you
think about this or any of our articles. Here are some}
\href{https://help.nytimes.com/hc/en-us/articles/115014925288-How-to-submit-a-letter-to-the-editor}{\emph{tips}}\emph{.
And here's our email:}
\href{mailto:letters@nytimes.com}{\emph{letters@nytimes.com}}\emph{.}

\emph{Follow The New York Times Opinion section on}
\href{https://www.facebook.com/nytopinion}{\emph{Facebook}}\emph{,}
\href{http://twitter.com/NYTOpinion}{\emph{Twitter (@NYTopinion)}}
\emph{and}
\href{https://www.instagram.com/nytopinion/}{\emph{Instagram}}\emph{.}

Advertisement

\protect\hyperlink{after-bottom}{Continue reading the main story}

\hypertarget{site-index}{%
\subsection{Site Index}\label{site-index}}

\hypertarget{site-information-navigation}{%
\subsection{Site Information
Navigation}\label{site-information-navigation}}

\begin{itemize}
\tightlist
\item
  \href{https://help.nytimes.com/hc/en-us/articles/115014792127-Copyright-notice}{©~2020~The
  New York Times Company}
\end{itemize}

\begin{itemize}
\tightlist
\item
  \href{https://www.nytco.com/}{NYTCo}
\item
  \href{https://help.nytimes.com/hc/en-us/articles/115015385887-Contact-Us}{Contact
  Us}
\item
  \href{https://www.nytco.com/careers/}{Work with us}
\item
  \href{https://nytmediakit.com/}{Advertise}
\item
  \href{http://www.tbrandstudio.com/}{T Brand Studio}
\item
  \href{https://www.nytimes.com/privacy/cookie-policy\#how-do-i-manage-trackers}{Your
  Ad Choices}
\item
  \href{https://www.nytimes.com/privacy}{Privacy}
\item
  \href{https://help.nytimes.com/hc/en-us/articles/115014893428-Terms-of-service}{Terms
  of Service}
\item
  \href{https://help.nytimes.com/hc/en-us/articles/115014893968-Terms-of-sale}{Terms
  of Sale}
\item
  \href{https://spiderbites.nytimes.com}{Site Map}
\item
  \href{https://help.nytimes.com/hc/en-us}{Help}
\item
  \href{https://www.nytimes.com/subscription?campaignId=37WXW}{Subscriptions}
\end{itemize}
