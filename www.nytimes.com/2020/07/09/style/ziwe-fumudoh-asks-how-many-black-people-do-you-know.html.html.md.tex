Sections

SEARCH

\protect\hyperlink{site-content}{Skip to
content}\protect\hyperlink{site-index}{Skip to site index}

\href{https://www.nytimes.com/section/style}{Style}

\href{https://myaccount.nytimes.com/auth/login?response_type=cookie\&client_id=vi}{}

\href{https://www.nytimes.com/section/todayspaper}{Today's Paper}

\href{/section/style}{Style}\textbar{}Ziwe Fumudoh Asks: `How Many Black
People Do You Know?'

\url{https://nyti.ms/3iOU2zI}

\begin{itemize}
\item
\item
\item
\item
\item
\end{itemize}

Advertisement

\protect\hyperlink{after-top}{Continue reading the main story}

Supported by

\protect\hyperlink{after-sponsor}{Continue reading the main story}

\hypertarget{ziwe-fumudoh-asks-how-many-black-people-do-you-know}{%
\section{Ziwe Fumudoh Asks: `How Many Black People Do You
Know?'}\label{ziwe-fumudoh-asks-how-many-black-people-do-you-know}}

The comedian performs a kind of racial high wire act on her Instagram
Live show, but she really just wants to heal.

\includegraphics{https://static01.nyt.com/images/2020/07/12/fashion/09ZIWE-1/09ZIWE-1-articleLarge.jpg?quality=75\&auto=webp\&disable=upscale}

\href{https://www.nytimes.com/by/sandra-e-garcia}{\includegraphics{https://static01.nyt.com/images/2020/07/10/reader-center/author-sandra-e-garcia/author-sandra-e-garcia-thumbLarge.png}}

By \href{https://www.nytimes.com/by/sandra-e-garcia}{Sandra E. Garcia}

\begin{itemize}
\item
  July 9, 2020
\item
  \begin{itemize}
  \item
  \item
  \item
  \item
  \item
  \end{itemize}
\end{itemize}

How many Black friends do you have? Is it ``between four and five?'' If
so, then you have something in common with several guests on
\href{https://www.instagram.com/ziwef/?hl=en}{Ziwe Fumudoh's Instagram
Live show}.

In it, she interviews people who have been
\href{https://www.nytimes.com/2018/06/28/style/is-it-canceled.html}{canceled}---
meaning that their personal views, whether political or artistic, are no
longer welcome --- but she also interviews comedians and people who
serve as mouthpieces for pop culture like
\href{https://www.nytimes.com/2016/08/18/style/jeremy-o-harris-actor-playwright-yale-james-franco.html}{Jeremy
O. Harris}, an actor and playwright. Ms. Fumudoh, a comedian who goes
professionally by just her first name, is still figuring out her guest
roster.

Her last two interviews in particular have made the show indelible. She
asks her white guests questions about race that often reveal their
ignorance or self-involvement.

This can include straightforward questions, like ``How many Black people
do you know?'' and more complicated ones, like ``Did your family own
slaves?'' There's also the fraught but impossible, including: ``When you
say Black people, do you capitalize the B?''

Caroline Calloway, a faux-chaotic Instagram figure powered by her
privilege and moored by her manufactured anarchy, was a recent guest.
She has been canceled many times --- once for charging \$165 to attend
her ``creativity workshops'' and then again when her ghostwriter
\href{https://www.nytimes.com/2019/09/11/style/caroline-calloway-explainer.html}{revealed
the details of their tumultuous friendship}.

Another recent guest, Alison Roman, is a cookbook author and food
personality who took on damage to her budding career when she insulted
Chrissy Teigen and Marie Kondo, two Asian women, in an interview. (Ms.
Roman is a columnist for The New York Times; the column is on temporary
leave.)

``It's just thinking, `What is the most absurd hyperbolic question that
I can confront someone with and how do I make them look me in the eyes
and either say an answer or not say anything at all?''' Ms. Fumudoh
said.

It's reasonable to wonder why many would agree to such an interview. Ms.
Fumudoh, 28, said she doesn't know why her guests do the interviews,
either. She books them simply by asking. When she asked Ms. Calloway to
come on her show --- because she attended Phillips Exeter Academy, Ms.
Fumudoh's rival high school in New Hampshire --- she called it a ``long
shot.'' When Ms. Calloway agreed, Ms. Fumudoh was surprised.

``It was never a long plan to book Caroline Calloway on the show,'' Ms.
Fumudoh said. ``She stands in contrast to a lot of my other guests, who
are comedians that I know and perform with in New York.''

In their interviews, Ms. Calloway and Ms. Roman appear candid with their
responses and comfortable with the line of questioning --- save for the
moment when Ms. Roman's face flushes after being asked how many Black
friends she has. The entire interview feels like an intimate
conversation between two people that we are allowed to listen in on. Ms.
Fumudoh comes off as inviting, funny, honest and warm and also smart and
no-nonsense. When Ms. Fumudoh interviewed the actress Rose McGowan, a
prominent figure in the \#MeToo movement who accused Harvey Weinstein of
sexual assault, Ms. McGowan sounded as if she was talking to a long-lost
friend.

When Ms. Calloway requests an ``ally cookie'' for buying books from a
Black-owned bookstore, Ms. Fumudoh is stern and sincere when she
responds: ``There are no cookies in this game.''

``Ultimately I'm not trying to make any of my guests look bad,'' Ms.
Fumudoh said. ``I'm just trying to start a really productive and healthy
conversation that promotes healing with the trauma that is our racist
history of this country.''

The point of the show is not to chastise people that have been publicly
canceled, Ms. Fumudoh said. For her, the most interesting thing about
the show is the comments she gets, particularly people saying, ``I
listened to those Ziwe questions and I've thought about how I would
answer them.''

One thing Ms. Fumudoh, who studied African-American studies, film and
poetry at Northwestern University, wants to be clear about is that she
is not calling anyone racist or demonizing them. She finds racism
against Black women to be the biggest obstacle in her life and wants to
contribute toward positive change and healing.

``I don't want to position myself as a racial authority because that's
not who I am,'' she said. ``I am just a Black woman who is an artist in
2020 during one of the biggest civil rights fights of a generation.''

\href{https://www.nytimes.com/2016/11/29/arts/television/larry-wilmore-after-comedy-central-signs-deal-with-abc.html}{Larry
Wilmore}, a comedian with deep credentials doing difficult work about
race (he was the ``senior Black correspondent'' on ``The Daily Show''
and helped create ``\href{https://www.hbo.com/insecure}{Insecure}'' with
Issa Rae), said that Ms. Fumudoh has mastered awkwardness in
conversations about race in a way he hasn't seen before.

``It's kind of a racial high-wire act having that kind of
conversation,'' Mr. Wilmore said. ``She seems generous in her approach
too, which is nice, but yet there's something else going on there.
There's a twinkle behind her eye that makes you go, `What is she doing
here exactly?'''

Her interviews are more of a wink to people who expect Ms. Calloway and
Ms. Roman to respond to the questions as they did, according to Mr.
Wilmore. In 2016, when he hosted the
\href{https://www.youtube.com/watch?v=1IDFt3BL7FA}{White House
Correspondents' Association Dinner}, he was criticized for calling
President Obama a racist slur; that was a risky wink.

``I took a lot of blowback from that,'' Mr. Wilmore said. ``But the
people who understood it, there was a lot of praise and support for
it.''

Ms. Fumudoh's wink is aimed more at Black women who feel awkward and
uncomfortable all the time simply because they are navigating the world.
She makes the discomfort visible and shareable by all.

``This is my way of seizing my authority and my autonomy,'' she said,
``and pushing that back onto society and saying, `Hey, I'm not going to
be the only one who's going through this life feeling discomfort.'''

The guests on the show do get uncomfortable, like when Ms. Fumudoh asks
them how many Black friends they have, or when they fumble if they know
who Marcus Garvey or Huey Newton are. Ms. Fumudoh, though, was quick on
her feet when asked to name five white people in 10 seconds.

``Three of the Haim sisters, John F. Kennedy, shout out to him, and the
fantastic actress Anne Hathaway,'' she answered immediately. It's not
about naming five Black friends, or five Black authors you've read. Ms.
Fumudoh simply does not want to be the lone carrier of the tension that
she faces as a Black woman.

``I'm just trying to heal,'' Ms. Fumudoh said. ``I'm just trying to make
people laugh and make people feel good.''

Advertisement

\protect\hyperlink{after-bottom}{Continue reading the main story}

\hypertarget{site-index}{%
\subsection{Site Index}\label{site-index}}

\hypertarget{site-information-navigation}{%
\subsection{Site Information
Navigation}\label{site-information-navigation}}

\begin{itemize}
\tightlist
\item
  \href{https://help.nytimes.com/hc/en-us/articles/115014792127-Copyright-notice}{©~2020~The
  New York Times Company}
\end{itemize}

\begin{itemize}
\tightlist
\item
  \href{https://www.nytco.com/}{NYTCo}
\item
  \href{https://help.nytimes.com/hc/en-us/articles/115015385887-Contact-Us}{Contact
  Us}
\item
  \href{https://www.nytco.com/careers/}{Work with us}
\item
  \href{https://nytmediakit.com/}{Advertise}
\item
  \href{http://www.tbrandstudio.com/}{T Brand Studio}
\item
  \href{https://www.nytimes.com/privacy/cookie-policy\#how-do-i-manage-trackers}{Your
  Ad Choices}
\item
  \href{https://www.nytimes.com/privacy}{Privacy}
\item
  \href{https://help.nytimes.com/hc/en-us/articles/115014893428-Terms-of-service}{Terms
  of Service}
\item
  \href{https://help.nytimes.com/hc/en-us/articles/115014893968-Terms-of-sale}{Terms
  of Sale}
\item
  \href{https://spiderbites.nytimes.com}{Site Map}
\item
  \href{https://help.nytimes.com/hc/en-us}{Help}
\item
  \href{https://www.nytimes.com/subscription?campaignId=37WXW}{Subscriptions}
\end{itemize}
