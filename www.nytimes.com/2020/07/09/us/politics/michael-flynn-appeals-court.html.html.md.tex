Sections

SEARCH

\protect\hyperlink{site-content}{Skip to
content}\protect\hyperlink{site-index}{Skip to site index}

\href{https://www.nytimes.com/section/politics}{Politics}

\href{https://myaccount.nytimes.com/auth/login?response_type=cookie\&client_id=vi}{}

\href{https://www.nytimes.com/section/todayspaper}{Today's Paper}

\href{/section/politics}{Politics}\textbar{}Judge in Michael Flynn Case
Asks for Full Appeals Court Review

\url{https://nyti.ms/325OnPJ}

\begin{itemize}
\item
\item
\item
\item
\item
\end{itemize}

Advertisement

\protect\hyperlink{after-top}{Continue reading the main story}

Supported by

\protect\hyperlink{after-sponsor}{Continue reading the main story}

\hypertarget{judge-in-michael-flynn-case-asks-for-full-appeals-court-review}{%
\section{Judge in Michael Flynn Case Asks for Full Appeals Court
Review}\label{judge-in-michael-flynn-case-asks-for-full-appeals-court-review}}

After a divided appeals court panel ordered him to drop the charge
against the former national security adviser, a judge asked the full
court to review the ruling.

\includegraphics{https://static01.nyt.com/images/2020/07/09/us/politics/09dc-flynn/merlin_172292805_8752e9b2-cf91-4d21-bf61-d404d36832d3-articleLarge.jpg?quality=75\&auto=webp\&disable=upscale}

\href{https://www.nytimes.com/by/adam-goldman}{\includegraphics{https://static01.nyt.com/images/2018/07/12/multimedia/author-adam-goldman/author-adam-goldman-thumbLarge.png}}

By \href{https://www.nytimes.com/by/adam-goldman}{Adam Goldman}

\begin{itemize}
\item
  July 9, 2020
\item
  \begin{itemize}
  \item
  \item
  \item
  \item
  \item
  \end{itemize}
\end{itemize}

WASHINGTON --- The judge overseeing the case of President Trump's former
national security adviser Michael T. Flynn asked a full appeals court on
Thursday to review
\href{https://www.nytimes.com/2020/06/24/us/politics/michael-flynn-appeals-court.html}{an
order} by a panel of its judges to end the prosecution, saying their
ruling marked ``a dramatic break from precedent that threatens the
orderly administration of justice.''

The request by the trial judge, Emmet G. Sullivan, was the latest turn
in
\href{https://www.nytimes.com/2020/06/28/us/politics/michael-flynn-sidney-powell.html}{an
extraordinary legal battle} over the case against Mr. Flynn, who twice
pleaded guilty to a charge of lying to the F.B.I. about his
conversations with a Russian diplomat during the presidential transition
in late 2016. The Justice Department sought in May
\href{https://www.nytimes.com/2020/05/07/us/politics/michael-flynn-case-dropped.html}{to
dismiss the case} in a highly unusual move that prompted
\href{https://www.nytimes.com/2020/05/08/us/politics/barr-flynn-case-justice-department.html}{accusations
of politicization}, and Judge Sullivan appointed an outsider to argue
against the department's request rather than granting it.

Mr. Flynn's lawyer, Sidney Powell, then asked the appeals panel to issue
an emergency ruling over whether Judge Sullivan had the legal authority
to scrutinize the Justice Department's move. Last month, a divided panel
ruled 2 to 1 in favor of Mr. Flynn, ordering Judge Sullivan to end the
case without further review.

``The panel's decision threatens to turn ordinary judicial process
upside down,'' a lawyer for Judge Sullivan wrote in
\href{https://pacer-documents.s3.amazonaws.com/207/20-05143/01208242956.pdf}{the
petition} asking the full appeals court to examine the ruling. ``It is
the district court's job to consider and rule on pending motions, even
ones that seem straightforward.''

Mr. Flynn was the only White House official to plead guilty to a
criminal charge in the Russia investigation. Judge Sullivan had been set
to sentence Mr. Flynn in late 2018 but granted him more time to maximize
his cooperation agreement and testify for the government against a
former business associate in a foreign lobbying case.

But after Mr. Flynn hired new lawyers, he changed his stance, eventually
declaring this year that he was innocent and seeking to withdraw his
guilty plea. He made allegations of misconduct by prosecutors and the
F.B.I., which Judge Sullivan rejected.

Attorney General William P. Barr stepped in, appointing the top federal
prosecutor in St. Louis, Jeff Jensen, to review the case. As part of his
review, Mr. Jensen handed over documents to Mr. Flynn's lawyers, who
declared them exculpatory.

That helped prompt the Justice Department to move to drop the case after
a long public campaign by Mr. Trump and his allies, leading to
accusations of political interference. None of the prosecutors who had
worked on the case over the previous two and a half years signed the
motion, and the lead prosecutor, Brandon L. Van Grack, withdrew from it
altogether.

Instead of granting the motion, Judge Sullivan appointed a former
federal judge and onetime mob prosecutor, John Gleeson, to argue against
it and invited legal experts to weigh in, suggesting that he was
skeptical of the government's rationale.

Experts broadly disputed the Justice Department's assertion that Mr.
Flynn's lies were not material since the F.B.I. was on the verge of
closing its investigation of him, noting that they bore on the broader
counterintelligence investigation into whether Trump campaign officials
had coordinated with Russia's 2016 election interference. Judge Sullivan
had previously ruled that Mr. Flynn's lies were relevant to the inquiry.

His decision to appoint Mr. Gleeson then spurred Ms. Powell's emergency
filing with the appeals panel seeking a so-called writ of mandamus, with
the Justice Department arguing that if the case was not dropped it would
harm the executive branch's exclusive prosecutorial power.

The dissenting judge in the panel's 2-to-1 decision said Mr. Sullivan
should be allowed to rule.

``The district court must be given a reasonable opportunity to consider
and hold a hearing on the government's request to ensure that it is not
clearly contrary to the public interest,'' Robert L. Wilkins, a 2014
appointee of President Barack Obama, wrote.

The order had handed Mr. Flynn and the Justice Department a crucial
victory as it meant that a hearing Judge Sullivan had scheduled for next
week would not take place. The judge most likely would have pressed the
Justice Department over its decision to drop the charge and why
prosecutors who had worked at length on the case had not signed the
motion.

In another development on Thursday, the Justice Department said it did
not raise objections to Mr. Trump's longtime friend Roger J. Stone Jr.
beginning a 40-month prison sentence later this month. Mr. Stone had
asked recently for a delay until Sept. 1 because of the coronavirus
pandemic, citing health concerns, but a judge partly rejected his
request, allowing him to put off the start of his sentence only until
next week.

Mr. Stone was convicted of seven felonies in a bid to impede a
congressional inquiry that threatened the president.

Another former aide to Mr. Trump, his onetime lawyer and fixer Michael
D. Cohen,
\href{https://www.nytimes.com/2020/07/09/nyregion/michael-cohen-arrested.html}{was
taken back into federal custody} on Thursday more than a month after
being granted a medical furlough from prison, where he was serving a
three-year sentence for campaign finance violations and other crimes.

The federal Bureau of Prisons said without elaborating that Mr. Cohen
``refused the conditions of his home confinement.'' A person briefed on
his legal status said he had refused to sign papers agreeing to
conditions related to media appearances and the writing of books.

Advertisement

\protect\hyperlink{after-bottom}{Continue reading the main story}

\hypertarget{site-index}{%
\subsection{Site Index}\label{site-index}}

\hypertarget{site-information-navigation}{%
\subsection{Site Information
Navigation}\label{site-information-navigation}}

\begin{itemize}
\tightlist
\item
  \href{https://help.nytimes.com/hc/en-us/articles/115014792127-Copyright-notice}{©~2020~The
  New York Times Company}
\end{itemize}

\begin{itemize}
\tightlist
\item
  \href{https://www.nytco.com/}{NYTCo}
\item
  \href{https://help.nytimes.com/hc/en-us/articles/115015385887-Contact-Us}{Contact
  Us}
\item
  \href{https://www.nytco.com/careers/}{Work with us}
\item
  \href{https://nytmediakit.com/}{Advertise}
\item
  \href{http://www.tbrandstudio.com/}{T Brand Studio}
\item
  \href{https://www.nytimes.com/privacy/cookie-policy\#how-do-i-manage-trackers}{Your
  Ad Choices}
\item
  \href{https://www.nytimes.com/privacy}{Privacy}
\item
  \href{https://help.nytimes.com/hc/en-us/articles/115014893428-Terms-of-service}{Terms
  of Service}
\item
  \href{https://help.nytimes.com/hc/en-us/articles/115014893968-Terms-of-sale}{Terms
  of Sale}
\item
  \href{https://spiderbites.nytimes.com}{Site Map}
\item
  \href{https://help.nytimes.com/hc/en-us}{Help}
\item
  \href{https://www.nytimes.com/subscription?campaignId=37WXW}{Subscriptions}
\end{itemize}
