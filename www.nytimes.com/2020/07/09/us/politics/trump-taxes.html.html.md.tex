Sections

SEARCH

\protect\hyperlink{site-content}{Skip to
content}\protect\hyperlink{site-index}{Skip to site index}

\href{https://www.nytimes.com/section/politics}{Politics}

\href{https://myaccount.nytimes.com/auth/login?response_type=cookie\&client_id=vi}{}

\href{https://www.nytimes.com/section/todayspaper}{Today's Paper}

\href{/section/politics}{Politics}\textbar{}Trump on Releasing His Tax
Returns: From `Absolutely' to `Political Prosecution'

\url{https://nyti.ms/3iX75zc}

\begin{itemize}
\item
\item
\item
\item
\item
\item
\end{itemize}

Advertisement

\protect\hyperlink{after-top}{Continue reading the main story}

Supported by

\protect\hyperlink{after-sponsor}{Continue reading the main story}

\hypertarget{trump-on-releasing-his-tax-returns-from-absolutely-to-political-prosecution}{%
\section{Trump on Releasing His Tax Returns: From `Absolutely' to
`Political
Prosecution'}\label{trump-on-releasing-his-tax-returns-from-absolutely-to-political-prosecution}}

The Supreme Court's decision adds another twist in a yearslong debate
over whether the president should release his tax returns.

\includegraphics{https://static01.nyt.com/images/2020/07/09/us/politics/09dc-trumptaxes/merlin_174367410_d85fafc1-6cb2-4a56-9f8d-5d3499e6ba3f-articleLarge.jpg?quality=75\&auto=webp\&disable=upscale}

\href{https://www.nytimes.com/by/katie-rogers}{\includegraphics{https://static01.nyt.com/images/2018/06/12/multimedia/author-katie-rogers/author-katie-rogers-thumbLarge-v2.png}}

By \href{https://www.nytimes.com/by/katie-rogers}{Katie Rogers}

\begin{itemize}
\item
  Published July 9, 2020Updated July 15, 2020
\item
  \begin{itemize}
  \item
  \item
  \item
  \item
  \item
  \item
  \end{itemize}
\end{itemize}

WASHINGTON --- In September 2016, Donald J. Trump stood on the debate
stage as a presidential candidate and addressed a question that had
dogged him on the campaign trail: When would he release his
\href{https://www.nytimes.com/2020/07/15/nyregion/donald-trump-taxes-cyrus-vance.html}{tax
return}?

``I'm under a routine audit, and it'll be released,'' Mr. Trump said.
``And as soon as the audit is finished, it will be released.''

Nearly four years later, the White House says the I.R.S. is still at it.

``His taxes are under audit, and when they're no longer under audit he
will release them,'' Kayleigh McEnany, the White House press secretary,
told reporters on Thursday.

In fact, every sitting president's returns are audited as a matter of
routine, and the I.R.S. has long said that nothing prevents an
individual from making tax returns public while an audit is underway.
Every president since Jimmy Carter has voluntarily released his returns.

So at this point, no one is expecting to see the
\href{https://www.nytimes.com/2020/08/03/nyregion/donald-trump-taxes-cyrus-vance.html}{president's
tax returns} anytime soon, even though the Supreme Court issued a
\href{https://www.nytimes.com/2020/07/09/us/trump-taxes-supreme-court.html}{major
ruling} on Thursday that cleared the way for New York prosecutors to
seek them. But there will be further skirmishing in the lower courts,
and there is little chance of a final decision before the next election.

\hypertarget{the-early-promises}{%
\subsection{The Early Promises}\label{the-early-promises}}

Mr. Trump has promised to release his tax returns under varying
conditions for nearly a decade.

In 2011, he began appearing on television to question whether President
Barack Obama was born in the United States --- spreading a lie that he
has never fully apologized for --- and suggesting that he would release
his returns when Mr. Obama released his birth certificate.

``Maybe I'm going to do the tax returns when Obama does his birth
certificate,'' he said in an interview with ABC in April 2011. ``I'd
love to give my tax returns. I may tie my tax returns into Obama's birth
certificate.''

Days after that interview, Mr. Obama released his long-form birth
certificate.

Mr. Trump did not keep his end of the deal. In 2014, an Irish journalist
pointed out that he had never released his tax returns, even though he
had coerced Mr. Obama into releasing his birth certificate. In that
interview, Mr. Trump then added a new qualifier: He would release them
if he ran for president.

``If I decide to run for office, I'll produce my tax returns,
absolutely,'' he said during a visit to Ireland, where he promoted his
golf club in Doonbeg. ``And I would love to do that.''

\hypertarget{the-audit}{%
\subsection{The Audit}\label{the-audit}}

By the time Mr. Trump was running for president in 2016, he had adopted
the audit as the reason he could not release his taxes. That spring,
\href{https://assets.donaldjtrump.com/Tax_Doc.pdf}{his lawyer Sheri A.
Dillon released a letter} that claimed Mr. Trump's tax returns had been
under ``continuous examination'' by the I.R.S. since 2002, and that the
audit for his tax returns since 2009 was ongoing.

Steven M. Rosenthal, a senior fellow at the Urban-Brookings Tax Policy
Center, said in an interview that it could be normal for an audit for a
taxpayer like Mr. Trump to take anywhere from six to eight years for
each year filed. He called Ms. Dillon's letter ``on the mark'' and said
that presidents were automatically audited each year while in office.
But he said there was no legal reason for Mr. Trump to hold back his tax
returns.

``The excuse that he's under audit is a non-excuse,'' Mr. Rosenthal
said. ``He's always under audit.''

After Mr. Trump won the election, he added another reason beyond the
audit for why he was withholding his returns. In May 2017, he told The
Economist that only journalists cared about his tax returns, and that he
might not release them until he left office.

``Maybe I'll release them after I'm finished because I'm very proud of
them actually,'' Mr. Trump said. ``I did a good job.''

In reality, polls show a majority of Americans believe that the public
has a right to see Mr. Trump's tax returns, as they have seen the
returns of every modern president over the last four decades. A June
\href{https://www.pewresearch.org/politics/2020/06/30/publics-mood-turns-grim-trump-trails-biden-on-most-personal-traits-major-issues/}{poll}
by the Pew Research Center found that 56 percent of Americans said Mr.
Trump had a responsibility to release them.

One critical question the returns would answer is how much Mr. Trump
paid in taxes, or whether he paid taxes at all. In October 2016, an
\href{https://www.nytimes.com/2016/10/02/us/politics/donald-trump-taxes.html}{investigation}
by The New York Times revealed that a \$916 million loss on Mr. Trump's
1995 returns would have allowed him to legally avoid paying income taxes
for 18 years. Mr. Trump declined to comment, and his campaign released a
statement that neither challenged nor confirmed the \$916 million loss.

\href{https://www.nytimes.com/interactive/2018/10/02/us/politics/donald-trump-tax-schemes-fred-trump.html}{Another
investigation} by The Times in 2018 found that Mr. Trump helped set up a
sham corporation to disguise millions of dollars in gifts from his
parents, assisted his father in taking improper tax deductions and
undervalued his family's real estate holdings.

On Thursday, a White House spokesman directed questions about the status
of the I.R.S. audit to the Trump Organization, which did not return a
request for comment.

\hypertarget{the-rulings}{%
\subsection{The Rulings}\label{the-rulings}}

One ruling delivered a victory to Cyrus R. Vance, the Manhattan district
attorney and a Democrat, whose office sought eight years of business and
personal tax records in connection with a state grand jury investigation
into Mr. Trump's role in hush-money payments made to a pornographic film
star before the 2016 election.

The court ruled 7-2 that Mr. Trump was not immune from criminal
proceedings while in office but sent the case back to the lower courts,
where Mr. Trump's lawyers will presumably make new arguments that the
subpoena for financial records is improper.

Should Mr. Vance eventually win, as many legal experts expect, the tax
returns will go to a secret grand jury that will delay the release of
that information to the public, if it is released at all.

Another ruling concerned subpoenas made by several House committees to
gather Mr. Trump's financial information from his accountants and banks.
The court ruled that Congress had limited power to issue those subpoenas
and again sent the case to the lower courts.

Mr. Trump, for his part,
\href{https://twitter.com/realDonaldTrump/status/1281236214646034432}{attacked}
the rulings on Twitter: ``This is all a political prosecution,'' he
wrote.

``Courts in the past have given `broad deference'. BUT NOT ME!''
\href{https://twitter.com/realDonaldTrump/status/1281236412667432961}{he
added}.

Advertisement

\protect\hyperlink{after-bottom}{Continue reading the main story}

\hypertarget{site-index}{%
\subsection{Site Index}\label{site-index}}

\hypertarget{site-information-navigation}{%
\subsection{Site Information
Navigation}\label{site-information-navigation}}

\begin{itemize}
\tightlist
\item
  \href{https://help.nytimes.com/hc/en-us/articles/115014792127-Copyright-notice}{©~2020~The
  New York Times Company}
\end{itemize}

\begin{itemize}
\tightlist
\item
  \href{https://www.nytco.com/}{NYTCo}
\item
  \href{https://help.nytimes.com/hc/en-us/articles/115015385887-Contact-Us}{Contact
  Us}
\item
  \href{https://www.nytco.com/careers/}{Work with us}
\item
  \href{https://nytmediakit.com/}{Advertise}
\item
  \href{http://www.tbrandstudio.com/}{T Brand Studio}
\item
  \href{https://www.nytimes.com/privacy/cookie-policy\#how-do-i-manage-trackers}{Your
  Ad Choices}
\item
  \href{https://www.nytimes.com/privacy}{Privacy}
\item
  \href{https://help.nytimes.com/hc/en-us/articles/115014893428-Terms-of-service}{Terms
  of Service}
\item
  \href{https://help.nytimes.com/hc/en-us/articles/115014893968-Terms-of-sale}{Terms
  of Sale}
\item
  \href{https://spiderbites.nytimes.com}{Site Map}
\item
  \href{https://help.nytimes.com/hc/en-us}{Help}
\item
  \href{https://www.nytimes.com/subscription?campaignId=37WXW}{Subscriptions}
\end{itemize}
