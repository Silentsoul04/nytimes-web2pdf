Sections

SEARCH

\protect\hyperlink{site-content}{Skip to
content}\protect\hyperlink{site-index}{Skip to site index}

\href{https://www.nytimes.com/section/us}{U.S.}

\href{https://myaccount.nytimes.com/auth/login?response_type=cookie\&client_id=vi}{}

\href{https://www.nytimes.com/section/todayspaper}{Today's Paper}

\href{/section/us}{U.S.}\textbar{}Supreme Court Rules Trump Cannot Block
Release of Financial Records

\url{https://nyti.ms/2ZU0vkg}

\begin{itemize}
\item
\item
\item
\item
\item
\item
\end{itemize}

Advertisement

\protect\hyperlink{after-top}{Continue reading the main story}

Supported by

\protect\hyperlink{after-sponsor}{Continue reading the main story}

\hypertarget{supreme-court-rules-trump-cannot-block-release-of-financial-records}{%
\section{Supreme Court Rules Trump Cannot Block Release of Financial
Records}\label{supreme-court-rules-trump-cannot-block-release-of-financial-records}}

Two rulings clear the way for prosecutors in New York to seek President
Trump's financial records, but the justices stopped Congress for now.

\includegraphics{https://static01.nyt.com/images/2020/07/06/us/politics/00dc-scotus-trump-sub2/merlin_174147207_d7424b77-fbb2-4f66-b7ed-693833d252fa-articleLarge.jpg?quality=75\&auto=webp\&disable=upscale}

\href{https://www.nytimes.com/by/adam-liptak}{\includegraphics{https://static01.nyt.com/images/2018/07/13/multimedia/author-adam-liptak/author-adam-liptak-thumbLarge-v3.png}}

By \href{https://www.nytimes.com/by/adam-liptak}{Adam Liptak}

\begin{itemize}
\item
  Published July 9, 2020Updated July 10, 2020
\item
  \begin{itemize}
  \item
  \item
  \item
  \item
  \item
  \item
  \end{itemize}
\end{itemize}

WASHINGTON --- The
\href{https://www.nytimes.com/2020/07/10/podcasts/the-daily/supreme-court-trump-taxes.html?action=click\&module=Briefings\&pgtype=Homepage}{Supreme
Court} cleared the way on Thursday for prosecutors in New York to seek
President
\href{https://www.nytimes.com/2020/07/10/nyregion/donald-trump-taxes-cy-vance.html}{Trump's
financial records} in a stunning defeat for Mr. Trump and a major
statement on the scope and limits of presidential power.

The decision in the case said Mr. Trump had no absolute right to block
release of the papers and would take its place with landmark rulings
that required President Richard M. Nixon to turn over tapes of Oval
Office conversations and that forced President Bill Clinton to provide
evidence in a sexual harassment suit.

``No citizen, not even the president, is categorically above the common
duty to produce evidence when called upon in a criminal proceeding,''
Chief Justice John G. Roberts Jr. wrote for the majority. He added that
Mr. Trump might still raise objections to the scope and relevance of the
subpoena requesting the records.

In a separate decision, the court
\href{https://www.supremecourt.gov/opinions/19pdf/19-715_febh.pdf}{ruled}
that Congress could not, at least for now, see many of the same records.
It said the case should be returned to lower courts to examine whether
Congress should narrow the parameters of the information it sought,
meaning that the practical effect of the two decisions is that the
records will not be made public before the elections this fall.

The chief justice wrote the majority opinions in both cases, and both
were decided by 7-to-2 votes. The court's four-member liberal wing voted
with him, as did Mr. Trump's two appointees, Justices Neil M. Gorsuch
and Brett M. Kavanaugh.

Justices Clarence Thomas and Samuel A. Alito Jr. dissented in both
cases.

How the court ruled

The court ruled in two cases, 7 to 2, that President Trump can for now
block the release of his financial records to Congress but that
prosecutors in New York may see them.

Liberal Bloc

\includegraphics{https://static01.nyt.com/newsgraphics/2020/06/09/scotus-key-cases/d2b681e70b06d896ef285c8211c468c460b7c04c/Sotomayor-c.png}

Sotomayor

\includegraphics{https://static01.nyt.com/newsgraphics/2020/06/09/scotus-key-cases/d2b681e70b06d896ef285c8211c468c460b7c04c/Ginsburg-c.png}

Ginsburg

\includegraphics{https://static01.nyt.com/newsgraphics/2020/06/09/scotus-key-cases/d2b681e70b06d896ef285c8211c468c460b7c04c/Kagan-c.png}

Kagan

\includegraphics{https://static01.nyt.com/newsgraphics/2020/06/09/scotus-key-cases/d2b681e70b06d896ef285c8211c468c460b7c04c/Breyer-c.png}

Breyer

Conservative Bloc

\includegraphics{https://static01.nyt.com/newsgraphics/2020/06/09/scotus-key-cases/d2b681e70b06d896ef285c8211c468c460b7c04c/Roberts-c.png}

Roberts

\includegraphics{https://static01.nyt.com/newsgraphics/2020/06/09/scotus-key-cases/d2b681e70b06d896ef285c8211c468c460b7c04c/Kavanaugh-c.png}

Kavanaugh

\includegraphics{https://static01.nyt.com/newsgraphics/2020/06/09/scotus-key-cases/d2b681e70b06d896ef285c8211c468c460b7c04c/Alito-c.png}

Alito

\includegraphics{https://static01.nyt.com/newsgraphics/2020/06/09/scotus-key-cases/d2b681e70b06d896ef285c8211c468c460b7c04c/Gorsuch-c.png}

Gorsuch

\includegraphics{https://static01.nyt.com/newsgraphics/2020/06/09/scotus-key-cases/d2b681e70b06d896ef285c8211c468c460b7c04c/Thomas-c.png}

Thomas

\href{https://www.nytimes.com/interactive/2020/06/15/us/supreme-court-major-cases-2020.html}{See
how the court voted on other major cases this term}

Mr. Trump immediately
\href{https://twitter.com/realDonaldTrump/status/1281236214646034432}{attacked}
the outcome on Twitter. ``This is all a political prosecution,'' he
wrote. ``I won the Mueller Witch Hunt, and others, and now I have to
keep fighting in a politically corrupt New York. Not fair to this
Presidency or Administration!''

\includegraphics{https://static01.nyt.com/images/2020/07/09/autossell/09dc-trump-scotus2/09dc-trump-scotus2-videoSixteenByNine3000.png}

\includegraphics{https://static01.nyt.com/images/2017/01/29/podcasts/the-daily-album-art/the-daily-album-art-articleInline-v2.jpg?quality=75\&auto=webp\&disable=upscale}

\hypertarget{listen-to-the-daily-the-fate-of-trumps-financial-records}{%
\subsubsection{Listen to `The Daily': The Fate of Trump's Financial
Records}\label{listen-to-the-daily-the-fate-of-trumps-financial-records}}

In a stunning defeat for the president, the Supreme Court ruled against
his claims that he was immune to an examination of his finances.

transcript

Back to The Daily

bars

0:00/27:53

-27:53

transcript

\hypertarget{listen-to-the-daily-the-fate-of-trumps-financial-records-1}{%
\subsection{Listen to `The Daily': The Fate of Trump's Financial
Records}\label{listen-to-the-daily-the-fate-of-trumps-financial-records-1}}

\hypertarget{hosted-by-michael-barbaro-produced-by-eric-krupke-daniel-guillemette-annie-brown-luke-vander-ploeg-and-stella-tan-and-edited-by-mj-davis-lin-and-wendy-dorr}{%
\subsubsection{Hosted by Michael Barbaro, produced by Eric Krupke,
Daniel Guillemette, Annie Brown, Luke Vander Ploeg and Stella Tan, and
edited by M.J. Davis Lin and Wendy
Dorr}\label{hosted-by-michael-barbaro-produced-by-eric-krupke-daniel-guillemette-annie-brown-luke-vander-ploeg-and-stella-tan-and-edited-by-mj-davis-lin-and-wendy-dorr}}

\hypertarget{in-a-stunning-defeat-for-the-president-the-supreme-court-ruled-against-his-claims-that-he-was-immune-to-an-examination-of-his-finances}{%
\paragraph{In a stunning defeat for the president, the Supreme Court
ruled against his claims that he was immune to an examination of his
finances.}\label{in-a-stunning-defeat-for-the-president-the-supreme-court-ruled-against-his-claims-that-he-was-immune-to-an-examination-of-his-finances}}

\begin{itemize}
\item
  michael barbaro\\
  From The New York Times, I'm Michael Barbaro. This is ``The Daily.''

  Today, the Supreme Court issues major rulings on whether President
  Trump must disclose his financial records. David Enrich on the story
  behind the cases, and Adam Liptak on the meaning of the decisions.

  It's Friday July 10th.

  David, tell us the story behind these three cases, all related to the
  president's financial records that reached the Supreme Court.
\item
  david enrich\\
  So this story starts back in 2015 with a decision by presidential
  candidate Donald Trump not to do something. For decades, there's a
  tradition among major presidential candidates to release their tax
  returns so that the public can understand better the finances of the
  people they might elect. So Carter did this. Reagan, Bush Clinton,
  Bush number two did it. Obama did it. And hopefully I'm not forgetting
  anyone there. But Trump refused to do so, saying that he was under IRS
  audit.
\item
  archived recording (donald trump)\\
  Every year they audit me, audit me, audit me.
\end{itemize}

david enrich

And it just wasn't going to be possible.

\begin{itemize}
\tightlist
\item
  archived recording (donald trump)\\
  I will absolutely give my return. But I'm being audited now. So I
  can't do it until the audit is finished, obviously. And I think people
  would understand that.
\end{itemize}

david enrich

This immediately raises a slew of questions in the public and in the
media about what's going on in Trump's finances. He is someone, who
among major presidential candidates in recent decades, is probably
unique in terms of the number and gravity of the entanglements he has.
He's been running for decades as multifaceted business that has
operations and assets all over the world. And so his refusal to release
even basic financial information about himself, it raises a lot of
suspicion about whether this presidential candidate has something in his
finances that he is trying to hide.

\begin{itemize}
\item
  archived recording (name)\\
  Do you believe voters have a right to see your tax returns before they
  make a final decision
\item
  archived recording (donald trump)\\
  I don't think they do. But I do say this. When the audit ends, I'm
  going to present them. That should be before the election. I hope it's
  before the election.
\end{itemize}

david enrich

And then, of course, he gets elected without ever having released his
tax returns.

michael barbaro

Right, and remind us what begins to change all this.

david enrich

Well, two things happen. The first is that in the fall of 2018, the New
York Times drops this massive investigation into the Trump family's
finances. And one of the things it reveals is that Donald Trump and
others committed what looks like fraud in an effort to reduce the amount
that they owe in taxes. So that really intensifies the clamor to get a
better understanding of what's going on inside the Trump finances. The
second thing that happens is that in November of 2018, Democrats regain
control of the House of Representatives. And that means that two of the
most powerful congressional committees are now going to be controlled by
Democrats instead of Republicans. And in particular, they're going to be
controlled by two of Trump's loudest critics on Capitol Hill.

\begin{itemize}
\tightlist
\item
  archived recording (adam schiff)\\
  When he says we have no contact with the Russians, that's a lie. When
  his son says I have no contacts with Wikileaks, that's a lie. And
  unfortunately the list goes on and on.
\end{itemize}

david enrich

Adam Schiff will be the chairman of the House Intelligence Committee.

\begin{itemize}
\tightlist
\item
  archived recording (maxine waters)\\
  He refuses to turn over the tax returns. What does he have to hide?
\end{itemize}

david enrich

And Maxine Waters will be Chairman of The House Financial Services
Committee.

\begin{itemize}
\tightlist
\item
  archived recording (maxine waters)\\
  Has he been compromised in any way? Is there money laundering going
  on?
\end{itemize}

david enrich

And they make very clear very early on that one of their priorities in
this new Congress is going to be to launch major investigations of
Donald Trump and his financial dealings.

\begin{itemize}
\tightlist
\item
  archived recording (maxine waters)\\
  There is enough that we know about him to have legitimate suspicion.
  And we need to have documentation.
\end{itemize}

david enrich

And it becomes very clear very quickly that one of the weapons that the
congressional Democrats investigators plan to use is to subpoena as much
information as they can, not from Trump himself, but from the financial
institutions that he has done business with over the years.

michael barbaro

And why are they interested in those financial institutions?

david enrich

Well, it's much easier for them to pry information out of a third party
than it is to pry it out of Trump himself. Among other things, Trump has
shown over the years to not put a whole lot of stock and adhere into
things like subpoenas or court orders. He delays. He kicks the can down
the road as much as he can.

michael barbaro

Right.

david enrich

So there are two financial institutions in particular. One is Mazars,
which is an accounting firm that has prepared that Trump's taxes over
the years. And the other is Deutsche Bank, which is the lender that has
provided billions of dollars in loans and other services to Trump and
his family over the past two decades.

michael barbaro

And, David, what was the legal rationale for these congressional
committees to subpoena these documents? Because I imagine the subpoenas
had to relate to the work of these committees.

david enrich

It depends whom you ask.

\begin{itemize}
\tightlist
\item
  archived recording (name)\\
  This needs to be looked into because if there is a financial leverage
  that the Russians hold over Donald Trump ---
\end{itemize}

david enrich

In the case of the Intelligence Committee ---

\begin{itemize}
\tightlist
\item
  archived recording (name)\\
  That could warp our policy in ways that are not in our national
  interests.
\end{itemize}

david enrich

--- it was conducting an investigation into foreign interference in the
2016 presidential election. And so their argument was that if there is
evidence inside Deutsche Bank's vault ---

\begin{itemize}
\tightlist
\item
  archived recording (name)\\
  If the president is claiming no business dealings with Russia while
  he's trying to make this Trump Tower deal take place, has he also been
  concealing money laundering with Russians?
\end{itemize}

david enrich

--- of Trump being in business with Russians or owing money to Russians
or being involved in money laundering with Russians.

\begin{itemize}
\tightlist
\item
  archived recording (name)\\
  That's some serious business that we need to look into. We need to
  follow the money.
\end{itemize}

david enrich

That was very relevant to the committee's investigation of the Russian
interference in the 2016 election. With the House Financial Services
Committee, the legal rationale was a little bit murkier.

\begin{itemize}
\tightlist
\item
  archived recording (name)\\
  And of course, he has a reputation for everything from filing,
  bankruptcies, to cheating people.
\end{itemize}

david enrich

The argument that they made was that this was part of a broader
investigation by the committee into the soundness and propriety of the
financial system overall and the banking system overall.

\begin{itemize}
\tightlist
\item
  archived recording (name)\\
  --- to basically being involved with the Russians and other foreign
  entities. And so there is a lot there.
\end{itemize}

david enrich

And so their argument was that it was highly relevant to figure out if a
major financial institution was engaged in wrongdoing as it related to
the bank accounts of one of the most powerful people in the world. And
the reality is that these also had a real tinge of politics. Maxine
Waters and Adam Schiff are two of the most outspoken critics of Donald
Trump and his administration on Capitol Hill. And I think it is quite
hard to escape the reality that they were looking for dirt on the
president. And one of the dirtiest things to look at in their mind was
his finances because Trump had kept them such a tightly held secret. And
if there wasn't anything bad to see, Trump presumably would have
released his tax returns and been much more transparent about this from
the get go.

michael barbaro

And so what is the president's response to these congressional
subpoenas?

david enrich

The president's response ---

\begin{itemize}
\tightlist
\item
  archived recording (name)\\
  The president and his three oldest children are now suing two of the
  banks that helped build their family empire.
\end{itemize}

david enrich

--- is to sue.

\begin{itemize}
\tightlist
\item
  archived recording (name)\\
  The president's attorneys are trying to circumvent anything the banks
  would provide to Congress, saying, quote, ``the subpoenas were issued
  to harass President Donald J Trump, to rummage through every aspect of
  his personal finances ---
\end{itemize}

david enrich

So he goes to federal court with his family and files lawsuits against
Deutsche Bank, against Mazars, seeking to block them from complying with
these congressional subpoenas, to block them from handing over his
records to Congress. And the argument that the Trump family makes is
kind of twofold. One is that there's no legislative or legal purpose to
these subpoenas. This amounts to a political fishing expedition on
Capitol Hill. And it doesn't actually have any proper rationale. And the
second argument essentially boils down to a separation of powers claim,
which is that the president is not exposed or should not be exposed to
endless congressional meddling in his personal affairs. So after the
family filed these lawsuits, federal courts basically hit pause on
anyone actually enforcing these subpoenas. And so the cases go into the
federal court system, and gradually over the next year, work their way
up to the Supreme Court.

michael barbaro

OK, so that accounts for two of the three cases before the Supreme
Court, Trump versus Mazars, his accounting firm, Trump versus Deutsche
Bank, his bank. And I think that leaves us with one more case, right?

david enrich

That's right. And that case involves the Manhattan District Attorney Cy
Vance.

\begin{itemize}
\tightlist
\item
  archived recording (cy vance)\\
  Good morning. Can you hear me?
\end{itemize}

david enrich

So at the same time that these congressional subpoenas are being
litigated in federal court, Cy Vance's office opens up a criminal
investigation into Trump and his company. And this investigation is
focused on Stormy Daniels.

\begin{itemize}
\tightlist
\item
  archived recording (stormy daniels)\\
  Who am I? I am ---
\end{itemize}

david enrich

Remember her?

\begin{itemize}
\tightlist
\item
  archived recording (stormy daniels)\\
  I am Stormy.
\end{itemize}

david enrich

She is the porn star with whom Trump was allegedly having an affair. And
what Stormy Daniels and Michael Cohen, Trump's former personal lawyers,
say is that they basically paid Stormy Daniels to keep quiet. They
provided hush money payments to her in the prelude to the 2016 election.

michael barbaro

Right.

david enrich

And the issue that Cy Vance and others have been looking into here is
whether those payments constituted essentially campaign spending and
whether that was an undisclosed and therefore violation of federal
campaign finance laws. So Cy Vance starts investigating not only this
issue, but also just a whole series of other potential financial
improprieties inside of the Trump organization and Inside Trump's
personal finances as well. And so as part of that, Cy Vance's office
issues a subpoena to Mazars, the accounting firm, the same one that's
been subpoenaed by congressional Democrats and seeks eight years of
President Trump's tax returns. And the hope from Cy Vance's office, I
think, is that that will shed some light on the inner workings of not
just Trump's bank accounts but also his company. And it could reveal
some of the potentially damaging information.

michael barbaro

So unlike the congressional subpoenas, which have a legislative
rationale, this subpoena from the District Attorney of Manhattan, it has
a criminal investigatory rationale?

david enrich

Yeah, that's right. This is a criminal investigation. And this is a
criminal subpoena that goes to Mazars. And very soon after the subpoena
is issued, Trump files a lawsuit against the district attorney's office,
arguing that Cy Vance simply does not have the legal authority to
conduct a criminal investigation of the President. Trump's argument is
that the sitting president is immune from this kind of criminal
investigation. So this case, along with the two involving the
congressional subpoenas, they start working their way through the
federal court system and eventually land before the Supreme Court, which
is where we are now. And the stakes here with these cases are really,
really high. And really, it boils down to does the public have the right
to know about the finances of its elected leaders. And the Trump
family's argument is that the answer to that is no. The public doesn't
have the right to know. These congressional committees don't have the
right to know. And not even district attorneys or prosecutors have the
right to know, not even in a criminal investigation such as this one.
And on the flip side, the Democrats and the prosecutors, and quite a few
others, argue that the right to know about your elected leaders
financial interests is absolutely core to your ability to determine
whether your president's interests are properly aligned with the
public's interest. And the only way to really know that is to have a
clear understanding of what the president's financial interests are.

michael barbaro

Right, and those are the theoretically lofty questions at hand here. But
to flip it around, isn't another version of these three cases, whether
or not a president is entitled to some kind of legal privacy, especially
when people seeking his personal financial records have a documented
political agenda?

david enrich

Yeah, and there's no question that there are politics at play here. So I
think that's a fair point to make. And that's certainly the point that
the Trump family is making. I think in practical terms, though, putting
aside all of the issues about separation of powers and presidential
powers, practically there is a very important issue at play here, which
is there is a lot of very detailed, very sensitive information sitting
inside Mazars and sitting inside Deutsche Bank that the congressional
Democrats and Cy Vance have demanded. And if that information does get
handed over, it could really provide a startling inside glimpse inside
this highly secretive companies operations and the president's innermost
financial secrets.

michael barbaro

And of course, this brings us to Thursday and to the Supreme Court
finally making a decision in these three cases.

david enrich

Yeah, and this is a really big day. I mean, I've been waiting for this
day for as long as I can remember I feel like. And the stakes, as we
said, are really high. There's these huge important questions of
presidential power and the rights of Congress to investigate. And at the
same time, this is happening four months before the presidential
election. So this is really the last chance that the media and the
public will have to get a glimpse of Donald Trump's innermost financial
secrets before election day. So it's really, really hard to overstate
the importance of Thursday's decision.

michael barbaro

David, thank you very much.

david enrich

Thanks for having me.

michael barbaro

We'll be right back.

So Adam Liptak, our colleague David Enrich just walked us through the
origins of these three cases before the court. So how did the Supreme
Court decide?

adam liptak

Michael, it was a bit of a split decision. The court adopted a lot of
what the president had to say in one set of cases and rejected his main
argument in the other case.

michael barbaro

OK, I wonder if we can start with the bucket of cases related to
Congress, these congressional subpoenas of the president's financial
records. What did the court decide in those cases?

adam liptak

The court decided that this was a novel issue, implicating the
separation of powers. It was hesitant to do anything particularly bold.
And it thought that lower courts should take a fresh look at the
question of whether the congressional committees had given adequate
reasons for seeking vast troves of material and whether they had shown
that they truly had a legislative need for them.

michael barbaro

And specifically what did the justices have to say about the rationale
Congress cited that documents like these relate to their work as
Congress. It would inform an investigation into Russia. It would inform
an understanding of the tax system.

adam liptak

The court did seem to suggest that the House's argument was so broad
that it could be a fishing expedition for all kinds of things, including
personal emails that the house hadn't really offered a limiting
principle for why it's entitled to all the stuff it sought. And the
truth is it sought quite a lot of information.

michael barbaro

Right, and so to be clear, this means that Congress will not be getting
its hands on these records for a long time?

adam liptak

Well, it goes back to the lower courts. And litigation takes months and
months. And the election will come and go. And a new house of
representatives will be sworn in. So it may be that as a practical
matter, this shuts down this particular set of subpoenas.

michael barbaro

OK, so that brings us to the second bucket, this case involving a
criminal investigation from the Manhattan District Attorney and his
subpoena of these financial records in pursuit of that criminal
investigation. How did the justices rule in that case?

adam liptak

In that case, the justices squarely rejected President Trump's central
argument. He had said that he is categorically and absolutely immune
from being investigated by state prosecutors while he remains in office.
And chief Justice Roberts, who wrote the majority opinion in both cases,
was having none of that.

michael barbaro

And what was his legal logic for rejecting that claim by the president?

adam liptak

The chief justice said there was lots of presidents, going back to the
treason trial of Aaron Burr where there was a subpoena for Thomas
Jefferson's information, the Nixon tapes case where Nixon was made to
give up tape recordings of Oval Office conversations, the Clinton
against Jones case where Bill Clinton was made to provide information in
a sexual harassment case. The chief justice said the idea that
presidents are above the law in that sense was nonsense.

michael barbaro

So in the case of the district attorney, he will be getting the records
he requested from Trump's accounting firm?

adam liptak

Well, let's not get ahead of ourselves. It certainly clears the way for
the Manhattan district attorney to seek the records. The chief justice
said, and this is unexceptional, that as with anyone else who's
subpoenaed, he can interpose the usual objections. The subpoenas doesn't
seek relevant material. The subpoena is meant to harass me. The subpoena
is too burdensome. And that may take some time to litigate. I don't
think the president's chances in New York courts are particularly good.
But that part of the decision may, as with the congressional subpoenas,
kick this can down the road past the election and maybe much longer.

michael barbaro

So this victory for the district attorney of New York, this defeat for
President Trump, is not likely to result in those records becoming
public?

adam liptak

That's right. So there's a school of thought that President Trump was a
winner in both decisions. If his goal was to keep stuff secret for a
while longer, he was a winner. If his goal was to have the court adopt
his sweeping view of presidential power, he was a loser.

michael barbaro

So Adam, I want to square this decision on Thursday with what David
Enrich said were the stakes in this case. And the first question is, is
the public entitled to financial transparency from our president? And
what did the court ultimately say to that question?

adam liptak

The court said, maybe not in so many words, but the thrust of the
decisions was sure. If they're relevant to the president's official
conduct, the public is entitled to know what's going on. Or if it's
relevant to a criminal prosecution, or investigation of the president,
or of one of his associates, sure, the public is entitled to that. But
it can't be overbroad. It's got to be targeted, and limited, and
sensible.

michael barbaro

And the second question is whether the president is entitled to
protection from partisan attempts to dig into his financial records. So
what did the ruling ultimately mean for that?

adam liptak

The ruling said in the criminal context, the court expected judges to
make sure that politically motivated prosecutors weren't doing something
unsavory and to protect the president from harassment. And in the
legislative context, the court also said that if there is a good enough,
strong enough link between legislative purpose and the need for this
information, that, too, was acceptable that presidents could be made to
turn over information.

michael barbaro

Hm, I mean, what's interesting about your answers to this is that it
feels like the principle that has been laid out in these decisions
doesn't feel entirely tied to the result of them. For example, if the
court is saying the public is entitled to information about the
president and his finances, then why are we living in a world where they
won't see them?

adam liptak

Well, the court is not about timing. The court is not about tomorrow.
It's about principles and processes. And the legal system takes its
time. And I know we're all eager to see this information before the
election. And of course, it could well inform the ability of citizens in
a democracy to choose their leaders correctly. But that's not what the
court is focused on. The court is focused on getting it right. And that
will take time. And as a practical matter, the amount of time it will
take may make it much less politically salient. But this is not a
decision about President Trump. This is a decision about the presidency.
And the court has laid down some important markers on that.

michael barbaro

Adam, reporter to reporter here, do you think we're ever going to see
these documents, I mean, ever?

adam liptak

I think it's probably more likely they see the light of day through good
investigative journalism, including from The New York Times than through
these convoluted mechanisms of congressional subpoenas and grand jury
subpoenas. That's not to say it's impossible. But they're more likely to
see the light of day in any kind of timely fashion through reporting
than through litigation.

michael barbaro

Thank you, Adam.

adam liptak

Thank you, Michael.

michael barbaro

On Thursday, the Supreme Court issued a separate ruling, finding that
much of Eastern Oklahoma, including most of Tulsa, falls within an
Indian reservation, meaning that state authorities there can not
prosecute offenses involving Native Americans. At issue in the case was
whether Congress had officially eliminated a reservation when Oklahoma
became a state in 1907. The justices ruled that it had not. The
five-to-four decision is one of the most consequential legal victories
for Native Americans in decades.

We'll be right back.

Here's what else you need to know today. On Thursday, the World Health
Organization formally acknowledged that droplets carrying the
coronavirus known as aerosols can linger in the air within indoor spaces
for long periods, raising the risk of infections. The acknowledgment,
which scientists said was long overdue, could complicate efforts to
socially distance indoors. At the same time, the W.H.O. also
acknowledged that seemingly healthy people can transmit the virus even
when they don't have symptoms --- a finding that the W.H.O. had long
resisted. The concept of symptomless spread was raised at least three
months ago by a doctor in Germany, Camilla Roth, whose report on a
patient there was widely dismissed by scientists, including those at the
W.H.O..

``The Daily'' is made by Theo Balcomb, Andy Mills, Lisa Tobin, Rachel
Quester, Lynsea Garrison, Annie Brown, Clare Toeniskoetter, Paige
Cowett, Michael Simon Johnson, Brad Fisher, Larissa Anderson, Wendy
Dorr, Chris Wood, Jessica Cheung, Stella Tan, Alexandra Leigh Young,
Jonathan Wolfe, Lisa Chow, Eric Krupke, Marc Georges, Luke Vander Ploeg,
Adizah Eghan, Kelly Prime, Julia Longoria, Sindhu Gnanasambandan, M.J.
Davis Lin, Austin Mitchell, Sayre Quevedo, Nina Pathak, Dan Powell, Dave
Shaw, Sydney Harper, Daniel Guillemette, Hans Buetow, Robert Jimison,
Mike Benoist, Bianca Giaever, Asthaa Chaturvedi and Rachelle Bonja. Our
theme music is by Jim Brunberg and Ben Landsverk of Wonderly. Special
thanks to Sam Dolnick, Mikayla Bouchard, Lauren Jackson, Julia Simon,
Mahima Chablani, Nora Keller and Kenneth Chang.

That's it for ``The Daily.'' I'm Michael Barbaro. See you on Monday.

Chief Justice Roberts implicitly addressed that question in his opinion.
There were ``200 years of precedent establishing that presidents, and
their official communications, are subject to judicial process, even
when the president is under investigation,'' he said.

Justice Kavanaugh put it another way: ``In our system of government, as
this court has often stated, no one is above the law. That principle
applies, of course, to a president.''

Mr. Trump had asked the court to block both sets of subpoenas, which had
sought information from his accountants and bankers, not from Mr. Trump
himself. The firms have indicated that they would comply with the
courts' ultimate rulings.

Mr. Trump's lawyers had argued that he was immune from all criminal
proceedings and investigations so long as he remained in office and that
Congress was powerless to obtain his records because it had no
legislative need for them.

Jay Sekulow, a lawyer for Mr. Trump, portrayed the decisions as at least
a temporary victory.

``We are pleased that in the decisions issued today, the Supreme Court
has temporarily blocked both Congress and New York prosecutors from
obtaining the president's
\href{https://www.nytimes.com/2020/07/09/us/politics/trump-taxes.html}{tax}
records,'' he said in a statement. ``We will now proceed to raise
additional constitutional and legal issues in the lower courts.''

Mr. Sekulow was right that the Supreme Court left open the possibility
that Mr. Trump could make new objections to the New York subpoena. But
the majority rejected the argument the president had made in the Supreme
Court: that he was categorically immune from being having his records
subpoenaed by state prosecutors.

The majority also rejected the Justice Department's more limited
argument that state prosecutors must satisfy a demanding standard when
they seek information concerning a sitting president.

The New York case concerned a subpoena to Mr. Trump's accounting firm,
Mazars USA, from the office of the Manhattan district attorney,
\href{https://www.nytimes.com/2020/07/10/nyregion/donald-trump-taxes-cy-vance.html}{Cyrus
R. Vance Jr}., a Democrat. It sought eight years of business and
personal tax records in connection with an investigation of the role
that Mr. Trump and the Trump Organization played in hush-money payments
made in the run-up to the 2016 election.

Mr. Vance expressed satisfaction with the ruling. ``This is a tremendous
victory for our nation's system of justice and its founding principle
that no one --- not even a president --- is above the law,'' he said in
a statement. ``Our investigation, which was delayed for almost a year by
this lawsuit, will resume, guided as always by the grand jury's solemn
obligation to follow the law and the facts, wherever they may lead.''

Both Mr. Trump and his company reimbursed Michael D. Cohen, the
president's former lawyer and fixer, for
\href{https://www.nytimes.com/2018/05/03/us/politics/stormy-daniels-trump-payment.html}{payments
made to the pornographic film actress Stormy Daniels}, who claimed that
she had an affair with Mr. Trump.

Mr. Cohen was also involved in payments to
\href{https://www.nytimes.com/2019/12/05/us/fox-news-mcdougal.html}{Karen
McDougal}, a Playboy model who had also claimed she had a relationship
with Mr. Trump. The president has denied the relationships.

Mr. Trump sued to stop the accounting firm from turning over the
records, but lower courts ruled against him. In
\href{https://www.nytimes.com/2019/11/04/nyregion/trump-taxes-vance-appeal.html}{a
unanimous ruling}, the United States Court of Appeals for the Second
Circuit, in New York, said state prosecutors may require third parties
to turn over a sitting president's financial records for use in a grand
jury investigation.

The Supreme Court affirmed that ruling.

Chief Justice Roberts drew on history to demonstrate that sitting
presidents have been forced to provide information in criminal
proceedings, starting with a subpoena to Thomas Jefferson in Aaron
Burr's 1807 trial for treason. Chief Justice John Marshall ruled that
the president could be subpoenaed.

``In the two centuries since the Burr trial,'' Chief Justice Roberts
wrote, ``successive presidents have accepted Marshall's ruling that the
chief executive is subject to subpoena.''

And in the Nixon and Clinton cases, the chief justice wrote, the court
relied on Chief Justice Marshall's ruling.

Chief Justice Roberts wrote that it was of no moment that the earlier
subpoenas were federal, while the one seeking Mr. Trump's documents came
from a state prosecutor.

He rejected three of the president's arguments: that such subpoenas
would distract him from his duties, that he would be stigmatized, and
that he would be subject to harassment from elected prosecutors around
the nation.

There was little reason to think that a subpoena for records held by
third parties would impose a significant burden on a president, the
chief justice wrote. There is nothing ``inherently stigmatizing,'' he
added, about furnishing information relevant to a criminal
investigation.

As for harassment, he wrote that the court had rejected a similar
argument in the Clinton case, and that state and federal courts could
address bad faith investigations.

``Two hundred years ago, a great jurist of our court established that no
citizen, not even the president, is categorically above the common duty
to produce evidence when called upon in a criminal proceeding,'' Chief
Justice Roberts wrote. ``We reaffirm that principle today and hold that
the president is neither absolutely immune from state criminal subpoenas
seeking his private papers nor entitled to a heightened standard of
need.''

Justices Ruth Bader Ginsburg, Stephen G. Breyer, Sonia Sotomayor and
Elena Kagan joined the chief justice's majority opinion in the case,
Trump v. Vance, No. 19-635.

Justice Kavanaugh, joined by Justice Gorsuch, voted with the majority
but did not adopt its reasoning. Justice Kavanaugh agreed that the
president was not absolutely immune from having his records subpoenaed
but said lower courts should require prosecutors to show a
``demonstrated, specific need'' for the information they sought.

In dissent, Justice Thomas also rejected the absolute immunity argument
but said lower courts should give great weight to the possibility that
the subpoena would prove distracting.

``The demands on the president's time and the importance of his tasks
are extraordinary, and the office of the president cannot be delegated
to subordinates,'' Justice Thomas wrote. ``A subpoena imposes both
demands on the president's limited time and a mental burden, even when
the president is not directly engaged in complying. This understanding
of the presidency should guide courts in deciding whether to enforce a
subpoena for the president's documents.''

In a separate dissent, Justice Alito wrote that the majority had
stripped Mr. Trump of his most important defenses.

``The point is that we should not treat this subpoena like an ordinary
grand jury subpoena and should not relegate a president to the meager
defenses that are available when an ordinary grand jury subpoena is
challenged,'' Justice Alito wrote. ``But that, at bottom, is the effect
of the court's decision.''

The subpoenas from congressional committees sought information from Mr.
Trump's accountants and two financial institutions ---
\href{https://www.nytimes.com/2020/02/04/magazine/deutsche-bank-trump.html}{Deutsche
Bank} and Capital One --- about hush-money payments, about whether Mr.
Trump inflated and deflated descriptions of his assets on financial
statements and about an array of financial records related to the
president, his companies and his family.

Daniel Hunter, a spokesman for Deutsche Bank, said it would comply with
the courts' ultimate rulings.

``Deutsche Bank has demonstrated full respect for the U.S. legal process
and remained neutral throughout these proceedings,'' he said in a
statement. ``We will of course abide by a final decision by the
courts.''

Mazars USA issued a similar statement.

In the decision on congressional subpoenas, Trump v. Mazars USA, No.
19-715, Chief Justice Roberts stressed the novelty of the question
before the court. Earlier disputes between Congress and the president,
he wrote, had been worked out by accommodation rather than litigation.

He wrote that the House had acknowledged ``essentially no limits on the
congressional power to subpoena the president's personal records.''

Under the House's theory, he wrote, ``Congress could declare open season
on the president's information held by schools, archives, internet
service providers, email clients and financial institutions.''

Chief Justice Roberts said lower courts should assess whether the
records were truly needed by performing ``a careful analysis that takes
adequate account of the separation of powers principles at stake,
including both the significant legislative interests of Congress and the
unique position of the president.''

The six justices who voted with the chief justice in the New York case
joined his opinion on the congressional subpoenas.

In dissent, Justice Thomas wrote that ``it is readily apparent that the
committees have no constitutional authority to subpoena private,
nonofficial documents.'' The House can seek information from the
president only in connection with impeachment proceedings, he wrote.

In a separate dissent, Justice Alito wrote that he would insist that the
committees explain how the information sought relates to their
legislative responsibilities.

The two cases tested the independence of the Supreme Court, which is
dominated by Republican appointees, including two named by Mr. Trump. In
earlier Supreme Court cases in which presidents sought to avoid
providing evidence, the rulings did not break along partisan lines.

To the contrary, the court was unanimous in ruling against Presidents
Nixon and Clinton, with their appointees voting against the presidents
who had placed them on the court.
\href{https://www.law.cornell.edu/supremecourt/text/418/683}{The Nixon
case} led to his resignation in the face of mounting calls for his
impeachment.
\href{https://www.law.cornell.edu/supremecourt/text/520/681}{The Clinton
case} led to Mr. Clinton's impeachment, though he survived a Senate vote
on his removal.

Mr. Trump's appointees, Justices Gorsuch and Kavanaugh, also voted
against the president who had placed them on the court, allying
themselves with the chief justice and the court's four-member liberal
wing rather than its two most conservative members in the case on the
New York subpoena. In the process, they repudiated a president who
prizes loyalty and sent an unmistakable message about judicial
independence.

Advertisement

\protect\hyperlink{after-bottom}{Continue reading the main story}

\hypertarget{site-index}{%
\subsection{Site Index}\label{site-index}}

\hypertarget{site-information-navigation}{%
\subsection{Site Information
Navigation}\label{site-information-navigation}}

\begin{itemize}
\tightlist
\item
  \href{https://help.nytimes.com/hc/en-us/articles/115014792127-Copyright-notice}{©~2020~The
  New York Times Company}
\end{itemize}

\begin{itemize}
\tightlist
\item
  \href{https://www.nytco.com/}{NYTCo}
\item
  \href{https://help.nytimes.com/hc/en-us/articles/115015385887-Contact-Us}{Contact
  Us}
\item
  \href{https://www.nytco.com/careers/}{Work with us}
\item
  \href{https://nytmediakit.com/}{Advertise}
\item
  \href{http://www.tbrandstudio.com/}{T Brand Studio}
\item
  \href{https://www.nytimes.com/privacy/cookie-policy\#how-do-i-manage-trackers}{Your
  Ad Choices}
\item
  \href{https://www.nytimes.com/privacy}{Privacy}
\item
  \href{https://help.nytimes.com/hc/en-us/articles/115014893428-Terms-of-service}{Terms
  of Service}
\item
  \href{https://help.nytimes.com/hc/en-us/articles/115014893968-Terms-of-sale}{Terms
  of Sale}
\item
  \href{https://spiderbites.nytimes.com}{Site Map}
\item
  \href{https://help.nytimes.com/hc/en-us}{Help}
\item
  \href{https://www.nytimes.com/subscription?campaignId=37WXW}{Subscriptions}
\end{itemize}
