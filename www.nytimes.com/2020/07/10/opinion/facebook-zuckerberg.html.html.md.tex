Sections

SEARCH

\protect\hyperlink{site-content}{Skip to
content}\protect\hyperlink{site-index}{Skip to site index}

\href{https://myaccount.nytimes.com/auth/login?response_type=cookie\&client_id=vi}{}

\href{https://www.nytimes.com/section/todayspaper}{Today's Paper}

\href{/section/opinion}{Opinion}\textbar{}Zuckerberg Never Fails to
Disappoint

\href{https://nyti.ms/2ZiQoGk}{https://nyti.ms/2ZiQoGk}

\begin{itemize}
\item
\item
\item
\item
\item
\item
\end{itemize}

Advertisement

\protect\hyperlink{after-top}{Continue reading the main story}

\href{/section/opinion}{Opinion}

Supported by

\protect\hyperlink{after-sponsor}{Continue reading the main story}

\hypertarget{zuckerberg-never-fails-to-disappoint}{%
\section{Zuckerberg Never Fails to
Disappoint}\label{zuckerberg-never-fails-to-disappoint}}

He cannot hold on to such enormous power and avoid responsibility when
things get tough.

\includegraphics{https://static01.nyt.com/images/2018/08/02/opinion/02swisher/02swisher-thumbLarge.png}

By Kara Swisher

Ms. Swisher covers technology and is a contributing opinion writer.

\begin{itemize}
\item
  July 10, 2020
\item
  \begin{itemize}
  \item
  \item
  \item
  \item
  \item
  \item
  \end{itemize}
\end{itemize}

\includegraphics{https://static01.nyt.com/images/2020/07/10/opinion/10Swisher/10Swisher-articleLarge.jpg?quality=75\&auto=webp\&disable=upscale}

I had hoped to write about electric bikes this week, as part of my
ongoing effort
\href{https://www.nytimes.com/2020/07/09/opinion/ban-cars-manhattan-cities.html}{to
live without a car}. I was also considering weighing in on Uber's
\href{https://www.nytimes.com/2020/07/05/technology/uber-postmates-deal.html?searchResultPosition=1}{purchase
of the Postmates} food delivery service and its implications for the
already beleaguered restaurant industry. Or, perhaps the rumors that
Twitter is considering a subscription service. Or, and this is
interesting, the spectacular I.P.O. of the insurance industry disrupter
Lemonade.

But Facebook. Always Facebook.

Every week, it seems that the giant social network makes news, typically
of the kind that makes the company look bad, and typically by declining
to get out of the way of the history that is being made.

Just last week, after hundreds of advertisers joined a boycott of
Facebook, its chief executive, Mark Zuckerberg, cavalierly shrugged off
the effort by a group of concerned civil rights groups and told his
employees that, ``My guess is that all these advertisers will be back on
the platform soon enough.''

He said this as the company's second in charge, the chief operating
officer Sheryl Sandberg, was reportedly going around trying to persuade
those marketers to do just that.

The pair also tried and failed, via Zoom, to appease a passel of those
civil rights organizations --- including the Anti-Defamation League, the
National Association for the Advancement of Colored People, Color of
Change and others --- who are justifiably sick and tired of all talk and
no action from Facebook.

These groups have been behind the advertiser boycott in which hundreds
of companies, including the giant Unilever, have temporarily pushed the
pause button on marketing on the platform. They also brought
\href{https://www.nytimes.com/2020/07/07/technology/facebook-ad-boycott-civil-rights.html}{a
list of 10 demands}, which they have pushed for before to no avail.

It went about as well as an appearance by President Trump at a
mask-lovers convention.

Among the comments from the attendees: ``spin,'' ``very disappointing,''
``functionally flawed.'' Rashad Robinson of Color of Change, summing it
all up for The Times, said: ``They showed up to the meeting expecting an
A for attendance. Attending alone is not enough.''

Facebook actually got an F, too, this week in an independent report that
the company had commissioned about itself. The report decried Facebook's
decisions about how to protect its users from discriminatory content,
including in ads. It called Facebook's actions --- including a recent
decision by Mr. Zuckerberg not to pull down incendiary posts by Mr.
Trump --- a ``significant setback for civil rights.''

Well, that's pretty disastrous --- and utterly right. Sadly, the 89-page
report was not much of a surprise to most critics of the company, which
has been slow-walking its responsibility over hate speech and a range of
other toxic waste on its platform since, \emph{well,} always.

Mr. Zuckerberg has tried for a while to wrap himself up in the First
Amendment --- getting the whole point of the words of that amendment
wrong nearly every time --- and he has insisted that he does not want to
be an ``arbiter of truth.'' Yet he has set up the company in such a way
--- completely under his sway --- that suggests he has to be, in fact,
an arbiter of truth.

With Mr. Zuckerberg's overwhelming voting and corporate power, there is
no reason to have a board --- which is why board members with backbones,
like Reed Hastings and Ken Chenault, have left --- and every reason to
put the responsibility for cleaning up the mess squarely at Mr.
Zuckerberg's feet.

I keep trying to figure out a way to explain what is happening ---
actually, to explain why nothing is happening --- with a fresh metaphor.
Once, I compared Facebook to a city manager who treats the streets like
\href{https://www.hulu.com/series/the-purge-c5c9e867-73c7-4c1b-a8e0-cc0e0e4b95ed?\&cmp=7958\&utm_source=google\&utm_medium=cpc\&utm_campaign=BM\%20Search\%20TV\%20Shows\&utm_term=the\%20purge\%20tv\%20show\&ds_rl=1263136\&gclid=Cj0KCQjwo6D4BRDgARIsAA6uN1-t7rwlRySiUXwOBwRGdSTAg4H2592FcPMyeq6vI4524MJaFUIyvHoaAmcTEALw_wcB\&gclsrc=aw.ds}{The
Purge}. The Salesforce chief executive, Marc Benioff, likened Facebook
to a cigarette company. And still others have likened it to a chemical
company that carelessly spews noxious information into the river of
society.

This week, I finally settled on a simpler comparison: Think about
Facebook as a seller of meat products.

Most of the meat is produced by others, and some of the cuts are
delicious and uncontaminated. But tainted meat --- say, Trump steaks ---
also gets out the door in ever increasing amounts and without regulatory
oversight.

The argument from the head butcher is this: People should be free to eat
rotten hamburger, even if it wreaks havoc on their gastrointestinal
tract, and the seller of the meat should not be the one to tell them
which meat is good and which is bad (even though the butcher can tell in
most cases).

Basically, the message is that you should find the truth through
vomiting and --- so sorry --- maybe even death.

In this, Mr. Zuckerberg is serving up a rancid meal that he says he's
not comfortable cooking himself, even as his hands control every aspect
of the operation. Which is why I say to him and every executive at
Facebook: You cannot hold on to such enormous power and avoid
responsibility when things get tough.

``Many in the civil rights community have become disheartened,
frustrated and angry after years of engagement where they implored the
company to do more to advance equality and fight discrimination, while
also safeguarding free expression,'' the report said. It was written by
civil rights lawyers Laura W. Murphy and Megan Cacace, who also flagged
worries about the impact on the 2020 election.

``Facebook has made policy and enforcement choices that leave our
election exposed to interference by the president and others who seek to
use misinformation to sow confusion and suppress voting,'' they wrote,
making it clear that what Facebook does counts a lot.

Of course, Ms. Sandberg, who has increasingly played blocker for Mr.
Zuckerberg's very bad calls, posted about the report in a who-me style
that has now become a joke for those of us who follow the company.
Noting that the report was ``the beginning of the journey, not the end''
for Facebook, she concluded that ``what has become increasingly clear is
that we have a long way to go.''

Beginning? Increasingly clear? That's just --- as they say in the meat
biz --- tripe. The real sizzle is that Mr. Zuckerberg told employees to
wait out the advertiser boycott.

As always, Facebook is making us more nauseous than ever, as its own
report, its own advertisers, its own employees are telling us. But we're
all still hungry for leadership from Mr. Zuckerberg.

I know I am. So next week, I promise, I am going to make a sweet column
about Lemonade out of all these bitter lemons.

\emph{The Times is committed to publishing}
\href{https://www.nytimes.com/2019/01/31/opinion/letters/letters-to-editor-new-york-times-women.html}{\emph{a
diversity of letters}} \emph{to the editor. We'd like to hear what you
think about this or any of our articles. Here are some}
\href{https://help.nytimes.com/hc/en-us/articles/115014925288-How-to-submit-a-letter-to-the-editor}{\emph{tips}}\emph{.
And here's our email:}
\href{mailto:letters@nytimes.com}{\emph{letters@nytimes.com}}\emph{.}

\emph{Follow The New York Times Opinion section on}
\href{https://www.facebook.com/nytopinion}{\emph{Facebook}}\emph{,}
\href{http://twitter.com/NYTOpinion}{\emph{Twitter (@NYTopinion)}}
\emph{and}
\href{https://www.instagram.com/nytopinion/}{\emph{Instagram}}\emph{,
and sign up for the}
\href{http://www.nytimes.com/newsletters/opiniontoday/}{\emph{Opinion
Today newsletter}}\emph{.}

Advertisement

\protect\hyperlink{after-bottom}{Continue reading the main story}

\hypertarget{site-index}{%
\subsection{Site Index}\label{site-index}}

\hypertarget{site-information-navigation}{%
\subsection{Site Information
Navigation}\label{site-information-navigation}}

\begin{itemize}
\tightlist
\item
  \href{https://help.nytimes.com/hc/en-us/articles/115014792127-Copyright-notice}{©~2020~The
  New York Times Company}
\end{itemize}

\begin{itemize}
\tightlist
\item
  \href{https://www.nytco.com/}{NYTCo}
\item
  \href{https://help.nytimes.com/hc/en-us/articles/115015385887-Contact-Us}{Contact
  Us}
\item
  \href{https://www.nytco.com/careers/}{Work with us}
\item
  \href{https://nytmediakit.com/}{Advertise}
\item
  \href{http://www.tbrandstudio.com/}{T Brand Studio}
\item
  \href{https://www.nytimes.com/privacy/cookie-policy\#how-do-i-manage-trackers}{Your
  Ad Choices}
\item
  \href{https://www.nytimes.com/privacy}{Privacy}
\item
  \href{https://help.nytimes.com/hc/en-us/articles/115014893428-Terms-of-service}{Terms
  of Service}
\item
  \href{https://help.nytimes.com/hc/en-us/articles/115014893968-Terms-of-sale}{Terms
  of Sale}
\item
  \href{https://spiderbites.nytimes.com}{Site Map}
\item
  \href{https://help.nytimes.com/hc/en-us}{Help}
\item
  \href{https://www.nytimes.com/subscription?campaignId=37WXW}{Subscriptions}
\end{itemize}
