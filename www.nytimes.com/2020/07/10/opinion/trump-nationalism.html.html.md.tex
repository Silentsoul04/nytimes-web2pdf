Sections

SEARCH

\protect\hyperlink{site-content}{Skip to
content}\protect\hyperlink{site-index}{Skip to site index}

\href{https://myaccount.nytimes.com/auth/login?response_type=cookie\&client_id=vi}{}

\href{https://www.nytimes.com/section/todayspaper}{Today's Paper}

\href{/section/opinion}{Opinion}\textbar{}The Most Dangerous Phase of
Trump's Rule

\href{https://nyti.ms/2CrBtB2}{https://nyti.ms/2CrBtB2}

\begin{itemize}
\item
\item
\item
\item
\item
\item
\end{itemize}

Advertisement

\protect\hyperlink{after-top}{Continue reading the main story}

\href{/section/opinion}{Opinion}

Supported by

\protect\hyperlink{after-sponsor}{Continue reading the main story}

\hypertarget{the-most-dangerous-phase-of-trumps-rule}{%
\section{The Most Dangerous Phase of Trump's
Rule}\label{the-most-dangerous-phase-of-trumps-rule}}

His threat to democracy is nothing to laugh at.

\href{https://www.nytimes.com/by/roger-cohen}{\includegraphics{https://static01.nyt.com/images/2014/11/01/opinion/cohen-circular/cohen-circular-thumbLarge-v6.png}}

By \href{https://www.nytimes.com/by/roger-cohen}{Roger Cohen}

Opinion Columnist

\begin{itemize}
\item
  July 10, 2020
\item
  \begin{itemize}
  \item
  \item
  \item
  \item
  \item
  \item
  \end{itemize}
\end{itemize}

\includegraphics{https://static01.nyt.com/images/2020/07/10/opinion/10cohen1/merlin_174228885_aff9c534-1f2c-4bc3-9500-2965e17daf10-articleLarge.jpg?quality=75\&auto=webp\&disable=upscale}

PARIS --- Think of postwar European institutions as an elaborate shield
against fascism. The European Union diluting nationalist identity; the
welfare state cushioning the social divisions dictators may exploit;
NATO transforming the United States into a European power and the
ultimate protector of democracy against totalitarian ideologies.

This was Europe's collective response to its double suicide in the first
half of the 20th century. It was not just Germany that had to resurrect
itself from the rubble of ``zero hour'' in 1945, but the whole
continent. Europeans owed it to the myriad corpses beneath their every
step to build societies and institutions that were fascism-proof.

No wonder President Trump, whose dictatorial inclinations are as hard to
suppress as Dr. Strangelove's Nazi salute, hates these European
institutions so much. His itch is to undermine, or even destroy, them.
``\href{https://www.nytimes.com/2018/10/23/us/politics/nationalist-president-trump.html}{I'm
a nationalist,'' he once said}. Yes, he is --- flags, military flyovers,
walls, monuments and all, in exaltation of ``the greatest, most
exceptional and most virtuous nation in the history of the world,'' as
he put it on July 4.

Since arriving in France, I've heard a couple of French people describe
Trump as ``funny.'' For Europeans, the novelty of America's showman has
worn off. He's a loudmouth. He's a fool. These observations have emerged
from societies that have settled their painful scores with history and
found a middling security. The United States, however, has not. In fact,
I think Trump has just entered the most dangerous phase of his
presidency.

It is important to see Trump in historical context. The country he took
over had been through a seesawing quarter-century of trauma. First the
giddy all-powerful interlude after the disappearance of the Soviet
Union, with its temptations of hubris. Then the disorienting shock of
Sept. 11 that shattered the idea of America-the-inviolable and propelled
the nation into its wars without victory. Then the Great Recession with
its indelible lesson that, as Leonard Cohen put it, ``the poor stay
poor, the rich get rich.'' Then the fact, irrefutable with the rise of
China, of America's relative decline, a development Barack Obama, the
first Black president, opted to manage with cool realism.

All this provided the perfect context for ``a clumsy, lurching and
undiscriminating American nationalism that would boomerang upon
itself,'' as Jacob Heilbrunn described it in his
\href{https://www.lowyinstitute.org/the-interpreter/owen-harries-never-ideologue}{tribute}
to Owen Harries, the Australian foreign policy intellectual, who
predicted such a fate after Sept. 11.

Trump, masterful media manipulator, is the vehicle of that nationalism.
He exploited a pervasive sense of American humiliation. It was out
there, in search of a voice. Trump is not funny. He is fiendish.

Nationalism is not fascism but is a necessary component of it. Both seek
to change the present in the name of an illusory past in order to create
a future vague in all respects except its glory.

One of the core characteristics of fascism is nostalgia, a pining for a
culture of masculinity and monumentalism, evident in Hitler's Nazi Party
and the architecture it embraced for the 1,000-year Reich. Trump's
nostalgia is for some unidentified moment of American greatness, when
white male property owners ruled alone, the nation's global dominance
was unchallenged, women stayed home, and gender was not 360. By choosing
to speak at Mount Rushmore on the eve of Independence Day, Trump
attempted to inscribe his nationalism in a monumental narrative of
American heroism. It was straight from the autocratic playbook.

Another central characteristic of nationalism and fascism is their need
to define themselves against an enemy. Trump has chosen his: China,
designated as the culprit for the coronavirus debacle (and the scapegoat
behind which the president can hide his own equal responsibility); and
the ``angry mobs'' he alluded to at Mount Rushmore who constitute, Trump
said, a new ``far-left fascism that demands absolute allegiance.''

It is Trump who demands ``absolute allegiance'' --- look at his
trembling cabinet --- and whose nationalism is fascist-tinged. He has
turned an uprising against racial injustice after the killing of George
Floyd into a pretext to lash out against ``criminal'' mobs.

There have been excesses among the protests. It is always better to try
to contextualize history than excise it. Cancel culture is inimical to
free speech. But the overarching threat the United States faces in the
run-up to the Nov. 3 election is from Trump. The fascism in the air is
on the far right of the political spectrum. If Trump could identify
national humiliation as his ace in the hole in 2016, he can also seize
the potential of the coronavirus pandemic to muddy the waters and stir
pervasive fear.

Last month,
\href{https://twitter.com/realdonaldtrump/status/1275024974579982336}{Trump
tweeted}: ``RIGGED 2020 ELECTION: MILLIONS OF MAIL-IN BALLOTS WILL BE
PRINTED BY FOREIGN COUNTRIES, AND OTHERS. IT WILL BE THE SCANDAL OF OUR
TIMES!'' Of course, that foreign country would be China.

Trump is preparing the ground to contest any loss to Joe Biden and
remain president, aided, no doubt, by Attorney General William Barr's
Justice Department.

I know, it's unthinkable. So was the
\href{https://www.history.com/topics/germany/reichstag-fire}{Reichstag
fire}. Europeans, like Americans, should focus on just how unfunny Trump
is.

\emph{The Times is committed to publishing}
\href{https://www.nytimes.com/2019/01/31/opinion/letters/letters-to-editor-new-york-times-women.html}{\emph{a
diversity of letters}} \emph{to the editor. We'd like to hear what you
think about this or any of our articles. Here are some}
\href{https://help.nytimes.com/hc/en-us/articles/115014925288-How-to-submit-a-letter-to-the-editor}{\emph{tips}}\emph{.
And here's our email:}
\href{mailto:letters@nytimes.com}{\emph{letters@nytimes.com}}\emph{.}

\emph{Follow The New York Times Opinion section on}
\href{https://www.facebook.com/nytopinion}{\emph{Facebook}}\emph{,}
\href{http://twitter.com/NYTOpinion}{\emph{Twitter (@NYTopinion)}}
\emph{and}
\href{https://www.instagram.com/nytopinion/}{\emph{Instagram}}\emph{.}

Advertisement

\protect\hyperlink{after-bottom}{Continue reading the main story}

\hypertarget{site-index}{%
\subsection{Site Index}\label{site-index}}

\hypertarget{site-information-navigation}{%
\subsection{Site Information
Navigation}\label{site-information-navigation}}

\begin{itemize}
\tightlist
\item
  \href{https://help.nytimes.com/hc/en-us/articles/115014792127-Copyright-notice}{©~2020~The
  New York Times Company}
\end{itemize}

\begin{itemize}
\tightlist
\item
  \href{https://www.nytco.com/}{NYTCo}
\item
  \href{https://help.nytimes.com/hc/en-us/articles/115015385887-Contact-Us}{Contact
  Us}
\item
  \href{https://www.nytco.com/careers/}{Work with us}
\item
  \href{https://nytmediakit.com/}{Advertise}
\item
  \href{http://www.tbrandstudio.com/}{T Brand Studio}
\item
  \href{https://www.nytimes.com/privacy/cookie-policy\#how-do-i-manage-trackers}{Your
  Ad Choices}
\item
  \href{https://www.nytimes.com/privacy}{Privacy}
\item
  \href{https://help.nytimes.com/hc/en-us/articles/115014893428-Terms-of-service}{Terms
  of Service}
\item
  \href{https://help.nytimes.com/hc/en-us/articles/115014893968-Terms-of-sale}{Terms
  of Sale}
\item
  \href{https://spiderbites.nytimes.com}{Site Map}
\item
  \href{https://help.nytimes.com/hc/en-us}{Help}
\item
  \href{https://www.nytimes.com/subscription?campaignId=37WXW}{Subscriptions}
\end{itemize}
