Sections

SEARCH

\protect\hyperlink{site-content}{Skip to
content}\protect\hyperlink{site-index}{Skip to site index}

\href{https://www.nytimes.com/section/us}{U.S.}

\href{https://myaccount.nytimes.com/auth/login?response_type=cookie\&client_id=vi}{}

\href{https://www.nytimes.com/section/todayspaper}{Today's Paper}

\href{/section/us}{U.S.}\textbar{}An Unexpected Struggle for Trump:
Defining an Elusive Biden

\url{https://nyti.ms/2W8zTuH}

\begin{itemize}
\item
\item
\item
\item
\item
\item
\end{itemize}

\begin{itemize}
\item
  \href{https://www.nytimes.com/2020/08/04/us/elections/primary-election-michigan-arizona-kansas.html?action=click\&pgtype=Article\&state=default\&region=TOP_BANNER\&context=storylines_menu}{Election
  Updates}
\item
  \href{https://www.nytimes.com/article/biden-vice-president-2020.html?action=click\&pgtype=Article\&state=default\&region=TOP_BANNER\&context=storylines_menu}{Biden's
  V.P. Search}
\item
  \href{https://www.nytimes.com/interactive/2020/07/24/us/politics/trump-biden-campaign-donors.html?action=click\&pgtype=Article\&state=default\&region=TOP_BANNER\&context=storylines_menu}{Map
  of Donations}
\item
  \href{https://www.nytimes.com/interactive/2020/us/elections/delegate-count-primary-results.html?action=click\&pgtype=Article\&state=default\&region=TOP_BANNER\&context=storylines_menu}{Delegate
  Count}
\item
  \href{https://www.nytimes.com/interactive/2019/us/politics/2020-presidential-candidates.html?action=click\&pgtype=Article\&state=default\&region=TOP_BANNER\&context=storylines_menu}{The
  Candidates}
\item
  \href{https://www.nytimes.com/newsletters/politics?action=click\&pgtype=Article\&state=default\&region=TOP_BANNER\&context=storylines_menu}{Politics
  Newsletter}
\end{itemize}

Advertisement

\protect\hyperlink{after-top}{Continue reading the main story}

Supported by

\protect\hyperlink{after-sponsor}{Continue reading the main story}

\hypertarget{an-unexpected-struggle-for-trump-defining-an-elusive-biden}{%
\section{An Unexpected Struggle for Trump: Defining an Elusive
Biden}\label{an-unexpected-struggle-for-trump-defining-an-elusive-biden}}

With only six weeks to the Republican National Convention, President
Trump has yet to find a framework for attacking his opponent.

\includegraphics{https://static01.nyt.com/images/2020/07/10/us/politics/10biden-trump1/10biden-trump1-articleLarge.jpg?quality=75\&auto=webp\&disable=upscale}

\href{https://www.nytimes.com/by/adam-nagourney}{\includegraphics{https://static01.nyt.com/images/2018/02/20/multimedia/author-adam-nagourney/author-adam-nagourney-thumbLarge-v3.png}}

By \href{https://www.nytimes.com/by/adam-nagourney}{Adam Nagourney}

\begin{itemize}
\item
  Published July 10, 2020Updated July 17, 2020
\item
  \begin{itemize}
  \item
  \item
  \item
  \item
  \item
  \item
  \end{itemize}
\end{itemize}

President
\href{https://www.nytimes.com/2020/07/17/us/trump-biden-2020-election.html}{Trump}
won the White House in no small part by seizing on Hillary Clinton's
missteps and using them to turn many voters against her. But after three
unsteady months, and with the Republican convention six weeks away, Mr.
Trump is struggling to define Joseph R.
\href{https://www.nytimes.com/2020/07/17/us/trump-biden-2020-election.html}{Biden}
Jr. to similarly devastating effect, a critical task at this stage of a
presidential race.

By a combination of design and circumstance, Mr. Biden, the presumptive
Democratic nominee, has managed so far to deny Mr. Trump the sort of
damaging offhand remarks, campaign clashes and clumsy encounters with
voters that he used as weapons against Mrs. Clinton in the last general
election, as well as his Republican opponents in the 2016 primary.

This is partly because Mr. Biden has run such a
\href{https://www.nytimes.com/2020/06/25/us/politics/biden-speech-trump-coronavirus.html}{low-profile
campaign during the pandemic}. He has had few public appearances and
news conferences, which can provide the unscripted moments opponents can
use to shape the public's perception of a candidate.

But there are other obstacles for Mr. Trump that have become clear since
Mr. Biden
\href{https://www.nytimes.com/2020/04/08/us/politics/bernie-sanders-drops-out.html}{effectively
won} his party's nomination in April. Mr. Biden, the former vice
president,
\href{https://www.nytimes.com/2020/06/24/us/politics/trump-biden-poll-nyt-upshot-siena-college.html}{is
viewed more favorably by voters} than Mrs. Clinton was in 2016. He is a
moderate Democrat who lacks a history of harsh partisanship or scandal.
And he has long appealed to white working-class voters, who are part of
Mr. Trump's base.

``It is going to be more difficult for the Trump campaign to go after a
man who really is a centrist, has dealings with people across the aisle
and knows how to talk to people who disagree with him,'' said Priscilla
Southwell, a professor emerita of political science at the University of
Oregon. ``And 2020 is a different kind of year. Donald Trump can appeal
to his core by being negative, but it's such a difficult time for
everybody. I don't think negativity is going to sell as well as it used
to.''

Defining an opponent --- putting them on the defensive with caricature
--- is a crucial and proven tactic for candidates in competitive races.
There is a graveyard of failed contenders --- names like Kennedy,
McGovern, Romney, Gore and Hillary Clinton --- who found themselves
branded by an opponent in portrayals, often unfair, that ricocheted
across the political playing field and the media.

Mr. Trump had been adept at this. But the kind of attacks that seemed so
effective when he was a new-to-politics outsider in 2016 also appear to
have less resonance coming from inside the White House. Four years of
tweets by Mr. Trump have numbed many voters.

``It's almost self-defeating,'' said Ron Christie, a Republican who was
a senior adviser to President George W. Bush and Vice President Dick
Cheney. ``People are exhausted. The president, with every tweet, every
insult, will move himself out of favor with the demographic that he
needs the most, which is the independent.''

Mr. Trump does have some avenues to use against Mr. Biden before voter
attitudes begin to harden. He has sought to tie Mr. Biden to the
political unrest that has swept the country since the killing of George
Floyd in Minneapolis on May 25 by the police. And Mr. Trump has tried to
portray his opponent as senile, ``sleepy,'' corrupt and an ally of
China, but none of those lines of attack has resonated with the public,
at least up to now.

His aides have signaled that Mr. Trump, incumbent or not, would run as
an outsider against Mr. Biden --- who has been a fixture in Washington
since he was elected to the Senate in 1972 --- the way he had against
Mrs. Clinton. Mr. Biden's long history of votes in the Senate, as well
as his eight years as an active vice president under President Barack
Obama, could give Mr. Trump plenty of material.

\hypertarget{latest-updates-2020-election}{%
\section{\texorpdfstring{\href{https://www.nytimes.com/2020/08/04/us/elections/primary-election-michigan-arizona-kansas.html?action=click\&pgtype=Article\&state=default\&region=MAIN_CONTENT_1\&context=storylines_live_updates}{Latest
Updates: 2020
Election}}{Latest Updates: 2020 Election}}\label{latest-updates-2020-election}}

Updated 2020-08-04T19:23:30.305Z

\begin{itemize}
\tightlist
\item
  \href{https://www.nytimes.com/2020/08/04/us/elections/primary-election-michigan-arizona-kansas.html?action=click\&pgtype=Article\&state=default\&region=MAIN_CONTENT_1\&context=storylines_live_updates\#link-3924dd44}{Two
  G.O.P. Senate primaries offer --- what else? --- a test of loyalty to
  Trump.}
\item
  \href{https://www.nytimes.com/2020/08/04/us/elections/primary-election-michigan-arizona-kansas.html?action=click\&pgtype=Article\&state=default\&region=MAIN_CONTENT_1\&context=storylines_live_updates\#link-32b39e33}{President
  Trump is suddenly a big supporter of mail-in voting --- in Florida.}
\item
  \href{https://www.nytimes.com/2020/08/04/us/elections/primary-election-michigan-arizona-kansas.html?action=click\&pgtype=Article\&state=default\&region=MAIN_CONTENT_1\&context=storylines_live_updates\#link-6d019753}{Election
  experts warn Congress about widespread disenfranchisement of voters of
  color in November.}
\end{itemize}

\href{https://www.nytimes.com/2020/08/04/us/elections/primary-election-michigan-arizona-kansas.html?action=click\&pgtype=Article\&state=default\&region=MAIN_CONTENT_1\&context=storylines_live_updates}{See
more updates}

And if Mr. Biden continues to escape definition, Mr. Trump is likely to
turn to Mr. Biden's running mate. Going after the vice-presidential
candidate would be an unusual but not unprecedented strategy, and might
have some resonance in this election given Mr. Biden's age; he is 77.
(Mr. Trump is 74.)

Mr. Trump's campaign had calculated that Mr. Biden, given his long
history and the stumbles in the early days of his primary, would be an
easier opponent to caricature.

But now time is running short. A series of national polls has shown Mr.
Trump trailing Mr. Biden, often by double digits. Even more alarming for
the president, he is trailing Mr. Biden in battleground states that he
won in 2016 and are likely critical to any re-election plan ---
including Wisconsin and Pennsylvania. Even states like Georgia, which
once seemed clearly in Mr. Trump's column, now
\href{https://cookpolitical.com/analysis/national/national-politics/new-july-2020-electoral-college-ratings}{appear
competitive}.

``Trump has much less time to pile up negatives on Biden,'' said Nelson
Warfield, a Republican consultant who served as press secretary for Bob
Dole's presidential campaign in 1996. ``I made my first negative ad
starring Hillary Clinton in 1992 and I kept doing ads criticizing her
across the next 24 years. And I was by no means alone. Republicans have
months to do to Biden what Republicans had over two decades to do to
Hillary.''

\includegraphics{https://static01.nyt.com/images/2020/07/10/us/politics/10biden-trump2/10biden-trump2-articleLarge.jpg?quality=75\&auto=webp\&disable=upscale}

Mr. Trump would certainly seem to have a few advantages here to make his
case. He has the platform of the White House and his Twitter account.
Until two months ago, he enjoyed a huge
\href{https://www.nytimes.com/2020/07/01/us/politics/trump-fundraising-2020.html}{financial
advantage} over Mr. Biden. And Mr. Trump, unlike Mr. Biden, never had to
worry this year about a primary or uniting his party behind him.

But to the frustration of Republicans, his attempts to define Mr. Biden
have seemed fitful. He barely mentioned Mr. Biden during two of his
campaign's highest-profile moments in months: his speech in front of
Mount Rushmore last week, followed by his July 4 address from the White
House South Lawn.

And when Mr. Biden has made mistakes, Mr. Trump's campaign has struggled
to turn them to its advantage. When Mr. Biden said that any
African-American voter who considered supporting Mr. Trump
\href{https://www.nytimes.com/2020/05/22/us/politics/joe-biden-black-breakfast-club.html}{``ain't
Black,''} the Trump campaign roared into action, but the fallout lasted
only a day, particularly after Mr. Biden apologized.

While Mr. Trump has filled the space on the stage that Mr. Biden has
left open, he has been the one to make campaign missteps that provided
fodder for Democrats, notably when he said at a rally that given the
increase in positive tests for Covid-19, ``I said to my people, `Slow
the testing down, please.'''

Mr. Trump has the obstacle of familiarity in trying to draw attention to
his attacks. And his credibility has suffered over these past four
years, which might make him an imperfect messenger to go after Mr.
Biden: 67 percent of voters in a New York Times/Siena College poll last
month said they think Mr. Trump promotes falsehoods or conspiracies very
or somewhat often.

``The truth is people have heard him mocking and demonizing for four
years now,'' said Mark Mellman, a Democratic pollster. ``They are
somewhat inured to it, they are sick and tired of it. And by being home,
Biden has given him less to shoot at.''

The period before the conventions is typically the time when candidates
make the kind of mistakes their opponents can use to set the frame for
the fall campaign.

Mr. Trump's campaign pounced when Mrs. Clinton was taped saying half of
Mr. Trump's supporters were a ``basket of deplorables.'' Mitt Romney,
the Republican presidential candidate in 2012, was caught at a private
fund-raiser saying 47 percent of Americans were ``people who pay no
income tax'' and were ``dependent upon government.''

John F. Kerry, as the Democratic presidential candidate in 2004, was
caught on video windsurfing back and forth across the Nantucket Sound.
That provided the perfect image for an
\href{https://www.youtube.com/watch?v=2QpS2Am51Wo}{attack advertisement}
by President George W. Bush, which portrayed him as a flip-flopper, his
policy positions changing with ``whichever way the wind blows,'' as the
announcer put it. (In the process, the advertisement underlined the Bush
campaign's effort to portray Mr. Kerry as elite.)

In Mrs. Clinton, Mr. Trump had an unpopular opponent easy to demonize.
In the final month before Election Day, 54 percent of respondents had an
unfavorable view of her, according to a
\href{https://www.nytimes.com/interactive/2016/11/03/us/politics/poll-times-cbs-news.html}{New
York Times/CBS News Poll}. Mr. Trump was in similar straits; he was
viewed unfavorably by 56 percent of voters, the same percentage he holds
today, according to the recent
\href{https://www.nytimes.com/2020/06/24/us/politics/trump-biden-poll-nyt-upshot-siena-college.html}{Times/Siena
poll}.

Mr. Biden, on the other hand, was viewed unfavorably by just 42 percent
of voters in the same
poll\href{-https:/www.nytimes.com/2020/06/24/us/politics/trump-biden-poll-nyt-upshot-siena-college.html}{;}
52 percent viewed him favorably.

``That was what was historic about that race,'' said Joel Benenson, who
was chief strategist for Mrs. Clinton. ``You don't have that now. You
have Trump with that high unfavorable rating. But Biden doesn't have
that.''

``And he's the president now,'' Mr. Benenson said. ``He ran as a
bomb-throwing presidential candidate. He could throw it at any
candidate, they were the establishment. He owns it now.''

Ms. Southwell said that Mr. Biden presented a different kind of target
than Mrs. Clinton. ``It's not that he hasn't been an insider,'' she
said. ``He's had a different kind of upbringing and a different kind of
career than Hillary Clinton.''

In many ways, Mr. Biden has turned the tables on Mr. Trump: He is
running as if he were the incumbent, while Mr. Trump is acting like the
challenger. The question is whether he will be able to maintain that
posture through the election.

``Biden's basement Rose Garden strategy has enabled him to play the role
of a generic Democratic candidate, without the microscopic scrutiny that
he would otherwise have been subjected to,'' said Neil Newhouse, a
Republican pollster who worked for Mr. Romney. ``Joe Biden is not
Hillary Clinton, in that he doesn't have the built-in negatives that
Hillary embodied. So, while we can absolutely still define Biden, we
have significantly less time to do so.''

\hypertarget{our-2020-election-guide}{%
\section{Our 2020 Election Guide}\label{our-2020-election-guide}}

Updated Aug. 4, 2020

\begin{itemize}
\item
  \begin{center}\rule{0.5\linewidth}{\linethickness}\end{center}

  \hypertarget{the-latest}{%
  \subsection{The Latest}\label{the-latest}}

  \begin{itemize}
  \tightlist
  \item
    Five states are holding primary elections Tuesday, with voters in
    Arizona, Kansas, Michigan, Missouri and Washington State choosing
    nominees for Congress and local offices.
    \href{https://www.nytimes.com/2020/08/04/us/elections/primary-election-michigan-arizona-kansas.html?action=click\&pgtype=Article\&state=default\&region=BELOW_MAIN_CONTENT\&context=storylines_guide}{Follow
    live election updates here.}
  \end{itemize}
\item
  \begin{center}\rule{0.5\linewidth}{\linethickness}\end{center}

  \hypertarget{bidens-vp-search}{%
  \subsection{Biden's V.P. Search}\label{bidens-vp-search}}

  \begin{itemize}
  \tightlist
  \item
    \href{https://www.nytimes.com/article/biden-vice-president-2020.html?action=click\&pgtype=Article\&state=default\&region=BELOW_MAIN_CONTENT\&context=storylines_guide}{Here
    are 13 women} who have been under consideration to be Joe Biden's
    running mate, and why each might be chosen --- and might not be.
  \end{itemize}
\item
  \begin{center}\rule{0.5\linewidth}{\linethickness}\end{center}

  \hypertarget{keep-up-with-our-coverage}{%
  \subsection{Keep Up With Our
  Coverage}\label{keep-up-with-our-coverage}}

  \begin{itemize}
  \tightlist
  \item
    Get an
    \href{https://www.nytimes.com/newsletters/politics?action=click\&pgtype=Article\&state=default\&region=BELOW_MAIN_CONTENT\&context=storylines_guide}{email}
    recapping the day's news
  \end{itemize}

  \begin{itemize}
  \tightlist
  \item
    Download our mobile app on
    \href{https://apps.apple.com/us/app/nytimes/id284862083?ls=1\&mat_click_id=5c79ae7455014fd1bd66b5610c05b8f2-20191112-16948\&referrer=mat_click_id\%3D5c79ae7455014fd1bd66b5610c05b8f2-20191112-16948\%26link_click_id\%3D722930677036718082}{iOS}
    and
    \href{http://a.localytics.com/android?id=com.nytimes.android\&referrer=utm_source\%3Dother_nyt_mobile_web\%26utm_medium\%3DWeb\%2520page\%26utm_term\%3DGeneral\%2520Mobile\%2520Page\%26utm_campaign\%3DNYT\%2520Mobile\%2520General\%2520Page}{Android}
    and turn on Breaking News and Politics alerts
  \end{itemize}
\end{itemize}

Advertisement

\protect\hyperlink{after-bottom}{Continue reading the main story}

\hypertarget{site-index}{%
\subsection{Site Index}\label{site-index}}

\hypertarget{site-information-navigation}{%
\subsection{Site Information
Navigation}\label{site-information-navigation}}

\begin{itemize}
\tightlist
\item
  \href{https://help.nytimes.com/hc/en-us/articles/115014792127-Copyright-notice}{©~2020~The
  New York Times Company}
\end{itemize}

\begin{itemize}
\tightlist
\item
  \href{https://www.nytco.com/}{NYTCo}
\item
  \href{https://help.nytimes.com/hc/en-us/articles/115015385887-Contact-Us}{Contact
  Us}
\item
  \href{https://www.nytco.com/careers/}{Work with us}
\item
  \href{https://nytmediakit.com/}{Advertise}
\item
  \href{http://www.tbrandstudio.com/}{T Brand Studio}
\item
  \href{https://www.nytimes.com/privacy/cookie-policy\#how-do-i-manage-trackers}{Your
  Ad Choices}
\item
  \href{https://www.nytimes.com/privacy}{Privacy}
\item
  \href{https://help.nytimes.com/hc/en-us/articles/115014893428-Terms-of-service}{Terms
  of Service}
\item
  \href{https://help.nytimes.com/hc/en-us/articles/115014893968-Terms-of-sale}{Terms
  of Sale}
\item
  \href{https://spiderbites.nytimes.com}{Site Map}
\item
  \href{https://help.nytimes.com/hc/en-us}{Help}
\item
  \href{https://www.nytimes.com/subscription?campaignId=37WXW}{Subscriptions}
\end{itemize}
