Sections

SEARCH

\protect\hyperlink{site-content}{Skip to
content}\protect\hyperlink{site-index}{Skip to site index}

\href{https://www.nytimes.com/section/politics}{Politics}

\href{https://myaccount.nytimes.com/auth/login?response_type=cookie\&client_id=vi}{}

\href{https://www.nytimes.com/section/todayspaper}{Today's Paper}

\href{/section/politics}{Politics}\textbar{}Trump Is Selling White
Grievance. The Suburbs Aren't Buying It.

\url{https://nyti.ms/3gH418g}

\begin{itemize}
\item
\item
\item
\item
\item
\item
\end{itemize}

\begin{itemize}
\item
  \href{https://www.nytimes.com/2020/07/31/us/elections/biden-vs-trump.html?action=click\&pgtype=Article\&state=default\&region=TOP_BANNER\&context=storylines_menu}{Election
  Updates}
\item
  \href{https://www.nytimes.com/article/biden-vice-president-2020.html?action=click\&pgtype=Article\&state=default\&region=TOP_BANNER\&context=storylines_menu}{Biden's
  V.P. Search}
\item
  \href{https://www.nytimes.com/interactive/2020/07/24/us/politics/trump-biden-campaign-donors.html?action=click\&pgtype=Article\&state=default\&region=TOP_BANNER\&context=storylines_menu}{Map
  of Donations}
\item
  \href{https://www.nytimes.com/interactive/2020/us/elections/delegate-count-primary-results.html?action=click\&pgtype=Article\&state=default\&region=TOP_BANNER\&context=storylines_menu}{Delegate
  Count}
\item
  \href{https://www.nytimes.com/interactive/2019/us/politics/2020-presidential-candidates.html?action=click\&pgtype=Article\&state=default\&region=TOP_BANNER\&context=storylines_menu}{The
  Candidates}
\item
  \href{https://www.nytimes.com/newsletters/politics?action=click\&pgtype=Article\&state=default\&region=TOP_BANNER\&context=storylines_menu}{Politics
  Newsletter}
\end{itemize}

Advertisement

\protect\hyperlink{after-top}{Continue reading the main story}

Supported by

\protect\hyperlink{after-sponsor}{Continue reading the main story}

\hypertarget{trump-is-selling-white-grievance-the-suburbs-arent-buying-it}{%
\section{Trump Is Selling White Grievance. The Suburbs Aren't Buying
It.}\label{trump-is-selling-white-grievance-the-suburbs-arent-buying-it}}

As the president casts himself as a bulwark against ``angry mobs,''
there are signs that he is alienating voters in bedroom communities who
view him as a deeply flawed messenger on issues of race.

\includegraphics{https://static01.nyt.com/images/2020/07/10/us/politics/10trump-suburbs1/merlin_174387516_b7844842-eda1-48da-8c24-4674dcba4f0d-articleLarge.jpg?quality=75\&auto=webp\&disable=upscale}

\href{https://www.nytimes.com/by/katie-glueck}{\includegraphics{https://static01.nyt.com/images/2020/01/29/reader-center/author-katie-glueck/author-katie-glueck-thumbLarge.png}}

By \href{https://www.nytimes.com/by/katie-glueck}{Katie Glueck}

\begin{itemize}
\item
  July 10, 2020
\item
  \begin{itemize}
  \item
  \item
  \item
  \item
  \item
  \item
  \end{itemize}
\end{itemize}

CORNELIUS, N.C. --- On a humid Wednesday morning in this leafy lakeside
suburb of Charlotte, American flags fluttered from porches along Main
Street, traffic was slow, and the occasional resident ambled out for a
walk.

There was only one visible sign of the anger and anxiety that have
coursed through this community and so many others across the nation in
recent weeks: ``Racist,'' read the faded black graffiti at the base of a
Confederate memorial, the kind of statue President Trump has
\href{https://www.nytimes.com/2020/06/26/us/politics/trump-monuments-executive-order.html}{vowed
to preserve} amid a national discussion of racism in America.

Down the street, as she loaded groceries into her car, Elizabeth Stewart
vented her frustrations about Mr. Trump's incendiary approach.

``He's trying to appeal to a base that's gotten more and more narrow,''
said Ms. Stewart of Davidson, N.C., a small-business owner who supported
Mitt Romney in the 2012 presidential race and Hillary Clinton in 2016
and will support Joseph R. Biden Jr. this year. ``It's just extremely
divisive.''

From North Carolina to Pennsylvania to Arizona, interviews this week
with more than two dozen suburban voters in critical swing states
revealed abhorrence for Mr. Trump's
\href{https://www.nytimes.com/2020/07/06/us/politics/trump-bubba-wallace-nascar.html}{growing
efforts to fuel white resentment} with inflammatory rhetoric on race and
cultural heritage. The discomfort was palpable even among voters who
also dislike the recent toppling of Confederate statues or who say they
agree with some of Mr. Trump's policies.

As the president increasingly stakes his candidacy on a message of ``law
and order,'' casting himself as a bulwark against
``\href{https://www.nytimes.com/2020/07/03/us/politics/trump-coronavirus-mount-rushmore.html}{angry
mobs}'' and ``thugs,'' there are signs that he is especially alienating
voters in bedroom communities who approach the debate over racial
justice with a far more nuanced perspective than the president does.

It's a tumultuous time in the country, and attitudes are fluid. A
\href{https://www.monmouth.edu/polling-institute/documents/monmouthpoll_us_070820.pdf/}{Monmouth
University poll} released this week found that Republicans --- who still
overwhelmingly support the president --- were less than half as likely
to express sympathy for the demonstrators' anger as they had been four
weeks earlier, when the protest movement was first gaining steam. And
some strategists warn that there is political risk for Democrats if
swing voters begin to perceive them as radical.

But among most Americans, the poll found,
\href{https://www.nytimes.com/2020/07/08/us/politics/polling-race-protesters.html}{support
for protesters continued to run high} --- and so did concern over Mr.
Trump's rhetoric, sentiment that was reflected on the ground in swing
state suburbs like Cornelius --- traditionally a conservative-leaning
area --- and along the Main Line outside Philadelphia.

``It's a disgrace,'' said Jane Scilovati, a schoolteacher from Devon,
Pa., along Philadelphia's wealthy Main Line. She voted for Mr. Trump in
2016 but said she now regrets the decision. She called the president's
recent handling of racial issues ``deplorable.''

``He doesn't have any compassion or empathy; he doesn't reference
historical facts correctly,'' she said in an interview outside a
supermarket. ``He's brought more division to this country than we've
seen since the Civil Rights Act.''

\hypertarget{latest-updates-2020-election}{%
\section{\texorpdfstring{\href{https://www.nytimes.com/2020/07/31/us/elections/biden-vs-trump.html?action=click\&pgtype=Article\&state=default\&region=MAIN_CONTENT_1\&context=storylines_live_updates}{Latest
Updates: 2020
Election}}{Latest Updates: 2020 Election}}\label{latest-updates-2020-election}}

Updated 2020-08-01T01:26:45.732Z

\begin{itemize}
\tightlist
\item
  \href{https://www.nytimes.com/2020/07/31/us/elections/biden-vs-trump.html?action=click\&pgtype=Article\&state=default\&region=MAIN_CONTENT_1\&context=storylines_live_updates\#link-29fdff45}{Kamala
  Harris, a top vice-presidential contender, confronts double
  standards.}
\item
  \href{https://www.nytimes.com/2020/07/31/us/elections/biden-vs-trump.html?action=click\&pgtype=Article\&state=default\&region=MAIN_CONTENT_1\&context=storylines_live_updates\#link-13ec3d9c}{Karen
  Bass and Susan Rice are rising on Biden's vice-presidential
  shortlist.}
\item
  \href{https://www.nytimes.com/2020/07/31/us/elections/biden-vs-trump.html?action=click\&pgtype=Article\&state=default\&region=MAIN_CONTENT_1\&context=storylines_live_updates\#link-49e9a016}{Trump
  says Russian bounties to kill U.S. troops `never took place.'}
\end{itemize}

\href{https://www.nytimes.com/2020/07/31/us/elections/biden-vs-trump.html?action=click\&pgtype=Article\&state=default\&region=MAIN_CONTENT_1\&context=storylines_live_updates}{See
more updates}

Ms. Scilovati, 54, said she would support ``Daffy Duck'' rather than the
president in this year's election.

While Mr. Trump
\href{https://www.cnn.com/election/2016/results/exit-polls}{won suburban
areas overall by four percentage}points in 2016, according to exit
polls, white college-educated suburban women have rapidly moved away
from his Republican Party, and they helped deliver the House of
Representatives to the Democrats in 2018. And now, as some polling shows
Mr. Trump facing competitive races even in deep-red states, he cannot
afford to lose all of those voters again.

In the Monmouth poll, among white Americans with college degrees, only
13 percent said that Mr. Trump's response to the protesters had improved
the situation. Seventy-six percent of these college-educated white
Americans said that he had made things worse.

In interview after interview this week, suburbanites who have been open
to voting for either party in recent years described Mr. Trump as a
polarizing and deeply flawed messenger on the most searing issue of the
day.

``College-educated suburban women do not want to support someone who is
perceived to be intolerant on racial issues,'' said Whit Ayres, the
veteran Republican pollster. ``That has been true for many years, and is
particularly true now, after the George Floyd killing.''

The killing of Mr. Floyd, a Black man who died after a white police
officer knelt on his neck, sparked a nationwide outcry this summer over
police brutality and racism. But in recent weeks, Mr. Trump has made
\href{https://www.nytimes.com/2020/07/06/us/politics/trump-bubba-wallace-nascar.html}{playing
on white fears} an explicit part of his campaign pitch in a way no other
major presidential campaign has approached in
\href{https://www.nytimes.com/2018/12/03/us/politics/bush-willie-horton.html}{at
least a generation}.

He has defended the Confederate flag and falsely accused a Black NASCAR
driver of perpetrating a ``hoax'' involving a noose. He has described
the phrase ``Black Lives Matter'' as a ``symbol of hate.'' In
\href{https://www.nytimes.com/2020/07/03/us/politics/trump-coronavirus-mount-rushmore.html}{an
address} last week at Mount Rushmore, the president painted a dark
portrait of a nation whose values were under attack by ``the radical
left, the Marxists, the anarchists, the agitators, the looters'' ---
\href{https://www.nytimes.com/2020/07/04/us/politics/trump-mt-rushmore.html}{an
echo} of his inaugural address depiction of ``American carnage.''

Taken together, it is an approach that is out of step with corporate
America, a number of Republican officials, military leaders and the
majority, polls show, of American voters, though there are partisan
divisions around views on racial injustice.

\includegraphics{https://static01.nyt.com/images/2020/07/10/us/politics/10trump-suburbs2/merlin_174387498_d8f8ee7d-7dbe-4e39-a47e-6008ed46e412-articleLarge.jpg?quality=75\&auto=webp\&disable=upscale}

At the Harris Teeter supermarket here in Cornelius, Marisa Pascucci, 45,
was another voter who had changed her mind since 2016. She did not vote
for either presidential candidate that year. Now a self-described
``recovering Republican,'' she intends to come off the sidelines to vote
for Mr. Biden, she said.

On issues of race, she said, Mr. Trump ``is biased and is purposely
saying things to spark things. I don't agree with the way he instigates
things.''

She held that view even as she also expressed discomfort with the
destruction of Confederate statues. They may belong in museums where
they can be ``put in context, not up on a pedestal,'' she said, but
should not be destroyed.

In North Carolina, a rare competitive battleground state in the South,
the controversies surrounding Confederate symbols, and Mr. Trump's views
on those issues, are especially fraught. In a state that is home to the
Research Triangle in the Raleigh area, a museum in Charlotte dedicated
to championing the ``New South'' and an influx of newcomers in recent
years, many residents recoil at Mr. Trump's defense of those symbols.

In a recent survey of North Carolina by The New York Times and Siena
College, 51 percent of registered voters in the Charlotte suburbs
disapproved of Mr. Trump's handling of recent protests, compared with 44
percent who approved.

``We're a changing and evolving district; the `Lost Cause' narrative is
no longer relevant,'' said Democratic State Representative Christy
Clark, who flipped this statehouse seat from Republican control in 2018
and faces a competitive re-election fight. ``We need to pay attention to
racism in our society.''

Asked about Mr. Trump's remarks in defense of the Confederate flag
(``Flag decision has caused lowest ratings EVER!'' he tweeted, in
reference to NASCAR's decision to ban Confederate flags from its
events), she said, ``Trying to hold tight to it as a national narrative
is tone deaf.''

Johanna Godlewski, 35, an occupational therapist from Radnor, Pa., said
she was uncomfortable with removing statues that commemorate heroes of
the Confederacy or others who perpetuated racism.

``That's a little different, because I still believe it's a part of
history,'' she said.

But she also said she believed that Mr. Trump was stoking intolerance
and that he is a racist --- ``The way he speaks about women so roughly,
I can see him saying the same thing racially,'' she said.

Ms. Godlewski voted for Mrs. Clinton and is undecided this year. But she
said she would be unlikely to support the president because her husband
is a police officer, and she fears he would confront more protests if
Mr. Trump is re-elected --- a direct rebuke to Mr. Trump's claim that he
is the candidate who can bring order.

The Trump campaign, for its part, is working to cast the Democratic
Party as filled with extremists who support unfettered property
destruction, violent protests and defunding of the police (Mr. Biden,
the presumptive Democratic nominee, has objected to all of that). The
campaign is working to play on the concerns of voters who believe some
acts of protest have gone too far.

In a statement, a Trump campaign spokesman, Tim Murtaugh, defended the
president's record on race, noting his work on criminal justice reform
and saying that his economic policies had benefited people of color.

``The president's unifying message at Mount Rushmore made clear that he
is proud of America as having done more to advance individual liberties
for all people than any nation in the history of the world,'' he said.

Image

Suburban voters, particularly college-educated women, have rapidly moved
away from the Republican Party in the years since President Trump's
election.Credit...Swikar Patel for The New York Times

Certainly, some suburban voters remain ardent Trump defenders, standing
by his response to the unrest.

They described Mr. Trump in familiar terms: They dislike his social
media presence but continue to trust him on the economy despite the
unemployment crisis amid the pandemic. They say that the news media and
the Democrats have not given him a fair shake. Some cheered his language
about protesters and want to see him embrace a tougher line.

``Trashing places shouldn't be our statement,'' said Patricia Hamilton,
a 40-year-old resident of Marana, a conservative suburb of Tucson, Ariz.
``We should not be throwing fits. And I get we've got issues in this
country, but we're still the greatest country in the world.''

Some white Trump supporters also said they saw racism as an intractable
problem, and chose to focus on other aspects of Mr. Trump's record,
saying that he was doing the best job possible under difficult
circumstances.

``It's one of those issues that's going to take time to resolve,'' said
Chris Berglund, 40, who said he had ``no opinion'' of the president's
handling of racial matters and emphasized instead Mr. Trump's handling
of the economy. ``He's done a great job so far, turned the economy
around, low unemployment --- well, not right now, but there's not
anybody who could do that with the pandemic going on.''

But in the supermarket parking lot near the Confederate monument in
Cornelius, Shaneika Guy couldn't overlook the statue --- or Mr. Trump's
painful approach, in her view, to race.

``I want it to come down, I feel like it's racism,'' she said of the
monument. Ms. Guy, 34, has not yet decided whether she will vote for Mr.
Biden, but she will not support Mr. Trump.

``I don't think he's very compassionate about either race, even his
own,'' she said.

Giovanni Russonello contributed reporting from New York. Jon Hurdle
contributed reporting from Radnor, Pa., Hank Stephenson contributed from
Oro Valley, Ariz., and Dave Umhoefer from Milwaukee.

\hypertarget{our-2020-election-guide}{%
\section{Our 2020 Election Guide}\label{our-2020-election-guide}}

Updated July 31, 2020

\begin{itemize}
\item
  \begin{center}\rule{0.5\linewidth}{\linethickness}\end{center}

  \hypertarget{the-latest}{%
  \subsection{The Latest}\label{the-latest}}

  \begin{itemize}
  \tightlist
  \item
    President Trump's assault on the Postal Service is intersecting with
    his attacks on mail-in voting.
    \href{https://www.nytimes.com/2020/07/31/us/politics/trump-usps-mail-delays.html?action=click\&pgtype=Article\&state=default\&region=BELOW_MAIN_CONTENT\&context=storylines_guide}{Voting
    rights groups say it is a recipe for disaster.}
  \end{itemize}
\item
  \begin{center}\rule{0.5\linewidth}{\linethickness}\end{center}

  \hypertarget{bidens-vp-search}{%
  \subsection{Biden's V.P. Search}\label{bidens-vp-search}}

  \begin{itemize}
  \tightlist
  \item
    \href{https://www.nytimes.com/article/biden-vice-president-2020.html?action=click\&pgtype=Article\&state=default\&region=BELOW_MAIN_CONTENT\&context=storylines_guide}{Here
    are 13 women} who have been under consideration to be Joe Biden's
    running mate, and why each might be chosen --- and might not be.
  \end{itemize}
\item
  \begin{center}\rule{0.5\linewidth}{\linethickness}\end{center}

  \hypertarget{keep-up-with-our-coverage}{%
  \subsection{Keep Up With Our
  Coverage}\label{keep-up-with-our-coverage}}

  \begin{itemize}
  \tightlist
  \item
    Get an
    \href{https://www.nytimes.com/newsletters/politics?action=click\&pgtype=Article\&state=default\&region=BELOW_MAIN_CONTENT\&context=storylines_guide}{email}
    recapping the day's news
  \end{itemize}

  \begin{itemize}
  \tightlist
  \item
    Download our mobile app on
    \href{https://apps.apple.com/us/app/nytimes/id284862083?ls=1\&mat_click_id=5c79ae7455014fd1bd66b5610c05b8f2-20191112-16948\&referrer=mat_click_id\%3D5c79ae7455014fd1bd66b5610c05b8f2-20191112-16948\%26link_click_id\%3D722930677036718082}{iOS}
    and
    \href{http://a.localytics.com/android?id=com.nytimes.android\&referrer=utm_source\%3Dother_nyt_mobile_web\%26utm_medium\%3DWeb\%2520page\%26utm_term\%3DGeneral\%2520Mobile\%2520Page\%26utm_campaign\%3DNYT\%2520Mobile\%2520General\%2520Page}{Android}
    and turn on Breaking News and Politics alerts
  \end{itemize}
\end{itemize}

Advertisement

\protect\hyperlink{after-bottom}{Continue reading the main story}

\hypertarget{site-index}{%
\subsection{Site Index}\label{site-index}}

\hypertarget{site-information-navigation}{%
\subsection{Site Information
Navigation}\label{site-information-navigation}}

\begin{itemize}
\tightlist
\item
  \href{https://help.nytimes.com/hc/en-us/articles/115014792127-Copyright-notice}{©~2020~The
  New York Times Company}
\end{itemize}

\begin{itemize}
\tightlist
\item
  \href{https://www.nytco.com/}{NYTCo}
\item
  \href{https://help.nytimes.com/hc/en-us/articles/115015385887-Contact-Us}{Contact
  Us}
\item
  \href{https://www.nytco.com/careers/}{Work with us}
\item
  \href{https://nytmediakit.com/}{Advertise}
\item
  \href{http://www.tbrandstudio.com/}{T Brand Studio}
\item
  \href{https://www.nytimes.com/privacy/cookie-policy\#how-do-i-manage-trackers}{Your
  Ad Choices}
\item
  \href{https://www.nytimes.com/privacy}{Privacy}
\item
  \href{https://help.nytimes.com/hc/en-us/articles/115014893428-Terms-of-service}{Terms
  of Service}
\item
  \href{https://help.nytimes.com/hc/en-us/articles/115014893968-Terms-of-sale}{Terms
  of Sale}
\item
  \href{https://spiderbites.nytimes.com}{Site Map}
\item
  \href{https://help.nytimes.com/hc/en-us}{Help}
\item
  \href{https://www.nytimes.com/subscription?campaignId=37WXW}{Subscriptions}
\end{itemize}
