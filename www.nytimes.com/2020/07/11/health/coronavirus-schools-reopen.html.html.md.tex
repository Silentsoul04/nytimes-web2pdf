Sections

SEARCH

\protect\hyperlink{site-content}{Skip to
content}\protect\hyperlink{site-index}{Skip to site index}

\href{https://www.nytimes.com/section/health}{Health}

\href{https://myaccount.nytimes.com/auth/login?response_type=cookie\&client_id=vi}{}

\href{https://www.nytimes.com/section/todayspaper}{Today's Paper}

\href{/section/health}{Health}\textbar{}How to Reopen Schools: What
Science and Other Countries Teach Us

\url{https://nyti.ms/2ZjbJj5}

\begin{itemize}
\item
\item
\item
\item
\item
\item
\end{itemize}

\href{https://www.nytimes.com/news-event/coronavirus?action=click\&pgtype=Article\&state=default\&region=TOP_BANNER\&context=storylines_menu}{The
Coronavirus Outbreak}

\begin{itemize}
\tightlist
\item
  live\href{https://www.nytimes.com/2020/08/04/world/coronavirus-cases.html?action=click\&pgtype=Article\&state=default\&region=TOP_BANNER\&context=storylines_menu}{Latest
  Updates}
\item
  \href{https://www.nytimes.com/interactive/2020/us/coronavirus-us-cases.html?action=click\&pgtype=Article\&state=default\&region=TOP_BANNER\&context=storylines_menu}{Maps
  and Cases}
\item
  \href{https://www.nytimes.com/interactive/2020/science/coronavirus-vaccine-tracker.html?action=click\&pgtype=Article\&state=default\&region=TOP_BANNER\&context=storylines_menu}{Vaccine
  Tracker}
\item
  \href{https://www.nytimes.com/2020/08/02/us/covid-college-reopening.html?action=click\&pgtype=Article\&state=default\&region=TOP_BANNER\&context=storylines_menu}{College
  Reopening}
\item
  \href{https://www.nytimes.com/live/2020/08/04/business/stock-market-today-coronavirus?action=click\&pgtype=Article\&state=default\&region=TOP_BANNER\&context=storylines_menu}{Economy}
\end{itemize}

Advertisement

\protect\hyperlink{after-top}{Continue reading the main story}

Supported by

\protect\hyperlink{after-sponsor}{Continue reading the main story}

\hypertarget{how-to-reopen-schools-what-science-and-other-countries-teach-us}{%
\section{How to Reopen Schools: What Science and Other Countries Teach
Us}\label{how-to-reopen-schools-what-science-and-other-countries-teach-us}}

The pressure to bring American students back to classrooms is intense,
but the calculus is tricky with infections still out of control in many
communities.

\includegraphics{https://static01.nyt.com/images/2020/07/12/science/00virus-schools-reopen01/merlin_170865825_2993c63a-7bb5-4ae4-853c-2f355b29af24-articleLarge.jpg?quality=75\&auto=webp\&disable=upscale}

By \href{https://www.nytimes.com/by/pam-belluck}{Pam Belluck},
\href{https://www.nytimes.com/by/apoorva-mandavilli}{Apoorva Mandavilli}
and \href{https://www.nytimes.com/by/benedict-carey}{Benedict Carey}

\begin{itemize}
\item
  July 11, 2020
\item
  \begin{itemize}
  \item
  \item
  \item
  \item
  \item
  \item
  \end{itemize}
\end{itemize}

\href{https://www.nytimes.com/es/2020/07/27/espanol/ciencia-y-tecnologia/regreso-a-clases-coronavirus.html}{Leer
en español}

As school districts across the United States consider whether and how to
restart in-person classes, their challenge is complicated by a pair of
fundamental uncertainties: No nation has tried to send children back to
school with the virus raging at levels like America's, and the
scientific research about transmission in classrooms is limited.

The World Health Organization has now concluded that
\href{https://slack-redir.net/link?url=https\%3A\%2F\%2Fwww.nytimes.com\%2F2020\%2F07\%2F09\%2Fhealth\%2Fvirus-aerosols-who.html}{the
virus is airborne} in crowded, indoor spaces with poor ventilation, a
description that fits many American schools. But there is enormous
pressure to bring students back --- from parents, from pediatricians and
child development specialists, and from President Trump.

``I'm just going to say it: It feels like we're playing Russian roulette
with our kids and our staff,'' said Robin Cogan, a nurse at the Yorkship
School in Camden, N.J., who serves on the state's committee on reopening
schools.

\href{https://www.cdc.gov/coronavirus/2019-ncov/hcp/pediatric-hcp.html\#burden-disease-risk-factors}{Data
from around the world} clearly shows that children are far less likely
to become seriously ill from the coronavirus than adults. But there are
big unanswered questions, including how often children become infected
and what role they play in transmitting the virus. Some research
suggests younger children are less likely to infect other people than
teenagers are, which would make opening elementary schools less risky
than high schools, but the evidence is not conclusive.

The experience abroad has shown that measures such as physical
distancing and wearing masks in schools can make a difference. Another
important variable is how widespread the virus is in the community over
all, because that will affect how many people potentially bring it into
a school.

For most districts, the solution won't be an all-or-nothing approach.
\href{https://bioethics.jhu.edu/research-and-outreach/projects/eschool-initiative/school-policy-tracker/}{Many
systems}, including the nation's largest, New York City, are devising
hybrids that involve spending some days in classrooms and other days
online.

``You have to do a lot more than just waving your hands and say make it
so,'' said Dr. Joshua Sharfstein, a professor of the practice at Johns
Hopkins Bloomberg School of Public Health. ``First you have to control
the community spread and then you have to open schools thoughtfully.''

\hypertarget{the-transmission-puzzle}{%
\subsection{The transmission puzzle}\label{the-transmission-puzzle}}

Though children are at much lower risk of getting seriously ill from the
coronavirus than adults, the risk is not zero. A small number of
children have died and others needed intensive care because they
\href{https://www.nytimes.com/2020/04/06/health/coronavirus-children-us.html}{suffered
respiratory failure} or an
\href{https://www.nytimes.com/2020/05/17/health/coronavirus-multisystem-fnflammatory-syndrome-children-teenagers.html}{inflammatory
syndrome} that caused heart or circulatory problems.

The larger concern with reopening schools is the potential for children
to become infected, many with no symptoms, and then spread the virus to
others, including family members, teachers and other school employees.
Most evidence to date suggests that even if children under 12 are
infected at the same rates as the adults around them, they are less
likely to spread it. The American Academy of Pediatrics has cited some
of this data to
\href{https://services.aap.org/en/pages/2019-novel-coronavirus-covid-19-infections/clinical-guidance/covid-19-planning-considerations-return-to-in-person-education-in-schools/}{recommend
that schools reopen} with proper safety precautions.

But the bulk of the evidence was collected in countries that were
already in lockdown or had begun to implement other preventive measures.
And few countries have systematically tested children for the virus or
for antibodies that would indicate whether they had been exposed to the
virus.

Infectious disease specialists have been modeling schools' impact on
community spread beginning as far back as February.

\includegraphics{https://static01.nyt.com/images/2020/07/10/science/00virus-schools-reopen02/merlin_170483466_ca28d6d9-7b78-4509-9b49-9fcddef888b3-articleLarge.jpg?quality=75\&auto=webp\&disable=upscale}

In March, most modelers agreed that closing schools
\href{https://www.nytimes.com/2020/05/05/health/coronavirus-children-transmission-school.html}{would
slow the progression of infections}. But wider measures, like social
distancing, proved to have a far greater containing effect,
overshadowing the results of school closings,
\href{https://www.medrxiv.org/content/10.1101/2020.04.16.20068403v1}{according
to recent analyses}.

The risk of reopening ``will depend on how well schools contain
transmission, with masks, for instance, or limiting occupancy,'' said
Lauren Ancel Meyers, a professor of biology and statistics at the
University of Texas, Austin, who has been consulting with the city and
school districts. ``The background community transmission rate in August
will also be a factor.''

\hypertarget{latest-updates-global-coronavirus-outbreak}{%
\section{\texorpdfstring{\href{https://www.nytimes.com/2020/08/04/world/coronavirus-cases.html?action=click\&pgtype=Article\&state=default\&region=MAIN_CONTENT_1\&context=storylines_live_updates}{Latest
Updates: Global Coronavirus
Outbreak}}{Latest Updates: Global Coronavirus Outbreak}}\label{latest-updates-global-coronavirus-outbreak}}

Updated 2020-08-04T20:42:41.838Z

\begin{itemize}
\tightlist
\item
  \href{https://www.nytimes.com/2020/08/04/world/coronavirus-cases.html?action=click\&pgtype=Article\&state=default\&region=MAIN_CONTENT_1\&context=storylines_live_updates\#link-1228a480}{Novavax
  sees encouraging results from two studies of its experimental
  vaccine.}
\item
  \href{https://www.nytimes.com/2020/08/04/world/coronavirus-cases.html?action=click\&pgtype=Article\&state=default\&region=MAIN_CONTENT_1\&context=storylines_live_updates\#link-4825b93}{Public
  and private schools in Maryland and elsewhere are divided over
  in-person instruction.}
\item
  \href{https://www.nytimes.com/2020/08/04/world/coronavirus-cases.html?action=click\&pgtype=Article\&state=default\&region=MAIN_CONTENT_1\&context=storylines_live_updates\#link-50f7386d}{The
  United Nations calls on policymakers to `plan thoroughly for school
  reopenings.'}
\end{itemize}

\href{https://www.nytimes.com/2020/08/04/world/coronavirus-cases.html?action=click\&pgtype=Article\&state=default\&region=MAIN_CONTENT_1\&context=storylines_live_updates}{See
more updates}

More live coverage:
\href{https://www.nytimes.com/live/2020/08/04/business/stock-market-today-coronavirus?action=click\&pgtype=Article\&state=default\&region=MAIN_CONTENT_1\&context=storylines_live_updates}{Markets}

In Austin, for example, which like cities in Florida and Arizona has
seen a recent acceleration in new cases, the estimated infection rate
now is about seven per 1,000 residents. That means a school with 500
students would have about four carrying the coronavirus. ``The school
might be able to contain those, depending on the measures it takes,''
Dr. Meyers said.

If not, schools could help incubate outbreaks, given that they're
enclosed facilities where students, especially younger ones, are likely
to have great difficulty social distancing, never mind wearing masks.
Even if it turns out that children do not spread the virus efficiently,
all it would take is one or two to seed new chains.

\hypertarget{the-evidence-from-abroad}{%
\subsection{The evidence from abroad}\label{the-evidence-from-abroad}}

So far, countries that reopened schools after reducing infection levels
--- and imposed requirements like physical distancing and limits on
class sizes ---
\href{https://globalhealth.washington.edu/sites/default/files/COVID-19\%20Schools\%20Summary\%20\%282\%29.pdf?mkt_tok=eyJpIjoiTkRreE5XWXlORFF3TXpNeCIsInQiOiJIbVNQTTVySEo0Vzk1cHVBZVVqWnFGVmR1UEJxRGdpd01mTXg4OGw3Mk5nTnpmaUoyMGt2UXIwWVZBOE5GVjIybHA5aStrbzJ3MUxsanoxamZibmlocmpSbXZyVFVoV0VHYU1aTGx0RnpsMXlmOEtXSVJqaDJsZ0RJU1BQcVZjZSJ9}{have
not seen a surge} in coronavirus cases.

Norway and Denmark are good examples. Both reopened their schools in
April, a month or so after they were closed, but they initially opened
them only for younger children, keeping high schools shut until later.
They strengthened sanitizing procedures, and have kept class size
limited, children in small groups at recess and space between desks.
Neither country has seen a significant increase in cases.

There have not yet been rigorous scientific studies on the potential for
school-based spread, but a smattering of case reports, most of them not
yet peer-reviewed, bolster the notion that it is not inevitably a high
risk.

Image

Students at a primary school in Bangkok returned on July 1, a delayed
start to their academic year.Credit...Adam Dean for The New York Times

\href{https://www.eurosurveillance.org/content/10.2807/1560-7917.ES.2020.25.21.2000903\#html_fulltext}{One
snapshot comes from a study in Ireland} of six infected people (two high
school students, an elementary student and three adults) who spent time
in schools before they were closed in March. The researchers analyzed
1,155 contacts of the six patients to see if any had been found to have
confirmed coronavirus infection. The contacts included participants in
school activities that could be fertile ground for transmission, like
music lessons on woodwind instruments, choir practice and sports. None
of the students appeared to have infected any other people, the authors
reported, adding that the only documented transmission of the virus was
to two adults who were in contact with one of the infected adults
outside of school.

But there have been school-based outbreaks in countries with higher
community infection levels and countries that apparently eased safety
guidelines too soon. In Israel, the virus infected more than 200
students and staff after schools reopened in early May and lifted limits
on class size a few weeks later, according to a
\href{https://globalhealth.washington.edu/sites/default/files/COVID-19\%20Schools\%20Summary\%20\%282\%29.pdf?mkt_tok=eyJpIjoiTkRreE5XWXlORFF3TXpNeCIsInQiOiJIbVNQTTVySEo0Vzk1cHVBZVVqWnFGVmR1UEJxRGdpd01mTXg4OGw3Mk5nTnpmaUoyMGt2UXIwWVZBOE5GVjIybHA5aStrbzJ3MUxsanoxamZibmlocmpSbXZyVFVoV0VHYU1aTGx0RnpsMXlmOEtXSVJqaDJsZ0RJU1BQcVZjZSJ9}{report
by University of Washington researchers}.

Case studies in some countries suggest differences in virus transmission
in younger children compared to older children.

In one community in northern France, Crépy-en-Valois, two high school
teachers became ill with Covid-19 in early February, before schools
closed. Scientists from the Institut Pasteur later tested the school's
students and staff for coronavirus antibodies. They found antibodies in
38 percent of the students, 43 percent of the teachers, and 59 percent
of other school staff, said Dr. Arnaud Fontanet, an epidemiologist at
the institute who led
\href{https://www.medrxiv.org/content/10.1101/2020.04.18.20071134v1}{the
study} and is a member of a committee advising the French government.

``Clearly you know that the virus circulated in the high school,'' Dr.
Fontanet said.

Later, the team tested students and staff from six
\href{https://www.medrxiv.org/content/10.1101/2020.06.25.20140178v2}{elementary
schools} in the community. The closure of schools in mid-February
provided an opportunity to see if younger children had become infected
when schools were in session, the point when the virus struck high
school students.

Researchers found antibodies in only 9 percent of elementary students, 7
percent of teachers and 4 percent of other staff. They identified three
students in three different elementary schools who had attended classes
with acute coronavirus symptoms before the schools closed. None appeared
to have infected other children, teachers or staff, Dr. Fontanet said.
Two of those symptomatic students had siblings in the high school and
the third had a sister who worked in the high school, he said.

The research also indicated that when an elementary school student
tested positive for coronavirus antibodies, there was a very high
probability that the student's parents had also been infected, Dr.
Fontanet said. The probability was not nearly as high for parents of
high school students. ``When I look at the timing, we think it started
in the high school, moved into the families and then to the young
students,'' he said.

Dr. Fontanet said that the findings suggest that older children may be
able to transmit the virus more easily than younger children.

That pattern may also be reflected by the experience in Israel, where
one of the largest school outbreaks, involving about 175 students and
staff, occurred in Gymnasia Rehavia, a middle and high school in
Jerusalem.

\href{https://www.nytimes.com/news-event/coronavirus?action=click\&pgtype=Article\&state=default\&region=MAIN_CONTENT_3\&context=storylines_faq}{}

\hypertarget{the-coronavirus-outbreak-}{%
\subsubsection{The Coronavirus Outbreak
›}\label{the-coronavirus-outbreak-}}

\hypertarget{frequently-asked-questions}{%
\paragraph{Frequently Asked
Questions}\label{frequently-asked-questions}}

Updated August 4, 2020

\begin{itemize}
\item ~
  \hypertarget{i-have-antibodies-am-i-now-immune}{%
  \paragraph{I have antibodies. Am I now
  immune?}\label{i-have-antibodies-am-i-now-immune}}

  \begin{itemize}
  \tightlist
  \item
    As of right
    now,\href{https://www.nytimes.com/2020/07/22/health/covid-antibodies-herd-immunity.html?action=click\&pgtype=Article\&state=default\&region=MAIN_CONTENT_3\&context=storylines_faq}{that
    seems likely, for at least several months.} There have been
    frightening accounts of people suffering what seems to be a second
    bout of Covid-19. But experts say these patients may have a
    drawn-out course of infection, with the virus taking a slow toll
    weeks to months after initial exposure. People infected with the
    coronavirus typically
    \href{https://www.nature.com/articles/s41586-020-2456-9}{produce}
    immune molecules called antibodies, which are
    \href{https://www.nytimes.com/2020/05/07/health/coronavirus-antibody-prevalence.html?action=click\&pgtype=Article\&state=default\&region=MAIN_CONTENT_3\&context=storylines_faq}{protective
    proteins made in response to an
    infection}\href{https://www.nytimes.com/2020/05/07/health/coronavirus-antibody-prevalence.html?action=click\&pgtype=Article\&state=default\&region=MAIN_CONTENT_3\&context=storylines_faq}{.
    These antibodies may} last in the body
    \href{https://www.nature.com/articles/s41591-020-0965-6}{only two to
    three months}, which may seem worrisome, but that's perfectly normal
    after an acute infection subsides, said Dr. Michael Mina, an
    immunologist at Harvard University. It may be possible to get the
    coronavirus again, but it's highly unlikely that it would be
    possible in a short window of time from initial infection or make
    people sicker the second time.
  \end{itemize}
\item ~
  \hypertarget{im-a-small-business-owner-can-i-get-relief}{%
  \paragraph{I'm a small-business owner. Can I get
  relief?}\label{im-a-small-business-owner-can-i-get-relief}}

  \begin{itemize}
  \tightlist
  \item
    The
    \href{https://www.nytimes.com/article/small-business-loans-stimulus-grants-freelancers-coronavirus.html?action=click\&pgtype=Article\&state=default\&region=MAIN_CONTENT_3\&context=storylines_faq}{stimulus
    bills enacted in March} offer help for the millions of American
    small businesses. Those eligible for aid are businesses and
    nonprofit organizations with fewer than 500 workers, including sole
    proprietorships, independent contractors and freelancers. Some
    larger companies in some industries are also eligible. The help
    being offered, which is being managed by the Small Business
    Administration, includes the Paycheck Protection Program and the
    Economic Injury Disaster Loan program. But lots of folks have
    \href{https://www.nytimes.com/interactive/2020/05/07/business/small-business-loans-coronavirus.html?action=click\&pgtype=Article\&state=default\&region=MAIN_CONTENT_3\&context=storylines_faq}{not
    yet seen payouts.} Even those who have received help are confused:
    The rules are draconian, and some are stuck sitting on
    \href{https://www.nytimes.com/2020/05/02/business/economy/loans-coronavirus-small-business.html?action=click\&pgtype=Article\&state=default\&region=MAIN_CONTENT_3\&context=storylines_faq}{money
    they don't know how to use.} Many small-business owners are getting
    less than they expected or
    \href{https://www.nytimes.com/2020/06/10/business/Small-business-loans-ppp.html?action=click\&pgtype=Article\&state=default\&region=MAIN_CONTENT_3\&context=storylines_faq}{not
    hearing anything at all.}
  \end{itemize}
\item ~
  \hypertarget{what-are-my-rights-if-i-am-worried-about-going-back-to-work}{%
  \paragraph{What are my rights if I am worried about going back to
  work?}\label{what-are-my-rights-if-i-am-worried-about-going-back-to-work}}

  \begin{itemize}
  \tightlist
  \item
    Employers have to provide
    \href{https://www.osha.gov/SLTC/covid-19/standards.html}{a safe
    workplace} with policies that protect everyone equally.
    \href{https://www.nytimes.com/article/coronavirus-money-unemployment.html?action=click\&pgtype=Article\&state=default\&region=MAIN_CONTENT_3\&context=storylines_faq}{And
    if one of your co-workers tests positive for the coronavirus, the
    C.D.C.} has said that
    \href{https://www.cdc.gov/coronavirus/2019-ncov/community/guidance-business-response.html}{employers
    should tell their employees} -\/- without giving you the sick
    employee's name -\/- that they may have been exposed to the virus.
  \end{itemize}
\item ~
  \hypertarget{should-i-refinance-my-mortgage}{%
  \paragraph{Should I refinance my
  mortgage?}\label{should-i-refinance-my-mortgage}}

  \begin{itemize}
  \tightlist
  \item
    \href{https://www.nytimes.com/article/coronavirus-money-unemployment.html?action=click\&pgtype=Article\&state=default\&region=MAIN_CONTENT_3\&context=storylines_faq}{It
    could be a good idea,} because mortgage rates have
    \href{https://www.nytimes.com/2020/07/16/business/mortgage-rates-below-3-percent.html?action=click\&pgtype=Article\&state=default\&region=MAIN_CONTENT_3\&context=storylines_faq}{never
    been lower.} Refinancing requests have pushed mortgage applications
    to some of the highest levels since 2008, so be prepared to get in
    line. But defaults are also up, so if you're thinking about buying a
    home, be aware that some lenders have tightened their standards.
  \end{itemize}
\item ~
  \hypertarget{what-is-school-going-to-look-like-in-september}{%
  \paragraph{What is school going to look like in
  September?}\label{what-is-school-going-to-look-like-in-september}}

  \begin{itemize}
  \tightlist
  \item
    It is unlikely that many schools will return to a normal schedule
    this fall, requiring the grind of
    \href{https://www.nytimes.com/2020/06/05/us/coronavirus-education-lost-learning.html?action=click\&pgtype=Article\&state=default\&region=MAIN_CONTENT_3\&context=storylines_faq}{online
    learning},
    \href{https://www.nytimes.com/2020/05/29/us/coronavirus-child-care-centers.html?action=click\&pgtype=Article\&state=default\&region=MAIN_CONTENT_3\&context=storylines_faq}{makeshift
    child care} and
    \href{https://www.nytimes.com/2020/06/03/business/economy/coronavirus-working-women.html?action=click\&pgtype=Article\&state=default\&region=MAIN_CONTENT_3\&context=storylines_faq}{stunted
    workdays} to continue. California's two largest public school
    districts --- Los Angeles and San Diego --- said on July 13, that
    \href{https://www.nytimes.com/2020/07/13/us/lausd-san-diego-school-reopening.html?action=click\&pgtype=Article\&state=default\&region=MAIN_CONTENT_3\&context=storylines_faq}{instruction
    will be remote-only in the fall}, citing concerns that surging
    coronavirus infections in their areas pose too dire a risk for
    students and teachers. Together, the two districts enroll some
    825,000 students. They are the largest in the country so far to
    abandon plans for even a partial physical return to classrooms when
    they reopen in August. For other districts, the solution won't be an
    all-or-nothing approach.
    \href{https://bioethics.jhu.edu/research-and-outreach/projects/eschool-initiative/school-policy-tracker/}{Many
    systems}, including the nation's largest, New York City, are
    devising
    \href{https://www.nytimes.com/2020/06/26/us/coronavirus-schools-reopen-fall.html?action=click\&pgtype=Article\&state=default\&region=MAIN_CONTENT_3\&context=storylines_faq}{hybrid
    plans} that involve spending some days in classrooms and other days
    online. There's no national policy on this yet, so check with your
    municipal school system regularly to see what is happening in your
    community.
  \end{itemize}
\end{itemize}

There are different theories about why older children would be more
likely to transmit the virus than younger children. Some scientists say
that younger children are less likely to have Covid-19 symptoms like
coughs and less likely to have strong speaking voices, both of which can
transmit the virus in droplets. Other researchers are examining whether
proteins that enable the virus to enter lung cells and replicate are
less abundant in children, limiting the severity of their infection and
potentially their ability to transmit the virus.

\hypertarget{what-schools-can-do}{%
\subsection{What schools can do}\label{what-schools-can-do}}

Testing for infections in schools is essential, public health experts
said. The Centers for Disease Control and Prevention recommends testing
of students or teachers based only on symptoms or a history of exposure.
But that will not catch everyone who is infected.

``We know that asymptomatic or pre-symptomatic spread is real, and we
know that kids are less likely to show symptoms if they're infected than
adults,'' said Dr. Megan Ranney, an emergency medicine doctor and expert
in adolescent health at Brown University. Schools should randomly test
students and teachers, she said, but that may be impossible given the
lack of funding and limited testing even in hospitals.

Countries that have reopened schools have implemented a range of safety
guidelines.

Image

Students returned to school in Thun, Switzerland, on May 11, after a
coronavirus lockdown kept them home.Credit...Peter Schneider/EPA, via
Shutterstock

Some countries initially brought back only a portion of their students
--- younger children in Denmark, Norway, Belgium, Switzerland and
Greece; older children in Germany, according to the
\href{https://globalhealth.washington.edu/sites/default/files/COVID-19\%20Schools\%20Summary\%20\%282\%29.pdf?mkt_tok=eyJpIjoiTkRreE5XWXlORFF3TXpNeCIsInQiOiJIbVNQTTVySEo0Vzk1cHVBZVVqWnFGVmR1UEJxRGdpd01mTXg4OGw3Mk5nTnpmaUoyMGt2UXIwWVZBOE5GVjIybHA5aStrbzJ3MUxsanoxamZibmlocmpSbXZyVFVoV0VHYU1aTGx0RnpsMXlmOEtXSVJqaDJsZ0RJU1BQcVZjZSJ9}{report
by University of Washington researchers}. Belgium brought back students
in shifts on alternate days.

Several countries limited class size, often allowing a maximum of 10 to
15 students in a classroom. Many place desks several feet apart. Several
countries group children in pods or cohorts with social interaction
largely restricted to those groups, especially at recess and lunchtime.

Mask-wearing policies vary. In Asia, where the practice of wearing masks
during flu season is common, many countries are requiring masks in
school. Elsewhere, some countries required masks for only some students
or staff, such as teachers in Belgium and high school students in
France, according to the University of Washington report.

In Germany, students who test negative for the virus do not have to wear
masks, according to the report, which said that since opening schools,
Germany has seen increased transmission of the virus among students, but
not school staff.

The C.D.C. has outlined steps schools can take to minimize the risks for
students, including maintaining a distance of six feet, washing hands
and wearing masks.

``The guidelines are already exceptionally weak,'' said Carl Bergstrom,
an infectious diseases expert at the University of Washington in
Seattle. He and others said they feared that the recommendations would
get watered down even more in response to political pressure.

The C.D.C. has been working on new recommendations for reopening schools
for several weeks, in consultation with organizations like the National
Association of School Nurses, according to a C.D.C. spokeswoman. The
five planned documents include guidance on symptom screening and face
masks, and a checklist for parents or guardians trying to decide whether
to send their children to school. But they do not include any
information on improving ventilation or curtailing airborne spread of
the virus.

Schools will need to ensure that they circulate fresh air, whether by
filtering the air, pumping it in from the outside, or simply by opening
windows, said Saskia Popescu, a hospital epidemiologist at The
University of Arizona. School nurses like Ms. Cogan will also need
protective equipment like gloves, gowns and N95 masks.

There are differences in how other countries are responding when
coronavirus cases are identified in schools, with some countries, like
Israel, closing entire schools for a single case and others taking the
more targeted approach of sending students and teachers in an affected
classroom into home quarantine for two weeks.

Dr. Kathryn Edwards, an infectious disease specialist and professor of
pediatrics at Vanderbilt University School of Medicine, is advising
Nashville schools on reopening approaches. She said the district is
still evaluating how far apart desks should be. ``Some people say you
only need three feet and others say you need six feet, and others wonder
with the aerosol issue, do we need more distance?''

Dr. Edwards said she was disappointed by Nashville's decision, announced
Thursday, to conduct classes
\href{https://www.tennessean.com/story/news/education/2020/07/09/metro-schools-academic-year-start-online-nashville-students/5383315002/}{online
for the first month of school}, at least until Labor Day.

Keeping schools closed for a prolonged stretch has worrisome
implications for social and academic development, child development
experts say. It also became evident this spring that denying children a
real school day deepened racial and economic inequalities.

``There is really damage to kids if they don't go to school,'' Dr.
Edwards said. ``I think we have got to think of the kids and getting
them back to school safely.''

Advertisement

\protect\hyperlink{after-bottom}{Continue reading the main story}

\hypertarget{site-index}{%
\subsection{Site Index}\label{site-index}}

\hypertarget{site-information-navigation}{%
\subsection{Site Information
Navigation}\label{site-information-navigation}}

\begin{itemize}
\tightlist
\item
  \href{https://help.nytimes.com/hc/en-us/articles/115014792127-Copyright-notice}{©~2020~The
  New York Times Company}
\end{itemize}

\begin{itemize}
\tightlist
\item
  \href{https://www.nytco.com/}{NYTCo}
\item
  \href{https://help.nytimes.com/hc/en-us/articles/115015385887-Contact-Us}{Contact
  Us}
\item
  \href{https://www.nytco.com/careers/}{Work with us}
\item
  \href{https://nytmediakit.com/}{Advertise}
\item
  \href{http://www.tbrandstudio.com/}{T Brand Studio}
\item
  \href{https://www.nytimes.com/privacy/cookie-policy\#how-do-i-manage-trackers}{Your
  Ad Choices}
\item
  \href{https://www.nytimes.com/privacy}{Privacy}
\item
  \href{https://help.nytimes.com/hc/en-us/articles/115014893428-Terms-of-service}{Terms
  of Service}
\item
  \href{https://help.nytimes.com/hc/en-us/articles/115014893968-Terms-of-sale}{Terms
  of Sale}
\item
  \href{https://spiderbites.nytimes.com}{Site Map}
\item
  \href{https://help.nytimes.com/hc/en-us}{Help}
\item
  \href{https://www.nytimes.com/subscription?campaignId=37WXW}{Subscriptions}
\end{itemize}
