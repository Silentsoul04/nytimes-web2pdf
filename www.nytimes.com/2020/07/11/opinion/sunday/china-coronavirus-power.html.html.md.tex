Sections

SEARCH

\protect\hyperlink{site-content}{Skip to
content}\protect\hyperlink{site-index}{Skip to site index}

\href{https://www.nytimes.com/section/opinion/sunday}{Sunday Review}

\href{https://myaccount.nytimes.com/auth/login?response_type=cookie\&client_id=vi}{}

\href{https://www.nytimes.com/section/todayspaper}{Today's Paper}

\href{/section/opinion/sunday}{Sunday Review}\textbar{}The Chinese
Decade

\href{https://nyti.ms/3gSkI0q}{https://nyti.ms/3gSkI0q}

\begin{itemize}
\item
\item
\item
\item
\item
\end{itemize}

Advertisement

\protect\hyperlink{after-top}{Continue reading the main story}

\href{/section/opinion}{Opinion}

Supported by

\protect\hyperlink{after-sponsor}{Continue reading the main story}

\hypertarget{the-chinese-decade}{%
\section{The Chinese Decade}\label{the-chinese-decade}}

The coronavirus has given Beijing a strategic opportunity --- but one
that might not last.

\href{https://www.nytimes.com/by/ross-douthat}{\includegraphics{https://static01.nyt.com/images/2018/04/03/opinion/ross-douthat/ross-douthat-thumbLarge.png}}

By \href{https://www.nytimes.com/by/ross-douthat}{Ross Douthat}

Opinion Columnist

\begin{itemize}
\item
  July 11, 2020
\item
  \begin{itemize}
  \item
  \item
  \item
  \item
  \item
  \end{itemize}
\end{itemize}

\includegraphics{https://static01.nyt.com/images/2020/07/12/opinion/12Douthat/merlin_174171105_9af456b2-5f8b-43db-a287-a6376a2133cc-articleLarge.jpg?quality=75\&auto=webp\&disable=upscale}

\href{https://cn.nytimes.com/opinion/20200713/china-coronavirus-power/}{阅读简体中文版}\href{https://cn.nytimes.com/opinion/20200713/china-coronavirus-power/zh-hant/}{閱讀繁體中文版}

It is quite extraordinary that a pandemic originating in a Chinese
province, a disease whose initial cover-up briefly seemed likely to deal
a grave blow to the Communist regime, has instead given China a
geopolitical opportunity unlike any enjoyed by an American rival since
at least the Vietnam War.

This opportunity has been a long time building. Across the 2000s and
early 2010s, China's ruling party reaped the benefits of globalization
without paying the cost, in political liberalization, that confident
Westerners expected the economic opening to impose. This
richer-but-not-freer China proved that it was possible for an
authoritarian power to tame the internet, to make its citizens
hardworking capitalists without granting them substantial political
freedoms, to buy allies across the developing world, and to establish
beachheads of influence --- in Hollywood, Silicon Valley, American
academia, the NBA, Washington, D.C. --- in the power centers of its
superpower rival.

Eventually, America responded to all this as you would expect a
superpower to react: It elected a China hawk who promised to get tough
on Beijing, to bring back jobs lost to the China shock, and to shift
foreign policy priorities from the Middle East to the Pacific. But there
was one small difficulty: This hawk was no Truman or Reagan, but rather
a reality-television mountebank whose real attitude toward China policy
was, basically, \emph{whatever}
\href{https://www.wsj.com/articles/john-bolton-the-scandal-of-trumps-china-policy-11592419564}{\emph{gets
me re-elected}} \emph{works}. A mountebank, and also a world-historical
incompetent, who was presented with exactly the challenge that his
nationalism was supposed to answer --- a dangerous disease carried by
global trade routes from our leading rival --- and managed to turn it
into an American calamity instead.

So China has won twice over: First rising with the active collaboration
of naïve American centrists, and then consolidating its gains with the
de facto collaboration of a feckless American populist. Four months into
the coronavirus era,
\href{https://www.ft.com/content/a0eac4d1-625d-4073-9eee-dcf1bacb749e}{Xi
Jinping's government} is throttling Hong Kong, taking tiny bites out of
India, saber-rattling with its other neighbors, and perpetrating a
near-genocide in its Muslim West. Meanwhile America is rudderless and
leaderless, consumed by protests and elite psychodrama and a moral
crusade whose zeal seems turned entirely inward, with no time to spare
for a rival power's crimes.

Furthermore, Trump's likely successor is a figure whose record and
instincts and family connections all belong to the recent period of
American illusions about China. Joe Biden speaks more hawkishly than he
did five years ago, but the very thing that makes him effective as a
foil to Trump --- his promise of a return to Obama-era normalcy --- also
makes him an unlikely person to drastically re-evaluate the choices that
gave China its advantages today.

If you were scripting a historical moment when a rising power overtakes
a fading hegemon, the cascade from establishment naïveté through
Trumpian folly to the coronavirus disaster would be almost too
on-the-nose. And foreign policy hands who fear a ``Thucydides trap'' ---
a scenario where a rising and an established power end up, like Athens
and Sparta, in a war --- have good reasons to be nervous about how the
current combination of Chinese ambition and American decline might play
out in, say, the Taiwan Strait.

But there is another way to look at things. It's possible that we're
nearing a peak of U.S.-China tension not because China is poised to
permanently overtake the United States as a global power, but because
China itself is peaking --- with
\href{https://www.wsj.com/articles/chinas-state-driven-growth-model-is-running-out-of-gas-11563372006}{a
slowing growth rate} that may leave it short of the prosperity achieved
by its Pacific neighbors, a swiftly aging population, and a combination
of self-limiting soft power and maxed-out hard power that's likely to
diminish, relative to the U.S. and India and others, in the 2040s and
beyond.

Instead of a Chinese Century, in other words, the coronavirus might be
ushering in a Chinese Decade, in which Xi Jinping's government behaves
with maximal aggression because it sees an opportunity that won't come
again.

That aggression has inward and outward manifestations. The inward form
is the attempt to lock in Han pre-eminence in China by forcibly
\href{https://foreignpolicy.com/2020/06/30/chinese-communist-party-han-baby-boom-sterilization-ethnic-minorities/}{suppressing
non-Han birthrates}, so that population decline doesn't lead to swings
in ethnic power. The outward form is what you see in Hong Kong and might
see with Taiwan soon --- an attempt to reach greedily for Greater China
goals because the odds of success look better now than in the further
future.

If this is China's true strategic calculus, it won't make the 2020s any
less dangerous. (History is thick with reckless decisions made because
great powers felt that long-term trends had turned against them.) But it
should condition the U.S. policy response, whether under a President
Biden or a future Republican with more capabilities than Trump, toward a
balance between resolve and caution, hawkishness and restraint.

If we show too much indecision and weakness, or just too obvious a
desire for the pre-Trump status quo, then Beijing's escalation will
continue, and the risks of war will rise.

But if we find a way to contain China for a decade, the Chinese century
could be permanently postponed.

\emph{The Times is committed to publishing}
\href{https://www.nytimes.com/2019/01/31/opinion/letters/letters-to-editor-new-york-times-women.html}{\emph{a
diversity of letters}} \emph{to the editor. We'd like to hear what you
think about this or any of our articles. Here are some}
\href{https://help.nytimes.com/hc/en-us/articles/115014925288-How-to-submit-a-letter-to-the-editor}{\emph{tips}}\emph{.
And here's our email:}
\href{mailto:letters@nytimes.com}{\emph{letters@nytimes.com}}\emph{.}

\emph{Follow The New York Times Opinion section on}
\href{https://www.facebook.com/nytopinion}{\emph{Facebook}}\emph{,}
\href{http://twitter.com/NYTOpinion}{\emph{Twitter (@NYTOpinion)}}
\emph{and}
\href{https://www.instagram.com/nytopinion/}{\emph{Instagram}}\emph{,
join the Facebook political discussion group,}
\href{https://www.facebook.com/groups/votingwhilefemale/}{\emph{Voting
While Female}}\emph{.}

Advertisement

\protect\hyperlink{after-bottom}{Continue reading the main story}

\hypertarget{site-index}{%
\subsection{Site Index}\label{site-index}}

\hypertarget{site-information-navigation}{%
\subsection{Site Information
Navigation}\label{site-information-navigation}}

\begin{itemize}
\tightlist
\item
  \href{https://help.nytimes.com/hc/en-us/articles/115014792127-Copyright-notice}{©~2020~The
  New York Times Company}
\end{itemize}

\begin{itemize}
\tightlist
\item
  \href{https://www.nytco.com/}{NYTCo}
\item
  \href{https://help.nytimes.com/hc/en-us/articles/115015385887-Contact-Us}{Contact
  Us}
\item
  \href{https://www.nytco.com/careers/}{Work with us}
\item
  \href{https://nytmediakit.com/}{Advertise}
\item
  \href{http://www.tbrandstudio.com/}{T Brand Studio}
\item
  \href{https://www.nytimes.com/privacy/cookie-policy\#how-do-i-manage-trackers}{Your
  Ad Choices}
\item
  \href{https://www.nytimes.com/privacy}{Privacy}
\item
  \href{https://help.nytimes.com/hc/en-us/articles/115014893428-Terms-of-service}{Terms
  of Service}
\item
  \href{https://help.nytimes.com/hc/en-us/articles/115014893968-Terms-of-sale}{Terms
  of Sale}
\item
  \href{https://spiderbites.nytimes.com}{Site Map}
\item
  \href{https://help.nytimes.com/hc/en-us}{Help}
\item
  \href{https://www.nytimes.com/subscription?campaignId=37WXW}{Subscriptions}
\end{itemize}
