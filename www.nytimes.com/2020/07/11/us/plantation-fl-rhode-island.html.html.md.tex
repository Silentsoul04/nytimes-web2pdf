Sections

SEARCH

\protect\hyperlink{site-content}{Skip to
content}\protect\hyperlink{site-index}{Skip to site index}

\href{https://www.nytimes.com/section/us}{U.S.}

\href{https://myaccount.nytimes.com/auth/login?response_type=cookie\&client_id=vi}{}

\href{https://www.nytimes.com/section/todayspaper}{Today's Paper}

\href{/section/us}{U.S.}\textbar{}`Not a Welcoming Name': Calls to Drop
`Plantation' Gain Steam Nationwide

\url{https://nyti.ms/302cMTE}

\begin{itemize}
\item
\item
\item
\item
\item
\end{itemize}

\href{https://www.nytimes.com/news-event/george-floyd-protests-minneapolis-new-york-los-angeles?action=click\&pgtype=Article\&state=default\&region=TOP_BANNER\&context=storylines_menu}{Race
and America}

\begin{itemize}
\tightlist
\item
  \href{https://www.nytimes.com/2020/07/26/us/protests-portland-seattle-trump.html?action=click\&pgtype=Article\&state=default\&region=TOP_BANNER\&context=storylines_menu}{Protesters
  Return to Other Cities}
\item
  \href{https://www.nytimes.com/2020/07/24/us/portland-oregon-protests-white-race.html?action=click\&pgtype=Article\&state=default\&region=TOP_BANNER\&context=storylines_menu}{Portland
  at the Center}
\item
  \href{https://www.nytimes.com/2020/07/23/podcasts/the-daily/portland-protests.html?action=click\&pgtype=Article\&state=default\&region=TOP_BANNER\&context=storylines_menu}{Podcast:
  Showdown in Portland}
\item
  \href{https://www.nytimes.com/interactive/2020/07/16/us/black-lives-matter-protests-louisville-breonna-taylor.html?action=click\&pgtype=Article\&state=default\&region=TOP_BANNER\&context=storylines_menu}{45
  Days in Louisville}
\end{itemize}

Advertisement

\protect\hyperlink{after-top}{Continue reading the main story}

Supported by

\protect\hyperlink{after-sponsor}{Continue reading the main story}

\hypertarget{not-a-welcoming-name-calls-to-drop-plantation-gain-steam-nationwide}{%
\section{`Not a Welcoming Name': Calls to Drop `Plantation' Gain Steam
Nationwide}\label{not-a-welcoming-name-calls-to-drop-plantation-gain-steam-nationwide}}

Across the country, people are pushing to change the names of cities,
neighborhoods and gated communities that include the word
``plantation.''

\includegraphics{https://static01.nyt.com/images/2020/07/08/multimedia/00xp-plantation/merlin_174367305_dbfc0e40-989b-4b3e-b29b-a5ae20ad939c-articleLarge.jpg?quality=75\&auto=webp\&disable=upscale}

\href{https://www.nytimes.com/by/sandra-e-garcia}{\includegraphics{https://static01.nyt.com/images/2020/07/10/reader-center/author-sandra-e-garcia/author-sandra-e-garcia-thumbLarge.png}}

By \href{https://www.nytimes.com/by/sandra-e-garcia}{Sandra E. Garcia}

\begin{itemize}
\item
  July 11, 2020
\item
  \begin{itemize}
  \item
  \item
  \item
  \item
  \item
  \end{itemize}
\end{itemize}

When Dharyl Auguste was 3 years old, he and his parents packed all of
their belongings and left their home in Port-au-Prince, Haiti, to
immigrate to the United States.

The family settled initially in Fort Lauderdale, Fla., before moving to
nearby Sunrise. When it was time for Mr. Auguste to attend middle
school, he and his parents relocated again, this time to Plantation,
Fla. Mr. Auguste welcomed the move, he said, because it was easier for
him to see his friends and access public transportation.

But something was not right in Plantation.

``It often came up as a topic between me and friends, and we all had the
same feeling that it's not a welcoming name,'' Mr. Auguste, 27, said.

In the weeks since the George Floyd protests began, neighborhoods and
subdivisions across the country have removed the word ``plantation''
from their names. In June, Rhode Island --- known formally as the State
of Rhode Island and Providence Plantations --- announced that it would
\href{https://www.nytimes.com/2020/06/23/us/politics/rhode-island-plantation.html}{drop
the second half of its official name from state documents and websites}.
(State lawmakers have introduced legislation that would put a
name-change referendum on the ballot in November.)

Inspired by the social unrest spurred by the death of Mr. Floyd, a Black
man who died after a Minneapolis police officer pinned him to the ground
by the neck for
\href{https://www.nytimes.com/2020/06/18/us/george-floyd-timing.html}{more
than eight minutes}, Mr. Auguste
\href{https://www.change.org/p/ron-desantis-changing-the-name-of-the-city-of-plantation}{started
a petition} to change the name of Plantation.

\includegraphics{https://static01.nyt.com/images/2020/07/09/multimedia/00xp-plantation-auguste/00xp-plantation-auguste-articleLarge.jpg?quality=75\&auto=webp\&disable=upscale}

``I was at home sitting in awe as our nation was going through a social
awakening,'' he said in an interview this week. According to Mr.
Auguste, images of toppled monuments to slaveholders and Confederate
generals fueled him to take action. The petition he created on June 7
has now been signed more than 11,000 times.

Strictly speaking, the word
\href{https://www.merriam-webster.com/dictionary/plantation}{``plantation''}
refers to a large group of plants or trees in a settlement. But the
association with slavery is inescapable.

``We can't ignore the image conjured by the word `plantation,''' Gov.
Gina Raimondo of Rhode Island
\href{https://www.nytimes.com/2020/06/23/us/politics/rhode-island-plantation.html}{said
last month}. ``We can't ignore how painful that is for Black Rhode
Islanders to see that and have to see that as part of their state's
name. It's demoralizing. It's a slap in the face. It's painful.''

Gabriela Koster, who moved to Plantation, Fla., in 2006, agrees.

``I have been saying for 15 years that I do not think it's an
appropriate name for our city,'' Ms. Koster said. ``I don't think it
serves us well.''

Ms. Koster, 42, who raised her three children in Plantation, described
the city as vibrant but said its name dulled some of the city's luster.

But Lynn Stoner, the mayor of Plantation, does not necessarily share
this opinion.

``If we change the name, it doesn't change the mind-set of what people
indicate the problem is,'' Ms. Stoner said. ``I think it is just the
optics.''

Ms. Stoner has lived in Plantation for 50 years, and she proposed
instead that residents be educated on the ``racial components and
diversity in the community.''

``I'm more about the education piece,'' Ms. Stoner said at a City
Council
\href{http://plantation.granicus.com/player/clip/575?view_id=2}{meeting}
on July 1, during which she also suggested that residents be taught
about what should be considered offensive and why. ``I feel like
changing the name doesn't change the philosophies --- I think that's
where the bigger issue is.''

At the meeting, Ms. Stoner criticized
\href{https://www.facebook.com/DharylAuguste/videos/10220288815443473}{an
interview} that Mr. Auguste had recently given on CNN, saying that ``he
didn't do real well.'' (She later apologized.) She also asked Mr.
Auguste during the meeting whether she should use the term
``African-Americans'' or ``Blacks''; claimed that the first time she
``ever really saw'' Black people was when she moved to Plantation; and
said that the last three people she had hired were not white.

She added that she was taught to treat everyone equally.

In response to Ms. Stoner's comments, Mr. Auguste told the mayor that
just because the city's name represented the status quo it did not mean
it should stay that way.

``I'm sure that was the same mentality when slavery was ended,'' Mr.
Auguste said. ``We have to be more than not racist --- we have to be
antiracist.''

Because the City of Plantation
\href{http://www.plantation.org/Plantation/history.html}{wasn't
incorporated until 1953}, many --- including Ms. Stoner --- believe that
its name is exempt from the correlation with slavery.

``This isn't just about Black people,'' the mayor said in an interview
on Tuesday. ``It is about how Black people and people from other
countries all relate to each other.''

Image

``I have been saying for 15 years that I do not think it's an
appropriate name for our city,'' said Gabriela Koster, who raised three
children in Plantation, Fla.Credit...Maria Alejandra Cardona for The New
York Times

Across the country, people are working to change the names of
neighborhoods, developments and subdivisions that include the word
``plantation.'' In Hilton Head, S.C., efforts to change the names of
gated communities and resorts are unequivocally about Black people.
Beaufort County, which includes the island of Hilton Head, was founded
in 1711. Before the Civil War, there were more than
\href{https://www.hiltonheadislandsc.gov/ourisland/history.cfm}{20
plantations} on the island where slave labor produced cotton, indigo,
sugar cane, rice and other crops, according to the local government.

Today, Hilton Head is a resort town with developments and gated
communities whose names often have the word ``plantation'' in them.

``It has been co-opted to mean a gated community in the area,'' said
Marisa Wojcikiewicz, who started
\href{https://www.change.org/p/beaufort-county-s-gated-communities-should-not-be-called-plantations?utm_source=share_petition\&utm_medium=custom_url\&recruited_by_id=7ef37120-b81e-0130-02cd-002219670981}{a
petition} last month to change the names of the resorts and gated
communities. ``It is very strange, to say the least, considering that
the island is inextricably linked to the plantation economy.''

According to Ms. Wojcikiewicz, whose petition currently has over 8,000
signatures, a manager of the Hilton Head Plantation development had not
entirely shot down the idea of changing the name of the development. Ms.
Wojcikiewicz said she was surprised to find that some residents of the
developments, who are mostly white, older and affluent, supported
changing the name.

Peter Kristian, general manager of the Hilton Head Plantation property
owners' association, did not respond to requests for comment on Friday.

In Plantation, Fla., Mr. Auguste has two options to get the city to
change its name. The City Council can vote to have a referendum added to
the November ballot for the name change or Mr. Auguste can go door to
door to collect signatures from at least 10 percent of the city's 94,000
residents, which would compel a City Council review. In Hilton Head,
because the developments and resorts are privately owned, the onus is on
the owners and investors to make any name changes.

Most people don't want to be told that something they are doing is
wrong, according to Ms. Wojcikiewicz, particularly when they have never
given any thought to how it might be hurtful.

``Many people are afraid to admit that they were blind to the fact that
it is racist,'' she said. ``They think a plantation is this beautiful,
expansive, green, calm, Southern idyllic life that everyone wishes they
could have. We have deluded ourselves.''

Advertisement

\protect\hyperlink{after-bottom}{Continue reading the main story}

\hypertarget{site-index}{%
\subsection{Site Index}\label{site-index}}

\hypertarget{site-information-navigation}{%
\subsection{Site Information
Navigation}\label{site-information-navigation}}

\begin{itemize}
\tightlist
\item
  \href{https://help.nytimes.com/hc/en-us/articles/115014792127-Copyright-notice}{©~2020~The
  New York Times Company}
\end{itemize}

\begin{itemize}
\tightlist
\item
  \href{https://www.nytco.com/}{NYTCo}
\item
  \href{https://help.nytimes.com/hc/en-us/articles/115015385887-Contact-Us}{Contact
  Us}
\item
  \href{https://www.nytco.com/careers/}{Work with us}
\item
  \href{https://nytmediakit.com/}{Advertise}
\item
  \href{http://www.tbrandstudio.com/}{T Brand Studio}
\item
  \href{https://www.nytimes.com/privacy/cookie-policy\#how-do-i-manage-trackers}{Your
  Ad Choices}
\item
  \href{https://www.nytimes.com/privacy}{Privacy}
\item
  \href{https://help.nytimes.com/hc/en-us/articles/115014893428-Terms-of-service}{Terms
  of Service}
\item
  \href{https://help.nytimes.com/hc/en-us/articles/115014893968-Terms-of-sale}{Terms
  of Sale}
\item
  \href{https://spiderbites.nytimes.com}{Site Map}
\item
  \href{https://help.nytimes.com/hc/en-us}{Help}
\item
  \href{https://www.nytimes.com/subscription?campaignId=37WXW}{Subscriptions}
\end{itemize}
