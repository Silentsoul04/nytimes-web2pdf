Sections

SEARCH

\protect\hyperlink{site-content}{Skip to
content}\protect\hyperlink{site-index}{Skip to site index}

\href{https://www.nytimes.com/section/world/asia}{Asia Pacific}

\href{https://myaccount.nytimes.com/auth/login?response_type=cookie\&client_id=vi}{}

\href{https://www.nytimes.com/section/todayspaper}{Today's Paper}

\href{/section/world/asia}{Asia Pacific}\textbar{}Defying U.S., China
and Iran Near Trade and Military Partnership

\url{https://nyti.ms/3iURNuM}

\begin{itemize}
\item
\item
\item
\item
\item
\end{itemize}

Advertisement

\protect\hyperlink{after-top}{Continue reading the main story}

Supported by

\protect\hyperlink{after-sponsor}{Continue reading the main story}

\hypertarget{defying-us-china-and-iran-near-trade-and-military-partnership}{%
\section{Defying U.S., China and Iran Near Trade and Military
Partnership}\label{defying-us-china-and-iran-near-trade-and-military-partnership}}

The investment and security pact would vastly extend China's influence
in the Middle East, throwing Iran an economic lifeline and creating new
flash points with the United States.

\includegraphics{https://static01.nyt.com/images/2020/07/12/world/12China-Iran1-print/merlin_103791543_51454d73-22ce-440c-bc70-0389f9207c57-articleLarge.jpg?quality=75\&auto=webp\&disable=upscale}

By \href{https://www.nytimes.com/by/farnaz-fassihi}{Farnaz Fassihi} and
\href{https://www.nytimes.com/by/steven-lee-myers}{Steven Lee Myers}

\begin{itemize}
\item
  Published July 11, 2020Updated July 22, 2020
\item
  \begin{itemize}
  \item
  \item
  \item
  \item
  \item
  \end{itemize}
\end{itemize}

\href{https://cn.nytimes.com/world/20200713/china-iran-trade-military-deal/}{阅读简体中文版}\href{https://cn.nytimes.com/world/20200713/china-iran-trade-military-deal/zh-hant/}{閱讀繁體中文版}

Iran and China have quietly drafted a sweeping economic and security
partnership that would clear the way for billions of dollars of Chinese
investments in energy and other sectors, undercutting the Trump
administration's efforts to isolate the Iranian government because of
its nuclear and military ambitions.

The partnership, detailed in an 18-page proposed agreement obtained by
The New York Times, would vastly expand Chinese presence in banking,
telecommunications, ports, railways and dozens of other projects. In
exchange, China would receive a regular --- and, according to an Iranian
official and an oil trader, heavily discounted --- supply of Iranian oil
over the next 25 years.

The document also describes deepening military cooperation, potentially
giving China a foothold in a region that has been a strategic
preoccupation of the United States for decades. It calls for joint
training and exercises, joint research and weapons development and
intelligence sharing --- all to fight ``the lopsided battle with
terrorism, drug and human trafficking and cross-border crimes.''

The partnership --- first proposed by China's leader, Xi Jinping, during
\href{https://www.nytimes.com/2016/01/31/world/asia/xi-jinping-visits-saudi-iran.html}{a
visit to Iran}in 2016 --- was approved by President Hassan Rouhani's
cabinet in June, Iran's foreign minister, Mohammad Javad Zarif, said
last week.

Iranian officials have publicly stated that there is a pending agreement
with China, and one Iranian official, as well as several people who have
discussed it with the Iranian government, confirmed that it is the
document obtained by The Times, which is labeled ``final version'' and
dated June 2020.

It has not yet been submitted to Iran's Parliament for approval or made
public, stoking suspicions in Iran about how much the government is
preparing to give away to China.

In Beijing, officials have not disclosed the terms of the agreement, and
it is not clear whether Mr. Xi's government has signed off or, if it
has, when it might announce it.

If put into effect as detailed, the partnership would create new and
potentially dangerous flash points in the deteriorating relationship
between China and the United States.

It represents a major blow to the Trump administration's
\href{https://www.nytimes.com/2020/01/08/world/middleeast/trump-iran-nuclear-sanctions.html}{aggressive
policy toward Iran} since abandoning the nuclear deal reached in 2015 by
President Barack Obama and the leaders of six other nations after two
years of grueling negotiations.

Renewed American sanctions, including the threat to cut off access to
the international banking system for any company that does business in
Iran, have succeeded in suffocating the Iranian economy by scaring away
badly needed foreign trade and investment.

\includegraphics{https://static01.nyt.com/images/2020/07/12/world/12China-Iran2-print/merlin_172517184_e76b7ef0-f99a-4328-a2a7-9cb8459f4ad7-articleLarge.jpg?quality=75\&auto=webp\&disable=upscale}

But Tehran's desperation has pushed it into the arms of China, which has
the technology and appetite for oil that Iran needs. Iran has been one
of the world's largest oil producers, but its exports, Tehran's largest
source of revenue, have plunged since the Trump administration began
imposing sanctions in 2018; China gets about 75 percent of its oil from
abroad and is the
\href{https://www.eia.gov/todayinenergy/detail.php?id=43216}{world's
largest importer}, at more than 10 million barrels a day last year.

At a time when the United States is reeling from recession and the
coronavirus, and increasingly isolated internationally, Beijing senses
American weakness. The draft agreement with Iran shows that unlike most
countries, China feels it is in a position to defy the United States,
powerful enough to withstand American penalties, as it has in the trade
war waged by President Trump.

``Two ancient Asian cultures, two partners in the sectors of trade,
economy, politics, culture and security with a similar outlook and many
mutual bilateral and multilateral interests will consider one another
strategic partners,'' the document says in its opening sentence.

The Chinese investments in Iran, which two people who have been briefed
on the deal said would total \$400 billion over 25 years, could spur
still more punitive actions against Chinese companies, which have
\href{https://www.nytimes.com/2019/07/22/world/asia/sanctions-china-iran-oil.html}{already
been targeted}by the administration in recent months.

``The United States will continue to impose costs on Chinese companies
that aid Iran, the world's largest state sponsor of terrorism,'' a State
Department spokeswoman wrote in response to questions about the draft
agreement.

``By allowing or encouraging Chinese companies to conduct sanctionable
activities with the Iranian regime, the Chinese government is
undermining its own stated goal of promoting stability and peace.''

The expansion of military assistance, training and intelligence-sharing
will also be viewed with alarm in Washington. American warships already
tangle regularly with Iranian forces in the crowded waters of the
Persian Gulf and challenge China's internationally disputed claim to
\href{https://www.nytimes.com/2020/07/04/us/politics/south-china-sea-aircraft-carrier.html}{much
of the South China Sea}, and the Pentagon's
\href{https://www.nytimes.com/2018/01/19/us/politics/military-china-russia-terrorism-focus.html}{national
security strategy} has declared China an adversary.

When reports of a long-term investment agreement with Iran surfaced last
September, China's foreign ministry dismissed the question out of hand.
Asked about it again last week, a spokesman, Zhao Lijian,
\href{https://www.fmprc.gov.cn/mfa_eng/xwfw_665399/s2510_665401/2511_665403/t1795337.shtml}{left
open} the possibility that a deal was in the works.

Image

A tanker carrying crude oil imported from Iran at the Port of Zhoushan,
China, in 2018.~Credit...Imaginechina, via Associated Press

``China and Iran enjoy traditional friendship, and the two sides have
been in communication on the development of bilateral relations,'' he
said. ``We stand ready to work with Iran to steadily advance practical
cooperation.''

The projects --- nearly 100 are cited in the draft agreement --- are
very much in keeping with Mr. Xi's ambitions to extend its economic and
strategic influence across Eurasia through the ``Belt and Road
Initiative,'' a vast aid and investment program.

The projects, including airports, high-speed railways and subways, would
touch the lives of millions of Iranians. China would develop free-trade
zones in Maku, in northwestern Iran; in Abadan, where the Shatt al-Arab
river flows into the Persian Gulf, and on the gulf island Qeshm.

The agreement also includes proposals for China to build the
infrastructure for a 5G telecommunications network, to offer the new
Chinese Global Positioning System, Beidou, and to help Iranian
authorities assert greater control over what circulates in cyberspace,
presumably as China's Great Firewall does.

The American campaign against a major Chinese telecommunications
company, Huawei, includes
\href{https://www.nytimes.com/2019/01/28/us/politics/meng-wanzhou-huawei-iran.html}{a
criminal case}against its chief financial officer, Meng Wanzhou, for
seeking to disguise investments in Iran in order to evade American
sanctions. The Trump administration has barred Huawei from involvement
in 5G development in the United States, and has tried, without great
success,
\href{https://www.nytimes.com/2020/02/17/us/politics/us-huawei-5g.html}{to
persuade other countries} to do the same.

Moving ahead with a broad investment program in Iran appears to signal
Beijing's growing impatience with the Trump administration after its
abandonment of the nuclear agreement. China has repeatedly called on the
administration to preserve the deal, which it was a party to, and has
sharply denounced the American use of unilateral sanctions.

Iran has traditionally looked west toward Europe for trade and
investment partners. Increasingly though, it has grown frustrated with
European countries that have opposed Mr. Trump's policy but quietly
withdrawn from the kinds of deals that the nuclear agreement once
promised.

Image

President Trump withdrew the United States from the Iran nuclear deal in
2018.Credit...Doug Mills/The New York Times

``Iran and China both view this deal as a strategic partnership in not
just expanding their own interests but confronting the U.S.,'' said Ali
Gholizadeh, an Iranian energy researcher at the University of Science
and Technology of China in Hefei. ``It is the first of its kind for Iran
keen on having a world power as an ally.''

The proposed partnership has nonetheless stoked a fierce debate within
Iran. Mr. Zarif, the foreign minister, who traveled to Beijing last
October to negotiate the agreement, faced hostile questioning about it
in Parliament last week.

The document was provided to The Times by someone familiar with its
drafting with the intention of showing the scope of the projects now
under consideration.

Mr. Zarif said the agreement would be submitted to Parliament for final
approval. It has the support of Iran's supreme leader, Ayatollah Ali
Khamenei, two Iranian officials said.

Ayatollah Khamenei's top economic adviser, Ali Agha Mohammadi, appeared
on state television recently to discuss the need for an economic
lifeline. He said Iran needs to increase its oil production to at least
8.5 million barrels a day in order to remain a player in the energy
market, and for that, it needs China.

Iranian supporters of the strategic partnership say that given the
country's limited economic options, the free-falling currency and the
dim prospect of U.S. sanctions being lifted, the deal with China could
provide a lifeline.

``Every road is closed to Iran,'' said Fereydoun Majlesi, a former
diplomat and a columnist for several Iranian newspapers on diplomacy.
``The only path open is China. Whatever it is, until sanctions are
lifted, this deal is the best option.''

But critics across the political spectrum in Iran have raised concerns
that the government is secretly ``selling off'' the country to China in
a moment of economic weakness and international isolation. In a speech
in late June, a former president, Mahmoud Ahmadinejad, called it a
suspicious secret deal that the people of Iran would never approve.

The critics have cited previous Chinese investment projects that have
left
\href{https://www.nytimes.com/2018/10/15/world/africa/kenya-china-racism.html}{countries
in Africa} and Asia
\href{https://www.nytimes.com/2018/06/25/world/asia/china-sri-lanka-port.html}{indebted
and ultimately beholden} to the authorities in Beijing. A particular
concern has been the proposed port facilities in Iran, including two
along the coast of the Sea of Oman.

One at Jask, just outside of the Strait of Hormuz, the entrance to the
Persian Gulf, would give the Chinese a strategic vantage point on the
waters through which much of the world's oil transits. The passage is of
critical strategic importance to the United States, whose Navy's Fifth
Fleet is headquartered in Bahrain, in the gulf.

Image

Jask, located at the entrance to the Persian Gulf, would give the
Chinese a strategic vantage point on the waters through which much of
the world's oil transits.Credit...Orbital Horizon/Gallo Images, via
Getty Images

China has already constructed a series of ports along the Indian Ocean,
creating a necklace of refueling and resupply stations from the South
China Sea to the Suez Canal. Ostensibly commercial in nature, the ports
potentially have military value, too, allowing China's
\href{https://www.nytimes.com/2018/08/29/world/asia/china-navy-aircraft-carrier-pacific.html}{rapidly
growing navy} to expand its reach.

Those include ports at Hambantota in Sri Lanka and Gwadar in Pakistan,
which are widely criticized as footholds for a potential military
presence, though no Chinese forces have officially been deployed at
them.

China opened its first overseas military base in Djibouti in 2015,
ostensibly to support its forces participating in international
antipiracy operations off the coast of Somalia. The outpost, which began
as a logistics base but is now more heavily fortified, is within miles
of the American base in that country.

China has also stepped up military cooperation with Iran. The People's
Liberation Army Navy has visited and participated in military exercises
at least three times, beginning in 2014. The most recent was last
December, when a Chinese missile destroyer, the Xining, joined a naval
exercise with the Russian and Iranian navies in the Gulf of Oman.

China's state-owned Xinhua news agency quoted the commander of Iran's
Navy, Rear Adm. Hossein Khanzadi, saying that the exercise showed ``the
era of American invasions in the region is over.''

David E. Sanger contributed reporting. Claire Fu in Beijing contributed
research.

Advertisement

\protect\hyperlink{after-bottom}{Continue reading the main story}

\hypertarget{site-index}{%
\subsection{Site Index}\label{site-index}}

\hypertarget{site-information-navigation}{%
\subsection{Site Information
Navigation}\label{site-information-navigation}}

\begin{itemize}
\tightlist
\item
  \href{https://help.nytimes.com/hc/en-us/articles/115014792127-Copyright-notice}{©~2020~The
  New York Times Company}
\end{itemize}

\begin{itemize}
\tightlist
\item
  \href{https://www.nytco.com/}{NYTCo}
\item
  \href{https://help.nytimes.com/hc/en-us/articles/115015385887-Contact-Us}{Contact
  Us}
\item
  \href{https://www.nytco.com/careers/}{Work with us}
\item
  \href{https://nytmediakit.com/}{Advertise}
\item
  \href{http://www.tbrandstudio.com/}{T Brand Studio}
\item
  \href{https://www.nytimes.com/privacy/cookie-policy\#how-do-i-manage-trackers}{Your
  Ad Choices}
\item
  \href{https://www.nytimes.com/privacy}{Privacy}
\item
  \href{https://help.nytimes.com/hc/en-us/articles/115014893428-Terms-of-service}{Terms
  of Service}
\item
  \href{https://help.nytimes.com/hc/en-us/articles/115014893968-Terms-of-sale}{Terms
  of Sale}
\item
  \href{https://spiderbites.nytimes.com}{Site Map}
\item
  \href{https://help.nytimes.com/hc/en-us}{Help}
\item
  \href{https://www.nytimes.com/subscription?campaignId=37WXW}{Subscriptions}
\end{itemize}
