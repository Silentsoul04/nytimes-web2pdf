Sections

SEARCH

\protect\hyperlink{site-content}{Skip to
content}\protect\hyperlink{site-index}{Skip to site index}

\href{https://www.nytimes.com/section/world/asia}{Asia Pacific}

\href{https://myaccount.nytimes.com/auth/login?response_type=cookie\&client_id=vi}{}

\href{https://www.nytimes.com/section/todayspaper}{Today's Paper}

\href{/section/world/asia}{Asia Pacific}\textbar{}Caught Between Indian
and Chinese Troops, at 15,000 Feet

\url{https://nyti.ms/2WpmY7Z}

\begin{itemize}
\item
\item
\item
\item
\item
\end{itemize}

Advertisement

\protect\hyperlink{after-top}{Continue reading the main story}

Supported by

\protect\hyperlink{after-sponsor}{Continue reading the main story}

\hypertarget{caught-between-indian-and-chinese-troops-at-15000-feet}{%
\section{Caught Between Indian and Chinese Troops, at 15,000
Feet}\label{caught-between-indian-and-chinese-troops-at-15000-feet}}

Ladakh is a place of high-altitude pastures, lonely herders and desolate
moonscapes --- and a byword for intractable geopolitical disputes.

\includegraphics{https://static01.nyt.com/images/2020/07/12/world/00india-china-border1/merlin_173696250_3e3803ea-1d5f-46f1-9613-73c7b2d23755-articleLarge.jpg?quality=75\&auto=webp\&disable=upscale}

\href{https://www.nytimes.com/by/jeffrey-gettleman}{\includegraphics{https://static01.nyt.com/images/2018/10/10/multimedia/author-jeffrey-gettleman/author-jeffrey-gettleman-thumbLarge.png}}

By \href{https://www.nytimes.com/by/jeffrey-gettleman}{Jeffrey
Gettleman}

\begin{itemize}
\item
  July 11, 2020
\item
  \begin{itemize}
  \item
  \item
  \item
  \item
  \item
  \end{itemize}
\end{itemize}

\href{https://cn.nytimes.com/world/20200713/india-china-border-ladakh/}{阅读简体中文版}\href{https://cn.nytimes.com/world/20200713/india-china-border-ladakh/zh-hant/}{閱讀繁體中文版}

NEW DELHI --- First it was cellphone towers, new roads and surveillance
cameras, popping up on the Chinese side of the disputed Himalayan border
with India.

Then it was more run-ins between troops on each side, pushing, shoving
and eventually getting into fistfights.

Then, about three years ago, Indian soldiers spotted their Chinese foes
carrying iron bars with little numbers written on them --- a weapon
apparently issued as standard gear, and a sign that the Chinese were
gearing up for hand-to-hand combat.

``This is how China operates,'' said J.P. Yadav, a recently retired
official with the Indo-Tibetan Border Police, on the Indian side.
``These are very planned things.''

Now, weeks after
\href{https://www.nytimes.com/2020/06/16/world/asia/indian-china-border-clash.html}{a
deadly brawl erupted along the border}, thousands of Chinese and Indian
troops are amassed over a contentious, jagged line in one of the most
remote places on earth. Satellite photos reveal a major Chinese buildup,
including a blizzard of new tents, new storage sheds, artillery pieces
and even tanks.

\includegraphics{https://static01.nyt.com/images/2020/07/12/world/00india-china-border2-sub/merlin_173694795_a38e35e9-b29c-4c84-b9d5-c09a9dcace82-articleLarge.jpg?quality=75\&auto=webp\&disable=upscale}

Each country has accused the other of provocative actions along the
murky border. But according to people who live and work in the region,
Ladakh, a Chinese push into Indian territory has been building for
years.

The area, high up in the Himalayas, has little obvious strategic value,
few resources and few people --- it's difficult to even breathe up
there, with much of the terrain above 15,000 feet. But India and China,
\href{https://www.nytimes.com/2020/06/17/world/asia/china-india-border.html}{both
in the grip of increasingly nationalistic governments}, will not give an
inch of territory, even along a border so remote that it has never been
conclusively mapped.

The Ladakhis caught in between are a fragile group, numbering perhaps a
few hundred thousand. They are Tibetan in culture, identify themselves
as Indian and have long been pulled in different directions at the edges
of empire.

``If we don't speak now, it will be too late,'' said Rigzin Spalbar, a
Ladakhi politician. ``The Chinese have intruded and encroached on our
land. Even the media is not telling the truth. They are only showing the
things that the government wants to them to show.''

Image

Herding sheep along the Srinagar-Leh Highway last month.Credit...Tauseef
Mustafa/Agence France-Presse --- Getty Images

Mr. Spalbar and other prominent Ladakhis insist that they have reported
Chinese incursions for years, but that the Indian military refused to do
anything about it. They say there was a code of silence, in which the
Indian media was complicit, and that the Indian armed forces didn't want
to face the fact that a more powerful and aggressive military was
steadily nibbling away at its territory.

Indian Army officials declined to comment for this article. Chinese
officials have been stingy with details as well, including about whether
any Chinese troops were killed in the clash in June. Western
intelligence agents, who see the border as one of Asia's most dangerous
flash points, say they think that China lost more than a dozen soldiers
in the fight.

In early July, Prime Minister Narendra Modi of India swooped into
Ladakh, rallying the troops while wearing a puffy green army jacket and
aviator-style shades.

``\href{https://www.youtube.com/watch?v=MAb4JWywUAA\&feature=youtu.be}{Friends,''
he vowed, ``the era of expansionism is over,''} implying that India was
willing to push back against China.

Years ago, the two countries agreed that their troops should not shoot
at each other during border standoffs. But the Chinese seem to be
testing the limits. In the June fighting, which left 20 Indians and an
unknown number of Chinese dead, Indian commanders say that Chinese
troops used iron clubs bristling with spikes.

Many analysts say that China's actions in Ladakh mirror the more
assertive approach China has taken across Asia, especially in the South
China Sea, since its leader,
\href{https://www.nytimes.com/2013/03/15/world/asia/chinas-new-leader-xi-jinping-takes-full-power.html}{Xi
Jinping, took over in 2012}.

Image

India's Prime Minister Narendra Modi, center, during a visit to Ladakh
last week.Credit...Indian Press Information Bureau

And
\href{https://www.nytimes.com/2020/06/17/world/asia/china-india-border.html}{Mr.
Modi's brand of renewed Indian nationalism}may also have provoked the
Chinese. The Indians, too, have also been building military roads along
the disputed border, known as the Line of Actual Control. And Indian
officials recently promised to take back Aksai Chin, a high-altitude
plateau that India says is part of Ladakh but that China controls and
claims as its own.

Aksai Chin is ``a very important strategic place'' to the Chinese
military, said Yue Gang, a retired colonel in the People's Liberation
Army. If India were to seize it, he said, it ``would cut the
transportation between Tibet and Xinjiang,'' two restive areas that
China is constantly concerned about.

In culture, language, history and Buddhism, Ladakh is close to Tibet.
But Ladakhi scholars are firm about one thing: They don't want to be
part of China.

``Ladakhis see themselves as Indians,'' said Sonam Joldan, a Ladakhi
political scientist.

Up until a few years ago, Ladakhi and Tibetan nomads roamed freely,
pushing their herds of goats, sheep and yak across the lonely,
high-altitude plains. They used to converge along a stretch of the Line
of Actual Control and barter.

The Ladakhis carried Indian products like basmati rice; the Tibetans
brought Chinese-made goods like plastic Thermoses. The trading sessions
ended, Ladakhis say, after
\href{https://www.indiatoday.in/india/north/story/chinese-army-occupied-640-square-km-three-ladakh-sectors-report-209992-2013-09-05}{Chinese
troops occupied the area}.

Image

Ladakhi students wearing traditional costume waved Indian flags during a
parade in Leh last year.Credit...Paula Bronstein/Getty Images

This is hardly the first time Ladakh has been swept up into geopolitics.

In the mid-19th century, the British helped set up the princely state of
Jammu and Kashmir, which seems to stretch endlessly across the
Himalayas. The British, who controlled the Indian subcontinent, believed
that the bigger the buffer zone against the Russian empire, the better.

So they allowed the maharajah of Jammu and Kashmir to also grab
neighboring Ladakh, enabling him to corner the lucrative trade in
pashmina wool. This part of Asia is known for its cashmere (the word for
which is derived from Kashmir), and Ladakh's longhaired Changthangi
goats produce especially fine pashmina.

But even after several treaties were signed, the border between Ladakh
and China was never neatly defined. It snakes across high mountains that
few people have ever climbed.

``There were different narratives during the British times,'' said
\href{https://thewire.in/diplomacy/global-dimension-india-china-confrontation-in-ladakh}{Siddiq
Wahid, a scholar of Central Asian history}. ``Aksai Chin was a part of
Tibet, and it was not a part of Tibet, it was part of Ladakh and not
part of Ladakh.''

Image

Buddhist monks blowing ceremonial horns inside the prayer hall at the
Thiksay monastery in Ladakh last year.Credit...Paula Bronstein/Getty
Images

Shortly after India gained independence in 1947 and Pakistan was
created, war erupted between the two countries over Jammu and Kashmir.
The princely state, which had hoped to stay independent, hurriedly
agreed to be part of India, and thus Ladakh became Indian.

In 1950, China invaded Tibet and soon built a road linking it to
Xinjiang, slicing through Aksai Chin. The area was so desolate that it
wasn't until several years later that India even found out about the
road. This triggered a brief war in 1962 that ended in a disastrous loss
for India, and China seized all of Aksai Chin, more than 14,000 square
miles.

By the mid-1970s, things had cooled down, at least on the China front. A
protocol evolved between Indian and Chinese troops, including a ban on
firearms during border standoffs and regular meetings to iron out
disputes.

Things were still hot with Pakistan, though. The same piece of
territory, Jammu and Kashmir, has propelled India into repeated
conflicts with both Pakistan and China --- two nations which, like
India, have nuclear arms today.

Indian soldiers who served along the China border in the 1980s and 1990s
remember friendly interactions with the Chinese troops.

``We used to shake hands and they would take photos with us and we would
take photos with them,'' said Sonam Murup, a retired officer.

Those visits with the Chinese were welcome distractions. Soldiers
stationed along the border had to tramp around a frozen moonscape for
weeks, with little food or water.

``We'd wash our face once maybe every 15 or 16 days,'' Mr. Murup
recalled.

Image

Indian soldiers walking in the foothills of a mountain range near Leh
last month.Credit...Tauseef Mustafa/Agence France-Presse --- Getty
Images

But the bonhomie with the Chinese ended years ago. Ladakhis say Chinese
troops have blocked herders' access to Indian territory in areas like
Demchok and Pangong Tso, a scenic lake where several brawls have
erupted.

Indian officers say they have tried to follow protocols for avoiding
confrontation, like unfurling banners that read ``This Is Indian
Territory'' in English and Chinese, but that the Chinese refuse to
listen. Indian commanders acknowledge that their soldiers, too, now
carry hand weapons, like bamboo sticks and sling shots.

The Chinese have clearly outpaced India in developing the region, Indian
commanders concede, which could give them a strategic advantage in a
conflict.

``They have better facilities,'' said Mr. Yadav, the former border
official. He said China had paved a highway running right along the
border and that Chinese border troops were resupplied by military
vehicles carrying supplemental oxygen.

But Mr. Yadav said the Indians had some advantages. He claimed the
Chinese troops were in poorer shape, saying, ``They don't walk much.''

More important, he added: ``They have not seen war, while on our side
our soldiers have been waging war every day in Kashmir.''

Hari Kumar and Sameer Yasir contributed reporting from New Delhi, Iqbal
Kirmani from Leh, India, and Steven Lee Myers from Seoul, South Korea.

Advertisement

\protect\hyperlink{after-bottom}{Continue reading the main story}

\hypertarget{site-index}{%
\subsection{Site Index}\label{site-index}}

\hypertarget{site-information-navigation}{%
\subsection{Site Information
Navigation}\label{site-information-navigation}}

\begin{itemize}
\tightlist
\item
  \href{https://help.nytimes.com/hc/en-us/articles/115014792127-Copyright-notice}{©~2020~The
  New York Times Company}
\end{itemize}

\begin{itemize}
\tightlist
\item
  \href{https://www.nytco.com/}{NYTCo}
\item
  \href{https://help.nytimes.com/hc/en-us/articles/115015385887-Contact-Us}{Contact
  Us}
\item
  \href{https://www.nytco.com/careers/}{Work with us}
\item
  \href{https://nytmediakit.com/}{Advertise}
\item
  \href{http://www.tbrandstudio.com/}{T Brand Studio}
\item
  \href{https://www.nytimes.com/privacy/cookie-policy\#how-do-i-manage-trackers}{Your
  Ad Choices}
\item
  \href{https://www.nytimes.com/privacy}{Privacy}
\item
  \href{https://help.nytimes.com/hc/en-us/articles/115014893428-Terms-of-service}{Terms
  of Service}
\item
  \href{https://help.nytimes.com/hc/en-us/articles/115014893968-Terms-of-sale}{Terms
  of Sale}
\item
  \href{https://spiderbites.nytimes.com}{Site Map}
\item
  \href{https://help.nytimes.com/hc/en-us}{Help}
\item
  \href{https://www.nytimes.com/subscription?campaignId=37WXW}{Subscriptions}
\end{itemize}
