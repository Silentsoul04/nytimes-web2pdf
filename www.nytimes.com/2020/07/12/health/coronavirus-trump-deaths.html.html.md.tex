Sections

SEARCH

\protect\hyperlink{site-content}{Skip to
content}\protect\hyperlink{site-index}{Skip to site index}

\href{https://www.nytimes.com/section/health}{Health}

\href{https://myaccount.nytimes.com/auth/login?response_type=cookie\&client_id=vi}{}

\href{https://www.nytimes.com/section/todayspaper}{Today's Paper}

\href{/section/health}{Health}\textbar{}Trump's Health Officials Warn
More Will Die as Covid Cases Rise

\url{https://nyti.ms/2C72k5t}

\begin{itemize}
\item
\item
\item
\item
\item
\end{itemize}

\href{https://www.nytimes.com/news-event/coronavirus?action=click\&pgtype=Article\&state=default\&region=TOP_BANNER\&context=storylines_menu}{The
Coronavirus Outbreak}

\begin{itemize}
\tightlist
\item
  live\href{https://www.nytimes.com/2020/08/04/world/coronavirus-covid-19.html?action=click\&pgtype=Article\&state=default\&region=TOP_BANNER\&context=storylines_menu}{Latest
  Updates}
\item
  \href{https://www.nytimes.com/interactive/2020/us/coronavirus-us-cases.html?action=click\&pgtype=Article\&state=default\&region=TOP_BANNER\&context=storylines_menu}{Maps
  and Cases}
\item
  \href{https://www.nytimes.com/interactive/2020/science/coronavirus-vaccine-tracker.html?action=click\&pgtype=Article\&state=default\&region=TOP_BANNER\&context=storylines_menu}{Vaccine
  Tracker}
\item
  \href{https://www.nytimes.com/2020/08/02/us/covid-college-reopening.html?action=click\&pgtype=Article\&state=default\&region=TOP_BANNER\&context=storylines_menu}{College
  Reopening}
\item
  \href{https://www.nytimes.com/live/2020/08/03/business/stock-market-today-coronavirus?action=click\&pgtype=Article\&state=default\&region=TOP_BANNER\&context=storylines_menu}{Economy}
\end{itemize}

Advertisement

\protect\hyperlink{after-top}{Continue reading the main story}

Supported by

\protect\hyperlink{after-sponsor}{Continue reading the main story}

\hypertarget{trumps-health-officials-warn-more-will-die-as-covid-cases-rise}{%
\section{Trump's Health Officials Warn More Will Die as Covid Cases
Rise}\label{trumps-health-officials-warn-more-will-die-as-covid-cases-rise}}

They struck a sober note on Sunday's news programs, strongly urging the
vast majority of people in hard-hit cities and states to wear masks and
avoid large gatherings.

\includegraphics{https://static01.nyt.com/images/2020/07/12/science/12virus-health/12virus-health-articleLarge.jpg?quality=75\&auto=webp\&disable=upscale}

\href{https://www.nytimes.com/by/pam-belluck}{\includegraphics{https://static01.nyt.com/images/2018/02/16/multimedia/author-pam-belluck/author-pam-belluck-thumbLarge-v2.png}}

By \href{https://www.nytimes.com/by/pam-belluck}{Pam Belluck}

\begin{itemize}
\item
  July 12, 2020
\item
  \begin{itemize}
  \item
  \item
  \item
  \item
  \item
  \end{itemize}
\end{itemize}

Two of the Trump administration's top health officials acknowledged
Sunday that the country is facing a very serious situation with the
onslaught of rising coronavirus cases in several states, striking a far
more sober tone than President Trump at this stage of the pandemic in
the United States.

Adm. Brett Giroir, an assistant secretary with the Health and Human
Services department, and Dr.
\href{https://www.nytimes.com/2020/07/21/us/politics/jerome-adams-surgeon-general-trump-coronavirus.html}{Jerome
Adams}, the surgeon general, both emphasized their concern about surging
outbreaks, many of them in areas where people have not followed
recommended public health guidelines to contain the spread of the virus.
Their remarks were in sharp contrast to Mr. Trump's contention just last
week that 99 percent of the cases were ``totally harmless'' and his
boast of the country's low death rate from the virus.

``We're all very concerned about the rise in cases, no doubt about
that,'' Admiral Giroir, the official who has been in charge of the
administration's coronavirus testing response, said on ABC's ``This
Week.''

``We do expect deaths to go up,'' he said. ``If you have more cases,
more hospitalizations, we do expect to see that over the next two or
three weeks before this turns around.''

Still, Admiral Giroir and Dr. Adams offered up a few optimistic notes.
Admiral Giroir said the percentage of positive test results was leveling
off, and both officials said that doctors had better tools to treat
people who become sick than they did at the start of the pandemic.

They steered clear of recommending widespread lockdowns in states with
heavy caseloads where hospitals are becoming overwhelmed. Instead, they
said those cities and states should consider closing bars and curtailing
mass social gatherings, and they strongly urged the vast majority of
people in those hard-hit areas to wear masks.

Masks have become a flash point in some areas of the country, especially
among members of Mr. Trump's political base. The president resisted
wearing a mask for months,
\href{https://www.washingtonpost.com/politics/trumps-mockery-of-wearing-masks-divides-republicans/2020/05/26/2c2bdc02-9f61-11ea-81bb-c2f70f01034b_story.html}{mocked
some people who did,} and only wore a mask in public for the first time
on Saturday during a visit to Walter Reed National Military Medical
Center.

\hypertarget{latest-updates-global-coronavirus-outbreak}{%
\section{\texorpdfstring{\href{https://www.nytimes.com/2020/08/04/world/coronavirus-covid-19.html?action=click\&pgtype=Article\&state=default\&region=MAIN_CONTENT_1\&context=storylines_live_updates}{Latest
Updates: Global Coronavirus
Outbreak}}{Latest Updates: Global Coronavirus Outbreak}}\label{latest-updates-global-coronavirus-outbreak}}

Updated 2020-08-04T09:15:14.275Z

\begin{itemize}
\tightlist
\item
  \href{https://www.nytimes.com/2020/08/04/world/coronavirus-covid-19.html?action=click\&pgtype=Article\&state=default\&region=MAIN_CONTENT_1\&context=storylines_live_updates\#link-6b644638}{`Long
  days, long nights': Washington prepares for a prolonged fight over
  virus relief.}
\item
  \href{https://www.nytimes.com/2020/08/04/world/coronavirus-covid-19.html?action=click\&pgtype=Article\&state=default\&region=MAIN_CONTENT_1\&context=storylines_live_updates\#link-7af9fca0}{Israel's
  rocky reopening of its schools may be a lesson for the U.S.}
\item
  \href{https://www.nytimes.com/2020/08/04/world/coronavirus-covid-19.html?action=click\&pgtype=Article\&state=default\&region=MAIN_CONTENT_1\&context=storylines_live_updates\#link-33bf9168}{Hurricane
  Isaias arrives in North Carolina as officials along the East Coast
  scramble.}
\end{itemize}

\href{https://www.nytimes.com/2020/08/04/world/coronavirus-covid-19.html?action=click\&pgtype=Article\&state=default\&region=MAIN_CONTENT_1\&context=storylines_live_updates}{See
more updates}

More live coverage:
\href{https://www.nytimes.com/live/2020/08/03/business/stock-market-today-coronavirus?action=click\&pgtype=Article\&state=default\&region=MAIN_CONTENT_1\&context=storylines_live_updates}{Markets}

``It's really essential to wear masks,'' Admiral Giroir said. ``We have
to have like 90 percent of people wearing the masks in public in the hot
spot areas. If we don't have that, we will not get control of the
virus.''

The host of ``This Week,'' George Stephanopoulos, asked him about
suggestions by Mr. Trump that there could be some harm in wearing masks.

``There's no downside to wearing a mask,'' Admiral Giroir responded.
``I'm a pediatric I.C.U. physician. I wore a mask 10 hours a day for
many many years.''

Dr. Adams wore a mask during his entire interview on the CBS program
``Face the Nation'' even though he was being interviewed remotely from
Indiana. He said measures like wearing face coverings were ``critically
important.''

Earlier in the pandemic, Dr. Adams had discouraged people from buying
masks, in part so there would be enough for medical workers, and he had
said ``masks do not work for the general public in preventing them from
getting coronavirus.''

On Sunday, when the host of ``Face the Nation,'' Margaret Brennan, asked
if he regretted saying that masks were not effective in keeping the
general population healthy, Dr. Adams replied: ``Once upon a time, we
prescribed cigarettes for asthmatics, and leeches and cocaine and heroin
for people as medical treatments,'' adding, ``When we learn better, we
do better.''

Dr. Adams, one of the highest-ranking Black officials in the Trump
administration, was also asked about his recent
\href{https://www.fox5dc.com/news/us-surgeon-general-stresses-importance-of-wearing-face-coverings-to-reduce-spread-of-covid-19}{comments}
that mask-wearing requirements should be enforced locally and not as a
national mandate. He had said, ``in the context of the Black Lives
Matter movement, when we have people being killed for handing out single
cigarettes or for falling asleep in a fast food line, I really worry
about over-policing and having a situation where you're giving people
one more reason to arrest a Black man.''

Ms. Brennan asked him on Sunday, ``Are you saying that racism makes it
too risky to mandate masks?''

Dr. Adams replied, ``So, to be very clear, I'm not saying it makes it
too risky. I'm saying, if we're going to have a mask mandate, we need to
understand that works best at the local and state level, along with
education. We need people to understand why they're doing it. We need
people to understand how they benefit from it, because if we just try to
mandate it, you have to have an enforcement mechanism, and we're in the
midst of a moment when over-policing has caused many different
individuals to be killed for very minor offenses.''

With record numbers of cases in states like Florida, which on Sunday
reported 15,000 new cases, the highest single-day total of any state
since the pandemic began, both health officials were questioned about
the administration's reluctance to consider returning to a lockdown in
some cities and states.

Asked on ``This Week'' if states with stark increases in cases, like
Florida, South Carolina, Arizona, Texas and Georgia, should consider
more stringent measures, Admiral Giroir said ``everything should be on
the table.'' He said closing bars and limiting the number of patrons
allowed in restaurants were ``two measures that really do need to be
done.''

\href{https://www.nytimes.com/news-event/coronavirus?action=click\&pgtype=Article\&state=default\&region=MAIN_CONTENT_3\&context=storylines_faq}{}

\hypertarget{the-coronavirus-outbreak-}{%
\subsubsection{The Coronavirus Outbreak
›}\label{the-coronavirus-outbreak-}}

\hypertarget{frequently-asked-questions}{%
\paragraph{Frequently Asked
Questions}\label{frequently-asked-questions}}

Updated August 3, 2020

\begin{itemize}
\item ~
  \hypertarget{im-a-small-business-owner-can-i-get-relief}{%
  \paragraph{I'm a small-business owner. Can I get
  relief?}\label{im-a-small-business-owner-can-i-get-relief}}

  \begin{itemize}
  \tightlist
  \item
    The
    \href{https://www.nytimes.com/article/small-business-loans-stimulus-grants-freelancers-coronavirus.html?action=click\&pgtype=Article\&state=default\&region=MAIN_CONTENT_3\&context=storylines_faq}{stimulus
    bills enacted in March} offer help for the millions of American
    small businesses. Those eligible for aid are businesses and
    nonprofit organizations with fewer than 500 workers, including sole
    proprietorships, independent contractors and freelancers. Some
    larger companies in some industries are also eligible. The help
    being offered, which is being managed by the Small Business
    Administration, includes the Paycheck Protection Program and the
    Economic Injury Disaster Loan program. But lots of folks have
    \href{https://www.nytimes.com/interactive/2020/05/07/business/small-business-loans-coronavirus.html?action=click\&pgtype=Article\&state=default\&region=MAIN_CONTENT_3\&context=storylines_faq}{not
    yet seen payouts.} Even those who have received help are confused:
    The rules are draconian, and some are stuck sitting on
    \href{https://www.nytimes.com/2020/05/02/business/economy/loans-coronavirus-small-business.html?action=click\&pgtype=Article\&state=default\&region=MAIN_CONTENT_3\&context=storylines_faq}{money
    they don't know how to use.} Many small-business owners are getting
    less than they expected or
    \href{https://www.nytimes.com/2020/06/10/business/Small-business-loans-ppp.html?action=click\&pgtype=Article\&state=default\&region=MAIN_CONTENT_3\&context=storylines_faq}{not
    hearing anything at all.}
  \end{itemize}
\item ~
  \hypertarget{what-are-my-rights-if-i-am-worried-about-going-back-to-work}{%
  \paragraph{What are my rights if I am worried about going back to
  work?}\label{what-are-my-rights-if-i-am-worried-about-going-back-to-work}}

  \begin{itemize}
  \tightlist
  \item
    Employers have to provide
    \href{https://www.osha.gov/SLTC/covid-19/standards.html}{a safe
    workplace} with policies that protect everyone equally.
    \href{https://www.nytimes.com/article/coronavirus-money-unemployment.html?action=click\&pgtype=Article\&state=default\&region=MAIN_CONTENT_3\&context=storylines_faq}{And
    if one of your co-workers tests positive for the coronavirus, the
    C.D.C.} has said that
    \href{https://www.cdc.gov/coronavirus/2019-ncov/community/guidance-business-response.html}{employers
    should tell their employees} -\/- without giving you the sick
    employee's name -\/- that they may have been exposed to the virus.
  \end{itemize}
\item ~
  \hypertarget{should-i-refinance-my-mortgage}{%
  \paragraph{Should I refinance my
  mortgage?}\label{should-i-refinance-my-mortgage}}

  \begin{itemize}
  \tightlist
  \item
    \href{https://www.nytimes.com/article/coronavirus-money-unemployment.html?action=click\&pgtype=Article\&state=default\&region=MAIN_CONTENT_3\&context=storylines_faq}{It
    could be a good idea,} because mortgage rates have
    \href{https://www.nytimes.com/2020/07/16/business/mortgage-rates-below-3-percent.html?action=click\&pgtype=Article\&state=default\&region=MAIN_CONTENT_3\&context=storylines_faq}{never
    been lower.} Refinancing requests have pushed mortgage applications
    to some of the highest levels since 2008, so be prepared to get in
    line. But defaults are also up, so if you're thinking about buying a
    home, be aware that some lenders have tightened their standards.
  \end{itemize}
\item ~
  \hypertarget{what-is-school-going-to-look-like-in-september}{%
  \paragraph{What is school going to look like in
  September?}\label{what-is-school-going-to-look-like-in-september}}

  \begin{itemize}
  \tightlist
  \item
    It is unlikely that many schools will return to a normal schedule
    this fall, requiring the grind of
    \href{https://www.nytimes.com/2020/06/05/us/coronavirus-education-lost-learning.html?action=click\&pgtype=Article\&state=default\&region=MAIN_CONTENT_3\&context=storylines_faq}{online
    learning},
    \href{https://www.nytimes.com/2020/05/29/us/coronavirus-child-care-centers.html?action=click\&pgtype=Article\&state=default\&region=MAIN_CONTENT_3\&context=storylines_faq}{makeshift
    child care} and
    \href{https://www.nytimes.com/2020/06/03/business/economy/coronavirus-working-women.html?action=click\&pgtype=Article\&state=default\&region=MAIN_CONTENT_3\&context=storylines_faq}{stunted
    workdays} to continue. California's two largest public school
    districts --- Los Angeles and San Diego --- said on July 13, that
    \href{https://www.nytimes.com/2020/07/13/us/lausd-san-diego-school-reopening.html?action=click\&pgtype=Article\&state=default\&region=MAIN_CONTENT_3\&context=storylines_faq}{instruction
    will be remote-only in the fall}, citing concerns that surging
    coronavirus infections in their areas pose too dire a risk for
    students and teachers. Together, the two districts enroll some
    825,000 students. They are the largest in the country so far to
    abandon plans for even a partial physical return to classrooms when
    they reopen in August. For other districts, the solution won't be an
    all-or-nothing approach.
    \href{https://bioethics.jhu.edu/research-and-outreach/projects/eschool-initiative/school-policy-tracker/}{Many
    systems}, including the nation's largest, New York City, are
    devising
    \href{https://www.nytimes.com/2020/06/26/us/coronavirus-schools-reopen-fall.html?action=click\&pgtype=Article\&state=default\&region=MAIN_CONTENT_3\&context=storylines_faq}{hybrid
    plans} that involve spending some days in classrooms and other days
    online. There's no national policy on this yet, so check with your
    municipal school system regularly to see what is happening in your
    community.
  \end{itemize}
\item ~
  \hypertarget{is-the-coronavirus-airborne}{%
  \paragraph{Is the coronavirus
  airborne?}\label{is-the-coronavirus-airborne}}

  \begin{itemize}
  \tightlist
  \item
    The coronavirus
    \href{https://www.nytimes.com/2020/07/04/health/239-experts-with-one-big-claim-the-coronavirus-is-airborne.html?action=click\&pgtype=Article\&state=default\&region=MAIN_CONTENT_3\&context=storylines_faq}{can
    stay aloft for hours in tiny droplets in stagnant air}, infecting
    people as they inhale, mounting scientific evidence suggests. This
    risk is highest in crowded indoor spaces with poor ventilation, and
    may help explain super-spreading events reported in meatpacking
    plants, churches and restaurants.
    \href{https://www.nytimes.com/2020/07/06/health/coronavirus-airborne-aerosols.html?action=click\&pgtype=Article\&state=default\&region=MAIN_CONTENT_3\&context=storylines_faq}{It's
    unclear how often the virus is spread} via these tiny droplets, or
    aerosols, compared with larger droplets that are expelled when a
    sick person coughs or sneezes, or transmitted through contact with
    contaminated surfaces, said Linsey Marr, an aerosol expert at
    Virginia Tech. Aerosols are released even when a person without
    symptoms exhales, talks or sings, according to Dr. Marr and more
    than 200 other experts, who
    \href{https://academic.oup.com/cid/article/doi/10.1093/cid/ciaa939/5867798}{have
    outlined the evidence in an open letter to the World Health
    Organization}.
  \end{itemize}
\end{itemize}

In another interview on Sunday, on NBC's ``Meet the Press,'' Admiral
Giroir said the rates of people testing positive for coronavirus were
``leveling off.''

But, he continued: ``I don't want to underestimate the seriousness of
this right now. It's all hands on deck and we have people in the field
assisting essentially every county, every hot spot, so we are in the
midst of this and we're taking it very seriously.''

Dr. Adams struck a similar tone. ``Please don't mistake me for saying
we're happy with where we are,'' he said on CBS. ``What I'm saying is
that we are working with states to make sure we can respond to this
incredibly contagious disease. And part of that, again, is making sure
we're slowing the spread, right? People understand the importance of
wearing face coverings and good hand hygiene and staying home when they
can.''

Appearing on ``Face the Nation,'' Dr. Scott Gottlieb, a former
commissioner of the Food and Drug Administration in the Trump
administration, said, ``I think things are going to get worse before
they get better.''

Dr. Gottlieb pointed to some reassuring signals --- in data on mobility
and restaurant reservations --- that people in some states with rising
cases are beginning to take social distancing guidelines seriously.
Still, he predicted that ``in the South, you're likely to see an
extended plateau'' of cases.

Unlike in New York, which had a huge surge early in the pandemic but got
it under control relatively quickly, Dr. Gottlieb said, ``I think the
Southern experience is more likely to mirror Brazil.'' That country has
been besieged with outbreaks and its case count is second only to the
United States. Last week, Brazil's president, Jair Bolsonaro, who had
downplayed the pandemic for months, tested positive for the virus.

Dr. Gottlieb said the states that were now hot spots had reopened too
early and ``people became complacent, especially younger people --- they
were going out, not taking precautions.''

He said that the surge in infections in younger people had now begun to
spread to more vulnerable populations. ``That's what we're seeing right
now, you're seeing rising cases in nursing homes.''

Advertisement

\protect\hyperlink{after-bottom}{Continue reading the main story}

\hypertarget{site-index}{%
\subsection{Site Index}\label{site-index}}

\hypertarget{site-information-navigation}{%
\subsection{Site Information
Navigation}\label{site-information-navigation}}

\begin{itemize}
\tightlist
\item
  \href{https://help.nytimes.com/hc/en-us/articles/115014792127-Copyright-notice}{©~2020~The
  New York Times Company}
\end{itemize}

\begin{itemize}
\tightlist
\item
  \href{https://www.nytco.com/}{NYTCo}
\item
  \href{https://help.nytimes.com/hc/en-us/articles/115015385887-Contact-Us}{Contact
  Us}
\item
  \href{https://www.nytco.com/careers/}{Work with us}
\item
  \href{https://nytmediakit.com/}{Advertise}
\item
  \href{http://www.tbrandstudio.com/}{T Brand Studio}
\item
  \href{https://www.nytimes.com/privacy/cookie-policy\#how-do-i-manage-trackers}{Your
  Ad Choices}
\item
  \href{https://www.nytimes.com/privacy}{Privacy}
\item
  \href{https://help.nytimes.com/hc/en-us/articles/115014893428-Terms-of-service}{Terms
  of Service}
\item
  \href{https://help.nytimes.com/hc/en-us/articles/115014893968-Terms-of-sale}{Terms
  of Sale}
\item
  \href{https://spiderbites.nytimes.com}{Site Map}
\item
  \href{https://help.nytimes.com/hc/en-us}{Help}
\item
  \href{https://www.nytimes.com/subscription?campaignId=37WXW}{Subscriptions}
\end{itemize}
