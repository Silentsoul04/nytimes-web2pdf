Sections

SEARCH

\protect\hyperlink{site-content}{Skip to
content}\protect\hyperlink{site-index}{Skip to site index}

\href{https://myaccount.nytimes.com/auth/login?response_type=cookie\&client_id=vi}{}

\href{https://www.nytimes.com/section/todayspaper}{Today's Paper}

\href{/section/opinion}{Opinion}\textbar{}You Should Start Writing
Letters

\href{https://nyti.ms/329r5IN}{https://nyti.ms/329r5IN}

\begin{itemize}
\item
\item
\item
\item
\item
\end{itemize}

\href{https://www.nytimes.com/spotlight/at-home?action=click\&pgtype=Article\&state=default\&region=TOP_BANNER\&context=at_home_menu}{At
Home}

\begin{itemize}
\tightlist
\item
  \href{https://www.nytimes.com/2020/08/04/arts/television/sam-jay-netflix-special.html?action=click\&pgtype=Article\&state=default\&region=TOP_BANNER\&context=at_home_menu}{Watch:
  Sam Jay}
\item
  \href{https://www.nytimes.com/interactive/2020/at-home/even-more-reporters-editors-diaries-lists-recommendations.html?action=click\&pgtype=Article\&state=default\&region=TOP_BANNER\&context=at_home_menu}{Peruse:
  Reporters' Google Docs}
\item
  \href{https://www.nytimes.com/2020/08/04/dining/colombian-empanadas-carlos-gaviria.html?action=click\&pgtype=Article\&state=default\&region=TOP_BANNER\&context=at_home_menu}{Make:
  Empanadas}
\item
  \href{https://www.nytimes.com/2020/08/06/arts/design/street-art-nyc-george-floyd.html?action=click\&pgtype=Article\&state=default\&region=TOP_BANNER\&context=at_home_menu}{Explore:
  N.Y.C. Street Art}
\end{itemize}

Advertisement

\protect\hyperlink{after-top}{Continue reading the main story}

\href{/section/opinion}{Opinion}

Supported by

\protect\hyperlink{after-sponsor}{Continue reading the main story}

\hypertarget{you-should-start-writing-letters}{%
\section{You Should Start Writing
Letters}\label{you-should-start-writing-letters}}

Zoom calls and texts are emotionally draining, but exchanging
handwritten notes can be sublime.

By Jordan Salama

Mr. Salama is a writer.

\begin{itemize}
\item
  July 12, 2020
\item
  \begin{itemize}
  \item
  \item
  \item
  \item
  \item
  \end{itemize}
\end{itemize}

\includegraphics{https://static01.nyt.com/images/2020/07/12/opinion/12salama2/12salama2-articleLarge.png?quality=75\&auto=webp\&disable=upscale}

The first letter arrived dated March 31, 2020. It was from a close
childhood friend, later turned college roommate, with whom I regularly
keep in touch via instant texts, FaceTimes and phone calls, as most
20-somethings do.

``The sun has set on our 15th day of quarantine/social distancing,'' my
friend wrote, his chicken scratch still familiar from our days in grade
school. ``Isn't it crazy how quickly this has become the new normal?''

He'd alerted me that the letter was coming in a text: After many days of
nonstop Zoom calls for work, the last thing he wanted to do was look at
another screen to catch up. Plus, he said, writing a letter could be a
fun creative exercise to break up the monotony.

So I wrote back. And then I wrote to another friend and another, and
lately not a week has gone by when there hasn't been a letter to respond
to. In most of these exchanges, there seems to exist this unspoken code
of slightly formal, performative language meant to evoke the past. My
childhood friend's first message, for instance, included a florid
analysis of John Keats's
\href{https://www.nytimes.com/2020/03/26/travel/coronavirus-essay-mayes-keats.html?searchResultPosition=1}{maritime
isolation} off the coast of typhus-plagued Naples in 1820.

``There's something about the ambience of the room,'' he wrote. ``The
gentle fire, the nautical aura, the fact that I'm writing a note --- it
makes me feel like a captain off on an expedition in a foreign land,
writing back home.''

It adds to a sense of emotion and escape, yet hardly detracts from the
ability to write candidly about our wide range of current experiences.
I've written about bird feeders, good movies and family; I've read
friends' letters about fishing and homesickness and Gabriel García
Márquez's ``Love in the Time of Cholera,'' in which the young Florentino
Ariza writes thousands of love letters during an epidemic in Colombia.

Frequent correspondence by mail is fairly new to me. When I was in fifth
grade, we had a pen-pals program with a class in Australia, but when the
school year ended, my pal and I fell out of touch. Anytime I travel
afar, I try to write to my family; somehow I always tend to get home
before my letters do.

But like so many other things in this otherwise-terrifying global
quarantine, I've found writing letters to be wonderful in the simplest
of ways. For each one, I sit at our dining room table for the better
part of an hour, away from my phone and computer, with only a sheet or
two of blank white printer paper in front of me. I'm hardly able to keep
a regular journal without it feeling like a chore, but writing to
someone else is sending a fresh entry off into the world without ever
having to look at it again.

In return, I'll be left with something far more interesting than a
mundane account of my own pandemic days: a patchwork of pages that were
sent to me by others, each one fresher than the next.

It's been deeply comforting to think that whatever I am writing will
soon be in the hands of someone else, especially in a time of so much
physical distancing. I've sent letters as far as Argentina and South
Korea, and as near as only a few blocks from my door. Some of the
handwriting I've seen, like mine, has been laughably illegible; other
letters are aesthetically works of art. One friend, an international
student isolating on an otherwise-emptied college campus in New Jersey,
enclosed a petal from a blossoming cherry tree. In these pages, I read
the smiles I cannot see.

I'm not alone in finding comfort in letter-writing these days. A recent
New York Times
\href{https://www.nytimes.com/2020/06/24/style/mail-letters-coronavirus.html}{article}
reported on the rise in snail mail and handwritten messages; the
practice seems to have caught on as people cope with grief from the
pandemic. That I've only started writing letters now is ironic and sad,
too, because the U.S. Postal Service is
\href{https://slate.com/news-and-politics/2020/05/coronavirus-postal-service-office-congress-money-trouble.html}{bleeding}.
The economic devastation of the pandemic could be the final blow that
ends one of our nation's oldest and most cherished institutions. While
an increase in package volume during the first few months of the
pandemic is
\href{https://www.washingtonpost.com/business/2020/06/25/postal-service-packages-coronavirus/}{providing
temporary relief}, no amount of letters we send can make up for the
billions in federal funding that are needed to save it.

And yet, as with so many other things these days, I'm holding out hope.
A Postal Service
\href{https://postalpro.usps.com/market-research/covid-mail-attitudes}{survey}
published in May suggested younger people in particular were more likely
to want to send cards and letters during this time. Though that doesn't
mean a lot of us are actually doing it, part of me likes to think
there's some Florentino Ariza out there, writing impassioned letters to
the girl he's not permitted to see.

More likely, it's because we're missing our friends and classmates;
we're so badly aching for the simple physical connections that the
coronavirus has taken away without a promise of near return. Perhaps
it's because a letter is an unhindered way of working through anxieties,
thoughts and emotions during a period of nonstop information and
tremendous grief. Perhaps it's simply a break from a screen or just
another way to mark the passage of time when the world seems to be on an
indefinite hold.

In that sense, there are plenty of reasons to start writing letters now
--- not least because there's something to be said for slowing down.
``When I got your letter, the first thing I wanted to do was text you a
picture \ldots{} but I quickly caught myself,'' another childhood friend
wrote. ``What an affront to letter-writing that would have been.'' I
smiled as I pulled out a blank sheet to start my response. I like to
think I'll keep this up for as long as I can, or at least as long as
someone is willing to write back.

Jordan Salama
(\href{https://twitter.com/JordanSalama19}{@jordansalama19}) is a writer
and the author of the forthcoming ``Every Day the River Changes.''

\emph{The Times is committed to publishing}
\href{https://www.nytimes.com/2019/01/31/opinion/letters/letters-to-editor-new-york-times-women.html}{\emph{a
diversity of letters}} \emph{to the editor. We'd like to hear what you
think about this or any of our articles. Here are some}
\href{https://help.nytimes.com/hc/en-us/articles/115014925288-How-to-submit-a-letter-to-the-editor}{\emph{tips}}\emph{.
And here's our email:}
\href{mailto:letters@nytimes.com}{\emph{letters@nytimes.com}}\emph{.}

\emph{Follow The New York Times Opinion section on}
\href{https://www.facebook.com/nytopinion}{\emph{Facebook}}\emph{,}
\href{http://twitter.com/NYTOpinion}{\emph{Twitter (@NYTopinion)}}
\emph{and}
\href{https://www.instagram.com/nytopinion/}{\emph{Instagram}}\emph{.}

Advertisement

\protect\hyperlink{after-bottom}{Continue reading the main story}

\hypertarget{site-index}{%
\subsection{Site Index}\label{site-index}}

\hypertarget{site-information-navigation}{%
\subsection{Site Information
Navigation}\label{site-information-navigation}}

\begin{itemize}
\tightlist
\item
  \href{https://help.nytimes.com/hc/en-us/articles/115014792127-Copyright-notice}{©~2020~The
  New York Times Company}
\end{itemize}

\begin{itemize}
\tightlist
\item
  \href{https://www.nytco.com/}{NYTCo}
\item
  \href{https://help.nytimes.com/hc/en-us/articles/115015385887-Contact-Us}{Contact
  Us}
\item
  \href{https://www.nytco.com/careers/}{Work with us}
\item
  \href{https://nytmediakit.com/}{Advertise}
\item
  \href{http://www.tbrandstudio.com/}{T Brand Studio}
\item
  \href{https://www.nytimes.com/privacy/cookie-policy\#how-do-i-manage-trackers}{Your
  Ad Choices}
\item
  \href{https://www.nytimes.com/privacy}{Privacy}
\item
  \href{https://help.nytimes.com/hc/en-us/articles/115014893428-Terms-of-service}{Terms
  of Service}
\item
  \href{https://help.nytimes.com/hc/en-us/articles/115014893968-Terms-of-sale}{Terms
  of Sale}
\item
  \href{https://spiderbites.nytimes.com}{Site Map}
\item
  \href{https://help.nytimes.com/hc/en-us}{Help}
\item
  \href{https://www.nytimes.com/subscription?campaignId=37WXW}{Subscriptions}
\end{itemize}
