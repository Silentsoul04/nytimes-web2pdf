Sections

SEARCH

\protect\hyperlink{site-content}{Skip to
content}\protect\hyperlink{site-index}{Skip to site index}

\href{https://myaccount.nytimes.com/auth/login?response_type=cookie\&client_id=vi}{}

\href{https://www.nytimes.com/section/todayspaper}{Today's Paper}

\href{/section/opinion}{Opinion}\textbar{}Attention All Women: Trump Is
Coming for Your Health Care

\href{https://nyti.ms/2BW9r0E}{https://nyti.ms/2BW9r0E}

\begin{itemize}
\item
\item
\item
\item
\item
\item
\end{itemize}

Advertisement

\protect\hyperlink{after-top}{Continue reading the main story}

\href{/section/opinion}{Opinion}

Supported by

\protect\hyperlink{after-sponsor}{Continue reading the main story}

\hypertarget{attention-all-women-trump-is-coming-for-your-health-care}{%
\section{Attention All Women: Trump Is Coming for Your Health
Care}\label{attention-all-women-trump-is-coming-for-your-health-care}}

Even with a pandemic raging, the president wants the Supreme Court to
strike down the Affordable Care Act.

By Kathleen Sebelius

Ms. Sebelius, a former governor of Kansas, was the secretary of health
and human services in the Obama administration.

\begin{itemize}
\item
  July 13, 2020
\item
  \begin{itemize}
  \item
  \item
  \item
  \item
  \item
  \item
  \end{itemize}
\end{itemize}

\includegraphics{https://static01.nyt.com/images/2020/07/13/opinion/13sebelius1/merlin_174017205_5a2fa982-1f61-4ccb-997d-a1fac38cfc39-articleLarge.jpg?quality=75\&auto=webp\&disable=upscale}

In the middle of a catastrophic public health crisis, the Trump
administration has asked the Supreme Court
\href{https://www.nytimes.com/2020/06/26/us/politics/obamacare-trump-administration-supreme-court.html}{to
overturn} the entire Affordable Care Act. This is dangerous for many
reasons --- but for women, it's devastating. They would be stripped of
the protections they have had in the decade since passage of the law,
known as Obamacare.

Before the law, insurance companies routinely discriminated against
women. Those who didn't work for employers that offered health coverage
or who weren't old enough or poor enough to qualify for Medicare or
Medicaid struggled to buy health insurance in the private market, where
insurance companies made all the rules.

In those days, insurers could charge women up to two or three times more
than men for identical health policies. Women discovered that many of
the services and medicines they needed were not included, like coverage
for pregnancy, which was not part of most individual policies and was
impossible to purchase once a woman became pregnant.

Insurance companies routinely denied coverage to Americans with
pre-existing conditions, a practice that affects more women than men.
About 30 million women have a pre-existing condition --- like side
effects from having taken Accutane as a teenager, depression or breast
cancer --- compared with about 24 million men, according to the
\href{https://www.kff.org/health-reform/issue-brief/pre-existing-condition-prevalence-for-individuals-and-families/}{Kaiser
Family Foundation}.

Obamacare put an end to that gender discrimination. Women young and old,
working in jobs or at home, gained coverage and health benefits that
they never had before. Insurance companies were required to sell
policies to women with pre-existing conditions and had to stop dropping
them off their health plans if they got sick. And all policies had to
include maternity coverage.

As the Affordable Care Act was being drawn up, Congress asked an expert
panel of doctors and scientists to identify services used by women that
were missing from most health policies. As a result, the law required
that coverage of preventive services like depression screening, breast
pumps for nursing mothers, various cancer screenings, well-woman visits
and all methods of contraception approved by the Food and Drug
Administration were included in health plans. Over 50 million women got
access to no-cost birth control, which has helped to reduce
\href{https://www.pewresearch.org/fact-tank/2019/08/02/why-is-the-teen-birth-rate-falling/}{teen
pregnancies} and
\href{https://www.guttmacher.org/fact-sheet/induced-abortion-united-states}{abortions}
in the United States to
\href{https://www.pewresearch.org/fact-tank/2019/08/02/why-is-the-teen-birth-rate-falling/}{record
low}s.

But the Trump administration has worked hard to limit birth control
benefits, and last week the Supreme Court upheld a Trump administration
rule allowing employers with moral or religious objections
\href{https://www.nytimes.com/2020/07/08/us/supreme-court-birth-control-obamacare.html}{to
opt out} of the Obamacare mandate to provide no-cost contraceptive
services for women.

Other coverage made possible by Obamacare would also disappear if the
Supreme Court overturns the law.

Before Obamacare, while federal Medicaid rules mandated coverage for all
women up to 60 days after childbirth, decisions about coverage then
reverted to the states. As a result, new mothers often saw their health
care terminated. Now in the 37 states that have expanded Medicaid,
Obamacare provides for continuing coverage for new mothers. And women in
poorly paid jobs, like nurses' aides and service workers, who often were
not entitled to any Medicaid coverage based on income, now have access
to low-cost health insurance in those states.

Women under the age of 26, whether they were married or single or had
children, became eligible to stay on their parents' insurance plans.
Women over 65, enrolled in Medicare, had annual well-woman visits added
to their benefits. And for those who take a lot of medications, their
prescription drug costs were greatly reduced.

With the Covid-19 economic crash, many women with employer-sponsored
health insurance are losing their coverage along with their jobs. But
thanks to Obamacare, many qualify for Medicaid or for subsidized
insurance, so they can continue to have health coverage as the economy
recovers.

If President Trump wins his case to eliminate Obamacare, millions of
women could lose coverage because of a pre-existing health condition,
and be deprived of access to expanded Medicaid insurance and no-cost
contraception and other preventive health services.

Women who own their own businesses or work in the gig economy could no
longer rely on federal help in buying health insurance for themselves
and their families. And once again, insurance companies could limit
health benefits that women need and charge them more than men for their
health care.

Women in America should make no mistake. The health progress we have
made in the last decade would be wiped out by one Supreme Court decision
if Donald Trump gets his way.

\href{http://www.sebeliusresources.com/welcome-1}{Kathleen Sebelius}, a
Democrat elected to two terms as governor of Kansas, was secretary of
health and human services from 2009 to 2014.

\emph{The Times is committed to publishing}
\href{https://www.nytimes.com/2019/01/31/opinion/letters/letters-to-editor-new-york-times-women.html}{\emph{a
diversity of letters}} \emph{to the editor. We'd like to hear what you
think about this or any of our articles. Here are some}
\href{https://help.nytimes.com/hc/en-us/articles/115014925288-How-to-submit-a-letter-to-the-editor}{\emph{tips}}\emph{.
And here's our email:}
\href{mailto:letters@nytimes.com}{\emph{letters@nytimes.com}}\emph{.}

\emph{Follow The New York Times Opinion section on}
\href{https://www.facebook.com/nytopinion}{\emph{Facebook}}\emph{,}
\href{http://twitter.com/NYTOpinion}{\emph{Twitter (@NYTopinion)}}
\emph{and}
\href{https://www.instagram.com/nytopinion/}{\emph{Instagram}}\emph{.}

Advertisement

\protect\hyperlink{after-bottom}{Continue reading the main story}

\hypertarget{site-index}{%
\subsection{Site Index}\label{site-index}}

\hypertarget{site-information-navigation}{%
\subsection{Site Information
Navigation}\label{site-information-navigation}}

\begin{itemize}
\tightlist
\item
  \href{https://help.nytimes.com/hc/en-us/articles/115014792127-Copyright-notice}{©~2020~The
  New York Times Company}
\end{itemize}

\begin{itemize}
\tightlist
\item
  \href{https://www.nytco.com/}{NYTCo}
\item
  \href{https://help.nytimes.com/hc/en-us/articles/115015385887-Contact-Us}{Contact
  Us}
\item
  \href{https://www.nytco.com/careers/}{Work with us}
\item
  \href{https://nytmediakit.com/}{Advertise}
\item
  \href{http://www.tbrandstudio.com/}{T Brand Studio}
\item
  \href{https://www.nytimes.com/privacy/cookie-policy\#how-do-i-manage-trackers}{Your
  Ad Choices}
\item
  \href{https://www.nytimes.com/privacy}{Privacy}
\item
  \href{https://help.nytimes.com/hc/en-us/articles/115014893428-Terms-of-service}{Terms
  of Service}
\item
  \href{https://help.nytimes.com/hc/en-us/articles/115014893968-Terms-of-sale}{Terms
  of Sale}
\item
  \href{https://spiderbites.nytimes.com}{Site Map}
\item
  \href{https://help.nytimes.com/hc/en-us}{Help}
\item
  \href{https://www.nytimes.com/subscription?campaignId=37WXW}{Subscriptions}
\end{itemize}
