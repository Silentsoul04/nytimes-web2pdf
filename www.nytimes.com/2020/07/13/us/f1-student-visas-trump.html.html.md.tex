Sections

SEARCH

\protect\hyperlink{site-content}{Skip to
content}\protect\hyperlink{site-index}{Skip to site index}

\href{https://www.nytimes.com/section/us}{U.S.}

\href{https://myaccount.nytimes.com/auth/login?response_type=cookie\&client_id=vi}{}

\href{https://www.nytimes.com/section/todayspaper}{Today's Paper}

\href{/section/us}{U.S.}\textbar{}17 States Sue to Block Student Visa
Rules

\url{https://nyti.ms/2Oiop3w}

\begin{itemize}
\item
\item
\item
\item
\item
\end{itemize}

\href{https://www.nytimes.com/news-event/coronavirus?action=click\&pgtype=Article\&state=default\&region=TOP_BANNER\&context=storylines_menu}{The
Coronavirus Outbreak}

\begin{itemize}
\tightlist
\item
  live\href{https://www.nytimes.com/2020/08/02/world/coronavirus-updates.html?action=click\&pgtype=Article\&state=default\&region=TOP_BANNER\&context=storylines_menu}{Latest
  Updates}
\item
  \href{https://www.nytimes.com/interactive/2020/us/coronavirus-us-cases.html?action=click\&pgtype=Article\&state=default\&region=TOP_BANNER\&context=storylines_menu}{Maps
  and Cases}
\item
  \href{https://www.nytimes.com/interactive/2020/science/coronavirus-vaccine-tracker.html?action=click\&pgtype=Article\&state=default\&region=TOP_BANNER\&context=storylines_menu}{Vaccine
  Tracker}
\item
  \href{https://www.nytimes.com/interactive/2020/07/29/us/schools-reopening-coronavirus.html?action=click\&pgtype=Article\&state=default\&region=TOP_BANNER\&context=storylines_menu}{What
  School May Look Like}
\item
  \href{https://www.nytimes.com/live/2020/07/31/business/stock-market-today-coronavirus?action=click\&pgtype=Article\&state=default\&region=TOP_BANNER\&context=storylines_menu}{Economy}
\end{itemize}

Advertisement

\protect\hyperlink{after-top}{Continue reading the main story}

Supported by

\protect\hyperlink{after-sponsor}{Continue reading the main story}

\hypertarget{17-states-sue-to-block-student-visa-rules}{%
\section{17 States Sue to Block Student Visa
Rules}\label{17-states-sue-to-block-student-visa-rules}}

A legal battle between universities and the Trump administration over
foreign students and online learning escalated on Monday, ahead of a
critical federal court hearing.

\includegraphics{https://static01.nyt.com/images/2020/07/13/us/13virus-studentvisas01/merlin_174372045_4af95db7-15eb-4928-b8b9-4be25f8be340-articleLarge.jpg?quality=75\&auto=webp\&disable=upscale}

\href{https://www.nytimes.com/by/anemona-hartocollis}{\includegraphics{https://static01.nyt.com/images/2018/06/13/multimedia/author-anemona-hartocollis/author-anemona-hartocollis-thumbLarge-v3.jpg}}

By \href{https://www.nytimes.com/by/anemona-hartocollis}{Anemona
Hartocollis}

\begin{itemize}
\item
  Published July 13, 2020Updated July 15, 2020
\item
  \begin{itemize}
  \item
  \item
  \item
  \item
  \item
  \end{itemize}
\end{itemize}

A Trump administration effort to force foreign college students to take
in-person classes in the fall or lose their visas has prompted a
high-stakes legal battle between the White House and some of America's
top universities, with 17 states and the District of Columbia joining
the fray on Monday in a lawsuit that calls the policy ``senseless and
cruel.''

The visa guidelines, issued a week ago, would upend months of careful
planning by colleges and universities and could force many students to
return to their home countries during the pandemic, where their ability
to study would be severely compromised.

The confrontation comes as the White House is pushing colleges and K-12
schools to throw open their doors to students, even as a growing number
\href{https://www.nytimes.com/2020/07/13/us/lausd-san-diego-school-reopening.html}{decide
that it's not safe}. Many universities have chosen to allow a limited
number of students on campus but to
\href{https://www.nytimes.com/2020/07/06/us/coronavirus-universities-colleges-reopening.html}{teach
most classes virtually} --- a decision that President Trump has derided
as ``ridiculous.'' Late last week, as his annoyance with universities
grew, Mr. Trump
\href{https://twitter.com/realDonaldTrump/status/1281616586273468416}{threatened
their nonprofit status}.

Harvard and the Massachusetts Institute of Technology, both of which
plan mostly online classes, were the first to challenge the new visa
rules in court, saying they were hastily implemented in violation of
federal procedures. Their lawsuit last week set up a high-stakes hearing
scheduled for Tuesday afternoon, a day before the government is
requiring schools to certify that students are taking in-person classes
to meet the visa requirements.

Dozens of universities have weighed in with Harvard and M.I.T., and
California's attorney general and several universities filed their own
suits in federal court late last week seeking to block the directive.

``The president is using foreign students as pawns to keep all schools
open, no matter the cost to the health and well-being of these students
and their communities,'' said Mark Rosenbaum, a lawyer with Public
Counsel, a legal aid organization in Los Angeles representing foreign
graduate students at three California universities. ``It's
temper-tantrum policymaking.''

\hypertarget{latest-updates-global-coronavirus-outbreak}{%
\section{\texorpdfstring{\href{https://www.nytimes.com/2020/08/01/world/coronavirus-covid-19.html?action=click\&pgtype=Article\&state=default\&region=MAIN_CONTENT_1\&context=storylines_live_updates}{Latest
Updates: Global Coronavirus
Outbreak}}{Latest Updates: Global Coronavirus Outbreak}}\label{latest-updates-global-coronavirus-outbreak}}

Updated 2020-08-02T17:52:35.962Z

\begin{itemize}
\tightlist
\item
  \href{https://www.nytimes.com/2020/08/01/world/coronavirus-covid-19.html?action=click\&pgtype=Article\&state=default\&region=MAIN_CONTENT_1\&context=storylines_live_updates\#link-34047410}{The
  U.S. reels as July cases more than double the total of any other
  month.}
\item
  \href{https://www.nytimes.com/2020/08/01/world/coronavirus-covid-19.html?action=click\&pgtype=Article\&state=default\&region=MAIN_CONTENT_1\&context=storylines_live_updates\#link-780ec966}{Top
  U.S. officials work to break an impasse over the federal jobless
  benefit.}
\item
  \href{https://www.nytimes.com/2020/08/01/world/coronavirus-covid-19.html?action=click\&pgtype=Article\&state=default\&region=MAIN_CONTENT_1\&context=storylines_live_updates\#link-2bc8948}{Its
  outbreak untamed, Melbourne goes into even greater lockdown.}
\end{itemize}

\href{https://www.nytimes.com/2020/08/01/world/coronavirus-covid-19.html?action=click\&pgtype=Article\&state=default\&region=MAIN_CONTENT_1\&context=storylines_live_updates}{See
more updates}

More live coverage:
\href{https://www.nytimes.com/live/2020/07/31/business/stock-market-today-coronavirus?action=click\&pgtype=Article\&state=default\&region=MAIN_CONTENT_1\&context=storylines_live_updates}{Markets}

The administration responded in court filings Monday that Immigration
and Customs Enforcement has the discretion to set student visa guidance,
and that just because universities don't like the requirements doesn't
make them against the law.

The government also pointed out that the directive allows foreign
students to take more online classes than they could have a year ago,
when only one virtual course was allowed. The agency had waived that
requirement in March, as the pandemic swept across the country and
forced college campuses to abruptly close.

Universities have argued that the state of emergency declared by the
president in March remains in effect, so the waived visa rules should,
too.

``The Trump administration didn't even attempt to explain the basis for
this senseless rule, which forces schools to choose between keeping
their international students enrolled and protecting the health and
safety of their campuses,'' Maura Healey, the Massachusetts attorney
general, said in a statement announcing the suit she filed Monday with
16 other states, which accuses the administration of violating the
Administrative Procedure Act.

At stake is the fate of possibly tens of thousands of students from all
over the world who are enrolled in American universities this fall,
where they represent both a major source of academic brainpower and a
vital revenue stream for institutions that face deep financial losses in
the pandemic.

The federal guidance issued July 6 has sent students scrambling to
enroll in in-person classes that are difficult to find, if they are
available at all.

Harvard Medical School's plan to move instruction online would force
students like Ayantu Temesgen of Ethiopia to return home. But the
Ethiopian government last month shut down the country's internet amid
deadly civil unrest, and even if it returns, the time difference would
be difficult to overcome.

``If I'm back home, I won't be able to wake up at 3 a.m. or 2 a.m. to
attend my classes,'' Ms. Temesgen said.

In their lawsuits, universities and the state attorneys general suggest
that the new guidance is part of a politically motivated attempt to
force universities to reopen, against their better judgment of the
health risks.

``The same day as the announcement of the administration's reversal, the
president of the United States made repeated public statements
expressing the view that schools must reopen in the fall,'' the lawsuit
filed Monday by the states says,
\href{https://twitter.com/realdonaldtrump/status/1280209946085339136}{citing
a tweet by the president} saying, in all capital letters, ``SCHOOLS MUST
OPEN IN THE FALL!!!''

The government said in its response that it had an interest in keeping
close tabs on foreign students, and allowing them to study online and
untethered from a university campus posed a security risk. ``A solely
online program of study provides a nonimmigrant student with enormous
flexibility to be present anywhere in the United States for up to an
entire academic term, whether that location has been reported to the
government, which raises significant national security concerns,'' it
said.

In a harbinger of how the case may fare, the Supreme Court ruled in June
that the Trump administration may not immediately proceed with its
\href{https://www.nytimes.com/2017/09/05/us/politics/trump-daca-dreamers-immigration.html}{plan
to end a program} protecting about 700,000 young immigrants known as
Dreamers from deportation. In the majority opinion,
\href{https://www.nytimes.com/2020/06/18/us/trump-daca-supreme-court.html}{Chief
Justice John G. Roberts Jr.} wrote that the court did not decide whether
rescinding the program was sound policy, only whether ``the agency
complied with the procedural requirement that it provide a reasoned
explanation for its action.''

The dispute illustrates the seismic shifts in how education is being
delivered during the pandemic, and the consequences of that change.
Online education has achieved new primacy even at prestigious
institutions like Harvard that previously relegated it mainly to
lower-status programs like extension courses.

\href{https://www.nytimes.com/news-event/coronavirus?action=click\&pgtype=Article\&state=default\&region=MAIN_CONTENT_3\&context=storylines_faq}{}

\hypertarget{the-coronavirus-outbreak-}{%
\subsubsection{The Coronavirus Outbreak
›}\label{the-coronavirus-outbreak-}}

\hypertarget{frequently-asked-questions}{%
\paragraph{Frequently Asked
Questions}\label{frequently-asked-questions}}

Updated July 27, 2020

\begin{itemize}
\item ~
  \hypertarget{should-i-refinance-my-mortgage}{%
  \paragraph{Should I refinance my
  mortgage?}\label{should-i-refinance-my-mortgage}}

  \begin{itemize}
  \tightlist
  \item
    \href{https://www.nytimes.com/article/coronavirus-money-unemployment.html?action=click\&pgtype=Article\&state=default\&region=MAIN_CONTENT_3\&context=storylines_faq}{It
    could be a good idea,} because mortgage rates have
    \href{https://www.nytimes.com/2020/07/16/business/mortgage-rates-below-3-percent.html?action=click\&pgtype=Article\&state=default\&region=MAIN_CONTENT_3\&context=storylines_faq}{never
    been lower.} Refinancing requests have pushed mortgage applications
    to some of the highest levels since 2008, so be prepared to get in
    line. But defaults are also up, so if you're thinking about buying a
    home, be aware that some lenders have tightened their standards.
  \end{itemize}
\item ~
  \hypertarget{what-is-school-going-to-look-like-in-september}{%
  \paragraph{What is school going to look like in
  September?}\label{what-is-school-going-to-look-like-in-september}}

  \begin{itemize}
  \tightlist
  \item
    It is unlikely that many schools will return to a normal schedule
    this fall, requiring the grind of
    \href{https://www.nytimes.com/2020/06/05/us/coronavirus-education-lost-learning.html?action=click\&pgtype=Article\&state=default\&region=MAIN_CONTENT_3\&context=storylines_faq}{online
    learning},
    \href{https://www.nytimes.com/2020/05/29/us/coronavirus-child-care-centers.html?action=click\&pgtype=Article\&state=default\&region=MAIN_CONTENT_3\&context=storylines_faq}{makeshift
    child care} and
    \href{https://www.nytimes.com/2020/06/03/business/economy/coronavirus-working-women.html?action=click\&pgtype=Article\&state=default\&region=MAIN_CONTENT_3\&context=storylines_faq}{stunted
    workdays} to continue. California's two largest public school
    districts --- Los Angeles and San Diego --- said on July 13, that
    \href{https://www.nytimes.com/2020/07/13/us/lausd-san-diego-school-reopening.html?action=click\&pgtype=Article\&state=default\&region=MAIN_CONTENT_3\&context=storylines_faq}{instruction
    will be remote-only in the fall}, citing concerns that surging
    coronavirus infections in their areas pose too dire a risk for
    students and teachers. Together, the two districts enroll some
    825,000 students. They are the largest in the country so far to
    abandon plans for even a partial physical return to classrooms when
    they reopen in August. For other districts, the solution won't be an
    all-or-nothing approach.
    \href{https://bioethics.jhu.edu/research-and-outreach/projects/eschool-initiative/school-policy-tracker/}{Many
    systems}, including the nation's largest, New York City, are
    devising
    \href{https://www.nytimes.com/2020/06/26/us/coronavirus-schools-reopen-fall.html?action=click\&pgtype=Article\&state=default\&region=MAIN_CONTENT_3\&context=storylines_faq}{hybrid
    plans} that involve spending some days in classrooms and other days
    online. There's no national policy on this yet, so check with your
    municipal school system regularly to see what is happening in your
    community.
  \end{itemize}
\item ~
  \hypertarget{is-the-coronavirus-airborne}{%
  \paragraph{Is the coronavirus
  airborne?}\label{is-the-coronavirus-airborne}}

  \begin{itemize}
  \tightlist
  \item
    The coronavirus
    \href{https://www.nytimes.com/2020/07/04/health/239-experts-with-one-big-claim-the-coronavirus-is-airborne.html?action=click\&pgtype=Article\&state=default\&region=MAIN_CONTENT_3\&context=storylines_faq}{can
    stay aloft for hours in tiny droplets in stagnant air}, infecting
    people as they inhale, mounting scientific evidence suggests. This
    risk is highest in crowded indoor spaces with poor ventilation, and
    may help explain super-spreading events reported in meatpacking
    plants, churches and restaurants.
    \href{https://www.nytimes.com/2020/07/06/health/coronavirus-airborne-aerosols.html?action=click\&pgtype=Article\&state=default\&region=MAIN_CONTENT_3\&context=storylines_faq}{It's
    unclear how often the virus is spread} via these tiny droplets, or
    aerosols, compared with larger droplets that are expelled when a
    sick person coughs or sneezes, or transmitted through contact with
    contaminated surfaces, said Linsey Marr, an aerosol expert at
    Virginia Tech. Aerosols are released even when a person without
    symptoms exhales, talks or sings, according to Dr. Marr and more
    than 200 other experts, who
    \href{https://academic.oup.com/cid/article/doi/10.1093/cid/ciaa939/5867798}{have
    outlined the evidence in an open letter to the World Health
    Organization}.
  \end{itemize}
\item ~
  \hypertarget{what-are-the-symptoms-of-coronavirus}{%
  \paragraph{What are the symptoms of
  coronavirus?}\label{what-are-the-symptoms-of-coronavirus}}

  \begin{itemize}
  \tightlist
  \item
    Common symptoms
    \href{https://www.nytimes.com/article/symptoms-coronavirus.html?action=click\&pgtype=Article\&state=default\&region=MAIN_CONTENT_3\&context=storylines_faq}{include
    fever, a dry cough, fatigue and difficulty breathing or shortness of
    breath.} Some of these symptoms overlap with those of the flu,
    making detection difficult, but runny noses and stuffy sinuses are
    less common.
    \href{https://www.nytimes.com/2020/04/27/health/coronavirus-symptoms-cdc.html?action=click\&pgtype=Article\&state=default\&region=MAIN_CONTENT_3\&context=storylines_faq}{The
    C.D.C. has also} added chills, muscle pain, sore throat, headache
    and a new loss of the sense of taste or smell as symptoms to look
    out for. Most people fall ill five to seven days after exposure, but
    symptoms may appear in as few as two days or as many as 14 days.
  \end{itemize}
\item ~
  \hypertarget{does-asymptomatic-transmission-of-covid-19-happen}{%
  \paragraph{Does asymptomatic transmission of Covid-19
  happen?}\label{does-asymptomatic-transmission-of-covid-19-happen}}

  \begin{itemize}
  \tightlist
  \item
    So far, the evidence seems to show it does. A widely cited
    \href{https://www.nature.com/articles/s41591-020-0869-5}{paper}
    published in April suggests that people are most infectious about
    two days before the onset of coronavirus symptoms and estimated that
    44 percent of new infections were a result of transmission from
    people who were not yet showing symptoms. Recently, a top expert at
    the World Health Organization stated that transmission of the
    coronavirus by people who did not have symptoms was ``very rare,''
    \href{https://www.nytimes.com/2020/06/09/world/coronavirus-updates.html?action=click\&pgtype=Article\&state=default\&region=MAIN_CONTENT_3\&context=storylines_faq\#link-1f302e21}{but
    she later walked back that statement.}
  \end{itemize}
\end{itemize}

But the move to online learning has prompted immigration authorities to
ask why, if classes can be attended from anywhere in the world, students
need to be in the United States to earn their degrees.

``You don't get a visa for taking online classes from, let's say,
University of Phoenix. So why would you if you were just taking online
classes, generally?'' the White House press secretary, Kayleigh McEnany,
told reporters at a news conference last week.

International students say this is a narrow view of what studying abroad
is all about, and that being in the United States may lead to research
and job opportunities that ultimately enrich the country and its
economic, cultural and intellectual life.

Meeting the demands of the government's new guidance would be extremely
difficult, universities say. Just weeks before school is scheduled to
begin, they would have to rejigger classes to make sure that hundreds of
thousands of international students would have an in-person option.

``The alternative is to lose significant numbers of students from their
campuses,'' the suit filed by state attorneys general says.

The area represented by the plaintiffs contains 1,124 colleges and
universities, with some approximately 373,000 international students
enrolled in 2019, who contributed an estimated \$14 billion to the
economy that year, according to the complaint.

About 40 higher education institutions filed declarations in support of
the lawsuit, including Yale, DePaul, the University of Chicago, Tufts,
Rutgers and state universities in Illinois, Maryland, Massachusetts,
Minnesota and Wisconsin.

Dan Levin contributed reporting.

Advertisement

\protect\hyperlink{after-bottom}{Continue reading the main story}

\hypertarget{site-index}{%
\subsection{Site Index}\label{site-index}}

\hypertarget{site-information-navigation}{%
\subsection{Site Information
Navigation}\label{site-information-navigation}}

\begin{itemize}
\tightlist
\item
  \href{https://help.nytimes.com/hc/en-us/articles/115014792127-Copyright-notice}{©~2020~The
  New York Times Company}
\end{itemize}

\begin{itemize}
\tightlist
\item
  \href{https://www.nytco.com/}{NYTCo}
\item
  \href{https://help.nytimes.com/hc/en-us/articles/115015385887-Contact-Us}{Contact
  Us}
\item
  \href{https://www.nytco.com/careers/}{Work with us}
\item
  \href{https://nytmediakit.com/}{Advertise}
\item
  \href{http://www.tbrandstudio.com/}{T Brand Studio}
\item
  \href{https://www.nytimes.com/privacy/cookie-policy\#how-do-i-manage-trackers}{Your
  Ad Choices}
\item
  \href{https://www.nytimes.com/privacy}{Privacy}
\item
  \href{https://help.nytimes.com/hc/en-us/articles/115014893428-Terms-of-service}{Terms
  of Service}
\item
  \href{https://help.nytimes.com/hc/en-us/articles/115014893968-Terms-of-sale}{Terms
  of Sale}
\item
  \href{https://spiderbites.nytimes.com}{Site Map}
\item
  \href{https://help.nytimes.com/hc/en-us}{Help}
\item
  \href{https://www.nytimes.com/subscription?campaignId=37WXW}{Subscriptions}
\end{itemize}
