Sections

SEARCH

\protect\hyperlink{site-content}{Skip to
content}\protect\hyperlink{site-index}{Skip to site index}

\href{https://www.nytimes.com/section/us}{U.S.}

\href{https://myaccount.nytimes.com/auth/login?response_type=cookie\&client_id=vi}{}

\href{https://www.nytimes.com/section/todayspaper}{Today's Paper}

\href{/section/us}{U.S.}\textbar{}Inside the White House, a Gun Industry
Lobbyist Delivers for His Former Patrons

\url{https://nyti.ms/307zeef}

\begin{itemize}
\item
\item
\item
\item
\item
\item
\end{itemize}

Advertisement

\protect\hyperlink{after-top}{Continue reading the main story}

Supported by

\protect\hyperlink{after-sponsor}{Continue reading the main story}

\hypertarget{inside-the-white-house-a-gun-industry-lobbyist-delivers-for-his-former-patrons}{%
\section{Inside the White House, a Gun Industry Lobbyist Delivers for
His Former
Patrons}\label{inside-the-white-house-a-gun-industry-lobbyist-delivers-for-his-former-patrons}}

The Trump administration lifted a ban on sales of silencers to private
overseas buyers that was intended to protect U.S. troops from ambushes.
The change was championed by a lawyer for the president who had worked
for a firearms trade group.

\includegraphics{https://static01.nyt.com/images/2020/07/12/multimedia/12trump-scilencers/merlin_156822702_5206652b-247b-47c1-887d-5b1e10f49b30-articleLarge.jpg?quality=75\&auto=webp\&disable=upscale}

By \href{https://www.nytimes.com/by/michael-laforgia}{Michael LaForgia}
and \href{https://www.nytimes.com/by/kenneth-p-vogel}{Kenneth P. Vogel}

\begin{itemize}
\item
  July 13, 2020
\item
  \begin{itemize}
  \item
  \item
  \item
  \item
  \item
  \item
  \end{itemize}
\end{itemize}

Michael B. Williams spent nearly two years helping to run a trade group
focused on expanding sales of firearm silencers by American
manufacturers.

But try as he might, he could not achieve one of the industry's main
goals: overturning a ban on sales to private foreign buyers enacted by
the State Department to protect American troops in Afghanistan and
elsewhere.

Then Mr. Williams joined the Trump administration.

As a White House lawyer, he pushed to overturn the prohibition, raising
the issue with influential administration officials and creating
pressure within the State Department, according to current and former
government officials.

On Friday, the State Department lifted the ban, and a longtime industry
goal was realized. The change paved the way for as much as \$250 million
a year in possible new overseas sales for companies that Mr. Williams
had championed as general counsel of the American Suppressor
Association.

His role in pushing to lift the ban, which has not been previously
reported, follows a well-established pattern in the Trump
administration, with the president handing over policymaking to allies
of special interest groups with a stake in those policies. And in this
case, Mr. Williams's victory comes for a key constituency as President
Trump seeks re-election.

Mr. Trump's cabinet includes a former coal lobbyist
\href{https://www.nytimes.com/2018/08/01/climate/andrew-wheeler-epa-lobbying.html}{as
administrator of the Environmental Protection Agency}, a former lobbyist
for the defense contractor Raytheon Technologies
\href{https://www.nytimes.com/2019/07/23/us/politics/mark-esper-secretary-defense.html}{as
defense secretary}, a lobbyist for the auto industry
\href{https://www.nytimes.com/2019/10/18/climate/trump-cabinet-lobbyists.html}{at
the helm of the Energy Department} and a former oil and gas lobbyist
\href{https://www.nytimes.com/2019/04/04/climate/david-bernhardt-interior-lobbying.html}{as
interior secretary}. Those industries have been sources of funds for Mr.
Trump's campaign and committees supporting it.

Mr. Williams's work, though lower-profile, has nevertheless been a boon
to another crucial political constituency: the gun lobby, which plays a
\href{https://www.nytimes.com/2018/02/24/us/politics/nra-gun-control-florida.html}{leading
role in Republican get-out-the-vote efforts} and views eliminating
silencer restrictions as an emerging issue. It's a subject that has been
embraced by the president's eldest son, Donald Trump Jr. --- an ally of
Mr. Williams's former trade group --- as well as by other powerful gun
industry groups.

\includegraphics{https://static01.nyt.com/images/2020/07/12/multimedia/12trump-silencers-wh/merlin_171695379_f9ec3aa0-d3e5-4e1d-b8d1-e50217bc9cdd-articleLarge.jpg?quality=75\&auto=webp\&disable=upscale}

``This is another win for the firearm and suppressor manufacturers by
the Trump administration,'' said Lawrence G. Keane, general counsel for
the National Shooting Sports Foundation, in a statement after the ban
was lifted Friday.

In an interview, Mr. Keane praised Mr. Williams, saying ``he understands
the product, obviously, having worked at the American Suppressor
Association.'' That association said it was
``\href{https://americansuppressorassociation.com/suppressor-exportation-now-legal/}{thrilled}''
with the ban's end; the group also dismissed safety concerns, noting
that the sales would be regulated by the State Department and that
foreign-made silencers were already available for purchase in other
countries.

But some in military, diplomatic and arms control circles defended the
ban and expressed alarm about its lifting, which was announced Friday
afternoon in a little-noticed posting on a State Department website.
Although the department's rules had long permitted selling silencers to
foreign governments, they did not allow sales to private companies or
individuals, whose use of the devices is more difficult to monitor.

Silencers, or sound suppressors, attach to a firearm's muzzle and reduce
the noise made by gunfire by trapping gas released when a bullet is
fired. Sales of suppressors in the United States, which are regulated by
federal authorities, have climbed in recent years.

Lincoln P. Bloomfield Jr., who was assistant secretary of state for
political-military affairs when the ban was enacted in 2002, said the
policy was intended to prevent American equipment from being used
against American service members, especially during the Afghanistan and
Iraq wars.

``Terrorist groups were using garage door openers to blow up U.S.
troops; you kind of think twice about what you are exporting,'' said Mr.
Bloomfield, who added that such dangers still exist today. ``Who are you
selling these silencers to?'' he said. ``I sure hope that none of these
are aimed at U.S. or allied forces.''

A State Department spokeswoman said the policy change was made to
benefit American manufacturers. ``U.S. companies should have the same
opportunity to compete in the international marketplace as other
manufacturers around the world,'' the spokeswoman said. She also said
that silencers were more readily available in foreign countries now than
when the ban was imposed.

The White House did not respond to a request for comment. Mr. Williams
declined to comment.

An examination of Mr. Williams's rise from trade group advocate to West
Wing lawyer reveals that White House tumult and turnover created
opportunities for him.

After joining Mr. Trump's campaign in 2016, Mr. Williams, at age 30,
became an assistant deputy general counsel at the Office of Management
and Budget, then led by Mick Mulvaney.

In the spring of 2019, not long after Mr. Mulvaney was
\href{https://www.nytimes.com/2018/12/14/us/politics/mick-mulvaney-trump-chief-of-staff.html}{elevated
to acting White House chief of staff}, Mr. Williams joined him as
counselor and a deputy assistant to the president. It was from that
perch that Mr. Williams began to press the gun issues in earnest,
according to the current and former officials, who were not authorized
to speak publicly.

While in the White House, Mr. Williams maintained close ties to the
suppressor association, which is funded by silencer manufacturers,
distributors, retailers and customers. His brother, Knox Williams,
started the organization and serves as its president and executive
director, and the two have remained in regular contact. ``We speak
almost every day,'' Knox Williams said in an interview.

Image

Knox Williams, president and executive director of the American
Suppressor Association.Credit...AP Photo/Lisa Marie Pane

Mr. Williams, his brother said, did not run afoul of
\href{https://www.nytimes.com/2017/01/28/us/politics/trump-toughens-some-facets-of-lobbying-ban-and-weakens-others.html}{Trump
administration ethics rules} that forbid government officials from
working on matters affecting their former employers within two years of
leaving. But in 2019, he set to work on gun issues without those
constraints.

He was involved in a successful push to shift responsibility for foreign
sales of semiautomatic weapons, including powerful .50-caliber sniper
rifles, to the Commerce Department from the State Department --- an
effort that had been underway since the Obama administration and that
had been blocked by Democratic members of Congress over concerns that it
would strip away oversight.

Once that was accomplished, Mr. Williams turned to the silencer sales
ban, even though in internal discussions Pentagon officials had warned
against lifting it. Silencers have become standard-issue among military
special operations forces because they offer an advantage in combat,
allowing American troops to shoot at an enemy but making it harder for
the enemy to determine where the gunfire is coming from. The officials
feared that a glut of high-quality silencers overseas might put American
forces at a similar disadvantage.Mr. Williams pressed the case anyway.

Knox Williams called the State Department decision a ``big victory'' for
his group but said the association had had no inside advantage in
seeking it --- though he acknowledged that his brother had played a
role. ``We work the issues that we work just the same as any other
organization does,'' he said.

Government watchdog groups, however, said the case raised concerns about
special interests gaining remarkable access in the Trump White House.

``When Michael Williams exits through the revolving door to return to
the gun industry, I'm sure he will be greeted with open arms,'' said
Austin Evers, executive director of American Oversight, a government
ethics advocacy group that has filed
\href{https://www.americanoversight.org/document/foia-to-omb-seeking-communications-between-michael-williams-and-gun-rights-advocates}{records
requests} for Mr. Williams's communications with the gun lobby from the
White House budget office.

Records obtained by Mr. Evers's group show that in early 2018, about a
year after his arrival at the White House, Mr. Williams was invited by
the National Shooting Sports Foundation to
\href{https://www.americanoversight.org/document/omb-calendar-invites-and-agenda-for-calls-regarding-the-parkland-shooting}{three
meetings} that another invitee described as being about countering gun
control measures after the mass shooting at a high school in Parkland,
Fla.

Mr. Keane, the shooting sports foundation's general counsel, said Mr.
Williams did not attend the meetings and had been invited in error.
Nevertheless, he said his group communicated with Mr. Williams about the
State Department's silencer policy, and other Second Amendment-related
issues. He said Mr. Williams took on what Mr. Keane called the ``hook
and bullet'' portfolio --- fishing and hunting issues --- at the White
House.

A Georgia native and Eagle Scout, Mr. Williams worked as a law clerk for
the National Rifle Association before
\href{https://www.documentcloud.org/documents/6987302-A-W-H-Lawyer-Who-Helped-Loosen-Gun-Silencer.html}{graduating
from George Washington University Law School} in 2014. Soon after, he
went to work at the American Suppressor Association, which his brother
had co-founded three years earlier. Mr. Williams managed the group's
budget, but he also helped draft legislation and lobby lawmakers, his
brother said. One of his main issues was the fight to open up sales of
silencers to private foreign buyers.

Intent on understanding the reasons for the sales ban, Mr. Williams
filed a Freedom of Information Act request for documents from the State
Department and then battled the agency over them for more than a year.
His association also sought to attack the ban from Capitol Hill, helping
to draft and push a bill introduced by Representative Chris Stewart,
Republican of Utah, in 2016 that
\href{https://www.congress.gov/bill/114th-congress/house-bill/5135/all-info}{would
have overturned the sales prohibition}, according to Knox Williams. The
bill never got out of committee.

Neither Mr. Williams nor his brother was required to register as a
lobbyist at the federal level, his brother said, because they did not
spend 20 percent or more of their time lobbying. ``We made sure that we
were not hitting those thresholds to require us individually to
register,'' Knox Williams said.

The Williams brothers also tried to influence silencer policies in
various states, including in New Hampshire, where both
\href{https://www.documentcloud.org/documents/6987325-Knox-Williams-s-New-Hampshire-Lobbying.html}{registered}
as
\href{https://www.documentcloud.org/documents/6987324-Before-Shaping-Silencer-Policy-In-The-Trump-W-H.html}{lobbyists
in 2015}.

After Mr. Trump accepted the Republican nomination in the summer of
2016, their cause got a boost from a prominent figure, Donald Trump Jr.

Image

Joshua G. Waldron, founder and former chief executive of
SilencerCo.Credit...Jim Urquhart/Reuters

The candidate's son, an avid hunter,
\href{https://www.youtube.com/watch?v=0vlu2G5UkXk}{recorded a video} in
September 2016 with Joshua G. Waldron, a founding board member of the
suppressor association, expressing support for making silencers easier
to buy in the United States.

Mr. Waldron, who founded a company called SilencerCo, tells Mr. Trump in
the video ``there is no better person than your father to protect our
Second Amendment,'' and says he wants to ``try to get the people that
love firearms in our community and our industry'' to back the Trump
campaign.

The same month, Mr. Wiliams left the suppressor association to become
director of Election Day operations for Mr. Trump's campaign in North
Carolina. He worked as associate counsel on Mr. Trump's inaugural
committee before joining the Office of Management and Budget in January
2017.

He returned to the budget office last month and was detailed to the U.S.
Agency for Global Media, operator of the Voice of America broadcasting
network and other federally funded media outlets, as Mr. Trump shook up
the agency's leadership,
\href{https://www.nytimes.com/2020/06/17/us/politics/michael-pack-media-agency.html}{raising
questions about its editorial independence}.

On Monday Mr. Williams started as principal deputy general counsel at
the Department of Housing and Urban Development. A department spokesman
did not answer a question about Mr. Williams's housing policy
experience, but called him ``a first-rate attorney with immense
experience in this administration and in the public policy sphere.''

John Ismay contributed reporting.

Advertisement

\protect\hyperlink{after-bottom}{Continue reading the main story}

\hypertarget{site-index}{%
\subsection{Site Index}\label{site-index}}

\hypertarget{site-information-navigation}{%
\subsection{Site Information
Navigation}\label{site-information-navigation}}

\begin{itemize}
\tightlist
\item
  \href{https://help.nytimes.com/hc/en-us/articles/115014792127-Copyright-notice}{©~2020~The
  New York Times Company}
\end{itemize}

\begin{itemize}
\tightlist
\item
  \href{https://www.nytco.com/}{NYTCo}
\item
  \href{https://help.nytimes.com/hc/en-us/articles/115015385887-Contact-Us}{Contact
  Us}
\item
  \href{https://www.nytco.com/careers/}{Work with us}
\item
  \href{https://nytmediakit.com/}{Advertise}
\item
  \href{http://www.tbrandstudio.com/}{T Brand Studio}
\item
  \href{https://www.nytimes.com/privacy/cookie-policy\#how-do-i-manage-trackers}{Your
  Ad Choices}
\item
  \href{https://www.nytimes.com/privacy}{Privacy}
\item
  \href{https://help.nytimes.com/hc/en-us/articles/115014893428-Terms-of-service}{Terms
  of Service}
\item
  \href{https://help.nytimes.com/hc/en-us/articles/115014893968-Terms-of-sale}{Terms
  of Sale}
\item
  \href{https://spiderbites.nytimes.com}{Site Map}
\item
  \href{https://help.nytimes.com/hc/en-us}{Help}
\item
  \href{https://www.nytimes.com/subscription?campaignId=37WXW}{Subscriptions}
\end{itemize}
