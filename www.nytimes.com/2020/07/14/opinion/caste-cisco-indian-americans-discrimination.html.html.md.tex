Sections

SEARCH

\protect\hyperlink{site-content}{Skip to
content}\protect\hyperlink{site-index}{Skip to site index}

\href{https://myaccount.nytimes.com/auth/login?response_type=cookie\&client_id=vi}{}

\href{https://www.nytimes.com/section/todayspaper}{Today's Paper}

\href{/section/opinion}{Opinion}\textbar{}The Specter of Caste in
Silicon Valley

\href{https://nyti.ms/3fA5mhe}{https://nyti.ms/3fA5mhe}

\begin{itemize}
\item
\item
\item
\item
\item
\end{itemize}

Advertisement

\protect\hyperlink{after-top}{Continue reading the main story}

\href{/section/opinion}{Opinion}

Supported by

\protect\hyperlink{after-sponsor}{Continue reading the main story}

\hypertarget{the-specter-of-caste-in-silicon-valley}{%
\section{The Specter of Caste in Silicon
Valley}\label{the-specter-of-caste-in-silicon-valley}}

Indian immigrants from Dalit backgrounds are rising up against caste
discrimination at their workplaces in the United States.

By Yashica Dutt

Ms. Dutt is the author of the memoir, ``Coming Out as a Dalit.''

\begin{itemize}
\item
  July 14, 2020
\item
  \begin{itemize}
  \item
  \item
  \item
  \item
  \item
  \end{itemize}
\end{itemize}

\includegraphics{https://static01.nyt.com/images/2020/07/14/opinion/14Dutt/14Dutt-articleLarge.jpg?quality=75\&auto=webp\&disable=upscale}

On June 30, California's Department of Fair Employment and Housing
regulators sued Cisco Systems Inc., for discrimination. The cause was
not, like most workplace discrimination lawsuits, based on race, gender,
age or sexual orientation. It was based on caste.

The lawsuit accuses Cisco, a multibillion-dollar tech conglomerate based
in San Jose, Calif., of denying an engineer, who immigrated from India
to the United States, **** professional opportunities, a raise and
promotions because he was from a low caste, or Dalit, background. The
lawsuit states that his Indian-American managers, Sundar Iyer and Ramana
Kompella, who are described as high-caste Brahmins, **** harassed the
engineer because of their sense of superiority rooted in the Hindu caste
system.

Many Indian-Americans reacted with disbelief that a giant corporation in
Silicon Valley could be mired in caste discrimination. For Dalit
Americans like me, it was just another Wednesday.

Dalit, which means ``oppressed,'' is a self-chosen identity for close to
25 percent of India's population, and it refers to former
``untouchables,'' the people who suffer the greatest violence,
discrimination and disenfranchisement under the centuries-old caste
system that structures Hindu society.

Caste is the gear that turns every system in India. ``If Hindus migrate
to other regions on earth, Indian Caste would become a world problem,''
\href{http://www.columbia.edu/itc/mealac/pritchett/00ambedkar/txt_ambedkar_castes.html}{B.R.
Ambedkar}, the greatest Dalit leader and one of the architects of the
India Constitution, wrote in 1916. He was prophetic.

Caste prejudice and discrimination is rife within the Indian communities
in the United States and other countries. Its chains are even turning
the work culture within multibillion-dollar American tech companies, and
beyond. The Cisco engineer, whose complaint led to the lawsuit and who
identifies himself as a Dalit, has not been named in the lawsuit.

From the mid-1990s, American companies, panicking at the feared
``millennial meltdown'' of computer systems, were
\href{https://www8.gsb.columbia.edu/articles/chazen-global-insights/singular-population-indian-immigrants-america}{hiring}
close to 100,000 technology workers a year from India. An overwhelming
majority of the Indian information technology professionals who moved to
the United States were from ``higher castes,'' and only a handful were
Dalits.

Over the Fourth of July weekend, I participated in a video call with
about 30 Dalit Indian immigrants. A Dalit information technology
professional on the video call spoke about moving to the United States
in 2000 and working at Cisco between 2007 and 2013. ``A large percentage
of the work force was already Indian," he told us. ``They openly
discussed their caste and would ask questions to figure out my caste
background.''

Higher caste Indians use the knowledge of a person's caste to place him
or her on the social hierarchy despite professional qualifications. ``I
usually ignored these conversations,'' the Dalit worker added. ``If they
knew I was Dalit, it could ruin my career.''

According to \href{https://regmedia.co.uk/2020/07/01/cisco.pdf}{the
lawsuit}, Mr. Iyer, one of the Brahmin engineers at Cisco, revealed to
his other higher-caste colleagues that the complainant had joined a top
engineering school in India through affirmative action. When the Dalit
engineer, the lawsuit says, confronted Mr. Iyer and contacted Cisco's
human resources to file a complaint, Mr. Iyer retaliated by taking away
the Dalit engineer's role as lead on two technologies.

For two years, the lawsuit says, Mr. Iyer isolated the Dalit engineer,
denied him bonuses and raises and stonewalled his promotions. Cisco's
human resources department responded by telling the Dalit engineer that
``caste discrimination was not unlawful'' and took no immediate
corrective action. Mr. Kompella, the other Brahmin manager named in the
lawsuit, replaced Mr. Iyer as the Dalit engineer's manager, and
according to the suit, ``continued to discriminate, harass, and
retaliate against'' him.

In 2019, Cisco was
\href{https://blogs.cisco.com/diversity/diversity-award}{ranked No. 2}
on Fortune's 100 Best Workplaces for Diversity. The technology giant got
away with ignoring the persistent caste discrimination because American
laws don't yet recognize Hindu caste discrimination as a valid form of
exclusion. Caste does not feature in Cisco's diversity practices in its
operations in India either. It reveals how the Indian information
technology sector often operates in willful ignorance of the terrifying
realities of caste.

In ``The Other One Percent: Indians in America,'' a 2016 study of people
of Indian descent in the United States, the authors Sanjoy Chakravorty,
Devesh Kapur and Nirvikar Singh estimated that
``\href{https://indianexpress.com/article/lifestyle/books/the-other-one-percent-indians-in-america-book-review-migration-4419534/}{over
90 percent of migrants}'' came from high castes or dominant castes.
According to a 2018
\href{https://static1.squarespace.com/static/58347d04bebafbb1e66df84c/t/5d9b4f9afbaef569c0a5c132/1570459664518/Caste_report_2018.pdf}{survey}
by Equality Labs, a Dalit-American led civil rights organization, 67
percent of Dalits in the Indian diaspora admitted to facing caste-based
harassment at the workplace.

In the backdrop of caste supremacy in the Indian diaspora in the United
States, when higher-caste Hindus often describe and demonize Dalits as
``inherently lazy/ opportunistic/ not talented,'' even apparently
innocuous practices like peer reviews for promotions (Cisco and several
other tech companies operate on this model), can turn into minefields,
ending in job losses and visa rejections for Dalits.

Almost every Dalit person I spoke to in the United States, after
California filed the lawsuit against Cisco, requested to remain
anonymous and feared that revealing their identity as a Dalit working in
the American tech industry filled with higher-caste Indians would ruin
their career.

Those words also governed my life until 2016, when I decided to publicly
reveal my caste identity and ``come out'' as Dalit. Growing up
``passing'' as a dominant-caste person in India while hiding my
``untouchable,'' caste I lived in the same fear that stops most Dalits
from articulating their harassment and asserting their identity in India
and the United States.

The overwhelmingly higher-caste Indian-American community is seen as a
``model minority'' with more than an
\href{https://www8.gsb.columbia.edu/articles/chazen-global-insights/singular-population-indian-immigrants-america}{average
\$100,000 median income} and rising cultural and political visibility.
But it has engendered a narrative that is as diabolical as it is in
India: insisting that they live in a ``post-caste world'' while
simultaneously upholding its hierarchical framework that benefits the
higher-caste people.

Ranging from seemingly harmless calls for ``vegetarian-only roommates''
(an easy way to assert caste purity), caste-based temple networks that
automatically exclude ``impure'' Dalits, and the more overt and
dangerous arm twisting of American norms --- right-wing Hindu activist
organizations
\href{https://www.nytimes.com/2016/05/06/us/debate-erupts-over-californias-india-history-curriculum.html}{tried}
to remove any mention of caste from California's textbooks in 2018 ---
caste supremacy is fiercely defended, almost as a core tenet of Indian
Hindu culture.

Yet after decades of being silenced, Dalit Americans are finally finding
a voice that cannot be ignored. I was able to come out as Dalit because
after moving to New York and avoiding Indian-only communities, for the
first time, I was not scared of someone finding out my caste. Finding
comfort and inspiration in movements like Black Lives Matter and Say Her
Name and the tragic institutional
\href{https://www.theguardian.com/global-development/2020/feb/19/coming-out-as-dalit-how-one-indian-author-finally-embraced-her-identity}{murder}
of a Dalit student activist in India, I was able to understand and
acknowledge that my history was a tapestry of pride, not shame.

Most Dalits in America still live with the fear of being exposed. But
the pending California vs. Cisco case is a major step in the right
direction.

Yashica Dutt is an Indian journalist and the author of the memoir,
``Coming Out as a Dalit.''

\emph{The Times is committed to publishing}
\href{https://www.nytimes.com/2019/01/31/opinion/letters/letters-to-editor-new-york-times-women.html}{\emph{a
diversity of letters}} \emph{to the editor. We'd like to hear what you
think about this or any of our articles. Here are some}
\href{https://help.nytimes.com/hc/en-us/articles/115014925288-How-to-submit-a-letter-to-the-editor}{\emph{tips}}\emph{.
And here's our email:}
\href{mailto:letters@nytimes.com}{\emph{letters@nytimes.com}}\emph{.}

\emph{Follow The New York Times Opinion section on}
\href{https://www.facebook.com/nytopinion}{\emph{Facebook}}\emph{,}
\href{http://twitter.com/NYTOpinion}{\emph{Twitter (@NYTopinion)}}
\emph{and}
\href{https://www.instagram.com/nytopinion/}{\emph{Instagram}}\emph{.}

Advertisement

\protect\hyperlink{after-bottom}{Continue reading the main story}

\hypertarget{site-index}{%
\subsection{Site Index}\label{site-index}}

\hypertarget{site-information-navigation}{%
\subsection{Site Information
Navigation}\label{site-information-navigation}}

\begin{itemize}
\tightlist
\item
  \href{https://help.nytimes.com/hc/en-us/articles/115014792127-Copyright-notice}{©~2020~The
  New York Times Company}
\end{itemize}

\begin{itemize}
\tightlist
\item
  \href{https://www.nytco.com/}{NYTCo}
\item
  \href{https://help.nytimes.com/hc/en-us/articles/115015385887-Contact-Us}{Contact
  Us}
\item
  \href{https://www.nytco.com/careers/}{Work with us}
\item
  \href{https://nytmediakit.com/}{Advertise}
\item
  \href{http://www.tbrandstudio.com/}{T Brand Studio}
\item
  \href{https://www.nytimes.com/privacy/cookie-policy\#how-do-i-manage-trackers}{Your
  Ad Choices}
\item
  \href{https://www.nytimes.com/privacy}{Privacy}
\item
  \href{https://help.nytimes.com/hc/en-us/articles/115014893428-Terms-of-service}{Terms
  of Service}
\item
  \href{https://help.nytimes.com/hc/en-us/articles/115014893968-Terms-of-sale}{Terms
  of Sale}
\item
  \href{https://spiderbites.nytimes.com}{Site Map}
\item
  \href{https://help.nytimes.com/hc/en-us}{Help}
\item
  \href{https://www.nytimes.com/subscription?campaignId=37WXW}{Subscriptions}
\end{itemize}
