Sections

SEARCH

\protect\hyperlink{site-content}{Skip to
content}\protect\hyperlink{site-index}{Skip to site index}

\href{https://myaccount.nytimes.com/auth/login?response_type=cookie\&client_id=vi}{}

\href{https://www.nytimes.com/section/todayspaper}{Today's Paper}

\href{/section/opinion}{Opinion}\textbar{}The Pandemic Could Get Much,
Much Worse. We Must Act Now.

\href{https://nyti.ms/38YdA06}{https://nyti.ms/38YdA06}

\begin{itemize}
\item
\item
\item
\item
\item
\item
\end{itemize}

Advertisement

\protect\hyperlink{after-top}{Continue reading the main story}

\href{/section/opinion}{Opinion}

Supported by

\protect\hyperlink{after-sponsor}{Continue reading the main story}

\hypertarget{the-pandemic-could-get-much-much-worse-we-must-act-now}{%
\section{The Pandemic Could Get Much, Much Worse. We Must Act
Now.}\label{the-pandemic-could-get-much-much-worse-we-must-act-now}}

A comprehensive shutdown may be required in much of the country.

By John M. Barry

Mr. Barry is the author of ``The Great Influenza: The Story of the
Deadliest Pandemic in History.''

\begin{itemize}
\item
  July 14, 2020
\item
  \begin{itemize}
  \item
  \item
  \item
  \item
  \item
  \item
  \end{itemize}
\end{itemize}

\includegraphics{https://static01.nyt.com/images/2020/07/14/opinion/sunday/14barry/14barry-articleLarge.jpg?quality=75\&auto=webp\&disable=upscale}

\href{https://www.nytimes.com/es/2020/07/16/espanol/opinion/coronavirus-cuarentena.html}{Leer
en español}

\emph{This article has been updated to reflect news developments.}

When you mix science and politics, you get politics. With the
coronavirus, the United States has proved politics hasn't worked. If we
are to fully reopen both the economy and schools safely --- which can be
done --- we have to return to science.

To understand just how bad things are in the United States and, more
important, what can be done about it requires comparison. At this
writing,
\href{https://www.nytimes.com/interactive/2020/world/coronavirus-maps.html}{Italy},
once the poster child of coronavirus devastation and with a population
twice that of Texas, has recently averaged about 200 new cases a day
when
\href{https://www.nytimes.com/interactive/2020/us/texas-coronavirus-cases.html}{Texas}
has had over 9,000. Germany, with a population four times that of
Florida, has had fewer than 400 new cases a day. On Sunday, Florida
\href{https://www.nytimes.com/2020/07/12/us/florida-coronavirus-covid-cases.html}{reported
over 15,300}, the highest single-day total of any state.

The White House says the country has to learn to live with the virus.
That's one thing if new cases occurred at the rates in Italy or Germany,
not to mention South Korea or Australia or Vietnam (which so far has
zero deaths). It's another thing when the United States has the
\href{https://coronavirus.jhu.edu/data/new-cases}{highest growth rate}
of new cases in the world, ahead even of Brazil.

Italy, Germany and dozens of other countries have reopened almost
entirely, and they had every reason to do so. They all took the virus
seriously and acted decisively, and they continue to: Australia just
issued fines totaling \$18,000 because too many people attended a
birthday party in someone's home.

In the United States, public health experts were virtually unanimous
that replicating European success required, first, maintaining the
shutdown until we achieved a steep downward slope in cases; second,
getting widespread compliance with public health advice; and third,
creating a work force of at least 100,000 --- some experts felt 300,000
were needed --- to test, trace and isolate cases. Nationally we came
nowhere near any of those goals, although some states did and are now
reopening carefully and safely. Other states fell far short but reopened
anyway. We now see the results.

The
\href{https://www.nytimes.com/interactive/2020/us/coronavirus-us-cases.html}{pandemic
is growing} across 39 states. In Miami-Dade County in Florida, six
hospitals have reached capacity. In Houston, where one of the country's
worst outbreaks rages, officials have called on the governor to issue a
stay-at-home order.

As if explosive growth in too many states isn't bad enough, we are also
suffering the same shortages that haunted hospitals in March and April.
In New Orleans, testing supplies are so limited that one site started
testing at 8 a.m. but had only enough to handle the people lined up by
7:33 a.m.

And testing by itself does little without an infrastructure to not only
trace and contact potentially infected people but also manage and
support those who test positive and are isolated along with those urged
to quarantine. Too often this has not been done; in Miami,
\href{https://miami.cbslocal.com/2020/07/09/mayors-coalition-wants-more-contact-tracers-miami-dade-county/}{only
17 percent} of those testing positive for the coronavirus had completed
questionnaires to help with contact tracing, critical to slowing spread.
Many states now have so many cases that contact tracing has become
impossible anyway.

What's the answer?

Social distancing, masks, hand washing and self-quarantine remain
crucial. Too little emphasis has been placed on ventilation, which also
matters. Ultraviolet lights can be installed in public areas. These
things will reduce spread, and President Trump finally wore a mask
publicly, which may somewhat depoliticize the issue. But at this point
all these things together, even with widespread compliance, can only
blunt dangerous trends where they are occurring. The virus is already
too widely disseminated for these actions to quickly bend the curve
downward.

To reopen schools in the safest way, which may be impossible in some
instances, and to get the economy fully back on track, we must get the
case counts down to manageable levels --- down to the levels of European
countries. The Trump administration's threat to withhold federal funds
from schools that don't reopen won't accomplish that goal. To do that,
only decisive action will work in places experiencing explosive growth
--- at the very least, limits even on private gatherings and selective
shutdowns that must include not just such obvious places as ****** bars
but ****** churches***,*** also a well-documented source of large-scale
spread.

Depending on local circumstances, that may prove insufficient; a
comprehensive April-like shutdown may be required. This could be on a
county-by-county basis, but half-measures will do little more than
prevent hospitals from being overrun. Half-measures will leave
transmission at a level vastly exceeding those of the many countries
that have contained the virus. Half-measures will leave too many
Americans not living with the virus but dying from it.

During the 1918 influenza pandemic, almost every city closed down much
of its activity. Fear and caring for sick family members did the rest;
absenteeism even in war industries exceeded 50 percent and eviscerated
the economy. Many cities reopened too soon and had to close a second
time --- sometimes a third time --- and faced intense resistance. But
lives were saved.

Had we done it right the first time, we'd be operating at near 100
percent now, schools would be preparing for a nearly normal school year,
football teams would be preparing to practice --- and tens of thousands
of Americans would not have died.

This is our second chance. We won't get a third. If we don't get the
growth of this pandemic under control now, in a few months, when the
weather turns cold and forces people to spend more time indoors, we
could face a disaster that dwarfs the situation today.

\href{http://www.johnmbarry.com/index.htm}{John M. Barry} is a professor
at the Tulane University School of Public Health and Tropical Medicine
and the author of ``The Great Influenza: The Story of the Deadliest
Pandemic in History.''

\emph{The Times is committed to publishing}
\href{https://www.nytimes.com/2019/01/31/opinion/letters/letters-to-editor-new-york-times-women.html}{\emph{a
diversity of letters}} \emph{to the editor. We'd like to hear what you
think about this or any of our articles. Here are some}
\href{https://help.nytimes.com/hc/en-us/articles/115014925288-How-to-submit-a-letter-to-the-editor}{\emph{tips}}\emph{.
And here's our email:}
\href{mailto:letters@nytimes.com}{\emph{letters@nytimes.com}}\emph{.}

\emph{Follow The New York Times Opinion section on}
\href{https://www.facebook.com/nytopinion}{\emph{Facebook}}\emph{,}
\href{http://twitter.com/NYTOpinion}{\emph{Twitter (@NYTopinion)}}
\emph{and}
\href{https://www.instagram.com/nytopinion/}{\emph{Instagram}}\emph{.}

Advertisement

\protect\hyperlink{after-bottom}{Continue reading the main story}

\hypertarget{site-index}{%
\subsection{Site Index}\label{site-index}}

\hypertarget{site-information-navigation}{%
\subsection{Site Information
Navigation}\label{site-information-navigation}}

\begin{itemize}
\tightlist
\item
  \href{https://help.nytimes.com/hc/en-us/articles/115014792127-Copyright-notice}{©~2020~The
  New York Times Company}
\end{itemize}

\begin{itemize}
\tightlist
\item
  \href{https://www.nytco.com/}{NYTCo}
\item
  \href{https://help.nytimes.com/hc/en-us/articles/115015385887-Contact-Us}{Contact
  Us}
\item
  \href{https://www.nytco.com/careers/}{Work with us}
\item
  \href{https://nytmediakit.com/}{Advertise}
\item
  \href{http://www.tbrandstudio.com/}{T Brand Studio}
\item
  \href{https://www.nytimes.com/privacy/cookie-policy\#how-do-i-manage-trackers}{Your
  Ad Choices}
\item
  \href{https://www.nytimes.com/privacy}{Privacy}
\item
  \href{https://help.nytimes.com/hc/en-us/articles/115014893428-Terms-of-service}{Terms
  of Service}
\item
  \href{https://help.nytimes.com/hc/en-us/articles/115014893968-Terms-of-sale}{Terms
  of Sale}
\item
  \href{https://spiderbites.nytimes.com}{Site Map}
\item
  \href{https://help.nytimes.com/hc/en-us}{Help}
\item
  \href{https://www.nytimes.com/subscription?campaignId=37WXW}{Subscriptions}
\end{itemize}
