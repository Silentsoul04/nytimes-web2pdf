Sections

SEARCH

\protect\hyperlink{site-content}{Skip to
content}\protect\hyperlink{site-index}{Skip to site index}

\href{https://www.nytimes.com/section/politics}{Politics}

\href{https://myaccount.nytimes.com/auth/login?response_type=cookie\&client_id=vi}{}

\href{https://www.nytimes.com/section/todayspaper}{Today's Paper}

\href{/section/politics}{Politics}\textbar{}Government Carries Out First
Federal Execution in 17 Years

\url{https://nyti.ms/2ZtqmQZ}

\begin{itemize}
\item
\item
\item
\item
\item
\item
\end{itemize}

\href{https://www.nytimes.com/news-event/coronavirus?action=click\&pgtype=Article\&state=default\&region=TOP_BANNER\&context=storylines_menu}{The
Coronavirus Outbreak}

\begin{itemize}
\tightlist
\item
  live\href{https://www.nytimes.com/2020/08/01/world/coronavirus-covid-19.html?action=click\&pgtype=Article\&state=default\&region=TOP_BANNER\&context=storylines_menu}{Latest
  Updates}
\item
  \href{https://www.nytimes.com/interactive/2020/us/coronavirus-us-cases.html?action=click\&pgtype=Article\&state=default\&region=TOP_BANNER\&context=storylines_menu}{Maps
  and Cases}
\item
  \href{https://www.nytimes.com/interactive/2020/science/coronavirus-vaccine-tracker.html?action=click\&pgtype=Article\&state=default\&region=TOP_BANNER\&context=storylines_menu}{Vaccine
  Tracker}
\item
  \href{https://www.nytimes.com/interactive/2020/07/29/us/schools-reopening-coronavirus.html?action=click\&pgtype=Article\&state=default\&region=TOP_BANNER\&context=storylines_menu}{What
  School May Look Like}
\item
  \href{https://www.nytimes.com/live/2020/07/31/business/stock-market-today-coronavirus?action=click\&pgtype=Article\&state=default\&region=TOP_BANNER\&context=storylines_menu}{Economy}
\end{itemize}

Advertisement

\protect\hyperlink{after-top}{Continue reading the main story}

Supported by

\protect\hyperlink{after-sponsor}{Continue reading the main story}

\hypertarget{government-carries-out-first-federal-execution-in-17-years}{%
\section{Government Carries Out First Federal Execution in 17
Years}\label{government-carries-out-first-federal-execution-in-17-years}}

Hours after a 5-to-4 vote by the Supreme Court, Daniel Lewis Lee was put
to death by lethal injection in Terre Haute, Ind., for his role in the
1996 murder of a family of three.

\includegraphics{https://static01.nyt.com/images/2020/07/14/us/politics/14dc-execution1/merlin_174534225_c73d2cd2-782c-4c9a-8da1-ccaf23306673-articleLarge.jpg?quality=75\&auto=webp\&disable=upscale}

By Hailey Fuchs

\begin{itemize}
\item
  July 14, 2020
\item
  \begin{itemize}
  \item
  \item
  \item
  \item
  \item
  \item
  \end{itemize}
\end{itemize}

WASHINGTON --- Hours after the Supreme Court
\href{https://www.nytimes.com/2020/07/12/us/politics/execution-daniel-lewis-lee.html}{rejected}
a last-minute legal challenge on a 5-to-4 vote, the Justice Department
put a man to death for his role in the 1996 murder of a family of three,
the first
\href{https://www.nytimes.com/2020/07/16/us/politics/wesley-ira-purkey-executed.html}{federal
execution} in more than 17 years.

The death row prisoner,
\href{https://www.nytimes.com/2019/10/29/us/arkansas-federal-death-penalty.html}{Daniel
Lewis Lee}, 47, a former white supremacist who renounced his ties to
that movement, was executed by lethal injection at the federal
penitentiary in Terre Haute, Ind., the Bureau of Prisons said. He is the
first of three federal inmates scheduled for execution this week.

Mr. Lee's death ended an informal moratorium on federal capital
punishment.

The Trump administration
\href{https://www.justice.gov/opa/pr/executions-scheduled-four-federal-inmates-convicted-murdering-children}{announced}
last summer its intention to resume the federal death penalty and to
employ a new procedure to carry it out --- using a single drug,
pentobarbital --- after several botched executions by lethal injection
renewed scrutiny of capital punishment.

But up until the final hours before Mr. Lee's death, the government had
to fight off legal challenges based on use of the single-drug technique
and the complications of carrying out the death penalty
\href{https://www.nytimes.com/2020/06/30/us/politics/federal-executions-pandemic.html}{during
a pandemic}.

On Monday, a federal judge had delayed the execution, saying that
questions about the constitutionality of the lethal injection procedure
had not been fully litigated.

The Justice Department immediately appealed the ruling by Judge Tanya S.
Chutkan of the United States District Court in Washington. In issuing a
\href{https://ecf.dcd.uscourts.gov/cgi-bin/show_public_doc?2019mc0145-135}{preliminary
injunction} against the execution of Mr. Lee, Judge Chutkan cited the
``extreme pain and needless suffering'' that could result from the
lethal injection protocol the government planned to use.

The Supreme Court's
\href{https://www.supremecourt.gov/opinions/19pdf/20a8_970e.pdf}{unsigned
5-to-4 ruling} early Tuesday morning said pentobarbital had been used in
over 100 executions ``without incident'' and had been upheld by the
Supreme Court and appeals courts.

``The plaintiffs in this case have not made the showing required to
justify last-minute intervention by a federal court,'' the unsigned
order said, quoting from a
\href{https://www.supremecourt.gov/opinions/18pdf/17-8151_1qm2.pdf}{decision
last year}. ```Last-minute stays' like that issued this morning `should
be the extreme exception, not the norm.'''

The court said it was its responsibility ``to ensure that
method-of-execution challenges to lawfully issued sentences are resolved
fairly and expeditiously,'' so that ``the question of capital
punishment'' can remain with ``the people and their representatives, not
the courts, to resolve.''

In dissent, Justice Stephen G. Breyer, joined by Justice Ruth Bader
Ginsburg, repeated their
\href{https://www.nytimes.com/2015/11/04/us/politics/death-penalty-opponents-split-over-taking-issue-to-supreme-court.html}{longstanding
doubts} about the constitutionality of the death penalty. ``The
resumption of federal executions promises to provide examples that
illustrate the difficulties of administering the death penalty
consistent with the Constitution,'' he wrote.

\includegraphics{https://static01.nyt.com/images/2020/07/14/us/politics/14dc-execution2/merlin_174529788_3a262795-6f57-4ac4-b98a-04a37697c497-articleLarge.jpg?quality=75\&auto=webp\&disable=upscale}

In a second dissent, Justice Sonia Sotomayor, joined by Justices
Ginsburg and Elena Kagan, said the court had acted with dangerous haste.

``Today's decision illustrates just how grave the consequences of such
accelerated decision making can be,'' Justice Sotomayor wrote. ``The
court forever deprives respondents of their ability to press a
constitutional challenge to their lethal injections, and prevents lower
courts from reviewing that challenge.''

\hypertarget{latest-updates-global-coronavirus-outbreak}{%
\section{\texorpdfstring{\href{https://www.nytimes.com/2020/08/01/world/coronavirus-covid-19.html?action=click\&pgtype=Article\&state=default\&region=MAIN_CONTENT_1\&context=storylines_live_updates}{Latest
Updates: Global Coronavirus
Outbreak}}{Latest Updates: Global Coronavirus Outbreak}}\label{latest-updates-global-coronavirus-outbreak}}

Updated 2020-08-02T10:04:29.623Z

\begin{itemize}
\tightlist
\item
  \href{https://www.nytimes.com/2020/08/01/world/coronavirus-covid-19.html?action=click\&pgtype=Article\&state=default\&region=MAIN_CONTENT_1\&context=storylines_live_updates\#link-34047410}{The
  U.S. reels as July cases more than double the total of any other
  month.}
\item
  \href{https://www.nytimes.com/2020/08/01/world/coronavirus-covid-19.html?action=click\&pgtype=Article\&state=default\&region=MAIN_CONTENT_1\&context=storylines_live_updates\#link-780ec966}{Top
  U.S. officials work to break an impasse over the federal jobless
  benefit.}
\item
  \href{https://www.nytimes.com/2020/08/01/world/coronavirus-covid-19.html?action=click\&pgtype=Article\&state=default\&region=MAIN_CONTENT_1\&context=storylines_live_updates\#link-2bc8948}{Its
  outbreak untamed, Melbourne goes into even greater lockdown.}
\end{itemize}

\href{https://www.nytimes.com/2020/08/01/world/coronavirus-covid-19.html?action=click\&pgtype=Article\&state=default\&region=MAIN_CONTENT_1\&context=storylines_live_updates}{See
more updates}

More live coverage:
\href{https://www.nytimes.com/live/2020/07/31/business/stock-market-today-coronavirus?action=click\&pgtype=Article\&state=default\&region=MAIN_CONTENT_1\&context=storylines_live_updates}{Markets}

Last month, the
\href{https://www.nytimes.com/2020/06/29/us/supreme-court-executions.html}{Supreme
Court let stand} an appeals court ruling that found the government was
in compliance with the
\href{https://www.law.cornell.edu/uscode/text/18/3596}{Federal Death
Penalty Act of 1994}, which requires executions to be carried out ``in
the manner prescribed by the law of the state in which the sentence is
imposed.'' Judge Chutkan had
\href{https://src.bna.com/MZD?_ga=2.258585482.1273884090.1575491003-907374773.1567693399}{found}
the government in violation of the law.

Last week, family members of Mr. Lee's victims sued the Justice
Department, arguing that traveling to the execution site would put them
at risk of contracting the coronavirus. A district court agreed and
\href{https://www.nytimes.com/2020/07/10/us/politics/first-federal-execution-delay.html}{granted
a temporary delay} in the execution. Late Sunday, the U.S. Court of
Appeals for the Seventh Circuit
\href{https://www.nytimes.com/2020/07/12/us/politics/execution-daniel-lewis-lee.html}{reversed
that decision}. The Supreme Court also declined the family members'
petition early Tuesday morning.

According to his lawyers, Mr. Lee was strapped to a gurney for the final
four hours of his life, while the legal challenges played out.

Two men stood beside him in the execution chamber, a U.S. Marshal and a
spiritual adviser, whom the Bureau of Prisons referred to as an
``Appalachian pagan minister.'' Neither wore a mask.

``I didn't do it,'' Mr. Lee said, according to a report from journalists
who were at the scene. ``I've made a lot of mistakes in my life, but I'm
not a murderer.'' He claimed the judge in his trial in Arkansas
overlooked DNA evidence that proved he was across the country at the
time of the murders.

After a senior bureau official told him that he was to be put to death,
Mr. Lee shook his head. As the drug was administered to his veins, he
raised his head to look around. Then his breaths became heavy. In just a
few short moments, his chest remained still, his lips turned blue, and
his fingers became ashy. He was pronounced dead at 8:07 a.m.

The federal government is scheduled to carry out two more executions
this week. A court has issued a
\href{http://media.ca7.uscourts.gov/cgi-bin/rssExec.pl?Submit=Display\&Path=Y2020/D07-13/C:19-3318:J:PerCuriam:aut:T:npDp:N:2544729:S:0}{temporary
stay} in the case of Wesley Ira Purkey, 68, whom the Justice Department
scheduled for execution on Wednesday for the 1998 killing and
dismembering of a teenage girl in Kansas City.

Dustin Lee Honken, 52, convicted of killing three adults and two young
girls in 1993, will be executed on Friday, barring any last-minute
stays. Another prisoner, Keith Dwayne Nelson, 46, will face execution in
August for the rape and murder of a 10-year-old girl in 1999.

\href{https://www.nytimes.com/news-event/coronavirus?action=click\&pgtype=Article\&state=default\&region=MAIN_CONTENT_3\&context=storylines_faq}{}

\hypertarget{the-coronavirus-outbreak-}{%
\subsubsection{The Coronavirus Outbreak
›}\label{the-coronavirus-outbreak-}}

\hypertarget{frequently-asked-questions}{%
\paragraph{Frequently Asked
Questions}\label{frequently-asked-questions}}

Updated July 27, 2020

\begin{itemize}
\item ~
  \hypertarget{should-i-refinance-my-mortgage}{%
  \paragraph{Should I refinance my
  mortgage?}\label{should-i-refinance-my-mortgage}}

  \begin{itemize}
  \tightlist
  \item
    \href{https://www.nytimes.com/article/coronavirus-money-unemployment.html?action=click\&pgtype=Article\&state=default\&region=MAIN_CONTENT_3\&context=storylines_faq}{It
    could be a good idea,} because mortgage rates have
    \href{https://www.nytimes.com/2020/07/16/business/mortgage-rates-below-3-percent.html?action=click\&pgtype=Article\&state=default\&region=MAIN_CONTENT_3\&context=storylines_faq}{never
    been lower.} Refinancing requests have pushed mortgage applications
    to some of the highest levels since 2008, so be prepared to get in
    line. But defaults are also up, so if you're thinking about buying a
    home, be aware that some lenders have tightened their standards.
  \end{itemize}
\item ~
  \hypertarget{what-is-school-going-to-look-like-in-september}{%
  \paragraph{What is school going to look like in
  September?}\label{what-is-school-going-to-look-like-in-september}}

  \begin{itemize}
  \tightlist
  \item
    It is unlikely that many schools will return to a normal schedule
    this fall, requiring the grind of
    \href{https://www.nytimes.com/2020/06/05/us/coronavirus-education-lost-learning.html?action=click\&pgtype=Article\&state=default\&region=MAIN_CONTENT_3\&context=storylines_faq}{online
    learning},
    \href{https://www.nytimes.com/2020/05/29/us/coronavirus-child-care-centers.html?action=click\&pgtype=Article\&state=default\&region=MAIN_CONTENT_3\&context=storylines_faq}{makeshift
    child care} and
    \href{https://www.nytimes.com/2020/06/03/business/economy/coronavirus-working-women.html?action=click\&pgtype=Article\&state=default\&region=MAIN_CONTENT_3\&context=storylines_faq}{stunted
    workdays} to continue. California's two largest public school
    districts --- Los Angeles and San Diego --- said on July 13, that
    \href{https://www.nytimes.com/2020/07/13/us/lausd-san-diego-school-reopening.html?action=click\&pgtype=Article\&state=default\&region=MAIN_CONTENT_3\&context=storylines_faq}{instruction
    will be remote-only in the fall}, citing concerns that surging
    coronavirus infections in their areas pose too dire a risk for
    students and teachers. Together, the two districts enroll some
    825,000 students. They are the largest in the country so far to
    abandon plans for even a partial physical return to classrooms when
    they reopen in August. For other districts, the solution won't be an
    all-or-nothing approach.
    \href{https://bioethics.jhu.edu/research-and-outreach/projects/eschool-initiative/school-policy-tracker/}{Many
    systems}, including the nation's largest, New York City, are
    devising
    \href{https://www.nytimes.com/2020/06/26/us/coronavirus-schools-reopen-fall.html?action=click\&pgtype=Article\&state=default\&region=MAIN_CONTENT_3\&context=storylines_faq}{hybrid
    plans} that involve spending some days in classrooms and other days
    online. There's no national policy on this yet, so check with your
    municipal school system regularly to see what is happening in your
    community.
  \end{itemize}
\item ~
  \hypertarget{is-the-coronavirus-airborne}{%
  \paragraph{Is the coronavirus
  airborne?}\label{is-the-coronavirus-airborne}}

  \begin{itemize}
  \tightlist
  \item
    The coronavirus
    \href{https://www.nytimes.com/2020/07/04/health/239-experts-with-one-big-claim-the-coronavirus-is-airborne.html?action=click\&pgtype=Article\&state=default\&region=MAIN_CONTENT_3\&context=storylines_faq}{can
    stay aloft for hours in tiny droplets in stagnant air}, infecting
    people as they inhale, mounting scientific evidence suggests. This
    risk is highest in crowded indoor spaces with poor ventilation, and
    may help explain super-spreading events reported in meatpacking
    plants, churches and restaurants.
    \href{https://www.nytimes.com/2020/07/06/health/coronavirus-airborne-aerosols.html?action=click\&pgtype=Article\&state=default\&region=MAIN_CONTENT_3\&context=storylines_faq}{It's
    unclear how often the virus is spread} via these tiny droplets, or
    aerosols, compared with larger droplets that are expelled when a
    sick person coughs or sneezes, or transmitted through contact with
    contaminated surfaces, said Linsey Marr, an aerosol expert at
    Virginia Tech. Aerosols are released even when a person without
    symptoms exhales, talks or sings, according to Dr. Marr and more
    than 200 other experts, who
    \href{https://academic.oup.com/cid/article/doi/10.1093/cid/ciaa939/5867798}{have
    outlined the evidence in an open letter to the World Health
    Organization}.
  \end{itemize}
\item ~
  \hypertarget{what-are-the-symptoms-of-coronavirus}{%
  \paragraph{What are the symptoms of
  coronavirus?}\label{what-are-the-symptoms-of-coronavirus}}

  \begin{itemize}
  \tightlist
  \item
    Common symptoms
    \href{https://www.nytimes.com/article/symptoms-coronavirus.html?action=click\&pgtype=Article\&state=default\&region=MAIN_CONTENT_3\&context=storylines_faq}{include
    fever, a dry cough, fatigue and difficulty breathing or shortness of
    breath.} Some of these symptoms overlap with those of the flu,
    making detection difficult, but runny noses and stuffy sinuses are
    less common.
    \href{https://www.nytimes.com/2020/04/27/health/coronavirus-symptoms-cdc.html?action=click\&pgtype=Article\&state=default\&region=MAIN_CONTENT_3\&context=storylines_faq}{The
    C.D.C. has also} added chills, muscle pain, sore throat, headache
    and a new loss of the sense of taste or smell as symptoms to look
    out for. Most people fall ill five to seven days after exposure, but
    symptoms may appear in as few as two days or as many as 14 days.
  \end{itemize}
\item ~
  \hypertarget{does-asymptomatic-transmission-of-covid-19-happen}{%
  \paragraph{Does asymptomatic transmission of Covid-19
  happen?}\label{does-asymptomatic-transmission-of-covid-19-happen}}

  \begin{itemize}
  \tightlist
  \item
    So far, the evidence seems to show it does. A widely cited
    \href{https://www.nature.com/articles/s41591-020-0869-5}{paper}
    published in April suggests that people are most infectious about
    two days before the onset of coronavirus symptoms and estimated that
    44 percent of new infections were a result of transmission from
    people who were not yet showing symptoms. Recently, a top expert at
    the World Health Organization stated that transmission of the
    coronavirus by people who did not have symptoms was ``very rare,''
    \href{https://www.nytimes.com/2020/06/09/world/coronavirus-updates.html?action=click\&pgtype=Article\&state=default\&region=MAIN_CONTENT_3\&context=storylines_faq\#link-1f302e21}{but
    she later walked back that statement.}
  \end{itemize}
\end{itemize}

On Tuesday, a federal judge in Indiana dismissed a suit brought by
spiritual advisers for Mr. Purkey and Mr. Honken that argued for a delay
on the grounds that their health could be jeopardized because of the
coronavirus crisis if they entered the Terre Haute penitentiary.

The Supreme Court struck down the death penalty in 1972, arguing that it
constituted ``cruel and unusual punishment.'' Four years later, it
reversed that decision, amid rising rates of violent crime. Since then,
some states have carried out regular executions, but only three men have
been put to death by the federal government. Most recently, Louis Jones
Jr. was executed in 2003 for the rape and murder of a female soldier.

Although dozens of federal inmates have been sentenced to death during
the hiatus, 22 states and the District of Columbia have abolished
capital punishment altogether, and an additional three have
governor-imposed moratoriums.

President Trump has advocated the death penalty since well before he
entered the Oval Office. In 1989, he called for the Central Park Five, a
group of Latino and Black teenagers wrongly convicted of raping a
jogger, to be put to death. More recently, he has called for the death
penalty for drug dealers and those convicted of killing police officers.

His campaign has criticized his Democratic opponent, former Vice
President Joseph R. Biden Jr., for opposing capital punishment
``\href{https://twitter.com/TrumpWarRoom/status/1274050283098734593}{even
for white supremacist murderers!}''

The executions scheduled for the summer come amid a presidential
election, a pandemic and intensive attention on policing and criminal
justice after the death of George Floyd.

Critics of capital punishment argue that Black offenders and
perpetrators of white victims are disproportionately sentenced to death.
The four men scheduled to be executed this summer are white, but 42
percent of prisoners on federal death row are Black.

Despite the race of the four inmates chosen by the Justice Department,
Samuel Spital, the director of litigation for the N.A.A.C.P. Legal
Defense Fund, said the resumption of the federal death penalty would
have the greatest effect on Black and Latino prisoners.

``The system is infected with racism,'' he said.

In the final years of Mr. Lee's life, several members of his victims'
family --- as well as the trial judge and prosecutor --- pleaded with
the government to commute his sentence to life in prison without the
possibility of parole. His accomplice in the crime, described as the
mastermind during the trial, was not sentenced to death.

``It is shameful that the government saw fit to carry out this execution
during a pandemic,'' Ruth Friedman, Mr. Lee's lawyer and the director of
the Federal Capital Habeas Project, said in a statement. ``It is
shameful that the government saw fit to carry out this execution when
counsel for Danny Lee could not be present with him, and when the judges
in his case and even the family of his victims urged against it.''

Adam Liptak contributed reporting.

Advertisement

\protect\hyperlink{after-bottom}{Continue reading the main story}

\hypertarget{site-index}{%
\subsection{Site Index}\label{site-index}}

\hypertarget{site-information-navigation}{%
\subsection{Site Information
Navigation}\label{site-information-navigation}}

\begin{itemize}
\tightlist
\item
  \href{https://help.nytimes.com/hc/en-us/articles/115014792127-Copyright-notice}{©~2020~The
  New York Times Company}
\end{itemize}

\begin{itemize}
\tightlist
\item
  \href{https://www.nytco.com/}{NYTCo}
\item
  \href{https://help.nytimes.com/hc/en-us/articles/115015385887-Contact-Us}{Contact
  Us}
\item
  \href{https://www.nytco.com/careers/}{Work with us}
\item
  \href{https://nytmediakit.com/}{Advertise}
\item
  \href{http://www.tbrandstudio.com/}{T Brand Studio}
\item
  \href{https://www.nytimes.com/privacy/cookie-policy\#how-do-i-manage-trackers}{Your
  Ad Choices}
\item
  \href{https://www.nytimes.com/privacy}{Privacy}
\item
  \href{https://help.nytimes.com/hc/en-us/articles/115014893428-Terms-of-service}{Terms
  of Service}
\item
  \href{https://help.nytimes.com/hc/en-us/articles/115014893968-Terms-of-sale}{Terms
  of Sale}
\item
  \href{https://spiderbites.nytimes.com}{Site Map}
\item
  \href{https://help.nytimes.com/hc/en-us}{Help}
\item
  \href{https://www.nytimes.com/subscription?campaignId=37WXW}{Subscriptions}
\end{itemize}
