Sections

SEARCH

\protect\hyperlink{site-content}{Skip to
content}\protect\hyperlink{site-index}{Skip to site index}

\href{https://www.nytimes.com/section/politics}{Politics}

\href{https://myaccount.nytimes.com/auth/login?response_type=cookie\&client_id=vi}{}

\href{https://www.nytimes.com/section/todayspaper}{Today's Paper}

\href{/section/politics}{Politics}\textbar{}The White House Called a
News Conference. Trump Turned It Into a Meandering Monologue.

\url{https://nyti.ms/3h0PRyZ}

\begin{itemize}
\item
\item
\item
\item
\item
\item
\end{itemize}

\begin{itemize}
\item
  \href{https://www.nytimes.com/2020/08/03/us/elections/biden-vs-trump.html?action=click\&pgtype=Article\&state=default\&region=TOP_BANNER\&context=storylines_menu}{Election
  Updates}
\item
  \href{https://www.nytimes.com/article/biden-vice-president-2020.html?action=click\&pgtype=Article\&state=default\&region=TOP_BANNER\&context=storylines_menu}{Biden's
  V.P. Search}
\item
  \href{https://www.nytimes.com/interactive/2020/07/24/us/politics/trump-biden-campaign-donors.html?action=click\&pgtype=Article\&state=default\&region=TOP_BANNER\&context=storylines_menu}{Map
  of Donations}
\item
  \href{https://www.nytimes.com/interactive/2020/us/elections/delegate-count-primary-results.html?action=click\&pgtype=Article\&state=default\&region=TOP_BANNER\&context=storylines_menu}{Delegate
  Count}
\item
  \href{https://www.nytimes.com/interactive/2019/us/politics/2020-presidential-candidates.html?action=click\&pgtype=Article\&state=default\&region=TOP_BANNER\&context=storylines_menu}{The
  Candidates}
\item
  \href{https://www.nytimes.com/newsletters/politics?action=click\&pgtype=Article\&state=default\&region=TOP_BANNER\&context=storylines_menu}{Politics
  Newsletter}
\end{itemize}

Advertisement

\protect\hyperlink{after-top}{Continue reading the main story}

Supported by

\protect\hyperlink{after-sponsor}{Continue reading the main story}

\hypertarget{the-white-house-called-a-news-conference-trump-turned-it-into-a-meandering-monologue}{%
\section{The White House Called a News Conference. Trump Turned It Into
a Meandering
Monologue.}\label{the-white-house-called-a-news-conference-trump-turned-it-into-a-meandering-monologue}}

The president spoke in the Rose Garden for 63 minutes. He spent only six
of those minutes answering questions from reporters.

\includegraphics{https://static01.nyt.com/images/2020/07/14/us/politics/14dc-trump2/14dc-trump2-articleLarge.jpg?quality=75\&auto=webp\&disable=upscale}

\href{https://www.nytimes.com/by/peter-baker}{\includegraphics{https://static01.nyt.com/images/2018/06/13/multimedia/peter-baker/peter-baker-thumbLarge-v2.png}}

By \href{https://www.nytimes.com/by/peter-baker}{Peter Baker}

\begin{itemize}
\item
  Published July 14, 2020Updated July 28, 2020
\item
  \begin{itemize}
  \item
  \item
  \item
  \item
  \item
  \item
  \end{itemize}
\end{itemize}

\href{https://cn.nytimes.com/usa/20200715/trump-news-conference/}{阅读简体中文版}\href{https://cn.nytimes.com/usa/20200715/trump-news-conference/zh-hant/}{閱讀繁體中文版}\href{https://www.nytimes.com/es/2020/07/15/espanol/estados-unidos/trump-conferencia-prensa.html}{Leer
en español}

WASHINGTON --- In theory,
\href{https://www.nytimes.com/2020/07/28/us/politics/donald-fred-trump.html}{President
Trump} summoned television cameras to the heat-baked Rose Garden early
Tuesday evening to announce new measures against China to punish it for
its oppression of Hong Kong. But that did not last long.

What followed instead was an hour of presidential stream of
consciousness as Mr. Trump drifted seemingly at random from one topic to
another, often in the same run-on sentence. Even for a president who
rarely sticks to the script and wanders from thought to thought, it was
one of the most rambling performances of his presidency.

He weighed in on China and the coronavirus and the Paris climate change
accord and crumbling highways. And then China again and military
spending and then China again and then the coronavirus again. And the
economy and energy taxes and trade with Europe and illegal immigration
and his friendship with Mexico's president. And the coronavirus again
and then immigration again and crime in Chicago and the death penalty
and back to climate change and education and historical statues. And
more.

``We could go on for days,'' he said at one point, and it sounded
plausible.

At times, it was hard to understand what he meant. He seemed to suggest
that his presumptive Democratic challenger, former Vice President Joseph
R. Biden Jr., would get rid of windows if elected and later said that
Mr. Biden would ``abolish the suburbs.'' He complained that Mr. Biden
had ``gone so far right.'' (He meant left.)

Even for those who follow Mr. Trump regularly and understand his
shorthand, it became challenging to follow his train of thought.

For instance, in discussing cooperation agreements with Central American
countries to stop illegal immigration, he had this to say: ``We have
great agreements where when Biden and Obama used to bring killers out,
they would say don't bring them back to our country, we don't want them.
Well, we have to, we don't want them. They wouldn't take them. Now with
us, they take them. Someday, I'll tell you why. Someday, I'll tell you
why. But they take them and they take them very gladly. They used to
bring them out and they wouldn't even let the airplanes land if they
brought them back by airplanes. They wouldn't let the buses into their
country. They said we don't want them. Said no, but they entered our
country illegally and they're murderers, they're killers in some
cases.''

\hypertarget{latest-updates-2020-election}{%
\section{\texorpdfstring{\href{https://www.nytimes.com/2020/08/03/us/elections/biden-vs-trump.html?action=click\&pgtype=Article\&state=default\&region=MAIN_CONTENT_1\&context=storylines_live_updates}{Latest
Updates: 2020
Election}}{Latest Updates: 2020 Election}}\label{latest-updates-2020-election}}

Updated 2020-08-04T01:23:51.312Z

\begin{itemize}
\tightlist
\item
  \href{https://www.nytimes.com/2020/08/03/us/elections/biden-vs-trump.html?action=click\&pgtype=Article\&state=default\&region=MAIN_CONTENT_1\&context=storylines_live_updates\#link-6494b448}{Trump
  assails mail-in voting anew, citing delays in declaring a winner in a
  New York congressional primary.}
\item
  \href{https://www.nytimes.com/2020/08/03/us/elections/biden-vs-trump.html?action=click\&pgtype=Article\&state=default\&region=MAIN_CONTENT_1\&context=storylines_live_updates\#link-3de249e6}{Obama
  issues his first slate of 2020 endorsements.}
\item
  \href{https://www.nytimes.com/2020/08/03/us/elections/biden-vs-trump.html?action=click\&pgtype=Article\&state=default\&region=MAIN_CONTENT_1\&context=storylines_live_updates\#link-54e34d20}{In
  a big shift, Trump is now encouraging mask-wearing in campaign
  emails.}
\end{itemize}

\href{https://www.nytimes.com/2020/08/03/us/elections/biden-vs-trump.html?action=click\&pgtype=Article\&state=default\&region=MAIN_CONTENT_1\&context=storylines_live_updates}{See
more updates}

At another point, he took a jab at Mr. Biden's mental acuity. ``Let him
define the word carbon, because he won't be able to,'' Mr. Trump said.
That has been a theme of his lately, unsubtly implying that Mr. Biden
has grown senile. Just last week, Mr. Trump, 74,
\href{https://www.nytimes.com/2020/07/10/us/politics/trump-cognitive-test-health.html}{boasted
that he had recently taken a cognitive test} and ``aced it,'' while
insisting that Mr. Biden, 77, ``couldn't pass'' such an exam.

The disjointed monologue, however, may not have been the most convincing
evidence. On Twitter, his critics quickly compared him to a grandfather
who had broken into the sherry cabinet.
\href{https://twitter.com/jonfavs/status/1283157746603356160}{``Trump is
a truly sick individual,''} wrote Jon Favreau, who was President Barack
Obama's chief speechwriter. Rick Wilson, a founder of the Lincoln
Project, a group of anti-Trump Republicans, called it
\href{https://twitter.com/TheRickWilson/status/1283165162988621839}{``rambling
verbal dysentery.''}

The appearance came on the same day that the president's estranged
niece, Mary L. Trump, a clinical psychologist,
\href{https://www.nytimes.com/2020/07/08/books/review-too-much-never-enough-mary-trump.html}{published
a scathing book} questioning his mental health and asserting that
pathologies stemming from his childhood are playing out now on the world
stage. Mr. Trump has not commented about the book, but in the past he
has rejected such contentions by describing himself as ``a very stable
genius.''

The focus of the evening session with reporters took a turn after Mr.
Biden received extensive television coverage earlier in the day for his
\$2 trillion climate plan, according to a senior official who spoke on
the condition of anonymity. The Hong Kong Autonomy Act, the ostensible
reason for his appearance, was treated as an afterthought.

In effect, the news conference turned into a campaign speech to
substitute for the one Mr. Trump was scheduled to give last weekend in
New Hampshire only to cancel amid concerns about flagging attendance,
\href{https://www.nytimes.com/2020/07/10/us/politics/trump-nh-rally-postponed.html}{citing
a possible storm} at the site of the rally. While presidents as a
general rule are not supposed to engage in overt campaigning from the
White House itself, Mr. Trump made little effort to disguise his intent
as he mentioned Mr. Biden's name more than 20 times as he spoke in the
Rose Garden.

Most of the time, the president paid little attention to the text he
seemed to have brought with him, but he eventually read from what he
claimed was Mr. Biden's campaign agenda but was in fact a misleading
compilation assembled by his own political advisers.

``Joe Biden's entire career has been a gift to the Chinese Communist
Party,'' Mr. Trump declared.

Reading from what he was given, he quoted Mr. Biden. ``He said that the
idea that China is our competition is really bizarre,'' the president
said. ``\emph{He's} really bizarre.''

The appearance came on a day when Mr. Trump seemed eager to challenge
convention and, at times, basic facts. During
\href{https://www.cbsnews.com/news/trump-black-americans-killed-police-white-too/}{an
earlier interview with CBS News,} he denied that Black Americans
suffered from police brutality more than white Americans.

Asked why Black Americans were ``still dying at the hands of law
enforcement in this country,'' Mr. Trump said: ``So are white people. So
are white people. What a terrible question to ask. So are white people.
More white people, by the way. More white people.''

Statistics show that while more white Americans are killed by the police
over all, people of color are killed at higher rates when accounting for
population differences. A federal
\href{https://www.ncbi.nlm.nih.gov/pmc/articles/PMC6080222/}{study} that
examined lethal force used by the police from 2009 to 2012 found that a
majority of victims were white, but that Black people were 2.8 times
likelier to be killed than white people.

In the same interview, Mr. Trump dismissed concerns about the
Confederate battle flag. ``With me, it's freedom of speech,'' he said.
``Very simple. Like it, don't like it, it's freedom of speech.''

Asked about those who see it as a painful symbol of slavery, he said:
``I know people that like the Confederate flag, and they're not thinking
about slavery.''

\href{https://www.youtube.com/watch?v=70Jzf0NhBv8}{In a separate
interview}, with the conservative website Townhall.com, that was
published on Tuesday, Mr. Trump falsely claimed that a white couple in
St. Louis who confronted peaceful marchers outside their home with guns
were on the verge of being attacked. ``They were going to be beat up
badly, and the house was going to be totally ransacked and probably
burned down,'' he said.

\href{https://www.nytimes.com/video/us/politics/100000007214585/trump-white-couple-guns-st-louis.html}{Video
of the episode}, which became a flash point in the national debate over
racial inequality, showed that the protesters at no point physically
threatened the couple.

Mr. Trump's Rose Garden appearance had its share of false or misleading
statements, as well. He complained once again that the rising cases of
the coronavirus in the United States were really because of an increase
in testing. ``If we did half the testing, we'd have half the cases,'' he
said. He likewise brushed off the
\href{https://www.nytimes.com/interactive/2020/us/coronavirus-us-cases.html}{death
toll of more than 136,000} by saying that he had saved as many as three
million people by taking the actions he did.

But he was eager to take on Mr. Biden after weeks of trailing him by
double digits in the polls, blaming the former vice president for
everything from crumbling highways to closed factories. ``Joe Biden is
pushing a platform that would demolish the U.S. economy, totally
demolish it,'' Mr. Trump said.

Mr. Biden, he added, has moved so far to the left that he has ``the most
extreme platform of any major-party nominee by far in American
history.'' He cited Mr. Biden's climate plan to reduce carbon emissions
for new homes and offices by 2030. ``That basically means no windows,''
the president said.

While advertised as a news conference, in fact Mr. Trump took only a few
questions, devoting six minutes of the 63-minute event to responding
before abruptly cutting it off. But he promised he was not finished:
``We will be having these conferences again.''

Reporting was contributed by Katie Rogers, Michael D. Shear and Ana
Swanson from Washington, and Jeremy Peters and Isabella Grullón Paz from
New York.

\hypertarget{our-2020-election-guide}{%
\section{Our 2020 Election Guide}\label{our-2020-election-guide}}

Updated Aug. 3, 2020

\begin{itemize}
\item
  \begin{center}\rule{0.5\linewidth}{\linethickness}\end{center}

  \hypertarget{the-latest}{%
  \subsection{The Latest}\label{the-latest}}

  \begin{itemize}
  \tightlist
  \item
    President Trump again assails mail-in voting,
    \href{https://www.nytimes.com/2020/08/03/us/politics/trump-mail-in-voting.html?action=click\&pgtype=Article\&state=default\&region=BELOW_MAIN_CONTENT\&context=storylines_guide}{claiming
    without evidence that the process is plagued by fraud}.
  \end{itemize}
\item
  \begin{center}\rule{0.5\linewidth}{\linethickness}\end{center}

  \hypertarget{bidens-vp-search}{%
  \subsection{Biden's V.P. Search}\label{bidens-vp-search}}

  \begin{itemize}
  \tightlist
  \item
    \href{https://www.nytimes.com/article/biden-vice-president-2020.html?action=click\&pgtype=Article\&state=default\&region=BELOW_MAIN_CONTENT\&context=storylines_guide}{Here
    are 13 women} who have been under consideration to be Joe Biden's
    running mate, and why each might be chosen --- and might not be.
  \end{itemize}
\item
  \begin{center}\rule{0.5\linewidth}{\linethickness}\end{center}

  \hypertarget{keep-up-with-our-coverage}{%
  \subsection{Keep Up With Our
  Coverage}\label{keep-up-with-our-coverage}}

  \begin{itemize}
  \tightlist
  \item
    Get an
    \href{https://www.nytimes.com/newsletters/politics?action=click\&pgtype=Article\&state=default\&region=BELOW_MAIN_CONTENT\&context=storylines_guide}{email}
    recapping the day's news
  \end{itemize}

  \begin{itemize}
  \tightlist
  \item
    Download our mobile app on
    \href{https://apps.apple.com/us/app/nytimes/id284862083?ls=1\&mat_click_id=5c79ae7455014fd1bd66b5610c05b8f2-20191112-16948\&referrer=mat_click_id\%3D5c79ae7455014fd1bd66b5610c05b8f2-20191112-16948\%26link_click_id\%3D722930677036718082}{iOS}
    and
    \href{http://a.localytics.com/android?id=com.nytimes.android\&referrer=utm_source\%3Dother_nyt_mobile_web\%26utm_medium\%3DWeb\%2520page\%26utm_term\%3DGeneral\%2520Mobile\%2520Page\%26utm_campaign\%3DNYT\%2520Mobile\%2520General\%2520Page}{Android}
    and turn on Breaking News and Politics alerts
  \end{itemize}
\end{itemize}

Advertisement

\protect\hyperlink{after-bottom}{Continue reading the main story}

\hypertarget{site-index}{%
\subsection{Site Index}\label{site-index}}

\hypertarget{site-information-navigation}{%
\subsection{Site Information
Navigation}\label{site-information-navigation}}

\begin{itemize}
\tightlist
\item
  \href{https://help.nytimes.com/hc/en-us/articles/115014792127-Copyright-notice}{©~2020~The
  New York Times Company}
\end{itemize}

\begin{itemize}
\tightlist
\item
  \href{https://www.nytco.com/}{NYTCo}
\item
  \href{https://help.nytimes.com/hc/en-us/articles/115015385887-Contact-Us}{Contact
  Us}
\item
  \href{https://www.nytco.com/careers/}{Work with us}
\item
  \href{https://nytmediakit.com/}{Advertise}
\item
  \href{http://www.tbrandstudio.com/}{T Brand Studio}
\item
  \href{https://www.nytimes.com/privacy/cookie-policy\#how-do-i-manage-trackers}{Your
  Ad Choices}
\item
  \href{https://www.nytimes.com/privacy}{Privacy}
\item
  \href{https://help.nytimes.com/hc/en-us/articles/115014893428-Terms-of-service}{Terms
  of Service}
\item
  \href{https://help.nytimes.com/hc/en-us/articles/115014893968-Terms-of-sale}{Terms
  of Sale}
\item
  \href{https://spiderbites.nytimes.com}{Site Map}
\item
  \href{https://help.nytimes.com/hc/en-us}{Help}
\item
  \href{https://www.nytimes.com/subscription?campaignId=37WXW}{Subscriptions}
\end{itemize}
