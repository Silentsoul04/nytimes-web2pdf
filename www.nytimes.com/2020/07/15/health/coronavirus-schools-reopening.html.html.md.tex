Sections

SEARCH

\protect\hyperlink{site-content}{Skip to
content}\protect\hyperlink{site-index}{Skip to site index}

\href{https://www.nytimes.com/section/health}{Health}

\href{https://myaccount.nytimes.com/auth/login?response_type=cookie\&client_id=vi}{}

\href{https://www.nytimes.com/section/todayspaper}{Today's Paper}

\href{/section/health}{Health}\textbar{}Citing Educational Risks,
Scientific Panel Urges That Schools Reopen

\url{https://nyti.ms/2CdzQXL}

\begin{itemize}
\item
\item
\item
\item
\item
\item
\end{itemize}

\href{https://www.nytimes.com/news-event/coronavirus?action=click\&pgtype=Article\&state=default\&region=TOP_BANNER\&context=storylines_menu}{The
Coronavirus Outbreak}

\begin{itemize}
\tightlist
\item
  live\href{https://www.nytimes.com/2020/08/02/world/coronavirus-updates.html?action=click\&pgtype=Article\&state=default\&region=TOP_BANNER\&context=storylines_menu}{Latest
  Updates}
\item
  \href{https://www.nytimes.com/interactive/2020/us/coronavirus-us-cases.html?action=click\&pgtype=Article\&state=default\&region=TOP_BANNER\&context=storylines_menu}{Maps
  and Cases}
\item
  \href{https://www.nytimes.com/interactive/2020/science/coronavirus-vaccine-tracker.html?action=click\&pgtype=Article\&state=default\&region=TOP_BANNER\&context=storylines_menu}{Vaccine
  Tracker}
\item
  \href{https://www.nytimes.com/interactive/2020/07/29/us/schools-reopening-coronavirus.html?action=click\&pgtype=Article\&state=default\&region=TOP_BANNER\&context=storylines_menu}{What
  School May Look Like}
\item
  \href{https://www.nytimes.com/live/2020/07/31/business/stock-market-today-coronavirus?action=click\&pgtype=Article\&state=default\&region=TOP_BANNER\&context=storylines_menu}{Economy}
\end{itemize}

Advertisement

\protect\hyperlink{after-top}{Continue reading the main story}

Supported by

\protect\hyperlink{after-sponsor}{Continue reading the main story}

\hypertarget{citing-educational-risks-scientific-panel-urges-that-schools-reopen}{%
\section{Citing Educational Risks, Scientific Panel Urges That Schools
Reopen}\label{citing-educational-risks-scientific-panel-urges-that-schools-reopen}}

Younger children in particular are ill-served by remote learning,
according to a report issued by the National Academies of Science,
Engineering and Medicine.

\includegraphics{https://static01.nyt.com/images/2020/07/15/science/15VIRUS-SCHOOLS1/merlin_174584724_88d1199d-1b71-4598-b4ad-cf0f652d30a5-articleLarge.jpg?quality=75\&auto=webp\&disable=upscale}

By \href{https://www.nytimes.com/by/apoorva-mandavilli}{Apoorva
Mandavilli}

\begin{itemize}
\item
  Published July 15, 2020Updated July 29, 2020
\item
  \begin{itemize}
  \item
  \item
  \item
  \item
  \item
  \item
  \end{itemize}
\end{itemize}

Wading into the contentious debate over
\href{https://www.nytimes.com/2020/07/29/magazine/schools-reopening-covid.html}{reopening
schools}, an influential committee of scientists and educators on
Wednesday recommended that, wherever possible, younger children and
those with special needs should attend school in person.

Their report --- issued by the prestigious National Academies of
Science, Engineering and Medicine, which advises the nation on issues
related to science --- is less prescriptive for middle and high schools,
\href{https://www.nationalacademies.org/news/2020/07/schools-should-prioritize-reopening-in-fall-2020-especially-for-grades-k-5-while-weighing-risks-and-benefits}{but
offered a framework for school districts to decide whether and how to
open}, with help from public health experts, families and teachers.

The committee emphasized common-sense precautions, such as hand-washing,
physical distancing and minimizing group activities, including lunch and
recess.

But the experts went further than guidelines issued by the Centers for
Disease Control and Prevention and other groups, also calling for
surgical masks to be worn by all teachers and staff members during
school hours, and for cloth face coverings to be worn by all students,
including those in elementary school.

Regular symptom checks should be conducted, the committee said, and not
just temperature checks. In the long term, schools will need upgrades to
ventilation and air-filtration systems, and federal and state
governments must fund these efforts, the report said.

Online learning is ineffective for most elementary-school children and
special-needs children, the panel of scientists and educators concluded.

To the extent possible, ``it should be a priority for districts to
reopen for in-person learning, especially for younger ages,'' said
Caitlin Rivers, an epidemiologist at Johns Hopkins and a member of the
committee.

Mary Kathryn Malone, a mother of three children, has been eager for
schools to reopen in Mount Vernon, Ohio, where she lives. Her 9-year-old
daughter is pining for her friends, and her 3-year-old has only
part-time day care --- and not while Dr. Malone works.

But she was most worried about her 7-year-old son, who needs help for
his attention deficit hyperactivity disorder and dyslexia.

\hypertarget{latest-updates-global-coronavirus-outbreak}{%
\section{\texorpdfstring{\href{https://www.nytimes.com/2020/08/01/world/coronavirus-covid-19.html?action=click\&pgtype=Article\&state=default\&region=MAIN_CONTENT_1\&context=storylines_live_updates}{Latest
Updates: Global Coronavirus
Outbreak}}{Latest Updates: Global Coronavirus Outbreak}}\label{latest-updates-global-coronavirus-outbreak}}

Updated 2020-08-02T17:52:35.962Z

\begin{itemize}
\tightlist
\item
  \href{https://www.nytimes.com/2020/08/01/world/coronavirus-covid-19.html?action=click\&pgtype=Article\&state=default\&region=MAIN_CONTENT_1\&context=storylines_live_updates\#link-34047410}{The
  U.S. reels as July cases more than double the total of any other
  month.}
\item
  \href{https://www.nytimes.com/2020/08/01/world/coronavirus-covid-19.html?action=click\&pgtype=Article\&state=default\&region=MAIN_CONTENT_1\&context=storylines_live_updates\#link-780ec966}{Top
  U.S. officials work to break an impasse over the federal jobless
  benefit.}
\item
  \href{https://www.nytimes.com/2020/08/01/world/coronavirus-covid-19.html?action=click\&pgtype=Article\&state=default\&region=MAIN_CONTENT_1\&context=storylines_live_updates\#link-2bc8948}{Its
  outbreak untamed, Melbourne goes into even greater lockdown.}
\end{itemize}

\href{https://www.nytimes.com/2020/08/01/world/coronavirus-covid-19.html?action=click\&pgtype=Article\&state=default\&region=MAIN_CONTENT_1\&context=storylines_live_updates}{See
more updates}

More live coverage:
\href{https://www.nytimes.com/live/2020/07/31/business/stock-market-today-coronavirus?action=click\&pgtype=Article\&state=default\&region=MAIN_CONTENT_1\&context=storylines_live_updates}{Markets}

``At one point, we were three full weeks behind on schoolwork,'' said
Dr. Malone, who teaches French at Kenyon College. ``I was working
through my own job, and when I looked at this mountain accumulating, it
was so stressful.''

\href{https://www.nytimes.com/2020/06/30/us/coronavirus-schools-reopening-guidelines-aap.html}{The
American Academy of Pediatrics last month also recommended} that schools
reopen, a position widely cited by the Trump administration, which has
been pushing hard for a return to something resembling normal life
despite soaring infection rates in many states.

Most studies suggest the virus poses
\href{https://www.nytimes.com/2020/07/11/health/coronavirus-schools-reopen.html}{minimal
health risks to children under 18}. And the report said that evidence
for how easily children become infected or spread the virus to others,
including teachers and parents, is ``insufficient'' to draw firm
conclusions.

Outside experts said they appreciated the report's distinction between
younger and older children. ``I think that's really smart,'' said Dr.
Ashish Jha, director of the Harvard Global Health Institute.

``The risk is different for a third grader than for a 10th grader, and I
say that as the dad of a third grader and a 10th grader.''

But Dr. Jha and other experts noted that the committee did not address
the level of
\href{https://www.nytimes.com/2020/07/14/us/coronavirus-schools-fall.html}{community
transmission at which opening schools might become unsafe} simply
because too much virus may be circulating. ``They punted the most
critical question,'' he said.

Committee members said the decision not to recommend a cutoff was
deliberate. ``There is no single prevalence or threshold that would be
appropriate for all communities,'' Dr. Rivers said.

Dr. Rivers said schools would need to decide how and when to open, close
and reopen schools by taking into account many factors, including
disease levels in the community --- and should plan for what to do when
students or teachers become infected.

``Even with extensive mitigation measures, it's not possible to reduce
the risk to zero, and that has to be part of the discussions,'' Dr.
Rivers said.

Reopening schools should be a priority because schools fulfill many
roles beyond providing an education, the authors said. ``It's child
care, it's nutrition, it's health services, it's social and emotional
support services,'' said Dr. Enriqueta Bond, the committee's chair.

``These functions are really undervalued, I think, in the conversation
that's been taking place.''

The report's recommendations are largely consistent with those from the
A.A.P., said Dr. Nathaniel Beers, who co-wrote the academy's guidance.
``The only nuanced difference is that they have acknowledged the
disproportionate impact on younger kids of not being in school,'' he
said.

While teenagers may be better able to learn online, they suffer the
social and emotional consequences of being separated from their peers,
Dr. Beers said.

``Adolescence is a period of time in life when you are to be exploring
your own sense of self and developing your identity,'' he said. ``It's
difficult to do that if you are at home with your parents all the
time.''

\includegraphics{https://static01.nyt.com/images/2020/07/15/science/15VIRUS-SCHOOLS2/merlin_174584700_33fb956d-7c8e-4261-bcbf-f4d9ab0dd1e8-articleLarge.jpg?quality=75\&auto=webp\&disable=upscale}

Daniel A. Domenech, executive director of the American Association of
School Administrators, said school superintendents were ``already
prioritizing in-person learning for the youngest learners.''

The new report is not ``groundbreaking,'' he said, ``but it is helpful
in affirming the touchy job ahead and the need for additional resources
to do right by kids, educators and communities during this school
year.''

\textbf{\emph{{[}}\href{http://on.fb.me/1paTQ1h}{\emph{Like the Science
Times page on Facebook.}}} ****** \emph{\textbar{} Sign up for the}
\textbf{\href{http://nyti.ms/1MbHaRU}{\emph{Science Times
newsletter.}}\emph{{]}}}

Some 54 percent of public school districts need to update or replace
facilities in their school buildings, and 41 percent should replace
heating, ventilation and air-conditioning systems in at least half of
their schools, according to an analysis by the Government Accountability
Office.

\href{https://www.nytimes.com/news-event/coronavirus?action=click\&pgtype=Article\&state=default\&region=MAIN_CONTENT_3\&context=storylines_faq}{}

\hypertarget{the-coronavirus-outbreak-}{%
\subsubsection{The Coronavirus Outbreak
›}\label{the-coronavirus-outbreak-}}

\hypertarget{frequently-asked-questions}{%
\paragraph{Frequently Asked
Questions}\label{frequently-asked-questions}}

Updated July 27, 2020

\begin{itemize}
\item ~
  \hypertarget{should-i-refinance-my-mortgage}{%
  \paragraph{Should I refinance my
  mortgage?}\label{should-i-refinance-my-mortgage}}

  \begin{itemize}
  \tightlist
  \item
    \href{https://www.nytimes.com/article/coronavirus-money-unemployment.html?action=click\&pgtype=Article\&state=default\&region=MAIN_CONTENT_3\&context=storylines_faq}{It
    could be a good idea,} because mortgage rates have
    \href{https://www.nytimes.com/2020/07/16/business/mortgage-rates-below-3-percent.html?action=click\&pgtype=Article\&state=default\&region=MAIN_CONTENT_3\&context=storylines_faq}{never
    been lower.} Refinancing requests have pushed mortgage applications
    to some of the highest levels since 2008, so be prepared to get in
    line. But defaults are also up, so if you're thinking about buying a
    home, be aware that some lenders have tightened their standards.
  \end{itemize}
\item ~
  \hypertarget{what-is-school-going-to-look-like-in-september}{%
  \paragraph{What is school going to look like in
  September?}\label{what-is-school-going-to-look-like-in-september}}

  \begin{itemize}
  \tightlist
  \item
    It is unlikely that many schools will return to a normal schedule
    this fall, requiring the grind of
    \href{https://www.nytimes.com/2020/06/05/us/coronavirus-education-lost-learning.html?action=click\&pgtype=Article\&state=default\&region=MAIN_CONTENT_3\&context=storylines_faq}{online
    learning},
    \href{https://www.nytimes.com/2020/05/29/us/coronavirus-child-care-centers.html?action=click\&pgtype=Article\&state=default\&region=MAIN_CONTENT_3\&context=storylines_faq}{makeshift
    child care} and
    \href{https://www.nytimes.com/2020/06/03/business/economy/coronavirus-working-women.html?action=click\&pgtype=Article\&state=default\&region=MAIN_CONTENT_3\&context=storylines_faq}{stunted
    workdays} to continue. California's two largest public school
    districts --- Los Angeles and San Diego --- said on July 13, that
    \href{https://www.nytimes.com/2020/07/13/us/lausd-san-diego-school-reopening.html?action=click\&pgtype=Article\&state=default\&region=MAIN_CONTENT_3\&context=storylines_faq}{instruction
    will be remote-only in the fall}, citing concerns that surging
    coronavirus infections in their areas pose too dire a risk for
    students and teachers. Together, the two districts enroll some
    825,000 students. They are the largest in the country so far to
    abandon plans for even a partial physical return to classrooms when
    they reopen in August. For other districts, the solution won't be an
    all-or-nothing approach.
    \href{https://bioethics.jhu.edu/research-and-outreach/projects/eschool-initiative/school-policy-tracker/}{Many
    systems}, including the nation's largest, New York City, are
    devising
    \href{https://www.nytimes.com/2020/06/26/us/coronavirus-schools-reopen-fall.html?action=click\&pgtype=Article\&state=default\&region=MAIN_CONTENT_3\&context=storylines_faq}{hybrid
    plans} that involve spending some days in classrooms and other days
    online. There's no national policy on this yet, so check with your
    municipal school system regularly to see what is happening in your
    community.
  \end{itemize}
\item ~
  \hypertarget{is-the-coronavirus-airborne}{%
  \paragraph{Is the coronavirus
  airborne?}\label{is-the-coronavirus-airborne}}

  \begin{itemize}
  \tightlist
  \item
    The coronavirus
    \href{https://www.nytimes.com/2020/07/04/health/239-experts-with-one-big-claim-the-coronavirus-is-airborne.html?action=click\&pgtype=Article\&state=default\&region=MAIN_CONTENT_3\&context=storylines_faq}{can
    stay aloft for hours in tiny droplets in stagnant air}, infecting
    people as they inhale, mounting scientific evidence suggests. This
    risk is highest in crowded indoor spaces with poor ventilation, and
    may help explain super-spreading events reported in meatpacking
    plants, churches and restaurants.
    \href{https://www.nytimes.com/2020/07/06/health/coronavirus-airborne-aerosols.html?action=click\&pgtype=Article\&state=default\&region=MAIN_CONTENT_3\&context=storylines_faq}{It's
    unclear how often the virus is spread} via these tiny droplets, or
    aerosols, compared with larger droplets that are expelled when a
    sick person coughs or sneezes, or transmitted through contact with
    contaminated surfaces, said Linsey Marr, an aerosol expert at
    Virginia Tech. Aerosols are released even when a person without
    symptoms exhales, talks or sings, according to Dr. Marr and more
    than 200 other experts, who
    \href{https://academic.oup.com/cid/article/doi/10.1093/cid/ciaa939/5867798}{have
    outlined the evidence in an open letter to the World Health
    Organization}.
  \end{itemize}
\item ~
  \hypertarget{what-are-the-symptoms-of-coronavirus}{%
  \paragraph{What are the symptoms of
  coronavirus?}\label{what-are-the-symptoms-of-coronavirus}}

  \begin{itemize}
  \tightlist
  \item
    Common symptoms
    \href{https://www.nytimes.com/article/symptoms-coronavirus.html?action=click\&pgtype=Article\&state=default\&region=MAIN_CONTENT_3\&context=storylines_faq}{include
    fever, a dry cough, fatigue and difficulty breathing or shortness of
    breath.} Some of these symptoms overlap with those of the flu,
    making detection difficult, but runny noses and stuffy sinuses are
    less common.
    \href{https://www.nytimes.com/2020/04/27/health/coronavirus-symptoms-cdc.html?action=click\&pgtype=Article\&state=default\&region=MAIN_CONTENT_3\&context=storylines_faq}{The
    C.D.C. has also} added chills, muscle pain, sore throat, headache
    and a new loss of the sense of taste or smell as symptoms to look
    out for. Most people fall ill five to seven days after exposure, but
    symptoms may appear in as few as two days or as many as 14 days.
  \end{itemize}
\item ~
  \hypertarget{does-asymptomatic-transmission-of-covid-19-happen}{%
  \paragraph{Does asymptomatic transmission of Covid-19
  happen?}\label{does-asymptomatic-transmission-of-covid-19-happen}}

  \begin{itemize}
  \tightlist
  \item
    So far, the evidence seems to show it does. A widely cited
    \href{https://www.nature.com/articles/s41591-020-0869-5}{paper}
    published in April suggests that people are most infectious about
    two days before the onset of coronavirus symptoms and estimated that
    44 percent of new infections were a result of transmission from
    people who were not yet showing symptoms. Recently, a top expert at
    the World Health Organization stated that transmission of the
    coronavirus by people who did not have symptoms was ``very rare,''
    \href{https://www.nytimes.com/2020/06/09/world/coronavirus-updates.html?action=click\&pgtype=Article\&state=default\&region=MAIN_CONTENT_3\&context=storylines_faq\#link-1f302e21}{but
    she later walked back that statement.}
  \end{itemize}
\end{itemize}

``One of the shocks to me is that over 50 percent of the school
buildings are awful,'' Dr. Bond said.

New evidence suggests that the coronavirus may be airborne, and that
many indoor spaces may need better air filtration to prevent infections.
``Between now and September, you're not going to be able to put in a new
ventilation system,'' she added.

In the meantime, schools may be able to opt for simpler solutions:
Before the weather cools, they might emulate their counterparts in
Europe and move classes outdoors, set up tents or build outdoor
classrooms, said Jennifer Nuzzo, an epidemiologist at the Johns Hopkins
Center for Health Security.

Schools may also need to hire additional staff to replace educators or
other staff members who may not wish to return, the report noted, and to
implement some of the recommendations, such as enforcing social
distancing in the classroom or ensuring that groups of children remain
with a particular teacher.

Some 28 percent of the more than 3.8 million full-time teachers in the
country are older than 50, and about a third of school principals are
over 55, age groups at high risk of severe illness from the coronavirus.

In one survey, 62 percent of educators and administrators reported that
they were somewhat or very concerned about returning to school while the
coronavirus continues to be a threat, according to the report. ``The
school work force issue is really not discussed that much,'' Dr. Bond
said.

Racial and socio-economic inequities are another prominent concern. The
communities where children struggle to learn in dilapidated,
understaffed schools are also those hit hardest by the pandemic, said
Keisha Scarlett, a committee member and chief of equity, partnerships
and engagement at the Seattle Public Schools.

\href{https://www.nytimes.com/2020/04/06/us/coronavirus-schools-attendance-absent.html}{Remote
learning is often difficult for children in low-income families}.
Nationwide, about 30 percent of Indigenous families and about 20 percent
of Black and Latino families do not have access to the internet or have
it only through a smartphone, compared with 7 percent of white families
and 4 percent of Asian families.

Adults in these communities are also more likely to be essential workers
who cannot stay home with their children, Dr. Scarlett said. Rates of
hospitalization for Covid-19 are four to five times higher in
\href{https://www.nytimes.com/interactive/2020/07/05/us/coronavirus-latinos-african-americans-cdc-data.html}{Black,
Latino and Indigenous populations than among whites}.

``Covid-19 exacerbates those disparities,'' Dr. Scarlett said.

The report also noted significant differences between rural and urban
schools. Some 26 percent of people in rural districts and 32 percent of
those living on tribal lands do not have reliable internet access.

Samuel Berry-Foster Sr., a sixth-grade science teacher, lives just
outside Asheville, N.C., in a pocket of the Appalachian Mountains, with
his wife and two school-age children.

For his family and for those of many of his students, Mr. Berry-Foster
said, even a simple phone call can be plagued with delays and hangups.
For more than one family member to be online at the same time is
``impossible.''

``What we end up doing is driving about eight miles to a little bitty
library for broadband,'' he said. ``We sit in the parking lot and do our
meetings and such.''

The C.D.C. has provided limited guidance on reopening schools and
largely puts the onus on district leaders to make judgments they may be
unequipped to make.

The new report offers more detailed guidance for how to reopen,
including a list of the kinds of experts to consult --- such as
epidemiologists who can interpret disease transmission rates. Local task
forces should take into account the number of coronavirus infections,
hospitalizations and deaths, and the percentage of diagnostic tests that
are positive.

President Trump has said that even the C.D.C.'s less detailed
recommendations were ``very tough and expensive.'' But the new report's
recommended retrofits are likely to be out of reach for most school
districts, costing roughly \$1.8 million for a school district with
eight school buildings and about 3,200 students.

``In my view, this has to be a top priority,'' Dr. Nuzzo said. ``The
economy depends on this, the future of our country depends on this.''

Advertisement

\protect\hyperlink{after-bottom}{Continue reading the main story}

\hypertarget{site-index}{%
\subsection{Site Index}\label{site-index}}

\hypertarget{site-information-navigation}{%
\subsection{Site Information
Navigation}\label{site-information-navigation}}

\begin{itemize}
\tightlist
\item
  \href{https://help.nytimes.com/hc/en-us/articles/115014792127-Copyright-notice}{©~2020~The
  New York Times Company}
\end{itemize}

\begin{itemize}
\tightlist
\item
  \href{https://www.nytco.com/}{NYTCo}
\item
  \href{https://help.nytimes.com/hc/en-us/articles/115015385887-Contact-Us}{Contact
  Us}
\item
  \href{https://www.nytco.com/careers/}{Work with us}
\item
  \href{https://nytmediakit.com/}{Advertise}
\item
  \href{http://www.tbrandstudio.com/}{T Brand Studio}
\item
  \href{https://www.nytimes.com/privacy/cookie-policy\#how-do-i-manage-trackers}{Your
  Ad Choices}
\item
  \href{https://www.nytimes.com/privacy}{Privacy}
\item
  \href{https://help.nytimes.com/hc/en-us/articles/115014893428-Terms-of-service}{Terms
  of Service}
\item
  \href{https://help.nytimes.com/hc/en-us/articles/115014893968-Terms-of-sale}{Terms
  of Sale}
\item
  \href{https://spiderbites.nytimes.com}{Site Map}
\item
  \href{https://help.nytimes.com/hc/en-us}{Help}
\item
  \href{https://www.nytimes.com/subscription?campaignId=37WXW}{Subscriptions}
\end{itemize}
