Sections

SEARCH

\protect\hyperlink{site-content}{Skip to
content}\protect\hyperlink{site-index}{Skip to site index}

\href{https://myaccount.nytimes.com/auth/login?response_type=cookie\&client_id=vi}{}

\href{https://www.nytimes.com/section/todayspaper}{Today's Paper}

Victory Gardens Were More About Solidarity Than Survival

\url{https://nyti.ms/3fwxWQt}

\begin{itemize}
\item
\item
\item
\item
\item
\item
\end{itemize}

Advertisement

\protect\hyperlink{after-top}{Continue reading the main story}

Supported by

\protect\hyperlink{after-sponsor}{Continue reading the main story}

Beyond the World War II We Know

\hypertarget{victory-gardens-were-more-about-solidarity-than-survival}{%
\section{Victory Gardens Were More About Solidarity Than
Survival}\label{victory-gardens-were-more-about-solidarity-than-survival}}

During World War II, millions of Americans grew their own vegetables,
but the movement was driven much more by government and corporate
messaging than by the threat of starvation.

\includegraphics{https://static01.nyt.com/images/2020/07/15/multimedia/15ww2-victory-gardens-01/15ww2-victory-gardens-01-articleLarge.png?quality=75\&auto=webp\&disable=upscale}

By \href{https://www.nytimes.com/by/jennifer-steinhauer}{Jennifer
Steinhauer}

\begin{itemize}
\item
  July 15, 2020
\item
  \begin{itemize}
  \item
  \item
  \item
  \item
  \item
  \item
  \end{itemize}
\end{itemize}

\emph{\emph{\emph{In the latest article from
``}\href{https://www.nytimes.com/spotlight/beyond-wwii}{\emph{Beyond the
World War II We Know}}},'' a series from The Times that documents
lesser-known stories from World War II, we recount the history of
victory gardens and some of the misconceptions of how they emerged after
the United States joined the conflict.}**

Of all the celebrated nostalgic markers of World War II, few are as
memorable as America's victory gardens --- those open lots, rooftops and
backyards made resplendent with beets, broccoli, kohlrabi, parsnips and
spinach to substitute for the commercial crops diverted to troops
overseas during the war.

The gardens were strongly encouraged by the American government during
World War I as part of the at-home efforts, yet they became immensely
more popular with the introduction of food rationing during the Second
World War as processed and canned foods were shipped abroad.

It's often said that this later era of victory gardens emerged out of
grass-roots collective action to prevent the risk of running out of
food, which was already hurting countries all over Europe. Despite the
millions of pounds of food being diverted from American kitchen tables
for the war effort, there was little threat of citizens going hungry.
Rather, the victory-garden movement was driven much more by government
and corporate messaging meant to invoke American solidarity.

``Americans like to portray that they worked hard and would have starved
had they not gardened,'' said Allan M. Winkler, a distinguished
professor emeritus of history at Miami University of Ohio. ``Victory
gardens were a symbol of abundance and doing it yourself, but that was
more symbolism than reality.''

Nearly two-thirds of American households participated in some form of
national harvest; even Eleanor Roosevelt
\href{https://library.si.edu/exhibition/cultivating-americas-gardens/gardening-for-the-common-good}{planted}
a victory garden on the White House lawn. By 1943, close to 20 million
families planted seven million acres of gardens across the United
States, producing more than 15 billion pounds, or roughly 40 percent, of
the fresh produce Americans consumed that year.

Public service advertisements urging Americans to grow vegetables and to
can them peppered libraries, community centers and newsreels in movie
theaters. They offered motivational messages such as ``Your country
needs soybeans,'' and
\href{https://www.history.nd.gov/exhibits/gardening/militaryevents16.html}{``Can
all you can. It's a real war job!''} One poster featured a fresh-faced
girl in overalls holding a hoe and a basket of bounty, with the tagline
``Grow vitamins at your kitchen door.''

Image

The victory-garden movement was driven by government and corporate
messaging meant to invoke American solidarity.Credit...National Archives

Image

Corporations received tax breaks for promoting the war efforts to
consumers.Credit...Library of Congress

Still, food-production levels throughout American involvement in the war
were pretty stable. The peak year of rationing in the United States was
in 1943, and food shortages never neared those in Europe and Asia. In
1942, for example, Americans consumed 138 pounds of meat per capita, a
mere three pounds less than the prior year, according to Amy Bentley's
1998 book ``Eating for Victory: Food Rationing and the Politics of
Domesticity.'' Americans were pressed to leave more for troops, with
government campaigns stressing that fighting men would get their
\href{https://books.google.com/books?id=InqSoenmQ0IC\&pg=PA85\&source=gbs_toc_r\&cad=4\#v=onepage\&q=soy\&f=false}{strength}from
meat.

``Look and Life magazines were where people got information,'' said
Bentley, a professor of food studies at New York University. ``Citizens
had a clear understanding of the threats of war and what their efforts
were supposed to be, and corporations wanted to be associated with
that.''

The National Victory Garden Program, which was created by the War Food
Administration in 1941, got early and strong support from corporations.
It was a very top-down movement, with a board composed of chief
executives from agriculture companies who saw the gardens as an exercise
both in expressing their patriotism and product placement, according to
Anastasia Day, a scholar in the University of Delaware's history
department. Many of the companies gave packs of seeds --- often labeled
``Victory Seeds'' --- with purchase of their products. In return,
corporations received tax breaks for promoting the war efforts to
consumers.

``I think one modern-day analogue is how big oil companies promote
alternative energies and green washing, ostensibly working against their
own interest,'' said Day. ``Just as Green Giant peas were big supporters
of victory gardens.''

The messaging from the top also attempted to shift American eating
habits through promotional campaigns and even changing nutritional
guidelines that often celebrated specific sectors of agriculture. For
example, as the government tried to further ration meat intended for
servicemen, Americans were pushed to enjoy soybeans, peanut butter, eggs
and organ meats. Newspapers printed how-to columns on building chicken
houses and caring for hens.

Image

As the government tried to ration meat intended for servicemen,
Americans were pushed to enjoy soybeans.Credit...National Archives

Image

By 1943, close to 20 million families had planted seven million acres of
gardens across the United States.Credit...National Archives

While it feels easy to draw narrative lines between victory gardens and
the organic, local food movement of today, in truth the fresh-vegetable
trends of World War II were almost immediately subsumed by postwar
Jell-O molds, cake mixes and frozen dinners --- all markers of modern
living at the time. Many women did most of the cooking and enjoyed being
free of domestic gardening and canning, and celebrated all forms of
culinary convenience during the baby boom era. That was especially true
of white families who populated the newly developing suburbs after the
war.

``The rise of suburbs was the culmination of this urge that owning
property and having your own space of land is something that is
inherently American,'' Day explained. ``Victory gardens were a
transitional phase on the way to the promise that was largely fulfilled
for white, upwardly mobile working-class Americans as they moved to the
suburbs,'' where victory gardens all but disappeared.

The gleaming new suburban developments tended not to include garden
plots. What is more, entrusting corporations with food preparation was
the ultimate postwar cultural shift (so cleverly captured in the show
``Mad Men'').

As a result of the new processed food trends, American tastes evolved
too, trending away from fresh flavors and seasonal produce. A generation
later, those preferences would return to become the centerpieces of
upscale restaurants in the contemporary United States. While many Black
families in the South and Latinos in the Southwest kept up gardening
traditions, predominantly white suburban homes were big on shelf-stable
products to fill newly expansive pantries, and technology that had gone
toward the war effort was transplanted to things used in the home.

``The golden age of food processing created a plethora of products, and
consumers were enamored by them,'' Bentley said. ``Fresh-tasting produce
becomes less important than convenience, shelf stability, price and
storage capacity. People also learned they like the heavy sugars and
salt used in canned vegetables and fruit.''

This spring, there was a spurt of
\href{https://www.nytimes.com/2020/04/15/magazine/gardening-quarantine-coronavirus.html}{new
attention} to the wartime victory gardens, and a search for lessons and
inspiration for Americans locked down in response to the coronavirus
pandemic. Some citizens were turning to their own gardens for dinner.
There was a spate of replanting
\href{https://www.eater.com/2020/4/16/21224012/quarantine-trend-growing-scallions-vegetables-jars-water-kitchen-scraps}{onions}
from scraps and a
\href{https://www.reuters.com/article/us-health-coronavirus-gardens/home-gardening-blooms-around-the-world-during-coronavirus-lockdowns-idUSKBN2220D3}{run
on seeds.} But that attention has been eclipsed by ominous news from the
home front, where coronavirus infections and deaths have surged, and
backyard gardening in 2020 has lacked a unified, depoliticized social
movement to fuel it.

``As I think about the victory gardens of World War II, I think their
most important value was in getting the public to feel involved in the
war,'' Winkler said. ``In the Covid-19 pandemic, there is some of that.
Wearing masks is protective, and necessary, to be sure, but it also
gives us a sense of doing our part.''

Advertisement

\protect\hyperlink{after-bottom}{Continue reading the main story}

\hypertarget{site-index}{%
\subsection{Site Index}\label{site-index}}

\hypertarget{site-information-navigation}{%
\subsection{Site Information
Navigation}\label{site-information-navigation}}

\begin{itemize}
\tightlist
\item
  \href{https://help.nytimes.com/hc/en-us/articles/115014792127-Copyright-notice}{©~2020~The
  New York Times Company}
\end{itemize}

\begin{itemize}
\tightlist
\item
  \href{https://www.nytco.com/}{NYTCo}
\item
  \href{https://help.nytimes.com/hc/en-us/articles/115015385887-Contact-Us}{Contact
  Us}
\item
  \href{https://www.nytco.com/careers/}{Work with us}
\item
  \href{https://nytmediakit.com/}{Advertise}
\item
  \href{http://www.tbrandstudio.com/}{T Brand Studio}
\item
  \href{https://www.nytimes.com/privacy/cookie-policy\#how-do-i-manage-trackers}{Your
  Ad Choices}
\item
  \href{https://www.nytimes.com/privacy}{Privacy}
\item
  \href{https://help.nytimes.com/hc/en-us/articles/115014893428-Terms-of-service}{Terms
  of Service}
\item
  \href{https://help.nytimes.com/hc/en-us/articles/115014893968-Terms-of-sale}{Terms
  of Sale}
\item
  \href{https://spiderbites.nytimes.com}{Site Map}
\item
  \href{https://help.nytimes.com/hc/en-us}{Help}
\item
  \href{https://www.nytimes.com/subscription?campaignId=37WXW}{Subscriptions}
\end{itemize}
