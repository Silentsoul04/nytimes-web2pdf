Sections

SEARCH

\protect\hyperlink{site-content}{Skip to
content}\protect\hyperlink{site-index}{Skip to site index}

\href{https://www.nytimes.com/section/technology}{Technology}

\href{https://myaccount.nytimes.com/auth/login?response_type=cookie\&client_id=vi}{}

\href{https://www.nytimes.com/section/todayspaper}{Today's Paper}

\href{/section/technology}{Technology}\textbar{}TikTok Enlists Army of
Lobbyists as Suspicions Over China Ties Grow

\url{https://nyti.ms/2OozBM2}

\begin{itemize}
\item
\item
\item
\item
\item
\end{itemize}

Advertisement

\protect\hyperlink{after-top}{Continue reading the main story}

Supported by

\protect\hyperlink{after-sponsor}{Continue reading the main story}

\hypertarget{tiktok-enlists-army-of-lobbyists-as-suspicions-over-china-ties-grow}{%
\section{TikTok Enlists Army of Lobbyists as Suspicions Over China Ties
Grow}\label{tiktok-enlists-army-of-lobbyists-as-suspicions-over-china-ties-grow}}

The viral social media app has beefed up its lobbying operation to
counter several potential actions in Washington that could threaten the
company's future.

\includegraphics{https://static01.nyt.com/images/2020/07/15/business/15TikTok-01/merlin_174342414_c131b731-fcbb-4911-942e-535857f70d3d-articleLarge.jpg?quality=75\&auto=webp\&disable=upscale}

By \href{https://www.nytimes.com/by/cecilia-kang}{Cecilia Kang},
\href{https://www.nytimes.com/by/lara-jakes}{Lara Jakes},
\href{https://www.nytimes.com/by/ana-swanson}{Ana Swanson} and
\href{https://www.nytimes.com/by/david-mccabe}{David McCabe}

\begin{itemize}
\item
  Published July 15, 2020Updated Aug. 3, 2020
\item
  \begin{itemize}
  \item
  \item
  \item
  \item
  \item
  \end{itemize}
\end{itemize}

\href{https://cn.nytimes.com/technology/20200716/tiktok-washington-lobbyist/}{阅读简体中文版}\href{https://cn.nytimes.com/technology/20200716/tiktok-washington-lobbyist/zh-hant/}{閱讀繁體中文版}

WASHINGTON ---
\href{https://www.nytimes.com/2020/08/01/technology/tiktok-trump-microsoft-bytedance-china-ban.html}{TikTok},
the wildly popular social media app known for its viral dance and lip
sync clips, has been embraced by millions of students, celebrities and
young adults across the United States. But the company's ties to China
could cripple its existence.

\href{https://www.nytimes.com/2020/08/03/technology/tiktok-trump-sale-microsoft.html}{TikTok,
which is owned by the China-based ByteDance}, has become the latest
target in the Trump administration's
\href{https://www.nytimes.com/2020/01/20/business/economy/trump-us-china-deal-micron-trade-war.html}{long
simmering security and economic battle with Beijing}. It is now
desperately trying to convince lawmakers and administration officials
that its allegiance lies with the United States, not China.

The social media company, which one year ago had virtually no lobbying
presence in the nation's capital, has hired a small army of more than 35
lobbyists to work on its behalf, including one with deep ties to
President Trump.

Behind that buildup is a growing threat to one of TikTok's most
important markets. Secretary of State Mike Pompeo has threatened to ban
Chinese apps like TikTok, which are downloaded to mobile phones, over
concerns they could be used for surveillance by the Chinese government.
Peter Navarro, the White House trade adviser, called TikTok's new chief
executive an ``American puppet'' during an interview on Fox News
Channel's ``Sunday Morning Futures'' ** and said the administration
would take ``strong action'' against the company and other Chinese
social media apps.

A powerful U.S. panel has opened a national security review into
Bytedance's 2018 purchase of Musical.ly, an app that was merged to form
TikTok. The Committee on Foreign Investment in the United States is
examining whether the merged companies could give the Chinese government
access to vast amounts of American data, including videos useful for
training facial recognition software. And the Trump administration is
weighing action against Chinese social media services like TikTok under
the International Emergency Economic Powers Act, which allows the
president to regulate international commerce in response to unusual and
extraordinary threats, people familiar with the deliberations say.

Speaking to reporters Wednesday evening, the White House chief of staff,
Mark Meadows, said a number of administration officials were ``looking
at the national security risk as it relates to TikTok, WeChat and other
apps.''

``I don't think there's any self-imposed deadline for action, but I
think we are looking at weeks, not months,'' he said.

In the past three months, lobbyists working on behalf of TikTok have
held at least 50 meetings with congressional staff and lawmakers,
including those on top committees like commerce, judiciary and
intelligence. Those meetings have included a slick presentation that
includes an organizational chart showing that TikTok does not operate in
China and that most of its top leaders reside in the United States and
are American citizens. For instance, TikTok's new chief executive, Kevin
Mayer, a former executive of Disney, lives in Los Angeles, they say.

\includegraphics{https://static01.nyt.com/images/2020/07/15/business/15dc-tiktok-02/15dc-tiktok-02-articleLarge.jpg?quality=75\&auto=webp\&disable=upscale}

ByteDance denies it shares data with the Chinese government and is
distancing itself from its roots in the communist nation. The company
stressed TikTok was not available in China --- it offers a similar app
called Douyin there instead --- and said user data was stored in
Virginia, with a backup in Singapore.

``There's a lot of misinformation about TikTok right now,'' said Michael
Beckerman, vice president and head of U.S. Public Policy. ``TikTok is
led by an American C.E.O., with hundreds of employees and key leaders
across safety, security, product, and public policy in the U.S.''

But some members of Congress still have suspicions. An aide to Senator
Marco Rubio, a Florida Republican who requested the Cfius review of
TikTok, said ByteDance had provided conflicting information in a meeting
with representatives of Mr. Rubio's office about where its data was
stored, as well as insufficient information about how it controls and
censors its content.

``It is no coincidence that every day more companies and organizations
are asking employees to delete TikTok,'' Mr. Rubio said in a statement,
referring to moves
\href{https://www.cnn.com/2020/07/13/tech/tiktok-wells-fargo/index.html}{by
Wells Fargo}
\href{https://www.nytimes.com/2020/07/10/technology/tiktok-amazon-security-risk.html}{and
others} to bar the app from company devices. ``TikTok has yet to provide
a real explanation to Americans about how they protect their data and
how much of it could be made available to the Chinese Communist Party.''

The United States provides a crucial audience for TikTok. American
influencers have global followings, and the app has become a center of
conversation
\href{https://www.nytimes.com/2020/02/27/style/tiktok-politics-bernie-trump.html}{about
politics}, the pandemic and racial inequality. TikTok users
\href{https://www.nytimes.com/2020/06/21/style/tiktok-trump-rally-tulsa.html}{claimed
credit} for reserving thousands of seats for Mr. Trump's campaign rally
in Tulsa, Okla., last month --- and then not showing up.

But it remains a high bar for ByteDance to convince the U.S. government
that it is not susceptible to the directives of the Chinese government.
Mr. Trump and his top advisers have increasingly focused on Chinese
technology companies, including Huawei and ZTE, saying those firms
threaten national security by providing a conduit for the Chinese
government to infiltrate American technology. The United States has
already
\href{https://www.nytimes.com/2019/10/23/business/trump-technology-china-trade.html}{barred
dozens}of high-tech Chinese companies --- including those specializing
in supercomputers, artificial intelligence and facial recognition ---
from gaining access to American technology products out of national
security concerns.

Image

The White House trade adviser Peter Navarro said the Trump
administration would take ``strong action'' against TikTok and other
Chinese social media apps.Credit...Samuel Corum for The New York Times

``What the American people have to understand is all the data that goes
into those mobile apps that kids have so much fun with and seem so
convenient, it goes right to servers in China, right to the Chinese
military, the Chinese Communist Party, and the agencies that want to
steal our intellectual property,'' Mr. Navarro said over the weekend.

The issue of whether TikTok should be curbed in the United States has
taken on new urgency, in part because of India's decision in late June
to
\href{https://www.state.gov/secretary-michael-r-pompeo-with-laura-ingraham-of-fox-news-3/}{ban
it and nearly 60 other Chinese apps}, a Trump administration official
said. TikTok has been downloaded two billion times, with its biggest
markets in India, the United States and Brazil, according to
\href{https://sensortower.com/blog/tiktok-downloads-2-billion}{SensorTower}.

Last December, the Pentagon ordered military personnel to delete the
TikTok app from their phones and some administration officials have
argued that the United States should retroactively block ByteDance's
acquisition of Musical.ly, which could force the company to divest its
American assets, or at least make changes to the way it moves and stores
data worldwide.

The State Department is considering expanding its
so-called\href{https://www.state.gov/the-tide-is-turning-toward-trusted-5g-vendors/}{clean
networks} program to include apps as it tries to steer foreign
governments away from unsecure Chinese telecommunications firms in the
name of protecting Americans' private information, according to
officials familiar with the internal discussions.

TikTok would be considered among those apps, although officials said the
State Department has not yet designated companies to be included in the
expansion.

``Whether it's TikTok or any of the other Chinese communications
platforms, apps, infrastructure, this administration has taken seriously
the requirement to protect the American people from having their
information end up in the hands of the Chinese Communist Party,'' Mr.
Pompeo said
\href{https://www.state.gov/secretary-michael-r-pompeo-with-bob-cusack-editor-in-chief-of-the-hill/}{Wednesday
in an interview with The Hill newspaper} in Washington.

He said he had heard from parents eager to see TikTok banned: ``That's
for the parents to decide their kids' usage on their cellphones. It's
our task to make sure that their children's information doesn't end up
in the hands of the Chinese Communist Party.''

Image

``This administration has taken seriously the requirement to protect the
American people from having their information end up in the hands of the
Chinese Communist Party,'' said the secretary of state, Mike
Pompeo.Credit...Andrew Harnik/Agence France-Presse --- Getty Images

Officials have also been considering potential national security risks
from other Chinese internet and social media services, including
Tencent's WeChat, which had more than a billion active monthly users
worldwide in the first quarter of 2020.

``These companies cannot claim that they don't follow the orders of the
party, that's just not credible,'' said Derek Scissors, a resident
scholar at the American Enterprise Institute who tracks Chinese
investment worldwide. ``Chinese firms don't have a choice.''

TikTok and the venture funds it counts as its major investors have tried
to reassure the Trump administration --- including Treasury Secretary
Steven Mnuchin, who is in charge of the national security review panel
--- that it has walled off its China operations from other global
activities, people familiar with the conversations said. The firm
recently pulled its operations out of Hong Kong after the city imposed
new national security laws that would bring Chinese-style censorship to
residents. Officials have also raised potential changes to its corporate
structure that could include moving its global headquarters during
discussions with U.S. officials, these people said.

The company has added well-connected lobbyists, including Mr. Beckerman,
the former president of the Internet Association and a longtime
Republican congressional aide, and David J. Urban, who ran Mr. Trump's
campaign in Pennsylvania and has been described by the president as
``one of my good friends.'' He is also a West Point classmate of Mr.
Pompeo and Mark T. Esper, the defense secretary.

Mr. Beckerman has hired 15 lobbyists and communications staff for
ByteDance, including aides to Paul Ryan, the former Wisconsin lawmaker
and speaker of the House, and Representative Jim Clyburn of South
Carolina, the Democratic whip.

ByteDance has also tapped its prominent investors for help. General
Atlantic, whose chief executive, William E. Ford, sits on ByteDance's
board, has been advising TikTok on lobbying strategy, and SoftBank,
which invested in ByteDance in 2018, has suggested new Washington hires
in the past, said two people familiar with the matter.

For the first three months of 2020, ByteDance spent \$300,000 on
lobbying, double the amount it spent in the previous quarter and the
equivalent of its two quarters of lobbying in 2019. TikTok's lobbying
force is not as large as those of other tech giants like Amazon,
Facebook and Google, but the company has deployed a defensive army with
astonishing speed.

Efforts to sway lawmakers have not always gone smoothly. The company
scheduled meetings last December between the then-head of TikTok,
\href{https://www.nytimes.com/2019/11/18/technology/tiktok-alex-zhu-interview.html}{Alex
Zhu}, and lawmakers critical of the company. It then canceled the
meetings, which irritated lawmakers, who promptly shared news of the
canceled meetings on Twitter. (TikTok
\href{https://www.washingtonpost.com/technology/2019/12/09/tiktok-leader-postpones-trip-washington-meet-with-members-congress/}{told
reporters} at the time that the meetings were postponed until after the
holidays.)

In meetings with lawmakers, lobbyists insist that the app is mainly for
entertainment and is not the type of content that is normally targeted
for government surveillance, according to two people with knowledge of
TikTok's lobbying activities. They point out that the most popular clips
are by young influencers like 16-year-old dancer Charli D'Amelio of
Connecticut, who has 70 million followers.

The company has also highlighted its American investors, like the
Chinese arm of the venture capital firm Sequoia and the private equity
firms KKR and General Atlantic, said one person familiar with the
matter.

Mr. Beckerman's staff sends a regular email newsletter to Capitol Hill
with uplifting stories about TikTok. They have highlighted fun videos
about the Netflix series ``Tiger King'' and clips related to Covid-19
prevention.

But in recent days, they have taken a more defensive tone. In the
newsletter sent last Friday, Mr. Beckerman highlighted TikTok's decision
to leave Hong Kong.

``We put action behind words,'' he said.

Raymond Zhong contributed reporting.

Advertisement

\protect\hyperlink{after-bottom}{Continue reading the main story}

\hypertarget{site-index}{%
\subsection{Site Index}\label{site-index}}

\hypertarget{site-information-navigation}{%
\subsection{Site Information
Navigation}\label{site-information-navigation}}

\begin{itemize}
\tightlist
\item
  \href{https://help.nytimes.com/hc/en-us/articles/115014792127-Copyright-notice}{©~2020~The
  New York Times Company}
\end{itemize}

\begin{itemize}
\tightlist
\item
  \href{https://www.nytco.com/}{NYTCo}
\item
  \href{https://help.nytimes.com/hc/en-us/articles/115015385887-Contact-Us}{Contact
  Us}
\item
  \href{https://www.nytco.com/careers/}{Work with us}
\item
  \href{https://nytmediakit.com/}{Advertise}
\item
  \href{http://www.tbrandstudio.com/}{T Brand Studio}
\item
  \href{https://www.nytimes.com/privacy/cookie-policy\#how-do-i-manage-trackers}{Your
  Ad Choices}
\item
  \href{https://www.nytimes.com/privacy}{Privacy}
\item
  \href{https://help.nytimes.com/hc/en-us/articles/115014893428-Terms-of-service}{Terms
  of Service}
\item
  \href{https://help.nytimes.com/hc/en-us/articles/115014893968-Terms-of-sale}{Terms
  of Sale}
\item
  \href{https://spiderbites.nytimes.com}{Site Map}
\item
  \href{https://help.nytimes.com/hc/en-us}{Help}
\item
  \href{https://www.nytimes.com/subscription?campaignId=37WXW}{Subscriptions}
\end{itemize}
