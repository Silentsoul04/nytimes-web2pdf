Sections

SEARCH

\protect\hyperlink{site-content}{Skip to
content}\protect\hyperlink{site-index}{Skip to site index}

\href{https://www.nytimes.com/section/world/asia}{Asia Pacific}

\href{https://myaccount.nytimes.com/auth/login?response_type=cookie\&client_id=vi}{}

\href{https://www.nytimes.com/section/todayspaper}{Today's Paper}

\href{/section/world/asia}{Asia Pacific}\textbar{}Monsoon Rains Pummel
South Asia, Displacing Millions

\url{https://nyti.ms/3fxSibR}

\begin{itemize}
\item
\item
\item
\item
\item
\item
\end{itemize}

Advertisement

\protect\hyperlink{after-top}{Continue reading the main story}

Supported by

\protect\hyperlink{after-sponsor}{Continue reading the main story}

\hypertarget{monsoon-rains-pummel-south-asia-displacing-millions}{%
\section{Monsoon Rains Pummel South Asia, Displacing
Millions}\label{monsoon-rains-pummel-south-asia-displacing-millions}}

Flooding in Bangladesh, Bhutan, India, Myanmar and Nepal has killed
scores of people, destroyed homes and structures, drowned entire
villages, and forced many to crouch on rooftops hoping for rescue.

\includegraphics{https://static01.nyt.com/images/2020/07/15/world/15bangladesh1/15bangladesh1-videoSixteenByNine3000.jpg}

\href{https://www.nytimes.com/by/sameer-yasir}{\includegraphics{https://static01.nyt.com/images/2019/11/22/reader-center/author-sameer-yasir/author-sameer-yasir-thumbLarge.png}}

By \href{https://www.nytimes.com/by/sameer-yasir}{Sameer Yasir}

\begin{itemize}
\item
  Published July 15, 2020Updated July 30, 2020
\item
  \begin{itemize}
  \item
  \item
  \item
  \item
  \item
  \item
  \end{itemize}
\end{itemize}

When heavy floods started inundating her riverside home last week in the
village of Madarganj, northern
\href{https://www.nytimes.com/2020/07/30/climate/bangladesh-floods.html}{Bangladesh},
Habiba Begum chose to stay put with her family instead of moving to
shelter.

Ms. Begum's family, desperate to save what few possessions they had,
chained their only suitcase to their house, a makeshift structure of
bamboo and banana leaves constructed after the last devastating flood in
the area, just two years ago.

As the waters rose, the house was marooned in muddied waters, and the
family had to cook meals on a raised area of dry ground nearby.

Then tragedy struck. Ms. Begum left her 1-year-old daughter, Lamia
Khatun, on a patch of higher ground while she washed clothes in
floodwaters on Tuesday. But the waters kept rising.

``When I came back, she was gone,'' Ms. Begum, 32, said. ``We found her
body hours later.''

Across southern Asia, more than four million people have been hit hard
by monsoon floods that have destroyed homes and structures, drowned
entire villages and forced people to crouch on rooftops hoping for
rescue.

The monsoon season --- usually June to September --- brings
\href{https://www.nytimes.com/2019/08/12/world/asia/india-pakistan-monsoons-karachi.html}{a
torrent of heavy rain}, a deluge that is crucial to South Asia's
agrarian economy.

But in recent years, the monsoon season has increasingly brought
\href{https://www.nytimes.com/2019/07/14/world/asia/monsoon-floods-nepal-india.html}{cyclones
and devastating floods}, causing the internal displacement of millions
of people in low-lying areas, particularly in Bangladesh.

Scientists say global warming has played a role by affecting rainfall
patterns across the subcontinent. Instead of more constant but less
intense rains, warming has
\href{https://slack-redir.net/link?url=https\%3A\%2F\%2Fwww.nature.com\%2Farticles\%2Fs41467-017-00744-9}{increased
the frequency of extreme rains,} which are more likely to cause
flooding.

Last year, at least
\href{https://www.indiatoday.in/india/story/600-people-killed-over-25-million-affected-flooding-india-bangladesh-nepal-myanmar-un-1574258-2019-07-27}{600
people were killed and more than 25 million} affected by flooding
because of the torrential monsoon rains in Bangladesh, India, Myanmar
and Nepal, according to the United Nations. And in 2017, more than
\href{https://www.nytimes.com/2017/08/29/world/asia/floods-south-asia-india-bangladesh-nepal-houston.html}{1,000
people died} in floods across South Asia.

Rainfall has been heaviest this year in northeast India, Bangladesh,
Bhutan, Myanmar and Nepal, according to the Southeast Asia Flash Flood
Forecast System, which is affiliated with the United Nations.

The Bangladeshi authorities say that the flooding started in late June,
inundations are expected to continue this month, and more areas will be
affected.

Enamur Rahman, the Bangladeshi minister for disaster management, said
the inundations were the worst in decades and that hundreds of thousands
of families had been marooned, forcing the authorities to open more than
a thousand emergency shelters.

``We are fighting the catastrophe with every possible resource
available,'' Mr. Rahman said. ``It seems rains and floods will be
prolonged this year.''

Researchers have warned that within a few decades, Bangladesh, with a
population of more than 160 million people, may lose more than 10
percent of its land to sea-level rise,
\href{https://www.nytimes.com/interactive/2019/04/11/magazine/climate-change-bangladesh-scavenging.html}{caused
by a warming climate}, displacing as many as 18 million.

\includegraphics{https://static01.nyt.com/images/2020/07/15/world/15bangladesh2/merlin_174551013_f3468397-df2a-4f0b-8101-a47953c82d3b-articleLarge.jpg?quality=75\&auto=webp\&disable=upscale}

India has also suffered immensely. Floods have swept across the states
of Assam, Bihar, Odisha, West Bengal and other areas in the eastern part
of the country. The authorities in the country have said that
\href{https://www.ndtv.com/india-news/assam-flood-6-more-dead-in-assam-as-flood-worsens-22-lakh-people-affected-2262196}{at
least 85 people have died}, with more than three million affected by the
deluge.

In the northeastern state of Assam, Kaziranga National Park, a World
Heritage site that is a home to the
\href{https://www.worldwildlife.org/species/greater-one-horned-rhino}{one-horned
Indian rhinoceros}, a species listed as vulnerable by the WWF, has been
\href{https://www.hindustantimes.com/india-news/tiger-takes-shelter-in-goat-shed-as-assam-s-kaziranga-national-park-gets-flooded/story-9nixAzckp91XhLv9MSPe6O.html}{completely
inundated}. Officials said that more than 50 animals had died in the
flooding, though some wildlife had been rescued.

With more than a dozen rivers and tributaries swelling above the danger
mark, rescue operations have been carried out in at least 22 districts
across Assam.

In Nepal, 67 people have died and 40 others are missing, according to
the National Emergency Operation Center.

That is in additional to the monsoons that have battered Bangladesh.
Low-lying and densely populated, the country is chronically ravaged by
flooding.

In Jamalpur, in the north, the flood situation has become critical, with
rivers flowing well above the danger level. Muneeb-ul-Islam, 42, who
lives in the area with his wife and three children, said he had lost his
home several times in 10 years, leaving him with nothing but the clothes
he was wearing.

Mr. ul-Islam and his family are among more than a million people in
Bangladesh left displaced or homeless by the floods.

``It is as if we have committed some sin,'' Mr. ul-Islam said. ``This is
the third time in the last few years that we will have to rebuild our
lives from scratch.''

Ms. Begum, who lost her 1-year-old, said her life had been completely
destroyed. She has now moved to a nearby shelter, a school building,
where hundreds of people were crammed in. Fear of
\href{https://www.nytimes.com/interactive/2020/world/coronavirus-maps.html}{the
coronavirus} spreading in such cramped quarters looms large. Ms. Begum's
family said there had not been enough warning about the magnitude of the
flooding.

``I will never go back to the place where we used to live,'' she said,
``The water has snatched everything from us.''

Advertisement

\protect\hyperlink{after-bottom}{Continue reading the main story}

\hypertarget{site-index}{%
\subsection{Site Index}\label{site-index}}

\hypertarget{site-information-navigation}{%
\subsection{Site Information
Navigation}\label{site-information-navigation}}

\begin{itemize}
\tightlist
\item
  \href{https://help.nytimes.com/hc/en-us/articles/115014792127-Copyright-notice}{©~2020~The
  New York Times Company}
\end{itemize}

\begin{itemize}
\tightlist
\item
  \href{https://www.nytco.com/}{NYTCo}
\item
  \href{https://help.nytimes.com/hc/en-us/articles/115015385887-Contact-Us}{Contact
  Us}
\item
  \href{https://www.nytco.com/careers/}{Work with us}
\item
  \href{https://nytmediakit.com/}{Advertise}
\item
  \href{http://www.tbrandstudio.com/}{T Brand Studio}
\item
  \href{https://www.nytimes.com/privacy/cookie-policy\#how-do-i-manage-trackers}{Your
  Ad Choices}
\item
  \href{https://www.nytimes.com/privacy}{Privacy}
\item
  \href{https://help.nytimes.com/hc/en-us/articles/115014893428-Terms-of-service}{Terms
  of Service}
\item
  \href{https://help.nytimes.com/hc/en-us/articles/115014893968-Terms-of-sale}{Terms
  of Sale}
\item
  \href{https://spiderbites.nytimes.com}{Site Map}
\item
  \href{https://help.nytimes.com/hc/en-us}{Help}
\item
  \href{https://www.nytimes.com/subscription?campaignId=37WXW}{Subscriptions}
\end{itemize}
