Sections

SEARCH

\protect\hyperlink{site-content}{Skip to
content}\protect\hyperlink{site-index}{Skip to site index}

\href{https://www.nytimes.com/section/arts/music}{Music}

\href{https://myaccount.nytimes.com/auth/login?response_type=cookie\&client_id=vi}{}

\href{https://www.nytimes.com/section/todayspaper}{Today's Paper}

\href{/section/arts/music}{Music}\textbar{}To Make Orchestras More
Diverse, End Blind Auditions

\href{https://nyti.ms/3eAwdbk}{https://nyti.ms/3eAwdbk}

\begin{itemize}
\item
\item
\item
\item
\item
\end{itemize}

Advertisement

\protect\hyperlink{after-top}{Continue reading the main story}

Supported by

\protect\hyperlink{after-sponsor}{Continue reading the main story}

critic's notebook

\hypertarget{to-make-orchestras-more-diverse-end-blind-auditions}{%
\section{To Make Orchestras More Diverse, End Blind
Auditions}\label{to-make-orchestras-more-diverse-end-blind-auditions}}

If ensembles are to reflect the communities they serve, the audition
process should take into account race, gender and other factors.

\includegraphics{https://static01.nyt.com/images/2020/07/19/arts/19race-tommasini-2/19race-tommasini-2-articleLarge.jpg?quality=75\&auto=webp\&disable=upscale}

\href{https://www.nytimes.com/by/anthony-tommasini}{\includegraphics{https://static01.nyt.com/images/2018/02/20/multimedia/author-anthony-tommasini/author-anthony-tommasini-thumbLarge.jpg}}

By \href{https://www.nytimes.com/by/anthony-tommasini}{Anthony
Tommasini}

\begin{itemize}
\item
  July 16, 2020
\item
  \begin{itemize}
  \item
  \item
  \item
  \item
  \item
  \end{itemize}
\end{itemize}

During the tumultuous summer of 1969, two Black musicians
\href{https://www.nytimes.com/2011/02/04/arts/music/04archive.html}{accused
the New York Philharmonic} of discrimination. Earl Madison, a cellist,
and J. Arthur Davis, a bassist, said they had been rejected for
positions because of their race.

The city's Commission on Human Rights decided against the musicians, but
found that aspects of the orchestra's hiring system, especially
regarding substitute and extra players, functioned as an old boys'
network and were discriminatory. The ruling helped prod American
orchestras, finally, to try and deal with the biases that had kept them
overwhelmingly white and male. The Philharmonic, and many other
ensembles, began to hold auditions behind a screen, so that factors like
race and gender wouldn't influence strictly musical appraisals.

Blind auditions, as they became known, proved transformative. The
percentage of women in orchestras, which
\href{https://gap.hks.harvard.edu/orchestrating-impartiality-impact-\%E2\%80\%9Cblind\%E2\%80\%9D-auditions-female-musicians}{hovered
under 6 percent in 1970}, grew. Today, women make up a third of the
Boston Symphony Orchestra, and they are half the New York Philharmonic.
Blind auditions changed the face of American orchestras.

But not enough.

American orchestras remain among the nation's
\href{https://www.nytimes.com/2018/04/18/arts/music/symphony-orchestra-diversity.html}{least
racially diverse institutions}, especially in regard to Black and Latino
artists. In a 2014 study, only 1.8 percent of the players in top
ensembles were Black; just 2.5 percent were Latino. At the time of the
Philharmonic's 1969 discrimination case, it had one Black player, the
first it ever hired: Sanford Allen, a violinist. Today, in a city that
is a quarter Black, just one out of 106 full-time players is Black:
Anthony McGill, the principal clarinet.

\includegraphics{https://static01.nyt.com/images/2020/07/19/arts/19race-tommasini-5/merlin_169681125_826fc99e-bfce-4525-b676-a9a91fc8c10e-articleLarge.jpg?quality=75\&auto=webp\&disable=upscale}

The status quo is not working. If things are to change, ensembles must
be able to take proactive steps to address the appalling racial
imbalance that remains in their ranks. Blind auditions are no longer
tenable.

This well-intentioned but restrictive practice has prevented substantive
action when it comes to the most essential element of maintaining an
orchestra: hiring musicians. Musicians' unions, which have in many ways
valiantly worked to protect their members in an economically tenuous
industry, have long been tenacious defenders of blind auditions,
asserting that they are the best way to ensure fairness.

But in sticking so stubbornly to the practice, unions may be hurting
themselves, their orchestras and their art form. Hanging on to a system
that has impeded diversity is particularly conspicuous at a moment when
the country has been galvanized by revulsion to police brutality against
Black Americans --- and when orchestras, largely shuttered by the
coronavirus pandemic, are brainstorming both how to be more relevant to
their communities and how to redress racial inequities among their
personnel when they re-emerge.

If the musicians onstage are going to better reflect the diversity of
the communities they serve, the audition process has to be altered to
take into fuller account artists' backgrounds and experiences. Removing
the screen is a crucial step.

Blind auditions are based on an appealing premise of pure meritocracy:
An orchestra should be built from the very best players, period. But ask
anyone in the field, and you'll learn that over the past century of
increasingly professionalized training, there has come to be remarkably
little difference between players at the top tier. There is an athletic
component to playing an instrument, and as with sprinters, gymnasts and
tennis pros, the basic level of technical skill among American
instrumentalists has steadily risen. A typical orchestral audition might
end up attracting dozens of people who are essentially indistinguishable
in their musicianship and technique.

It's like an elite college facing a sea of applicants with straight A's
and perfect test scores. Such a school can move past those marks,
embrace diversity as a social virtue and assemble a freshman class that
advances other values along with academic achievement. For orchestras,
the qualities of an ideal player might well include talent as an
educator, interest in unusual repertoire or willingness to program
innovative chamber events as well as pure musicianship. American
orchestras should be able to foster these values, and a diverse
complement of musicians, rather than passively waiting for
representation to emerge from behind the audition screen.

Some leaders in the field I've spoken with over the years have argued
that the problem starts earlier than auditions. They say racial
diversity is missing in the so-called pipeline that leads from learning
an instrument to summer programs to conservatories to graduate education
to elite jobs. In this view, even that strong pool of equally talented
hypothetical auditioners might have few, if any, Black or Latino players
in it.

Yet Afa S. Dworkin, the president of the
\href{http://www.sphinxmusic.org/}{Sphinx Organization}, which is
dedicated to encouraging diversity in classical music by fostering young
artists, argues that the pipeline is not the problem, and that talented
musicians of color are out there and ready.

Image

Afa S. Dworkin, the president of the Sphinx Organization (top left),
participating in an online panel discussion among leading Black
musicians.

``As we speak,'' she said in a recent online
\href{https://www.facebook.com/watch/?v=693343974780937}{roundtable
discussion} among leading Black musicians, ``about 96 Black and brown
students who were competitively selected from hundreds who auditioned
for Sphinx's summer programs are going to go through intensive solo and
chamber music training.''

She added that any of those young artists would soon be worthy of
entrance to an elite conservatory and, in just a few years, ready for
top-tier auditions.

Sphinx has been attempting to change the auditions landscape. Two years
ago, alongside the \href{https://www.nws.edu/}{New World Symphony}, a
prestigious --- and notably diverse --- training orchestra for
post-college musicians, and the
\href{https://americanorchestras.org/}{League of American Orchestras}, a
trade group, Sphinx began a program to train musicians for auditions by
pairing them with mentors, giving them performance opportunities and
awarding them stipends to travel to auditions. (The heavy costs
associated with auditions disproportionately affect younger musicians of
color; if you can't afford to buy many flights and hotel rooms each
year, it doesn't matter how well you play.)

But orchestras must be a part of changing the landscape, too, by getting
rid of blind auditions.

Change can be unnerving. Might the gains female players have made be
reversed if the screen comes down? Might old habits of favoring the
students of veteran players return? Orchestras will need to be
transparent about their goals and procedures if they are to move forward
with a new approach to auditions --- one that takes race and gender into
account, along with the full spectrum of a musician's experience.

Image

Mr. McGill performing as a soloist with the Philharmonic in
2018.Credit...Hiroyuki Ito for The New York Times

I put the question to Mr. McGill, the Philharmonic's principal clarinet
since 2014, who was more ambivalent about blind auditions than I am.

``I don't know what the right answers are,'' he said, adding that the
screen has proved effective at eliminating the coziness that can creep
into the auditions process when members of the jury have worked with the
person playing.

Yet, he added, ``representation matters more than people know.'' He
recalled how crucial it was to his early development as a clarinetist,
growing up on the South Side of Chicago, to be part of the Chicago Teen
Ensemble, a small group of young Black musicians who worked with a
coach, made their own musical arrangements and toured the city giving
concerts. It gave Mr. McGill, he said, a sense that classical music ``is
very normal,'' the same sense his presence could give to a young Black
person watching the Philharmonic.

``Is slow and steady change fast enough?'' he asked. ``The world has
changed around us.''

When the Philharmonic plays, Mr. McGill stands out, not just for his
magnificent playing but also as the kind of role model he looked to as a
young artist. Yet, now more than ever, the spectacle of a lone Black
musician on a huge, packed stage at Lincoln Center is unbearably
depressing. Slow and steady change is no longer fast enough.

Advertisement

\protect\hyperlink{after-bottom}{Continue reading the main story}

\hypertarget{site-index}{%
\subsection{Site Index}\label{site-index}}

\hypertarget{site-information-navigation}{%
\subsection{Site Information
Navigation}\label{site-information-navigation}}

\begin{itemize}
\tightlist
\item
  \href{https://help.nytimes.com/hc/en-us/articles/115014792127-Copyright-notice}{©~2020~The
  New York Times Company}
\end{itemize}

\begin{itemize}
\tightlist
\item
  \href{https://www.nytco.com/}{NYTCo}
\item
  \href{https://help.nytimes.com/hc/en-us/articles/115015385887-Contact-Us}{Contact
  Us}
\item
  \href{https://www.nytco.com/careers/}{Work with us}
\item
  \href{https://nytmediakit.com/}{Advertise}
\item
  \href{http://www.tbrandstudio.com/}{T Brand Studio}
\item
  \href{https://www.nytimes.com/privacy/cookie-policy\#how-do-i-manage-trackers}{Your
  Ad Choices}
\item
  \href{https://www.nytimes.com/privacy}{Privacy}
\item
  \href{https://help.nytimes.com/hc/en-us/articles/115014893428-Terms-of-service}{Terms
  of Service}
\item
  \href{https://help.nytimes.com/hc/en-us/articles/115014893968-Terms-of-sale}{Terms
  of Sale}
\item
  \href{https://spiderbites.nytimes.com}{Site Map}
\item
  \href{https://help.nytimes.com/hc/en-us}{Help}
\item
  \href{https://www.nytimes.com/subscription?campaignId=37WXW}{Subscriptions}
\end{itemize}
