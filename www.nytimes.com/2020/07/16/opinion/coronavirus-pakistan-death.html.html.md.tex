Sections

SEARCH

\protect\hyperlink{site-content}{Skip to
content}\protect\hyperlink{site-index}{Skip to site index}

\href{https://myaccount.nytimes.com/auth/login?response_type=cookie\&client_id=vi}{}

\href{https://www.nytimes.com/section/todayspaper}{Today's Paper}

\href{/section/opinion}{Opinion}\textbar{}What's With All the
Covid-Death Shaming?

\href{https://nyti.ms/2Wlo2ti}{https://nyti.ms/2Wlo2ti}

\begin{itemize}
\item
\item
\item
\item
\item
\end{itemize}

Advertisement

\protect\hyperlink{after-top}{Continue reading the main story}

\href{/section/opinion}{Opinion}

Supported by

\protect\hyperlink{after-sponsor}{Continue reading the main story}

\hypertarget{whats-with-all-the-covid-death-shaming}{%
\section{What's With All the Covid-Death
Shaming?}\label{whats-with-all-the-covid-death-shaming}}

Some Pakistanis won't say they are losing family members to the pandemic
because they don't want to bury the bodies alone.

\href{https://www.nytimes.com/by/mohammed-hanif}{\includegraphics{https://static01.nyt.com/images/2016/10/21/opinion/HANIF-CIRCLE-GRAY/HANIF-CIRCLE-GRAY-thumbLarge-v2.jpg}}

By \href{https://www.nytimes.com/by/mohammed-hanif}{Mohammed Hanif}

Mr. Hanif is a novelist.

\begin{itemize}
\item
  July 16, 2020
\item
  \begin{itemize}
  \item
  \item
  \item
  \item
  \item
  \end{itemize}
\end{itemize}

\includegraphics{https://static01.nyt.com/images/2020/07/15/opinion/15Hanif/merlin_173282280_e597d4be-ff4c-4d96-8409-bee9801ddd02-articleLarge.jpg?quality=75\&auto=webp\&disable=upscale}

KARACHI, Pakistan --- I've made a few condolence calls during the past
few months. None of the people I called to condole about had died of
Covid-19. I was always given another reason.

It started with the death of an uncle two months ago. I called to
condole with my cousin. We had not spoken in more than a decade. My
uncle was in his 80s, and I thought it'd be a life-well-lived type of
conversation, but soon it turned into a detailed and uncalled-for
denial: My dad, your uncle, didn't die of Covid-19, my cousin said.

My uncle had tested negative; he died of a heart attack, I was told. I
hadn't even mentioned the C-word.

I managed to get in my life-well-lived bit and hung up.

Since then I have seen a pattern: I have made a half-dozen calls, the
subject of my condolences were five men and one woman, between the ages
of 31 and 82, and without my ever mentioning the subject, I was told,
``It wasn't what you think.''

There was one exception: A friend's father died and my friend told me
his father had died of Covid-19 but that the family was trying to cover
that up.

My friend is a doctor who has lived and worked in the United States for
the past three decades. When I called him, he told me about the
circumstances of his father's death: His father was in his late 70s and
like many, many Pakistanis had chosen to believe that either the new
coronavirus is a hoax or it would shy away from good Muslims.

My friend's father had been socializing in his neighborhood of Karachi
and offering his prayers in public, going into the mosque through the
back door when, for a time, people his age were banned from visiting. He
fell sick, was hospitalized, tested positive for the virus and died a
couple of days later.

This was in early May, around the time that the funerals of people who
were dying of Covid-19 were being managed by the local authorities
rather than relatives, because some members of the medical community
believed then that the dead could infect the living. My friend's
relatives had to bribe the police to get the body, and although
initially they tried to hide the cause of death, the news leaked.

``There were people coming to condole,'' my friend said, ``but others
were coming to complain: `If your father had Covid, why did you let him
mingle with our elders? What if he has infected us all?'''

The family denied vehemently that he had died of Covid-19. ``We have
lived on that street for generations. I can understand why my brothers
did it.''

In Karachi, I called up a doctor to ask why people were hiding Covid-19
deaths. ``People don't want to see some dead body and the police arrive
at their homes at the same time,'' I was told.

Another doctor said to me: ``There is fear and shame and stigma attached
to death by Covid, as if you got infected by this virus by doing
something immoral or filthy, rather than by touching a door knob.''

And what if nobody will come to your loved one's funeral? What if nobody
comes to condole? What if people think that because you are related to
the deceased you might pass the disease on to them and their children?

It's best to just deny what happened. Weren't people dying before this
virus anyway? What's the harm in slightly shifting the cause of death?
Even when someone dies of Covid-19 they don't really, really die of
Covid-19; they die of the complications caused by Covid-19.

And so you lie and move on. After all, 98 percent of people who get
infected do survive.

The longest condolence conversation I've had was with a close friend of
mine whose sister passed away at the beginning of the month. She had
been suffering from some unknown illness and was admitted to a private
hospital in Lahore.

When her condition worsened, the family was asked to take her to
Services Hospital, a major government facility with a designated ward
for Covid-19 patients. My friend insisted that his sister had tested
negative for the coronavirus, but when they arrived at the hospital a
doctor directed them to the Covid section.

``If we had let her into that ward, they wouldn't even have given us the
body,'' my friend said.

Some of the myths around the virus have been propagated by the very
people who were supposed to guide us through this crisis. Prime Minister
Imran Khan has said that 90 percent of Covid-19 cases were like a
``\href{https://asia.nikkei.com/Spotlight/Coronavirus/Conspiracy-theories-help-coronavirus-take-root-in-Pakistan}{normal
flu}'' and
\href{https://indianexpress.com/article/trending/trending-globally/pak-pm-imran-khan-coronavirus-address-criticism-6320050/}{you
don't go to the hospital when you have the flu}.

In early May, Asad Umar, the federal minister for planning, development
and special initiatives and the man in charge of handling the
coronavirus crisis, gave a briefing about how many more people in
Pakistan
\href{https://arynews.tv/en/coronavirus-fatal-other-countries-asad-umar/}{die
in traffic accidents} than they do because of Covid-19 and yet,
\href{https://nayadaur.tv/2020/05/minister-asad-umar-compares-covid-19-to-road-accidents/}{he
said}, ``we still allow cars on roads, because their necessity is
greater than the danger of those accidents.''

The prime minister's health adviser, Zafar Mirza, said early this month
that the coming monsoon season
\href{https://www.samaa.tv/news/2020/07/monsoon-will-decrease-coronavirus-cases-zafar-mirza/}{would
wash away the coronavirus}.
``\href{https://nayadaur.tv/2020/07/monsoon-rains-could-help-mitigate-covid-pandemic-in-pakistan-zafar-mirza/}{But
it's too early to tell},'' he added. As the first rains fell a few days
later news came that Dr. Mirza had
\href{https://www.dawn.com/news/1567337/sapm-dr-zafar-mirza-tests-positive-for-covid-19-says-he-has-mild-symptoms}{tested
positive for the virus}.

In late March, during the early weeks of the pandemic's spread in
Pakistan, a telephone call between Nadeem Afzal Chan, one of the prime
minister's advisers and spokesmen, and a political lieutenant of Mr.
Chan's
\href{https://nayadaur.tv/2020/03/govt-hiding-thousands-of-coronavirus-deaths-says-pms-assistant-in-alleged-audio-clip/}{was
leaked} on social media. Mr. Chan is heard telling his aide,
\href{https://nayadaur.tv/2020/03/govt-hiding-thousands-of-coronavirus-deaths-says-pms-assistant-in-alleged-audio-clip/}{in
rather colorful language}, to forget about political campaigning, go
home and stay there with his kids.

After a Punjabi invective that can't be translated here, Mr. Chan said:
``Thousands are dying, but the government is hiding it.''

Later, Mr. Chan
\href{https://nayadaur.tv/2020/03/govt-hiding-thousands-of-coronavirus-deaths-says-pms-assistant-in-alleged-audio-clip/}{admitted}
that the conversation was genuine but said he had exaggerated the facts
because his lieutenant, a friend, wasn't taking the threat seriously.
``Political workers like him still want to campaign, inaugurate roads,
hold rallies.''

Mr. Chan was lying to save a friend from maybe dying, and now people
around me are lying because they don't want their friends to abandon
them while they bury their dead.

I called up a doctor friend who has lost two colleagues during the past
month, both of them because of Covid-19. Why are people still lying
about this? I asked, again.

``I am, too,'' he said. ``On all the death certificates I issue I write
`cardiac pulmonary failure' as the cause of death. That saves them the
embarrassment.''

And maybe then at least some people will be around when the last spade
of earth is shoveled into those graves.

Mohammed Hanif
(\href{https://twitter.com/mohammedhanif}{@mohammedhanif}) is the author
of the novels ``A Case of Exploding Mangoes,'' ``Our Lady of Alice
Bhatti'' and ``Red Birds.'' He is a contributing opinion writer.

\emph{The Times is committed to publishing}
\href{https://www.nytimes.com/2019/01/31/opinion/letters/letters-to-editor-new-york-times-women.html}{\emph{a
diversity of letters}} \emph{to the editor. We'd like to hear what you
think about this or any of our articles. Here are some}
\href{https://help.nytimes.com/hc/en-us/articles/115014925288-How-to-submit-a-letter-to-the-editor}{\emph{tips}}\emph{.
And here's our email:}
\href{mailto:letters@nytimes.com}{\emph{letters@nytimes.com}}\emph{.}

\emph{Follow The New York Times Opinion section on}
\href{https://www.facebook.com/nytopinion}{\emph{Facebook}}\emph{,}
\href{http://twitter.com/NYTOpinion}{\emph{Twitter (@NYTopinion)}}
\emph{and}
\href{https://www.instagram.com/nytopinion/}{\emph{Instagram}}\emph{.}

Advertisement

\protect\hyperlink{after-bottom}{Continue reading the main story}

\hypertarget{site-index}{%
\subsection{Site Index}\label{site-index}}

\hypertarget{site-information-navigation}{%
\subsection{Site Information
Navigation}\label{site-information-navigation}}

\begin{itemize}
\tightlist
\item
  \href{https://help.nytimes.com/hc/en-us/articles/115014792127-Copyright-notice}{©~2020~The
  New York Times Company}
\end{itemize}

\begin{itemize}
\tightlist
\item
  \href{https://www.nytco.com/}{NYTCo}
\item
  \href{https://help.nytimes.com/hc/en-us/articles/115015385887-Contact-Us}{Contact
  Us}
\item
  \href{https://www.nytco.com/careers/}{Work with us}
\item
  \href{https://nytmediakit.com/}{Advertise}
\item
  \href{http://www.tbrandstudio.com/}{T Brand Studio}
\item
  \href{https://www.nytimes.com/privacy/cookie-policy\#how-do-i-manage-trackers}{Your
  Ad Choices}
\item
  \href{https://www.nytimes.com/privacy}{Privacy}
\item
  \href{https://help.nytimes.com/hc/en-us/articles/115014893428-Terms-of-service}{Terms
  of Service}
\item
  \href{https://help.nytimes.com/hc/en-us/articles/115014893968-Terms-of-sale}{Terms
  of Sale}
\item
  \href{https://spiderbites.nytimes.com}{Site Map}
\item
  \href{https://help.nytimes.com/hc/en-us}{Help}
\item
  \href{https://www.nytimes.com/subscription?campaignId=37WXW}{Subscriptions}
\end{itemize}
