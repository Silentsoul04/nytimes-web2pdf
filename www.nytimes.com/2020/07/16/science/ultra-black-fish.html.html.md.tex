Sections

SEARCH

\protect\hyperlink{site-content}{Skip to
content}\protect\hyperlink{site-index}{Skip to site index}

\href{https://www.nytimes.com/section/science}{Science}

\href{https://myaccount.nytimes.com/auth/login?response_type=cookie\&client_id=vi}{}

\href{https://www.nytimes.com/section/todayspaper}{Today's Paper}

\href{/section/science}{Science}\textbar{}How Ultra-Black Fish Disappear
in the Deepest Seas

\url{https://nyti.ms/3jedsyi}

\begin{itemize}
\item
\item
\item
\item
\item
\item
\end{itemize}

Advertisement

\protect\hyperlink{after-top}{Continue reading the main story}

Supported by

\protect\hyperlink{after-sponsor}{Continue reading the main story}

Trilobites

\hypertarget{how-ultra-black-fish-disappear-in-the-deepest-seas}{%
\section{How Ultra-Black Fish Disappear in the Deepest
Seas}\label{how-ultra-black-fish-disappear-in-the-deepest-seas}}

Researchers have found fish that absorb more than 99.9 percent of the
light that hits their skin.

\includegraphics{https://static01.nyt.com/images/2020/07/21/science/16TB-ULTRABLACKFISH1/16TB-ULTRABLACKFISH1-articleLarge.jpg?quality=75\&auto=webp\&disable=upscale}

By \href{https://www.nytimes.com/by/katherine-j--wu}{Katherine J. Wu}

\begin{itemize}
\item
  July 16, 2020
\item
  \begin{itemize}
  \item
  \item
  \item
  \item
  \item
  \item
  \end{itemize}
\end{itemize}

Alexander Davis admits that he can be a glutton for punishment. He
staked part of his Ph.D. on finding some of the world's best-camouflaged
fishes in the ocean's deepest depths. These animals are so keen on not
being found that they've evolved the ability to absorb more than 99.9
percent of the light that hits their skin.

To locate and study these so-called ultra-black fishes, Mr. Davis, a
biologist at Duke University, said he relied largely on the luck of the
draw. ``We basically just drop nets and see what we get,'' he said.
``You never know what you're going to pull up.''

When he and his colleagues did cash in, they cashed in big. In a paper
published Thursday in
\href{http://dx.doi.org/10.1016/j.cub.2020.06.044}{Current Biology},
they report snaring the first documented ultra-black animals in the
ocean, and some of the darkest creatures ever found: 16 types of
deep-sea fish that are so black, they manifest as permanent silhouettes
--- light-devouring voids that almost seem to shred the fabric of
space-time.

``It's like looking at a black hole,'' Mr. Davis said.

To qualify as
\href{https://www.nytimes.com/2019/11/11/science/black-fashion-physics-animals.html}{ultra-black},
a substance has to reflect less than 0.5 percent of the light that hits
it. Some
\href{https://cdn.mos.cms.futurecdn.net/qpDcS5xokG7zrWesXcXd8V.jpg}{birds
of paradise} manage this, beaming back
\href{https://www.nature.com/articles/s41467-017-02088-w}{as little as
0.05 percent}, as do certain types of
\href{https://www.nature.com/articles/s41467-020-15033-1}{butterflies}
(0.06 percent) and
\href{https://royalsocietypublishing.org/doi/10.1098/rspb.2019.0589}{spiders}
(0.35 percent). A feat of engineering allowed humans to best them all
with \href{https://www.pnas.org/content/106/15/6044}{synthetic
materials}, some of which reflect only 0.045 percent of incoming light.
(``Black'' paper, on the other hand, returns a whopping 10 percent of
the light it meets.)

\includegraphics{https://static01.nyt.com/images/2020/07/21/science/16TB-ULTRABLACKFISH2/16TB-ULTRABLACKFISH2-articleLarge.jpg?quality=75\&auto=webp\&disable=upscale}

Now, it seems fish may come close to trouncing them all.

One species profiled in the paper, a bioluminescent anglerfish in the
genus Oneirodes, reflects as little as 0.044 to 0.051 percent of the
deep-sea light it encounters. The other 99.95 percent, Mr. Davis and his
colleagues found, gets lost in a labyrinth of light-swallowing pigments
until it effectively disappears.

``I'm always arguing with bird people on the internet,'' said Kory
Evans, a fish biologist at Rice University who wasn't involved in the
study. ``I say, `I bet these deep-sea fish are as dark as your birds of
paradise.' And then boom, they checked, and that was exactly the case.''

Super-dark skin might seem redundant hundreds or thousands of feet
beneath the surface of the sea, where the sun's rays don't reach. But
thanks to the D.I.Y. light cooked up by bioluminescent creatures, this
part of the ocean can actually ``sparkle like the sky,'' said Prosanta
Chakrabarty, a fish biologist at Louisiana State University who wasn't
involved in the study.

Birds, butterflies and spiders tend to use ultra-black for contrast,
making vibrant patches of color pop against an extreme backdrop. Some
fish may do this, too. But in a world where many deep-sea lurkers use
their homemade glow to lure in prey, ultra-black may function more as a
disappearing act for swimmers that don't want to be spotted, Dr. Evans
said.

To suss out how deep-sea fishes conjure their cloaks of invisibility,
the researchers took skin samples from nine species of ultra-black fish
and analyzed them under the microscope.

Like many other animals, including humans, fish pigment their skin with
melanin, a light-absorbing compound stored in microscopic compartments
called melanosomes. Typically colored fish scatter these pockets of
pigment into a sparse, even layer held up by a protein called collagen.
Any light that hits the melanin head-on is gobbled up, while light that
misses the mark ricochets back toward the viewer.

To maintain their stealth, the researchers found, ultra-black fishes
skimp on the collagen. That allows them to pack their melanosomes
together like piled grains of rice. When light contacts the clutter,
what's not absorbed is deflected sideways --- straight into the path of
another ravenous melanosome.

Image

A crested bigscale, whose ultra-black skin covers its scales, which can
detach if a predator tries to grab it.Credit...Karen Osborn, Smithsonian

Ultra-black birds, butterflies and spiders do something similar, but
perhaps in a less efficient way, said Karen Osborn, a zoologist at the
Smithsonian National Museum of Natural History and an author on the
study, which she began in 2014. Rather than using the same structure ---
melanosomes ---~to absorb and deflect light, as fish do, these
land-living animals embed their melanin in mazes of bumps, boxes or
spikes that bounce photons back and forth. What deep-sea fish do ``is a
much simpler system,'' Dr. Osborn said.

That could be a saving grace for creatures that must eke out a living in
an environment as harsh and unforgiving as the deep sea, said Anela
Choy, a deep-sea researcher at the Scripps Institution of Oceanography
in San Diego who wasn't involved in the study.

Down there, Dr. Choy said, everything ``has to do with survival: eating,
not being eaten and reproducing yourself.''

Some of the ocean's deepest dwellers might be even darker than what Mr.
Davis and colleagues have dredged up.

``I would not be surprised if we have not yet found the blackest fish in
the sea,'' Dr. Chakrabarty said.

Advertisement

\protect\hyperlink{after-bottom}{Continue reading the main story}

\hypertarget{site-index}{%
\subsection{Site Index}\label{site-index}}

\hypertarget{site-information-navigation}{%
\subsection{Site Information
Navigation}\label{site-information-navigation}}

\begin{itemize}
\tightlist
\item
  \href{https://help.nytimes.com/hc/en-us/articles/115014792127-Copyright-notice}{©~2020~The
  New York Times Company}
\end{itemize}

\begin{itemize}
\tightlist
\item
  \href{https://www.nytco.com/}{NYTCo}
\item
  \href{https://help.nytimes.com/hc/en-us/articles/115015385887-Contact-Us}{Contact
  Us}
\item
  \href{https://www.nytco.com/careers/}{Work with us}
\item
  \href{https://nytmediakit.com/}{Advertise}
\item
  \href{http://www.tbrandstudio.com/}{T Brand Studio}
\item
  \href{https://www.nytimes.com/privacy/cookie-policy\#how-do-i-manage-trackers}{Your
  Ad Choices}
\item
  \href{https://www.nytimes.com/privacy}{Privacy}
\item
  \href{https://help.nytimes.com/hc/en-us/articles/115014893428-Terms-of-service}{Terms
  of Service}
\item
  \href{https://help.nytimes.com/hc/en-us/articles/115014893968-Terms-of-sale}{Terms
  of Sale}
\item
  \href{https://spiderbites.nytimes.com}{Site Map}
\item
  \href{https://help.nytimes.com/hc/en-us}{Help}
\item
  \href{https://www.nytimes.com/subscription?campaignId=37WXW}{Subscriptions}
\end{itemize}
