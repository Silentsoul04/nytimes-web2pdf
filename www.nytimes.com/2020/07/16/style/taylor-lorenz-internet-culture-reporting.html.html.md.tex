Sections

SEARCH

\protect\hyperlink{site-content}{Skip to
content}\protect\hyperlink{site-index}{Skip to site index}

\href{https://www.nytimes.com/section/style}{Style}

\href{https://myaccount.nytimes.com/auth/login?response_type=cookie\&client_id=vi}{}

\href{https://www.nytimes.com/section/todayspaper}{Today's Paper}

\href{/section/style}{Style}\textbar{}How to Report on Internet Culture
and the Teens Who Rule It

\url{https://nyti.ms/3h25x4U}

\begin{itemize}
\item
\item
\item
\item
\item
\item
\end{itemize}

Advertisement

\protect\hyperlink{after-top}{Continue reading the main story}

Supported by

\protect\hyperlink{after-sponsor}{Continue reading the main story}

\hypertarget{how-to-report-on-internet-culture-and-the-teens-who-rule-it}{%
\section{How to Report on Internet Culture and the Teens Who Rule
It}\label{how-to-report-on-internet-culture-and-the-teens-who-rule-it}}

Taylor Lorenz spends most of her days reporting on the ways we use the
internet. Here's how she does it.

\includegraphics{https://static01.nyt.com/images/2020/07/16/fashion/16WAIT-1-SUB/16WAIT-1-SUB-articleLarge-v4.jpg?quality=75\&auto=webp\&disable=upscale}

\begin{itemize}
\item
  July 16, 2020
\item
  \begin{itemize}
  \item
  \item
  \item
  \item
  \item
  \item
  \end{itemize}
\end{itemize}

\emph{We spend a lot of our days chatting with each other about things
we see online, trying to make sense of it all. This week in the Styles
newsletter, Wait \ldots{}, Taylor Lorenz, a Styles reporter, Lindsey
Underwood,} \href{https://www.nytimes.com/section/style}{\emph{a Styles
editor}}\emph{, and Jessica Grose, the}
\href{https://www.nytimes.com/section/parenting}{\emph{Parenting
editor}}\emph{, discussed how we report on teenagers and the internet.}

\textbf{Jessica:}
\href{https://www.nytimes.com/by/taylor-lorenz}{Taylor,} I feel
\href{https://www.nytimes.com/2020/07/14/style/what-is-the-cake-meme.html}{your}
\href{https://www.nytimes.com/2020/07/02/style/tati-devin-tiktok.html}{work}
in
\href{https://www.nytimes.com/2020/06/24/style/roller-skating-is-back-baby-taylor-lorenz.html}{explaining}
the
\href{https://www.nytimes.com/2020/06/16/style/blm-accounts-social-media-high-school.html}{way}
teens
\href{https://www.nytimes.com/2020/06/02/style/police-protests-video.html}{use}
the
\href{https://www.nytimes.com/2020/05/19/style/call-her-daddy-podcast.html}{internet}
is a public service to parents like myself. How do you find your topics
in their vast online world?

\textbf{Taylor:} I write stories about culture and technology; a lot of
my stories have
\href{https://www.nytimes.com/2020/06/24/style/roller-skating-is-back-baby-taylor-lorenz.html}{nothing
to do} with young people. But young people are often
\href{https://www.nytimes.com/2020/04/30/style/instagram-yearbook-coronavirus.html}{leading
the way online}, so I do end up talking to them a lot. I also cover the
\href{https://www.nytimes.com/2020/06/29/style/shane-dawson-jeffree-star-youtube-taylor-lorenz.html}{online
creator} and
\href{https://www.nytimes.com/2020/05/21/style/tiktok-collab-houses-quarantine-coronavirus.html}{influencer
industry}, which is dominated by young people. But I'm not a youth
culture reporter or a reporter on teens. I'm a technology reporter with
a focus on culture.

\textbf{Lindsey:} And how do you find your stories?

\textbf{Taylor:} I spend all day, every day on the internet.

\textbf{Lindsey:} I can attest to this! I sat next to you in the office
in The Before Times.

\textbf{Taylor:} It's very rare that I unplug! I keep my DMs open on
every platform and I spend a lot of time just listening to people. I try
to be one of the most accessible reporters at The New York Times. I want
people, when they see something interesting or newsworthy online, to
think ``Oh, I have to send this to Taylor.''

\textbf{Jessica:} How do you get in contact with teenagers? And do they
try to catfish you?

\textbf{Taylor:} I never do interviews over text or DM. My goal with
everyone I speak to, whether adult or teenager, is to get them on the
phone or a Zoom. Often, I'll DM a meme account, for instance, and not
have any idea of the person's age or identity behind it until I do more
reporting and verify their identity.

I regularly hop on the phone with parents and answer any questions. I've
become very friendly with the parents of several kids I've covered in
the past and keep in touch with the parents of many more.

When I've
\href{https://www.nytimes.com/2020/01/03/style/hype-house-los-angeles-tik-tok.html}{visited
teenager's houses}, for instance, for an in-person interview, that is
always coordinated with their parents either directly or through a
manager or agent who works with their parents.

I consider it part of my job to very clearly and explicitly explain the
ramifications that coverage in The Times will have. I have had
conversations about this with parents, for instance, about what would
likely happen when a story went up. I talk to kids about this a lot when
I'm discussing how we will refer to them and why. I would never write
about someone without communicating what an article about them means.

Absolutely, people lie to me and try to catfish me every day. Sometimes
that
\href{https://www.thedailybeast.com/reddits-favorite-high-school-porn-magnate-is-actually-22}{becomes
a story in itself}. Like any beat, you have to do an enormous amount of
due diligence throughout the reporting process.

\textbf{Jessica:} I want to know more about these managers!

\textbf{Taylor:} The people I work with most when covering teenagers are
actually managers and agents. More often than not, when I DM a teenager
looking to get in touch for a piece, it is a manager, agent or PR
representative who will respond. Teenagers with as few as 10,000
followers on a platform have talent managers now.

A lot of these kids' parents aren't aware of how famous their children
are, and don't always understand what their child is doing online that
made them so famous. (That is often
\href{https://www.nytimes.com/2020/02/13/style/the-original-renegade.html}{part
of the story}!) Some parents live quite far away, while their children
live in Los Angeles with a guardian or manager. I wrote
\href{https://www.theatlantic.com/technology/archive/2018/01/raising-a-social-media-star/550418/}{a
feature for The Atlantic several years ago} on what it's like for the
parents of kids who become influencers and social media stars.

\textbf{Lindsey:} What's your goal --- if you have one?

\textbf{Taylor:} I want to reveal the real work taking place in the
online creator world. I want people to recognize that influencing and
creating things online is a valid career and should be taken seriously.

A lot of young people, even the famous ones, are wary of the press. My
beat is based on taking young people seriously --- and that means not
exploiting them, or being condescending, or mocking them. I want them to
know they can trust me from what they see in my work --- but I also
never want any kid to mistake me as their peer. I don't try to speak
like a teenager or act like I'm in their world. It's important that they
recognize that I'm a professional reporter and an adult, just like their
parents and teachers.

That said, I approach all interviews, especially those with young
people, from a place of empathy. It is important that interview subjects
feel comfortable talking to me and trust me enough to tell the truth. I
take confidentiality very seriously and I take issues around identity
very seriously.

\textbf{Lindsey:} Do teens read your stories or is our coverage for
their parents?

\textbf{Taylor:} I receive a ton emails and messages from kids with
feedback and thoughts about my stories. I want young people to feel
represented by this global news organization.

I want kids to read us and feel like they matter to The New York Times
and that The Times is actually working to understand them. Growing up, I
looked up to women in media like Nancy Jo Sales and Atoosa Rubenstein
because I felt like they really understood me. I want to be that person
for the next generation.

\textbf{Lindsey:} Do you think teens are thinking about their Google
results or what it means to be in The Times?

\textbf{Taylor:} Oh yes. (And I've
\href{https://www.theatlantic.com/technology/archive/2019/02/when-kids-realize-their-whole-life-already-online/582916/}{written
about}this.) Young people are
\href{https://www.theatlantic.com/technology/archive/2019/02/when-kids-realize-their-whole-life-already-online/582916/}{hyper
aware of their digital footprint} from a very, very young age. Often,
they know that any story written about them could be the most prominent
(or Google-able) information about them for years to come.

\textbf{Jessica:} How much money is sloshing around these kids and where
is it coming from?

\textbf{Taylor:} Influencer marketing is a \$10 billion industry and
that only represents a fraction of the money flowing on the internet.
Teenagers today have unprecedented access to career and moneymaking
opportunities. They can
\href{https://www.theatlantic.com/technology/archive/2018/08/posting-instagram-sponsored-content-is-the-new-summer-job/568108/}{sell
sponsored content on Instagram as a summer job}, for instance. Others
are building media
\href{https://www.nytimes.com/2019/11/29/style/the-clout.html}{empires
from their bedrooms}.

\textbf{Lindsey:} What happened to teen magazines? I read YM and Teen
People. God, what I would have I given to be a CosmoGirl. What does a
cool teen read now?

\textbf{Taylor:} The entire media landscape has been upended by social
media and the internet. That's led to the decimation of the print
industry and a huge shift in how people consume media. That shift to
digital and social is even more apparent in the younger generation.
Before writing full time, I worked as a social strategist for brands and
other media companies. My job was to program content for younger people
and meet consumers where they are these days: Instagram, YouTube, etc.

Teens today don't have YM or Teen People, but they do have Instagram
accounts like \href{https://www.instagram.com/tiktokroom/}{TikTok room}
and \href{https://www.instagram.com/m3ssym0nday/}{Messy Monday}, and a
plethora of YouTubers and podcasters to keep them in the loop on what's
going on in the world. The number one thing kids ask me when I reach out
for an interview is: ``Will this be on
\href{https://www.youtube.com/user/TheNewYorkTimes}{The New York Times
YouTube channel}?''

\emph{(This conversation has been condensed and edited for clarity).}

Advertisement

\protect\hyperlink{after-bottom}{Continue reading the main story}

\hypertarget{site-index}{%
\subsection{Site Index}\label{site-index}}

\hypertarget{site-information-navigation}{%
\subsection{Site Information
Navigation}\label{site-information-navigation}}

\begin{itemize}
\tightlist
\item
  \href{https://help.nytimes.com/hc/en-us/articles/115014792127-Copyright-notice}{©~2020~The
  New York Times Company}
\end{itemize}

\begin{itemize}
\tightlist
\item
  \href{https://www.nytco.com/}{NYTCo}
\item
  \href{https://help.nytimes.com/hc/en-us/articles/115015385887-Contact-Us}{Contact
  Us}
\item
  \href{https://www.nytco.com/careers/}{Work with us}
\item
  \href{https://nytmediakit.com/}{Advertise}
\item
  \href{http://www.tbrandstudio.com/}{T Brand Studio}
\item
  \href{https://www.nytimes.com/privacy/cookie-policy\#how-do-i-manage-trackers}{Your
  Ad Choices}
\item
  \href{https://www.nytimes.com/privacy}{Privacy}
\item
  \href{https://help.nytimes.com/hc/en-us/articles/115014893428-Terms-of-service}{Terms
  of Service}
\item
  \href{https://help.nytimes.com/hc/en-us/articles/115014893968-Terms-of-sale}{Terms
  of Sale}
\item
  \href{https://spiderbites.nytimes.com}{Site Map}
\item
  \href{https://help.nytimes.com/hc/en-us}{Help}
\item
  \href{https://www.nytimes.com/subscription?campaignId=37WXW}{Subscriptions}
\end{itemize}
