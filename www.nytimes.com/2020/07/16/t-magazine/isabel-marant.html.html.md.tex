The Designer Who Defined Modern Parisian Cool

\url{https://nyti.ms/390yU54}

\begin{itemize}
\item
\item
\item
\item
\item
\end{itemize}

\includegraphics{https://static01.nyt.com/images/2020/07/16/t-magazine/16tmag-isabel-marant-slide-B9VG/16tmag-isabel-marant-slide-B9VG-articleLarge.jpg?quality=75\&auto=webp\&disable=upscale}

Sections

\protect\hyperlink{site-content}{Skip to
content}\protect\hyperlink{site-index}{Skip to site index}

Profile in Style

\hypertarget{the-designer-who-defined-modern-parisian-cool}{%
\section{The Designer Who Defined Modern Parisian
Cool}\label{the-designer-who-defined-modern-parisian-cool}}

Isabel Marant has always known exactly how she wants to dress. In the
decades since she founded her brand, people all over the world have
adopted her tastes as their own.

Credit...Dudi Hasson

Supported by

\protect\hyperlink{after-sponsor}{Continue reading the main story}

By \href{https://www.nytimes.com/by/lindsay-talbot}{Lindsay Talbot}

\begin{itemize}
\item
  July 16, 2020
\item
  \begin{itemize}
  \item
  \item
  \item
  \item
  \item
  \end{itemize}
\end{itemize}

``I was a strange and rebellious child,'' says the designer
\href{https://www.nytimes.com/slideshow/2019/03/02/t-magazine/48-hours-with-isabel-marant-in-paris.html}{Isabel
Marant}, who, after her French father and German mother separated when
she was 7, was largely raised --- in the Parisian suburb of
Neuilly-sur-Seine --- by her stepmother, who was from Martinique and
dressed in head-to-toe
\href{https://www.nytimes.com/topic/person/yves-saint-laurent}{Yves
Saint Laurent}. The glamour wasn't wasted on Marant, but she had her own
ideas. ``That era of '80s super-chic French fashion was really getting
on my nerves,'' she says. ``By 10, I knew exactly what I wanted to wear,
but it didn't exist in any shop.'' So she started making her own
clothes, reworking her dad's castoff dressing gowns, sweaters, vintage
military fatigues and silk slippers. At one point she thought she might
study economics or become a veterinarian, but her friends loved her
creations and kept requesting versions for themselves. After studying
fashion design at Paris's Studio Berçot, Marant launched Twen, a
knitwear and jersey line, with her mother in 1989. ``I was determined to
work for myself,'' she says. Her eponymous label debuted five years
later, with a collection of men's-wear-inspired overcoats and striped
suits that drew from Indonesian batiks and ikats. Her credo was simple:
She'd only design pieces she herself would wear. ``To this day,'' says
Marant, 53, ``nothing leaves the workshop without me trying it on.''

Her preference for eye-catching pieces that exude a sense of bohemian
nonchalance has also remained consistent, and those familiar with the
brand are able to spot a Marant schoolboy- style button-down, floral
georgette dress or pair of dip-dyed motorcycle jeans from a block away.
``I think of my collections as versatile satellites of each other that
are meant to transcend time,'' adds Marant. Today, she has 52 boutiques
in cities ranging from Copenhagen to Chengdu and, in addition to women's
wear, oversees accessories, a diffusion line called Étoile and
\href{https://www.nytimes.com/2017/09/29/fashion/isabel-marant-menswear-paris-fashion-week.html}{men's
wear}. Still, along with her husband, the handbag designer Jérôme
Dreyfuss, and their 17-year-old son, she manages to split her time
between an apartment in Paris's Belleville neighborhood and a small
cabin in Fontainebleau. ``Far from the madness of Paris, I can clear my
head,'' says the designer, ``which actually speeds up my ideas.''

*At top: ``*A recent portrait of me taken by Dudi Hasson in front of our
Paris atelier. I'm dressed in my typical uniform, which consists of a
lot of light gray and off-white, and I'm wearing my father's watch from
the '60s. I always have my hair up in a chignon. Sometimes I think I
should dye it, but then I say, `No, no, Isabel --- that's just the way
it is.'''

\includegraphics{https://static01.nyt.com/images/2020/07/16/t-magazine/16tmag-isabel-marant-slide-HYSS/16tmag-isabel-marant-slide-HYSS-articleLarge.jpg?quality=75\&auto=webp\&disable=upscale}

\emph{Left:} ``My mother, Christa Fiedler, with my little brother (left)
and me in Neuilly-sur-Seine. I must be 5 or 6 here. At the time, my mom
was modeling in fashion magazines; later she ran Elite, the modeling
agency. I think my dad hoped I'd be beautiful like her, but alas, I was
very tomboyish and always hiding myself behind a dark fringe of bangs.''

\emph{Center:} ``In France, it's quite typical to have a classic Breton
bowl with your name on it for breakfast and things. I carry this
`Isabel' one around with me and fill it with tobacco when I roll my
cigarettes. A few years ago, we produced a small run of them, sort of as
an inside company joke, and I sometimes give them to friends as gifts.''

\emph{Right:} ``\href{https://www.manoloballesteros.com/}{Manolo
Ballesteros}, who's now in his 50s, was a young artist when I first
discovered his pieces, at a gallery in Barcelona. I've always enjoyed
the work of Catalan artists from the 1960s, like Miró and Dalí, and I
love the way that Ballesteros nods to their creations with his own use
of geometric shapes. This work, `Untitled' (2019), is made of folded
painted paper and feels somewhere between a painting and a sculpture. I
don't own it, but I keep two others of his at my cabin.''

Image

Credit...From left: Ed Alcock for The New York Times; Matthieu
Salvaing/OTTO.

\emph{Left:} ``I can be extremely French --- I often eat steak and
French fries with red wine for lunch at Chez Georges, which is not very
far away from my office. It's a very old-school bistro. More Parisian,
you simply cannot do.''

\emph{Right:} ``My atelier in Paris is a two-story atrium just off the
Place des Victoires. The armchairs are a 1950s design by Pierre
Jeanneret from his Kangaroo series. I bought the carpet during a trip to
Morocco and found the table in Los Angeles. On the walls is framed
jewelry I've designed. The skylights bring in the best light, but,
depending on the season and time of day, it can also get wonderfully
moody.''

Image

Credit...From left: Juergen Teller for Isabel Marant; courtesy of Isabel
Marant.

*Left: ``*Our
\href{https://www.isabelmarant.com/us/isabel-marant/women/naoko}{Naoko
bag}, launching this summer, is made of thick red cow leather and has
two different handle lengths. I'm not the kind of person who changes
their bag every day --- I have one for spring and one for fall, so maybe
this will become my fall bag. The photo was shot in Tel Aviv by Juergen
Teller, whom I've been working with for five seasons. My German side
appreciates the straightforward roughness of his images. He doesn't
cheat.''

\emph{Right:} ``In 1987, when I was 20, I backpacked with three friends
through India, and took this photo of a streetscape in the pink city of
Jaipur. On the metal tray on the left is a small black-and-white
portrait of me, a Polaroid that was taken behind the curtain with an
old-fashioned camera. You could get these sorts of pictures of yourself
all over the city, and they make for sweet keepsakes.''

Image

Credit...From left:~Charles Allmon/Nat Geo Image Collection; courtesy of
Isabel Marant.

``Music is often a jumping-off point for me. When I was thinking about
my spring 2020 collection, I was listening to a lot of baile funk and
contemporary electronic Brazilian music, and it got me thinking about
Brazil's beaches, joyful colors and design. I admire the work of the
architect
\href{https://www.nytimes.com/2018/08/17/t-magazine/niemeyer-house-adriana-varejao.html}{Oscar
Niemeyer} and the landscape designer
\href{https://www.nytimes.com/2019/06/20/arts/design/roberto-burle-marx-botanical-garden.html}{Roberto
Burle Marx}, and this men's wear look (right), though it could be
unisex, was inspired by Marx's trippy, undulating Copacabana boardwalk
in Rio de Janeiro (left).''

Image

Credit...From left: Courtesy of Isabel Marant;~Oasis at Image Locations.

\emph{Left:} ``This was the invitation to my first-ever runway show. I
found a square of beautiful abandoned Parisian buildings on Rue St.
Sabin and got permission from the township to hold my show there. Two
days before, they called and said, `You can't do it.' Well, all the
invites had been sent out, so I pretended I'd call it off but did it
anyway --- illegally!''

\emph{Right:} ``I'm very keen on modernism, particularly when it comes
to architecture. Richard Neutra, Alvar Aalto and Frank Lloyd Wright are
all favorites, and I would totally live in this modernist-inspired 1993
Los Angeles house by
\href{https://www.architecturaldigest.com/gallery/inside-the-most-instagrammed-house-in-los-angeles}{Philip
Dixon}. I'm drawn to the rawness of the concrete, which is juxtaposed by
the cactuses and lush palm trees.''

Image

Credit...From left:~Photograph by Armin Linke/courtesy of Maurizio
Cattelan's Archive; courtesy of Isabel Marant.

\emph{Left:} ``This 1996 black-and-white photograph, `Untitled' by
\href{https://www.nytimes.com/2020/05/14/arts/maurizio-cattelan-bedtime-stories-virus.html}{Maurizio
Cattelan}, is one of my favorites. I love how the hands form a star and
are also peace signs. For me, it symbolizes what we can achieve with our
hands, and how we can be peaceful if we're connected.''

\emph{Right:} ``My family and I go to our place outside of Paris most
weekends when the weather is good --- in the picture, taken five years
ago, you can see my son, Tal. The cabin sits by a deep river and used to
be a fishing shack --- there's still no running water or electricity so
I do a lot of gardening and barbecues. Being there is like an escape to
a past era.''

Advertisement

\protect\hyperlink{after-bottom}{Continue reading the main story}

\hypertarget{site-index}{%
\subsection{Site Index}\label{site-index}}

\hypertarget{site-information-navigation}{%
\subsection{Site Information
Navigation}\label{site-information-navigation}}

\begin{itemize}
\tightlist
\item
  \href{https://help.nytimes.com/hc/en-us/articles/115014792127-Copyright-notice}{©~2020~The
  New York Times Company}
\end{itemize}

\begin{itemize}
\tightlist
\item
  \href{https://www.nytco.com/}{NYTCo}
\item
  \href{https://help.nytimes.com/hc/en-us/articles/115015385887-Contact-Us}{Contact
  Us}
\item
  \href{https://www.nytco.com/careers/}{Work with us}
\item
  \href{https://nytmediakit.com/}{Advertise}
\item
  \href{http://www.tbrandstudio.com/}{T Brand Studio}
\item
  \href{https://www.nytimes.com/privacy/cookie-policy\#how-do-i-manage-trackers}{Your
  Ad Choices}
\item
  \href{https://www.nytimes.com/privacy}{Privacy}
\item
  \href{https://help.nytimes.com/hc/en-us/articles/115014893428-Terms-of-service}{Terms
  of Service}
\item
  \href{https://help.nytimes.com/hc/en-us/articles/115014893968-Terms-of-sale}{Terms
  of Sale}
\item
  \href{https://spiderbites.nytimes.com}{Site Map}
\item
  \href{https://help.nytimes.com/hc/en-us}{Help}
\item
  \href{https://www.nytimes.com/subscription?campaignId=37WXW}{Subscriptions}
\end{itemize}
