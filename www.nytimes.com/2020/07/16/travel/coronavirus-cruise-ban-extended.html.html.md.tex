Sections

SEARCH

\protect\hyperlink{site-content}{Skip to
content}\protect\hyperlink{site-index}{Skip to site index}

\href{https://www.nytimes.com/section/travel}{Travel}

\href{https://myaccount.nytimes.com/auth/login?response_type=cookie\&client_id=vi}{}

\href{https://www.nytimes.com/section/todayspaper}{Today's Paper}

\href{/section/travel}{Travel}\textbar{}Extending Cruise Ban, C.D.C.
Slams Industry for Spreading Coronavirus

\url{https://nyti.ms/30jj815}

\begin{itemize}
\item
\item
\item
\item
\item
\end{itemize}

\href{https://www.nytimes.com/news-event/coronavirus?action=click\&pgtype=Article\&state=default\&region=TOP_BANNER\&context=storylines_menu}{The
Coronavirus Outbreak}

\begin{itemize}
\tightlist
\item
  live\href{https://www.nytimes.com/2020/08/01/world/coronavirus-covid-19.html?action=click\&pgtype=Article\&state=default\&region=TOP_BANNER\&context=storylines_menu}{Latest
  Updates}
\item
  \href{https://www.nytimes.com/interactive/2020/us/coronavirus-us-cases.html?action=click\&pgtype=Article\&state=default\&region=TOP_BANNER\&context=storylines_menu}{Maps
  and Cases}
\item
  \href{https://www.nytimes.com/interactive/2020/science/coronavirus-vaccine-tracker.html?action=click\&pgtype=Article\&state=default\&region=TOP_BANNER\&context=storylines_menu}{Vaccine
  Tracker}
\item
  \href{https://www.nytimes.com/interactive/2020/07/29/us/schools-reopening-coronavirus.html?action=click\&pgtype=Article\&state=default\&region=TOP_BANNER\&context=storylines_menu}{What
  School May Look Like}
\item
  \href{https://www.nytimes.com/live/2020/07/31/business/stock-market-today-coronavirus?action=click\&pgtype=Article\&state=default\&region=TOP_BANNER\&context=storylines_menu}{Economy}
\end{itemize}

Advertisement

\protect\hyperlink{after-top}{Continue reading the main story}

Supported by

\protect\hyperlink{after-sponsor}{Continue reading the main story}

\hypertarget{extending-cruise-ban-cdc-slams-industry-for-spreading-coronavirus}{%
\section{Extending Cruise Ban, C.D.C. Slams Industry for Spreading
Coronavirus}\label{extending-cruise-ban-cdc-slams-industry-for-spreading-coronavirus}}

In a scathing order extending the current ``no sail'' order on U.S.
cruise lines, the agency said it spent 38,000 hours managing the
outbreaks on ships.

\includegraphics{https://static01.nyt.com/images/2020/07/16/travel/16-cruise-ban/16-cruise-ban-articleLarge.jpg?quality=75\&auto=webp\&disable=upscale}

By \href{https://www.nytimes.com/by/frances-robles}{Frances Robles}

\begin{itemize}
\item
  July 16, 2020
\item
  \begin{itemize}
  \item
  \item
  \item
  \item
  \item
  \end{itemize}
\end{itemize}

As the coronavirus pandemic raged around the world, cruise ship
companies continued to allow their crews to attend social gatherings,
work out at gyms and share buffet-style meals, violating basic protocols
designed to stop the spread of the highly transmissible virus, the
Centers for Disease Control and Prevention said in a
\href{https://www.cdc.gov/quarantine/pdf/No-Sail-Order-Cruise-Ships-Second-Extension_07_16_2020-p.pdf}{scathing
20-page order}, released Thursday, that extended the suspension of
cruise operations until Sept. 30.

In a rebuke of the cruise ship companies, Robert R. Redfield, the
director of the C.D.C., blamed them for widespread transmission of the
virus. The C.D.C. said there were 99 outbreaks aboard 123 cruise ships
in United States waters alone, the agency said in the statement. From
March 1 until July 10, 80 percent of the ships in the C.D.C.'s
jurisdiction were affected by the coronavirus. The agency said there had
been nearly 3,000 suspected and confirmed cases and 34 deaths on ships
in U.S. waters.

As of July 3, nine ships still had ongoing or resolving outbreaks.

The C.D.C. spent at least 38,000 hours managing the crisis, the order
said. Public health authorities had to do contact tracing for some
11,000 passengers, more than the number of contacts identified from
airplane flights since the beginning of the pandemic, the C.D.C. said.

The cruise industry has struggled to manage the coronavirus pandemic
since the start, when the
\href{https://www.nytimes.com/2020/03/08/world/asia/coronavirus-cruise-ship.html}{Diamond
Princess}, part of the cruise giant Carnival Corporation, moored in the
Japanese harbor of Yokohama, Japan, amid an outbreak that eventually
infected 712 people and killed nine of them. Even as warnings were
issued about the dangers of cruise-ship travel, passengers kept boarding
and ships kept sailing.

Though more and more cruise passengers fell ill, companies
\href{https://www.nytimes.com/2020/03/19/travel/coronavirus-cruise-costa-luminosa.html}{continued
their voyages}, offering entertainment that included live music and pool
parties. The industry ultimately suspended operations in mid-March, but
as ships made their way to port, many passengers and crew were stranded
around the world, as countries refused the ships entry.

\hypertarget{latest-updates-global-coronavirus-outbreak}{%
\section{\texorpdfstring{\href{https://www.nytimes.com/2020/08/01/world/coronavirus-covid-19.html?action=click\&pgtype=Article\&state=default\&region=MAIN_CONTENT_1\&context=storylines_live_updates}{Latest
Updates: Global Coronavirus
Outbreak}}{Latest Updates: Global Coronavirus Outbreak}}\label{latest-updates-global-coronavirus-outbreak}}

Updated 2020-08-02T07:14:05.841Z

\begin{itemize}
\tightlist
\item
  \href{https://www.nytimes.com/2020/08/01/world/coronavirus-covid-19.html?action=click\&pgtype=Article\&state=default\&region=MAIN_CONTENT_1\&context=storylines_live_updates\#link-34047410}{The
  U.S. reels as July cases more than double the total of any other
  month.}
\item
  \href{https://www.nytimes.com/2020/08/01/world/coronavirus-covid-19.html?action=click\&pgtype=Article\&state=default\&region=MAIN_CONTENT_1\&context=storylines_live_updates\#link-780ec966}{Top
  U.S. officials work to break an impasse over the federal jobless
  benefit.}
\item
  \href{https://www.nytimes.com/2020/08/01/world/coronavirus-covid-19.html?action=click\&pgtype=Article\&state=default\&region=MAIN_CONTENT_1\&context=storylines_live_updates\#link-2bc8948}{Its
  outbreak untamed, Melbourne goes into even greater lockdown.}
\end{itemize}

\href{https://www.nytimes.com/2020/08/01/world/coronavirus-covid-19.html?action=click\&pgtype=Article\&state=default\&region=MAIN_CONTENT_1\&context=storylines_live_updates}{See
more updates}

More live coverage:
\href{https://www.nytimes.com/live/2020/07/31/business/stock-market-today-coronavirus?action=click\&pgtype=Article\&state=default\&region=MAIN_CONTENT_1\&context=storylines_live_updates}{Markets}

One ship arrived in Fort Lauderdale with four
\href{https://www.nytimes.com/2020/04/02/us/coronavirus-cruise-zaandam-rotterdam-florida.html}{dead
passengers} on board.

Many of those passengers who were allowed to disembark from contaminated
ships ``traversed international airports, boarded planes and returned to
their homes,'' the C.D.C. said, potentially spreading the virus further.

The cruise industry had already voluntarily suspended operations until
Sept. 15, and many companies withdrew their ships from United States
waters, removing them from the C.D.C.'s jurisdiction. But the order from
Dr. Redfield underscores the gap between the industry and the public
health agency. The companies cannot begin to sail again until they come
up with cohesive plans for prevention and mitigation of the illness.

Cruise ship companies submitted plans on how to safely evacuate crews,
but nearly all the companies failed to meet the basic requirements
necessary to stop the spread of the coronavirus, the C.D.C. said. Crew
members still bunked together and shared bathrooms. Even ships that
seemed to have gone a month without any coronavirus cases had crew
members who tested positive upon reaching shore, Dr. Redfield said.

One company, Norwegian Cruise Lines, said it felt it had exceeded
recommended C.D.C. guidance, because crew members were not just asked
but ``encouraged'' to wear face coverings, the order said. Disney
acknowledged that some of its asymptomatic-infected crew members had not
quarantined until after the results of shipwide testing came in.

The companies created a task force to come up with recommendations on
how to safely sail, but according to the C.D.C., the group will not
produce its findings for several months.

If unrestricted cruise-ship passenger operations were permitted to
resume, it would put ``substantial unnecessary risk'' on communities,
health care workers, port personnel and federal employees, the order
said, as well as placing passengers and crew members at increased risk.

The agency's previous no-sail order was set to expire July 24.

Disney said only one of its four ships, the Disney Wonder, had an
outbreak on board ---but only after passengers had disembarked. The
company tested every crew member on board and isolated non-essential
crew to their cabins for three weeks in April. Half the 174 crew who
tested positive had no symptoms, the company said.

\href{https://www.nytimes.com/news-event/coronavirus?action=click\&pgtype=Article\&state=default\&region=MAIN_CONTENT_3\&context=storylines_faq}{}

\hypertarget{the-coronavirus-outbreak-}{%
\subsubsection{The Coronavirus Outbreak
›}\label{the-coronavirus-outbreak-}}

\hypertarget{frequently-asked-questions}{%
\paragraph{Frequently Asked
Questions}\label{frequently-asked-questions}}

Updated July 27, 2020

\begin{itemize}
\item ~
  \hypertarget{should-i-refinance-my-mortgage}{%
  \paragraph{Should I refinance my
  mortgage?}\label{should-i-refinance-my-mortgage}}

  \begin{itemize}
  \tightlist
  \item
    \href{https://www.nytimes.com/article/coronavirus-money-unemployment.html?action=click\&pgtype=Article\&state=default\&region=MAIN_CONTENT_3\&context=storylines_faq}{It
    could be a good idea,} because mortgage rates have
    \href{https://www.nytimes.com/2020/07/16/business/mortgage-rates-below-3-percent.html?action=click\&pgtype=Article\&state=default\&region=MAIN_CONTENT_3\&context=storylines_faq}{never
    been lower.} Refinancing requests have pushed mortgage applications
    to some of the highest levels since 2008, so be prepared to get in
    line. But defaults are also up, so if you're thinking about buying a
    home, be aware that some lenders have tightened their standards.
  \end{itemize}
\item ~
  \hypertarget{what-is-school-going-to-look-like-in-september}{%
  \paragraph{What is school going to look like in
  September?}\label{what-is-school-going-to-look-like-in-september}}

  \begin{itemize}
  \tightlist
  \item
    It is unlikely that many schools will return to a normal schedule
    this fall, requiring the grind of
    \href{https://www.nytimes.com/2020/06/05/us/coronavirus-education-lost-learning.html?action=click\&pgtype=Article\&state=default\&region=MAIN_CONTENT_3\&context=storylines_faq}{online
    learning},
    \href{https://www.nytimes.com/2020/05/29/us/coronavirus-child-care-centers.html?action=click\&pgtype=Article\&state=default\&region=MAIN_CONTENT_3\&context=storylines_faq}{makeshift
    child care} and
    \href{https://www.nytimes.com/2020/06/03/business/economy/coronavirus-working-women.html?action=click\&pgtype=Article\&state=default\&region=MAIN_CONTENT_3\&context=storylines_faq}{stunted
    workdays} to continue. California's two largest public school
    districts --- Los Angeles and San Diego --- said on July 13, that
    \href{https://www.nytimes.com/2020/07/13/us/lausd-san-diego-school-reopening.html?action=click\&pgtype=Article\&state=default\&region=MAIN_CONTENT_3\&context=storylines_faq}{instruction
    will be remote-only in the fall}, citing concerns that surging
    coronavirus infections in their areas pose too dire a risk for
    students and teachers. Together, the two districts enroll some
    825,000 students. They are the largest in the country so far to
    abandon plans for even a partial physical return to classrooms when
    they reopen in August. For other districts, the solution won't be an
    all-or-nothing approach.
    \href{https://bioethics.jhu.edu/research-and-outreach/projects/eschool-initiative/school-policy-tracker/}{Many
    systems}, including the nation's largest, New York City, are
    devising
    \href{https://www.nytimes.com/2020/06/26/us/coronavirus-schools-reopen-fall.html?action=click\&pgtype=Article\&state=default\&region=MAIN_CONTENT_3\&context=storylines_faq}{hybrid
    plans} that involve spending some days in classrooms and other days
    online. There's no national policy on this yet, so check with your
    municipal school system regularly to see what is happening in your
    community.
  \end{itemize}
\item ~
  \hypertarget{is-the-coronavirus-airborne}{%
  \paragraph{Is the coronavirus
  airborne?}\label{is-the-coronavirus-airborne}}

  \begin{itemize}
  \tightlist
  \item
    The coronavirus
    \href{https://www.nytimes.com/2020/07/04/health/239-experts-with-one-big-claim-the-coronavirus-is-airborne.html?action=click\&pgtype=Article\&state=default\&region=MAIN_CONTENT_3\&context=storylines_faq}{can
    stay aloft for hours in tiny droplets in stagnant air}, infecting
    people as they inhale, mounting scientific evidence suggests. This
    risk is highest in crowded indoor spaces with poor ventilation, and
    may help explain super-spreading events reported in meatpacking
    plants, churches and restaurants.
    \href{https://www.nytimes.com/2020/07/06/health/coronavirus-airborne-aerosols.html?action=click\&pgtype=Article\&state=default\&region=MAIN_CONTENT_3\&context=storylines_faq}{It's
    unclear how often the virus is spread} via these tiny droplets, or
    aerosols, compared with larger droplets that are expelled when a
    sick person coughs or sneezes, or transmitted through contact with
    contaminated surfaces, said Linsey Marr, an aerosol expert at
    Virginia Tech. Aerosols are released even when a person without
    symptoms exhales, talks or sings, according to Dr. Marr and more
    than 200 other experts, who
    \href{https://academic.oup.com/cid/article/doi/10.1093/cid/ciaa939/5867798}{have
    outlined the evidence in an open letter to the World Health
    Organization}.
  \end{itemize}
\item ~
  \hypertarget{what-are-the-symptoms-of-coronavirus}{%
  \paragraph{What are the symptoms of
  coronavirus?}\label{what-are-the-symptoms-of-coronavirus}}

  \begin{itemize}
  \tightlist
  \item
    Common symptoms
    \href{https://www.nytimes.com/article/symptoms-coronavirus.html?action=click\&pgtype=Article\&state=default\&region=MAIN_CONTENT_3\&context=storylines_faq}{include
    fever, a dry cough, fatigue and difficulty breathing or shortness of
    breath.} Some of these symptoms overlap with those of the flu,
    making detection difficult, but runny noses and stuffy sinuses are
    less common.
    \href{https://www.nytimes.com/2020/04/27/health/coronavirus-symptoms-cdc.html?action=click\&pgtype=Article\&state=default\&region=MAIN_CONTENT_3\&context=storylines_faq}{The
    C.D.C. has also} added chills, muscle pain, sore throat, headache
    and a new loss of the sense of taste or smell as symptoms to look
    out for. Most people fall ill five to seven days after exposure, but
    symptoms may appear in as few as two days or as many as 14 days.
  \end{itemize}
\item ~
  \hypertarget{does-asymptomatic-transmission-of-covid-19-happen}{%
  \paragraph{Does asymptomatic transmission of Covid-19
  happen?}\label{does-asymptomatic-transmission-of-covid-19-happen}}

  \begin{itemize}
  \tightlist
  \item
    So far, the evidence seems to show it does. A widely cited
    \href{https://www.nature.com/articles/s41591-020-0869-5}{paper}
    published in April suggests that people are most infectious about
    two days before the onset of coronavirus symptoms and estimated that
    44 percent of new infections were a result of transmission from
    people who were not yet showing symptoms. Recently, a top expert at
    the World Health Organization stated that transmission of the
    coronavirus by people who did not have symptoms was ``very rare,''
    \href{https://www.nytimes.com/2020/06/09/world/coronavirus-updates.html?action=click\&pgtype=Article\&state=default\&region=MAIN_CONTENT_3\&context=storylines_faq\#link-1f302e21}{but
    she later walked back that statement.}
  \end{itemize}
\end{itemize}

The ship has not had a positive case since May 8, Disney said.

Royal Caribbean and Norwegian Cruise Line, whose failures were
specifically cited in the C.D.C. document, released statements in
response to the order that did not specifically address the allegations.

Norwegian said it canceled trips through September, as well as cruises
embarking from or calling on ports in Canada in October. ``We continue
to partner with the C.D.C. and other authorities to mitigate the impact
of COVID-19 by prioritizing the health and safety of our passengers and
crew,'' the company said.

Royal Caribbean said it would suspend operations through September to
comply with the order. ``The health and safety of our guests, crew and
the communities we visit is our top priority,'' the company said.

Carnival Cruises said that it had already extended its suspension
through September. But the company plans three voyages in Germany next
month through a European line, and Italy trips are also expected soon, a
spokesman said.

Bari Golin-Blaugrund, a spokeswoman for the Cruise Line Industry
Association, a trade organization that represents most of the major
cruise companies, released a statement that did not address the C.D.C.
criticisms.

``As we continue to work towards the development of enhanced protocols
to support the safe resumption of cruise operations around the world, we
look forward to timely and productive dialogue with the C.D.C. to
determine measures that will be appropriate for ocean-going cruise
operations to resume in the United States when the time is right,'' she
said.

\emph{\textbf{Follow New York Times Travel}}
\emph{on}\href{https://www.instagram.com/nytimestravel/}{\emph{Instagram}}\emph{,}\href{https://twitter.com/nytimestravel}{\emph{Twitter}}
\emph{and}\href{https://www.facebook.com/nytimestravel/}{\emph{Facebook}}\emph{.
And}\href{https://www.nytimes.com/newsletters/traveldispatch}{\emph{sign
up for our weekly Travel Dispatch newsletter}} \emph{to receive expert
tips on traveling smarter and inspiration for your next vacation.}

Advertisement

\protect\hyperlink{after-bottom}{Continue reading the main story}

\hypertarget{site-index}{%
\subsection{Site Index}\label{site-index}}

\hypertarget{site-information-navigation}{%
\subsection{Site Information
Navigation}\label{site-information-navigation}}

\begin{itemize}
\tightlist
\item
  \href{https://help.nytimes.com/hc/en-us/articles/115014792127-Copyright-notice}{©~2020~The
  New York Times Company}
\end{itemize}

\begin{itemize}
\tightlist
\item
  \href{https://www.nytco.com/}{NYTCo}
\item
  \href{https://help.nytimes.com/hc/en-us/articles/115015385887-Contact-Us}{Contact
  Us}
\item
  \href{https://www.nytco.com/careers/}{Work with us}
\item
  \href{https://nytmediakit.com/}{Advertise}
\item
  \href{http://www.tbrandstudio.com/}{T Brand Studio}
\item
  \href{https://www.nytimes.com/privacy/cookie-policy\#how-do-i-manage-trackers}{Your
  Ad Choices}
\item
  \href{https://www.nytimes.com/privacy}{Privacy}
\item
  \href{https://help.nytimes.com/hc/en-us/articles/115014893428-Terms-of-service}{Terms
  of Service}
\item
  \href{https://help.nytimes.com/hc/en-us/articles/115014893968-Terms-of-sale}{Terms
  of Sale}
\item
  \href{https://spiderbites.nytimes.com}{Site Map}
\item
  \href{https://help.nytimes.com/hc/en-us}{Help}
\item
  \href{https://www.nytimes.com/subscription?campaignId=37WXW}{Subscriptions}
\end{itemize}
