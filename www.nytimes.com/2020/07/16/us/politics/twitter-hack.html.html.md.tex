Sections

SEARCH

\protect\hyperlink{site-content}{Skip to
content}\protect\hyperlink{site-index}{Skip to site index}

\href{https://www.nytimes.com/section/politics}{Politics}

\href{https://myaccount.nytimes.com/auth/login?response_type=cookie\&client_id=vi}{}

\href{https://www.nytimes.com/section/todayspaper}{Today's Paper}

\href{/section/politics}{Politics}\textbar{}What the Twitter Hack
Revealed: An Election System Teeming With Risks

\href{https://nyti.ms/3fBL1rJ}{https://nyti.ms/3fBL1rJ}

\begin{itemize}
\item
\item
\item
\item
\item
\end{itemize}

\begin{itemize}
\item
  \href{https://www.nytimes.com/2020/08/07/us/elections/biden-vs-trump.html?action=click\&pgtype=Article\&state=default\&region=TOP_BANNER\&context=storylines_menu}{Election
  Updates}
\item
  \href{https://www.nytimes.com/interactive/2020/08/08/us/elections/results-hawaii-primary-elections.html?action=click\&pgtype=Article\&state=default\&region=TOP_BANNER\&context=storylines_menu}{Hawaii
  Results}
\item
  \href{https://www.nytimes.com/article/biden-vice-president-2020.html?action=click\&pgtype=Article\&state=default\&region=TOP_BANNER\&context=storylines_menu}{Biden's
  V.P. Search}
\item
  \href{https://www.nytimes.com/interactive/2019/us/politics/2020-presidential-candidates.html?action=click\&pgtype=Article\&state=default\&region=TOP_BANNER\&context=storylines_menu}{The
  Candidates}
\item
  \href{https://www.nytimes.com/newsletters/politics?action=click\&pgtype=Article\&state=default\&region=TOP_BANNER\&context=storylines_menu}{Politics
  Newsletter}
\end{itemize}

Advertisement

\protect\hyperlink{after-top}{Continue reading the main story}

Supported by

\protect\hyperlink{after-sponsor}{Continue reading the main story}

News Analysis

\hypertarget{what-the-twitter-hack-revealed-an-election-system-teeming-with-risks}{%
\section{What the Twitter Hack Revealed: An Election System Teeming With
Risks}\label{what-the-twitter-hack-revealed-an-election-system-teeming-with-risks}}

The breach that targeted Joe Biden, Barack Obama and others served as a
warning: Had it happened on Nov. 3, hoping to upend the election, the
political fallout could have been quite different.

\includegraphics{https://static01.nyt.com/images/2020/07/16/us/politics/16elections-hack1/merlin_174559221_c63d8626-3f52-4183-887d-f3e4d1fd565a-articleLarge.jpg?quality=75\&auto=webp\&disable=upscale}

\href{https://www.nytimes.com/by/david-e-sanger}{\includegraphics{https://static01.nyt.com/images/2018/10/03/multimedia/author-david-e-sanger/author-david-e-sanger-thumbLarge.png}}\href{https://www.nytimes.com/by/nicole-perlroth}{\includegraphics{https://static01.nyt.com/images/2018/02/20/multimedia/author-nicole-perlroth/author-nicole-perlroth-thumbLarge.jpg}}\href{https://www.nytimes.com/by/nick-corasaniti}{\includegraphics{https://static01.nyt.com/images/2018/06/13/multimedia/author-nick-corasaniti/author-nick-corasaniti-thumbLarge-v2.png}}

By \href{https://www.nytimes.com/by/david-e-sanger}{David E. Sanger},
\href{https://www.nytimes.com/by/nicole-perlroth}{Nicole Perlroth} and
\href{https://www.nytimes.com/by/nick-corasaniti}{Nick Corasaniti}

\begin{itemize}
\item
  July 16, 2020
\item
  \begin{itemize}
  \item
  \item
  \item
  \item
  \item
  \end{itemize}
\end{itemize}

Over the past year government officials have raced to help states
replace voting machines that leave no paper trail, and to harden
vulnerable online voter registration systems that many fear Russia, or
others, could hijack to trigger chaos on Election Day.

But this week, the country got a startling vision of other perils in
political disinformation --- and how many other ways there may be to
manipulate turnout, if not votes.

The breach by a
\href{https://www.nytimes.com/2020/07/17/technology/twitter-hackers-interview.html}{hacker
or hackers} who
\href{https://www.nytimes.com/2020/07/15/technology/twitter-hack-bill-gates-elon-musk.html}{bored
into the command center of Twitter} on Wednesday --- seizing control of
\href{https://www.nytimes.com/interactive/2020/us/elections/joe-biden.html}{Joseph
R. Biden Jr.}'s and Barack Obama's blue-checked accounts, among many
others --- served as a warning that some of the most critical
infrastructure that could influence the election is not in the hands of
government experts, and is far less protected than anyone assumed even a
day ago.

The hackers probably did the nation a favor. With a crude scheme to
deceive users into thinking that Mr. Biden and Mr. Obama were asking
them for donations in Bitcoin --- which sent more than \$120,000 flowing
into their cryptocurrency wallets --- they revealed how simple it may be
to imitate the powerful and the trusted.

Had saboteurs infiltrated Twitter on Nov. 3 instead of in the middle of
July, with the goal of upending the election, the political fallout
could have been quite different. False warnings of a coronavirus
outbreak in key precincts in Wisconsin or Pennsylvania could have untold
impact on a close vote in a battleground state. Deceptive tweets from
political party accounts saying polling places were closed could sow
confusion.

Or imagine a fake declaration, under Mr. Biden's account, that he was
dropping out of the race --- a nightmare scenario for Democrats that
some federal officials said they were talking about hypothetically among
themselves on Wednesday night as the scope of Twitter's failure became
clear.

Similar war gaming about social media and election interference has
played out in classified simulations conducted by the Department of
Homeland Security, which is responsible for securing the 2020 election,
and at Fort Meade, Md., home of the National Security Agency and United
States Cyber Command. The results have never fully been made public.

But the nation is now getting a very public look at the impact of
disinformation when trusted accounts of politicians and prominent
Americans are hacked --- with voters confused and more wary than ever of
who is telling the truth, blue check or no blue check. The disruption
revealed that the social media platform favored by the president --- one
that the
\href{https://www.nytimes.com/2019/07/09/us/politics/trump-twitter-first-amendment.html}{federal
courts concluded a year ago is a conduit for official messages about
national policy} --- was as vulnerable, in its own way, as the aging
registration databases that Russian intelligence invaded four years ago
in Arizona, Illinois and other states.

Investigators are still trying to determine exactly how the hackers got
inside Twitter's systems and took such command of the platform that,
when Twitter employees took the Bitcoin-seeking messages down, the
disinformation popped right back up. Many of the details remain unclear:
Investigators are still trying to determine if the hackers tricked a
Twitter employee into handing over login information. Twitter suggested
on Wednesday that the hackers had used ``social engineering,'' a
strategy to gain passwords or other personal information by posing as a
trusted person like a company representative.

But another line of inquiry includes whether a Twitter employee was
bribed for his or her credentials, something one person who claimed
responsibility for the hacking told the technology site
\href{https://slack-redir.net/link?url=https\%3A\%2F\%2Fwww.vice.com\%2Fen_us\%2Farticle\%2Fjgxd3d\%2Ftwitter-insider-access-panel-account-hacks-biden-uber-bezos}{Motherboard}.

In the end, it may matter less how they did it than that they succeeded.
As Christopher Krebs, who leads the Cybersecurity and Infrastructure
Agency at the Department of Homeland Security, has often noted,
influencing an election requires either hacking into voter systems or
hacking into voters' brains. The Twitter breach demonstrated yet another
way to accomplish the latter, what Mr. Krebs called on Thursday ``the
more likely, less costly way'' to mount an attack.

Until Wednesday's attack, most of the officials and analysts at the
array of federal agencies confronting election threats were focused
heavily on voting systems --- because that is the area over which
governments have most control. Their particular worry was a convergence
of cybercriminals and national intelligence agencies, particularly in
Russia, deploying ransomware against underprotected American cities and
towns.

A leaked F.B.I. warning from May 1 said ransomware hackers could seek to
lock up registration databases, a move that would disrupt both in-person
voting and the mailing and processing of mail-in ballots. The F.B.I.
warning suggested that ransomware attacks ``will likely threaten the
availability of data on interconnected election servers, even if that is
not the actors' intention.''

The bureau had reason to worry:
\href{https://www.nytimes.com/2018/03/27/us/cyberattack-atlanta-ransomware.html}{Atlanta},
\href{https://www.nytimes.com/2019/05/22/us/baltimore-ransomware.html}{Baltimore}
and towns across Florida and Texas have been victims of attacks that
locked up their data, making it impossible to pay taxes, get potholes
fixed or obtain a building permit. The advisory noted that
cybercriminals broke into the American companies that provide internet
services to Louisiana election officials late last year, then carefully
timed their ransomware attack to a week before an election.

It was a wake-up call, F.B.I. analysts said, to what American states and
counties might expect in 2020.

But the Twitter hacking suggested yet another vector for attack. And it
was a reminder of three particular challenges facing those trying to
secure the election. The first is assessing possible vulnerabilities so
the country is not playing catch-up once again, long after Election Day,
to outside interference with the election system or on social media.
(The extent of Russia's manipulation of Facebook posts in 2016, for
example, became clear only after
\href{https://www.nytimes.com/interactive/2020/us/elections/donald-trump.html}{President
Trump} had been elected.)

The second is how well the country can lock down these systems in the
100-plus days left before the election, beyond the obvious ``critical
infrastructure'' that will enable the Nov. 3 vote. And the third is
whether it is possible to build some national resilience to respond
quickly, as Twitter tried, if something goes wrong.

\includegraphics{https://static01.nyt.com/images/2020/07/16/us/politics/16elections-hack2/merlin_173164302_a4091605-fbbd-4fdf-b05f-a24b83dfb707-articleLarge.jpg?quality=75\&auto=webp\&disable=upscale}

Since 2016, thousands of pages of federal investigative reports have
been published on what went wrong in the presidential election that
year, and a congressional Cyberspace Solarium Commission has produced
long lists of recommendations of how private enterprise and the
government can work together.

But then there are days like Wednesday, when it seems as if all the
studies were insufficient.

``We have seen disconcerting incidents of account takeovers before,''
said John Hultquist, the senior director of intelligence analysis at
FireEye, one of the leading cybersecurity firms, ``but we are very
concerned about the possibility of real foreign actors hijacking
legitimate sources of information --- key media accounts for instance
--- and using that to push out disinformation'' close to Election Day.

``By the time we unwind everything to figure out what happened, it could
be too late,'' he added. ``That's a very real scenario.''

Or, as Laura Rosenberger, a former State Department official who now
directs the Alliance for Securing Democracy project at the German
Marshall Fund, noted, ``What hasn't changed is our failure to think
ahead. Our adversaries have an ability to turn this infrastructure,
which we have created, against us, and we need to be better at
anticipating the threat vectors.''

Similar thoughts haunt state election officials on both sides of the
aisle, who say they are alarmed about what could happen if the
mega-microphones of accounts belonging to the likes of Mr. Biden or Mr.
Obama broadcast a bit of electoral disinformation.

Alex Padilla, the secretary of state in California, home to Twitter
headquarters, said that while state officials had run simulations of a
social media disinformation campaign disrupting an election day, they
hadn't imagined a situation in which Twitter itself was hacked. Still,
he said, threats posed by disinformation motivated him to set up
\href{https://www.sos.ca.gov/elections/vote-sure/}{VoteSure}, a
statewide voting information effort sparked by the special counsel's
investigation of Russian interference in the 2016 election.

``I wouldn't say it was a new concern, but I would say it's a big
reminder given what we've all been through over the last four years,''
Mr. Padilla said.

In Ohio, Frank LaRose, the secretary of state, has been conducting
seminars to inform local officials about disinformation tactics and how
to respond, and directing much of Ohio's federal election funds to
shoring up election security. But the attack on Twitter opened a new
front, he said.

``From my time in the Army, I learned that the enemy is always going to
be innovating to try to find our vulnerabilities,'' Mr. LaRose said in a
statement. ``We're doing everything we can to stay ahead of the curve,
including going straight into targeted communities and arming them with
the tools they need to fight back against disinformation.''

Of course, no one should be shocked at high-profile account takeovers:
The account of Jack Dorsey, Twitter's chief executive, was
\href{https://www.nytimes.com/2019/08/30/technology/jack-dorsey-twitter-account-hacked.html}{compromised
last year}. Last year, two Twitter employees were accused of abusing
their access to aid Saudi Arabia's efforts to spy on dissidents abroad.

And as far back as 2013, the Syrian Electronic Army hacked The
Associated Press's Twitter account, issuing false warnings that an
explosion at the White House had injured Mr. Obama. By the time the
tweet could be corrected, and the hackers exposed, the stock market had
plunged.

Seven years later, fears are heightened by the uncertainty over how to
deal with life in a so-called dirty network, where data and information
are coursing through Americans' phones on apps of questionable security
--- Twitter is now in that category --- or under foreign control.

That is why companies ranging from PayPal and Wells Fargo, and political
organizations like the Democratic National Committee, have told
employees to delete the Chinese social media app TikTok from their
corporate devices. On Wednesday, as the Twitter drama was unfolding, Mr.
Trump's chief of staff, Mark Meadows, said the government was
considering banning TikTok entirely.

``There are a number of administration officials who are looking at the
national security risk as it relates to TikTok, WeChat and other apps
that have the potential for national security exposure,'' Mr. Meadows
said on Air Force One, ``specifically as it relates to the gathering of
information on American citizens by a foreign adversary.''

But Twitter is an American company --- no one is going to ban it --- and
it's the way that Mr. Meadows's boss communicates with his constituents
and, often, with his own government. The question is whether its
security flaws can be fixed in the next 16 weeks.

\hypertarget{our-2020-election-guide}{%
\section{Our 2020 Election Guide}\label{our-2020-election-guide}}

Updated Aug. 8, 2020

\begin{itemize}
\item
  \begin{center}\rule{0.5\linewidth}{\linethickness}\end{center}

  \hypertarget{the-latest}{%
  \subsection{The Latest}\label{the-latest}}

  \begin{itemize}
  \tightlist
  \item
    With 160 lawsuits filed over voting rules and President Trump's
    baseless claims of fraud, Election Day in America
    \href{https://www.nytimes.com/2020/08/08/us/politics/voting-nov-3-election.html?action=click\&pgtype=Article\&state=default\&region=BELOW_MAIN_CONTENT\&context=storylines_guide}{could
    become Election Month}.
  \end{itemize}
\item
  \begin{center}\rule{0.5\linewidth}{\linethickness}\end{center}

  \hypertarget{bidens-vp-search}{%
  \subsection{Biden's V.P. Search}\label{bidens-vp-search}}

  \begin{itemize}
  \tightlist
  \item
    \href{https://www.nytimes.com/article/biden-vice-president-2020.html?action=click\&pgtype=Article\&state=default\&region=BELOW_MAIN_CONTENT\&context=storylines_guide}{Here
    are 13 women} who have been under consideration to be Joe Biden's
    running mate, and why each might be chosen --- and might not be.
  \end{itemize}
\item
  \begin{center}\rule{0.5\linewidth}{\linethickness}\end{center}

  \hypertarget{keep-up-with-our-coverage}{%
  \subsection{Keep Up With Our
  Coverage}\label{keep-up-with-our-coverage}}

  \begin{itemize}
  \tightlist
  \item
    Get an
    \href{https://www.nytimes.com/newsletters/politics?action=click\&pgtype=Article\&state=default\&region=BELOW_MAIN_CONTENT\&context=storylines_guide}{email}
    recapping the day's news
  \end{itemize}

  \begin{itemize}
  \tightlist
  \item
    Download our mobile app on
    \href{https://apps.apple.com/us/app/nytimes/id284862083?ls=1\&mat_click_id=5c79ae7455014fd1bd66b5610c05b8f2-20191112-16948\&referrer=mat_click_id\%3D5c79ae7455014fd1bd66b5610c05b8f2-20191112-16948\%26link_click_id\%3D722930677036718082}{iOS}
    and
    \href{http://a.localytics.com/android?id=com.nytimes.android\&referrer=utm_source\%3Dother_nyt_mobile_web\%26utm_medium\%3DWeb\%2520page\%26utm_term\%3DGeneral\%2520Mobile\%2520Page\%26utm_campaign\%3DNYT\%2520Mobile\%2520General\%2520Page}{Android}
    and turn on Breaking News and Politics alerts
  \end{itemize}
\end{itemize}

Advertisement

\protect\hyperlink{after-bottom}{Continue reading the main story}

\hypertarget{site-index}{%
\subsection{Site Index}\label{site-index}}

\hypertarget{site-information-navigation}{%
\subsection{Site Information
Navigation}\label{site-information-navigation}}

\begin{itemize}
\tightlist
\item
  \href{https://help.nytimes.com/hc/en-us/articles/115014792127-Copyright-notice}{©~2020~The
  New York Times Company}
\end{itemize}

\begin{itemize}
\tightlist
\item
  \href{https://www.nytco.com/}{NYTCo}
\item
  \href{https://help.nytimes.com/hc/en-us/articles/115015385887-Contact-Us}{Contact
  Us}
\item
  \href{https://www.nytco.com/careers/}{Work with us}
\item
  \href{https://nytmediakit.com/}{Advertise}
\item
  \href{http://www.tbrandstudio.com/}{T Brand Studio}
\item
  \href{https://www.nytimes.com/privacy/cookie-policy\#how-do-i-manage-trackers}{Your
  Ad Choices}
\item
  \href{https://www.nytimes.com/privacy}{Privacy}
\item
  \href{https://help.nytimes.com/hc/en-us/articles/115014893428-Terms-of-service}{Terms
  of Service}
\item
  \href{https://help.nytimes.com/hc/en-us/articles/115014893968-Terms-of-sale}{Terms
  of Sale}
\item
  \href{https://spiderbites.nytimes.com}{Site Map}
\item
  \href{https://help.nytimes.com/hc/en-us}{Help}
\item
  \href{https://www.nytimes.com/subscription?campaignId=37WXW}{Subscriptions}
\end{itemize}
