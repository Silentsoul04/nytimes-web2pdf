Sections

SEARCH

\protect\hyperlink{site-content}{Skip to
content}\protect\hyperlink{site-index}{Skip to site index}

\href{https://www.nytimes.com/section/politics}{Politics}

\href{https://myaccount.nytimes.com/auth/login?response_type=cookie\&client_id=vi}{}

\href{https://www.nytimes.com/section/todayspaper}{Today's Paper}

\href{/section/politics}{Politics}\textbar{}Government Executes Second
Federal Death Row Prisoner in a Week

\url{https://nyti.ms/2ZzBVGj}

\begin{itemize}
\item
\item
\item
\item
\item
\end{itemize}

\href{https://www.nytimes.com/news-event/coronavirus?action=click\&pgtype=Article\&state=default\&region=TOP_BANNER\&context=storylines_menu}{The
Coronavirus Outbreak}

\begin{itemize}
\tightlist
\item
  live\href{https://www.nytimes.com/2020/08/01/world/coronavirus-covid-19.html?action=click\&pgtype=Article\&state=default\&region=TOP_BANNER\&context=storylines_menu}{Latest
  Updates}
\item
  \href{https://www.nytimes.com/interactive/2020/us/coronavirus-us-cases.html?action=click\&pgtype=Article\&state=default\&region=TOP_BANNER\&context=storylines_menu}{Maps
  and Cases}
\item
  \href{https://www.nytimes.com/interactive/2020/science/coronavirus-vaccine-tracker.html?action=click\&pgtype=Article\&state=default\&region=TOP_BANNER\&context=storylines_menu}{Vaccine
  Tracker}
\item
  \href{https://www.nytimes.com/interactive/2020/07/29/us/schools-reopening-coronavirus.html?action=click\&pgtype=Article\&state=default\&region=TOP_BANNER\&context=storylines_menu}{What
  School May Look Like}
\item
  \href{https://www.nytimes.com/live/2020/07/31/business/stock-market-today-coronavirus?action=click\&pgtype=Article\&state=default\&region=TOP_BANNER\&context=storylines_menu}{Economy}
\end{itemize}

Advertisement

\protect\hyperlink{after-top}{Continue reading the main story}

Supported by

\protect\hyperlink{after-sponsor}{Continue reading the main story}

\hypertarget{government-executes-second-federal-death-row-prisoner-in-a-week}{%
\section{Government Executes Second Federal Death Row Prisoner in a
Week}\label{government-executes-second-federal-death-row-prisoner-in-a-week}}

Wesley Ira Purkey was killed by lethal injection at the penitentiary in
Terre Haute, Ind., for killing a teenage girl more than two decades ago.

\includegraphics{https://static01.nyt.com/images/2020/07/16/us/politics/16dc-execute/merlin_174569643_d1a604e3-4a0b-4a79-8842-e2b21a495b5f-articleLarge.jpg?quality=75\&auto=webp\&disable=upscale}

By Hailey Fuchs

\begin{itemize}
\item
  July 16, 2020
\item
  \begin{itemize}
  \item
  \item
  \item
  \item
  \item
  \end{itemize}
\end{itemize}

WASHINGTON --- The Justice Department executed a 68-year-old man on
Thursday for the gruesome murder in 1998 of a teenage girl, the second
time this week the federal government has carried out capital punishment
after a 17-year hiatus.

Wesley Ira Purkey was put to death at the federal penitentiary in Terre
Haute, Ind., by lethal injection after the Supreme Court rejected the
last of a number of legal challenges including assertions that he was
not mentally competent.

He was pronounced dead at 8:19 a.m.

In the moments before his death, Mr. Purkey's arms were strapped to a
gurney, and a light blue blanket covered his body, according to a
journalist who witnessed the execution. His spiritual adviser was nearby
in a face mask and surgical mask, praying with gloved palms pressed
together.

``I deeply regret the pain and suffering I caused to Jennifer's
family,'' Mr. Purkey said before the execution, referring to his victim,
Jennifer Long, 16, whom he raped, murdered and dismembered in Kansas
City, Mo.

``I am deeply sorry,'' Mr. Purkey said after a Bureau of Prisons
official removed a mask from his face so he could speak. ``I deeply
regret the pain I caused to my daughter, who I love so very much. This
sanitized murder really does not serve no purpose whatsoever. Thank
you.''

He took several deep breaths and blinked repeatedly before he lay his
head back down on the gurney. With his mouth ajar and eyes closed, Mr.
Purkey lay motionless for several minutes before prison officials
reported a time of death.

The Justice Department announced its intention to bring back federal
capital punishment last summer. After a series of court fights over its
plan to use a single drug for lethal injection, Attorney General William
P. Barr scheduled four executions for this summer, the first by the
federal government since 2003.

\hypertarget{latest-updates-global-coronavirus-outbreak}{%
\section{\texorpdfstring{\href{https://www.nytimes.com/2020/08/01/world/coronavirus-covid-19.html?action=click\&pgtype=Article\&state=default\&region=MAIN_CONTENT_1\&context=storylines_live_updates}{Latest
Updates: Global Coronavirus
Outbreak}}{Latest Updates: Global Coronavirus Outbreak}}\label{latest-updates-global-coronavirus-outbreak}}

Updated 2020-08-02T10:04:29.623Z

\begin{itemize}
\tightlist
\item
  \href{https://www.nytimes.com/2020/08/01/world/coronavirus-covid-19.html?action=click\&pgtype=Article\&state=default\&region=MAIN_CONTENT_1\&context=storylines_live_updates\#link-34047410}{The
  U.S. reels as July cases more than double the total of any other
  month.}
\item
  \href{https://www.nytimes.com/2020/08/01/world/coronavirus-covid-19.html?action=click\&pgtype=Article\&state=default\&region=MAIN_CONTENT_1\&context=storylines_live_updates\#link-780ec966}{Top
  U.S. officials work to break an impasse over the federal jobless
  benefit.}
\item
  \href{https://www.nytimes.com/2020/08/01/world/coronavirus-covid-19.html?action=click\&pgtype=Article\&state=default\&region=MAIN_CONTENT_1\&context=storylines_live_updates\#link-2bc8948}{Its
  outbreak untamed, Melbourne goes into even greater lockdown.}
\end{itemize}

\href{https://www.nytimes.com/2020/08/01/world/coronavirus-covid-19.html?action=click\&pgtype=Article\&state=default\&region=MAIN_CONTENT_1\&context=storylines_live_updates}{See
more updates}

More live coverage:
\href{https://www.nytimes.com/live/2020/07/31/business/stock-market-today-coronavirus?action=click\&pgtype=Article\&state=default\&region=MAIN_CONTENT_1\&context=storylines_live_updates}{Markets}

Mr. Purkey was the second on the list.

Jennifer's father and stepmother told reporters that Mr. Purkey's
execution was long overdue.

``He needed to take his last breath, he took my daughter's last
breath,'' William Long, Jennifer's father, said.

The Bureau of Prisons
\href{https://www.nytimes.com/2020/07/14/us/politics/daniel-lewis-lee-execution-crime.html}{put
to death Daniel Lewis Lee, 47, on Tuesday morning} for his part in the
murder of a family of three. Just hours before, the Supreme Court issued
the final go-ahead,
\href{https://www.nytimes.com/2020/07/13/us/politics/federal-execution.html}{ruling
in a split 5-4 decision} in the dead of night that the federal
government's single-drug execution protocol was constitutional. The
verdict cleared the way for Mr. Lee, Mr. Purkey, and Dustin Lee Honken,
scheduled to die on Friday, to be executed this week.

Mr. Purkey's lawyers had argued that he was incompetent to be executed.
They said he suffered from schizophrenia, Alzheimer's disease and brain
damage, which left him unable to comprehend why he was sentenced to
death. He believed that his execution was intended as retaliation by the
federal government for his frequent complaints about prison conditions,
they said.

Judge Tanya Chutkan of the Federal District Court in Washington, D.C.,
ruled on Wednesday that Mr. Purkey's execution should be delayed until
the court could determine whether he was fit to be executed.

In a separate decision, Judge Chutkan also ruled on Wednesday that the
lethal injection protocol required additional litigation to determine
whether it violated several federal statutes and the inmates'
constitutional rights. She cited the potential for ``irreparable harm''
if the inmates were put to death before their claims could be resolved
by the court.

The government filed an immediate appeal in both cases. Judge Chutkan's
delays forced the Bureau to briefly postpone Mr. Purkey's death,
scheduled for Wednesday evening. The Supreme Court, which overturned a
similar ruling from Judge Chutkan about the single-drug protocol in Mr.
Lee's case earlier this week, vacated both preliminary injunctions early
Thursday.

In the case concerning Mr. Purkey's mental fitness, the court split 5-4
with the liberal justices dissenting, as they did earlier this week in
Mr. Lee's case. Justice Sonia Sotomayor, joined by Justices Ruth Bader
Ginsburg, Stephen Breyer, and Elena Kagan, wrote that continuing with
Mr. Purkey's execution ``despite the grave questions and factual
findings regarding his mental competency, casts a shroud of
constitutional doubt over the most irrevocable of injuries.''

In a separate dissent, Justice Stephen Breyer, joined by Justice
Ginsburg, reiterated questions about the effectiveness of the death
penalty as a form of deterrence and retribution.

``A modern system of criminal justice must be reasonably accurate, fair,
humane, and timely. Our recent experience with the federal government's
resumption of executions adds to the mounting body of evidence that the
death penalty cannot be reconciled with those values,'' he wrote. ``I
remain convinced of the importance of reconsidering the
constitutionality of the death penalty itself.''

\href{https://www.nytimes.com/news-event/coronavirus?action=click\&pgtype=Article\&state=default\&region=MAIN_CONTENT_3\&context=storylines_faq}{}

\hypertarget{the-coronavirus-outbreak-}{%
\subsubsection{The Coronavirus Outbreak
›}\label{the-coronavirus-outbreak-}}

\hypertarget{frequently-asked-questions}{%
\paragraph{Frequently Asked
Questions}\label{frequently-asked-questions}}

Updated July 27, 2020

\begin{itemize}
\item ~
  \hypertarget{should-i-refinance-my-mortgage}{%
  \paragraph{Should I refinance my
  mortgage?}\label{should-i-refinance-my-mortgage}}

  \begin{itemize}
  \tightlist
  \item
    \href{https://www.nytimes.com/article/coronavirus-money-unemployment.html?action=click\&pgtype=Article\&state=default\&region=MAIN_CONTENT_3\&context=storylines_faq}{It
    could be a good idea,} because mortgage rates have
    \href{https://www.nytimes.com/2020/07/16/business/mortgage-rates-below-3-percent.html?action=click\&pgtype=Article\&state=default\&region=MAIN_CONTENT_3\&context=storylines_faq}{never
    been lower.} Refinancing requests have pushed mortgage applications
    to some of the highest levels since 2008, so be prepared to get in
    line. But defaults are also up, so if you're thinking about buying a
    home, be aware that some lenders have tightened their standards.
  \end{itemize}
\item ~
  \hypertarget{what-is-school-going-to-look-like-in-september}{%
  \paragraph{What is school going to look like in
  September?}\label{what-is-school-going-to-look-like-in-september}}

  \begin{itemize}
  \tightlist
  \item
    It is unlikely that many schools will return to a normal schedule
    this fall, requiring the grind of
    \href{https://www.nytimes.com/2020/06/05/us/coronavirus-education-lost-learning.html?action=click\&pgtype=Article\&state=default\&region=MAIN_CONTENT_3\&context=storylines_faq}{online
    learning},
    \href{https://www.nytimes.com/2020/05/29/us/coronavirus-child-care-centers.html?action=click\&pgtype=Article\&state=default\&region=MAIN_CONTENT_3\&context=storylines_faq}{makeshift
    child care} and
    \href{https://www.nytimes.com/2020/06/03/business/economy/coronavirus-working-women.html?action=click\&pgtype=Article\&state=default\&region=MAIN_CONTENT_3\&context=storylines_faq}{stunted
    workdays} to continue. California's two largest public school
    districts --- Los Angeles and San Diego --- said on July 13, that
    \href{https://www.nytimes.com/2020/07/13/us/lausd-san-diego-school-reopening.html?action=click\&pgtype=Article\&state=default\&region=MAIN_CONTENT_3\&context=storylines_faq}{instruction
    will be remote-only in the fall}, citing concerns that surging
    coronavirus infections in their areas pose too dire a risk for
    students and teachers. Together, the two districts enroll some
    825,000 students. They are the largest in the country so far to
    abandon plans for even a partial physical return to classrooms when
    they reopen in August. For other districts, the solution won't be an
    all-or-nothing approach.
    \href{https://bioethics.jhu.edu/research-and-outreach/projects/eschool-initiative/school-policy-tracker/}{Many
    systems}, including the nation's largest, New York City, are
    devising
    \href{https://www.nytimes.com/2020/06/26/us/coronavirus-schools-reopen-fall.html?action=click\&pgtype=Article\&state=default\&region=MAIN_CONTENT_3\&context=storylines_faq}{hybrid
    plans} that involve spending some days in classrooms and other days
    online. There's no national policy on this yet, so check with your
    municipal school system regularly to see what is happening in your
    community.
  \end{itemize}
\item ~
  \hypertarget{is-the-coronavirus-airborne}{%
  \paragraph{Is the coronavirus
  airborne?}\label{is-the-coronavirus-airborne}}

  \begin{itemize}
  \tightlist
  \item
    The coronavirus
    \href{https://www.nytimes.com/2020/07/04/health/239-experts-with-one-big-claim-the-coronavirus-is-airborne.html?action=click\&pgtype=Article\&state=default\&region=MAIN_CONTENT_3\&context=storylines_faq}{can
    stay aloft for hours in tiny droplets in stagnant air}, infecting
    people as they inhale, mounting scientific evidence suggests. This
    risk is highest in crowded indoor spaces with poor ventilation, and
    may help explain super-spreading events reported in meatpacking
    plants, churches and restaurants.
    \href{https://www.nytimes.com/2020/07/06/health/coronavirus-airborne-aerosols.html?action=click\&pgtype=Article\&state=default\&region=MAIN_CONTENT_3\&context=storylines_faq}{It's
    unclear how often the virus is spread} via these tiny droplets, or
    aerosols, compared with larger droplets that are expelled when a
    sick person coughs or sneezes, or transmitted through contact with
    contaminated surfaces, said Linsey Marr, an aerosol expert at
    Virginia Tech. Aerosols are released even when a person without
    symptoms exhales, talks or sings, according to Dr. Marr and more
    than 200 other experts, who
    \href{https://academic.oup.com/cid/article/doi/10.1093/cid/ciaa939/5867798}{have
    outlined the evidence in an open letter to the World Health
    Organization}.
  \end{itemize}
\item ~
  \hypertarget{what-are-the-symptoms-of-coronavirus}{%
  \paragraph{What are the symptoms of
  coronavirus?}\label{what-are-the-symptoms-of-coronavirus}}

  \begin{itemize}
  \tightlist
  \item
    Common symptoms
    \href{https://www.nytimes.com/article/symptoms-coronavirus.html?action=click\&pgtype=Article\&state=default\&region=MAIN_CONTENT_3\&context=storylines_faq}{include
    fever, a dry cough, fatigue and difficulty breathing or shortness of
    breath.} Some of these symptoms overlap with those of the flu,
    making detection difficult, but runny noses and stuffy sinuses are
    less common.
    \href{https://www.nytimes.com/2020/04/27/health/coronavirus-symptoms-cdc.html?action=click\&pgtype=Article\&state=default\&region=MAIN_CONTENT_3\&context=storylines_faq}{The
    C.D.C. has also} added chills, muscle pain, sore throat, headache
    and a new loss of the sense of taste or smell as symptoms to look
    out for. Most people fall ill five to seven days after exposure, but
    symptoms may appear in as few as two days or as many as 14 days.
  \end{itemize}
\item ~
  \hypertarget{does-asymptomatic-transmission-of-covid-19-happen}{%
  \paragraph{Does asymptomatic transmission of Covid-19
  happen?}\label{does-asymptomatic-transmission-of-covid-19-happen}}

  \begin{itemize}
  \tightlist
  \item
    So far, the evidence seems to show it does. A widely cited
    \href{https://www.nature.com/articles/s41591-020-0869-5}{paper}
    published in April suggests that people are most infectious about
    two days before the onset of coronavirus symptoms and estimated that
    44 percent of new infections were a result of transmission from
    people who were not yet showing symptoms. Recently, a top expert at
    the World Health Organization stated that transmission of the
    coronavirus by people who did not have symptoms was ``very rare,''
    \href{https://www.nytimes.com/2020/06/09/world/coronavirus-updates.html?action=click\&pgtype=Article\&state=default\&region=MAIN_CONTENT_3\&context=storylines_faq\#link-1f302e21}{but
    she later walked back that statement.}
  \end{itemize}
\end{itemize}

A federal appeals court panel had issued a stay in Mr. Purkey's case
that remained in effect until the day before his death. In that
proceeding, Mr. Purkey claimed that his legal counsel failed to
adequately defend him at trial and during his habeas proceedings, but
the government argued that federal law barred him from raising the issue
so late in his case.

The Seventh Circuit ruled that his claims deserved another look by the
court before he was executed. The Supreme Court rejected that argument.

The coronavirus pandemic also complicated Mr. Purkey's case. The Rev.
Dale Hartkemeyer, a Buddhist priest and his spiritual adviser, said he
could not attend Mr. Purkey's execution without exposing himself to the
virus. The priest, who goes by the name of Seigen, had developed a
relationship with Mr. Purkey over more than a decade. But he said his
history of bronchitis and pleurisy, lung-related illnesses, left him at
greater risk from the coronavirus.

Mr. Hartkemeyer sued the Justice Department, joined by a spiritual
adviser for Mr. Honken. They argued that by forcing them to choose
between their religious duties and health, the government violated the
Religious Freedom Restoration Act. The law prohibits the government from
burdening an individual's exercise of religion.

A federal judge in Indiana rejected the lawsuit Tuesday, and the Supreme
Court dismissed Mr. Hartkemeyer's case Thursday morning.

Robert Dunham, executive director of the Death Penalty Information
Center, said the court failed to give Mr. Purkey the opportunity to
adequately defend his claims before his death. If the justices had
followed the rules of judicial decision-making, Mr. Purkey would not
have been executed, Mr. Dunham said.

``They did it again,'' he said. ``Five members of the court knew what
outcome they wanted before they decided, and the merits didn't matter.''

Advertisement

\protect\hyperlink{after-bottom}{Continue reading the main story}

\hypertarget{site-index}{%
\subsection{Site Index}\label{site-index}}

\hypertarget{site-information-navigation}{%
\subsection{Site Information
Navigation}\label{site-information-navigation}}

\begin{itemize}
\tightlist
\item
  \href{https://help.nytimes.com/hc/en-us/articles/115014792127-Copyright-notice}{©~2020~The
  New York Times Company}
\end{itemize}

\begin{itemize}
\tightlist
\item
  \href{https://www.nytco.com/}{NYTCo}
\item
  \href{https://help.nytimes.com/hc/en-us/articles/115015385887-Contact-Us}{Contact
  Us}
\item
  \href{https://www.nytco.com/careers/}{Work with us}
\item
  \href{https://nytmediakit.com/}{Advertise}
\item
  \href{http://www.tbrandstudio.com/}{T Brand Studio}
\item
  \href{https://www.nytimes.com/privacy/cookie-policy\#how-do-i-manage-trackers}{Your
  Ad Choices}
\item
  \href{https://www.nytimes.com/privacy}{Privacy}
\item
  \href{https://help.nytimes.com/hc/en-us/articles/115014893428-Terms-of-service}{Terms
  of Service}
\item
  \href{https://help.nytimes.com/hc/en-us/articles/115014893968-Terms-of-sale}{Terms
  of Sale}
\item
  \href{https://spiderbites.nytimes.com}{Site Map}
\item
  \href{https://help.nytimes.com/hc/en-us}{Help}
\item
  \href{https://www.nytimes.com/subscription?campaignId=37WXW}{Subscriptions}
\end{itemize}
