Sections

SEARCH

\protect\hyperlink{site-content}{Skip to
content}\protect\hyperlink{site-index}{Skip to site index}

\href{https://www.nytimes.com/section/world/asia}{Asia Pacific}

\href{https://myaccount.nytimes.com/auth/login?response_type=cookie\&client_id=vi}{}

\href{https://www.nytimes.com/section/todayspaper}{Today's Paper}

\href{/section/world/asia}{Asia Pacific}\textbar{}U.S. Wants to Bar
Members of China's Communist Party. Who Are They?

\url{https://nyti.ms/3jaJGuf}

\begin{itemize}
\item
\item
\item
\item
\item
\end{itemize}

Advertisement

\protect\hyperlink{after-top}{Continue reading the main story}

Supported by

\protect\hyperlink{after-sponsor}{Continue reading the main story}

\hypertarget{us-wants-to-bar-members-of-chinas-communist-party-who-are-they}{%
\section{U.S. Wants to Bar Members of China's Communist Party. Who Are
They?}\label{us-wants-to-bar-members-of-chinas-communist-party-who-are-they}}

With more than 90 million members and led by Xi Jinping, the party
encompasses people at the heights of Chinese power and the civil
servants of everyday life.

\includegraphics{https://static01.nyt.com/images/2020/07/17/world/16china-explain1/merlin_172914009_247b0f19-b874-4bee-9e31-d61c2b93465d-articleLarge.jpg?quality=75\&auto=webp\&disable=upscale}

\href{https://www.nytimes.com/by/paul-mozur}{\includegraphics{https://static01.nyt.com/images/2018/10/15/multimedia/author-paul-mozur/author-paul-mozur-thumbLarge.png}}

By \href{https://www.nytimes.com/by/paul-mozur}{Paul Mozur}

\begin{itemize}
\item
  July 16, 2020
\item
  \begin{itemize}
  \item
  \item
  \item
  \item
  \item
  \end{itemize}
\end{itemize}

\href{https://cn.nytimes.com/usa/20200717/china-communist-party-travel-ban-explain/}{阅读简体中文版}\href{https://cn.nytimes.com/usa/20200717/china-communist-party-travel-ban-explain/zh-han}{閱讀繁體中文版}

As the Trump administration weighs
\href{https://www.nytimes.com/2020/07/15/us/politics/china-travel-ban.html}{a
travel ban against the members of the Chinese Communist Party and their
relatives}, it is considering cutting off a vast sector of Chinese
society --- 92 million people --- that often defies stereotypes,
including those who walk the halls of power in Beijing, supervise
China's schools and run major companies.

Barring them would change the topography of the
\href{https://www.nytimes.com/2020/07/23/world/asia/us-china-consulate.html}{U.S.-China
relationship} in the most prosaic of ways: It would cut off huge numbers
of regular Chinese who, in pre-coronavirus pandemic times, traveled to
the United States by the millions to do business, see the sights, shop
at high-end department stores and study at some of the country's most
elite universities.

Blocking the party members would not only turn off an economic spigot
for the United States, but also plunge the relationship between the
world's two largest economies into a new phase of deeper isolation. Who
are the members of the Communist Party? Here's what we know.

\hypertarget{leaders-prize-winners-dissenters-and-caretakers}{%
\subsection{Leaders, prize winners, dissenters and
caretakers}\label{leaders-prize-winners-dissenters-and-caretakers}}

Some Communist Party members are the stolid apparatchiks of the
Communist stereotype; many are not. At the heady heights of political
power in Beijing, members craft harsh crackdowns,
\href{https://www.nytimes.com/2020/06/07/world/asia/china-coronavirus.html}{misleading
propaganda} and
\href{https://www.nytimes.com/2020/06/17/world/asia/China-DNA-surveillance.html}{sweeping
surveillance} designed to preserve the party's autocratic rule over the
country.

They keep tabs on people they consider political troublemakers and
control China's government in Beijing. They enforce rules that have led
to
\href{https://www.nytimes.com/interactive/2019/11/16/world/asia/china-xinjiang-documents.html}{the
internment of more than a million members of minorities} like the
\href{https://www.nytimes.com/2020/07/06/world/asia/china-xinjiang-uighur-court.html}{Uighurs
in the country's west}.

Yet voices of dissent have also come from the party. Dr. Li Wenliang,
who sounded the alarm online about a mysterious virus that emerged in
China and was
\href{https://www.nytimes.com/2020/02/06/world/asia/chinese-doctor-Li-Wenliang-coronavirus.html}{interrogated
by the police} for his trouble before dying of Covid-19, was a party
member.

So, too, is the Uighur
\href{https://www.nytimes.com/2016/10/12/world/europe/ilham-tohti-uighur-human-rights-award.html}{economist
Ilham Tohti}, a winner of the Sakharov Prize.

\href{http://www.12371.cn/2020/06/30/ARTI1593514894217396.shtml}{Recent
statistics} showed 12.3 million of them are 30 or younger, about half
have college or university degrees, and 27.9 percent are female. Many
party members also offer child care services, run schools, manage
technology companies, organize beach cleanups, act in blockbuster movies
and do outreach to older Chinese citizens. Alongside them are academics,
scientists and business people --- lifelines for a U.S.-Chinese economic
relationship that has persisted despite souring ties.

For those not at the top rungs of power, membership in the party is
often a way to fuel one's career by making the right connections. During
the boom years from the 1980s to the early 2010s, many Chinese joined
the party to get a leg up in business, academics and the arts.

\hypertarget{a-party-born-of-a-civil-war}{%
\subsection{A party born of a civil
war}\label{a-party-born-of-a-civil-war}}

Founded in 1921, the Communist Party has dominated politics in China
since it won a civil war against the Nationalists of the Republic of
China in 1949. Since then, it has gone through many evolutions, some
dictated by practicality, others by the small-minded calculations of
power grabs.

In recent decades, the party has appeared to emerge as a bastion of
technocrats wielding industrial policy and close ties to business to
emphasize economic growth, even as they sharply punished those who
defied their power.

\href{https://www.nytimes.com/2017/10/24/world/asia/china-xi-jinping-constitution.html}{Under
Xi Jinping}, China's top leader, a party that 10 years ago was often
jokingly called a business group masquerading as Marxists has reaffirmed
its Communist roots. Members must engage in study sessions of high
theory, at times with the
\href{https://www.nytimes.com/2019/04/07/world/asia/china-xi-jinping-study-the-great-nation-app.html}{tracking
power} of apps to monitor their reading habits.

Mr. Xi has emphasized political loyalty over economic benefits, and a
ferocious anticorruption crackdown has taken some of the shine out of
joining. He has also made the selection process more rigorous: What was
once a dull formality has become more
\href{https://macropolo.org/analysis/members-only-recruitment-trends-in-the-chinese-communist-party/}{difficult
and selective}. Applicants are subjected to an investigation and a
battery of tests and interviews, before years of waiting for full
membership.

\hypertarget{a-new-shinier-hammer-and-sickle}{%
\subsection{A new, shinier hammer and
sickle}\label{a-new-shinier-hammer-and-sickle}}

\includegraphics{https://static01.nyt.com/images/2020/07/16/world/16china-explain2/merlin_135784293_fc46910f-c172-48f2-bb04-741457dfc206-articleLarge.jpg?quality=75\&auto=webp\&disable=upscale}

The power and symbols of the Communist Party of China loom inside
companies and other organizations, and new, shinier hammer-and-sickle
signs have popped up at community centers in towns and cities across the
country.

Inside the party's multiplicities, there are books and playgrounds for
children. But surveillance is a given, with officials neatly tracking
local happenings, reporting political troublemakers via databases.

Party committees, once ceremonial and dormant at private companies, have
gained new powers. Many top executives, like Jack Ma, co-founder of
Alibaba, are
\href{https://www.nytimes.com/2018/11/27/business/jack-ma-communist-party-alibaba.html}{members}.
Overseas, the party's structure has helped tie together institutions
that run
\href{https://www.nytimes.com/2019/05/20/world/australia/australia-china.html}{influence
campaigns} to drum up support for China.

Yet even as Mr. Xi has brought back many Mao trappings, like study
sessions and surveillance, people still join for the professional perks,
not the stultifying ideology. Tempted by the prospects of better jobs,
many students
\href{https://supchina.com/2019/07/31/communism-is-a-faith/}{sign up in
university}, well before they have a fully developed political outlook.
Admittance is often seen as a sign of excellence. In the southern tech
hub of Shenzhen in 2018, a sign encouraged entrepreneurs with a slogan
that would boggle the mind of the orthodox Marxist: ``Follow our party,
start your business.''

\hypertarget{barring-millions-may-be-impossible}{%
\subsection{Barring millions may be
impossible}\label{barring-millions-may-be-impossible}}

With party members making up a bewilderingly huge portion of society in
China, some tell stories of the party losing the records of their
membership. Mr. Xi, in seeking to revive the party, has gone after
myriad members who have not paid dues for years.

If even Beijing is struggling to track the 92 million party members and
their families, it's not clear that the United States could do a much
better job if it decides to carry out its travel ban. Experts cautioned
that the draft ban would be all but impossible to enforce on a wide
scale.

Even so, the United States could set up new mechanisms by which the
State Department and the Department of Homeland Security could more
closely track party membership. Several Chinese citizens who recently
traveled to the United States said they did not recall being asked about
party affiliation. Though some travel applications from the State
Department explicitly ask.

Any new rule would also be easier to apply to more prominent Chinese
political leaders and their families. Children of top leaders might
struggle to gain entry to the United States if the order is signed. Mr.
Xi's daughter, Xi Mingze, for instance, attended Harvard under a
pseudonym several years ago.

Yet, if enforced strictly, the visa ban could make life difficult for
many others. Scholars and business people who regularly visit the United
States might have to either disclose their membership or risk running
afoul of laws that punish falsification of visa applications.

A spokeswoman for the Chinese Ministry of Foreign Affairs, Hua Chunying,
called the possible move by the Trump administration ``very pathetic.''

``The United States, as the most powerful country, what has it left?
What kind of impression does it want to leave the world? We hope that
the United States will stop doing such things that do not respect the
basic norms of international relations.''

The U.S. threat alone could keep many from coming to the States and do
more to push business conferences and other events that include
Americans and Chinese to other countries --- like Canada.

Lin Qiqing contributed research.

Advertisement

\protect\hyperlink{after-bottom}{Continue reading the main story}

\hypertarget{site-index}{%
\subsection{Site Index}\label{site-index}}

\hypertarget{site-information-navigation}{%
\subsection{Site Information
Navigation}\label{site-information-navigation}}

\begin{itemize}
\tightlist
\item
  \href{https://help.nytimes.com/hc/en-us/articles/115014792127-Copyright-notice}{©~2020~The
  New York Times Company}
\end{itemize}

\begin{itemize}
\tightlist
\item
  \href{https://www.nytco.com/}{NYTCo}
\item
  \href{https://help.nytimes.com/hc/en-us/articles/115015385887-Contact-Us}{Contact
  Us}
\item
  \href{https://www.nytco.com/careers/}{Work with us}
\item
  \href{https://nytmediakit.com/}{Advertise}
\item
  \href{http://www.tbrandstudio.com/}{T Brand Studio}
\item
  \href{https://www.nytimes.com/privacy/cookie-policy\#how-do-i-manage-trackers}{Your
  Ad Choices}
\item
  \href{https://www.nytimes.com/privacy}{Privacy}
\item
  \href{https://help.nytimes.com/hc/en-us/articles/115014893428-Terms-of-service}{Terms
  of Service}
\item
  \href{https://help.nytimes.com/hc/en-us/articles/115014893968-Terms-of-sale}{Terms
  of Sale}
\item
  \href{https://spiderbites.nytimes.com}{Site Map}
\item
  \href{https://help.nytimes.com/hc/en-us}{Help}
\item
  \href{https://www.nytimes.com/subscription?campaignId=37WXW}{Subscriptions}
\end{itemize}
