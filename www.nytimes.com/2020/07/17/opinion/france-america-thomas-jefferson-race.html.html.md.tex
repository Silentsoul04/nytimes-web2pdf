Sections

SEARCH

\protect\hyperlink{site-content}{Skip to
content}\protect\hyperlink{site-index}{Skip to site index}

\href{https://myaccount.nytimes.com/auth/login?response_type=cookie\&client_id=vi}{}

\href{https://www.nytimes.com/section/todayspaper}{Today's Paper}

\href{/section/opinion}{Opinion}\textbar{}The Tenacity of the
Franco-American Ideal

\href{https://nyti.ms/2ZF1GFc}{https://nyti.ms/2ZF1GFc}

\begin{itemize}
\item
\item
\item
\item
\item
\item
\end{itemize}

Advertisement

\protect\hyperlink{after-top}{Continue reading the main story}

\href{/section/opinion}{Opinion}

Supported by

\protect\hyperlink{after-sponsor}{Continue reading the main story}

\hypertarget{the-tenacity-of-the-franco-american-ideal}{%
\section{The Tenacity of the Franco-American
Ideal}\label{the-tenacity-of-the-franco-american-ideal}}

Can a slave owner be celebrated for writing a liberating sentence?

\href{https://www.nytimes.com/by/roger-cohen}{\includegraphics{https://static01.nyt.com/images/2014/11/01/opinion/cohen-circular/cohen-circular-thumbLarge-v6.png}}

By \href{https://www.nytimes.com/by/roger-cohen}{Roger Cohen}

Opinion Columnist

\begin{itemize}
\item
  July 17, 2020
\item
  \begin{itemize}
  \item
  \item
  \item
  \item
  \item
  \item
  \end{itemize}
\end{itemize}

\includegraphics{https://static01.nyt.com/images/2020/07/20/opinion/20cohen_print/17cohenWeb-articleLarge.jpg?quality=75\&auto=webp\&disable=upscale}

PARIS --- Perhaps the root of the mutual fascination that binds France
and the United States is that each sees itself as an idea, a model of
some kind for the rest of the world. This is an immodest but tenacious
notion, bound up with the founding articles and myths of both republics.
No other countries make such claims for the universality of their
virtue.

These are now unfashionable ideas, having their roots in the white
patriarchal societies of the late 18th century.

Beware of fashion. It may overcorrect. I will try to explain.

In France,
\href{https://avalon.law.yale.edu/18th_century/rightsof.asp}{the
Declaration of the Rights of Man and of the Citizen}, adopted in 1789 as
the expression of the ideals of the French Revolution, states in its
first article: ``Men are born and remain free and equal in rights.'' The
declaration defines these natural rights as ``liberty, property,
security, and resistance against oppression,'' and says that liberty
``consists of doing anything which does not harm others.''

Thirteen years earlier, in its Declaration of Independence, the United
States set out certain ``self-evident'' truths: ``that all men are
created equal, that they are endowed by their Creator with certain
unalienable Rights, that among these are Life, Liberty and the pursuit
of Happiness.'' The right to govern stemmed ``from the consent of the
governed.'' Over the ensuing 15 years, these ideas were enshrined in the
United States Constitution and Bill of Rights.

France and the United States were intertwined as political allies, but
also as twinned sources of Enlightenment principles. Thomas Jefferson, a
slave owner, influenced the formulation of the French Declaration and
was an author of America's founding laws.

The revolutions were sweeping. There was nothing ``self-evident'' about
them. Out with monarchy, in with ``We the people.'' Out with divine
right, in with human rights. Out with rule by edict, in with the
separation of powers and the rule of law. So, falteringly, began the
liberal democratic experiment, now under attack.

The experiment was as flawed as Jefferson himself. All men are created
equal. Sounds good, but what about women?
(\href{https://www.bl.uk/collection-items/the-declaration-of-the-rights-of-woman-and-the-citizen}{A
Declaration of the Rights of Woman and the Female Citizen} was written
in France in 1791 by the French feminist Olympe de Gouges.) And what of
Black slaves, their value set in the Constitution at 60 percent of a
free human being? Let's rephrase the sentence: \emph{All white male
property owners are created equal}. Not much of a ring to it, but has
the merit of accuracy.

And what of France,
\href{https://www.google.com/url?q=https://www.nytimes.com/2020/06/24/world/europe/france-george-floyd-racism-slave-trade.html\&sa=D\&ust=1595005716313000\&usg=AFQjCNGxqT9Pi2mklxELxZ0kkuOVj1qY0g}{trading
in slaves well into the 19th century}, ushering Jews to emancipation
through the principles of the revolution only to contribute to their
mass murder during World War II, fighting a savage colonial war in
Algeria between 1954 and 1962?

So, a cry goes up. These pretensions of embodying ennobling ideals for
humankind were false, reflecting no more than the narrow worldview of
18th-century white males whose talk of equal rights was shot through
with exploitative hypocrisy.

The perfect becomes the enemy of the good. In an age of absolutist moral
certainty, the most conspicuous feature of humankind --- its fallibility
--- becomes unpardonable. Can a slave owner be celebrated for penning a
liberating sentence? How can a progressive socialist French president,
François Mitterrand, have been an official of the Vichy regime? Because
the second-most conspicuous feature of human beings is their
contradictory natures.

``I don't think any people enjoys rooting around in the unpleasant parts
of their past,'' Robert Paxton, a prominent American historian whose
groundbreaking work helped bring France to a full
\href{https://www.nytimes.com/1997/11/01/world/us-historian-relates-how-vichy-france-served-nazis.html}{understanding
of the crimes of the Vichy regime}, told me. ``Denial is often
ineradicable. I think on the whole the French came out of it quicker
than we did.''

It took more than a half-century, until 1995, for France, in the person
of President Jacques Chirac, to acknowledge that the French state, and
not some handful of misguided Vichy operatives,
\href{https://www.nytimes.com/1995/07/17/world/chirac-affirms-france-s-guilt-in-fate-of-jews.html}{had
``committed the irreparable''} in sending some 76,000 French and foreign
Jews to their deaths. It was more than a half-century after France left
Algeria that Emmanuel Macron, while a candidate for the French
presidency in 2017, called the French colonization of Algeria ``a crime
against humanity'' and later, as president, acknowledged French
``atrocities.''

The United States has never formally apologized for slavery. President
Clinton, in Africa more than two decades ago,
\href{https://www.google.com/url?q=https://www.nytimes.com/1998/03/25/world/clinton-africa-overview-uganda-clinton-expresses-regret-slavery-us.html\&sa=D\&ust=1595008239423000\&usg=AFQjCNGGXNqmVDJsq_LlHavJsYM61duUkQ}{managed
to say that ``we were wrong''} to have ``received the fruits of the
slave trade.'' That was all he could muster.

Now, in the midst of another push to overcome America's original sin,
would be a good moment for such an apology.

That, after all, is what democracies like France and the United States
are capable of: continuous adjustment, improvement, recognition of past
mistakes, atonement, progress toward their ideals. If they are, it is
thanks in large part to the flawed brilliance of the architects, direct
or indirect, of the two republics.

We can and should acknowledge their flaws without denigrating their
achievement in spreading the ideas of liberty, free expression and the
rule of law across the face of the earth. The words that issued from
Paris and Philadelphia between 1776 and 1791 have served the cause of
freedom, even if they were the product of minds and cultures foreign to
the Great Awokening of recent years, whose own chief flaw may prove to
be self-righteous intolerance.

\emph{The Times is committed to publishing}
\href{https://www.nytimes.com/2019/01/31/opinion/letters/letters-to-editor-new-york-times-women.html}{\emph{a
diversity of letters}} \emph{to the editor. We'd like to hear what you
think about this or any of our articles. Here are some}
\href{https://help.nytimes.com/hc/en-us/articles/115014925288-How-to-submit-a-letter-to-the-editor}{\emph{tips}}\emph{.
And here's our email:}
\href{mailto:letters@nytimes.com}{\emph{letters@nytimes.com}}\emph{.}

\emph{Follow The New York Times Opinion section on}
\href{https://www.facebook.com/nytopinion}{\emph{Facebook}}\emph{,}
\href{http://twitter.com/NYTOpinion}{\emph{Twitter (@NYTopinion)}}
\emph{and}
\href{https://www.instagram.com/nytopinion/}{\emph{Instagram}}\emph{.}

Advertisement

\protect\hyperlink{after-bottom}{Continue reading the main story}

\hypertarget{site-index}{%
\subsection{Site Index}\label{site-index}}

\hypertarget{site-information-navigation}{%
\subsection{Site Information
Navigation}\label{site-information-navigation}}

\begin{itemize}
\tightlist
\item
  \href{https://help.nytimes.com/hc/en-us/articles/115014792127-Copyright-notice}{©~2020~The
  New York Times Company}
\end{itemize}

\begin{itemize}
\tightlist
\item
  \href{https://www.nytco.com/}{NYTCo}
\item
  \href{https://help.nytimes.com/hc/en-us/articles/115015385887-Contact-Us}{Contact
  Us}
\item
  \href{https://www.nytco.com/careers/}{Work with us}
\item
  \href{https://nytmediakit.com/}{Advertise}
\item
  \href{http://www.tbrandstudio.com/}{T Brand Studio}
\item
  \href{https://www.nytimes.com/privacy/cookie-policy\#how-do-i-manage-trackers}{Your
  Ad Choices}
\item
  \href{https://www.nytimes.com/privacy}{Privacy}
\item
  \href{https://help.nytimes.com/hc/en-us/articles/115014893428-Terms-of-service}{Terms
  of Service}
\item
  \href{https://help.nytimes.com/hc/en-us/articles/115014893968-Terms-of-sale}{Terms
  of Sale}
\item
  \href{https://spiderbites.nytimes.com}{Site Map}
\item
  \href{https://help.nytimes.com/hc/en-us}{Help}
\item
  \href{https://www.nytimes.com/subscription?campaignId=37WXW}{Subscriptions}
\end{itemize}
