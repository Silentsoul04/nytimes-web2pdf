Sections

SEARCH

\protect\hyperlink{site-content}{Skip to
content}\protect\hyperlink{site-index}{Skip to site index}

\href{https://www.nytimes.com/section/opinion/sunday}{Sunday Review}

\href{https://myaccount.nytimes.com/auth/login?response_type=cookie\&client_id=vi}{}

\href{https://www.nytimes.com/section/todayspaper}{Today's Paper}

\href{/section/opinion/sunday}{Sunday Review}\textbar{}Do Progressives
Have a Free Speech Problem?

\href{https://nyti.ms/3jcp0SB}{https://nyti.ms/3jcp0SB}

\begin{itemize}
\item
\item
\item
\item
\item
\item
\end{itemize}

Advertisement

\protect\hyperlink{after-top}{Continue reading the main story}

\href{/section/opinion}{Opinion}

Supported by

\protect\hyperlink{after-sponsor}{Continue reading the main story}

\hypertarget{do-progressives-have-a-free-speech-problem}{%
\section{Do Progressives Have a Free Speech
Problem?}\label{do-progressives-have-a-free-speech-problem}}

The illiberal left is a lot less threatening than the right. That
doesn't mean it doesn't exist.

\href{https://www.nytimes.com/by/michelle-goldberg}{\includegraphics{https://static01.nyt.com/images/2018/04/02/opinion/michelle-goldberg/michelle-goldberg-thumbLarge.png}}

By \href{https://www.nytimes.com/by/michelle-goldberg}{Michelle
Goldberg}

Opinion Columnist

\begin{itemize}
\item
  July 17, 2020
\item
  \begin{itemize}
  \item
  \item
  \item
  \item
  \item
  \item
  \end{itemize}
\end{itemize}

\includegraphics{https://static01.nyt.com/images/2020/07/17/opinion/17goldberg_sub2/17goldberg_sub2-articleLarge-v18.jpg?quality=75\&auto=webp\&disable=upscale}

An acquaintance came to me a few weeks ago with the rough draft of a
letter about free speech and asked me to sign. I declined, in part
because it denounced ``cancel culture.'' As I wrote in an email, the
phrase ```cancel culture,' while it describes something real, has been
rendered sort of useless because it's so often used by right-wing
whiners like Ivanka Trump who think protests against them violate their
free speech.''

A little later my acquaintance came back to me with a new version, which
didn't mention ``cancel culture.'' Like the people who wrote the letter,
I think
\href{https://www.thenation.com/article/archive/cancelcolbert-and-return-anti-liberal-left/}{left-wing
illiberalism} is a problem, though I've mostly stopped writing about it
since Donald Trump was elected, because it seems like complaining about
a bee sting when you have Stage IV cancer.

So I signed. The statement, published in Harper's Magazine as
``\href{https://harpers.org/a-letter-on-justice-and-open-debate/}{A
Letter on Justice and Open Debate},'' spawned takes and countertakes,
most of them, despite my modest effort, about ``cancel culture.''

At first I avoided wading into discourse about what's now called the
Letter. It seemed self-indulgent to write about media angst when the
country is self-immolating because of unchecked disease and an economic
catastrophe that's about to get much worse. But as the debate over free
speech grew and grew, I started to think I was using the burning world
as an excuse to avoid personal discomfort.

From my (privileged) vantage point, several things are happening
simultaneously. The mass uprising following the killing of George Floyd
has led to a necessary expansion of the boundaries of mainstream speech.
Space has been created for daring left-wing ideas, like abolishing the
police, that were once marginalized. Cultural institutions are reckoning
with the racism that leads to mostly white leadership.

At the same time, a climate of punitive heretic-hunting, a recurrent
feature of left-wing politics, has set in, enforced, in some cases,
through workplace discipline, including firings. It's the involvement of
human resources departments in compelling adherence with rapidly
changing new norms of speech and debate that worries me the most.

In her
\href{https://www.theatlantic.com/culture/archive/2020/07/harpers-letter-free-speech/614080/}{scathing
rejoinder} to the Letter in The Atlantic, Hannah Giorgis wrote, ``Facing
widespread criticism on Twitter, undergoing an internal workplace
review, or having
\href{https://www.theatlantic.com/ideas/archive/2020/01/american-dirt-controversy/605725/?gclid=EAIaIQobChMIo5rIn5fI6gIVDorICh1e2wb8EAAYASAAEgJl1_D_BwE}{one's
book}
\href{https://www.bookforum.com/print/2603/thomas-chatterton-williams-s-confused-argument-for-a-post-racial-society-23610}{panned}
does not, in fact, erode one's constitutional rights or endanger a
liberal society.''

This sentence brought me up short; one of these things is not like the
others. Anyone venturing ideas in public should be prepared to endure
negative reviews and pushback on social media. Internal workplace
reviews are something else. If people fear for their livelihoods for
relatively minor ideological transgressions, it may not violate the
Constitution --- the workplace is not the state --- but it does create a
climate of self-censorship and grudging conformity.

One of the more egregious recent examples of left-wing illiberalism is
the firing of David Shor, a data analyst at the progressive consulting
firm Civis Analytics. Amid the protests over Floyd's killing, Shor was
called out online for tweeting about work by Omar Wasow, an assistant
professor of politics at Princeton, that shows a link between violent
protest in the 1960s and Richard Nixon's vote share.

Shor was accused of ``anti-Blackness'' for seeming to suggest, via
Wasow's research, that violent protest is counterproductive. (Wasow is
Black.) ``At least some employees and clients of Civis Analytics
complained that Shor's tweet threatened their safety,''
\href{https://nymag.com/intelligencer/2020/06/case-for-liberalism-tom-cotton-new-york-times-james-bennet.html}{reported}
New York Magazine's Jonathan Chait. After an internal review, Shor was
let go; he was also
\href{https://nymag.com/intelligencer/2020/06/white-fragility-racism-racism-progressive-progressphiles-david-shor.html}{kicked
off} a progressive industry listserv.

Civis has denied that Shor was fired for a tweet, but an employee
\href{https://www.theatlantic.com/ideas/archive/2020/06/stop-firing-innocent/613615/}{told
The Atlantic's Yascha Mounk} that the company's chief executive said, in
a staff meeting, ``something along the lines of freedom of speech is
important, but he had to take a stand with our staff, clients, and
people of color.''

It should be said that many people on the left, including some who are
often dismissive of the idea of left-wing illiberalism, condemned Shor's
firing. Surely one reason this episode has been invoked so often is that
there aren't many comparable examples of such obvious social justice
overreach.

Still, there's no question that many people feel intimidated. John
McWhorter, an associate professor of English and comparative literature
at Columbia who signed the Harper's Letter, told me that in recent days
he's heard from over 100 graduate students and professors, most of them
left of center, who fear for their professional prospects if they get on
the wrong side of left-wing opinion.

Some on the left have argued, fairly, that those worried about people
losing their jobs for running afoul of progressive orthodoxies should do
more to strengthen labor protections, since all sorts of employees are
vulnerable to capricious termination.

In a much-discussed essay on what he called ``reactionary liberalism,''
\href{https://newrepublic.com/article/158346/willful-blindness-reactionary-liberalism}{The
New Republic's} Osita Nwanevu wrote, ``In practice, workers of all
stripes often lack the means and opportunity to defend themselves from
unjust firings⁠ --- all the more reason for those preoccupied with
`cancel culture' and social media-driven dismissals to support
just-cause provisions and an end to at-will employment.''

This is true; as
\href{https://www.persuasion.community/p/a-better-remedy-for-cancel-culture}{Zaid
Jilani wrote recently}, ``If it were harder for employers to fire people
for frivolous reasons, Americans would have less reason to fear that
expressing their views might cost them their livelihoods.'' But it seems
strange to me to argue that in the absence of better labor law, the left
is justified in taking advantage of precarity to punish people for
political disagreements.

None of this is an argument for a totally laissez-faire approach to
speech; some ideas \emph{should} be stigmatized.

I recently spoke to Wasow about the reaction to Shor tweeting his paper.
``Much of what we call `cancel culture' is just culture,'' he said.
``Culture has boundaries. Every community has boundaries. Those
boundaries are always shifting. In the age of the internet, they move
faster, and therefore where those boundaries are is less clear and less
stable, and it makes it easier for people to cross those lines.''

But it's a problem when the range of proscribed speech is so wide that
the rules are hard to even explain to those not steeped in left-wing
mores.

Writing in the 1990s, at a time when feminists like Catharine MacKinnon
sought to curtail free speech in the name of equality, the great
left-libertarian Ellen Willis described how progressive movements sow
the seeds of their own destruction when they become censorious. It's
impossible, Willis wrote, ``to censor the speech of the dominant without
stifling debate among all social groups and reinforcing orthodoxy within
left movements. Under such conditions a movement can neither integrate
new ideas nor build support based on genuine transformations of
consciousness rather than guilt or fear of ostracism.''

It's not always easy to draw a clear line between what Willis described
as ``reinforcing orthodoxy'' and agitating to make language and society
more democratic and inclusive. As Nicholas Grossman pointed out in
\href{https://arcdigital.media/free-speech-defenders-dont-understand-the-critique-against-them-4ed8327c0879}{Arc
Digital}, most signatories to the Letter probably agree that it's a good
thing that the casual use of racist and homophobic slurs is no longer
socially acceptable. ``But those changes came about through private
sanction, social pressure and cultural change, driven by activists and
younger generations,'' he wrote.

Willis reminds us that when these changes were happening, the right
denounced them as violations of free expression. Of the conservative
campaign against political correctness in the 1990s, she wrote,
``Predictably, their valid critique of left authoritarianism has segued
all too smoothly into a campaign of moral intimidation,'' one ``aimed at
demonizing egalitarian ideas, per se, as repressive.''

The same is happening today; the president throws tantrums about
``cancel culture'' while regularly trying to use the power of the state
to quash speech he dislikes. Because Trump poisons everything he
touches, his movement's hypocritical embrace of the mantle of free
speech threatens to devalue it, turning it into the rhetorical
equivalent of ``All Lives Matter.''

But to let this occur is to surrender what has historically been a
sacred left-wing value. One reason many on the right want to be seen as
free speech defenders is that they understand that the power to break
taboos can be even more potent than the power to create them. Even
sympathetic people will come to resent a left that refuses to make
distinctions between deliberate slurs, awkward mistakes and legitimate
disagreements. Cowing people is not the same as converting them.

\emph{The Times is committed to publishing}
\href{https://www.nytimes.com/2019/01/31/opinion/letters/letters-to-editor-new-york-times-women.html}{\emph{a
diversity of letters}} \emph{to the editor. We'd like to hear what you
think about this or any of our articles. Here are some}
\href{https://help.nytimes.com/hc/en-us/articles/115014925288-How-to-submit-a-letter-to-the-editor}{\emph{tips}}\emph{.
And here's our email:}
\href{mailto:letters@nytimes.com}{\emph{letters@nytimes.com}}\emph{.}

\emph{Follow The New York Times Opinion section on}
\href{https://www.facebook.com/nytopinion}{\emph{Facebook}}\emph{,}
\href{http://twitter.com/NYTOpinion}{\emph{Twitter (@NYTopinion)}}
\emph{and}
\href{https://www.instagram.com/nytopinion/}{\emph{Instagram}}\emph{.}

Advertisement

\protect\hyperlink{after-bottom}{Continue reading the main story}

\hypertarget{site-index}{%
\subsection{Site Index}\label{site-index}}

\hypertarget{site-information-navigation}{%
\subsection{Site Information
Navigation}\label{site-information-navigation}}

\begin{itemize}
\tightlist
\item
  \href{https://help.nytimes.com/hc/en-us/articles/115014792127-Copyright-notice}{©~2020~The
  New York Times Company}
\end{itemize}

\begin{itemize}
\tightlist
\item
  \href{https://www.nytco.com/}{NYTCo}
\item
  \href{https://help.nytimes.com/hc/en-us/articles/115015385887-Contact-Us}{Contact
  Us}
\item
  \href{https://www.nytco.com/careers/}{Work with us}
\item
  \href{https://nytmediakit.com/}{Advertise}
\item
  \href{http://www.tbrandstudio.com/}{T Brand Studio}
\item
  \href{https://www.nytimes.com/privacy/cookie-policy\#how-do-i-manage-trackers}{Your
  Ad Choices}
\item
  \href{https://www.nytimes.com/privacy}{Privacy}
\item
  \href{https://help.nytimes.com/hc/en-us/articles/115014893428-Terms-of-service}{Terms
  of Service}
\item
  \href{https://help.nytimes.com/hc/en-us/articles/115014893968-Terms-of-sale}{Terms
  of Sale}
\item
  \href{https://spiderbites.nytimes.com}{Site Map}
\item
  \href{https://help.nytimes.com/hc/en-us}{Help}
\item
  \href{https://www.nytimes.com/subscription?campaignId=37WXW}{Subscriptions}
\end{itemize}
