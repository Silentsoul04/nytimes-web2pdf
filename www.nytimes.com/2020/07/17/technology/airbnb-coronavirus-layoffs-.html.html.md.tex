Sections

SEARCH

\protect\hyperlink{site-content}{Skip to
content}\protect\hyperlink{site-index}{Skip to site index}

\href{/section/technology}{Technology}\textbar{}Airbnb Was Like a
Family, Until the Layoffs Started

\url{https://nyti.ms/2DM2IGR}

\begin{itemize}
\item
\item
\item
\item
\item
\item
\end{itemize}

\href{https://www.nytimes.com/news-event/coronavirus?action=click\&pgtype=Article\&state=default\&region=TOP_BANNER\&context=storylines_menu}{The
Coronavirus Outbreak}

\begin{itemize}
\tightlist
\item
  live\href{https://www.nytimes.com/2020/08/04/world/coronavirus-cases.html?action=click\&pgtype=Article\&state=default\&region=TOP_BANNER\&context=storylines_menu}{Latest
  Updates}
\item
  \href{https://www.nytimes.com/interactive/2020/us/coronavirus-us-cases.html?action=click\&pgtype=Article\&state=default\&region=TOP_BANNER\&context=storylines_menu}{Maps
  and Cases}
\item
  \href{https://www.nytimes.com/interactive/2020/science/coronavirus-vaccine-tracker.html?action=click\&pgtype=Article\&state=default\&region=TOP_BANNER\&context=storylines_menu}{Vaccine
  Tracker}
\item
  \href{https://www.nytimes.com/2020/08/02/us/covid-college-reopening.html?action=click\&pgtype=Article\&state=default\&region=TOP_BANNER\&context=storylines_menu}{College
  Reopening}
\item
  \href{https://www.nytimes.com/live/2020/08/04/business/stock-market-today-coronavirus?action=click\&pgtype=Article\&state=default\&region=TOP_BANNER\&context=storylines_menu}{Economy}
\end{itemize}

\includegraphics{https://static01.nyt.com/images/2020/07/19/business/00airbnb1/merlin_173400378_c8e2a75e-fc10-4a7b-83e3-0a448fe486e1-articleLarge.jpg?quality=75\&auto=webp\&disable=upscale}

\hypertarget{airbnb-was-like-a-family-until-the-layoffs-started}{%
\section{Airbnb Was Like a Family, Until the Layoffs
Started}\label{airbnb-was-like-a-family-until-the-layoffs-started}}

What happens when a kumbaya office culture meets the business realities
of a pandemic?

``I have a deep feeling of love for all of you,'' Brian Chesky, Airbnb's
chief executive, told employees in May.Credit...Jessica Chou for The New
York Times

Supported by

\protect\hyperlink{after-sponsor}{Continue reading the main story}

\href{https://www.nytimes.com/by/erin-griffith}{\includegraphics{https://static01.nyt.com/images/2019/06/18/reader-center/author-erin-griffith/author-erin-griffith-thumbLarge.png}}

By \href{https://www.nytimes.com/by/erin-griffith}{Erin Griffith}

\begin{itemize}
\item
  July 17, 2020
\item
  \begin{itemize}
  \item
  \item
  \item
  \item
  \item
  \item
  \end{itemize}
\end{itemize}

SAN FRANCISCO --- On May 5, after almost two months of working alone in
his San Francisco apartment, Brian Chesky, Airbnb's chief executive,
cried into his video camera.

It was a Tuesday, not that it mattered because the days had blurred
together, and Mr. Chesky was addressing thousands of his employees.
Looking into his webcam, he read from a script that he had written to
tell them that
\href{https://www.nytimes.com/news-event/coronavirus?action=click\&pgtype=Article\&state=default\&module=styln-coronavirus\&variant=show\&region=TOP_BANNER\&context=storylines_menu}{the
coronavirus} had crushed the travel industry,
\href{https://www.nytimes.com/2020/03/10/technology/airbnb-hosts-coronavirus.html}{including
their home rental start-up}. Divisions would have to be cut and workers
laid off.

``I have a deep feeling of love for all of you,'' Mr. Chesky said, his
voice cracking. ``What we are about is belonging, and at the center of
belonging is love.'' Within a few hours, 1,900 employees --- a quarter
of Airbnb's work force --- were told they were out.

The moves thrust Airbnb into the center of a growing debate in Silicon
Valley: What happens when a company that has positioned itself as family
to its employees reveals that it is just a regular business with the
same capitalist concerns --- namely, survival --- as any other?

Start-ups that sell everything from mattresses to data-warehousing
software have long used ``making the world a better place''-style
mission statements to energize and motivate their workers. But as the
economic fallout from the coronavirus persists, many of those gauzy
mantras have given way to harsh realities like
\href{https://www.nytimes.com/2020/04/01/technology/virus-start-ups-pummeled-layoffs-unwinding.html}{budget
cuts, layoffs and bottom lines}.

That now puts companies with a ``commitment'' culture at the highest
risk of losing what made them successful, said Ethan Mollick, an
entrepreneurship professor at the University of Pennsylvania's Wharton
School.

``Part of the compensation is being part of this family,'' Mr. Mollick
said. ``Now the family goes away, and the deal is sort of changed. It
just becomes a job.''

In many ways, Airbnb was the ideal example of a commitment culture
company. Founded by Mr. Chesky, Nathan Blecharczyk and Joe Gebbia in
2008, the start-up grew quickly as an online platform that helped
homeowners rent out rooms to travelers. Along the way to a \$31 billion
valuation, it built a reputation as the polar opposite of its sharing
economy peers such as Uber, which prized
\href{https://www.nytimes.com/2017/02/22/technology/uber-workplace-culture.html}{ruthless
competition}, and WeWork, which collapsed under a
\href{https://www.nytimes.com/2019/11/02/business/adam-neumann-wework-exit-package.html}{partying
culture} and its founder's self-dealing.

Instead, Airbnb stood for earnest idealism. Mr. Chesky, 38, a stocky
designer from upstate New York, spoke frequently of
\href{https://news.airbnb.com/in-the-business-of-trust/}{trustworthiness},
authenticity and a desire to build a business that valued
\href{https://news.airbnb.com/serving-all-stakeholders/}{principles and
people} over the short-termism of Wall Street. Mr. Gebbia delivered a
TED Talk on
\href{https://www.ted.com/talks/joe_gebbia_how_airbnb_designs_for_trust?language=en}{designing
for trust}. And Airbnb's former chief ethics officer, Rob Chesnut, wrote
a book called ``Intentional Integrity.''

\includegraphics{https://static01.nyt.com/images/2020/07/19/business/00airbnb4/merlin_47366927_22bfc94d-43d8-4a6c-a319-d08fb97c6d3a-articleLarge.jpg?quality=75\&auto=webp\&disable=upscale}

Inside the San Francisco company's airy, plant-filled offices, the
posivibes were also plentiful. Employees surprised one another by
raising their arms to form celebratory human tunnels, held dog
``pawties'' in conference rooms designed to look like actual Airbnb
listings and were serenaded on their birthdays by the company's a
cappella group, Airbnbeats. New employees, who were screened for empathy
in job interviews, were welcomed ``home'' and told: ``You belong here.''

So in March, when the coronavirus hurtled in, the rupturing of the
``Airfam'' was painful. Airbnb, which had been
\href{https://www.nytimes.com/2019/09/19/technology/airbnb-ipo-2020.html}{on
track to go public this year}, suddenly faced an avalanche of travel
cancellations. Revenue evaporated. Weeks later, Mr. Chesky announced the
layoffs and scaled back the company's ambitions.

``Everything that kind of could go wrong did go wrong,'' he said in an
interview. ``It felt like everything stopped working at the same time.''

\hypertarget{latest-updates-economy}{%
\section{\texorpdfstring{\href{https://www.nytimes.com/live/2020/08/04/business/stock-market-today-coronavirus?action=click\&pgtype=Article\&state=default\&region=MAIN_CONTENT_1\&context=storylines_live_updates}{Latest
Updates:
Economy}}{Latest Updates: Economy}}\label{latest-updates-economy}}

\href{https://www.nytimes.com/live/2020/08/04/business/stock-market-today-coronavirus?action=click\&pgtype=Article\&state=default\&region=MAIN_CONTENT_1\&context=storylines_live_updates\#fox-corporations-plunging-profit-is-cushioned-by-fox-news}{11h
ago}

\href{https://www.nytimes.com/live/2020/08/04/business/stock-market-today-coronavirus?action=click\&pgtype=Article\&state=default\&region=MAIN_CONTENT_1\&context=storylines_live_updates\#fox-corporations-plunging-profit-is-cushioned-by-fox-news}{Fox
Corporation's plunging profit is cushioned by Fox News.}

\href{https://www.nytimes.com/live/2020/08/04/business/stock-market-today-coronavirus?action=click\&pgtype=Article\&state=default\&region=MAIN_CONTENT_1\&context=storylines_live_updates\#trading-in-kodak-shares-comes-under-scrutiny}{11h
ago}

\href{https://www.nytimes.com/live/2020/08/04/business/stock-market-today-coronavirus?action=click\&pgtype=Article\&state=default\&region=MAIN_CONTENT_1\&context=storylines_live_updates\#trading-in-kodak-shares-comes-under-scrutiny}{Trading
in Kodak shares comes under scrutiny.}

\href{https://www.nytimes.com/live/2020/08/04/business/stock-market-today-coronavirus?action=click\&pgtype=Article\&state=default\&region=MAIN_CONTENT_1\&context=storylines_live_updates\#disney-lost-4-7-billion-last-quarter-but-its-newest-business-was-a-big-hit}{12h
ago}

\href{https://www.nytimes.com/live/2020/08/04/business/stock-market-today-coronavirus?action=click\&pgtype=Article\&state=default\&region=MAIN_CONTENT_1\&context=storylines_live_updates\#disney-lost-4-7-billion-last-quarter-but-its-newest-business-was-a-big-hit}{Disney
lost \$4.7 billion last quarter, but its newest business was a big hit.}

\href{https://www.nytimes.com/live/2020/08/04/business/stock-market-today-coronavirus?action=click\&pgtype=Article\&state=default\&region=MAIN_CONTENT_1\&context=storylines_live_updates}{See
more updates}

More live coverage:
\href{https://www.nytimes.com/2020/08/04/world/coronavirus-cases.html?action=click\&pgtype=Article\&state=default\&region=MAIN_CONTENT_1\&context=storylines_live_updates}{Global}

From the outside, Airbnb's commitment culture appeared intact. Mr.
Chesky's layoffs script, which was published on the company
\href{https://news.airbnb.com/a-message-from-co-founder-and-ceo-brian-chesky/}{blog},
got more than one million views and was praised as
\href{https://www.businessinsider.com/airbnb-ceo-brian-chesky-layoffs-show-respect-compassion-for-employees-2020-5}{compassionate},
\href{https://www.prnewsonline.com/airbnb-ceo-delivers-empathetic-transparent-message-regarding-layoffs/}{empathetic}
and a
``\href{https://www.inc.com/jason-aten/lessons-behind-airbnb-ceos-email-about-laying-off-1900-workers.html}{lesson
in leadership}.'' At a question-and-answer session about the job cuts
later, Mr. Chesky and his co-founders offered a standing ovation to the
employees they had let go. Clapping and heart emojis from audience
members filled the screen.

But more than a dozen current and former Airbnb employees, most of whom
declined to be identified because they had signed nondisparagement
agreements with the company, said in interviews that they had
experienced a sudden disillusionment when the carefully crafted
corporate idealism cracked.

Kaspian Clark, 38, who worked in customer support in Portland, Ore., for
around two years, said he had fully bought into Airbnb's mission and
felt denial and grief when he was let go.

``There are a lot of people who feel very betrayed by this,'' he said.
``I deeply hope that Airbnb is able to remain the thing that I believed
in.''

A company spokesman said it ``has been a difficult time for everyone.''
He added, ``The more than 5,000 people who work at Airbnb are incredibly
motivated and enthusiastic because they believe in our mission.''

In a
\href{https://podcasts.apple.com/us/podcast/brian-chesky-part-one-the-heros-journey/id1505392824?i=1000475093717}{podcast
interview} in May with Eric Ries, a fellow entrepreneur, Mr. Chesky
acknowledged a disconnect.

``How does a company whose mission is centered around belonging have to
tell thousands of people they can't be at the company anymore?'' he
said. ``It was a very, very difficult thing to face.''

\hypertarget{embrace-the-adventure-champion-the-mission}{%
\subsection{Embrace the adventure, champion the
mission}\label{embrace-the-adventure-champion-the-mission}}

Image

Inside Airbnb's headquarters, where workers are encouraged to embody the
company's core values, former employees said.Credit...Jason Henry for
The New York Times

Airbnb was built not on a genius technological innovation or a
meticulous business school PowerPoint, but on the idea that people might
trust one another enough to stay in strangers' houses. Basically, the
goodness of humanity.

Its network of home rentals quickly spread across the United States and
into almost every country. Airbnb raised more than \$3 billion in
venture capital and expanded into activities, luxury vacations,
experiments with flights and even a print magazine.

As the company grew, Mr. Chesky began talking of a world where digital
nomads healed divisions with in-person connections.

``I think in the future, people won't travel --- they'll just be
mobile,'' he predicted in 2013. ``People are going to be living a month
here, a few weeks there, four months somewhere else.'' Airbnb was not
just renting vacation homes, the idea went, it was building a
``\href{https://slate.com/business/2014/02/airbnb-gentrification-how-the-sharing-economy-drives-up-housing-prices.html}{United
Nations around the kitchen table}.''

His philosophy crystallized in 2018 when he
\href{https://news.airbnb.com/brian-cheskys-open-letter-to-the-airbnb-community-about-building-a-21st-century-company/}{presented
a plan} for something called ``stakeholder'' capitalism. In contrast to
Wall Street's focus on quarterly financial reports and daily stock
moves, Mr. Chesky aspired to a capitalism that had an ``infinite time
horizon'' and was good for society.

That philosophy imbued many areas of work for Airbnb employees. Part of
their performance reviews, for instance, were based on how well they
embodied the start-up's core values, three former employees said.
``Embrace the adventure'' was sometimes used to justify difficult
situations, they said, and ``champion the mission'' was code for putting
a positive spin on things. (A company spokesman disputed the
characterization.)

Airbnb's rental listings
\href{https://press.airbnb.com/wp-content/uploads/sites/4/2018/08/The-Airbnb-Story-Timeline-EN-GLOBAL.pdf}{grew
from 2,500 in 2009} to seven million this year. The company landed
funding from top venture firms including Andreessen Horowitz, Founders
Fund and Sequoia Capital. Its valuation, which topped \$2 billion in
2012, skyrocketed to \$31 billion by 2017. An
\href{https://www.nytimes.com/2019/09/20/technology/airbnb-employees-ipo-payouts.html}{initial
public offering this year} was set to make its executives, investors and
employees rich.

Enter the virus. As travel ground to a halt in March, Airbnb cut its
2020 revenue projection to less than half of the \$4.8 billion it hauled
in last year. Its I.P.O. filing, which Mr. Chesky had been tweaking with
ideas for stakeholder capitalism and planned to submit by late March,
went into a drawer.

Instead, Mr. Chesky said, he drew up a list of principles for operating
in the virus. They included being decisive and emerging ``on the right
side of history.''

He compared the situation to a fire. ``You're in a house, it's burning,
you have to put out the fire while getting the furniture out of the
house and also rebuilding the house,'' he said.

Mr. Chesky asked Airbnb's board of directors to call in to virtual
meetings every Sunday and set up a daily ``war room'' meeting with his
executive team. He said he had remained glued to his computer most days
till around midnight, occasionally baking chocolate chip cookies or
going on walks during calls.

There were stumbles. When guests wanted out of nonrefundable bookings
because the pandemic had forced them to change their plans, Airbnb
changed its policy to allow refunds. But the move outraged the company's
rental operators, who relied on the income. Mr. Chesky eventually
apologized for how Airbnb had communicated the decision.

``Was everything done perfectly? No,'' said Alfred Lin, an Airbnb board
member and investor at Sequoia Capital. ``It was about speed and being
directionally right.''

Airbnb soon cut \$800 million in marketing costs, dropped bonuses and
halved executive pay for six months. It also ended contracts with
roughly 490 full-time freelancers. With cancellations pouring in and
call centers closed because of the virus, Airbnb directed employees
across the company, including its recruiters, who had frozen hiring, to
assist customers. The backlog took weeks to get through.

In April, the company
\href{https://www.nytimes.com/2020/04/06/technology/airbnb-coronavirus-valuation.html}{raised}
\$1 billion in emergency funding, followed by another \$1 billion in
debt.

Then came the May 5 layoffs. To blunt the shock, Airbnb's severance
packages included three months of salary and a year of health benefits,
which was more generous than many other
\href{https://www.nytimes.com/2020/04/01/technology/virus-start-ups-pummeled-layoffs-unwinding.html}{start-ups}
doing layoffs.

Mr. Chesky has since described a ``second founding,'' in which Airbnb
will be more focused on its core home rental business. It will look
different, he said, with fewer customers booking international travel,
less flocking to crowded cities, more local trips and more long-term
stays.

\hypertarget{dissent-in-the-airfam}{%
\subsection{Dissent in the `Airfam'}\label{dissent-in-the-airfam}}

Image

Mr. Chesky, working from his home office,~was glued to his computer most
days till around midnight.Credit...Jessica Chou for The New York Times

Two days after the layoffs, the questions came thick and fast in the
employee Q. and A. inside Awedience, Airbnb's virtual meeting software,
according to five people who attended.

Some workers asked why there weren't furloughs or broader pay cuts
instead of layoffs. Others asked why certain groups had been chosen for
cuts and why the company couldn't trim more perks, like its budget for
renting office plants.

Mr. Chesky said the situation was too uncertain for furloughs and pay
cuts, calling those temporary measures. Layoffs were mapped to the
future business strategy, he added. A spokesman said the company spent
only a small amount on landscaping and related services.

One area hit by layoffs was Airbnb's safety team, which handles
situations like shootings and assaults at its rentals. When
\href{https://www.nytimes.com/2019/11/01/us/orinda-shooting.html}{a
fatal shooting at a party} in Orinda, Calif., made national headlines
last fall, the company banned unauthorized parties at rentals and
announced plans to confirm that all of its listings were what they
advertised.

In the employee Q. and A., Mr. Chesky reiterated past statements that
safety was a priority for the company. Workers piped up with written
heckles --- the equivalent of shouting in a crowded theater --- with
messages like ``Safety was never a priority!'' It was an unusual public
show of dissent.

Within a week of the layoffs, new safety cases had piled up, two people
with knowledge of the situation said. Airbnb asked some laid-off
employees to return temporarily to work through the cases, they said.
Workers on the regulatory response and payments teams were asked to come
back temporarily as well, they said.

An Airbnb spokesman said that the groups focused on user safety were the
same size as before the layoffs and that the company assessed its
support staffing levels daily. ``Brian has always made clear that safety
is our priority,'' he said.

During that time, Leonardo Baca, an information technology professional
who was laid off, joined colleagues to attend a virtual magic
performance presented by Airbnb Experience --- part of the company's
activities booking service, which had moved online because of the virus.
It was meant to be a team-building exercise but instead became a goodbye
party.

Some laid-off colleagues were devastated, Mr. Baca said, while those who
remained expressed dismay over why they had been spared. ``We don't know
why people were cut,'' he said. ``You lose a piece of the team.''

Later, on a Slack channel for former employees, some lamented that
Airbnb was gutting its culture, according to messages viewed by The New
York Times. In June, an Airbnb contractor who had recently been let go
wrote an
\href{https://www.wired.com/story/airbnb-quietly-fired-hundreds-of-contract-workers-im-one-of-them/}{editorial
for Wired} that quoted peers calling the company ``hypocritical'' for
its ``remarkably callous'' treatment of contract labor during the
pandemic.

An Airbnb spokesman said its contractors ``were more than contractors,
they were our teammates and friends.'' He said the company had provided
them two weeks of pay and other benefits.

Other issues bubbled up. In a chat room for female Airbnb employees
after the layoffs, one laid-off worker described three instances of
sexual harassment while at the company, saying that human resources was
unhelpful and that co-workers brushed it off, according to an image of
the conversation viewed by The Times. The latter, the person wrote,
``hurt the most.''

The company said it does not tolerate harassment and discrimination and
investigates all claims.

Last month, some employees in Airbnb's China division sent a letter to
management outlining what they said was inappropriate behavior by Yanxin
Shi, engineering director for its China business, according to one of
the employees responsible for the letter, which The Times viewed. They
alleged that Mr. Shi had ranked female colleagues by attractiveness and
had said he didn't believe in the company's ``core values'' but could
perform them well enough to pass the job interview and teach others to
do the same.

Airbnb said it had concluded that the letter's ``most serious
allegations'' were not supported and had taken ``appropriate action,''
but it did not specify what that was. Mr. Shi did not respond to a
request for comment. Skift earlier reported on the
\href{https://skift.com/2020/06/12/airbnb-investigates-allegations-of-sexual-harassment-against-a-top-chinese-exec/}{letter}.

Mr. Chesky said he remained optimistic. The company has been promoting
signs of recovery, like a growing number of bookings within driving
distance and adoption of its ``virtual experiences.'' In a virtual
meeting on Wednesday afternoon, Mr. Chesky told Airbnb workers that the
company would resume work on its
\href{https://www.nytimes.com/2020/07/15/technology/airbnb-ipo.html}{plans
to go public}.

He also reflected on the last four months, which he said had been
``traumatizing in some ways.'' The crisis showed him that Airbnb had
strayed from its roots as a place for people to connect, and he planned
to rectify that.

``Something we can never lose,'' Mr. Chesky said, ``is being true to
ourselves, being different, being special.''

Advertisement

\protect\hyperlink{after-bottom}{Continue reading the main story}

\hypertarget{site-index}{%
\subsection{Site Index}\label{site-index}}

\hypertarget{site-information-navigation}{%
\subsection{Site Information
Navigation}\label{site-information-navigation}}

\begin{itemize}
\tightlist
\item
  \href{https://help.nytimes.com/hc/en-us/articles/115014792127-Copyright-notice}{©~2020~The
  New York Times Company}
\end{itemize}

\begin{itemize}
\tightlist
\item
  \href{https://www.nytco.com/}{NYTCo}
\item
  \href{https://help.nytimes.com/hc/en-us/articles/115015385887-Contact-Us}{Contact
  Us}
\item
  \href{https://www.nytco.com/careers/}{Work with us}
\item
  \href{https://nytmediakit.com/}{Advertise}
\item
  \href{http://www.tbrandstudio.com/}{T Brand Studio}
\item
  \href{https://www.nytimes.com/privacy/cookie-policy\#how-do-i-manage-trackers}{Your
  Ad Choices}
\item
  \href{https://www.nytimes.com/privacy}{Privacy}
\item
  \href{https://help.nytimes.com/hc/en-us/articles/115014893428-Terms-of-service}{Terms
  of Service}
\item
  \href{https://help.nytimes.com/hc/en-us/articles/115014893968-Terms-of-sale}{Terms
  of Sale}
\item
  \href{https://spiderbites.nytimes.com}{Site Map}
\item
  \href{https://help.nytimes.com/hc/en-us}{Help}
\item
  \href{https://www.nytimes.com/subscription?campaignId=37WXW}{Subscriptions}
\end{itemize}
