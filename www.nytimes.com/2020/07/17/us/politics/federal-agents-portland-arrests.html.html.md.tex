Sections

SEARCH

\protect\hyperlink{site-content}{Skip to
content}\protect\hyperlink{site-index}{Skip to site index}

\href{https://www.nytimes.com/section/politics}{Politics}

\href{https://myaccount.nytimes.com/auth/login?response_type=cookie\&client_id=vi}{}

\href{https://www.nytimes.com/section/todayspaper}{Today's Paper}

\href{/section/politics}{Politics}\textbar{}Were the Actions of Federal
Agents in Portland Legal?

\url{https://nyti.ms/30lFbEp}

\begin{itemize}
\item
\item
\item
\item
\item
\end{itemize}

Advertisement

\protect\hyperlink{after-top}{Continue reading the main story}

Supported by

\protect\hyperlink{after-sponsor}{Continue reading the main story}

\hypertarget{were-the-actions-of-federal-agents-in-portland-legal}{%
\section{Were the Actions of Federal Agents in Portland
Legal?}\label{were-the-actions-of-federal-agents-in-portland-legal}}

The Department of Homeland Security can point to federal statutes
protecting property to justify the arrests of protesters in Portland,
Ore., but whether they stretched the law would be up to a judge.

\includegraphics{https://static01.nyt.com/images/2020/07/17/autossell/portland-v1-2/portland-v1-2-videoSixteenByNineJumbo1600.jpg}

\href{https://www.nytimes.com/by/zolan-kanno-youngs}{\includegraphics{https://static01.nyt.com/images/2019/12/13/reader-center/author-zolan-kanno-youngs/author-zolan-kanno-youngs-thumbLarge.png}}

By \href{https://www.nytimes.com/by/zolan-kanno-youngs}{Zolan
Kanno-Youngs}

\begin{itemize}
\item
  Published July 17, 2020Updated July 24, 2020
\item
  \begin{itemize}
  \item
  \item
  \item
  \item
  \item
  \end{itemize}
\end{itemize}

The Department of Homeland Security's
\href{https://www.nytimes.com/2020/07/20/us/politics/portland-federal-agents-trump.html}{deployment
of federal agents to Portland}, Ore., has shown the broad legal
authority an agency created to protect the United States from national
security threats has to crack down on American citizens.

After President Trump signed an executive order directing federal
agencies to send personnel to protect monuments, statues and federal
property during continuing protests against racism and police brutality,
the Department of Homeland Security formed ``rapid deployment teams.''
Those are made up of officers from Customs and Border Protection, the
Transportation Security Administration, the Coast Guard and Immigration
and Customs Enforcement who back up the Federal Protective Service,
which is already responsible for protecting federal property.

Videos showing federal agents using tear gas on protesters and
complaints that federal agents lacking insignia are pulling people from
the streets have raised questions over the legal authority that homeland
security officials have to crack down on citizens. In Portland, federal
agents have acted against the expressed opposition of the local
authorities.

But officials in Washington said they had clear authority. Customs and
Border Protection, which sent tactical border agents to Portland,
\href{https://www.law.cornell.edu/uscode/text/40/1315}{cited 40 U.S.
Code 1315}, which under the Homeland Security Act of 2002 gives the
department's secretary the power to deputize other federal agents to
assist the Federal Protective Service in protecting federal property,
such as the courthouse in Portland.

Those agents can carry firearms, arrest those accused of committing a
crime without a warrant and conduct investigations ``on and off the
property in question.''

``An interpretation of that authority so broadly seems to undermine all
the other careful checks and balances on D.H.S.'s power because the
officers' power is effectively limitless and all encompassing,'' said
Garrett Graff, a historian who studies the Department of Homeland
Security's history and development.

The department has justified the tactics of the federal agents in
Portland by pointing to
\href{https://www.dhs.gov/news/2020/07/16/acting-secretary-wolf-condemns-rampant-long-lasting-violence-portland}{dozens
of episodes}, including the defacement of federal property with
graffiti, the damaging of buildings with fireworks and the throwing of
rocks and bottles at officers.

Detaining demonstrators away from federal properties has also raised
questions. Former officials at the Department of Homeland Security said
it would normally only dispatch agents to assist with local incidents if
the state or municipal governments asked for help and deputized that
responsibility. In Portland, local leaders have done the opposite.

But the lack of any consent from local officials just means federal
agents cannot rely on state and local laws to justify the arrests.
Federal agents can still detain the demonstrators away from federal
property if they can assert probable cause that a federal crime was
violated, according to Peter Vincent, a former top lawyer with
Immigration and Customs Enforcement, which has also sent agents to
cities across the United States.

``Homeland security's authorities are so extraordinarily broad that they
can find federal laws that they are authorized to enforce across the
spectrum, so long as it has some national security, public safety, human
trafficking, criminal street gang conspiracy,'' Mr. Vincent said.

But civil rights lawyers and demonstrators have questioned whether the
department has used that authority to violate protesters' right to free
speech.

``What is happening now in Portland should concern everyone in the
United States,'' said Jann Carson, the interim executive director of the
American Civil Liberties Union of Oregon. ``Usually when we see people
in unmarked cars forcibly grab someone off the street, we call it
kidnapping. The actions of the militarized federal officers are flat-out
unconstitutional and will not go unanswered.''

The American Civil Liberties Union Foundation of Oregon on Friday
\href{https://aclu-or.org/sites/default/files/field_documents/07-17-20_-_second_amended_complaint_0.pdf}{also
sued the Department of Homeland Security and the Marshal's Service} for
``indiscriminately using tear gas, rubber bullets and acoustic
weapons.''

One demonstrator, Mark Pettibone, 29, said agents who were in camouflage
but lacked any insignia forced him into an unmarked van and did not tell
him why he was being arrested. Deploying agents without any
identification violates the protocols of police departments across the
United States.

Mark Morgan, the acting secretary of Customs and Border Protection, said
the agents did display signs that they were federal agents but withheld
their names for their own safety.

Advertisement

\protect\hyperlink{after-bottom}{Continue reading the main story}

\hypertarget{site-index}{%
\subsection{Site Index}\label{site-index}}

\hypertarget{site-information-navigation}{%
\subsection{Site Information
Navigation}\label{site-information-navigation}}

\begin{itemize}
\tightlist
\item
  \href{https://help.nytimes.com/hc/en-us/articles/115014792127-Copyright-notice}{©~2020~The
  New York Times Company}
\end{itemize}

\begin{itemize}
\tightlist
\item
  \href{https://www.nytco.com/}{NYTCo}
\item
  \href{https://help.nytimes.com/hc/en-us/articles/115015385887-Contact-Us}{Contact
  Us}
\item
  \href{https://www.nytco.com/careers/}{Work with us}
\item
  \href{https://nytmediakit.com/}{Advertise}
\item
  \href{http://www.tbrandstudio.com/}{T Brand Studio}
\item
  \href{https://www.nytimes.com/privacy/cookie-policy\#how-do-i-manage-trackers}{Your
  Ad Choices}
\item
  \href{https://www.nytimes.com/privacy}{Privacy}
\item
  \href{https://help.nytimes.com/hc/en-us/articles/115014893428-Terms-of-service}{Terms
  of Service}
\item
  \href{https://help.nytimes.com/hc/en-us/articles/115014893968-Terms-of-sale}{Terms
  of Sale}
\item
  \href{https://spiderbites.nytimes.com}{Site Map}
\item
  \href{https://help.nytimes.com/hc/en-us}{Help}
\item
  \href{https://www.nytimes.com/subscription?campaignId=37WXW}{Subscriptions}
\end{itemize}
