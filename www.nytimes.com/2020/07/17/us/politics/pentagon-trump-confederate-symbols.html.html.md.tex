Sections

SEARCH

\protect\hyperlink{site-content}{Skip to
content}\protect\hyperlink{site-index}{Skip to site index}

\href{https://www.nytimes.com/section/politics}{Politics}

\href{https://myaccount.nytimes.com/auth/login?response_type=cookie\&client_id=vi}{}

\href{https://www.nytimes.com/section/todayspaper}{Today's Paper}

\href{/section/politics}{Politics}\textbar{}Pentagon Sidesteps Trump to
Ban the Confederate Flag

\url{https://nyti.ms/32tZD8I}

\begin{itemize}
\item
\item
\item
\item
\item
\item
\end{itemize}

Advertisement

\protect\hyperlink{after-top}{Continue reading the main story}

Supported by

\protect\hyperlink{after-sponsor}{Continue reading the main story}

\hypertarget{pentagon-sidesteps-trump-to-ban-the-confederate-flag}{%
\section{Pentagon Sidesteps Trump to Ban the Confederate
Flag}\label{pentagon-sidesteps-trump-to-ban-the-confederate-flag}}

The defense secretary lists the types of flags that are allowed to
appear on bases worldwide. That flag does not fit.

\includegraphics{https://static01.nyt.com/images/2020/07/17/us/politics/17dc-military/merlin_174401388_dcce9b5b-fed3-4c62-b8b5-3fd0a0a219db-articleLarge.jpg?quality=75\&auto=webp\&disable=upscale}

\href{https://www.nytimes.com/by/helene-cooper}{\includegraphics{https://static01.nyt.com/images/2018/08/24/multimedia/author-helene-cooper/author-helene-cooper-thumbLarge.png}}

By \href{https://www.nytimes.com/by/helene-cooper}{Helene Cooper}

\begin{itemize}
\item
  July 17, 2020
\item
  \begin{itemize}
  \item
  \item
  \item
  \item
  \item
  \item
  \end{itemize}
\end{itemize}

WASHINGTON --- The Pentagon, without once mentioning the word
``Confederate,'' announced a policy on Friday that essentially banned
displays of the Confederate flag on military installations around the
world.

In a carefully worded memo that Defense Department officials said they
hoped would avoid igniting another
\href{https://www.nytimes.com/2020/07/06/us/politics/trump-bubba-wallace-nascar.html}{defense
of the flag from President Trump}, Defense Secretary Mark T. Esper
issued guidance that listed the types of flags that could be displayed
on military installations --- in barracks, on cars and on signs.

According to the guidance, appropriate flags include those of American
states and territories, military services and other countries that are
allies of the United States. The guidance never specifically says that
Confederate flags are banned, but they do not fit in any of the approved
categories --- and any such flags are prohibited.

``Problem solved --- we hope,'' one Defense Department official said on
Friday, speaking on the condition of anonymity so as not to anger Mr.
Trump.

That senior military leaders are contorting themselves to such an extent
shows the gap that has developed between the White House and the
\href{https://www.nytimes.com/2020/06/07/us/unrest-protests-minneapolis-ending.html}{movement
for racial justice} that has swept across the country since the killing
of George Floyd while in police custody in May in Minneapolis. As
protests ignited, senior Defense Department officials began grappling
with the legacy of racism in the military.

Gen. Mark A. Milley, the chairman of the Joint Chiefs of Staff, said
last week at a House hearing that the Pentagon needed to take ``a hard
look'' at changing the names of Army bases honoring Confederate officers
who had fought against the Union during the Civil War, explicitly laying
out a course that
\href{https://www.nytimes.com/2020/06/10/us/politics/trump-rejects-renaming-military-bases.html}{diverges
from his commander in chief}.

Ten Army bases that
\href{https://www.nytimes.com/2020/06/11/us/military-bases-confederates.html}{honor
Confederate generals} who fought to defend the slaveholding South have
been the focus of a growing movement for change. Mr. Trump, for his
part, has sided with those who want symbols of the Confederacy to remain
in place.

``The United States of America trained and deployed our HEROES on these
Hallowed Grounds, and won two World Wars,''
\href{https://twitter.com/realDonaldTrump/status/1270787974880526337}{Mr.
Trump wrote} in
\href{https://twitter.com/realDonaldTrump/status/1270787975719391233}{a
string} of
\href{https://twitter.com/realDonaldTrump/status/1270787978626052096}{Twitter
posts}. ``Therefore, my Administration will not even consider the
renaming of these Magnificent and Fabled Military Installations. Our
history as the Greatest Nation in the World will not be tampered with.
Respect our Military!''

In the hearing, General Milley echoed senior military leaders who also
wanted to remove the
\href{https://www.nytimes.com/2020/07/20/us/politics/congress-trump-confederate-base-names.html}{Confederate
symbols and base names}. ``There is no place in our armed forces for
manifestations or symbols of racism, bias or discrimination,'' he said.

Mr. Esper's memo on Friday did not address the issue of bases named
after Confederate generals; one senior military official said this week
that the Pentagon would wait until after the November election before
further raising the issue. But the memo goes after the many American
soldiers, Marines and airmen who display Confederate flags and other
symbols in their barracks and in parking lots on military installations.

``Flags are powerful symbols, particularly in the military community for
whom flags embody common mission, common histories, and the special,
timeless bond of warriors,'' Mr. Esper said in his memo, before quoting
former Justice John Paul Stevens that the United States flag ``is a
symbol of freedom, of equal opportunity, of religious tolerance, and of
good will for other peoples who share our aspirations.''

Mr. Esper added in his memo that ``the flags we fly must accord with the
military imperatives of good order and discipline, treating all our
people with dignity and respect, and rejecting divisive symbols.''

A Defense Department official said that the new directive meant that
Black Lives Matter and L.G.B.T.Q. flags would not be allowed, either.
The ban applies to public and shared spaces; troops and military
officials can display Confederate flags in areas deemed private and
personal, like lockers and single rooms.

``It's absolutely outrageous that Defense Secretary Mark Esper would ban
the Pride flag --- the very symbol of inclusion and diversity,'' said
Jennifer Dane, the interim executive director of the advocacy group
Modern Military Association of America. ``In what universe is it OK to
turn an opportunity to ban a racist symbol like the Confederate flag
into an opportunity to ban the symbol of diversity? This decision sends
an alarming message to L.G.B.T.Q. service members, their families, and
future recruits.''

Next week, senators will continue their own bipartisan push to strip
military bases of Confederate symbols, advancing an amendment to the
annual defense bill spearheaded by Senator Elizabeth Warren, Democrat of
Massachusetts, that would require the Pentagon to eliminate
\href{https://www.nytimes.com/2020/06/11/us/politics/senate-confederate-names-military-bases.html}{Confederate
names, monuments or symbols} from military assets in three years. The
House is expected to press ahead on a similar measure as lawmakers
consider their version of the military policy legislation.

Top Republican leaders in Congress have indicated they would broadly
support such measures. Senator Mitch McConnell, Republican of Kentucky
and the majority leader, said this week in an interview
\href{https://slack-redir.net/link?url=https\%3A\%2F\%2Fwww.wsj.com\%2Farticles\%2Fmitch-mcconnell-signals-limits-on-race-related-policy-changes-11594733555}{with
The Wall Street Journal} that he would not block the effort to rename
the bases despite Mr. Trump's pledge to veto the broader defense bill if
Ms. Warren's amendment was included. Representative Kevin McCarthy,
Republican of California and the House minority leader, told reporters
last month that he was ``not opposed'' to renaming the bases.

The Marine Corps this year banned the Confederate flag, and the Army was
moving to do the same until Mr. Esper intervened, saying that he wanted
to issue uniform guidance across all services.

Catie Edmondson contributed reporting.

Advertisement

\protect\hyperlink{after-bottom}{Continue reading the main story}

\hypertarget{site-index}{%
\subsection{Site Index}\label{site-index}}

\hypertarget{site-information-navigation}{%
\subsection{Site Information
Navigation}\label{site-information-navigation}}

\begin{itemize}
\tightlist
\item
  \href{https://help.nytimes.com/hc/en-us/articles/115014792127-Copyright-notice}{©~2020~The
  New York Times Company}
\end{itemize}

\begin{itemize}
\tightlist
\item
  \href{https://www.nytco.com/}{NYTCo}
\item
  \href{https://help.nytimes.com/hc/en-us/articles/115015385887-Contact-Us}{Contact
  Us}
\item
  \href{https://www.nytco.com/careers/}{Work with us}
\item
  \href{https://nytmediakit.com/}{Advertise}
\item
  \href{http://www.tbrandstudio.com/}{T Brand Studio}
\item
  \href{https://www.nytimes.com/privacy/cookie-policy\#how-do-i-manage-trackers}{Your
  Ad Choices}
\item
  \href{https://www.nytimes.com/privacy}{Privacy}
\item
  \href{https://help.nytimes.com/hc/en-us/articles/115014893428-Terms-of-service}{Terms
  of Service}
\item
  \href{https://help.nytimes.com/hc/en-us/articles/115014893968-Terms-of-sale}{Terms
  of Sale}
\item
  \href{https://spiderbites.nytimes.com}{Site Map}
\item
  \href{https://help.nytimes.com/hc/en-us}{Help}
\item
  \href{https://www.nytimes.com/subscription?campaignId=37WXW}{Subscriptions}
\end{itemize}
