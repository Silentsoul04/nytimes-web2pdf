Sections

SEARCH

\protect\hyperlink{site-content}{Skip to
content}\protect\hyperlink{site-index}{Skip to site index}

\href{https://www.nytimes.com/section/politics}{Politics}

\href{https://myaccount.nytimes.com/auth/login?response_type=cookie\&client_id=vi}{}

\href{https://www.nytimes.com/section/todayspaper}{Today's Paper}

\href{/section/politics}{Politics}\textbar{}F.B.I. Agent in Russia
Inquiry Saw Basis in Early 2017 to Doubt Dossier

\url{https://nyti.ms/2Bfju0o}

\begin{itemize}
\item
\item
\item
\item
\item
\end{itemize}

Advertisement

\protect\hyperlink{after-top}{Continue reading the main story}

Supported by

\protect\hyperlink{after-sponsor}{Continue reading the main story}

\hypertarget{fbi-agent-in-russia-inquiry-saw-basis-in-early-2017-to-doubt-dossier}{%
\section{F.B.I. Agent in Russia Inquiry Saw Basis in Early 2017 to Doubt
Dossier}\label{fbi-agent-in-russia-inquiry-saw-basis-in-early-2017-to-doubt-dossier}}

Newly declassified documents added more fodder for the continuing
political fight over an aspect of the Trump-Russia investigation.

\includegraphics{https://static01.nyt.com/images/2020/07/17/us/politics/17dc-fbi1/merlin_141183813_22c27433-e3da-4f5b-b7f4-fe0ccf0b9b6f-articleLarge.jpg?quality=75\&auto=webp\&disable=upscale}

\href{https://www.nytimes.com/by/charlie-savage}{\includegraphics{https://static01.nyt.com/images/2018/06/12/multimedia/author-charlie-savage/author-charlie-savage-thumbLarge-v2.png}}\href{https://www.nytimes.com/by/adam-goldman}{\includegraphics{https://static01.nyt.com/images/2018/07/12/multimedia/author-adam-goldman/author-adam-goldman-thumbLarge.png}}

By \href{https://www.nytimes.com/by/charlie-savage}{Charlie Savage} and
\href{https://www.nytimes.com/by/adam-goldman}{Adam Goldman}

\begin{itemize}
\item
  July 17, 2020
\item
  \begin{itemize}
  \item
  \item
  \item
  \item
  \item
  \end{itemize}
\end{itemize}

WASHINGTON --- A top F.B.I. agent recognized by February 2017 that a now
notorious dossier of claims about purported Trump-Russia ties had
credibility problems, but the Justice Department continued to rely on it
as part of its basis to renew permission to wiretap a former Trump
campaign adviser, documents released on Friday showed.

The documents included
\href{https://www.judiciary.senate.gov/download/february-9-2017-electronic-communication}{an
F.B.I. memo} recounting a three-day interview in January 2017 with a
person who served as a primary source for Christopher Steele, a former
British intelligence officer who compiled the dossier for a research
firm paid by Democrats. They also included an F.B.I. agent's
\href{https://www.judiciary.senate.gov/imo/media/doc/Annotated\%20New\%20York\%20Times\%20Article.pdf}{notes
disputing aspects} of
\href{https://www.nytimes.com/2017/02/14/us/politics/russia-intelligence-communications-trump.html}{a
New York Times article} the next month.

The agent, Peter Strzok, had not participated in the interview of Mr.
Steele's source, in which the source had suggested that the dossier
misstated or exaggerated certain information that the source had
gathered from a network of contacts in Russia and relayed to Mr. Steele.
But Mr. Strzok appeared to be aware of aspects of it.

In his annotations about two weeks later, Mr. Strzok questioned the
reliability of the dossier.

Reacting to a line in the newspaper article that senior F.B.I. officials
believed that Mr. Steele had a credible track record, Mr. Strzok wrote
in the margins: ``Recent interviews and investigation, however, reveal
Steele may not be in a position to judge the reliability of his
subsource network.''

Nevertheless, in the ensuing months, the Justice Department twice sought
and obtained a court's permission to renew a wiretap of the former Trump
campaign adviser, Carter Page, recycling language from earlier
applications that relied in part on information from the Steele dossier.

An inspector general report last year
\href{https://www.nytimes.com/2019/12/11/us/politics/fisa-surveillance-fbi.html}{sharply
criticized the F.B.I.} for not telling judges that the interview had
raised doubts about the credibility of the Steele information. The
bureau has since
\href{https://www.nytimes.com/2020/01/23/us/politics/carter-page-fbi-surveillance.html}{conceded
to the court} that oversees national security surveillance that the
available evidence about Mr. Page was legally insufficient to justify
the last two wiretaps.

The documents were released on Friday by Senator Lindsey Graham,
Republican of South Carolina and the chairman of the Judiciary
Committee. A close ally to President Trump, Mr. Graham has been
\href{https://www.nytimes.com/2020/06/11/us/politics/republicans-subpoena-russia-inquiry.html}{using
his position to try to discredit the Russia inquiry in an election
year.}

In a statement announcing the release of the documents, Mr. Graham
called the F.B.I.'s investigation into the Trump campaign ``corrupt.''
An accompanying
\href{https://www.judiciary.senate.gov/press/rep/releases/judiciary-committee-releases-declassified-documents-that-substantially-undercut-steele-dossier-page-fisa-warrants}{news
release} from his office said that ``the document demonstrates that
Peter Strzok and others in F.B.I. leadership positions must have been
aware of the issues with the Steele dossier that the F.B.I.'s interview
with Steele's `primary subsource' revealed.''

``Senator Graham's statement represents another attempt by President
Trump's congressional lackeys to use Pete's work product to paint the
Russia investigation as a political witch hunt,'' Aitan Goelman, a
lawyer for Mr. Strzok, said in a statement. He described Mr. Strzok's
notes as ``nothing more than a dedicated counterintelligence
professional diligently vetting public reports of intelligence
information.''

While Mr. Strzok was still working on other aspects of the larger Russia
investigation, he was not part of the team working on the wiretap
renewals, his lawyer said. Another senior F.B.I. counterintelligence
official, Jennifer Boone, was supervising a team in charge of
determining the sources of information for the dossier and of handling
the wiretap targeting Mr. Page, according to people familiar with the
investigation.

Mr. Strzok was later removed from the Russia investigation after the
Justice Department inspector general discovered numerous texts on his
work phone expressing animus toward the election of Mr. Trump. The
inspector general, however, did not find evidence that he took or
withheld any official action because of his personal opinions.

Mr. Strzok's skeptical annotations of the
\href{https://www.nytimes.com/2017/02/14/us/politics/russia-intelligence-communications-trump.html}{Times
article}, headlined ``Trump Campaign Aides Had Repeated Contacts With
Russian Intelligence,'' were similar to congressional testimony months
later by the former F.B.I. director James B. Comey
\href{https://www.nytimes.com/2017/06/08/us/politics/james-comey-new-york-times-article-russia.html}{disputing
it.} Mr. Comey did not say exactly what he thought was incorrect about
the article, which cited four current and former American officials who
spoke on the condition of anonymity to discuss classified information.

Mr. Strzok's annotations disputed the article's premise and other
aspects. He wrote, ``We are unaware of ANY Trump advisers engaging in
conversations with Russian intelligence officials.''

Still, he also added, the bureau had identified contacts between
\href{https://www.nytimes.com/2017/04/04/us/politics/carter-page-trump-russia.html}{Mr.
Page and Russian intelligence officials} before the campaign; contacts
between an associate of Paul Manafort, the onetime campaign chairman,
and
\href{https://www.nytimes.com/2019/02/23/us/politics/konstantin-kilimnik-russia.html}{Russian
intelligence}; and contacts between two
\href{https://www.nytimes.com/2017/06/12/us/politics/sessions-is-likely-to-be-grilled-on-reports-of-meeting-with-russian-envoy.html}{campaign
advisers}, Jeff Sessions and Michael T. Flynn, and
\href{https://www.nytimes.com/2020/05/29/us/politics/flynn-russian-ambassador-transcripts.html}{Russia's
ambassador to the United States.}

Eileen Murphy, a Times spokeswoman, said, ``We stand by our reporting.''

The wiretapping of Mr. Page was a small part of the overall
investigation into Russia's covert attempt to help tilt the election in
Mr. Trump's favor and whether any Trump campaign affiliates had
conspired in that effort. The inspector general report found that
\href{https://www.nytimes.com/2019/12/09/us/politics/fbi-ig-report-russia-investigation.html}{the
opening of the investigation met legal standards and that the Steele
dossier had played no role in that decision}; the agents working on it
did not learn of its existence until later.

Still, the inspector general report's uncovering of serious flaws in the
wiretap applications --- including numerous errors and omissions, among
them the failure to alert the court to the doubts raised by the
interview of Mr. Steele's source --- has made them a political focus.

The report eventually issued by Robert S. Mueller III, the special
counsel who later took over the investigation, did not rely on
information from the dossier. It laid out how Russians hacked Democratic
emails and sought to covertly sow discord on American social media.
While it also found that the Russian government wanted Mr. Trump to win,
and that the Trump campaign welcomed the interference and expected to
benefit from it, it did not find sufficient evidence to establish any
criminal conspiracy between Trump campaign associates and Russia.

Advertisement

\protect\hyperlink{after-bottom}{Continue reading the main story}

\hypertarget{site-index}{%
\subsection{Site Index}\label{site-index}}

\hypertarget{site-information-navigation}{%
\subsection{Site Information
Navigation}\label{site-information-navigation}}

\begin{itemize}
\tightlist
\item
  \href{https://help.nytimes.com/hc/en-us/articles/115014792127-Copyright-notice}{©~2020~The
  New York Times Company}
\end{itemize}

\begin{itemize}
\tightlist
\item
  \href{https://www.nytco.com/}{NYTCo}
\item
  \href{https://help.nytimes.com/hc/en-us/articles/115015385887-Contact-Us}{Contact
  Us}
\item
  \href{https://www.nytco.com/careers/}{Work with us}
\item
  \href{https://nytmediakit.com/}{Advertise}
\item
  \href{http://www.tbrandstudio.com/}{T Brand Studio}
\item
  \href{https://www.nytimes.com/privacy/cookie-policy\#how-do-i-manage-trackers}{Your
  Ad Choices}
\item
  \href{https://www.nytimes.com/privacy}{Privacy}
\item
  \href{https://help.nytimes.com/hc/en-us/articles/115014893428-Terms-of-service}{Terms
  of Service}
\item
  \href{https://help.nytimes.com/hc/en-us/articles/115014893968-Terms-of-sale}{Terms
  of Sale}
\item
  \href{https://spiderbites.nytimes.com}{Site Map}
\item
  \href{https://help.nytimes.com/hc/en-us}{Help}
\item
  \href{https://www.nytimes.com/subscription?campaignId=37WXW}{Subscriptions}
\end{itemize}
