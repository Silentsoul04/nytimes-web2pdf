Sections

SEARCH

\protect\hyperlink{site-content}{Skip to
content}\protect\hyperlink{site-index}{Skip to site index}

\href{https://www.nytimes.com/section/arts}{Arts}

\href{https://myaccount.nytimes.com/auth/login?response_type=cookie\&client_id=vi}{}

\href{https://www.nytimes.com/section/todayspaper}{Today's Paper}

\href{/section/arts}{Arts}\textbar{}Brigid Berlin, Socialite Who Joined
Warhol's World, Dies at 80

\url{https://nyti.ms/3h7LWQW}

\begin{itemize}
\item
\item
\item
\item
\item
\end{itemize}

Advertisement

\protect\hyperlink{after-top}{Continue reading the main story}

Supported by

\protect\hyperlink{after-sponsor}{Continue reading the main story}

\hypertarget{brigid-berlin-socialite-who-joined-warhols-world-dies-at-80}{%
\section{Brigid Berlin, Socialite Who Joined Warhol's World, Dies at
80}\label{brigid-berlin-socialite-who-joined-warhols-world-dies-at-80}}

Born into privilege, she rejected her upbringing to become a fixture on
the New York underground art scene of the 1960s and '70s.

\href{https://www.nytimes.com/by/john-leland}{\includegraphics{https://static01.nyt.com/images/2018/02/20/multimedia/author-john-leland/author-john-leland-thumbLarge.jpg}}

By \href{https://www.nytimes.com/by/john-leland}{John Leland}

\begin{itemize}
\item
  July 18, 2020
\item
  \begin{itemize}
  \item
  \item
  \item
  \item
  \item
  \end{itemize}
\end{itemize}

\includegraphics{https://static01.nyt.com/images/2020/07/20/obituaries/20Berlin-obit4/17Berlin-articleLarge.jpg?quality=75\&auto=webp\&disable=upscale}

Here is one way to make a first impression.

``A car pulled up outside my apartment and out came Brigid, topless,
wearing a red sarong around her waist and a toy stethoscope around her
neck, and carrying a fake alligator doctor's bag filled with
amphetamines and a giant syringe. She came up and chased me around the
room trying to poke me with the needle.

``We became everyday friends.''

That was how
\href{https://www.theguardian.com/film/2016/oct/07/danny-fields-documentary-film-danny-says-ramones}{Danny
Fields}, a member of Andy Warhol's inner circle, remembered meeting
Brigid Berlin, a former socialite who became one of Warhol's closest
friends and a figure in the New York art world of the 1960s and '70s.

She acted in Warhol films. She recorded the Velvet Underground. She
befriended Robert Rauschenberg, Jasper Johns, John Chamberlain and Larry
Rivers, who all embraced her as a fellow artist, even as she rejected
the label for herself.

``As near as you can get to the genesis of the art of the '60s,'' Mr.
Fields said in an interview, ``she was there.''

Ms. Berlin died on Friday at a hospital in Manhattan, after several
years of health problems that largely confined her to her bed. She was
80.

The cause was cardiac arrest brought on by a pulmonary embolism, said
Rob Vaczy, a close friend and neighbor.

In a New York scene filled with large, self-invented characters, Ms.
Berlin --- also known as Brigid Polk, because she liked to administer
amphetamine injections, or pokes, to herself and others --- was a
runaway freight train, oversize both physically (she once weighed close
to 300 pounds) and in her personality, which alternately terrorized and
delighted people.

``I was scared of her,'' the filmmaker John Waters wrote about meeting
her, ``in the best way.''

The daughter of Richard E. Berlin, who ran the Hearst publishing empire
for more than 30 years, and Muriel Berlin, an uptown socialite known as
Honey, Ms. Berlin flamboyantly celebrated everything her parents
opposed, making art out of her naked body and selling family artifacts
to buy drugs.

She took mountains of speed and made thousands of recordings and
Polaroid photographs of New York's bohemian demimonde, when such a thing
existed. She made prints of her breasts and of well-known artists'
penises. She once rejected a Christmas gift of a Warhol painting, saying
she would rather have a washer-dryer.

Perhaps her most radical act, late in life, was to become a near replica
of her mother, with a similar apartment, identical pug dogs and
conservative political views.

``Brigid was a force and liked to fight,'' her younger brother and only
immediate survivor, Richard, said in an interview. ``She was
complicated, but she was a hell of a lot of fun.''

Brigid Emmett Berlin, the oldest of four children, was born in Manhattan
on Sept. 6, 1939. She grew up on Fifth Avenue in a home frequented by
political figures, including Presidents Lyndon B. Johnson and Richard M.
Nixon and the F.B.I. director J. Edgar Hoover.

``Brigid remembered once sitting in the back of our Town Car between Roy
Cohn and Joseph McCarthy, with my dad in the jump seat,'' Richard Berlin
said. ``I thought everyone had houseguests like that.''

Her mother was severely critical of Brigid's weight and sent her to Dr.
Max Jacobson, a Kennedy family physician known as Dr. Feelgood because
of his liberal administration of amphetamines. Thus began Ms. Berlin's
twinned obsession --- with speed and with her body --- which she
channeled into her recreational life and her art, appearing nude in
photographs and making prints of her breasts by rolling paint on them
and pressing them onto paper.

She was kicked out of several Roman Catholic high schools and numerous
colleges. Between Brigid and her sister Richie, who also joined the
Warhol set, ``they went to 17 different colleges,'' her brother said.
``Once my parents sent Brigid to a convent in Spain, thinking, `These
Spanish nuns will knock her into shape.' But they were no match for
Brigid.''

At 19, she married John Parker, a window dresser who was gay, in large
part to shock her parents. According to Warhol, her mother gave her a
wedding present of \$100 and told her to buy some new underwear, adding,
``Good luck with that fairy.'' The two ran through much of her money and
shortly divorced.

She found a better match in Warhol, with whom she shared a special bond,
said Pat Hackett, who edited Warhol's diaries and co-wrote some of his
books.

Image

Ms. Berlin with Andy Warhol in a scene from the 2000 documentary ``Pie
in the Sky: The Brigid Berlin Story.'' ``Andy enjoyed Brigid more than
anyone else I can think of,'' a fellow Warhol associate
said.Credit...Vincent Fremont Enterprise

``Andy enjoyed Brigid more than anyone else I can think of,'' Ms.
Hackett said. ``He was fascinated by the Berlin family. He and Brigid
loved each other. Andy used to say, `If you ever want to learn what's
wrong with you, don't look in the mirror; give Brigid a glass of wine
and she'll tell you.'''

From her tiny room in the old George Washington Hotel near Gramercy
Park, Ms. Berlin recorded telephone conversations with everyone,
meticulously labeling and storing the thousands of cassettes in
alphabetical and chronological order. Her Polaroid photographs of people
on the scene, and of herself, were equally voluminous.

``Her mind-set was that she was collecting people,'' said
\href{https://www.interviewmagazine.com/art/gerard-malanga-adam-kimmel}{Gerard
Malanga,} a poet and Warhol collaborator who sometimes helped Ms. Berlin
get artists to pose for her. ``If you said she was an artist, she'd say
she wasn't. It seemed she didn't take herself seriously --- but
ironically she did take herself seriously, but didn't go out and promote
herself until the last few years.''

In the late 1960s, Ms. Berlin presented a running performance called
``\href{http://www.andypresentsnothingspecial.com/interviews/brigid-berlin.html\#:~:text=After\%20moving\%20to\%20the\%20Chelsea,Her\%20Satanic\%20Majesty\%20in\%20Person.}{Brigid
Polk Strikes! Her Satanic Majesty in Person,}'' in which she called
people on the telephone and --- unbeknown to them --- amplified the
conversation for the audience. At one performance, she told a friend she
needed \$100 for an abortion, then left the nightclub and returned with
the money. At another, Mr. Fields recalled, she called her mother and
got into an argument, then asked the audience, ``You see, did I ever
exaggerate what a monster my mother was?''

Warhol cast her in his movies ``Bad'' and ``Chelsea Girls.'' Her mother
went to see ``Chelsea Girls''
\href{https://www.nytimes.com/1996/03/07/garden/warhol-star-is-back-for-15-more-minutes.html}{incognito}
and was appalled to see her daughter shoot amphetamines into her
backside during a monologue. As a prank, Ms. Berlin and Warhol once told
a reporter that she actually created his paintings; their value
instantly fell.

In a preface to her book
``\href{https://www.reelartpress.com/catalog/edition/82/brigid-berlin-polaroids}{Brigid
Berlin Polaroids},'' Mr. Waters, who cast her in several of his movies
after Warhol's death, called her ``old money combined with danger'' and
``my favorite underground movie star; big, often naked and ornery as
hell.''

Ms. Berlin befriended the mostly heterosexual, hard-drinking Abstract
Expressionist painters who held court in the front room at Max's Kansas
City and the more druggy, gayer Warhol crowd in the back room. She
recorded Lou Reed's last performance with the Velvet Underground at the
club, which was released as the album
``\href{https://www.rollingstone.com/music/music-album-reviews/live-at-maxs-kansas-city-94148/}{Live
at Max's Kansas City}.''

``She was wild,''
\href{https://www.nytimes.com/2017/09/27/fashion/viva-andy-warhol.html}{Viva
Hoffmann}, another of Warhol's actresses, said in an interview. ``Brigid
came to my family's house in the Thousand Islands and water-skied naked
in the middle channel of the St. Lawrence Waterway. People still talk
about that.''

Image

Ms. Berlin in about 1967. Credit...Digne Meller Marcovicz/Vincent
Fremont Enterprises

In later years, Ms. Berlin gave up amphetamines and mounted serious
exhibitions of her photographs, which now had an added veneer of art
history. She also made elaborate
\href{https://glennhorowitz.com/events/brigid-berlin-needlepoint/}{needlepoint}
pillows replicating tabloid front pages and exhibited them in galleries.
Three of her books of penis art --- sometimes artists' drawings of their
own anatomy that she solicited, sometimes ink prints --- sold to the
painter Richard Prince for \$175,000.

She moved from the hotel room to a proper
\href{https://www.nytimes.com/2001/04/22/movies/film-a-warholian-finds-a-new-center-of-gravity.html}{apartment}
in 1986 and gradually began dressing like her socialite mother. She
cleaned obsessively, then cleaned some more. The conceptual artist
Richard Dupont, a Warhol associate who lived with her in the mid-2000s,
remembered her rarely leaving the apartment except to buy yarn.

``She said, `I only go out on the phone,''' Mr. Dupont said.

Mr. Vaczy, her closest companion, said she would perform ad hoc
monologues in restaurants or bars when the musicians were on a break,
talking about her childhood, Alcoholics Anonymous meetings, politics,
whatever was on her mind. She watched Fox News a lot, friends said, and
was heartened by the election of Donald J. Trump.

``She was underappreciated as an artist,'' said Vincent Fremont, a
longtime friend and Warhol associate who made a documentary about her,
``Pie in the Sky: The Brigid Berlin Story,'' in 2000.

``She could've been anything she wanted to be. She was an incredibly
good conceptual thinker, and that's why Andy appreciated her. She was
obsessive, but had an incredible imagination. She could really take you
down with her mouth.''

He added, ``She didn't do anything halfway.''

Advertisement

\protect\hyperlink{after-bottom}{Continue reading the main story}

\hypertarget{site-index}{%
\subsection{Site Index}\label{site-index}}

\hypertarget{site-information-navigation}{%
\subsection{Site Information
Navigation}\label{site-information-navigation}}

\begin{itemize}
\tightlist
\item
  \href{https://help.nytimes.com/hc/en-us/articles/115014792127-Copyright-notice}{©~2020~The
  New York Times Company}
\end{itemize}

\begin{itemize}
\tightlist
\item
  \href{https://www.nytco.com/}{NYTCo}
\item
  \href{https://help.nytimes.com/hc/en-us/articles/115015385887-Contact-Us}{Contact
  Us}
\item
  \href{https://www.nytco.com/careers/}{Work with us}
\item
  \href{https://nytmediakit.com/}{Advertise}
\item
  \href{http://www.tbrandstudio.com/}{T Brand Studio}
\item
  \href{https://www.nytimes.com/privacy/cookie-policy\#how-do-i-manage-trackers}{Your
  Ad Choices}
\item
  \href{https://www.nytimes.com/privacy}{Privacy}
\item
  \href{https://help.nytimes.com/hc/en-us/articles/115014893428-Terms-of-service}{Terms
  of Service}
\item
  \href{https://help.nytimes.com/hc/en-us/articles/115014893968-Terms-of-sale}{Terms
  of Sale}
\item
  \href{https://spiderbites.nytimes.com}{Site Map}
\item
  \href{https://help.nytimes.com/hc/en-us}{Help}
\item
  \href{https://www.nytimes.com/subscription?campaignId=37WXW}{Subscriptions}
\end{itemize}
