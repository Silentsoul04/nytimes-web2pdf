Sections

SEARCH

\protect\hyperlink{site-content}{Skip to
content}\protect\hyperlink{site-index}{Skip to site index}

\href{https://www.nytimes.com/section/realestate}{Real Estate}

\href{https://myaccount.nytimes.com/auth/login?response_type=cookie\&client_id=vi}{}

\href{https://www.nytimes.com/section/todayspaper}{Today's Paper}

\href{/section/realestate}{Real Estate}\textbar{}Can I Stop Paying for
Amenities That Closed Because of the Pandemic?

\url{https://nyti.ms/3hge4kO}

\begin{itemize}
\item
\item
\item
\item
\item
\item
\end{itemize}

\href{https://www.nytimes.com/spotlight/at-home?action=click\&pgtype=Article\&state=default\&region=TOP_BANNER\&context=at_home_menu}{At
Home}

\begin{itemize}
\tightlist
\item
  \href{https://www.nytimes.com/2020/08/03/well/family/the-benefits-of-talking-to-strangers.html?action=click\&pgtype=Article\&state=default\&region=TOP_BANNER\&context=at_home_menu}{Talk:
  To Strangers}
\item
  \href{https://www.nytimes.com/2020/08/01/at-home/coronavirus-make-pizza-on-a-grill.html?action=click\&pgtype=Article\&state=default\&region=TOP_BANNER\&context=at_home_menu}{Make:
  Grilled Pizza}
\item
  \href{https://www.nytimes.com/2020/07/31/arts/television/goldbergs-abc-stream.html?action=click\&pgtype=Article\&state=default\&region=TOP_BANNER\&context=at_home_menu}{Watch:
  'The Goldbergs'}
\item
  \href{https://www.nytimes.com/interactive/2020/at-home/even-more-reporters-editors-diaries-lists-recommendations.html?action=click\&pgtype=Article\&state=default\&region=TOP_BANNER\&context=at_home_menu}{Explore:
  Reporters' Google Docs}
\end{itemize}

Advertisement

\protect\hyperlink{after-top}{Continue reading the main story}

Supported by

\protect\hyperlink{after-sponsor}{Continue reading the main story}

Ask Real Estate

\hypertarget{can-i-stop-paying-for-amenities-that-closed-because-of-the-pandemic}{%
\section{Can I Stop Paying for Amenities That Closed Because of the
Pandemic?}\label{can-i-stop-paying-for-amenities-that-closed-because-of-the-pandemic}}

Gyms, playrooms and other commons spaces may be off limits, but that may
not relieve your financial obligations.

\includegraphics{https://static01.nyt.com/images/2020/07/19/realestate/19ask/18ask-articleLarge.jpg?quality=75\&auto=webp\&disable=upscale}

\href{https://www.nytimes.com/by/ronda-kaysen}{\includegraphics{https://static01.nyt.com/images/2018/07/16/multimedia/author-ronda-kaysen/author-ronda-kaysen-thumbLarge-v2.png}}

By \href{https://www.nytimes.com/by/ronda-kaysen}{Ronda Kaysen}

\begin{itemize}
\item
  July 18, 2020
\item
  \begin{itemize}
  \item
  \item
  \item
  \item
  \item
  \item
  \end{itemize}
\end{itemize}

\textbf{Q: The gym, billiards room and children's playroom in our
apartment building have been closed for months, with no prospect of
reopening anytime soon. We paid a premium for this apartment partly
because of the amenity package. Now, we can't use it. Should we get a
credit for loss of services?}

\textbf{A:} New Yorkers who own or rent apartments in buildings with
fitness centers and other high-contact indoor amenities are facing the
reality that those spaces will be closed indefinitely. Gyms are not
included in Phase 4 of the state's reopening plan. (The city entered
Phase 3 on July 6.)

Despite the long-term closures, most buildings are not offering relief
for residents. You could try to get a credit, but your options are
limited and depend on what kind of building you live in.

\hypertarget{in-a-rental}{%
\subsubsection{\texorpdfstring{\textbf{In a
Rental}}{In a Rental}}\label{in-a-rental}}

If you pay a monthly fee, you could stop --- but it's risky, according
to Samuel J. Himmelstein, a Manhattan lawyer who represents tenants.
Before you do, check your lease to see if the terms allow it, because
your landlord could potentially sue you in housing court for nonpayment
of rent. If your amenity costs are included, do not deduct them from
your rent, because doing so could also land you in housing court.

Instead, you are better off paying your rent in full and then trying to
recoup the money after successfully navigating legal channels. A
market-rate tenant could sue the landlord in small-claims court for
breach of contract. And a rent-stabilized tenant could file a complaint
with the
state's\href{https://hcr.ny.gov/division-housing-and-community-renewal}{Division
of Housing and Community Renewal,} which oversees such apartments, and
potentially be awarded a rent abatement for a reduction of services.

\hypertarget{in-a-condo-or-co-op}{%
\subsubsection{\texorpdfstring{\textbf{In a Condo or
Co-op}}{In a Condo or Co-op}}\label{in-a-condo-or-co-op}}

In most condos and co-ops, the cost of amenities is baked into the
common charges or maintenance fees. Buildings are not likely to be
offering residents discounts or credits because they too may be under
considerable financial strain. Even if your building is saving money
with the gym closed, other costs have likely risen to cover expenses
like personal protective equipment, extra cleaning supplies and overtime
pay for staff.

``Even if the board was willing to provide some kind of credit on
everyone's monthly statement, it would just get collected later on with
a higher charge next year or an assessment to make up any budget
shortfall,'' said Lisa A. Smith, a real estate lawyer and a partner in
the Manhattan office of the law firm Smith, Gambrell \& Russell.

If you pay for your amenities separately, you could withhold that
line-item fee --- but still pay your monthly charges for the building
--- and see if the board pushes back. ``It will be an easier discussion
if the dispute is only over the amenity fees and not the common charges
themselves,'' Ms. Smith said.

The good news for owners is that those amenities will eventually be
desirable again, so that premium isn't lost. Renters with market-rate
leases, on the other hand, may want to try to renegotiate when their
lease expires, arguing that they lost the benefits of the building.

For weekly email updates on residential real estate news,
\href{http://www.nytimes.com/newsletters/realestate/}{sign up here}.
Follow us on Twitter:
\href{https://twitter.com/nytrealestate}{@nytrealestate}.

Advertisement

\protect\hyperlink{after-bottom}{Continue reading the main story}

\hypertarget{site-index}{%
\subsection{Site Index}\label{site-index}}

\hypertarget{site-information-navigation}{%
\subsection{Site Information
Navigation}\label{site-information-navigation}}

\begin{itemize}
\tightlist
\item
  \href{https://help.nytimes.com/hc/en-us/articles/115014792127-Copyright-notice}{©~2020~The
  New York Times Company}
\end{itemize}

\begin{itemize}
\tightlist
\item
  \href{https://www.nytco.com/}{NYTCo}
\item
  \href{https://help.nytimes.com/hc/en-us/articles/115015385887-Contact-Us}{Contact
  Us}
\item
  \href{https://www.nytco.com/careers/}{Work with us}
\item
  \href{https://nytmediakit.com/}{Advertise}
\item
  \href{http://www.tbrandstudio.com/}{T Brand Studio}
\item
  \href{https://www.nytimes.com/privacy/cookie-policy\#how-do-i-manage-trackers}{Your
  Ad Choices}
\item
  \href{https://www.nytimes.com/privacy}{Privacy}
\item
  \href{https://help.nytimes.com/hc/en-us/articles/115014893428-Terms-of-service}{Terms
  of Service}
\item
  \href{https://help.nytimes.com/hc/en-us/articles/115014893968-Terms-of-sale}{Terms
  of Sale}
\item
  \href{https://spiderbites.nytimes.com}{Site Map}
\item
  \href{https://help.nytimes.com/hc/en-us}{Help}
\item
  \href{https://www.nytimes.com/subscription?campaignId=37WXW}{Subscriptions}
\end{itemize}
