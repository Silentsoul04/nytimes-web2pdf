Sections

SEARCH

\protect\hyperlink{site-content}{Skip to
content}\protect\hyperlink{site-index}{Skip to site index}

\href{https://www.nytimes.com/section/us}{U.S.}

\href{https://myaccount.nytimes.com/auth/login?response_type=cookie\&client_id=vi}{}

\href{https://www.nytimes.com/section/todayspaper}{Today's Paper}

\href{/section/us}{U.S.}\textbar{}Federal Officers Deployed in Portland
Didn't Have Proper Training, D.H.S. Memo Said

\url{https://nyti.ms/30hC8gk}

\begin{itemize}
\item
\item
\item
\item
\item
\end{itemize}

\href{https://www.nytimes.com/news-event/george-floyd-protests-minneapolis-new-york-los-angeles?action=click\&pgtype=Article\&state=default\&region=TOP_BANNER\&context=storylines_menu}{Race
and America}

\begin{itemize}
\tightlist
\item
  \href{https://www.nytimes.com/2020/07/26/us/protests-portland-seattle-trump.html?action=click\&pgtype=Article\&state=default\&region=TOP_BANNER\&context=storylines_menu}{Protesters
  Return to Other Cities}
\item
  \href{https://www.nytimes.com/2020/07/24/us/portland-oregon-protests-white-race.html?action=click\&pgtype=Article\&state=default\&region=TOP_BANNER\&context=storylines_menu}{Portland
  at the Center}
\item
  \href{https://www.nytimes.com/2020/07/23/podcasts/the-daily/portland-protests.html?action=click\&pgtype=Article\&state=default\&region=TOP_BANNER\&context=storylines_menu}{Podcast:
  Showdown in Portland}
\item
  \href{https://www.nytimes.com/interactive/2020/07/16/us/black-lives-matter-protests-louisville-breonna-taylor.html?action=click\&pgtype=Article\&state=default\&region=TOP_BANNER\&context=storylines_menu}{45
  Days in Louisville}
\end{itemize}

Advertisement

\protect\hyperlink{after-top}{Continue reading the main story}

Supported by

\protect\hyperlink{after-sponsor}{Continue reading the main story}

\hypertarget{federal-officers-deployed-in-portland-didnt-have-proper-training-dhs-memo-said}{%
\section{Federal Officers Deployed in Portland Didn't Have Proper
Training, D.H.S. Memo
Said}\label{federal-officers-deployed-in-portland-didnt-have-proper-training-dhs-memo-said}}

Rather than tamping down persistent protests in Portland, Ore., a
militarized presence from federal officers seems to have re-energized
them.

\includegraphics{https://static01.nyt.com/images/2020/07/18/us/18unrest-portland-1/merlin_174697392_435eafd8-b9c5-4edf-a542-9107c1d8e866-articleLarge.jpg?quality=75\&auto=webp\&disable=upscale}

By Sergio Olmos, \href{https://www.nytimes.com/by/mike-baker}{Mike
Baker} and \href{https://www.nytimes.com/by/zolan-kanno-youngs}{Zolan
Kanno-Youngs}

\begin{itemize}
\item
  Published July 18, 2020Updated July 21, 2020
\item
  \begin{itemize}
  \item
  \item
  \item
  \item
  \item
  \end{itemize}
\end{itemize}

PORTLAND, Ore. --- The federal agents facing a growing backlash for
their militarized approach to weeks of unrest in
\href{https://www.nytimes.com/2020/07/21/us/portland-protests.html}{Portland}
were not specifically trained in riot control or mass demonstrations, an
internal Department of Homeland Security memo warned this week.

The message, dated Thursday, was prepared by the agency for Chad F.
Wolf, the acting secretary of homeland security, as he arrived in
\href{https://www.nytimes.com/2020/07/21/us/portland-protests.html}{Portland}
to view the scene in person, according to a copy of the memo obtained by
The New York Times. It listed federal buildings in the city and issues
officers faced in protecting them.

The memo, seemingly anticipating future encounters with protesters in
other cities as the department follows President Trump's guidance to
crack down on unrest, warns: ``Moving forward, if this type of response
is going to be the norm, specialized training and standardized equipment
should be deployed to responding agencies.''

The tactical agents deployed by homeland security include officials from
a group known as BORTAC, the Border Patrol's equivalent of a SWAT team,
a highly trained group that normally is tasked with investigating drug
smuggling organizations, as opposed to protesters in cities.

Alexei Woltornist, a spokesman for the Department of Homeland Security,
said on Sunday that the missions of the federal agents in Portland
``aligned with their appropriate training'' and that officers received
``additional training for their deployment in the city'' to assist the
Federal Protective Service.

The statement did not specifically mention the memo that said the agents
lacked sufficient training in riot control or mass demonstrations. The
agency did not respond to follow-up questions about the information in
the memo.

The issue is playing out as the aggressive federal campaign to suppress
\href{https://www.nytimes.com/2020/07/20/us/portland-protests-navy-christopher-david.html}{protests
in Portland} appears to have instead rejuvenated the city's movement, as
protesters gathered by the hundreds late Friday and into Saturday
morning --- the largest crowd in weeks.

Federal officers at times flooded street corridors with tear gas and
shot projectiles from paintball guns, while demonstrators responded by
shouting that the officers in fatigues were ``terrorists'' and chanting:
``Whose streets? Our streets.''

A court ruling has largely prohibited the local police from using tear
gas during the recent protests, which have played out for more than 50
consecutive nights.

With one Portland protester severely injured in front of the federal
courthouse and others pulled by unidentified federal agents into
unmarked vans, the extraordinary campaign to subdue protesters has led
to widespread condemnation of the federal response in Portland and
beyond.

While the protesters have repeatedly decried the city's own police
tactics, Mayor Ted Wheeler, who also serves as police commissioner, and
other leaders have united in calls for federal agencies to stay away. Jo
Ann Hardesty, a city commissioner, went to join protesters gathered
outside the county Justice Center downtown, saying the city would ``not
allow armed military forces to attack our people.''

``Today we show the country and the world that the city of Portland,
even as much as we fight among ourselves, will come together to stand up
for our constitutional rights,'' Ms. Hardesty said on Friday.

While officials from the Department of Homeland Security have described
the stepped-up involvement of federal officers as part of an effort to
oppose lawlessness in the city, state and local leaders contended that
the federal officers themselves may be violating the law.

\includegraphics{https://static01.nyt.com/images/2020/07/19/us/19unrest-portland-print1/merlin_174698766_7985ae78-5ca4-4a9d-982d-f6225c4e7553-articleLarge.jpg?quality=75\&auto=webp\&disable=upscale}

Prosecutors have opened a criminal investigation into the injury of one
protester, who appeared to have been shot in the head with a less-lethal
weapon outside the federal courthouse in downtown Portland. Ellen
Rosenblum, the state's attorney general, has filed a lawsuit, accusing
federal officers of unlawful tactics in how they went about detaining
people by pulling them into unmarked vans.

The pushback against the
\href{https://www.nytimes.com/2020/07/17/us/portland-protests.html}{militarized
federal deployment} involving officers in fatigues and tactical gear has
also extended to the streets, where the presence of those federal agents
has rejuvenated a movement that had shown signs of finally slowing down
after weeks of protest against police violence and militarization.

Hundreds continued to demonstrate after midnight on Saturday, playing
music, holding shields, tearing down temporary fences and throwing
fireworks at the county's Justice Center.

Along with street medics, protesters also have the support of a snack
van that offers free Gatorade and instant noodles, and a makeshift
kitchen called Riot Ribs that cooks bratwursts and Beyond Meat sausage.
Someone on Saturday had set up a stand selling T-shirts promoting racial
equity and handwashing.

The protests have long featured a mix of tactics, with some there
displaying signs to sustain a Black Lives Matter movement that emerged
in the aftermath of George Floyd's death in May. Others have engaged in
more unruly responses, such as graffiti or throwing objects at officers.
Dozens have been arrested over the weeks, including some by federal
officers, such as a man accused of hitting an officer with a hammer last
week.

Protests around the federal courthouse --- tagged with messages such as
``Stop Using Violence on Us'' and ``History Has Its Eye on You'' ---
have drawn the ire of federal leaders. Mr. Wolf got a tour there this
week and shared images of himself in front of graffitied walls.

Image

Protesters gathered by the hundreds late Friday and into Saturday
morning --- the largest crowd in weeks.Credit...Mason Trinca/Getty
Images

The arrival of a more aggressive federal presence came after President
Trump, who at one point called on states to ``dominate'' protesters,
directed federal agencies to increase their presence to protect federal
properties, including statues and monuments that have at times been the
target of protesters. Mr. Trump said last week that he had sent
personnel to Portland because ``the locals couldn't handle it.''

Gov. Kate Brown said in an interview that she believed that the protests
were starting to ease before the federal officers waded into the scene.
She said that she had asked Mr. Wolf to keep federal agents off the
streets but that he rejected the suggestion.

Mayor Wheeler said he got the same response. But he said he believed
that the unified local response could change the federal tactics and
keep federal officers off the streets.

``I can't recall a single instance where we have had federal, state and
local officials all in alignment, saying the presence of federal troops
in our city is harmful to our residents,'' Mr. Wheeler said.

Mr. Wheeler himself has been the target of protests, with crowds at
times gathering outside of his condo. For weeks, he has called for an
end to destructive demonstrations, saying he was concerned about
``groups who continue to perpetrate violence and vandalism on our
streets.''

Senator Jeff Merkley, Democrat of Oregon,
\href{https://twitter.com/SenJeffMerkley/status/1284560125483798529}{said
in a tweet} that he and Oregon's other Democratic senator, Ron Wyden,
next week would introduce an amendment to the defense bill to stop the
Trump administration ``from sending its paramilitary squads'' onto
America's streets.

Ms. Rosenblum said her office was working with the Multnomah County
district attorney, Rod Underhill, on a criminal investigation focused on
the injury of a protester on July 12. In that case, video appeared to
show a man being struck in the head by an impact munition near the
federal courthouse, and his family said he subsequently needed surgery.

The attorney general's office also filed a lawsuit late Friday accusing
federal officers of using unlawful tactics. Protesters, along with
videos posted on social media, have described scenes of federal officers
seizing people and pulling them into unmarked vans.

The American Civil Liberties Union Foundation of Oregon has also filed
in court to curtail the actions of federal officers, and the group said
``many'' more lawsuits that would be forthcoming.

Mary B. McCord, a professor at Georgetown Law and former national
security official at the U.S. Department of Justice, said the federal
tactics and use of unmarked vehicles were reminiscent of the
\href{https://www.nytimes.com/2020/06/02/us/politics/trump-walk-lafayette-square.html}{much-criticized
federal response to demonstrations in Washington in June}.

Ms. McCord said federal officials were on dangerous ground with the
tactics they were using, including seizing and detaining protesters off
the streets and seemingly portraying all protesters as part of a
dangerous movement.

``It sends the message that these people are terrorists and need to be
treated like terrorists,'' Ms. McCord said.

She added: ``This is the kind of thing we see in authoritarian
regimes.''

Sergio Olmos reported from Portland and Mike Baker from Seattle. Neil
MacFarquhar contributed reporting.

Advertisement

\protect\hyperlink{after-bottom}{Continue reading the main story}

\hypertarget{site-index}{%
\subsection{Site Index}\label{site-index}}

\hypertarget{site-information-navigation}{%
\subsection{Site Information
Navigation}\label{site-information-navigation}}

\begin{itemize}
\tightlist
\item
  \href{https://help.nytimes.com/hc/en-us/articles/115014792127-Copyright-notice}{©~2020~The
  New York Times Company}
\end{itemize}

\begin{itemize}
\tightlist
\item
  \href{https://www.nytco.com/}{NYTCo}
\item
  \href{https://help.nytimes.com/hc/en-us/articles/115015385887-Contact-Us}{Contact
  Us}
\item
  \href{https://www.nytco.com/careers/}{Work with us}
\item
  \href{https://nytmediakit.com/}{Advertise}
\item
  \href{http://www.tbrandstudio.com/}{T Brand Studio}
\item
  \href{https://www.nytimes.com/privacy/cookie-policy\#how-do-i-manage-trackers}{Your
  Ad Choices}
\item
  \href{https://www.nytimes.com/privacy}{Privacy}
\item
  \href{https://help.nytimes.com/hc/en-us/articles/115014893428-Terms-of-service}{Terms
  of Service}
\item
  \href{https://help.nytimes.com/hc/en-us/articles/115014893968-Terms-of-sale}{Terms
  of Sale}
\item
  \href{https://spiderbites.nytimes.com}{Site Map}
\item
  \href{https://help.nytimes.com/hc/en-us}{Help}
\item
  \href{https://www.nytimes.com/subscription?campaignId=37WXW}{Subscriptions}
\end{itemize}
