Sections

SEARCH

\protect\hyperlink{site-content}{Skip to
content}\protect\hyperlink{site-index}{Skip to site index}

\href{https://www.nytimes.com/section/books/review}{Book Review}

\href{https://myaccount.nytimes.com/auth/login?response_type=cookie\&client_id=vi}{}

\href{https://www.nytimes.com/section/todayspaper}{Today's Paper}

\href{/section/books/review}{Book Review}\textbar{}Why Intellectuals
Support Dictators

\url{https://nyti.ms/30otBZ7}

\begin{itemize}
\item
\item
\item
\item
\item
\end{itemize}

Advertisement

\protect\hyperlink{after-top}{Continue reading the main story}

Supported by

\protect\hyperlink{after-sponsor}{Continue reading the main story}

nonfiction

\hypertarget{why-intellectuals-support-dictators}{%
\section{Why Intellectuals Support
Dictators}\label{why-intellectuals-support-dictators}}

\includegraphics{https://static01.nyt.com/images/2020/08/02/books/review/02Keller/Keller-articleLarge.jpg?quality=75\&auto=webp\&disable=upscale}

Buy Book ▾

\begin{itemize}
\tightlist
\item
  \href{https://www.amazon.com/gp/search?index=books\&tag=NYTBSREV-20\&field-keywords=Twilight+of+Democracy+Anne+Applebaum}{Amazon}
\item
  \href{https://du-gae-books-dot-nyt-du-prd.appspot.com/buy?title=Twilight+of+Democracy\&author=Anne+Applebaum}{Apple
  Books}
\item
  \href{https://www.anrdoezrs.net/click-7990613-11819508?url=https\%3A\%2F\%2Fwww.barnesandnoble.com\%2Fw\%2F\%3Fean\%3D9780385545808}{Barnes
  and Noble}
\item
  \href{https://www.anrdoezrs.net/click-7990613-35140?url=https\%3A\%2F\%2Fwww.booksamillion.com\%2Fp\%2FTwilight\%2Bof\%2BDemocracy\%2FAnne\%2BApplebaum\%2F9780385545808}{Books-A-Million}
\item
  \href{https://bookshop.org/a/3546/9780385545808}{Bookshop}
\item
  \href{https://www.indiebound.org/book/9780385545808?aff=NYT}{Indiebound}
\end{itemize}

When you purchase an independently reviewed book through our site, we
earn an affiliate commission.

By
\href{https://topics.nytimes.com/top/reference/timestopics/people/k/bill_keller/index.html}{Bill
Keller}

\begin{itemize}
\item
  Published July 19, 2020Updated July 20, 2020
\item
  \begin{itemize}
  \item
  \item
  \item
  \item
  \item
  \end{itemize}
\end{itemize}

\textbf{TWILIGHT OF DEMOCRACY}\\
\textbf{The Seductive Lure of Authoritarianism}\\
By Anne Applebaum

Even before the coronavirus began to test our social order, the world
was experiencing another plague, a pandemic of authoritarianism. Over
the past decade it has infected democracies around the globe, including
our own. Among the first responders were writers offering dystopian
fiction and apocalyptic nonfiction, all questioning the durability of
democracy under stress.

\href{https://www.nytimes.com/2018/06/14/books/review/benjamin-carter-hett-death-of-democracy.html}{``The
Death of Democracy,''}Benjamin Carter Hett's reconsideration of Weimar
Germany, explored how partisan intransigence enabled the rise of Hitler,
a lesson clearly intended as a timely warning. In their all-too-credible
alarum,
\href{https://www.nytimes.com/2018/01/10/books/review-trumpocracy-david-frum-how-democracies-die-steven-levitsky-daniel-ziblatt.html}{``How
Democracies Die,''} the Harvard political scientists Steven Levitsky and
Daniel Ziblatt drew on a global roster of recently failed democracies to
identify symptoms of would-be autocrats. (Donald Trump checks all the
boxes.) In ``Surviving Autocracy,'' the journalist Masha Gessen, having
sharpened a scalpel on Vladimir Putin, dissected Trumpism and concluded
that curing it will take more than an election.

Anne Applebaum's contribution to this discussion, ``Twilight of
Democracy: The Seductive Lure of Authoritarianism,'' is concerned less
with the aspiring autocrats and their compliant mobs than with the
mentality of the courtiers who make a tyrant possible: ``the writers,
intellectuals, pamphleteers, bloggers, spin doctors, producers of
television programs and creators of memes who can sell his image to the
public.''

Are these enablers true believers or just cynical opportunists? Do they
believe the lies they tell and the conspiracies they invent or are they
simply greedy for wealth and power? The answers she reaches are frankly
equivocal, which in our era of dueling absolutes is commendable if
sometimes a little frustrating.

\href{https://www.anneapplebaum.com}{Applebaum, an American
journalist}who lives mostly in Poland, has earned accolades (including a
Pulitzer Prize) for prodigiously researched popular histories of the
Cold War, the Gulag and Stalin's forced famine in Ukraine. ``Twilight of
Democracy'' is less substantial, a magazine essay expanded into a book
that is part rumination, part memoir.

\emph{{[} This book was one of our most anticipated titles of July.}
\href{https://www.nytimes.com/2020/06/24/books/new-july-books.html}{\emph{See
the full list}}\emph{.{]}}

The book, like the magazine piece, begins with a party she and her
Polish husband (who was then a deputy foreign minister in a center-right
government) hosted on New Year's Eve, 1999, at their home in the Polish
countryside. The guest list was multinational and politically diverse,
united by the afterglow of the Cold War victory over Communism and a
shared belief in ``democracy, in the rule of law, in checks and
balances, and in \ldots{} a Poland that was an integrated part of modern
Europe.''

``Nearly two decades later, I would now cross the street to avoid some
of the people who were at my New Year's Eve party,'' Applebaum writes.
``They, in turn, would not only refuse to enter my house, they would be
embarrassed to admit they had ever been there.''

These erstwhile friends, classmates and colleagues have lost faith in
democracy and gravitated to right-wing nationalist regimes and
movements. She calls them ``clercs,'' borrowing from
\href{https://www.britannica.com/biography/Julien-Benda}{the French
philosopher Julien Benda}, who a century ago seems to have meant a
sarcastic fusion of ``clerks'' and ``clerics,'' functionaries and
evangelists.

Applebaum believes the usual explanations for how authoritarians come to
power --- economic distress, fear of terrorism, the pressures of
immigration --- while important, do not fully explain the clercs. After
all, when Poland, where she begins her investigation, brought the
right-wing nativists of the
\href{https://foreignpolicy.com/2019/10/11/pis-centuries-old-divides-polands-east-west-elections/}{Law
and Justice Party} to power in 2015, the country was prosperous, was not
a migrant destination, faced no terrorist threat. ``Something else is
going on right now, something that is affecting very different
democracies, with very different economics and very different
demographics, all over the world,'' she writes.

She introduces the Polish brothers Jacek and Jaroslaw Kurski, who
marched with the dissident labor union Solidarity in the 1980s. After
the Soviet empire dissolved, Jaroslaw kept the liberal faith and now
edits a major opposition newspaper, but Jacek hooked up with Law and
Justice and became the director of Polish state television and ``chief
ideologist of the would-be one-party state.'' In Jacek, Applebaum
diagnoses a toxic sense of entitlement, a conviction that he had not
been aptly rewarded for standing up to Communism.

``Resentment, envy and above all the belief that the `system' is unfair
--- not just to the country, but to you --- these are important
sentiments among the nativist ideologues of the Polish right, so much so
that it is not easy to pick apart their personal and political
motives.''

A recurring problem in this book is that most of the clercs refuse to
talk to Applebaum, leaving her dependent on the public record and the
wisdom of mutual acquaintances. But she makes the best of what she's
got. She is most sure-footed when appraising intellectuals who have
lived in, and escaped, the Soviet orbit. From Poland, she moves on to
Hungary, then to Britain and finally to Trump's United States, with
detours to Spain and Greece, in pursuit of the fallen intellectuals.

She identifies layers of disenchantment: nostalgia for the moral purpose
of the Cold War, disappointment with meritocracy, the appeal of
conspiracy theories (often involving George Soros, the
Hungarian-American and, not incidentally, Jewish billionaire). She adds
that part of the answer lies in the ``cantankerous nature of modern
discourse itself,'' the mixed blessing of the internet, which has
deprived us of a shared narrative and diminished the responsible media
elite that used to filter out conspiracy theories and temper partisan
passions. This is hardly an original complaint, but no less true for
that.

``As polarization increases, the employees of the state are invariably
portrayed as having been `captured' by their opponents. It is not an
accident that the Law and Justice Party in Poland, the Brexiteers in
Britain and the Trump administration in the United States have launched
verbal assaults on civil servants and professional diplomats.''

Virulent populist movements have always existed in America, on the right
(the Klan, say) and the left (the Weather Underground, say). Applebaum
finds it surprising that its current incarnation emerged in the
Republican Party. ``For the party of Reagan to become the party of Trump
--- for Republicans to abandon American idealism and to adopt, instead,
the rhetoric of despair --- a sea change had to take place, not just
among the party's voters, but among the party's clercs.'' This is
probably the place to note that Applebaum deserted the Republican Party
in 2008, over the nomination of the ``proto-Trump'' Sarah Palin.

Her sampling of the American clercs consists mainly of Pat Buchanan,
Franklin Graham, Steve Bannon and Laura Ingraham, none of whom talked to
her, but all of whom are copiously on the record. She is struck by the
way their Reaganite optimism gave way to a dark sense of a decadent and
doomed America ``where universities teach people to hate their country,
where victims are more celebrated than heroes, where older values have
been discarded. Any price should be paid, any crime should be forgiven,
any outrage should be ignored if that's what it takes to get the real
America, the old America, back.''

Applebaum spends several pages trying to explain how someone as smart
and strong-willed as Ingraham became a shill for Donald Trump.
Professional ambition? Her midlife conversion to devout Catholicism? Or
perhaps she shouts so loud to drown out her own doubts. Applebaum
concedes that ``picking apart the personal and the political is a fool's
game.''

``Twilight of Democracy'' apparently was supposed to have finished with
a hopeful appraisal of her children's generation, but that finale was
interrupted by the coronavirus, and it leaves her --- like the rest of
us --- at a loss. She notes how populist leaders have seized on the
virus to justify emergency powers.

``Maybe fear of disease will create fear of freedom,'' she concludes.
``Or maybe the coronavirus will inspire a new sense of global
solidarity. \ldots{} Maddeningly, we have to accept that both futures
are possible.''

Advertisement

\protect\hyperlink{after-bottom}{Continue reading the main story}

\hypertarget{site-index}{%
\subsection{Site Index}\label{site-index}}

\hypertarget{site-information-navigation}{%
\subsection{Site Information
Navigation}\label{site-information-navigation}}

\begin{itemize}
\tightlist
\item
  \href{https://help.nytimes.com/hc/en-us/articles/115014792127-Copyright-notice}{©~2020~The
  New York Times Company}
\end{itemize}

\begin{itemize}
\tightlist
\item
  \href{https://www.nytco.com/}{NYTCo}
\item
  \href{https://help.nytimes.com/hc/en-us/articles/115015385887-Contact-Us}{Contact
  Us}
\item
  \href{https://www.nytco.com/careers/}{Work with us}
\item
  \href{https://nytmediakit.com/}{Advertise}
\item
  \href{http://www.tbrandstudio.com/}{T Brand Studio}
\item
  \href{https://www.nytimes.com/privacy/cookie-policy\#how-do-i-manage-trackers}{Your
  Ad Choices}
\item
  \href{https://www.nytimes.com/privacy}{Privacy}
\item
  \href{https://help.nytimes.com/hc/en-us/articles/115014893428-Terms-of-service}{Terms
  of Service}
\item
  \href{https://help.nytimes.com/hc/en-us/articles/115014893968-Terms-of-sale}{Terms
  of Sale}
\item
  \href{https://spiderbites.nytimes.com}{Site Map}
\item
  \href{https://help.nytimes.com/hc/en-us}{Help}
\item
  \href{https://www.nytimes.com/subscription?campaignId=37WXW}{Subscriptions}
\end{itemize}
