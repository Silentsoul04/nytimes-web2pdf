Sections

SEARCH

\protect\hyperlink{site-content}{Skip to
content}\protect\hyperlink{site-index}{Skip to site index}

\href{https://www.nytimes.com/section/business/media}{Media}

\href{https://myaccount.nytimes.com/auth/login?response_type=cookie\&client_id=vi}{}

\href{https://www.nytimes.com/section/todayspaper}{Today's Paper}

\href{/section/business/media}{Media}\textbar{}A Rush to Use Black Art
Leaves the Artists Feeling Used

\url{https://nyti.ms/2ZHJBq0}

\begin{itemize}
\item
\item
\item
\item
\item
\item
\end{itemize}

\href{https://www.nytimes.com/news-event/george-floyd-protests-minneapolis-new-york-los-angeles?action=click\&pgtype=Article\&state=default\&region=TOP_BANNER\&context=storylines_menu}{Race
and America}

\begin{itemize}
\tightlist
\item
  \href{https://www.nytimes.com/2020/07/26/us/protests-portland-seattle-trump.html?action=click\&pgtype=Article\&state=default\&region=TOP_BANNER\&context=storylines_menu}{Protesters
  Return to Other Cities}
\item
  \href{https://www.nytimes.com/2020/07/24/us/portland-oregon-protests-white-race.html?action=click\&pgtype=Article\&state=default\&region=TOP_BANNER\&context=storylines_menu}{Portland
  at the Center}
\item
  \href{https://www.nytimes.com/2020/07/23/podcasts/the-daily/portland-protests.html?action=click\&pgtype=Article\&state=default\&region=TOP_BANNER\&context=storylines_menu}{Podcast:
  Showdown in Portland}
\item
  \href{https://www.nytimes.com/interactive/2020/07/16/us/black-lives-matter-protests-louisville-breonna-taylor.html?action=click\&pgtype=Article\&state=default\&region=TOP_BANNER\&context=storylines_menu}{45
  Days in Louisville}
\end{itemize}

Advertisement

\protect\hyperlink{after-top}{Continue reading the main story}

Supported by

\protect\hyperlink{after-sponsor}{Continue reading the main story}

\hypertarget{a-rush-to-use-black-art-leaves-the-artists-feeling-used}{%
\section{A Rush to Use Black Art Leaves the Artists Feeling
Used}\label{a-rush-to-use-black-art-leaves-the-artists-feeling-used}}

Black creative professionals say they have been used to lend legitimacy
to diversity campaigns while being underpaid and pigeonholed.

\includegraphics{https://static01.nyt.com/images/2020/07/14/business/14Black-Creatives-martin/merlin_174170583_61b66f0b-f3f0-4902-97de-a9733ec707f9-articleLarge.jpg?quality=75\&auto=webp\&disable=upscale}

\href{https://www.nytimes.com/by/tiffany-hsu}{\includegraphics{https://static01.nyt.com/images/2018/12/06/multimedia/author-tiffany-hsu/author-tiffany-hsu-thumbLarge.png}}\href{https://www.nytimes.com/by/sandra-e-garcia}{\includegraphics{https://static01.nyt.com/images/2020/07/10/reader-center/author-sandra-e-garcia/author-sandra-e-garcia-thumbLarge.png}}

By \href{https://www.nytimes.com/by/tiffany-hsu}{Tiffany Hsu} and
\href{https://www.nytimes.com/by/sandra-e-garcia}{Sandra E. Garcia}

\begin{itemize}
\item
  July 20, 2020
\item
  \begin{itemize}
  \item
  \item
  \item
  \item
  \item
  \item
  \end{itemize}
\end{itemize}

The streets of New York were crowded with protesters when Shantell
Martin received an email from an ad agency last month.

M:United, a firm owned by the global advertising company McCann, wanted
to know if Ms. Martin, a Black artist, would be interested in creating a
mural about the Black Lives Matter movement on Microsoft's boarded-up
Fifth Avenue storefront. And could she do it, the email said, ``while
the protests are still relevant and the boards are still up, ideally no
later than this coming Sunday?''

Several other Black artists received the same email. In
\href{https://www.wearerelevant.art/}{an open letter} to Microsoft and
McCann, Ms. Martin and the other artists described the invitation as
``both shocking and somehow predictable.'' They also wrote that it
``betrays a telling and dangerous opportunism.''

``In their rush to portray a public solidarity with the Black Lives
Matter movement, companies risk reinscribing what got us all here: the
instrumentalization and exploitation of Black labor, ideas and talent
for what is ultimately their own benefit and safety,'' the group wrote.

The efforts of major companies to
\href{https://www.nytimes.com/2020/05/31/business/media/companies-marketing-black-lives-matter-george-floyd.html}{publicly
support} the protests against racism and police brutality have rung
hollow for some Black workers in creative fields.

Artists, models, designers, copywriters and others said they had been
drafted to lend legitimacy to companies that fail to live up to
principles of diversity and inclusion. They said they had
been\href{https://www.adweek.com/agencies/deutsch-la-fires-cco-after-offensive-email-about-casting-black-talent-resurfaces/}{pigeonholed
for roles in ad campaigns} or penalized when they raised objections
about efforts they felt were insensitive, and had been underpaid, or not
given proper credit for their work.

After Ms. Martin \href{https://www.instagram.com/p/CBGn0IXA9Q9/}{posted
on Instagram} on June 6 about the mural request, several McCann
employees told her that the ad agency had reached out to her and other
artists despite some internal objections about how the project was being
handled, she said in an interview. Both
\href{https://twitter.com/chriscapossela/status/1269391530843729920?s=20}{Chris
Capossela}, the chief marketing officer of Microsoft, and
\href{https://twitter.com/HDiamond_McCann/status/1269396926899916808?s=20}{Harris
Diamond}, the chief executive of McCann, apologized publicly to Ms.
Martin on Twitter.

The language used in the email to Ms. Martin ``was flat out wrong,'' Mr.
Diamond wrote. Microsoft said in a statement that the message was ``an
unacceptable mistake'' and that the company took ``full
accountability.''

A group of marketing professionals, Lexie Pérez, Julian Cole, Stephanie
Vitacca and Davis Ballard, began tracking the flood of company
statements of solidarity in
\href{https://docs.google.com/presentation/d/19d2SDI4yEbkSyPnFqHNwcc7TAb_4PaVEza3FprS_2Nk/edit?fbclid=IwAR0wTovNMezg-Cw9j9QTG-U7K6Z8koq-xCudwIcbJy8BL_sdmulXaLwjxrE\#slide=id.g882d183bdc_1002_63}{an
open Google Slides document} that they released on June 5. They noted
that companies often seemed to be ``seeking participation trophies'' and
trivializing the Black Lives Matter movement with ``empty and vague
platitudes,'' providing no concrete plans for change and ignoring
complaints of inequality internally.

``This is the current issue of the day,'' said Sonya Grier, a marketing
professor at American University. ``It has become almost standard for
companies to jump in, because everyone expects them to have some kind of
social presence explaining how they align on race.''

So-called
\href{https://www.nytimes.com/2020/06/22/opinion/corporate-brands-protest-art.html?smid=tw-share}{protest
art} has appeared on the doors and boarded windows of upscale brands
like Free People, 7 For All Mankind and Hugo Boss. Scores of companies
participated in \#BlackoutTuesday on Instagram last month,
\href{https://www.nytimes.com/2020/06/20/travel/travel-brands-black-lives-matter.html}{posting
black squares} on their feeds with captions expressing solidarity with
the movement.

But consumers are increasingly sensitive to how companies express their
positions. Twenty percent of U.S. adults surveyed in late June said they
would stop buying from a company deemed to be acting hypocritically on
the issues of police violence and racial injustice, the polling and
market research firm Opinium said last week.

After the publishing giant
\href{https://www.nytimes.com/2020/06/13/business/media/conde-nast-racial.html}{Condé
Nast} and the website
\href{https://www.nytimes.com/2020/06/08/business/media/refinery-29-christene-barberich.html}{Refinery29}
publicly backed the Black Lives Matter movement, they faced accusations
of mistreating employees of color. Leaked
\href{https://www.instagram.com/p/CBL9hp9nTNr/?utm_source=ig_embed}{grooming
guidelines} for store employees of the Australian fashion label
Zimmerman, which recently denounced racism and
\href{https://www.instagram.com/p/CA4dPxXpThf/?utm_source=ig_web_copy_link}{quoted}
Archbishop Desmond Tutu on its Instagram account, were found to
discriminate against Black women who wear their hair naturally.

In a statement, Zimmerman said it condemned racism and was ``determined
to be part of meaningful and positive change in the global fashion
industry.''

Ifeoma Ozoma, a former manager for the image-sharing web service
Pinterest, said on Twitter that she and another Black woman had recently
left the company after they were
\href{https://twitter.com/erikashimizu/status/1272547227177713664}{subjected
to racist and sexist behavior}. That behavior included negative feedback
from a manager after Ms. Ozoma pushed back against the promotion of
plantation weddings on the platform, she said.

The company said in a statement that it planned to diversify its board
and commission an external review of employee pay.

Many creative workers are self-employed and are not protected by human
resources departments or represented in corporate surveys. Many
independent Black artists, like Ms. Martin, said they were frequently
asked to provide input on diversity initiatives, but were not
compensated as consultants.

\includegraphics{https://static01.nyt.com/images/2020/07/14/business/14Black-Creatives-clouser/merlin_174476799_ca160f46-2c1a-4d41-98ff-bf51db02bcd5-articleLarge.jpg?quality=75\&auto=webp\&disable=upscale}

Last month, the fashion designer **** Dionne Clouser saw a design from
her Dionne by T brand
\href{https://twitter.com/_tdionne/status/1275836655027384320?s=20}{replicated
on the Instagram account} of the fast fashion label Pretty Little Thing.
She had seen her work borrowed without credit before, but this time, the
theft seemed especially brazen. Just a few months earlier, Ms. Clouser
said she had turned down an offer to be a brand ambassador for Pretty
Little Thing.

But as a small-business owner with limited funds, she opted not to take
on the far larger company in court. Pretty Little Thing declined to
comment.

``I've gotten used to it, but it leaves a bad taste for me,'' Ms.
Clouser said.

Lydia Okello, a Black queer influencer who uses the pronouns they and
them, said they also felt powerless pushing back against large fashion
companies. Mx. Okello received an offer from Anthropologie of a free
outfit if they published content on Instagram and provided several
images to the company for a social media campaign pegged to Pride month.
Mx. Okello responded with their standard rates, but said the producer
who had reached out repeatedly evaded their request for payment ---
treatment that they did not believe a straight, white influencer would
have experienced.

URBN, the company that owns Anthropologie, said in a statement that it
``handled our overture to Lydia poorly.'' The company said it was
evaluating how to make future interactions with influencers more
transparent and respectful while clarifying guidelines for compensation.

``I've worked as a Black creative all my adult life, and I've noticed
that there's often an assumption that you should feel flattered that
this large company is reaching out to you, that it has noticed you, and
that reflects a greater cultural narrative that the creative work of
marginalized groups is less valuable,'' Mx. Okello said. ``It's like,
`Just shut up and take it, or we'll find someone else.'''

Image

Lydia Okello, an influencer, said Anthropologie had evaded requests for
payment for a social media campaign.Credit...Hannah Rebecca Ackeral

Exacerbating the problem is a lack of diversity in leadership roles in
the industry. Ad agencies and marketing executives from companies such
as General Motors, McDonald's and Walmart
\href{https://www.anaaimm.net/campaigns/commitment-to-systemic-change-letter}{vowed
in a public letter to address} the issue.

But the messages of solidarity, while encouraging, ``ring hollow in the
face of our daily lived experiences,'' according to
\href{https://docs.google.com/document/d/1yVH9y0geSyJmpqUH2utaK7x1Salq5CE5IIkx0HHCGuM/preview?pru=AAABcr356cw*LhF4-mllXS4dVdIVEszy9Q}{a
letter} signed by hundreds of Black advertising employees in June.

``You have extremely limited people of color in positions of authority
at the same time that the marketplace itself is becoming much more
diverse,'' said Judy Foster Davis, a marketing professor at Eastern
Michigan University, who has studied
\href{https://www.nytimes.com/2020/06/17/business/aunt-jemima-racial-stereotype.html}{the
troubled history} of brands like
\href{https://journals.sagepub.com/doi/10.1177/0276146706296709}{Aunt
Jemima}. ``Then, over the past few years, you see all sorts of marketing
blunders.''

Recent gaffes have included racist ads and images from
\href{https://www.nytimes.com/2020/05/21/business/volkswagen-ad.html}{Volkswagen},
\href{https://twitter.com/Dove/status/916731793927278592}{Dove} and
\href{https://www.nytimes.com/2018/03/29/business/fast-fashion-diversity.html}{H\&M}.

Saturday Morning, a creative collective focused on racial justice, which
has worked with companies like Procter \& Gamble and Spotify,
\href{https://www.saturdaymorning.co/}{issued a call} last month for
brands to ``take bold steps.''

``In order for us to find true equality, there has to be sacrifice and
not just sympathy,'' said the group of Black advertising executives
behind Saturday Morning. ``Otherwise this moment will fade away like so
many before it.''

Elizabeth Paton contributed reporting.

Advertisement

\protect\hyperlink{after-bottom}{Continue reading the main story}

\hypertarget{site-index}{%
\subsection{Site Index}\label{site-index}}

\hypertarget{site-information-navigation}{%
\subsection{Site Information
Navigation}\label{site-information-navigation}}

\begin{itemize}
\tightlist
\item
  \href{https://help.nytimes.com/hc/en-us/articles/115014792127-Copyright-notice}{©~2020~The
  New York Times Company}
\end{itemize}

\begin{itemize}
\tightlist
\item
  \href{https://www.nytco.com/}{NYTCo}
\item
  \href{https://help.nytimes.com/hc/en-us/articles/115015385887-Contact-Us}{Contact
  Us}
\item
  \href{https://www.nytco.com/careers/}{Work with us}
\item
  \href{https://nytmediakit.com/}{Advertise}
\item
  \href{http://www.tbrandstudio.com/}{T Brand Studio}
\item
  \href{https://www.nytimes.com/privacy/cookie-policy\#how-do-i-manage-trackers}{Your
  Ad Choices}
\item
  \href{https://www.nytimes.com/privacy}{Privacy}
\item
  \href{https://help.nytimes.com/hc/en-us/articles/115014893428-Terms-of-service}{Terms
  of Service}
\item
  \href{https://help.nytimes.com/hc/en-us/articles/115014893968-Terms-of-sale}{Terms
  of Sale}
\item
  \href{https://spiderbites.nytimes.com}{Site Map}
\item
  \href{https://help.nytimes.com/hc/en-us}{Help}
\item
  \href{https://www.nytimes.com/subscription?campaignId=37WXW}{Subscriptions}
\end{itemize}
