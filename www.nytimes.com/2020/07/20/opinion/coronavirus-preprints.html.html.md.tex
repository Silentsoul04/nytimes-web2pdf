Sections

SEARCH

\protect\hyperlink{site-content}{Skip to
content}\protect\hyperlink{site-index}{Skip to site index}

\href{https://myaccount.nytimes.com/auth/login?response_type=cookie\&client_id=vi}{}

\href{https://www.nytimes.com/section/todayspaper}{Today's Paper}

\href{/section/opinion}{Opinion}\textbar{}How to Identify Flawed
Research Before It Becomes Dangerous

\href{https://nyti.ms/2CV1M2o}{https://nyti.ms/2CV1M2o}

\begin{itemize}
\item
\item
\item
\item
\item
\end{itemize}

Advertisement

\protect\hyperlink{after-top}{Continue reading the main story}

\href{/section/opinion}{Opinion}

Supported by

\protect\hyperlink{after-sponsor}{Continue reading the main story}

\hypertarget{how-to-identify-flawed-research-before-it-becomes-dangerous}{%
\section{How to Identify Flawed Research Before It Becomes
Dangerous}\label{how-to-identify-flawed-research-before-it-becomes-dangerous}}

Scientists and journalists need to establish a service to review
research that's publicized before it is peer reviewed.

By Michael B. Eisen and Robert Tibshirani

Mr. Eisen is a biologist at the University of California, Berkeley. Mr.
Tibshirani is a statistician at Stanford University.

\begin{itemize}
\item
  July 20, 2020
\item
  \begin{itemize}
  \item
  \item
  \item
  \item
  \item
  \end{itemize}
\end{itemize}

\includegraphics{https://static01.nyt.com/images/2020/07/20/autossell/Preprints-NYT-22_still/Preprints-NYT-22_still-mediumSquareAt3X.jpg}

Researchers have responded to the challenge of the coronavirus with a
commitment to speed and cooperation, featuring the rapid sharing of
preliminary findings through
``\href{https://connect.medrxiv.org/relate/content/181}{preprints},''
scientific manuscripts that have not yet undergone formal peer review.

We are thrilled that researchers have embraced preprints, which are
making new ideas, data and discoveries about the pandemic available to
scientists and the public in almost real time. An example is the work of
Bhramar Mukherjee and her team at the University of Michigan, whose
research modeling the Covid-19 outbreak in India helped guide that
government's lockdown policies.

But the open dissemination of early versions of papers has created a
challenge: how to ensure that policymakers and the public do not act too
hastily on early studies that are soon shown to have serious errors.

A case in point occurred in April when a group of scientists from
Stanford University posted the results of a study of previous exposure
to the coronavirus in 3,300 residents of Santa Clara, Calif. Their
paper,
\href{https://www.medrxiv.org/content/10.1101/2020.04.14.20062463v1?versioned=true}{published}
on the preprint server medRxiv, concluded that ``the number of
infections is 50- to 85-fold larger than the number of cases detected,''
suggesting that the fatality rates due to the coronavirus were much
lower than previously thought.

Given the policy implications of such a result, it is no surprise that
the study received immediate attention on social media, and in the local
and national press. But the methods and results of the study were
quickly questioned by scientists, who used
\href{https://twitter.com/wfithian/status/1252692357788479488?s=20}{Twitter},
blogs and online comments on medRxiv to air
\href{https://undark.org/2020/04/24/john-ioannidis-covid-19-death-rate-critics/}{concerns}
about the study's
\href{https://statmodeling.stat.columbia.edu/2020/04/19/fatal-flaws-in-stanford-study-of-coronavirus-prevalence/}{design},
the reliability of the antibody tests and the statistical methodology,
including important errors in the basic mathematical formulas.

This coupling of rapid dissemination with an informal, crowdsourced form
of peer review reflects a new and potentially transformative way to do
science. But the speed of modern journalism, and the lack of familiarity
of the press and public with preprints, meant that despite being largely
debunked, the results were pretty much taken at face value.

It has always been a challenge for science journalists to balance the
results of individual studies against the complex and often contentious
process by which science converges on a better understanding of reality.
In the past, because they were generally reporting on studies that had
been through peer review at a scholarly journal, journalists could be
confident that the work they were describing had received at least some
scrutiny from independent scientists, even if that did not guarantee its
accuracy.

But the slow and staid system of journal peer review in its current form
offers little help to journalists in the rapid-fire world of preprints,
especially amid a pandemic when there are strong forces aligned against
patience. The best initial reporting on the Stanford study incorporated
the concerns raised by scientists on Twitter. But we realize that we
cannot rely on this as the sole means of guarding against the overly
hasty application of science reported in preprints.

That is why we and
\href{https://docs.google.com/spreadsheets/d/1NgNbxbMq6X4EBY3Zv8sFQhpqLNNOHRce9VfaWbynI5U/edit?usp=sharing}{a
group of over 100 scientists} are calling for American scientists and
journalists to join forces to create a rapid-review service for
preprints of broad public interest. It would corral a diverse contingent
of scientists ready to comment on new preprints and to be responsive to
reporters on deadline. This would provide journalists reliable access to
independent scientists to help deal with today's growing stream of
preprints.

Such a service would collaborate with journals and other respected
organizations working at the interface between science and journalism.
For example, \href{https://www.sciline.org/}{SciLine} --- a
philanthropically supported free service for journalists based at the
American Association for the Advancement of Science --- mediates
hundreds of media interviews with scientists every year, and similar
organizations exist in
\href{https://www.sciencemediacentre.org/}{Britain},
\href{https://www.sciencemediacenter.de/en/our-offers/portfolio}{Germany},
\href{https://www.smc.org.au/}{Australia} and
\href{https://www.sciencemediacentre.co.nz/}{New Zealand}. The service
would also collaborate with professional organizations, such as the
American Statistical Association, to recruit a team of volunteers with
the expertise needed to assess new preprints on journalists' timelines.

We hope that scientists will step up to this need and provide
journalists with the tools they need to better understand the research
and convey its practical message. The public and policymakers must
demand this kind of scrutiny before they turn the latest science on
Covid-19 or anything else into policy or individual action.

Michael Eisen is a biologist at the University of California, Berkeley.
Robert Tibshirani is a statistician at Stanford University.

\emph{The Times is committed to publishing}
\href{https://www.nytimes.com/2019/01/31/opinion/letters/letters-to-editor-new-york-times-women.html}{\emph{a
diversity of letters}} \emph{to the editor. We'd like to hear what you
think about this or any of our articles. Here are some}
\href{https://help.nytimes.com/hc/en-us/articles/115014925288-How-to-submit-a-letter-to-the-editor}{\emph{tips}}\emph{.
And here's our email:}
\href{mailto:letters@nytimes.com}{\emph{letters@nytimes.com}}\emph{.}

\emph{Follow The New York Times Opinion section on}
\href{https://www.facebook.com/nytopinion}{\emph{Facebook}}\emph{,}
\href{http://twitter.com/NYTOpinion}{\emph{Twitter (@NYTopinion)}}
\emph{and}
\href{https://www.instagram.com/nytopinion/}{\emph{Instagram}}\emph{.}

Advertisement

\protect\hyperlink{after-bottom}{Continue reading the main story}

\hypertarget{site-index}{%
\subsection{Site Index}\label{site-index}}

\hypertarget{site-information-navigation}{%
\subsection{Site Information
Navigation}\label{site-information-navigation}}

\begin{itemize}
\tightlist
\item
  \href{https://help.nytimes.com/hc/en-us/articles/115014792127-Copyright-notice}{©~2020~The
  New York Times Company}
\end{itemize}

\begin{itemize}
\tightlist
\item
  \href{https://www.nytco.com/}{NYTCo}
\item
  \href{https://help.nytimes.com/hc/en-us/articles/115015385887-Contact-Us}{Contact
  Us}
\item
  \href{https://www.nytco.com/careers/}{Work with us}
\item
  \href{https://nytmediakit.com/}{Advertise}
\item
  \href{http://www.tbrandstudio.com/}{T Brand Studio}
\item
  \href{https://www.nytimes.com/privacy/cookie-policy\#how-do-i-manage-trackers}{Your
  Ad Choices}
\item
  \href{https://www.nytimes.com/privacy}{Privacy}
\item
  \href{https://help.nytimes.com/hc/en-us/articles/115014893428-Terms-of-service}{Terms
  of Service}
\item
  \href{https://help.nytimes.com/hc/en-us/articles/115014893968-Terms-of-sale}{Terms
  of Sale}
\item
  \href{https://spiderbites.nytimes.com}{Site Map}
\item
  \href{https://help.nytimes.com/hc/en-us}{Help}
\item
  \href{https://www.nytimes.com/subscription?campaignId=37WXW}{Subscriptions}
\end{itemize}
