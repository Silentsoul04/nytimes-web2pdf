Sections

SEARCH

\protect\hyperlink{site-content}{Skip to
content}\protect\hyperlink{site-index}{Skip to site index}

\href{https://myaccount.nytimes.com/auth/login?response_type=cookie\&client_id=vi}{}

\href{https://www.nytimes.com/section/todayspaper}{Today's Paper}

\href{/section/opinion}{Opinion}\textbar{}Will Clean Energy Projects
Face Troubles That Have Bedeviled Pipelines?

\href{https://nyti.ms/3fL4eqN}{https://nyti.ms/3fL4eqN}

\begin{itemize}
\item
\item
\item
\item
\item
\end{itemize}

Advertisement

\protect\hyperlink{after-top}{Continue reading the main story}

\href{/section/opinion}{Opinion}

Supported by

\protect\hyperlink{after-sponsor}{Continue reading the main story}

\hypertarget{will-clean-energy-projects-face-troubles-that-have-bedeviled-pipelines}{%
\section{Will Clean Energy Projects Face Troubles That Have Bedeviled
Pipelines?}\label{will-clean-energy-projects-face-troubles-that-have-bedeviled-pipelines}}

If the permit process is not improved, solar and wind energy efforts may
face protracted delays or shutdowns.

By Jason Bordoff

Mr. Bordoff is director of Columbia University's Center on Global Energy
Policy

\begin{itemize}
\item
  July 20, 2020
\item
  \begin{itemize}
  \item
  \item
  \item
  \item
  \item
  \end{itemize}
\end{itemize}

\includegraphics{https://static01.nyt.com/images/2020/07/17/opinion/17Bordoff/merlin_127646141_2f53b6f9-2dca-4c89-8762-bfe225beb311-articleLarge.jpg?quality=75\&auto=webp\&disable=upscale}

This month, environmentalists celebrated setbacks to
\href{https://www.nytimes.com/reuters/2020/07/08/us/08reuters-usa-pipelines.html}{three
major oil and gas pipelines}, nearly a decade after the protests against
the Keystone XL pipeline began. Yet what goes around can come around.
Legal strategies that have derailed pipelines can also be turned against
clean energy projects urgently needed to combat climate change.

The path forward is not to gut the environmental review process, as
President Trump
\href{https://www.nytimes.com/2020/07/15/climate/trump-environment-nepa.html}{proposed}
last week. Rather, the government must make the permitting process work
better if it is to quickly develop renewable energy projects and improve
the grid infrastructure while upholding the nation's landmark
environmental laws.

This is especially important now. With no time to delay in combating
climate change, and the economy in need of a boost, we need large
government investment in clean energy infrastructure from electric car
charging to mass transit and rail, as former Vice President Joe Biden
\href{https://www.nytimes.com/2020/07/14/us/politics/biden-climate-plan.html}{proposed}
last week.

The three pipeline cases followed a boom in oil and gas output in the
United States over the past decade. Environmental groups have developed
sophisticated legal strategies to block the pipelines that get fuel to
market. Major pipeline projects usually require federal permits, such as
when they cross certain water bodies, wetlands or public lands. But
before those permits can be issued, federal law requires the government
to conduct a review of the project's environmental impacts. This process
has become increasingly expensive, time consuming and susceptible to
litigation, especially for large energy infrastructure projects.

The recent pipeline setbacks occurred, among other reasons, because the
courts found that the Trump administration's environmental reviews cut
corners. Its effort to fast-track projects in the name of ``energy
dominance'' came back to bite the very projects it was meant to help.

First, a Montana court ruled in the Keystone XL pipeline litigation that
the federal government had failed to properly consult on threatened and
endangered species when it reauthorized a program to streamline water
permits. The Supreme Court then
\href{https://www.supremecourt.gov/orders/courtorders/070620zr_2d83.pdf}{rejected
a request} from the Trump administration to allow construction of parts
of the pipeline that had been blocked by the judge.

In addition, two of the nation's largest utilities announced they were
\href{https://www.nytimes.com/2020/07/05/business/atlantic-coast-pipeline-cancel-dominion-energy-berkshire-hathaway.html}{abandoning
plans} to build a 600-mile natural gas pipeline crossing the Appalachian
Trail because of rising costs, ongoing delays and regulatory
\href{https://news.dominionenergy.com/2020-07-05-Dominion-Energy-and-Duke-Energy-Cancel-the-Atlantic-Coast-Pipeline}{uncertainty}
arising from the Montana ruling. And that was followed by a federal
court ordering another high-profile oil pipeline, the Dakota Access
project, which has been a focus of protests from Native Americans and
environmentalists, to
\href{https://www.nytimes.com/aponline/2020/07/06/business/ap-us-dakota-access-pipeline.html}{shut
down} pending revisions to an environmental review that the court found
to be deficient,
\href{https://www.marketwatch.com/story/federal-appeals-court-temporarily-halts-dakota-access-pipeline-shutdown-2020-07-14}{an
order since stayed} pending appeal.

These pipeline defeats reflect an increasingly sophisticated legal
strategy to use complex environmental laws designed to protect public
lands, water, endangered species and air quality to block fossil fuels
projects based on flaws in environmental reviews and regulatory
approvals. Yet those very tactics can now be used to impede clean energy
projects, which have impacts as well. For example, solar projects sited
in the deserts of Western states may affect the habitat of the
endangered desert tortoise. Commercial fishermen in the Northeast have
opposed offshore wind projects they claim will interfere with their
fisheries.

Delay is problematic not only because time is short to curb emissions,
but also because the need for large government spending to boost an
economy reeling from Covid-19 shutdowns may create a unique opportunity
to invest in clean energy. The very nature of economic stimulus means
the faster funds can be deployed, the more effective they can be.

The Trump administration's approach of shortcutting environmental
reviews and avoiding meaningful community engagement has been
self-defeating. His so-called reforms --- which would set arbitrary time
limits, exclude from review certain project impacts such as climate
change, and give short shrift to input from local communities --- are
misguided and provide grist for years of litigation.

The environmental review process needs to be improved, but not by
cutting corners. There is no simple solution, but there are several
actions the federal government can take. First, more should be done to
invest the resources necessary and use the authorities available under
the
\href{https://law.stanford.edu/2015/12/10/congress-just-enacted-new-permitting-requirements-for-energy-projects-did-you-miss-it/}{FAST
Act of 2015} to achieve the law's aim to improve coordination among
federal agencies on major infrastructure projects, hold them accountable
to reasonable timetables, and track progress. Second, to see the full
consequences of environmental impacts, environmental benefits of clean
energy projects like lower emissions should be considered along with
potential harms. Finally, communities near these projects should be
included to address concerns, develop solutions and defuse opposition.

Environmental activists will view the legal blows landed on pipelines as
a victory. But if the federal environmental review and permitting
processes that stymied those projects are not improved, the massive
clean energy investments required to transform our energy economy may
fall victim next.

\href{https://sipa.columbia.edu/faculty-research/faculty-directory/jason-bordoff}{Jason
Bordoff} is director of Columbia University's Center on Global Energy
Policy and was on the staffs of the National Security Council and
Council on Environmental Quality under President Barack Obama*.*

\emph{The Times is committed to publishing}
\href{https://www.nytimes.com/2019/01/31/opinion/letters/letters-to-editor-new-york-times-women.html}{\emph{a
diversity of letters}} \emph{to the editor. We'd like to hear what you
think about this or any of our articles. Here are some}
\href{https://help.nytimes.com/hc/en-us/articles/115014925288-How-to-submit-a-letter-to-the-editor}{\emph{tips}}\emph{.
And here's our email:}
\href{mailto:letters@nytimes.com}{\emph{letters@nytimes.com}}\emph{.}

\emph{Follow The New York Times Opinion section on}
\href{https://www.facebook.com/nytopinion}{\emph{Facebook}}\emph{,}
\href{http://twitter.com/NYTOpinion}{\emph{Twitter (@NYTopinion)}}
\emph{and}
\href{https://www.instagram.com/nytopinion/}{\emph{Instagram}}\emph{.}

Advertisement

\protect\hyperlink{after-bottom}{Continue reading the main story}

\hypertarget{site-index}{%
\subsection{Site Index}\label{site-index}}

\hypertarget{site-information-navigation}{%
\subsection{Site Information
Navigation}\label{site-information-navigation}}

\begin{itemize}
\tightlist
\item
  \href{https://help.nytimes.com/hc/en-us/articles/115014792127-Copyright-notice}{©~2020~The
  New York Times Company}
\end{itemize}

\begin{itemize}
\tightlist
\item
  \href{https://www.nytco.com/}{NYTCo}
\item
  \href{https://help.nytimes.com/hc/en-us/articles/115015385887-Contact-Us}{Contact
  Us}
\item
  \href{https://www.nytco.com/careers/}{Work with us}
\item
  \href{https://nytmediakit.com/}{Advertise}
\item
  \href{http://www.tbrandstudio.com/}{T Brand Studio}
\item
  \href{https://www.nytimes.com/privacy/cookie-policy\#how-do-i-manage-trackers}{Your
  Ad Choices}
\item
  \href{https://www.nytimes.com/privacy}{Privacy}
\item
  \href{https://help.nytimes.com/hc/en-us/articles/115014893428-Terms-of-service}{Terms
  of Service}
\item
  \href{https://help.nytimes.com/hc/en-us/articles/115014893968-Terms-of-sale}{Terms
  of Sale}
\item
  \href{https://spiderbites.nytimes.com}{Site Map}
\item
  \href{https://help.nytimes.com/hc/en-us}{Help}
\item
  \href{https://www.nytimes.com/subscription?campaignId=37WXW}{Subscriptions}
\end{itemize}
