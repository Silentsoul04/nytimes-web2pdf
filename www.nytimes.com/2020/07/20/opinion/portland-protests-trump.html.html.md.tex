Sections

SEARCH

\protect\hyperlink{site-content}{Skip to
content}\protect\hyperlink{site-index}{Skip to site index}

\href{https://myaccount.nytimes.com/auth/login?response_type=cookie\&client_id=vi}{}

\href{https://www.nytimes.com/section/todayspaper}{Today's Paper}

\href{/section/opinion}{Opinion}\textbar{}Trump's Occupation of American
Cities Has Begun

\href{https://nyti.ms/3eHOymV}{https://nyti.ms/3eHOymV}

\begin{itemize}
\item
\item
\item
\item
\item
\item
\end{itemize}

Advertisement

\protect\hyperlink{after-top}{Continue reading the main story}

\href{/section/opinion}{Opinion}

Supported by

\protect\hyperlink{after-sponsor}{Continue reading the main story}

\hypertarget{trumps-occupation-of-american-cities-has-begun}{%
\section{Trump's Occupation of American Cities Has
Begun}\label{trumps-occupation-of-american-cities-has-begun}}

Protesters are being snatched from the streets without warrants. Can we
call it fascism yet?

\href{https://www.nytimes.com/by/michelle-goldberg}{\includegraphics{https://static01.nyt.com/images/2018/04/02/opinion/michelle-goldberg/michelle-goldberg-thumbLarge.png}}

By \href{https://www.nytimes.com/by/michelle-goldberg}{Michelle
Goldberg}

Opinion Columnist

\begin{itemize}
\item
  July 20, 2020
\item
  \begin{itemize}
  \item
  \item
  \item
  \item
  \item
  \item
  \end{itemize}
\end{itemize}

\includegraphics{https://static01.nyt.com/images/2020/07/20/opinion/20goldbergWeb/merlin_174758376_bccbfa52-6fd1-4e3b-ba2a-fc349006e4b4-articleLarge.jpg?quality=75\&auto=webp\&disable=upscale}

\href{https://www.nytimes.com/es/2020/07/22/espanol/opinion/portland-protestas-trump.html}{Leer
en español}

The month after Donald Trump's inauguration, the Yale historian Timothy
Snyder published the best-selling book ``On Tyranny: Twenty Lessons From
the Twentieth Century.'' It was part of a small flood of titles meant to
help Americans find their bearings as the new president laid siege to
liberal democracy.

One of Snyder's lessons was, ``Be wary of paramilitaries.'' He wrote,
``When the pro-leader paramilitary and the official police and military
intermingle, the end has come.'' In 2017, the idea of unidentified
agents in camouflage snatching leftists off the streets without warrants
might have seemed like a febrile Resistance fantasy. Now it's happening.

According to a
\href{http://opb-imgserve-production.s3-website-us-west-2.amazonaws.com/original/ag_rosenblum_xxxx_updated_complaint_1595086491349.pdf}{lawsuit}
filed by Oregon's attorney general, Ellen Rosenblum, on Friday, federal
agents ``have been using unmarked vehicles to drive around downtown
Portland, detain protesters, and place them into the officers' unmarked
vehicles'' since at least last Tuesday. The protesters are neither
arrested nor told why they're being held.

There's no way to know the affiliation of all the agents --- they've
been wearing military fatigues with patches that just say ``Police'' ---
but The Times reported that some of them are
\href{https://www.nytimes.com/2020/07/18/us/portland-protests.html}{part
of a specialized Border Patrol group} ``that normally is tasked with
investigating drug smuggling organizations.''

The Trump administration has announced that it intends to send a similar
force to
\href{https://www.motherjones.com/anti-racism-police-protest/2020/07/trump-border-patrol-cities-portland-chicago/}{other
cities}; on Monday,
\href{https://www.chicagotribune.com/news/criminal-justice/ct-chicago-police-dhs-deployment-20200720-dftu5ychwbcxtg4ltarh5qnwma-story.html}{The
Chicago Tribune reported} on plans to deploy about 150 federal agents to
Chicago. ``I don't need invitations by the state,'' Chad Wolf, acting
secretary of the Department of Homeland Security,
\href{https://twitter.com/atrupar/status/1285224329878306817?s=20}{said
on Fox News} Monday, adding, ``We're going to do that whether they like
us there or not.''

In Portland, we see what such an occupation looks like. Oregon Public
Broadcasting reported on
\href{https://www.opb.org/news/article/federal-law-enforcement-unmarked-vehicles-portland-protesters/}{29-year-old
Mark Pettibone}, who early last Wednesday was grabbed off the street by
unidentified men, hustled into an unmarked minivan and taken to a
holding cell in the federal courthouse. He was eventually released
without learning who had abducted him.

A federal agent shot 26-year-old Donavan La Bella in the head with an
impact munition; he was hospitalized and needed reconstructive surgery.
In a widely circulated video, a 53-year-old Navy veteran was
\href{https://www.washingtonpost.com/nation/2020/07/20/christopher-david-portland-protest-video/}{pepper
sprayed and beaten} after approaching federal agents to ask them about
their oaths to the Constitution, leaving him with two broken bones.

There's something particularly terrifying in the use of Border Patrol
agents against American dissidents. After the
\href{https://www.nytimes.com/2020/06/02/us/politics/trump-walk-lafayette-square.html}{attack
on protesters} near the White House last month, the military pushed back
on Trump's attempts to turn it against the citizenry. Police officers in
many cities are willing to brutalize demonstrators, but they're under
local control. U.S. Customs and Border Protection, however, is under
federal authority, has leadership that's
\href{https://www.newyorker.com/news/news-desk/the-border-patrol-was-primed-for-president-trump}{fanatically
devoted} to Trump and is saturated with far-right politics.

``It doesn't surprise me that Donald Trump picked C.B.P. to be the ones
to go over to Portland and do this,'' Representative Joaquin Castro,
Democrat of Texas, told me. ``It has been a very problematic agency in
terms of respecting human rights and in terms of respecting the law.''

It is true that C.B.P. is not an extragovernmental militia, and so might
not fit precisely into Snyder's ``On Tyranny'' schema. But when I spoke
to Snyder on Monday, he suggested the distinction isn't that
significant. ``The state is allowed to use force, but the state is
allowed to use force according to rules,'' he said. These agents,
operating outside their normal roles, are by all appearances behaving
lawlessly.

Snyder pointed out that the history of autocracy offers several examples
of border agents being used against regime enemies.

``This is a classic way that violence happens in authoritarian regimes,
whether it's Franco's Spain or whether it's the Russian Empire,'' said
Snyder. ``The people who are getting used to committing violence on the
border are then brought in to commit violence against people in the
interior.''

\includegraphics{https://static01.nyt.com/images/2018/10/15/autossell/15op-fascism2/15op-fascism2-videoSixteenByNineJumbo1600.jpg}

Castro worries that since the agents are unidentified, far-right groups
could
\href{https://twitter.com/JoaquinCastrotx/status/1284956181400899585?s=20}{easily
masquerade as them} to go after their enemies on the left. ``It becomes
more likely the more that this tactic is used,'' he said. ``I think it's
unconstitutional and dangerous and heading towards fascism.''

On Friday, the House speaker, Nancy Pelosi,
\href{https://twitter.com/SpeakerPelosi/status/1284294427654197248?s=20}{tweeted
about what's happening in Portland}: ``Trump and his storm troopers must
be stopped.'' She didn't mention what Congress plans to do to stop them,
but the House will soon vote on a homeland security appropriations bill.
People outraged about the administration's police-state tactics should
demand, at a minimum, that Congress hold up the department's funding
until those tactics are halted.

Through the Trump years, there's been a debate about whether the
president's authoritarianism is tempered by his incompetence. Those who
think concern about fascism is overblown can cite several instances when
the administration has been beaten back after overreaching. But all too
often the White House has persevered, deforming American life until what
once seemed like worst-case scenarios become the status quo.

Trump has already established that his allies, like Michael Flynn and
Roger Stone, are above the law. What happens now will tell us how many
of us are below it.

\emph{The Times is committed to publishing}
\href{https://www.nytimes.com/2019/01/31/opinion/letters/letters-to-editor-new-york-times-women.html}{\emph{a
diversity of letters}} \emph{to the editor. We'd like to hear what you
think about this or any of our articles. Here are some}
\href{https://help.nytimes.com/hc/en-us/articles/115014925288-How-to-submit-a-letter-to-the-editor}{\emph{tips}}\emph{.
And here's our email:}
\href{mailto:letters@nytimes.com}{\emph{letters@nytimes.com}}\emph{.}

\emph{Follow The New York Times Opinion section on}
\href{https://www.facebook.com/nytopinion}{\emph{Facebook}}\emph{,}
\href{http://twitter.com/NYTOpinion}{\emph{Twitter (@NYTopinion)}}
\emph{and}
\href{https://www.instagram.com/nytopinion/}{\emph{Instagram}}\emph{.}

Advertisement

\protect\hyperlink{after-bottom}{Continue reading the main story}

\hypertarget{site-index}{%
\subsection{Site Index}\label{site-index}}

\hypertarget{site-information-navigation}{%
\subsection{Site Information
Navigation}\label{site-information-navigation}}

\begin{itemize}
\tightlist
\item
  \href{https://help.nytimes.com/hc/en-us/articles/115014792127-Copyright-notice}{©~2020~The
  New York Times Company}
\end{itemize}

\begin{itemize}
\tightlist
\item
  \href{https://www.nytco.com/}{NYTCo}
\item
  \href{https://help.nytimes.com/hc/en-us/articles/115015385887-Contact-Us}{Contact
  Us}
\item
  \href{https://www.nytco.com/careers/}{Work with us}
\item
  \href{https://nytmediakit.com/}{Advertise}
\item
  \href{http://www.tbrandstudio.com/}{T Brand Studio}
\item
  \href{https://www.nytimes.com/privacy/cookie-policy\#how-do-i-manage-trackers}{Your
  Ad Choices}
\item
  \href{https://www.nytimes.com/privacy}{Privacy}
\item
  \href{https://help.nytimes.com/hc/en-us/articles/115014893428-Terms-of-service}{Terms
  of Service}
\item
  \href{https://help.nytimes.com/hc/en-us/articles/115014893968-Terms-of-sale}{Terms
  of Sale}
\item
  \href{https://spiderbites.nytimes.com}{Site Map}
\item
  \href{https://help.nytimes.com/hc/en-us}{Help}
\item
  \href{https://www.nytimes.com/subscription?campaignId=37WXW}{Subscriptions}
\end{itemize}
