Sections

SEARCH

\protect\hyperlink{site-content}{Skip to
content}\protect\hyperlink{site-index}{Skip to site index}

\href{https://www.nytimes.com/section/world/europe}{Europe}

\href{https://myaccount.nytimes.com/auth/login?response_type=cookie\&client_id=vi}{}

\href{https://www.nytimes.com/section/todayspaper}{Today's Paper}

\href{/section/world/europe}{Europe}\textbar{}`A Place Where Everybody
Can Shop' Is Closing Its Doors

\url{https://nyti.ms/2ZIpkR6}

\begin{itemize}
\item
\item
\item
\item
\item
\item
\end{itemize}

\href{https://www.nytimes.com/news-event/coronavirus?action=click\&pgtype=Article\&state=default\&region=TOP_BANNER\&context=storylines_menu}{The
Coronavirus Outbreak}

\begin{itemize}
\tightlist
\item
  live\href{https://www.nytimes.com/2020/08/01/world/coronavirus-covid-19.html?action=click\&pgtype=Article\&state=default\&region=TOP_BANNER\&context=storylines_menu}{Latest
  Updates}
\item
  \href{https://www.nytimes.com/interactive/2020/us/coronavirus-us-cases.html?action=click\&pgtype=Article\&state=default\&region=TOP_BANNER\&context=storylines_menu}{Maps
  and Cases}
\item
  \href{https://www.nytimes.com/interactive/2020/science/coronavirus-vaccine-tracker.html?action=click\&pgtype=Article\&state=default\&region=TOP_BANNER\&context=storylines_menu}{Vaccine
  Tracker}
\item
  \href{https://www.nytimes.com/interactive/2020/07/29/us/schools-reopening-coronavirus.html?action=click\&pgtype=Article\&state=default\&region=TOP_BANNER\&context=storylines_menu}{What
  School May Look Like}
\item
  \href{https://www.nytimes.com/live/2020/07/31/business/stock-market-today-coronavirus?action=click\&pgtype=Article\&state=default\&region=TOP_BANNER\&context=storylines_menu}{Economy}
\end{itemize}

Advertisement

\protect\hyperlink{after-top}{Continue reading the main story}

Supported by

\protect\hyperlink{after-sponsor}{Continue reading the main story}

Paris Dispatch

\hypertarget{a-place-where-everybody-can-shop-is-closing-its-doors}{%
\section{`A Place Where Everybody Can Shop' Is Closing Its
Doors}\label{a-place-where-everybody-can-shop-is-closing-its-doors}}

The famed department store Tati is shutting. And a lot of people are
very sad.

\includegraphics{https://static01.nyt.com/images/2020/07/19/world/xxparis-store-dispatch01sub/xxparis-store-dispatch01sub-articleLarge.jpg?quality=75\&auto=webp\&disable=upscale}

\href{https://www.nytimes.com/by/adam-nossiter}{\includegraphics{https://static01.nyt.com/images/2018/10/15/multimedia/author-adam-nossiter/author-adam-nossiter-thumbLarge.png}}

By \href{https://www.nytimes.com/by/adam-nossiter}{Adam Nossiter}

\begin{itemize}
\item
  July 20, 2020
\item
  \begin{itemize}
  \item
  \item
  \item
  \item
  \item
  \item
  \end{itemize}
\end{itemize}

PARIS --- Tati is a ghost of what it once was, and soon it won't even be
that. Shelves that were crammed with dish-scrubbers and brassieres for
\$1.50 and plaster figurines of smiling bunnies are now bare. Exits are
still numbered --- how else to find your way out through the crowds ---
but the aisles are empty. There are no crowds.

It's as though the discount department store, a once-thronged wonder of
Paris more visited than the Eiffel Tower, is willing itself out of
existence, fading like a Cheshire cat's smile back into the hinterlands
on both sides of the Mediterranean where many of its famous pink gingham
shopping bags ended up. The official end, in any case, is not far-off
for Tati, which revolutionized postwar shopping in France and stamped
its identity on the entire vibrant neighborhood of Barbès, on the edge
of Montmartre.

Tati is a victim of Covid-19, its latest owners say, and sharply
declining sales. But the trends that have killed it go back much
further.

The great bales of 50-cent stockings and dollar underwear that once
crowded the sidewalks in front of the store, on the Boulevard de
Rochechouart, are no longer enough. Bargain-basement shoppers have long
since had more socially enhancing options among the international
chains.

\includegraphics{https://static01.nyt.com/images/2020/07/20/world/00paris-store-dispatch6/merlin_174755208_34b4f44f-4db5-4ee3-a192-46c730b5aec8-articleLarge.jpg?quality=75\&auto=webp\&disable=upscale}

``Bring me your little thieves!'' Tati's founder, Jules Ouaki, a Jewish
immigrant from Tunisia who had fought for the Free French in World War
II, used to tell the press, a sly nod to both his humble clientele and
the raffish immigrant district of Barbès that he transformed. Both made
him rich. His success was once studied in French business schools.

The vicissitudes of the Ouaki family --- the bankruptcies, the failed
ventures into high fashion by Mr. Ouaki's son, his rock bands, horses
and coauthorship of a book with the Dalai Lama --- were covered over the
years by the French media like episodes of ``Dallas.''

Now Tati, named by Jules Ouaki after his Tunisian mother, is losing
money fast. The last Tati store in France, the Barbès flagship, will be
gone by the end of the year. But despite its problems, it still fills a
niche and will be missed.

``Even people on the dole could shop here,'' said Alger Djemila,
clutching the vaunted gingham as she emerged one day last week from the
annex off Boulevard de Rochechouart.

``Equality, for everybody,'' said her friend Moana Ndjassiri, who works
in an Air France lounge at the airport. ``This is a place where
everybody can shop. You never leave with your hands empty.''

\hypertarget{latest-updates-global-coronavirus-outbreak}{%
\section{\texorpdfstring{\href{https://www.nytimes.com/2020/08/01/world/coronavirus-covid-19.html?action=click\&pgtype=Article\&state=default\&region=MAIN_CONTENT_1\&context=storylines_live_updates}{Latest
Updates: Global Coronavirus
Outbreak}}{Latest Updates: Global Coronavirus Outbreak}}\label{latest-updates-global-coronavirus-outbreak}}

Updated 2020-08-02T07:11:27.880Z

\begin{itemize}
\tightlist
\item
  \href{https://www.nytimes.com/2020/08/01/world/coronavirus-covid-19.html?action=click\&pgtype=Article\&state=default\&region=MAIN_CONTENT_1\&context=storylines_live_updates\#link-34047410}{The
  U.S. reels as July cases more than double the total of any other
  month.}
\item
  \href{https://www.nytimes.com/2020/08/01/world/coronavirus-covid-19.html?action=click\&pgtype=Article\&state=default\&region=MAIN_CONTENT_1\&context=storylines_live_updates\#link-780ec966}{Top
  U.S. officials work to break an impasse over the federal jobless
  benefit.}
\item
  \href{https://www.nytimes.com/2020/08/01/world/coronavirus-covid-19.html?action=click\&pgtype=Article\&state=default\&region=MAIN_CONTENT_1\&context=storylines_live_updates\#link-2bc8948}{Its
  outbreak untamed, Melbourne goes into even greater lockdown.}
\end{itemize}

\href{https://www.nytimes.com/2020/08/01/world/coronavirus-covid-19.html?action=click\&pgtype=Article\&state=default\&region=MAIN_CONTENT_1\&context=storylines_live_updates}{See
more updates}

More live coverage:
\href{https://www.nytimes.com/live/2020/07/31/business/stock-market-today-coronavirus?action=click\&pgtype=Article\&state=default\&region=MAIN_CONTENT_1\&context=storylines_live_updates}{Markets}

So infused is Barbès with the spirit of Tati ** that residents and the
few shoppers left have trouble imagining the neighborhood without it,
after over 70 years of fruitful coexistence.

``Without Tati, Barbès is dead,'' proclaimed Nawel Bahiti, another
shopper. ``When Tati closes, the whole neighborhood will shut down,''
insisted Bernard Le Huerou, a retiree who had ridden up on his bike.

These are exaggerations: This rambunctious mix of Algiers, Dakar and
Paris at the foot of tourist-dominated Montmartre is unkillable. But at
the very least the neighborhood over which Mr. Ouaki ruled for decades
``like a customary prince,'' as L'Express ** put it in a 1980 profile,
will be changed, just as he himself changed it.

Image

Tati's founder, Jules Ouaki, in Paris in July 1980.Credit...Bernard
Charlon/Gamma-Rapho, via Getty Images

Newsreels from the 1970s show him beaming over the jostling
elbow-to-elbow shoppers, few of them, at that time, immigrants. Now,
``the entire neighborhood will be disfigured,'' said Odette Osmanovic, a
retiree who had come to commiserate at Tati's doorstep last week with a
fellow Barbès resident.

``Thanks to the presence of Tati, Barbès itself, today, seems almost
like a brand name,'' wrote Emmanuelle Lallement, a University of Paris
anthropologist who has closely studied the intimate relationship between
the two.

She has focused on how much of what surrounds Tati --- the Barbès
culture of street vendors hawking foodstuffs and clothing --- appears as
an extension of the store, and inspired by it.

Even today the first thing you see off the metro, from the elevated
platform at Barbès-Rochechouart station, is the giant old ``Tati, The
Lowest Prices'' sign on top of its central seven-story Haussmann-style
building, a former cafe transformed by Mr. Ouaki.

The sign dominates the meek towers of Sacré-Coeur church in the near
distance, commerce taking precedence over the sacred.

After the war, shopping in France still inhabited a 19th-century world;
if you entered a store, you generally had to buy something. Often you
had to ring a bell to enter. It was an intimidating experience.

Mr. Ouaki democratized it, reinventing shopping for a postwar generation
of French eager to buy but limited in means. He put enormous bales of
discount clothing out on the sidewalks of Barbès. The customers were
free to rummage through them, feel the merchandise, not even enter the
store.

``He got rid of the door, and the window,'' Ms. Lallement said.

On the boulevard, Tati ** extends nearly a full block, incorporating the
old bordellos and hotels that Mr. Ouaki gradually bought up, going from
the 150 square feet he started with in 1948, to over 9,000 square feet
of shopping space by the 1980s.

Image

Inside Tati's first historic store on the Boulevard de Rochechouart in
Paris, in 2013.Credit...Lucas Schifres/Getty Image

``He changed it from a nighttime to a daytime district,'' said Ms.
Lallement, and for his efforts was awarded the Légion d'Honneur,
France's highest civil distinction.

By the early 1980s, Tati was drawing 25 million customers a year and had
expanded all over France. Specially chartered buses would come up from
the provinces for day shopping trips.

\href{https://www.nytimes.com/news-event/coronavirus?action=click\&pgtype=Article\&state=default\&region=MAIN_CONTENT_3\&context=storylines_faq}{}

\hypertarget{the-coronavirus-outbreak-}{%
\subsubsection{The Coronavirus Outbreak
›}\label{the-coronavirus-outbreak-}}

\hypertarget{frequently-asked-questions}{%
\paragraph{Frequently Asked
Questions}\label{frequently-asked-questions}}

Updated July 27, 2020

\begin{itemize}
\item ~
  \hypertarget{should-i-refinance-my-mortgage}{%
  \paragraph{Should I refinance my
  mortgage?}\label{should-i-refinance-my-mortgage}}

  \begin{itemize}
  \tightlist
  \item
    \href{https://www.nytimes.com/article/coronavirus-money-unemployment.html?action=click\&pgtype=Article\&state=default\&region=MAIN_CONTENT_3\&context=storylines_faq}{It
    could be a good idea,} because mortgage rates have
    \href{https://www.nytimes.com/2020/07/16/business/mortgage-rates-below-3-percent.html?action=click\&pgtype=Article\&state=default\&region=MAIN_CONTENT_3\&context=storylines_faq}{never
    been lower.} Refinancing requests have pushed mortgage applications
    to some of the highest levels since 2008, so be prepared to get in
    line. But defaults are also up, so if you're thinking about buying a
    home, be aware that some lenders have tightened their standards.
  \end{itemize}
\item ~
  \hypertarget{what-is-school-going-to-look-like-in-september}{%
  \paragraph{What is school going to look like in
  September?}\label{what-is-school-going-to-look-like-in-september}}

  \begin{itemize}
  \tightlist
  \item
    It is unlikely that many schools will return to a normal schedule
    this fall, requiring the grind of
    \href{https://www.nytimes.com/2020/06/05/us/coronavirus-education-lost-learning.html?action=click\&pgtype=Article\&state=default\&region=MAIN_CONTENT_3\&context=storylines_faq}{online
    learning},
    \href{https://www.nytimes.com/2020/05/29/us/coronavirus-child-care-centers.html?action=click\&pgtype=Article\&state=default\&region=MAIN_CONTENT_3\&context=storylines_faq}{makeshift
    child care} and
    \href{https://www.nytimes.com/2020/06/03/business/economy/coronavirus-working-women.html?action=click\&pgtype=Article\&state=default\&region=MAIN_CONTENT_3\&context=storylines_faq}{stunted
    workdays} to continue. California's two largest public school
    districts --- Los Angeles and San Diego --- said on July 13, that
    \href{https://www.nytimes.com/2020/07/13/us/lausd-san-diego-school-reopening.html?action=click\&pgtype=Article\&state=default\&region=MAIN_CONTENT_3\&context=storylines_faq}{instruction
    will be remote-only in the fall}, citing concerns that surging
    coronavirus infections in their areas pose too dire a risk for
    students and teachers. Together, the two districts enroll some
    825,000 students. They are the largest in the country so far to
    abandon plans for even a partial physical return to classrooms when
    they reopen in August. For other districts, the solution won't be an
    all-or-nothing approach.
    \href{https://bioethics.jhu.edu/research-and-outreach/projects/eschool-initiative/school-policy-tracker/}{Many
    systems}, including the nation's largest, New York City, are
    devising
    \href{https://www.nytimes.com/2020/06/26/us/coronavirus-schools-reopen-fall.html?action=click\&pgtype=Article\&state=default\&region=MAIN_CONTENT_3\&context=storylines_faq}{hybrid
    plans} that involve spending some days in classrooms and other days
    online. There's no national policy on this yet, so check with your
    municipal school system regularly to see what is happening in your
    community.
  \end{itemize}
\item ~
  \hypertarget{is-the-coronavirus-airborne}{%
  \paragraph{Is the coronavirus
  airborne?}\label{is-the-coronavirus-airborne}}

  \begin{itemize}
  \tightlist
  \item
    The coronavirus
    \href{https://www.nytimes.com/2020/07/04/health/239-experts-with-one-big-claim-the-coronavirus-is-airborne.html?action=click\&pgtype=Article\&state=default\&region=MAIN_CONTENT_3\&context=storylines_faq}{can
    stay aloft for hours in tiny droplets in stagnant air}, infecting
    people as they inhale, mounting scientific evidence suggests. This
    risk is highest in crowded indoor spaces with poor ventilation, and
    may help explain super-spreading events reported in meatpacking
    plants, churches and restaurants.
    \href{https://www.nytimes.com/2020/07/06/health/coronavirus-airborne-aerosols.html?action=click\&pgtype=Article\&state=default\&region=MAIN_CONTENT_3\&context=storylines_faq}{It's
    unclear how often the virus is spread} via these tiny droplets, or
    aerosols, compared with larger droplets that are expelled when a
    sick person coughs or sneezes, or transmitted through contact with
    contaminated surfaces, said Linsey Marr, an aerosol expert at
    Virginia Tech. Aerosols are released even when a person without
    symptoms exhales, talks or sings, according to Dr. Marr and more
    than 200 other experts, who
    \href{https://academic.oup.com/cid/article/doi/10.1093/cid/ciaa939/5867798}{have
    outlined the evidence in an open letter to the World Health
    Organization}.
  \end{itemize}
\item ~
  \hypertarget{what-are-the-symptoms-of-coronavirus}{%
  \paragraph{What are the symptoms of
  coronavirus?}\label{what-are-the-symptoms-of-coronavirus}}

  \begin{itemize}
  \tightlist
  \item
    Common symptoms
    \href{https://www.nytimes.com/article/symptoms-coronavirus.html?action=click\&pgtype=Article\&state=default\&region=MAIN_CONTENT_3\&context=storylines_faq}{include
    fever, a dry cough, fatigue and difficulty breathing or shortness of
    breath.} Some of these symptoms overlap with those of the flu,
    making detection difficult, but runny noses and stuffy sinuses are
    less common.
    \href{https://www.nytimes.com/2020/04/27/health/coronavirus-symptoms-cdc.html?action=click\&pgtype=Article\&state=default\&region=MAIN_CONTENT_3\&context=storylines_faq}{The
    C.D.C. has also} added chills, muscle pain, sore throat, headache
    and a new loss of the sense of taste or smell as symptoms to look
    out for. Most people fall ill five to seven days after exposure, but
    symptoms may appear in as few as two days or as many as 14 days.
  \end{itemize}
\item ~
  \hypertarget{does-asymptomatic-transmission-of-covid-19-happen}{%
  \paragraph{Does asymptomatic transmission of Covid-19
  happen?}\label{does-asymptomatic-transmission-of-covid-19-happen}}

  \begin{itemize}
  \tightlist
  \item
    So far, the evidence seems to show it does. A widely cited
    \href{https://www.nature.com/articles/s41591-020-0869-5}{paper}
    published in April suggests that people are most infectious about
    two days before the onset of coronavirus symptoms and estimated that
    44 percent of new infections were a result of transmission from
    people who were not yet showing symptoms. Recently, a top expert at
    the World Health Organization stated that transmission of the
    coronavirus by people who did not have symptoms was ``very rare,''
    \href{https://www.nytimes.com/2020/06/09/world/coronavirus-updates.html?action=click\&pgtype=Article\&state=default\&region=MAIN_CONTENT_3\&context=storylines_faq\#link-1f302e21}{but
    she later walked back that statement.}
  \end{itemize}
\end{itemize}

``We had half the population of France that shopped with us,'' said
Hubert Assous, the former chief executive of Tati, now retired in
Israel, who married into the family. ``It was complete madness, for
sales. When we opened in the provinces, we would be taken by storm.''

When the sons took over after Mr. Ouaki's death from cancer in 1983,
``they didn't adapt to the times,'' Mr. Assous said. ``Nothing was
anticipated. We were navigating from day to day.''

A succession of failed ventures and expansions --- Russia, South Africa,
Fifth Avenue in New York, jewelry, a travel agency, high fashion ---
failed to pan out.

``No longer cheap but chic,'' Fabien Ouaki, Jules Ouaki's son, insisted
was the new mantra for the store. It was a fatal miscalculation, and the
company was sold in receivership in 2004.

Subsequent owners failed to invest, said the workers' union and Mr.
Assous. The company that owns the store now, the GPG Group, said sales
dropped 60 percent between October and May. ``The store must close, for
good,'' it said in a news release.

When the store closes, 34 people will lose their jobs, a union official
said.

Image

Barbès is so infused with the spirit of Tati that residents and the few
shoppers left have trouble imagining the neighborhood without
it.Credit...Dmitry Kostyukov for The New York Times

For the remaining customers, it is one more unwelcome transformation at
a time when too many are afoot. There is a kind of disbelief.

``No, no,'' said Sabrina Miloua, a customer toting the gingham. ``This
is too good here. They sell stuff, you can afford it.''

``We bought all of our dishes here,'' she said. ``All the clothing for
the children.''

``Everything, everything,'' her friend, Ms. Bahiti, said.

Théophile Larcher contributed reporting.

Advertisement

\protect\hyperlink{after-bottom}{Continue reading the main story}

\hypertarget{site-index}{%
\subsection{Site Index}\label{site-index}}

\hypertarget{site-information-navigation}{%
\subsection{Site Information
Navigation}\label{site-information-navigation}}

\begin{itemize}
\tightlist
\item
  \href{https://help.nytimes.com/hc/en-us/articles/115014792127-Copyright-notice}{©~2020~The
  New York Times Company}
\end{itemize}

\begin{itemize}
\tightlist
\item
  \href{https://www.nytco.com/}{NYTCo}
\item
  \href{https://help.nytimes.com/hc/en-us/articles/115015385887-Contact-Us}{Contact
  Us}
\item
  \href{https://www.nytco.com/careers/}{Work with us}
\item
  \href{https://nytmediakit.com/}{Advertise}
\item
  \href{http://www.tbrandstudio.com/}{T Brand Studio}
\item
  \href{https://www.nytimes.com/privacy/cookie-policy\#how-do-i-manage-trackers}{Your
  Ad Choices}
\item
  \href{https://www.nytimes.com/privacy}{Privacy}
\item
  \href{https://help.nytimes.com/hc/en-us/articles/115014893428-Terms-of-service}{Terms
  of Service}
\item
  \href{https://help.nytimes.com/hc/en-us/articles/115014893968-Terms-of-sale}{Terms
  of Sale}
\item
  \href{https://spiderbites.nytimes.com}{Site Map}
\item
  \href{https://help.nytimes.com/hc/en-us}{Help}
\item
  \href{https://www.nytimes.com/subscription?campaignId=37WXW}{Subscriptions}
\end{itemize}
