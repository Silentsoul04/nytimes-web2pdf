Sections

SEARCH

\protect\hyperlink{site-content}{Skip to
content}\protect\hyperlink{site-index}{Skip to site index}

\href{https://myaccount.nytimes.com/auth/login?response_type=cookie\&client_id=vi}{}

\href{https://www.nytimes.com/section/todayspaper}{Today's Paper}

\href{/section/opinion}{Opinion}\textbar{}Can Trump Come Back?

\href{https://nyti.ms/2ZKikDu}{https://nyti.ms/2ZKikDu}

\begin{itemize}
\item
\item
\item
\item
\item
\item
\end{itemize}

Advertisement

\protect\hyperlink{after-top}{Continue reading the main story}

\href{/section/opinion}{Opinion}

Supported by

\protect\hyperlink{after-sponsor}{Continue reading the main story}

\hypertarget{can-trump-come-back}{%
\section{Can Trump Come Back?}\label{can-trump-come-back}}

Not under his own power. Only events can save him now.

\href{https://www.nytimes.com/by/ross-douthat}{\includegraphics{https://static01.nyt.com/images/2018/04/03/opinion/ross-douthat/ross-douthat-thumbLarge.png}}

By \href{https://www.nytimes.com/by/ross-douthat}{Ross Douthat}

Opinion Columnist

\begin{itemize}
\item
  July 21, 2020
\item
  \begin{itemize}
  \item
  \item
  \item
  \item
  \item
  \item
  \end{itemize}
\end{itemize}

\includegraphics{https://static01.nyt.com/images/2020/07/21/opinion/21douthatWeb/21douthatWeb-articleLarge.jpg?quality=75\&auto=webp\&disable=upscale}

Here is what Donald Trump can realistically do to right his poll
numbers, regain his lost supporters and make the 2020 presidential race
close instead of a 10-point Joe Biden blowout.

(voice drops to a whisper)

\emph{Nothing.}

The word ``realistically'' is crucial here. On paper, Trump is president
of the United States, the most powerful man in the world, possessed of
all the advantages of incumbency and all the powers of his office. In
theory he could initiate, tomorrow, a new nationwide anti-Covid-19
effort aimed at full suppression of the coronavirus, which if successful
could have the economy running hot again by Election Day. In theory he
could do many other things as well: barnstorm the country giving
eloquent speeches on the American idea and racial reconciliation, launch
a dozen clever policy gambits to wrong-foot the Democrats, cruise into
Hong Kong's harbor with a ``Free Xinjiang'' banner hanging from the
presidential yacht.

But realistically one need only watch his Sunday interview with Chris
Wallace to see that the president is still the man that he has spent his
entire life becoming: a character out of C.S. Lewis's ``The Great
Divorce,'' in which the narrator visits a gray suburb of hell and finds
it populated by souls who are self-imprisoned, incapable of freedom,
their personalities reduced to grievances and grumbles. A couple of
inhabitants go in search of Napoleon, who has ``built himself a huge
house all in the Empire style --- rows of windows flaming with light,''
and inside all he does is pace:

\begin{quote}
``Walking up and down --- up and down all the time --- left-right,
left-right --- never stopping for a moment. The two chaps watched him
for about a year and he never rested. And muttering to himself all the
time. `It was Soult's fault. It was Ney's fault. It was Josephine's
fault. It was the fault of the Russians. It was the fault of the
English.' Like that all the time. Never stopped for a moment. A little,
fat man and he looked kind of tired. But he didn't seem able to stop
it.''
\end{quote}

This is Trump as his first term as president concludes. At least 140,000
Americans are dead in a still-burning pandemic, the unemployment rate is
11 percent, and when asked by Wallace how his presidency will be
remembered, all he offered was the old pre-Covid litany of grievance,
starting with ``I think I was very unfairly treated'' and continuing for
paragraphs in the same self-pitying vein.

So let's be real: There is no strategy that this president could adopt,
no policy choice that he could make, no
\href{https://twitter.com/realDonaldTrump/status/1285299379746811915}{tweet}
of himself in a mask that he could issue, that would fundamentally alter
his political position. Trump is incapable of normal presidential
action, and even if his aides and handlers concocted such a strategy,
the man in charge would make sure it would fail. A new suppression
program? A vision for economic recovery? Forget it. ``It was the media's
fault. It was NeverTrump's fault. It was Obama's fault. It was the fault
of Robert Mueller. It was the fault of Jeff Sessions. \ldots''

But that doesn't make his defeat inevitable. It only means that to
speculate about a Trump comeback is to necessarily speculate about
possibilities that are outside the president's control.

Some of these possibilities are outside the realm of easy pundit
extrapolation, too. (None of last year's 2020 punditry predicted the
pandemic.) But if we just stick with the two most important issues of
the moment, the coronavirus and the protests and unrest in American
cities, we can imagine that Trump might benefit politically if the first
gets suddenly better and the second gets much worse.

This president isn't going to suppress the pandemic, so he needs it to
do what he keeps suggesting, wistfully, that it might do, and simply go
away. That's unlikely, but it's not quite impossible. Sweden, ground
zero for the quest for herd immunity, has seen its caseload decline and
its death rate fall without having reached the terrifying fatality rates
that you would expect if the virus needed to infect 70 percent of
society before burning itself out. New York, our nightmare state three
months ago, has partly reopened (and hosted major protests!) while
\href{https://www1.nyc.gov/site/doh/covid/covid-19-data-deaths.page}{keeping
its death curve flat.}

Both of these case studies are modest evidence for the theory that herd
immunity for this virus starts to kick in once around 20 percent of a
population is infected, not 60 or 70 percent.

The Atlantic's James Hamblin had a good
\href{https://www.theatlantic.com/health/archive/2020/07/herd-immunity-coronavirus/614035/}{write-up}
of these speculations recently: The idea is basically that if there's a
wide variation in susceptibility to a disease, and the sickness burns
through most of the high-susceptibility targets in its first big sweep,
subsequent sweeps will be much slower, and their case rates and death
rates far lower than in a scenario where most people have roughly
similar susceptibility. Which could --- \emph{could} --- be what we're
seeing in New York and some of the formerly hardest-hit European
countries now.

More likely, herd immunity is somewhere above 20 percent and below 70
percent, in the vast space between the most-hopeful and the worst-case
possibilities.

But the 20 percent scenario offers Trump's best hope for an autumn
comeback. It would mean that this summer's surges need not be replicated
in the fall, and that by the time Election Day arrives there could be a
feeling of normalization, a recovery of growth, even a safe reopening of
schools, that right now seems altogether out of reach.

Note that I'm not saying this would save Trump, just that it would
clearly help him on a scale far beyond anything he can do himself. At
the very least it would help solve one of his leading political
problems, which is that a certain number of Republican-leaning voters
feel straightforwardly afraid to vote for him again, lest the United
States end up stuck in its current viral agony for months or years to
come.

But then there still remains the difficulty that not enough Americans
are afraid to vote for Joe Biden, notwithstanding the Trump campaign's
attempt to brand him as the candidate of Antifa. For this to change, the
threat of crime and disorder would need to become much more palpable to
the suburban voters who have abandoned Trump, so that he ceases to look
like a goon overreacting to peaceful protests and begins to look like
the hard man for hard times he imagines himself to be. And Biden, who
right now doesn't have to do much more than speak about race and crime
in comforting platitudes, would need to find himself a bit more
squeezed, between the activist wing of his party and the ``racism is
terrible but so is crime'' voters who are giving him his current lead.

More violence is entirely possible, with the
\href{https://www.nytimes.com/2020/07/06/upshot/murders-rising-crime-coronavirus.html}{steady
worsening of homicide rates} in many cities probably a more important
indicator than the mobs at statues or at Amazon stores. There is a lot
of late-1960s naïveté at work in progressive politics right now, and the
basic sympathy a majority of Americans feel for the protests and the
Black Lives Matters movement might not survive a 1970s-style degradation
of public safety.

But a degradation that happens fast enough to doom a candidate like
Biden seems even more unlikely than the ``early herd immunity''
possibility.

First, because the way Americans live now means that swing voters in the
suburbs are a lot more buffered from urban crime than were white-ethnic
swing voters in the 1970s, and the liberal gentrifiers who feel unsafe
when crime spikes in Chicago or Washington D.C. are extremely unlikely
to vote for any Republican, let alone for Trump.

Second, because Trump's own overreactions, in Lafayette Park especially,
have locked in an image of him as an instigator in his own right, an
arsonist in the White House whose presence there can only make matters
that much worse. Maybe there is a threshold of violence where this image
changes and his instigation starts to look like necessary toughness. But
it's also possible that Trump's incapacities now extend to an inability
to \emph{ever} look like the law-and-order candidate, no matter how many
times he tweets the phrase.

Finally, because if Trump gets lucky and gets the first form of help he
needs, the earlier-than-expected coronavirus herd immunity, it's likely
that the subsequent normalization will have a calming effect on urban
unrest, pulling people off the streets and turning down political
temperatures at the moment when literal temperatures begin to fall as
well. A scenario where the virus goes away but the suburbs feel
\emph{more} besieged than they do right now seems like one of the least
likely combinations one could conjure.

So having set out to imagine two possibilities that could help make this
a much closer race, at best I've imagined one and a half. Which doesn't
mean, again, that there aren't other, wilder ones. But the exercise has
clarified the limits of my own imagination: Even when I'm waxing
speculative, I have a hard time imagining a plausible November in which
Donald Trump can win.

\emph{The Times is committed to publishing}
\href{https://www.nytimes.com/2019/01/31/opinion/letters/letters-to-editor-new-york-times-women.html}{\emph{a
diversity of letters}} \emph{to the editor. We'd like to hear what you
think about this or any of our articles. Here are some}
\href{https://help.nytimes.com/hc/en-us/articles/115014925288-How-to-submit-a-letter-to-the-editor}{\emph{tips}}\emph{.
And here's our email:}
\href{mailto:letters@nytimes.com}{\emph{letters@nytimes.com}}\emph{.}

\emph{Follow The New York Times Opinion section on}
\href{https://www.facebook.com/nytopinion}{\emph{Facebook}}\emph{,}
\href{http://twitter.com/NYTOpinion}{\emph{Twitter (@NYTOpinion)}}
\emph{and}
\href{https://www.instagram.com/nytopinion/}{\emph{Instagram}}\emph{,
join the Facebook political discussion group,}
\href{https://www.facebook.com/groups/votingwhilefemale/}{\emph{Voting
While Female}}\emph{.}

Advertisement

\protect\hyperlink{after-bottom}{Continue reading the main story}

\hypertarget{site-index}{%
\subsection{Site Index}\label{site-index}}

\hypertarget{site-information-navigation}{%
\subsection{Site Information
Navigation}\label{site-information-navigation}}

\begin{itemize}
\tightlist
\item
  \href{https://help.nytimes.com/hc/en-us/articles/115014792127-Copyright-notice}{©~2020~The
  New York Times Company}
\end{itemize}

\begin{itemize}
\tightlist
\item
  \href{https://www.nytco.com/}{NYTCo}
\item
  \href{https://help.nytimes.com/hc/en-us/articles/115015385887-Contact-Us}{Contact
  Us}
\item
  \href{https://www.nytco.com/careers/}{Work with us}
\item
  \href{https://nytmediakit.com/}{Advertise}
\item
  \href{http://www.tbrandstudio.com/}{T Brand Studio}
\item
  \href{https://www.nytimes.com/privacy/cookie-policy\#how-do-i-manage-trackers}{Your
  Ad Choices}
\item
  \href{https://www.nytimes.com/privacy}{Privacy}
\item
  \href{https://help.nytimes.com/hc/en-us/articles/115014893428-Terms-of-service}{Terms
  of Service}
\item
  \href{https://help.nytimes.com/hc/en-us/articles/115014893968-Terms-of-sale}{Terms
  of Sale}
\item
  \href{https://spiderbites.nytimes.com}{Site Map}
\item
  \href{https://help.nytimes.com/hc/en-us}{Help}
\item
  \href{https://www.nytimes.com/subscription?campaignId=37WXW}{Subscriptions}
\end{itemize}
