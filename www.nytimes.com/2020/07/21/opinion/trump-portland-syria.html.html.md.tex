Sections

SEARCH

\protect\hyperlink{site-content}{Skip to
content}\protect\hyperlink{site-index}{Skip to site index}

\href{https://myaccount.nytimes.com/auth/login?response_type=cookie\&client_id=vi}{}

\href{https://www.nytimes.com/section/todayspaper}{Today's Paper}

\href{/section/opinion}{Opinion}\textbar{}Trump's Wag-the-Dog War

\href{https://nyti.ms/3jqKriM}{https://nyti.ms/3jqKriM}

\begin{itemize}
\item
\item
\item
\item
\item
\item
\end{itemize}

Advertisement

\protect\hyperlink{after-top}{Continue reading the main story}

\href{/section/opinion}{Opinion}

Supported by

\protect\hyperlink{after-sponsor}{Continue reading the main story}

\hypertarget{trumps-wag-the-dog-war}{%
\section{Trump's Wag-the-Dog War}\label{trumps-wag-the-dog-war}}

The president is looking for a dangerous domestic enemy to fight.

\href{https://www.nytimes.com/by/thomas-l-friedman}{\includegraphics{https://static01.nyt.com/images/2018/04/02/opinion/thomas-l-friedman/thomas-l-friedman-thumbLarge.png}}

By \href{https://www.nytimes.com/by/thomas-l-friedman}{Thomas L.
Friedman}

Opinion Columnist

\begin{itemize}
\item
  July 21, 2020
\item
  \begin{itemize}
  \item
  \item
  \item
  \item
  \item
  \item
  \end{itemize}
\end{itemize}

\includegraphics{https://static01.nyt.com/images/2020/07/21/opinion/21friedman1/merlin_174530535_b753e93b-5486-4db4-927b-278f18253604-articleLarge.jpg?quality=75\&auto=webp\&disable=upscale}

Some presidents, when they get into trouble before an election, try to
``wag the dog'' by starting a war abroad. Donald Trump seems ready to
wag the dog by starting a war at home. Be afraid --- he just might get
his wish.

How did we get here? Well, when historians summarize the Trump team's
approach to dealing with the coronavirus, it will take only a few
paragraphs:

``They talked as if they were locking down like China. They acted as if
they were going for herd immunity like Sweden. They prepared for
neither. And they claimed to be superior to both. In the end, they got
the worst of all worlds --- uncontrolled viral spread and an
unemployment catastrophe.

``And then the story turned really dark.

``As the virus spread, and businesses had to shut down again and schools
and universities were paralyzed as to whether to open or stay closed in
the fall, Trump's poll numbers nose-dived. Joe Biden opened up a
15-point lead in a national head-to-head survey.

``So, in a desperate effort to salvage his campaign, Trump turned to the
Middle East Dictator's Official Handbook and found just what he was
looking for, the chapter titled, `What to Do When Your People Turn
Against You?'

``Answer: Turn them against each other and then present yourself as the
only source of law and order.''

America blessedly is not Syria, yet, but Trump is adopting the same
broad approach that Bashar al-Assad did back in 2011, when peaceful
protests broke out in the southern Syrian town of Dara'a, calling for
democratic reforms; the protests then spread throughout the country.

Had al-Assad responded with even the mildest offer of more participatory
politics, he would have been hailed as a savior by a majority of
Syrians. One of their main chants during the demonstrations was,
``Silmiya, silmiya'' (``Peaceful, peaceful'').

But al-Assad did not want to share power, and so he made sure that the
protests were not peaceful. He had his soldiers open fire on and arrest
nonviolent demonstrators, many of them Sunni Muslims. Over time, the
peaceful, secular elements of the Syrian democracy movement were
sidelined, as hardened Islamists began to spearhead the fight against
al-Assad. In the process, the uprising was transformed into a naked,
rule-or-die sectarian civil war between al-Assad's Alawite Shiite forces
and various Sunni jihadist groups.

Al-Assad got exactly what he wanted --- not a war between his
dictatorship and his people peacefully asking to have their voices
heard, but a war with Islamic radicals in which he could play the
law-and-order president, backed by Russia and Iran. In the end, his
country was destroyed and hundreds of thousands of Syrians were killed
or forced to flee. But al-Assad stayed in power. Today, he's the top dog
on a pile of rubble.

Image

A banner depicting the Syrian dictator Bashar al-Assad hanging in late
2018 in Douma, Syria. The town was retaken months earlier by the
government from rebels after heavy fighting and
airstrikes.Credit...Marko Djurica/Reuters

Image

Syrians walking amid the rubble of damaged buildings after an airstrike
in 2018.Credit...Mohammed Badra/EPA, via Shutterstock

I have zero tolerance for any American protesters who resort to violence
in any U.S. city, because it damages homes and businesses already
hammered by the coronavirus --- many of them minority-owned --- and
because violence will only turn off and repel the majority needed to
drive change.

But when I heard Trump suggest, as he did in the Oval Office on Monday,
that he was going to send federal forces into U.S. cities, where the
local mayors have not invited him, the first word that popped into my
head was ``Syria.''

Listen to
\href{https://www.whitehouse.gov/briefings-statements/remarks-president-trump-phase-four-negotiations/}{how
Trump put it}: ``I'm going to do something --- that, I can tell you.
Because we're not going to let New York and Chicago and Philadelphia and
Detroit and Baltimore and all of these --- Oakland is a mess. We're not
going to let this happen in our country.''

These cities, Trump stressed, are ``all run by very liberal Democrats.
All run, really, by radical left. If Biden got in, that would be true
for the country. The whole country would go to hell. And we're not going
to let it go to hell.''

This is coming so straight from the Middle East Dictator's Handbook,
it's chilling. In Syria, al-Assad used plainclothes, pro-regime thugs,
known as the shabiha (``the apparitions'') to make protesters disappear.
In Portland, Ore., we saw militarized federal forces wearing battle
fatigues, but no identifiable markings, arresting people and putting
them into unmarked vans. How can this happen in America?

Authoritarian populists --- whether Recep Tayyip Erdogan in Turkey, Jair
Bolsonaro in Brazil, Rodrigo Duterte in the Philippines, Vladimir Putin
in Russia, Viktor Orban in Hungary, Jaroslaw Kaczynski in Poland, or
al-Assad --- ``win by dividing the people and presenting themselves as
the savior of the good and ordinary citizens against the undeserving
agents of subversion and `cultural pollution,''' explained Stanford's
Larry Diamond, author of
``\href{https://diamond-democracy.stanford.edu/publications/ill-winds-saving-democracy-russian-rage-chinese-ambition-and-american-complacency}{Ill
Winds}: Saving Democracy From Russian Rage, Chinese Ambition, and
American Complacency.''

In the face of such a threat, the left needs to be smart. Stop calling
for ``defunding the police'' and then saying that ``defunding'' doesn't
mean disbanding. If it doesn't mean that then say what it means:
``reform.'' **** Defunding the police, calling police officers ``pigs,''
taking over whole neighborhoods with barricades --- these are terrible
messages, not to mention strategies, easily exploitable by Trump.

The scene that The Times's
\href{https://www.nytimes.com/2020/07/21/us/portland-protests.html}{Mike
Baker described} from Portland in the early hours of Tuesday --- Day 54
of the protests there --- is not good: ``Some leaders in the Black
community, grateful for a reckoning on race, worry that what should be a
moment for racial justice could be squandered by violence. Businesses
supportive of reforms have been left demoralized by the mayhem the
protests have brought. \ldots{} On Tuesday morning, police said another
jewelry store had been looted. As federal agents appeared to try
detaining one person, others in the crowd rushed to free the person.''

\includegraphics{https://static01.nyt.com/images/2020/07/21/opinion/21friedman4/merlin_174794280_39d41952-5fff-4e3e-b3f2-fdee72b4c3f9-articleLarge.jpg?quality=75\&auto=webp\&disable=upscale}

A new \href{https://wapo.st/3jiOiyp}{Washington Post-ABC News poll,}
according to The Post, found that a ``majority of Americans support the
Black Lives Matter movement and a record 69 percent say Black people and
other minorities are not treated as equal to white people in the
criminal justice system. But the public generally opposes calls to shift
some police funding to social services or remove statues of Confederate
generals or presidents who enslaved people.''

All of this street violence and defund-the-police rhetoric plays into
the only
\href{https://www.nytimes.com/2020/07/21/us/politics/trump-portland-federal-agents.html}{effective
Trump ad} that I've seen on television. It goes like this: A phone rings
and a recording begins: ``You have reached the 911 police emergency
line. Due to defunding of the police department, we're sorry but no one
is here to take your call. If you're calling to report a rape, please
press 1. To report a murder, press 2. To report a home invasion, press
3. For all other crimes, leave your name and number and someone will get
back to you. Our estimated wait time is currently five days. Goodbye.''

Today's protesters need to trump Trump by taking a page from another
foreign leader --- a liberal --- Ekrem Imamoglu, who managed to win the
2019 election to become the mayor of Istanbul, despite the illiberal
Erdogan using every dirty trick possible to steal the election.
Imamoglu's campaign strategy was called ``radical love.''

Radical love meant reaching out to the more traditional and religious
Erdogan supporters, listening to them, showing them respect and making
clear that they were not ``the enemy'' --- that Erdogan was the enemy,
because he was the enemy of unity and mutual respect, and there could be
no progress without them.

As a
\href{https://www.journalofdemocracy.org/articles/the-pushback-against-populism-running-on-radical-love-in-turkey/}{recent
essay} on Imamoglu's strategy in The Journal of Democracy noted, he
overcame Erdogan with a ``message of inclusiveness, an attitude of
respect toward {[}Erdogan{]} supporters, and a focus on bread-and-butter
issues that could unite voters across opposing political camps. On June
23, Imamoglu was again elected mayor of Istanbul, but this time with
more than 54 percent of the vote --- the largest mandate obtained by an
Istanbul mayor since 1984 --- against 45 percent for his opponent.''

Radical love. Wow. I bet that could work in America, too. It's the
perfect answer to Trump's politics of division --- and it's the one
strategy he'll never imitate.

\emph{The Times is committed to publishing}
\href{https://www.nytimes.com/2019/01/31/opinion/letters/letters-to-editor-new-york-times-women.html}{\emph{a
diversity of letters}} \emph{to the editor. We'd like to hear what you
think about this or any of our articles. Here are some}
\href{https://help.nytimes.com/hc/en-us/articles/115014925288-How-to-submit-a-letter-to-the-editor}{\emph{tips}}\emph{.
And here's our email:}
\href{mailto:letters@nytimes.com}{\emph{letters@nytimes.com}}\emph{.}

\emph{Follow The New York Times Opinion section on}
\href{https://www.facebook.com/nytopinion}{\emph{Facebook}}\emph{,}
\href{http://twitter.com/NYTOpinion}{\emph{Twitter (@NYTopinion)}}
\emph{and}
\href{https://www.instagram.com/nytopinion/}{\emph{Instagram}}\emph{.}

Advertisement

\protect\hyperlink{after-bottom}{Continue reading the main story}

\hypertarget{site-index}{%
\subsection{Site Index}\label{site-index}}

\hypertarget{site-information-navigation}{%
\subsection{Site Information
Navigation}\label{site-information-navigation}}

\begin{itemize}
\tightlist
\item
  \href{https://help.nytimes.com/hc/en-us/articles/115014792127-Copyright-notice}{©~2020~The
  New York Times Company}
\end{itemize}

\begin{itemize}
\tightlist
\item
  \href{https://www.nytco.com/}{NYTCo}
\item
  \href{https://help.nytimes.com/hc/en-us/articles/115015385887-Contact-Us}{Contact
  Us}
\item
  \href{https://www.nytco.com/careers/}{Work with us}
\item
  \href{https://nytmediakit.com/}{Advertise}
\item
  \href{http://www.tbrandstudio.com/}{T Brand Studio}
\item
  \href{https://www.nytimes.com/privacy/cookie-policy\#how-do-i-manage-trackers}{Your
  Ad Choices}
\item
  \href{https://www.nytimes.com/privacy}{Privacy}
\item
  \href{https://help.nytimes.com/hc/en-us/articles/115014893428-Terms-of-service}{Terms
  of Service}
\item
  \href{https://help.nytimes.com/hc/en-us/articles/115014893968-Terms-of-sale}{Terms
  of Sale}
\item
  \href{https://spiderbites.nytimes.com}{Site Map}
\item
  \href{https://help.nytimes.com/hc/en-us}{Help}
\item
  \href{https://www.nytimes.com/subscription?campaignId=37WXW}{Subscriptions}
\end{itemize}
