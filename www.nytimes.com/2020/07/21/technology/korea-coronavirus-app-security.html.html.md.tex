Sections

SEARCH

\protect\hyperlink{site-content}{Skip to
content}\protect\hyperlink{site-index}{Skip to site index}

\href{https://www.nytimes.com/section/technology}{Technology}

\href{https://myaccount.nytimes.com/auth/login?response_type=cookie\&client_id=vi}{}

\href{https://www.nytimes.com/section/todayspaper}{Today's Paper}

\href{/section/technology}{Technology}\textbar{}Major Security Flaws
Found in South Korea Quarantine App

\url{https://nyti.ms/2ZMeXf3}

\begin{itemize}
\item
\item
\item
\item
\item
\end{itemize}

\href{https://www.nytimes.com/news-event/coronavirus?action=click\&pgtype=Article\&state=default\&region=TOP_BANNER\&context=storylines_menu}{The
Coronavirus Outbreak}

\begin{itemize}
\tightlist
\item
  live\href{https://www.nytimes.com/2020/08/01/world/coronavirus-covid-19.html?action=click\&pgtype=Article\&state=default\&region=TOP_BANNER\&context=storylines_menu}{Latest
  Updates}
\item
  \href{https://www.nytimes.com/interactive/2020/us/coronavirus-us-cases.html?action=click\&pgtype=Article\&state=default\&region=TOP_BANNER\&context=storylines_menu}{Maps
  and Cases}
\item
  \href{https://www.nytimes.com/interactive/2020/science/coronavirus-vaccine-tracker.html?action=click\&pgtype=Article\&state=default\&region=TOP_BANNER\&context=storylines_menu}{Vaccine
  Tracker}
\item
  \href{https://www.nytimes.com/interactive/2020/07/29/us/schools-reopening-coronavirus.html?action=click\&pgtype=Article\&state=default\&region=TOP_BANNER\&context=storylines_menu}{What
  School May Look Like}
\item
  \href{https://www.nytimes.com/live/2020/07/31/business/stock-market-today-coronavirus?action=click\&pgtype=Article\&state=default\&region=TOP_BANNER\&context=storylines_menu}{Economy}
\end{itemize}

Advertisement

\protect\hyperlink{after-top}{Continue reading the main story}

Supported by

\protect\hyperlink{after-sponsor}{Continue reading the main story}

\hypertarget{major-security-flaws-found-in-south-korea-quarantine-app}{%
\section{Major Security Flaws Found in South Korea Quarantine
App}\label{major-security-flaws-found-in-south-korea-quarantine-app}}

The defects, which have been fixed, exposed private details of people in
quarantine. The country has been hailed as a pioneer in digital public
health.

\includegraphics{https://static01.nyt.com/images/2020/07/22/business/22virus-korea-app-1/merlin_170765895_bedadef7-1f80-440f-a0d8-3063d0083c6f-articleLarge.jpg?quality=75\&auto=webp\&disable=upscale}

By \href{https://www.nytimes.com/by/choe-sang-hun}{Choe Sang-Hun}, Aaron
Krolik, \href{https://www.nytimes.com/by/raymond-zhong}{Raymond Zhong}
and \href{https://www.nytimes.com/by/natasha-singer}{Natasha Singer}

\begin{itemize}
\item
  Published July 21, 2020Updated July 28, 2020
\item
  \begin{itemize}
  \item
  \item
  \item
  \item
  \item
  \end{itemize}
\end{itemize}

SEOUL, South Korea ---
\href{https://www.nytimes.com/2020/07/28/world/asia/south-korea-satellites-rockets.html}{South
Korea} has been praised for making effective use of digital tools to
\href{https://www.nytimes.com/2020/03/23/world/asia/coronavirus-south-korea-flatten-curve.html}{contain
the coronavirus}, from emergency phone alerts to aggressive contact
tracing based on a variety of data.

But one pillar of that strategy, a mobile app that helps enforce
quarantines, had serious security flaws that made private information
vulnerable to hackers, a software engineer has found.

The defects, which were confirmed by The New York Times and have now
been fixed, could have let attackers retrieve the names, real-time
locations and other details of people in quarantine. The flaws could
also have allowed hackers to tamper with data to make it look as if
users of the app were either violating quarantine orders or still in
quarantine despite being somewhere else.

In interviews, South Korean officials acknowledged that they had become
aware of the security lapses only after the engineer, Frédéric
Rechtenstein, and The Times notified them.

``We were really in a hurry to make and deploy this app as quickly as
possible to help slow down the spread of the virus,'' said Jung
Chan-hyun, an official at the Ministry of the Interior and Safety's
disaster response division, which oversees the app. ``We could not
afford a time-consuming security check on the app that would delay its
deployment.''

The ministry fixed the flaws in the latest version of the app, which was
released in
\href{https://play.google.com/store/apps/details?id=kr.go.safekorea.sqsm\&hl=ko}{Google}
and
\href{https://apps.apple.com/us/app/\%EC\%9E\%90\%EA\%B0\%80\%EA\%B2\%A9\%EB\%A6\%AC\%EC\%9E\%90-\%EC\%95\%88\%EC\%A0\%84\%EB\%B3\%B4\%ED\%98\%B8/id1502372537}{Apple}
stores last week. South Korean officials said they had not received any
reports that personal information was improperly retrieved or misused
before the vulnerabilities were patched.

Governments worldwide have raced to deploy virus-tracing apps only to
face complaints about
\href{https://www.nytimes.com/2020/07/08/technology/virus-tracing-apps-privacy.html}{poor
security practices}. With the software gathering so many details about
users, their health and their locations, the apps are prime targets for
hackers. But pressure to act quickly appears to have allowed software
with inadequate security features to be rushed out in several nations.

\hypertarget{latest-updates-economy}{%
\section{\texorpdfstring{\href{https://www.nytimes.com/live/2020/07/31/business/stock-market-today-coronavirus?action=click\&pgtype=Article\&state=default\&region=MAIN_CONTENT_1\&context=storylines_live_updates}{Latest
Updates:
Economy}}{Latest Updates: Economy}}\label{latest-updates-economy}}

\href{https://www.nytimes.com/live/2020/07/31/business/stock-market-today-coronavirus?action=click\&pgtype=Article\&state=default\&region=MAIN_CONTENT_1\&context=storylines_live_updates\#kodaks-chief-executive-was-given-stock-options-then-the-share-price-spiked-1000-percent}{34h
ago}

\href{https://www.nytimes.com/live/2020/07/31/business/stock-market-today-coronavirus?action=click\&pgtype=Article\&state=default\&region=MAIN_CONTENT_1\&context=storylines_live_updates\#kodaks-chief-executive-was-given-stock-options-then-the-share-price-spiked-1000-percent}{Kodak's
chief executive was given stock options. Then the share price spiked
1,000 percent.}

\href{https://www.nytimes.com/live/2020/07/31/business/stock-market-today-coronavirus?action=click\&pgtype=Article\&state=default\&region=MAIN_CONTENT_1\&context=storylines_live_updates\#fitch-ratings-downgrades-its-outlook-on-us-debt}{37h
ago}

\href{https://www.nytimes.com/live/2020/07/31/business/stock-market-today-coronavirus?action=click\&pgtype=Article\&state=default\&region=MAIN_CONTENT_1\&context=storylines_live_updates\#fitch-ratings-downgrades-its-outlook-on-us-debt}{Fitch
Ratings downgrades its outlook on U.S. debt.}

\href{https://www.nytimes.com/live/2020/07/31/business/stock-market-today-coronavirus?action=click\&pgtype=Article\&state=default\&region=MAIN_CONTENT_1\&context=storylines_live_updates\#us-sanctions-more-chinese-officials-over-human-rights-violations-as-tensions-flare}{44h
ago}

\href{https://www.nytimes.com/live/2020/07/31/business/stock-market-today-coronavirus?action=click\&pgtype=Article\&state=default\&region=MAIN_CONTENT_1\&context=storylines_live_updates\#us-sanctions-more-chinese-officials-over-human-rights-violations-as-tensions-flare}{U.S.
sanctions more Chinese officials over human rights violations as
tensions flare}

\href{https://www.nytimes.com/live/2020/07/31/business/stock-market-today-coronavirus?action=click\&pgtype=Article\&state=default\&region=MAIN_CONTENT_1\&context=storylines_live_updates}{See
more updates}

More live coverage:
\href{https://www.nytimes.com/2020/08/01/world/coronavirus-covid-19.html?action=click\&pgtype=Article\&state=default\&region=MAIN_CONTENT_1\&context=storylines_live_updates}{Global}

The Times found this spring that a
\href{https://www.nytimes.com/2020/04/29/business/coronavirus-cellphone-apps-contact-tracing.html}{virus-tracing
app in India} could leak users' precise locations, prompting the Indian
government to fix the problem. Amnesty International discovered flaws in
an
\href{https://www.amnesty.org/en/latest/news/2020/05/qatar-covid19-contact-tracing-app-security-flaw/}{exposure-alert
app in Qatar}, which the authorities there quickly updated. Other
nations, including Norway and Britain, have had to change course on
their virus apps after public outcry about privacy.

In April, South Korea began requiring all visitors and residents
arriving from abroad to isolate themselves for two weeks. To monitor
compliance, they had to install an app whose name in Korean means
Self-Quarantine Safety Protection.

As of last month, more than 162,000 people had downloaded the app, which
tracks users' locations to ensure they remain in quarantine areas.
Violators might be required to wear tracking wristbands or pay steep
fines.

\includegraphics{https://static01.nyt.com/images/2020/07/10/business/00virus-korea-app-2/merlin_170621106_4d1a7f35-d52e-4bc0-bf85-630d1a1dedb0-articleLarge.jpg?quality=75\&auto=webp\&disable=upscale}

In May, Mr. Rechtenstein returned to his home in Seoul from a trip
abroad. While self-isolating at home, he became curious about the
government's seemingly simple app and what extra features it might have.
That prompted Mr. Rechtenstein to peek under the hood of the code, which
is how he discovered several major security flaws.

He found that the software's developers were assigning users ID numbers
that were easily guessable. After guessing a person's credentials, a
hacker could have retrieved the information provided upon registration,
including name, date of birth, sex, nationality, address, phone number,
real-time location and medical symptoms.

Mr. Rechtenstein also found that the developers were using an insecure
method to scramble, or encrypt, the app's communications with the server
where data was stored. Instead of HTTPS, the security standard used by
apps like Gmail and Twitter, the app used an encryption key written
directly into its code.

Doing so meant hackers could easily find the key and decode the data if
they had tried. It also meant the key did not change depending on the
message being sent or on the user sending it.

The key was also far from random: It was ``1234567890123456.''

With such weak encryption, monitoring all of the app's communications
with the server would be possible simply, for instance, by being on the
same unprotected Wi-Fi network as someone else using the app.

The Times examined the app's code and confirmed Mr. Rechtenstein's
findings. After The Times approached the South Korean authorities about
the security flaws last month, officials said they had put a priority on
deploying the app quickly ``to save lives.''

Mr. Jung, the Interior Ministry official, said his team had developed
the app with Winitech, a software maintenance and repair company in
Daegu, a South Korean city that became
\href{https://www.nytimes.com/2020/02/25/world/asia/daegu-south-korea-coronavirus.html}{a
center of the outbreak} in February.

Winitech's senior managing director, Hong Seong-bok, said that when the
company first developed the app, it expected that only a small number of
South Koreans would ever use the software.

``We had never thought that it would be used by so many people, becoming
a must-install app for all arrivals at the airport,'' Mr. Hong said.

Mr. Jung said that while the group had worked around the clock to
develop the app and train officials on how to use it, they lacked the
expertise to make the software secure.

Over time, the government also asked Mr. Jung's team to add surveillance
functions to the app, which officials said had increased their workload
and prevented them from spending time hunting for security flaws.

A feature was added, for instance, that caused a quarantined person's
phone to emit a noise or vibrate when it was not physically moved for
more than two hours. If the user did not respond by picking up the
device, it was a potential sign that the person had ventured out and
left the phone behind. The app would then alert the authorities.

To keep a closer watch on quarantine violators, another function was
added to connect tracking wristbands to the app.

``We were simply overwhelmed with work,'' said Koo Chang-kyu, a South
Korean official.

In meetings last month with Mr. Rechtenstein and a Times reporter, South
Korean officials initially played down the security issues, saying that
they had deleted personal data and disabled the app once a user
completed the two-week quarantine.

But Mr. Rechtenstein demonstrated in the meeting that his data could
still be retrieved from the government server by using the app on his
phone, even though his quarantine had ended more than a week earlier.
South Korean officials later said they had fixed the problem.

South Korea has become a global poster child for its creative and
transparent handling of the coronavirus pandemic. But the app's security
flaws show how the country lags in protecting personal data, Mr.
Rechtenstein said. He also expressed disappointment at how long it took
the authorities to fix the problems.

The episode could ``affect perceptions about the Korean model'' for
combating the pandemic, Mr. Rechtenstein said.

Advertisement

\protect\hyperlink{after-bottom}{Continue reading the main story}

\hypertarget{site-index}{%
\subsection{Site Index}\label{site-index}}

\hypertarget{site-information-navigation}{%
\subsection{Site Information
Navigation}\label{site-information-navigation}}

\begin{itemize}
\tightlist
\item
  \href{https://help.nytimes.com/hc/en-us/articles/115014792127-Copyright-notice}{©~2020~The
  New York Times Company}
\end{itemize}

\begin{itemize}
\tightlist
\item
  \href{https://www.nytco.com/}{NYTCo}
\item
  \href{https://help.nytimes.com/hc/en-us/articles/115015385887-Contact-Us}{Contact
  Us}
\item
  \href{https://www.nytco.com/careers/}{Work with us}
\item
  \href{https://nytmediakit.com/}{Advertise}
\item
  \href{http://www.tbrandstudio.com/}{T Brand Studio}
\item
  \href{https://www.nytimes.com/privacy/cookie-policy\#how-do-i-manage-trackers}{Your
  Ad Choices}
\item
  \href{https://www.nytimes.com/privacy}{Privacy}
\item
  \href{https://help.nytimes.com/hc/en-us/articles/115014893428-Terms-of-service}{Terms
  of Service}
\item
  \href{https://help.nytimes.com/hc/en-us/articles/115014893968-Terms-of-sale}{Terms
  of Sale}
\item
  \href{https://spiderbites.nytimes.com}{Site Map}
\item
  \href{https://help.nytimes.com/hc/en-us}{Help}
\item
  \href{https://www.nytimes.com/subscription?campaignId=37WXW}{Subscriptions}
\end{itemize}
