Sections

SEARCH

\protect\hyperlink{site-content}{Skip to
content}\protect\hyperlink{site-index}{Skip to site index}

\href{https://www.nytimes.com/section/politics}{Politics}

\href{https://myaccount.nytimes.com/auth/login?response_type=cookie\&client_id=vi}{}

\href{https://www.nytimes.com/section/todayspaper}{Today's Paper}

\href{/section/politics}{Politics}\textbar{}U.S. Accuses Hackers of
Trying to Steal Coronavirus Vaccine Data for China

\href{https://nyti.ms/2CyTMED}{https://nyti.ms/2CyTMED}

\begin{itemize}
\item
\item
\item
\item
\item
\item
\end{itemize}

\href{https://www.nytimes.com/news-event/coronavirus?action=click\&pgtype=Article\&state=default\&region=TOP_BANNER\&context=storylines_menu}{The
Coronavirus Outbreak}

\begin{itemize}
\tightlist
\item
  live\href{https://www.nytimes.com/2020/08/08/world/coronavirus-updates.html?action=click\&pgtype=Article\&state=default\&region=TOP_BANNER\&context=storylines_menu}{Latest
  Updates}
\item
  \href{https://www.nytimes.com/interactive/2020/us/coronavirus-us-cases.html?action=click\&pgtype=Article\&state=default\&region=TOP_BANNER\&context=storylines_menu}{Maps
  and Cases}
\item
  \href{https://www.nytimes.com/interactive/2020/science/coronavirus-vaccine-tracker.html?action=click\&pgtype=Article\&state=default\&region=TOP_BANNER\&context=storylines_menu}{Vaccine
  Tracker}
\item
  \href{https://www.nytimes.com/interactive/2020/world/coronavirus-tips-advice.html?action=click\&pgtype=Article\&state=default\&region=TOP_BANNER\&context=storylines_menu}{F.A.Q.}
\item
  \href{https://www.nytimes.com/live/2020/08/07/business/stock-market-today-coronavirus?action=click\&pgtype=Article\&state=default\&region=TOP_BANNER\&context=storylines_menu}{Markets
  \& Economy}
\end{itemize}

Advertisement

\protect\hyperlink{after-top}{Continue reading the main story}

Supported by

\protect\hyperlink{after-sponsor}{Continue reading the main story}

\hypertarget{us-accuses-hackers-of-trying-to-steal-coronavirus-vaccine-data-for-china}{%
\section{U.S. Accuses Hackers of Trying to Steal Coronavirus Vaccine
Data for
China}\label{us-accuses-hackers-of-trying-to-steal-coronavirus-vaccine-data-for-china}}

Two suspects in China targeted companies working on vaccines as part of
a broader cybertheft campaign to enrich themselves and aid the Chinese
government, officials said.

\includegraphics{https://static01.nyt.com/images/2020/07/21/us/politics/21dc-cyber/merlin_173396472_2e20a751-fe4b-4b0c-a3b5-772fb49e2fdd-articleLarge.jpg?quality=75\&auto=webp\&disable=upscale}

\href{https://www.nytimes.com/by/julian-e-barnes}{\includegraphics{https://static01.nyt.com/images/2019/12/13/reader-center/author-julian-barnes/author-julian-barnes-thumbLarge.png}}

By \href{https://www.nytimes.com/by/julian-e-barnes}{Julian E. Barnes}

\begin{itemize}
\item
  July 21, 2020
\item
  \begin{itemize}
  \item
  \item
  \item
  \item
  \item
  \item
  \end{itemize}
\end{itemize}

\href{https://cn.nytimes.com/usa/20200722/china-hacking-coronavirus-vaccine/}{阅读简体中文版}\href{https://cn.nytimes.com/usa/20200722/china-hacking-coronavirus-vaccine/zh-hant/}{閱讀繁體中文版}

WASHINGTON --- The Justice Department accused a pair of Chinese hackers
on Tuesday of targeting vaccine development on behalf of the country's
intelligence service as part of a broader yearslong campaign of global
cybertheft aimed at industries such as defense contractors, high-end
manufacturing and solar energy companies.

Justice Department officials labeled the suspects, Li Xiaoyu and Dong
Jiazhi, as a blended threat who sometimes worked on behalf of China's
spy services and sometimes to enrich themselves. The officials said that
\href{https://www.courtlistener.com/recap/gov.uscourts.waed.91446/gov.uscourts.waed.91446.15.0.pdf}{an
indictment} secured against them this month and unsealed on Tuesday was
the first to target such a threat.

United States government officials said that the suspects had previously
stolen information about other Chinese intelligence targets like human
rights activists and, at the behest of the Ministry of State Security
spy service, shifted focus this year to trying to acquire coronavirus
vaccine research.

The indictment comes as the Trump administration has stepped up its
criticism of Beijing, both for its theft of secrets and its failure to
contain the spread of the coronavirus, and is a significant escalation
of that campaign to denounce Beijing. The Justice Department said that
China's covert activity could potentially set back vaccine research
efforts.

The accusations also came days after the United States and allied
countries
\href{https://www.nytimes.com/2020/07/16/us/politics/vaccine-hacking-russia.html}{accused
Russia} of trying to steal information on vaccine development.

The indictment also suggests that China did far less to curb its spying
than it had vowed to as part of a
\href{https://www.nytimes.com/2015/09/26/world/asia/xi-jinping-white-house.html}{nonaggression
pact} signed with the United States in late 2015 that was aimed at
curbing China's efforts to steal American technological know-how.

The agreement was thought to have slowed China's hacking for
\href{https://www.nytimes.com/2018/11/29/us/politics/china-trump-cyberespionage.html}{about
18 months}, reducing the industrial espionage work done by the Chinese
military. But Mr. Li and Mr. Dong, guided by the Chinese intelligence
agency, tried to steal secrets in 2016 and 2017, even as the agreement
was purportedly being honored.

Asked for comment on the accusations, a press officer for the Chinese
Embassy pointed on Tuesday to earlier comments by a foreign ministry
spokeswoman, Hua Chunying, who said that the government opposed all
forms of cyberattacks and threats.

The suspects are unlikely to be brought to trial because China does not
have an extradition treaty with the United States. The charges were the
latest in a continuing effort by the Justice Department to secure
indictments against private groups and intelligence officials involved
in hacking campaigns as a deterrent and to raise awareness of the threat
that such groups pose.

\hypertarget{latest-updates-the-coronavirus-outbreak}{%
\section{\texorpdfstring{\href{https://www.nytimes.com/2020/08/07/world/covid-19-news.html?action=click\&pgtype=Article\&state=default\&region=MAIN_CONTENT_1\&context=storylines_live_updates}{Latest
Updates: The Coronavirus
Outbreak}}{Latest Updates: The Coronavirus Outbreak}}\label{latest-updates-the-coronavirus-outbreak}}

Updated 2020-08-08T12:04:28.992Z

\begin{itemize}
\tightlist
\item
  \href{https://www.nytimes.com/2020/08/07/world/covid-19-news.html?action=click\&pgtype=Article\&state=default\&region=MAIN_CONTENT_1\&context=storylines_live_updates\#link-1f86d03a}{As
  the U.S. relief talks falter again, Trump says he is prepared to act
  on his own.}
\item
  \href{https://www.nytimes.com/2020/08/07/world/covid-19-news.html?action=click\&pgtype=Article\&state=default\&region=MAIN_CONTENT_1\&context=storylines_live_updates\#link-3f64a70a}{Cuomo
  says N.Y. schools can reopen in-person but leaves it up to districts
  to determine if, when and how.}
\item
  \href{https://www.nytimes.com/2020/08/07/world/covid-19-news.html?action=click\&pgtype=Article\&state=default\&region=MAIN_CONTENT_1\&context=storylines_live_updates\#link-14e70066}{Thousands
  of cases went unreported in California when a computer server failed.}
\end{itemize}

\href{https://www.nytimes.com/2020/08/07/world/covid-19-news.html?action=click\&pgtype=Article\&state=default\&region=MAIN_CONTENT_1\&context=storylines_live_updates}{See
more updates}

More live coverage:
\href{https://www.nytimes.com/live/2020/08/07/business/stock-market-today-coronavirus?action=click\&pgtype=Article\&state=default\&region=MAIN_CONTENT_1\&context=storylines_live_updates}{Markets}

On Tuesday, David L. Bowdich, the F.B.I. deputy director, called the
hacks part of a campaign of economic coercion akin to ``what we expect
from an organized criminal syndicate.''

The suspects targeted hundreds of computer networks around the world and
caused unnamed companies to lose hundreds of millions of dollars of
intellectual property, according to the indictment. For example, they
stole research on radio and laser technology from a California defense
firm and engineering drawings for a gas turbine from a company working
in the United States and Japan, court papers showed.

Justice Department and F.B.I. officials said the hackers were pursuing
information and research about the coronavirus vaccine from American
biotech firms but described it as an attempt to steal the data. The
indictment, which was filed in the Eastern District of Washington, did
not say that the hackers successfully stole information or research on
the vaccine.

The pair did try to hack a Massachusetts biotech firm researching a
vaccine as early as Jan. 27, according to the indictment. On Feb. 1, the
pair tried to find vulnerabilities on the networks of a California
biotech firm that had announced it was researching coronavirus antiviral
drugs. Then, in May, Mr. Li investigated a California diagnostic firm
developing virus testing kits.

While the indictment named only the two suspects, unlike the larger
group of Russian hackers accused of seeking vaccine data, the Justice
Department portrayed their work as far-reaching and long-running, going
back to at least 2009.

American officials first detected the suspects five years ago, when they
stole a gigabyte of information including personnel and administrator
accounts from the Hanford Site, an Energy Department facility in
Washington State where plutonium was produced during World War II,
according to the indictment.

In some cases, the suspects tried to extort money from companies,
according to the indictment. In 2017, Mr. Li threatened to publish the
source code of a Massachusetts software company if it did not give him
\$15,000 in cryptocurrency.

Like the Russian group, the Chinese hackers operated with the assistance
of their country's intelligence agencies. Their interests were broad,
covering manufacturing firms, defense contractors, government agencies,
game developers and medical device makers; they recently grew to include
information about coronavirus vaccine development and other
virus-related data.

The suspects also tried to steal other information on Chinese activists
for the Ministry of State Security, Beijing's civilian spy agency, said
John C. Demers, the assistant attorney general for national security.
The suspects handed over account information and passwords belonging to
a Hong Kong community organizer, a former Tiananmen Square protester and
a pastor of a Christian church in China.

``You can see by the variety of the hacks that they did how they were
being directed by the government,'' Mr. Demers said at a news conference
at the Justice Department. ``Extorting someone for cryptocurrency is not
something that the government is usually interested in, nor are criminal
hackers usually interested in human rights activists and clergymen.''

The hackers broke into computer networks by researching personal
identifying information about employees and customers, which helped them
gain unauthorized access, according to law enforcement officials. Once
inside, they stole information from pharmaceutical companies about drugs
under development and source code from software companies, the
indictment said.

\href{https://www.nytimes.com/news-event/coronavirus?action=click\&pgtype=Article\&state=default\&region=MAIN_CONTENT_3\&context=storylines_faq}{}

\hypertarget{the-coronavirus-outbreak-}{%
\subsubsection{The Coronavirus Outbreak
›}\label{the-coronavirus-outbreak-}}

\hypertarget{frequently-asked-questions}{%
\paragraph{Frequently Asked
Questions}\label{frequently-asked-questions}}

Updated August 6, 2020

\begin{itemize}
\item ~
  \hypertarget{why-are-bars-linked-to-outbreaks}{%
  \paragraph{Why are bars linked to
  outbreaks?}\label{why-are-bars-linked-to-outbreaks}}

  \begin{itemize}
  \tightlist
  \item
    Think about a bar. Alcohol is flowing. It can be loud, but it's
    definitely intimate, and you often need to lean in close to hear
    your friend. And strangers have way, way fewer reservations about
    coming up to people in a bar. That's sort of the point of a bar.
    Feeling good and close to strangers. It's no surprise, then, that
    \href{https://www.nytimes.com/2020/07/02/us/coronavirus-bars.html?action=click\&pgtype=Article\&state=default\&region=MAIN_CONTENT_3\&context=storylines_faq}{bars
    have been linked to outbreaks in several states.} Louisiana health
    officials have tied
    \href{https://www.nytimes.com/2020/06/22/us/new-coronavirus-phase.html?action=click\&pgtype=Article\&state=default\&region=MAIN_CONTENT_3\&context=storylines_faq}{at
    least 100 coronavirus cases} to bars in the Tigerland nightlife
    district in Baton Rouge. Minnesota has traced 328 recent cases to
    bars across the state.
    \href{https://www.boisestatepublicradio.org/post/bars-large-venues-close-ada-county-after-surge-coronavirus-prompts-rollback\#stream/0}{In
    Idaho}, health officials shut down bars in Ada County after
    reporting clusters of infections among young adults who had visited
    several bars in downtown Boise. Governors in
    \href{https://www.nytimes.com/2020/07/01/us/california-coronavirus-reopening.html?action=click\&pgtype=Article\&state=default\&region=MAIN_CONTENT_3\&context=storylines_faq}{California},
    \href{https://www.nytimes.com/2020/06/14/us/coronavirus-united-states.html?action=click\&pgtype=Article\&state=default\&region=MAIN_CONTENT_3\&context=storylines_faq}{Texas
    and Arizona}, where coronavirus cases are soaring, have ordered
    hundreds of newly reopened bars to shut down. Less than two weeks
    after Colorado's bars reopened at limited capacity, Gov. Jared Polis
    \href{https://www.denverpost.com/2020/06/30/colorado-bars-closed-coronavirus/}{ordered
    them to close}.
  \end{itemize}
\item ~
  \hypertarget{i-have-antibodies-am-i-now-immune}{%
  \paragraph{I have antibodies. Am I now
  immune?}\label{i-have-antibodies-am-i-now-immune}}

  \begin{itemize}
  \tightlist
  \item
    As of right now,
    \href{https://www.nytimes.com/2020/07/22/health/covid-antibodies-herd-immunity.html?action=click\&pgtype=Article\&state=default\&region=MAIN_CONTENT_3\&context=storylines_faq}{that
    seems likely, for at least several months.} There have been
    frightening accounts of people suffering what seems to be a second
    bout of Covid-19. But experts say these patients may have a
    drawn-out course of infection, with the virus taking a slow toll
    weeks to months after initial exposure. People infected with the
    coronavirus typically
    \href{https://www.nature.com/articles/s41586-020-2456-9}{produce}
    immune molecules called antibodies, which are
    \href{https://www.nytimes.com/2020/05/07/health/coronavirus-antibody-prevalence.html?action=click\&pgtype=Article\&state=default\&region=MAIN_CONTENT_3\&context=storylines_faq}{protective
    proteins made in response to an
    infection}\href{https://www.nytimes.com/2020/05/07/health/coronavirus-antibody-prevalence.html?action=click\&pgtype=Article\&state=default\&region=MAIN_CONTENT_3\&context=storylines_faq}{.
    These antibodies may} last in the body
    \href{https://www.nature.com/articles/s41591-020-0965-6}{only two to
    three months}, which may seem worrisome, but that's perfectly normal
    after an acute infection subsides, said Dr. Michael Mina, an
    immunologist at Harvard University. It may be possible to get the
    coronavirus again, but it's highly unlikely that it would be
    possible in a short window of time from initial infection or make
    people sicker the second time.
  \end{itemize}
\item ~
  \hypertarget{im-a-small-business-owner-can-i-get-relief}{%
  \paragraph{I'm a small-business owner. Can I get
  relief?}\label{im-a-small-business-owner-can-i-get-relief}}

  \begin{itemize}
  \tightlist
  \item
    The
    \href{https://www.nytimes.com/article/small-business-loans-stimulus-grants-freelancers-coronavirus.html?action=click\&pgtype=Article\&state=default\&region=MAIN_CONTENT_3\&context=storylines_faq}{stimulus
    bills enacted in March} offer help for the millions of American
    small businesses. Those eligible for aid are businesses and
    nonprofit organizations with fewer than 500 workers, including sole
    proprietorships, independent contractors and freelancers. Some
    larger companies in some industries are also eligible. The help
    being offered, which is being managed by the Small Business
    Administration, includes the Paycheck Protection Program and the
    Economic Injury Disaster Loan program. But lots of folks have
    \href{https://www.nytimes.com/interactive/2020/05/07/business/small-business-loans-coronavirus.html?action=click\&pgtype=Article\&state=default\&region=MAIN_CONTENT_3\&context=storylines_faq}{not
    yet seen payouts.} Even those who have received help are confused:
    The rules are draconian, and some are stuck sitting on
    \href{https://www.nytimes.com/2020/05/02/business/economy/loans-coronavirus-small-business.html?action=click\&pgtype=Article\&state=default\&region=MAIN_CONTENT_3\&context=storylines_faq}{money
    they don't know how to use.} Many small-business owners are getting
    less than they expected or
    \href{https://www.nytimes.com/2020/06/10/business/Small-business-loans-ppp.html?action=click\&pgtype=Article\&state=default\&region=MAIN_CONTENT_3\&context=storylines_faq}{not
    hearing anything at all.}
  \end{itemize}
\item ~
  \hypertarget{what-are-my-rights-if-i-am-worried-about-going-back-to-work}{%
  \paragraph{What are my rights if I am worried about going back to
  work?}\label{what-are-my-rights-if-i-am-worried-about-going-back-to-work}}

  \begin{itemize}
  \tightlist
  \item
    Employers have to provide
    \href{https://www.osha.gov/SLTC/covid-19/standards.html}{a safe
    workplace} with policies that protect everyone equally.
    \href{https://www.nytimes.com/article/coronavirus-money-unemployment.html?action=click\&pgtype=Article\&state=default\&region=MAIN_CONTENT_3\&context=storylines_faq}{And
    if one of your co-workers tests positive for the coronavirus, the
    C.D.C.} has said that
    \href{https://www.cdc.gov/coronavirus/2019-ncov/community/guidance-business-response.html}{employers
    should tell their employees} -\/- without giving you the sick
    employee's name -\/- that they may have been exposed to the virus.
  \end{itemize}
\item ~
  \hypertarget{what-is-school-going-to-look-like-in-september}{%
  \paragraph{What is school going to look like in
  September?}\label{what-is-school-going-to-look-like-in-september}}

  \begin{itemize}
  \tightlist
  \item
    It is unlikely that many schools will return to a normal schedule
    this fall, requiring the grind of
    \href{https://www.nytimes.com/2020/06/05/us/coronavirus-education-lost-learning.html?action=click\&pgtype=Article\&state=default\&region=MAIN_CONTENT_3\&context=storylines_faq}{online
    learning},
    \href{https://www.nytimes.com/2020/05/29/us/coronavirus-child-care-centers.html?action=click\&pgtype=Article\&state=default\&region=MAIN_CONTENT_3\&context=storylines_faq}{makeshift
    child care} and
    \href{https://www.nytimes.com/2020/06/03/business/economy/coronavirus-working-women.html?action=click\&pgtype=Article\&state=default\&region=MAIN_CONTENT_3\&context=storylines_faq}{stunted
    workdays} to continue. California's two largest public school
    districts --- Los Angeles and San Diego --- said on July 13, that
    \href{https://www.nytimes.com/2020/07/13/us/lausd-san-diego-school-reopening.html?action=click\&pgtype=Article\&state=default\&region=MAIN_CONTENT_3\&context=storylines_faq}{instruction
    will be remote-only in the fall}, citing concerns that surging
    coronavirus infections in their areas pose too dire a risk for
    students and teachers. Together, the two districts enroll some
    825,000 students. They are the largest in the country so far to
    abandon plans for even a partial physical return to classrooms when
    they reopen in August. For other districts, the solution won't be an
    all-or-nothing approach.
    \href{https://bioethics.jhu.edu/research-and-outreach/projects/eschool-initiative/school-policy-tracker/}{Many
    systems}, including the nation's largest, New York City, are
    devising
    \href{https://www.nytimes.com/2020/06/26/us/coronavirus-schools-reopen-fall.html?action=click\&pgtype=Article\&state=default\&region=MAIN_CONTENT_3\&context=storylines_faq}{hybrid
    plans} that involve spending some days in classrooms and other days
    online. There's no national policy on this yet, so check with your
    municipal school system regularly to see what is happening in your
    community.
  \end{itemize}
\end{itemize}

Although the Chinese intelligence service in some cases provided them
with hacking tools, much of their work was done using more common
methods to breach publicly known software vulnerabilities.

The hackers also worked to cover their tracks, sometimes in ways that
could damage the data they were stealing, like by changing the file
names of information they downloaded, according to court papers. To
further avoid detection, the two hackers worked inside computers'
``recycle bins,'' where files are hidden by default and harder for
system administrators to see.

Mr. Demers said an attempted breach could slow down research because it
must be secured, but researchers also must make sure their data has not
been corrupted or altered by the intruders. The government officials did
not say they had evidence that such manipulation had occurred, however.

``Once someone is in your system, they cannot only take the data, they
can manipulate the data,'' Mr. Demers said. ``So what you have to focus
on is making sure through backup or other systems that nothing has
changed about your data.''

The indictment contained 11 criminal charges against Mr. Li and Mr.
Dong, including conspiracies to commit computer fraud and theft as well
as multiple counts of aggravated identity theft.

Trump administration officials, both in public speeches and classified
briefings to Congress, have stepped up warnings in recent weeks about
Chinese intelligence services and their campaign to steal information
and influence American politics.

Lawmakers have been wrestling with how to better deter China, Russia and
other nations from trying to hack into pharmaceutical companies,
technology firms and other organizations.

``We need a comprehensive strategy to deter the serial theft of
strategic U.S. secrets,'' Senator Chris Van Hollen, Democrat of
Maryland, said in an interview. ``It is not enough to have these one-off
indictments. We need to make it clear upfront that there will be a very
high price to pay for foreign actors that attempt to steal important
trade secrets, whether it relates to the coronavirus or semiconductors
or 5G networks.''

Mr. Van Hollen and Senator Ben Sasse, Republican of Nebraska and a
member of the Senate Intelligence Committee, have pushed a bill that
would impose sanctions on foreigners and foreign companies that try to
steal American intellectual property. The two are hoping the measure
could be considered as part of congressional debate this week over a
defense policy bill, though there is no guarantee of a vote on the
proposal.

``This indictment reveals yet again that Chairman Xi leads an army of
hackers that steal and attempt to steal --- every single day, in almost
every country and industry,'' Mr. Sasse said, referring to President Xi
Jinping of China.

David E. Sanger contributed reporting.

Advertisement

\protect\hyperlink{after-bottom}{Continue reading the main story}

\hypertarget{site-index}{%
\subsection{Site Index}\label{site-index}}

\hypertarget{site-information-navigation}{%
\subsection{Site Information
Navigation}\label{site-information-navigation}}

\begin{itemize}
\tightlist
\item
  \href{https://help.nytimes.com/hc/en-us/articles/115014792127-Copyright-notice}{©~2020~The
  New York Times Company}
\end{itemize}

\begin{itemize}
\tightlist
\item
  \href{https://www.nytco.com/}{NYTCo}
\item
  \href{https://help.nytimes.com/hc/en-us/articles/115015385887-Contact-Us}{Contact
  Us}
\item
  \href{https://www.nytco.com/careers/}{Work with us}
\item
  \href{https://nytmediakit.com/}{Advertise}
\item
  \href{http://www.tbrandstudio.com/}{T Brand Studio}
\item
  \href{https://www.nytimes.com/privacy/cookie-policy\#how-do-i-manage-trackers}{Your
  Ad Choices}
\item
  \href{https://www.nytimes.com/privacy}{Privacy}
\item
  \href{https://help.nytimes.com/hc/en-us/articles/115014893428-Terms-of-service}{Terms
  of Service}
\item
  \href{https://help.nytimes.com/hc/en-us/articles/115014893968-Terms-of-sale}{Terms
  of Sale}
\item
  \href{https://spiderbites.nytimes.com}{Site Map}
\item
  \href{https://help.nytimes.com/hc/en-us}{Help}
\item
  \href{https://www.nytimes.com/subscription?campaignId=37WXW}{Subscriptions}
\end{itemize}
