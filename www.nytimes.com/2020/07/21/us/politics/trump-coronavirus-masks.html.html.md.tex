Sections

SEARCH

\protect\hyperlink{site-content}{Skip to
content}\protect\hyperlink{site-index}{Skip to site index}

\href{https://www.nytimes.com/section/politics}{Politics}

\href{https://myaccount.nytimes.com/auth/login?response_type=cookie\&client_id=vi}{}

\href{https://www.nytimes.com/section/todayspaper}{Today's Paper}

\href{/section/politics}{Politics}\textbar{}Trump, in a Shift, Endorses
Masks and Says Virus Will Get Worse

\href{https://nyti.ms/3fSR5fq}{https://nyti.ms/3fSR5fq}

\begin{itemize}
\item
\item
\item
\item
\item
\item
\end{itemize}

\begin{itemize}
\item
  \href{https://www.nytimes.com/2020/08/07/us/elections/biden-vs-trump.html?action=click\&pgtype=Article\&state=default\&region=TOP_BANNER\&context=storylines_menu}{Election
  Updates}
\item
  \href{https://www.nytimes.com/interactive/2020/08/08/us/elections/results-hawaii-primary-elections.html?action=click\&pgtype=Article\&state=default\&region=TOP_BANNER\&context=storylines_menu}{Hawaii
  Results}
\item
  \href{https://www.nytimes.com/article/biden-vice-president-2020.html?action=click\&pgtype=Article\&state=default\&region=TOP_BANNER\&context=storylines_menu}{Biden's
  V.P. Search}
\item
  \href{https://www.nytimes.com/interactive/2019/us/politics/2020-presidential-candidates.html?action=click\&pgtype=Article\&state=default\&region=TOP_BANNER\&context=storylines_menu}{The
  Candidates}
\item
  \href{https://www.nytimes.com/newsletters/politics?action=click\&pgtype=Article\&state=default\&region=TOP_BANNER\&context=storylines_menu}{Politics
  Newsletter}
\end{itemize}

Advertisement

\protect\hyperlink{after-top}{Continue reading the main story}

Supported by

\protect\hyperlink{after-sponsor}{Continue reading the main story}

\hypertarget{trump-in-a-shift-endorses-masks-and-says-virus-will-get-worse}{%
\section{Trump, in a Shift, Endorses Masks and Says Virus Will Get
Worse}\label{trump-in-a-shift-endorses-masks-and-says-virus-will-get-worse}}

Rather than just ``embers'' of the disease, as he has repeatedly
characterized recent outbreaks afflicting much of the country, President
Trump conceded that there were now ``big fires.''

\includegraphics{https://static01.nyt.com/images/2020/07/21/us/politics/21dc-trump/21dc-trump-articleLarge.jpg?quality=75\&auto=webp\&disable=upscale}

\href{https://www.nytimes.com/by/peter-baker}{\includegraphics{https://static01.nyt.com/images/2018/06/13/multimedia/peter-baker/peter-baker-thumbLarge-v2.png}}

By \href{https://www.nytimes.com/by/peter-baker}{Peter Baker}

\begin{itemize}
\item
  Published July 21, 2020Updated July 28, 2020
\item
  \begin{itemize}
  \item
  \item
  \item
  \item
  \item
  \item
  \end{itemize}
\end{itemize}

WASHINGTON --- President Trump acknowledged on Tuesday that
\href{https://www.nytimes.com/2020/07/21/world/coronavirus-covid-19.html}{the
coronavirus pandemic} was growing more severe in the United States and
endorsed mask wearing in a shift after weeks of playing down the
seriousness of the crisis that has
\href{https://www.nytimes.com/interactive/2020/us/coronavirus-us-cases.html}{killed
more than 140,000 Americans}.

Rather than just ``embers'' of the virus, as he has repeatedly
characterized recent outbreaks afflicting much of the country,
\href{https://www.nytimes.com/2020/07/25/us/politics/trump-florida-convention.html}{Mr.
Trump} conceded that there were now ``big fires,'' particularly in
Florida and elsewhere across the South and West. He vowed to press a
``relentless'' campaign to curb the spread without offering any new
specific plans for how to do so.

``It will probably, unfortunately, get worse before it gets better,''
Mr. Trump told reporters as he resumed the televised
\href{https://www.nytimes.com/2020/07/28/us/politics/trump-nobody-likes-me-walks-out-briefing.html}{coronavirus
briefings} that he had called off in late April. ``Something I don't
like saying about things, but that's the way it is. It's what we have.''

The president's shift had its limits, however, as he again congratulated
himself on his handling of the pandemic, admitted no missteps and made a
number of specious claims. He included none of his public health experts
in the briefing and falsely asserted that he had never resisted wearing
a mask. And he contradicted his own press secretary, who had told
reporters just hours earlier that the president was sometimes tested for
the virus multiple times a day; in fact, he said, he has never been
tested more than once in a single day.

But Mr. Trump was notably less dismissive about the pandemic, a
reflection of the dawning realization within his team that the virus not
only is not going away but has badly damaged his standing with the
public heading into the election in November. Approval of his handling
of the pandemic has fallen from 51 percent in late March to 38 percent
last week in
\href{https://www.langerresearch.com/wp-content/uploads/1214a22020Election.pdf}{polling
by The Washington Post and ABC News}.

Former Vice President Joseph R. Biden Jr., the presumptive Democratic
nominee who now leads Mr. Trump by double digits, has assailed him in
recent days for ignoring a devastating threat to the United States.

On Monday, Mr. Biden said the president had ``raised the white flag'' in
the fight against the virus. On Tuesday, he said the incumbent had
failed to help working families hurt by the economic collapse.

``You know, he's quit on you, and he's quit on this country,'' Mr. Biden
said as he released
\href{https://www.nytimes.com/2020/07/21/us/politics/biden-workplace-childcare.html}{a
plan for child and elder care}. ``But this election is not just about
him. It's about us. It's about you. It's about what we'll do, what a
president's supposed to do.''

Mr. Trump's briefing was more tightly disciplined than the daily
performances in March and April when he would talk for more than an
hour, picking fights with governors and reporters and making
ill-considered remarks like
\href{https://www.nytimes.com/2020/04/24/us/politics/trump-inject-disinfectant-bleach-coronavirus.html}{suggesting
bleach as a treatment} for the coronavirus. On Tuesday, he read from a
prepared script, took fewer questions than in the past and ended the
session in 27 minutes, shorter than all but one of the 50 briefings he
did in the spring, according to Factba.se, which tracks his public
appearances.

Advisers have urged him to be less combative, demonstrate more concern
over the latest surge in infections and avoid straying into areas that
have been counterproductive. Even so, the president wandered far afield
when he offered supportive words to
\href{https://www.nytimes.com/2020/07/02/nyregion/ghislaine-maxwell-arrest-jeffrey-epstein.html}{Ghislaine
Maxwell}, who was charged with luring underage girls into the orbit of
the financier
\href{https://www.nytimes.com/2019/08/10/nyregion/jeffrey-epstein-suicide.html}{Jeffrey
Epstein}, who killed himself in August after he was charged with sex
trafficking.

``I've met her numerous times over the years, especially since I lived
in Palm Beach, and I guess they lived in Palm Beach,'' Mr. Trump said.
``But I wish her well.''

The White House did not invite to the briefing Dr. Anthony S. Fauci, the
government's top infectious disease expert, who has come under fire from
the president and his team. Dr. Deborah L. Birx, the White House
coronavirus response coordinator, was not in the room, either.

But Dr. Fauci appeared on CNN an hour before Mr. Trump's briefing with a
message that contradicted the president's assertions.

While Mr. Trump has boasted of the number of tests conducted to assert
that the virus is under control, Dr. Fauci said that was not enough if
it took days to get results. ``Just the number of tests that you do
doesn't always give you a right reflection of how well things are
working or not,'' he told Jake Tapper in an interview.

Dr. Fauci also pushed back on the president's description of him as ``a
little bit of an alarmist,'' as
\href{https://www.nytimes.com/2020/07/19/us/politics/trump-fox-interview-coronavirus-race.html}{Mr.
Trump put it on Fox News} over the weekend. ``I consider myself more a
realist than an alarmist, but people do have their opinions other than
that,'' Dr. Fauci said mildly.

Mr. Trump again bragged about the ``tremendous amount of testing,'' but
when asked about the delays in results, he agreed that it should be
fixed. ``If the doctors and the professionals feel that even though we
are at a level that nobody ever dreamt possible that they would like to
do more, I'm OK with it,'' he said.

``Ultimately, our goal is not merely to manage the pandemic but to end
it,'' he added. ``We want to get rid of it as soon as we can.''

Mr. Trump's decision to resume televised virus briefings came as the
number of new cases soared far above what it was when he was last
addressed the country about the pandemic on a daily basis. The United
States is recording about
\href{https://www.nytimes.com/interactive/2020/us/coronavirus-us-cases.html}{60,000
new infections a day}, far more than the increase in tests in some
states. The number of deaths, after falling substantially, is up 64
percent over the past two weeks.

The president again insisted the virus would ``disappear'' but conceded
that it remained serious. ``We have embers and fires, and we have big
fires, and unfortunately now Florida is in a little tough or in a big
tough position,'' he said.

Weeks after claiming that ``99 percent'' of coronavirus cases were
``totally harmless,'' the president sounded less sanguine on Tuesday,
calling it ``a nasty horrible disease,'' although he continued to
falsely insist that the mortality rate in the United States was among
the lowest in the world.

Mr. Trump urged Americans to avoid packed bars and offered his most
robust endorsement of masks, saying, ``When you can, use a mask,'' even
as he falsely claimed he had always been supportive. ``I have no problem
with the masks,'' he said, holding up a blue one with a presidential
seal. ``I view it this way: Anything that potentially can help, and that
certainly can potentially help, is a good thing. I have no problem. I
carry it. I wear it. You saw me wearing it a number of times, and I'll
continue.''

In fact, Mr. Trump has worn a mask in public on only one occasion ---
during a recent visit to Walter Reed National Military Medical Center in
Maryland. Until then, he often disparaged masks: In April, after public
health advisers recommended wearing them, he said,
``\href{https://www.nytimes.com/video/us/politics/100000007070943/trump-mask-coronavirus.html}{I
don't think I'm going to be doing it}.'' Mr. Trump
\href{https://www.nytimes.com/2020/05/26/us/politics/joe-biden-facemasks-trump-coronavirus.html}{mocked
Mr. Biden} in May for donning one, calling them
``\href{https://www.wsj.com/articles/trump-talks-juneteenth-john-bolton-economy-in-wsj-interview-11592493771}{a
double-edged sword}'' and even suggesting that wearing a mask was a
political statement against him.

He shifted his stance only after many senior Republicans, including
Senator Mitch McConnell of Kentucky, the majority leader, and several
governors began promoting them more vigorously. By this week, Mr. Trump
began saying that it was ``patriotic'' to wear a mask, in contrast to
his supporters, who have claimed that mandating masks is an infringement
on their civil liberties.

But even after
\href{https://twitter.com/realDonaldTrump/status/1285299379746811915}{posting
a Twitter message} on Monday urging masks, the president was spotted
that night at his Washington hotel mingling with guests without wearing
one.

And even as the president sought to recalibrate his message on the virus
on Tuesday, he was struggling to reconcile it with the rest of his team.
Asked at an earlier news briefing about Mr. Trump's failure to wear a
mask at his hotel, Kayleigh McEnany, the White House press secretary,
told reporters that it was not as urgent for the president to wear one
since he was tested regularly for the virus.

``The president is the most tested man in America,'' she said. ``He's
tested more than anyone, multiple times a day. And we believe that he's
acting appropriately.''

Mr. Trump offered a different account from the same lectern. ``I do take
probably on average a test every two days, three days,'' he said. ``I
don't know of any time I've taken two tests in one day.''

\hypertarget{our-2020-election-guide}{%
\section{Our 2020 Election Guide}\label{our-2020-election-guide}}

Updated Aug. 8, 2020

\begin{itemize}
\item
  \begin{center}\rule{0.5\linewidth}{\linethickness}\end{center}

  \hypertarget{the-latest}{%
  \subsection{The Latest}\label{the-latest}}

  \begin{itemize}
  \tightlist
  \item
    With 160 lawsuits filed over voting rules and President Trump's
    baseless claims of fraud, Election Day in America
    \href{https://www.nytimes.com/2020/08/08/us/politics/voting-nov-3-election.html?action=click\&pgtype=Article\&state=default\&region=BELOW_MAIN_CONTENT\&context=storylines_guide}{could
    become Election Month}.
  \end{itemize}
\item
  \begin{center}\rule{0.5\linewidth}{\linethickness}\end{center}

  \hypertarget{bidens-vp-search}{%
  \subsection{Biden's V.P. Search}\label{bidens-vp-search}}

  \begin{itemize}
  \tightlist
  \item
    \href{https://www.nytimes.com/article/biden-vice-president-2020.html?action=click\&pgtype=Article\&state=default\&region=BELOW_MAIN_CONTENT\&context=storylines_guide}{Here
    are 13 women} who have been under consideration to be Joe Biden's
    running mate, and why each might be chosen --- and might not be.
  \end{itemize}
\item
  \begin{center}\rule{0.5\linewidth}{\linethickness}\end{center}

  \hypertarget{keep-up-with-our-coverage}{%
  \subsection{Keep Up With Our
  Coverage}\label{keep-up-with-our-coverage}}

  \begin{itemize}
  \tightlist
  \item
    Get an
    \href{https://www.nytimes.com/newsletters/politics?action=click\&pgtype=Article\&state=default\&region=BELOW_MAIN_CONTENT\&context=storylines_guide}{email}
    recapping the day's news
  \end{itemize}

  \begin{itemize}
  \tightlist
  \item
    Download our mobile app on
    \href{https://apps.apple.com/us/app/nytimes/id284862083?ls=1\&mat_click_id=5c79ae7455014fd1bd66b5610c05b8f2-20191112-16948\&referrer=mat_click_id\%3D5c79ae7455014fd1bd66b5610c05b8f2-20191112-16948\%26link_click_id\%3D722930677036718082}{iOS}
    and
    \href{http://a.localytics.com/android?id=com.nytimes.android\&referrer=utm_source\%3Dother_nyt_mobile_web\%26utm_medium\%3DWeb\%2520page\%26utm_term\%3DGeneral\%2520Mobile\%2520Page\%26utm_campaign\%3DNYT\%2520Mobile\%2520General\%2520Page}{Android}
    and turn on Breaking News and Politics alerts
  \end{itemize}
\end{itemize}

Advertisement

\protect\hyperlink{after-bottom}{Continue reading the main story}

\hypertarget{site-index}{%
\subsection{Site Index}\label{site-index}}

\hypertarget{site-information-navigation}{%
\subsection{Site Information
Navigation}\label{site-information-navigation}}

\begin{itemize}
\tightlist
\item
  \href{https://help.nytimes.com/hc/en-us/articles/115014792127-Copyright-notice}{©~2020~The
  New York Times Company}
\end{itemize}

\begin{itemize}
\tightlist
\item
  \href{https://www.nytco.com/}{NYTCo}
\item
  \href{https://help.nytimes.com/hc/en-us/articles/115015385887-Contact-Us}{Contact
  Us}
\item
  \href{https://www.nytco.com/careers/}{Work with us}
\item
  \href{https://nytmediakit.com/}{Advertise}
\item
  \href{http://www.tbrandstudio.com/}{T Brand Studio}
\item
  \href{https://www.nytimes.com/privacy/cookie-policy\#how-do-i-manage-trackers}{Your
  Ad Choices}
\item
  \href{https://www.nytimes.com/privacy}{Privacy}
\item
  \href{https://help.nytimes.com/hc/en-us/articles/115014893428-Terms-of-service}{Terms
  of Service}
\item
  \href{https://help.nytimes.com/hc/en-us/articles/115014893968-Terms-of-sale}{Terms
  of Sale}
\item
  \href{https://spiderbites.nytimes.com}{Site Map}
\item
  \href{https://help.nytimes.com/hc/en-us}{Help}
\item
  \href{https://www.nytimes.com/subscription?campaignId=37WXW}{Subscriptions}
\end{itemize}
