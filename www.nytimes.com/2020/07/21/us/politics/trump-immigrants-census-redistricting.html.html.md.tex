Sections

SEARCH

\protect\hyperlink{site-content}{Skip to
content}\protect\hyperlink{site-index}{Skip to site index}

\href{https://www.nytimes.com/section/politics}{Politics}

\href{https://myaccount.nytimes.com/auth/login?response_type=cookie\&client_id=vi}{}

\href{https://www.nytimes.com/section/todayspaper}{Today's Paper}

\href{/section/politics}{Politics}\textbar{}Trump Seeks to Stop Counting
Unauthorized Immigrants in Drawing House Districts

\url{https://nyti.ms/3fO7BO1}

\begin{itemize}
\item
\item
\item
\item
\item
\end{itemize}

Advertisement

\protect\hyperlink{after-top}{Continue reading the main story}

Supported by

\protect\hyperlink{after-sponsor}{Continue reading the main story}

\hypertarget{trump-seeks-to-stop-counting-unauthorized-immigrants-in-drawing-house-districts}{%
\section{Trump Seeks to Stop Counting Unauthorized Immigrants in Drawing
House
Districts}\label{trump-seeks-to-stop-counting-unauthorized-immigrants-in-drawing-house-districts}}

Critics described the move as unconstitutional and a transparent attempt
to help Republicans.

\includegraphics{https://static01.nyt.com/images/2020/08/20/us/politics/20dc-immig-eo/merlin_174766299_d164c2c5-1528-4c70-9b1b-f4405db24dfd-articleLarge.jpg?quality=75\&auto=webp\&disable=upscale}

\href{https://www.nytimes.com/by/katie-rogers}{\includegraphics{https://static01.nyt.com/images/2018/06/12/multimedia/author-katie-rogers/author-katie-rogers-thumbLarge-v2.png}}\href{https://www.nytimes.com/by/peter-baker}{\includegraphics{https://static01.nyt.com/images/2018/06/13/multimedia/peter-baker/peter-baker-thumbLarge-v2.png}}

By \href{https://www.nytimes.com/by/katie-rogers}{Katie Rogers} and
\href{https://www.nytimes.com/by/peter-baker}{Peter Baker}

\begin{itemize}
\item
  Published July 21, 2020Updated July 23, 2020
\item
  \begin{itemize}
  \item
  \item
  \item
  \item
  \item
  \end{itemize}
\end{itemize}

WASHINGTON ---
\href{https://www.nytimes.com/2020/07/23/us/trump-immigration-nation-netflix.html}{President
Trump} directed the federal government on Tuesday not to count
undocumented
\href{https://www.nytimes.com/2020/07/23/us/trump-immigration-nation-netflix.html}{immigrants}
when allocating the nation's House districts, a move that critics called
a transparent political ploy to help Republicans in violation of the
Constitution.

The president's directive would exclude millions of people when
determining how many House seats each state should have based on the
once-a-decade census, reversing the longstanding policy of counting
everyone regardless of citizenship or legal status. The effect would
likely shift several seats from Democratic states to Republican states.

``There used to be a time when you could proudly declare, `I am a
citizen of the United States,''' Mr. Trump said in a written statement
after signing a memorandum to the Commerce Department, which oversees
the Census Bureau. ``But now, the radical left is trying to erase the
existence of this concept and conceal the number of illegal aliens in
our country. This is all part of a broader left-wing effort to erode the
rights of Americans citizens, and I will not stand for it.''

The action directly conflicts with the traditional consensus
interpretation of the Constitution and will almost surely be challenged
in court, potentially delaying its effect if not blocking its enactment
altogether. But it fit into Mr. Trump's efforts to curb both legal and
illegal immigration at a time when he is anxiously trying to galvanize
his political base heading into a fall election season trailing his
Democratic opponent.

``I think the Donald Trump view is: `I can look like I'm trying to do
something by stoking anti-immigrant fervor, and if I lose in court then,
I just stoke anti-court fervor too,''' Joshua A. Geltzer, the director
of the Institute for Constitutional Advocacy and Protection at
Georgetown, said in an interview. ``It should be legally impossible as
well as factually difficult to do.''

As a practical matter, Mr. Trump's order could not be carried out even
were it legal, because no official tally of undocumented immigrants
exists, and federal law bars the use of population estimates for
reapportionment purposes*.*

The move comes a year after Mr. Trump
\href{https://www.nytimes.com/2019/07/02/us/trump-census-citizenship-question.html}{was
blocked by the Supreme Court} from adding a citizenship question to the
census on the grounds that its ostensible reasoning ``seems to have been
contrived.'' The administration has been trying ever since to collect
information on undocumented immigrants through separate means
\href{https://www.nytimes.com/aponline/2020/07/16/us/ap-us-census-citizenship.html}{like
driver's license files}.

\href{https://cis.org/Report/Impact-Legal-and-Illegal-Immigration-Apportionment-Seats-US-House-Representatives-2020}{A
study last year by the Center for Immigration Studies}, a group that
supports limits on immigration, found that excluding immigrants from the
count for purposes of drawing congressional districts would take away
seats from some states while giving more to others.

Excluding unauthorized immigrants in 2020 would redistribute three
seats, the study found, with California, New York and Texas all losing a
seat that they would have had otherwise, while Ohio, Alabama and
Minnesota would each gain one. The study found even more sweeping
effects if the U.S.-born children of undocumented immigrants were
excluded, but the president's directive made no mention of them.

Steven Camarota, the research director for the center, said the
administration's effort would be difficult administratively and likely
tied up in court. ``Nevertheless,'' he said, ``the president has done
the country an important service by reminding us that tolerating
large-scale illegal immigration creates a number of unavoidable
consequences, including diluting the political representation of
American citizens in Congress and the Electoral College.''

The White House separately asked congressional appropriators last
weekend to include \$1 billion into the next coronavirus relief package
for the purpose of conducting a ``timely census.'' The Census Bureau had
previously sought permission to extend the tally of the hardest-to-count
people into October and delay delivery of reapportionment population
totals to next year.

The \$1 billion could allow the bureau to abandon that plan and
accelerate the counting to deliver a reapportionment count to Congress
in December, before Mr. Trump leaves office if he loses the election to
former Vice President Joseph R. Biden Jr. It could mean that less time
is devoted to counting the marginalized people than in a normal census,
which experts believe would benefit Republicans.

The president's directive on Tuesday amounted to his latest
election-year effort to restrict immigration and immigration rights in
the United States, lately predicated on the need to stem the spread of
the coronavirus.

The administration decided last month
\href{https://www.nytimes.com/2020/06/22/us/politics/trump-h1b-work-visas.html}{to
suspend new work visas} and bar hundreds of thousands of foreigners from
seeking employment in the United States, drawing immediate opposition
from business leaders and several states.

But last week administration officials
\href{https://www.nytimes.com/2020/06/22/us/politics/trump-h1b-work-visas.html}{backed
away} from a separate plan to strip international college students of
their visas if they did not attend at least some classes in person.
Earlier this month, Mr. Trump told Telemundo that he would sign a ``much
bigger bill on immigration'' through an executive order, although that
has not come to fruition.

The president's move to exclude unauthorized immigrants from
congressional apportionment upends a long history. Even as he signed his
memorandum on Tuesday,
\href{https://www.census.gov/population/apportionment/about/faq.html\#Q16}{the
Census Bureau's own website} continued to say in a question-and-answer
section that undocumented residents are to be counted: ``Yes, all people
(citizens and noncitizens) with a usual residence in the 50 states are
to be included in the census and thus in the apportionment counts.''

The president's policy appeared at odds with the Constitution, which
\href{https://www.census.gov/programs-surveys/decennial-census/about/census-constitution.html}{requires
the government to conduct} an ``actual enumeration'' of all people
living in the United States without distinguishing whether they are
citizens. But the memorandum signed by Mr. Trump argued that the
government has always made distinctions like not counting foreign
diplomats or temporary visitors even though they are in the United
States physically. Therefore, the memorandum argued, the government can
make the further distinction of not counting people who have no legal
right to be in the country in the first place.

The argument that immigrants can be excluded from reapportionment counts
also runs counter to legal opinions that the Department of Justice
issued during the administrations of Presidents George H.W. Bush and
Bill Clinton, when some in Congress sought to put that exclusion into
law.

Critics said the administration's efforts first to include a citizenship
question and now to disregard undocumented immigrants from apportionment
would lead to undercounts of even legal noncitizens and minority
residents, resulting in less representation and federal funding in areas
where they live, which tend to vote Democratic.

Marielena Hincapié, the executive director of the National Immigration
Law Center Immigrant Justice Fund, said that regardless of whether Mr.
Trump's latest action was legal, it would discourage compliance with the
census among Latinos, who already complete the survey at lower rates
than people of other races.

``This is his go-to play every time that he's feeling cornered or he's
feeling like he's losing,'' Ms. Hincapié said. ``He uses immigrants and
immigration to divide and distract, and at the same time he sends that
chilling effect through all immigrant communities who have already been
living in fear under his administration.''

Michael Wines contributed reporting.

Advertisement

\protect\hyperlink{after-bottom}{Continue reading the main story}

\hypertarget{site-index}{%
\subsection{Site Index}\label{site-index}}

\hypertarget{site-information-navigation}{%
\subsection{Site Information
Navigation}\label{site-information-navigation}}

\begin{itemize}
\tightlist
\item
  \href{https://help.nytimes.com/hc/en-us/articles/115014792127-Copyright-notice}{©~2020~The
  New York Times Company}
\end{itemize}

\begin{itemize}
\tightlist
\item
  \href{https://www.nytco.com/}{NYTCo}
\item
  \href{https://help.nytimes.com/hc/en-us/articles/115015385887-Contact-Us}{Contact
  Us}
\item
  \href{https://www.nytco.com/careers/}{Work with us}
\item
  \href{https://nytmediakit.com/}{Advertise}
\item
  \href{http://www.tbrandstudio.com/}{T Brand Studio}
\item
  \href{https://www.nytimes.com/privacy/cookie-policy\#how-do-i-manage-trackers}{Your
  Ad Choices}
\item
  \href{https://www.nytimes.com/privacy}{Privacy}
\item
  \href{https://help.nytimes.com/hc/en-us/articles/115014893428-Terms-of-service}{Terms
  of Service}
\item
  \href{https://help.nytimes.com/hc/en-us/articles/115014893968-Terms-of-sale}{Terms
  of Sale}
\item
  \href{https://spiderbites.nytimes.com}{Site Map}
\item
  \href{https://help.nytimes.com/hc/en-us}{Help}
\item
  \href{https://www.nytimes.com/subscription?campaignId=37WXW}{Subscriptions}
\end{itemize}
