Sections

SEARCH

\protect\hyperlink{site-content}{Skip to
content}\protect\hyperlink{site-index}{Skip to site index}

\href{https://www.nytimes.com/section/world/europe}{Europe}

\href{https://myaccount.nytimes.com/auth/login?response_type=cookie\&client_id=vi}{}

\href{https://www.nytimes.com/section/todayspaper}{Today's Paper}

\href{/section/world/europe}{Europe}\textbar{}`No One' Protected British
Democracy From Russia, U.K. Report Concludes

\url{https://nyti.ms/2DXEZnc}

\begin{itemize}
\item
\item
\item
\item
\item
\item
\end{itemize}

Advertisement

\protect\hyperlink{after-top}{Continue reading the main story}

Supported by

\protect\hyperlink{after-sponsor}{Continue reading the main story}

\hypertarget{no-one-protected-british-democracy-from-russia-uk-report-concludes}{%
\section{`No One' Protected British Democracy From Russia, U.K. Report
Concludes}\label{no-one-protected-british-democracy-from-russia-uk-report-concludes}}

Russian efforts to interfere in the British political system were widely
ignored by successive governments, according to a long-awaited report by
Parliament.

\includegraphics{https://static01.nyt.com/images/2020/07/21/world/21uk-russia1/merlin_160190313_fbf10f2f-65e7-4f4b-aab7-383521aabab5-articleLarge.jpg?quality=75\&auto=webp\&disable=upscale}

\href{https://www.nytimes.com/by/mark-landler}{\includegraphics{https://static01.nyt.com/images/2019/10/22/reader-center/author-mark-landler/author-mark-landler-thumbLarge-v3.png}}\href{https://www.nytimes.com/by/stephen-castle}{\includegraphics{https://static01.nyt.com/images/2018/10/08/multimedia/author-stephen-castle/author-stephen-castle-thumbLarge.png}}

By \href{https://www.nytimes.com/by/mark-landler}{Mark Landler} and
\href{https://www.nytimes.com/by/stephen-castle}{Stephen Castle}

\begin{itemize}
\item
  July 21, 2020
\item
  \begin{itemize}
  \item
  \item
  \item
  \item
  \item
  \item
  \end{itemize}
\end{itemize}

LONDON --- Russia has mounted a prolonged, sophisticated campaign to
undermine Britain's democracy and corrupt its politics, while successive
British governments have looked the other way, according to a
long-delayed report released on Tuesday by a British parliamentary
committee.

From meddling in elections and spreading disinformation to funneling
dirty money and employing members of the House of Lords, the Russians
have tried to co-opt politicians and corrode institutions, often with
little resistance from law enforcement or intelligence agencies that
ignored years of warning signs.

The report, in many ways harder on British officials than the Russians,
did not answer the question of whether Russia swayed one of the most
consequential votes in modern British history: the 2016 referendum on
leaving the European Union. But it was unforgiving about who is
protecting British democracy.

``No one is,'' the report's authors said.

``The outrage isn't if there is interference,'' said Kevan Jones, a
Labour Party member of Parliament who served on the intelligence
committee that released the report. ``The outrage is no one wanted to
know if there was interference.''

The release of the report came more than seven months after Prime
Minister Boris Johnson's Conservative Party
\href{https://www.nytimes.com/2019/12/12/world/europe/uk-election-boris-johnson.html}{racked
up an 80-seat majority} in Parliament and almost 18 months after the end
of the inquiry by the Intelligence and Security Committee, a
parliamentary body that oversees the country's spy agencies.

Still, it was eagerly awaited in Britain, where anxieties about Russia's
behavior range from
\href{https://www.nytimes.com/2018/05/21/world/europe/uk-russia-money-laundering-london.html}{influence-peddling
with oligarchs} in London to the
\href{https://www.nytimes.com/2018/09/09/world/europe/sergei-skripal-russian-spy-poisoning.html}{poisoning
of a former Russian intelligence agent} and his daughter in Salisbury,
England.

\includegraphics{https://static01.nyt.com/images/2020/07/21/world/21uk-russia3/merlin_174580659_c200ecf9-48d5-48f3-a9e4-b9c1c152708c-articleLarge.jpg?quality=75\&auto=webp\&disable=upscale}

The report also landed in the heat of an American presidential election,
\href{https://www.nytimes.com/2020/02/20/us/politics/russian-interference-trump-democrats.html}{shadowed
by questions} about ties between President Trump and Russia, as well as
fears of renewed foreign tampering, not just by Russia, but also by
China and Iran.

The committee's account characterized Russia as a reckless country bent
on recapturing its status as a ``great power,'' primarily by
destabilizing those in the West. ``The security threat posed by Russia
is difficult for the West to manage as, in our view and that of many
others, it appears fundamentally nihilistic,'' the authors said.

Experts said the report showed parallels between Britain and the United
States in the failure to pick up warning signs, but also important
differences. The F.B.I. and other American agencies, they said, had
investigated election interference more aggressively than their British
counterparts, while the British were ahead of the United States in
scrutinizing how Russian money had corrupted politics.

``This is one of the pieces that is not really well understood in the
U.S.,'' said Laura Rosenberger, director of the
\href{https://securingdemocracy.gmfus.org/}{Alliance for Securing
Democracy}, which tracks Russian disinformation efforts in the United
States. ``Whether there is dirty Russian money that has flowed into our
political system.''

The report described how British politicians had welcomed oligarchs to
London, allowing them to launder their illicit money through what it
called the London ``laundromat.'' A growth industry of ``enablers'' ---
lawyers, accountants, real estate agents, and public relations
consultants --- sprang up to serve their needs.

Image

British politicians welcomed Russian oligarchs to London, the report
said.Credit...Andrew Testa for The New York Times

These people, the report said, ``played a role, wittingly or
unwittingly, in the extension of Russian influence which is often linked
to promoting the nefarious interests of the Russian state.''

Several members of the House of Lords, the report said, had business
interests linked to Russia or worked for companies with Russian ties. It
urged an investigation of them, though it did not name any names. That
information, as well as the names of politicians who received donations,
was redacted from the public report, along with other sensitive
intelligence.

``The most disturbing thing is the recognition of what the Russian
government has gotten away, under our eyes,'' said
\href{https://www.nytimes.com/2018/07/16/world/europe/putin-bill-browder-magnitsky-investor.html}{William
F. Browder}, an American-born British financier who has worked
extensively in Russia and provided evidence to the committee. ``The
government, and particularly law enforcement, has been toothless.''

The report painted a picture of years of Russian interference through
disinformation spread by traditional media outlets, like the cable-TV
channel RT, and by the use of internet bots and trolls. This activity
dated back to the
\href{https://www.nytimes.com/2014/09/19/world/europe/scotland-independence-vote.html}{Scottish
independence referendum} in 2014, but it was never confronted by the
country's political establishment or by an intelligence community with
other priorities.

Image

RT, Russia's state-financed international cable network, has studios in
London.Credit...Sergey Ponomarev for The New York Times

Focused more on clandestine operations, the spy agencies were anxious to
keep their distance from political campaigns, regarding them as a ``hot
potato,'' the report said. Nor was it clear who in the government was in
charge of countering the Russian threat to destabilize Britain's
political process. ``It has been surprisingly difficult to establish who
has responsibility for what,'' the report said.

Despite pressing questions, the report said the government had shown
little interest in investigating whether the Brexit referendum was
targeted by Russia. The government responded that it had ``seen no
evidence of successful interference in the E.U. referendum'' and
dismissed the need for further investigation.

But the committee suggested that the reason no evidence had been
uncovered was because nobody had looked for it.

``In response to our request for written evidence at the outset of the
inquiry, MI5 initially provided just six lines of text,'' the committee
said. Had the intelligence agencies conducted a threat assessment before
the vote, it added, it was ``inconceivable'' that they would not have
concluded there was a Russian threat.

Among the report's most politically salient conclusions might be about a
Russian influence campaign during the Scottish independence referendum.
Nationalist sentiment is surging again in Scotland, partly because many
voters consider the Scottish authorities to have handled the coronavirus
pandemic better than the government in England. Based on its previous
behavior, some experts said, Russia would try again to encourage the
fracturing of the United Kingdom.

``That obviously has implications for next year's Scottish elections,
and the polling on referendums,'' said
\href{https://www.instituteforgovernment.org.uk/person/bronwen-maddox}{Bronwen
Maddox}, director of the Institute for Government, a research institute
in London. ``All this is very, very relevant.''

Concerns about Russian meddling and aggression stretch back more than a
decade to the death in 2006 of
\href{https://www.nytimes.com/topic/person/alexander-v-litvinenko}{Alexander
V. Litvinenko}, a former K.G.B. officer and critic of the Kremlin, who
was killed in London using a radioactive poison, polonium-210, believed
to have been administered in a cup of tea. An inquiry concluded that his
killing ``was probably approved'' by
\href{https://www.nytimes.com/topic/person/vladimir-putin}{President
Vladimir V. Putin.}

In 2018, another former Russian spy, Sergei Skripal, and his 33-year-old
daughter, Yulia, were found seriously ill on a bench in Salisbury, after
a poisoning attack that left them hospitalized for weeks. Britain
accused two Russians of using a rare nerve agent to try to kill Mr.
Skripal, and
\href{https://www.nytimes.com/2018/03/14/world/europe/uk-russia-spy-punitive-measures.html}{expelled
23 Russian diplomats} in retaliation.

Image

Investigating the poisoning of a former Russian spy and his daughter in
Salisbury, England, in 2018.Credit...Chris J Ratcliffe/Getty Images

Although the report was approved by Downing Street in 2019, its release
was held up before the election that gave Mr. Johnson his decisive
parliamentary majority. Critics said he had been compromised by
donations to his party from wealthy Russians living in Britain and they
argued that the report was delayed unnecessarily.

After the election, there was a second delay while Downing Street agreed
on the membership of a new Intelligence and Security Committee.

While the publicly available part of the report unearthed little new
material, one expert said that it underscored the need to widen the
focus and improve the coordination of Britain's intelligence apparatus.

``We did know most of this,'' said Martin Innes, director of the
\href{https://www.cardiff.ac.uk/crime-security-research-institute}{Crime
and Security Research Institute at Cardiff University}, ``but people
were not joining the dots and seeing that quite a serious situation was
developing.''

``What Russia wants is to be able to play great power politics,''
Professor Innes said. ``And one of the ways of doing that is by
destabilizing the U.K. and some of its close allies to create that space
to maneuver.''

Advertisement

\protect\hyperlink{after-bottom}{Continue reading the main story}

\hypertarget{site-index}{%
\subsection{Site Index}\label{site-index}}

\hypertarget{site-information-navigation}{%
\subsection{Site Information
Navigation}\label{site-information-navigation}}

\begin{itemize}
\tightlist
\item
  \href{https://help.nytimes.com/hc/en-us/articles/115014792127-Copyright-notice}{©~2020~The
  New York Times Company}
\end{itemize}

\begin{itemize}
\tightlist
\item
  \href{https://www.nytco.com/}{NYTCo}
\item
  \href{https://help.nytimes.com/hc/en-us/articles/115015385887-Contact-Us}{Contact
  Us}
\item
  \href{https://www.nytco.com/careers/}{Work with us}
\item
  \href{https://nytmediakit.com/}{Advertise}
\item
  \href{http://www.tbrandstudio.com/}{T Brand Studio}
\item
  \href{https://www.nytimes.com/privacy/cookie-policy\#how-do-i-manage-trackers}{Your
  Ad Choices}
\item
  \href{https://www.nytimes.com/privacy}{Privacy}
\item
  \href{https://help.nytimes.com/hc/en-us/articles/115014893428-Terms-of-service}{Terms
  of Service}
\item
  \href{https://help.nytimes.com/hc/en-us/articles/115014893968-Terms-of-sale}{Terms
  of Sale}
\item
  \href{https://spiderbites.nytimes.com}{Site Map}
\item
  \href{https://help.nytimes.com/hc/en-us}{Help}
\item
  \href{https://www.nytimes.com/subscription?campaignId=37WXW}{Subscriptions}
\end{itemize}
