Sections

SEARCH

\protect\hyperlink{site-content}{Skip to
content}\protect\hyperlink{site-index}{Skip to site index}

\href{https://www.nytimes.com/section/books}{Books}

\href{https://myaccount.nytimes.com/auth/login?response_type=cookie\&client_id=vi}{}

\href{https://www.nytimes.com/section/todayspaper}{Today's Paper}

\href{/section/books}{Books}\textbar{}In `Intimations,' Zadie Smith
Applies Her Even Temper to Tumultuous Times

\url{https://nyti.ms/39kqsh3}

\begin{itemize}
\item
\item
\item
\item
\item
\end{itemize}

Advertisement

\protect\hyperlink{after-top}{Continue reading the main story}

Supported by

\protect\hyperlink{after-sponsor}{Continue reading the main story}

\href{/column/books-of-the-times}{Books of The Times}

\hypertarget{in-intimations-zadie-smith-applies-her-even-temper-to-tumultuous-times}{%
\section{In `Intimations,' Zadie Smith Applies Her Even Temper to
Tumultuous
Times}\label{in-intimations-zadie-smith-applies-her-even-temper-to-tumultuous-times}}

By \href{https://www.nytimes.com/by/john-williams}{John Williams}

\begin{itemize}
\item
  July 22, 2020
\item
  \begin{itemize}
  \item
  \item
  \item
  \item
  \item
  \end{itemize}
\end{itemize}

\includegraphics{https://static01.nyt.com/images/2020/07/23/books/22BOOKSMITH1/22BOOKSMITH1-articleLarge.png?quality=75\&auto=webp\&disable=upscale}

Buy Book ▾

\begin{itemize}
\tightlist
\item
  \href{https://www.amazon.com/gp/search?index=books\&tag=NYTBSREV-20\&field-keywords=Intimations+Zadie+Smith}{Amazon}
\item
  \href{https://du-gae-books-dot-nyt-du-prd.appspot.com/buy?title=Intimations\&author=Zadie+Smith}{Apple
  Books}
\item
  \href{https://www.anrdoezrs.net/click-7990613-11819508?url=https\%3A\%2F\%2Fwww.barnesandnoble.com\%2Fw\%2F\%3Fean\%3D9780593297612}{Barnes
  and Noble}
\item
  \href{https://www.anrdoezrs.net/click-7990613-35140?url=https\%3A\%2F\%2Fwww.booksamillion.com\%2Fp\%2FIntimations\%2FZadie\%2BSmith\%2F9780593297612}{Books-A-Million}
\item
  \href{https://bookshop.org/a/3546/9780593297612}{Bookshop}
\item
  \href{https://www.indiebound.org/book/9780593297612?aff=NYT}{Indiebound}
\end{itemize}

When you purchase an independently reviewed book through our site, we
earn an affiliate commission.

Wunderkinds, even those who deservedly stick around a long time, don't
seem to age normally. Zadie Smith's presence will always carry a
significant memory of the 24-year-old who published
\href{https://archive.nytimes.com/www.nytimes.com/books/00/04/30/reviews/000430.30quinnt.html}{``White
Teeth''} to international acclaim. But, being subject to the space-time
continuum, Smith is in her mid-40s now, and has the temperament and
perspective of someone who could be (a compliment, in this case) 105.

It's never a boom time for wisdom --- almost by definition; if it were
more common, it wouldn't be valued so highly --- but this is an
especially arid era for it. We're in the Age of Certainty, at least in
the bellowing of its various constituents. And maybe that's fine; maybe
some times are just for fighting ideological fire with ideological fire.
But ``polemic'' is too generous a word for the dominant cultural tone.

All of which makes Smith feel especially out of time. In the very brief
foreword to her first book of essays,
\href{https://www.nytimes.com/2010/01/17/books/review/Mishra-t.html}{``Changing
My Mind,''} she wrote: ``Ideological inconsistency is, for me,
practically an article of faith.'' That faith doesn't seem to have
wavered in the 10 years since that book was published. ``Intimations,''
her slender new collection (less than 100 pages) of ultra-timely essays
(several written in the past few momentous months), showcases her
trademark levelheadedness.

This cast of mind doesn't mean that Smith avoids moral stances. In
``Intimations,'' she speaks clearly and forcefully about the murder of
George Floyd and the legacy of slavery and the systemic sins revealed by
Covid-19. ``The virus map of the New York boroughs turns redder along
precisely the same lines as it would if the relative shade of crimson
counted not infection and death but income brackets and middle-school
ratings,'' she writes. ``Death comes to all --- but in America it has
long been considered reasonable to offer the best chance of delay to the
highest bidder.''

At her most withering, on the subject of race, she writes about the
many, ``even in the bluest states in America,'' who ``are very happy to
`blackout' their social media for a day, to read all-Black books and
`educate' themselves about Black issues --- as long as this education
does not occur in the form of actual Black children attending their
actual schools.''

But despite these jabs, Smith remains unmistakably noncombative. This
spirit appears born not of a fear of confrontation but a genuine
perplexity (of a searching, brilliant kind) at the nature of experience
and people, including herself. She says that the art of writing, though
it's often advertised as ``creative,'' is really about ``control.''
``The part of the university in which I teach,'' she says, ``should
properly be called the Controlling Experience Department.''

As a writer and reader, what she finds --- in a phrase perfectly suited
to her sensibility --- is ``a wide repertoire of possible attitudes.''
But out in the world, living eludes control; it's ``mystifying,
overwhelming, conscious, subconscious,'' and ``it just keeps coming at
you.''

Image

Zadie Smith, whose new essay collection is
``Intimations.''Credit...Dominique Nabokov

Smith's gifts as a novelist animate her essays. Writing about the
neighborhood nail place where she regularly goes for stress relief, we
get a portrait of her masseur, Ben, who teases her for always reading
during their sessions and who asks where her hair ``comes from.''
(``Jamaica and England --- via Africa,'' she tells him. Ben replies:
``Ho ho ho! Interesting mix!'') As the essay closes, Smith watches Ben
from afar, the optimistic face she's used to seeing changed into ``a
stern portrait of calculation and concern,'' worried, Smith assumes,
about the constant heavy traffic needed for the place to pay the rent.

Anxiety lurks through these few pages. This is a work of minor
dimensions at --- and about --- a major time. (Royalties from the book
will go to two charities, The Equal Justice Initiative and The COVID-19
Emergency Relief Fund for New York.)

Smith herself left New York early on in the pandemic's course, and she
expresses guilt about this without over-performing it. She bemoans, when
thinking about the apocalypse or anything even approaching it, her lack
of a survival instinct. ``A book like `The Road' is as incomprehensible
to me as a Norse myth cycle in the original language,'' she writes.
``Suicide would hold out its quiet hand to me on the first day --- the
first hour.''

In an essay called ``Suffering Like Mel Gibson'' (its title is a play on
a popular meme), she writes provocatively of Christ on the cross,
looking at those crucified beside him and wondering ``whether his
agonies, when all was said and done, were relatively speaking in fact
better than those of the thieves and beggars to his left and right whose
sufferings long predated their present crucifixions and who had no hope
(unlike Christ) of an improved post-cross situation.'' This thought
comes in a passage addressing the word of this century so far,
``privilege,'' which she does with her usual many-sidedness: She notes
her own advantages; parses the stubbornness of inequality; and outlines
the explanatory (and experiential) limitations of privilege, including
its ultimate inability to shield anyone from suffering, sometimes to the
point of suicide. In Zadie Smith's universe --- meaning, for my money,
the one we're all living in --- complexity is king.

She sympathizes with the generations coming up behind her, born into a
beleaguered century and now living through the current crises with
worried eyes on a deeply tenuous future. In one of the finest lines in
``Intimations,'' Smith writes: ``The infinite promise of American youth
--- a promise elaborately articulated by movies and advertisements and
university prospectuses --- has been an empty lie for so long that I
notice my students joking about it with a black humor more appropriate
to old men, to the veterans of wars.''

It might be engrossing to hear Smith in conversation with those she now
teaches, to see where their ideas overlap and diverge. Toward the end of
the book, she writes elliptically of identity as an ``area of
interest.'' Elsewhere she argues for solidarity among ``the plague class
--- that is, all economically exploited people, whatever their race.''

Interested in what she once called ``coalition across difference,''
Smith has some opinions that she defines as commonplace but that she
must know are now hotly debated.

She resists, for instance, the idea of ``hate crime'' as a desirable
distinction, calling it ``an elevation of importance in what strikes me
as the wrong direction,'' lending an undeserved power to the bigotry
that inspires the term.

``The hatred of a group \emph{qua} group is, after all, the most debased
and irrational of hatreds, the weakest, the \emph{most} banal,'' she
writes. ``It shouldn't radiate a special aura, lifting it into a
separate epistemological category. For this is exactly what the killer
believes.''

Advertisement

\protect\hyperlink{after-bottom}{Continue reading the main story}

\hypertarget{site-index}{%
\subsection{Site Index}\label{site-index}}

\hypertarget{site-information-navigation}{%
\subsection{Site Information
Navigation}\label{site-information-navigation}}

\begin{itemize}
\tightlist
\item
  \href{https://help.nytimes.com/hc/en-us/articles/115014792127-Copyright-notice}{©~2020~The
  New York Times Company}
\end{itemize}

\begin{itemize}
\tightlist
\item
  \href{https://www.nytco.com/}{NYTCo}
\item
  \href{https://help.nytimes.com/hc/en-us/articles/115015385887-Contact-Us}{Contact
  Us}
\item
  \href{https://www.nytco.com/careers/}{Work with us}
\item
  \href{https://nytmediakit.com/}{Advertise}
\item
  \href{http://www.tbrandstudio.com/}{T Brand Studio}
\item
  \href{https://www.nytimes.com/privacy/cookie-policy\#how-do-i-manage-trackers}{Your
  Ad Choices}
\item
  \href{https://www.nytimes.com/privacy}{Privacy}
\item
  \href{https://help.nytimes.com/hc/en-us/articles/115014893428-Terms-of-service}{Terms
  of Service}
\item
  \href{https://help.nytimes.com/hc/en-us/articles/115014893968-Terms-of-sale}{Terms
  of Sale}
\item
  \href{https://spiderbites.nytimes.com}{Site Map}
\item
  \href{https://help.nytimes.com/hc/en-us}{Help}
\item
  \href{https://www.nytimes.com/subscription?campaignId=37WXW}{Subscriptions}
\end{itemize}
