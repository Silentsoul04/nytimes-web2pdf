Sections

SEARCH

\protect\hyperlink{site-content}{Skip to
content}\protect\hyperlink{site-index}{Skip to site index}

\href{https://myaccount.nytimes.com/auth/login?response_type=cookie\&client_id=vi}{}

\href{https://www.nytimes.com/section/todayspaper}{Today's Paper}

Four Artists on the Future of Video Art

\url{https://nyti.ms/3eUZm13}

\begin{itemize}
\item
\item
\item
\item
\item
\end{itemize}

\href{https://www.nytimes.com/spotlight/at-home?action=click\&pgtype=Article\&state=default\&region=TOP_BANNER\&context=at_home_menu}{At
Home}

\begin{itemize}
\tightlist
\item
  \href{https://www.nytimes.com/2020/08/03/well/family/the-benefits-of-talking-to-strangers.html?action=click\&pgtype=Article\&state=default\&region=TOP_BANNER\&context=at_home_menu}{Talk:
  To Strangers}
\item
  \href{https://www.nytimes.com/2020/08/01/at-home/coronavirus-make-pizza-on-a-grill.html?action=click\&pgtype=Article\&state=default\&region=TOP_BANNER\&context=at_home_menu}{Make:
  Grilled Pizza}
\item
  \href{https://www.nytimes.com/2020/07/31/arts/television/goldbergs-abc-stream.html?action=click\&pgtype=Article\&state=default\&region=TOP_BANNER\&context=at_home_menu}{Watch:
  'The Goldbergs'}
\item
  \href{https://www.nytimes.com/interactive/2020/at-home/even-more-reporters-editors-diaries-lists-recommendations.html?action=click\&pgtype=Article\&state=default\&region=TOP_BANNER\&context=at_home_menu}{Explore:
  Reporters' Google Docs}
\end{itemize}

Advertisement

\protect\hyperlink{after-top}{Continue reading the main story}

Supported by

\protect\hyperlink{after-sponsor}{Continue reading the main story}

True Believers

\hypertarget{four-artists-on-the-future-of-video-art}{%
\section{Four Artists on the Future of Video
Art}\label{four-artists-on-the-future-of-video-art}}

Hito Steyerl, Rachel Rose, Isaac Julien and Lynn Hershman Leeson talk
about how they've been spending quarantine and just where, in this era
of never-ending screen time, their work should live.

\includegraphics{https://static01.nyt.com/images/2020/07/10/t-magazine/art/digital-artists-slide-VVNP/digital-artists-slide-VVNP-articleLarge.jpg?quality=75\&auto=webp\&disable=upscale}

By Andrew Russeth

\begin{itemize}
\item
  July 22, 2020
\item
  \begin{itemize}
  \item
  \item
  \item
  \item
  \item
  \end{itemize}
\end{itemize}

When coronavirus shuttered just about every gallery in the United States
and confined many to their homes, museum curators and dealers
\href{https://www.nytimes.com/2020/03/16/arts/design/art-galleries-online-viewing-coronavirus.html}{had
to improvise}. Overnight, the only way they could show art was
digitally. In some cases, this meant posting photos of paintings or
using cameras to offer 360-degree virtual tours of exhibitions --- to
varying degrees of success. But then there were the works that were
\emph{designed} to be viewed on a screen, which have enjoyed a sort of a
renaissance.

On many Friday nights since April, the
\href{https://www.nytimes.com/topic/organization/whitney-museum-of-american-art}{Whitney
Museum of American Art} in New York has streamed video art by
\href{https://www.nytimes.com/2018/02/16/t-magazine/alex-da-corte-st-vincent.html}{Alex
Da Corte}, \href{https://whitney.org/artists/18064}{Juan Antonio
Olivares} and
\href{https://www.newmuseum.org/exhibitions/view/adelita-husni-bey-chiron}{Adelita
Husni-Bey}, to name a few. \href{https://gagosian.com/}{Gagosian
Gallery} staged web shows with moving-image work by artists including
\href{https://www.nytimes.com/2020/01/15/arts/design/ed-ruscha.html}{Ed
Ruscha} and \href{https://gagosian.com/artists/douglas-gordon/}{Douglas
Gordon}, Metro Pictures hosted a digital film festival over more than a
dozen weekends, and the artist
\href{https://www.nytimes.com/2018/06/13/t-magazine/artist-work-habits-camille-henrot-nina-chanel-abney.html}{Nina
Chanel Abney} curated a two-week run of pieces by
\href{https://www.nytimes.com/2019/05/09/arts/design/whitney-museum-biennial-artists.html}{Tiona
Nekkia McClodden},
\href{https://www.nytimes.com/2018/10/15/t-magazine/solange-interview.html}{Solange
Knowles} and others at Brooklyn's \href{https://webuygold.wtf/}{We Buy
Gold}. Pittsburgh's
\href{https://www.nytimes.com/2015/07/10/arts/design/carnegie-museum-to-open-a-survey-of-the-designer-peter-muller-munk.html}{Carnegie
Museum of Art} launched an online exhibition series with ``Lake
Valley,'' a 2016 cartoon-collage animation by
\href{https://www.artsy.net/artist/rachel-rose}{Rachel Rose} that
follows a rabbitlike animal as it explores an enchanted world, seeking
community. ``Two-dimensional work or sculpture all comes out a bit flat
on social media,'' says the filmmaker
\href{https://www.nytimes.com/2014/02/18/arts/international/facing-the-camera.html}{Isaac
Julien}. ``I'm not saying people can't sell works. They do, in fact. But
I think the moving image --- it becomes its own form. It's not really
compromised.''

\href{https://www.nytimes.com/issue/t-magazine/2020/07/02/true-believers-art-issue}{\includegraphics{https://static01.nyt.com/newsgraphics/2020/06/29/tmag-art-embeds-new/assets/images/art_issue_gif_special_editon.gif}}

Since its emergence in the 1960s, it's video art that has typically been
harder to present and sell and perhaps to take in, too, often requiring
time and patience of its viewer. When pioneers like
\href{https://www.nytimes.com/2018/10/15/t-magazine/bruce-nauman-art-interview.html}{Bruce
Nauman},
\href{https://www.nytimes.com/2017/04/28/arts/design/vito-acconci-dead-performance-artist.html}{Vito
Acconci} and
\href{https://www.nytimes.com/2015/04/05/t-magazine/joan-jonas-reanimation-venice-biennale.html}{Joan
Jonas} wielded cumbersome cameras in the 1970s to make scrappy, poetic
or otherwise bizarre tapes intended for gallery display, the prominent
dealers
\href{https://www.nytimes.com/1999/08/23/arts/leo-castelli-influential-art-dealer-dies-at-91.html}{Leo
Castelli} and
\href{https://www.nytimes.com/2007/10/24/arts/24sonnabend.html}{Ileana
Sonnabend} started a service to rent and sell them. It never turned a
profit. However, as production values increased and minds opened in the
1980s and '90s, stars were minted:
\href{https://www.nytimes.com/2017/03/14/t-magazine/art/bill-viola-palazzo-strozzi-florence.html}{Bill
Viola},
\href{https://www.nytimes.com/1999/10/10/magazine/the-importance-of-matthew-barney.html}{Matthew
Barney},
\href{https://www.nytimes.com/2016/10/22/arts/design/pipilotti-rist-provoking-with-delight.html}{Pipilotti
Rist} and
\href{https://www.nytimes.com/2018/11/19/arts/music/christian-marclay-huddersfield-music-festival.html}{Christian
Marclay} were among those to see their work sold in limited editions and
featured in major museums and international biennials, just as that of
\href{https://www.guggenheim.org/artwork/artist/ryan-trecartin}{Ryan
Trecartin},
\href{https://www.nytimes.com/2019/08/14/t-magazine/arthur-jafa-in-bloom.html}{Arthur
Jafa} and
\href{https://www.nytimes.com/2016/12/06/t-magazine/art/martine-syms-artist-poster-phrase.html}{Martine
Syms} has been more recently.

\includegraphics{https://static01.nyt.com/images/2020/07/10/t-magazine/art/digital-artists-slide-U4Y9/digital-artists-slide-U4Y9-articleLarge.jpg?quality=75\&auto=webp\&disable=upscale}

But even in 2020, with screens glowing all around us, video art remains
a relatively niche field. It is still rare, for instance, to come across
a figure like \href{https://www.jsc.art/}{Julia Stoschek}, a German
collector who focuses exclusively on multimedia and video. When the art
spaces Stoschek runs in Berlin and Düsseldorf, Germany, were closed in
March, she collaborated with artists and technicians to upload more than
70 works from her 800-some-piece collection online. Her aim, she says,
is ``to make it accessible for everyone, all the time, everywhere.''
\href{https://carnegiemuseums.org/expert/eric-crosby/}{Eric Crosby}, the
Carnegie's director, feels similarly, saying that a lesson from lockdown
is that ``audiences should be able to encounter art regardless of
whether our museum doors are open or closed.'' Their comments underscore
the diffuse and potentially democratic nature of video art in comparison
to, say, oil painting.

At the same time, readily available offerings raise questions about
where exactly video art should live, and how its future might be shaped.
Is its star turn on the web a sign of things to come, or just a
momentary detour? Here, four video and digital artists talk about
working during the pandemic and share thoughts on the field as a whole.

Image

An installation view of Hito Steyerl's ``Liquidity Inc.'' (2014) at
Artists Space in New York in 2015.Credit...Courtesy of the artist,
Andrew Kreps Gallery, New York, and Esther Schipper, Berlin. Photo:
Matthew Septimus

\hypertarget{hito-steyerl}{%
\subsubsection{\texorpdfstring{\textbf{Hito
Steyerl}}{Hito Steyerl}}\label{hito-steyerl}}

\emph{For more than 20 years,}
\href{https://www.nytimes.com/2017/12/15/arts/design/hito-steyerl.html}{\emph{Hito
Steyerl}}\emph{, 53, has made incisive, deliriously entertaining videos
that examine how images and ideas circulate. Based in Berlin, she is set
to have her first}
\href{https://www.kunstsammlung.de/de/exhibitions/hito-steyerl}{\emph{survey
exhibition}} \emph{in Germany --- at the Kunstsammlung
Nordrhein-Westfalen, in Düsseldorf --- in September.}

Most of my work has always been online. But I try not to share it on
commercial or corporate platforms, so you will find very little of it on
YouTube. It's mostly hidden in plain sight, meaning that it's on
platforms that are not Google searchable, which technically makes them
part of the dark web.

The concept of the digital sphere is not in a very fortunate moment.
I've been talking for the past year about how the digital has been
privatized and gentrified. It's basically monetized by four or five big
corporations, and none of this has changed just because of lockdown, so
I'm very ambivalent about the digital euphoria.

What institutions or galleries assume is that videos are available and
that they can just stream them for free without taking into account any
sort of production cost or anything like that because the web is seen as
a sphere of distribution of things that are free. There is a sort of
digital fatigue now, and I think there is a whole oversupply, and it's
not good for the artwork. It doesn't get it any more attention. On the
other hand, if people were to give it more time and attention, and also
resources, I think it could develop a lot. The over-traveling that the
art world has experienced could be reduced if there were some convincing
digital platforms. But I'm not sure whether that will happen.

Basically, I think it's a good idea to expand digital distribution, but
then we also need public digital infrastructure. We need, let's say,
municipal digital platforms, for teaching, for schooling, for education,
but also, of course, for the arts. We should also keep in mind that most
of these means of production are not easily accessible --- especially if
we go into this video sphere, with 3-D modeling and VR, it quickly
becomes very complex and expensive. And it stifles the development of
modes of expression because it's just too expensive and young people
have a hard time getting access to it. There is a huge difference to 20
or 30 years ago, because the camcorders were really cheap. That was my
only shot at ever entering this kind of activity.

But if people in Minneapolis can start thinking about
\href{https://www.nytimes.com/2020/06/07/us/minneapolis-police-abolish.html}{how
to build their security institutions}, then why not a municipal digital
platform? I think that's really easy in comparison --- even cheap. Many
things that once seemed impossible are suddenly possible, so why not
try?

\begin{center}\rule{0.5\linewidth}{\linethickness}\end{center}

Image

An installation view of Rachel Rose's ``Lake Valley'' (2016) at
Lafayette Anticipations in Paris this year.Credit...Courtesy of the
artist and Gavin Brown's enterprise New York/Rome. Photo: Andrea
Rossetti

\hypertarget{rachel-rose}{%
\subsubsection{\texorpdfstring{\textbf{Rachel
Rose}}{Rachel Rose}}\label{rachel-rose}}

\emph{Based in New York, though she's been working upstate during this
period, Rachel Rose, 33, shows transfixing, research-intensive videos in
intimate, carefully crafted environments. Her 2016 work ``Lake Valley''}
\href{https://cmoa.org/exhibition/rachel-rose/}{\emph{is on view}}
\emph{on the Carnegie Museum of Art's website through Aug. 16, and at a
solo show at Lafayette Anticipations in Paris through Sept. 13.}

The biggest change during lockdown has really been not having time. Not
having child care for Eden, my 10-month-old daughter, I don't have the
same amount of time to myself every day. It's more broken up and
staccato. But I'm also planning for an exhibition, and I'm working on a
screenplay, so the computer is all I need.

The Carnegie proposed doing an online exhibition of ``Lake Valley,'' and
I was excited about that because it felt like a nice opportunity to show
it to families at home with kids, which I had always wanted to do. All
of my works thus far have been developed in relationship, at least in
part, to the site where I'm first showing them, because so many of them
have been commissioned by institutions. So for me, the installation has
always been a condition of the work. One of the gifts of the art
universe is the opportunity to think through the physicality of how you
view something. That's not something you get with a feature film,
because it's going out to theaters or it's going out to streaming
platforms. But, because none of us can be in physical spaces together, I
feel it might as well be shared online.

I wonder about this period as a time for slower kinds of producing. Art
is on this very seasonal cycle that parallels fashion. There are the
September shows, there are the art fairs --- it's this constant
saturation, and I think for artists, that can often make them feel as
though they need to keep up with that pace, which keeps getting faster
and faster. Because of what I do, and the way in which my work has been
commissioned, I've been able to work more slowly, but I notice how
detrimental that cycle can be, and this could be the beginning of a new
sort of time scale.

Online is essential right now, but as a long-term strategy, I'm not
sure. When I open up Netflix and look at the main page, I've never heard
of most of the TV shows it lists, and most of them are like six seasons
in. It feels like there's so much entertainment and moving-image stuff
online. In a way, I feel like the value that art can hold is parallel to
that of live concerts. We can all listen to anything on Spotify or
iTunes, but going to a concert is an entirely different thing. That is
one of the things that museums can do for us, for art. Maybe, after
Covid, art should become \emph{less} visible on the internet. At the
same time, the immediacy of
``\href{https://www.nytimes.com/2020/04/22/arts/design/lizards-instagram-coronavirus-stars.html}{Two
Lizards}'' {[}a serialized video project the artists
\href{https://www.nytimes.com/2015/10/28/t-magazine/art-meriem-bennani-hijab-video.html}{Meriem
Bennani} and \href{https://www.orianbarki.com/}{Orian Barki} debuted in
March{]} --- what is that? It's like watching art live. And there's
something really exhilarating and beautiful about that, too.

\begin{center}\rule{0.5\linewidth}{\linethickness}\end{center}

\includegraphics{https://static01.nyt.com/images/2020/07/10/t-magazine/art/digital-artists-slide-QOUF/digital-artists-slide-QOUF-videoSixteenByNine3000.jpg}

\hypertarget{isaac-julien}{%
\subsubsection{\texorpdfstring{\textbf{Isaac
Julien}}{Isaac Julien}}\label{isaac-julien}}

\emph{After having a solo show at}
\href{https://www.nytimes.com/2020/01/22/t-magazine/jessica-silverman-gallery.html}{\emph{Jessica
Silverman Gallery}} \emph{in San Francisco cut short because of
shelter-in-place orders, Isaac Julien, 60, was working in Santa Cruz,
whose University of California branch is home to the}
\href{https://danm.ucsc.edu/project_group/isaac-julien-lab}{\emph{Isaac
Julien Lab}}\emph{. Known for sumptuous videos centered on radical
histories, which he releases in multiple formats, from single-screen
cuts to immersive multiscreen installations, Julien is scheduled to have
a show at the McEvoy Foundation for the Arts in San Francisco in the
fall.}

Filming would be impossible right now. Maybe the gods were looking down
on me in 2019, because I had the kind of mad intention of making these
two gigantic projects --- ``A Marvellous Entanglement,'' on the
architect Lina Bo Bardi, and ``Lessons of the Hour,'' on Frederick
Douglass --- and I did. That has meant that 2020 has been the kind of
year with more exhibitions and my making single-screen versions of work,
so a time when it's more postproduction. If I had not done that, this
would be very disruptive. There really are some things you can continue
doing while social distancing, and there are some things that you
cannot.

And yet, we're seeing this flourishing of video art and media works on
social-media platforms. I participated in this Metro Pictures film
festival, which I think was really successful. I really enjoyed seeing
``Baltimore'' {[}his 2003 short starring
\href{https://www.nytimes.com/2013/09/20/arts/design/melvin-van-peebles-headlines-a-group-art-show.html}{Melvin
Van Peebles}{]} during that time, and it was great to be able to post
about it on Facebook and Instagram, to have all the responses to the
work. I realized that a lot of the video artworks that one makes ---
they become, in a way, connected to the time when they were made. People
have the memory. But it's great to be able to redistribute them on
social-media platforms and to introduce the work to new audiences.
Viewers were really excited, and this made me think about the
possibility of how those works could live in a different capacity. When
we have an exhibition of work showing in a museum, maybe we can have a
single-screen version on social media simultaneously and think about
both platforms as exhibition spaces.

What was good about the festival was that you got excited about who was
going to be the next artist, and then you looked at the films, and you
had time to look at them, and you could really learn things. Since then,
other works of mine have been shown at special events. For example, in
Brazil, the
\href{https://www.goethe.de/ins/br/pt/sta/sal.html}{Goethe-Institut} in
\href{https://www.nytimes.com/2019/01/24/travel/what-to-do-in-salvador-brazil.html}{Salvador}
showed my Fanon film
{[}``\href{https://www.isaacjulien.com/projects/frantz-fanon-black-skin-white-mask/}{Frantz
Fanon: Black Skin, White Mask}'' (1995){]} --- it was just on for 24
hours, and it was watched by over 37,000 people. We kind of couldn't
believe it when the figure came out. Funnily enough, the Fanon film was
also showing in an exhibition in Singapore, and they showed it online,
too, and some friends from Germany saw it. So you have this
internationalization of the platform, of different people watching
different works.

This kind of lit the fuse. I got approached by lots of other
institutions and museums, and I thought to myself, ``OK, hang on here a
minute. I think I might stop and think about it a little bit more,
because maybe we can do it ourselves, in the studio.''

\begin{center}\rule{0.5\linewidth}{\linethickness}\end{center}

Image

A screenshot of Lynn Hershman Leeson's ``Agent Ruby's EDream Portal''
(2002).Credit...© Lynn Hershman Leeson, courtesy of the artist and
Bridget Donahue, N.Y.C.

\hypertarget{lynn-hershman-leeson}{%
\subsubsection{\texorpdfstring{\textbf{Lynn Hershman
Leeson}}{Lynn Hershman Leeson}}\label{lynn-hershman-leeson}}

\emph{For more than half a century, the San Francisco-based}
\href{https://www.nytimes.com/2019/11/08/arts/design/Lynn-Hershman-Leeson-Shed-art-technology.html}{\emph{Lynn
Hershman Leeson}}\emph{, 79, has released feature films,
cross-disciplinary scientific endeavors, interactive videos (like
1979-84's
``}\href{https://www.lynnhershman.com/lorna/}{\emph{Lorna}}\emph{,''
which concerns an agoraphobic woman) and web projects such as
``}\href{http://www.lynnhershman.com/agent-ruby/}{\emph{Agent
Ruby}}\emph{'' (1998-2002), a chatbot with artificial intelligence that
can converse with internet visitors. A survey of her work at the}
\href{https://www.nytimes.com/topic/organization/new-museum-of-contemporary-art}{\emph{New
Museum}} \emph{in New York has been postponed because of the pandemic,
though she is currently developing what she calls ``an occultish online
game.''}

I probably have more time to work now because usually I run around a
lot. In the past few months, I was able to finish a lot of projects I
started in the '60s. I'm doing calls on Zoom, which I didn't have
before, but my practice hasn't changed all that much because these
things do take a long time to develop. I have the luxury of staying home
and being able to commute digitally, which is something that
\href{https://www.nytimes.com/2006/01/31/arts/design/nam-june-paik-73-dies-pioneer-of-video-art-whose-work-broke.html}{Nam
June Paik} talked about --- how we're becoming stationary nomads. He
talked about that in the early 1980s, going all around the world without
leaving your house.

I started a project called ``Agent Ruby'' in 1998, and nobody knew what
to do with it, and I finally gave it to SFMOMA because I couldn't afford
her upkeep on the net. I was told that the piece is the most-visited
artwork in their collection. What really astounded me, because we did an
exhibition of it in 2013, was that you do these things on the internet
and they never die. The museum had collected something like 80 tons of
global responses that became a portrait of what the world was thinking
about all of these years.

One of the things I want to do when I have my next museum exhibition ---
and I don't think that any museum can afford to not do this --- is
really design a way it can be seen online. I think that there are ways
that we can design almost telerobotic surveillance systems that allow
you to see works better --- really going into the piece and being able
to understand it and see details of it. I haven't seen anybody, any
museum, take advantage of the possibilities of how a work can be seen
online. The viewer needs to be in control of what they're seeing and
have access to the tools that will allow them to do that. The way that
museums have been portraying exhibitions is that they're in control.
They let you fly through a gallery, but they don't let you stop or go
into something to understand it in a more tactile way.

Artists use the tools of their time and if, for instance, people who
were shooting in 8-millimeter or 16-millimeter then want to convert it
to video or maybe a form that may last longer, it's not the same piece.
It doesn't look the same, it doesn't feel the same. You don't breathe
the same way when you watch it. The light is different. We actually took
``Lorna'' from LaserDisc and migrated it to a DVD just because it could
be shown that way. We have the original in an archive, but I kept all
the mistakes in --- I wanted it to look like it was made then. I think
glitches are the key to discovery. They're underrated.

\emph{These interviews have been edited and condensed.}

\hypertarget{true-believers-art-issue}{%
\subsubsection{\texorpdfstring{\href{https://www.nytimes.com/issue/t-magazine/2020/07/02/true-believers-art-issue}{True
Believers Art
Issue}}{True Believers Art Issue}}\label{true-believers-art-issue}}

Advertisement

\protect\hyperlink{after-bottom}{Continue reading the main story}

\hypertarget{site-index}{%
\subsection{Site Index}\label{site-index}}

\hypertarget{site-information-navigation}{%
\subsection{Site Information
Navigation}\label{site-information-navigation}}

\begin{itemize}
\tightlist
\item
  \href{https://help.nytimes.com/hc/en-us/articles/115014792127-Copyright-notice}{©~2020~The
  New York Times Company}
\end{itemize}

\begin{itemize}
\tightlist
\item
  \href{https://www.nytco.com/}{NYTCo}
\item
  \href{https://help.nytimes.com/hc/en-us/articles/115015385887-Contact-Us}{Contact
  Us}
\item
  \href{https://www.nytco.com/careers/}{Work with us}
\item
  \href{https://nytmediakit.com/}{Advertise}
\item
  \href{http://www.tbrandstudio.com/}{T Brand Studio}
\item
  \href{https://www.nytimes.com/privacy/cookie-policy\#how-do-i-manage-trackers}{Your
  Ad Choices}
\item
  \href{https://www.nytimes.com/privacy}{Privacy}
\item
  \href{https://help.nytimes.com/hc/en-us/articles/115014893428-Terms-of-service}{Terms
  of Service}
\item
  \href{https://help.nytimes.com/hc/en-us/articles/115014893968-Terms-of-sale}{Terms
  of Sale}
\item
  \href{https://spiderbites.nytimes.com}{Site Map}
\item
  \href{https://help.nytimes.com/hc/en-us}{Help}
\item
  \href{https://www.nytimes.com/subscription?campaignId=37WXW}{Subscriptions}
\end{itemize}
