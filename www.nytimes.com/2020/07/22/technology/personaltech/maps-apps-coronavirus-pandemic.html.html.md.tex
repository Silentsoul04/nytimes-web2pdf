Sections

SEARCH

\protect\hyperlink{site-content}{Skip to
content}\protect\hyperlink{site-index}{Skip to site index}

\href{https://www.nytimes.com/section/technology/personaltech}{Personal
Tech}

\href{https://myaccount.nytimes.com/auth/login?response_type=cookie\&client_id=vi}{}

\href{https://www.nytimes.com/section/todayspaper}{Today's Paper}

\href{/section/technology/personaltech}{Personal Tech}\textbar{}Your
Trusty Maps App Can Help You Navigate the Pandemic

\url{https://nyti.ms/2ZP9JiF}

\begin{itemize}
\item
\item
\item
\item
\item
\end{itemize}

\href{https://www.nytimes.com/spotlight/at-home?action=click\&pgtype=Article\&state=default\&region=TOP_BANNER\&context=at_home_menu}{At
Home}

\begin{itemize}
\tightlist
\item
  \href{https://www.nytimes.com/2020/07/28/books/time-for-a-literary-road-trip.html?action=click\&pgtype=Article\&state=default\&region=TOP_BANNER\&context=at_home_menu}{Take:
  A Literary Road Trip}
\item
  \href{https://www.nytimes.com/2020/07/29/magazine/bored-with-your-home-cooking-some-smoky-eggplant-will-fix-that.html?action=click\&pgtype=Article\&state=default\&region=TOP_BANNER\&context=at_home_menu}{Cook:
  Smoky Eggplant}
\item
  \href{https://www.nytimes.com/2020/07/27/travel/moose-michigan-isle-royale.html?action=click\&pgtype=Article\&state=default\&region=TOP_BANNER\&context=at_home_menu}{Look
  Out: For Moose}
\item
  \href{https://www.nytimes.com/interactive/2020/at-home/even-more-reporters-editors-diaries-lists-recommendations.html?action=click\&pgtype=Article\&state=default\&region=TOP_BANNER\&context=at_home_menu}{Explore:
  Reporters' Obsessions}
\end{itemize}

Advertisement

\protect\hyperlink{after-top}{Continue reading the main story}

Supported by

\protect\hyperlink{after-sponsor}{Continue reading the main story}

Tech Tip

\hypertarget{your-trusty-maps-app-can-help-you-navigate-the-pandemic}{%
\section{Your Trusty Maps App Can Help You Navigate the
Pandemic}\label{your-trusty-maps-app-can-help-you-navigate-the-pandemic}}

Apple and Google have added handy features for these uncertain times.

\includegraphics{https://static01.nyt.com/images/2020/07/22/technology/personaltech/22TECHTIP_TOP/22TECHTIP_TOP-articleLarge.jpg?quality=75\&auto=webp\&disable=upscale}

\href{https://www.nytimes.com/by/j-d-biersdorfer}{\includegraphics{https://static01.nyt.com/images/2018/06/14/multimedia/author-j-d-biersdorfer/author-j-d-biersdorfer-thumbLarge.png}}

By \href{https://www.nytimes.com/by/j-d-biersdorfer}{J. D. Biersdorfer}

\begin{itemize}
\item
  July 22, 2020
\item
  \begin{itemize}
  \item
  \item
  \item
  \item
  \item
  \end{itemize}
\end{itemize}

If your town is partly closed or you're wary of travel during the
Covid-19 pandemic, it might feel as if your phone's map app is just
sitting there gathering digital dust. But even if you're not tapping
\href{https://www.apple.com/ios/maps/}{Apple's Maps} or
\href{https://www.google.com/maps/about/\#!/}{Google Maps} to explore an
exotic vacation spot or to belt out turn-by-turn directions on a long
road trip this summer, your interactive travel aid can be useful. Here
are a few things you can do.

\hypertarget{find-whats-open-or-closed}{%
\subsection{Find What's Open (or
Closed)}\label{find-whats-open-or-closed}}

Major American cities have been in varying stages of closure for months,
and it may be hard to remember which businesses are open. While a local
government's website should have general guidelines posted, both the iOS
Maps app from Apple and Google Maps (for
\href{https://play.google.com/store/apps/details?id=com.google.android.apps.maps\&hl=en_US}{Android}
and
\href{https://apps.apple.com/us/app/google-maps-transit-food/id585027354}{iOS})
have been updating their map labels and listings pages for specific
businesses to note adjusted hours, any curbside pickup service and
temporary closures.

\includegraphics{https://static01.nyt.com/images/2020/07/22/technology/personaltech/22TECHTIP_01/22TECHTIP_01-articleLarge.jpg?quality=75\&auto=webp\&disable=upscale}

But what if you find outdated details? In Apple's Maps app, tap the name
of the business on the map and, when its information page opens, scroll
down and tap Report an Issue; you can
\href{https://support.apple.com/en-us/HT203080}{report other
cartographic issues} by tapping the encircled ``i'' in the top-right
corner of the map itself. In Google Maps, select a business and scroll
down on its information page to the
\href{https://support.google.com/local-guides/answer/7084895?co=GENIE.Platform\%3DAndroid\&hl=en-GB\&oco=1}{``Suggest
an edit''} option.

\hypertarget{find-restaurants}{%
\subsection{Find Restaurants}\label{find-restaurants}}

Many dining establishments have struggled during the pandemic, as some
have stayed open with reduced service while others have been forced to
close. Apple's Maps app often notes temporary or permanent closures and
operating hours on its Yelp-assisted restaurant listings pages. As part
of its \href{https://support.google.com/maps/answer/9795160}{Covid-19
updates}, Google now adds a line on a restaurant's info page that lists
the status of dine-in, takeout and delivery service.

Image

Apple's Maps app and Google Maps both show operating hours and website
links for restaurants, but Google adds information about dine-in,
takeout and delivery options.Credit...The New York Times

Like Google Maps, Apple's Maps includes the restaurant's phone number
and website for details straight from the source. Use this contact
information to confirm current delivery and takeout services --- along
with any outdoor-dining options.

\hypertarget{find-a-covid-19-testing-site}{%
\subsection{Find a Covid-19 Testing
Site}\label{find-a-covid-19-testing-site}}

\href{https://www.cdc.gov/coronavirus/2019-ncov/symptoms-testing/testing.html}{State
and local health departments manage} testing, but if you have
\href{https://www.cdc.gov/coronavirus/2019-ncov/symptoms-testing/symptoms.html}{coronavirus
symptoms} or your medical provider advises you to
\href{https://www.cdc.gov/coronavirus/2019-ncov/lab/testing.html}{get
tested}, find a facility. Apple and Google now include the locations of
Covid-19 testing sites in their maps apps using data gleaned from
government agencies, public-health departments and health care
institutions.

Image

Both Apple and Google have added locations and other information for
Covid-19 testing sites around the country.Credit...The New York Times

To see places where you can potentially be tested, enter a variation of
``Covid-19 testing'' into the search box in the maps app. When you
select a facility from the resulting list, it should show any additional
requirements for getting a test there, like an appointment or a doctor's
referral.

\hypertarget{find-socially-distant-activity}{%
\subsection{Find Socially Distant
Activity}\label{find-socially-distant-activity}}

If you need to leave home for work, errands or other reasons and don't
drive, both maps apps provide information on the current status of local
public-transit service. But if your destination is walkable, going by
foot offers exercise and a change of scenery --- just tap the icon for
walking directions.

A quick search in either maps app for ``parks near me'' can lead you to
local greenery; maps for larger parks often include footpaths and
attractions within the park.

Image

You can plan your outdoor exercise in nearby parks as shown in Apple's
Maps app, left, or have Google Maps show you bike trails when you tap
the Layers icon in the upper-right corner and select
Bicycling.Credit...The New York Times

For cyclists, Google Maps displays the
\href{https://support.google.com/maps/answer/3092439?co=GENIE.Platform\%3DAndroid\&hl=en\&oco=1}{types
of bicycle trails} available on a route (like dedicated traffic lanes or
off-road dirt paths), along with landscape and terrain details when
selected in the Layers menu. Apple's
\href{https://www.apple.com/ios/ios-14-preview/features/}{iOS 14
software}, coming this year, will enhance its Maps app features with
routes showing things like bicycle lanes, bike-friendly roads, elevation
and traffic information.

\hypertarget{find-a-digital-diversion}{%
\subsection{Find a Digital Diversion}\label{find-a-digital-diversion}}

Preventing the spread of the virus may discourage many people from
taking trips this summer. But if you still want to explore new places,
you can do a bit of virtual travel right in your maps app.

For example, when you search for a major city or landmark in Apple's
Maps app, look for the
\href{https://support.apple.com/guide/iphone/take-flyover-tours-in-maps-iph81a3f978/ios}{Flyover}
button or 3-D icon on the information page, and tap it to take an aerial
tour. You can also do a street-level exploration of major cities by
tapping the binoculars icon and using the app's
\href{https://support.apple.com/guide/iphone/look-around-iph65703a702/ios}{Look
Around} feature.

Image

No, it's not the same as being there, but the Flyover tours in Apple's
Maps app or the Street View feature in Google Maps can virtually take
you places.Credit...The New York Times

For years, Google Maps has included a similar
\href{https://support.google.com/maps/answer/3093484?co=GENIE.Platform\%3DAndroid\&hl=en\&oco=1}{Street
View} feature that shows panoramic photos of a location as if you were
standing there. But if you really want to see the world, check out
\href{https://www.google.com/earth/versions/\#earth-for-mobile}{Google
Earth}, the company's other free maps app for
\href{https://play.google.com/store/apps/details?id=com.google.earth\&hl=en_US}{Android}
and \href{https://apps.apple.com/us/app/google-earth/id293622097}{iOS}.
Once installed, Google Earth lets you ``fly'' around the globe, explore
cities in 3-D and wander through map-based tours in its
\href{https://support.google.com/earth/answer/7365064?hl=en\&ref_topic=7364880\&co=GENIE.Platform\%3DAndroid\&oco=1}{Voyager}
section. True, virtual travel in any app can't match the thrill of going
in person --- but for now, it's safer.

Advertisement

\protect\hyperlink{after-bottom}{Continue reading the main story}

\hypertarget{site-index}{%
\subsection{Site Index}\label{site-index}}

\hypertarget{site-information-navigation}{%
\subsection{Site Information
Navigation}\label{site-information-navigation}}

\begin{itemize}
\tightlist
\item
  \href{https://help.nytimes.com/hc/en-us/articles/115014792127-Copyright-notice}{©~2020~The
  New York Times Company}
\end{itemize}

\begin{itemize}
\tightlist
\item
  \href{https://www.nytco.com/}{NYTCo}
\item
  \href{https://help.nytimes.com/hc/en-us/articles/115015385887-Contact-Us}{Contact
  Us}
\item
  \href{https://www.nytco.com/careers/}{Work with us}
\item
  \href{https://nytmediakit.com/}{Advertise}
\item
  \href{http://www.tbrandstudio.com/}{T Brand Studio}
\item
  \href{https://www.nytimes.com/privacy/cookie-policy\#how-do-i-manage-trackers}{Your
  Ad Choices}
\item
  \href{https://www.nytimes.com/privacy}{Privacy}
\item
  \href{https://help.nytimes.com/hc/en-us/articles/115014893428-Terms-of-service}{Terms
  of Service}
\item
  \href{https://help.nytimes.com/hc/en-us/articles/115014893968-Terms-of-sale}{Terms
  of Sale}
\item
  \href{https://spiderbites.nytimes.com}{Site Map}
\item
  \href{https://help.nytimes.com/hc/en-us}{Help}
\item
  \href{https://www.nytimes.com/subscription?campaignId=37WXW}{Subscriptions}
\end{itemize}
