Sections

SEARCH

\protect\hyperlink{site-content}{Skip to
content}\protect\hyperlink{site-index}{Skip to site index}

\href{https://www.nytimes.com/section/us}{U.S.}

\href{https://myaccount.nytimes.com/auth/login?response_type=cookie\&client_id=vi}{}

\href{https://www.nytimes.com/section/todayspaper}{Today's Paper}

\href{/section/us}{U.S.}\textbar{}U.S. Hospitalizations for the
Coronavirus Near April Peak

\url{https://nyti.ms/2WOwXnj}

\begin{itemize}
\item
\item
\item
\item
\item
\end{itemize}

\href{https://www.nytimes.com/news-event/coronavirus?action=click\&pgtype=Article\&state=default\&region=TOP_BANNER\&context=storylines_menu}{The
Coronavirus Outbreak}

\begin{itemize}
\tightlist
\item
  live\href{https://www.nytimes.com/2020/08/01/world/coronavirus-covid-19.html?action=click\&pgtype=Article\&state=default\&region=TOP_BANNER\&context=storylines_menu}{Latest
  Updates}
\item
  \href{https://www.nytimes.com/interactive/2020/us/coronavirus-us-cases.html?action=click\&pgtype=Article\&state=default\&region=TOP_BANNER\&context=storylines_menu}{Maps
  and Cases}
\item
  \href{https://www.nytimes.com/interactive/2020/science/coronavirus-vaccine-tracker.html?action=click\&pgtype=Article\&state=default\&region=TOP_BANNER\&context=storylines_menu}{Vaccine
  Tracker}
\item
  \href{https://www.nytimes.com/interactive/2020/07/29/us/schools-reopening-coronavirus.html?action=click\&pgtype=Article\&state=default\&region=TOP_BANNER\&context=storylines_menu}{What
  School May Look Like}
\item
  \href{https://www.nytimes.com/live/2020/07/31/business/stock-market-today-coronavirus?action=click\&pgtype=Article\&state=default\&region=TOP_BANNER\&context=storylines_menu}{Economy}
\end{itemize}

Advertisement

\protect\hyperlink{after-top}{Continue reading the main story}

Supported by

\protect\hyperlink{after-sponsor}{Continue reading the main story}

\hypertarget{us-hospitalizations-for-the-coronavirus-near-april-peak}{%
\section{U.S. Hospitalizations for the Coronavirus Near April
Peak}\label{us-hospitalizations-for-the-coronavirus-near-april-peak}}

The rising hospitalizations reflect the scale of serious illnesses:
Nearly as many people are in hospitals now as there were when New York
was at its worst.

\includegraphics{https://static01.nyt.com/images/2020/07/22/us/22VIRUS-HOSPITALIZATIONS-nc/merlin_174692424_192772b6-f34a-41a3-987b-de5ddef9aedd-articleLarge.jpg?quality=75\&auto=webp\&disable=upscale}

By \href{https://www.nytimes.com/by/nicholas-bogel-burroughs}{Nicholas
Bogel-Burroughs} and
\href{https://www.nytimes.com/by/sarah-mervosh}{Sarah Mervosh}

\begin{itemize}
\item
  Published July 22, 2020Updated July 25, 2020
\item
  \begin{itemize}
  \item
  \item
  \item
  \item
  \item
  \end{itemize}
\end{itemize}

They are hooked up to ventilators, relying on the machines to breathe.
They are taking experimental drugs that doctors hope will ease their
agony. They are isolated from their families, fighting to recover on
their own.

More people are on track to be hospitalized with the coronavirus in the
United States than at any point in the pandemic, a disturbing sign of
how the current surge has spread widely and is seriously sickening as
many people as ever.

Across the country, 59,628 **** people were being treated in hospitals
on Wednesday, according to
\href{https://covidtracking.com/data/us-daily}{the Covid Tracking
Project}, nearing an earlier peak of 59,940 on April 15, when the center
of the outbreak was New York.

The country is
\href{https://www.nytimes.com/interactive/2020/us/coronavirus-us-cases.html}{averaging
more than 66,000 new virus cases per day}, more than twice as many as a
month ago, and deaths have also started trending upward, with an average
of more than 800 daily. But hospitalizations may be the clearest measure
of how widely the virus is causing the most serious illnesses, and could
offer a glimpse of what is ahead.

``Once you get to the point of being hospitalized or in the I.C.U., some
notable portion of those people will die,'' said Natalie E. Dean, an
infectious-disease expert at the University of Florida. Even when
patients walk out of the hospital, ``we don't know what the long-term
consequences are,'' she said. ``Surviving doesn't mean thriving.''

Not long ago, things seemed to be improving. Fewer than 28,000 patients
were hospitalized as of mid-June, when a new surge of cases was
appearing throughout the Sun Belt.

\hypertarget{latest-updates-global-coronavirus-outbreak}{%
\section{\texorpdfstring{\href{https://www.nytimes.com/2020/08/01/world/coronavirus-covid-19.html?action=click\&pgtype=Article\&state=default\&region=MAIN_CONTENT_1\&context=storylines_live_updates}{Latest
Updates: Global Coronavirus
Outbreak}}{Latest Updates: Global Coronavirus Outbreak}}\label{latest-updates-global-coronavirus-outbreak}}

Updated 2020-08-02T07:42:09.613Z

\begin{itemize}
\tightlist
\item
  \href{https://www.nytimes.com/2020/08/01/world/coronavirus-covid-19.html?action=click\&pgtype=Article\&state=default\&region=MAIN_CONTENT_1\&context=storylines_live_updates\#link-34047410}{The
  U.S. reels as July cases more than double the total of any other
  month.}
\item
  \href{https://www.nytimes.com/2020/08/01/world/coronavirus-covid-19.html?action=click\&pgtype=Article\&state=default\&region=MAIN_CONTENT_1\&context=storylines_live_updates\#link-780ec966}{Top
  U.S. officials work to break an impasse over the federal jobless
  benefit.}
\item
  \href{https://www.nytimes.com/2020/08/01/world/coronavirus-covid-19.html?action=click\&pgtype=Article\&state=default\&region=MAIN_CONTENT_1\&context=storylines_live_updates\#link-2bc8948}{Its
  outbreak untamed, Melbourne goes into even greater lockdown.}
\end{itemize}

\href{https://www.nytimes.com/2020/08/01/world/coronavirus-covid-19.html?action=click\&pgtype=Article\&state=default\&region=MAIN_CONTENT_1\&context=storylines_live_updates}{See
more updates}

More live coverage:
\href{https://www.nytimes.com/live/2020/07/31/business/stock-market-today-coronavirus?action=click\&pgtype=Article\&state=default\&region=MAIN_CONTENT_1\&context=storylines_live_updates}{Markets}

The uptick in hospitalized patients around the country reflects a
different phase of the pandemic --- a widening geographic area,
especially across the South, for the most serious illnesses compared
with what had been a relatively concentrated crisis in the spring.

Back then, nearly one in five hospitalized patients were in New York,
and the
\href{https://www.nytimes.com/2020/04/02/nyregion/coronavirus-new-york-bodies.html}{city
had to set up mobile morgues} to keep up with the rising deaths. Now,
the situation looks far different, and California, where the virus is
surging, has reported more cases than New York. In the past week, both
Florida and Texas
\href{https://www.nytimes.com/interactive/2020/us/coronavirus-us-cases.html}{have
added an average of more than 10,000 reported cases each day}, with
California not far behind.

The spike in hospitalizations for the virus has been driven in part by
people younger than 50. That group made up nearly 40 percent of the
hospitalizations as of earlier this month, compared with 26 percent in
late April,
\href{https://gis.cdc.gov/grasp/COVIDNet/COVID19_5.html}{according to
the Centers for Disease Control and Prevention}. The shift offers one
hopeful sign: Young people are less likely to die from the virus.

The majority of hospitalizations --- about 60 percent --- are now in the
South,
\href{https://covidtracking.com/data/charts/regional-current-hospitalizations}{according
to the Covid Tracking Project}, which is run by The Atlantic and
collects state hospitalization data.

\includegraphics{https://static01.nyt.com/images/2020/07/22/us/22VIRUS-HOSPITALIZATIONS-ga/merlin_174628587_009fa2c1-4880-4686-8560-b6ebf19d9800-articleLarge.jpg?quality=75\&auto=webp\&disable=upscale}

The tracking project's data is the most complete tally of
hospitalizations available and relies largely on figures reported by
state and local health officials. Still, there are gaps in available
information. Florida
\href{https://www.miamiherald.com/news/coronavirus/article243899367.html}{began
reporting daily hospitalization numbers only this month}, and the
tracking project's website does not include hospitalization figures for
Hawaii and Kansas because it says those states do not provide the data.

In some places with lots of seriously ill patients, the rise in
hospitalizations has tested the limits of local hospital systems. Some
of the most severely battered communities are in Texas, where more than
10,800 people were being treated in hospitals for the virus on
Wednesday, a record high for the state. The situation is most acute in
the Rio Grande Valley on the Texas-Mexico border, where more than a
third of families live in poverty.

``Our hospitals are maxed out. They are at 100 percent capacity,'' said
Frank Torres, the emergency management coordinator for Willacy County.
He described paramedics and patients spending hours in ambulances before
they could enter hospitals. Inside, patients are left in the halls of
chaotic emergency departments, waiting for a bed.

``It's just an overwhelming mess,'' he said.

But in a sprawling state like Texas, the crisis is not always as
apparent as it was in densely packed New York City, where sirens were
frequently heard and
\href{https://www.nytimes.com/interactive/2020/04/10/nyregion/nyc-7pm-cheer-thank-you-coronavirus.html}{people
emerged from windows and balconies each night} to support front-line
workers, sending cheers all around the city.

In San Antonio, an official with Christus Health told reporters that its
hospital system was running out of space for morgues and was preparing
to store bodies in refrigerated trucks even as ordinary life is
proceeding in other parts of the city. One of the biggest movie theaters
in the area, the Palladium, remains open for discounted \$5 showings.

\href{https://www.nytimes.com/news-event/coronavirus?action=click\&pgtype=Article\&state=default\&region=MAIN_CONTENT_3\&context=storylines_faq}{}

\hypertarget{the-coronavirus-outbreak-}{%
\subsubsection{The Coronavirus Outbreak
›}\label{the-coronavirus-outbreak-}}

\hypertarget{frequently-asked-questions}{%
\paragraph{Frequently Asked
Questions}\label{frequently-asked-questions}}

Updated July 27, 2020

\begin{itemize}
\item ~
  \hypertarget{should-i-refinance-my-mortgage}{%
  \paragraph{Should I refinance my
  mortgage?}\label{should-i-refinance-my-mortgage}}

  \begin{itemize}
  \tightlist
  \item
    \href{https://www.nytimes.com/article/coronavirus-money-unemployment.html?action=click\&pgtype=Article\&state=default\&region=MAIN_CONTENT_3\&context=storylines_faq}{It
    could be a good idea,} because mortgage rates have
    \href{https://www.nytimes.com/2020/07/16/business/mortgage-rates-below-3-percent.html?action=click\&pgtype=Article\&state=default\&region=MAIN_CONTENT_3\&context=storylines_faq}{never
    been lower.} Refinancing requests have pushed mortgage applications
    to some of the highest levels since 2008, so be prepared to get in
    line. But defaults are also up, so if you're thinking about buying a
    home, be aware that some lenders have tightened their standards.
  \end{itemize}
\item ~
  \hypertarget{what-is-school-going-to-look-like-in-september}{%
  \paragraph{What is school going to look like in
  September?}\label{what-is-school-going-to-look-like-in-september}}

  \begin{itemize}
  \tightlist
  \item
    It is unlikely that many schools will return to a normal schedule
    this fall, requiring the grind of
    \href{https://www.nytimes.com/2020/06/05/us/coronavirus-education-lost-learning.html?action=click\&pgtype=Article\&state=default\&region=MAIN_CONTENT_3\&context=storylines_faq}{online
    learning},
    \href{https://www.nytimes.com/2020/05/29/us/coronavirus-child-care-centers.html?action=click\&pgtype=Article\&state=default\&region=MAIN_CONTENT_3\&context=storylines_faq}{makeshift
    child care} and
    \href{https://www.nytimes.com/2020/06/03/business/economy/coronavirus-working-women.html?action=click\&pgtype=Article\&state=default\&region=MAIN_CONTENT_3\&context=storylines_faq}{stunted
    workdays} to continue. California's two largest public school
    districts --- Los Angeles and San Diego --- said on July 13, that
    \href{https://www.nytimes.com/2020/07/13/us/lausd-san-diego-school-reopening.html?action=click\&pgtype=Article\&state=default\&region=MAIN_CONTENT_3\&context=storylines_faq}{instruction
    will be remote-only in the fall}, citing concerns that surging
    coronavirus infections in their areas pose too dire a risk for
    students and teachers. Together, the two districts enroll some
    825,000 students. They are the largest in the country so far to
    abandon plans for even a partial physical return to classrooms when
    they reopen in August. For other districts, the solution won't be an
    all-or-nothing approach.
    \href{https://bioethics.jhu.edu/research-and-outreach/projects/eschool-initiative/school-policy-tracker/}{Many
    systems}, including the nation's largest, New York City, are
    devising
    \href{https://www.nytimes.com/2020/06/26/us/coronavirus-schools-reopen-fall.html?action=click\&pgtype=Article\&state=default\&region=MAIN_CONTENT_3\&context=storylines_faq}{hybrid
    plans} that involve spending some days in classrooms and other days
    online. There's no national policy on this yet, so check with your
    municipal school system regularly to see what is happening in your
    community.
  \end{itemize}
\item ~
  \hypertarget{is-the-coronavirus-airborne}{%
  \paragraph{Is the coronavirus
  airborne?}\label{is-the-coronavirus-airborne}}

  \begin{itemize}
  \tightlist
  \item
    The coronavirus
    \href{https://www.nytimes.com/2020/07/04/health/239-experts-with-one-big-claim-the-coronavirus-is-airborne.html?action=click\&pgtype=Article\&state=default\&region=MAIN_CONTENT_3\&context=storylines_faq}{can
    stay aloft for hours in tiny droplets in stagnant air}, infecting
    people as they inhale, mounting scientific evidence suggests. This
    risk is highest in crowded indoor spaces with poor ventilation, and
    may help explain super-spreading events reported in meatpacking
    plants, churches and restaurants.
    \href{https://www.nytimes.com/2020/07/06/health/coronavirus-airborne-aerosols.html?action=click\&pgtype=Article\&state=default\&region=MAIN_CONTENT_3\&context=storylines_faq}{It's
    unclear how often the virus is spread} via these tiny droplets, or
    aerosols, compared with larger droplets that are expelled when a
    sick person coughs or sneezes, or transmitted through contact with
    contaminated surfaces, said Linsey Marr, an aerosol expert at
    Virginia Tech. Aerosols are released even when a person without
    symptoms exhales, talks or sings, according to Dr. Marr and more
    than 200 other experts, who
    \href{https://academic.oup.com/cid/article/doi/10.1093/cid/ciaa939/5867798}{have
    outlined the evidence in an open letter to the World Health
    Organization}.
  \end{itemize}
\item ~
  \hypertarget{what-are-the-symptoms-of-coronavirus}{%
  \paragraph{What are the symptoms of
  coronavirus?}\label{what-are-the-symptoms-of-coronavirus}}

  \begin{itemize}
  \tightlist
  \item
    Common symptoms
    \href{https://www.nytimes.com/article/symptoms-coronavirus.html?action=click\&pgtype=Article\&state=default\&region=MAIN_CONTENT_3\&context=storylines_faq}{include
    fever, a dry cough, fatigue and difficulty breathing or shortness of
    breath.} Some of these symptoms overlap with those of the flu,
    making detection difficult, but runny noses and stuffy sinuses are
    less common.
    \href{https://www.nytimes.com/2020/04/27/health/coronavirus-symptoms-cdc.html?action=click\&pgtype=Article\&state=default\&region=MAIN_CONTENT_3\&context=storylines_faq}{The
    C.D.C. has also} added chills, muscle pain, sore throat, headache
    and a new loss of the sense of taste or smell as symptoms to look
    out for. Most people fall ill five to seven days after exposure, but
    symptoms may appear in as few as two days or as many as 14 days.
  \end{itemize}
\item ~
  \hypertarget{does-asymptomatic-transmission-of-covid-19-happen}{%
  \paragraph{Does asymptomatic transmission of Covid-19
  happen?}\label{does-asymptomatic-transmission-of-covid-19-happen}}

  \begin{itemize}
  \tightlist
  \item
    So far, the evidence seems to show it does. A widely cited
    \href{https://www.nature.com/articles/s41591-020-0869-5}{paper}
    published in April suggests that people are most infectious about
    two days before the onset of coronavirus symptoms and estimated that
    44 percent of new infections were a result of transmission from
    people who were not yet showing symptoms. Recently, a top expert at
    the World Health Organization stated that transmission of the
    coronavirus by people who did not have symptoms was ``very rare,''
    \href{https://www.nytimes.com/2020/06/09/world/coronavirus-updates.html?action=click\&pgtype=Article\&state=default\&region=MAIN_CONTENT_3\&context=storylines_faq\#link-1f302e21}{but
    she later walked back that statement.}
  \end{itemize}
\end{itemize}

The broadening crisis means that even hospitals in states without the
worst outbreaks are feeling the strain. In Oklahoma, hospitalizations
have been on the rise since mid-June, and about 80 to 90 percent of
hospital beds have been filled in recent days. Even so, its situation
seemed enviable compared with neighboring Texas.

``Houston hospitals are calling Oklahoma hospitals trying to place their
sick patients,'' said Dr. George Monks, president of the Oklahoma State
Medical Association. Summers are typically a slower season for
hospitals, Dr. Monks said, which has left him worried about the
well-being of health care workers and the long months ahead.

``There is just no pool of doctors and nurses available that we can hire
to come in and help,'' he said. ``You can't hire them from Texas.
They've got their hands full.''

Aubree Gordon, an associate professor of epidemiology at the University
of Michigan, said the rise in cases and hospitalizations appeared to
reflect people returning to pre-pandemic behaviors, such as going to
bars or gathering with friends. The number of hospitalizations, she
said, was sure to rise further, and she said some political leaders
seemed wary of imposing restrictions again.

``Those locations are more hesitant than before to shut down because
they feel like they shut down before and it didn't prevent them from
having the big outbreak,'' Dr. Gordon said.

In and around Miami, so many patients have packed hospitals that daily
reports routinely show wards at over 100 percent capacity --- a sign the
facilities have filled their regular beds and are operating with
additional beds set up solely to accommodate the surge.

Image

People lined up to be tested at a mobile site in Miami Beach on
Tuesday.Credit...Saul Martinez for The New York Times

``I always worry about having a scenario like we saw in New York where
the hospital system is overwhelmed,'' said Dr. Yvonne Johnson, the chief
medical officer at South Miami Hospital. ``Right now, we are handling
things.''

At Jackson Memorial Hospital in Miami, some coronavirus patients are
being treated in a large, air-conditioned tent in the hospital parking
lot until a room opens up, said Dr. Lilly Lee, the hospital's director
of emergency medicine. She said that the hospital had treated a wave of
younger patients in June, but that more older people are now coming in
with the virus.

``Over the past week, we've noticed that the patients are more 60, 70,''
she said. ``That was our concern initially: That some of these younger
people out there would cause illness in the older and more vulnerable
population.''

Contributing reporting were Manny Fernandez, Josh Keller, Patricia
Mazzei and Mitch Smith.

Advertisement

\protect\hyperlink{after-bottom}{Continue reading the main story}

\hypertarget{site-index}{%
\subsection{Site Index}\label{site-index}}

\hypertarget{site-information-navigation}{%
\subsection{Site Information
Navigation}\label{site-information-navigation}}

\begin{itemize}
\tightlist
\item
  \href{https://help.nytimes.com/hc/en-us/articles/115014792127-Copyright-notice}{©~2020~The
  New York Times Company}
\end{itemize}

\begin{itemize}
\tightlist
\item
  \href{https://www.nytco.com/}{NYTCo}
\item
  \href{https://help.nytimes.com/hc/en-us/articles/115015385887-Contact-Us}{Contact
  Us}
\item
  \href{https://www.nytco.com/careers/}{Work with us}
\item
  \href{https://nytmediakit.com/}{Advertise}
\item
  \href{http://www.tbrandstudio.com/}{T Brand Studio}
\item
  \href{https://www.nytimes.com/privacy/cookie-policy\#how-do-i-manage-trackers}{Your
  Ad Choices}
\item
  \href{https://www.nytimes.com/privacy}{Privacy}
\item
  \href{https://help.nytimes.com/hc/en-us/articles/115014893428-Terms-of-service}{Terms
  of Service}
\item
  \href{https://help.nytimes.com/hc/en-us/articles/115014893968-Terms-of-sale}{Terms
  of Sale}
\item
  \href{https://spiderbites.nytimes.com}{Site Map}
\item
  \href{https://help.nytimes.com/hc/en-us}{Help}
\item
  \href{https://www.nytimes.com/subscription?campaignId=37WXW}{Subscriptions}
\end{itemize}
