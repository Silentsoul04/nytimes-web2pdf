Sections

SEARCH

\protect\hyperlink{site-content}{Skip to
content}\protect\hyperlink{site-index}{Skip to site index}

\href{https://www.nytimes.com/section/us}{U.S.}

\href{https://myaccount.nytimes.com/auth/login?response_type=cookie\&client_id=vi}{}

\href{https://www.nytimes.com/section/todayspaper}{Today's Paper}

\href{/section/us}{U.S.}\textbar{}U.S. Northeast, Pummeled in the
Spring, Now Stands Out in Virus Control

\url{https://nyti.ms/32I8PXh}

\begin{itemize}
\item
\item
\item
\item
\item
\item
\end{itemize}

\href{https://www.nytimes.com/news-event/coronavirus?action=click\&pgtype=Article\&state=default\&region=TOP_BANNER\&context=storylines_menu}{The
Coronavirus Outbreak}

\begin{itemize}
\tightlist
\item
  live\href{https://www.nytimes.com/2020/08/01/world/coronavirus-covid-19.html?action=click\&pgtype=Article\&state=default\&region=TOP_BANNER\&context=storylines_menu}{Latest
  Updates}
\item
  \href{https://www.nytimes.com/interactive/2020/us/coronavirus-us-cases.html?action=click\&pgtype=Article\&state=default\&region=TOP_BANNER\&context=storylines_menu}{Maps
  and Cases}
\item
  \href{https://www.nytimes.com/interactive/2020/science/coronavirus-vaccine-tracker.html?action=click\&pgtype=Article\&state=default\&region=TOP_BANNER\&context=storylines_menu}{Vaccine
  Tracker}
\item
  \href{https://www.nytimes.com/interactive/2020/07/29/us/schools-reopening-coronavirus.html?action=click\&pgtype=Article\&state=default\&region=TOP_BANNER\&context=storylines_menu}{What
  School May Look Like}
\item
  \href{https://www.nytimes.com/live/2020/07/31/business/stock-market-today-coronavirus?action=click\&pgtype=Article\&state=default\&region=TOP_BANNER\&context=storylines_menu}{Economy}
\end{itemize}

Advertisement

\protect\hyperlink{after-top}{Continue reading the main story}

Supported by

\protect\hyperlink{after-sponsor}{Continue reading the main story}

\hypertarget{us-northeast-pummeled-in-the-spring-now-stands-out-in-virus-control}{%
\section{U.S. Northeast, Pummeled in the Spring, Now Stands Out in Virus
Control}\label{us-northeast-pummeled-in-the-spring-now-stands-out-in-virus-control}}

In just over two months, the Northeast has gone from the country's worst
coronavirus hot spot to its most controlled. ``It's acting like
Europe,'' one expert said.

\includegraphics{https://static01.nyt.com/images/2020/07/21/us/00newengland1/merlin_173853822_f5307ec7-7453-4eef-bf1c-125c9c58b4ab-articleLarge.jpg?quality=75\&auto=webp\&disable=upscale}

\href{https://www.nytimes.com/by/ellen-barry}{\includegraphics{https://static01.nyt.com/images/2018/10/08/multimedia/author-ellen-barry/author-ellen-barry-thumbLarge.png}}

By \href{https://www.nytimes.com/by/ellen-barry}{Ellen Barry}

\begin{itemize}
\item
  Published July 22, 2020Updated July 23, 2020
\item
  \begin{itemize}
  \item
  \item
  \item
  \item
  \item
  \item
  \end{itemize}
\end{itemize}

BOSTON --- Last week, as Dr. Emily Wroe left her home in Boston and
drove west to see her parents in Idaho, she watched as signs of the
pandemic became fewer and farther between.

After she left Ohio, customers at gas stations no longer wore masks. In
Nebraska, when she needed a repair to her truck, the mechanic seemed to
look at her strangely because she was wearing one. In Montana, there
were no masks in sight, and motorcyclists clustered in groups of 20.

The farther she was from the East Coast, the more she found Americans
treating the threat of the virus as ``faraway, and not important.''

``I wasn't surprised, but it is striking,'' said Dr. Wroe, who spent the
spring preparing contact tracers for Massachusetts. ``We're in the
middle of a global pandemic, and there are a lot of towns and businesses
along the way where nothing has changed.''

Six months since the coronavirus crisis was first detected in the United
States, the Northeast stands in sharp contrast with the rest of the
nation.

Along the East Coast, from Delaware through Maine, new case reports
remain at a low level, a small fraction of their April peak. Six of the
country's 11 states with flat or falling case levels are in that
Northeastern corridor.

New York State announced on Wednesday that just over 700 people were
hospitalized because of the virus, the fewest since mid-March and a huge
drop from a peak of over 18,000 people. Deaths have also slowed
significantly, hovering around 20 for the past six days, compared with
the nearly 800 fatalities in a single day at its peak.

``It's acting like Europe,'' Dr. Ashish Jha, the director of the Harvard
Global Health Institute, said of the Northeastern United States.

Like Europe, the Northeast suffered a devastating wave of illnesses and
deaths in March and April, and state leaders responded, after some
hesitation, with aggressive lockdowns and big investments in testing and
tracing efforts. Residents have largely followed rules and been
surprisingly supportive of tough measures, even at the cost of economic
pain.

\hypertarget{latest-updates-global-coronavirus-outbreak}{%
\section{\texorpdfstring{\href{https://www.nytimes.com/2020/08/01/world/coronavirus-covid-19.html?action=click\&pgtype=Article\&state=default\&region=MAIN_CONTENT_1\&context=storylines_live_updates}{Latest
Updates: Global Coronavirus
Outbreak}}{Latest Updates: Global Coronavirus Outbreak}}\label{latest-updates-global-coronavirus-outbreak}}

Updated 2020-08-02T07:42:09.613Z

\begin{itemize}
\tightlist
\item
  \href{https://www.nytimes.com/2020/08/01/world/coronavirus-covid-19.html?action=click\&pgtype=Article\&state=default\&region=MAIN_CONTENT_1\&context=storylines_live_updates\#link-34047410}{The
  U.S. reels as July cases more than double the total of any other
  month.}
\item
  \href{https://www.nytimes.com/2020/08/01/world/coronavirus-covid-19.html?action=click\&pgtype=Article\&state=default\&region=MAIN_CONTENT_1\&context=storylines_live_updates\#link-780ec966}{Top
  U.S. officials work to break an impasse over the federal jobless
  benefit.}
\item
  \href{https://www.nytimes.com/2020/08/01/world/coronavirus-covid-19.html?action=click\&pgtype=Article\&state=default\&region=MAIN_CONTENT_1\&context=storylines_live_updates\#link-2bc8948}{Its
  outbreak untamed, Melbourne goes into even greater lockdown.}
\end{itemize}

\href{https://www.nytimes.com/2020/08/01/world/coronavirus-covid-19.html?action=click\&pgtype=Article\&state=default\&region=MAIN_CONTENT_1\&context=storylines_live_updates}{See
more updates}

More live coverage:
\href{https://www.nytimes.com/live/2020/07/31/business/stock-market-today-coronavirus?action=click\&pgtype=Article\&state=default\&region=MAIN_CONTENT_1\&context=storylines_live_updates}{Markets}

Dr. Jha said the difference in regional trajectories was so pronounced
that, by the time flu season rolls around in the late fall, ``I would
not be surprised if what we have is two countries, one which is
neck-deep in coronavirus, its hospitals overwhelmed, and another part of
the country that is struggling a little, but largely doing OK with their
economy.''

It is also true that the Northeast remains the corner of America that
has suffered most from the virus. New Jersey, New York, Connecticut,
Massachusetts and Rhode Island have reported the country's most deaths
per capita over the course of the pandemic, with more than 61,000
combined. And the economic wounds from prolonged shutdowns are deep:
Massachusetts's unemployment rate in June
\href{https://www.wbur.org/bostonomix/2020/07/17/massachusetts-unemployment-rate-worst-country}{climbed
to 17.4 percent}, the worst in the country, according to federal data
released on Friday.

But polls, so far, suggest that voters in the Northeast are prepared to
tolerate prolonged economic pain in order to stop the spread of the
virus. Governors from the states that were hit early in the pandemic
have sustained
\href{https://covidstates.net/COVID19\%20CONSORTIUM\%20REPORT\%20APPROVAL\%20JULY\%202020.pdf}{the
highest approval ratings in the country}.

And in May, when a poll by Suffolk University Political Research Center
asked Massachusetts residents how long they could endure the hardships
of a shutdown, 38 percent of those surveyed answered ``indefinitely.''

``This isn't an economic policy, this is life or death,'' said David
Paleologos, the center's director. ``That is at the core of why people
are saying, `I'll do whatever it takes.'''

The crisis has drawn out key regional differences in how Americans view
the role of government in their lives, said Wendy J. Schiller, chair of
the political science department at Brown University in Providence, R.I.
The Northeast, she said, with its 400-year tradition of localized,
participatory government, has been less affected by decades of
antigovernment rhetoric.

\includegraphics{https://static01.nyt.com/images/2020/07/21/us/00newengland2/merlin_174119331_e0155449-b2b3-43ad-9988-ac47c6c977a0-articleLarge.jpg?quality=75\&auto=webp\&disable=upscale}

``In New England and the Northeast, it is easier to say, `Let's put on a
mask and lock down, we're all in this together, we know each other,'''
she said. ``It's this reservoir of belief that the government exists to
be good.''

Four months ago, all of the New England governors were scrambling to
contain the spread of the virus. They had hesitated to impose shutdowns
in early March, when many in the public health community were urging
immediate action, Dr. Jha said.

``It took longer than it should,'' he said.

But the responses that followed were aggressive. Charlie Baker, the
Republican governor of Massachusetts,
\href{https://www.nytimes.com/2020/04/16/us/coronavirus-massachusetts-contact-tracing.html}{decided
after a late-night phone call} with Jim Yong Kim, co-founder of the
nonprofit Partners in Health, to budget \$55 million for contact-tracing
programs that would recruit and train a corps of 1,900 newly minted
public health workers. The program was up and running within weeks.

``I certainly felt under the gun --- and I know many of my colleagues
did --- to make decisions with less than perfect information,'' Mr.
Baker said.

By this month, tracers were able to reach 90 percent of contacts within
24 hours. New cases had fallen so steeply that the corps was reduced to
500.

There was a similar scramble to acquire personal protective equipment,
which included chartering six flights from China to carry shipments of
masks. In early April, Robert K. Kraft, the owner of the New England
Patriots,
\href{https://www.bostonglobe.com/2020/04/02/nation/kraft-family-used-patriots-team-plane-shuttle-protective-masks-china-boston-wsj-reports/}{transported
a million N95 masks} from China to Boston Logan International Airport on
a team plane.

``There's all kinds of things that happened over this period of time
that were unusual decisions and risky ones, but for most of us, we felt
we were doing what we had to do,'' said Mr. Baker, whose job approval
ratings rose to 81 percent in late June, according to a
\href{https://www.suffolk.edu/news-features/news/2020/06/24/11/14/poll-4-out-of-5-bay-staters-say-police-dont-treat-blacks-equally}{Suffolk
University poll}.

In Rhode Island, Gov. Gina Raimondo, a Democrat, took a stern approach
beginning in late March, at one point
\href{https://www.nytimes.com/2020/03/28/us/coronavirus-rhode-island-checkpoint.html}{ordering
State Police to stop cars} entering the state from New York to enforce
quarantine requirements. She regularly warns that expanding freedoms
will be curtailed if residents fail to observe social distancing rules.

\href{https://www.nytimes.com/news-event/coronavirus?action=click\&pgtype=Article\&state=default\&region=MAIN_CONTENT_3\&context=storylines_faq}{}

\hypertarget{the-coronavirus-outbreak-}{%
\subsubsection{The Coronavirus Outbreak
›}\label{the-coronavirus-outbreak-}}

\hypertarget{frequently-asked-questions}{%
\paragraph{Frequently Asked
Questions}\label{frequently-asked-questions}}

Updated July 27, 2020

\begin{itemize}
\item ~
  \hypertarget{should-i-refinance-my-mortgage}{%
  \paragraph{Should I refinance my
  mortgage?}\label{should-i-refinance-my-mortgage}}

  \begin{itemize}
  \tightlist
  \item
    \href{https://www.nytimes.com/article/coronavirus-money-unemployment.html?action=click\&pgtype=Article\&state=default\&region=MAIN_CONTENT_3\&context=storylines_faq}{It
    could be a good idea,} because mortgage rates have
    \href{https://www.nytimes.com/2020/07/16/business/mortgage-rates-below-3-percent.html?action=click\&pgtype=Article\&state=default\&region=MAIN_CONTENT_3\&context=storylines_faq}{never
    been lower.} Refinancing requests have pushed mortgage applications
    to some of the highest levels since 2008, so be prepared to get in
    line. But defaults are also up, so if you're thinking about buying a
    home, be aware that some lenders have tightened their standards.
  \end{itemize}
\item ~
  \hypertarget{what-is-school-going-to-look-like-in-september}{%
  \paragraph{What is school going to look like in
  September?}\label{what-is-school-going-to-look-like-in-september}}

  \begin{itemize}
  \tightlist
  \item
    It is unlikely that many schools will return to a normal schedule
    this fall, requiring the grind of
    \href{https://www.nytimes.com/2020/06/05/us/coronavirus-education-lost-learning.html?action=click\&pgtype=Article\&state=default\&region=MAIN_CONTENT_3\&context=storylines_faq}{online
    learning},
    \href{https://www.nytimes.com/2020/05/29/us/coronavirus-child-care-centers.html?action=click\&pgtype=Article\&state=default\&region=MAIN_CONTENT_3\&context=storylines_faq}{makeshift
    child care} and
    \href{https://www.nytimes.com/2020/06/03/business/economy/coronavirus-working-women.html?action=click\&pgtype=Article\&state=default\&region=MAIN_CONTENT_3\&context=storylines_faq}{stunted
    workdays} to continue. California's two largest public school
    districts --- Los Angeles and San Diego --- said on July 13, that
    \href{https://www.nytimes.com/2020/07/13/us/lausd-san-diego-school-reopening.html?action=click\&pgtype=Article\&state=default\&region=MAIN_CONTENT_3\&context=storylines_faq}{instruction
    will be remote-only in the fall}, citing concerns that surging
    coronavirus infections in their areas pose too dire a risk for
    students and teachers. Together, the two districts enroll some
    825,000 students. They are the largest in the country so far to
    abandon plans for even a partial physical return to classrooms when
    they reopen in August. For other districts, the solution won't be an
    all-or-nothing approach.
    \href{https://bioethics.jhu.edu/research-and-outreach/projects/eschool-initiative/school-policy-tracker/}{Many
    systems}, including the nation's largest, New York City, are
    devising
    \href{https://www.nytimes.com/2020/06/26/us/coronavirus-schools-reopen-fall.html?action=click\&pgtype=Article\&state=default\&region=MAIN_CONTENT_3\&context=storylines_faq}{hybrid
    plans} that involve spending some days in classrooms and other days
    online. There's no national policy on this yet, so check with your
    municipal school system regularly to see what is happening in your
    community.
  \end{itemize}
\item ~
  \hypertarget{is-the-coronavirus-airborne}{%
  \paragraph{Is the coronavirus
  airborne?}\label{is-the-coronavirus-airborne}}

  \begin{itemize}
  \tightlist
  \item
    The coronavirus
    \href{https://www.nytimes.com/2020/07/04/health/239-experts-with-one-big-claim-the-coronavirus-is-airborne.html?action=click\&pgtype=Article\&state=default\&region=MAIN_CONTENT_3\&context=storylines_faq}{can
    stay aloft for hours in tiny droplets in stagnant air}, infecting
    people as they inhale, mounting scientific evidence suggests. This
    risk is highest in crowded indoor spaces with poor ventilation, and
    may help explain super-spreading events reported in meatpacking
    plants, churches and restaurants.
    \href{https://www.nytimes.com/2020/07/06/health/coronavirus-airborne-aerosols.html?action=click\&pgtype=Article\&state=default\&region=MAIN_CONTENT_3\&context=storylines_faq}{It's
    unclear how often the virus is spread} via these tiny droplets, or
    aerosols, compared with larger droplets that are expelled when a
    sick person coughs or sneezes, or transmitted through contact with
    contaminated surfaces, said Linsey Marr, an aerosol expert at
    Virginia Tech. Aerosols are released even when a person without
    symptoms exhales, talks or sings, according to Dr. Marr and more
    than 200 other experts, who
    \href{https://academic.oup.com/cid/article/doi/10.1093/cid/ciaa939/5867798}{have
    outlined the evidence in an open letter to the World Health
    Organization}.
  \end{itemize}
\item ~
  \hypertarget{what-are-the-symptoms-of-coronavirus}{%
  \paragraph{What are the symptoms of
  coronavirus?}\label{what-are-the-symptoms-of-coronavirus}}

  \begin{itemize}
  \tightlist
  \item
    Common symptoms
    \href{https://www.nytimes.com/article/symptoms-coronavirus.html?action=click\&pgtype=Article\&state=default\&region=MAIN_CONTENT_3\&context=storylines_faq}{include
    fever, a dry cough, fatigue and difficulty breathing or shortness of
    breath.} Some of these symptoms overlap with those of the flu,
    making detection difficult, but runny noses and stuffy sinuses are
    less common.
    \href{https://www.nytimes.com/2020/04/27/health/coronavirus-symptoms-cdc.html?action=click\&pgtype=Article\&state=default\&region=MAIN_CONTENT_3\&context=storylines_faq}{The
    C.D.C. has also} added chills, muscle pain, sore throat, headache
    and a new loss of the sense of taste or smell as symptoms to look
    out for. Most people fall ill five to seven days after exposure, but
    symptoms may appear in as few as two days or as many as 14 days.
  \end{itemize}
\item ~
  \hypertarget{does-asymptomatic-transmission-of-covid-19-happen}{%
  \paragraph{Does asymptomatic transmission of Covid-19
  happen?}\label{does-asymptomatic-transmission-of-covid-19-happen}}

  \begin{itemize}
  \tightlist
  \item
    So far, the evidence seems to show it does. A widely cited
    \href{https://www.nature.com/articles/s41591-020-0869-5}{paper}
    published in April suggests that people are most infectious about
    two days before the onset of coronavirus symptoms and estimated that
    44 percent of new infections were a result of transmission from
    people who were not yet showing symptoms. Recently, a top expert at
    the World Health Organization stated that transmission of the
    coronavirus by people who did not have symptoms was ``very rare,''
    \href{https://www.nytimes.com/2020/06/09/world/coronavirus-updates.html?action=click\&pgtype=Article\&state=default\&region=MAIN_CONTENT_3\&context=storylines_faq\#link-1f302e21}{but
    she later walked back that statement.}
  \end{itemize}
\end{itemize}

It has been a transformational political moment for Ms. Raimondo,
lifting her approval ratings to
\href{https://www.wpri.com/news/politics/poll-81-in-ri-back-gov-raimondos-handling-of-coronavirus-crisis/}{81
percent in late April}, at the height of the pandemic, from
\href{https://www.bostonglobe.com/2020/01/17/metro/poll-raimondo-is-no-longer-most-unpopular-governor-us/?s_campaign=rhodemap:newsletter}{35
percent in January}. Her signature admonition --- ``knock it off'' ---
became so popular that a gift shop in Providence
\href{https://www.bostonglobe.com/2020/03/30/metro/ri-shop-selling-t-shirts-with-raimondos-message-social-distancing-scofflaws-knock-it-off/}{printed
it on T-shirts}.

Image

Gov. Gina Raimondo of Rhode Island spoke at a news conference on the
coronavirus in April.Credit...Kayana Szymczak for The New York Times

Mr. Baker said the high level of compliance with quarantine measures was
natural, given how badly the region was battered in the spring.

``Like everybody else, I know people who have been directly affected by
this thing,'' he said. ``I've had very close friends almost die. I've
had good friends who have lost family members because of it. I went
100-odd days without seeing my father because he's 92 years old and in
an assisted living facility.''

It is not certain what the months ahead will hold for the Northeast. A
new surge of cases in the South and the West has spread across other
states, and as of this week, cases were rising in 41 states. The number
of people hospitalized for the coronavirus across the country was
nearing an earlier peak in the spring, according to the
\href{https://covidtracking.com/data/us-daily}{Covid Tracking Project}.
Among states where cases were slightly rising in recent days were Rhode
Island and Massachusetts.

In New York, the drop in cases has held even as the state has gone
through a gradual reopening process, one that began with officials
warning that they would not hesitate to restore restrictions if the
virus showed signs of returning. Nearly two months after the state began
reopening, New York has seen about 1 percent of Covid-19 tests return
positive.

Dr. Jha said he was optimistic that Northeastern states could maintain
control over the virus's spread through the summer, reopening gradually
while closely monitoring shifts in the data.

``I think they're watching what's happening in the South and they're
horrified,'' he said.

Gov. Janet Mills of Maine, a Democrat, sounded cautious. She said
officials in her state were ``exhaling, but safely, with masks.''

``The last few weeks, in particular, have felt good, but we're not out
of the woods,'' she said.

The coming months will bring new waves of difficulty as well, as the
economic impact of the spring shutdowns ripples outward, unemployment
benefits expire and an expected wave of evictions begins.

Of all the difficult decisions that she faced this year, Ms. Mills said,
none has been more ``gut-wrenching'' than her first stay-at-home order.

``Nobody wants to be the governor who puts the kibosh on graduations,
weddings, beach parties, bars,'' she said. ``Nobody wants to be the
governor the tourist industry rails against. Nobody wants to be that
governor.''

Mitch Smith contributed reporting from Chicago, and Michael Gold from
New York.

Advertisement

\protect\hyperlink{after-bottom}{Continue reading the main story}

\hypertarget{site-index}{%
\subsection{Site Index}\label{site-index}}

\hypertarget{site-information-navigation}{%
\subsection{Site Information
Navigation}\label{site-information-navigation}}

\begin{itemize}
\tightlist
\item
  \href{https://help.nytimes.com/hc/en-us/articles/115014792127-Copyright-notice}{©~2020~The
  New York Times Company}
\end{itemize}

\begin{itemize}
\tightlist
\item
  \href{https://www.nytco.com/}{NYTCo}
\item
  \href{https://help.nytimes.com/hc/en-us/articles/115015385887-Contact-Us}{Contact
  Us}
\item
  \href{https://www.nytco.com/careers/}{Work with us}
\item
  \href{https://nytmediakit.com/}{Advertise}
\item
  \href{http://www.tbrandstudio.com/}{T Brand Studio}
\item
  \href{https://www.nytimes.com/privacy/cookie-policy\#how-do-i-manage-trackers}{Your
  Ad Choices}
\item
  \href{https://www.nytimes.com/privacy}{Privacy}
\item
  \href{https://help.nytimes.com/hc/en-us/articles/115014893428-Terms-of-service}{Terms
  of Service}
\item
  \href{https://help.nytimes.com/hc/en-us/articles/115014893968-Terms-of-sale}{Terms
  of Sale}
\item
  \href{https://spiderbites.nytimes.com}{Site Map}
\item
  \href{https://help.nytimes.com/hc/en-us}{Help}
\item
  \href{https://www.nytimes.com/subscription?campaignId=37WXW}{Subscriptions}
\end{itemize}
