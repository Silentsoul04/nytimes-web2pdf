Sections

SEARCH

\protect\hyperlink{site-content}{Skip to
content}\protect\hyperlink{site-index}{Skip to site index}

\href{https://www.nytimes.com/section/politics}{Politics}

\href{https://myaccount.nytimes.com/auth/login?response_type=cookie\&client_id=vi}{}

\href{https://www.nytimes.com/section/todayspaper}{Today's Paper}

\href{/section/politics}{Politics}\textbar{}House Votes to Remove
Confederate Statues From U.S. Capitol

\url{https://nyti.ms/30Ha1ra}

\begin{itemize}
\item
\item
\item
\item
\item
\end{itemize}

\href{https://www.nytimes.com/news-event/george-floyd-protests-minneapolis-new-york-los-angeles?action=click\&pgtype=Article\&state=default\&region=TOP_BANNER\&context=storylines_menu}{Race
and America}

\begin{itemize}
\tightlist
\item
  \href{https://www.nytimes.com/2020/07/26/us/protests-portland-seattle-trump.html?action=click\&pgtype=Article\&state=default\&region=TOP_BANNER\&context=storylines_menu}{Protesters
  Return to Other Cities}
\item
  \href{https://www.nytimes.com/2020/07/24/us/portland-oregon-protests-white-race.html?action=click\&pgtype=Article\&state=default\&region=TOP_BANNER\&context=storylines_menu}{Portland
  at the Center}
\item
  \href{https://www.nytimes.com/2020/07/23/podcasts/the-daily/portland-protests.html?action=click\&pgtype=Article\&state=default\&region=TOP_BANNER\&context=storylines_menu}{Podcast:
  Showdown in Portland}
\item
  \href{https://www.nytimes.com/interactive/2020/07/16/us/black-lives-matter-protests-louisville-breonna-taylor.html?action=click\&pgtype=Article\&state=default\&region=TOP_BANNER\&context=storylines_menu}{45
  Days in Louisville}
\end{itemize}

Advertisement

\protect\hyperlink{after-top}{Continue reading the main story}

Supported by

\protect\hyperlink{after-sponsor}{Continue reading the main story}

\hypertarget{house-votes-to-remove-confederate-statues-from-us-capitol}{%
\section{House Votes to Remove Confederate Statues From U.S.
Capitol}\label{house-votes-to-remove-confederate-statues-from-us-capitol}}

The bipartisan vote to banish the statues from display was the latest
step in a nationwide push to remove historical symbols of racism and
oppression from public places.

\includegraphics{https://static01.nyt.com/images/2020/07/22/us/politics/22dc-statues-sub/merlin_174837648_f3aaaec6-6dd2-42ac-9623-0cf6e57c530e-articleLarge.jpg?quality=75\&auto=webp\&disable=upscale}

\href{https://www.nytimes.com/by/catie-edmondson}{\includegraphics{https://static01.nyt.com/images/2019/11/20/us/politics/catie-edmonson-twitter-chatblog/catie-edmonson-twitter-chatblog-thumbLarge.png}}

By \href{https://www.nytimes.com/by/catie-edmondson}{Catie Edmondson}

\begin{itemize}
\item
  July 22, 2020
\item
  \begin{itemize}
  \item
  \item
  \item
  \item
  \item
  \end{itemize}
\end{itemize}

WASHINGTON --- The House voted on Wednesday to banish from the Capitol
statues of Confederate figures and leaders who pushed white supremacist
agendas, part of a broader effort to remove historical symbols of racism
and oppression from public spaces.

The bipartisan vote, 305 to 113, came amid a national discussion about
racism and justice that has led to the
\href{https://www.nytimes.com/news-event/confederate-flags-monuments-statues}{toppling
of Confederate statues} across the country and left lawmakers
scrutinizing how their predecessors are honored in their own halls.
Speaker Nancy Pelosi last month ordered that the portraits of four
speakers who served the Confederacy be removed from the ornate hall just
outside the House chamber.

``These painful symbols of bigotry and racism --- they have no place in
our society, and certainly should not be enshrined in the United States
Capitol,'' said Representative Barbara Lee, Democrat of California and a
co-sponsor of the bill. ``It's past time that we end the glorification
of men who committed treason against the United States in a concerted
effort to keep African-Americans in chains.''

\href{https://docs.house.gov/billsthisweek/20200720/BILLS-116hr7573-SUSv2.pdf}{The
legislation}, spearheaded by Representative Steny H. Hoyer, Democrat of
Maryland and the majority leader, would mandate the removal of ``all
statues of individuals who voluntarily served'' the Confederacy. It
specifically identifies five statues for removal, including a bust of
Chief Justice Roger B. Taney, who delivered the majority Supreme Court
opinion in the landmark Dred Scott v. Sandford case, which ruled that
slaves were not American citizens and could not sue in federal court.
Mr. Hoyer's bill would replace the bust with one of Thurgood Marshall,
the first Black Supreme Court justice.

Also targeted for removal are the statues of John C. Calhoun of South
Carolina, the former vice president who led the pro-slavery faction in
the Senate; John C. Breckinridge of Kentucky, a former vice president
who served as the Confederate secretary of war and was expelled from the
Senate for joining for the Confederate Army; Charles Brantley Aycock,
the former governor of North Carolina and an architect of a violent coup
d'état in Wilmington led by white supremacists; and James Paul Clarke, a
senator and governor of Arkansas who extolled the need to ``preserve the
white standards of civilization.''

Each state is allowed to send two statues to the Capitol to be featured
in the
\href{https://www.aoc.gov/explore-capitol-campus/art/about-national-statuary-hall-collection}{National
Statuary Hall collection}, which is typically visited by thousands of
tourists every day.
\href{https://uscode.house.gov/view.xhtml?req=\%28title:2\%20section:2132\%20edition:prelim\%29}{Federal
law} gives state leaders, not members of Congress, the authority to
replace them. Because Republican lawmakers have long argued that states
should retain that right, House Democrats, even though they are in the
majority, have been unable to remove the statues.

Senator Mitch McConnell, Republican of Kentucky and the majority leader,
is unlikely to allow the bill to receive a vote in the Senate, calling
the move ``clearly a bridge too far'' and an attempt to ``airbrush the
Capitol.'' He has also contended that the decision should be left to the
states, though in 2015 he called for a statue of Jefferson Davis
displayed prominently in front of Kentucky's State Capitol to be moved
to a museum.

But in a striking display of bipartisanship, 72 Republicans voted in
favor of the measure on Wednesday, arguing that it was an important
symbolic step toward reconciliation.

``The history of this nation is so fraught with racial division, with
hatred,'' said Representative Paul Mitchell, Republican of Michigan, who
supported the bill. ``The only way to overcome that is to recognize
that, acknowledge it for what it is.''

Representative James E. Clyburn, Democrat of South Carolina and the
majority whip, suggested on Wednesday that the statues in the Capitol,
once removed, should also be put in a museum. He issued a broad warning
against the destruction of Confederate monuments.

``I do not advocate and don't want anybody tearing down any statues,''
Mr. Clyburn said. ``I want them put in their proper perspective.''

Some states, responding to local outcries over their representation in
Congress, have already moved to replace the statues they sent. Arkansas,
for example, is set to replace the statue of Mr. Clarke with a likeness
of Johnny Cash.

Democratic lawmakers have
\href{https://www.nytimes.com/2015/06/26/us/politics/search-for-confederate-symbols-finds-them-aplenty-in-washington.html}{agonized
for years} over the presence of Confederate symbols in the nation's
Capitol. During her last speakership, Ms. Pelosi moved Robert E. Lee
from
\href{http://www.aoc.gov/capitol-buildings/national-statuary-hall}{Statuary
Hall} to a more remote area of the building and placed in his stead a
statue of Rosa Parks. In the wake of a
\href{https://www.nytimes.com/2017/07/08/us/kkk-rally-charlottesville-robert-e-lee-statue.html}{2017
white nationalist rally in Charlottesville, Va.}, Ms. Pelosi, then the
minority leader, called on Speaker Paul D. Ryan to remove the statues.

But the issue never reached the House floor until now, reignited by a
sea change in public opinion around issues of race and justice amid
nationwide
\href{https://www.nytimes.com/news-event/george-floyd-protests-minneapolis-new-york-los-angeles}{protests
in honor of George Floyd}, who was killed in May during a confrontation
with Minneapolis police, and other Black Americans.

``Imagine what it feels like as an African American to know that my
ancestors built the Capitol, but yet there are monuments to the very
people that enslaved my ancestors,'' said Representative Karen Bass,
Democrat of California and the chairwoman of the Congressional Black
Caucus. ``Statues are not just historical markers but are tributes, a
way to honor an individual. These individuals do not deserve to be
honored.''

Advertisement

\protect\hyperlink{after-bottom}{Continue reading the main story}

\hypertarget{site-index}{%
\subsection{Site Index}\label{site-index}}

\hypertarget{site-information-navigation}{%
\subsection{Site Information
Navigation}\label{site-information-navigation}}

\begin{itemize}
\tightlist
\item
  \href{https://help.nytimes.com/hc/en-us/articles/115014792127-Copyright-notice}{©~2020~The
  New York Times Company}
\end{itemize}

\begin{itemize}
\tightlist
\item
  \href{https://www.nytco.com/}{NYTCo}
\item
  \href{https://help.nytimes.com/hc/en-us/articles/115015385887-Contact-Us}{Contact
  Us}
\item
  \href{https://www.nytco.com/careers/}{Work with us}
\item
  \href{https://nytmediakit.com/}{Advertise}
\item
  \href{http://www.tbrandstudio.com/}{T Brand Studio}
\item
  \href{https://www.nytimes.com/privacy/cookie-policy\#how-do-i-manage-trackers}{Your
  Ad Choices}
\item
  \href{https://www.nytimes.com/privacy}{Privacy}
\item
  \href{https://help.nytimes.com/hc/en-us/articles/115014893428-Terms-of-service}{Terms
  of Service}
\item
  \href{https://help.nytimes.com/hc/en-us/articles/115014893968-Terms-of-sale}{Terms
  of Sale}
\item
  \href{https://spiderbites.nytimes.com}{Site Map}
\item
  \href{https://help.nytimes.com/hc/en-us}{Help}
\item
  \href{https://www.nytimes.com/subscription?campaignId=37WXW}{Subscriptions}
\end{itemize}
