Sections

SEARCH

\protect\hyperlink{site-content}{Skip to
content}\protect\hyperlink{site-index}{Skip to site index}

\href{https://www.nytimes.com/section/politics}{Politics}

\href{https://myaccount.nytimes.com/auth/login?response_type=cookie\&client_id=vi}{}

\href{https://www.nytimes.com/section/todayspaper}{Today's Paper}

\href{/section/politics}{Politics}\textbar{}Pfizer Gets \$1.95 Billion
to Produce Coronavirus Vaccine by Year's End

\url{https://nyti.ms/2OOem6k}

\begin{itemize}
\item
\item
\item
\item
\item
\item
\end{itemize}

\href{https://www.nytimes.com/news-event/coronavirus?action=click\&pgtype=Article\&state=default\&region=TOP_BANNER\&context=storylines_menu}{The
Coronavirus Outbreak}

\begin{itemize}
\tightlist
\item
  live\href{https://www.nytimes.com/2020/08/02/world/coronavirus-updates.html?action=click\&pgtype=Article\&state=default\&region=TOP_BANNER\&context=storylines_menu}{Latest
  Updates}
\item
  \href{https://www.nytimes.com/interactive/2020/us/coronavirus-us-cases.html?action=click\&pgtype=Article\&state=default\&region=TOP_BANNER\&context=storylines_menu}{Maps
  and Cases}
\item
  \href{https://www.nytimes.com/interactive/2020/science/coronavirus-vaccine-tracker.html?action=click\&pgtype=Article\&state=default\&region=TOP_BANNER\&context=storylines_menu}{Vaccine
  Tracker}
\item
  \href{https://www.nytimes.com/interactive/2020/07/29/us/schools-reopening-coronavirus.html?action=click\&pgtype=Article\&state=default\&region=TOP_BANNER\&context=storylines_menu}{What
  School May Look Like}
\item
  \href{https://www.nytimes.com/live/2020/07/31/business/stock-market-today-coronavirus?action=click\&pgtype=Article\&state=default\&region=TOP_BANNER\&context=storylines_menu}{Economy}
\end{itemize}

Advertisement

\protect\hyperlink{after-top}{Continue reading the main story}

Supported by

\protect\hyperlink{after-sponsor}{Continue reading the main story}

\hypertarget{pfizer-gets-195-billion-to-produce-coronavirus-vaccine-by-years-end}{%
\section{Pfizer Gets \$1.95 Billion to Produce Coronavirus Vaccine by
Year's
End}\label{pfizer-gets-195-billion-to-produce-coronavirus-vaccine-by-years-end}}

Two pharmaceutical companies announced a nearly \$2 billion contract for
600 million doses of a vaccine, with the first 100 million promised
before the end of the year.

\includegraphics{https://static01.nyt.com/images/2020/07/22/us/politics/22dc-virus-vaccine/merlin_174113568_f0395436-1ab7-4df9-86d6-fd1b39b08578-articleLarge.jpg?quality=75\&auto=webp\&disable=upscale}

By \href{https://www.nytimes.com/by/noah-weiland}{Noah Weiland},
\href{https://www.nytimes.com/by/denise-grady}{Denise Grady} and
\href{https://www.nytimes.com/by/david-e-sanger}{David E. Sanger}

\begin{itemize}
\item
  Published July 22, 2020Updated July 30, 2020
\item
  \begin{itemize}
  \item
  \item
  \item
  \item
  \item
  \item
  \end{itemize}
\end{itemize}

WASHINGTON --- As nations around the world race to lock up
\href{https://www.nytimes.com/2020/07/27/health/moderna-vaccine-covid.html}{coronavirus
vaccines} even before they are ready, the Trump administration on
Wednesday made one of the largest investments yet, announcing a nearly
\$2 billion contract with
\href{https://www.nytimes.com/2020/07/27/health/moderna-vaccine-covid.html}{Pfizer}
and a German biotechnology company for 100 million doses by December.

The contract is part of what the White House calls the Warp Speed
project, an effort to drastically shorten the time it would take to
manufacture and distribute a working
\href{https://www.nytimes.com/2020/07/22/upshot/vaccine-coronavirus-government-purchase.html}{vaccine}.
So far, the United States has put money into more than a half dozen
efforts, hoping to build manufacturing ability for an eventual
breakthrough.

Europe has a parallel effort underway. Germany recently took a 23
percent stake in a German firm, CureVac, that
\href{https://www.nytimes.com/2020/03/15/world/europe/cornonavirus-vaccine-us-germany.html}{President
Trump once tried to lure} to American shores in hopes that its vaccine,
if successful, would be distributed in the United States first. A
\href{https://www.nytimes.com/2020/05/04/world/europe/eu-coronavirus-vaccine.html}{European-led
fund-raising effort in May} brought \$8 billion in pledges from the
world's governments, philanthropists and leaders for coronavirus vaccine
research, even with the United States sitting out the conference.

China, meantime, has militarized the effort: Researchers associated with
the Academy of Military Medical Sciences have developed one of China's
leading vaccine candidates, and another Chinese firm, Sinopharm Group,
announced in June that it was beginning Phase 3 trials in the United
Arab Emirates.

The Pfizer contract, an agreement to ensure the pharmaceutical giant has
a market for its work, is the biggest splash yet by the Americans. No
vaccine has yet been developed, and it is not clear whether the Pfizer
version will work. But if the vaccine being produced by Pfizer and
BioNTech, the German firm, proves to be safe and effective in clinical
trials, the companies say they could manufacture those first 100 million
doses by the end of the year.

Under
\href{https://www.businesswire.com/news/home/20200722005438/en/Pfizer-BioNTech-Announce-Agreement-U.S.-Government-600}{the
arrangement}, the federal government would obtain that first batch for
\$1.95 billion, or about \$20 a dose, with the rights to acquire up to
500 million more, or 600 million total. Americans would receive the
vaccine for free. Before it could be distributed, it would need
emergency approval by the Food and Drug Administration. But the U.S.
government does not pay the nearly \$2 billion until the drug is
approved and the first 100 million doses are delivered.

Pfizer said that large-scale safety and efficacy trials were to begin
this month, with regulatory review set for as early as October, although
nothing was guaranteed.

``Depending on success in clinical trials, today's agreement will enable
the delivery of approximately 100 million doses of vaccine being
developed by Pfizer and BioNTech,'' Alex M. Azar II, the health
secretary, said in a statement announcing the deal.

On Monday, Pfizer and
\href{https://www.nytimes.com/2020/05/21/health/coronavirus-vaccine-astrazeneca.html}{AstraZeneca},
a British-Swedish drug company developing a potential
\href{https://www.nytimes.com/interactive/2020/06/09/magazine/covid-vaccine.html}{vaccine}
with Oxford University,
\href{https://www.nytimes.com/2020/07/20/world/covid-coronavirus-vaccine.html}{released
data} suggesting that their vaccines could stimulate strong immune
responses with only minor side effects.

But unlike AstraZeneca, which has also obtained funding from the U.S.
government, Pfizer did not receive a contract for its earlier research
and development efforts ---~only for the doses and their distribution.

\hypertarget{latest-updates-global-coronavirus-outbreak}{%
\section{\texorpdfstring{\href{https://www.nytimes.com/2020/08/01/world/coronavirus-covid-19.html?action=click\&pgtype=Article\&state=default\&region=MAIN_CONTENT_1\&context=storylines_live_updates}{Latest
Updates: Global Coronavirus
Outbreak}}{Latest Updates: Global Coronavirus Outbreak}}\label{latest-updates-global-coronavirus-outbreak}}

Updated 2020-08-02T17:52:35.962Z

\begin{itemize}
\tightlist
\item
  \href{https://www.nytimes.com/2020/08/01/world/coronavirus-covid-19.html?action=click\&pgtype=Article\&state=default\&region=MAIN_CONTENT_1\&context=storylines_live_updates\#link-34047410}{The
  U.S. reels as July cases more than double the total of any other
  month.}
\item
  \href{https://www.nytimes.com/2020/08/01/world/coronavirus-covid-19.html?action=click\&pgtype=Article\&state=default\&region=MAIN_CONTENT_1\&context=storylines_live_updates\#link-780ec966}{Top
  U.S. officials work to break an impasse over the federal jobless
  benefit.}
\item
  \href{https://www.nytimes.com/2020/08/01/world/coronavirus-covid-19.html?action=click\&pgtype=Article\&state=default\&region=MAIN_CONTENT_1\&context=storylines_live_updates\#link-2bc8948}{Its
  outbreak untamed, Melbourne goes into even greater lockdown.}
\end{itemize}

\href{https://www.nytimes.com/2020/08/01/world/coronavirus-covid-19.html?action=click\&pgtype=Article\&state=default\&region=MAIN_CONTENT_1\&context=storylines_live_updates}{See
more updates}

More live coverage:
\href{https://www.nytimes.com/live/2020/07/31/business/stock-market-today-coronavirus?action=click\&pgtype=Article\&state=default\&region=MAIN_CONTENT_1\&context=storylines_live_updates}{Markets}

By refusing funding up until now, Pfizer was able to avoid drawn-out
contractual negotiations and get its vaccine to trials, company
officials say.

``We didn't accept the federal government funding solely for the reason
that we wanted to be able to move as quickly as possible with our
vaccine candidate into the clinic,'' John Young, Pfizer's chief business
officer, said on Tuesday at
\href{https://www.nytimes.com/2020/07/21/health/covid-19-vaccine-coronavirus-moderna-pfizer.html}{a
congressional hearing} with executives from five vaccine manufacturers.

Pfizer and BioNTech are developing a
\href{https://www.nytimes.com/2020/05/15/us/politics/coronavirus-vaccine-timeline.html}{vaccine}
candidate that uses genetic material from the virus,
\href{https://www.nytimes.com/2020/05/05/health/pfizer-vaccine-coronavirus.html}{known
as messenger RNA}, to stimulate the immune system without making the
recipient sick. The technology can create a vaccine quickly, but has not
yet produced one that has been approved and marketed.

\href{https://www.nytimes.com/2020/07/27/health/moderna-vaccine-covid.html}{Moderna},
a Massachusetts biotech company, received \$483 million from the U.S.
government for its vaccine development and is also using mRNA
technology. By putting the might of an industry giant behind it, Pfizer
is making the technology mainstream.

The lack of a track record has prompted some skepticism about this
approach, but Dr. Kathrin Jansen, a senior vice president and the head
of vaccine research and development at Pfizer, dismissed the criticism.

``That's not a scientific mind-set --- that just because it's new, it
will fail,'' she said in an interview.

Earlier in her career, Dr. Jansen worked for Merck, where she led its
development of a vaccine to prevent cervical cancer, which is caused by
a virus. The vaccine, Gardasil, has been successful. It, too, used a
technology that was new at the time and faced considerable skepticism.

Dr. Jansen said Pfizer had placed its bet on messenger RNA not just
because the technology could produce a vaccine quickly, but also because
its review of previous work by BioNTech on experimental cancer vaccines
suggested the approach could cause a powerful immune response. Before
the coronavirus pandemic, the two companies had been collaborating on
flu vaccines.

Vaccines using mRNA consist of genetic material from part of the virus,
encased in tiny particles made of fat that help it get into human cells.
The messenger RNA then prompts the cells to churn out a tiny piece of
the virus, causing the immune system to attack the real virus if the
person is exposed. In essence, the patient's cells become factories for
a harmless fragment of the virus.

These vaccines set off several different kinds of immune responses, Dr.
Jansen said, which is important because scientists do not know yet which
type will be most potent against the coronavirus.

Dr. Jansen described making such a vaccine as a clean, fast process that
required a relatively small footprint to produce many doses.

\href{https://www.nytimes.com/news-event/coronavirus?action=click\&pgtype=Article\&state=default\&region=MAIN_CONTENT_3\&context=storylines_faq}{}

\hypertarget{the-coronavirus-outbreak-}{%
\subsubsection{The Coronavirus Outbreak
›}\label{the-coronavirus-outbreak-}}

\hypertarget{frequently-asked-questions}{%
\paragraph{Frequently Asked
Questions}\label{frequently-asked-questions}}

Updated July 27, 2020

\begin{itemize}
\item ~
  \hypertarget{should-i-refinance-my-mortgage}{%
  \paragraph{Should I refinance my
  mortgage?}\label{should-i-refinance-my-mortgage}}

  \begin{itemize}
  \tightlist
  \item
    \href{https://www.nytimes.com/article/coronavirus-money-unemployment.html?action=click\&pgtype=Article\&state=default\&region=MAIN_CONTENT_3\&context=storylines_faq}{It
    could be a good idea,} because mortgage rates have
    \href{https://www.nytimes.com/2020/07/16/business/mortgage-rates-below-3-percent.html?action=click\&pgtype=Article\&state=default\&region=MAIN_CONTENT_3\&context=storylines_faq}{never
    been lower.} Refinancing requests have pushed mortgage applications
    to some of the highest levels since 2008, so be prepared to get in
    line. But defaults are also up, so if you're thinking about buying a
    home, be aware that some lenders have tightened their standards.
  \end{itemize}
\item ~
  \hypertarget{what-is-school-going-to-look-like-in-september}{%
  \paragraph{What is school going to look like in
  September?}\label{what-is-school-going-to-look-like-in-september}}

  \begin{itemize}
  \tightlist
  \item
    It is unlikely that many schools will return to a normal schedule
    this fall, requiring the grind of
    \href{https://www.nytimes.com/2020/06/05/us/coronavirus-education-lost-learning.html?action=click\&pgtype=Article\&state=default\&region=MAIN_CONTENT_3\&context=storylines_faq}{online
    learning},
    \href{https://www.nytimes.com/2020/05/29/us/coronavirus-child-care-centers.html?action=click\&pgtype=Article\&state=default\&region=MAIN_CONTENT_3\&context=storylines_faq}{makeshift
    child care} and
    \href{https://www.nytimes.com/2020/06/03/business/economy/coronavirus-working-women.html?action=click\&pgtype=Article\&state=default\&region=MAIN_CONTENT_3\&context=storylines_faq}{stunted
    workdays} to continue. California's two largest public school
    districts --- Los Angeles and San Diego --- said on July 13, that
    \href{https://www.nytimes.com/2020/07/13/us/lausd-san-diego-school-reopening.html?action=click\&pgtype=Article\&state=default\&region=MAIN_CONTENT_3\&context=storylines_faq}{instruction
    will be remote-only in the fall}, citing concerns that surging
    coronavirus infections in their areas pose too dire a risk for
    students and teachers. Together, the two districts enroll some
    825,000 students. They are the largest in the country so far to
    abandon plans for even a partial physical return to classrooms when
    they reopen in August. For other districts, the solution won't be an
    all-or-nothing approach.
    \href{https://bioethics.jhu.edu/research-and-outreach/projects/eschool-initiative/school-policy-tracker/}{Many
    systems}, including the nation's largest, New York City, are
    devising
    \href{https://www.nytimes.com/2020/06/26/us/coronavirus-schools-reopen-fall.html?action=click\&pgtype=Article\&state=default\&region=MAIN_CONTENT_3\&context=storylines_faq}{hybrid
    plans} that involve spending some days in classrooms and other days
    online. There's no national policy on this yet, so check with your
    municipal school system regularly to see what is happening in your
    community.
  \end{itemize}
\item ~
  \hypertarget{is-the-coronavirus-airborne}{%
  \paragraph{Is the coronavirus
  airborne?}\label{is-the-coronavirus-airborne}}

  \begin{itemize}
  \tightlist
  \item
    The coronavirus
    \href{https://www.nytimes.com/2020/07/04/health/239-experts-with-one-big-claim-the-coronavirus-is-airborne.html?action=click\&pgtype=Article\&state=default\&region=MAIN_CONTENT_3\&context=storylines_faq}{can
    stay aloft for hours in tiny droplets in stagnant air}, infecting
    people as they inhale, mounting scientific evidence suggests. This
    risk is highest in crowded indoor spaces with poor ventilation, and
    may help explain super-spreading events reported in meatpacking
    plants, churches and restaurants.
    \href{https://www.nytimes.com/2020/07/06/health/coronavirus-airborne-aerosols.html?action=click\&pgtype=Article\&state=default\&region=MAIN_CONTENT_3\&context=storylines_faq}{It's
    unclear how often the virus is spread} via these tiny droplets, or
    aerosols, compared with larger droplets that are expelled when a
    sick person coughs or sneezes, or transmitted through contact with
    contaminated surfaces, said Linsey Marr, an aerosol expert at
    Virginia Tech. Aerosols are released even when a person without
    symptoms exhales, talks or sings, according to Dr. Marr and more
    than 200 other experts, who
    \href{https://academic.oup.com/cid/article/doi/10.1093/cid/ciaa939/5867798}{have
    outlined the evidence in an open letter to the World Health
    Organization}.
  \end{itemize}
\item ~
  \hypertarget{what-are-the-symptoms-of-coronavirus}{%
  \paragraph{What are the symptoms of
  coronavirus?}\label{what-are-the-symptoms-of-coronavirus}}

  \begin{itemize}
  \tightlist
  \item
    Common symptoms
    \href{https://www.nytimes.com/article/symptoms-coronavirus.html?action=click\&pgtype=Article\&state=default\&region=MAIN_CONTENT_3\&context=storylines_faq}{include
    fever, a dry cough, fatigue and difficulty breathing or shortness of
    breath.} Some of these symptoms overlap with those of the flu,
    making detection difficult, but runny noses and stuffy sinuses are
    less common.
    \href{https://www.nytimes.com/2020/04/27/health/coronavirus-symptoms-cdc.html?action=click\&pgtype=Article\&state=default\&region=MAIN_CONTENT_3\&context=storylines_faq}{The
    C.D.C. has also} added chills, muscle pain, sore throat, headache
    and a new loss of the sense of taste or smell as symptoms to look
    out for. Most people fall ill five to seven days after exposure, but
    symptoms may appear in as few as two days or as many as 14 days.
  \end{itemize}
\item ~
  \hypertarget{does-asymptomatic-transmission-of-covid-19-happen}{%
  \paragraph{Does asymptomatic transmission of Covid-19
  happen?}\label{does-asymptomatic-transmission-of-covid-19-happen}}

  \begin{itemize}
  \tightlist
  \item
    So far, the evidence seems to show it does. A widely cited
    \href{https://www.nature.com/articles/s41591-020-0869-5}{paper}
    published in April suggests that people are most infectious about
    two days before the onset of coronavirus symptoms and estimated that
    44 percent of new infections were a result of transmission from
    people who were not yet showing symptoms. Recently, a top expert at
    the World Health Organization stated that transmission of the
    coronavirus by people who did not have symptoms was ``very rare,''
    \href{https://www.nytimes.com/2020/06/09/world/coronavirus-updates.html?action=click\&pgtype=Article\&state=default\&region=MAIN_CONTENT_3\&context=storylines_faq\#link-1f302e21}{but
    she later walked back that statement.}
  \end{itemize}
\end{itemize}

She added that it ``has the potential to be fast to produce a product
that is very well defined and very pure.''

Several other companies are also making such vaccines, and each has its
own formulation of the genetic material and types of fat used to encase
it.

The large vaccine studies set to begin this month will each include
30,000 people, with some getting placebo shots. The
\href{https://www.fda.gov/regulatory-information/search-fda-guidance-documents/development-and-licensure-vaccines-prevent-covid-19}{Food
and Drug Administration has said} that to be considered effective, a
coronavirus vaccine should protect 50 percent of the people who receive
it.

Companies hope to show proof of effectiveness by the fall, but that will
depend on enrolling enough volunteers in areas where the infection rate
is high enough to see a significant difference between the vaccinated
people and the placebo group.

``We think we will see the end points, given that the infection rates
are going up, up, up,'' Dr. Jansen said. ``If the stars are aligned, it
could be next fall. But everything has to be right.''

Dr. Amesh Adalja, an infectious disease physician and senior scholar at
the Johns Hopkins University Center for Health Security, said that
Pfizer, unlike some smaller pharmaceutical companies that the government
had contracted with, did not need research money because it was likely
to have the infrastructure and early data it needed to speed its vaccine
to trials without federal assistance.

``Pfizer is a company that has a lot of expertise in making vaccines,''
he said. ``They knew that any negotiation with the government could have
delayed the start'' of trials, which he said the company knew how to set
up rapidly.

He added that the \$1.95 billion agreement was a way to guarantee a
market for the vaccine at the end of production, since prominent
drugmakers have historically been hesitant to spend on infectious
disease outbreaks.

``Advance purchase agreements have been one way we've been able to
acquire vaccines and countermeasures against certain threats that
pharmaceutical companies have traditionally stayed away from,'' he said.

The agreement with Pfizer, which the company and the Department of
Health and Human Services announced Wednesday morning, is the largest
one yet for Operation Warp Speed. The federal government announced this
month that it would pay the Maryland-based company
\href{https://www.nytimes.com/2020/07/16/health/coronavirus-vaccine-novavax.html}{Novavax}
\$1.6 billion to expedite the development of a
\href{https://www.nytimes.com/interactive/2020/science/coronavirus-vaccine-tracker.html}{coronavirus
vaccine}.

``We've been committed to making the impossible possible by working
tirelessly to develop and produce in record time a safe and effective
vaccine to help bring an end to this global health crisis,'' Dr. Albert
Bourla, Pfizer's chairman and chief executive officer, said in a news
release.

Advertisement

\protect\hyperlink{after-bottom}{Continue reading the main story}

\hypertarget{site-index}{%
\subsection{Site Index}\label{site-index}}

\hypertarget{site-information-navigation}{%
\subsection{Site Information
Navigation}\label{site-information-navigation}}

\begin{itemize}
\tightlist
\item
  \href{https://help.nytimes.com/hc/en-us/articles/115014792127-Copyright-notice}{©~2020~The
  New York Times Company}
\end{itemize}

\begin{itemize}
\tightlist
\item
  \href{https://www.nytco.com/}{NYTCo}
\item
  \href{https://help.nytimes.com/hc/en-us/articles/115015385887-Contact-Us}{Contact
  Us}
\item
  \href{https://www.nytco.com/careers/}{Work with us}
\item
  \href{https://nytmediakit.com/}{Advertise}
\item
  \href{http://www.tbrandstudio.com/}{T Brand Studio}
\item
  \href{https://www.nytimes.com/privacy/cookie-policy\#how-do-i-manage-trackers}{Your
  Ad Choices}
\item
  \href{https://www.nytimes.com/privacy}{Privacy}
\item
  \href{https://help.nytimes.com/hc/en-us/articles/115014893428-Terms-of-service}{Terms
  of Service}
\item
  \href{https://help.nytimes.com/hc/en-us/articles/115014893968-Terms-of-sale}{Terms
  of Sale}
\item
  \href{https://spiderbites.nytimes.com}{Site Map}
\item
  \href{https://help.nytimes.com/hc/en-us}{Help}
\item
  \href{https://www.nytimes.com/subscription?campaignId=37WXW}{Subscriptions}
\end{itemize}
