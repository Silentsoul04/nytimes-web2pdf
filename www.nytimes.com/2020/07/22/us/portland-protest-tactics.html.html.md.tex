Sections

SEARCH

\protect\hyperlink{site-content}{Skip to
content}\protect\hyperlink{site-index}{Skip to site index}

\href{/section/us}{U.S.}\textbar{}To Battle a Militarized Foe, Portland
Protesters Use Umbrellas, Pool Noodles and Fire

\url{https://nyti.ms/3jt08pN}

\begin{itemize}
\item
\item
\item
\item
\item
\item
\end{itemize}

\href{https://www.nytimes.com/news-event/george-floyd-protests-minneapolis-new-york-los-angeles?action=click\&pgtype=Article\&state=default\&region=TOP_BANNER\&context=storylines_menu}{Race
and America}

\begin{itemize}
\tightlist
\item
  \href{https://www.nytimes.com/2020/07/26/us/protests-portland-seattle-trump.html?action=click\&pgtype=Article\&state=default\&region=TOP_BANNER\&context=storylines_menu}{Protesters
  Return to Other Cities}
\item
  \href{https://www.nytimes.com/2020/07/24/us/portland-oregon-protests-white-race.html?action=click\&pgtype=Article\&state=default\&region=TOP_BANNER\&context=storylines_menu}{Portland
  at the Center}
\item
  \href{https://www.nytimes.com/2020/07/23/podcasts/the-daily/portland-protests.html?action=click\&pgtype=Article\&state=default\&region=TOP_BANNER\&context=storylines_menu}{Podcast:
  Showdown in Portland}
\item
  \href{https://www.nytimes.com/interactive/2020/07/16/us/black-lives-matter-protests-louisville-breonna-taylor.html?action=click\&pgtype=Article\&state=default\&region=TOP_BANNER\&context=storylines_menu}{45
  Days in Louisville}
\end{itemize}

\includegraphics{https://static01.nyt.com/images/2020/07/22/us/22portland-tactics02/merlin_174824010_af4dff3f-946f-4425-8e7e-cd15432c80bb-articleLarge.jpg?quality=75\&auto=webp\&disable=upscale}

\hypertarget{to-battle-a-militarized-foe-portland-protesters-use-umbrellas-pool-noodles-and-fire}{%
\section{To Battle a Militarized Foe, Portland Protesters Use Umbrellas,
Pool Noodles and
Fire}\label{to-battle-a-militarized-foe-portland-protesters-use-umbrellas-pool-noodles-and-fire}}

With no clear leaders or blueprints, demonstrators have scrounged for
items from home and largely embraced a strategy of spontaneous
consensus.

Federal forces and protesters clashed near the federal courthouse in
Portland, Ore., early Wednesday.Credit...Mason Trinca for The New York
Times

Supported by

\protect\hyperlink{after-sponsor}{Continue reading the main story}

\href{https://www.nytimes.com/by/mike-baker}{\includegraphics{https://static01.nyt.com/images/2020/05/19/reader-center/author-mike-baker/author-mike-baker-thumbLarge.png}}\href{https://www.nytimes.com/by/thomas-fuller}{\includegraphics{https://static01.nyt.com/images/2018/06/12/multimedia/author-thomas-fuller/author-thomas-fuller-thumbLarge.png}}

By \href{https://www.nytimes.com/by/mike-baker}{Mike Baker} and
\href{https://www.nytimes.com/by/thomas-fuller}{Thomas Fuller}

\begin{itemize}
\item
  Published July 22, 2020Updated July 24, 2020
\item
  \begin{itemize}
  \item
  \item
  \item
  \item
  \item
  \item
  \end{itemize}
\end{itemize}

PORTLAND, Ore. --- Shields were made of pool noodles, umbrellas and
sleds. The body armor was pieced together with bicycle helmets and
football pads. The weapons included water bottles and cigarette
lighters.

Facing federal forces who came to
\href{https://www.nytimes.com/article/portland-protests-explained-protesters.html}{Portland}
to subdue them, many of the city's
\href{https://www.nytimes.com/interactive/2020/07/22/us/portland-protests.html}{protesters}
have taken to the streets this week with items scrounged from home. Then
they have assembled at the federal courthouse each night with sometimes
starkly different visions of how to put their tools to use.

In 55 consecutive
\href{https://www.nytimes.com/2020/07/23/upshot/trump-portland.html}{nights
of protest in Portland}, no two have been alike. The protests began on
May 29, after the killing of
\href{https://www.nytimes.com/article/george-floyd-who-is.html}{George
Floyd} in police custody in Minneapolis. They have continued ever since,
night after night, and they show no signs of letting up.

With no clear leaders or blueprints, the demonstrators have largely
embraced a strategy of spontaneous consensus, which follows the whims of
the crowd and, sometimes, the resolutions of their internal
disagreements --- such as whether fire is an appropriate addition to
their protest.

``It's really organic and non-centralized,'' said Luke Meyer, who walked
through the streets overnight with a plywood shield. ``You almost vote
with your actions.''

For weeks, city officials who have been the target of much of the ire
have been unable to find a way to bring the demonstrations to an end.
Now those protesters are giving grief to federal agents who were
assigned to maintain calm in the city but have instead watched the
number of protesters outside the federal courthouse swell into the
thousands.

\includegraphics{https://static01.nyt.com/images/2020/07/22/us/22portland-tactics01/merlin_174824037_756322f0-b4f0-4faf-b923-092682898cf0-videoSixteenByNine3000.jpg}

``Whose streets?'' they shout, in one of their signature chants. ``Our
streets. Whose lives matter? Black lives matter.''

The protests have drawn diverse crowds, including teenagers,
grandmothers, longtime activists and those joining the fray for the
first time.

In the early morning hours on Wednesday, protesters used a range of
tactics. While some spent their time chanting, dancing or locking arms,
others slowly chipped away at the temporary wooden facade of the
courthouse building, eventually pulling away a whole panel to expose an
entryway. Some used lasers or bright lights to shine at holes from which
federal officers sometimes fire projectiles into the crowd. Others
dragged in barricades to cover the building's entrances.

\includegraphics{https://static01.nyt.com/images/2020/07/22/us/22portland-tactics03/22portland-tactics03-articleLarge.jpg?quality=75\&auto=webp\&disable=upscale}

At one point during the protests that raged from late Tuesday night into
Wednesday, federal agents rushed out of the courthouse and shoved a
group of mothers wearing matching yellow shirts as tear gas billowed
around them. In another moment, the agents knocked one protester down
onto a pole. After clearing a park, the agents rummaged through a
makeshift kitchen that has been used to feed the protesters, knocking
food and supplies to the ground.

Facing volleys of tear gas that left many coughing, the protesters
retreated up Main Street. But they soon regrouped and returned as the
authorities backed off. In an echo of the
\href{https://www.nytimes.com/interactive/2014/10/01/world/asia/hong-kong-protest-photos.html}{``umbrella
revolution'' in Hong Kong}, protesters with shields and umbrellas took
the front of the line to protect themselves and others from the weaponry
of federal forces in tactical gear.

The authorities have accused protesters of at times using dangerous
methods, including launching projectiles from sling shots. In one of
dozens of arrests during the demonstrations, a protester was charged
recently with hitting an officer with a hammer.

In court documents, the federal government has said protesters have
committed ``vandalism, destruction of property, looting, arson and
assault.'' A Molotov cocktail was thrown on May 28, according to court
papers, and security cameras on federal buildings have been vandalized.
One protester was accused of pushing a glass door on July 2, causing the
glass to break and injure a deputy marshal.

The federal efforts to confront the demonstrators in recent days were
often preceded by attempts by some protesters to light fires on the
facade of the federal courthouse, which has been boarded up with wood
that has been painted and targeted by protesters.

Image

Protesters and federal agents faced off outside the
courthouse.Credit...Mason Trinca for The New York Times

On Wednesday morning, someone started trying to light a portion of the
facade but was blocked by another person. Later, a debate developed in
the crowd over whether to allow the fire, with one protester arguing
that being peaceful was the better route.

Megan Smith was among those who opposed the use of fire, saying she
douses any flames she encounters.

``I would prefer fires to be controlled,'' Ms. Smith said. ``I get it
that it's how you get your anger out, but it makes it dangerous for
everybody else.''

In part, she said, starting a fire raises the risks for everyone because
it is clear that it will prompt federal officers to come out to confront
the crowds. And on Wednesday, it did just that.

The federal authorities have seized on the lighting of fires as further
justification for their presence in Portland.

``Individuals trying to set fire to a building are no longer
protesting,'' Chad F. Wolf, the acting secretary of homeland security,
said at a news conference on Tuesday. ``They are criminals.''

Early on Tuesday morning protesters had lit bonfires in the streets
around the federal courthouse and tried to ignite the plywood protecting
the building. There were fewer fires at the protests early on Wednesday
morning.

``If we left tomorrow, they would burn that building down,'' said Mr.
Wolf, who added that the agency was compiling video evidence of
protesters lighting fires.

Image

Protesters are giving grief to federal agents who were assigned to
maintain calm in the city but have instead watched the number of
protesters outside the federal courthouse swell into the
thousands.Credit...Mason Trinca for The New York Times

Yet Mr. Wolf's insistence that federal agents stay to protect federal
property --- both the mayor of Portland and the governor of Oregon have
asked that the federal agents leave the state --- is part of the dynamic
that is causing the protests to swell into the thousands.

Among the protesters on Tuesday night was Ronda Jordan, 63, who works as
a quality control officer for a company that makes granola bars. None of
her friends felt bold enough to attend the demonstrations, so she came
alone.

``Frankly, I was afraid to come down here, too --- the feds intimidate
me,'' she said, clutching a sunflower as a symbol of peace. ``But the
more they come at us with federal officers, the more people are going to
come out here.''

In recent days ``Feds go home!'' has become a rallying cry for
protesters. Laurie Vandenburgh, a school counselor, and her daughter
Emily Vanlaningham, a social worker, said their presence was one of the
main reasons they joined the protests on Tuesday night.

``I'm personally upset that the feds are here,'' Ms. Vandenburgh said.
``Our city is fine without them.''

Sergio Olmos contributed reporting.

Advertisement

\protect\hyperlink{after-bottom}{Continue reading the main story}

\hypertarget{site-index}{%
\subsection{Site Index}\label{site-index}}

\hypertarget{site-information-navigation}{%
\subsection{Site Information
Navigation}\label{site-information-navigation}}

\begin{itemize}
\tightlist
\item
  \href{https://help.nytimes.com/hc/en-us/articles/115014792127-Copyright-notice}{©~2020~The
  New York Times Company}
\end{itemize}

\begin{itemize}
\tightlist
\item
  \href{https://www.nytco.com/}{NYTCo}
\item
  \href{https://help.nytimes.com/hc/en-us/articles/115015385887-Contact-Us}{Contact
  Us}
\item
  \href{https://www.nytco.com/careers/}{Work with us}
\item
  \href{https://nytmediakit.com/}{Advertise}
\item
  \href{http://www.tbrandstudio.com/}{T Brand Studio}
\item
  \href{https://www.nytimes.com/privacy/cookie-policy\#how-do-i-manage-trackers}{Your
  Ad Choices}
\item
  \href{https://www.nytimes.com/privacy}{Privacy}
\item
  \href{https://help.nytimes.com/hc/en-us/articles/115014893428-Terms-of-service}{Terms
  of Service}
\item
  \href{https://help.nytimes.com/hc/en-us/articles/115014893968-Terms-of-sale}{Terms
  of Sale}
\item
  \href{https://spiderbites.nytimes.com}{Site Map}
\item
  \href{https://help.nytimes.com/hc/en-us}{Help}
\item
  \href{https://www.nytimes.com/subscription?campaignId=37WXW}{Subscriptions}
\end{itemize}
