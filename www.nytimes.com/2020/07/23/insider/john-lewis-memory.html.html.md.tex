Sections

SEARCH

\protect\hyperlink{site-content}{Skip to
content}\protect\hyperlink{site-index}{Skip to site index}

\href{https://www.nytimes.com/section/reader-center}{Times Insider}

\href{https://myaccount.nytimes.com/auth/login?response_type=cookie\&client_id=vi}{}

\href{https://www.nytimes.com/section/todayspaper}{Today's Paper}

\href{/section/reader-center}{Times Insider}\textbar{}The Blessing and
Burden of Being John Lewis

\url{https://nyti.ms/32KneCa}

\begin{itemize}
\item
\item
\item
\item
\item
\item
\end{itemize}

Advertisement

\protect\hyperlink{after-top}{Continue reading the main story}

Supported by

\protect\hyperlink{after-sponsor}{Continue reading the main story}

Times Insider

\hypertarget{the-blessing-and-burden-of-being-john-lewis}{%
\section{The Blessing and Burden of Being John
Lewis}\label{the-blessing-and-burden-of-being-john-lewis}}

During a reporting assignment in 2013, I received a rare glimpse of both
the legend and the man. Neither one disappointed.

\includegraphics{https://static01.nyt.com/images/2020/07/23/us/politics/23insider/video-johnlewis-videoSixteenByNineJumbo1600-v2.jpg}

\href{https://www.nytimes.com/by/sheryl-gay-stolberg}{\includegraphics{https://static01.nyt.com/images/2018/11/26/multimedia/author-sheryl-gay-stolberg/author-sheryl-gay-stolberg-thumbLarge.png}}

By \href{https://www.nytimes.com/by/sheryl-gay-stolberg}{Sheryl Gay
Stolberg}

\begin{itemize}
\item
  July 23, 2020
\item
  \begin{itemize}
  \item
  \item
  \item
  \item
  \item
  \item
  \end{itemize}
\end{itemize}

\href{https://www.nytimes.com/series/times-insider}{\emph{Times
Insider}} \emph{explains who we are and what we do, and delivers
behind-the-scenes insights into how our journalism comes together.}

The Supreme Court had just
\href{https://www.nytimes.com/2013/06/26/us/supreme-court-ruling.html}{eviscerated
the Voting Rights Act}, and the 50th anniversary of the 1963 March on
Washington was approaching when I traveled to Philadelphia to meet with
\href{https://www.nytimes.com/2013/08/14/us/politics/50-years-later-fighting-the-same-civil-rights-battle.html}{Representative
John Lewis} in the summer of 2013. A singular moment sticks out in my
mind: a woman who wept at the sight of him.

We were riding down an escalator in the city's packed convention center,
where Mr. Lewis was to give a speech to the Urban League. The woman was
riding up. She spotted Mr. Lewis and began waving her hands excitedly,
and as they passed each other the tears started to flow. He looked up at
her, holding her gaze, and patted his hand on his heart in gratitude.

Later, during the first of two lengthy interviews, I asked him about it.
He told me, with a hint of embarrassment and not the slightest trace of
ego, that this was not the first time.

``People cry, they cannot believe they're talking to me,'' he said,
speaking softly, as he always did --- except during his speeches, when
he roared. ``I think a lot of people think, in some way, somehow, I
don't exist, like they think I worked in another time.''

It struck me that this was both the blessing and the burden of being Mr.
Lewis. He was fixed in the American mind as the young Freedom Rider (he
was one of the original 13) who was beaten, bloodied and left with a
cracked skull by a state trooper on a bridge in Alabama. It had earned
him titles: ``Civil rights icon.'' ``American hero.'' ``Conscience of
the Congress.''

But somewhere in there was John Lewis the man --- living in the present,
carrying the past --- and I wondered how he dealt with it.

``I've felt guilty from time to time,'' he told me --- especially about
the disappearance and brutal murder of three civil rights workers,
Andrew Goodman, Michael Schwerner and James Chaney, in Mississippi in
the summer of 1964. Mr. Goodman and Mr. Schwerner had come from New York
to join the fight for voting rights. Mr. Lewis, just 24 at the time, had
recruited them.

``I learned to survive,'' he told me. ``But it took me a long time, a
very long time, to go back to the place where these three young men came
up missing.''

How long? I asked. About 30 years, he said. And when he did, ``I just
broke down and cried. It was just too much.''

For nearly two hours, on two separate occasions, we talked about Mr.
Lewis's life. He told me that he was ``somewhat shy'' as a boy, growing
up in small town Alabama. Aspiring to a life in the ministry, he
preached to his chickens. (Unlike members of Congress, he joked, they
listened to him.)

He told me about the letter he wrote to the Rev. Dr. Martin Luther King
Jr. when he was 18, and how he took a bus trip to Montgomery to meet the
civil rights leader, and was ``scared'' walking in the door. He told me
his parents and grandparents, frightened for his safety, discouraged his
early civil rights work, telling him to ``be quiet,'' and ``don't get in
the way.'' He told me how, decades later, he cried when Barack Obama was
inaugurated as president.

Mr. Lewis's Capitol Hill office was like a civil rights museum, filled
with black-and-white photographs. I asked him to give me a tour. He
paused at an image of movement leaders at the Alabama State Capitol,
where Gov. George Wallace had refused to allow Black people on the
grounds.

\includegraphics{https://static01.nyt.com/images/2020/07/23/us/23insider-lewis-print/merlin_174833262_0472135e-1d3f-453e-b477-4cbea8970b74-articleLarge.jpg?quality=75\&auto=webp\&disable=upscale}

There was Dr. King and his wife, Coretta. There was the Rev. Ralph
Abernathy, and the writer James Baldwin. There was Charles Evers, who
died Wednesday and whose brother, Medgar, had been assassinated by a Ku
Klux Klansman. There, he said, was ``young John Lewis.''

His use of the third person startled me, but it also made sense. It
imposed a distance between then and now. ``Like, it's not me,'' he later
said.

When I was a child, Dr. King was one of my heroes. Now, here I was,
sitting with the sole surviving speaker from the march that produced Dr.
King's ``I Have a Dream'' speech. I am fascinated by civil rights
history, and working as a reporter for The New York Times has afforded
me a rare opportunity to pursue that interest. In 2010, I did a similar
interview with one of Mr. Lewis's elders in the movement,
\href{https://www.nytimes.com/video/obituaries/1247467659411/an-interview-with-dorothy-height.html}{Dorothy
Height}, then 97.

Mr. Lewis was 73 when we met. He knew then, he told me, that ``I won't
be in this position forever.'' He wanted to continue making change while
he could, ``to make our own country better, to leave the world a little
cleaner.''

It was a gift to have spent so much time with him, and when we parted, I
thanked him. ``Some stories I do for the paper and some stories I do for
me,'' I told him. ``This one I'm doing for me, and everybody else can
read it.''

Advertisement

\protect\hyperlink{after-bottom}{Continue reading the main story}

\hypertarget{site-index}{%
\subsection{Site Index}\label{site-index}}

\hypertarget{site-information-navigation}{%
\subsection{Site Information
Navigation}\label{site-information-navigation}}

\begin{itemize}
\tightlist
\item
  \href{https://help.nytimes.com/hc/en-us/articles/115014792127-Copyright-notice}{©~2020~The
  New York Times Company}
\end{itemize}

\begin{itemize}
\tightlist
\item
  \href{https://www.nytco.com/}{NYTCo}
\item
  \href{https://help.nytimes.com/hc/en-us/articles/115015385887-Contact-Us}{Contact
  Us}
\item
  \href{https://www.nytco.com/careers/}{Work with us}
\item
  \href{https://nytmediakit.com/}{Advertise}
\item
  \href{http://www.tbrandstudio.com/}{T Brand Studio}
\item
  \href{https://www.nytimes.com/privacy/cookie-policy\#how-do-i-manage-trackers}{Your
  Ad Choices}
\item
  \href{https://www.nytimes.com/privacy}{Privacy}
\item
  \href{https://help.nytimes.com/hc/en-us/articles/115014893428-Terms-of-service}{Terms
  of Service}
\item
  \href{https://help.nytimes.com/hc/en-us/articles/115014893968-Terms-of-sale}{Terms
  of Sale}
\item
  \href{https://spiderbites.nytimes.com}{Site Map}
\item
  \href{https://help.nytimes.com/hc/en-us}{Help}
\item
  \href{https://www.nytimes.com/subscription?campaignId=37WXW}{Subscriptions}
\end{itemize}
