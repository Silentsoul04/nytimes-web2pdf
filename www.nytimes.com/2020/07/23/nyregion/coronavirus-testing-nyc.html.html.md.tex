Sections

SEARCH

\protect\hyperlink{site-content}{Skip to
content}\protect\hyperlink{site-index}{Skip to site index}

\href{https://www.nytimes.com/section/nyregion}{New York}

\href{https://myaccount.nytimes.com/auth/login?response_type=cookie\&client_id=vi}{}

\href{https://www.nytimes.com/section/todayspaper}{Today's Paper}

\href{/section/nyregion}{New York}\textbar{}Testing Bottlenecks Threaten
N.Y.C.'s Ability to Contain Virus

\href{https://nyti.ms/32K6DP7}{https://nyti.ms/32K6DP7}

\begin{itemize}
\item
\item
\item
\item
\item
\end{itemize}

\href{https://www.nytimes.com/news-event/coronavirus?action=click\&pgtype=Article\&state=default\&region=TOP_BANNER\&context=storylines_menu}{The
Coronavirus Outbreak}

\begin{itemize}
\tightlist
\item
  live\href{https://www.nytimes.com/2020/08/08/world/coronavirus-updates.html?action=click\&pgtype=Article\&state=default\&region=TOP_BANNER\&context=storylines_menu}{Latest
  Updates}
\item
  \href{https://www.nytimes.com/interactive/2020/us/coronavirus-us-cases.html?action=click\&pgtype=Article\&state=default\&region=TOP_BANNER\&context=storylines_menu}{Maps
  and Cases}
\item
  \href{https://www.nytimes.com/interactive/2020/science/coronavirus-vaccine-tracker.html?action=click\&pgtype=Article\&state=default\&region=TOP_BANNER\&context=storylines_menu}{Vaccine
  Tracker}
\item
  \href{https://www.nytimes.com/interactive/2020/world/coronavirus-tips-advice.html?action=click\&pgtype=Article\&state=default\&region=TOP_BANNER\&context=storylines_menu}{F.A.Q.}
\item
  \href{https://www.nytimes.com/live/2020/08/07/business/stock-market-today-coronavirus?action=click\&pgtype=Article\&state=default\&region=TOP_BANNER\&context=storylines_menu}{Markets
  \& Economy}
\end{itemize}

Advertisement

\protect\hyperlink{after-top}{Continue reading the main story}

Supported by

\protect\hyperlink{after-sponsor}{Continue reading the main story}

\hypertarget{testing-bottlenecks-threaten-nycs-ability-to-contain-virus}{%
\section{Testing Bottlenecks Threaten N.Y.C.'s Ability to Contain
Virus}\label{testing-bottlenecks-threaten-nycs-ability-to-contain-virus}}

``Honestly, I don't even really see the point in getting tested,'' said
one New Yorker who has waited nearly two weeks, with still no results.

\includegraphics{https://static01.nyt.com/images/2020/07/22/nyregion/00nyvirus-testing-02/00nyvirus-testing-02-articleLarge.jpg?quality=75\&auto=webp\&disable=upscale}

\href{https://www.nytimes.com/by/joseph-goldstein}{\includegraphics{https://static01.nyt.com/images/2018/07/16/multimedia/author-joseph-goldstein/author-joseph-goldstein-thumbLarge.png}}\href{https://www.nytimes.com/by/jesse-mckinley}{\includegraphics{https://static01.nyt.com/images/2018/02/20/multimedia/author-jesse-mckinley/author-jesse-mckinley-thumbLarge.jpg}}

By \href{https://www.nytimes.com/by/joseph-goldstein}{Joseph Goldstein}
and \href{https://www.nytimes.com/by/jesse-mckinley}{Jesse McKinley}

\begin{itemize}
\item
  Published July 23, 2020Updated Aug. 5, 2020
\item
  \begin{itemize}
  \item
  \item
  \item
  \item
  \item
  \end{itemize}
\end{itemize}

Nearly four months after the pandemic's peak, New York City is facing
such serious delays in returning
\href{https://www.nytimes.com/2020/08/05/world/europe/germany-coronavirus-test-travelers.html}{coronavirus
test} results that public health experts are warning that the problems
could hinder efforts to reopen the local economy and schools.

Despite repeated pledges from Gov. Andrew M. Cuomo and Mayor Bill de
Blasio that testing would be both widely accessible and effective,
thousands of New Yorkers have had to wait a week or more for results,
and at some clinics the median wait time is nine days. One prominent
local official has even proposed the drastic step of limiting testing.

The delays in New York City are caused in part by the outbreak's spike
in states like California, Florida and Texas, which has strained
laboratories across the country and touched off a renewed national
testing crisis.

But officials have also been unable to adequately expand the capacity of
state and city government laboratories in New York to
\href{https://www.nytimes.com/2020/08/06/health/rapid-Covid-tests.html}{test
rapidly} at a time when they are asking more New Yorkers to get tested
to guard against a second wave.

The delays limit the ability of public health officials to quickly
identify --- and isolate --- people who are infected while also
diminishing the usefulness of
\href{https://www.nytimes.com/2020/06/21/nyregion/nyc-contact-tracing.html}{New
York City's contact-tracing program}. They also can lead to growing
blind spots that obscure the extent of the virus's spread, which could
spell trouble as the city tries to reopen.

As a result, some public officials and laboratory executives say New
York's strategy of allowing anyone and everyone who wants a test to get
one is unsustainable.

``I'm afraid that we have to prioritize people with symptoms, someone
who has been exposed, or someone whose work puts them in contact with a
lot of people,'' said a city councilman, Mark Levine, a Manhattan
Democrat who heads the Council's health committee. ``That's what I'm
about to call for, but I don't think City Hall wants to.''

New York was once the epicenter of the pandemic in the United States,
\href{https://www.nytimes.com/interactive/2020/us/new-york-coronavirus-cases.html}{suffering
more than 30,000 deaths}, far more than any other state. But by shutting
down in March and April, it has significantly slowed the spread of the
virus. The
\href{https://www.nytimes.com/2020/07/22/us/coronavirus-northeast-governors.html}{Northeast
has emerged as the only region} in the country to beat back the
outbreak.

When the pandemic peaked in the city, testing was relatively scarce, and
prominent elected officials, from President Trump to Mr. Cuomo and Mr.
de Blasio, said they would ensure that in the future, there would be
more than enough.

As capacity expanded, New York City authorities began encouraging
\href{https://www.nychealthandhospitals.org/covid-19-testing-sites/}{everyone
to get tested, and urged people} to get tested repeatedly, setting a
target of 50,000 tests per day.

In recent weeks, about
\href{https://www1.nyc.gov/site/doh/covid/covid-19-data.page}{20,000 to
35,000} people are tested most weekdays, a demand that has put a strain
on local labs.

City public health officials said they were growing increasingly alarmed
by the delays, pointing out that widespread testing and quick turnaround
times were needed to reduce transmission by asymptomatic and
pre-symptomatic patients,
\href{https://www.nytimes.com/2020/06/27/world/europe/coronavirus-spread-asymptomatic.html}{who
are believed to play a major part} in the virus's spread.

``This is becoming a problem,'' said Dr. Jay Varma, a City Hall adviser
who has a critical role in the city's testing and contact-tracing
program. ``Any lag in this process can make it more difficult to have
case and contact tracing be effective.''

\hypertarget{latest-updates-the-coronavirus-outbreak}{%
\section{\texorpdfstring{\href{https://www.nytimes.com/2020/08/07/world/covid-19-news.html?action=click\&pgtype=Article\&state=default\&region=MAIN_CONTENT_1\&context=storylines_live_updates}{Latest
Updates: The Coronavirus
Outbreak}}{Latest Updates: The Coronavirus Outbreak}}\label{latest-updates-the-coronavirus-outbreak}}

Updated 2020-08-08T12:04:28.992Z

\begin{itemize}
\tightlist
\item
  \href{https://www.nytimes.com/2020/08/07/world/covid-19-news.html?action=click\&pgtype=Article\&state=default\&region=MAIN_CONTENT_1\&context=storylines_live_updates\#link-1f86d03a}{As
  the U.S. relief talks falter again, Trump says he is prepared to act
  on his own.}
\item
  \href{https://www.nytimes.com/2020/08/07/world/covid-19-news.html?action=click\&pgtype=Article\&state=default\&region=MAIN_CONTENT_1\&context=storylines_live_updates\#link-3f64a70a}{Cuomo
  says N.Y. schools can reopen in-person but leaves it up to districts
  to determine if, when and how.}
\item
  \href{https://www.nytimes.com/2020/08/07/world/covid-19-news.html?action=click\&pgtype=Article\&state=default\&region=MAIN_CONTENT_1\&context=storylines_live_updates\#link-14e70066}{Thousands
  of cases went unreported in California when a computer server failed.}
\end{itemize}

\href{https://www.nytimes.com/2020/08/07/world/covid-19-news.html?action=click\&pgtype=Article\&state=default\&region=MAIN_CONTENT_1\&context=storylines_live_updates}{See
more updates}

More live coverage:
\href{https://www.nytimes.com/live/2020/08/07/business/stock-market-today-coronavirus?action=click\&pgtype=Article\&state=default\&region=MAIN_CONTENT_1\&context=storylines_live_updates}{Markets}

Still, Gareth Rhodes, an aide to Mr. Cuomo and a member of his virus
response team, said that there was a priority placed on tests for people
with Covid-19 symptoms and those who reported being exposed to someone
who is positive for the virus. He added that some labs could return such
tests in less than 24 hours.

But if someone has no symptoms and no known exposure, he said, ``I'm
less concerned if the result comes back in five, six, seven days.''

\includegraphics{https://static01.nyt.com/images/2020/07/22/nyregion/00nyvirus-testing/00nyvirus-testing-articleLarge.jpg?quality=75\&auto=webp\&disable=upscale}

Mr. de Blasio said on Thursday that he was moving to address delays in
testing. He blamed the national surge in cases for the waits and said
labs were overwhelmed.

Asked how delays were able to mount in New York City after the mayor
pledged to prioritize testing, Mr. de Blasio said the city had to
``reset the equation'' after cases spiked across the country.

``I've been consistent --- we want fast turnaround times and we want the
maximum number of people tested, and that has been working
overwhelmingly until we hit this glitch,'' the mayor told reporters.

On Thursday, the governor defended the state's performance, noting that
longer delays were being seen because some heavily used labs were
``getting overwhelmed'' by demand for results from other states.

New York processes about 70 percent of its tests at a network of more
than 200 private labs, which the state has enlisted to process
specimens, Mr. Cuomo said. It was redirecting some samples to
underutilized facilities, he added, which resulted in average wait times
for results from those labs of 2.6 days.

But Mr. Cuomo conceded that some samples sent to busy national labs had
wait times that averaged six to 10 days, and sometimes even longer.

And the governor said the problems could get even worse in the fall,
during flu season, when labs would be asked to process samples looking
for that infection. ``The flu tests will eat at the capacity,'' he said.

New York City is finding ways to lessen its reliance on commercial
laboratories, like Quest Diagnostics, where backlogs sometimes mean
waits of up to two weeks. The city's Department of Health and Mental
Hygiene is vowing to expand its own capacity to conduct tests.

But the delays may get worse before they get better. The reasons are
complex, but are largely driven by a simple fact: Demand for coronavirus
tests has grown faster than laboratory capacity. And demand is likely to
increase with the start of the school year, particularly with some
universities requiring the tests for students.

``The pressure put on us by the higher-ed community, who wants every kid
to have a negative test to show up on campus, will soon put a strain on
the testing system,'' said Scott J. Becker, the chief executive of the
Association of Public Health Laboratories.

Image

The city's Health Department is working on transforming nine clinics
that had been used to test and treat patients for sexually transmitted
diseases and tuberculosis into coronavirus testing sites.Credit...John
Minchillo/Associated Press

The current crisis carries some echoes of February and March, when
limited testing capacity and a disastrous
\href{https://www.nytimes.com/2020/04/18/health/cdc-coronavirus-lab-contamination-testing.html}{series}
of
\href{https://www.nytimes.com/2020/03/28/us/testing-coronavirus-pandemic.html}{missteps}
by the
\href{https://www.nytimes.com/2020/06/03/us/cdc-coronavirus.html}{federal
government} meant relatively few sick New Yorkers were able to get
tested. The virus spread rapidly --- and largely undetected.

The nation's testing capacity has expanded significantly since then.
More than
\href{https://coronavirus.jhu.edu/testing/individual-states/usa}{750,000
tests} are administered across the nation on some days.

But even as laboratories try to increase capacity, some supplies are
getting more difficult to obtain. In particular, the cartridges that
have been critical for quick testing at many hospitals are growing
scarcer, said Dr. Dwayne Breining, who oversees laboratories at
Northwell Health, New York's largest hospital system.

``There are effects in our area from what's going on in the rest of the
country,'' he said. ``All of those companies are kind of allocating
their supply to the places that are hot spots, which is clinically
appropriate.''

As of early July, results for about a quarter of coronavirus tests in
New York City were returned within 24 hours, Avery Cohen, a spokeswoman
for the mayor, said. But a quarter of tests took more than six days, she
said.

\href{https://www.nytimes.com/news-event/coronavirus?action=click\&pgtype=Article\&state=default\&region=MAIN_CONTENT_3\&context=storylines_faq}{}

\hypertarget{the-coronavirus-outbreak-}{%
\subsubsection{The Coronavirus Outbreak
›}\label{the-coronavirus-outbreak-}}

\hypertarget{frequently-asked-questions}{%
\paragraph{Frequently Asked
Questions}\label{frequently-asked-questions}}

Updated August 6, 2020

\begin{itemize}
\item ~
  \hypertarget{why-are-bars-linked-to-outbreaks}{%
  \paragraph{Why are bars linked to
  outbreaks?}\label{why-are-bars-linked-to-outbreaks}}

  \begin{itemize}
  \tightlist
  \item
    Think about a bar. Alcohol is flowing. It can be loud, but it's
    definitely intimate, and you often need to lean in close to hear
    your friend. And strangers have way, way fewer reservations about
    coming up to people in a bar. That's sort of the point of a bar.
    Feeling good and close to strangers. It's no surprise, then, that
    \href{https://www.nytimes.com/2020/07/02/us/coronavirus-bars.html?action=click\&pgtype=Article\&state=default\&region=MAIN_CONTENT_3\&context=storylines_faq}{bars
    have been linked to outbreaks in several states.} Louisiana health
    officials have tied
    \href{https://www.nytimes.com/2020/06/22/us/new-coronavirus-phase.html?action=click\&pgtype=Article\&state=default\&region=MAIN_CONTENT_3\&context=storylines_faq}{at
    least 100 coronavirus cases} to bars in the Tigerland nightlife
    district in Baton Rouge. Minnesota has traced 328 recent cases to
    bars across the state.
    \href{https://www.boisestatepublicradio.org/post/bars-large-venues-close-ada-county-after-surge-coronavirus-prompts-rollback\#stream/0}{In
    Idaho}, health officials shut down bars in Ada County after
    reporting clusters of infections among young adults who had visited
    several bars in downtown Boise. Governors in
    \href{https://www.nytimes.com/2020/07/01/us/california-coronavirus-reopening.html?action=click\&pgtype=Article\&state=default\&region=MAIN_CONTENT_3\&context=storylines_faq}{California},
    \href{https://www.nytimes.com/2020/06/14/us/coronavirus-united-states.html?action=click\&pgtype=Article\&state=default\&region=MAIN_CONTENT_3\&context=storylines_faq}{Texas
    and Arizona}, where coronavirus cases are soaring, have ordered
    hundreds of newly reopened bars to shut down. Less than two weeks
    after Colorado's bars reopened at limited capacity, Gov. Jared Polis
    \href{https://www.denverpost.com/2020/06/30/colorado-bars-closed-coronavirus/}{ordered
    them to close}.
  \end{itemize}
\item ~
  \hypertarget{i-have-antibodies-am-i-now-immune}{%
  \paragraph{I have antibodies. Am I now
  immune?}\label{i-have-antibodies-am-i-now-immune}}

  \begin{itemize}
  \tightlist
  \item
    As of right now,
    \href{https://www.nytimes.com/2020/07/22/health/covid-antibodies-herd-immunity.html?action=click\&pgtype=Article\&state=default\&region=MAIN_CONTENT_3\&context=storylines_faq}{that
    seems likely, for at least several months.} There have been
    frightening accounts of people suffering what seems to be a second
    bout of Covid-19. But experts say these patients may have a
    drawn-out course of infection, with the virus taking a slow toll
    weeks to months after initial exposure. People infected with the
    coronavirus typically
    \href{https://www.nature.com/articles/s41586-020-2456-9}{produce}
    immune molecules called antibodies, which are
    \href{https://www.nytimes.com/2020/05/07/health/coronavirus-antibody-prevalence.html?action=click\&pgtype=Article\&state=default\&region=MAIN_CONTENT_3\&context=storylines_faq}{protective
    proteins made in response to an
    infection}\href{https://www.nytimes.com/2020/05/07/health/coronavirus-antibody-prevalence.html?action=click\&pgtype=Article\&state=default\&region=MAIN_CONTENT_3\&context=storylines_faq}{.
    These antibodies may} last in the body
    \href{https://www.nature.com/articles/s41591-020-0965-6}{only two to
    three months}, which may seem worrisome, but that's perfectly normal
    after an acute infection subsides, said Dr. Michael Mina, an
    immunologist at Harvard University. It may be possible to get the
    coronavirus again, but it's highly unlikely that it would be
    possible in a short window of time from initial infection or make
    people sicker the second time.
  \end{itemize}
\item ~
  \hypertarget{im-a-small-business-owner-can-i-get-relief}{%
  \paragraph{I'm a small-business owner. Can I get
  relief?}\label{im-a-small-business-owner-can-i-get-relief}}

  \begin{itemize}
  \tightlist
  \item
    The
    \href{https://www.nytimes.com/article/small-business-loans-stimulus-grants-freelancers-coronavirus.html?action=click\&pgtype=Article\&state=default\&region=MAIN_CONTENT_3\&context=storylines_faq}{stimulus
    bills enacted in March} offer help for the millions of American
    small businesses. Those eligible for aid are businesses and
    nonprofit organizations with fewer than 500 workers, including sole
    proprietorships, independent contractors and freelancers. Some
    larger companies in some industries are also eligible. The help
    being offered, which is being managed by the Small Business
    Administration, includes the Paycheck Protection Program and the
    Economic Injury Disaster Loan program. But lots of folks have
    \href{https://www.nytimes.com/interactive/2020/05/07/business/small-business-loans-coronavirus.html?action=click\&pgtype=Article\&state=default\&region=MAIN_CONTENT_3\&context=storylines_faq}{not
    yet seen payouts.} Even those who have received help are confused:
    The rules are draconian, and some are stuck sitting on
    \href{https://www.nytimes.com/2020/05/02/business/economy/loans-coronavirus-small-business.html?action=click\&pgtype=Article\&state=default\&region=MAIN_CONTENT_3\&context=storylines_faq}{money
    they don't know how to use.} Many small-business owners are getting
    less than they expected or
    \href{https://www.nytimes.com/2020/06/10/business/Small-business-loans-ppp.html?action=click\&pgtype=Article\&state=default\&region=MAIN_CONTENT_3\&context=storylines_faq}{not
    hearing anything at all.}
  \end{itemize}
\item ~
  \hypertarget{what-are-my-rights-if-i-am-worried-about-going-back-to-work}{%
  \paragraph{What are my rights if I am worried about going back to
  work?}\label{what-are-my-rights-if-i-am-worried-about-going-back-to-work}}

  \begin{itemize}
  \tightlist
  \item
    Employers have to provide
    \href{https://www.osha.gov/SLTC/covid-19/standards.html}{a safe
    workplace} with policies that protect everyone equally.
    \href{https://www.nytimes.com/article/coronavirus-money-unemployment.html?action=click\&pgtype=Article\&state=default\&region=MAIN_CONTENT_3\&context=storylines_faq}{And
    if one of your co-workers tests positive for the coronavirus, the
    C.D.C.} has said that
    \href{https://www.cdc.gov/coronavirus/2019-ncov/community/guidance-business-response.html}{employers
    should tell their employees} -\/- without giving you the sick
    employee's name -\/- that they may have been exposed to the virus.
  \end{itemize}
\item ~
  \hypertarget{what-is-school-going-to-look-like-in-september}{%
  \paragraph{What is school going to look like in
  September?}\label{what-is-school-going-to-look-like-in-september}}

  \begin{itemize}
  \tightlist
  \item
    It is unlikely that many schools will return to a normal schedule
    this fall, requiring the grind of
    \href{https://www.nytimes.com/2020/06/05/us/coronavirus-education-lost-learning.html?action=click\&pgtype=Article\&state=default\&region=MAIN_CONTENT_3\&context=storylines_faq}{online
    learning},
    \href{https://www.nytimes.com/2020/05/29/us/coronavirus-child-care-centers.html?action=click\&pgtype=Article\&state=default\&region=MAIN_CONTENT_3\&context=storylines_faq}{makeshift
    child care} and
    \href{https://www.nytimes.com/2020/06/03/business/economy/coronavirus-working-women.html?action=click\&pgtype=Article\&state=default\&region=MAIN_CONTENT_3\&context=storylines_faq}{stunted
    workdays} to continue. California's two largest public school
    districts --- Los Angeles and San Diego --- said on July 13, that
    \href{https://www.nytimes.com/2020/07/13/us/lausd-san-diego-school-reopening.html?action=click\&pgtype=Article\&state=default\&region=MAIN_CONTENT_3\&context=storylines_faq}{instruction
    will be remote-only in the fall}, citing concerns that surging
    coronavirus infections in their areas pose too dire a risk for
    students and teachers. Together, the two districts enroll some
    825,000 students. They are the largest in the country so far to
    abandon plans for even a partial physical return to classrooms when
    they reopen in August. For other districts, the solution won't be an
    all-or-nothing approach.
    \href{https://bioethics.jhu.edu/research-and-outreach/projects/eschool-initiative/school-policy-tracker/}{Many
    systems}, including the nation's largest, New York City, are
    devising
    \href{https://www.nytimes.com/2020/06/26/us/coronavirus-schools-reopen-fall.html?action=click\&pgtype=Article\&state=default\&region=MAIN_CONTENT_3\&context=storylines_faq}{hybrid
    plans} that involve spending some days in classrooms and other days
    online. There's no national policy on this yet, so check with your
    municipal school system regularly to see what is happening in your
    community.
  \end{itemize}
\end{itemize}

Some of the longest delays are at the dozens of CityMD walk-in clinics
that have blanketed the city in recent years. Thousands of New Yorkers
seek tests there each day.

CityMD sends many coronavirus tests to a Quest Diagnostics laboratory in
Teterboro, N.J., for processing. Quest Diagnostics has cited
\href{https://newsroom.questdiagnostics.com/COVIDTestingUpdates}{a range
of factors} for why turnaround times have been doubling or tripling,
including the growing number of orders from employees returning to work
and from hospitals that have resumed elective surgeries, but need first
to screen patients.

``We're not taking specimens from southern parts of the country and
moving those to the Teterboro, New Jersey, lab,'' Wendy Bost, a
spokeswoman for Quest Diagnostics, said. ``The Teterboro lab is dealing
with volume that is coming from the region.''

Across the nation, test results on average take
\href{https://newsroom.questdiagnostics.com/COVIDTestingUpdates}{slightly
more than two days} for priority samples, Quest Diagnostics said. But
for everyone else, the wait times have been getting much longer --- up
two weeks in some instances.

Some other laboratories have managed to keep turnaround times shorter.
The city's public hospital system, which runs a network of health
clinics and community testing sites, sends many samples to BioReference
Laboratories, which currently has a turnaround time of two to three days
for nonpriority samples, said Dr. Jon R. Cohen, executive chairman of
BioReference, one of the nation's largest commercial laboratories.

BioReference is using a ``pooling'' technique, where if a batch tests
negative, all the samples are deemed negative. If it tests positive,
each sample must be individually tested.

For now, City Hall's strategy for reducing turnaround times has been to
advertise free testing at city-run sites, where the waits tend to be
shorter, city officials said.

Image

At some clinics, the median wait time for test results is nine
days.Credit...Spencer Platt/Getty Images

The Health Department is also working on transforming nine clinics that
had been used to test and treat patients for sexually transmitted
diseases and tuberculosis into coronavirus testing sites. That would
expand the city's public laboratory capacity considerably.

The health commissioner, Dr. Oxiris Barbot, said these sites would be
able to process ``a couple of thousand tests a day.''

Stories of long waits for results have become common among New Yorkers.

Lee Ziesche, 31, said she went to get a virus test on July 5 at a CityMD
location in her Brooklyn neighborhood, Bedford-Stuyvesant, as a
precaution after her boyfriend's roommate started feeling sick. She said
she got her results on July 20.

``I wasn't that worried about myself, but the reflection on how the
system is working was super concerning,'' she said. ``It makes it really
hard for us to return to normal when it takes two weeks for us to get
tests.''

Zach Honig, 34, who lives in the Financial District, said he was tested
on July 12 in anticipation of a trip to Maine and still had not gotten
his results.

``Honestly, I don't even really see the point in getting tested,'' he
said. ``Even if I get a positive result, I imagine I wouldn't even be
contagious anymore. It's just not really practical.''

So far, the delays do not seem to have contributed to an uptick of
transmission, but a second wave of the outbreak looms.

``That seriously undermines the entire purpose of testing --- both to
inform people they are contagious so they are quarantined and also to
trigger the contact tracing to find out who else may have been
exposed,'' Mr. Levine, the councilman, said. ``With a delay of seven
days, you can be pretty certain the virus will spread.''

Dr. Varma, the City Hall adviser, said it would be a mistake to restrict
testing to only the symptomatic. Despite the lag times, he said, it made
sense ``to push through and stick with your strategy of expanding
testing as much as possible.''

Testing and contact tracing are tightly linked: After people with active
virus infections are discovered through testing, contact tracers
interview them about whom they in turn may have infected. Then contact
tracers try to get these contacts into quarantine before they become
contagious.

The longer test results take, the more likely contact tracers will
simply be tracking the spread of the virus from person to person, rather
than stopping it, said Charles King, an AIDS activist, and member of a
City Hall appointed group advising the contact-tracing program.

``Frankly, if you can't get results within 24 hours,'' Mr. King said,
``you do start losing the utility of the exercise.''

Emma G. Fitzsimmons and Troy Closson contributed reporting.

Advertisement

\protect\hyperlink{after-bottom}{Continue reading the main story}

\hypertarget{site-index}{%
\subsection{Site Index}\label{site-index}}

\hypertarget{site-information-navigation}{%
\subsection{Site Information
Navigation}\label{site-information-navigation}}

\begin{itemize}
\tightlist
\item
  \href{https://help.nytimes.com/hc/en-us/articles/115014792127-Copyright-notice}{©~2020~The
  New York Times Company}
\end{itemize}

\begin{itemize}
\tightlist
\item
  \href{https://www.nytco.com/}{NYTCo}
\item
  \href{https://help.nytimes.com/hc/en-us/articles/115015385887-Contact-Us}{Contact
  Us}
\item
  \href{https://www.nytco.com/careers/}{Work with us}
\item
  \href{https://nytmediakit.com/}{Advertise}
\item
  \href{http://www.tbrandstudio.com/}{T Brand Studio}
\item
  \href{https://www.nytimes.com/privacy/cookie-policy\#how-do-i-manage-trackers}{Your
  Ad Choices}
\item
  \href{https://www.nytimes.com/privacy}{Privacy}
\item
  \href{https://help.nytimes.com/hc/en-us/articles/115014893428-Terms-of-service}{Terms
  of Service}
\item
  \href{https://help.nytimes.com/hc/en-us/articles/115014893968-Terms-of-sale}{Terms
  of Sale}
\item
  \href{https://spiderbites.nytimes.com}{Site Map}
\item
  \href{https://help.nytimes.com/hc/en-us}{Help}
\item
  \href{https://www.nytimes.com/subscription?campaignId=37WXW}{Subscriptions}
\end{itemize}
