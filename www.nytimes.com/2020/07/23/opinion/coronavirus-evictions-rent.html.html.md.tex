Sections

SEARCH

\protect\hyperlink{site-content}{Skip to
content}\protect\hyperlink{site-index}{Skip to site index}

\href{https://myaccount.nytimes.com/auth/login?response_type=cookie\&client_id=vi}{}

\href{https://www.nytimes.com/section/todayspaper}{Today's Paper}

\href{/section/opinion}{Opinion}\textbar{}Millions of Americans Are
About to Lose Their Homes. Congress Must Help Them.

\url{https://nyti.ms/3f3iDxL}

\begin{itemize}
\item
\item
\item
\item
\item
\item
\end{itemize}

Advertisement

\protect\hyperlink{after-top}{Continue reading the main story}

\href{/section/opinion}{Opinion}

Supported by

\protect\hyperlink{after-sponsor}{Continue reading the main story}

\hypertarget{millions-of-americans-are-about-to-lose-their-homes-congress-must-help-them}{%
\section{Millions of Americans Are About to Lose Their Homes. Congress
Must Help
Them.}\label{millions-of-americans-are-about-to-lose-their-homes-congress-must-help-them}}

Here's what can be done to help people avoid eviction.

By
\href{https://www.nytimes.com/interactive/opinion/editorialboard.html}{The
Editorial Board}

The editorial board is a group of opinion journalists whose views are
informed by expertise, research, debate and certain longstanding ****
\href{https://www.nytimes.com/interactive/2018/opinion/editorialboard.html}{values}.
It is separate from the newsroom.

\begin{itemize}
\item
  July 23, 2020
\item
  \begin{itemize}
  \item
  \item
  \item
  \item
  \item
  \item
  \end{itemize}
\end{itemize}

\includegraphics{https://static01.nyt.com/images/2020/07/23/opinion/23evictions_big_house/23evictions_big_house-mediumSquareAt3X.jpg}

By failing to contain the coronavirus, the United States is allowing
what began as a temporary disruption of economic life to do lasting
damage to the nation's prosperity and prospects. With little chance of
an imminent economic rebound, millions of Americans who have lost their
jobs during the pandemic are now in grave danger of losing their homes,
too.

Twenty-two percent of households say that they don't expect to be able
to make their next monthly rent or mortgage payment, according to a
\href{https://www.census.gov/householdpulsedata}{Census Bureau survey}.

Temporary limits on evictions, imposed in the early weeks of the U.S.
crisis, are gradually ending, and a growing number of lenders and
landlords are seeking to evict those who cannot pay.

The plight of desperate tenants and homeowners is attracting far less
attention than it did during the housing crisis that peaked in 2008,
perhaps because this time the problems did not begin in the housing
market, or perhaps because this crisis arrived so abruptly. But the
alarm bells ought to be ringing: The United States is on the verge of
allowing a mass dislocation of lower-income households that could dwarf
the last crisis.

The immediate need is for Congress to impose a nationwide moratorium on
evictions and then to give people who have lost their jobs the money
required for rent or mortgage payments. The moratorium is necessary
because it takes time to distribute aid; it would protect people from
losing homes while help is on the way. The aid is necessary because
erasing obligations, as some have proposed, would merely move the crisis
up the food chain. About 47 percent of rental units are owned by
individual investors, who must pay their debts, too.

Much of the necessary money can be provided by continuing the \$600
weekly payments that the federal government has made to unemployed
workers since April. The Urban Institute
\href{https://www.urban.org/research/publication/how-much-assistance-needed-support-renters-through-covid-19-crisis}{calculates}
that those payments provide roughly two-thirds of the \$5.5 billion in
monthly aid required to keep people in their homes. (Even with those
payments, the institute estimates that the government needs to provide
another \$1.8 billion in monthly housing aid. That number is a minimum:
It would not cover other obligations, especially missed utility
payments, which can also lead to eviction. North Carolina residents, for
example, underpaid utility bills by \$218 million from April through
June, The Washington Post
\href{https://www.washingtonpost.com/business/2020/07/23/north-carolina-utility-bills-coronavirus/?hpid=hp_hp-top-table-high_bailoutnorthcarolina-1215pm\%3Ahomepage\%2Fstory-ans}{reported}.)

The House passed a bill in May that more than addresses these needs. In
addition to extending supplementary unemployment benefits through
January, it provides \$100 billion in aid for renters --- about \$16
billion a month for the next six months --- and another \$75 billion in
aid for homeowners, both substantially reserved for lower-income
households. It also imposes a 12-month moratorium on tenant evictions
and a 60-day grace period for homeowners facing foreclosure. It includes
funding to help people who do lose their homes, including \$11.5 billion
for homeless shelters and support services.

Senate Republicans have not offered a counterproposal. After insisting
for months that additional federal aid was not required, Republicans
have acknowledged the need to do something, but even as existing
measures begin to expire this weekend, they have yet to agree on the
details. Notably,
\href{https://www.nytimes.com/interactive/2020/07/23/us/republican-draft-virus-aid-bill-july-23.html}{a
draft proposal} that circulated Thursday included no mention of direct
housing aid, while calling for a sharp reduction in unemployment
benefits.

In the absence of aid, millions of Americans could lose their homes in
the coming months.

But even the House legislation is not sufficient to address the crisis.
Congress also needs to provide expert assistance to tenants and
homeowners facing the loss of homes.

People regularly get evicted
\href{https://www.nytimes.com/2020/07/23/business/evictions-moratorium-cares-act.html}{even
when the law is on their side}. The federal government, for example, has
imposed a moratorium on evictions from properties with mortgages backed
by the federal government, but only 14 states require landlords to
certify that their property is not covered. Everywhere else, courts are
basically operating on the honor system.

Landlords are almost always represented by legal counsel, while tenants
rarely have professional help ---~which, predictably, does not go well
for many of them. A study of eviction cases from 2006 to 2016 in Kansas
City found that tenants
\href{https://static1.squarespace.com/static/59ba0bd359cc68f015b7ff8a/t/5a68e811e4966bee3fb5d6cd/1516824594549/KC+Eviction+Project+-+Courts+Analysis.pdf}{prevailed}
in just 161 out of 77,000 cases in that time.

In 2017, New York City began a new program to provide lawyers to
low-income tenants facing eviction, initially in about 10 percent of the
city's neighborhoods. The early evidence suggests that it makes a
difference. In the first year,
\href{https://www.cssny.org/news/entry/nyc-right-to-counsel}{84 percent}
of tenants who received legal representation were able to avoid
eviction.

Congress also should resurrect the
\href{https://neighborworks.org/NFMCCapstoneReport}{National Foreclosure
Mitigation Counseling} program, created in 2008 in response to the last
housing crisis. The program, which ended in 2018, provided counseling to
more than two million homeowners, helping many to avoid foreclosure
through loan modifications or negotiated sales.

The price for such counseling is relatively modest: The foreclosure
mitigation program cost \$853 million over 10 years. The benefits can be
enormous. The loss of a home is also the loss of an investment, and of a
community. Moving can make it harder to keep a job. It can force
children to transfer to a new school. And the black mark of an eviction
or a foreclosure makes it harder to rent a new home, let alone buy one.

The need for such measures is not merely a product of an unexpected
public health crisis. It also reflects the fact that millions of
lower-income households were teetering on the brink of eviction during
the previous decade of economic growth.

Affordable housing is in desperately short supply. Roughly one in four
low-income households spend more than half their income on rent, leaving
little cushion for any loss of income. Even before the coronavirus
struck, more than half a million Americans were homeless, many of them
forced to sleep on the street. Even the most generous proposals before
Congress would leave that harsh reality largely unchanged. They would
address the immediate crisis, but not the enduring crisis.

That is an appropriate priority for now, but in the coming months,
Congress ought to take the lessons of this crisis, and the last one, and
act to ensure every American has access to affordable housing. There is
no justification for providing aid to people facing eviction during a
public health crisis, but not to those who face eviction during an
ordinary July.

Food stamps are available to every American who demonstrates need,
because people need food. Housing aid ought to be available on the same
terms, because people need shelter, too.

\emph{The Times is committed to publishing}
\href{https://www.nytimes.com/2019/01/31/opinion/letters/letters-to-editor-new-york-times-women.html}{\emph{a
diversity of letters}} \emph{to the editor. We'd like to hear what you
think about this or any of our articles. Here are some}
\href{https://help.nytimes.com/hc/en-us/articles/115014925288-How-to-submit-a-letter-to-the-editor}{\emph{tips}}\emph{.
And here's our email:}
\href{mailto:letters@nytimes.com}{\emph{letters@nytimes.com}}\emph{.}

\emph{Follow The New York Times Opinion section on}
\href{https://www.facebook.com/nytopinion}{\emph{Facebook}}\emph{,}
\href{http://twitter.com/NYTOpinion}{\emph{Twitter (@NYTopinion)}}
\emph{and}
\href{https://www.instagram.com/nytopinion/}{\emph{Instagram}}\emph{.}

Advertisement

\protect\hyperlink{after-bottom}{Continue reading the main story}

\hypertarget{site-index}{%
\subsection{Site Index}\label{site-index}}

\hypertarget{site-information-navigation}{%
\subsection{Site Information
Navigation}\label{site-information-navigation}}

\begin{itemize}
\tightlist
\item
  \href{https://help.nytimes.com/hc/en-us/articles/115014792127-Copyright-notice}{©~2020~The
  New York Times Company}
\end{itemize}

\begin{itemize}
\tightlist
\item
  \href{https://www.nytco.com/}{NYTCo}
\item
  \href{https://help.nytimes.com/hc/en-us/articles/115015385887-Contact-Us}{Contact
  Us}
\item
  \href{https://www.nytco.com/careers/}{Work with us}
\item
  \href{https://nytmediakit.com/}{Advertise}
\item
  \href{http://www.tbrandstudio.com/}{T Brand Studio}
\item
  \href{https://www.nytimes.com/privacy/cookie-policy\#how-do-i-manage-trackers}{Your
  Ad Choices}
\item
  \href{https://www.nytimes.com/privacy}{Privacy}
\item
  \href{https://help.nytimes.com/hc/en-us/articles/115014893428-Terms-of-service}{Terms
  of Service}
\item
  \href{https://help.nytimes.com/hc/en-us/articles/115014893968-Terms-of-sale}{Terms
  of Sale}
\item
  \href{https://spiderbites.nytimes.com}{Site Map}
\item
  \href{https://help.nytimes.com/hc/en-us}{Help}
\item
  \href{https://www.nytimes.com/subscription?campaignId=37WXW}{Subscriptions}
\end{itemize}
