Sections

SEARCH

\protect\hyperlink{site-content}{Skip to
content}\protect\hyperlink{site-index}{Skip to site index}

\href{https://myaccount.nytimes.com/auth/login?response_type=cookie\&client_id=vi}{}

\href{https://www.nytimes.com/section/todayspaper}{Today's Paper}

Eight Artists on the Influence of Tom of Finland

\url{https://nyti.ms/3fVxqeT}

\begin{itemize}
\item
\item
\item
\item
\item
\end{itemize}

Advertisement

\protect\hyperlink{after-top}{Continue reading the main story}

Supported by

\protect\hyperlink{after-sponsor}{Continue reading the main story}

True Believers

\hypertarget{eight-artists-on-the-influence-of-tom-of-finland}{%
\section{Eight Artists on the Influence of Tom of
Finland}\label{eight-artists-on-the-influence-of-tom-of-finland}}

Touko Valio Laaksonen, who would have been 100 this year,~transformed
depictions of queer eroticism in art through his hyper-real,
hypermasculine style.

\includegraphics{https://static01.nyt.com/images/2020/07/13/t-magazine/13tmag-tomoffinland-slide-Q1LS/13tmag-tomoffinland-slide-Q1LS-articleLarge.jpg?quality=75\&auto=webp\&disable=upscale}

By John Chiaverina

\begin{itemize}
\item
  July 23, 2020
\item
  \begin{itemize}
  \item
  \item
  \item
  \item
  \item
  \end{itemize}
\end{itemize}

\href{https://www.nytimes.com/2016/02/22/t-magazine/art/robert-mapplethorpe-tom-of-finland.html}{Tom
of Finland}'s influence is so vast that it can be hard to calculate.
Through sheer force of imagination, the artist was able to manifest a
hyper-real, hypermasculine style of queer erotic illustration that would
end up inspiring not just legions of visual artists but entire
subcultures. Any time a stylist puts a young pop star in a leather biker
cap for a magazine shoot, the impact of Tom of Finland is not far-off.

The year 2020 marks the centennial of the birth of this artist, born
Touko Valio Laaksonen, who died in 1991. Exhibitions, both virtual and
in reopened spaces, have been staged or planned in locales as far-flung
as
\href{https://www.nytimes.com/interactive/2015/08/20/t-magazine/los-angeles-city-guide.html}{Los
Angeles},
\href{https://www.nytimes.com/2019/02/15/t-magazine/london-wellness.html}{London},
Tokyo,
\href{https://www.nytimes.com/2019/02/26/t-magazine/paris-wellness.html}{Paris},
\href{https://www.nytimes.com/2018/08/23/t-magazine/berlin-guide.html}{Berlin}
and
\href{https://www.nytimes.com/2018/07/24/travel/tallinn-estonia-vilnius-lithuania-52-places.html}{Tallinn,
Estonia}. These commemorative shows are part of a larger, slower shift
over the past few decades, one that has seen the artist's leather-clad
figurative work recast more firmly into an institutionally approved art
cannon.

\href{https://www.nytimes.com/issue/t-magazine/2020/07/02/true-believers-art-issue}{\includegraphics{https://static01.nyt.com/newsgraphics/2020/06/29/tmag-art-embeds-new/assets/images/art_issue_gif_special_editon.gif}}

In 2013, the artist's work was included in a two-person exhibition with
the gay erotic art trailblazer --- and, as it happens, the originator of
Laaksonen's famous pseudonym --- \href{http://bobmizer.org/}{Bob Mizer}
at the \href{https://www.moca.org/}{Museum of Contemporary Art, Los
Angeles}. Tom of Finland drawings are in the permanent collection at the
\href{https://www.nytimes.com/topic/organization/museum-of-modern-art}{Museum
of Modern Art} in New York. Even Laaksonen's home country and namesake,
in which homosexuality was criminalized until 1971, has come around to
the artist's importance: In 2014, it made a series of stamps honoring
Tom of Finland, and a successful 2017 biopic was produced in the
country.

Image

Tom of Finland's ``Buddies'' (1973).Credit...Photo by Brian Forrest.
Courtesy of David Kordansky Gallery, Los Angeles.

Image

The artist's ``Untitled (From Kake Vol. 20 --- `Pleasure
Park').''Credit...Courtesy of the Tom of Finland Foundation, Los
Angeles, and David Kordansky Gallery, Los Angeles

But the artist's work has had a long road to wider acceptance. From a
young age, he took an interest in leather and uniforms --- particularly
those of local loggers and farmers --- which would become his primary
stylistic touchstone: Sailors flex and embrace in his work, and bikers
touch bulges. This early attraction was amplified during a stint **** in
the Finnish military, in which Laaksonen saw action in Finland's 1941
Continuation War against the U.S.S.R., which landed his country on the
wrong side of World War II history until it **** switched sides late in
1944, and later through the emergent biker subculture, inspired by
\href{https://www.nytimes.com/topic/person/marlon-brando}{Marlon Brando}
in the 1953 film ``The Wild One.'' (It should be noted that though the
uniforms of the German military were an influence on the artist,
Laaksonen was decidedly anti-racist.)

An initially secretive postwar art practice begun **** while the artist
was working a day job at an advertising agency developed into a career,
spurred on **** by a successful submission, in 1956, **** to Mizer's
magazine, Physique Pictorial, which had to be branded as a fitness
magazine as a cover, though that didn't always work (Mizer was charged
with obscenity in 1954). Early pieces published under the Tom of Finland
moniker were more suggestive than explicit, but the artist's work
evolved with the loosening of both legal and social constraints. Even
so, many of Laaksonen's later, more explicit drawings retained the
winking affability seen in his more formative work.

In 1978, Laaksonen made his first trip to Los Angeles, where he would
end up establishing the Tom of Finland Company with his muse and close
friend
\href{https://www.nytimes.com/slideshow/2016/02/23/t-magazine/inside-tom-house-tom-of-finland-in-los-angeles.html}{Durk
Dehner}, in order to fight rampant copyright infringement. That company
would expand into the nonprofit \href{https://www.tomoffinland.org/}{Tom
of Finland Foundation}, which to this day retains headquarters in an
Echo Park Craftsman house and continues to be an important community
hub.

\includegraphics{https://static01.nyt.com/images/2020/07/13/t-magazine/13tmag-tomoffinland-slide-3QNG/13tmag-tomoffinland-slide-3QNG-articleLarge.jpg?quality=75\&auto=webp\&disable=upscale}

Throughout this timeline, Tom of Finland has remained a quintessential
artist's artist. In the early 1960s, the pioneering, boundary-pushing
gay artist
\href{https://www.nytimes.com/2018/11/23/t-magazine/robert-mappelthorpe-michael-cunningham-elif-batuman-hilton-als.html}{Robert
Mapplethorpe}, according to
\href{https://www.nytimes.com/2015/10/07/t-magazine/patti-smith-m-train-objects.html}{Patti
Smith}, discovered Tom of Finland's work in a used bookstall in Times
Square. Mapplethorpe would become a crucial link in exposing Laaksonen's
work to the contemporary art world. Mapplethorpe attended Laaksonen's
debut San Francisco exhibition at the pioneering queer art gallery
Fey-Way Studios. Dehner facilitated the show, and Mapplethorpe's
enthusiasm helped the artist land an exhibition at Robert Samuel Gallery
in New York two years later.

In 1985, the artist
\href{https://www.nytimes.com/2017/03/08/t-magazine/art/mike-kelley-mobile-homestead.html}{Mike
Kelley} brought Tom of Finland to CalArts, the legendary Southern
California art school, to give a talk. In his introduction, Kelley
called Tom of Finland ``an incredible inspiration in my work.'' In
context, it was a bold statement. ``CalArts was steeped within the dogma
of conceptual art, and Tom, of course, was anything but that,'' the
gallerist
\href{https://tmagazine.blogs.nytimes.com/2014/09/10/david-kordansky-art-dealer-profile/}{David
Kordansky}, who represents Tom of Finland's work through the foundation,
says.

S.R. Sharp, who is the vice-president and curator at the Tom of Finland
Foundation, says artists like Kelley revered Tom because his art did
nothing less than offer permission to explore sexuality and explicit
imagery in their own work. ``And they always have remembered that,''
Sharp says. ``And they've carried his legacy for many, many, many
years.''

T talked to a wide range of artists about Tom of Finland's influence for
**** what would have been his 100th birthday.

Image

An installation view of ``The Collectors,'' curated by Elmgreen \&
Dragset, at the Danish and Nordic Pavilions during the 2009 Venice
Biennale.Credit...Photo by Anders Sune Berg

\hypertarget{elmgreen--dragset-berlin-based-artist-duo}{%
\subsubsection{\texorpdfstring{\textbf{\href{https://tmagazine.blogs.nytimes.com/2011/04/22/modern-amusement/}{Elmgreen
\& Dragset}, Berlin-based artist
duo}}{Elmgreen \& Dragset, Berlin-based artist duo}}\label{elmgreen--dragset-berlin-based-artist-duo}}

Tom of Finland's art is unabashedly gay and celebratory of a subculture
and sexual rituals that were considered perverse when his drawings first
appeared in public. It seems absolutely devoid of the Protestant
reservedness, darkness, angst and pietism that has otherwise affected
the Nordic culture. In spite of the depictions of rough sexual
practices, there is something almost innocent and sweet about Tom of
Finland's drawings, like it's all playacting. Seen in today's light, his
leather-clad muscle men don't seem that different from
\href{https://www.nytimes.com/2016/12/15/t-magazine/art/moomin-tove-jansson-adventures-moominland.html}{Tove
Jansson's Moomintroll} fantasy figures.

When we curated the Nordic Pavilion at the 53rd Venice Biennale in 2009,
we installed a whole wall with Tom of Finland drawings. Even at that
time his art was considered controversial. It's funny to think that only
a few years later Tom of Finland's drawings appeared on national stamps
and on bedsheets and cushion covers from the traditional Finnish textile
company Finlayson, founded in 1820.

\hypertarget{cassils-los-angeles-based-performance-artist}{%
\subsubsection{\texorpdfstring{\textbf{Cassils, Los Angeles-based
performance
artist}}{Cassils, Los Angeles-based performance artist}}\label{cassils-los-angeles-based-performance-artist}}

Tom of Finland's is probably the only artwork that I've ever jerked off
to. Those hot drawings scalded an impression onto my tender, young queer
brain fairly early on. I lived in Echo Park from 2009 to 2016. Visiting
Tom's house was a refuge; knowing that it is still care taken by his
former lover enacts and makes present a rich, deep history. Its like a
portal to the queer culture I always aspired to but has mostly been
erased these days by digital platforms and capitalism.

Image

A portion of the ceiling inside Tom of Finland's Los Angeles
home.Credit...Photography from the Rizzoli book ``Tom House: Tom of
Finland in Los Angeles'' (2016). Photo © Martyn Thompson

His formal mastery as a draftsman is really remarkable. You don't see
people with that kind of skill set anymore. Forget about the subject
matter, the ability to draw that well is a pleasure to witness. Also,
the absurdity of Tom's house as a living, breathing kinky institution: I
recall going there, and seeing this huge, huge butt plug holding open
the front door with this ancient Lab snoozing on the mat, and then,
looking up to the ceiling and instead of fixing a crack, they'd ****
hired a young queer artist,
\href{http://www.worldoftomoffinland.com/tomsblog/weekly-artist-focus-hector-silva/}{Hector
Silva}, to come in and paint a facade that's as if you're looking up
somebody's kilt. That incredible amount of detail and labor and
eroticism went into absolutely every part of his life.

For a long time, there was no language around transness, or folks that
were gender nonconforming or nonbinary. And I think, similarly, perhaps
when Tom of Finland was forging this iconic style, he really took
ownership over his definition of what it was to be a homosexual, which
was perhaps, at that time, a term that was viewed as weak or derogatory.
For him to manifest this totally fantastical, empowered erotic vision,
it was completely contrary to that. So, I think that aspect of his
imagining is something that has definitely influenced me as an artist,
in terms of me being able to understand and forge a possibility for
myself.

\hypertarget{john-waters-baltimore-based-filmmaker}{%
\subsubsection{\texorpdfstring{\textbf{\href{https://www.nytimes.com/2015/11/20/t-magazine/my-10-favorite-books-john-waters.html}{John
Waters}, Baltimore-based
filmmaker}}{John Waters, Baltimore-based filmmaker}}\label{john-waters-baltimore-based-filmmaker}}

\href{https://www.nytimes.com/2019/11/27/style/peter-berlin-the-70s-gay-sex-symbol-takes-new-york.html}{Peter
Berlin}, \href{https://www.kennethanger.org/}{Kenneth Anger}, Joe
Dallesandro, Jeff Stryker,
\href{https://www.nytimes.com/1971/07/09/archives/jim-morrison-25-lead-singer-with-doors-rock-group-dies.html}{Jim
Morrison},
\href{https://www.nytimes.com/2011/03/20/magazine/mag-20Bidgood-t.html}{James
Bidgood},
\href{https://www.nytimes.com/2013/12/01/us/a-first-gay-novel-a-poor-latino-boyhood-and-the-confluence.html}{John
Rechy}, even
\href{https://www.nytimes.com/topic/person/elvis-presley}{Elvis} and
\href{https://www.nytimes.com/topic/person/james-dean}{James Dean}. None
of them could have existed without Tom Of Finland's art coming first. He
took the word ``butch'' and turned it into a lifestyle. No, a reason to
live.

\hypertarget{richard-hawkins-los-angeles-based-artist}{%
\subsubsection{\texorpdfstring{\textbf{\href{https://tmagazine.blogs.nytimes.com/2010/05/27/seeing-things-richard-hawkinss-haunted-houses/}{Richard
Hawkins}, Los Angeles-based
artist}}{Richard Hawkins, Los Angeles-based artist}}\label{richard-hawkins-los-angeles-based-artist}}

Working for the Tom of Finland Company for several years, I was able to
see firsthand not only the breadth and amazing development of Tom's
characters and story lines but also how widely seen the work became ---
through distribution of Tom's own publications but, more important, hand
to hand and fan to fan. I take from that a very valuable lesson about
artistic practice: By pursuing and portraying the particulars of your
own personal desires --- as idiosyncratic, abhorrent, irresponsible or
far too subjective as your current situation may make it seem --- you
might just someday inspire the lives of others, many of whom may be
worlds and lifetimes away.

Image

G.B. Jones's ``Girls Who Are Fans of Sailor Moon''
(1996).Credit...Courtesy of the artist and Cooper Cole Gallery, Toronto

Image

Jones's ``Tom Girls Go West'' (2001).Credit...Courtesy of the artist and
Cooper Cole Gallery, Toronto

\hypertarget{catherine-opie-los-angeles-based-photographer}{%
\subsubsection{\texorpdfstring{\textbf{\href{https://www.nytimes.com/2019/10/02/t-magazine/catherine-opie.html}{Catherine
Opie}, Los Angeles-based
photographer}}{Catherine Opie, Los Angeles-based photographer}}\label{catherine-opie-los-angeles-based-photographer}}

As a longtime Angeleno and definitely someone who has been a part of a
larger queer leather community here, I know how important Tom of Finland
was in terms of brotherhood. So even though it wasn't necessarily for
me, Tom's house always provided an amazing community resource. But for
me as a dyke, I could not find myself in Tom of Finland's work beyond
drag.

In a certain way, there was always a position of separatism with the
leather men compared to the leather dykes. Which is why I'm so
interested in the influence that Tom of Finland had on {[}the Canadian
artist and publisher{]}
\href{https://coopercolegallery.com/artist/g-b-jones/}{G. B. Jones}. For
the first time within G. B. Jones's zines, in which she adopted the
style of Tom of Finland, I was able to see my own community and my own
self, versus the fantasies that many of us carried of being leather
daddies.

Tom of Finland, what he modeled for us in his drawings, was actually a
butch drag. We ended up adopting this --- it was a way for us to do drag
as a community. But G. B. Jones, with her drawings, all of a sudden made
it part of our queer culture --- we could think of ourselves as being
women and leather dykes versus just doing drag.

\hypertarget{simon-haas-of-the-los-angeles-based-artist-duo-the-haas-brothers}{%
\subsubsection{\texorpdfstring{\textbf{Simon Haas, of the Los
Angeles-based artist duo the}
\textbf{\href{http://www.thehaasbrothers.com/}{Haas
Brothers}}}{Simon Haas, of the Los Angeles-based artist duo the Haas Brothers}}\label{simon-haas-of-the-los-angeles-based-artist-duo-the-haas-brothers}}

My college boyfriend gave me a Tom of Finland Kake comic for my 21st
birthday, when I was studying at the Rhode Island School of Design. I
was a recently out-of-the-closet painting student filled with angst
about my sexuality and my art, and this was my first exposure to art
that made me feel like I belonged. Tom of Finland's deft pencil work and
the immediate eroticism are enough to make any young gay boy a quick
fan, but after a decade of looking at his drawings, I understand that
his work transcends pornography and occupies a space of queer
spirituality. I came for the giant phalluses and stayed for the joy of
being a gay person. Tom's drawings are unapologetically happy and have
not a shred of shame in them --- an incredible rarity in any depiction
of homosexuality, even now. Tom had the fortitude of spirit to celebrate
men at \emph{play} at a time when most of the world considered gay
people to be an abomination. I am 64 years his junior, and I have yet to
discover within myself the kind of fearless happiness that Tom
manifested in his work. Tom had such an abundance of radical
self-acceptance that his work continues to impart the spirit of
self-love onto gay men everywhere. I will never know Tom, but I can
sincerely say that I love him with all my heart.

Image

Tom of Finland's ``Sailor's Dream'' (1959).Credit...Photo by Brian
Forrest.~Courtesy of David Kordansky Gallery, Los Angeles

\hypertarget{brontez-purnell-oakland-calif-based-writer-and-artist}{%
\subsubsection{\texorpdfstring{\textbf{\href{https://www.whiting.org/awards/winners/brontez-purnell}{Brontez
Purnell}, Oakland, Calif.-based writer and
artist}}{Brontez Purnell, Oakland, Calif.-based writer and artist}}\label{brontez-purnell-oakland-calif-based-writer-and-artist}}

We kind of take Tom of Finland for granted, because, let's be honest, as
gay men, do we really need any more images of super muscular white
dudes? No, of course not. But, also, he was an excellent portraitist,
probably the last of the greatest of them, in a world where the camera
has become omni-accessible.

Also, when he was creating, I don't think anybody really understood how
out of vogue or how hyper-questioned hypermasculinity would become. But
the thing that was absolutely radical was that he was doing this in the
'40s. When you do the residency {[}Purnell was a resident at the Tom of
Finland Foundation in 2019{]}, you get access to his room, and I saw
drawings from when he was like 8 years old, and he's doing these little
comics about cops and robbers. So, he was definitely all about dudes in
uniforms.

Image

Tom of Finland's ``The New Biker Stud'' (1969).Credit...Courtesy of the
Tom of Finland Foundation, Los Angeles, and David Kordansky Gallery, Los
Angeles

When I was the artist in residence there, it says Tom of Finland, so you
would expect a bunch of German dudes in leather, but it was a pretty
diverse group of people. The people who run that organization are very,
very near and dear to me. I think they still have a very, very deep and
intentional hand in L.A. queer radical art. The month I was there, I saw
that house be a welcoming spot for so many different people --- so many
walks of queer life.

It's a thing that I think is seemingly dead in San Francisco --- the
house is maybe the last bastion of the radical, queer, underground
meeting place. But also through the filter of these still amazing
drawings. With Tom of Finland, it's important to be able to place him in
his time period. He was definitely doing something that was going to get
his ass killed, but he said, ``This is my art. This is the type of
beauty I want to enact in the world,'' and there is no way to not be in
awe of that.

Image

Tom of Finland's ``Untitled (Preparatory Drawing for Kake Vol. 16 ---
`Sex on the Train')'' (1974).Credit...Courtesy of David Kordansky
Gallery, Los Angeles

\hypertarget{tom-bianchi-palm-springs-calif-based-photographer}{%
\subsubsection{\texorpdfstring{\textbf{\href{https://tmagazine.blogs.nytimes.com/2013/05/02/feeling-for-fire-island-memories/}{Tom
Bianchi}, Palm Springs, Calif.-based
photographer}}{Tom Bianchi, Palm Springs, Calif.-based photographer}}\label{tom-bianchi-palm-springs-calif-based-photographer}}

In the pocket-size Physique Pictorial magazines, I first saw Tom of
Finland drawings. As a frustrated, horny adolescent thinking myself
alone in my perverse desires, I reveled in his mind-blowing sex
fantasies. But I never thought those men or what they were up to could
be real. In the late '50s and early '60s, few men had the physiques he
idealized. And I saw no evidence that Tom's world existed beyond his
imagination. But that didn't stop me from joining in his adventures with
my dick in hand. Years later, I learned that Tom drew with one hand and
held his dick in the other. That revelation speaks to the authenticity
of his art.

Tom wanted us to feel the charge of his desires in our loins. I agree
with
\href{https://www.nytimes.com/1987/05/14/obituaries/richard-ellmann-dies-at-69-eminent-james-joyce-scholar.html}{Richard
Ellmann}'s observations on
\href{https://www.nytimes.com/2018/10/02/t-magazine/oscar-wilde-temple-london.html}{Oscar
Wilde}, that life would repeat itself tediously were it not for the
daemonic changes art forces upon it. I also agree with Ellmann's idea
that the artist makes models of experience that people rush to try out.
Tom of Finland perfectly exemplifies this observation. Before Tom, the
homosexual stereotype was a narrowly limited negative one. But Tom
opened a door to an alternate, robust way of being queer. He invited us
to dress and play with hypermasculine images of ourselves and
illustrated myriad sexual adventures we could realize. Tom of Finland
expanded our vision of what was possible for us to experience. How many
artists' work has changed our culture so profoundly? That we celebrate
Tom's 100th birthday today is a testament to the transformative power of
his work.

\emph{These interviews have been edited and condensed.}

\hypertarget{true-believers-art-issue}{%
\subsubsection{\texorpdfstring{\href{https://www.nytimes.com/issue/t-magazine/2020/07/02/true-believers-art-issue}{True
Believers Art
Issue}}{True Believers Art Issue}}\label{true-believers-art-issue}}

Advertisement

\protect\hyperlink{after-bottom}{Continue reading the main story}

\hypertarget{site-index}{%
\subsection{Site Index}\label{site-index}}

\hypertarget{site-information-navigation}{%
\subsection{Site Information
Navigation}\label{site-information-navigation}}

\begin{itemize}
\tightlist
\item
  \href{https://help.nytimes.com/hc/en-us/articles/115014792127-Copyright-notice}{©~2020~The
  New York Times Company}
\end{itemize}

\begin{itemize}
\tightlist
\item
  \href{https://www.nytco.com/}{NYTCo}
\item
  \href{https://help.nytimes.com/hc/en-us/articles/115015385887-Contact-Us}{Contact
  Us}
\item
  \href{https://www.nytco.com/careers/}{Work with us}
\item
  \href{https://nytmediakit.com/}{Advertise}
\item
  \href{http://www.tbrandstudio.com/}{T Brand Studio}
\item
  \href{https://www.nytimes.com/privacy/cookie-policy\#how-do-i-manage-trackers}{Your
  Ad Choices}
\item
  \href{https://www.nytimes.com/privacy}{Privacy}
\item
  \href{https://help.nytimes.com/hc/en-us/articles/115014893428-Terms-of-service}{Terms
  of Service}
\item
  \href{https://help.nytimes.com/hc/en-us/articles/115014893968-Terms-of-sale}{Terms
  of Sale}
\item
  \href{https://spiderbites.nytimes.com}{Site Map}
\item
  \href{https://help.nytimes.com/hc/en-us}{Help}
\item
  \href{https://www.nytimes.com/subscription?campaignId=37WXW}{Subscriptions}
\end{itemize}
