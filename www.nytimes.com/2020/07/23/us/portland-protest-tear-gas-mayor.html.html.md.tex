Sections

SEARCH

\protect\hyperlink{site-content}{Skip to
content}\protect\hyperlink{site-index}{Skip to site index}

\href{https://www.nytimes.com/section/us}{U.S.}

\href{https://myaccount.nytimes.com/auth/login?response_type=cookie\&client_id=vi}{}

\href{https://www.nytimes.com/section/todayspaper}{Today's Paper}

\href{/section/us}{U.S.}\textbar{}Federal Agents Envelop Portland
Protest, and City's Mayor, in Tear Gas

\url{https://nyti.ms/2WMX1z0}

\begin{itemize}
\item
\item
\item
\item
\item
\item
\end{itemize}

\href{https://www.nytimes.com/news-event/george-floyd-protests-minneapolis-new-york-los-angeles?action=click\&pgtype=Article\&state=default\&region=TOP_BANNER\&context=storylines_menu}{Race
and America}

\begin{itemize}
\tightlist
\item
  \href{https://www.nytimes.com/2020/07/26/us/protests-portland-seattle-trump.html?action=click\&pgtype=Article\&state=default\&region=TOP_BANNER\&context=storylines_menu}{Protesters
  Return to Other Cities}
\item
  \href{https://www.nytimes.com/2020/07/24/us/portland-oregon-protests-white-race.html?action=click\&pgtype=Article\&state=default\&region=TOP_BANNER\&context=storylines_menu}{Portland
  at the Center}
\item
  \href{https://www.nytimes.com/2020/07/23/podcasts/the-daily/portland-protests.html?action=click\&pgtype=Article\&state=default\&region=TOP_BANNER\&context=storylines_menu}{Podcast:
  Showdown in Portland}
\item
  \href{https://www.nytimes.com/interactive/2020/07/16/us/black-lives-matter-protests-louisville-breonna-taylor.html?action=click\&pgtype=Article\&state=default\&region=TOP_BANNER\&context=storylines_menu}{45
  Days in Louisville}
\end{itemize}

Advertisement

\protect\hyperlink{after-top}{Continue reading the main story}

Supported by

\protect\hyperlink{after-sponsor}{Continue reading the main story}

\hypertarget{federal-agents-envelop-portland-protest-and-citys-mayor-in-tear-gas}{%
\section{Federal Agents Envelop Portland Protest, and City's Mayor, in
Tear
Gas}\label{federal-agents-envelop-portland-protest-and-citys-mayor-in-tear-gas}}

Mayor Ted Wheeler denounced federal agents for an ``egregious
overreaction,'' and official reviews were started in Washington, D.C.

\includegraphics{https://static01.nyt.com/images/2020/07/24/world/23unrest-portland2-sub/23unrest-portland2-sub-videoSixteenByNine3000-v2.jpg}

\href{https://www.nytimes.com/by/mike-baker}{\includegraphics{https://static01.nyt.com/images/2020/05/19/reader-center/author-mike-baker/author-mike-baker-thumbLarge.png}}

By \href{https://www.nytimes.com/by/mike-baker}{Mike Baker}

\begin{itemize}
\item
  Published July 23, 2020Updated July 25, 2020
\item
  \begin{itemize}
  \item
  \item
  \item
  \item
  \item
  \item
  \end{itemize}
\end{itemize}

PORTLAND, Ore. --- The mayor of Portland, Ted Wheeler, was left coughing
and wincing in the middle of his own city on Wednesday night after
federal officers deployed tear gas into a crowd of protesters that Mr.
Wheeler had joined outside the federal courthouse.

Mr. Wheeler, who scrambled to put on goggles while denouncing what he
called the ``urban warfare'' tactic of the federal agents in Oregon,
said that he was outraged by the use of tear gas and that it only made
protesters more angry.

``I'm not going to lie --- it stings; it's hard to breathe,''
\href{https://twitter.com/ByMikeBaker/status/1286186731763412993}{Mr.
Wheeler said.} ``And I can tell you with 100 percent honesty, I saw
nothing which provoked this response.''

He called it an ``egregious overreaction'' on the part of the federal
officers, and not a de-escalation strategy.

``It's got to stop now,'' he declared.

But the Democratic mayor, 57, has also long been the target of
\href{https://www.nytimes.com/article/portland-protests-explained-protesters.html}{Portland
protesters} infuriated by the city police's own use of tear gas, which
was persistent until a federal judge ordered the city to use it only
when there was a safety issue. As Mr. Wheeler went through the crowd,
some threw objects in his direction and others called for his
resignation, chanting, ``Tear Gas Teddy.''

Mr. Wheeler joined the crowd at the front of the protest, against a
barrier
\href{https://www.nytimes.com/2020/07/22/us/portland-protests-courthouse.html}{around
the federal courthouse}. As a first volley of tear gas wafted over them,
Mr. Wheeler stayed put, watching the actions of federal agents.

\href{https://twitter.com/ByMikeBaker/status/1286194497559252992}{After
another large wave of tear gas sent Mr. Wheeler away from the scene},
some protesters mocked him, asking how it felt. Mr. Wheeler said that
joining the protesters at the front of the line was just one way he was
going to try to rid the city of the federal tactical teams.

``A lot of these people hate my guts,'' Mr. Wheeler said in an
interview, looking around at the crowd. But he said they were unified in
wanting federal officers gone.

The mayor
\href{https://www.nytimes.com/2020/07/21/us/portland-protests.html}{has
called for federal agents to leave the city} after they arrived to
subdue long-running unrest. Dressed in camouflage and tactical gear and
unleashing tear gas, federal officers have clashed violently with
protesters and pulled some people into unmarked vans in
\href{https://www.nytimes.com/2020/07/17/us/portland-protests.html}{what
Gov. Kate Brown called ``a blatant abuse of power.''}

Customs and Border Protection sent
\href{https://www.nytimes.com/2020/07/23/us/seattle-federal-agents-police.html}{tactical
border officers to Seattle} on Thursday in case they were needed to help
control protests expected this weekend, the Federal Protective Service
said. The city's mayor, Jenny Durkan, said it did not need the help of
federal agents.

The inspectors general of the Departments of Justice and Homeland
Security also announced the start of an investigation into the actions
of federal law enforcement agents responding to protests and civil
unrest in Portland and Washington, D.C., after the police killing of
George Floyd in Minneapolis.

\href{https://twitter.com/DanielStrauss4/status/1286375148115963904}{In
a letter to Congress and a public statement}, Michael E. Horowitz, the
Justice Department's inspector general, said he and his counterpart
would scrutinize the actions of law enforcement officials who cleared
\href{https://www.nytimes.com/2020/07/28/us/politics/lafayette-square-park-police-protests.html}{Lafayette
Square}, outside the White House, on June 1. Federal agents violently
pushed protesters out of the square shortly before President Trump
walked across it for a photo op in front of a church.

\includegraphics{https://static01.nyt.com/images/2020/07/23/us/23unrest-portland02/23unrest-portland02-articleLarge.jpg?quality=75\&auto=webp\&disable=upscale}

Image

Federal agents shot tear gas at protesters early
Thursday.Credit...Octavio Jones for The New York Times

Mr. Horowitz and his counterpart at the Department of Homeland Security,
Joseph V. Cuffari, said they would also examine the actions of agents in
camouflage uniforms who have confronted protesters outside the federal
courthouse in Portland. A video of two such agents seizing a protester
and taking him away in an unmarked vehicle has sparked a furor.

Mr. Horowitz said he and his colleagues were undertaking the inquiries
in response to requests by lawmakers, complaints his office had
received, and a referral by the United States attorney in Oregon, Billy
J. Williams.

``The review will include examining the training and instruction that
was provided to the DOJ law enforcement personnel; compliance with
applicable identification requirements, rules of engagement, and legal
authorities; and adherence to DOJ policies regarding the use of
less-lethal munitions, chemical agents, and other uses of force,'' Mr.
Horowitz wrote.

On the streets of Portland, some demonstrators called the mayor's
arrival at the protest scene a photo op. Sean Smith, who has been at the
protests for weeks, said Mr. Wheeler, who also serves as police
commissioner, needed to take more action to control the Portland Police
Bureau and align with protesters.

``He should probably be out here every night,'' Mr. Smith said.

Kenneth T. Cuccinelli II, the acting deputy secretary of homeland
security, mocked Mr. Wheeler for his appearance at the protest,
\href{https://twitter.com/HomelandKen/status/1286361863433879552?s=20}{saying
on Twitter} that Mr. Wheeler had the choice to ``stand with the rioters
or stand with the rule of law.''

``He chose to stand with the rioters and STILL can't appease them,'' Mr.
Cuccinelli wrote.

By early Thursday morning, with protesters still outside both the
federal courthouse and the county justice center across the street,
federal agents continued deploying tear gas; the Portland police
repeatedly warned that city officers might also use it but did not do
so.

Image

Protesters tried to help a man who had been tear-gassed.Credit...Mason
Trinca for The New York Times

Image

Fires burned outside of the courthouse.Credit...Octavio Jones for The
New York Times

The demonstrations, fueled by a wide array of grievances, including
against police brutality, have rocked Portland for 55 consecutive nights
and counting, persisting even as other protests across the country have
waned.

The city has become a target of Mr. Trump,
\href{https://www.nytimes.com/2020/07/21/us/politics/trump-portland-federal-agents.html}{who
has embraced a law-and-order message in his re-election campaign}. While
federal officers were deployed to Portland to purportedly quell unrest
and protect federal property, their arrival has only galvanized the
movement, with the numbers of protesters each night swelling into the
thousands.

On Wednesday night and into Thursday morning, protesters gathered around
a temporary fence that federal officers had erected during the day. They
shot fireworks at the building, and some breached the fence. Federal
officers, wearing camouflage and tactical gear, emerged to fire tear gas
and less-lethal munitions, and to arrest those who breached the fence.

While the Trump administration has labeled the protesters ``violent
anarchists,'' Mr. Wheeler decided to go into the crowd on Wednesday
night for what he deemed a ``listening session.'' Even though people
followed and cursed him, he spent three hours there.

At times he was jeered, such as when he told the crowd he would not
promise to abolish the Police Department. Other times, he drew cheers,
such as when he demanded that the federal government ``stop occupying
our city.''

``If they launch the tear gas against you,'' Mr. Wheeler said, ``they
are launching the tear gas against me.''

Charlie Savage contributed reporting from Washington.

Advertisement

\protect\hyperlink{after-bottom}{Continue reading the main story}

\hypertarget{site-index}{%
\subsection{Site Index}\label{site-index}}

\hypertarget{site-information-navigation}{%
\subsection{Site Information
Navigation}\label{site-information-navigation}}

\begin{itemize}
\tightlist
\item
  \href{https://help.nytimes.com/hc/en-us/articles/115014792127-Copyright-notice}{©~2020~The
  New York Times Company}
\end{itemize}

\begin{itemize}
\tightlist
\item
  \href{https://www.nytco.com/}{NYTCo}
\item
  \href{https://help.nytimes.com/hc/en-us/articles/115015385887-Contact-Us}{Contact
  Us}
\item
  \href{https://www.nytco.com/careers/}{Work with us}
\item
  \href{https://nytmediakit.com/}{Advertise}
\item
  \href{http://www.tbrandstudio.com/}{T Brand Studio}
\item
  \href{https://www.nytimes.com/privacy/cookie-policy\#how-do-i-manage-trackers}{Your
  Ad Choices}
\item
  \href{https://www.nytimes.com/privacy}{Privacy}
\item
  \href{https://help.nytimes.com/hc/en-us/articles/115014893428-Terms-of-service}{Terms
  of Service}
\item
  \href{https://help.nytimes.com/hc/en-us/articles/115014893968-Terms-of-sale}{Terms
  of Sale}
\item
  \href{https://spiderbites.nytimes.com}{Site Map}
\item
  \href{https://help.nytimes.com/hc/en-us}{Help}
\item
  \href{https://www.nytimes.com/subscription?campaignId=37WXW}{Subscriptions}
\end{itemize}
