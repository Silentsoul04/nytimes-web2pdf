Sections

SEARCH

\protect\hyperlink{site-content}{Skip to
content}\protect\hyperlink{site-index}{Skip to site index}

\href{https://www.nytimes.com/section/opinion/sunday}{Sunday Review}

\href{https://myaccount.nytimes.com/auth/login?response_type=cookie\&client_id=vi}{}

\href{https://www.nytimes.com/section/todayspaper}{Today's Paper}

\href{/section/opinion/sunday}{Sunday Review}\textbar{}Just How Far Will
Joe Biden Go?

\url{https://nyti.ms/3jFpWz1}

\begin{itemize}
\item
\item
\item
\item
\item
\item
\end{itemize}

Advertisement

\protect\hyperlink{after-top}{Continue reading the main story}

\href{/section/opinion}{Opinion}

Supported by

\protect\hyperlink{after-sponsor}{Continue reading the main story}

\hypertarget{just-how-far-will-joe-biden-go}{%
\section{Just How Far Will Joe Biden
Go?}\label{just-how-far-will-joe-biden-go}}

He's not listening to Twitter. So who does have the candidate's ear?

\href{https://www.nytimes.com/by/michelle-cottle}{\includegraphics{https://static01.nyt.com/images/2019/06/25/opinion/michelle-cottle-circular/michelle-cottle-circular-thumbLarge-v2.png}}

By \href{https://www.nytimes.com/by/michelle-cottle}{Michelle Cottle}

Ms. Cottle is a member of the
\href{https://www.nytimes.com/interactive/2018/opinion/editorialboard.html}{editorial
board}.

\begin{itemize}
\item
  July 24, 2020
\item
  \begin{itemize}
  \item
  \item
  \item
  \item
  \item
  \item
  \end{itemize}
\end{itemize}

\includegraphics{https://static01.nyt.com/images/2020/07/26/opinion/sunday/26CottleCover/26CottleCover-articleLarge.jpg?quality=75\&auto=webp\&disable=upscale}

For months, President Trump's re-election team has been
\href{https://apnews.com/698d634366a78b7197a14784605c76a8}{trying out}
possible
\href{https://www.theatlantic.com/ideas/archive/2020/06/trump-cant-figure-out-how-attack-biden/613402/}{lines
of attack} against
\href{https://www.nytimes.com/2020/07/24/business/joe-biden-stocks-taxes.html}{Joe
Biden}: He's sleepy. He's creepy. He's corrupt. He's soft on China. He's
soft in the head. So far, nothing has stuck. Which may explain why one
theme Mr. Trump and his supporters have latched onto is that, whatever
you think of Mr. Biden personally, he is ``a helpless puppet of the
radical left.''

To which the radical left would surely respond: If only!

As even Mr. Trump admits, the former vice president is no progressive
revolutionary. The Democratic Party's activist base, especially its
younger members, harbors grave doubts about Mr. Biden and has vowed to
keep the pressure on as he charts a path forward. One big, basic
question on many people's minds is, Just how far left will Joe go?

Looking to get a sense of how Mr. Biden's governing vision is shaping
up, I spent several weeks talking with his advisers, his allies, his
critics and other party players. I wanted to know how the rolling crises
have, for instance, impacted his search for the perfect running mate ---
the big reveal of which is expected any day now! --- as well as how
various policy proposals are being revised and expanded.

It was clear that, fundamentally, Joe is gonna be Joe. But he recognizes
the need to respond to all the turbulence --- and if there's one thing
Team Biden has a surfeit of, it's people looking to influence how he
does that.

``I've really never been in a campaign where so many people every day
are reaching out to me with offers of assistance, advice, input,
suggestions about everything,'' said Ron Klain, who served as Mr.
Biden's chief of staff when he was vice president.

``Everybody wants to win,'' said Representative Cedric Richmond of
Louisiana, one of the campaign's co-chairmen. ``And everybody wants to
give their ideas on what they think it takes to win.''

All that input has its downsides. ``At some point you have to be able to
make a decision and execute a strategy,'' said Representative Richmond.
He credits the nominee's tight inner circle for keeping the campaign on
track. ``You just can't have a million coaches.''

Some close to Mr. Biden have adopted a more absolutist approach. ``I get
letters and telephone calls from people saying, `This is what Biden
needs to do,' or, `This is what you need to tell Biden to do,''' said
Representative Jim Clyburn, the South Carolina Democrat, chuckling. ``I
don't tell him any of it.''

\hypertarget{ii}{%
\subsubsection{II.}\label{ii}}

Presidential campaigns are always intense. This cycle, with a sitting
president as incendiary as Donald Trump, the Democrats' desperation to
reclaim the White House is at Level 11. This was the case even before
the nation got hit with the triple whammy of a pandemic, an economic
meltdown and nationwide protests over racial injustice. America is on
edge, and many Democrats are jittery --- read: panicked --- about
whether their champion can meet the moment.

The Democratic presidential field started out sprawling. But the final
decision boiled down to whether voters preferred a left-wing
revolutionary (Senator Bernie Sanders) or an ideological and
dispositional moderate (Mr. Biden). Not that Mr. Biden doesn't share
progressive goals such as achieving universal health care or combating
climate change --- his campaign \emph{hates} it when people call him a
centrist --- but his vision for how to achieve them is more evolutionary
than revolutionary.

For all the passion on the party's left flank, the milder Biden won out.
On April 8, Mr. Sanders
\href{https://www.nytimes.com/2020/04/08/us/politics/bernie-sanders-drops-out.html}{suspended
his campaign}.

But the end of the primary did not end the tug of war over the direction
of the party.

Progressives in particular see the current turmoil as proof that bold
change is essential, and they were buoyed when Mr. Biden recently said
that the times called for an ambitious, F.D.R.-size response.

``He seems to be recognizing that, in the midst of Covid-19, simply
going back to normal, which was his original orientation, is
insufficient,'' said Maurice Mitchell, the national director of the
Working Families Party. ``This is actually an opportunity for a
transformational agenda.''

The question, of course, is what exactly this transformational agenda
will look like --- and who will get to shape it.

Skepticism about Mr. Biden runs deep on the left. During more than four
decades in public office, he earned a reputation as a pragmatic centrist
(sorry!) --- the guy President Obama sent to negotiate deals with
congressional Republicans that no one else wanted to be in the room
with. Some progressives regard him as just the sort of compromised,
compromising, politics-as-usual establishment tool standing in the way
of meaningful change, and they fear that he has surrounded himself with
other establishment tools who see the activist base as a threat to the
existing power structure that must be neutralized.

``There's a whole wing of the Democratic Party establishment that
doesn't simply want an electoral victory,'' they want it on terms that
let them ``weave a narrative'' to discredit the left, said Mr. Mitchell.
``They want to defeat Trump and progressives in one fell swoop.''

This may be an overstatement. But plenty of Biden allies clearly weren't
sorry to have the party's revolutionary wing taken down a notch in the
primaries. They consider it a vindication of the nominee's politics and
persona --- and a thumb in the eye of all the cool kids who said he
could never win. ``If the Biden campaign had bent to the collective
political advice of Twitter, it would have done a lot of things
differently and probably would not have been successful,'' said Mr.
Klain. ``In the end, he's run this campaign on these instincts, his
judgment and his experience.''

So he's not listening to Twitter. Who \emph{is} he listening to?

\hypertarget{iii}{%
\subsubsection{III.}\label{iii}}

As the saying goes: Personnel is policy. But the campaign has been cagey
about who is advising it and how the policy sausage gets made. Members
of its extended economics team, for instance, were ordered to keep quiet
about their campaign work. They can tell friends and colleagues,
according to a
\href{https://www.nytimes.com/2020/06/11/us/politics/joe-biden-campaign-economy.html}{memo}
acquired by The Times, but should not mention their affiliation ``on
social media such as Facebook or LinkedIn or in your professional bio.''
And they should steer clear of the news media. Period.

Some names have trickled out. Progressives are not happy that Rahm
Emanuel, the former White House chief of staff/congressman/mayor of
Chicago is advising the campaign on economic policy and political
strategy. (The left's
\href{https://www.salon.com/2015/03/23/why_the_left_hates_this_man_rahm_emanuels_sins_against_the_progressive_movement/}{grievance
list} against this former Clintonite is long, and his mayoral tenure was
marred by serious police scandals, including the 2014 shooting of Laquan
McDonald, which prompted protests and an investigation by the Justice
Department.) ``Not the sign we want to see,'' said Rahna Epting, the
executive director of the grass roots group MoveOn.

Even more explosive was the
\href{https://www.bloomberg.com/news/articles/2020-04-23/larry-summers-advising-biden-campaign-on-economic-recovery}{April
news} that Lawrence Summers has been offering his economic insights. A
veteran of the Clinton and Obama White Houses, Mr. Summers is viewed as
a neoliberal, business-cozy monster by the left, his name invoked with a
level of distaste normally reserved for child predators.

In early May, more than two dozen progressive groups sent an
\href{https://assets.bwbx.io/documents/users/iqjWHBFdfxIU/rq88fcCcNRP8/v0}{open
letter} to Mr. Biden, demanding that he remove Mr. Summers from any
campaign advisory role and ``exclude him from a future Biden
administration.'' Charging that Mr. Summers had ``put the interests of
large corporations ahead of working families in the United States and
around the world, fueled the climate crisis, and undermined efforts to
ensure gender equality,'' they declared it ``hard to imagine a worse
person than Larry Summers to guide the next President toward an economy
that works for all.''

The Biden campaign has met such criticisms with assurances that it is
listening to a wide range of voices.

With Mr. Biden having spent the last half-century collecting friends,
aides and advisers, not to mention this campaign's fast-growing official
staff, the org chart for Team Biden can be hard to decipher. His inner
circle is defined differently depending on whom you ask, and even
reasonably senior staffers aren't always clear about who does what. But
whether you think in terms of concentric circles or Venn diagrams or
pyramids of power, there are legions of people offering counsel.

For instance,
\href{https://www.nytimes.com/2020/06/11/us/politics/joe-biden-campaign-economy.html}{the
campaign is consulting} with more than 100 left-leaning experts on
economic policy. The nominee's regular briefings are conducted by a
smaller core of liberal economists, former Obama officials and advisers
to Hillary Clinton's 2016 campaign.

On foreign policy, the nominee has a large
\href{https://www.theatlantic.com/ideas/archive/2020/05/bidens-grand-ambitions-dont-extend-foreign-policy/611863/}{network
of working groups} subdivided according to specialty: nuclear
proliferation, the Middle East, China, etc. Who is running these groups,
and how much real influence they have, is hard to pin down. For all Mr.
Trump's ravings about China, international matters typically receive
less play in presidential races than do domestic issues such as jobs or
health care --- meaning the Biden campaign is facing relatively little
leftward pressure. When Mr. Biden and Mr. Sanders formed a collection of
working groups in the spring to hammer out joint proposals on various
policy issues, foreign policy was not even among the topics tackled.

This likely suits Mr. Biden just fine. Foreign policy is kind of his
thing. His expertise runs deep. He knows the players and the issues. As
vice president, his instincts were more cautious and minimalist than
those of Secretary of State Hillary Clinton. The Times once
\href{https://www.nytimes.com/2015/10/10/us/politics/a-biden-run-would-expose-foreign-policy-differences-with-hillary-clinton.html}{described}
the two as representing ``the yin and the yang of Mr. Obama's foreign
policy.''

But, in this as in so many areas, Mr. Biden is a solidly establishment
player, and he relies on a clutch of trusted hands, including Julie
Smith, Tom Donilon and Tony Blinken, who sits atop the campaign's
foreign policy shop. Mr. Blinken has been with Mr. Biden for nearly two
decades and served as his national security adviser in the Obama White
House.

\href{https://www.theatlantic.com/ideas/archive/2020/05/bidens-grand-ambitions-dont-extend-foreign-policy/611863/}{Don't
expect} his team to be taking on the military-industrial complex or
taking up
\href{https://foreignpolicy.com/2020/05/11/biden-left-leaning-groups-slash-pentagon-budget/}{calls
to slash funding for the Pentagon}. The nominee's message thus far has
been mainstream and soothing, with talk of rebuilding frayed alliances
and restoring American leadership on issues ranging from nuclear arms to
the Middle East to global warming.

Other top policy dogs: Stef Feldman is the campaign's official policy
director, while Jake Sullivan serves as a combination gatekeeper and air
traffic controller, gathering input, coordinating info and bringing
order to the chaos across fields and working groups. Bruce Reed, one of
Mr. Biden's chiefs of staff in the Obama White House and a former head
of the now-defunct centrist Democratic Leadership Council, also plays a
central advisory role. (He used to brief Mr. Biden on campaign trips ---
in the pre-Covid days when people could still travel.)

\hypertarget{iv}{%
\subsubsection{IV.}\label{iv}}

Many of those with the most influence operate outside any official lines
of authority. Mr. Biden's inner circle includes longtime loyalists like
Mr. Klain; Mike Donilon (brother of the aforementioned Tom), Mr. Biden's
political guru; Steve Ricchetti, who was another of his chiefs of staff
in the Obama administration, and Ted Kaufman, who has been with Mr.
Biden since his 1972 Senate race. These are the kitchen cabinet folks
who make progressives super nervous. They are considered establishment
fogies unlikely to challenge the nominee or push him to think big.

The inner ranks are not entirely closed to newcomers. Anita Dunn, a
veteran of Obamaworld,
\href{https://www.nytimes.com/2020/02/07/us/politics/joe-biden-anita-dunn.html}{effectively
took control} of Mr. Biden's primary campaign in the shake-up following
his loss in Iowa, and continues to wield serious clout. But Ms. Dunn is
herself a Washington fixture and an object of suspicion for some on the
left.

``He's not listening to the folks he needs to listen to,'' said Yvette
Simpson, who leads the political action committee Democracy for America.

In some cases, these innermost insiders take on specific tasks. Mr.
Klain is the point-person on debate prep. When there is an important
speech to be given, Mr. Biden huddles up with Mike Donilon. But, more
important, they're around to provide general support and counsel. These
days, that tends to mean lots of video meetings and phone calls ---
\emph{so many} virtual meetings, say team members. There are still only
three people staffing Mr. Biden's Delaware home. For the most part,
campaign business, including senior staff meetings, continues to be
conducted remotely.

When it comes to political strategy, Mike Donilon is Joe's go-to guy. He
has been with Mr. Biden since the early 1980s and has been called Mr.
Biden's ``alter ego.'' He was central in helping the vice president
explore --- and ultimately opt against --- running for the White House
in 2016. In ``Promise Me, Dad,'' Mr. Biden's 2017 book about his son
Beau's battle with brain cancer, he recalls Mr. Donilon studying his
face one night, not quite five months after Beau's death, and realizing
that the vice president wasn't up to a campaign. ``I don't think you
should do this,'' he told Mr. Biden, who announced his decision not to
run the next day.

Mr. Donilon was also
\href{https://www.politico.com/news/magazine/2019/12/19/biden-2020-campaign-president-advisers-087410}{the
brains behind} the current campaign's core message that this is a battle
for ``the soul of America'' --- spend much time in Biden world and you
\emph{will} get sick of this phrase --- and that Mr. Biden was the
candidate to unify a wounded nation. Some others on the team initially
found the approach hackneyed, but not Mr. Biden. So far, cheesiness
seems to be the comfort food many voters are craving.

Mr. Ricchetti tends to keep a low profile, but he is the indispensable
man. He handles much of the delicate outreach to --- and fields plenty
of incoming from --- all the twitchy governors, mayors, members of
Congress, and other political eminences who need hand-holding.

And he is among the handful of confidants to whom Mr. Biden turns when
hard decisions get made. As one top adviser put it, ``Mike is the last
person he talks to on message, strategy, advertising and polling. Steve
is the person he talks to about everything else.'' When Beau was dying,
Mr. Biden asked Mr. Ricchetti, then his chief of staff, to keep him
overscheduled as a way to power through the pain and fear. This at times
put Mr. Ricchetti at odds with Jill Biden, who worried that her husband
was running himself into the ground. Often, ``the two of them would
conspire'' to get the vice president ``to ease off for a while,'' Mr.
Biden wrote in ``Promise Me, Dad.''

Family has always played a central role in Mr. Biden's life and
political career. His sister, Valerie Biden Owens, managed all of his
Senate races and his first two presidential runs. This time, she's
serving as a key surrogate and confidante.

So too is the nominee's wife. Mr. Klain, who has known Jill Biden for
more than three decades, says she has morphed from a reluctant political
spouse into an enthusiastic participant, ``particularly on issues of
education.'' (Dr. Biden is a longtime educator who, until recently,
taught English at a community college in Northern Virginia.) ``I think
the White House experience really changed things for her,'' he said,
explaining that she came to appreciate the contributions she could make
by engaging the public.

And, of course, in a pinch, Jill can double as security. One of the
primary's more charming episodes was on the night of Super Tuesday, when
Dr. Biden body blocked an anti-dairy protester who stormed the stage
during Mr. Biden's victory speech. He later
\href{https://thehill.com/homenews/campaign/486064-biden-on-wife-blocking-protester-whoa-you-dont-screw-around-with-a-philly}{joke}d:
``Whoa, you don't screw around with a Philly girl, I'll tell you what.''

\hypertarget{v}{%
\subsubsection{V.}\label{v}}

Mr. Biden, age 77, knows that he is seen by many as a dinosaur. During
the primary, he explicitly pitched himself as a
``\href{https://www.nytimes.com/2020/05/03/us/politics/joe-biden-vice-president-pick.html}{transition
candidate}'' who aimed to serve as a
``\href{https://www.cnn.com/2020/03/09/politics/joe-biden-bridge-new-generation-of-leaders/index.html}{bridge}''
for a new generation of leaders. Though his team does not like to
discuss it, the conventional wisdom is that Mr. Biden most likely would
be a one-term president. This has fueled a greater-than-normal frenzy
around the vice-presidential pick, which the campaign has said it plans
to announce around Aug. 1.

Supercharging speculation, Mr. Biden vowed to put a woman on the ticket.
But he provided few other hints as to what he is looking for, touching
off a lobbying free-for-all by the hopefuls and their cheering sections.

Some progressive groups are pressing for Elizabeth Warren, even as some
business interests have
\href{https://www.cnbc.com/2020/04/30/donors-pressure-joe-biden-to-not-pick-elizabeth-warren-as-vp.html}{argued
against her}. Many people think that, since Black voters rescued Mr.
Biden's primary candidacy, he should put a Black woman on the ticket ---
a drum beat that has grown louder with the fresh focus on racial
justice.

This has brought leaders such as Keisha Lance Bottoms, the mayor of
Atlanta, and Representative Karen Bass of California, the head of the
Congressional Black Caucus, into the spotlight. Senator Kamala Harris is
thought to have an edge on the rest of the pack, even if her stint as
California's attorney general concerns many progressives.

But other political watchers have made the case for Representative Val
Demings, Senator Tammy Duckworth, Senator Tammy Baldwin, Gov. Gretchen
Whitmer, the former Georgia state legislator and candidate for governor
Stacey Abrams, the former U.N. ambassador Susan Rice, Michelle Obama
\ldots{} At this point, it's hard to find a major female political
figure who has \emph{not} been floated as a V.P. possibility.

Which brings us back to Mr. Biden's governing vision and policy plans.

In the closing weeks of the primary, Mr. Biden began gently inching left
on certain issues in a reassuring gesture to progressives. Shortly
before the final debate, he
\href{https://www.vox.com/2020/3/16/21181500/joe-biden-elizabeth-warren-bankruptcy}{endorsed}
Senator Warren's bankruptcy reform and
\href{https://www.nytimes.com/2020/03/15/us/politics/biden-backs-free-college.html}{embraced}
a version of Senator Sanders' plan to make four-year public colleges
tuition-free for many students. The day after Mr. Sanders dropped out of
the race, Mr. Biden
\href{https://www.npr.org/sections/health-shots/2020/04/11/832025550/bidens-health-play-in-a-covid-19-economy-lower-medicares-eligibility-age-to-60}{called
for} lowering the eligibility age for Medicare from 65 to 60.

After receiving Mr. Sanders'
\href{https://www.nytimes.com/2020/04/13/us/politics/bernie-sanders-joe-biden-endorsement.html}{endorsement},
Mr. Biden kicked things up a notch. As proof of their commitment to
party harmony, the former rivals created a half-dozen of those working
groups, called ``unity task forces.'' The groups --- each with five or
six appointees from the Biden camp and three from the Sanders camp ---
were charged with drawing up recommendations on health care, climate
change, criminal justice reform, immigration, education and the economy.

This gave the ideological wings of the party a safe space in which to
come together, listen to each other and hammer out ideas everyone could
live with, say Biden insiders. With the pandemic having derailed the
usual modes of outreach, the groups were a way to productively channel
the energy of the Sanders revolution.

The
\href{https://www.nytimes.com/aponline/2020/07/08/us/politics/ap-us-election-2020-biden.html}{groups}'
final
\href{https://joebiden.com/wp-content/uploads/2020/07/UNITY-TASK-FORCE-RECOMMENDATIONS.pdf}{recommendations},
released in an 110-page document on July 8, featured some
\href{https://www.politico.com/news/2020/07/08/biden-legal-marijuana-police-protections-353585?nname=playbook\&nid=0000014f-1646-d88f-a1cf-5f46b7bd0000\&nrid=00000163-9937-d365-aff3-fdbffcd70000\&nlid=630318}{wins}
for the left, such as the withholding of federal funds from states that
use cash bail and an accelerated timetable for achieving net-zero
emissions.

Progressive activists I spoke to pointed to the experiment as a hopeful
sign that the campaign was taking their ideas seriously and they were
pleased that some of their influential allies, like Representative
Alexandria Ocasio-Cortez of New York, were included.

``There is improvement in the climate crisis and criminal justice
sections, compared to Biden's previous positions on the subject,'' said
Joseph Geevarghese, the executive director of Our Revolution, a grass
roots group spun out of Mr. Sanders' 2016 presidential run. ``There
should be no doubt that this is a direct result of outside pressure at
this moment.''

But the proposals stopped short of endorsing systemic overhauls like the
Green New Deal or Medicare for All, and steered clear of hot button
issues like abolishing ICE, decriminalizing border crossings, fully
legalizing pot and banning fracking. The document is a statement of
progressive goals --- but it is not pushing for seismic disruption.

``There are some lines we're not going to cross,'' said Mr. Klain.
``He's not going to embrace Medicare For All. He did not run on Medicare
for All. He ran on a campaign that critiqued Medicare for All, and
that's not going to change.''

Mr. Biden is ``not a revolutionary who is going to blow everything up,''
said Senator Chris Coons of Delaware, an old friend of Mr. Biden's who
now holds his old Senate seat.

Mr. Biden also swiftly
\href{https://www.nytimes.com/2020/06/08/us/politics/biden-defund-the-police.html}{came
out against} the ``defund police'' movement. In fact, the reform plan he
put forward included a funding boost for community policing --- which
did not endear him to some activists. Around 50 progressive groups sent
Mr. Biden a letter warning that failing to back a more aggressive
overhaul could cost him support among Black voters. Maybe. Maybe not.
Black voters' views on
\href{https://www.vox.com/2020/6/17/21292046/black-people-abolish-defund-dismantle-police-george-floyd-breonna-taylor-black-lives-matter-protest}{policing
are complicated}. And Mr. Biden's basic instinct remains not to raze but
to ``Build Back Better,'' as he has named his economic plan.

On July 9, Mr. Biden visited a metal-works factory on the outskirts of
Scranton, Pa., his hometown, to talk up Part 1 of that plan. This first
plank focused on reviving manufacturing and included measures such as a
\$300 billion increase in R\&D investment and \$400 billion in
procurement spending on American-made goods. He promised more to come in
a
\href{https://www.nytimes.com/2020/07/09/us/politics/biden-buy-american.html}{populist
speech} with a touch of nationalism.

\href{https://www.nytimes.com/2020/07/14/us/politics/biden-climate-plan.html}{The
following week}, he debuted his four-year, \$2 trillion plan for
investing in infrastructure and clean energy. And on Tuesday, he
proposed a
\href{https://www.nytimes.com/2020/07/21/us/politics/biden-workplace-childcare.html}{\$775
billion investment} to tackle the nation's ``caregiving crisis.'' His
next big announcement is expected to be his plan to address racial
inequity.

Election Day is just over three months away. As it nears, Democrats'
attention will shift toward the transition process and who should do
what in a possible Biden government. At that point, say insiders, things
will \emph{really} get crazy.

\emph{The Times is committed to publishing}
\href{https://www.nytimes.com/2019/01/31/opinion/letters/letters-to-editor-new-york-times-women.html}{\emph{a
diversity of letters}} \emph{to the editor. We'd like to hear what you
think about this or any of our articles. Here are some}
\href{https://help.nytimes.com/hc/en-us/articles/115014925288-How-to-submit-a-letter-to-the-editor}{\emph{tips}}\emph{.
And here's our email:}
\href{mailto:letters@nytimes.com}{\emph{letters@nytimes.com}}\emph{.}

\emph{Follow The New York Times Opinion section on}
\href{https://www.facebook.com/nytopinion}{\emph{Facebook}}\emph{,}
\href{http://twitter.com/NYTOpinion}{\emph{Twitter (@NYTopinion)}}
\emph{and}
\href{https://www.instagram.com/nytopinion/}{\emph{Instagram}}\emph{.}

Advertisement

\protect\hyperlink{after-bottom}{Continue reading the main story}

\hypertarget{site-index}{%
\subsection{Site Index}\label{site-index}}

\hypertarget{site-information-navigation}{%
\subsection{Site Information
Navigation}\label{site-information-navigation}}

\begin{itemize}
\tightlist
\item
  \href{https://help.nytimes.com/hc/en-us/articles/115014792127-Copyright-notice}{©~2020~The
  New York Times Company}
\end{itemize}

\begin{itemize}
\tightlist
\item
  \href{https://www.nytco.com/}{NYTCo}
\item
  \href{https://help.nytimes.com/hc/en-us/articles/115015385887-Contact-Us}{Contact
  Us}
\item
  \href{https://www.nytco.com/careers/}{Work with us}
\item
  \href{https://nytmediakit.com/}{Advertise}
\item
  \href{http://www.tbrandstudio.com/}{T Brand Studio}
\item
  \href{https://www.nytimes.com/privacy/cookie-policy\#how-do-i-manage-trackers}{Your
  Ad Choices}
\item
  \href{https://www.nytimes.com/privacy}{Privacy}
\item
  \href{https://help.nytimes.com/hc/en-us/articles/115014893428-Terms-of-service}{Terms
  of Service}
\item
  \href{https://help.nytimes.com/hc/en-us/articles/115014893968-Terms-of-sale}{Terms
  of Sale}
\item
  \href{https://spiderbites.nytimes.com}{Site Map}
\item
  \href{https://help.nytimes.com/hc/en-us}{Help}
\item
  \href{https://www.nytimes.com/subscription?campaignId=37WXW}{Subscriptions}
\end{itemize}
