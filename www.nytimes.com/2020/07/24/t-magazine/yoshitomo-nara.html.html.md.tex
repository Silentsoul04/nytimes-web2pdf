Yoshitomo Nara Paints What He Hears

\url{https://nyti.ms/2CAc7RT}

\begin{itemize}
\item
\item
\item
\item
\item
\end{itemize}

\includegraphics{https://static01.nyt.com/images/2020/06/30/t-magazine/30tmag-nara-slide-67ZV/30tmag-nara-slide-67ZV-articleLarge.jpg?quality=75\&auto=webp\&disable=upscale}

Sections

\protect\hyperlink{site-content}{Skip to
content}\protect\hyperlink{site-index}{Skip to site index}

True Believers

\hypertarget{yoshitomo-nara-paints-what-he-hears}{%
\section{Yoshitomo Nara Paints What He
Hears}\label{yoshitomo-nara-paints-what-he-hears}}

Ahead of a major retrospective in Los Angeles, the Japanese artist
discusses his musical education.

Yoshitomo Nara in his home studio in Tochigi Prefecture, Japan, with one
of his bigheaded girl works, ``Miss Moonlight'' (2020).Credit...Tetsuya
Miura

Supported by

\protect\hyperlink{after-sponsor}{Continue reading the main story}

By Nick Marino

\begin{itemize}
\item
  July 24, 2020
\item
  \begin{itemize}
  \item
  \item
  \item
  \item
  \item
  \end{itemize}
\end{itemize}

Growing up in the far northern reaches of Honshu, Japan's largest
island,
\href{https://www.pacegallery.com/artists/yoshitomo-nara/}{Yoshitomo
Nara} discovered the outside world through his ears. This was many years
before he'd leave to study painting at the prestigious Kunstakademie
Düsseldorf in the late '80s, and decades before Sotheby's would sell his
work
``\href{https://www.sothebys.com/en/auctions/ecatalogue/2019/contemporary-art-evening-sale-hk0885/lot.1142.html}{Knife
Behind Back}'' (2000) for \$25 million in 2019. Back then, in the 1960s
and '70s, he was a latchkey kid who whiled away the afternoons playing
in an abandoned ammunition depot on a former Japanese Imperial Army
site. At night, he'd listen obsessively, either using the family radio
or one he'd built himself at age 8, to the Far East Network, an American
station that served the area with news and tunes. Over these airwaves,
he found Western music. Folk music. Rock music. He heard voices in a
strange foreign language --- English --- and because he couldn't
understand the lyrics, these voices became just another sound alongside
the guitars. And so he became an improbable, insatiable witness to
Western pop's evolution from the flower-child bliss of the mid-60s to
the ecstatic thrash of late '70s punk.

\href{https://www.nytimes.com/issue/t-magazine/2020/07/02/true-believers-art-issue}{\includegraphics{https://static01.nyt.com/newsgraphics/2020/06/29/tmag-art-embeds-new/assets/images/art_issue_gif_special_editon.gif}}

As he collected records (his first single in English was 1967's
``Massachusetts,'' by
\href{https://www.nytimes.com/1979/02/18/archives/why-the-bee-gees-sound-so-good-the-bee-gees.html}{the
Bee Gees}), he scrutinized the album jackets, which he considered
wondrous artworks. He adored the cover of
\href{https://www.nytimes.com/topic/person/joni-mitchell}{Joni
Mitchell}'s ``Song to a Seagull'' (1968), which he was pleased to learn
she'd painted herself, and how the jacket of Luke Gibson's ``Another
Perfect Day'' (1971) appeared to be embroidered with wildflowers. These
combinations of sounds and visuals trained Nara's imagination and
foreshadowed the time when, as a grown man and cultural figure in his
own right, he'd provide cover art for bands including
\href{https://www.moma.org/collection/works/88514}{Shonen Knife},
\href{https://en.wikipedia.org/wiki/I\%27ll_Take_the_Rain\#/media/File:R.E.M._-_I'll_Take_the_Rain.jpg}{R.E.M.}
and
\href{https://en.wikipedia.org/wiki/Bloodthirsty_butchers_vs_\%2B/-_PLUS/MINUS\#/media/File:Butcherssplit.jpg}{Bloodthirsty
Butchers}.

\includegraphics{https://static01.nyt.com/images/2020/06/30/t-magazine/30tmag-nara-slide-PJB9/30tmag-nara-slide-PJB9-articleLarge.jpg?quality=75\&auto=webp\&disable=upscale}

Nara developed his signature style in the 1990s, during art school, when
he began painting what a
\href{https://www.phaidon.com/store/art/yoshitomo-nara-9780714879949/}{new
Phaidon monograph} calls ``those big-headed girls.'' Rendered in acrylic
with cartoonish proportions, these cherubic figures seemed, at first
glance, indebted to both American twee and Japanese \emph{kawaii} but
were far from innocents. With their slit mouths and saucer eyes, their
faces radiated exquisite ambivalence. ``People refer to them as
portraits of girls or children,'' says the curator Mika Yoshitake, an
expert on postwar Japanese art. ``But they're really all, I think,
self-portraits.''

By the time of Nara's breakthrough 1995 show, ``In the Deepest Puddle,''
at Scai the Bathhouse gallery in Tokyo, these imaginary characters had
cemented their place as his muses. Over the next two decades, he would
paint them time and again, often against solid milky backgrounds on
canvasses five feet high or more. Other times, in pencil, he'd
transmogrify them into
\href{https://www.nytimes.com/2016/04/15/arts/music/his-brothers-keeper-a-ramones-tour-of-queens.html}{Joey
or Dee Dee Ramone} with the exuberance of a teenager drawing on his
jeans. Nara's imps lived rich musical lives, too: They bashed drums and
throttled microphone stands. And even when they weren't literal punks
(though they often were), they had a punk-rock attitude. They came off
as gremlin Kewpies, often wielding a disturbing totem --- a saw, a
pistol, an unlit match --- while wearing a baby-doll dress or pageboy
haircut.

Image

The artist's characters often hold props --- flowers, instruments or
something more menacing.Credit...Tetsuya Miura

Image

``Visitors, when they come by, apparently find some things odd,'' Nara
says. ``They will ask, `Why do you have all of these strange dolls
everywhere?'''Credit...Tetsuya Miura

Many of them will be present in a major new retrospective at the
\href{https://www.lacma.org/art/exhibition/yoshitomo-nara}{Los Angeles
County Museum of Art} (scheduled for as soon as the institution can
reopen to visitors), which will bring together more than 100 of the
artist's works from the past 36 years, with an emphasis on pieces
inspired by music. ``It's sort of a class reunion,'' says Nara, who is
now 60. ``It's not my children having a reunion. It's more like my
grandchildren.'' **** In addition to his paintings and drawings --- and
a 26-foot painted bronze sculpture of a girl whose head sprouts into a
towering evergreen that will stand outside the museum on Wilshire
Boulevard --- the exhibit, curated by Yoshitake, will also include
several hundred vinyl album covers from Nara's personal collection. A
limited edition of the exhibition catalog will come with a custom vinyl
record, featuring six tracks (five covers and an original) by the
stalwart American indie band and Nara favorite
\href{https://www.nytimes.com/2015/08/25/t-magazine/yo-la-tengo-ira-kaplan-inspirations.html}{Yo
La Tengo}, and a B-side of vintage folk songs from artists including
Karen Dalton and
\href{https://www.nytimes.com/2014/06/11/arts/music/donovan-to-enter-songwriters-hall-of-fame.html}{Donovan}.
While the show could be seen as an example of music and art coming
together, for Nara, the two were never apart. ``When I'm working on
drawings,'' he says, ``music just comes into my ear and goes straight
out of my hand.''

Today, Nara lives 300 miles south of his childhood home, in the
mountainous countryside of Tochigi Prefecture, and works in an airy,
white-walled home studio filled with toy figurines and cat-shaped clocks
with dangling pendulum tails. Speaking in Japanese, via a translator,
from the Tokyo office of his gallery,
\href{https://www.blumandpoe.com/}{Blum \& Poe}, he answered T's Artist
Questionnaire.

Image

Nara is such a music fanatic that he still has CDs, in addition to his
beloved vinyl collection.Credit...Tetsuya Miura

Image

Though the artist also draws and sculpts, his most celebrated works tend
to be painted in acrylic.Credit...Tetsuya Miura

\textbf{What's your day like? How much do you sleep, and what's your
work schedule?}

In my daily life, I don't have to interact with people. So, my schedule
is all over the place. For example, yesterday I woke up at midnight. But
on a regular day, I do sleep between eight and 10 hours.

\textbf{How many hours of creative work do you think you do in a day?}

If it's a good day, I might work from the moment I get up all the way
until the time I go to sleep. I'll spend a whole day in the studio. And
there are some days where I don't do any work at all, and I just go on a
walk or read a book.

\textbf{What's the first piece of art you ever made?}

When I was 6 years old, I made an illustrated \emph{kamishibai} story
about my cat and me traveling together to the North Pole, and then going
all the way down to the South Pole.

\textbf{What's the worst studio you ever had?}

When I was young, my studios were really terrible --- but I enjoyed all
of them. For example, when I was in Germany, I had a studio where I had
no shower. But I just went to the pool all the time and I washed my hair
there.

Image

The artist in his reading room.Credit...Tetsuya Miura

\textbf{What's the first work you ever sold? For how much?}

When I was 24 years old, I had a show in a very, very small space, and
it had a painting of mine that's about the size of a record jacket. I
sold it for about 2,000 yen, which is basically 20 bucks.

\textbf{When you start a new piece, where do you begin?}

It's really different each time. The inspiration might be the shape of a
cloud or a piece of music or a scene from a movie.

\textbf{How do you know when you're done?}

It varies each time as well. But when I'm happy with it, it's done. I
don't worry about or think about what other people might see.

\textbf{How many assistants do you have?}

Just one: me, myself. So I can totally slack off. I feel like if I had
assistants, I'd feel pressure to always be working.

Image

Nara's aesthetic in composite: pageboy haircuts, cartoonish cat clocks
and Kurt Cobain.Credit...Tetsuya Miura

\textbf{What music do you play when you're making art?}

Whenever I don't know what to turn to, I usually go back to
\href{https://www.nytimes.com/2020/06/12/arts/music/bob-dylan-rough-and-rowdy-ways.html}{Bob
Dylan} or \href{https://www.nytimes.com/topic/person/neil-young}{Neil
Young}.

\textbf{When did you first feel comfortable saying you're a professional
artist?}

When I graduated from the Kunstakademie. Up until then, when I would
check in at a hotel and have to register my profession, I would always
write ``student.'' But after I graduated, I could no longer write
``student.'' And so I thought, ``OK, well, I guess I have to write
`artist.'''

\textbf{Is there a meal you eat on repeat when you're working?}

When I lived in Germany, it was Haribo gummy bears. But when I moved
back to Japan, it became chocolate.

\textbf{Are you bingeing on any shows right now?}

Netflix's recent remake of ``Anne of Green Gables,'' called
``\href{https://www.netflix.com/title/80136311}{Anne With an E}.'' They
do an incredible job with the historical re-creation that really gives
you a good sense of what life was like at the time, in that place, and
what the people were like. I enjoy that they've found a way to weave in
issues that are relevant to today, like issues facing Indigenous people
and the Black **** community in that world. It really addresses all
kinds of interesting things, and should be watched by children and
adults alike.

Image

Nara in the stockroom.Credit...Tetsuya Miura

Image

One of the walls of the studio's entrance hall.Credit...Tetsuya Miura

\textbf{What's the weirdest object in your studio?}

I don't think anything in my studio is weird at all. But visitors, when
they come by, apparently find some things odd. They will ask, ``Why do
you have all of these strange dolls everywhere?''

\textbf{How often do you talk to other artists?}

I very rarely meet with other artists. Artists tend to only want to talk
about art. I'd rather talk to people with other interests: people who
love movies, or people who love to read, or people who are in
professions completely different from mine. People who work as
fishermen, people who work as hunters, people who work in forestry.

\textbf{What's the last thing that made you cry?}

``Anne With an E.'' When I was younger, I rarely cried. But as I've
gotten older, sometimes just the smallest things are enough to set me
off.

\textbf{If you have windows, what do they look out on?}

Mountains, forests and grassland. No people.

\textbf{What do you pay for rent?}

I don't pay rent. I built this studio myself.

\textbf{What do you bulk buy with most frequency?}

Probably chocolate. The brand is called
\href{https://www.peopletree.co.jp/choco/index.html}{People Tree}. I
actually once put a picture of the chocolate on my Twitter, and then the
company sent me a bunch --- so now I don't have to buy it.

\emph{This interview has been condensed and edited.}

\hypertarget{true-believers-art-issue}{%
\subsubsection{\texorpdfstring{\href{https://www.nytimes.com/issue/t-magazine/2020/07/02/true-believers-art-issue}{True
Believers Art
Issue}}{True Believers Art Issue}}\label{true-believers-art-issue}}

Advertisement

\protect\hyperlink{after-bottom}{Continue reading the main story}

\hypertarget{site-index}{%
\subsection{Site Index}\label{site-index}}

\hypertarget{site-information-navigation}{%
\subsection{Site Information
Navigation}\label{site-information-navigation}}

\begin{itemize}
\tightlist
\item
  \href{https://help.nytimes.com/hc/en-us/articles/115014792127-Copyright-notice}{©~2020~The
  New York Times Company}
\end{itemize}

\begin{itemize}
\tightlist
\item
  \href{https://www.nytco.com/}{NYTCo}
\item
  \href{https://help.nytimes.com/hc/en-us/articles/115015385887-Contact-Us}{Contact
  Us}
\item
  \href{https://www.nytco.com/careers/}{Work with us}
\item
  \href{https://nytmediakit.com/}{Advertise}
\item
  \href{http://www.tbrandstudio.com/}{T Brand Studio}
\item
  \href{https://www.nytimes.com/privacy/cookie-policy\#how-do-i-manage-trackers}{Your
  Ad Choices}
\item
  \href{https://www.nytimes.com/privacy}{Privacy}
\item
  \href{https://help.nytimes.com/hc/en-us/articles/115014893428-Terms-of-service}{Terms
  of Service}
\item
  \href{https://help.nytimes.com/hc/en-us/articles/115014893968-Terms-of-sale}{Terms
  of Sale}
\item
  \href{https://spiderbites.nytimes.com}{Site Map}
\item
  \href{https://help.nytimes.com/hc/en-us}{Help}
\item
  \href{https://www.nytimes.com/subscription?campaignId=37WXW}{Subscriptions}
\end{itemize}
