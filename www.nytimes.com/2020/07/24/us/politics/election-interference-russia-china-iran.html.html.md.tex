Sections

SEARCH

\protect\hyperlink{site-content}{Skip to
content}\protect\hyperlink{site-index}{Skip to site index}

\href{https://www.nytimes.com/section/politics}{Politics}

\href{https://myaccount.nytimes.com/auth/login?response_type=cookie\&client_id=vi}{}

\href{https://www.nytimes.com/section/todayspaper}{Today's Paper}

\href{/section/politics}{Politics}\textbar{}U.S. Warns Russia, China and
Iran Are Trying to Interfere in the Election. Democrats Say It's Far
Worse.

\url{https://nyti.ms/3jDBrac}

\begin{itemize}
\item
\item
\item
\item
\item
\end{itemize}

\begin{itemize}
\item
  \href{https://www.nytimes.com/2020/07/31/us/elections/biden-vs-trump.html?action=click\&pgtype=Article\&state=default\&region=TOP_BANNER\&context=storylines_menu}{Election
  Updates}
\item
  \href{https://www.nytimes.com/article/biden-vice-president-2020.html?action=click\&pgtype=Article\&state=default\&region=TOP_BANNER\&context=storylines_menu}{Biden's
  V.P. Search}
\item
  \href{https://www.nytimes.com/interactive/2020/07/24/us/politics/trump-biden-campaign-donors.html?action=click\&pgtype=Article\&state=default\&region=TOP_BANNER\&context=storylines_menu}{Map
  of Donations}
\item
  \href{https://www.nytimes.com/interactive/2020/us/elections/delegate-count-primary-results.html?action=click\&pgtype=Article\&state=default\&region=TOP_BANNER\&context=storylines_menu}{Delegate
  Count}
\item
  \href{https://www.nytimes.com/interactive/2019/us/politics/2020-presidential-candidates.html?action=click\&pgtype=Article\&state=default\&region=TOP_BANNER\&context=storylines_menu}{The
  Candidates}
\item
  \href{https://www.nytimes.com/newsletters/politics?action=click\&pgtype=Article\&state=default\&region=TOP_BANNER\&context=storylines_menu}{Politics
  Newsletter}
\end{itemize}

Advertisement

\protect\hyperlink{after-top}{Continue reading the main story}

Supported by

\protect\hyperlink{after-sponsor}{Continue reading the main story}

\hypertarget{us-warns-russia-china-and-iran-are-trying-to-interfere-in-the-election-democrats-say-its-far-worse}{%
\section{U.S. Warns Russia, China and Iran Are Trying to Interfere in
the Election. Democrats Say It's Far
Worse.}\label{us-warns-russia-china-and-iran-are-trying-to-interfere-in-the-election-democrats-say-its-far-worse}}

The government statement was short on details, reminiscent of the vague
warnings in 2016 that, in retrospect, failed to seize the attention of
officials and voters before the last presidential election.

\includegraphics{https://static01.nyt.com/images/2020/07/24/us/politics/24dc-intel/merlin_173855703_1804d4e1-7897-4312-ad2c-b9f3f4d63801-articleLarge.jpg?quality=75\&auto=webp\&disable=upscale}

\href{https://www.nytimes.com/by/david-e-sanger}{\includegraphics{https://static01.nyt.com/images/2018/10/03/multimedia/author-david-e-sanger/author-david-e-sanger-thumbLarge.png}}\href{https://www.nytimes.com/by/julian-e-barnes}{\includegraphics{https://static01.nyt.com/images/2019/12/13/reader-center/author-julian-barnes/author-julian-barnes-thumbLarge.png}}

By \href{https://www.nytimes.com/by/david-e-sanger}{David E. Sanger} and
\href{https://www.nytimes.com/by/julian-e-barnes}{Julian E. Barnes}

\begin{itemize}
\item
  July 24, 2020
\item
  \begin{itemize}
  \item
  \item
  \item
  \item
  \item
  \end{itemize}
\end{itemize}

American intelligence officials issued a public warning on Friday that
China was ``expanding its influence efforts'' in the United States ahead
of the presidential election, along with Russia and Iran, but Democrats
briefed on the matter said the threat was far more urgent than what the
administration described.

The warning came from William R. Evanina, the director of the National
Counterintelligence and Security Center, in a statement 100 days before
Americans go to the polls. ``We're primarily concerned with China,
Russia and Iran --- although other nation-states and nonstate actors
could also do harm to our electoral process,'' the statement said.

The warning about China came at a moment of extraordinary tension
between Beijing and Washington, only days after the United States
indicted two Chinese hackers on charges of stealing intellectual
property, including for the country's main intelligence service, and
\href{https://www.nytimes.com/2020/07/22/world/asia/us-china-houston-consulate.html}{evicted
Chinese diplomats} from their consulate in Houston.

The intelligence warning on Friday did not accuse the Chinese of trying
to hack the vote; instead it said they were using their influence ``to
shape the policy environment in the United States'' and to pressure
politicians ``it views as opposed to China's interests.''

Russia, the warning said, was continuing to ``spread disinformation in
the U.S. that is designed to undermine confidence in our democratic
process,'' and it described Iran as an emerging actor in election
interference, seeking to spread disinformation and ``recirculating
anti-U.S. content.''

The statement was short on details, reminiscent of the vague warnings
that the director of national intelligence turned out starting in
October 2016 that, in retrospect, failed to seize the attention of
officials and voters before the last presidential election.

In a statement issued a few hours later, Speaker Nancy Pelosi was joined
by the Senate Democratic leader, Senator Chuck Schumer of New York, and
two key Democrats on intelligence oversight committees, Senator Mark
Warner of Virginia and Representative Adam B. Schiff of California, in
saying that the descriptions of malign activity were ``so generic as to
be almost meaningless.''

\hypertarget{latest-updates-2020-election}{%
\section{\texorpdfstring{\href{https://www.nytimes.com/2020/07/31/us/elections/biden-vs-trump.html?action=click\&pgtype=Article\&state=default\&region=MAIN_CONTENT_1\&context=storylines_live_updates}{Latest
Updates: 2020
Election}}{Latest Updates: 2020 Election}}\label{latest-updates-2020-election}}

Updated 2020-08-01T01:26:45.732Z

\begin{itemize}
\tightlist
\item
  \href{https://www.nytimes.com/2020/07/31/us/elections/biden-vs-trump.html?action=click\&pgtype=Article\&state=default\&region=MAIN_CONTENT_1\&context=storylines_live_updates\#link-29fdff45}{Kamala
  Harris, a top vice-presidential contender, confronts double
  standards.}
\item
  \href{https://www.nytimes.com/2020/07/31/us/elections/biden-vs-trump.html?action=click\&pgtype=Article\&state=default\&region=MAIN_CONTENT_1\&context=storylines_live_updates\#link-13ec3d9c}{Karen
  Bass and Susan Rice are rising on Biden's vice-presidential
  shortlist.}
\item
  \href{https://www.nytimes.com/2020/07/31/us/elections/biden-vs-trump.html?action=click\&pgtype=Article\&state=default\&region=MAIN_CONTENT_1\&context=storylines_live_updates\#link-49e9a016}{Trump
  says Russian bounties to kill U.S. troops `never took place.'}
\end{itemize}

\href{https://www.nytimes.com/2020/07/31/us/elections/biden-vs-trump.html?action=click\&pgtype=Article\&state=default\&region=MAIN_CONTENT_1\&context=storylines_live_updates}{See
more updates}

Mr. Evanina's statement, said the four Democrats, who earlier called on
the F.B.I. to give a briefing on disinformation campaigns to the entire
Congress, ``does not go nearly far enough in arming the American people
with the knowledge they need about how foreign powers are seeking to
influence our political process.''

Their letter was particularly critical of the description of Russian
activity, the most politically delicate topic because of President
Trump's own unwillingness to acknowledge Russia's actions four years
ago. The Democrats wrote that ``to say without more, for example, that
Russia seeks to denigrate what it sees as an anti-Russia `establishment'
in America is so generic as to be almost meaningless.''

Mr. Schiff, who is the chairman of the House Intelligence Committee,
said Friday on MSNBC that he had been ``urging Bill Evanina and others
in the intelligence community to level with the American people about
what's going on.'' He said the warning gave ``a false sense of
equivalence between what Russia is doing, what China is doing, what Iran
is doing.''

Mr. Schiff and the other three authors of the letter
\href{https://www.nytimes.com/2020/02/20/us/politics/russian-interference-trump-democrats.html}{have
been briefed}extensively on the intelligence, and thus are prohibited
from violating classification rules by describing what they have seen.

But Mr. Schiff, a frequent target of harsh criticism from Mr. Trump
because he was the Democrats' manager in the impeachment trial in the
Senate, added, ``I think that our adversaries, in particular the
Russians, are going to amplify the false messages that the president is
putting out about, `Well, you can't trust absentee ballots,' even though
that's how the president votes.''

Some intelligence officials expressed surprise at the lawmakers' letter
and insisted they were not trying to play down the threat of
interference from Moscow or signal that China was a greater challenge.
They said Mr. Evanina's statement was meant to be the beginning of a
series of public statements, according to an official from the Office of
the Director of National Intelligence.

The official said the statement did not play down the threat of Russian
interference, but lawmakers had to understand that the 2020 contest
would be different from 2016's.

It is unclear whether those statements, however, deter further action by
American adversaries. But it is clear that 2020 will not be the same as
2016 ---
\href{https://www.nytimes.com/2020/03/10/us/politics/election-interference-briefing-trump.html}{the
Russians know that they cannot use the same playbook}, and Iran and
China both seem poised to play a greater role.

The question is whether they will be on the same side, or working
against each other.

After the 2016 election, American intelligence assessments concluded
that the Russians ultimately intervened on Mr. Trump's behalf. But this
year, Republicans and Democrats who have reviewed the intelligence have
come to different assessments about whether Russia hopes to swing the
election to Mr. Trump, or if President Vladimir V. Putin is simply
intent on eroding confidence in the American electoral system.

The threat of Iran to the election is harder to judge. Senior American
officials said it was intent on trying to hurt Mr. Trump's re-election
campaign. Some believe Iran would stage attacks on oil shipping this
fall, to try to cause economic calamity. But with the global economy
already in turmoil from the pandemic, Iran's room to try to influence
the election through such attacks may be more limited.

China is not new to presidential elections. In 2008, intelligence
officials warned the campaigns of both Barack Obama and John McCain that
Chinese hackers had penetrated their campaign computer systems. But that
was intelligence gathering, it appears, not an effort to influence the
outcome in the way Russia tried eight years later.

This year, intelligence officials do not believe China will try the same
kind of brazen techniques Moscow has employed. Instead, intelligence
officials said Friday, China is playing a long game, trying to cultivate
local politicians who may ultimately win election to Congress.

Mick Baccio, a former information security official with Pete
Buttigieg's campaign, said that with large numbers of absentee ballots
and a potentially long counting period, foreign interference could
intensify in November. As votes are being counted, foreign powers could
seek to undermine confidence in the vote.

The 2020 election ``is like every disaster movie sequel rolled into
one,'' said Mr. Baccio, now an adviser with the cybersecurity firm
Splunk. ``The postelection period is what I am most concerned about. The
window of time where we are uncertain, that is when they will drop their
madness.''

The intensified work by China, Russia, Iran and others provides a major
challenge to the campaigns.

``Nothing is unhackable,'' Mr. Baccio said. ``You raise the bar as best
you can. You identify your crown jewels of data and you lock it down the
best you can.''

In recent weeks, intelligence officials have briefed lawmakers on
Capitol Hill about election interference threats from China, according
to American officials.

Those classified briefings included warnings about how the Chinese
government was trying to influence the broader political debate in the
United States as well as in Congress.

F.B.I. and intelligence officials have warned lawmakers about China's
so-called mask diplomacy, in which Beijing's diplomats have made
donations of personal protective equipment useful to fight the
coronavirus pandemic and have demanded support in return.

A Chinese diplomat had asked a
\href{https://www.nytimes.com/2020/04/14/us/politics/coronavirus-china-trump-donation.html}{Wisconsin
state Republican lawmaker} to pass a resolution lauding China's work to
fight the virus. The state senator refused and instead introduced a
measure criticizing the Chinese Communist Party.

Until now, China has focused on local and congressional races and was
less interested in influencing the presidential campaigns, officials
said. Local influence campaigns are less likely to receive national
attention, and are therefore more likely to succeed, officials said.

But those localized campaigns could influence national or
presidential-level politics, current and former intelligence officials
said. ``Beijing recognizes its efforts might affect the presidential
race,'' Mr. Evanina wrote in his warning on Friday.

Beijing has been conducting cybersurveillance of the presidential
campaigns and related political efforts, said an intelligence official,
and it could decide in the coming weeks to expand its influence campaign
to try to influence presidential politics more directly.

\hypertarget{our-2020-election-guide}{%
\section{Our 2020 Election Guide}\label{our-2020-election-guide}}

Updated July 31, 2020

\begin{itemize}
\item
  \begin{center}\rule{0.5\linewidth}{\linethickness}\end{center}

  \hypertarget{the-latest}{%
  \subsection{The Latest}\label{the-latest}}

  \begin{itemize}
  \tightlist
  \item
    President Trump's assault on the Postal Service is intersecting with
    his attacks on mail-in voting.
    \href{https://www.nytimes.com/2020/07/31/us/politics/trump-usps-mail-delays.html?action=click\&pgtype=Article\&state=default\&region=BELOW_MAIN_CONTENT\&context=storylines_guide}{Voting
    rights groups say it is a recipe for disaster.}
  \end{itemize}
\item
  \begin{center}\rule{0.5\linewidth}{\linethickness}\end{center}

  \hypertarget{bidens-vp-search}{%
  \subsection{Biden's V.P. Search}\label{bidens-vp-search}}

  \begin{itemize}
  \tightlist
  \item
    \href{https://www.nytimes.com/article/biden-vice-president-2020.html?action=click\&pgtype=Article\&state=default\&region=BELOW_MAIN_CONTENT\&context=storylines_guide}{Here
    are 13 women} who have been under consideration to be Joe Biden's
    running mate, and why each might be chosen --- and might not be.
  \end{itemize}
\item
  \begin{center}\rule{0.5\linewidth}{\linethickness}\end{center}

  \hypertarget{keep-up-with-our-coverage}{%
  \subsection{Keep Up With Our
  Coverage}\label{keep-up-with-our-coverage}}

  \begin{itemize}
  \tightlist
  \item
    Get an
    \href{https://www.nytimes.com/newsletters/politics?action=click\&pgtype=Article\&state=default\&region=BELOW_MAIN_CONTENT\&context=storylines_guide}{email}
    recapping the day's news
  \end{itemize}

  \begin{itemize}
  \tightlist
  \item
    Download our mobile app on
    \href{https://apps.apple.com/us/app/nytimes/id284862083?ls=1\&mat_click_id=5c79ae7455014fd1bd66b5610c05b8f2-20191112-16948\&referrer=mat_click_id\%3D5c79ae7455014fd1bd66b5610c05b8f2-20191112-16948\%26link_click_id\%3D722930677036718082}{iOS}
    and
    \href{http://a.localytics.com/android?id=com.nytimes.android\&referrer=utm_source\%3Dother_nyt_mobile_web\%26utm_medium\%3DWeb\%2520page\%26utm_term\%3DGeneral\%2520Mobile\%2520Page\%26utm_campaign\%3DNYT\%2520Mobile\%2520General\%2520Page}{Android}
    and turn on Breaking News and Politics alerts
  \end{itemize}
\end{itemize}

Advertisement

\protect\hyperlink{after-bottom}{Continue reading the main story}

\hypertarget{site-index}{%
\subsection{Site Index}\label{site-index}}

\hypertarget{site-information-navigation}{%
\subsection{Site Information
Navigation}\label{site-information-navigation}}

\begin{itemize}
\tightlist
\item
  \href{https://help.nytimes.com/hc/en-us/articles/115014792127-Copyright-notice}{©~2020~The
  New York Times Company}
\end{itemize}

\begin{itemize}
\tightlist
\item
  \href{https://www.nytco.com/}{NYTCo}
\item
  \href{https://help.nytimes.com/hc/en-us/articles/115015385887-Contact-Us}{Contact
  Us}
\item
  \href{https://www.nytco.com/careers/}{Work with us}
\item
  \href{https://nytmediakit.com/}{Advertise}
\item
  \href{http://www.tbrandstudio.com/}{T Brand Studio}
\item
  \href{https://www.nytimes.com/privacy/cookie-policy\#how-do-i-manage-trackers}{Your
  Ad Choices}
\item
  \href{https://www.nytimes.com/privacy}{Privacy}
\item
  \href{https://help.nytimes.com/hc/en-us/articles/115014893428-Terms-of-service}{Terms
  of Service}
\item
  \href{https://help.nytimes.com/hc/en-us/articles/115014893968-Terms-of-sale}{Terms
  of Sale}
\item
  \href{https://spiderbites.nytimes.com}{Site Map}
\item
  \href{https://help.nytimes.com/hc/en-us}{Help}
\item
  \href{https://www.nytimes.com/subscription?campaignId=37WXW}{Subscriptions}
\end{itemize}
