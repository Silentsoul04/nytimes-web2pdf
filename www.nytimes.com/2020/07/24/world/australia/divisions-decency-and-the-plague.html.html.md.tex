Sections

SEARCH

\protect\hyperlink{site-content}{Skip to
content}\protect\hyperlink{site-index}{Skip to site index}

\href{https://www.nytimes.com/section/world/australia}{Australia}

\href{https://myaccount.nytimes.com/auth/login?response_type=cookie\&client_id=vi}{}

\href{https://www.nytimes.com/section/todayspaper}{Today's Paper}

\href{/section/world/australia}{Australia}\textbar{}Divisions, Decency
and `The Plague'

\url{https://nyti.ms/32Qbb6o}

\begin{itemize}
\item
\item
\item
\item
\item
\end{itemize}

\href{https://www.nytimes.com/news-event/coronavirus?action=click\&pgtype=Article\&state=default\&region=TOP_BANNER\&context=storylines_menu}{The
Coronavirus Outbreak}

\begin{itemize}
\tightlist
\item
  live\href{https://www.nytimes.com/2020/08/04/world/coronavirus-cases.html?action=click\&pgtype=Article\&state=default\&region=TOP_BANNER\&context=storylines_menu}{Latest
  Updates}
\item
  \href{https://www.nytimes.com/interactive/2020/us/coronavirus-us-cases.html?action=click\&pgtype=Article\&state=default\&region=TOP_BANNER\&context=storylines_menu}{Maps
  and Cases}
\item
  \href{https://www.nytimes.com/interactive/2020/science/coronavirus-vaccine-tracker.html?action=click\&pgtype=Article\&state=default\&region=TOP_BANNER\&context=storylines_menu}{Vaccine
  Tracker}
\item
  \href{https://www.nytimes.com/2020/08/02/us/covid-college-reopening.html?action=click\&pgtype=Article\&state=default\&region=TOP_BANNER\&context=storylines_menu}{College
  Reopening}
\item
  \href{https://www.nytimes.com/live/2020/08/04/business/stock-market-today-coronavirus?action=click\&pgtype=Article\&state=default\&region=TOP_BANNER\&context=storylines_menu}{Economy}
\end{itemize}

Advertisement

\protect\hyperlink{after-top}{Continue reading the main story}

Supported by

\protect\hyperlink{after-sponsor}{Continue reading the main story}

letter 168

\hypertarget{divisions-decency-and-the-plague}{%
\section{Divisions, Decency and `The
Plague'}\label{divisions-decency-and-the-plague}}

What should we do about the uneven distribution of pandemic
consequences?

\includegraphics{https://static01.nyt.com/images/2020/07/24/world/24australialetter-northernnsw/merlin_142321659_e1a5f92d-6b22-416b-87d0-17ac716cd29e-articleLarge.jpg?quality=75\&auto=webp\&disable=upscale}

\href{https://www.nytimes.com/by/damien-cave}{\includegraphics{https://static01.nyt.com/images/2018/10/08/multimedia/author-damien-cave/author-damien-cave-thumbLarge.png}}

By \href{https://www.nytimes.com/by/damien-cave}{Damien Cave}

\begin{itemize}
\item
  July 24, 2020
\item
  \begin{itemize}
  \item
  \item
  \item
  \item
  \item
  \end{itemize}
\end{itemize}

\href{https://www.nytimes.com/series/nyt-australia-newsletter?module=inline}{\emph{The
Australia Letter}} \emph{is a weekly newsletter from our Australia
bureau.}
\href{https://www.nytimes.com/newsletters/australia-letter?module=inline}{\emph{Sign
up}} \emph{to get it by email.}

\begin{center}\rule{0.5\linewidth}{\linethickness}\end{center}

There's a point early on in ``The Plague'' when a journalist stuck in
the sickened city of Oran tries desperately to get out --- beyond the
walls, to the rest of healthy, happy French Algeria.

It's meant to be for love. Raymond Rambert, the would-be escapee in
Albert Camus's 1947 novel, is eager to reach the woman he intends to
marry.

But I found myself recalling the scene last week because it reminded me
of the divide that pandemics create between places and people, and the
enormous difference between life inside with the contagion and life
beyond, in someplace that feels almost normal.

At the time, I was catching up on news of the surging coronavirus
outbreak in Victoria while enjoying a gourmet lunch in northern New
South Wales after stopping at a busy local bookstore. It felt odd and,
if I'm honest, exhilarating, to be enjoying such luxuries. The tables in
the restaurant,
\href{https://www.nytimes.com/2018/08/16/dining/restaurants-northern-rivers-byron-bay.html}{Shelter
in Lennox Head}, were far enough apart to feel safe but not so distant
to feel abnormal. The food was great. The bookstore, too, felt almost
pre-pandemic.

Our family of four felt as though we had escaped to the other side, and
in many ways we had, especially compared to our loved ones in the United
States. I recently talked to a friend in South Carolina who told me he
had been in isolation for 110 days and spoken to only five people in
that time who were not members of his family.

\hypertarget{latest-updates-global-coronavirus-outbreak}{%
\section{\texorpdfstring{\href{https://www.nytimes.com/2020/08/04/world/coronavirus-cases.html?action=click\&pgtype=Article\&state=default\&region=MAIN_CONTENT_1\&context=storylines_live_updates}{Latest
Updates: Global Coronavirus
Outbreak}}{Latest Updates: Global Coronavirus Outbreak}}\label{latest-updates-global-coronavirus-outbreak}}

Updated 2020-08-04T20:57:54.346Z

\begin{itemize}
\tightlist
\item
  \href{https://www.nytimes.com/2020/08/04/world/coronavirus-cases.html?action=click\&pgtype=Article\&state=default\&region=MAIN_CONTENT_1\&context=storylines_live_updates\#link-1228a480}{Novavax
  sees encouraging results from two studies of its experimental
  vaccine.}
\item
  \href{https://www.nytimes.com/2020/08/04/world/coronavirus-cases.html?action=click\&pgtype=Article\&state=default\&region=MAIN_CONTENT_1\&context=storylines_live_updates\#link-4825b93}{Public
  and private schools in Maryland and elsewhere are divided over
  in-person instruction.}
\item
  \href{https://www.nytimes.com/2020/08/04/world/coronavirus-cases.html?action=click\&pgtype=Article\&state=default\&region=MAIN_CONTENT_1\&context=storylines_live_updates\#link-50f7386d}{The
  United Nations calls on policymakers to `plan thoroughly for school
  reopenings.'}
\end{itemize}

\href{https://www.nytimes.com/2020/08/04/world/coronavirus-cases.html?action=click\&pgtype=Article\&state=default\&region=MAIN_CONTENT_1\&context=storylines_live_updates}{See
more updates}

More live coverage:
\href{https://www.nytimes.com/live/2020/08/04/business/stock-market-today-coronavirus?action=click\&pgtype=Article\&state=default\&region=MAIN_CONTENT_1\&context=storylines_live_updates}{Markets}

And yet, even for the lucky ones, it all feels so fragile. The
difference between the imaginary world of Camus and our very real world
today is that the virus is not --- and may never be --- truly contained.
Our societies are porous. Our lives are networked and global.

So in the bookstore, the clerk, when she heard our American accents,
told us that she had Australian friends who were forced to leave the
U.S. because the Trump administration --- its anti-immigration urges
empowered by the pandemic --- had canceled their visas.

At the restaurant, one of the owners told me that while they were doing
well there, he had to close another restaurant he had opened only
recently. ``The places that are not well established, they're not going
to make it,'' he said.

Another divide, another wall: between the businesses that survive and
those that do not; between those with work and those without.

What should we do and how should we feel about the uneven distribution
of pandemic consequences?

This, it seems to me, is a question that we'll be grappling with for a
while. The Australian government's
\href{https://www.abc.net.au/news/2020-07-21/jobkeeper-jobseeker-extended-rates-cut-coronavirus-morrison/12475716}{adjustments
to the JobKeeper program} this week, extending wage subsidies into next
year but curtailing the amounts, signals that officials have begun to
recognize what it took the residents of Oran a long time to accept ---
that pestilence requires endurance.

Here's Prime Minister Scott Morrison: ``The virus will plot its own
course.''

Here's Camus: ``The preceding months, though they had increased the
desire for liberation, had also taught them to count less and less on a
rapid end to the epidemic.''

In this middle moment between initial infection and final eradication
--- the very lengthy moment ``The Plague'' focused on --- Camus's
characters prioritized perseverance and small acts of kindness. Flipping
back through the book as I wrote this, the power of consistency
re-emerged. Dr. Bernard Rieux, ``struggling against the world as it
was,'' provided both treatment for the sick and an ear for the struggles
of the anxious.

He was a hero of small things, a hero of careful carrying on. Perhaps
that's what we all need to aim for as well.

\href{https://www.nytimes.com/news-event/coronavirus?action=click\&pgtype=Article\&state=default\&region=MAIN_CONTENT_3\&context=storylines_faq}{}

\hypertarget{the-coronavirus-outbreak-}{%
\subsubsection{The Coronavirus Outbreak
›}\label{the-coronavirus-outbreak-}}

\hypertarget{frequently-asked-questions}{%
\paragraph{Frequently Asked
Questions}\label{frequently-asked-questions}}

Updated August 4, 2020

\begin{itemize}
\item ~
  \hypertarget{i-have-antibodies-am-i-now-immune}{%
  \paragraph{I have antibodies. Am I now
  immune?}\label{i-have-antibodies-am-i-now-immune}}

  \begin{itemize}
  \tightlist
  \item
    As of right
    now,\href{https://www.nytimes.com/2020/07/22/health/covid-antibodies-herd-immunity.html?action=click\&pgtype=Article\&state=default\&region=MAIN_CONTENT_3\&context=storylines_faq}{that
    seems likely, for at least several months.} There have been
    frightening accounts of people suffering what seems to be a second
    bout of Covid-19. But experts say these patients may have a
    drawn-out course of infection, with the virus taking a slow toll
    weeks to months after initial exposure. People infected with the
    coronavirus typically
    \href{https://www.nature.com/articles/s41586-020-2456-9}{produce}
    immune molecules called antibodies, which are
    \href{https://www.nytimes.com/2020/05/07/health/coronavirus-antibody-prevalence.html?action=click\&pgtype=Article\&state=default\&region=MAIN_CONTENT_3\&context=storylines_faq}{protective
    proteins made in response to an
    infection}\href{https://www.nytimes.com/2020/05/07/health/coronavirus-antibody-prevalence.html?action=click\&pgtype=Article\&state=default\&region=MAIN_CONTENT_3\&context=storylines_faq}{.
    These antibodies may} last in the body
    \href{https://www.nature.com/articles/s41591-020-0965-6}{only two to
    three months}, which may seem worrisome, but that's perfectly normal
    after an acute infection subsides, said Dr. Michael Mina, an
    immunologist at Harvard University. It may be possible to get the
    coronavirus again, but it's highly unlikely that it would be
    possible in a short window of time from initial infection or make
    people sicker the second time.
  \end{itemize}
\item ~
  \hypertarget{im-a-small-business-owner-can-i-get-relief}{%
  \paragraph{I'm a small-business owner. Can I get
  relief?}\label{im-a-small-business-owner-can-i-get-relief}}

  \begin{itemize}
  \tightlist
  \item
    The
    \href{https://www.nytimes.com/article/small-business-loans-stimulus-grants-freelancers-coronavirus.html?action=click\&pgtype=Article\&state=default\&region=MAIN_CONTENT_3\&context=storylines_faq}{stimulus
    bills enacted in March} offer help for the millions of American
    small businesses. Those eligible for aid are businesses and
    nonprofit organizations with fewer than 500 workers, including sole
    proprietorships, independent contractors and freelancers. Some
    larger companies in some industries are also eligible. The help
    being offered, which is being managed by the Small Business
    Administration, includes the Paycheck Protection Program and the
    Economic Injury Disaster Loan program. But lots of folks have
    \href{https://www.nytimes.com/interactive/2020/05/07/business/small-business-loans-coronavirus.html?action=click\&pgtype=Article\&state=default\&region=MAIN_CONTENT_3\&context=storylines_faq}{not
    yet seen payouts.} Even those who have received help are confused:
    The rules are draconian, and some are stuck sitting on
    \href{https://www.nytimes.com/2020/05/02/business/economy/loans-coronavirus-small-business.html?action=click\&pgtype=Article\&state=default\&region=MAIN_CONTENT_3\&context=storylines_faq}{money
    they don't know how to use.} Many small-business owners are getting
    less than they expected or
    \href{https://www.nytimes.com/2020/06/10/business/Small-business-loans-ppp.html?action=click\&pgtype=Article\&state=default\&region=MAIN_CONTENT_3\&context=storylines_faq}{not
    hearing anything at all.}
  \end{itemize}
\item ~
  \hypertarget{what-are-my-rights-if-i-am-worried-about-going-back-to-work}{%
  \paragraph{What are my rights if I am worried about going back to
  work?}\label{what-are-my-rights-if-i-am-worried-about-going-back-to-work}}

  \begin{itemize}
  \tightlist
  \item
    Employers have to provide
    \href{https://www.osha.gov/SLTC/covid-19/standards.html}{a safe
    workplace} with policies that protect everyone equally.
    \href{https://www.nytimes.com/article/coronavirus-money-unemployment.html?action=click\&pgtype=Article\&state=default\&region=MAIN_CONTENT_3\&context=storylines_faq}{And
    if one of your co-workers tests positive for the coronavirus, the
    C.D.C.} has said that
    \href{https://www.cdc.gov/coronavirus/2019-ncov/community/guidance-business-response.html}{employers
    should tell their employees} -\/- without giving you the sick
    employee's name -\/- that they may have been exposed to the virus.
  \end{itemize}
\item ~
  \hypertarget{should-i-refinance-my-mortgage}{%
  \paragraph{Should I refinance my
  mortgage?}\label{should-i-refinance-my-mortgage}}

  \begin{itemize}
  \tightlist
  \item
    \href{https://www.nytimes.com/article/coronavirus-money-unemployment.html?action=click\&pgtype=Article\&state=default\&region=MAIN_CONTENT_3\&context=storylines_faq}{It
    could be a good idea,} because mortgage rates have
    \href{https://www.nytimes.com/2020/07/16/business/mortgage-rates-below-3-percent.html?action=click\&pgtype=Article\&state=default\&region=MAIN_CONTENT_3\&context=storylines_faq}{never
    been lower.} Refinancing requests have pushed mortgage applications
    to some of the highest levels since 2008, so be prepared to get in
    line. But defaults are also up, so if you're thinking about buying a
    home, be aware that some lenders have tightened their standards.
  \end{itemize}
\item ~
  \hypertarget{what-is-school-going-to-look-like-in-september}{%
  \paragraph{What is school going to look like in
  September?}\label{what-is-school-going-to-look-like-in-september}}

  \begin{itemize}
  \tightlist
  \item
    It is unlikely that many schools will return to a normal schedule
    this fall, requiring the grind of
    \href{https://www.nytimes.com/2020/06/05/us/coronavirus-education-lost-learning.html?action=click\&pgtype=Article\&state=default\&region=MAIN_CONTENT_3\&context=storylines_faq}{online
    learning},
    \href{https://www.nytimes.com/2020/05/29/us/coronavirus-child-care-centers.html?action=click\&pgtype=Article\&state=default\&region=MAIN_CONTENT_3\&context=storylines_faq}{makeshift
    child care} and
    \href{https://www.nytimes.com/2020/06/03/business/economy/coronavirus-working-women.html?action=click\&pgtype=Article\&state=default\&region=MAIN_CONTENT_3\&context=storylines_faq}{stunted
    workdays} to continue. California's two largest public school
    districts --- Los Angeles and San Diego --- said on July 13, that
    \href{https://www.nytimes.com/2020/07/13/us/lausd-san-diego-school-reopening.html?action=click\&pgtype=Article\&state=default\&region=MAIN_CONTENT_3\&context=storylines_faq}{instruction
    will be remote-only in the fall}, citing concerns that surging
    coronavirus infections in their areas pose too dire a risk for
    students and teachers. Together, the two districts enroll some
    825,000 students. They are the largest in the country so far to
    abandon plans for even a partial physical return to classrooms when
    they reopen in August. For other districts, the solution won't be an
    all-or-nothing approach.
    \href{https://bioethics.jhu.edu/research-and-outreach/projects/eschool-initiative/school-policy-tracker/}{Many
    systems}, including the nation's largest, New York City, are
    devising
    \href{https://www.nytimes.com/2020/06/26/us/coronavirus-schools-reopen-fall.html?action=click\&pgtype=Article\&state=default\&region=MAIN_CONTENT_3\&context=storylines_faq}{hybrid
    plans} that involve spending some days in classrooms and other days
    online. There's no national policy on this yet, so check with your
    municipal school system regularly to see what is happening in your
    community.
  \end{itemize}
\end{itemize}

``The decent man,'' Camus wrote, ``the one who doesn't infect anybody,
is the one who concentrates most.''

What are you reading for insight on the pandemic? Tell us at
\href{mailto:nytaustralia@nytimes.com}{\nolinkurl{nytaustralia@nytimes.com}}.
Our Book Review editors also have a few suggestions:
\href{https://www.nytimes.com/2020/02/24/books/pandemic-books-coronavirus.html}{nonfiction}
and
\href{https://www.nytimes.com/2020/03/12/books/coronavirus-reading.html}{fiction}.

Now here are our stories of the week.

\begin{center}\rule{0.5\linewidth}{\linethickness}\end{center}

\hypertarget{around-the-times}{%
\subsection{Around the Times}\label{around-the-times}}

\includegraphics{https://static01.nyt.com/images/2020/07/18/world/00virus-thailand-photos-01/merlin_173899668_3cb53f8d-f789-4496-9f25-0218e93a0037-articleLarge.jpg?quality=75\&auto=webp\&disable=upscale}

\begin{itemize}
\item
  \textbf{\href{https://www.nytimes.com/2020/07/16/world/asia/coronavirus-thailand-photos.html}{No
  One Knows What Thailand Is Doing Right, but So Far, It's Working}:}
  Can the country's low rate of coronavirus infections be attributed to
  culture? Genetics? Face masks? Or a combination of all three?
\item
  \textbf{\href{https://www.nytimes.com/2020/07/22/health/covid-antibodies-herd-immunity.html}{Can
  You Get Covid-19 Again? It's Very Unlikely}:} Reports of reinfection
  instead may be cases of drawn-out illness. A decline in antibodies is
  normal after a few weeks, and people are protected from the
  coronavirus in other ways.
\item
  \href{https://www.nytimes.com/2020/07/22/arts/comedy-funny-podcasts.html}{7}\textbf{\href{https://www.nytimes.com/2020/07/22/arts/comedy-funny-podcasts.html}{Podcasts
  to Lighten the Mood}:} With TV production mostly on pause and comedy
  clubs off-limits because of the pandemic, check out these podcasts for
  some much-needed comic relief.
\item
  \textbf{\href{https://www.nytimes.com/interactive/2020/science/coronavirus-vaccine-tracker.html}{Coronavirus
  Vaccine Tracker}:} Researchers around the world are developing more
  than 165 vaccines against the coronavirus, and 27 vaccines are in
  human trials. Let us help you keep up with the latest developments.
\item
  \textbf{\href{https://www.nytimes.com/2020/07/23/world/australia/lawsuit-climate-change-bonds.html}{Australian
  Student Sues Government Over Financial Risks of Climate Change}:} A
  23-year-old law student filed a class-action suit accusing Australia
  of failing to disclose financial risks from climate change. Experts
  say it is the first of its kind.
\item
  \textbf{\href{https://www.nytimes.com/2020/07/22/books/review-intimations-essays-zadie-smith.html}{In
  `Intimations,' Zadie Smith Applies Her Even Temper to Tumultuous
  Times}:} This short essay collection includes Smith's recent thoughts
  on the coronavirus pandemic, race relations in America and other
  subjects.
\item
  \textbf{\href{https://www.nytimes.com/interactive/2020/07/22/magazine/zambia-kariba-dam.html}{Learning
  From the Kariba Dam}:} Climate change and neglect have brought the
  mammoth structure at the border of Zambia and Zimbabwe to the brink of
  calamity --- a crisis prefigured in the dam's troubling colonial
  history.
\end{itemize}

\begin{center}\rule{0.5\linewidth}{\linethickness}\end{center}

Enjoying the Australia Letter?
\href{https://www.nytimes.com/newsletters/australia-letter?utm_source=ausend}{Sign
up here} or forward to a friend.

For more Australia coverage and discussion, start your day with your
local
\href{https://www.nytimes.com/interactive/2018/briefing/global-morning-briefing-newsletter-signup.html?utm_source=ausend}{Morning
Briefing} and join us in our
\href{https://www.facebook.com/groups/nytaustralia/}{Facebook group}.

Advertisement

\protect\hyperlink{after-bottom}{Continue reading the main story}

\hypertarget{site-index}{%
\subsection{Site Index}\label{site-index}}

\hypertarget{site-information-navigation}{%
\subsection{Site Information
Navigation}\label{site-information-navigation}}

\begin{itemize}
\tightlist
\item
  \href{https://help.nytimes.com/hc/en-us/articles/115014792127-Copyright-notice}{©~2020~The
  New York Times Company}
\end{itemize}

\begin{itemize}
\tightlist
\item
  \href{https://www.nytco.com/}{NYTCo}
\item
  \href{https://help.nytimes.com/hc/en-us/articles/115015385887-Contact-Us}{Contact
  Us}
\item
  \href{https://www.nytco.com/careers/}{Work with us}
\item
  \href{https://nytmediakit.com/}{Advertise}
\item
  \href{http://www.tbrandstudio.com/}{T Brand Studio}
\item
  \href{https://www.nytimes.com/privacy/cookie-policy\#how-do-i-manage-trackers}{Your
  Ad Choices}
\item
  \href{https://www.nytimes.com/privacy}{Privacy}
\item
  \href{https://help.nytimes.com/hc/en-us/articles/115014893428-Terms-of-service}{Terms
  of Service}
\item
  \href{https://help.nytimes.com/hc/en-us/articles/115014893968-Terms-of-sale}{Terms
  of Sale}
\item
  \href{https://spiderbites.nytimes.com}{Site Map}
\item
  \href{https://help.nytimes.com/hc/en-us}{Help}
\item
  \href{https://www.nytimes.com/subscription?campaignId=37WXW}{Subscriptions}
\end{itemize}
