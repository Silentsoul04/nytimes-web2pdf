Sections

SEARCH

\protect\hyperlink{site-content}{Skip to
content}\protect\hyperlink{site-index}{Skip to site index}

\href{https://www.nytimes.com/section/world/middleeast}{Middle East}

\href{https://myaccount.nytimes.com/auth/login?response_type=cookie\&client_id=vi}{}

\href{https://www.nytimes.com/section/todayspaper}{Today's Paper}

\href{/section/world/middleeast}{Middle East}\textbar{}A Saudi Spy Chief
Hid Abroad. With Appeals and Threats, M.B.S. Tried to Bring Him Back

\url{https://nyti.ms/2ZTRd96}

\begin{itemize}
\item
\item
\item
\item
\item
\end{itemize}

Advertisement

\protect\hyperlink{after-top}{Continue reading the main story}

Supported by

\protect\hyperlink{after-sponsor}{Continue reading the main story}

\hypertarget{a-saudi-spy-chief-hid-abroad-with-appeals-and-threats-mbs-tried-to-bring-him-back}{%
\section{A Saudi Spy Chief Hid Abroad. With Appeals and Threats, M.B.S.
Tried to Bring Him
Back}\label{a-saudi-spy-chief-hid-abroad-with-appeals-and-threats-mbs-tried-to-bring-him-back}}

To try to force a former Saudi intelligence officer to return to the
kingdom, Crown Prince Mohammed bin Salman asked for his help, targeted
his family and sought to have him arrested abroad.

\includegraphics{https://static01.nyt.com/images/2020/07/24/world/24saudi-interpoltop/24saudi-interpoltop-articleLarge.jpg?quality=75\&auto=webp\&disable=upscale}

\href{https://www.nytimes.com/by/ben-hubbard}{\includegraphics{https://static01.nyt.com/images/2018/10/10/multimedia/author-ben-hubbard/author-ben-hubbard-thumbLarge.png}}

By \href{https://www.nytimes.com/by/ben-hubbard}{Ben Hubbard}

\begin{itemize}
\item
  July 24, 2020
\item
  \begin{itemize}
  \item
  \item
  \item
  \item
  \item
  \end{itemize}
\end{itemize}

BEIRUT, Lebanon --- As Crown Prince Mohammed bin Salman of Saudi Arabia
sidelined rivals to consolidate power a few years ago, a former Saudi
intelligence official feared that he would end up in the prince's sights
and slipped out of the kingdom.

The prince has been trying to get him back since, first asking the
former official, Saad Aljabri, to come home for a new job, then trying
unsuccessfully to have him extradited on corruption charges through
Interpol, according to text messages and legal documents reviewed by The
New York Times.

``You are involved in many large cases of corruption that have been
proven,'' Prince Mohammed wrote to the former official in September
2017. ``There is no state in the world that would refuse to turn you
over.''

But Interpol questioned the Saudi commitment to due process and human
rights in the kingdom's handling of corruption cases and deemed the
Saudi request for Mr. Aljabri politically motivated, a violation of the
organization's rules, according to Interpol documents. So it removed Mr.
Aljabri's name from its system.

The text messages and documents reviewed by The Times, which have not
been previously reported, shed new light on how far Prince Mohammed has
reached to exert control over Saudis he fears could subvert him.

The struggle has accelerated this year. In March, Saudi Arabia detained
\href{https://www.nytimes.com/2020/05/21/world/middleeast/saudi-aljabri-detain.html}{two
of Mr. Aljabri's adult children and his brother}, prompting accusations
by relatives and United States officials that they were being held
hostage to secure Mr. Aljabri's return.

And last week, the kingdom's state-controlled news media seized upon
\href{https://www.wsj.com/articles/a-spymaster-ran-off-after-saudis-say-billions-went-missing-they-want-him-back-11595004443}{an
article} in the The Wall Street Journal that cited unidentified Saudi
officials accusing Mr. Aljabri of misspending billions of dollars in
state funds to enrich himself and relatives. One Saudi newspaper
published a wanted poster with Mr. Aljabri's face on it, part of an
apparent effort to tarnish his reputation in the kingdom.

The revelations come amid concerns about the health of Prince Mohammed's
father, King Salman, whose death could put the prince in charge of Saudi
Arabia for decades. The king, 84, was hospitalized over the weekend and
underwent successful gall bladder surgery,
\href{https://english.alarabiya.net/en/News/gulf/2020/07/23/Saudi-Arabia-s-King-Salman-undergoes-successful-gallbladder-surgery-SPA.html}{Saudi
state media reported Thursday.}

Since his father became king in 2015, Prince Mohammed, 34, has taken
charge of military, economic and social policies while targeting critics
and foes with travel bans, detentions and lawsuits.

These increasingly authoritarian tactics caught global attention when
Saudi agents
\href{https://www.nytimes.com/video/world/middleeast/100000006154117/khashoggi-istanbul-death-saudi-consulate.html}{killed}
\href{https://www.nytimes.com/2018/10/14/world/middleeast/jamal-khashoggi-saudi-arabia.html}{Jamal
Khashoggi}, the dissident Saudi writer, inside the Saudi consulate in
Istanbul in 2018, bringing widespread condemnation.

The Saudi moves against Mr. Aljabri have drawn attention in Washington,
where many officials considered him a valuable intelligence partner.

\includegraphics{https://static01.nyt.com/images/2020/07/23/world/00Saudi-Interpol05/merlin_172709409_2c09d0af-6275-472a-a50f-7f6405b6c8d4-articleLarge.jpg?quality=75\&auto=webp\&disable=upscale}

In \href{https://twitter.com/SenatorLeahy/status/1281281725499408386}{a
letter to President Trump} this month, four senators referred to Mr.
Aljabri as ``a close U.S. ally and friend'' and said the United States
had ``a moral obligation to do what it can to assist in securing his
children's freedom.''

Officials at the Saudi Embassy in Washington did not respond to requests
for comment about the text messages between Prince Mohammed and Mr.
Aljabri, the Saudi Interpol request or the kingdom's corruption
allegations.

The Times reviewed scores of text messages between the two men provided
by a law firm working for Mr. Aljabri,
\href{https://www.nortonrosefulbright.com/en-ca}{Norton Rose Fulbright
Canada}, and Interpol documents informing Mr. Aljabri of its decision
about the Saudi request against him.

Mr. Aljabri's rise and fall were tied to his association with Prince
Mohammed's primary rival for the Saudi throne, Prince Mohammed bin
Nayef, who headed the Interior Ministry and became crown prince in 2015.

A linguist with a doctorate in artificial intelligence, Mr. Aljabri
became a top official at the ministry, which handles security and
counterterrorism, putting him in regular contact with U.S. diplomats and
officials from the Central Intelligence Agency. Many have praised his
professionalism.

``Aljabri is really smart, and he has encyclopedic knowledge,'' said
Douglas London, a former officer in the C.I.A.'s
\href{https://www.cia.gov/offices-of-cia/clandestine-service/index.html}{Clandestine
Service} and nonresident scholar at the Middle East Institute in
Washington. ``He lived up to his word, he did not over-promise and he
delivered.''

But Mr. Aljabri's star fell as Prince Mohammed's rose. Mr. Aljabri was
dismissed by royal decree in 2015.

In 2017, Mr. Aljabri began to fear that Prince Mohammed intended to
replace Mohammed bin Nayef as crown prince and target his domestic
allies, so Mr. Aljabri left the kingdom, settling in Turkey.

On June 18 of that year, Prince Mohammed texted him, asking Mr. Aljabri
to return to help solve an unspecified issue with Mohammed bin Nayef,
according to translated versions of texts provided by Mr. Aljabri's law
firm.

``I want to explain to you what has happened recently and come to an
agreement with you about a strategy to solve all these difficulties,''
Prince Mohammed wrote.

Mr. Aljabri replied that he was ``prepared to accept whatever you
command.''

Prince Mohammed said he wanted the three men to meet so they could
``reconcile and everything can return to the way it was.''

On June 20, Mr. Aljabri said he could not return to Saudi Arabia
immediately because of medical treatment. Prince Mohammed said he had
only summoned him because he was ``in dire need of your assistance.''

The next day, however, Prince Mohammed
\href{https://www.nytimes.com/2017/07/18/world/middleeast/saudi-arabia-mohammed-bin-nayef-mohammed-bin-salman.html}{ousted
Mohammed bin Nayef as crown prince} and took his place. Mohammed bin
Nayef was placed under house arrest, and two of Mr. Aljabri's children,
Sarah, who was 17 at the time, and Omar, who was 18, were barred from
leaving Saudi Arabia.

Image

Omar Aljabri, 21, and his sister Sarah, 20, right, were arrested at
their home in Riyadh in March.Credit...via Aljabri family

Image

The two adult children of Saad Aljabri have not been seen since their
arrest.Credit...via Aljabri family

Mr. Aljabri wrote to pledge allegiance to Prince Mohammed as crown
prince, and Prince Mohammed encouraged him to return for an important
new job.

``When you return safely, I will explain to you the background to the
problem,'' Prince Mohammed wrote. ``I will still need you to deal with
anyone who attempts to create disorder and conflict.''

Mr. Aljabri asked Prince Mohammed to lift the travel ban on his
children. Prince Mohammed did not respond.

Three months later, Mr. Aljabri asked Prince Mohammed again to lift the
travel ban ``to allow them to leave so that they may finish their
studies.''

``When I see you, I will explain to you the background,'' Prince
Mohammed responded.

Mr. Aljabri repeated his request.

``When I see you, I will explain everything to you,'' Prince Mohammed
wrote.

A few days later, Prince Mohammed asked Mr. Aljabri to return to Saudi
Arabia the next day, linking his return to the travel ban on Mr.
Aljabri's children.

``I want to resolve this problem of your son and daughter, but this is a
very sensitive file here'' related to Mohammed bin Nayef, Prince
Mohammed wrote. ``I want your opinion about it as well as information
from you concerning it. I also want to come to an understanding with you
regarding your future situation and what the details should be.''

Soon after, Prince Mohammed texted again, this time threatening to have
Mr. Aljabri arrested abroad.

With the danger now clear, Mr. Aljabri moved from Turkey to Canada,
according to his son, Khalid Aljabri, a cardiologist also based in
Canada.

To try to force him home, the Saudi authorities filed a notice with
Interpol, the international police organization, asking other nations to
help with Mr. Aljabri's extradition, according to Interpol documents.

But instead of filing for a
\href{https://www.interpol.int/en/How-we-work/Notices/Red-Notices}{Red
Notice}, which acts like an international arrest warrant, the Saudis
filed a
\href{https://www.interpol.int/en/How-we-work/Notices/About-Notices}{diffusion},
which Interpol describes as a less formal way for Interpol members to
request help from other nations.

Image

Mr. Aljabri's rise and fall were tied to his association with Prince
Mohammed's primary rival for the Saudi throne, Prince Mohammed bin
Nayef, who was ousted in 2017.Credit...Ahmet Bolat/Anadolu Agency, via
Getty Images

Mr. Aljabri confirmed that his name was in the Interpol system in
December 2017, when his wife and other relatives were barred from flying
from Turkey to Canada because their party contained another Saad
Aljabri: Mr. Aljabri's infant grandson and namesake, Dr. Aljabri said.

The family nonetheless managed to get to Canada via the United States
and appealed the inclusion of Mr. Aljabri's name in the Interpol system.

They won in July 2018, according to an Interpol document about the
decision.

It did not detail the charges Saudi Arabia had made against Mr. Aljabri
or any evidence the kingdom had provided.

But in rejecting the Saudi request, the commission criticized the
kingdom's previous handling of corruption cases for ``the lack of due
process and human rights guarantees.''

The commission cited Prince Mohammed's crackdown in 2017, when hundreds
of the kingdom's richest and most prominent businessmen were locked
\href{https://www.nytimes.com/2018/03/11/world/middleeast/saudi-arabia-corruption-mohammed-bin-salman.html}{in
the Riyadh Ritz-Carlton} and accused of corruption. Many were abused and
at least one died from mistreatment, medics and associates of the
detainees said.

The Interpol commission wrote that the anti-corruption committee that
oversaw that crackdown was ``part of a political strategy by MBS to
target any potential political rival or opposition.''

The kingdom soon found other ways to pressure Mr. Aljabri.

In March, his two adult children who had been barred from leaving the
kingdom were arrested in their Riyadh home. In May, Mr. Aljabri's
brother was arrested. None have contacted their relatives since, Dr.
Aljabri said.

``The Saudi royal family is holding Sarah and Omar Aljabri as
hostages,'' Senator Patrick J. Leahy of Vermont
\href{https://twitter.com/SenatorLeahy/status/1281281725499408386}{wrote
on Twitter} this month with the letter from him and three other senators
to Mr. Trump. ``For a government to use such tactics is abhorrent. They
should be released immediately.''

Advertisement

\protect\hyperlink{after-bottom}{Continue reading the main story}

\hypertarget{site-index}{%
\subsection{Site Index}\label{site-index}}

\hypertarget{site-information-navigation}{%
\subsection{Site Information
Navigation}\label{site-information-navigation}}

\begin{itemize}
\tightlist
\item
  \href{https://help.nytimes.com/hc/en-us/articles/115014792127-Copyright-notice}{©~2020~The
  New York Times Company}
\end{itemize}

\begin{itemize}
\tightlist
\item
  \href{https://www.nytco.com/}{NYTCo}
\item
  \href{https://help.nytimes.com/hc/en-us/articles/115015385887-Contact-Us}{Contact
  Us}
\item
  \href{https://www.nytco.com/careers/}{Work with us}
\item
  \href{https://nytmediakit.com/}{Advertise}
\item
  \href{http://www.tbrandstudio.com/}{T Brand Studio}
\item
  \href{https://www.nytimes.com/privacy/cookie-policy\#how-do-i-manage-trackers}{Your
  Ad Choices}
\item
  \href{https://www.nytimes.com/privacy}{Privacy}
\item
  \href{https://help.nytimes.com/hc/en-us/articles/115014893428-Terms-of-service}{Terms
  of Service}
\item
  \href{https://help.nytimes.com/hc/en-us/articles/115014893968-Terms-of-sale}{Terms
  of Sale}
\item
  \href{https://spiderbites.nytimes.com}{Site Map}
\item
  \href{https://help.nytimes.com/hc/en-us}{Help}
\item
  \href{https://www.nytimes.com/subscription?campaignId=37WXW}{Subscriptions}
\end{itemize}
