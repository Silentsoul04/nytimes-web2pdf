Sections

SEARCH

\protect\hyperlink{site-content}{Skip to
content}\protect\hyperlink{site-index}{Skip to site index}

\href{https://www.nytimes.com/section/business}{Business}

\href{https://myaccount.nytimes.com/auth/login?response_type=cookie\&client_id=vi}{}

\href{https://www.nytimes.com/section/todayspaper}{Today's Paper}

\href{/section/business}{Business}\textbar{}Corporate Insiders Pocket
\$1 Billion in Rush for Coronavirus Vaccine

\url{https://nyti.ms/32UN9Hw}

\begin{itemize}
\item
\item
\item
\item
\item
\item
\end{itemize}

\href{https://www.nytimes.com/news-event/coronavirus?action=click\&pgtype=Article\&state=default\&region=TOP_BANNER\&context=storylines_menu}{The
Coronavirus Outbreak}

\begin{itemize}
\tightlist
\item
  live\href{https://www.nytimes.com/2020/08/04/world/coronavirus-cases.html?action=click\&pgtype=Article\&state=default\&region=TOP_BANNER\&context=storylines_menu}{Latest
  Updates}
\item
  \href{https://www.nytimes.com/interactive/2020/us/coronavirus-us-cases.html?action=click\&pgtype=Article\&state=default\&region=TOP_BANNER\&context=storylines_menu}{Maps
  and Cases}
\item
  \href{https://www.nytimes.com/interactive/2020/science/coronavirus-vaccine-tracker.html?action=click\&pgtype=Article\&state=default\&region=TOP_BANNER\&context=storylines_menu}{Vaccine
  Tracker}
\item
  \href{https://www.nytimes.com/2020/08/02/us/covid-college-reopening.html?action=click\&pgtype=Article\&state=default\&region=TOP_BANNER\&context=storylines_menu}{College
  Reopening}
\item
  \href{https://www.nytimes.com/live/2020/08/04/business/stock-market-today-coronavirus?action=click\&pgtype=Article\&state=default\&region=TOP_BANNER\&context=storylines_menu}{Economy}
\end{itemize}

Advertisement

\protect\hyperlink{after-top}{Continue reading the main story}

Supported by

\protect\hyperlink{after-sponsor}{Continue reading the main story}

\hypertarget{corporate-insiders-pocket-1-billion-in-rush-for-coronavirus-vaccine}{%
\section{Corporate Insiders Pocket \$1 Billion in Rush for Coronavirus
Vaccine}\label{corporate-insiders-pocket-1-billion-in-rush-for-coronavirus-vaccine}}

Well-timed stock bets have generated big profits for senior executives
and board members at companies developing vaccines and treatments.

\includegraphics{https://static01.nyt.com/images/2020/07/26/business/26Virus-Vaccine-Payday-shot/merlin_174630441_1f65a11a-cfd7-48ef-9c5c-8454b39c7e17-articleLarge.jpg?quality=75\&auto=webp\&disable=upscale}

By \href{https://www.nytimes.com/by/david-gelles}{David Gelles} and
\href{https://www.nytimes.com/by/jesse-drucker}{Jesse Drucker}

\begin{itemize}
\item
  July 25, 2020
\item
  \begin{itemize}
  \item
  \item
  \item
  \item
  \item
  \item
  \end{itemize}
\end{itemize}

On June 26, a small South San Francisco company called Vaxart made a
surprise announcement: A coronavirus vaccine it was working on had been
selected by the U.S. government to be part of Operation Warp Speed, the
flagship federal initiative to quickly develop drugs to combat Covid-19.

Vaxart's shares soared. Company insiders, who weeks earlier had received
stock options worth a few million dollars, saw the value of those awards
increase sixfold. And a hedge fund that partly controlled the company
walked away with more than \$200 million in instant profits.

The race is on to develop a coronavirus vaccine, and some companies and
investors are betting that the winners stand to earn vast profits from
selling hundreds of millions --- or even billions --- of doses to a
desperate public.

Across the pharmaceutical and medical industries, senior executives and
board members are capitalizing on that dynamic.

They are making millions of dollars after announcing positive
developments, including support from the government, in their efforts to
fight Covid-19. After such announcements, insiders from at least 11
companies --- most of them smaller firms whose fortunes often hinge on
the success or failure of a single drug --- have sold shares worth well
over \$1 billion since March, according to figures compiled for The New
York Times by Equilar, a data provider.

In some cases, company insiders are profiting from regularly scheduled
compensation or automatic stock trades. But in other situations, senior
officials appear to be pouncing on opportunities to cash out while their
stock prices are sky high. And some companies have awarded stock options
to executives shortly before market-moving announcements about their
vaccine progress.

The sudden windfalls highlight the powerful financial incentives for
company officials to generate positive headlines in
\href{https://www.nytimes.com/interactive/2020/science/coronavirus-vaccine-tracker.html}{the
race for coronavirus vaccines and treatments}, even if the drugs might
never pan out.

Some companies are attracting government scrutiny for potentially using
their associations with Operation Warp Speed as marketing ploys.

For example, the headline on Vaxart's news release declared: ``Vaxart's
Covid-19 Vaccine Selected for the U.S. Government's Operation Warp
Speed.'' But the reality is more complex.

Vaxart's vaccine candidate was included in a trial on primates that a
federal agency was organizing in conjunction with Operation Warp Speed.
But Vaxart is not among the companies selected to receive significant
financial support from Warp Speed to produce hundreds of millions of
vaccine doses.

``The U.S. Department of Health and Human Services has entered into
funding agreements with certain vaccine manufacturers, and we are
negotiating with others. Neither is the case with Vaxart,'' said Michael
R. Caputo, the department's assistant secretary for public affairs.
``Vaxart's vaccine candidate was selected to participate in preliminary
U.S. government studies to determine potential areas for possible
Operation Warp Speed partnership and support. At this time, those
studies are ongoing, and no determinations have been made.''

\hypertarget{latest-updates-economy}{%
\section{\texorpdfstring{\href{https://www.nytimes.com/live/2020/08/04/business/stock-market-today-coronavirus?action=click\&pgtype=Article\&state=default\&region=MAIN_CONTENT_1\&context=storylines_live_updates}{Latest
Updates:
Economy}}{Latest Updates: Economy}}\label{latest-updates-economy}}

\href{https://www.nytimes.com/live/2020/08/04/business/stock-market-today-coronavirus?action=click\&pgtype=Article\&state=default\&region=MAIN_CONTENT_1\&context=storylines_live_updates\#fox-corporations-plunging-profit-is-cushioned-by-fox-news}{11h
ago}

\href{https://www.nytimes.com/live/2020/08/04/business/stock-market-today-coronavirus?action=click\&pgtype=Article\&state=default\&region=MAIN_CONTENT_1\&context=storylines_live_updates\#fox-corporations-plunging-profit-is-cushioned-by-fox-news}{Fox
Corporation's plunging profit is cushioned by Fox News.}

\href{https://www.nytimes.com/live/2020/08/04/business/stock-market-today-coronavirus?action=click\&pgtype=Article\&state=default\&region=MAIN_CONTENT_1\&context=storylines_live_updates\#trading-in-kodak-shares-comes-under-scrutiny}{11h
ago}

\href{https://www.nytimes.com/live/2020/08/04/business/stock-market-today-coronavirus?action=click\&pgtype=Article\&state=default\&region=MAIN_CONTENT_1\&context=storylines_live_updates\#trading-in-kodak-shares-comes-under-scrutiny}{Trading
in Kodak shares comes under scrutiny.}

\href{https://www.nytimes.com/live/2020/08/04/business/stock-market-today-coronavirus?action=click\&pgtype=Article\&state=default\&region=MAIN_CONTENT_1\&context=storylines_live_updates\#disney-lost-4-7-billion-last-quarter-but-its-newest-business-was-a-big-hit}{12h
ago}

\href{https://www.nytimes.com/live/2020/08/04/business/stock-market-today-coronavirus?action=click\&pgtype=Article\&state=default\&region=MAIN_CONTENT_1\&context=storylines_live_updates\#disney-lost-4-7-billion-last-quarter-but-its-newest-business-was-a-big-hit}{Disney
lost \$4.7 billion last quarter, but its newest business was a big hit.}

\href{https://www.nytimes.com/live/2020/08/04/business/stock-market-today-coronavirus?action=click\&pgtype=Article\&state=default\&region=MAIN_CONTENT_1\&context=storylines_live_updates}{See
more updates}

More live coverage:
\href{https://www.nytimes.com/2020/08/04/world/coronavirus-cases.html?action=click\&pgtype=Article\&state=default\&region=MAIN_CONTENT_1\&context=storylines_live_updates}{Global}

Some officials at the Department of Health and Human Services have grown
concerned about whether companies including Vaxart are trying to inflate
their stock prices by exaggerating their roles in Warp Speed, a senior
Trump administration official said. The department has relayed those
concerns to the Securities and Exchange Commission, said the official,
who spoke on the condition of anonymity.

It isn't clear if the commission is looking into the matter. An S.E.C.
spokeswoman declined to comment.

\includegraphics{https://static01.nyt.com/images/2020/07/26/business/26Virus-Vaccine-payday-floroiu/26Virus-Vaccine-payday-floroiu-articleLarge.jpg?quality=75\&auto=webp\&disable=upscale}

``Vaxart abides by good corporate governance guidelines and policies and
makes decisions in accordance with the best interests of the company and
its shareholders,'' Vaxart's chief executive, Andrei Floroiu, said in a
statement on Friday. Referring to Operation Warp Speed, he added, ``We
believe that Vaxart's Covid-19 vaccine is the most exciting one in
O.W.S. because it is the only oral vaccine (a pill) in O.W.S.''

Well-timed stock transactions are generally legal. But investors and
corporate governance experts say they can create the appearance that
executives are profiting from inside information, and could erode public
confidence in the pharmaceutical industry when the world is looking to
these companies to cure Covid-19.

``It is inappropriate for drug company executives to cash in on a
crisis,'' said Ben Wakana, executive director of Patients for Affordable
Drugs, a nonprofit advocacy group. ``Every day, Americans wake up and
make sacrifices during this pandemic. Drug companies see this as a
payday.''

Executives at a long list of companies have reaped seven- or
eight-figure profits thanks to their work on coronavirus vaccines and
treatments.

Shares of Regeneron, a biotech company in Tarrytown, N.Y., have climbed
nearly 80 percent since early February, when it announced a
collaboration with the Department of Health and Human Services to
develop a Covid-19 treatment. Since then, the company's top executives
and board members have sold nearly \$700 million in stock. The chief
executive, Leonard Schleifer, sold \$178 million of shares on a single
day in May.

Alexandra Bowie, a spokeswoman for Regeneron, said most of those sales
had been scheduled in advance through programs that automatically sell
executives' shares if the stock hits a certain price.

Moderna, a 10-year-old vaccine developer based in Cambridge, Mass., that
has never brought a product to market, announced in late January that it
was working on a coronavirus vaccine. It has issued a stream of news
releases hailing its vaccine progress, and its stock has more than
tripled, giving the company a market value of almost \$30 billion.

Moderna insiders have sold about \$248 million of shares since that
January announcement, most of it after the company
\href{https://investors.modernatx.com/news-releases/news-release-details/moderna-announces-award-us-government-agency-barda-483-million}{was
selected in April} to receive federal funding to support its vaccine
efforts.

Image

The stock of Moderna, which has its headquarters in Cambridge, Mass.,
has more than tripled during its work on a vaccine.~Credit...Adam
Glanzman/Bloomberg

While some of those sales were scheduled in advance, others were more
spur of the moment. Flagship Ventures, an investment fund run by the
company's founder and chairman, Noubar Afeyan, sold more than \$68
million worth of Moderna shares on May 21. Those transactions were not
scheduled in advance, according to securities filings.

Executives and board members at Luminex, Quidel and Emergent
BioSolutions have sold shares worth a combined \$85 million after
announcing they were working on vaccines, treatments or testing
solutions.

At other companies, executives and board members received large grants
of stock options shortly before the companies announced good news that
lifted the value of those options.

Novavax,
\href{https://www.nytimes.com/2020/07/16/health/coronavirus-vaccine-novavax.html}{a
drugmaker} in Gaithersburg, Md., began working on a vaccine early this
year. This spring, the company reported promising preliminary test
results and a
\href{https://www.nytimes.com/2020/07/07/health/novavax-coronavirus-vaccine-warp-speed.html}{\$1.6
billion deal} with the Trump administration.

In April, with its shares below \$24, Novavax issued a batch of new
stock awards to all its employees ``in acknowledgment of the
extraordinary work of our employees to implement a new vaccine
program.'' Four senior executives, including the chief executive,
Stanley Erck, received stock options that were worth less than \$20
million at the time.

Since then, Novavax's stock has rocketed to more than \$130 a share. At
least on paper, the four executives' stock options are worth more than
\$100 million.

So long as the company hits a milestone with its vaccine testing, which
it is expected to achieve soon, the executives will be able to use the
options to buy discounted Novavax shares as early as next year,
regardless of whether the company develops a successful vaccine.

Silvia Taylor, a Novavax spokeswoman, said the stock awards were
designed ``to incentivize and retain our employees during this critical
time.'' She added that ``there is no guarantee they will retain their
value.''

Two other drugmakers, Translate Bio and Inovio, awarded large batches of
stock options to executives and board members shortly before they
announced progress on their coronavirus vaccines, sending shares higher.
Representatives of the companies said the options were regularly
scheduled annual grants.

Vaxart, though, is where the most money was made the fastest.

At the start of the year, its shares were around 35 cents. Then in late
January, Vaxart began working on an orally administered coronavirus
vaccine, and its shares started rising.

Vaxart's largest shareholder was a New York hedge fund, Armistice
Capital, which last year acquired nearly two-thirds of the company's
shares. Two Armistice executives, including the hedge fund's founder,
Steven Boyd,
\href{https://investors.vaxart.com/news-releases/news-release-details/vaxart-inc-announces-changes-its-board-directors}{joined}
Vaxart's board of directors. The hedge fund also purchased rights, known
as warrants, to buy 21 million more Vaxart shares at some point in the
future for as little as 30 cents each.

Image

Selling Vaxart stock made more than \$197 million in profit for
Armistice Capital, a hedge fund that owned two-thirds of the company's
shares.Credit... Rafael Henrique/Getty Images

Vaxart has never brought a vaccine to market. It has just 15 employees.
But throughout the spring, Vaxart announced positive preliminary data
for its vaccine, along with a partnership with a company that could
manufacture it. By late April, with investors sensing the potential for
big profits, the company's shares had reached \$3.66 --- a tenfold
increase from January.

On June 8, Vaxart changed the terms of its warrants agreement with
Armistice, making it easier for the hedge fund to rapidly acquire the 21
million shares, rather than having to buy and sell in smaller batches.

One week later, Vaxart announced that its chief executive was stepping
down, though he would remain chairman. The new C.E.O., Mr. Floroiu, had
previously worked with Mr. Boyd, Armistice's founder, at the hedge fund
and the consulting firm McKinsey.

On June 25, Vaxart announced that it had signed a letter of intent with
another company that might help it mass-produce a coronavirus vaccine.
Vaxart's shares nearly doubled that day.

The next day, Vaxart issued its news release saying it had been
\href{https://investors.vaxart.com/news-releases/news-release-details/vaxarts-covid-19-vaccine-selected-us-governments-operation-warp}{selected}
for Operation Warp Speed. Its shares instantly doubled again, at one
pointing hitting \$14, their highest level in years.

``We are very pleased to be one of the few companies selected by
Operation Warp Speed, and that ours is the only oral vaccine being
evaluated,'' Mr. Floroiu said.

Armistice took advantage of the stock's exponential increase --- at that
point up more than 3,600 percent since January. On June 26, a Friday,
and the next Monday, the hedge fund exercised its warrants to buy nearly
21 million Vaxart shares for either 30 cents or \$1.10 a share ---
purchases it would not have been able to make as quickly had its
agreement with Vaxart not been modified weeks earlier.

Armistice then immediately sold the shares at prices from \$6.58 to
\$12.89 a share, according to securities filings. The hedge fund's
profits were immense: more than \$197 million.

``It looks like the warrants may have been reconfigured at a time when
they knew good news was coming,'' said Robert Daines, a professor at
Stanford Law School who is an expert on corporate governance. ``That's a
valuable change, made right as the company's stock price was about to
rise.''

At the same time, the hedge fund also unloaded some of the Vaxart shares
it had previously bought, notching tens of millions of dollars in
additional profits.

By the end of that Monday, June 29, Armistice had sold almost all of its
Vaxart shares.

Mr. Boyd and Armistice declined to comment.

Mr. Floroiu said the change to the Armistice agreement ``was in the best
interests of Vaxart and its stockholders'' and helped it raise money to
work on the Covid-19 vaccine.

He and other Vaxart board members also were positioned for big personal
profits. When he became chief executive in mid-June, Mr. Floroiu
received stock options that were worth about \$4.3 million. A month
later, those options were worth more than \$28 million.

Normally when companies issue stock options to executives, the options
can't be exercised for months or years. Because of the unusual terms and
the run-up in Vaxart's stock price, most of Mr. Floroiu's can be cashed
in now.

Vaxart's board members also received large grants of stock options,
giving them the right to buy shares in the company at prices well below
where the stock is now trading. The higher the shares fly, the bigger
the profits.

``Vaxart is disrupting the vaccine world,'' Mr. Floroiu boasted during a
virtual investor conference on Thursday. He added that his impression
was that ``it's OK to make a profit from Covid vaccines, as long as
you're not profiteering.''

Noah Weiland contributed reporting.

Advertisement

\protect\hyperlink{after-bottom}{Continue reading the main story}

\hypertarget{site-index}{%
\subsection{Site Index}\label{site-index}}

\hypertarget{site-information-navigation}{%
\subsection{Site Information
Navigation}\label{site-information-navigation}}

\begin{itemize}
\tightlist
\item
  \href{https://help.nytimes.com/hc/en-us/articles/115014792127-Copyright-notice}{©~2020~The
  New York Times Company}
\end{itemize}

\begin{itemize}
\tightlist
\item
  \href{https://www.nytco.com/}{NYTCo}
\item
  \href{https://help.nytimes.com/hc/en-us/articles/115015385887-Contact-Us}{Contact
  Us}
\item
  \href{https://www.nytco.com/careers/}{Work with us}
\item
  \href{https://nytmediakit.com/}{Advertise}
\item
  \href{http://www.tbrandstudio.com/}{T Brand Studio}
\item
  \href{https://www.nytimes.com/privacy/cookie-policy\#how-do-i-manage-trackers}{Your
  Ad Choices}
\item
  \href{https://www.nytimes.com/privacy}{Privacy}
\item
  \href{https://help.nytimes.com/hc/en-us/articles/115014893428-Terms-of-service}{Terms
  of Service}
\item
  \href{https://help.nytimes.com/hc/en-us/articles/115014893968-Terms-of-sale}{Terms
  of Sale}
\item
  \href{https://spiderbites.nytimes.com}{Site Map}
\item
  \href{https://help.nytimes.com/hc/en-us}{Help}
\item
  \href{https://www.nytimes.com/subscription?campaignId=37WXW}{Subscriptions}
\end{itemize}
