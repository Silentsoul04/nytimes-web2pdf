Sections

SEARCH

\protect\hyperlink{site-content}{Skip to
content}\protect\hyperlink{site-index}{Skip to site index}

\href{https://www.nytimes.com/section/business/economy}{Economy}

\href{https://myaccount.nytimes.com/auth/login?response_type=cookie\&client_id=vi}{}

\href{https://www.nytimes.com/section/todayspaper}{Today's Paper}

\href{/section/business/economy}{Economy}\textbar{}Once a Source of
U.S.-China Tension, Trade Emerges as an Area of Calm

\url{https://nyti.ms/2CLZDXi}

\begin{itemize}
\item
\item
\item
\item
\item
\end{itemize}

Advertisement

\protect\hyperlink{after-top}{Continue reading the main story}

Supported by

\protect\hyperlink{after-sponsor}{Continue reading the main story}

\hypertarget{once-a-source-of-us-china-tension-trade-emerges-as-an-area-of-calm}{%
\section{Once a Source of U.S.-China Tension, Trade Emerges as an Area
of
Calm}\label{once-a-source-of-us-china-tension-trade-emerges-as-an-area-of-calm}}

The trade deal is providing a rare point of stability as relations
between the United States and China fray over Hong Kong, the coronavirus
and accusations of espionage.

\includegraphics{https://static01.nyt.com/images/2020/07/25/us/politics/25dc-chinatrade/merlin_174935004_06e5124a-b1d9-4d5d-9c08-ea5fdfe9e812-articleLarge.jpg?quality=75\&auto=webp\&disable=upscale}

\href{https://www.nytimes.com/by/ana-swanson}{\includegraphics{https://static01.nyt.com/images/2018/12/10/multimedia/author-ana-swanson/author-ana-swanson-thumbLarge.png}}\href{https://www.nytimes.com/by/keith-bradsher}{\includegraphics{https://static01.nyt.com/images/2018/10/08/multimedia/author-keith-bradsher/author-keith-bradsher-thumbLarge.png}}

By \href{https://www.nytimes.com/by/ana-swanson}{Ana Swanson} and
\href{https://www.nytimes.com/by/keith-bradsher}{Keith Bradsher}

\begin{itemize}
\item
  July 25, 2020
\item
  \begin{itemize}
  \item
  \item
  \item
  \item
  \item
  \end{itemize}
\end{itemize}

\href{https://cn.nytimes.com/business/20200727/us-china-trade-diplomacy/}{阅读简体中文版}\href{https://cn.nytimes.com/business/20200727/us-china-trade-diplomacy/zh-hant/}{閱讀繁體中文版}

WASHINGTON --- For the better part of three years, President Trump's
trade war with China strained relations between the world's largest
economies. Now, the trade pact the two countries signed in January
appears to be the most durable part of the U.S.-China relationship.

Tensions between the United States and China are flaring over the
coronavirus, which the Trump administration accuses China of failing to
control, as well as accusations of espionage, intellectual property
theft and human rights violations. American officials on Tuesday ordered
\href{https://www.nytimes.com/2020/07/22/world/asia/us-china-houston-consulate.html}{the
closure of China's consulate in Houston}, saying that diplomats there
had aided in economic espionage, prompting China to
\href{https://www.nytimes.com/2020/07/24/world/asia/china-us-consulate-chengdu.html}{order
the closure} of the American consulate in Chengdu.

Earlier in the week, the Trump administration
\href{https://www.nytimes.com/2020/07/20/business/economy/china-sanctions-uighurs-labor.html}{added
another 11 Chinese companies} to a government list barring them from
buying American technology and other products, citing human rights
abuses against predominantly Muslim ethnic minorities in the Xinjiang
region in China's far west. The two countries are also clashing over
China's security crackdown in Hong Kong, its global 5G ambitions and its
territorial claims in the South China Sea.

But unlike
\href{https://www.nytimes.com/2019/08/23/business/china-tariffs-trump.html}{previous
moments} of heightened tensions between the United States and China, Mr.
Trump has not threatened to impose additional tariffs on Chinese goods
or take other steps to punish companies that export their products to
America. And neither side is threatening to rip up the initial trade
deal they signed in January, which took years of painful negotiations to
complete.

Trade, long the most contentious part of the U.S.-China relationship,
has suddenly become an area of surprising stability.

The reasons have more to do with politics than diplomacy. Both the Trump
administration and Chinese leaders invested time and political capital
in reaching
\href{https://www.nytimes.com/2020/01/15/business/economy/china-trade-deal.html}{their
initial trade deal}, which removed barriers for foreign firms doing
business in China and strengthened the country's intellectual property
protections. The deal also required China to purchase an additional
\$200 billion of American goods by the end of next year, including some
agricultural goods like soybeans, pork and corn from farm states that
are crucial to Mr. Trump's re-election chances.

As tensions between the two countries rise again, both sides appear to
think they have more to lose from rupturing the agreement than they
would gain.

``Ironically, trade has become an area of cooperation or stability,''
said Michael Pillsbury, a China expert at the Hudson Institute who
advises the Trump administration.

In some ways, the signing of the sought-after trade deal in January has
paved the way for the Trump administration to press China on other
fronts. In pursuit of a trade deal, the Trump administration had long
shelved various actions to address other concerns about China, including
\href{https://www.nytimes.com/2019/05/04/world/asia/trump-china-uighurs-trade-deal.html}{its
human rights abuses in Xinjiang},
\href{https://www.nytimes.com/2019/11/21/us/politics/trump-hong-kong-china.html}{its
crackdown in Hong Kong} and
\href{https://www.nytimes.com/2019/11/15/business/us-reprieve-huawei.html}{security
threats} and
\href{https://www.nytimes.com/2018/06/07/business/us-china-zte-deal.html}{sanctions
violations} by Chinese technology and telecommunications companies like
Huawei and ZTE.

\includegraphics{https://static01.nyt.com/images/2020/07/25/us/politics/25dc-chinatrade2/merlin_174556137_60e70096-cfc8-4e5f-97a7-a415ff0bc9e4-articleLarge.jpg?quality=75\&auto=webp\&disable=upscale}

As painful as the trade war was for companies on both sides of the
Pacific, it provided a strange source of stability for the U.S.-China
relationship by focusing the conflict between the countries on purely
economic matters. With trade no longer the source of friction, tensions
are spilling over into diplomacy, security and technology,
\href{https://www.nytimes.com/2020/07/14/world/asia/cold-war-china-us.html}{issues
that are more contentious and trickier to solve.}

The technological divide between the United States and China, which
censors its internet with the help of its so-called Great Firewall, is
growing larger. China is considering whether to extend new internet
controls to Hong Kong, while the Trump administration is weighing
\href{https://www.nytimes.com/2020/07/15/technology/tiktok-washington-lobbyist.html}{potential
action against Chinese-owned social media services}, like TikTok and
Tencent's WeChat.

The United States has been pressuring other countries to ban equipment
from Huawei, which it views as a surveillance and national security
threat, from wireless networks around the globe, including
\href{https://www.nytimes.com/2020/07/14/business/huawei-uk-5g.html}{in
Britain}, which recently barred the Chinese telecom giant. On Thursday,
cybersecurity researchers revealed that a popular Chinese-made drone was
\href{https://www.nytimes.com/2020/07/23/us/politics/dji-drones-security-vulnerability.html}{collecting
large amounts of personal information} that could be exploited by
Beijing.

Geopolitical tensions are rising, too.
\href{https://www.nytimes.com/interactive/2020/07/18/world/asia/china-india-border-conflict.html}{Clashes
between Chinese and Indian troops} over their disputed border in the
Himalayas have resulted in fatalities. American officials have increased
their criticisms of
\href{https://www.nytimes.com/2020/07/13/world/asia/south-china-sea-pompeo.html}{China's
actions in the South China Sea}, calling Beijing's claims to the
disputed waters ``completely unlawful.'' In April, the United States
\href{https://www.nytimes.com/2020/04/21/world/asia/coronavirus-south-china-sea-warships.html}{sent
two warships} into disputed waters near Malaysia as a show of force
after a Chinese government vessel tailed a Malaysian state oil company
ship for days.

American officials say that China's behavior has become increasingly
provocative in recent months, prompting tougher action not just from the
United States but also Australia, Britain, India and other nations. Some
public figures in China have blamed increasing tensions on Democrats and
Republicans in the United States competing to appear tougher on China
ahead of the general election in November.

Jia Qingguo, a professor of Peking University's the School of
International Studies, said in an online panel hosted by the National
Press Foundation on Thursday that the United States and China were not
yet in a new Cold War, but they were heading in that direction with
``accelerating speed, thanks to the Trump administration.''

``If the current momentum continues, I think the two countries are
likely to end up in a Cold War and maybe even in a hot one,'' Mr. Jia
said.

The Chinese government is trying to keep trade matters separate from
other frictions in the bilateral relationship, though that has proved
more difficult as the two countries begin closing each other's
consulates.

``Comparatively speaking, trade I think is more stable and more quiet,''
said He Weiwen, a former Chinese commerce ministry official and now a
senior fellow at the Center for China and Globalization, a nonprofit
research group in Beijing. But he said there are reasons to be worried
going forward.

``I'm quite concerned about the trade relationship ahead, because we
need a calm, stable political environment,'' said Mr. He, who is also an
executive council member of the China Association of International
Trade.

Chinese officials and experts argue that recent difficulties in
bilateral relations between Washington and Beijing are caused by the
Trump administration and not by the Chinese government, which has tried
to address different challenges in the relationship individually, rather
than linking them together for leverage.

``China has made all efforts to smooth the relationship with the U.S.,''
said Tu Xinquan, the dean of the China Institute for World Trade
Organization Studies at the University of International Business and
Economics in Beijing.

``Though it is admitted that there are problems between the two
countries, China has never planned an all-out whole-government strategy
against the U.S.,'' he said.

While the trade truce is holding for now, that could prove fleeting if
Mr. Trump decides Beijing is not living up to its side of the deal. The
agreement left tariffs in place on more than \$360 billion of Chinese
goods and ushered in a détente that forestalled further tariff increases
by either side.

But the president views tariffs as one of his most effective and
reliable tools, a powerful cudgel to wield against foreign countries
that doesn't require the approval of Congress. And China appears to be
\href{https://www.nytimes.com/2020/06/19/business/economy/trump-china-trade-war-farmers.html}{lagging
far behind on the purchases of American products} it pledged to make as
part of the trade deal, partly as a result of the pandemic.

Analysts have long viewed those targets as unrealistic. But Mr. Trump
sees those purchases as crucial to narrowing the U.S. trade deficit and
boosting the fortunes of farmers and businesses, and thus his
re-election prospects.

``The president has repeatedly said if they don't make the purchases, I
will terminate the deal,'' Mr. Pillsbury said.

As China shakes off the coronavirus, its purchases of American products
appear to be ticking up. Data from China's General Administration of
Customs shows that the country's imports from the United States were up
15.1 percent in June from the same month last year, when calculated in
China's currency, the renminbi, compared to a 5.2 percent increase in
China's exports to the United States.

Agricultural imports from the United States have been especially strong
this summer, with two of the three largest Chinese purchases ever of
American grain
\href{https://www.bloomberg.com/news/articles/2020-07-14/china-books-record-deal-for-u-s-corn-stepping-up-buying-spree}{occurring}this
month.

On Thursday, Mr. Trump pointed to
\href{https://www.fas.usda.gov/newsroom/private-exporters-report-sales-activity-china-251}{record-setting
purchases of corn} made by China. But, he added that ``the trade deal
means less to me now than it did when I made it.''

``Can you understand that? It just means much less to me,'' the
president said.

For now,
\href{https://www.nytimes.com/2020/06/23/business/economy/trump-navarro-china-trade-deal.html}{Mr.
Trump}
\href{https://www.nytimes.com/2020/06/17/business/economy/us-trade-china-tariffs.html}{and
his trade advisers} are mostly defending China's efforts to live up to
the trade deal, saying China has been taking crucial steps to open its
agricultural markets and financial system.

``I don't think it's going away,'' Jamieson L. Greer, a former chief of
staff to the United States Trade Representative, who is now a partner at
the law firm King \& Spalding, said of the initial trade deal. ``The
Chinese really needed it, and the administration has every incentive to
keep it, too.''

Ana Swanson reported from Washington, and Keith Bradsher from Beijing.
Paul Mozur contributed reporting.

Advertisement

\protect\hyperlink{after-bottom}{Continue reading the main story}

\hypertarget{site-index}{%
\subsection{Site Index}\label{site-index}}

\hypertarget{site-information-navigation}{%
\subsection{Site Information
Navigation}\label{site-information-navigation}}

\begin{itemize}
\tightlist
\item
  \href{https://help.nytimes.com/hc/en-us/articles/115014792127-Copyright-notice}{©~2020~The
  New York Times Company}
\end{itemize}

\begin{itemize}
\tightlist
\item
  \href{https://www.nytco.com/}{NYTCo}
\item
  \href{https://help.nytimes.com/hc/en-us/articles/115015385887-Contact-Us}{Contact
  Us}
\item
  \href{https://www.nytco.com/careers/}{Work with us}
\item
  \href{https://nytmediakit.com/}{Advertise}
\item
  \href{http://www.tbrandstudio.com/}{T Brand Studio}
\item
  \href{https://www.nytimes.com/privacy/cookie-policy\#how-do-i-manage-trackers}{Your
  Ad Choices}
\item
  \href{https://www.nytimes.com/privacy}{Privacy}
\item
  \href{https://help.nytimes.com/hc/en-us/articles/115014893428-Terms-of-service}{Terms
  of Service}
\item
  \href{https://help.nytimes.com/hc/en-us/articles/115014893968-Terms-of-sale}{Terms
  of Sale}
\item
  \href{https://spiderbites.nytimes.com}{Site Map}
\item
  \href{https://help.nytimes.com/hc/en-us}{Help}
\item
  \href{https://www.nytimes.com/subscription?campaignId=37WXW}{Subscriptions}
\end{itemize}
