Sections

SEARCH

\protect\hyperlink{site-content}{Skip to
content}\protect\hyperlink{site-index}{Skip to site index}

\href{https://www.nytimes.com/section/books}{Books}

\href{https://myaccount.nytimes.com/auth/login?response_type=cookie\&client_id=vi}{}

\href{https://www.nytimes.com/section/todayspaper}{Today's Paper}

\href{/section/books}{Books}\textbar{}`Memorial Drive' Powerfully
Recalls a Southern Childhood and a Mother's Murder

\url{https://nyti.ms/3jG8v1r}

\begin{itemize}
\item
\item
\item
\item
\item
\end{itemize}

\href{https://www.nytimes.com/spotlight/at-home?action=click\&pgtype=Article\&state=default\&region=TOP_BANNER\&context=at_home_menu}{At
Home}

\begin{itemize}
\tightlist
\item
  \href{https://www.nytimes.com/2020/08/03/well/family/the-benefits-of-talking-to-strangers.html?action=click\&pgtype=Article\&state=default\&region=TOP_BANNER\&context=at_home_menu}{Talk:
  To Strangers}
\item
  \href{https://www.nytimes.com/2020/08/01/at-home/coronavirus-make-pizza-on-a-grill.html?action=click\&pgtype=Article\&state=default\&region=TOP_BANNER\&context=at_home_menu}{Make:
  Grilled Pizza}
\item
  \href{https://www.nytimes.com/2020/07/31/arts/television/goldbergs-abc-stream.html?action=click\&pgtype=Article\&state=default\&region=TOP_BANNER\&context=at_home_menu}{Watch:
  'The Goldbergs'}
\item
  \href{https://www.nytimes.com/interactive/2020/at-home/even-more-reporters-editors-diaries-lists-recommendations.html?action=click\&pgtype=Article\&state=default\&region=TOP_BANNER\&context=at_home_menu}{Explore:
  Reporters' Google Docs}
\end{itemize}

Advertisement

\protect\hyperlink{after-top}{Continue reading the main story}

Supported by

\protect\hyperlink{after-sponsor}{Continue reading the main story}

\href{/column/books-of-the-times}{Books of The Times}

\hypertarget{memorial-drive-powerfully-recalls-a-southern-childhood-and-a-mothers-murder}{%
\section{`Memorial Drive' Powerfully Recalls a Southern Childhood and a
Mother's
Murder}\label{memorial-drive-powerfully-recalls-a-southern-childhood-and-a-mothers-murder}}

By \href{https://www.nytimes.com/by/dwight-garner}{Dwight Garner}

\begin{itemize}
\item
  July 27, 2020
\item
  \begin{itemize}
  \item
  \item
  \item
  \item
  \item
  \end{itemize}
\end{itemize}

\includegraphics{https://static01.nyt.com/images/2020/07/28/books/27BOOKTRETHEWEY1/27BOOKTRETHEWEY1-articleLarge.png?quality=75\&auto=webp\&disable=upscale}

Buy Book ▾

\begin{itemize}
\tightlist
\item
  \href{https://www.amazon.com/gp/search?index=books\&tag=NYTBSREV-20\&field-keywords=Memorial+Drive+Natasha+Trethewey}{Amazon}
\item
  \href{https://du-gae-books-dot-nyt-du-prd.appspot.com/buy?title=Memorial+Drive\&author=Natasha+Trethewey}{Apple
  Books}
\item
  \href{https://www.anrdoezrs.net/click-7990613-11819508?url=https\%3A\%2F\%2Fwww.barnesandnoble.com\%2Fw\%2F\%3Fean\%3D9780062248572}{Barnes
  and Noble}
\item
  \href{https://www.anrdoezrs.net/click-7990613-35140?url=https\%3A\%2F\%2Fwww.booksamillion.com\%2Fp\%2FMemorial\%2BDrive\%2FNatasha\%2BTrethewey\%2F9780062248572}{Books-A-Million}
\item
  \href{https://bookshop.org/a/3546/9780062248572}{Bookshop}
\item
  \href{https://www.indiebound.org/book/9780062248572?aff=NYT}{Indiebound}
\end{itemize}

When you purchase an independently reviewed book through our site, we
earn an affiliate commission.

The poet Natasha Trethewey was born in Mississippi and grew up there and
in Atlanta. She became accustomed, she writes in her new memoir,
``Memorial Drive,'' to the ``hair rising on the back of my neck when I'd
hear a certain kind of Southern accent, a tensing in my spine when I'd
see the Confederate flag or the gun rack on a truck following us too
closely down the road.''

Trethewey won a Pulitzer Prize in 2007 for her collection ``Native
Guard,'' and she served two terms as poet laureate. Some of her
\href{https://www.nytimes.com/2018/11/13/books/review-monument-natasha-trethewey.html}{dexterous
poetry} touches on the autobiographical details of her life, and she is
the author of a previous memoir, ``Beyond Katrina: A Meditation on the
Mississippi Gulf Coast.''

Nothing she has written drills down into her past, and her family's, as
powerfully as ``Memorial Drive.'' It is a controlled burn of chaos and
intellection; it is a memoir that will really lay you out.

``Memorial Drive'' is about the murder of her mother, Gwendolyn, who was
40, by Gwendolyn's second husband, a troubled Vietnam veteran named
Joel. The author was 19. She was led from a dorm room to the crime
scene, where she was filmed entering by a local news crew.

\emph{{[} This book was one of our most anticipated titles of July.}
\href{https://www.nytimes.com/2020/06/24/books/new-july-books.html}{\emph{See
the full list}}\emph{. {]}}

The murder of Trethewey's mother followed months of beatings and threats
by Joel. Gwendolyn and Natasha escaped to hotels and shelters. It is
among this book's ironies that Gwendolyn had a master's in social work,
and made more money than the shelter employees. ``Maybe you can help me
get a job,'' one of the workers said to her.

This is a book with a slow, steady build. This is restraint in service
to release. Among its first scenes is that of the author's birth in
Gulfport, Miss., in 1966. Her father was a white man, a future academic
born in Nova Scotia. The author was thus, she writes, ``a child of
miscegenation, an interracial marriage still illegal in Mississippi and
in as many as 20 other states.''

Trethewey was born on the hundredth anniversary of Confederate Memorial
Day, which paid homage to the Lost Cause. As her mother made the trip to
Gulfport Memorial Hospital, the author writes, she could not help but
witness ``the barrage of rebel flags lining the streets: private
citizens, lawmakers, Klansmen (often one and the same) raising them in
Gulfport and small towns all across Mississippi.''

When Trethewey was young and out with her parents, she grew used to
hostility. This often crossed the line into intimidation. Men followed
them out of shops. There was the ``stream of headlights searching the
front windows of the house at night'' and ``sexually charged calls from
white men driving by in broad daylight.'' The Klan burned a cross in the
family's driveway.

After her divorce from the author's father, who had grown distant while
finishing his studies in New Orleans, Gwendolyn moved with Natasha to
Atlanta, hoping for a better life.

Image

The poet Natasha Trethewey, whose new memoir is ``Memorial
Drive.''Credit...Nancy Crampton

This memoir has eddies of joy and celebration. Trethewey writes
memorably about the music Gwendolyn loved. She describes a photograph of
her mother and Joel in which they ``look like performers in a 1970s soul
band, bell-bottoms and Afros, both of them posed with one hand on the
stair railing and one foot trailing behind on the step as if they are
walking in unison down the stairs.''

They're both dressed in white, she adds, ``like Al Green on the album
cover propped up against the wall.''

By its midpoint, ``Memorial Drive'' is merely a quite good memoir. The
book's second half, like the wall of a hurricane after the eye calmly
passes over, is the destructor.

The second half, unexpectedly, dumps a bag of harrowing receipts on the
table. Thanks to a police officer who had been the first on the scene,
Trethewey has access to transcripts of her mother's police statements
before her murder; transcripts of telephone calls with Joel that
Gwendolyn taped, in hopes of getting an arrest warrant; and a short
journal her mother kept.

Trethewey dispenses this material to powerful effect. Some readers will
be put in mind of Norman Mailer's epic
\href{https://archive.nytimes.com/www.nytimes.com/books/97/05/04/reviews/mailer-song.html}{``The
Executioner's Song,''} about the surreal events surrounding the
execution of the convicted killer Gary Gilmore in Utah in the 1970s.

On the telephone recordings, Gwendolyn hangs on as Joel says things
like: ``You created this monster inside of me. It's your baby, it's
yours''; ``I have embedded these things in my head that only you can
take out''; ``Gwen, you forgot I spent two years in Vietnam. I can
explode anything''; ``I'm gonna come out there and I'm gonna shoot a
round through the window, OK. All right?''

Gwendolyn did get an arrest warrant. Joel killed her after a cop left
his post before his shift was up. One of the bullets went through her
raised right hand and into her head. ``Memorial Drive'' closes like a
door sucked shut by the wind.

Among this memoir's themes is the development of the author's
sensibility, her solitude of spirit. She is honest about what she
remembers and what she does not.

``If you had told me early on how much of my life I would lose to
forgetting --- most of those years when my mother was still alive ---
maybe I'd have begun then trying to save as much as I could.'' She had
to jettison a lot, she writes, ``out of a kind of necessity.''

Even though you intuit what is coming, the moment you learn of
Gwendolyn's death is as stunning as the moment when Anna Magnani is shot
in the street in Roberto Rossellini's ``Rome, Open City.''

Rita Dove said this about memory in a poem called ``Primer for the
Nuclear Age'':

\begin{quote}
if you've\\
got a heart at all, someday\\
it will kill you.
\end{quote}

Advertisement

\protect\hyperlink{after-bottom}{Continue reading the main story}

\hypertarget{site-index}{%
\subsection{Site Index}\label{site-index}}

\hypertarget{site-information-navigation}{%
\subsection{Site Information
Navigation}\label{site-information-navigation}}

\begin{itemize}
\tightlist
\item
  \href{https://help.nytimes.com/hc/en-us/articles/115014792127-Copyright-notice}{©~2020~The
  New York Times Company}
\end{itemize}

\begin{itemize}
\tightlist
\item
  \href{https://www.nytco.com/}{NYTCo}
\item
  \href{https://help.nytimes.com/hc/en-us/articles/115015385887-Contact-Us}{Contact
  Us}
\item
  \href{https://www.nytco.com/careers/}{Work with us}
\item
  \href{https://nytmediakit.com/}{Advertise}
\item
  \href{http://www.tbrandstudio.com/}{T Brand Studio}
\item
  \href{https://www.nytimes.com/privacy/cookie-policy\#how-do-i-manage-trackers}{Your
  Ad Choices}
\item
  \href{https://www.nytimes.com/privacy}{Privacy}
\item
  \href{https://help.nytimes.com/hc/en-us/articles/115014893428-Terms-of-service}{Terms
  of Service}
\item
  \href{https://help.nytimes.com/hc/en-us/articles/115014893968-Terms-of-sale}{Terms
  of Sale}
\item
  \href{https://spiderbites.nytimes.com}{Site Map}
\item
  \href{https://help.nytimes.com/hc/en-us}{Help}
\item
  \href{https://www.nytimes.com/subscription?campaignId=37WXW}{Subscriptions}
\end{itemize}
