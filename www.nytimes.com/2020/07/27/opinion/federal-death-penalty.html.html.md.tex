Sections

SEARCH

\protect\hyperlink{site-content}{Skip to
content}\protect\hyperlink{site-index}{Skip to site index}

\href{https://myaccount.nytimes.com/auth/login?response_type=cookie\&client_id=vi}{}

\href{https://www.nytimes.com/section/todayspaper}{Today's Paper}

\href{/section/opinion}{Opinion}\textbar{}The Death Penalty Can Ensure
`Justice Is Being Done'

\href{https://nyti.ms/39uChS1}{https://nyti.ms/39uChS1}

\begin{itemize}
\item
\item
\item
\item
\item
\end{itemize}

Advertisement

\protect\hyperlink{after-top}{Continue reading the main story}

\href{/section/opinion}{Opinion}

Supported by

\protect\hyperlink{after-sponsor}{Continue reading the main story}

\hypertarget{the-death-penalty-can-ensure-justice-is-being-done}{%
\section{The Death Penalty Can Ensure `Justice Is Being
Done'}\label{the-death-penalty-can-ensure-justice-is-being-done}}

A top Justice Department official says for many Americans the death
penalty is a difficult issue on moral, religious and policy grounds. But
as a legal issue, it is straightforward.

By Jeffrey A. Rosen

Mr. Rosen is the deputy attorney general.

\begin{itemize}
\item
  July 27, 2020
\item
  \begin{itemize}
  \item
  \item
  \item
  \item
  \item
  \end{itemize}
\end{itemize}

\includegraphics{https://static01.nyt.com/images/2020/07/28/opinion/27rosenWeb/merlin_156995043_43641184-5008-4224-8476-fb25d85a3462-articleLarge.jpg?quality=75\&auto=webp\&disable=upscale}

This month, for the first time in
\href{https://www.nytimes.com/2020/07/14/us/politics/daniel-lewis-lee-execution-crime.html}{17
years}, the United States resumed carrying out death sentences for
federal crimes.

On July 14,
\href{https://www.nytimes.com/2020/07/14/us/politics/daniel-lewis-lee-execution-crime.html}{Daniel
Lewis Lee} was executed for the 1996 murder of a family, including an
8-year-old girl, by suffocating and drowning them in the Illinois Bayou
after robbing them to fund a white-supremacist organization. On July 16,
\href{https://www.nytimes.com/2020/07/16/us/politics/wesley-ira-purkey-executed.html}{Wesley
Purkey} was executed for the 1998 murder of a teenage girl, whom he
kidnapped, raped, killed, dismembered and discarded in a septic pond.
The next day,
\href{https://www.nytimes.com/2020/07/17/us/dustin-honken-federal-execution.html}{Dustin
Honken} was executed for five murders committed in 1993, including the
execution-style shooting of two young girls, their mother, and two
prospective witnesses against him in a federal prosecution for
methamphetamine trafficking.

The death penalty is a difficult issue for many Americans on moral,
religious and policy grounds. But as a legal issue, it is
straightforward. The United States Constitution expressly contemplates
``capital'' crimes, and Congress has authorized the death penalty for
serious federal offenses since President George Washington signed the
Crimes Act of 1790. The American people have repeatedly ratified that
decision, including through the
\href{https://www.justice.gov/archives/jm/criminal-resource-manual-69-federal-death-penalty-act-1994}{Federal
Death Penalty Act of 1994} signed by President Bill Clinton, the federal
execution of Timothy McVeigh under President George W. Bush and the
decision by President Barack Obama's Justice Department to seek the
death penalty against the
\href{https://www.wsj.com/articles/boston-marathon-bomber-appeals-death-penalty-11576152005}{Boston
Marathon bomber} and
\href{https://www.nbcnews.com/news/us-news/dylann-roof-appeals-death-penalty-south-carolina-church-massacre-n1125341}{Dylann
Roof}.

The recent executions reflect that consensus, as the Justice Department
has an obligation to carry out the law. The decision to seek the death
penalty against Mr. Lee was made by Attorney General Janet Reno (who
said she personally opposed the death penalty but was bound by the law)
and reaffirmed by Deputy Attorney General Eric Holder.

Mr. Purkey was prosecuted during the George W. Bush administration, and
his conviction and sentence were vigorously defended throughout the
Obama administration. The judge who imposed the death sentence on Mr.
Honken,
\href{https://www.desmoinesregister.com/story/news/crime-and-courts/2020/07/17/judge-who-oversaw-dustin-honken-trial-says-evidence-overwhelming-dealth-penalty-iowa/5457829002/}{Mark
Bennett, said} that while he generally opposed the death penalty, he
would not lose any sleep over Mr. Honken's execution.

In a New York Times Op-Ed essay
\href{https://www.nytimes.com/2020/07/17/opinion/justice-department-federal-execution.html}{published
on July 17}, two of Mr. Lee's lawyers criticized the execution of their
client, which they contend was carried out in a ``shameful rush.'' That
objection overlooks that Mr. Lee was sentenced more than 20 years ago,
and his appeals and other permissible challenges failed, up to and
including the day of his execution.

Mr. Lee's lawyers seem to endorse a system of endless delays that
prevent a death sentence from ever becoming real. But his execution date
was announced almost a year ago, and was initially set for last
December. It was delayed when his lawyers obtained six more months of
review by unsuccessfully challenging the procedures used to carry out
his lethal injection.

After an appellate court rejected their claim as ``without merit,'' the
Justice Department rescheduled Mr. Lee's execution, providing an
additional four weeks of notice. Yet on the day of the rescheduled
execution, after family members of his victims had traveled to Terre
Haute, Ind., to witness the execution, a District Court granted Mr.
Lee's request for further review. That court entered a last-minute
reprieve that the Supreme Court has said should be an ``extreme
exception.''

Given the long delay that had already occurred, the Justice Department
asked the Supreme Court to lift the order so the execution could
proceed. Mr. Lee's lawyers opposed that request, insisting that
overturning the order would result in their client's imminent execution.
After reviewing the matter, the court
\href{https://www.nytimes.com/2020/07/13/us/politics/federal-execution.html}{granted
the government's request}, rebuked the District Court for creating an
unjustified last-minute barrier, and directed that the execution could
proceed.

In the final minutes before the execution was to occur, Mr. Lee's
lawyers claimed the execution could not proceed because Mr. Lee still
had time to seek further review of an appellate court decision six weeks
earlier lifting a prior stay of execution. The Justice Department
decided to pause the execution for several hours while the appellate
court considered and promptly rejected Mr. Lee's request. That cautious
step, taken to ensure undoubted compliance with court orders, is
irreconcilable with the suggestion that the department ``rushed'' the
execution or disregarded any law. Mr. Lee's final hours awaiting his
fate were a result of his own lawyers' choice to assert a
non-meritorious objection at the last moment.

Mr. Lee's lawyers also disregarded the cost to victims' families of
continued delay. Although they note that
\href{https://www.nytimes.com/2019/10/29/us/arkansas-federal-death-penalty.html?action=click\&module=RelatedLinks\&pgtype=Article}{some
members of Mr. Lee's victims' families opposed his execution,} others
did not. Nor did the family members of Wesley Purkey's victim, Jennifer
Long, who were in Terre Haute on Wednesday afternoon. When the District
Court again imposed another last-minute stoppage, granting more time for
Mr. Purkey's lawyers to argue (among other things) that he did not
understand the reason for his execution, the Justice Department again
sought Supreme Court review.

As the hours wore on, Justice Department officials asked Ms. Long's
father if he would prefer to wait for another day. The answer was
unequivocal: He would stay as long as it took. As Ms. Long's stepmother
later said, ``We just shouldn't have had to wait this long.'' The
Supreme Court ultimately authorized the execution just before 3 a.m. In
his final statement, Mr. Purkey apologized to ``Jennifer's family'' for
the pain he had caused, contradicting the claim of his lawyers that he
did not understand the reason for his execution.

The third execution, of Dustin Honken, occurred on schedule, but still
too late for some of his victims' families. John Duncan --- the father
of the victim Lori Duncan and grandfather of her slain daughters,
Kandace (age 10) and Amber (age 6) --- had urged Mr. Honken's execution
for years. As John Duncan was dying of cancer in 2018, he asked family
members to promise they would witness the execution on his behalf. On
July 17, they did. ``Finally,''
\href{https://www.desmoinesregister.com/story/news/local/columnists/courtney-crowder/2020/07/17/dustin-honken-execution-iowa-man-executed-federal-prison-terre-haute-indiana-methamphetamine-murder/5456933002/}{they
said in a statement,} ``justice is being done.''

Mr. Lee's lawyers and other death penalty opponents are entitled to
disagree with that sentiment. But if the United States is going to allow
capital punishment, a white-supremacist triple murderer would seem the
textbook example of a justified case. And if death sentences are going
to be imposed, they cannot just be hypothetical; they eventually have to
be carried out, or the punishment will lose its deterrent and
retributive effects.

Rather than forthrightly opposing the death penalty and attempting to
change the law through democratic means, however, Mr. Lee's lawyers and
others have chosen the legal and public-relations equivalent of
guerrilla war. They sought to obstruct by any means the administration
of sentences that Congress permitted, juries supported and the Supreme
Court approved. And when those tactics failed, they accused the Justice
Department of ``a grave threat to the rule of law,'' even though it
operated entirely within the law enacted by Congress and approved by the
Supreme Court. The American people can decide for themselves which
aspects of that process should be considered ``shameful.''

Jeffrey A. Rosen is the deputy attorney general.

\emph{The Times is committed to publishing}
\href{https://www.nytimes.com/2019/01/31/opinion/letters/letters-to-editor-new-york-times-women.html}{\emph{a
diversity of letters}} \emph{to the editor. We'd like to hear what you
think about this or any of our articles. Here are some}
\href{https://help.nytimes.com/hc/en-us/articles/115014925288-How-to-submit-a-letter-to-the-editor}{\emph{tips}}\emph{.
And here's our email:}
\href{mailto:letters@nytimes.com}{\emph{letters@nytimes.com}}\emph{.}

\emph{Follow The New York Times Opinion section on}
\href{https://www.facebook.com/nytopinion}{\emph{Facebook}}\emph{,}
\href{http://twitter.com/NYTOpinion}{\emph{Twitter (@NYTopinion)}}
\emph{and}
\href{https://www.instagram.com/nytopinion/}{\emph{Instagram}}\emph{.}

Advertisement

\protect\hyperlink{after-bottom}{Continue reading the main story}

\hypertarget{site-index}{%
\subsection{Site Index}\label{site-index}}

\hypertarget{site-information-navigation}{%
\subsection{Site Information
Navigation}\label{site-information-navigation}}

\begin{itemize}
\tightlist
\item
  \href{https://help.nytimes.com/hc/en-us/articles/115014792127-Copyright-notice}{©~2020~The
  New York Times Company}
\end{itemize}

\begin{itemize}
\tightlist
\item
  \href{https://www.nytco.com/}{NYTCo}
\item
  \href{https://help.nytimes.com/hc/en-us/articles/115015385887-Contact-Us}{Contact
  Us}
\item
  \href{https://www.nytco.com/careers/}{Work with us}
\item
  \href{https://nytmediakit.com/}{Advertise}
\item
  \href{http://www.tbrandstudio.com/}{T Brand Studio}
\item
  \href{https://www.nytimes.com/privacy/cookie-policy\#how-do-i-manage-trackers}{Your
  Ad Choices}
\item
  \href{https://www.nytimes.com/privacy}{Privacy}
\item
  \href{https://help.nytimes.com/hc/en-us/articles/115014893428-Terms-of-service}{Terms
  of Service}
\item
  \href{https://help.nytimes.com/hc/en-us/articles/115014893968-Terms-of-sale}{Terms
  of Sale}
\item
  \href{https://spiderbites.nytimes.com}{Site Map}
\item
  \href{https://help.nytimes.com/hc/en-us}{Help}
\item
  \href{https://www.nytimes.com/subscription?campaignId=37WXW}{Subscriptions}
\end{itemize}
