Sections

SEARCH

\protect\hyperlink{site-content}{Skip to
content}\protect\hyperlink{site-index}{Skip to site index}

\href{https://myaccount.nytimes.com/auth/login?response_type=cookie\&client_id=vi}{}

\href{https://www.nytimes.com/section/todayspaper}{Today's Paper}

\href{/section/opinion}{Opinion}\textbar{}The Cult of Selfishness Is
Killing America

\href{https://nyti.ms/302ecPo}{https://nyti.ms/302ecPo}

\begin{itemize}
\item
\item
\item
\item
\item
\item
\end{itemize}

Advertisement

\protect\hyperlink{after-top}{Continue reading the main story}

\href{/section/opinion}{Opinion}

Supported by

\protect\hyperlink{after-sponsor}{Continue reading the main story}

\hypertarget{the-cult-of-selfishness-is-killing-america}{%
\section{The Cult of Selfishness Is Killing
America}\label{the-cult-of-selfishness-is-killing-america}}

The right has made irresponsible behavior a key principle.

\href{https://www.nytimes.com/by/paul-krugman}{\includegraphics{https://static01.nyt.com/images/2018/04/02/opinion/paul-krugman/paul-krugman-thumbLarge.png}}

By \href{https://www.nytimes.com/by/paul-krugman}{Paul Krugman}

Opinion Columnist

\begin{itemize}
\item
  July 27, 2020
\item
  \begin{itemize}
  \item
  \item
  \item
  \item
  \item
  \item
  \end{itemize}
\end{itemize}

\includegraphics{https://static01.nyt.com/images/2020/07/27/opinion/27krugmanWeb/merlin_174715356_80bcd3ca-8fb9-4b27-868b-5cc41b02c7c9-articleLarge.jpg?quality=75\&auto=webp\&disable=upscale}

\href{https://cn.nytimes.com/opinion/20200728/us-republicans-coronavirus/}{阅读简体中文版}\href{https://cn.nytimes.com/opinion/20200728/us-republicans-coronavirus/zh-hant/}{閱讀繁體中文版}

America's response to the coronavirus has been a lose-lose proposition.

The Trump administration and governors like Florida's Ron DeSantis
insisted that there was no trade-off between economic growth and
controlling the disease, and they were right --- but not in the way they
expected.

Premature reopening led to a surge in infections: Adjusted for
population, Americans are currently dying from Covid-19 at around
\href{https://ourworldindata.org/coronavirus-data-explorer?zoomToSelection=true\&deathsMetric=true\&dailyFreq=true\&perCapita=true\&smoothing=7\&country=USA~CAN~EuropeanUnion\&pickerMetric=location\&pickerSort=asc}{15
times} the rate in the European Union or Canada. Yet the
``\href{https://www.nytimes.com/2020/07/01/business/economic-recovery-virus-surge.html}{rocket
ship}'' recovery Donald Trump promised has crashed and burned: Job
growth appears to have
\href{https://twitter.com/ernietedeschi/status/1286740199796596743}{stalled
or reversed}, especially in states that were
\href{https://www.washingtonpost.com/business/2020/07/21/arizona-struggles-neighboring-new-mexico-found-more-cautious-path-sustained-growth/}{most
aggressive} about lifting social distancing mandates, and early
indications are that the U.S. economy is
\href{https://www.bloomberg.com/news/articles/2020-07-26/europe-s-economy-set-to-outpace-u-s-in-upending-of-past-roles?srnd=premium\&sref=qzusa8bC}{lagging
behind} the economies of major European nations.

So we're failing dismally on both the epidemiological and the economic
fronts. But why?

On the face of it, the answer is that Trump and allies were so eager to
see big jobs numbers that they ignored both infection risks and the way
a resurgent pandemic would undermine the economy. As I and others have
said, they failed the
\href{https://www.nytimes.com/2020/06/09/opinion/coronavirus-reopening-marshmallow-test.html}{marshmallow
test}, sacrificing the future because they weren't willing to show a
little patience.

And there's surely a lot to that explanation. But it isn't the whole
story.

For one thing, people truly focused on restarting the economy should
have been big supporters of measures to limit infections without hurting
business --- above all, getting Americans to wear face masks. Instead,
Trump ridiculed those in masks as
``\href{https://www.cbsnews.com/video/trump-mocks-those-wearing-face-masks-calling-it-politically-correct/}{politically
correct},'' while Republican governors not only refused to mandate
mask-wearing, but they prevented mayors from imposing
\href{https://www.cbsnews.com/news/georgia-governor-brian-kemp-bans-city-face-mask-orders-coronavirus-pandemic/}{local
mask rules}.

Also, politicians eager to see the economy bounce back should have
wanted to sustain consumer purchasing power until wages recovered.
Instead, Senate Republicans ignored the looming July 31 expiration of
special unemployment benefits, which means that tens of millions of
workers are about to see a huge hit to their incomes, damaging the
economy as a whole.

So what was going on? Were our leaders just stupid? Well, maybe. But
there's a deeper explanation of the profoundly self-destructive behavior
of Trump and his allies: They were all members of America's cult of
selfishness.

You see, the modern U.S. right is committed to the proposition that
greed is good, that we're all better off when individuals engage in the
untrammeled pursuit of self-interest. In their vision, unrestricted
profit maximization by businesses and unregulated consumer choice is the
recipe for a good society.

Support for this proposition is, if anything, more emotional than
intellectual. I've long been struck by the intensity of right-wing anger
against relatively trivial regulations, like bans on
\href{https://krugman.blogs.nytimes.com/2014/08/05/phosphate-memories/}{phosphates}
in detergent and efficiency standards for
\href{https://www.desmogblog.com/light-bulb-madness-new-case-study-right-wing-misinformation}{light
bulbs}. It's the principle of the thing: Many on the right are enraged
at any suggestion that their actions should take other people's welfare
into account.

This rage is sometimes portrayed as love of freedom. But people who
insist on the right to pollute are notably unbothered by, say, federal
agents tear-gassing peaceful protesters. What they call ``freedom'' is
actually absence of responsibility.

Rational policy in a pandemic, however, is all about taking
responsibility. The main reason you shouldn't go to a bar and should
wear a mask isn't self-protection, although that's part of it; the point
is that congregating in noisy, crowded spaces or exhaling droplets into
shared air puts \emph{others} at risk. And that's the kind of thing
America's right just hates, hates to hear.

Indeed, it sometimes seems as if right-wingers actually make a point of
behaving irresponsibly. Remember how Senator Rand Paul, who was worried
that he might have Covid-19 (he did), wandered around the Senate and
even
\href{https://www.theatlantic.com/politics/archive/2020/03/rand-paul-coronavirus-test-reckless/608593/}{used
the gym} while waiting for his test results?

Anger at any suggestion of social responsibility also helps explain the
looming fiscal catastrophe. It's striking how emotional many Republicans
get in their opposition to the temporary rise in unemployment benefits;
for example, Senator Lindsey Graham declared that these benefits would
be extended
``\href{https://www.businessinsider.com/lindsey-graham-congress-coronavirus-unemployment-benefit-over-our-dead-bodies-2020-4}{over
our dead bodies}.'' Why such hatred?

It's not because the benefits are making workers unwilling to take jobs.
There's no evidence that
\href{https://twitter.com/ernietedeschi/status/1285687440058064903}{this
is happening} --- it's just something Republicans want to believe. And
in any case, economic arguments can't explain the rage.

Again, it's the principle. Aiding the unemployed, even if their
joblessness isn't their own fault, is a tacit admission that lucky
Americans should help their less-fortunate fellow citizens. And that's
an admission the right doesn't want to make.

Just to be clear, I'm not saying that Republicans are selfish. We'd be
doing much better if that were all there were to it. The point, instead,
is that they've sacralized selfishness, hurting their own political
prospects by insisting on the right to act selfishly even when it hurts
others.

What the coronavirus has revealed is the power of America's cult of
selfishness. And this cult is killing us.

\emph{The Times is committed to publishing}
\href{https://www.nytimes.com/2019/01/31/opinion/letters/letters-to-editor-new-york-times-women.html}{\emph{a
diversity of letters}} \emph{to the editor. We'd like to hear what you
think about this or any of our articles. Here are some}
\href{https://help.nytimes.com/hc/en-us/articles/115014925288-How-to-submit-a-letter-to-the-editor}{\emph{tips}}\emph{.
And here's our email:}
\href{mailto:letters@nytimes.com}{\emph{letters@nytimes.com}}\emph{.}

\emph{Follow The New York Times Opinion section on}
\href{https://www.facebook.com/nytopinion}{\emph{Facebook}}\emph{,}
\href{http://twitter.com/NYTOpinion}{\emph{Twitter (@NYTopinion)}}
\emph{and}
\href{https://www.instagram.com/nytopinion/}{\emph{Instagram}}\emph{.}

Advertisement

\protect\hyperlink{after-bottom}{Continue reading the main story}

\hypertarget{site-index}{%
\subsection{Site Index}\label{site-index}}

\hypertarget{site-information-navigation}{%
\subsection{Site Information
Navigation}\label{site-information-navigation}}

\begin{itemize}
\tightlist
\item
  \href{https://help.nytimes.com/hc/en-us/articles/115014792127-Copyright-notice}{©~2020~The
  New York Times Company}
\end{itemize}

\begin{itemize}
\tightlist
\item
  \href{https://www.nytco.com/}{NYTCo}
\item
  \href{https://help.nytimes.com/hc/en-us/articles/115015385887-Contact-Us}{Contact
  Us}
\item
  \href{https://www.nytco.com/careers/}{Work with us}
\item
  \href{https://nytmediakit.com/}{Advertise}
\item
  \href{http://www.tbrandstudio.com/}{T Brand Studio}
\item
  \href{https://www.nytimes.com/privacy/cookie-policy\#how-do-i-manage-trackers}{Your
  Ad Choices}
\item
  \href{https://www.nytimes.com/privacy}{Privacy}
\item
  \href{https://help.nytimes.com/hc/en-us/articles/115014893428-Terms-of-service}{Terms
  of Service}
\item
  \href{https://help.nytimes.com/hc/en-us/articles/115014893968-Terms-of-sale}{Terms
  of Sale}
\item
  \href{https://spiderbites.nytimes.com}{Site Map}
\item
  \href{https://help.nytimes.com/hc/en-us}{Help}
\item
  \href{https://www.nytimes.com/subscription?campaignId=37WXW}{Subscriptions}
\end{itemize}
