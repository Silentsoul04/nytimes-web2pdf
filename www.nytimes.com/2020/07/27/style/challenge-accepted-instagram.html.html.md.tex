Sections

SEARCH

\protect\hyperlink{site-content}{Skip to
content}\protect\hyperlink{site-index}{Skip to site index}

\href{https://www.nytimes.com/section/style}{Style}

\href{https://myaccount.nytimes.com/auth/login?response_type=cookie\&client_id=vi}{}

\href{https://www.nytimes.com/section/todayspaper}{Today's Paper}

\href{/section/style}{Style}\textbar{}`Challenge Accepted': Why Women
Are Posting Black-and-White Selfies

\url{https://nyti.ms/305xof0}

\begin{itemize}
\item
\item
\item
\item
\item
\item
\end{itemize}

Advertisement

\protect\hyperlink{after-top}{Continue reading the main story}

Supported by

\protect\hyperlink{after-sponsor}{Continue reading the main story}

\hypertarget{challenge-accepted-why-women-are-posting-black-and-white-selfies}{%
\section{`Challenge Accepted': Why Women Are Posting Black-and-White
Selfies}\label{challenge-accepted-why-women-are-posting-black-and-white-selfies}}

A campaign that purports to be about ``women supporting women'' is
filling people's Instagram feeds with striking, if opaque, imagery.

\includegraphics{https://static01.nyt.com/images/2020/07/27/fashion/27CHALLENGEACCEPTED-2/27CHALLENGEACCEPTED-2-articleLarge.jpg?quality=75\&auto=webp\&disable=upscale}

\href{https://www.nytimes.com/by/taylor-lorenz}{\includegraphics{https://static01.nyt.com/images/2020/03/18/reader-center/author-taylor-lorenz/author-taylor-lorenz-thumbLarge.png}}

By \href{https://www.nytimes.com/by/taylor-lorenz}{Taylor Lorenz}

\begin{itemize}
\item
  Published July 27, 2020Updated July 29, 2020
\item
  \begin{itemize}
  \item
  \item
  \item
  \item
  \item
  \item
  \end{itemize}
\end{itemize}

\href{https://www.nytimes.com/es/2020/07/28/espanol/estilos-de-vida/reto-selfi-blanco-negro.html}{Leer
en español}

Over the past several days, many Instagram feeds have been overrun with
black-and-white images of women both famous and not.

These photographs are often posed and filtered, taken from flattering
angles and accompanied by benign captions about ``supporting women.''

``Love this simple way to lift each other up. \#challengeaccepted. Thank
you for nominating me @vanessabryant,'' the model Cindy Crawford posted
on Monday along with a black-and-white
\href{https://www.instagram.com/p/CDJnHlQFuQi/?igshid=1vzk7hu5crx15}{photo
of herself strolling on a beach} that looks ripped from a Calvin Klein
advertisement.

The premise of the ``challenge accepted'' trend is that these photos
promote female empowerment, and that nominating friends to take part in
the campaign is a way for
\href{https://twitter.com/SoniAggarwal/status/1287784564262223872}{women
to support each other}.

So far, more than 3 million photos have been uploaded with the
\#ChallengeAccepted hashtag; many more have appeared without it.

``The trend is still picking up with usage of the hashtag on Instagram
doubling in the last day alone,'' an Instagram spokeswoman said on
Monday. ``Based on the posts, we're seeing that most of the participants
are posting with notes relating to strength and support for their
communities.''

Many women have included the hashtag \#womensupportingwomen in their
posts. ``Challenge Accepted,'' Khloe Kardashian wrote in an
\href{https://www.instagram.com/p/CDH3xn1BB3W/}{Instagram post on
Sunday}. ``To all my Queens- Let's spread love and remember to be a
little kinder to one another. \#womensupportingwomen.''

This is not the first time Instagram users have leveraged
black-and-white selfies in support of a vague cause. Back in 2016,
black-and-white photos with the hashtag \#ChallengeAccepted were meant
to spread a message of
``\href{https://metro.co.uk/2016/08/29/what-is-the-black-and-white-photo-challenge-6097096/}{cancer
awareness}.'' Over the years the photo trend has also been used to
``spread positivity.''

The challenge has circulated like chain mail. Participants nominate at
least one other woman (and often several) to post her own
black-and-white portrait. Celebrities including the actresses Kerry
Washington, Jennifer Garner, Kristen Bell and Eva Longoria have helped
the campaign gain visibility.

Cristine Abram, a public relations and influencer marketing manager for
\href{https://later.com/}{Later}, a social media marketing firm, said
that
\href{https://www.nytimes.com/2020/07/23/us/alexandria-ocasio-cortez-sexism-congress.html}{a
video} of Representative Alexandria Ocasio-Cortez speaking out against
Representative Ted Yoho's
\href{https://www.nytimes.com/2020/07/25/sunday-review/aoc-daughters-ted-yoho.html}{sexist
remarks against her} on the floor of Congress last week led to a spike
in social media posts about feminism and female empowerment, which may
have contributed to the latest round of black-and-white photos.

``That was the spark that led to the resurgence of the hashtag
challenge,'' said Ms. Abram. ``It's all to do with female empowerment.
There was this hashtag that already existed to raise awareness around
other large issues. Tapping into that allowed participants to gain
traction a lot faster because the algorithm was already familiar with
the hashtag.''

A representative from Instagram said that the earliest post the company
could surface for this current cycle of the challenge was posted
\href{https://www.instagram.com/p/CCxGfzTBmXP/}{a week and a half ago}
by the Brazilian journalist Ana Paula Padrão. Others have noted that
women in Turkey began sharing black-and-white photos recently to
\href{https://twitter.com/imaann_patel/status/1288080743198068736?s=21}{raise
awareness about femicide}.

Though the portraits have spread widely, the posts themselves say very
little. Like
\href{https://www.nytimes.com/2020/06/02/arts/music/what-blackout-tuesday.html}{the
black square}, which became a symbol of solidarity with Black people but
asked very little of those who shared it, the black-and-white selfie
allows users to feel as if they're taking a stand while saying almost
nothing. Influencers and celebrities love these types of ``challenges''
because they don't require actual advocacy, which might alienate certain
factions of their fan base.

``Ladies,'' Alana Levinson, a writer,
\href{https://twitter.com/alanalevinson/status/1287818379818971136}{tweeted
on Monday}, ``instead of posting that hot black-and-white selfie, why
don't we ease into feminism with something low stakes, like cutting off
your friend who's an abuser?''

Other women have spoken out about the backlash they have faced for
critiquing the trend. ``Currently getting hate mail on instagram from
complete strangers because i said black and white selfies aren't a
cause,''
\href{https://twitter.com/OnlineAlison/status/1287804859773677568}{tweeted}
the \href{https://www.instagram.com/p/CDJw9mpgqc1/}{podcast host Ali
Segel}. ``Apparently i hate women and don't love myself!!!!!! I'm
minding my own business for the rest of my life!!!!!!''

``I think that if this `movement' featured trans women or differently
abled women, or showcased female businesses or accomplishments or women
in history, it would make more sense,'' Ms. Segel explained further, in
a direct message on Twitter. ``But the idea of this as a challenge or
cause is really lost on me.''

Brooke Hammerling, 46, the founder of the New New Thing, an advisory to
technology C.E.O.s, questioned the trend's efficacy
\href{https://medium.com/popculturemondays/pop-culture-mondays-7-20-20-3d2276f3c16}{in
her weekly pop culture newsletter} on Monday.

``I just don't know what it stands for,'' she said by phone. ``Virtually
everyone in my life has done the challenge, a lot of my friends and a
lot of people I love. I'm 100 percent for women supporting women and I'm
grateful to the women who nominated me, but I don't understand how a
black-and-white vanity selfie does that. If we could do portraits of the
women who inspired us, that would be a little bit more in line with what
this is trying to accomplish.''

Other women online
\href{https://twitter.com/itsbooyeah/status/1287807365614534661}{suggested}
that, instead of a black-and-white selfie, women should share photos of
books, articles, products and charities that benefit women. A few people
wondered whether the trend was
\href{https://twitter.com/wolfiecomedy/status/1287813475096387584}{started
by men}.

Camilla Blackett, a TV writer, suggested that the campaign was little
more than a vehicle for attractive photos. ``What is the point of this
\href{https://twitter.com/hashtag/ChallengeAccepted?src=hashtag_click}{\#ChallengeAccepted}
thing?''
she\href{https://twitter.com/camillard/status/1287768130140246022}{tweeted
on Monday}. ``Do people not know you can just post a hot selfie for no
reason?''

Advertisement

\protect\hyperlink{after-bottom}{Continue reading the main story}

\hypertarget{site-index}{%
\subsection{Site Index}\label{site-index}}

\hypertarget{site-information-navigation}{%
\subsection{Site Information
Navigation}\label{site-information-navigation}}

\begin{itemize}
\tightlist
\item
  \href{https://help.nytimes.com/hc/en-us/articles/115014792127-Copyright-notice}{©~2020~The
  New York Times Company}
\end{itemize}

\begin{itemize}
\tightlist
\item
  \href{https://www.nytco.com/}{NYTCo}
\item
  \href{https://help.nytimes.com/hc/en-us/articles/115015385887-Contact-Us}{Contact
  Us}
\item
  \href{https://www.nytco.com/careers/}{Work with us}
\item
  \href{https://nytmediakit.com/}{Advertise}
\item
  \href{http://www.tbrandstudio.com/}{T Brand Studio}
\item
  \href{https://www.nytimes.com/privacy/cookie-policy\#how-do-i-manage-trackers}{Your
  Ad Choices}
\item
  \href{https://www.nytimes.com/privacy}{Privacy}
\item
  \href{https://help.nytimes.com/hc/en-us/articles/115014893428-Terms-of-service}{Terms
  of Service}
\item
  \href{https://help.nytimes.com/hc/en-us/articles/115014893968-Terms-of-sale}{Terms
  of Sale}
\item
  \href{https://spiderbites.nytimes.com}{Site Map}
\item
  \href{https://help.nytimes.com/hc/en-us}{Help}
\item
  \href{https://www.nytimes.com/subscription?campaignId=37WXW}{Subscriptions}
\end{itemize}
