Sections

SEARCH

\protect\hyperlink{site-content}{Skip to
content}\protect\hyperlink{site-index}{Skip to site index}

\href{https://www.nytimes.com/section/us}{U.S.}

\href{https://myaccount.nytimes.com/auth/login?response_type=cookie\&client_id=vi}{}

\href{https://www.nytimes.com/section/todayspaper}{Today's Paper}

\href{/section/us}{U.S.}\textbar{}A Small Georgia City Plans to Put
Students in Classrooms This Week

\url{https://nyti.ms/3hFoYB2}

\begin{itemize}
\item
\item
\item
\item
\item
\item
\end{itemize}

\href{https://www.nytimes.com/news-event/coronavirus?action=click\&pgtype=Article\&state=default\&region=TOP_BANNER\&context=storylines_menu}{The
Coronavirus Outbreak}

\begin{itemize}
\tightlist
\item
  live\href{https://www.nytimes.com/2020/08/01/world/coronavirus-covid-19.html?action=click\&pgtype=Article\&state=default\&region=TOP_BANNER\&context=storylines_menu}{Latest
  Updates}
\item
  \href{https://www.nytimes.com/interactive/2020/us/coronavirus-us-cases.html?action=click\&pgtype=Article\&state=default\&region=TOP_BANNER\&context=storylines_menu}{Maps
  and Cases}
\item
  \href{https://www.nytimes.com/interactive/2020/science/coronavirus-vaccine-tracker.html?action=click\&pgtype=Article\&state=default\&region=TOP_BANNER\&context=storylines_menu}{Vaccine
  Tracker}
\item
  \href{https://www.nytimes.com/interactive/2020/07/29/us/schools-reopening-coronavirus.html?action=click\&pgtype=Article\&state=default\&region=TOP_BANNER\&context=storylines_menu}{What
  School May Look Like}
\item
  \href{https://www.nytimes.com/live/2020/07/31/business/stock-market-today-coronavirus?action=click\&pgtype=Article\&state=default\&region=TOP_BANNER\&context=storylines_menu}{Economy}
\end{itemize}

Advertisement

\protect\hyperlink{after-top}{Continue reading the main story}

Supported by

\protect\hyperlink{after-sponsor}{Continue reading the main story}

\hypertarget{a-small-georgia-city-plans-to-put-students-in-classrooms-this-week}{%
\section{A Small Georgia City Plans to Put Students in Classrooms This
Week}\label{a-small-georgia-city-plans-to-put-students-in-classrooms-this-week}}

In-person classes without a mask requirement are scheduled to begin on
Friday in Jefferson, worrying some parents, students and teachers as the
state confronts the coronavirus.

\includegraphics{https://static01.nyt.com/images/2020/07/26/us/26georgia-school-slide-MLDE/26georgia-school-slide-MLDE-articleLarge.jpg?quality=75\&auto=webp\&disable=upscale}

By \href{https://www.nytimes.com/by/richard-fausset}{Richard Fausset}

\begin{itemize}
\item
  Published July 27, 2020Updated July 29, 2020
\item
  \begin{itemize}
  \item
  \item
  \item
  \item
  \item
  \item
  \end{itemize}
\end{itemize}

JEFFERSON, Ga. --- When Jennifer Fogle and her family moved from Indiana
to Georgia 13 years ago, they settled in Jefferson, a small, handsome
city an hour's drive from Atlanta, because they had heard about the
excellent schools. And until recently, they had little to complain
about. The teachers are passionate and committed, and the facilities
rival those found at some private schools.

But in recent days Ms. Fogle found herself uncharacteristically anxious,
after learning that Jefferson City Schools planned to offer face-to-face
instruction in the midst of a resurgent coronavirus pandemic that has
seen thousands of new cases reported daily in Georgia.

As other districts around the state delayed their back-to-school days or
moved to all-remote learning, Jefferson school officials announced that
they were sticking with their Friday start date, one of the earliest in
the nation. And while school officials said they would ``strongly
encourage'' masks for students and teachers, they stopped short of
making masks mandatory.

Ms. Fogle, 46, a stay-at-home mother, thinks these decisions are unwise.
But after weighing her options, including online education promoted by
the district but taught by a private company or the state, she decided
it best to let her two teenage children embrace the risks and physically
attend Jefferson High School. It seemed futile, she said, to go against
the grain in a heavily pro-Trump community where many see masks as an
infringement of their personal freedom --- and in a state where the
Republican governor, Brian Kemp, has been urging districts to reopen
their classrooms despite the pandemic's growing toll.

``I can't fix it,'' Ms. Fogle said. ``So I have to learn, how do we live
life as normal as possible and still try to protect ourselves?''

The reopening plans have starkly divided Jefferson, a middle-class city
of about 12,000 people, offering a likely preview of the contentious
debates ahead for many other communities whose school years start closer
to the end of summer.

An
\href{https://www.change.org/p/jefferson-board-of-education-mandate-masks-at-jefferson-city-schools}{online
petition} created by two Jefferson High seniors calling for a
mandatory-mask rule has garnered more than 600 signatures. But a
\href{https://www.change.org/p/jefferson-board-of-education-banned-use-of-mask-in-jefferson-city-schools?utm_content=cl_sharecopy_23575435_en-US\%3A3\&recruiter=1133092121\&recruited_by_id=947a76d0-ca11-11ea-be25-252352018b69\&utm_source=share_petition\&utm_medium=copylink\&utm_campaign=psf_combo_share_initial\&utm_term=tap_basic_share}{competing
petition} demanding that masks remain a choice for students has
attracted more than 200 signers, some of whom have left comments that
underscore the politicized nature of the disagreement. ``Only liberals
can get rona and I'm not a liberal,'' wrote one, using a slang term for
the coronavirus. ``TRUMP2020 no mask fo me.''

\includegraphics{https://static01.nyt.com/images/2020/07/26/us/26georgia-school-slide-LGNT/26georgia-school-slide-LGNT-articleLarge.jpg?quality=75\&auto=webp\&disable=upscale}

The virus has been surging in the United States since mid-June. And the
possibility of more online-only schooling in the fall --- after a spring
in which many people were forced to learn from home --- is raising
concerns about the quality of students' education, the possible harm it
may cause them psychologically and socially, and the child-care problems
that working parents will face.

\hypertarget{latest-updates-global-coronavirus-outbreak}{%
\section{\texorpdfstring{\href{https://www.nytimes.com/2020/08/01/world/coronavirus-covid-19.html?action=click\&pgtype=Article\&state=default\&region=MAIN_CONTENT_1\&context=storylines_live_updates}{Latest
Updates: Global Coronavirus
Outbreak}}{Latest Updates: Global Coronavirus Outbreak}}\label{latest-updates-global-coronavirus-outbreak}}

Updated 2020-08-02T04:42:44.717Z

\begin{itemize}
\tightlist
\item
  \href{https://www.nytimes.com/2020/08/01/world/coronavirus-covid-19.html?action=click\&pgtype=Article\&state=default\&region=MAIN_CONTENT_1\&context=storylines_live_updates\#link-34047410}{The
  U.S. reels as July cases more than double the total of any other
  month.}
\item
  \href{https://www.nytimes.com/2020/08/01/world/coronavirus-covid-19.html?action=click\&pgtype=Article\&state=default\&region=MAIN_CONTENT_1\&context=storylines_live_updates\#link-780ec966}{Top
  U.S. officials work to break an impasse over the federal jobless
  benefit.}
\item
  \href{https://www.nytimes.com/2020/08/01/world/coronavirus-covid-19.html?action=click\&pgtype=Article\&state=default\&region=MAIN_CONTENT_1\&context=storylines_live_updates\#link-25930521}{Thousands
  in Berlin protest Germany's coronavirus measures.}
\end{itemize}

\href{https://www.nytimes.com/2020/08/01/world/coronavirus-covid-19.html?action=click\&pgtype=Article\&state=default\&region=MAIN_CONTENT_1\&context=storylines_live_updates}{See
more updates}

More live coverage:
\href{https://www.nytimes.com/live/2020/07/31/business/stock-market-today-coronavirus?action=click\&pgtype=Article\&state=default\&region=MAIN_CONTENT_1\&context=storylines_live_updates}{Markets}

President Trump said on Thursday that schools in areas hit hard by the
coronavirus should delay reopening. But he also said schools should not
be able to partake in a proposed multibillion-dollar aid package unless
they opened for in-person learning. That same day, the Centers for
Disease Control and Prevention
\href{https://www.nytimes.com/2020/07/24/health/cdc-schools-coronavirus.html}{issued
a full-throated call to reopen schools}.

\href{https://www.nytimes.com/2020/07/11/health/coronavirus-schools-reopen.html}{Researchers
have painted an incomplete picture} about the wisdom of physically
opening schools. Children are less likely than older people to get
seriously ill from Covid-19, and some research suggests that younger
children may be less likely than teenagers to infect people.

Image

A shuttered gas station in downtown Jefferson, northeast of Atlanta,
displays local school pride.Credit...Melissa Golden for The New York
Times

Jefferson sits northeast of Atlanta and is the seat of semirural Jackson
County, which has had 13 coronavirus-related deaths, and an infection
rate of 1,067 per 100,000 people. But in nearby Gwinnett County, which
has about 12 times as many people, the infection rate is considerably
higher and 216 people have died. More broadly, Georgia, in the week
ending July 23, has seen an average of 3,287 new cases per day --- an
increase of 42 percent from the average two weeks earlier. Many
Jefferson residents traditionally commute for work to Atlanta and
beyond.

The president won almost 80 percent of the vote in Jackson County in
2016, and he has
\href{https://www.nytimes.com/2020/03/27/us/politics/trump-coronavirus-factcheck.html}{repeatedly
downplayed the seriousness of the virus}. Similar sentiments have been a
staple on a Facebook forum for Jefferson residents that has been flaring
with passionate disagreements about the pandemic and the school system's
response to it.

``My kids have been to baseball, wrestling and cheerleading practices,''
one commenter wrote recently. ``We have been out to eat and shopping.
Yes I will be taking precautions but locking my kids up and making sure
they are 6ft from their friends is ridiculous. What about their mental
health. It's not normal for children to have no interactions.''

Jefferson's public school system, which dates to 1818,
\href{https://www.wsbtv.com/news/local/local-school-district-s-student-population-explodes-in-recent-years/799563106/}{has
been growing quickly}. Today four schools service the 3,800-student
system. The district, which is 78 percent white, boasts of high test
scores and other accolades on its website, and makes some spots
available to out-of-district students for \$900 to \$1,000 per year.

Donna McMullan, the district superintendent, acknowledged the
nervousness. But she said the reopening plan was carefully devised to
comply with state and federal guidelines, and was developed after
consulting parents.

Image

Dana Phillips, who has three children in the school system, would prefer
a mask mandate.Credit...Melissa Golden for The New York Times

``Obviously there are different viewpoints about wearing the masks,''
Dr. McMullan said. The reason they were not being mandated had nothing
to do with politics, she said, but because students with disabilities or
other medical conditions may not be able to wear them.

She also said the plan could change. Last week, reports surfaced that
the state board of education was
\href{https://www.ajc.com/education/get-schooled-blog/state-may-impose-statewide-start-date-of-sept-8/O5WRRZX3EFCL5KKFPVQ3KYRURU/\#:~:text=Activate\%20My\%20Account-,Georgia\%20may\%20push\%20statewide\%20start\%20date\%20of\%20Sept.\%208\%20for,start\%20of\%20school\%20until\%20September.\&text=At\%20its\%20meeting\%20Thursday\%2C\%20the,school\%20statewide\%20until\%20Sept.\%208.}{considering
pushing all school openings to Sept. 8}, though nothing came of the
idea.

A copy of notes from a July 15 high school department chair meeting
describes how masks will be required on school buses, hallways will be
marked to encourage walking on the right side, and teachers will be
required to send children with coronavirus symptoms to a school nurse in
an ``isolation room.''

\href{https://www.nytimes.com/news-event/coronavirus?action=click\&pgtype=Article\&state=default\&region=MAIN_CONTENT_3\&context=storylines_faq}{}

\hypertarget{the-coronavirus-outbreak-}{%
\subsubsection{The Coronavirus Outbreak
›}\label{the-coronavirus-outbreak-}}

\hypertarget{frequently-asked-questions}{%
\paragraph{Frequently Asked
Questions}\label{frequently-asked-questions}}

Updated July 27, 2020

\begin{itemize}
\item ~
  \hypertarget{should-i-refinance-my-mortgage}{%
  \paragraph{Should I refinance my
  mortgage?}\label{should-i-refinance-my-mortgage}}

  \begin{itemize}
  \tightlist
  \item
    \href{https://www.nytimes.com/article/coronavirus-money-unemployment.html?action=click\&pgtype=Article\&state=default\&region=MAIN_CONTENT_3\&context=storylines_faq}{It
    could be a good idea,} because mortgage rates have
    \href{https://www.nytimes.com/2020/07/16/business/mortgage-rates-below-3-percent.html?action=click\&pgtype=Article\&state=default\&region=MAIN_CONTENT_3\&context=storylines_faq}{never
    been lower.} Refinancing requests have pushed mortgage applications
    to some of the highest levels since 2008, so be prepared to get in
    line. But defaults are also up, so if you're thinking about buying a
    home, be aware that some lenders have tightened their standards.
  \end{itemize}
\item ~
  \hypertarget{what-is-school-going-to-look-like-in-september}{%
  \paragraph{What is school going to look like in
  September?}\label{what-is-school-going-to-look-like-in-september}}

  \begin{itemize}
  \tightlist
  \item
    It is unlikely that many schools will return to a normal schedule
    this fall, requiring the grind of
    \href{https://www.nytimes.com/2020/06/05/us/coronavirus-education-lost-learning.html?action=click\&pgtype=Article\&state=default\&region=MAIN_CONTENT_3\&context=storylines_faq}{online
    learning},
    \href{https://www.nytimes.com/2020/05/29/us/coronavirus-child-care-centers.html?action=click\&pgtype=Article\&state=default\&region=MAIN_CONTENT_3\&context=storylines_faq}{makeshift
    child care} and
    \href{https://www.nytimes.com/2020/06/03/business/economy/coronavirus-working-women.html?action=click\&pgtype=Article\&state=default\&region=MAIN_CONTENT_3\&context=storylines_faq}{stunted
    workdays} to continue. California's two largest public school
    districts --- Los Angeles and San Diego --- said on July 13, that
    \href{https://www.nytimes.com/2020/07/13/us/lausd-san-diego-school-reopening.html?action=click\&pgtype=Article\&state=default\&region=MAIN_CONTENT_3\&context=storylines_faq}{instruction
    will be remote-only in the fall}, citing concerns that surging
    coronavirus infections in their areas pose too dire a risk for
    students and teachers. Together, the two districts enroll some
    825,000 students. They are the largest in the country so far to
    abandon plans for even a partial physical return to classrooms when
    they reopen in August. For other districts, the solution won't be an
    all-or-nothing approach.
    \href{https://bioethics.jhu.edu/research-and-outreach/projects/eschool-initiative/school-policy-tracker/}{Many
    systems}, including the nation's largest, New York City, are
    devising
    \href{https://www.nytimes.com/2020/06/26/us/coronavirus-schools-reopen-fall.html?action=click\&pgtype=Article\&state=default\&region=MAIN_CONTENT_3\&context=storylines_faq}{hybrid
    plans} that involve spending some days in classrooms and other days
    online. There's no national policy on this yet, so check with your
    municipal school system regularly to see what is happening in your
    community.
  \end{itemize}
\item ~
  \hypertarget{is-the-coronavirus-airborne}{%
  \paragraph{Is the coronavirus
  airborne?}\label{is-the-coronavirus-airborne}}

  \begin{itemize}
  \tightlist
  \item
    The coronavirus
    \href{https://www.nytimes.com/2020/07/04/health/239-experts-with-one-big-claim-the-coronavirus-is-airborne.html?action=click\&pgtype=Article\&state=default\&region=MAIN_CONTENT_3\&context=storylines_faq}{can
    stay aloft for hours in tiny droplets in stagnant air}, infecting
    people as they inhale, mounting scientific evidence suggests. This
    risk is highest in crowded indoor spaces with poor ventilation, and
    may help explain super-spreading events reported in meatpacking
    plants, churches and restaurants.
    \href{https://www.nytimes.com/2020/07/06/health/coronavirus-airborne-aerosols.html?action=click\&pgtype=Article\&state=default\&region=MAIN_CONTENT_3\&context=storylines_faq}{It's
    unclear how often the virus is spread} via these tiny droplets, or
    aerosols, compared with larger droplets that are expelled when a
    sick person coughs or sneezes, or transmitted through contact with
    contaminated surfaces, said Linsey Marr, an aerosol expert at
    Virginia Tech. Aerosols are released even when a person without
    symptoms exhales, talks or sings, according to Dr. Marr and more
    than 200 other experts, who
    \href{https://academic.oup.com/cid/article/doi/10.1093/cid/ciaa939/5867798}{have
    outlined the evidence in an open letter to the World Health
    Organization}.
  \end{itemize}
\item ~
  \hypertarget{what-are-the-symptoms-of-coronavirus}{%
  \paragraph{What are the symptoms of
  coronavirus?}\label{what-are-the-symptoms-of-coronavirus}}

  \begin{itemize}
  \tightlist
  \item
    Common symptoms
    \href{https://www.nytimes.com/article/symptoms-coronavirus.html?action=click\&pgtype=Article\&state=default\&region=MAIN_CONTENT_3\&context=storylines_faq}{include
    fever, a dry cough, fatigue and difficulty breathing or shortness of
    breath.} Some of these symptoms overlap with those of the flu,
    making detection difficult, but runny noses and stuffy sinuses are
    less common.
    \href{https://www.nytimes.com/2020/04/27/health/coronavirus-symptoms-cdc.html?action=click\&pgtype=Article\&state=default\&region=MAIN_CONTENT_3\&context=storylines_faq}{The
    C.D.C. has also} added chills, muscle pain, sore throat, headache
    and a new loss of the sense of taste or smell as symptoms to look
    out for. Most people fall ill five to seven days after exposure, but
    symptoms may appear in as few as two days or as many as 14 days.
  \end{itemize}
\item ~
  \hypertarget{does-asymptomatic-transmission-of-covid-19-happen}{%
  \paragraph{Does asymptomatic transmission of Covid-19
  happen?}\label{does-asymptomatic-transmission-of-covid-19-happen}}

  \begin{itemize}
  \tightlist
  \item
    So far, the evidence seems to show it does. A widely cited
    \href{https://www.nature.com/articles/s41591-020-0869-5}{paper}
    published in April suggests that people are most infectious about
    two days before the onset of coronavirus symptoms and estimated that
    44 percent of new infections were a result of transmission from
    people who were not yet showing symptoms. Recently, a top expert at
    the World Health Organization stated that transmission of the
    coronavirus by people who did not have symptoms was ``very rare,''
    \href{https://www.nytimes.com/2020/06/09/world/coronavirus-updates.html?action=click\&pgtype=Article\&state=default\&region=MAIN_CONTENT_3\&context=storylines_faq\#link-1f302e21}{but
    she later walked back that statement.}
  \end{itemize}
\end{itemize}

``Teachers cannot require students to wear masks in their classroom,''
the document says, though it also encourages teachers to ``make masks
the culture.''

Several teachers told The New York Times they were concerned about their
health and the health of others. All of them requested anonymity because
they feared retaliation. One teacher said he had numerous underlying
health issues and was afraid to go back into the classroom.

Image

Masks with a Jefferson High School logo.Credit...Melissa Golden for The
New York Times

``I think they're worried about upsetting people who aren't taking Covid
seriously,'' the teacher said. He noted that Governor Kemp is suing
Mayor Keisha Lance Bottoms of Atlanta, a Democrat, over her efforts to
mandate masks in the city. ``So many Republicans at all levels of
government aren't taking this seriously.''

But other teachers say they are ready to go back to work. ``I'm not
paranoid, actually, of the virus. I'm more worried about our kids and
their well-being if we don't get them back into the classroom,'' said
Katie Sellers, an eighth-grade physical science teacher at Jefferson
Middle School.

Ms. Sellers has two students in the district, an eighth grader and a
senior. She said she was letting them make up their own minds about
whether to wear masks. ``My senior has absolutely said no'' on the
grounds that school feels like a ``safe space'' for him, she said.

Dr. McMullan said 2 percent of students have selected the
remote-learning option. Those students will be able to choose between at
least two programs, one provided by a private company called
\href{https://www.edgenuity.com/}{Edgenuity} and the other by
\href{https://www.gavirtualschool.org/}{Georgia Virtual School}, which
is run by the state.

The teachers of online classes will be state certified, but some parents
were disappointed they were not teachers from the Jefferson system. Pete
Fuller, a candidate for a local seat in the State Legislature, said his
two children, an eighth and a ninth grader, would be starting the year
learning from home. ``The choice was basically given and it's not the
choice I want to make,'' he said.

Mr. Fuller said his ninth grader, Rainey Fuller, 14, a trombone player,
tried to attend marching band practice this month, and became
uncomfortable when many students stopped wearing masks by the second
day.

Image

Pete Fuller is having his two children, including Rainey Fuller, start
the school year with home learning. The district superintendent said 2
percent of students are doing so.Credit...Melissa Golden for The New
York Times

Last week, about 30 incoming freshmen and their parents arrived in the
big school auditorium for an introductory session led by the principal,
Brian Moore. Normally, the hundreds of incoming freshmen would show up
in one session, but because of the virus they had broken into smaller
sessions. About a third of them were wearing masks.

As he set out to find his new classrooms with his mother, Hunter Walker,
14, said he did not plan to wear a mask. ``No one around my age has
really been affected by it as much,'' he said. Then he noted that a
football player who had been exposed to the coronavirus had been at a
weight training session last Monday, forcing coaches to shut down
ninth-grade football practice for the week.

It was one of many indications that this could be a very long year on
campus --- or, in fact, a very short one.

Advertisement

\protect\hyperlink{after-bottom}{Continue reading the main story}

\hypertarget{site-index}{%
\subsection{Site Index}\label{site-index}}

\hypertarget{site-information-navigation}{%
\subsection{Site Information
Navigation}\label{site-information-navigation}}

\begin{itemize}
\tightlist
\item
  \href{https://help.nytimes.com/hc/en-us/articles/115014792127-Copyright-notice}{©~2020~The
  New York Times Company}
\end{itemize}

\begin{itemize}
\tightlist
\item
  \href{https://www.nytco.com/}{NYTCo}
\item
  \href{https://help.nytimes.com/hc/en-us/articles/115015385887-Contact-Us}{Contact
  Us}
\item
  \href{https://www.nytco.com/careers/}{Work with us}
\item
  \href{https://nytmediakit.com/}{Advertise}
\item
  \href{http://www.tbrandstudio.com/}{T Brand Studio}
\item
  \href{https://www.nytimes.com/privacy/cookie-policy\#how-do-i-manage-trackers}{Your
  Ad Choices}
\item
  \href{https://www.nytimes.com/privacy}{Privacy}
\item
  \href{https://help.nytimes.com/hc/en-us/articles/115014893428-Terms-of-service}{Terms
  of Service}
\item
  \href{https://help.nytimes.com/hc/en-us/articles/115014893968-Terms-of-sale}{Terms
  of Sale}
\item
  \href{https://spiderbites.nytimes.com}{Site Map}
\item
  \href{https://help.nytimes.com/hc/en-us}{Help}
\item
  \href{https://www.nytimes.com/subscription?campaignId=37WXW}{Subscriptions}
\end{itemize}
