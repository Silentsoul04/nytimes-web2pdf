Sections

SEARCH

\protect\hyperlink{site-content}{Skip to
content}\protect\hyperlink{site-index}{Skip to site index}

\href{https://www.nytimes.com/section/politics}{Politics}

\href{https://myaccount.nytimes.com/auth/login?response_type=cookie\&client_id=vi}{}

\href{https://www.nytimes.com/section/todayspaper}{Today's Paper}

\href{/section/politics}{Politics}\textbar{}Susan Rice Wants to Run for
Office. Will Her First Campaign Be for V.P.?

\url{https://nyti.ms/30R9p2n}

\begin{itemize}
\item
\item
\item
\item
\item
\item
\end{itemize}

\begin{itemize}
\item
  \href{https://www.nytimes.com/2020/07/31/us/elections/biden-vs-trump.html?action=click\&pgtype=Article\&state=default\&region=TOP_BANNER\&context=storylines_menu}{Election
  Updates}
\item
  \href{https://www.nytimes.com/article/biden-vice-president-2020.html?action=click\&pgtype=Article\&state=default\&region=TOP_BANNER\&context=storylines_menu}{Biden's
  V.P. Search}
\item
  \href{https://www.nytimes.com/interactive/2020/07/24/us/politics/trump-biden-campaign-donors.html?action=click\&pgtype=Article\&state=default\&region=TOP_BANNER\&context=storylines_menu}{Map
  of Donations}
\item
  \href{https://www.nytimes.com/interactive/2020/us/elections/delegate-count-primary-results.html?action=click\&pgtype=Article\&state=default\&region=TOP_BANNER\&context=storylines_menu}{Delegate
  Count}
\item
  \href{https://www.nytimes.com/interactive/2019/us/politics/2020-presidential-candidates.html?action=click\&pgtype=Article\&state=default\&region=TOP_BANNER\&context=storylines_menu}{The
  Candidates}
\item
  \href{https://www.nytimes.com/newsletters/politics?action=click\&pgtype=Article\&state=default\&region=TOP_BANNER\&context=storylines_menu}{Politics
  Newsletter}
\end{itemize}

Advertisement

\protect\hyperlink{after-top}{Continue reading the main story}

Supported by

\protect\hyperlink{after-sponsor}{Continue reading the main story}

\hypertarget{susan-rice-wants-to-run-for-office-will-her-first-campaign-be-for-vp}{%
\section{Susan Rice Wants to Run for Office. Will Her First Campaign Be
for
V.P.?}\label{susan-rice-wants-to-run-for-office-will-her-first-campaign-be-for-vp}}

The former national security adviser is on the short list to be Joe
Biden's running mate. She has never been elected to public office, but
in 2018 she took a close look at running for the Senate --- and at her
own political strengths and vulnerabilities.

\includegraphics{https://static01.nyt.com/images/2020/07/24/us/politics/00susanrice-01/00susanrice-01-articleLarge.jpg?quality=75\&auto=webp\&disable=upscale}

\href{https://www.nytimes.com/by/alexander-burns}{\includegraphics{https://static01.nyt.com/images/2018/09/25/multimedia/author-alexander-burns/author-alexander-burns-thumbLarge-v2.png}}

By \href{https://www.nytimes.com/by/alexander-burns}{Alexander Burns}

\begin{itemize}
\item
  Published July 27, 2020Updated July 29, 2020
\item
  \begin{itemize}
  \item
  \item
  \item
  \item
  \item
  \item
  \end{itemize}
\end{itemize}

On an autumn Friday not long before the 2018 elections, Susan E. Rice
was traveling through the Phoenix airport and watching from afar as
Justice Brett M. Kavanaugh moved steadily toward confirmation. The
convulsive Senate battle had reached a climax, and for Ms. Rice's party
an unhappy one: Senator Susan Collins, the Maine Republican, had just
announced her support for Justice Kavanaugh, effectively sealing his
victory.

When a former White House colleague tweeted plaintively, asking who
might take down Ms. Collins in the 2020 election, Ms. Rice
\href{https://twitter.com/ambassadorrice/status/1048305496732491777}{fired
off} a two-letter reply: ``Me.''

The message excited Ms. Rice's followers, startled her friends and
puzzled Democratic Party leaders, most of whom were surprised to learn
the former national security adviser had any interest in electoral
politics. Party strategists were already in the process of recruiting a
challenger for Ms. Collins, and Ms. Rice had not been on their radar as
an option. Though she had family roots in Maine, she did not even live
in the state.

In public, Ms. Rice did little to clarify her intentions, and she made
no overtures to the Democratic Senatorial Campaign Committee. When Ms.
Rice announced several months later that she had decided against running
for family reasons, most Democrats concluded she had never given it real
consideration.

They were wrong: Before ruling out the race, Ms. Rice had quietly
explored the idea of battling Ms. Collins for weeks, seeking advice from
seasoned politicians in Maine, friendly operatives in Washington and top
advisers to former President Barack Obama, including Valerie Jarrett and
the pollster Joel Benenson. Within her political circle, the sincerity
of her interest was clear.

\includegraphics{https://static01.nyt.com/images/2020/07/24/us/politics/00susanrice-5/merlin_144243435_c507077e-2089-4394-adbb-8ac3dd2b49f5-articleLarge.jpg?quality=75\&auto=webp\&disable=upscale}

In the end, Ms. Rice did not run. But her exploration of the race
represented an emphatic declaration of new political aspirations. It was
Ms. Rice's first and only examination of what it would mean to become a
candidate, and test the appeal of her formidable credentials not to her
fellow experts but to voters for whom the National Security Council is a
distant and obscure institution.

Ms. Rice, 55, is now among
\href{https://www.nytimes.com/article/biden-vice-president-2020.html}{a
handful of women} under consideration to become
\href{https://www.nytimes.com/interactive/2020/us/elections/joe-biden.html}{Joseph
R. Biden Jr.}'s running mate. It is the latest stage in a path to power
that has seen Ms. Rice chosen to be a Rhodes scholar at 21, an assistant
secretary of state at 32 and ambassador to the United Nations little
more than a decade later.

The questions that faced Ms. Rice in 2018 presaged, in some respects,
those that now surround her as a vice-presidential contender: How much
do voters prize government experience, or care about the international
stage? Is the country ready, just years after seeming to reject elite
expertise with the election of
\href{https://www.nytimes.com/interactive/2020/us/elections/donald-trump.html}{President
Trump}, to embrace a candidate defined chiefly as an analytical policy
mind?

And how eager, after all, is Ms. Rice to emerge from the halls of
Washington and plunge into the undignified melee of a national political
campaign?

\hypertarget{latest-updates-2020-election}{%
\section{\texorpdfstring{\href{https://www.nytimes.com/2020/07/31/us/elections/biden-vs-trump.html?action=click\&pgtype=Article\&state=default\&region=MAIN_CONTENT_1\&context=storylines_live_updates}{Latest
Updates: 2020
Election}}{Latest Updates: 2020 Election}}\label{latest-updates-2020-election}}

Updated 2020-08-01T01:26:45.732Z

\begin{itemize}
\tightlist
\item
  \href{https://www.nytimes.com/2020/07/31/us/elections/biden-vs-trump.html?action=click\&pgtype=Article\&state=default\&region=MAIN_CONTENT_1\&context=storylines_live_updates\#link-29fdff45}{Kamala
  Harris, a top vice-presidential contender, confronts double
  standards.}
\item
  \href{https://www.nytimes.com/2020/07/31/us/elections/biden-vs-trump.html?action=click\&pgtype=Article\&state=default\&region=MAIN_CONTENT_1\&context=storylines_live_updates\#link-13ec3d9c}{Karen
  Bass and Susan Rice are rising on Biden's vice-presidential
  shortlist.}
\item
  \href{https://www.nytimes.com/2020/07/31/us/elections/biden-vs-trump.html?action=click\&pgtype=Article\&state=default\&region=MAIN_CONTENT_1\&context=storylines_live_updates\#link-49e9a016}{Trump
  says Russian bounties to kill U.S. troops `never took place.'}
\end{itemize}

\href{https://www.nytimes.com/2020/07/31/us/elections/biden-vs-trump.html?action=click\&pgtype=Article\&state=default\&region=MAIN_CONTENT_1\&context=storylines_live_updates}{See
more updates}

In 2018, at least, Ms. Jarrett said she believed Ms. Rice was
``relishing the chance to actually run for office.''

``She loves a good battle,'' Ms. Jarrett said, adding of Ms. Rice's
deliberations: ``It wasn't just talking to her friends and family. It
was talking to people who would have advised her on the nuts and bolts
of a campaign.''

\hypertarget{a-personal-reckoning}{%
\subsection{`A personal reckoning'}\label{a-personal-reckoning}}

Ms. Rice's electoral inexperience is not the only possible mark against
her in the vice-presidential process: In an election dominated by a
public-health disaster and economic recession, it is unclear how much a
candidate best known for her foreign policy credentials would improve
Mr. Biden's chances. And there are people close to Mr. Biden who fear
that choosing her would force the campaign to spend precious days
relitigating her role in responding to the 2012 terrorist attack on the
American mission in Benghazi, Libya, that left four Americans dead and
prompted months of Republican-led congressional hearings.

While a galaxy of conspiracy theories about the attack has been
discredited, Ms. Rice ended up taking the political fall for appearing
on the Sunday shows to deliver a set of flawed administration talking
points describing it as an outburst of spontaneous violence rather than
organized terrorism. In her 2019 memoir, Ms. Rice wrote that the episode
turned her ``from being a respected if relatively low-profile cabinet
official to a nationally notorious villain or heroine, depending on
one's political perspective.''

Image

~Joseph R. Biden Jr., then vice president, with Ms. Rice in the Oval
Office in 2015. She would bring strong foreign policy experience if
selected, but her abilities on the campaign trail remain an open
question.~Credit...Jacquelyn Martin/Associated Press

She would bring clear strengths to a ticket and administration,
reinforcing Mr. Biden's message of sober and seasoned leadership and
appealing further to Americans who pine for the Obama years. While she
and Mr. Biden have had policy disagreements over the years, they share a
deeply held view of the importance of diplomacy and international
institutions, a concern for promoting democracy and human rights and a
common pride in Obama-era achievements that they helped shape, like the
Paris climate agreement and the Iran nuclear deal.

Should Mr. Biden become president, few other potential
\href{https://www.nytimes.com/2020/07/29/nyregion/val-demings-biden-vp.html}{vice
presidents} might be dispatched as easily on important missions around
the world. Ms. Rice could confidently play that role, Mr. Benenson
suggested, ``while President Biden would do a lot of the repair,
certainly in the early days of the administration, on the national
stage.''

But hanging over everything is the question of Ms. Rice's abilities as a
campaigner. She would be the first person chosen for vice president
without prior elected experience since 1972, when the Democratic ticket
included R. Sargent Shriver, the former Peace Corps director and John F.
Kennedy's brother-in-law --- like Ms. Rice, a diplomat closely linked to
a president sorely missed by his party.

Ms. Rice is up against multiple candidates who have run for president
themselves, including Senators Kamala Harris and Elizabeth Warren, and
others, like Senator Tammy Duckworth of Illinois and Gov. Gretchen
Whitmer of Michigan, who have endured grueling statewide campaigns.

Allies of Ms. Rice have argued privately to Biden advisers that the
learning curve for a first-time candidate might be smoother than normal
given the strictures of a pandemic-era campaign. If a town-hall meeting
or rally might be a relatively new setting for Ms. Rice, a television
studio or webinar surely would not. They point, too, to the electoral
inexperience on the opposing ticket: Ms. Rice, after all, has won
exactly as many elections as Mr. Trump did before defeating Hillary
Clinton in 2016.

Ertharin Cousin, the former executive director of the United Nations
World Food Program who is friends with Ms. Rice, said Ms. Rice had
confided not long after Mr. Obama left office that she was intrigued by
electoral politics, though she did not specify Maine as a venue. More
recently, Ms. Cousin said, Ms. Rice had confirmed her interest in the
vice presidency.

``She said to me: Joe Biden knows me and he knows my capabilities and if
he thinks I'm right for him, then I'd be honored to serve with him, full
stop,'' Ms. Cousin said.

Ms. Cousin, who traveled with Ms. Rice in South Carolina during the 2008
presidential primary there, said that even then voters recognized her
from her media appearances and connected with her as ``a smart Black
woman.'' The country has few Black diplomats, Ms. Cousin noted, and
voters rarely see them up close.

Still, Ms. Cousin allowed that becoming a national candidate was a
daunting hurdle.

Image

Ertharin Cousin in Rome in 2017 as executive director of the United
Nations World Food Program. She said Ms. Rice had confirmed her interest
in the vice presidency.~Credit...Andrew Medichini/Associated Press

Even for people who have been deeply involved in presidential politics,
Ms. Cousin said, ``I think the experience for the candidate is quite
different.''

In an interview, Ms. Rice said she was comfortable on the campaign
trail, pointing to her activities for Mr. Obama. Without addressing the
vice presidency explicitly, Ms. Rice said she remained interested in
running for office. She left open the door to seeking a Senate seat in
Washington, D.C.,~where she grew up and has spent most of her
professional life,~if the city were to achieve statehood.

Though not a Mainer herself, Ms. Rice's family is closely tied to the
state: Her maternal grandparents emigrated there from Jamaica in the
early 20th century, her mother was raised in Maine, and all the men of
that generation attended Bowdoin College. Ms. Rice's mother, Lois
Dickson Rice, who died in 2017, grew up in the state before graduating
from Radcliffe College and settling down in Washington.

Exploring the race in Maine, Ms. Rice said she had come away convinced
she understood the needs of the state. She had a clear sense of what it
would have taken to beat Ms. Collins, a dogged campaigner long viewed in
Maine as a careful moderate. Ms. Rice's message, she said, would have
been about Maine's ``real socioeconomic challenges,'' like broadband
access and providing health care to an aging population.

``It is true I have never run for office on my own behalf, but I've run
for office on behalf of others,'' Ms. Rice said in an interview from a
vacation home in Maine's Midcoast region. ``If I were to decide to do
it, there's nothing about it that on its face would feel uncomfortable
or unfamiliar.''

The decision not to run for Senate, she said, had been about ``a
personal reckoning'' with not wanting to uproot her family in her
daughter's final years of high school.

``I would have, I think, been able to raise a formidable amount of
money,'' Ms. Rice said. ``And this is a state that twice voted for
Barack Obama, so Maine is capable of supporting people with his
perspective and people who look like me.''

In her memoir, Ms. Rice revealed that as a 10-year-old girl growing up
in Washington she had dreamed of one day becoming a senator. But she
soon learned that her city lacked representation in Congress and, after
spending summers on Capitol Hill, found herself put off by ``many
members' unabashed egotism.''

In the same book, Ms. Rice expressed bluntly critical views of several
senators who had thrown up strong resistance in 2012 to her possible
nomination for secretary of state, effectively blocking her selection.
She named one Republican senator as perhaps her most ``disingenuous''
adversary: Susan Collins.

\hypertarget{candor-and-caution}{%
\subsection{Candor and caution}\label{candor-and-caution}}

Ms. Rice declined to go into detail about the conversations she had
about the Senate race. Several people who spoke to her at the time said
they had stressed the great difficulty of winning office as an outsider
in a state where newcomers are often described as being ``from away.''

Among those cautionary voices was Tom Allen, a former Democratic
congressman and mayor of Portland who ran against Ms. Collins in 2008.
Mr. Allen, who said he had known Ms. Rice's mother, called himself an
admirer of the diplomat and said she had given no definitive signal
about her level of interest in the race.

``When you make inquiries,'' he said, ``you're always serious at some
level.''

One person who did take Ms. Rice seriously was Ms. Collins, who just
days after Ms. Rice's tweet assailed her in a television interview as
lacking even the basic credential of Maine residency. Two Republicans
close to the Collins campaign said that the senator had been excited at
the possibility of facing Ms. Rice, whose identification with the
Benghazi attack and the Iran nuclear deal might have helped Ms. Collins
soothe the discontent she has faced from conservatives who see her as
inadequately loyal to Mr. Trump.

The president himself has often joined in those attacks on Ms. Rice over
the years, most recently having accused her, without evidence, of having
participated in an Obama administration plot against Michael T. Flynn,
the retired general and disgraced former national security adviser. No
such effort has been documented, and **** Ms. Rice has denied being
involved in any such maneuvering against Mr. Flynn, who later pleaded
guilty to lying to federal investigators in a case that is still in
court.

Image

Senator Susan Collins was one of several senators who resisted Ms.
Rice's possible nomination for secretary of state.~Credit...Erin
Schaff/The New York Times

In February 2019, Ms. Collins's campaign took a poll and came away
unimpressed by Ms. Rice's standing: It found the senator leading her by
16 percentage points, 47 percent to 31 percent, people briefed on the
data said. (``I'm glad she wasted money on it,'' Ms. Rice said of the
poll.)

In her overtures to Democrats, Ms. Rice relied on a network of contacts
from her time in the Obama administration, conferring with Michael
Cuzzi, a former strategist for Mr. Obama's campaign in Maine. About a
week after her tweet, she made an unadvertised appearance at a Rockport
fund-raiser for Janet Mills, now the governor of Maine.

But Ms. Rice did not seek to court a swelling grass-roots movement, in
Maine and Washington, that was mobilizing against Ms. Collins after her
vote for Justice Kavanaugh. Other candidates were doing so, including
Sara Gideon, the State House speaker, who would soon win support from
the Democratic Senatorial Campaign Committee and build a grass-roots
following online that helped her raise \$9 million in just a few months
this spring.

Ms. Rice eventually confirmed in April 2019, six months after her tweet,
that she would not run for Senate. Some time after that, Ms. Rice said
in an interview with
\href{https://www.pressherald.com/2019/10/23/susan-rice-to-talk-about-new-book-maine-roots-during-portland-event/}{The
Portland Press-Herald} that Maine ``deserves senators who live there.''

Jim Mitchell, a Democratic lobbyist in Maine and former state party
chair, said that there had been a swirl of excitement about Ms. Rice,
but that few people in the state felt as if they could take her measure
as a candidate.

``Retail politics still matters in a place like Maine, because there
aren't a lot of people,'' he said. ``I have no idea if the ambassador
has those skills.''

Should the vice-presidential nomination go to another candidate, Ms.
Rice would most likely be a top candidate for other offices in a Biden
administration, perhaps including secretary of state.

There is at least, in theory, another job prospect on the horizon:
senator from the newly admitted state of Washington, D.C. It is perhaps
an unlikely prospect, but so, too, were the ideas of moving to Maine and
toppling a tenacious local Republican, or ascending more or less
directly from the National Security Council to the vice presidency. In
June, the Democratic-controlled House
\href{https://www.nytimes.com/2020/06/26/us/politics/dc-statehood-house-vote.html}{voted
in favor of statehood}.

Ms. Rice, who in June wrote
\href{https://www.nytimes.com/2020/06/09/opinion/trump-military-washington-statehood.html}{a
New York Times Op-Ed column} calling for an end to ``the enduring
oppression of the citizens of the District of Columbia,'' said she might
be open to an eventual Senate run there.

``I'm a huge champion of statehood for D.C.,'' she said, ``separate and
apart from my own interests.''

\hypertarget{our-2020-election-guide}{%
\section{Our 2020 Election Guide}\label{our-2020-election-guide}}

Updated July 31, 2020

\begin{itemize}
\item
  \begin{center}\rule{0.5\linewidth}{\linethickness}\end{center}

  \hypertarget{the-latest}{%
  \subsection{The Latest}\label{the-latest}}

  \begin{itemize}
  \tightlist
  \item
    President Trump's assault on the Postal Service is intersecting with
    his attacks on mail-in voting.
    \href{https://www.nytimes.com/2020/07/31/us/politics/trump-usps-mail-delays.html?action=click\&pgtype=Article\&state=default\&region=BELOW_MAIN_CONTENT\&context=storylines_guide}{Voting
    rights groups say it is a recipe for disaster.}
  \end{itemize}
\item
  \begin{center}\rule{0.5\linewidth}{\linethickness}\end{center}

  \hypertarget{bidens-vp-search}{%
  \subsection{Biden's V.P. Search}\label{bidens-vp-search}}

  \begin{itemize}
  \tightlist
  \item
    \href{https://www.nytimes.com/article/biden-vice-president-2020.html?action=click\&pgtype=Article\&state=default\&region=BELOW_MAIN_CONTENT\&context=storylines_guide}{Here
    are 13 women} who have been under consideration to be Joe Biden's
    running mate, and why each might be chosen --- and might not be.
  \end{itemize}
\item
  \begin{center}\rule{0.5\linewidth}{\linethickness}\end{center}

  \hypertarget{keep-up-with-our-coverage}{%
  \subsection{Keep Up With Our
  Coverage}\label{keep-up-with-our-coverage}}

  \begin{itemize}
  \tightlist
  \item
    Get an
    \href{https://www.nytimes.com/newsletters/politics?action=click\&pgtype=Article\&state=default\&region=BELOW_MAIN_CONTENT\&context=storylines_guide}{email}
    recapping the day's news
  \end{itemize}

  \begin{itemize}
  \tightlist
  \item
    Download our mobile app on
    \href{https://apps.apple.com/us/app/nytimes/id284862083?ls=1\&mat_click_id=5c79ae7455014fd1bd66b5610c05b8f2-20191112-16948\&referrer=mat_click_id\%3D5c79ae7455014fd1bd66b5610c05b8f2-20191112-16948\%26link_click_id\%3D722930677036718082}{iOS}
    and
    \href{http://a.localytics.com/android?id=com.nytimes.android\&referrer=utm_source\%3Dother_nyt_mobile_web\%26utm_medium\%3DWeb\%2520page\%26utm_term\%3DGeneral\%2520Mobile\%2520Page\%26utm_campaign\%3DNYT\%2520Mobile\%2520General\%2520Page}{Android}
    and turn on Breaking News and Politics alerts
  \end{itemize}
\end{itemize}

Advertisement

\protect\hyperlink{after-bottom}{Continue reading the main story}

\hypertarget{site-index}{%
\subsection{Site Index}\label{site-index}}

\hypertarget{site-information-navigation}{%
\subsection{Site Information
Navigation}\label{site-information-navigation}}

\begin{itemize}
\tightlist
\item
  \href{https://help.nytimes.com/hc/en-us/articles/115014792127-Copyright-notice}{©~2020~The
  New York Times Company}
\end{itemize}

\begin{itemize}
\tightlist
\item
  \href{https://www.nytco.com/}{NYTCo}
\item
  \href{https://help.nytimes.com/hc/en-us/articles/115015385887-Contact-Us}{Contact
  Us}
\item
  \href{https://www.nytco.com/careers/}{Work with us}
\item
  \href{https://nytmediakit.com/}{Advertise}
\item
  \href{http://www.tbrandstudio.com/}{T Brand Studio}
\item
  \href{https://www.nytimes.com/privacy/cookie-policy\#how-do-i-manage-trackers}{Your
  Ad Choices}
\item
  \href{https://www.nytimes.com/privacy}{Privacy}
\item
  \href{https://help.nytimes.com/hc/en-us/articles/115014893428-Terms-of-service}{Terms
  of Service}
\item
  \href{https://help.nytimes.com/hc/en-us/articles/115014893968-Terms-of-sale}{Terms
  of Sale}
\item
  \href{https://spiderbites.nytimes.com}{Site Map}
\item
  \href{https://help.nytimes.com/hc/en-us}{Help}
\item
  \href{https://www.nytimes.com/subscription?campaignId=37WXW}{Subscriptions}
\end{itemize}
