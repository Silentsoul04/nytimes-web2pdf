Sections

SEARCH

\protect\hyperlink{site-content}{Skip to
content}\protect\hyperlink{site-index}{Skip to site index}

\href{https://www.nytimes.com/section/arts/television}{Television}

\href{https://myaccount.nytimes.com/auth/login?response_type=cookie\&client_id=vi}{}

\href{https://www.nytimes.com/section/todayspaper}{Today's Paper}

\href{/section/arts/television}{Television}\textbar{}`Last Chance U'
Travels to Oakland, Just Like the Players

\url{https://nyti.ms/30WBiWU}

\begin{itemize}
\item
\item
\item
\item
\item
\item
\end{itemize}

Advertisement

\protect\hyperlink{after-top}{Continue reading the main story}

Supported by

\protect\hyperlink{after-sponsor}{Continue reading the main story}

\hypertarget{last-chance-u-travels-to-oakland-just-like-the-players}{%
\section{`Last Chance U' Travels to Oakland, Just Like the
Players}\label{last-chance-u-travels-to-oakland-just-like-the-players}}

For its final season about junior college football, the Netflix series
looks at a commuter college in a gentrifying city where many of the
players can't afford to live.

\includegraphics{https://static01.nyt.com/images/2020/08/02/arts/02last-chance1/02last-chance1-articleLarge-v2.jpg?quality=75\&auto=webp\&disable=upscale}

By Scott Tobias

\begin{itemize}
\item
  July 28, 2020
\item
  \begin{itemize}
  \item
  \item
  \item
  \item
  \item
  \item
  \end{itemize}
\end{itemize}

It was supposed to be a breakout season for Dior Walker-Scott. As a
returning sophomore for the Laney College football team in Oakland,
which had won the California state championship the year before,
Walker-Scott was the centerpiece of the young Eagles offense --- a
tough, compact, 5'8'' wide receiver whose quickness and strength allowed
him to get him separation from much taller cornerbacks and safeties.

Now suddenly he was on the floor, seized by a sharp pain in his chest.

``I felt like I was dying, to be honest,'' he said in a phone interview
last week. The team's three quarterbacks had all been injured, and
Walker-Scott had been asked to step in and play quarterback for the
first time since high school --- a move that risked costing him the
crucial game film he needed to secure a scholarship.

``It was the stress,'' he added. ``It was just me getting in my own
head, saying, `I'm not good enough.'''

The pain persisted throughout the day, reasserting itself later as
Walker-Scott tried to will his way through practice --- a struggle
captured in the fifth season of
``\href{https://www.netflix.com/title/80091742}{Last Chance U,}'' a
documentary series about the smash-mouth abattoir of junior college
football, which returned to Netflix on Tuesday. (The show will shift to
junior college basketball next season.) But the cameras missed when he
dragged himself to the coach's office that morning, in the grips of a
full-on anxiety attack.

Coach John Beam, a fixture in Oakland football for about 40 years,
remembered stopping a meeting to tend to his star receiver.

``He and I went off and did 30 minutes of breathing exercises,'' Beam
said from his home in Oakland, where he's managing a season delayed
indefinitely by the coronavirus. ``We didn't know if we'd have to call
911.''

Fans of ``Last Chance U'' will recognize certain recurring elements:
young, mostly Black athletes committed to a desperate bid to play at the
next level; a larger-than-life coach dashing up and down the sidelines,
peppering them with invective; the ups-and-downs of each game, rendered
with bone-crunching immediacy. And yet, the Laney College season is a
significant departure for the show, feeding off a coach who is given to
hugging his players and talking to them about the importance of being
vulnerable and ``letting the pain out.''

\includegraphics{https://static01.nyt.com/images/2020/08/02/arts/02last-chance5/merlin_174858495_a036978f-a9c2-46e2-8fe0-2955e5b36dc0-articleLarge.jpg?quality=75\&auto=webp\&disable=upscale}

In its move to Oakland, the new season also sharpens the show's focus on
contemporary racial and socio-economic issues at a moment of intensified
demand for better and more honest stories about Black lives like
Walker-Scott's. Laney athletes struggle to avoid career-ending injuries
in a violent game while juggling children, jobs and even hunger in a
rapidly gentrifying city where many can't afford to live. Walker-Scott
himself is shown as homeless, crawling each night into the back seat of
his car to sleep.

``My main focus was not to worry about now,'' he said. ``Just keep
grinding. Just keep doing what you're doing because eventually it's
going to get better.''

Junior college football is by nature a high-stakes proposition: Athletes
who can't attend a four-year college after high school often pursue a
two-year associate degree with the goal of transferring junior year. Any
hopes of playing professionally --- often seen as the only ticket out of
grim circumstances --- hinge on locking down a scholarship from a
Division 1 recruiter, or at least enough highlight footage to attempt a
walk-on.

The challenges at Laney are particularly acute. Previous seasons of
``Last Chance'' were set on rural campuses --- East Mississippi
Community College in Scooba, Miss., and Independence Community College
in Independence, Kan. --- where players faced similarly narrow prospects
but there was never a question of where they would sleep or get their
next meal.

At Laney, a commuter college, players are not on scholarship or eligible
for benefits like on-campus housing. Those difficulties are due in part
to rules specific to the California Community College Athletic
Association, noted the show's creator, Greg Whiteley, which is cordoned
off from national governing bodies. That was a major reason for his
decision this time to pick a California school for Season 5.

Image

Nu'u Taugavau quit his job as a Walmart greeter to try his chances at
junior college football. He saw the sport as a possible ticket out of
tough circumstances for him, his wife and two young
children.~Credit...Netflix

``A lot of times these kids are piling up in a single-bedroom apartment
or a studio apartment,'' Whiteley said of California junior college
athletes. ``In some cases, as we documented this year, there's even more
extreme living circumstances.''

For the six-foot-two, 300-pound offensive lineman Nu'u Taugavau, who
quit his job as a Walmart greeter to try his chances at Laney, going to
school means applying for food stamps to help feed his two young
children. For Laney's star cornerback, Rejzohn Wright, it means driving
hours every day to and from the suburbs while managing the trauma of a
violent family tragedy.

Coach Beam bristled at some of the strict rules enforced by the
association, which until last year prohibited its sports programs from
providing special food assistance to athletes.

``When a kid was hungry, I couldn't go out and get him a Cup O' Noodles
--- that was considered an impermissible benefit because I'm giving it
to an athlete and not everybody,'' Beam said. ``When Dior is sleeping in
his car, I legally cannot help him.''

Still, the larger challenge for Beam, as it was for the coaches in
previous seasons, has been to navigate what Whiteley called ``two
competing objectives'' of the job: ``One is to win and have a successful
program, and the other is to do what's best for the young men that
they're coaching.''

Image

``My main focus was not to worry about now,'' said Walker-Scott, who had
to sleep each night in his car. ``Just keep grinding. Just keep doing
what you're doing because eventually it's going to get
better.''Credit...Netflix

Those goals are ostensibly aligned because the players get recruitment
benefits from succeeding on the field and in the classroom. But the
players, many of them teenagers from hardscrabble backgrounds, are often
more sensitive than their sturdy frames suggest.

``To many young men, especially young men of color, showing your
vulnerability is a sign of weakness,'' said Beam. ``And we're trying to
say: `No. It's not. In fact, it's a sign of strength. Asking for help is
a strength move, a power move.'''

Beam acknowledged that there is only so much a coach can do. He
sometimes turns athletes like Walker-Scott over to his wife, Cindi
Rivera-Beam, a therapist who
\href{https://cindiriveratherapy.com}{specializes} in services for
people of color, or to a social worker, Carlisa Harris, who was brought
into the program last year.

``In a masculine game like football, it's OK to be vulnerable,'' Beam
said. ``It's OK to hug. It's OK to cry. At the same time, I'm a
competitive guy. But you always have to remember that there's a
fragility there that we have to be careful of.''

Beam's long history working with underprivileged young men makes a
difference. The ESPN sportswriter Tim Keown, who appears throughout the
season, has known Beam for nearly 30 years, going back to when Beam was
head coach at Skyline High School in Oakland.

``He just got it with the kids,'' Keown said by phone. ``He understood
where they were coming from, what they were, what obstacles were in
their way, what they had to overcome just to get to school.''

Keown added: ``Some of them had to make three bus transfers just to get
to school every day. And he never lost sight of that.''

Image

Coach John Beam takes a tough-love approach that includes plenty of
yelling but also a lot of hugs. ``In a masculine game like football,
it's OK to be vulnerable,'' he said.Credit...Andres Gonzalez for The New
York Times

The long-shot N.F.L. dreams of disadvantaged teenagers have always been
the dramatic lifeblood of ``Last Chance U'': The show portrays junior
college football as a kind of purgatory, where athletes labor under
tempestuous coaches who want to deliver them to Division I programs but
also need to win games.

But a lot has changed since last season. The global outcry following the
killing of George Floyd has intensified calls to re-examine insidious
racial dynamics --- and with that, changed the context in which such
player-coach relationships might be understood.

Earlier seasons already underscored the inherent tensions. Coach Buddy
Stephens of East Mississippi, a booming field-general type who was once
suspended for getting into a fistfight with a referee, had to walk back
an ugly moment when he upbraided a locker room of mostly Black players
for acting like ``thugs.'' Coach Jason Brown of Independence, who is
also white, believed that his past as a tough guy from Compton, Calif.,
helped him recruit star Black prospects, but it didn't insulate him from
developing dysfunctional relationships with his players.

Beam, who does no shortage of yelling but also puts his players through
mindfulness workshops, talked about the need for coaches like him to
recognize the injustices and systemic racism players face --- to ``help
absorb some of that pain.'' But he acknowledged that players of color
ultimately carry the burden, which he can only try to understand.

``That's a weight on their shoulders daily,'' he said. ``And yet they
still put a smile on their face and they show up ready to go.''

In making ``Last Chance U,'' Whiteley, who is white and also created the
Netflix series ``Cheer'' about a competitive college cheer squad in
Texas, said he has tried to be mindful of his own responsibilities in
conveying these players' stories. (``I consider it a genuine honor that
so many compelling young people of varying backgrounds have opened up
their lives and trusted me with their stories,'' he wrote in an email.)

For his part, Walker-Scott seems determined to write his own. Nights, he
is shown working at the fast-food chain Wingstop, where he logs 10 to 20
hours a week during the season. He fends off relatives who pressure him
to reconcile with a father who he says beat him with a belt and
sabotaged his senior year of high school football. (``You need to
understand that I was in the Navy,'' his father, Jarvis Walker, responds
in the series during a separate interview. ``So in my house, it's a
tight ship.'')

Image

Season 5 follows the Laney Eagles through the 2019 season, when the
pressure is high to defend its state championship from the year
before.Credit...Netflix

Sharing those experiences on camera ``was an overwhelming experience at
first,'' Walker-Scott said, and he was scared to open up. But Whiteley
talked him into occupying a central role, which he came to find
empowering.

``He said, `Look, your story is tremendous --- I know you're going to be
scared to talk about it at first, but you're really what JuCo football
is about,''' Walker-Scott said. ``Later on in the season, I felt more
comfortable. I felt great knowing my story would be out there and I
could help somebody else.''

Throughout the 2019 season, Walker-Scott has an eye on playing at the
University of Hawaii, where Beam recently placed another wide receiver.
The coronavirus has since added more uncertainty to an already cloudy
future, but Beam was persuaded by Walker-Scott's grit.

``You've got to root for him,'' Beam said. ``He's going to be successful
in life. I'm not sure how or what he's going to do, but he will because
he has the resolve to get through this.''

Advertisement

\protect\hyperlink{after-bottom}{Continue reading the main story}

\hypertarget{site-index}{%
\subsection{Site Index}\label{site-index}}

\hypertarget{site-information-navigation}{%
\subsection{Site Information
Navigation}\label{site-information-navigation}}

\begin{itemize}
\tightlist
\item
  \href{https://help.nytimes.com/hc/en-us/articles/115014792127-Copyright-notice}{©~2020~The
  New York Times Company}
\end{itemize}

\begin{itemize}
\tightlist
\item
  \href{https://www.nytco.com/}{NYTCo}
\item
  \href{https://help.nytimes.com/hc/en-us/articles/115015385887-Contact-Us}{Contact
  Us}
\item
  \href{https://www.nytco.com/careers/}{Work with us}
\item
  \href{https://nytmediakit.com/}{Advertise}
\item
  \href{http://www.tbrandstudio.com/}{T Brand Studio}
\item
  \href{https://www.nytimes.com/privacy/cookie-policy\#how-do-i-manage-trackers}{Your
  Ad Choices}
\item
  \href{https://www.nytimes.com/privacy}{Privacy}
\item
  \href{https://help.nytimes.com/hc/en-us/articles/115014893428-Terms-of-service}{Terms
  of Service}
\item
  \href{https://help.nytimes.com/hc/en-us/articles/115014893968-Terms-of-sale}{Terms
  of Sale}
\item
  \href{https://spiderbites.nytimes.com}{Site Map}
\item
  \href{https://help.nytimes.com/hc/en-us}{Help}
\item
  \href{https://www.nytimes.com/subscription?campaignId=37WXW}{Subscriptions}
\end{itemize}
