Sections

SEARCH

\protect\hyperlink{site-content}{Skip to
content}\protect\hyperlink{site-index}{Skip to site index}

\href{https://www.nytimes.com/section/books/review}{Book Review}

\href{https://myaccount.nytimes.com/auth/login?response_type=cookie\&client_id=vi}{}

\href{https://www.nytimes.com/section/todayspaper}{Today's Paper}

\href{/section/books/review}{Book Review}\textbar{}The Chinese Town That
Became the Self-Immolation Capital of the World

\url{https://nyti.ms/3hJRKR6}

\begin{itemize}
\item
\item
\item
\item
\item
\end{itemize}

Advertisement

\protect\hyperlink{after-top}{Continue reading the main story}

Supported by

\protect\hyperlink{after-sponsor}{Continue reading the main story}

Nonfiction

\hypertarget{the-chinese-town-that-became-the-self-immolation-capital-of-the-world}{%
\section{The Chinese Town That Became the Self-Immolation Capital of the
World}\label{the-chinese-town-that-became-the-self-immolation-capital-of-the-world}}

\includegraphics{https://static01.nyt.com/images/2020/08/02/books/review/02Fadiman1/merlin_174383289_6c33eaff-ba59-4395-afe1-28e40b892f3d-articleLarge.jpg?quality=75\&auto=webp\&disable=upscale}

Buy Book ▾

\begin{itemize}
\tightlist
\item
  \href{https://www.amazon.com/gp/search?index=books\&tag=NYTBSREV-20\&field-keywords=Eat+the+Buddha+Barbara+Demick}{Amazon}
\item
  \href{https://du-gae-books-dot-nyt-du-prd.appspot.com/buy?title=Eat+the+Buddha\&author=Barbara+Demick}{Apple
  Books}
\item
  \href{https://www.anrdoezrs.net/click-7990613-11819508?url=https\%3A\%2F\%2Fwww.barnesandnoble.com\%2Fw\%2F\%3Fean\%3D0812998758}{Barnes
  and Noble}
\item
  \href{https://www.anrdoezrs.net/click-7990613-35140?url=https\%3A\%2F\%2Fwww.booksamillion.com\%2Fp\%2FEat\%2Bthe\%2BBuddha\%2FBarbara\%2BDemick\%2F0812998758}{Books-A-Million}
\item
  \href{https://bookshop.org/a/3546/0812998758}{Bookshop}
\item
  \href{https://www.indiebound.org/book/0812998758?aff=NYT}{Indiebound}
\end{itemize}

When you purchase an independently reviewed book through our site, we
earn an affiliate commission.

By Anne Fadiman

\begin{itemize}
\item
  July 28, 2020
\item
  \begin{itemize}
  \item
  \item
  \item
  \item
  \item
  \end{itemize}
\end{itemize}

\textbf{EAT THE BUDDHA}\\
\textbf{Life and Death in a Tibetan Town}\\
By Barbara Demick

Tibetans encountered Chinese Communists for the first time during the
Long March of the mid-1930s, when Mao's Red Army evaded the Nationalist
forces by heading west and north through the Tibetan plateau. The
famished Chinese soldiers picked the fields bare. They stole yaks, sheep
and grain (though some of them, reluctant to jettison the Communist
principle of helping the rural poor, left i.o.u.s). They swept through
monasteries, melting down copper urns for shrapnel, ripping up
floorboards for firewood, sitting on sacred scroll paintings and eating
boiled yak hide torn from temple drums. They were delighted to discover
that \emph{tormas} --- votive offerings made of barley flour and butter
--- were also edible. Some \emph{tormas} are sculpted in human form, and
the soldiers, assuming they were committing a sacrilege but too hungry
to care, believed they were eating statues of the Buddha.

Hence the title of ``Eat the Buddha,'' a brilliantly reported and
eye-opening work of narrative nonfiction by Barbara Demick, the former
Beijing bureau chief of The Los Angeles Times, on the history of Tibetan
resistance to Chinese domination. Demick centers the book in and around
the town of Ngaba, on the eastern plateau. I was initially disappointed
to learn that Ngaba isn't in the Tibet Autonomous Region --- the
territory, governed by China, whose capital is Lhasa and which most of
us think of as Tibet --- but rather in Sichuan, one of the four Chinese
provinces in which the majority of Tibetans live. I assumed that Demick
hadn't focused on the TAR because of access problems: Visiting
journalists must obtain permission from the Chinese government, which is
rarely granted, and are usually required to travel with supervised
tours. But it soon became apparent that Ngaba --- which has access
challenges of its own, though more surmountable ones --- was
\emph{exactly} the right place to write about. Nowhere else, inside or
outside the TAR, has been a more intense hotbed of Tibetan political
unrest.

Ngaba currently has steel barricades at the entrances to town and
surveillance cameras that record the license plates of all cars arriving
and leaving, and, by one count, some 50,000 security personnel. (The
town's population is around 15,000.) Demick guides us through the phases
of oppression and defiance, decade by appalling decade, which have led
the Chinese government to exert such heavy-handed control.

\includegraphics{https://static01.nyt.com/images/2020/05/27/books/review/Fadiman1/Fadiman1-articleLarge.jpg?quality=75\&auto=webp\&disable=upscale}

In the 1930s, the Red Army brought famine; the local residents fought
back with spears, flintlocks and muskets. In 1958, at the beginning of
Mao's Great Leap Forward, the Chinese government deposed a beloved
regional king, forced the local people into collective farms,
confiscated livestock, closed markets, requisitioned or destroyed the
monasteries and beat or shot those who refused to fall in line.
Thousands starved. Demick writes, ``Tibetans of this generation refer to
this period simply as \emph{ngabgay} --- '58. Like 9/11, it is shorthand
for a catastrophe so overwhelming that words cannot express it, only the
number.''

Ten years later, the people of Ngaba rose up in a bloody rebellion that
ended with mass arrests and more than 50 deaths. During the late 1980s,
Ngaba residents who made or posted fliers supporting the Dalai Lama ---
their spiritual leader, who had fled Tibet for India in 1959 --- were
imprisoned. In 2008, in another Ngaba uprising, at least a dozen people
were killed.

The cycle of resistance, crackdown, resistance, crackdown --- with the
crackdowns serving mainly as goads for further resistance --- culminated
when locals, most of them current or former monks from Ngaba's Kirti
Monastery, found a new and uniquely public way to protest Chinese rule
and call for the return of the Dalai Lama. In 2009, they started setting
themselves on fire. Over the next 10 years, nearly a third of Tibet's
156 self-immolations would take place in or near Ngaba. Many of the
self-immolators have been the grandchildren of men who bore arms in
earlier uprisings. ``The older generation produced the fighters,''
Demick writes. ``The younger people, educated during the time of the
14th Dalai Lama, took his teachings about nonviolence to heart. They
couldn't bring themselves to kill anyone but themselves.''

Image

The first monk to attempt self-immolation survived, but his successors
upped their chances of success by swallowing gasoline as well as dousing
themselves in it and wrapping themselves in wire-trussed quilts. Ngaba
--- ``this nothing little town that had just gotten its first traffic
light'' --- became the self-immolation capital of the world.

The Chinese government, angered by the latest threat to stability in a
chronically troublesome region, barricaded the monastery, brought in
paramilitary troops, made hundreds of arrests and cut off Ngaba's
internet. Nonetheless, the deaths of many self-immolators found their
way to YouTube. ``In the videos from Ngaba,'' Demick writes, ``one
streaks down a dimly lit gray street like a fireball. Another twitches
and crumples like a piece of paper thrown into a fireplace. Those whose
bodies are completely consumed shrivel as small as children, blackened
and twisting.''

The chapters on the self-immolations are the heart of ``Eat the Buddha''
--- the terrible climax for which Demick has prepared us through her
recounting of more than 60 years of religious repression and human
rights abuses. There's a good deal of exposition, all of it essential,
but whenever possible, she presents Ngaba's brutal history through the
stories of individual characters, the technique pioneered by John Hersey
in ``Hiroshima.'' (Hersey took the idea from the Thornton Wilder novel
``The Bridge of San Luis Rey,'' which he read on his ship en route to
Japan.) I occasionally felt I needed an Excel spreadsheet to keep track
of all the players, some of whom vanish for long stretches --- in one
case, for more than 100 pages --- before re-entering the narrative. But
Demick, who used the same technique to excellent effect in her previous
books (``Logavina Street,'' about Sarajevo, and ``Nothing to Envy,''
about North Korea), knows what she's doing.

As ``Eat the Buddha'' unfolds, we come to understand why she has
introduced this particular cast in sufficient detail to make us care
about them. They aren't just a representative sampling of Ngaba
residents; they are people who have intersected with history. For
instance, the woman who sold counterfeit Nike sneakers turns out to be a
witness to Ngaba's first, failed self-immolation; after the would-be
martyr's robes burned and his face turned black, she saw Chinese
soldiers toss him into the back of a truck, ``like an animal.'' The
young monk who was given his first bath at age 7 after his mother
brought him to the Kirti Monastery turns out to be a close friend of the
second self-immolator and the half brother of the 21st. We are
heartbroken by that last death, as we couldn't be if we read about it in
a newspaper headline, because we've heard about the summer the brothers
spent herding yaks in the hills, sharing a black felt tent and, when
September came, making snow angels together.

I realized early on --- though probably later than some more alert
readers --- that the end of the story for all the major characters would
be Dharamsala, India, the community of 100,000 Tibetans that is the home
of their government-in-exile. Of course it would be. They
\emph{couldn't} still live in Ngaba, since they would not have been safe
from retribution if Demick had interviewed them there. (She visited
Ngaba three times, but almost all her local sources are unnamed.)
Because they couldn't obtain passports, most of them made their way to
Dharamsala via various illegal trajectories, some of them extortionately
expensive, some hair-raising, some both. They are now able to discuss
politics, to worship without restrictions, to display portraits of their
spiritual leader. (And to see him in person. The Dalai Lama has lived
there since 1960.)

But India is no paradise; more exiles are returning to Tibet than
leaving it. Demick writes of Dharamsala, ``I met many Tibetans spinning
with indecision. Their families send them photos on WeChat of new cars
and motorcycles, remodeled houses and appliances'' --- the perks of
China's economic boom. On the other side of the balance there is the
businessman in Ngaba, the owner of an SUV, an iPhone and an iPad, who
tells Demick in the final chapter of this harrowing but necessary book,
``I have everything I might possibly want in life, but my freedom.''

Advertisement

\protect\hyperlink{after-bottom}{Continue reading the main story}

\hypertarget{site-index}{%
\subsection{Site Index}\label{site-index}}

\hypertarget{site-information-navigation}{%
\subsection{Site Information
Navigation}\label{site-information-navigation}}

\begin{itemize}
\tightlist
\item
  \href{https://help.nytimes.com/hc/en-us/articles/115014792127-Copyright-notice}{©~2020~The
  New York Times Company}
\end{itemize}

\begin{itemize}
\tightlist
\item
  \href{https://www.nytco.com/}{NYTCo}
\item
  \href{https://help.nytimes.com/hc/en-us/articles/115015385887-Contact-Us}{Contact
  Us}
\item
  \href{https://www.nytco.com/careers/}{Work with us}
\item
  \href{https://nytmediakit.com/}{Advertise}
\item
  \href{http://www.tbrandstudio.com/}{T Brand Studio}
\item
  \href{https://www.nytimes.com/privacy/cookie-policy\#how-do-i-manage-trackers}{Your
  Ad Choices}
\item
  \href{https://www.nytimes.com/privacy}{Privacy}
\item
  \href{https://help.nytimes.com/hc/en-us/articles/115014893428-Terms-of-service}{Terms
  of Service}
\item
  \href{https://help.nytimes.com/hc/en-us/articles/115014893968-Terms-of-sale}{Terms
  of Sale}
\item
  \href{https://spiderbites.nytimes.com}{Site Map}
\item
  \href{https://help.nytimes.com/hc/en-us}{Help}
\item
  \href{https://www.nytimes.com/subscription?campaignId=37WXW}{Subscriptions}
\end{itemize}
