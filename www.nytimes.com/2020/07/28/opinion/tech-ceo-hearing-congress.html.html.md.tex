Sections

SEARCH

\protect\hyperlink{site-content}{Skip to
content}\protect\hyperlink{site-index}{Skip to site index}

\href{https://myaccount.nytimes.com/auth/login?response_type=cookie\&client_id=vi}{}

\href{https://www.nytimes.com/section/todayspaper}{Today's Paper}

\href{/section/opinion}{Opinion}\textbar{}Four of the World's Wealthiest
Men Are Preparing for Battle

\url{https://nyti.ms/2EljP2y}

\begin{itemize}
\item
\item
\item
\item
\item
\item
\end{itemize}

Advertisement

\protect\hyperlink{after-top}{Continue reading the main story}

\href{/section/opinion}{Opinion}

Supported by

\protect\hyperlink{after-sponsor}{Continue reading the main story}

\hypertarget{four-of-the-worlds-wealthiest-men-are-preparing-for-battle}{%
\section{Four of the World's Wealthiest Men Are Preparing for
Battle}\label{four-of-the-worlds-wealthiest-men-are-preparing-for-battle}}

Members of Congress will be able to grill tech C.E.O.s at a hearing.
Let's hope they don't waste the opportunity.

By
\href{https://www.nytimes.com/interactive/opinion/editorialboard.html}{The
Editorial Board}

The editorial board is a group of opinion journalists whose views are
informed by expertise, research, debate and certain longstanding ****
\href{https://www.nytimes.com/interactive/2018/opinion/editorialboard.html}{values}.
It is separate from the newsroom.

\begin{itemize}
\item
  July 28, 2020
\item
  \begin{itemize}
  \item
  \item
  \item
  \item
  \item
  \item
  \end{itemize}
\end{itemize}

\includegraphics{https://static01.nyt.com/images/2020/07/29/opinion/28tech-editorial-sub/28tech-editorial-articleLarge.jpg?quality=75\&auto=webp\&disable=upscale}

Amid the coronavirus pandemic, Congress's attention will turn briefly to
another matter: world domination. On Wednesday, the chief executives of
Facebook, Amazon, Apple and Google's parent company, Alphabet, will
appear together to defend themselves in a continuing inquiry into their
market power by the congressional subcommittee focused on antitrust.

The virus's reach means
\href{https://www.nytimes.com/live/2020/07/29/technology/tech-ceos-hearing-testimony}{Mark
Zuckerberg of Facebook, Jeff Bezos of Amazon, Tim Cook of Apple and
Sundar Pichai of Alphabet} will dial into the hearing via video chat,
avoiding the glare of photographers, members of Congress and the gallery
as they are peppered with questions. While in their home offices, the
men also could receive messages from aides about how to respond.

Nonetheless, the one-day hearing, conducted by the House Judiciary
Committee's antitrust subcommittee, could give the public a rare glimpse
into the inner workings of some of the world's most valuable companies.
Panel members are likely to turn a good deal of attention to Mr. Bezos,
the world's wealthiest person, in part because he's never appeared
before Congress. (He initially
\href{https://www.nytimes.com/2020/05/15/technology/jeff-bezos-amazon-testimony-congress.html}{tried
to avoid it} this time.)

The challenge for the subcommittee will be in establishing whether these
tech companies ---~which have amassed immeasurable power --- operate as
illegal monopolies in certain domains, such as online search (Google),
online marketplaces (Amazon), mobile phone app stores (Apple), the
dissemination of information (Facebook), advertising sales (Google and
Facebook) and mergers and acquisitions.

It's not against the law to be the largest search engine, online
marketplace or social media network. But antitrust laws, meant to
protect against outsize market power, do not allow companies to suppress
competition --- a practice known as
\href{https://www.ftc.gov/tips-advice/competition-guidance/guide-antitrust-laws/single-firm-conduct/monopolization-defined}{exclusionary
conduct} --- by, for instance, quashing or gobbling up potential rivals.
Freed from competition, companies may also cross the line by squeezing
suppliers or imposing higher prices on consumers.

Wednesday's hearing will be the culmination of a roughly yearlong
investigation into the businesses' operations and is likely to be the
final public forum before the subcommittee releases its findings,
expected in the fall.

Representative David Cicilline, Democrat of Rhode Island and the
subcommittee's chairman, has indicated the tech companies may not like
the results, this month calling the companies' power
\href{https://www.nytimes.com/2020/07/01/opinion/anti-trust-tech-hearing-facebook.html}{``terrifying.''}

It's up to the subcommittee to help the public understand the breadth of
the companies' power and, potentially, to recommend that regulators
break them up or take other action. Also at issue is determining if
existing rules are sufficient given the tech companies' market power and
whether the laws should be updated because the companies behave in ways
that should be illegal.

Here are some questions subcommittee members ought to consider:

\hypertarget{amazon}{%
\subsubsection{Amazon}\label{amazon}}

The subcommittee will probably focus on the company's relationship with
third-party merchants that use the site to sell directly to consumers.
Such merchants represent about 60 percent of Amazon's sales. The company
also operates an enormous shipping network, an advertising sales
business and a cloud computing service that may raise alarms among
regulators. Amazon's trove of sales data gives it incredibly detailed
insights into both customers and merchants.

\begin{itemize}
\tightlist
\item
  After an
  \href{https://www.cnbc.com/2019/07/17/amazon-in-deal-with-german-watchdog-to-overhaul-marketplace-terms.html}{investigation
  by German regulators}, Amazon vowed last year to overhaul its
  contracts with third-party merchants. Did the company adequately do
  so? Does Amazon have contracts that require lower prices than other
  retailers'? Does it require exclusivity, meaning merchants cannot
  offer their goods on other sellers' websites?
\end{itemize}

\begin{itemize}
\tightlist
\item
  An Amazon lawyer told the panel, ``We don't use individual seller data
  directly to compete'' with other businesses on Amazon's site. But a
  Wall Street Journal
  \href{https://www.wsj.com/articles/amazon-scooped-up-data-from-its-own-sellers-to-launch-competing-products-11587650015}{report}
  showed evidence that Amazon does just that, helping it create tailored
  private-label products that undercut competitors. What is the extent
  of Amazon's use of seller data?
\end{itemize}

\begin{itemize}
\tightlist
\item
  Amazon offers its sellers warehousing and shipping services worldwide.
  What does it seek in return, beyond a commission? Does Amazon use
  sales data from small merchants to source new products or to help
  larger sellers succeed, forcing out
  \href{https://www.usatoday.com/story/tech/news/2019/07/16/amazon-faces-antitrust-probe-business-practices-european-union/1742868001/}{smaller
  ones}?
\end{itemize}

\begin{itemize}
\tightlist
\item
  In 2010, Amazon
  \href{http://www.slate.com/blogs/future_tense/2013/10/10/amazon_book_how_jeff_bezos_went_thermonuclear_on_diapers_com.html/}{dropped
  diaper prices} well below profitability, in a successful effort to
  force a competitor, Diapers.com, into acquisition talks. Amazon has
  since shuttered that site. Does Amazon view such actions as
  exclusionary? And is the company engaged in other such pricing wars in
  order to force a competitor to sell?
\end{itemize}

\begin{itemize}
\tightlist
\item
  A Washington Post
  \href{https://www.washingtonpost.com/technology/2019/08/27/aggressive-amazon-tactic-pushes-you-consider-its-own-brand-before-you-click-buy/}{investigation}
  showed that Amazon pushes consumers toward its private-label products
  even when they appear to want to buy name brands. Does Amazon favor
  its own products in consumers' searches? Does it require fees or
  advertising purchases from merchants or brands to ensure their
  products rise to the top of searches?
\end{itemize}

\hypertarget{apple}{%
\subsubsection{Apple}\label{apple}}

While Apple is best known for its iPhones and laptops, it also has
healthy competition from companies like Samsung and Lenovo in hardware
sales. As a result, Mr. Cook is most likely to be asked about the
structure of Apple's App Store, where millions of software developers
offer their apps for download.

\begin{itemize}
\tightlist
\item
  Why does Apple permit only its own app store on iPhones?
\end{itemize}

\begin{itemize}
\tightlist
\item
  Developers are generally required to offer their in-app purchases and
  paid subscriptions through Apple's App Store, rather than on their own
  websites, where they may avoid Apple's commissions. Apple has
  \href{https://www.nytimes.com/2020/06/19/opinion/apple-app-store-hey.html}{threatened
  to remove apps} that don't abide. How is this in the best interest of
  consumers and app developers?
\end{itemize}

\begin{itemize}
\tightlist
\item
  Some app developers have alleged that Apple uses the detailed data it
  collects about app downloads to
  \href{https://appleinsider.com/articles/19/06/06/developers-talk-about-being-sherlocked-as-apple-uses-them-for-market-research}{copy
  their ideas} and that the company
  \href{https://www.nytimes.com/interactive/2019/09/09/technology/apple-app-store-competition.html}{favors
  its own apps} in searches. Is this true? If so, how does the company
  defend such practices?
\end{itemize}

\hypertarget{facebook}{%
\subsubsection{Facebook}\label{facebook}}

Facebook's aggressive acquisition strategy --- including the giants
Instagram and WhatsApp ---~makes it vulnerable to a breakup if
regulators find that it was trying to rid the market of real
competition.

\begin{itemize}
\tightlist
\item
  \href{https://nypost.com/2019/02/26/facebook-boasted-of-buying-instagram-to-kill-the-competition-sources/}{Reportedly},
  the Federal Trade Commission had documents demonstrating Facebook
  acquired Instagram in 2012 in an explicit bid to stifle a competitor.
  Were those documents mischaracterized? How did Facebook's buying
  Instagram benefit consumers, and how did it determine the
  \href{https://dealbook.nytimes.com/2012/04/09/facebook-buys-instagram-for-1-billion/}{\$1
  billion price}?
\end{itemize}

\begin{itemize}
\item
  British lawmakers
  \href{https://www.nytimes.com/2018/12/05/technology/facebook-emails-privacy-data.html}{released
  emails} showing Facebook used an analytics app to collect detailed
  data about competitors in order to snuff them out. That helped
  Facebook decide to buy WhatsApp for \$19 billion, the emails show.
  Couldn't that be called an abuse of market power? Does Facebook still
  cull proprietary data on rivals in order to protect its market
  leadership?
\item
  Advertisers can target customers on Facebook with incredible accuracy,
  in part because of the platform's ability to track users' internet
  browsing activity across the web. Shouldn't users consider those terms
  onerous? Also, has Facebook made assurances about the privacy of
  customer data that it later reneged on? What assurances do consumers
  have that their data will remain private and not be repurposed for
  Facebook's benefit?
\end{itemize}

\begin{itemize}
\tightlist
\item
  \href{https://www.wsj.com/articles/facebook-knows-it-encourages-division-top-executives-nixed-solutions-11590507499}{According
  to The Wall Street Journal}, Facebook quashed efforts to make its site
  less politically divisive because partisan content drives more use of
  the site, which is beneficial to its advertising business. How can
  suppressing opposing views for users be viewed as anything but an
  abuse of power?
\end{itemize}

\includegraphics{https://static01.nyt.com/images/2019/05/09/autossell/op-chris/op-chris-videoSixteenByNineJumbo1600.jpg}

\hypertarget{alphabet}{%
\subsubsection{Alphabet}\label{alphabet}}

Alphabet's signature product, Google, is central to nearly every
activity on the web --- and increasingly, many activities off the web as
well. It is the dominant search engine by far and operates sprawling
advertising and cloud computing businesses. Subcommittee members are
likely to home in on how Google's search business potentially stifles
competition by favoring its own services and how YouTube, a Google
subsidiary, pushes content to users.

\begin{itemize}
\tightlist
\item
  Frequently, answers to common Google search questions can be found in
  so-called answer boxes at the top of a results page. Google culls that
  information from other websites, eliminating the need for users to
  visit those sites. Doesn't that starve other websites of valuable
  traffic? Don't answer boxes simply buttress Google's market leadership
  by stifling competitors?
\end{itemize}

\begin{itemize}
\tightlist
\item
  Google has paid Apple
  \href{https://www.theverge.com/2020/7/1/21310591/apple-google-search-engine-safari-iphone-deal-billions-regulation-antitrust}{billions
  of dollars} to be the default search engine on the Safari web browser
  on iPhones and iPads. How does Google expect rivals to compete on an
  even playing field if it has cornered such a significant share of the
  market?
\end{itemize}

\begin{itemize}
\tightlist
\item
  Google controls multiple levers in the process of placing
  advertisements on the web, including analytics. The company also
  requires some businesses to use its ad technology if they want to use
  Google services. How does Google's ad sales technology benefit
  advertisers? Why doesn't Google allow marketers to see what prices
  others pay to place ads? Doesn't that stifle bargaining power?
\end{itemize}

\begin{itemize}
\tightlist
\item
  Critics say YouTube
  \href{https://www.nytimes.com/2020/07/13/technology/google-ads-antitrust.html}{pushes
  videos} that are politically slanted, based on data about its users,
  meaning viewers aren't likely to see competing viewpoints. Is it
  correct that YouTube's software is designed to reinforce biases? Other
  than keeping users on YouTube longer, what is the purpose of that
  system?
\end{itemize}

\emph{The Times is committed to publishing}
\href{https://www.nytimes.com/2019/01/31/opinion/letters/letters-to-editor-new-york-times-women.html}{\emph{a
diversity of letters}} \emph{to the editor. We'd like to hear what you
think about this or any of our articles. Here are some}
\href{https://help.nytimes.com/hc/en-us/articles/115014925288-How-to-submit-a-letter-to-the-editor}{\emph{tips}}\emph{.
And here's our email:}
\href{mailto:letters@nytimes.com}{\emph{letters@nytimes.com}}\emph{.}

\emph{Follow The New York Times Opinion section on}
\href{https://www.facebook.com/nytopinion}{\emph{Facebook}}\emph{,}
\href{http://twitter.com/NYTOpinion}{\emph{Twitter (@NYTopinion)}}
\emph{and}
\href{https://www.instagram.com/nytopinion/}{\emph{Instagram}}\emph{.}

Advertisement

\protect\hyperlink{after-bottom}{Continue reading the main story}

\hypertarget{site-index}{%
\subsection{Site Index}\label{site-index}}

\hypertarget{site-information-navigation}{%
\subsection{Site Information
Navigation}\label{site-information-navigation}}

\begin{itemize}
\tightlist
\item
  \href{https://help.nytimes.com/hc/en-us/articles/115014792127-Copyright-notice}{©~2020~The
  New York Times Company}
\end{itemize}

\begin{itemize}
\tightlist
\item
  \href{https://www.nytco.com/}{NYTCo}
\item
  \href{https://help.nytimes.com/hc/en-us/articles/115015385887-Contact-Us}{Contact
  Us}
\item
  \href{https://www.nytco.com/careers/}{Work with us}
\item
  \href{https://nytmediakit.com/}{Advertise}
\item
  \href{http://www.tbrandstudio.com/}{T Brand Studio}
\item
  \href{https://www.nytimes.com/privacy/cookie-policy\#how-do-i-manage-trackers}{Your
  Ad Choices}
\item
  \href{https://www.nytimes.com/privacy}{Privacy}
\item
  \href{https://help.nytimes.com/hc/en-us/articles/115014893428-Terms-of-service}{Terms
  of Service}
\item
  \href{https://help.nytimes.com/hc/en-us/articles/115014893968-Terms-of-sale}{Terms
  of Sale}
\item
  \href{https://spiderbites.nytimes.com}{Site Map}
\item
  \href{https://help.nytimes.com/hc/en-us}{Help}
\item
  \href{https://www.nytimes.com/subscription?campaignId=37WXW}{Subscriptions}
\end{itemize}
