Sections

SEARCH

\protect\hyperlink{site-content}{Skip to
content}\protect\hyperlink{site-index}{Skip to site index}

\href{https://www.nytimes.com/section/smarter-living}{Smarter Living}

\href{https://myaccount.nytimes.com/auth/login?response_type=cookie\&client_id=vi}{}

\href{https://www.nytimes.com/section/todayspaper}{Today's Paper}

\href{/section/smarter-living}{Smarter Living}\textbar{}How to Ask if
Everything Is OK When It's Clearly Not

\url{https://nyti.ms/30Sw3HH}

\begin{itemize}
\item
\item
\item
\item
\item
\end{itemize}

\href{https://www.nytimes.com/spotlight/at-home?action=click\&pgtype=Article\&state=default\&region=TOP_BANNER\&context=at_home_menu}{At
Home}

\begin{itemize}
\tightlist
\item
  \href{https://www.nytimes.com/2020/08/03/well/family/the-benefits-of-talking-to-strangers.html?action=click\&pgtype=Article\&state=default\&region=TOP_BANNER\&context=at_home_menu}{Talk:
  To Strangers}
\item
  \href{https://www.nytimes.com/2020/08/01/at-home/coronavirus-make-pizza-on-a-grill.html?action=click\&pgtype=Article\&state=default\&region=TOP_BANNER\&context=at_home_menu}{Make:
  Grilled Pizza}
\item
  \href{https://www.nytimes.com/2020/07/31/arts/television/goldbergs-abc-stream.html?action=click\&pgtype=Article\&state=default\&region=TOP_BANNER\&context=at_home_menu}{Watch:
  'The Goldbergs'}
\item
  \href{https://www.nytimes.com/interactive/2020/at-home/even-more-reporters-editors-diaries-lists-recommendations.html?action=click\&pgtype=Article\&state=default\&region=TOP_BANNER\&context=at_home_menu}{Explore:
  Reporters' Google Docs}
\end{itemize}

Advertisement

\protect\hyperlink{after-top}{Continue reading the main story}

Supported by

\protect\hyperlink{after-sponsor}{Continue reading the main story}

July 28, 2020

\hypertarget{how-to-ask-if-everything-is-ok-when-its-clearly-not}{%
\section{How to Ask if Everything Is OK When It's Clearly
Not}\label{how-to-ask-if-everything-is-ok-when-its-clearly-not}}

\includegraphics{https://static01.nyt.com/images/2020/07/27/multimedia/24sl-virus-ok/24sl-virus-ok-articleLarge.png?quality=75\&auto=webp\&disable=upscale}

By \href{https://www.nytimes.com/by/anna-goldfarb}{Anna Goldfarb}

In a perfect world, when you're checking in with someone who's
struggling, you'd have your conversation together in a calm, private
setting. Phones and devices would be silenced and stashed out of sight.
Food and drinks tend to put people at ease, so you'd nosh on snacks or
sip a beverage together, too.

But this, of course, isn't a perfect world, and we're still in the
throes of a pandemic, so this idyllic social scenario may not be
possible anytime soon. So it's even more important you choose the right
moment to check in, as it will determine the quality of the interaction
you have.

While we may not be able to be physically present when we approach a
troubled friend, we can create an atmosphere --- and cultivate the right
mind-set within ourselves --- so the other person will feel comfortable
opening up when they need support most.

\hypertarget{look-for-signs-of-distress}{%
\subsection{Look for signs of
distress}\label{look-for-signs-of-distress}}

When you chat with a friend,
\href{https://psy.fsu.edu/faculty/joinert/joiner.dp.php}{Thomas Joiner},
a psychology professor at Florida State University, said you should be
on the lookout for noticeable changes in their demeanor, such as an
irritable mood or a disheveled appearance. If your friend has recently
experienced relationship issues, health problems or workplace stress, or
has faced financial difficulties, they may be especially vulnerable to
anguish right now.

\hypertarget{be-mindful-of-any-power-dynamics}{%
\subsection{Be mindful of any power
dynamics}\label{be-mindful-of-any-power-dynamics}}

Depending on your relationship, you might want to tread carefully.
Personal friends, work colleagues, classmates and family members all
require different approaches, said
\href{https://phoenixjacksontherapy.com/about}{Phoenix Jackson}, a
licensed marriage and family therapist. She recommends carefully
considering the power dynamics before you approach, as it's easier to be
vulnerable with someone if you're on equal footing.

In some cases, even asking if someone is OK, ``depending on how, where
and when it's posed, could be seen as an affront or even something where
a case is being built to dismiss that person,'' she said. She recommends
reassuring the other person that you're asking from a place of real
concern. If the person doesn't want to engage, say you respect their
decision. Assure them you'll drop the issue.

\hypertarget{check-in-with-yourself-first}{%
\subsection{Check in with yourself
first}\label{check-in-with-yourself-first}}

``When you check in with others, you are opening some vulnerability
there and that takes some insight,'' said Dr.
\href{https://www.semel.ucla.edu/autism/team/jena-lee-md}{Jena Lee}, a
child and adult psychiatrist and clinical instructor at the David Geffen
School of Medicine at U.C.L.A. So it's important to make sure you're in
a healthy place to be present and engage with someone who's struggling.

\hypertarget{when-youre-ready-to-have-a-conversation-pinpoint-why-youre-concerned}{%
\subsection{When you're ready to have a conversation, pinpoint why
you're
concerned}\label{when-youre-ready-to-have-a-conversation-pinpoint-why-youre-concerned}}

Be explicit: ``I notice you've been slower to respond to my text
messages.'' Or, ``I see you've been sleeping a lot more than usual. Is
there anything you want to talk about?''

By indicating you've noticed a change in their behavior, ``you give them
the opportunity to either confirm what you're saying or deny it,'' said
\href{https://www.drukuku.com/}{Uche Ukuku}, a psychologist. You're not
telling the other person how they feel, but you're initiating a
conversation and giving them a chance to address the change, she said.

\hypertarget{offer-confidentiality}{%
\subsection{Offer confidentiality}\label{offer-confidentiality}}

If you have the kind of relationship where you can honor
confidentiality, Ms. Jackson suggests offering it. Your promise might
help them feel more secure confiding in you. If there's potential for
embarrassment or shame, she suggests letting the person know you
understand if they're not ready to have a conversation. Just reiterate
that you care about them, which is why you're asking.

\hypertarget{ask-open-ended-nonjudgmental-questions}{%
\subsection{Ask open-ended, nonjudgmental
questions}\label{ask-open-ended-nonjudgmental-questions}}

When asking someone if they're OK, the other person may reflexively
reply they're fine, which shuts the conversation down. Dr. Ukuku
suggests keeping your questions open-ended:

\begin{quote}
``How are things?''

``Is anything on your mind?''

``What's the most difficult thing you've experienced lately?''
\end{quote}

If you are more familiar with this person, Dr. Lee suggests asking
specific questions to show you care:

\begin{quote}
``How did your meeting go?''

``How are your kids adjusting to so many changes at school?''
\end{quote}

This way, she said, your questions come out naturally. ``What you're
trying to do is actually show that you want to know what their life is
like and how they're actually experiencing their circumstances,'' she
said.

\hypertarget{reveal-a-bit-about-your-own-struggles}{%
\subsection{Reveal a bit about your own
struggles}\label{reveal-a-bit-about-your-own-struggles}}

Dr. Lee also recommends sharing a little bit about yourself to get the
conversation rolling. Saying something like: ``I've been so stressed.
How have things been for you?'' Or ``I'm sick of cooking meals. How have
you been handling staying home?'' Opening the conversation this way, she
said, gives the other person permission to air their own grievances and
worries.

\hypertarget{or-you-dont-have-to-pose-a-question-at-all}{%
\subsection{Or, you don't have to pose a question at
all}\label{or-you-dont-have-to-pose-a-question-at-all}}

Ms. Jackson suggests sending a letter or postcard to someone as a way to
let them know you're thinking about them. You could write: ``I'm
wondering how you are.'' The phrasing leaves a lot of room for people to
choose whether to engage, Ms. Jackson said.

\hypertarget{dont-be-preoccupied-with-what-to-say-in-response}{%
\subsection{Don't be preoccupied with what to say in
response}\label{dont-be-preoccupied-with-what-to-say-in-response}}

``When you're in the conversation and someone is sharing with you a
horrible situation that they're going through, the first thing that most
people think is, `What do I say? How can I help them?''' Dr. Lee said.
It's an understandable reaction, ``but thinking about those things
distracts your mind and you actually aren't able to be empathetic,'' she
said.

She suggests putting yourself in their shoes. Even if you sit in
silence, your facial expressions and body language will convey your
empathetic reaction, Dr. Lee said. Validate your friend. Say that yes,
their situation \emph{is} painful. ``The most helpful thing that we can
do for each other is just share that you're actually burdened
together,'' she said.

\hypertarget{dont-set-out-to-solve-your-friends-problem}{%
\subsection{Don't set out to solve your friend's
problem}\label{dont-set-out-to-solve-your-friends-problem}}

For complex problems with no easy solutions, you shouldn't expect that
you can resolve these issues on your own. If your friend is experiencing
distress, Dr. Joiner suggests telling them to reach out to their primary
care physician or family doctor for added support. If your friend is
religious, encourage them to reach out to a clergy person as ``they're
often really helpful with things like this,'' he said.

\hypertarget{make-a-date-to-follow-up}{%
\subsection{Make a date to follow up}\label{make-a-date-to-follow-up}}

Coming up with a follow-up plan --- a phone call in a few days, a
socially distanced picnic, a Zoom call --- not only gives the other
person something to look forward to, but it also sends the message that
this checkup isn't going to be a one-time thing. It also takes the
pressure off the other person from feeling as if they have to provide
daily updates and gives you both space to process your conversation, Dr.
Ukuku said.

``We don't realize how much being seen can really change somebody's
mood,'' she said. ``The idea that you checked in on them is telling
them, `Hey, not only am I seen, but also that I'm known and I'm
loved.'''

There will be times when your friends aren't able to communicate what
they need from you. The goal, Dr. Ukuku said, is to plant a seed so that
when they do need support, they'll know you're somebody they can reach
out to.

Advertisement

\protect\hyperlink{after-bottom}{Continue reading the main story}

\hypertarget{site-index}{%
\subsection{Site Index}\label{site-index}}

\hypertarget{site-information-navigation}{%
\subsection{Site Information
Navigation}\label{site-information-navigation}}

\begin{itemize}
\tightlist
\item
  \href{https://help.nytimes.com/hc/en-us/articles/115014792127-Copyright-notice}{©~2020~The
  New York Times Company}
\end{itemize}

\begin{itemize}
\tightlist
\item
  \href{https://www.nytco.com/}{NYTCo}
\item
  \href{https://help.nytimes.com/hc/en-us/articles/115015385887-Contact-Us}{Contact
  Us}
\item
  \href{https://www.nytco.com/careers/}{Work with us}
\item
  \href{https://nytmediakit.com/}{Advertise}
\item
  \href{http://www.tbrandstudio.com/}{T Brand Studio}
\item
  \href{https://www.nytimes.com/privacy/cookie-policy\#how-do-i-manage-trackers}{Your
  Ad Choices}
\item
  \href{https://www.nytimes.com/privacy}{Privacy}
\item
  \href{https://help.nytimes.com/hc/en-us/articles/115014893428-Terms-of-service}{Terms
  of Service}
\item
  \href{https://help.nytimes.com/hc/en-us/articles/115014893968-Terms-of-sale}{Terms
  of Sale}
\item
  \href{https://spiderbites.nytimes.com}{Site Map}
\item
  \href{https://help.nytimes.com/hc/en-us}{Help}
\item
  \href{https://www.nytimes.com/subscription?campaignId=37WXW}{Subscriptions}
\end{itemize}
