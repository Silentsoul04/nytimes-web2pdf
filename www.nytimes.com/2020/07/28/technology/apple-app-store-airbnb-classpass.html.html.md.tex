Sections

SEARCH

\protect\hyperlink{site-content}{Skip to
content}\protect\hyperlink{site-index}{Skip to site index}

\href{https://www.nytimes.com/section/technology}{Technology}

\href{https://myaccount.nytimes.com/auth/login?response_type=cookie\&client_id=vi}{}

\href{https://www.nytimes.com/section/todayspaper}{Today's Paper}

\href{/section/technology}{Technology}\textbar{}Their Businesses Went
Virtual. Then Apple Wanted a Cut.

\url{https://nyti.ms/2CMhnC5}

\begin{itemize}
\item
\item
\item
\item
\item
\item
\end{itemize}

\href{https://www.nytimes.com/news-event/coronavirus?action=click\&pgtype=Article\&state=default\&region=TOP_BANNER\&context=storylines_menu}{The
Coronavirus Outbreak}

\begin{itemize}
\tightlist
\item
  live\href{https://www.nytimes.com/2020/08/01/world/coronavirus-covid-19.html?action=click\&pgtype=Article\&state=default\&region=TOP_BANNER\&context=storylines_menu}{Latest
  Updates}
\item
  \href{https://www.nytimes.com/interactive/2020/us/coronavirus-us-cases.html?action=click\&pgtype=Article\&state=default\&region=TOP_BANNER\&context=storylines_menu}{Maps
  and Cases}
\item
  \href{https://www.nytimes.com/interactive/2020/science/coronavirus-vaccine-tracker.html?action=click\&pgtype=Article\&state=default\&region=TOP_BANNER\&context=storylines_menu}{Vaccine
  Tracker}
\item
  \href{https://www.nytimes.com/interactive/2020/07/29/us/schools-reopening-coronavirus.html?action=click\&pgtype=Article\&state=default\&region=TOP_BANNER\&context=storylines_menu}{What
  School May Look Like}
\item
  \href{https://www.nytimes.com/live/2020/07/31/business/stock-market-today-coronavirus?action=click\&pgtype=Article\&state=default\&region=TOP_BANNER\&context=storylines_menu}{Economy}
\end{itemize}

Advertisement

\protect\hyperlink{after-top}{Continue reading the main story}

Supported by

\protect\hyperlink{after-sponsor}{Continue reading the main story}

\hypertarget{their-businesses-went-virtual-then-apple-wanted-a-cut}{%
\section{Their Businesses Went Virtual. Then Apple Wanted a
Cut.}\label{their-businesses-went-virtual-then-apple-wanted-a-cut}}

After Airbnb and ClassPass began selling virtual classes because of the
pandemic, Apple tried to collect its commission on the sales.

\includegraphics{https://static01.nyt.com/images/2020/07/28/business/28virus-apple1/merlin_175008315_b71321c6-4491-4190-a975-7fa298305db0-articleLarge.jpg?quality=75\&auto=webp\&disable=upscale}

By \href{https://www.nytimes.com/by/jack-nicas}{Jack Nicas} and
\href{https://www.nytimes.com/by/david-mccabe}{David McCabe}

\begin{itemize}
\item
  July 28, 2020
\item
  \begin{itemize}
  \item
  \item
  \item
  \item
  \item
  \item
  \end{itemize}
\end{itemize}

ClassPass built its business on helping people book exercise classes at
local gyms. So when the pandemic forced gyms across the United States to
close, the company shifted to virtual classes.

Then ClassPass received a concerning message from Apple. Because the
classes it sold on its iPhone app were now virtual, Apple said it was
entitled to 30 percent of the sales, up from no fee previously,
according to a person close to ClassPass who spoke on the condition of
anonymity for fear of upsetting Apple. The iPhone maker said it was
merely enforcing a decade-old rule.

Airbnb experienced similar demands from Apple after it began an ``online
experiences'' business that offered virtual cooking classes, meditation
sessions and drag-queen shows, augmenting the in-person experiences it
started selling in 2016, according to two people familiar with the
issues.

Airbnb **** discussed Apple's demands with House lawmakers' offices that
are investigating how Apple controls its App Store, according to three
people who spoke on the condition of anonymity to discuss private
conversations. Those lawmakers are now considering Apple's efforts to
collect a commission from Airbnb and ClassPass **** as part of their
yearlong antitrust inquiry into the biggest tech companies, ****
according to a person with **** knowledge **** of their investigation.

Those lawmakers are set to grill Tim Cook, Apple's chief executive, and
the chief executives of Amazon, Facebook and Google in a high-profile
hearing on Wednesday.

Apple's disputes with the smaller companies point to the control the
world's largest tech companies have had over the shift to online life
brought on by the pandemic. While much of the rest of the economy is
struggling, the pandemic has further entrenched their businesses.

With millions more employees working from home, Amazon and Google are
selling more online cloud space, with revenue for Amazon Web Services
and Google Cloud soaring in the first quarter of the year, which
included the start of the pandemic. Facebook and YouTube, which is part
of Google, some of the internet's largest gathering places,
\href{https://www.nytimes.com/interactive/2020/04/07/technology/coronavirus-internet-use.html}{had
traffic surge} as people couldn't socialize in person.

Apple
\href{https://www.nytimes.com/2020/04/30/technology/apple-sales-earnings-coronavirus.html}{has
also brought in more revenue} from its online-services business, mostly
on the back of its App Store, and its Macs, iPads and iPhones have
become even more important tools.

\includegraphics{https://static01.nyt.com/images/2020/07/28/business/28virus-apple4/merlin_175008231_7e074a93-9df2-46f3-82be-b01ef4ae5fdc-articleLarge.jpg?quality=75\&auto=webp\&disable=upscale}

With gyms shut down, ClassPass dropped its typical commission on virtual
classes, passing along 100 percent of sales to gyms, the person close to
the company said. That meant Apple would have taken its cut from
hundreds of struggling independent fitness centers, yoga studios and
boxing gyms.

Apple said that with Airbnb and ClassPass, it was not trying to generate
revenue --- though that is a side effect --- but instead was trying to
enforce a rule that has been in place since it first published its app
guidelines in 2010.

\hypertarget{latest-updates-economy}{%
\section{\texorpdfstring{\href{https://www.nytimes.com/live/2020/07/31/business/stock-market-today-coronavirus?action=click\&pgtype=Article\&state=default\&region=MAIN_CONTENT_1\&context=storylines_live_updates}{Latest
Updates:
Economy}}{Latest Updates: Economy}}\label{latest-updates-economy}}

\href{https://www.nytimes.com/live/2020/07/31/business/stock-market-today-coronavirus?action=click\&pgtype=Article\&state=default\&region=MAIN_CONTENT_1\&context=storylines_live_updates\#kodaks-chief-executive-was-given-stock-options-then-the-share-price-spiked-1000-percent}{33h
ago}

\href{https://www.nytimes.com/live/2020/07/31/business/stock-market-today-coronavirus?action=click\&pgtype=Article\&state=default\&region=MAIN_CONTENT_1\&context=storylines_live_updates\#kodaks-chief-executive-was-given-stock-options-then-the-share-price-spiked-1000-percent}{Kodak's
chief executive was given stock options. Then the share price spiked
1,000 percent.}

\href{https://www.nytimes.com/live/2020/07/31/business/stock-market-today-coronavirus?action=click\&pgtype=Article\&state=default\&region=MAIN_CONTENT_1\&context=storylines_live_updates\#fitch-ratings-downgrades-its-outlook-on-us-debt}{36h
ago}

\href{https://www.nytimes.com/live/2020/07/31/business/stock-market-today-coronavirus?action=click\&pgtype=Article\&state=default\&region=MAIN_CONTENT_1\&context=storylines_live_updates\#fitch-ratings-downgrades-its-outlook-on-us-debt}{Fitch
Ratings downgrades its outlook on U.S. debt.}

\href{https://www.nytimes.com/live/2020/07/31/business/stock-market-today-coronavirus?action=click\&pgtype=Article\&state=default\&region=MAIN_CONTENT_1\&context=storylines_live_updates\#us-sanctions-more-chinese-officials-over-human-rights-violations-as-tensions-flare}{43h
ago}

\href{https://www.nytimes.com/live/2020/07/31/business/stock-market-today-coronavirus?action=click\&pgtype=Article\&state=default\&region=MAIN_CONTENT_1\&context=storylines_live_updates\#us-sanctions-more-chinese-officials-over-human-rights-violations-as-tensions-flare}{U.S.
sanctions more Chinese officials over human rights violations as
tensions flare}

\href{https://www.nytimes.com/live/2020/07/31/business/stock-market-today-coronavirus?action=click\&pgtype=Article\&state=default\&region=MAIN_CONTENT_1\&context=storylines_live_updates}{See
more updates}

More live coverage:
\href{https://www.nytimes.com/2020/08/01/world/coronavirus-covid-19.html?action=click\&pgtype=Article\&state=default\&region=MAIN_CONTENT_1\&context=storylines_live_updates}{Global}

Apple said waiving the commission in these cases would not be fair to
the many other app developers that have paid the fee for similar
businesses for years. Because of the pandemic, Apple said that it gave
ClassPass until the end of the year to comply and that it was continuing
to negotiate with Airbnb.

``To ensure every developer can create and grow a successful business,
Apple maintains a clear, consistent set of guidelines that apply equally
to everyone,'' the company said in a statement.

ClassPass was told it must comply with the rule this month, according to
the person close to the company. Instead, it stopped offering virtual
classes in its iPhone app, since those classes were subject to Apple's
commission, according to Apple. As a result, fewer potential customers
now see the classes advertised by its gym partners.

In 2016, Airbnb started a business offering in-person ``experiences'' to
travelers, such as guided tours, bar crawls and cooking classes with
locals in their vacation destinations. In early April, as the pandemic
gutted travel plans and the company's bottom line, Airbnb began selling
virtual versions of similar experiences, though it quickly expanded that
business to more prominent offerings, like cooking classes with famous
chefs and training sessions with Olympic athletes.

Later that month, Apple reached out to say that when the online
experiences were sold in Airbnb's iPhone app, the company would have to
pay Apple's fees, said a person familiar with their exchanges.

Apple said it believed that Airbnb had long intended to offer virtual
experiences --- not that the business was created simply because of the
pandemic --- and that it would continue to do so once the world has
resumed to normal. Apple also pointed out that Airbnb had never paid
Apple any money despite the fact that it built its multibillion-dollar
business with the help of its iPhone app.

Airbnb is still negotiating with Apple. In June, Brian Chesky, Airbnb's
chief executive, said that the online experiences offering was the
company's ``fastest growing product ever'' and had earned \$1 million in
revenue. Apple said that if the two companies could not come to terms,
it could remove Airbnb's app from the App Store.

Image

Tim Cook at an Apple Store event in Manhattan last year. Mr. Cook is set
to testify at an antitrust hearing on Wednesday.Credit...James
Estrin/The New York Times

Many companies and app developers complain that Apple forces them to pay
its commission to be included in the App Store, which is crucial to
reaching the roughly 900 million people with iPhones. Apple said the App
Store had 500 million visitors from 175 countries each week.

For months, economists and lawyers at the Justice Department have held
meetings with companies and app developers about the App Store as part
of its antitrust investigation into Apple. The music service Spotify and
another large company that declined to be named also said they have had
recent conversations with attorneys general from several states about
the issue.

Unlike Spotify, Airbnb and ClassPass do not offer services that directly
compete with one of Apple's digital products.

Many companies complain that they are also subject to what they call
Apple's capricious enforcement of its rules, which can lead to their
apps' removal from the App Store, killing some of their business. If
Apple removes an app from the App Store, the developer couldn't gain new
app users and couldn't update the apps already on people's phones,
eventually rendering them broken.

Apple said a small fraction of iPhone apps were subject to its
commission, which is in line with the fees other platforms charge,
according to a study released by Apple last Wednesday. Airbnb, for
instance, charges a 20 percent commission on experiences.

``If you're not in the App Store today, you're not online. Your business
cannot function. So they're the gatekeepers of something that every
single company wants,'' said Andy Yen, the chief executive of
ProtonMail, an encrypted email service based in Switzerland that
effectively competes with Apple's own email service. ``If you want to
pass through their gates, they're going to charge you 30 percent of your
revenue.''

Mr. Yen said his company had been battling with Apple since 2017 over
its commission, with Apple sometimes restricting the ProtonMail app on
iPhones. To account for Apple's fee, ProtonMail began charging 30
percent more for subscriptions bought on its iPhone app versus those
bought on its website, which aren't subject to Apple's fee. ``The only
way that we could support this fee was actually by passing on the cost
to the customer,'' he said.

But when ProtonMail told iPhone users about the lower price on its
website, Apple restricted its app. Then, when the company instead tried
to make clear that 30 percent of the subscription price went to Apple,
Apple restricted its app again. ``You only hide something like this if
it's wrong,'' Mr. Yen said.

Asked about ProtonMail's experience, Apple said its rules require
certain apps to use its payment system and ban them from directing
people to buy their products or services elsewhere.

Advertisement

\protect\hyperlink{after-bottom}{Continue reading the main story}

\hypertarget{site-index}{%
\subsection{Site Index}\label{site-index}}

\hypertarget{site-information-navigation}{%
\subsection{Site Information
Navigation}\label{site-information-navigation}}

\begin{itemize}
\tightlist
\item
  \href{https://help.nytimes.com/hc/en-us/articles/115014792127-Copyright-notice}{©~2020~The
  New York Times Company}
\end{itemize}

\begin{itemize}
\tightlist
\item
  \href{https://www.nytco.com/}{NYTCo}
\item
  \href{https://help.nytimes.com/hc/en-us/articles/115015385887-Contact-Us}{Contact
  Us}
\item
  \href{https://www.nytco.com/careers/}{Work with us}
\item
  \href{https://nytmediakit.com/}{Advertise}
\item
  \href{http://www.tbrandstudio.com/}{T Brand Studio}
\item
  \href{https://www.nytimes.com/privacy/cookie-policy\#how-do-i-manage-trackers}{Your
  Ad Choices}
\item
  \href{https://www.nytimes.com/privacy}{Privacy}
\item
  \href{https://help.nytimes.com/hc/en-us/articles/115014893428-Terms-of-service}{Terms
  of Service}
\item
  \href{https://help.nytimes.com/hc/en-us/articles/115014893968-Terms-of-sale}{Terms
  of Sale}
\item
  \href{https://spiderbites.nytimes.com}{Site Map}
\item
  \href{https://help.nytimes.com/hc/en-us}{Help}
\item
  \href{https://www.nytimes.com/subscription?campaignId=37WXW}{Subscriptions}
\end{itemize}
