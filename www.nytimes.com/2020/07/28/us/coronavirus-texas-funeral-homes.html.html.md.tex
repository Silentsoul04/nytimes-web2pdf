Sections

SEARCH

\protect\hyperlink{site-content}{Skip to
content}\protect\hyperlink{site-index}{Skip to site index}

\href{/section/us}{U.S.}\textbar{}`Not Sparing Anyone': Texas Funeral
Homes Can't Escape Virus

\url{https://nyti.ms/39AO5SL}

\begin{itemize}
\item
\item
\item
\item
\item
\end{itemize}

\href{https://www.nytimes.com/news-event/coronavirus?action=click\&pgtype=Article\&state=default\&region=TOP_BANNER\&context=storylines_menu}{The
Coronavirus Outbreak}

\begin{itemize}
\tightlist
\item
  live\href{https://www.nytimes.com/2020/08/01/world/coronavirus-covid-19.html?action=click\&pgtype=Article\&state=default\&region=TOP_BANNER\&context=storylines_menu}{Latest
  Updates}
\item
  \href{https://www.nytimes.com/interactive/2020/us/coronavirus-us-cases.html?action=click\&pgtype=Article\&state=default\&region=TOP_BANNER\&context=storylines_menu}{Maps
  and Cases}
\item
  \href{https://www.nytimes.com/interactive/2020/science/coronavirus-vaccine-tracker.html?action=click\&pgtype=Article\&state=default\&region=TOP_BANNER\&context=storylines_menu}{Vaccine
  Tracker}
\item
  \href{https://www.nytimes.com/interactive/2020/07/29/us/schools-reopening-coronavirus.html?action=click\&pgtype=Article\&state=default\&region=TOP_BANNER\&context=storylines_menu}{What
  School May Look Like}
\item
  \href{https://www.nytimes.com/live/2020/07/31/business/stock-market-today-coronavirus?action=click\&pgtype=Article\&state=default\&region=TOP_BANNER\&context=storylines_menu}{Economy}
\end{itemize}

\includegraphics{https://static01.nyt.com/images/2020/07/24/us/00VIRUS-FUNERALHOMES-tafollabefore/merlin_174909369_6d85dc75-fd67-4930-965f-5e86c92758ca-articleLarge.jpg?quality=75\&auto=webp\&disable=upscale}

\hypertarget{not-sparing-anyone-texas-funeral-homes-cant-escape-virus}{%
\section{`Not Sparing Anyone': Texas Funeral Homes Can't Escape
Virus}\label{not-sparing-anyone-texas-funeral-homes-cant-escape-virus}}

The death toll in the Rio Grande Valley is forcing funeral directors to
buy refrigerated trucks and bypass traditional services such as
velorios.

Credit...Tamir Kalifa for The New York Times

Supported by

\protect\hyperlink{after-sponsor}{Continue reading the main story}

By \href{https://www.nytimes.com/by/edgar-sandoval}{Edgar Sandoval}

\begin{itemize}
\item
  Published July 28, 2020Updated July 30, 2020
\item
  \begin{itemize}
  \item
  \item
  \item
  \item
  \item
  \end{itemize}
\end{itemize}

ELSA, Texas --- Johnny Salinas Jr., the owner of Salinas Funeral Home,
typically handles five funerals a week. But on a recent day, with the
coronavirus tearing through his community, he saw that many grieving
families in a single day.

A sixth family was waiting, too. His own.

Mr. Salinas changed from a polo shirt into a crisp black suit and left
his office for the chapel next door. The light blue coffin of his
great-uncle, who died of Covid-19, sat at the front of the room, adorned
with white flower arrangements and a wooden crucifix.

``The virus is not sparing anyone,'' Mr. Salinas said. ``Not even my
family.''

In the Rio Grande Valley of Texas, where a surge of virus cases has set
off a flood of deaths this month, funeral homes --- like hospitals ---
are overloaded and struggling to carry out basic services and keep up
with the expanding crisis. Local funeral homes, officials said, have not
experienced such demand in decades.

About one in 60 residents of Hidalgo County is known to have had the
virus, and about one of every 2,000 people has died from the virus, a
New York Times database shows. Hidalgo County now has one of the highest
per capita death rates in the state.

At the start of July, fewer than 50 deaths in Hidalgo County had been
attributed to the virus, according to the database. By Monday, there had
been almost 470.

``It's like a bad dream,'' said Linda Ceballos, a co-director of
Ceballos Funeral Home in McAllen. ``You want to wake up, but you
can't.''

The death toll is forcing funeral directors to bypass traditional
services such as velorios, viewings that sometimes last for days and are
filled with prayers, hugs and sorrowful Spanish-language songs. Instead,
many funeral homes now are shortening viewing times and limiting
attendance. Some have ordered large refrigerator trucks to store bodies
until they can get to them.

The virus's spread seemed relatively under control in the area until the
state reopened the economy in time for Memorial Day, local health
officials said. Richard Cortez, the Hidalgo County judge, said the virus
soon was wreaking damage through the region, where chronic disease and
widespread poverty were already significant problems. More than 14,000
people have contracted the virus in El Valle, as the area is known to
its mostly Latino residents.

Mr. Cortez has been a constant presence on Spanish and English TV and
radio, urging people to wear masks, wash their hands and --- most
important, he says --- keep distant from older and vulnerable relatives.

\hypertarget{latest-updates-global-coronavirus-outbreak}{%
\section{\texorpdfstring{\href{https://www.nytimes.com/2020/08/01/world/coronavirus-covid-19.html?action=click\&pgtype=Article\&state=default\&region=MAIN_CONTENT_1\&context=storylines_live_updates}{Latest
Updates: Global Coronavirus
Outbreak}}{Latest Updates: Global Coronavirus Outbreak}}\label{latest-updates-global-coronavirus-outbreak}}

Updated 2020-08-02T10:04:29.623Z

\begin{itemize}
\tightlist
\item
  \href{https://www.nytimes.com/2020/08/01/world/coronavirus-covid-19.html?action=click\&pgtype=Article\&state=default\&region=MAIN_CONTENT_1\&context=storylines_live_updates\#link-34047410}{The
  U.S. reels as July cases more than double the total of any other
  month.}
\item
  \href{https://www.nytimes.com/2020/08/01/world/coronavirus-covid-19.html?action=click\&pgtype=Article\&state=default\&region=MAIN_CONTENT_1\&context=storylines_live_updates\#link-780ec966}{Top
  U.S. officials work to break an impasse over the federal jobless
  benefit.}
\item
  \href{https://www.nytimes.com/2020/08/01/world/coronavirus-covid-19.html?action=click\&pgtype=Article\&state=default\&region=MAIN_CONTENT_1\&context=storylines_live_updates\#link-2bc8948}{Its
  outbreak untamed, Melbourne goes into even greater lockdown.}
\end{itemize}

\href{https://www.nytimes.com/2020/08/01/world/coronavirus-covid-19.html?action=click\&pgtype=Article\&state=default\&region=MAIN_CONTENT_1\&context=storylines_live_updates}{See
more updates}

More live coverage:
\href{https://www.nytimes.com/live/2020/07/31/business/stock-market-today-coronavirus?action=click\&pgtype=Article\&state=default\&region=MAIN_CONTENT_1\&context=storylines_live_updates}{Markets}

It is a tall request in an area where
\href{https://www.nytimes.com/2020/07/14/us/coronavirus-texas-rio-grande-valley-border.html}{family
gatherings} and pachangas, or backyard barbecues, are hallmarks of
social life.

In desperation, Mr. Cortez last week instituted a voluntary stay-at-home
order, hoping that it would send a message.

``If 10 percent follow it, we are making progress. We need to protect
our grandparents, our aunts, our uncles,'' Mr. Cortez said in an
interview. ``Too many people are dying, too many people in misery.''

\includegraphics{https://static01.nyt.com/images/2020/07/24/us/00VIRUS-FUNERALHOMES-salinas-tafolla/merlin_174916491_172e8214-e521-4132-8a62-c4bc5ddf8b3c-articleLarge.jpg?quality=75\&auto=webp\&disable=upscale}

Image

Cesar Guerra, left, and Leonel Caballero, right, waited to perform at a
visitation for Mr. Tafolla on Tuesday.Credit...Tamir Kalifa for The New
York Times

Image

A cowboy hat belonging to Mr. Tafolla was displayed during a
visitation.Credit...Tamir Kalifa for The New York Times

Image

A box of tissues waited on a pew at Salinas Funeral Home.Credit...Tamir
Kalifa for The New York Times

At his great-uncle's velorio, Mr. Salinas, 30, had two roles at once:
funeral director and mourning relative. Before family arrived, Mr.
Salinas paced around the room, making sure it complied with hygiene
guidelines.

Ever since Mr. Salinas was a teenager, he knew what work he wanted to
do.

``Death is a journey,'' he said. ``God is the destination.''

When Mr. Salinas was 16, he had his own brush with tragedy. He said he
was driving his 15-year-old sister and another teenage friend when a dog
suddenly appeared in front of the car. In the chaos, the car flipped,
Mr. Salinas said. His sister was killed.

``Now she's always here with me, not physically, but spiritually,'' said
Mr. Salinas, who keeps an oversize photo of his sister, Deborah Lynn
Salinas, bearing the date of her death --- Dec. 6, 2006 --- in the
funeral home.

On this day, around the chapel, every other pew was sealed off with blue
tape so people would sit apart. A plexiglass barrier shielded his
great-uncle's upper body in the open coffin to keep mourners from
leaning in.

``People tend to want to hug and cry on their loved ones,'' Mr. Salinas
said.

Not long after, he talked a family member out of placing a rosary in the
hands of Francisco Tafolla Sr., the great-uncle whom Mr. Salinas grew up
simply calling uncle, who died of the virus at 85.

``It's human instinct to want to touch the body, but they can't,'' he
said. ``It's for their safety.''

White flower arrangements adorned a wall beside a large photo of Mr.
Tafolla. Mr. Tafolla's daughter, Gloria Tafolla Gomez, stood silently,
nodding her head softly to the lyrics of ``Un Dia a la Vez'' --- ``One
Day at a Time'' --- by the Tejano band Siggno.

``He always delivered what he promised, even if that was a spanking,''
Ms. Tafolla whispered to a relative. They both chuckled. ``We were so
afraid of him.''

Mr. Tafolla, who worked manual jobs in the gas industry, saved money to
ensure that all of his 12 children went to college, Ms. Tafolla said.
She became a nurse.

\href{https://www.nytimes.com/news-event/coronavirus?action=click\&pgtype=Article\&state=default\&region=MAIN_CONTENT_3\&context=storylines_faq}{}

\hypertarget{the-coronavirus-outbreak-}{%
\subsubsection{The Coronavirus Outbreak
›}\label{the-coronavirus-outbreak-}}

\hypertarget{frequently-asked-questions}{%
\paragraph{Frequently Asked
Questions}\label{frequently-asked-questions}}

Updated July 27, 2020

\begin{itemize}
\item ~
  \hypertarget{should-i-refinance-my-mortgage}{%
  \paragraph{Should I refinance my
  mortgage?}\label{should-i-refinance-my-mortgage}}

  \begin{itemize}
  \tightlist
  \item
    \href{https://www.nytimes.com/article/coronavirus-money-unemployment.html?action=click\&pgtype=Article\&state=default\&region=MAIN_CONTENT_3\&context=storylines_faq}{It
    could be a good idea,} because mortgage rates have
    \href{https://www.nytimes.com/2020/07/16/business/mortgage-rates-below-3-percent.html?action=click\&pgtype=Article\&state=default\&region=MAIN_CONTENT_3\&context=storylines_faq}{never
    been lower.} Refinancing requests have pushed mortgage applications
    to some of the highest levels since 2008, so be prepared to get in
    line. But defaults are also up, so if you're thinking about buying a
    home, be aware that some lenders have tightened their standards.
  \end{itemize}
\item ~
  \hypertarget{what-is-school-going-to-look-like-in-september}{%
  \paragraph{What is school going to look like in
  September?}\label{what-is-school-going-to-look-like-in-september}}

  \begin{itemize}
  \tightlist
  \item
    It is unlikely that many schools will return to a normal schedule
    this fall, requiring the grind of
    \href{https://www.nytimes.com/2020/06/05/us/coronavirus-education-lost-learning.html?action=click\&pgtype=Article\&state=default\&region=MAIN_CONTENT_3\&context=storylines_faq}{online
    learning},
    \href{https://www.nytimes.com/2020/05/29/us/coronavirus-child-care-centers.html?action=click\&pgtype=Article\&state=default\&region=MAIN_CONTENT_3\&context=storylines_faq}{makeshift
    child care} and
    \href{https://www.nytimes.com/2020/06/03/business/economy/coronavirus-working-women.html?action=click\&pgtype=Article\&state=default\&region=MAIN_CONTENT_3\&context=storylines_faq}{stunted
    workdays} to continue. California's two largest public school
    districts --- Los Angeles and San Diego --- said on July 13, that
    \href{https://www.nytimes.com/2020/07/13/us/lausd-san-diego-school-reopening.html?action=click\&pgtype=Article\&state=default\&region=MAIN_CONTENT_3\&context=storylines_faq}{instruction
    will be remote-only in the fall}, citing concerns that surging
    coronavirus infections in their areas pose too dire a risk for
    students and teachers. Together, the two districts enroll some
    825,000 students. They are the largest in the country so far to
    abandon plans for even a partial physical return to classrooms when
    they reopen in August. For other districts, the solution won't be an
    all-or-nothing approach.
    \href{https://bioethics.jhu.edu/research-and-outreach/projects/eschool-initiative/school-policy-tracker/}{Many
    systems}, including the nation's largest, New York City, are
    devising
    \href{https://www.nytimes.com/2020/06/26/us/coronavirus-schools-reopen-fall.html?action=click\&pgtype=Article\&state=default\&region=MAIN_CONTENT_3\&context=storylines_faq}{hybrid
    plans} that involve spending some days in classrooms and other days
    online. There's no national policy on this yet, so check with your
    municipal school system regularly to see what is happening in your
    community.
  \end{itemize}
\item ~
  \hypertarget{is-the-coronavirus-airborne}{%
  \paragraph{Is the coronavirus
  airborne?}\label{is-the-coronavirus-airborne}}

  \begin{itemize}
  \tightlist
  \item
    The coronavirus
    \href{https://www.nytimes.com/2020/07/04/health/239-experts-with-one-big-claim-the-coronavirus-is-airborne.html?action=click\&pgtype=Article\&state=default\&region=MAIN_CONTENT_3\&context=storylines_faq}{can
    stay aloft for hours in tiny droplets in stagnant air}, infecting
    people as they inhale, mounting scientific evidence suggests. This
    risk is highest in crowded indoor spaces with poor ventilation, and
    may help explain super-spreading events reported in meatpacking
    plants, churches and restaurants.
    \href{https://www.nytimes.com/2020/07/06/health/coronavirus-airborne-aerosols.html?action=click\&pgtype=Article\&state=default\&region=MAIN_CONTENT_3\&context=storylines_faq}{It's
    unclear how often the virus is spread} via these tiny droplets, or
    aerosols, compared with larger droplets that are expelled when a
    sick person coughs or sneezes, or transmitted through contact with
    contaminated surfaces, said Linsey Marr, an aerosol expert at
    Virginia Tech. Aerosols are released even when a person without
    symptoms exhales, talks or sings, according to Dr. Marr and more
    than 200 other experts, who
    \href{https://academic.oup.com/cid/article/doi/10.1093/cid/ciaa939/5867798}{have
    outlined the evidence in an open letter to the World Health
    Organization}.
  \end{itemize}
\item ~
  \hypertarget{what-are-the-symptoms-of-coronavirus}{%
  \paragraph{What are the symptoms of
  coronavirus?}\label{what-are-the-symptoms-of-coronavirus}}

  \begin{itemize}
  \tightlist
  \item
    Common symptoms
    \href{https://www.nytimes.com/article/symptoms-coronavirus.html?action=click\&pgtype=Article\&state=default\&region=MAIN_CONTENT_3\&context=storylines_faq}{include
    fever, a dry cough, fatigue and difficulty breathing or shortness of
    breath.} Some of these symptoms overlap with those of the flu,
    making detection difficult, but runny noses and stuffy sinuses are
    less common.
    \href{https://www.nytimes.com/2020/04/27/health/coronavirus-symptoms-cdc.html?action=click\&pgtype=Article\&state=default\&region=MAIN_CONTENT_3\&context=storylines_faq}{The
    C.D.C. has also} added chills, muscle pain, sore throat, headache
    and a new loss of the sense of taste or smell as symptoms to look
    out for. Most people fall ill five to seven days after exposure, but
    symptoms may appear in as few as two days or as many as 14 days.
  \end{itemize}
\item ~
  \hypertarget{does-asymptomatic-transmission-of-covid-19-happen}{%
  \paragraph{Does asymptomatic transmission of Covid-19
  happen?}\label{does-asymptomatic-transmission-of-covid-19-happen}}

  \begin{itemize}
  \tightlist
  \item
    So far, the evidence seems to show it does. A widely cited
    \href{https://www.nature.com/articles/s41591-020-0869-5}{paper}
    published in April suggests that people are most infectious about
    two days before the onset of coronavirus symptoms and estimated that
    44 percent of new infections were a result of transmission from
    people who were not yet showing symptoms. Recently, a top expert at
    the World Health Organization stated that transmission of the
    coronavirus by people who did not have symptoms was ``very rare,''
    \href{https://www.nytimes.com/2020/06/09/world/coronavirus-updates.html?action=click\&pgtype=Article\&state=default\&region=MAIN_CONTENT_3\&context=storylines_faq\#link-1f302e21}{but
    she later walked back that statement.}
  \end{itemize}
\end{itemize}

Family members still do not know how he got the virus. ``He never left
the house,'' said Ms. Tafolla, who is 63 and is Mr. Salinas's aunt. Even
before the virus, Mr. Tafolla had been battling other medical issues,
and recently had heart surgery, she said. Days after his first virus
symptoms, he was gone.

``We kind of knew that if he ended up getting sick, that he would end up
leaving us,'' she said.

Image

Armando Tafolla held a rosary that belonged to his father, who died of
the coronavirus.Credit...Tamir Kalifa for The New York Times

Image

Gloria Tafolla Gomez beside the coffin of her father.Credit...Tamir
Kalifa for The New York Times

At Ceballos Funeral Home in McAllen, people seeking funerals during the
pandemic have to wait several days, sometimes a week, Ms. Ceballos said.
She has seen young victims, too, she said.

``Nothing is like it used to be,'' Ms. Ceballos said.

Aaron Rivera, a funeral director and embalmer at Rivera Funeral Home in
McAllen, said he ordered a refrigerated truck with a capacity for about
100 bodies to avoid turning people away. The volume has tripled in the
last month, he said.

``They don't get to see their loved ones when they are taken to the
hospital,'' Mr. Rivera said. ``They should see them at the funeral.''

Mr. Salinas has been working around the clock, sleeping only a handful
of hours a day. He owns two funeral home locations, one in Hidalgo
County and a second in neighboring Cameron County. Smaller funeral homes
have been referring grieving families to him when they have no more
space.

``I tell them, stop sending them,'' he said. ``We are tired. We haven't
stopped. We need to sleep.''

Image

Flooded with demand for his services, Mr. Salinas put in place a new
shelf, which he said would allow him to store bodies more
efficiently.Credit...Tamir Kalifa for The New York Times

Image

Imprints from the straps of a face mask were visible on Mr. Salinas's
skin as he worked at his funeral home.Credit...Tamir Kalifa for The New
York Times

On a recent afternoon, Monica Garcia told Mr. Salinas of the day that
her mother texted her from a hospital, pleading to ``get me out of
here.''

Relatives rushed to the hospital parking lot, where a doctor explained
that her mother, 61-year-old Sylvia N. Fuentes, had a better chance at
beating the coronavirus in an intensive care unit than at home. They
left feeling hopeful. The next day --- Wednesday --- her kidneys gave
out and she died.

During a two-hour meeting with Mr. Salinas, Ms. Garcia and other family
members picked out a burial outfit. Ms. Fuentes would be dressed in her
favorite dress. She would wear her favorite hair brooch she wore on
special occasions.

``Peinala to the side,'' Ms. Garcia said, mixing Spanish and English,
comb her hair to the side. ``She loved taking photos and posing,'' Ms.
Garcia said. ``She'd like to look pretty.''

The relatives picked a pearl-white coffin adorned with a medallion:
``Querida Madre,'' beloved mother. They shared a tender look that, in
some other year, might have turned into a group hug. Instead, they
looked down and returned to Mr. Salinas's office to complete paperwork.

Image

Estrella Martinez, left, was overcome with emotion as she spoke to Mr.
Salinas about funeral arrangements for her sister.Credit...Tamir Kalifa
for The New York Times

Image

Mr. Salinas positioned a coffin.Credit...Tamir Kalifa for The New York
Times

For Mr. Salinas, long days have begun to blend, one into the next.

When a call came in on a recent afternoon, it was from inside the
funeral home. The virus had hit close again. The cousin of a funeral
home worker had died from Covid-19. His body waited in the nearby
chapel. Mr. Salinas excused himself and disappeared down the hall.

Mitch Smith contributed reporting from Chicago.

Advertisement

\protect\hyperlink{after-bottom}{Continue reading the main story}

\hypertarget{site-index}{%
\subsection{Site Index}\label{site-index}}

\hypertarget{site-information-navigation}{%
\subsection{Site Information
Navigation}\label{site-information-navigation}}

\begin{itemize}
\tightlist
\item
  \href{https://help.nytimes.com/hc/en-us/articles/115014792127-Copyright-notice}{©~2020~The
  New York Times Company}
\end{itemize}

\begin{itemize}
\tightlist
\item
  \href{https://www.nytco.com/}{NYTCo}
\item
  \href{https://help.nytimes.com/hc/en-us/articles/115015385887-Contact-Us}{Contact
  Us}
\item
  \href{https://www.nytco.com/careers/}{Work with us}
\item
  \href{https://nytmediakit.com/}{Advertise}
\item
  \href{http://www.tbrandstudio.com/}{T Brand Studio}
\item
  \href{https://www.nytimes.com/privacy/cookie-policy\#how-do-i-manage-trackers}{Your
  Ad Choices}
\item
  \href{https://www.nytimes.com/privacy}{Privacy}
\item
  \href{https://help.nytimes.com/hc/en-us/articles/115014893428-Terms-of-service}{Terms
  of Service}
\item
  \href{https://help.nytimes.com/hc/en-us/articles/115014893968-Terms-of-sale}{Terms
  of Sale}
\item
  \href{https://spiderbites.nytimes.com}{Site Map}
\item
  \href{https://help.nytimes.com/hc/en-us}{Help}
\item
  \href{https://www.nytimes.com/subscription?campaignId=37WXW}{Subscriptions}
\end{itemize}
