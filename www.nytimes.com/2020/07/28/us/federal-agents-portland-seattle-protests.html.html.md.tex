Sections

SEARCH

\protect\hyperlink{site-content}{Skip to
content}\protect\hyperlink{site-index}{Skip to site index}

\href{https://www.nytimes.com/section/us}{U.S.}

\href{https://myaccount.nytimes.com/auth/login?response_type=cookie\&client_id=vi}{}

\href{https://www.nytimes.com/section/todayspaper}{Today's Paper}

\href{/section/us}{U.S.}\textbar{}From the Start, Federal Agents
Demanded a Role in Suppressing Anti-Racism Protests

\url{https://nyti.ms/2X8yODM}

\begin{itemize}
\item
\item
\item
\item
\item
\end{itemize}

\href{https://www.nytimes.com/news-event/george-floyd-protests-minneapolis-new-york-los-angeles?action=click\&pgtype=Article\&state=default\&region=TOP_BANNER\&context=storylines_menu}{Race
and America}

\begin{itemize}
\tightlist
\item
  \href{https://www.nytimes.com/2020/07/26/us/protests-portland-seattle-trump.html?action=click\&pgtype=Article\&state=default\&region=TOP_BANNER\&context=storylines_menu}{Protesters
  Return to Other Cities}
\item
  \href{https://www.nytimes.com/2020/07/24/us/portland-oregon-protests-white-race.html?action=click\&pgtype=Article\&state=default\&region=TOP_BANNER\&context=storylines_menu}{Portland
  at the Center}
\item
  \href{https://www.nytimes.com/2020/07/23/podcasts/the-daily/portland-protests.html?action=click\&pgtype=Article\&state=default\&region=TOP_BANNER\&context=storylines_menu}{Podcast:
  Showdown in Portland}
\item
  \href{https://www.nytimes.com/interactive/2020/07/16/us/black-lives-matter-protests-louisville-breonna-taylor.html?action=click\&pgtype=Article\&state=default\&region=TOP_BANNER\&context=storylines_menu}{45
  Days in Louisville}
\end{itemize}

Advertisement

\protect\hyperlink{after-top}{Continue reading the main story}

Supported by

\protect\hyperlink{after-sponsor}{Continue reading the main story}

\hypertarget{from-the-start-federal-agents-demanded-a-role-in-suppressing-anti-racism-protests}{%
\section{From the Start, Federal Agents Demanded a Role in Suppressing
Anti-Racism
Protests}\label{from-the-start-federal-agents-demanded-a-role-in-suppressing-anti-racism-protests}}

Twin government memos show how a gung-ho federal law enforcement
response to anti-racism protests may have been driven by a shaky
understanding of the demonstrations' roots.

\includegraphics{https://static01.nyt.com/images/2020/07/28/us/politics/28dc-unrest-feds/28dc-unrest-feds-articleLarge.jpg?quality=75\&auto=webp\&disable=upscale}

By \href{https://www.nytimes.com/by/zolan-kanno-youngs}{Zolan
Kanno-Youngs}, Sergio Olmos,
\href{https://www.nytimes.com/by/mike-baker}{Mike Baker} and
\href{https://www.nytimes.com/by/adam-goldman}{Adam Goldman}

\begin{itemize}
\item
  July 28, 2020
\item
  \begin{itemize}
  \item
  \item
  \item
  \item
  \item
  \end{itemize}
\end{itemize}

WASHINGTON --- From the earliest days of the recent protests against
police brutality and racism, some top
\href{https://www.nytimes.com/2020/07/30/nyregion/nypd-protester-van.html}{federal
law enforcement officials} viewed the demonstrators with alarm and
called for an aggressive federal response that two months later
continues to escalate.

A memo from the deputy director of the F.B.I., dated June 2, demanded an
immediate mobilization as protests gathered
\href{https://www.nytimes.com/2020/05/31/us/george-floyd-investigation.html}{after
George Floyd's death while he was in police custody a week earlier}.
David L. Bowdich, the F.B.I.'s No. 2, declared the situation ``a
national crisis,'' and wrote that in addition to investigating ``violent
protesters, instigators'' and ``inciters,'' bureau leaders should
collect information with ``robust social media exploitation teams'' and
examine what appeared to be ``highly organized behavior.''

Mr. Bowdich suggested that the bureau could make use of the Hobbs Act,
put into place in the 1940s to punish racketeering in labor groups, to
charge the protesters.

``When 9/11 occurred, our folks did not quibble about whether there was
danger ahead for them,'' he wrote, telling aides that the
\href{https://www.nytimes.com/interactive/2020/us/coronavirus-us-cases.html}{continuing
coronavirus pandemic} should not hold them back. ``They ran head-on into
peril.''

The memo came after a weekend in which protests gave way to looting in
some cities and the day after federal agents forcibly cleared peaceful
protesters from the White House so
\href{https://www.nytimes.com/2020/06/02/us/politics/trump-walk-lafayette-square.html}{President
Trump could walk through Lafayette Square}. Since then, the federal
response has become a focal point of the Trump administration
\href{https://www.nytimes.com/2020/07/21/us/politics/trump-portland-federal-agents.html}{and
of Mr. Trump's re-election campaign}. The Bowdich memo suggests agencies
need little prodding to adopt the president's forceful posture.

``Think differently, out of the box,'' the memo demanded.

On Tuesday, Attorney General William P. Barr took the same tone, saying
strife in
\href{https://www.nytimes.com/2020/07/29/us/protests-portland-federal-withdrawal.html}{Portland},
Ore., was not a protest at all, but ``an assault on the government of
the United States.''

``Remarkably, the response from many in the media and local elected
offices to this organized assault has been to blame the federal
government,'' Mr. Barr told the House Judiciary Committee. ``To state
what should be obvious, peaceful protesters do not throw explosives into
federal courthouses.''

Privately, domestic intelligence agents are uncertain about the root
causes of those actions. Another internal government memo, from
Department of Homeland Security intelligence officers, indicated that
even as federal agents in camouflage deployed to quell the unrest in
Portland, the administration had little understanding of what it was
facing.

\includegraphics{https://static01.nyt.com/images/2020/07/28/us/politics/28dc-unrest-feds2/28dc-unrest-feds2-articleLarge.jpg?quality=75\&auto=webp\&disable=upscale}

The memo tried to put the recent conflict into historical context,
describing how ``anarchist extremists'' have committed crimes in the
Pacific Northwest for years and asserting that ``sustained violence
against government personnel and facilities'' had longstanding roots.

But even as it laid out a timeline of violence extending back to 2015,
the intelligence briefing, dated July 16, admitted, ``We have low
confidence in our assessment'' when it comes to the present day.

``We lack insight into the motives for the most recent attacks,'' it
read.

\hypertarget{theyve-become-the-story}{%
\subsection{`They've Become the Story'}\label{theyve-become-the-story}}

At the end of May, as protests against police brutality and racism
sprang up across the country, Mr. Trump decided the unrest was the work
of ``antifa,'' a leaderless coalition of people who oppose fascism but
have at times used vandalism and violence to make their points.

Since then, federal prosecutors have brought charges against
demonstrators across the country for crimes that would typically be
handled locally. In Delaware and Alabama, prosecutors brought charges
against people who each smashed the window of a police car. In Ohio,
prosecutors charged someone for burning a parking-attendant booth.

In Missouri, one man was charged for Facebook posts that authorities
said were inciting violence. The charges were later dropped.

``I don't think this type of charge would have been filed under any
other administration,'' said Marleen Menendez Suarez, a lawyer for the
Missouri man, Michael Avery.

Federal law enforcement has a duty to investigate the organizing of
crimes across state lines, said John McKay, a former U.S. attorney
appointed under President George W. Bush. Federal crimes around banks or
guns are also common targets. But, he said, a broken window or robberies
would typically be left to the local authorities.

``The feds shouldn't come in unless there is a clear indication of
federal crime and a federal interest,'' he said.

Nowhere has the federal response been as aggressive and obvious as
Portland, where for years now leftist groups and white extremists have
faced off. Since Mr. Floyd's death, protesters have held demonstrations
every night for 60 consecutive days, and those gatherings have grown in
size and intensity since federal authorities arrived. Some in the crowds
have lobbed commercial-grade fireworks toward the officers and pointed
lasers at the faces of federal agents. Agents have clubbed demonstrators
with batons, used tear gas indiscriminately and forced protesters into
unmarked vans, prompting investigations by the inspectors general for
the Departments of Homeland Security and Justice.

The feud between the Trump administration and the local officials has
grown so tense that on Monday night, Portland's mayor, Ted Wheeler,
called for a meeting with Department of Homeland Security officials to
discuss ``a cease-fire'' with his own federal government.

On Tuesday, Mr. Barr was adamant that the administration had a right and
a duty to protect federal property and defend its agents.

Image

Attorney General William P. Barr testifying on Tuesday before the House
Judiciary Committee.Credit...Pool photo by Matt McClain

But outside the administration, a consensus is emerging: The deployment
of the federal agents is perpetuating the unrest.

``They've become the story,'' John Sandweg, a former acting general
counsel for the Department of Homeland Security, said of the federal
deployments. ``The protests are feeding off their presence.''

In New York, F.B.I. agents were deployed to guard New York Police
Department precincts for a few days during the height of the protests in
early June.

Last month, the F.B.I.'s elite hostage rescue team was deployed to stand
by in Washington, an unnecessary show of public force, some agents say
they thought, that miscast a lethal unit that conducts the bureau's most
dangerous missions. When a group of F.B.I. agents in Washington knelt in
front of protesters last month, some former agents saw cowardice while
others applauded the de-escalation effort.

The F.B.I. referred to a statement issued in early June that said agents
were committed to defending First Amendment rights.

\hypertarget{low-confidence-in-our-assessment}{%
\subsection{`Low Confidence in Our
Assessment'}\label{low-confidence-in-our-assessment}}

But de-escalation tactics have been the exception. Over the Fourth of
July weekend, after Mr. Trump
\href{https://www.nytimes.com/2020/06/26/us/politics/trump-monuments-executive-order.html}{directed
federal agencies to strengthen security} at statues and federal
property, the Department of Homeland Security put about 2,000 agents
from various agencies on standby and deployed more than 200 tactical
agents to multiple cities, but many of them outside a single federal
courthouse in Portland, already covered with graffiti and marred by
broken windows.

In the July 16 intelligence briefing memo, the department concluded the
``sustained violence against government personnel and facilities in
Portland, Ore., since May reflects the enduring threat environment in
the region since at least 2015.'' The memo included a timeline of
violent episodes in Seattle and Olympia, Wash., as well as in Portland.
(The timeline specified two episodes involving ``white supremacist
extremists.'')

Image

Protesters last week outside the Multnomah County Justice Center in
Portland.Credit...Mason Trinca for The New York Times

But the memo, prepared by the Counterterrorism Mission Center, also
admitted that ``we have low confidence in our assessment that sustained
violence against government personnel and facilities in Portland, Ore.,
since May reflects the enduring threat environment in the region because
we lack insight into the motives for the most recent attacks.''

The agency may have been raising questions about the accuracy of its
findings, but its leaders have shown no such qualms. Last week, Chad F.
Wolf, the acting secretary of homeland security, referred to protests in
Portland in 2018 that prompted the department to temporarily shut down a
detention center run by its Immigration Customs Enforcement.

``There's a little bit of a pattern here that obviously I'm concerned
about,'' Mr. Wolf said.

The agency does not appear ready to
\href{https://www.opb.org/news/article/more-federal-officers-deploying-portland/}{pull
back its forces}. The U.S. Marshals Service said Monday it had
identified 100 officials to send to Portland to relieve or back up
marshals at the courthouse. The Department of Homeland Security is also
considering sending more than three dozen Customs and Border Protection
agents to the city to back up the tactical agents from ICE and BORTAC,
the Border Patrol's equivalent of a S.W.A.T. team, as well as the
Federal Protective Service.

\hypertarget{out-into-the-streets}{%
\subsection{Out Into the Streets}\label{out-into-the-streets}}

The deployment has not only outraged local officials who have asked
federal agents to leave, but it has also raised questions about the
authority that federal officers are operating under.

The Department of Homeland Security has cited a law that permits federal
agents to ``conduct investigations'' into crimes against federal
property or officers. But in recent days, officers have left the grounds
of the courthouse in Portland and
\href{https://www.nytimes.com/2020/07/25/us/portland-federal-legal-jurisdiction-courts.html}{pursued
protesters through the streets, firing tear gas and pepper balls,
advancing to areas where the courthouse was no longer visible}.

Image

The deployment of federal agents in Portland outraged local officials
who have asked them to leave, and raised questions about the authority
the officers are operating under.Credit...Mason Trinca for The New York
Times

The department also
\href{https://www.nytimes.com/2020/07/23/us/seattle-protests-feds.html}{sent
a tactical team to stand by in Seattle last week}, hours after
department officials told the mayor there no such deployment would
occur. After pushback from those local officials, the administration
told the Seattle government on Tuesday the team had left the city.

Federal agencies generally reach an agreement with local governments
before dispatching tactical agents to respond to local crime, and
\href{https://www.nytimes.com/2020/07/18/us/portland-protests.html}{an
internal Department of Homeland Security memo warned that deployed teams
were not trained to confront the unrest}. Department officials have said
the training the teams received to handle crowds of migrants at the
border and riots in detention facilities has prepared them for Portland.

Chuck Wexler, the director of the Police Executive Research Forum, said
federal officials could be engaging in a dialogue with activist leaders
and local politicians during the day to calm tensions ahead of nighttime
protests.

The opposite is occurring.

``I don't know who's benefiting from this,'' Mr. Wexler said.

Zolan Kanno-Youngs and Adam Goldman reported from Washington, Sergio
Olmos from Portland, Ore., and Mike Baker from Seattle. William K.
Rashbaum contributed reporting from New York.

Advertisement

\protect\hyperlink{after-bottom}{Continue reading the main story}

\hypertarget{site-index}{%
\subsection{Site Index}\label{site-index}}

\hypertarget{site-information-navigation}{%
\subsection{Site Information
Navigation}\label{site-information-navigation}}

\begin{itemize}
\tightlist
\item
  \href{https://help.nytimes.com/hc/en-us/articles/115014792127-Copyright-notice}{©~2020~The
  New York Times Company}
\end{itemize}

\begin{itemize}
\tightlist
\item
  \href{https://www.nytco.com/}{NYTCo}
\item
  \href{https://help.nytimes.com/hc/en-us/articles/115015385887-Contact-Us}{Contact
  Us}
\item
  \href{https://www.nytco.com/careers/}{Work with us}
\item
  \href{https://nytmediakit.com/}{Advertise}
\item
  \href{http://www.tbrandstudio.com/}{T Brand Studio}
\item
  \href{https://www.nytimes.com/privacy/cookie-policy\#how-do-i-manage-trackers}{Your
  Ad Choices}
\item
  \href{https://www.nytimes.com/privacy}{Privacy}
\item
  \href{https://help.nytimes.com/hc/en-us/articles/115014893428-Terms-of-service}{Terms
  of Service}
\item
  \href{https://help.nytimes.com/hc/en-us/articles/115014893968-Terms-of-sale}{Terms
  of Sale}
\item
  \href{https://spiderbites.nytimes.com}{Site Map}
\item
  \href{https://help.nytimes.com/hc/en-us}{Help}
\item
  \href{https://www.nytimes.com/subscription?campaignId=37WXW}{Subscriptions}
\end{itemize}
