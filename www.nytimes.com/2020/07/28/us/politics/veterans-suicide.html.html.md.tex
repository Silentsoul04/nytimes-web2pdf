Sections

SEARCH

\protect\hyperlink{site-content}{Skip to
content}\protect\hyperlink{site-index}{Skip to site index}

\href{https://www.nytimes.com/section/politics}{Politics}

\href{https://myaccount.nytimes.com/auth/login?response_type=cookie\&client_id=vi}{}

\href{https://www.nytimes.com/section/todayspaper}{Today's Paper}

\href{/section/politics}{Politics}\textbar{}Report Slams Doctor at V.A.
for Dismissing Suicide Risk of Patient Who Later Killed Himself

\url{https://nyti.ms/39xKMvF}

\begin{itemize}
\item
\item
\item
\item
\item
\end{itemize}

Advertisement

\protect\hyperlink{after-top}{Continue reading the main story}

Supported by

\protect\hyperlink{after-sponsor}{Continue reading the main story}

\hypertarget{report-slams-doctor-at-va-for-dismissing-suicide-risk-of-patient-who-later-killed-himself}{%
\section{Report Slams Doctor at V.A. for Dismissing Suicide Risk of
Patient Who Later Killed
Himself}\label{report-slams-doctor-at-va-for-dismissing-suicide-risk-of-patient-who-later-killed-himself}}

Six days after being removed by the police from a veterans hospital in
Washington, the man died from a self-inflicted gunshot wound.

\includegraphics{https://static01.nyt.com/images/2020/07/28/us/politics/28dc-va/merlin_169646094_23e0bbc5-c4fe-44ca-bca5-3417234a3ac1-articleLarge.jpg?quality=75\&auto=webp\&disable=upscale}

By \href{https://www.nytimes.com/by/jennifer-steinhauer}{Jennifer
Steinhauer}

\begin{itemize}
\item
  July 28, 2020
\item
  \begin{itemize}
  \item
  \item
  \item
  \item
  \item
  \end{itemize}
\end{itemize}

WASHINGTON --- The internal watchdog for the Department of Veterans
Affairs said Tuesday that a veteran who came through the department's
medical center in Washington last year seeking psychiatric treatment
died by suicide a few days later, after a doctor there ordered him
forcibly removed and was heard saying that she did ``not care'' if he
killed himself.

The report, which details many failings at a center that has been the
subject of repeated criticism, comes a few weeks after Robert L. Wilkie,
the secretary of veterans affairs,
\href{https://www.stripes.com/news/us/wilkie-trump-is-the-first-president-since-1890s-to-recognize-veteran-suicide-crisis-1.637847}{told}
a conservative news outlet that ``President Trump is the first president
since the 1890s who recognized the scourge of veteran suicide.''

In 2017, the suicide rate was about 28 per 100,000 for veterans,
compared with 18 per 100,000 people in the overall American population,
according to
a\href{https://www.mentalhealth.va.gov/docs/data-sheets/2019/2019_National_Veteran_Suicide_Prevention_Annual_Report_508.pdf}{2019
department report}.

But even as the department
\href{https://www.nytimes.com/2019/04/14/us/politics/veterans-suicide.html}{struggles}
to lower those numbers, the case at the veterans medical center last
year appears egregious. A patient in his 60s, who had a long history of
panic attacks, pain killer addiction and various injuries came into the
hospital's emergency room.

He described pain from drug withdrawal and insomnia, and asked to be
admitted for detoxification. An outpatient psychiatrist assessed him as
being at ``moderate risk for suicide'' and recommended he be admitted.
He was sent back to the emergency room.

Later, a consulting psychiatry resident, who said the patient could be
served in outpatient care, recommended he be discharged and sent home,
but the patient refused to leave, according to the inspector general
report.

Police officers from the department were called to escort him out, and
at least three hospital staff members said they heard the doctor say
that veterans ``can go shoot themselves. I do not care.''

The man died six days later from a self-inflicted gunshot wound.

The inspector general's report found many missteps in the patient's
care, including communications breakdowns among staff members, as well
as the fact that the ``Emergency Department and consulting psychiatry
staff failed to complete required suicide prevention planning prior to
the patient's discharge.''

``Emergency Department staff's failure to manage this patient's care,
according to Veteran Health Administration suicide prevention policies,
contributed to an inadequate assessment of suicide risk,'' the report
stated.

According to the report, the actions ``could also be considered
misconduct according to V.A. policy and patient abuse according to
facility policy.''

The doctor, who was not named in the report, worked at the veterans
center through a contract with the George Washington University
Hospital, an arrangement that is common across veterans medical care.
The report faults the center for not moving to immediately discipline or
remove the doctor, who ``had a history of verbal misconduct.''

The doctor highlighted in the report ``was never a V.A. employee, only
worked on a contract basis and is no longer welcome at the facility,''
Dr. Michael S. Heimall, the center's director, said in an emailed
statement.

The hospital ``grieves for the loss of this veteran and extends our
deepest condolences to their loved ones,'' he added.

The episode prompted the hospital to make changes to its policies, Dr.
Heimall said, including requiring weekly audits of how some
suicide-related patients' visits to the emergency room were handled and
instituting a ``comprehensive education program regarding employee
misconduct and patient abuse.''

The center has come under fire before from Michael J. Missal, the
department's inspector general.

In
\href{https://www.stripes.com/news/inspector-general-poor-leadership-led-to-widespread-problems-at-dc-va-1.515388}{2018,}
he cited poor management and other factors, which included a lack of
medical supplies, less-than-sterile conditions and chronic
understaffing. Last year, a senior policy adviser on female veterans'
issues on Capitol Hill said she was assaulted at the center in
Washington by a man who slammed his body against hers and then pressed
himself against her in the center's cafe. That episode remains the
subject of investigation by Mr. Missal's office.

Since the suicide, officials at the hospital told the inspector general
they had ``instituted a comprehensive educational program for clinical
staff working in the emergency room to ensure staff's understanding of
the Veterans Health Administration's local policies surrounding suicide
prevention.''

The doctor was removed several months after the episode.

Advertisement

\protect\hyperlink{after-bottom}{Continue reading the main story}

\hypertarget{site-index}{%
\subsection{Site Index}\label{site-index}}

\hypertarget{site-information-navigation}{%
\subsection{Site Information
Navigation}\label{site-information-navigation}}

\begin{itemize}
\tightlist
\item
  \href{https://help.nytimes.com/hc/en-us/articles/115014792127-Copyright-notice}{©~2020~The
  New York Times Company}
\end{itemize}

\begin{itemize}
\tightlist
\item
  \href{https://www.nytco.com/}{NYTCo}
\item
  \href{https://help.nytimes.com/hc/en-us/articles/115015385887-Contact-Us}{Contact
  Us}
\item
  \href{https://www.nytco.com/careers/}{Work with us}
\item
  \href{https://nytmediakit.com/}{Advertise}
\item
  \href{http://www.tbrandstudio.com/}{T Brand Studio}
\item
  \href{https://www.nytimes.com/privacy/cookie-policy\#how-do-i-manage-trackers}{Your
  Ad Choices}
\item
  \href{https://www.nytimes.com/privacy}{Privacy}
\item
  \href{https://help.nytimes.com/hc/en-us/articles/115014893428-Terms-of-service}{Terms
  of Service}
\item
  \href{https://help.nytimes.com/hc/en-us/articles/115014893968-Terms-of-sale}{Terms
  of Sale}
\item
  \href{https://spiderbites.nytimes.com}{Site Map}
\item
  \href{https://help.nytimes.com/hc/en-us}{Help}
\item
  \href{https://www.nytimes.com/subscription?campaignId=37WXW}{Subscriptions}
\end{itemize}
