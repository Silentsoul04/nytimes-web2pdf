Sections

SEARCH

\protect\hyperlink{site-content}{Skip to
content}\protect\hyperlink{site-index}{Skip to site index}

\href{https://www.nytimes.com/section/us}{U.S.}

\href{https://myaccount.nytimes.com/auth/login?response_type=cookie\&client_id=vi}{}

\href{https://www.nytimes.com/section/todayspaper}{Today's Paper}

\href{/section/us}{U.S.}\textbar{}New Census Worry: A Rushed Count Could
Mean a Botched One

\url{https://nyti.ms/3hKqQbO}

\begin{itemize}
\item
\item
\item
\item
\item
\end{itemize}

Advertisement

\protect\hyperlink{after-top}{Continue reading the main story}

Supported by

\protect\hyperlink{after-sponsor}{Continue reading the main story}

\hypertarget{new-census-worry-a-rushed-count-could-mean-a-botched-one}{%
\section{New Census Worry: A Rushed Count Could Mean a Botched
One}\label{new-census-worry-a-rushed-count-could-mean-a-botched-one}}

Stalled by the pandemic, the count is supposed to resume soon. But
census experts are rattled by signs of a push from the White House to
finish it early.

\includegraphics{https://static01.nyt.com/images/2020/07/26/us/26census-2/merlin_170599233_81fbb0a1-5d87-49a2-a226-710f1fe70719-articleLarge.jpg?quality=75\&auto=webp\&disable=upscale}

By \href{https://www.nytimes.com/by/michael-wines}{Michael Wines}

\begin{itemize}
\item
  July 28, 2020
\item
  \begin{itemize}
  \item
  \item
  \item
  \item
  \item
  \end{itemize}
\end{itemize}

WASHINGTON --- As the 2020 census struggles to find its footing amid the
coronavirus outbreak and public reluctance to give the government
personal data, officials have a new worry: The Trump administration and
Senate Republicans appear to be signaling that they want the census
finished well ahead of schedule, pandemic or not.

With almost 40 percent of the nation's households still uncounted,
including the hardest-to-reach populations that are disproportionately
poor, people of color and young, the Trump administration took the
Census Bureau by surprise last week. It asked the Senate Appropriations
Committee to set aside \$448 million in the next coronavirus relief
package for a ``timely'' completion of the census.

The request did not define what ``timely'' meant, and legislation
released on Monday said only that the money would be used for nationwide
census operations and data processing. But it comes as census workers
and former officials say the White House and the Commerce Department,
which oversees the Census Bureau, are asking how the bureau can compress
its schedule to wrap up the count of households earlier than expected
--- perhaps by the end of September. The aim, they say, may be to speed
up the delivery of key data for political reapportionment to the
president by the end of December.

The administration has yet to announce a compressed schedule and may not
find a way to do so. But the prospect already has alarmed an array of
experts, who warned in recent days that an expedited census risks a
deeply flawed count of the nation's population. The census is
constitutionally required to count all residents of the country every 10
years.

``There's a lot of uncertainty, but one thing is absolutely sure: There
will be egregious undercounts if the Census Bureau has to produce this
data by December,'' said Robert Santos, the vice president of the Urban
Institute and the incoming president of the American Statistical
Association.

Some, including former Census Bureau directors, raised the prospect that
the final totals could be so skewed that a future Congress might order
the bureau to do further work on the 2020 population data, or even
consider another census in five years, which federal law allows but
which has never been conducted nationwide.

The numbers are enormously important. They are used to reapportion all
435 House seats and thousands of state and local districts, as well as
divvy up trillions of dollars in federal grants and aid.

At issue is how fast, and how precisely, the Census Bureau will track
down and count the 60 million households that have not filled out census
forms.

Slightly more than six in 10 households have completed forms. The
remainder are the very hardest to count. To reach them, the bureau has
planned to deploy up to 500,000 census takers, each with an iPhone that
can securely relay census data to the bureau's computers.

In 2010, census takers worked from May to August to count hard-to-find
households. This spring, with the start of that count delayed by the
pandemic, the bureau said it was pushing back the start of that work to
August, ending on Oct. 31.

With White House approval, the bureau also
\href{https://www.nytimes.com/2020/04/13/us/census-coronavirus-delay.html?searchResultPosition=9}{asked
Congress for a four-month extension} --- to April 2021 --- of the Dec.
31 statutory deadline for delivering to the president the population
totals required to reapportion the House of Representatives.

But that plan now appears to be in flux. Census Bureau workers have been
asked whether that Oct. 31 deadline for collecting data can be moved to
September, giving them six or seven weeks to finish a count that was
supposed to take 10 weeks.

At the same time, the administration's commitment to extending the
delivery of reapportionment statistics beyond the statutory Dec. 31
deadline also appears in doubt.

In Congress, the House has approved the four-month delay. The Senate has
not.

Legislation extending the Dec. 31 deadline was widely expected to be
included in the coronavirus relief package that Republicans in the
Senate unveiled on Monday. But a review of the voluminous legislative
package found no evidence that it had been.

Asked on Saturday whether Senator Mitch McConnell, the majority leader,
still supports extending the deadline, a spokesman for the senator said
in an email: ``Don't think I'm going to be able to help you out on
this.''

\includegraphics{https://static01.nyt.com/images/2020/07/26/us/26census-1/merlin_173932341_3ad63e30-b188-48c2-8d67-aa24e798e1f2-articleLarge.jpg?quality=75\&auto=webp\&disable=upscale}

The White House declined to address questions about its census plans.
Responding to a reporter's questions, the Census Bureau issued a
statement on Monday that neither confirmed nor denied an effort to
hasten the completion of the count and the delivery of reapportionment
figures.

``The Census Bureau is working toward the plan to complete field data
collection by October 31,'' it said. It then added that its staff would
``continue to evaluate and plan for all contingencies, including the
impact of delivering statutorily required data products at the current
legislative deadlines'' --- a reference to the Dec. 31 date to produce
reapportionment figures.

In fact, top Census Bureau officials already have said that meeting that
deadline is impossible.

``We have passed the point where we could even meet the current
legislative requirement of Dec. 31. We can't do that anymore,'' the
census official leading field operations for the count, Tim Olson,
\href{https://www.youtube.com/watch?v=F6IyJMtDDgY\&feature=youtu.be\&t=4688}{told
a Native American organization} during a webinar in May.

And in a webinar this month for groups with a stake in census results,
the associate director of the census, Albert E. Fontenot Jr., said, ``we
are past the window of being able to get those counts'' by year's end.

The new concerns come atop a growing record of political interference in
census decisions by the Trump administration.

The Supreme Court last year, in a 5-to-4 vote,
\href{https://www.nytimes.com/2019/06/27/us/politics/supreme-court-gerrymandering-census.html}{rejected
the administration's effort to add a citizenship question} to the census
that experts said would surely depress the count of immigrants and
minorities, documented and otherwise.

On White House orders, the Census Bureau last month
\href{https://www.nytimes.com/2020/06/23/us/census-bureau-cogley-korzeniewski.html}{created
two top-level positions} and filled them with political appointees from
outside, a remarkable move in an agency renowned for its nonpartisan
culture.

Some critics say
\href{https://www.nytimes.com/2020/07/21/us/politics/trump-immigrants-census-redistricting.html}{Mr.
Trump's order last week to exclude undocumented immigrants} from
state-by-state population totals used for reapportionment totals
explains the administration's apparent desire to speed up census work.

The order, which is already being challenged in court, is widely viewed
as unconstitutional by legal scholars. But for the order to have any
chance of succeeding, they say, the census totals used for
reapportionment must be delivered to Mr. Trump while he is still in
office --- as he almost certainly will be on Dec. 31, but may well not
be in April 2021.

``I think it's entirely about that,'' Thomas A. Saenz, the president of
the Mexican American Legal Defense and Educational Fund, said on Monday.
``He wants to exclude undocumented immigrants because he believes it
will shift representation away from blue states to red states. In the
end, it's entirely about trying to stem Latino political power.''

Others say Mr. Trump's order, regardless of whether it is upheld, could
have an impact on representation by making noncitizens worry that their
answers on a census survey could be used against them.

``They clearly have an agenda for not counting undocumented immigrants
in the apportionment count,'' said Vanita Gupta, the president of the
Leadership Conference on Civil and Human Rights, a coalition of more
than 200 advocacy groups. ``I think the administration knows their order
isn't going to be constitutional. Maybe through fear of it, they're
trying to get to the same place.''

Image

The issue in the latest debate is how fast, and how precisely, the
Census Bureau will track down and count the remaining 60 million
households that have failed to fill out census forms. Credit...Ted S.
Warren/Associated Press

Experts said a rush to wrap up the census would force the bureau into
shortcuts that would make population totals significantly less accurate.
Months of post-census analysis and accuracy checks also would be at risk
were population totals required by December.

``It won't be finished unless they can quickly ramp up something, like
using administrative records'' instead of census takers to count
households, said Kenneth Prewitt, a Columbia University public affairs
professor who led the Census Bureau during the 2000 census. ``Otherwise,
you end up with a census that's 10 percent uncounted, or 12 percent.''

Mr. Prewitt and John Thompson, a career Census Bureau official who
directed the agency from 2013 to 2017, said the bureau also could be
forced to expand its use of a statistical method called imputation, in
which an algorithm makes an educated guess about who lives in a
household by looking at who lives nearby.

Past censuses have relied on imputation for a tiny fraction of
households --- about 1 percent, in most cases --- that could not be
otherwise counted. But ``it could get a lot bigger, maybe 10 or 15
percent in some areas of the country, if they have to cut it short,''
Mr. Thompson said.

If past censuses are any indication, the Census Bureau will state
clearly where it believes inaccuracies lie, and how large they might be.
After the count, the bureau conducts a massive accuracy check, called a
post-enumeration survey, in which experts revisit a sample of households
to see whether reported data was correct.

But while the bureau will say how inaccurate its numbers are, it will
not, in all likelihood, say whether it believes they can be relied on.

``What it means to fail to have a census has never been tested,'' said
Justin Levitt, an expert on the topic at Loyola Law School in Los
Angeles. ``How bad it has to be before it's not a census anymore is
something we have yet to decide.''

Should it come to that, he said, that judgment would probably be hashed
out in Congress --- and later in the courts.

Advertisement

\protect\hyperlink{after-bottom}{Continue reading the main story}

\hypertarget{site-index}{%
\subsection{Site Index}\label{site-index}}

\hypertarget{site-information-navigation}{%
\subsection{Site Information
Navigation}\label{site-information-navigation}}

\begin{itemize}
\tightlist
\item
  \href{https://help.nytimes.com/hc/en-us/articles/115014792127-Copyright-notice}{©~2020~The
  New York Times Company}
\end{itemize}

\begin{itemize}
\tightlist
\item
  \href{https://www.nytco.com/}{NYTCo}
\item
  \href{https://help.nytimes.com/hc/en-us/articles/115015385887-Contact-Us}{Contact
  Us}
\item
  \href{https://www.nytco.com/careers/}{Work with us}
\item
  \href{https://nytmediakit.com/}{Advertise}
\item
  \href{http://www.tbrandstudio.com/}{T Brand Studio}
\item
  \href{https://www.nytimes.com/privacy/cookie-policy\#how-do-i-manage-trackers}{Your
  Ad Choices}
\item
  \href{https://www.nytimes.com/privacy}{Privacy}
\item
  \href{https://help.nytimes.com/hc/en-us/articles/115014893428-Terms-of-service}{Terms
  of Service}
\item
  \href{https://help.nytimes.com/hc/en-us/articles/115014893968-Terms-of-sale}{Terms
  of Sale}
\item
  \href{https://spiderbites.nytimes.com}{Site Map}
\item
  \href{https://help.nytimes.com/hc/en-us}{Help}
\item
  \href{https://www.nytimes.com/subscription?campaignId=37WXW}{Subscriptions}
\end{itemize}
