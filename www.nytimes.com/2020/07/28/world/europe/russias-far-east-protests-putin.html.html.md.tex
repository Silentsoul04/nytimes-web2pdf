Sections

SEARCH

\protect\hyperlink{site-content}{Skip to
content}\protect\hyperlink{site-index}{Skip to site index}

\href{https://www.nytimes.com/section/world/europe}{Europe}

\href{https://myaccount.nytimes.com/auth/login?response_type=cookie\&client_id=vi}{}

\href{https://www.nytimes.com/section/todayspaper}{Today's Paper}

\href{/section/world/europe}{Europe}\textbar{}In Russia's Far East, a
New Face of Resistance to Putin's Reign

\url{https://nyti.ms/3gjT7Wt}

\begin{itemize}
\item
\item
\item
\item
\item
\item
\end{itemize}

Advertisement

\protect\hyperlink{after-top}{Continue reading the main story}

Supported by

\protect\hyperlink{after-sponsor}{Continue reading the main story}

\hypertarget{in-russias-far-east-a-new-face-of-resistance-to-putins-reign}{%
\section{In Russia's Far East, a New Face of Resistance to Putin's
Reign}\label{in-russias-far-east-a-new-face-of-resistance-to-putins-reign}}

As the protests swell in the city of Khabarovsk, 4,000 miles from
Moscow, residents who had never before found a public outlet for anger
are becoming activists.

\includegraphics{https://static01.nyt.com/images/2020/07/27/world/27russia-protests1/merlin_174995703_0bcc4bf2-5058-4c07-bb9c-d8963c57e499-articleLarge.jpg?quality=75\&auto=webp\&disable=upscale}

\href{https://www.nytimes.com/by/anton-troianovski}{\includegraphics{https://static01.nyt.com/images/2019/09/24/reader-center/author-anton-troianovski/author-anton-troianovski-thumbLarge.png}}

By \href{https://www.nytimes.com/by/anton-troianovski}{Anton
Troianovski}

\begin{itemize}
\item
  July 28, 2020
\item
  \begin{itemize}
  \item
  \item
  \item
  \item
  \item
  \item
  \end{itemize}
\end{itemize}

KHABAROVSK, Russia --- Valentin Kvashnikov, a construction worker and
recovering heroin addict, lives near the railway depot in a wooden
shack, with a plastic trash can in the corner that serves as his toilet.

But he has risen from obscurity into a celebrity in far eastern Russia
by helping to energize the antigovernment demonstrations that have
gotten bigger and bolder in the past three weeks.

``It's him!'' a passing woman, Natasha Gordiyenko, said after she
spotted Mr. Kvashnikov outside his house on Sunday, before unleashing a
tirade of profanity against Russian officialdom.

The protests in Khabarovsk reached well into the tens of thousands over
the weekend, establishing this distant city --- some 4,000 miles from
Moscow --- as the site of
\href{https://www.nytimes.com/2020/07/25/world/europe/russia-protests-putin-khabarovsk.html}{the
biggest popular challenge} to President Vladimir V. Putin's authority
that a city in Russia's far-flung regions has produced in his 20 years
in power.

The protests have no leader and few concrete demands. But they have
electrified a quiet city half a world away from the capital, turning
apolitical residents into activists overnight and showing how quickly
the embers of discontent over corruption, poverty and the stranglehold
of Mr. Putin's rule can ignite a conflagration.

``It's not that there is something wrong with us,'' said Elena
Okhrimenko, a retired accountant, who has been protesting with homemade
signs along with her husband, a retired truck driver. ``We realized that
there is something wrong with the country.''

The involvement of protesters from a broad cross-section of the city, an
eight-hour flight from Moscow and only 15 miles from China, is a new
kind of warning for the Kremlin. For years, large-scale protests have
mainly been limited to Moscow and St. Petersburg, making them easy to
dismiss as the work of an out-of-touch urban elite.

Yet the well of popular anger so far from the capital undercuts the
Kremlin's narrative of Mr. Putin's Russia, which he has essentially
ruled for the past two decades.

Mr. Putin won a
\href{https://www.nytimes.com/2020/07/01/world/europe/putin-referendum-vote-russia.html}{heavily
orchestrated referendum} less than a month ago that rewrote the
Constitution to allow him to stay in office until 2036. But many
analysts have called the vote fraudulent, and while pollsters have
identified rising discontent among Russians in recent years, the anger
has never spilled into the streets with such force outside the nation's
biggest cities.

``For now, society doesn't appear to be so radicalized as to storm the
gates, if you will,'' said Tatiana Stanovaya, a nonresident scholar at
the Carnegie Moscow Center, a research organization focused on politics
and policy. ``But from my point of view, that is only a question of time
if the authorities are not able to see what is really happening in the
country.''

Mr. Kvashnikov, long struggling with poverty and grousing at the state's
injustice, turned into a bullhorn-carrying cheerleader of protesters who
have marched through the city each day since July 11 in defense of their
popular governor, Sergei I. Furgal, who was arrested by the federal
authorities this month.

The protesters gather in Lenin Square in front of the marble-sheathed
hulk of the regional government headquarters --- known locally as the
White House --- before spilling into the road for a three-mile loop
above the sprawling Amur River.

Cars honk in support, drivers offer high-fives and marveling bystanders
--- the ice-cream vendor, the cosmetics shop security guard, the officer
in front of the railway-company building ---~have their phones out to
record the scene.

``I never believed our people were so united,'' Mr. Kvashnikov said,
describing the protests.

\includegraphics{https://static01.nyt.com/images/2020/07/27/world/27russia-protests2/merlin_174994851_7139618f-3c7f-4445-8496-a7f74e16f2a6-articleLarge.jpg?quality=75\&auto=webp\&disable=upscale}

The protests have drawn their ranks from political novices like Elena
Skorodumova, a 23-year-old kindergarten teacher's assistant. On July 9,
she was scrolling through a social media page devoted to local news and
pets when she
\href{https://vk.com/t_khabarovsk?w=wall-33988878_1469520}{saw a post}
about the arrest of Mr. Furgal, the governor. In a sharp blue suit, Mr.
Furgal was pictured being led away by a masked Federal Security Service
officer in camouflage gear, a gloved hand pressing down on the
governor's head.

Ms. Skorodumova recalls that she got goose bumps from her anger. The
``only way'' to support the governor, she wrote in the comments, was to
``go out in the streets.''

The arrest of the governor, on suspicion of having organized murders
some 15 years ago, seemed to many residents a blatant power play by the
Kremlin to get rid of a regional leader seen as insufficiently loyal.

Mr. Furgal, a former scrap metal trader, defeated the incumbent, a
widely disliked ally of Mr. Putin's, in the 2018 regional election. Then
Mr. Furgal won over residents with a populist style that his staff
assiduously \href{https://www.instagram.com/s.furgal/}{documented on
Instagram}.

Officially dismissed by Mr. Putin last week, Mr. Furgal had highlighted
how he set aside millions of dollars for school lunches, cut his own pay
and put the governor's yacht on the market.

More calls for protest over his arrest coursed through social media,
often in the coded language of invitations for a stroll or ``feeding the
pigeons'' in the central square.

On July 11, a Saturday, Ms. Skorodumova, the teacher's assistant, packed
sanitizing wipes and a toothbrush in case she got arrested and went to
Lenin Square. She had never protested before.

Image

The protesters widely see the arrest of Sergei I. Furgal, the governor
of Khabarovsk, as a blatant power play by the Kremlin.Credit...Evgenia
Novozhenina/Reuters

\href{https://www.nytimes.com/2020/07/10/world/europe/russian-governor-arrested-murders.html}{Tens
of thousands} of her fellow residents also came. And they keep coming
back.

Mr. Kvashnikov, the construction worker, found a wellspring of people
who shared his disdain for Mr. Putin and what he sees as a system that
enriches the few. He has scarcely enough money to eat, he said, and had
been involved in criminal groups and done time in prison in an earlier
life.

``You rabid dog, why don't you deal with what is under your own nose?''
he said of Mr. Putin. ``Your people are hungry. Look at how your people
live.''

Mr. Kvashnikov drew the attention of the many YouTubers livestreaming
the protests by his almost daily attendance, his loud chants and his
readiness to defy the police. In one
\href{https://www.youtube.com/watch?v=NxQxQICbcTU}{widely viewed video},
he can be seen shouting at a police officer that the Russian
Constitution guarantees freedom of assembly. The crowd next to him
starts chanting ``We're the ones in charge here!''

The crowds of demonstrators have grown for three consecutive Saturdays,
with some estimates putting last weekend's crowd at more than 50,000 ---
a spurt of spontaneous political activism that is rare in Russia.

Alyona Panteleyeva, 22, and her mother run a cramped sewing workshop and
fabric store. She said she had never been involved in politics until one
of her employees suggested producing face masks that say ``I Am/We Are
Furgal.''

For the past week, the workshop has been producing about 50 masks a day,
and Ms. Panteleyeva said she was selling them at cost over Instagram and
in her store. The first person who bought one, she said, paid in cash
marked with a pro-Furgal slogan; such bills are increasingly in
circulation in the city, she explained.

``I am sure that the protests will continue until the citizens get what
they want,'' including a public trial for Mr. Furgal in Khabarovsk,
rather than in Moscow, she said. ``We are fighting for the truth.''

Image

The protests have electrified a quiet city thousands of miles from the
capital.Credit...Sergey Ponomarev for The New York Times

Mr. Furgal's popularity as a regional elected official is unique, so the
Khabarovsk protests are not likely to be replicated elsewhere, the
social scientist Sergei Belanovsky
\href{https://theins.ru/opinions/belanovsky/narodnyi-gubernator}{wrote
recently}. But they show an increased willingness to protest in response
to any number of slights.

``Given the overall unfavorable economic and social situation, the
reasons to protest keep growing in number,'' Mr. Belanovsky said. ``The
fabric of the state has thinned, and to tear it requires less and less
effort.''

Mr. Putin remains in control of the country's powerful security
services, and, though in decline, his approval rating
\href{https://www.levada.ru/en/ratings/}{stands at 60 percent}. A major
question is to what extent the Kremlin will be prepared to use force to
put down protests --- it has done so in Moscow but not yet in
Khabarovsk. At one point on Monday, a sole police officer followed the
column of roughly 1,000 protesters, apparently to keep the cars at bay.

Many protesters assume that some police officers sympathize with them.
Analysts also say that the Kremlin seems to be hoping the protests will
fade on their own, and the state media has largely ignored them.
Meanwhile, the authorities seem to be putting pressure on some
activists.

Late Sunday evening in Lenin Square, videos showed Mr. Kvashnikov
haranguing a man in plainclothes who he said had threatened him, then
being wrestled to the ground by other people in plainclothes; he was
carried by his ankles, chest and elbows to a waiting police car.

Hours later, the authorities released Mr. Kvashnikov. Waiting video
bloggers were there to record his walk from the police station. Mr.
Kvashnikov had already let his fans know that he was taking a break from
protesting, for his family's safety.

``Don't be afraid and keep at it, friends,'' Mr. Kvashnikov said in a
video message recorded on Sunday. ``Most important, don't abandon what
we started together.''

Image

Along with the crowds has come a spurt of spontaneous political activism
that is rare in Russia.Credit...Sergey Ponomarev for The New York Times

Oleg Matsnev contributed research from Moscow.

Advertisement

\protect\hyperlink{after-bottom}{Continue reading the main story}

\hypertarget{site-index}{%
\subsection{Site Index}\label{site-index}}

\hypertarget{site-information-navigation}{%
\subsection{Site Information
Navigation}\label{site-information-navigation}}

\begin{itemize}
\tightlist
\item
  \href{https://help.nytimes.com/hc/en-us/articles/115014792127-Copyright-notice}{©~2020~The
  New York Times Company}
\end{itemize}

\begin{itemize}
\tightlist
\item
  \href{https://www.nytco.com/}{NYTCo}
\item
  \href{https://help.nytimes.com/hc/en-us/articles/115015385887-Contact-Us}{Contact
  Us}
\item
  \href{https://www.nytco.com/careers/}{Work with us}
\item
  \href{https://nytmediakit.com/}{Advertise}
\item
  \href{http://www.tbrandstudio.com/}{T Brand Studio}
\item
  \href{https://www.nytimes.com/privacy/cookie-policy\#how-do-i-manage-trackers}{Your
  Ad Choices}
\item
  \href{https://www.nytimes.com/privacy}{Privacy}
\item
  \href{https://help.nytimes.com/hc/en-us/articles/115014893428-Terms-of-service}{Terms
  of Service}
\item
  \href{https://help.nytimes.com/hc/en-us/articles/115014893968-Terms-of-sale}{Terms
  of Sale}
\item
  \href{https://spiderbites.nytimes.com}{Site Map}
\item
  \href{https://help.nytimes.com/hc/en-us}{Help}
\item
  \href{https://www.nytimes.com/subscription?campaignId=37WXW}{Subscriptions}
\end{itemize}
