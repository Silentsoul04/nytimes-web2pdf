Sections

SEARCH

\protect\hyperlink{site-content}{Skip to
content}\protect\hyperlink{site-index}{Skip to site index}

\href{/section/health}{Health}\textbar{}A Viral Epidemic Splintering
Into Deadly Pieces

\url{https://nyti.ms/3gaI81m}

\begin{itemize}
\item
\item
\item
\item
\item
\item
\end{itemize}

\href{https://www.nytimes.com/news-event/coronavirus?action=click\&pgtype=Article\&state=default\&region=TOP_BANNER\&context=storylines_menu}{The
Coronavirus Outbreak}

\begin{itemize}
\tightlist
\item
  live\href{https://www.nytimes.com/2020/07/29/world/coronavirus-covid-19.html?action=click\&pgtype=Article\&state=default\&region=TOP_BANNER\&context=storylines_menu}{Latest
  Updates}
\item
  \href{https://www.nytimes.com/interactive/2020/us/coronavirus-us-cases.html?action=click\&pgtype=Article\&state=default\&region=TOP_BANNER\&context=storylines_menu}{Maps
  and Cases}
\item
  \href{https://www.nytimes.com/interactive/2020/science/coronavirus-vaccine-tracker.html?action=click\&pgtype=Article\&state=default\&region=TOP_BANNER\&context=storylines_menu}{Vaccine
  Tracker}
\item
  \href{https://www.nytimes.com/interactive/2020/07/27/upshot/coronavirus-pooled-testing.html?action=click\&pgtype=Article\&state=default\&region=TOP_BANNER\&context=storylines_menu}{Understand
  Pooled Testing}
\item
  \href{https://www.nytimes.com/live/2020/07/29/business/stock-market-today-coronavirus?action=click\&pgtype=Article\&state=default\&region=TOP_BANNER\&context=storylines_menu}{Economy}
\end{itemize}

\includegraphics{https://static01.nyt.com/images/2020/07/30/science/30VIRUS-FUTURE6-jump/merlin_174438045_0ea9fc15-a773-4e95-909e-863972c145d7-articleLarge.jpg?quality=75\&auto=webp\&disable=upscale}

\hypertarget{a-viral-epidemic-splintering-into-deadly-pieces}{%
\section{A Viral Epidemic Splintering Into Deadly
Pieces}\label{a-viral-epidemic-splintering-into-deadly-pieces}}

There's not just one coronavirus outbreak in the United States. Now
there are many, each requiring its own mix of solutions.

Coronavirus testing at a site in Los Angeles last week. The pathogen has
infected at least 4.3 million Americans, killing almost
150,000.Credit...Jenna Schoenefeld for The New York Times

Supported by

\protect\hyperlink{after-sponsor}{Continue reading the main story}

\href{https://www.nytimes.com/by/donald-g-mcneil-jr}{\includegraphics{https://static01.nyt.com/images/2018/06/13/multimedia/author-donald-g-mcneil-jr/author-donald-g-mcneil-jr-thumbLarge-v4.png}}

By \href{https://www.nytimes.com/by/donald-g-mcneil-jr}{Donald G. McNeil
Jr.}

\begin{itemize}
\item
  July 29, 2020, 5:00 a.m. ET
\item
  \begin{itemize}
  \item
  \item
  \item
  \item
  \item
  \item
  \end{itemize}
\end{itemize}

Once again, the coronavirus is ascendant. As infections mount across the
country, it is dawning on Americans that the epidemic is now
unstoppable, and that no corner of the nation will be left untouched.

As of Tuesday, the pathogen had infected at least 4.3 million Americans,
killing almost 150,000. Many experts fear the virus could kill
\href{https://www.forbes.com/sites/mattperez/2020/07/07/imhe-model-projects-208255-us-deaths-by-november-but-estimate-falls-sharply-if-mask-use-increases/\#3c8ee9616f2e}{200,000}or
\href{https://www.cnbc.com/2020/07/22/dr-scott-gottlieb-us-coronavirus-deaths-may-hit-300000-by-year-end.html}{even
300,000} by year's end. Even President Trump has donned a mask, after
resisting for months, and has
\href{https://www.nytimes.com/2020/07/23/us/politics/jacksonville-rnc.html}{canceled
the Republican National Convention} celebrations in Florida.

Each state, each city has its own crisis driven by its own risk factors:
vacation crowds in one, bars reopened too soon in another, a revolt
against masks in a third.

``We are in a worse place than we were in March,'' when the virus
coursed through New York, said
\href{https://www.gwumc.edu/smhs/facultydirectory/profile.cfm?empName=Leana\%20Wen\&FacID=2073685428}{Dr.
Leana S. Wen}, a former Baltimore health commissioner. ``Back then we
had one epicenter. Now we have lots.''

To assess where the country is heading now, The New York Times
interviewed 20 public health experts --- not just clinicians and
epidemiologists, but also historians and sociologists, because the
spread of the virus is now influenced as much by human behavior as it is
by the pathogen itself.

Not only are American cities in the South and West facing deadly
outbreaks like those that struck Northeastern cities in the spring, but
\href{https://www.nytimes.com/2020/07/14/us/coronavirus-texas-rio-grande-valley-border.html}{rural
areas} are being hurt, too. In every region,
\href{https://www.nytimes.com/interactive/2020/07/05/us/coronavirus-latinos-african-americans-cdc-data.html}{people
of color will continue to suffer disproportionately}, experts said.

While there may be no appetite for a national lockdown, local
restrictions must be tightened when required, the researchers said, and
governors and mayors must have identical goals. Testing must become more
targeted.

In most states, contact tracing is now moot --- there are simply too
many cases to track.
\href{https://www.nytimes.com/interactive/2020/science/coronavirus-vaccine-tracker.html}{And
while progress has been made on vaccines}, none is expected to arrive
this winter in time to stave off what many fear will be a new wave of
deaths.

Overall, the scientists conveyed a pervasive sense of sadness and
exhaustion. Where
\href{https://www.nytimes.com/2020/03/22/health/coronavirus-restrictions-us.html}{once
there was defiance},
\href{https://www.nytimes.com/2020/04/18/health/coronavirus-america-future.html}{and
then a growing sense of dread}, now there seems to be sorrow and
frustration, a feeling that so many funerals never had to happen and
that nothing is going well. The United States is a wounded giant, while
much of Europe, which was hit first, is
\href{https://www.nytimes.com/2020/07/14/business/europe-consumer-spending.html}{recovering
and reopening} ---
\href{https://www.nytimes.com/article/eu-travel-ban-explained-usa.html}{although
not to us}.

``We're all incredibly depressed and in shock at how out of control the
virus is in the U.S.,'' said
\href{https://profiles.stanford.edu/michele-barry}{Dr. Michele Barry},
the director of the Center for Innovation in Global Health at Stanford
University.

With so much wealth and medical talent, they asked, how could we have
\href{https://www.nytimes.com/2020/07/10/us/daily-virus-death-toll-rises-in-some-states.html}{done
so poorly}? How did we fare not just worse than autocratic China and
isolated New Zealand, but also worse than tiny, much poorer nations like
Vietnam and Rwanda?

``National hubris and belief in American exceptionalism have served us
badly,'' said
\href{https://anthropology.sfsu.edu/people/faculty/martha-lincoln}{Martha
L. Lincoln}, a medical anthropologist and historian at San Francisco
State University. ``We were not prepared to see the risk of failure.''

\hypertarget{what-weve-learned}{%
\subsubsection{What We've Learned}\label{what-weve-learned}}

\includegraphics{https://static01.nyt.com/images/2020/07/30/science/30VIRUS-FUTURE3-jump/merlin_174267405_2f8e4d59-b785-4231-aea5-476014cc6306-articleLarge.jpg?quality=75\&auto=webp\&disable=upscale}

Since the coronavirus was first found to be the cause of lethal
pneumonias in Wuhan, China, in late 2019, scientists have gained a
better understanding of the enemy.

It is extremely transmissible, through not just coughed droplets but
also a fine aerosol mist that is expelled when people
\href{https://www.nytimes.com/2020/05/14/health/coronavirus-infections.html}{talk
loudly, laugh or sing} and that can
\href{https://www.nytimes.com/2020/07/04/health/239-experts-with-one-big-claim-the-coronavirus-is-airborne.html}{linger
in indoor air}. As a result,
\href{https://www.nytimes.com/2020/07/27/health/coronavirus-mask-protection.html}{masks
are far more effective} than scientists once believed.

Virus carriers with mild or no symptoms can be infectious, and there
\href{https://khn.org/morning-breakout/number-of-americans-infected-with-virus-could-be-10-times-higher-than-official-count-cdc-chief-warns/}{may
be 10 times}as many
\href{https://www.nytimes.com/2020/07/21/health/coronavirus-infections-us.html}{people
spreading the illness} as have tested positive for it.

The infection may start in the lungs, but it is very different from
influenza, a respiratory virus. In severely ill patients, the
coronavirus may attach to receptors inside the veins and arteries, and
move on to attack the kidneys, the heart, the gut and even the brain,
choking off these organs with hundreds of tiny blood clots.

Most of the virus's victims are elderly, but it
\href{https://www.nytimes.com/2020/06/25/us/coronavirus-cases-young-people.html}{has
not spared young adults}, especially those with obesity, high blood
pressure or diabetes. Adults aged 18 to 49 now
\href{https://gis.cdc.gov/grasp/COVIDNet/COVID19_5.html}{account for
more hospitalized cases} than people aged 50 to 64 or those 65 and
older.

Children are
\href{https://www.nytimes.com/2020/04/06/health/coronavirus-children-us.html}{usually
not harmed} by the virus, although clinicians were dismayed to discover
a few who were struck by a
\href{https://www.nytimes.com/2020/05/09/nyregion/coronavirus-new-york-update.html}{rare
but dangerous}
\href{https://www.nytimes.com/2020/05/13/health/coronavirus-children-kawasaki-pmis.html}{inflammatory
version}.
\href{https://www.nytimes.com/2020/02/05/health/coronavirus-children.html}{Young
children} appear to transmit the virus
\href{https://www.nytimes.com/2020/07/18/health/coronavirus-children-schools.html}{less
often than teenagers}, which may affect how schools can be opened.

Among adults, a very different picture has emerged. Growing evidence
suggests that perhaps 10 percent of the infected account for
\href{https://www.nytimes.com/2020/06/30/science/how-coronavirus-spreads.html}{80
percent of new transmissions}. Unpredictable superspreading events in
nursing homes, meatpacking plants, churches, prisons and bars are major
drivers of the epidemic.

Thus far, none of the medicines for which hopes were once high ---
repurposed malaria drugs, AIDS drugs and antivirals --- have proved to
be rapid cures. One antiviral, remdesivir,
\href{https://www.nytimes.com/2020/04/29/health/gilead-remdesivir-coronavirus.html}{has
been shown to shorten hospital stays}, while a common steroid,
dexamethasone, has
\href{https://www.nytimes.com/2020/06/16/world/europe/dexamethasone-coronavirus-covid.html}{helped
save some severely ill} patients.

\hypertarget{latest-updates-global-coronavirus-outbreak}{%
\section{\texorpdfstring{\href{https://www.nytimes.com/2020/07/29/world/coronavirus-covid-19.html?action=click\&pgtype=Article\&state=default\&region=MAIN_CONTENT_1\&context=storylines_live_updates}{Latest
Updates: Global Coronavirus
Outbreak}}{Latest Updates: Global Coronavirus Outbreak}}\label{latest-updates-global-coronavirus-outbreak}}

Updated 2020-07-29T18:24:41.028Z

\begin{itemize}
\tightlist
\item
  \href{https://www.nytimes.com/2020/07/29/world/coronavirus-covid-19.html?action=click\&pgtype=Article\&state=default\&region=MAIN_CONTENT_1\&context=storylines_live_updates\#link-1fc03c4a}{The
  virus death toll in the U.S. reaches 150,000.}
\item
  \href{https://www.nytimes.com/2020/07/29/world/coronavirus-covid-19.html?action=click\&pgtype=Article\&state=default\&region=MAIN_CONTENT_1\&context=storylines_live_updates\#link-6644b9da}{Gohmert
  tests positive for the virus (and blames a mask for it), sending
  Capitol Hill racing to trace contacts.}
\item
  \href{https://www.nytimes.com/2020/07/29/world/coronavirus-covid-19.html?action=click\&pgtype=Article\&state=default\&region=MAIN_CONTENT_1\&context=storylines_live_updates\#link-73760ee2}{Trump
  says `we really don't care' about negotiating a big recovery bill,
  instead pushing for a narrower aid package.}
\end{itemize}

\href{https://www.nytimes.com/2020/07/29/world/coronavirus-covid-19.html?action=click\&pgtype=Article\&state=default\&region=MAIN_CONTENT_1\&context=storylines_live_updates}{See
more updates}

More live coverage:
\href{https://www.nytimes.com/live/2020/07/29/business/stock-market-today-coronavirus?action=click\&pgtype=Article\&state=default\&region=MAIN_CONTENT_1\&context=storylines_live_updates}{Markets}

One or even several vaccines may be available by year's end, which would
be a spectacular achievement. But by then the virus may have in its grip
virtually every village and city on the globe.

\hypertarget{solutions-must-be-localized}{%
\subsubsection{Solutions Must Be
Localized}\label{solutions-must-be-localized}}

Image

A closed outdoor gym in Miami earlier this month. Florida and California
now have reported more coronavirus infections than New York State, once
the epicenter of the epidemic.Credit...Scott McIntyre for The New York
Times

Some experts, like Michael T. Osterholm, the director of the University
of Minnesota's Center for Infectious Disease Research and Policy, argue
that only a nationwide lockdown can completely contain the virus now.
Other researchers think that is politically impossible, but emphasize
that localities must be free to act quickly and enforce strong measures
with support from their state capitols.

\href{https://scholar.harvard.edu/danielleallen/home}{Danielle Allen},
the director of Harvard University's Edmond J. Safra Center for Ethics,
which has issued
\href{https://ethics.harvard.edu/news/path-zero-key-metrics}{pandemic
response plans}, said that finding less than one case per 100,000 people
means a community should continue testing, contact tracing and isolating
cases --- with financial support for those who need it.

Up to 25 cases per 100,000 requires greater restrictions, like closing
bars and limiting gatherings. Above that number, authorities should
issue stay-at-home orders, she said.

Testing must be focused, not just offered at convenient parking lots,
experts said, and it should be most intense in institutions like nursing
homes, prisons, factories or other places at risk of superspreading
events.

Testing must be free in places where people are poor or uninsured, such
as public housing projects, Native American reservations and churches
and grocery stores in impoverished neighborhoods.

None of this will be possible unless the nation's capacity for testing,
a continuing disaster, is greatly expanded. By the end of summer, the
administration hopes to start using ``pooling,'' in which
\href{https://www.nytimes.com/2020/07/01/health/coronavirus-pooled-testing.html}{tests
are combined in batches} to speed up the process.

But the method only works in communities with lower infection rates,
where large numbers of pooled tests turn up relatively few positive
results. It fails where the virus has spread everywhere, because too
many batches turn up positive results that require retesting.

At the moment, the United States
\href{https://www.nytimes.com/interactive/2020/us/coronavirus-testing.html}{tests
roughly 800,000 people per day}, about 38 percent of the number some
experts think is needed.

Above all, researchers said, mask use should be universal indoors ---
including airplanes, subway cars and every other enclosed space --- and
outdoors anywhere people are less than six feet apart.

Dr. Emily Landon, an infection control specialist at the University of
Chicago Pritzker School of Medicine, said it was ``sad that something as
simple as a mask got politicized.''

``It's not a statement, it's a piece of clothing,'' she added. ``You get
used to it the way you got used to wearing pants.''

Arguments that masks infringe on personal rights must be countered both
by legal orders and by persuasion. ``We need more credible messengers
endorsing masks,'' Dr. Wen said --- just before the president himself
became a messenger.

``They could include C.E.O.s or celebrities or religious leaders.
Different people are influencers to different demographics.''

Although this feels like a new debate, it is actually an old one. Masks
were
\href{https://www.history.com/news/1918-spanish-flu-mask-wearing-resistance}{common
in some Western cities} during the 1918 flu pandemic and mandatory in
San Francisco. There was
\href{https://www.pbs.org/wgbh/americanexperience/features/influenza-san-francisco/\#:~:text=One\%20of\%20the\%20more\%20highly,the\%20spread\%20of\%20flu\%20germs.\&text=A\%20Mask\%20is\%2099\%25\%20Proof,laws\%2C\%20and\%20wear\%20the\%20gauze.}{even
a jingle}: ``Obey the laws, wear the gauze. Protect your jaws from
septic paws.''

``A libertarian movement, the
\href{https://en.wikipedia.org/wiki/Anti-Mask_League_of_San_Francisco}{Anti-Mask
League}, emerged,'' Dr. Lincoln of San Francisco State said. ``There
were fistfights with police officers over it.'' Ultimately, city
officials ``waffled'' and compliance faded.

``I wonder what this issue would be like today,'' she mused, ``if that
hadn't happened.''

Images of Americans disregarding social distancing requirements have
become a daily news staple. But the pictures are deceptive: Americans
are more accepting of social distancing than the media sometimes
portrays, said
\href{https://magazine.northwestern.edu/exclusives/covid-19-impact-research/}{Beth
Redbird, a Northwestern University sociologist} who since March has
conducted \href{https://coronadata.us/data/}{regular surveys} of 8,000
adults about the impact of the virus.

``About 70 percent of Americans report using all forms of it,'' she
said. ``And when we give them adjective choices, they describe people
who won't distance as mean, selfish or unintelligent, not as generous,
open-minded or patriotic.''

The key predictor, she said in early July, was whether or not the poll
respondent trusted Mr. Trump. Those who trusted him were less likely to
practice social distancing. That was true of Republicans and
independents, ``and there's no such thing as a Democrat who trusts
Donald Trump,'' she added.

Whether or not people support coercive measures like stay-at-home orders
or bar closures depended on how scared the respondent was.

``When rising case numbers make people more afraid, they have more taste
for liberty-constraining actions,'' Dr. Redbird said. And no economic
recovery will occur, she added, ``until people aren't afraid. If they
are, they won't go out and spend money even if they're allowed to.''

\hypertarget{the-danger-indoors}{%
\subsubsection{The Danger Indoors}\label{the-danger-indoors}}

Image

Closing a bar in Houston on June 27, after Gov. Greg Abbott ordered a
partial re-closing of Texas.Credit...Erin Trieb for The New York Times

As of Tuesday, new infections were still rising in 28 states, according
to a database maintained by The Times.

Weeks ago, experts like Dr. Anthony S. Fauci, the director of the
National Institute for Allergy and Infectious Diseases,
\href{https://www.pbs.org/newshour/show/how-fauci-says-the-u-s-can-get-control-of-the-pandemic}{were
advising}states where the virus was surging to pull back from reopening
by closing down bars, forbidding large gatherings and requiring mask
usage.

Many of those states are finally taking that advice, but it is not yet
clear whether this national change of heart has happened in time to stop
the newest wave of deaths from ultimately exceeding the 2,750-a-day peak
of mid-April. Through Tuesday, the seven-day average was 1,078 virus
deaths nationwide.

Deaths may surge even higher, experts warned, when cold weather, rain
and snow force Americans to meet indoors, eat indoors and crowd into
public transit.

Oddly, states that are now hard-hit might become safer, some experts
suggested. In the South and Southwest, summers are so hot that diners
seek air-conditioning indoors, but eating outdoors in December can be
pleasant.

Several studies have confirmed transmission in air-conditioned rooms. In
one
\href{https://www.nytimes.com/2020/04/20/health/airflow-coronavirus-restaurants.html}{well-known
case cluster} in a restaurant in Guangzhou, China, researchers concluded
that air-conditioners blew around a viral cloud, infecting patrons as
far as 10 feet from a sick diner.

Rural areas face another risk. Almost 80 percent of the country's
counties lack even one infectious disease specialist, according to
\href{https://www.acpjournals.org/doi/10.7326/M20-2684}{a study} led by
\href{https://www.massgeneral.org/doctors/17245/rochelle-walensky}{Dr.
Rochelle Walensky}, the chief of infectious diseases at Massachusetts
General Hospital in Boston.

At the moment, the crisis is most acute in Southern and Southwestern
states. But \href{https://rt.live/}{websites that track transmission
rates} show that hot spots can turn up anywhere. For three weeks, for
example, Alaska's small outbreak has been one of the country's
fastest-spreading, while transmission in Texas and Arizona has
dramatically slowed.

Deaths now may rise more slowly than they did in spring, because
hospitalized patients are, on average, younger this time. But
overwhelmed hospitals can lead to
\href{https://www.nytimes.com/interactive/2020/06/01/us/coronavirus-deaths-new-york-new-jersey.html}{excess
deaths from many causes} all over a community, as ambulances are delayed
and people having health crises
\href{https://www.nytimes.com/2020/06/09/opinion/coronavirus-hospitals-deaths.html}{avoid
hospitals out of fear}.

The experts were divided as to what role influenza will play in the
fall. A harsh flu season could flood hospitals with pneumonia patients
needing ventilators. But some said the flu season could be mild or
almost nonexistent this year.

Normally, the flu virus migrates from the Northern Hemisphere to the
Southern Hemisphere in the spring --- presumably in air travelers ---
and then returns in the fall, with new mutations that may make it a poor
match for the annual vaccine.

But this year, the national lockdown abruptly ended flu transmission in
late April, according to
\href{https://www.cdc.gov/flu/weekly/index.htm}{weekly Fluview reports}
from the Centers for Disease Control and Prevention. International air
travel has been sharply curtailed, and there has been
\href{https://www.abc.net.au/news/2020-06-13/flu-cases-drop-amid-coronavirus-restrictions-statistics-show/12332204}{almost
no flu activity} in
\href{https://www.wsj.com/articles/covid-19-measures-have-all-but-wiped-out-the-flu-in-the-southern-hemisphere-11595440682}{the
whole southern hemisphere} this year.

Assuming there is still little air travel to the United States this
fall, there may be little ``reseeding'' of the flu virus here. But in
case that prediction turns out be wrong, all the researchers advised
getting flu shots anyway.

``There's no reason to be caught unprepared for two respiratory
viruses,'' said
\href{https://www.kent.edu/publichealth/profile/tara-c-smith-phd}{Tara
C. Smith}, an epidemiologist at Kent State University's School of Public
Health.

\hypertarget{partially-effective-remedies}{%
\subsubsection{Partially Effective
Remedies}\label{partially-effective-remedies}}

Image

Blood samples for coronavirus research in a lab in New York
City.Credit...Misha Friedman for The New York Times

Experts familiar with vaccine and drug manufacturing were disappointed
that, thus far, only dexamethasone and remdesivir have proved to be
effective treatments, and then only partially.

Most felt that monoclonal antibodies --- cloned human proteins that can
be grown in cell culture --- represented the best hope until vaccines
arrive.
\href{https://www.nytimes.com/2020/07/09/health/regeneron-monoclonal-antibodies.html}{Regeneron},
Eli Lilly and other drugmakers are working on candidates.

\href{https://www.nytimes.com/news-event/coronavirus?action=click\&pgtype=Article\&state=default\&region=MAIN_CONTENT_3\&context=storylines_faq}{}

\hypertarget{the-coronavirus-outbreak-}{%
\subsubsection{The Coronavirus Outbreak
›}\label{the-coronavirus-outbreak-}}

\hypertarget{frequently-asked-questions}{%
\paragraph{Frequently Asked
Questions}\label{frequently-asked-questions}}

Updated July 27, 2020

\begin{itemize}
\item ~
  \hypertarget{should-i-refinance-my-mortgage}{%
  \paragraph{Should I refinance my
  mortgage?}\label{should-i-refinance-my-mortgage}}

  \begin{itemize}
  \tightlist
  \item
    \href{https://www.nytimes.com/article/coronavirus-money-unemployment.html?action=click\&pgtype=Article\&state=default\&region=MAIN_CONTENT_3\&context=storylines_faq}{It
    could be a good idea,} because mortgage rates have
    \href{https://www.nytimes.com/2020/07/16/business/mortgage-rates-below-3-percent.html?action=click\&pgtype=Article\&state=default\&region=MAIN_CONTENT_3\&context=storylines_faq}{never
    been lower.} Refinancing requests have pushed mortgage applications
    to some of the highest levels since 2008, so be prepared to get in
    line. But defaults are also up, so if you're thinking about buying a
    home, be aware that some lenders have tightened their standards.
  \end{itemize}
\item ~
  \hypertarget{what-is-school-going-to-look-like-in-september}{%
  \paragraph{What is school going to look like in
  September?}\label{what-is-school-going-to-look-like-in-september}}

  \begin{itemize}
  \tightlist
  \item
    It is unlikely that many schools will return to a normal schedule
    this fall, requiring the grind of
    \href{https://www.nytimes.com/2020/06/05/us/coronavirus-education-lost-learning.html?action=click\&pgtype=Article\&state=default\&region=MAIN_CONTENT_3\&context=storylines_faq}{online
    learning},
    \href{https://www.nytimes.com/2020/05/29/us/coronavirus-child-care-centers.html?action=click\&pgtype=Article\&state=default\&region=MAIN_CONTENT_3\&context=storylines_faq}{makeshift
    child care} and
    \href{https://www.nytimes.com/2020/06/03/business/economy/coronavirus-working-women.html?action=click\&pgtype=Article\&state=default\&region=MAIN_CONTENT_3\&context=storylines_faq}{stunted
    workdays} to continue. California's two largest public school
    districts --- Los Angeles and San Diego --- said on July 13, that
    \href{https://www.nytimes.com/2020/07/13/us/lausd-san-diego-school-reopening.html?action=click\&pgtype=Article\&state=default\&region=MAIN_CONTENT_3\&context=storylines_faq}{instruction
    will be remote-only in the fall}, citing concerns that surging
    coronavirus infections in their areas pose too dire a risk for
    students and teachers. Together, the two districts enroll some
    825,000 students. They are the largest in the country so far to
    abandon plans for even a partial physical return to classrooms when
    they reopen in August. For other districts, the solution won't be an
    all-or-nothing approach.
    \href{https://bioethics.jhu.edu/research-and-outreach/projects/eschool-initiative/school-policy-tracker/}{Many
    systems}, including the nation's largest, New York City, are
    devising
    \href{https://www.nytimes.com/2020/06/26/us/coronavirus-schools-reopen-fall.html?action=click\&pgtype=Article\&state=default\&region=MAIN_CONTENT_3\&context=storylines_faq}{hybrid
    plans} that involve spending some days in classrooms and other days
    online. There's no national policy on this yet, so check with your
    municipal school system regularly to see what is happening in your
    community.
  \end{itemize}
\item ~
  \hypertarget{is-the-coronavirus-airborne}{%
  \paragraph{Is the coronavirus
  airborne?}\label{is-the-coronavirus-airborne}}

  \begin{itemize}
  \tightlist
  \item
    The coronavirus
    \href{https://www.nytimes.com/2020/07/04/health/239-experts-with-one-big-claim-the-coronavirus-is-airborne.html?action=click\&pgtype=Article\&state=default\&region=MAIN_CONTENT_3\&context=storylines_faq}{can
    stay aloft for hours in tiny droplets in stagnant air}, infecting
    people as they inhale, mounting scientific evidence suggests. This
    risk is highest in crowded indoor spaces with poor ventilation, and
    may help explain super-spreading events reported in meatpacking
    plants, churches and restaurants.
    \href{https://www.nytimes.com/2020/07/06/health/coronavirus-airborne-aerosols.html?action=click\&pgtype=Article\&state=default\&region=MAIN_CONTENT_3\&context=storylines_faq}{It's
    unclear how often the virus is spread} via these tiny droplets, or
    aerosols, compared with larger droplets that are expelled when a
    sick person coughs or sneezes, or transmitted through contact with
    contaminated surfaces, said Linsey Marr, an aerosol expert at
    Virginia Tech. Aerosols are released even when a person without
    symptoms exhales, talks or sings, according to Dr. Marr and more
    than 200 other experts, who
    \href{https://academic.oup.com/cid/article/doi/10.1093/cid/ciaa939/5867798}{have
    outlined the evidence in an open letter to the World Health
    Organization}.
  \end{itemize}
\item ~
  \hypertarget{what-are-the-symptoms-of-coronavirus}{%
  \paragraph{What are the symptoms of
  coronavirus?}\label{what-are-the-symptoms-of-coronavirus}}

  \begin{itemize}
  \tightlist
  \item
    Common symptoms
    \href{https://www.nytimes.com/article/symptoms-coronavirus.html?action=click\&pgtype=Article\&state=default\&region=MAIN_CONTENT_3\&context=storylines_faq}{include
    fever, a dry cough, fatigue and difficulty breathing or shortness of
    breath.} Some of these symptoms overlap with those of the flu,
    making detection difficult, but runny noses and stuffy sinuses are
    less common.
    \href{https://www.nytimes.com/2020/04/27/health/coronavirus-symptoms-cdc.html?action=click\&pgtype=Article\&state=default\&region=MAIN_CONTENT_3\&context=storylines_faq}{The
    C.D.C. has also} added chills, muscle pain, sore throat, headache
    and a new loss of the sense of taste or smell as symptoms to look
    out for. Most people fall ill five to seven days after exposure, but
    symptoms may appear in as few as two days or as many as 14 days.
  \end{itemize}
\item ~
  \hypertarget{does-asymptomatic-transmission-of-covid-19-happen}{%
  \paragraph{Does asymptomatic transmission of Covid-19
  happen?}\label{does-asymptomatic-transmission-of-covid-19-happen}}

  \begin{itemize}
  \tightlist
  \item
    So far, the evidence seems to show it does. A widely cited
    \href{https://www.nature.com/articles/s41591-020-0869-5}{paper}
    published in April suggests that people are most infectious about
    two days before the onset of coronavirus symptoms and estimated that
    44 percent of new infections were a result of transmission from
    people who were not yet showing symptoms. Recently, a top expert at
    the World Health Organization stated that transmission of the
    coronavirus by people who did not have symptoms was ``very rare,''
    \href{https://www.nytimes.com/2020/06/09/world/coronavirus-updates.html?action=click\&pgtype=Article\&state=default\&region=MAIN_CONTENT_3\&context=storylines_faq\#link-1f302e21}{but
    she later walked back that statement.}
  \end{itemize}
\end{itemize}

``They're promising both for treatment and for prophylaxis, and there
are companies with track records and manufacturing platforms,'' said
\href{http://leighbureau.com/speakers/lborio}{Dr. Luciana Borio}, a
former director of medical and biodefense preparedness at the National
Security Council. ``But manufacturing capacity is limited.''

According to a database compiled by The Times,
\href{https://www.nytimes.com/interactive/2020/science/coronavirus-vaccine-tracker.html}{researchers
worldwide are developing more than 165 vaccine candidates}, and 27 are
in human trials.

New announcements are pouring in, and the pressure to hurry is intense:
The Trump administration
\href{https://www.nytimes.com/2020/07/22/us/politics/pfizer-coronavirus-vaccine.html}{just
awarded nearly \$2 billion to a Pfizer-led consortium} that promised 100
million doses by December, assuming trials succeed.

Because the virus is still spreading rapidly, most experts said
``challenge trials,'' in which a small number of volunteers are
vaccinated and then deliberately infected, would probably not be needed.

Absent a known cure, ``challenges'' can be ethically fraught, and some
doctors
\href{https://www.statnews.com/2020/06/23/challenge-trials-live-coronavirus-speedy-covid-19-vaccine/}{oppose
doing them for}this virus. ``They don't tell you anything about
safety,'' Dr. Borio said.

And when a virus is circulating unchecked, a typical placebo-controlled
trial with up to 30,000 participants can be done efficiently, she added.
Moderna and Pfizer have already begun such trials.

The Food and Drug Administration has said a vaccine will pass muster
even
\href{https://www.washingtonpost.com/health/2020/06/30/coronavirus-vaccine-approval-fda/}{if
it is only 50 percent effective}. Experts said they could accept that,
at least initially, because the first vaccine approved could save lives
while testing continued on better alternatives.

``A vaccine doesn't have to work perfectly to be useful,'' Dr. Walensky
said. ``Even with measles vaccine, you can sometimes still get measles
--- but it's mild, and you aren't infectious.''

``We don't know if a vaccine will work in older folks. We don't know
exactly what level of herd immunity we'll need to stop the epidemic. But
anything safe and fairly effective should help.''

Still, haste is risky, experts warned, especially
\href{https://www.nytimes.com/2020/07/18/health/coronavirus-anti-vaccine.html}{when
opponents of vaccines are spreading fear}. If a vaccine is rushed
\href{https://www.nytimes.com/2020/06/08/opinion/trump-coronavirus-vaccine.html}{to
market} without thorough safety testing and recipients are hurt by it,
all vaccines could be set back for years.

\hypertarget{a-focus-on-people-of-color}{%
\subsubsection{A Focus on People of
Color}\label{a-focus-on-people-of-color}}

Image

Ayub Farah working at a drive-through testing site in Houston earlier
this month.Credit...Callaghan O'Hare for The New York Times

No matter what state the virus reaches, one risk remains constant. Even
in states with few Black and Hispanic residents,
they\href{https://www.cdc.gov/coronavirus/2019-ncov/need-extra-precautions/racial-ethnic-minorities.html}{are
usually hit hardest}, experts said.

People of color are more likely to have jobs that require physical
presence and sometimes close contact, such as construction work, store
clerking and nursing. They are more likely to rely on public transit and
to live in neighborhoods where grocery stores are scarce and crowded.

\href{https://www.nytimes.com/interactive/2020/07/05/us/coronavirus-latinos-african-americans-cdc-data.html}{They
are}
\href{https://www.pewresearch.org/fact-tank/2018/04/05/a-record-64-million-americans-live-in-multigenerational-households/}{more
likely to live in crowded housing and multigenerational homes}, some
with only one bathroom, making safe home isolation impossible when
sickness strikes. They have higher rates of obesity, high blood
pressure, diabetes and asthma.

\href{https://www.nytimes.com/interactive/2020/07/05/us/coronavirus-latinos-african-americans-cdc-data.html}{Federal
data} gathered through May 28 shows that Black and Hispanic Americans
were three times as likely to get infected as their white neighbors, and
twice as likely to die, even if they lived in remote rural counties with
few Black or Hispanic residents.

``By the time that minority patient sets foot in a hospital, he is
already on an unequal footing,'' said
\href{https://www.newswise.com/coronavirus/iu-professor-available-to-discuss-social-bias-and-inequality-in-covid-crisis/?article_id=729760}{Elaine
Hernandez}, a sociologist at Indiana University.

The differences persist even though Black and Hispanic adults
drastically altered their behavior. One study found that through the
beginning of May, the average Black American
\href{https://www.medrxiv.org/content/10.1101/2020.06.04.20119131v1}{practiced
more social distancing} than the average white American.

Officials in
\href{https://blockclubchicago.org/2020/04/07/black-people-are-not-immune-to-coronavirus-debunking-deadly-social-media-myths/}{Chicago},
\href{https://www.baltimoresun.com/coronavirus/bs-md-baltimore-coronavirus-black-messaging-testing-20200414-cgqbwz6cmffabel364ixfktlje-story.html}{Baltimore}
and other communities faced another threat: rumors flying about social
media that Black people were somehow immune.

The top factor making people adopt self-protective behavior is
personally knowing someone who fell ill, said Dr. Redbird. By the end of
spring, Black and Hispanic Americans were 50 percent more likely than
white Americans to know someone who had been ill, her surveys found.

Dr. Hernandez, whose parents live in Arizona, said their neighbors who
had not been scared in June had since changed their attitudes.

Her father, a physician, had set an example. Early on, he wore a mask
with a silly mustache when he and his wife took walks, and they would
decline friends' invitations, saying, ``No, we're staying in our
bubble.''

Now, she said, their neighbors are wearing masks, ``and people are
telling my father, `You were right,''' Dr. Hernandez said.

\hypertarget{this-is-the-beginning}{%
\subsubsection{This Is the Beginning}\label{this-is-the-beginning}}

Image

A line for coronavirus testing in Atlanta on July 6.Credit...Dustin
Chambers for The New York Times

There was no widespread agreement among experts about what is likely to
happen in the years after the pandemic. Some scientists expected a quick
economic recovery; others thought the damage could persist for years.

Working at home will become more common, some predicted, while crowded,
open-plan offices may be changed. The just-in-time supply chains on
which many businesses depend will need fixing because the processes
failed to deliver adequate protective gear, ventilators and test
materials.

A disease-modeling system like that used by the
\href{https://www.weather.gov/}{National Weather Service} to predict
storms is needed, said
\href{https://www.centerforhealthsecurity.org/our-people/rivers/}{Caitlin
Rivers}, an epidemiologist at the Johns Hopkins Center for Health
Security. Right now, the country has surveillance for seasonal flu but
no national map tracking all disease outbreaks. As Dr. Thomas R.
Frieden, a former C.D.C. director, recently pointed out, states are
\href{https://www.nytimes.com/2020/07/21/health/coronavirus-data-states-cdc.html}{not
even required to track} the same data.

Several experts said they assumed that millions of Americans who have
been left without health insurance or forced to line up at food banks
would vote for politicians favoring universal health care, paid sick
leave, greater income equality and other changes.

But given the country's deep political divisions, no researcher was
certain what the outcome of the coming election would be.

Dr. Redbird said her polling of Americans showed ``little faith in
institutions across the board --- we're not seeing an increase in trust
in science or an appetite for universal health care or workers equity.''

The Trump administration did little to earn trust. More than six months
into the worst health crisis in a century, Mr. Trump only last week
urged Americans to wear masks and canceled the Republican convention in
Florida, the kind of high-risk indoor event that states have been
banning since mid-March.

``It will probably, unfortunately, get worse before it gets better,''
Mr. Trump said at the first of the resurrected coronavirus task force
briefings earlier this month, which included no scientists or health
officials. The briefings were discontinued in April amid his rosy
predications that the epidemic would soon be over.

Mr. Trump has
\href{https://www.nytimes.com/2020/03/23/us/politics/coronavirus-trump-fauci.html}{ignored},
\href{https://www.youtube.com/watch?v=yYhriqvJMSw}{contradicted} or
\href{https://www.cnn.com/2020/07/19/politics/trump-fauci-alarmist-coronavirus/index.html}{disparaged}
his scientific advisers, repeatedly saying that the virus simply would
go away, touting unproven drugs like hydroxychloroquine even after they
were shown to be ineffective and sometimes dangerous, and suggesting
that disinfectants or lethal ultraviolet light might be used inside the
body.

Millions of Americans have
\href{https://www.nytimes.com/live/2020/07/23/business/stock-market-today-coronavirus\#roughly-one-in-five-workers-are-collecting-unemployment-benefits}{lost
their jobs} and
\href{https://www.nytimes.com/2020/07/13/us/politics/coronavirus-health-insurance-trump.html}{their
health insurance}, and are in danger of
\href{https://www.nytimes.com/2020/07/23/opinion/coronavirus-evictions-rent.html}{losing
their homes,}even as they find themselves in the path of a lethal
disease. The Trump presidency ``is the symptom of the denigration of
science and the gutting of the public contract about what we owe each
other as citizens,'' said
\href{https://ghsm.hms.harvard.edu/faculty-staff/joia-stapleton-mukherjee}{Dr.
Joia S. Mukherjee}, the chief medical officer of Partners in Health in
Boston.

One lesson that will surely be learned is that the country needs to be
better prepared for microbial assaults, said Dr. Julie Gerberding, a
former director of the C.D.C.

``This is not a once-in-a-century event. It's a harbinger of things to
come.''

Image

Social-distancing signage at Coney Island in Brooklyn in
May.Credit...Brittainy Newman/The New York Times

Advertisement

\protect\hyperlink{after-bottom}{Continue reading the main story}

\hypertarget{site-index}{%
\subsection{Site Index}\label{site-index}}

\hypertarget{site-information-navigation}{%
\subsection{Site Information
Navigation}\label{site-information-navigation}}

\begin{itemize}
\tightlist
\item
  \href{https://help.nytimes.com/hc/en-us/articles/115014792127-Copyright-notice}{©~2020~The
  New York Times Company}
\end{itemize}

\begin{itemize}
\tightlist
\item
  \href{https://www.nytco.com/}{NYTCo}
\item
  \href{https://help.nytimes.com/hc/en-us/articles/115015385887-Contact-Us}{Contact
  Us}
\item
  \href{https://www.nytco.com/careers/}{Work with us}
\item
  \href{https://nytmediakit.com/}{Advertise}
\item
  \href{http://www.tbrandstudio.com/}{T Brand Studio}
\item
  \href{https://www.nytimes.com/privacy/cookie-policy\#how-do-i-manage-trackers}{Your
  Ad Choices}
\item
  \href{https://www.nytimes.com/privacy}{Privacy}
\item
  \href{https://help.nytimes.com/hc/en-us/articles/115014893428-Terms-of-service}{Terms
  of Service}
\item
  \href{https://help.nytimes.com/hc/en-us/articles/115014893968-Terms-of-sale}{Terms
  of Sale}
\item
  \href{https://spiderbites.nytimes.com}{Site Map}
\item
  \href{https://help.nytimes.com/hc/en-us}{Help}
\item
  \href{https://www.nytimes.com/subscription?campaignId=37WXW}{Subscriptions}
\end{itemize}
