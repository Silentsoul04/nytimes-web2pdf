Sections

SEARCH

\protect\hyperlink{site-content}{Skip to
content}\protect\hyperlink{site-index}{Skip to site index}

\href{https://myaccount.nytimes.com/auth/login?response_type=cookie\&client_id=vi}{}

\href{https://www.nytimes.com/section/todayspaper}{Today's Paper}

`I May Destroy You' Is Perfect TV for an Anxious World

\url{https://nyti.ms/2P7veFv}

\begin{itemize}
\item
\item
\item
\item
\item
\item
\end{itemize}

Advertisement

\protect\hyperlink{after-top}{Continue reading the main story}

Supported by

\protect\hyperlink{after-sponsor}{Continue reading the main story}

\href{/column/screenland}{Screenland}

\hypertarget{i-may-destroy-you-is-perfect-tv-for-an-anxious-world}{%
\section{`I May Destroy You' Is Perfect TV for an Anxious
World}\label{i-may-destroy-you-is-perfect-tv-for-an-anxious-world}}

\includegraphics{https://static01.nyt.com/images/2020/08/02/magazine/02mag-screenland-1/02mag-screenland-1-articleLarge.jpg?quality=75\&auto=webp\&disable=upscale}

By Carina Chocano

\begin{itemize}
\item
  July 29, 2020
\item
  \begin{itemize}
  \item
  \item
  \item
  \item
  \item
  \item
  \end{itemize}
\end{itemize}

The sixth episode of HBO's ``I May Destroy You'' opens on a bucolic
urban scene: a park under a train overpass, where three young friends
are attempting some plein-air spray-painting on canvas. Terry and Kwame
have come to support Arabella, whose therapist has recommended painting
to help her heal from a recent sexual assault. But Arabella doesn't
paint. She stands apart from her friends, engrossed in her phone. Terry,
concerned, points this out to Kwame, but he shrugs: Arabella looks fine
to him.

This sets Terry off. She launches into a monologue about how trauma acts
on the body, overwhelming the nervous system and causing it to shut down
for safety. ``She's not fine,'' Terry says, as Kwame stares at her
blankly, flinching occasionally. ``She's vacant, she's empty. She's a
shell of herself. She's dying inside. But if you aren't looking for it,
you ain't gonna see it.'' The irony here is that Terry is proving her
own point: She delivers this lecture without ever noticing that Kwame is
exhibiting the very same behavior. He, too, was recently raped.

``I May Destroy You,'' created by the British-Ghanaian writer and actor
Michaela Coel, has been described as a drama about consent, but mostly
it's a show about trauma --- how mutable and contagious it is, how
insidious and pervasive. The story doesn't build so much as it burrows,
digging into crevices to reveal an infinite regress of damage. With each
new trauma its characters endure, another is set off, or uncovered, or
recalled, revealing a system of abuse so ubiquitous, so normalized as to
be invisible, hiding in plain sight.

Arabella, an up-and-coming East London author of Ghanaian descent,
starts the series trying to avoid a looming book deadline. The night
before her draft is due, she decides to meet up with a friend, and she's
at a bar with him --- she thinks --- when somebody drugs her drink and
rapes her in a toilet stall. She wakes from her fugue with a cut on her
forehead, a smashed phone and no memory of how she made it back to her
publisher's office. Soon, despite her best efforts to repress her
feelings, she is suffering from classic symptoms of PTSD --- flashbacks
and intrusive thoughts, hyperarousal and insomnia, avoidance and
withdrawal. She even disavows her own memories of the event, describing
them to the police as images in her head that don't belong there.

\includegraphics{https://static01.nyt.com/images/2020/08/02/magazine/02mag-screenland-image-2/02mag-screenland-image-2-articleLarge.png?quality=75\&auto=webp\&disable=upscale}

The carefree, independent sense of herself Arabella is trying to protect
--- the safe, salable self she's carefully constructed and put forth in
a book called ``Chronicles of a Fed-Up Millennial'' --- is perhaps not
as solid or secure in the world as she would like to believe. Her
beloved friends are not always trustworthy. We learn that she is
estranged from her family. Her long-distance boyfriend --- sweet but
traumatized himself --- refuses to talk about their relationship. After
the assault, her anxious editors pay for therapy and hire a more
established writer to help with the book, but he resents and belittles
Arabella's success, which he sees as fluky and undeserved. (He went to
Cambridge, while she got a book deal based on a popular Twitter
account.) He ends up raping her himself, then gaslighting her into
thinking he hasn't --- which she nearly goes along with, because she,
too, wants to believe everything is fine.

The person Arabella is texting during that spray-painting session opens
up the door into an especially fraught chain of guilt, complicity and
emotional damage. It's a former classmate, a white woman named Theo, who
has formed a support group for survivors of sexual abuse, which Arabella
joins. In high school, we learn, Theo was incensed when the Black
classmate she thought was her boyfriend took her picture during sex and,
when she asked him to delete it, offered her money instead. She then
falsely accused him of trying to rape her --- an echo of the lie her
mother once forced her to tell about her father during a custody battle.

\textbf{All this pinballing} of trauma is not just confined to the world
of interpersonal relationships. Six episodes in, Arabella is coming to
understand how trauma works not just on the body, but on the body
politic --- how it ricochets through populations and generations,
transforming everything it touches, revealing the world to be a scarier
and more complex place than she had allowed herself to imagine. ``I May
Destroy You'' is about consent in the sexual sense, yes. But it is also
about the broader sense, the one that encompasses any proposal, desire
or situation we are asked to agree to --- negotiations that grow
complicated in a society whose norms don't favor everyone equally, and
where your standing can be shifting and unstable. We talk about cultures
of abuse, but this show is about nothing less than what it's like to
live in an abusive culture: a system of dominance in which almost no one
is safe, in which everyone's trust is violated, in all kinds of ways,
all the time.

To be a person in this world is to be subjected to all sorts of unwanted
desires, expectations, rules and systems of coercion. Our bodies ---
more so for some of us than for others --- are not entirely our own, a
reality the overlapping horrors of 2020 have laid especially bare. Strip
the veil of familiarity off the world, as Percy Bysshe Shelley once put
it, and you expose a dark map of corruption, abuse, predation and
precarity underneath the veneer of civility. The threats Americans feel
right now, both real and perceived, act on us like trauma: As a nation,
as a social body, we're activated, hypervigilant, anxious, triggered.
We're exhibiting all the symptoms of complex PTSD.

In that sense, ``I May Destroy You'' is perfectly suited to the moment;
it is possibly the most emblematic show of 2020. It examines how, by
avoiding the truth, we pass fear and suffering on to others. It reminds
us that everyone is vulnerable, that nobody is entirely above avoidance
or self-delusion. It makes the case for facing even those truths that,
when confronted, might reveal an altogether different reality from the
one we thought we inhabited. But as Terry tells her friend: If you
aren't looking for it, you ain't gonna see it.

In an earlier episode of ``I May Destroy You,'' Arabella tracks down
someone else who was with her the night of the assault: Alissa, whom her
partnered friend Simon has been seeing on the side. Alissa is sure her
drink was drugged as well, but when Arabella suggests that Simon may
have had something to do with it, Alissa explodes, calling Arabella
crazy. Her image of Simon as safe and trustworthy trumps her own bodily
experience; the alternative is too overwhelming, too annihilating to
handle. It's hard to confront the truth when it forces us to re-evaluate
everything we think we know about who and what we are. We struggle with
this every day. We run away and avoid it. It may destroy us.

Advertisement

\protect\hyperlink{after-bottom}{Continue reading the main story}

\hypertarget{site-index}{%
\subsection{Site Index}\label{site-index}}

\hypertarget{site-information-navigation}{%
\subsection{Site Information
Navigation}\label{site-information-navigation}}

\begin{itemize}
\tightlist
\item
  \href{https://help.nytimes.com/hc/en-us/articles/115014792127-Copyright-notice}{©~2020~The
  New York Times Company}
\end{itemize}

\begin{itemize}
\tightlist
\item
  \href{https://www.nytco.com/}{NYTCo}
\item
  \href{https://help.nytimes.com/hc/en-us/articles/115015385887-Contact-Us}{Contact
  Us}
\item
  \href{https://www.nytco.com/careers/}{Work with us}
\item
  \href{https://nytmediakit.com/}{Advertise}
\item
  \href{http://www.tbrandstudio.com/}{T Brand Studio}
\item
  \href{https://www.nytimes.com/privacy/cookie-policy\#how-do-i-manage-trackers}{Your
  Ad Choices}
\item
  \href{https://www.nytimes.com/privacy}{Privacy}
\item
  \href{https://help.nytimes.com/hc/en-us/articles/115014893428-Terms-of-service}{Terms
  of Service}
\item
  \href{https://help.nytimes.com/hc/en-us/articles/115014893968-Terms-of-sale}{Terms
  of Sale}
\item
  \href{https://spiderbites.nytimes.com}{Site Map}
\item
  \href{https://help.nytimes.com/hc/en-us}{Help}
\item
  \href{https://www.nytimes.com/subscription?campaignId=37WXW}{Subscriptions}
\end{itemize}
