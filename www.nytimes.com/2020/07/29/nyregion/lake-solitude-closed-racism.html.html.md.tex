Sections

SEARCH

\protect\hyperlink{site-content}{Skip to
content}\protect\hyperlink{site-index}{Skip to site index}

\href{https://www.nytimes.com/section/nyregion}{New York}

\href{https://myaccount.nytimes.com/auth/login?response_type=cookie\&client_id=vi}{}

\href{https://www.nytimes.com/section/todayspaper}{Today's Paper}

\href{/section/nyregion}{New York}\textbar{}`Hidden Gem' Made Popular by
TikTok Is Shut to Keep Out-of-Towners Away

\url{https://nyti.ms/3f77NGY}

\begin{itemize}
\item
\item
\item
\item
\item
\item
\end{itemize}

\href{https://www.nytimes.com/news-event/coronavirus?action=click\&pgtype=Article\&state=default\&region=TOP_BANNER\&context=storylines_menu}{The
Coronavirus Outbreak}

\begin{itemize}
\tightlist
\item
  live\href{https://www.nytimes.com/2020/08/04/world/coronavirus-cases.html?action=click\&pgtype=Article\&state=default\&region=TOP_BANNER\&context=storylines_menu}{Latest
  Updates}
\item
  \href{https://www.nytimes.com/interactive/2020/us/coronavirus-us-cases.html?action=click\&pgtype=Article\&state=default\&region=TOP_BANNER\&context=storylines_menu}{Maps
  and Cases}
\item
  \href{https://www.nytimes.com/interactive/2020/science/coronavirus-vaccine-tracker.html?action=click\&pgtype=Article\&state=default\&region=TOP_BANNER\&context=storylines_menu}{Vaccine
  Tracker}
\item
  \href{https://www.nytimes.com/2020/08/02/us/covid-college-reopening.html?action=click\&pgtype=Article\&state=default\&region=TOP_BANNER\&context=storylines_menu}{College
  Reopening}
\item
  \href{https://www.nytimes.com/live/2020/08/04/business/stock-market-today-coronavirus?action=click\&pgtype=Article\&state=default\&region=TOP_BANNER\&context=storylines_menu}{Economy}
\end{itemize}

Advertisement

\protect\hyperlink{after-top}{Continue reading the main story}

Supported by

\protect\hyperlink{after-sponsor}{Continue reading the main story}

\hypertarget{hidden-gem-made-popular-by-tiktok-is-shut-to-keep-out-of-towners-away}{%
\section{`Hidden Gem' Made Popular by TikTok Is Shut to Keep
Out-of-Towners
Away}\label{hidden-gem-made-popular-by-tiktok-is-shut-to-keep-out-of-towners-away}}

A lake in New Jersey was closed to curb the spread of the coronavirus,
but some complaints about recent crowding there focused on the ethnicity
of visitors.

\includegraphics{https://static01.nyt.com/images/2020/07/27/nyregion/00LakeSolitude-1/00LakeSolitude-1-articleLarge-v2.jpg?quality=75\&auto=webp\&disable=upscale}

\href{https://www.nytimes.com/by/sarah-maslin-nir}{\includegraphics{https://static01.nyt.com/images/2018/06/13/multimedia/author-sarah-maslin-nir/author-sarah-maslin-nir-thumbLarge.jpg}}

By \href{https://www.nytimes.com/by/sarah-maslin-nir}{Sarah Maslin Nir}

\begin{itemize}
\item
  July 29, 2020
\item
  \begin{itemize}
  \item
  \item
  \item
  \item
  \item
  \item
  \end{itemize}
\end{itemize}

HIGH BRIDGE, N.J. --- Lake Solitude it was not.

For years, the 35-acre picturesque lake, waterfall and century-old dam
had been an unspoiled treasure for local residents, but through the
power of social media, the secret got out.

People began pouring in, bringing portable speakers, children and food,
and leaving behind trash. Borough officials installed extra garbage cans
and portable toilets, and brought in police officers to direct traffic
--- many vehicles with New York license plates. On a recent Sunday, some
visitors had to be turned away.

Residents had seen enough. They swarmed a virtual town-hall-style
meeting this month, demanding that Lake Solitude be shut. Last week, the
Borough of High Bridge complied, closing the area to all visitors.

Some of the complaints stemmed from fears that visitors might bring the
coronavirus from New York City, about 50 miles east of the borough. But
some of them focused bluntly on ethnicity.

\includegraphics{https://static01.nyt.com/images/2020/07/27/nyregion/00LakeSolitude-2/merlin_174760581_c1b61b5d-ecff-41cb-9df6-f258b0747b93-articleLarge.jpg?quality=75\&auto=webp\&disable=upscale}

High Bridge is nearly 95 percent white, but the lake attracts a much
more diverse group.

On the last weekend that Lake Solitude was open, mothers and fathers
were paddling with toddlers underneath the waterfall's spray,
grandmothers were basking with their feet in the bucolic river and
20-somethings were taking drone photos with the imposing dam.

``We have droves of out-of-state Spanish people and they leave their
crap lying on the ground,'' said Lester Tomson, 58, who regularly fished
the stream.

Mr. Tomson, a registered Democrat, is one of a number of people who, on
social media and in conversation, have suggested that Immigration and
Customs Enforcement should have been called to the park.

``It's not a racist thing,'' he said in an interview. ``It's a thing
where you observe things, and your observations are based in facts and
not in racism.''

Lake Solitude is one of several normally quiet oases for locals that
have been recently overrun by day trippers from New York City ---
\href{https://www.nytimes.com/2020/07/03/nyregion/beaches-open-nyc.html}{where
public beaches} and pools were mostly closed until this month --- who
are looking for closer places to visit because of the pandemic.

Image

Many of the recent visitors to Lake Solitude have arrived in vehicles
with New York license plates.~Credit...Bryan Anselm for The New York
Times

In Woodstock, N.Y., about two hours north of Manhattan, town officials
said they had to
\href{https://hudsonvalleyone.com/2020/07/08/woodstock-cracks-down-use-of-closed-swimming-holes/}{shut
their popular Big Deep and Little Deep} swimming holes because of the
``littering and messes left behind by visitors'' that made it difficult
to ``maintain safety during the pandemic.''

Elsewhere in the Catskills, more than 300 people attended a
town-hall-style meeting held outdoors last week to complain about the
traffic and trash brought by outsiders at Kaaterskill Falls, especially
at its popular swimming holes, Fawns Leap and Dog Hole.

\hypertarget{latest-updates-global-coronavirus-outbreak}{%
\section{\texorpdfstring{\href{https://www.nytimes.com/2020/08/04/world/coronavirus-cases.html?action=click\&pgtype=Article\&state=default\&region=MAIN_CONTENT_1\&context=storylines_live_updates}{Latest
Updates: Global Coronavirus
Outbreak}}{Latest Updates: Global Coronavirus Outbreak}}\label{latest-updates-global-coronavirus-outbreak}}

Updated 2020-08-05T07:58:24.076Z

\begin{itemize}
\tightlist
\item
  \href{https://www.nytimes.com/2020/08/04/world/coronavirus-cases.html?action=click\&pgtype=Article\&state=default\&region=MAIN_CONTENT_1\&context=storylines_live_updates\#link-762df92}{As
  talks drag on, McConnell signals openness to jobless aid extension,
  and negotiators agree on a deadline.}
\item
  \href{https://www.nytimes.com/2020/08/04/world/coronavirus-cases.html?action=click\&pgtype=Article\&state=default\&region=MAIN_CONTENT_1\&context=storylines_live_updates\#link-1228a480}{Novavax
  sees encouraging results from two studies of its experimental
  vaccine.}
\item
  \href{https://www.nytimes.com/2020/08/04/world/coronavirus-cases.html?action=click\&pgtype=Article\&state=default\&region=MAIN_CONTENT_1\&context=storylines_live_updates\#link-794484ed}{Mississippians
  must now wear masks in public, governor says.}
\end{itemize}

\href{https://www.nytimes.com/2020/08/04/world/coronavirus-cases.html?action=click\&pgtype=Article\&state=default\&region=MAIN_CONTENT_1\&context=storylines_live_updates}{See
more updates}

More live coverage:
\href{https://www.nytimes.com/live/2020/08/04/business/stock-market-today-coronavirus?action=click\&pgtype=Article\&state=default\&region=MAIN_CONTENT_1\&context=storylines_live_updates}{Markets}

Greene County, home to Kaaterskill Falls, is over 96 percent white. But
Daryl Legg, the town supervisor of Hunter, where the falls are, rejected
the idea that race had any part to play in the complaints.

``People come here for the scenery and beauty of the place,'' he said,
``but leave red Solo cups at the bottom of the swim hole, and people
defecate and pee in the woods and it smells like a latrine after
Woodstock.''

But the tensions have been especially sharp in New Jersey, where state
officials in April gave municipalities broad discretion to close public
parks as a way of curbing the coronavirus.

In Ocean County, borough officials in Lakehurst cited overcrowding in
their decision to
\href{https://nj1015.com/lake-in-nj-closes-after-one-day-over-crowds-not-social-distancing/}{shut
Lake Horicon} to visitors in May, a day after the state had reopened
county and state parks. In Long Branch, in Monmouth County, the police
temporarily blocked beach access this month after a deluge of beachgoers
made social distancing impossible.

In High Bridge, the decision last week to close Lake Solitude was made
for health precautions and because the parking lots were at capacity,
according to the borough's mayor, Michele Lee.

But three days earlier, during the virtual meeting, some public comments
cited other reasons: One man expressed that he felt unsafe after a male
visitor to the lake said, ``Hola, señorita,'' to his wife.

Image

An unattended cooler on the lake's grounds. Earlier this year, state
officials had given municipalities broad discretion to close public
parks as a way of curbing the coronavirus.Credit...Bryan Anselm for The
New York Times

``We are an inclusive community. We are going to be accepting of
everybody, regardless of race or faith or who you love,'' Mayor Lee, a
Democrat, said. ``We did what we have to do because it was really
becoming a safety concern.''

The mayor said that the crowds grew drastically after a
\href{https://www.tiktok.com/@dkaur171/video/6845337773726043398}{TikTok
video} that called the site a ``hidden gem'' went viral.

\href{https://www.nytimes.com/news-event/coronavirus?action=click\&pgtype=Article\&state=default\&region=MAIN_CONTENT_3\&context=storylines_faq}{}

\hypertarget{the-coronavirus-outbreak-}{%
\subsubsection{The Coronavirus Outbreak
›}\label{the-coronavirus-outbreak-}}

\hypertarget{frequently-asked-questions}{%
\paragraph{Frequently Asked
Questions}\label{frequently-asked-questions}}

Updated August 4, 2020

\begin{itemize}
\item ~
  \hypertarget{i-have-antibodies-am-i-now-immune}{%
  \paragraph{I have antibodies. Am I now
  immune?}\label{i-have-antibodies-am-i-now-immune}}

  \begin{itemize}
  \tightlist
  \item
    As of right
    now,\href{https://www.nytimes.com/2020/07/22/health/covid-antibodies-herd-immunity.html?action=click\&pgtype=Article\&state=default\&region=MAIN_CONTENT_3\&context=storylines_faq}{that
    seems likely, for at least several months.} There have been
    frightening accounts of people suffering what seems to be a second
    bout of Covid-19. But experts say these patients may have a
    drawn-out course of infection, with the virus taking a slow toll
    weeks to months after initial exposure. People infected with the
    coronavirus typically
    \href{https://www.nature.com/articles/s41586-020-2456-9}{produce}
    immune molecules called antibodies, which are
    \href{https://www.nytimes.com/2020/05/07/health/coronavirus-antibody-prevalence.html?action=click\&pgtype=Article\&state=default\&region=MAIN_CONTENT_3\&context=storylines_faq}{protective
    proteins made in response to an
    infection}\href{https://www.nytimes.com/2020/05/07/health/coronavirus-antibody-prevalence.html?action=click\&pgtype=Article\&state=default\&region=MAIN_CONTENT_3\&context=storylines_faq}{.
    These antibodies may} last in the body
    \href{https://www.nature.com/articles/s41591-020-0965-6}{only two to
    three months}, which may seem worrisome, but that's perfectly normal
    after an acute infection subsides, said Dr. Michael Mina, an
    immunologist at Harvard University. It may be possible to get the
    coronavirus again, but it's highly unlikely that it would be
    possible in a short window of time from initial infection or make
    people sicker the second time.
  \end{itemize}
\item ~
  \hypertarget{im-a-small-business-owner-can-i-get-relief}{%
  \paragraph{I'm a small-business owner. Can I get
  relief?}\label{im-a-small-business-owner-can-i-get-relief}}

  \begin{itemize}
  \tightlist
  \item
    The
    \href{https://www.nytimes.com/article/small-business-loans-stimulus-grants-freelancers-coronavirus.html?action=click\&pgtype=Article\&state=default\&region=MAIN_CONTENT_3\&context=storylines_faq}{stimulus
    bills enacted in March} offer help for the millions of American
    small businesses. Those eligible for aid are businesses and
    nonprofit organizations with fewer than 500 workers, including sole
    proprietorships, independent contractors and freelancers. Some
    larger companies in some industries are also eligible. The help
    being offered, which is being managed by the Small Business
    Administration, includes the Paycheck Protection Program and the
    Economic Injury Disaster Loan program. But lots of folks have
    \href{https://www.nytimes.com/interactive/2020/05/07/business/small-business-loans-coronavirus.html?action=click\&pgtype=Article\&state=default\&region=MAIN_CONTENT_3\&context=storylines_faq}{not
    yet seen payouts.} Even those who have received help are confused:
    The rules are draconian, and some are stuck sitting on
    \href{https://www.nytimes.com/2020/05/02/business/economy/loans-coronavirus-small-business.html?action=click\&pgtype=Article\&state=default\&region=MAIN_CONTENT_3\&context=storylines_faq}{money
    they don't know how to use.} Many small-business owners are getting
    less than they expected or
    \href{https://www.nytimes.com/2020/06/10/business/Small-business-loans-ppp.html?action=click\&pgtype=Article\&state=default\&region=MAIN_CONTENT_3\&context=storylines_faq}{not
    hearing anything at all.}
  \end{itemize}
\item ~
  \hypertarget{what-are-my-rights-if-i-am-worried-about-going-back-to-work}{%
  \paragraph{What are my rights if I am worried about going back to
  work?}\label{what-are-my-rights-if-i-am-worried-about-going-back-to-work}}

  \begin{itemize}
  \tightlist
  \item
    Employers have to provide
    \href{https://www.osha.gov/SLTC/covid-19/standards.html}{a safe
    workplace} with policies that protect everyone equally.
    \href{https://www.nytimes.com/article/coronavirus-money-unemployment.html?action=click\&pgtype=Article\&state=default\&region=MAIN_CONTENT_3\&context=storylines_faq}{And
    if one of your co-workers tests positive for the coronavirus, the
    C.D.C.} has said that
    \href{https://www.cdc.gov/coronavirus/2019-ncov/community/guidance-business-response.html}{employers
    should tell their employees} -\/- without giving you the sick
    employee's name -\/- that they may have been exposed to the virus.
  \end{itemize}
\item ~
  \hypertarget{should-i-refinance-my-mortgage}{%
  \paragraph{Should I refinance my
  mortgage?}\label{should-i-refinance-my-mortgage}}

  \begin{itemize}
  \tightlist
  \item
    \href{https://www.nytimes.com/article/coronavirus-money-unemployment.html?action=click\&pgtype=Article\&state=default\&region=MAIN_CONTENT_3\&context=storylines_faq}{It
    could be a good idea,} because mortgage rates have
    \href{https://www.nytimes.com/2020/07/16/business/mortgage-rates-below-3-percent.html?action=click\&pgtype=Article\&state=default\&region=MAIN_CONTENT_3\&context=storylines_faq}{never
    been lower.} Refinancing requests have pushed mortgage applications
    to some of the highest levels since 2008, so be prepared to get in
    line. But defaults are also up, so if you're thinking about buying a
    home, be aware that some lenders have tightened their standards.
  \end{itemize}
\item ~
  \hypertarget{what-is-school-going-to-look-like-in-september}{%
  \paragraph{What is school going to look like in
  September?}\label{what-is-school-going-to-look-like-in-september}}

  \begin{itemize}
  \tightlist
  \item
    It is unlikely that many schools will return to a normal schedule
    this fall, requiring the grind of
    \href{https://www.nytimes.com/2020/06/05/us/coronavirus-education-lost-learning.html?action=click\&pgtype=Article\&state=default\&region=MAIN_CONTENT_3\&context=storylines_faq}{online
    learning},
    \href{https://www.nytimes.com/2020/05/29/us/coronavirus-child-care-centers.html?action=click\&pgtype=Article\&state=default\&region=MAIN_CONTENT_3\&context=storylines_faq}{makeshift
    child care} and
    \href{https://www.nytimes.com/2020/06/03/business/economy/coronavirus-working-women.html?action=click\&pgtype=Article\&state=default\&region=MAIN_CONTENT_3\&context=storylines_faq}{stunted
    workdays} to continue. California's two largest public school
    districts --- Los Angeles and San Diego --- said on July 13, that
    \href{https://www.nytimes.com/2020/07/13/us/lausd-san-diego-school-reopening.html?action=click\&pgtype=Article\&state=default\&region=MAIN_CONTENT_3\&context=storylines_faq}{instruction
    will be remote-only in the fall}, citing concerns that surging
    coronavirus infections in their areas pose too dire a risk for
    students and teachers. Together, the two districts enroll some
    825,000 students. They are the largest in the country so far to
    abandon plans for even a partial physical return to classrooms when
    they reopen in August. For other districts, the solution won't be an
    all-or-nothing approach.
    \href{https://bioethics.jhu.edu/research-and-outreach/projects/eschool-initiative/school-policy-tracker/}{Many
    systems}, including the nation's largest, New York City, are
    devising
    \href{https://www.nytimes.com/2020/06/26/us/coronavirus-schools-reopen-fall.html?action=click\&pgtype=Article\&state=default\&region=MAIN_CONTENT_3\&context=storylines_faq}{hybrid
    plans} that involve spending some days in classrooms and other days
    online. There's no national policy on this yet, so check with your
    municipal school system regularly to see what is happening in your
    community.
  \end{itemize}
\end{itemize}

Ms. Lee said that the borough's decision had nothing to do with any
overt or subtle xenophobia or racism --- like the discussions about the
cleanliness of ``those people'' that could recently be overheard over
pints of Keepin' Local beer on the patio of a local brewery and taproom.

``I find those kinds of comments more disgusting than any of the garbage
I saw left behind at the lake,'' Mayor Lee said.

At Lake Solitude the day before it was shut, the ground was pristine,
and few people enjoying the park felt there was a problem.

``People are just looking for an excuse not to have colored people
around, to get us out of their town,'' said Alej Rodriguez, 26, a truck
driver who drove in from Upper Manhattan with his family, to visit the
lake and the sights on the rolling grounds, like the remains of the
Union Iron Works forge, which smelted cannonballs for the Revolutionary
War.

``You've always got a target on your back as a colored person,'' he
said. ``You've always got to watch your back, even at a beautiful lake
where we come to have fun.''

Not far from Mr. Rodriguez, a man, who identified himself as a High
Bridge resident but would give only his first name, Mike, was taking
photos of people swimming with a long range lens.

``I'm documenting the problem,'' said the man, who was white, explaining
he was angry that the bathers were not wearing masks as they swam, and
worried that the people playing in the water were contaminating it.

Image

The decision to close Lake Solitude was made because of health
precautions tied to the coronavirus, according to Michele Lee, the mayor
of High Bridge.Credit...Bryan Anselm for The New York Times

Edward Bielcik, 74, had heard talk of the overcrowding and wanted to see
for himself, he said. He was one of several residents strolling the park
with cameras to document the claims. ``They said the Latin Kings tagged
the area,'' Mr. Bielcik said.

For several weeks, Mayor Lee, a financial adviser who does not take a
salary for her borough position, had pushed to keep the park open, under
her belief the newcomers could help make High Bridge a tourist
destination. ``If we get this right, it's a great situation for the
town,'' she said.

Plans are underway to figure out how to reopen and accommodate any
crowds, the mayor said, but there is no timeline yet to do so.

Some of the borough's residents say they can't help but feel that uglier
impulses are behind the desire to close the lake.

At Scout's Coffee Bar \& Mercantile on Main Street, the owner's eyes
filled with tears when she recounted the words used about the visitors
that she had overheard at her barista's counter. Just a month before, a
Black Lives Matter rally had taken place down the street.

``We just went through all the protests, and we are all learning about
how we can be better allies to people of color, and this is our
opportunity. It's disheartening,'' said the owner, Nicole Poko, 38, who
is white. ``It just feels like there is a lot of work to be done.''

Advertisement

\protect\hyperlink{after-bottom}{Continue reading the main story}

\hypertarget{site-index}{%
\subsection{Site Index}\label{site-index}}

\hypertarget{site-information-navigation}{%
\subsection{Site Information
Navigation}\label{site-information-navigation}}

\begin{itemize}
\tightlist
\item
  \href{https://help.nytimes.com/hc/en-us/articles/115014792127-Copyright-notice}{©~2020~The
  New York Times Company}
\end{itemize}

\begin{itemize}
\tightlist
\item
  \href{https://www.nytco.com/}{NYTCo}
\item
  \href{https://help.nytimes.com/hc/en-us/articles/115015385887-Contact-Us}{Contact
  Us}
\item
  \href{https://www.nytco.com/careers/}{Work with us}
\item
  \href{https://nytmediakit.com/}{Advertise}
\item
  \href{http://www.tbrandstudio.com/}{T Brand Studio}
\item
  \href{https://www.nytimes.com/privacy/cookie-policy\#how-do-i-manage-trackers}{Your
  Ad Choices}
\item
  \href{https://www.nytimes.com/privacy}{Privacy}
\item
  \href{https://help.nytimes.com/hc/en-us/articles/115014893428-Terms-of-service}{Terms
  of Service}
\item
  \href{https://help.nytimes.com/hc/en-us/articles/115014893968-Terms-of-sale}{Terms
  of Sale}
\item
  \href{https://spiderbites.nytimes.com}{Site Map}
\item
  \href{https://help.nytimes.com/hc/en-us}{Help}
\item
  \href{https://www.nytimes.com/subscription?campaignId=37WXW}{Subscriptions}
\end{itemize}
