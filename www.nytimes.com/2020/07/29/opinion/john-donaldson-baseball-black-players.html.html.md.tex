Sections

SEARCH

\protect\hyperlink{site-content}{Skip to
content}\protect\hyperlink{site-index}{Skip to site index}

\href{https://myaccount.nytimes.com/auth/login?response_type=cookie\&client_id=vi}{}

\href{https://www.nytimes.com/section/todayspaper}{Today's Paper}

\href{/section/opinion}{Opinion}\textbar{}This Great Black Baseball
Player Still Isn't in the Hall of Fame

\url{https://nyti.ms/39Et1uo}

\begin{itemize}
\item
\item
\item
\item
\item
\end{itemize}

Advertisement

\protect\hyperlink{after-top}{Continue reading the main story}

\href{/section/opinion}{Opinion}

Supported by

\protect\hyperlink{after-sponsor}{Continue reading the main story}

\hypertarget{this-great-black-baseball-player-still-isnt-in-the-hall-of-fame}{%
\section{This Great Black Baseball Player Still Isn't in the Hall of
Fame}\label{this-great-black-baseball-player-still-isnt-in-the-hall-of-fame}}

John Donaldson died in obscurity, his statistics devalued because he
played before baseball was integrated.

By \href{https://www.nytimes.com/by/mary-pilon}{Mary Pilon} and Travon
Free

Ms. Pilon is a former sports reporter for The Times. Mr. Free is a
television writer.

\begin{itemize}
\item
  July 29, 2020, 5:00 a.m. ET
\item
  \begin{itemize}
  \item
  \item
  \item
  \item
  \item
  \end{itemize}
\end{itemize}

\includegraphics{https://static01.nyt.com/images/2020/07/23/opinion/23pilon/23pilon-articleLarge.jpg?quality=75\&auto=webp\&disable=upscale}

Most baseball fans have probably never heard of John Donaldson, a
hard-throwing pitcher who drew sold-out crowds around the country for an
astonishing three decades before he hung up his glove in 1941. His
statistics establish him as one of the greatest to ever play America's
pastime. Yet he died in obscurity.

Now, Donaldson's towering contributions to the Negro Leagues are being
slowly resurrected after decades of racial injustice and institutional
neglect. That's thanks to the efforts of a white guy who drives an Uber
in Minnesota and to a network of amateur researchers that he organized
to reconstruct Donaldson's career and push for his admission into the
Hall of Fame.

It's been an arduous task; records about his life and career were
scattered and often difficult to find. An earlier effort to elect him to
the hall failed when a panel of historians considered experts on the
Negro Leagues declined to select him in 2006. At the time, many of his
career numbers were still not known. No explanation was given.

But Donaldson may have another shot in December, when the hall's Early
Baseball Era Committee meets to consider a roster of players, managers,
umpires and executives whose greatest contributions to baseball took
place before 1950. Any candidate whose name appears on at least 75
percent of the ballots will be inducted next year into the National
Baseball Hall of Fame, joining 35 other Negro League players. The 10
candidates will be announced this fall.

The first step to righting an injustice is to admit that it occurred.
That's why this small group of baseball activists, led by Peter Gorton,
the Uber driver, have been assembling their evidence and telling
Donaldson's story.

Perhaps one day the Hall of Fame will listen. It certainly should. His
story is more than about baseball. It's about the pain of social change.
It's probably not a coincidence that Donaldson attended seminary, but,
against his mother's wishes, ultimately chose to preach by example the
gospel of change on the field of dreams.

Mr. Gorton has so far uncovered 413 wins by Donaldson and 5,091
strikeouts. This means, according to Mr. Gorton, that Donaldson has more
wins and strikeouts than any pitcher in segregated baseball --- in the
Negro Leagues, on barnstorming teams and in the semi-pros --- before
Jackie Robinson broke the color barrier in 1947.

He was one of the biggest stars in the game's
\href{https://baseballhall.org/discover-more/stories/baseball-history/road-to-equality}{barnstorming
era}, a time when Black players risked their lives to play in towns
where lynchings were carried out with impunity. Barnstorming players
competed in matchups between Black and white teams that included major
leaguers in the off-season, among them Babe Ruth. Donaldson played in at
least 724 cities in the United States and Canada, according to Mr.
Gorton's research, and pitched 14 no-hitters and two perfect games. A
power pitcher, he was far ahead of his time in his technique.

He was also a leader. He was among the founders of the Kansas City
Monarchs (he is credited with coming up with its name), the Negro
Leagues team that was a training ground for the Hall of Famers Ernie
Banks, Satchel Paige, Robinson and other great players.

After Donaldson's famed pitching arm wore out, he became one of the
first Black scouts for Major League Baseball, working for the Chicago
White Sox, where he spotted talents like the young Willie Mays (though
the White Sox didn't sign him). He mentored many players on and off the
field, including Robinson.

Yet for all of his accomplishments, he spent his final years on the
overnight shift as a postal worker in Chicago and his days teaching
baseball to children in Chicago's parks system. For decades after his
death, his grave was unmarked.

The slow detective work that pieced together this story came about by
accident, when Mr. Gorton saw Donaldson's photo in a museum in Minnesota
in the early 2000s. He stared at the photo, startled by the sight of a
racially integrated baseball team (there were a few, like the All
Nations team, for which Donaldson once played) well before Robinson put
on a Brooklyn Dodgers uniform.

While most are familiar with the Negro Leagues, the practice of Black
teams playing white teams of that era is often overlooked but was
important in showing that racially integrated baseball could succeed.
Or, in the case of the All Nations team, that a single team could be
racially integrated, play nationwide, and thrive.

For Mr. Gorton, a middle-aged white man and impassioned baseball fan,
the reconstruction of Donaldson's life and career has meant confronting
inconvenient truths about the country and the sport that he loves, and
about the nation's history of redlining, violence, segregationist school
policies and racist unions. He has spent 20 years and filled his
basement with towers of paperwork as he amasses ever more evidence for
his second appeal to the Hall of Fame to admit Donaldson this year, the
\href{https://www.mlb.com/news/negro-leaguers-in-the-national-baseball-hall-of-fame}{centennial
year} of the Negro Leagues.

Donaldson's story had long been buried. It's part of a larger story
about the wall that kept Black players out of ``the Show''--- the major
leagues. Even today, statistics from the barnstorming era and the Negro
Leagues are played down, even though those players had nowhere else to
showcase their talents.

Mr. Gorton said he is trying to call attention to Donaldson because his
life was full of lessons about resilience and being an agent of social
change. ``Donaldson's story was a life-or-death struggle,'' Mr. Gorton
said. ``A huge part of the Black baseball struggle in America is
misunderstood. Everyone thinks that it's chiseled in stone in 1947 but
we're learning something new every day. We need to figure out that
history.''

Donaldson's story also reveals a hard truth about progress --- that it's
messy, complicated and almost instantly rewritten by those who had tried
to slow it down. We love to trumpet Robinson's career, for example, and
should, but seldom mention his
\href{https://prospect.org/civil-rights/jackie-robinson-legacy-activism/}{post-retirement
advocacy work} against racial injustices including redlining.

The world is full of Donaldsons, people who change things for the
better, their contributions unnoticed. Telling their stories matters
immensely not merely in building the arc of progress, but in also
showing the world as it really is.

``The ability of John Donaldson to have a lasting legacy was
systematically eliminated by both baseball and the society he lived
through,'' Mr. Gorton told us.

``He never had a chance,'' he added. ``Not only with `on the field
opportunities' but in life as well. History cannot remember what it
knows little about, and actively tries to minimize.''

Perhaps the truth about baseball, and any kind of seismic change, is an
inversion of the fabled line from ``Field of Dreams.''

No one built it for players like Donaldson. But they came anyway.

And we are all are better for it.

\href{http://marypilon.com/}{Mary Pilon}, a former sports reporter for
The Times and author of ``The Monopolists,'' and
\href{https://www.travonfree.com/}{Travon Free}, a television writer,
are the screenwriters of ``Barnstormers," a film in development about
John Donaldson and Peter Gorton.

\emph{The Times is committed to publishing}
\href{https://www.nytimes.com/2019/01/31/opinion/letters/letters-to-editor-new-york-times-women.html}{\emph{a
diversity of letters}} \emph{to the editor. We'd like to hear what you
think about this or any of our articles. Here are some}
\href{https://help.nytimes.com/hc/en-us/articles/115014925288-How-to-submit-a-letter-to-the-editor}{\emph{tips}}\emph{.
And here's our email:}
\href{mailto:letters@nytimes.com}{\emph{letters@nytimes.com}}\emph{.}

\emph{Follow The New York Times Opinion section on}
\href{https://www.facebook.com/nytopinion}{\emph{Facebook}}\emph{,}
\href{http://twitter.com/NYTOpinion}{\emph{Twitter (@NYTopinion)}}
\emph{and}
\href{https://www.instagram.com/nytopinion/}{\emph{Instagram}}\emph{.}

Advertisement

\protect\hyperlink{after-bottom}{Continue reading the main story}

\hypertarget{site-index}{%
\subsection{Site Index}\label{site-index}}

\hypertarget{site-information-navigation}{%
\subsection{Site Information
Navigation}\label{site-information-navigation}}

\begin{itemize}
\tightlist
\item
  \href{https://help.nytimes.com/hc/en-us/articles/115014792127-Copyright-notice}{©~2020~The
  New York Times Company}
\end{itemize}

\begin{itemize}
\tightlist
\item
  \href{https://www.nytco.com/}{NYTCo}
\item
  \href{https://help.nytimes.com/hc/en-us/articles/115015385887-Contact-Us}{Contact
  Us}
\item
  \href{https://www.nytco.com/careers/}{Work with us}
\item
  \href{https://nytmediakit.com/}{Advertise}
\item
  \href{http://www.tbrandstudio.com/}{T Brand Studio}
\item
  \href{https://www.nytimes.com/privacy/cookie-policy\#how-do-i-manage-trackers}{Your
  Ad Choices}
\item
  \href{https://www.nytimes.com/privacy}{Privacy}
\item
  \href{https://help.nytimes.com/hc/en-us/articles/115014893428-Terms-of-service}{Terms
  of Service}
\item
  \href{https://help.nytimes.com/hc/en-us/articles/115014893968-Terms-of-sale}{Terms
  of Sale}
\item
  \href{https://spiderbites.nytimes.com}{Site Map}
\item
  \href{https://help.nytimes.com/hc/en-us}{Help}
\item
  \href{https://www.nytimes.com/subscription?campaignId=37WXW}{Subscriptions}
\end{itemize}
