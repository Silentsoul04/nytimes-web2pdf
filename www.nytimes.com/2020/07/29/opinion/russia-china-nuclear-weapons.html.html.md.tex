Sections

SEARCH

\protect\hyperlink{site-content}{Skip to
content}\protect\hyperlink{site-index}{Skip to site index}

\href{https://myaccount.nytimes.com/auth/login?response_type=cookie\&client_id=vi}{}

\href{https://www.nytimes.com/section/todayspaper}{Today's Paper}

\href{/section/opinion}{Opinion}\textbar{}China's Arms Buildup Threatens
the Nuclear Balance

\url{https://nyti.ms/3f6A4NH}

\begin{itemize}
\item
\item
\item
\item
\item
\end{itemize}

Advertisement

\protect\hyperlink{after-top}{Continue reading the main story}

\href{/section/opinion}{Opinion}

Supported by

\protect\hyperlink{after-sponsor}{Continue reading the main story}

\hypertarget{chinas-arms-buildup-threatens-the-nuclear-balance}{%
\section{China's Arms Buildup Threatens the Nuclear
Balance}\label{chinas-arms-buildup-threatens-the-nuclear-balance}}

A Pentagon leader argues that as Beijing's weapons grow in size and
sophistication, the U.S. and Russia will have to reassess their own
arsenals.

By James Anderson

Dr. Anderson is the acting under secretary of defense for policy.

\begin{itemize}
\item
  July 29, 2020, 5:00 a.m. ET
\item
  \begin{itemize}
  \item
  \item
  \item
  \item
  \item
  \end{itemize}
\end{itemize}

\includegraphics{https://static01.nyt.com/images/2020/07/29/opinion/29Anderson/29Anderson-articleLarge.jpg?quality=75\&auto=webp\&disable=upscale}

Nuclear arms control is at a crossroads --- not because we are
approaching the deadline on an extension of the
\href{https://www.nytimes.com/topic/subject/new-start-treaty}{2010 New
Strategic Arms Reduction Treaty}, but because
\href{https://www.nytimes.com/2020/06/30/us/politics/trump-russia-china-nuclear.html}{China's
nuclear expansion} threatens to upend decades of relative nuclear
stability between the United States and Russia.

The United States and Russia have been reducing their strategic nuclear
arsenals since the end of the Cold War.
\href{https://www.nytimes.com/1991/05/03/opinion/start-treaty-finish-it-lose-it-richard-burt-was-chief-negotiator-strategic-arms.html}{The
1991 Start Treaty} allowed each side 6,000 deployable strategic nuclear
warheads; the 2010 treaty, known as New Start, lowered that limit to
1,550 operationally deployed strategic nuclear warheads.

But stability at these lower force levels will be challenged by
\href{https://www.nti.org/learn/countries/china/}{China's nuclear
ambitions}. China is clearly moving away from the small, limited nuclear
force of its past. It is fielding modern land- and sea-based strategic
systems and plans to introduce an air-launched ballistic missile
delivered by heavy bombers in the near future, achieving its own
strategic nuclear triad.

The
\href{https://www.dia.mil/News/Speeches-and-Testimonies/Article-View/Article/1859890/russian-and-chinese-nuclear-modernization-trends/}{Defense
Intelligence Agency} estimates that China will at least double the size
of its nuclear arsenal over the next decade and is building the
production capacity to expand it further. Given China's secrecy about
its nuclear forces, and its manifestly aggressive strategic intentions,
this nuclear expansion may go even further, well beyond Beijing's old
``\href{https://carnegieendowment.org/2016/06/30/china-s-nuclear-doctrine-debates-and-evolution-pub-63967}{minimum
deterrence}'' doctrine.

Still, it is in China's interest to reverse its dangerous nuclear
buildup, lest it set off
\href{https://www.nytimes.com/2019/08/08/world/europe/arms-race-russia-china.html}{a
nuclear arms race} involving the United States and Russia, and perhaps
encourage other nuclear powers to increase their forces to keep pace.

Meanwhile, the United States is
\href{https://www.armscontrol.org/factsheets/USNuclearModernization}{replacing
its aging nuclear weapons systems.}Our intention is to remain within the
New Start limits of 700 strategic missiles and bombers and 1,550
deployed strategic warheads.

But as Chinese nuclear forces grow in size and sophistication, the
United States will have no choice but to reassess and adjust its own
nuclear force requirements. In the past, the United States classified
China's small nuclear arsenal as a subset of U.S. nuclear force
requirements, which have been largely driven by the Soviet and then
Russian threat.

But this will not remain the case if U.S. nuclear forces remain at
historically low levels and China's continue to expand with no
discernible constraint. And the less we know about what China is doing
and why, the more the United States must rely on worst case scenarios to
size its nuclear forces accordingly.

China's nuclear expansion and its refusal to engage in meaningful
dialogue will affect stability on multiple levels. Increased U.S.
nuclear force requirements to ensure credible deterrence against China
would affect the United States-Russia strategic nuclear balance and
threaten to undermine the prospects for further negotiated reductions.
We should assume that Russia will also assess the implications of
China's expansion.

The American special envoy for arms control, Marshall Billingslea, made
these points to his Russian counterpart during a
\href{https://www.state.gov/special-presidential-envoy-ambassador-marshall-billingslea-travels-to-austria/}{meeting
in June in Vienna.} Russia should clearly see its own self-interest in
helping to bring China into discussions on arms control.

These talks need not focus on making China part of an extended New Start
agreement. But renewing the treaty for the United States and Russia
without conditions for bringing China into a broader arms control
process carries risks for future security, even if today it seems the
easiest course to take. All the great powers must be invested in such a
process.

We ask China to recognize its obligations under the Nuclear
Nonproliferation Treaty to negotiate in good faith on limiting and
reducing nuclear arms and, more generally, to take steps toward greater
transparency. Transparency is important to foster greater trust and
lessen the chance of miscalculation during a crisis. That first step is
joining the United States and Russia at the table in Vienna.

Those of us charged with ensuring the defense of the United States call
on Congress and our allies to help make the case to Russia and China
that it is in the interests of all nations to broaden the current arms
control framework to verifiably limit the nuclear weapons of all three
major powers to secure a more stable and prosperous future.

\emph{The Times is committed to publishing}
\href{https://www.nytimes.com/2019/01/31/opinion/letters/letters-to-editor-new-york-times-women.html}{\emph{a
diversity of letters}} \emph{to the editor. We'd like to hear what you
think about this or any of our articles. Here are some}
\href{https://help.nytimes.com/hc/en-us/articles/115014925288-How-to-submit-a-letter-to-the-editor}{\emph{tips}}\emph{.
And here's our email:}
\href{mailto:letters@nytimes.com}{\emph{letters@nytimes.com}}\emph{.}

\emph{Follow The New York Times Opinion section on}
\href{https://www.facebook.com/nytopinion}{\emph{Facebook}}\emph{,}
\href{http://twitter.com/NYTOpinion}{\emph{Twitter (@NYTopinion)}}
\emph{and}
\href{https://www.instagram.com/nytopinion/}{\emph{Instagram}}\emph{.}

James Anderson is the acting under secretary of defense for policy.

Advertisement

\protect\hyperlink{after-bottom}{Continue reading the main story}

\hypertarget{site-index}{%
\subsection{Site Index}\label{site-index}}

\hypertarget{site-information-navigation}{%
\subsection{Site Information
Navigation}\label{site-information-navigation}}

\begin{itemize}
\tightlist
\item
  \href{https://help.nytimes.com/hc/en-us/articles/115014792127-Copyright-notice}{©~2020~The
  New York Times Company}
\end{itemize}

\begin{itemize}
\tightlist
\item
  \href{https://www.nytco.com/}{NYTCo}
\item
  \href{https://help.nytimes.com/hc/en-us/articles/115015385887-Contact-Us}{Contact
  Us}
\item
  \href{https://www.nytco.com/careers/}{Work with us}
\item
  \href{https://nytmediakit.com/}{Advertise}
\item
  \href{http://www.tbrandstudio.com/}{T Brand Studio}
\item
  \href{https://www.nytimes.com/privacy/cookie-policy\#how-do-i-manage-trackers}{Your
  Ad Choices}
\item
  \href{https://www.nytimes.com/privacy}{Privacy}
\item
  \href{https://help.nytimes.com/hc/en-us/articles/115014893428-Terms-of-service}{Terms
  of Service}
\item
  \href{https://help.nytimes.com/hc/en-us/articles/115014893968-Terms-of-sale}{Terms
  of Sale}
\item
  \href{https://spiderbites.nytimes.com}{Site Map}
\item
  \href{https://help.nytimes.com/hc/en-us}{Help}
\item
  \href{https://www.nytimes.com/subscription?campaignId=37WXW}{Subscriptions}
\end{itemize}
