\href{/section/science}{Science}\textbar{}This Moss Uses Quartz as a
Parasol

\url{https://nyti.ms/2P5Tsji}

\begin{itemize}
\item
\item
\item
\item
\item
\item
\end{itemize}

\includegraphics{https://static01.nyt.com/images/2020/07/28/science/00SCI-MOSS1/merlin_175035663_0bc43f89-865e-423b-a1e7-258beaaaa1d4-articleLarge.jpg?quality=75\&auto=webp\&disable=upscale}

Sections

\protect\hyperlink{site-content}{Skip to
content}\protect\hyperlink{site-index}{Skip to site index}

Trilobites

\hypertarget{this-moss-uses-quartz-as-a-parasol}{%
\section{This Moss Uses Quartz as a
Parasol}\label{this-moss-uses-quartz-as-a-parasol}}

In the Mojave Desert, a translucent crystal offers bryophytes
much-needed respite from the heat of the sun.

A piece of quartz hiding a species of moss, Syntrichia caninervis, in
the Mojave Desert of California.Credit...Kirsten Fisher

Supported by

\protect\hyperlink{after-sponsor}{Continue reading the main story}

By Sabrina Imbler

\begin{itemize}
\item
  July 29, 2020
\item
  \begin{itemize}
  \item
  \item
  \item
  \item
  \item
  \item
  \end{itemize}
\end{itemize}

To humans, a desert oasis may conjure an image of a blue pool encircled
by a coronet of palm trees. But to certain mosses, an oasis takes the
form of a pebble of milky quartz. The cloudy crystal dilutes the sun's
piercing ultraviolet rays and, in the dry desert heat, traps moisture
beneath it, creating a microclimate perfect for a moss.

Kirsten Fisher, a biologist at California State University, Los Angeles,
first spotted these miniature oases in 2014, somewhere off a highway in
the western Mojave Desert in Wrightwood, Calif. She was studying the sex
life of the moss Syntrichia caninervis, and progress was slow.

``They're never having sex,'' she said. ``For a moss, it's maybe once
every 30 years.''

The site was studded with crystals eroded from a nearby mountain striped
with pearlescent quartz. While waiting for Jenna Ekwealor, now a
doctoral student at the University of California, Berkeley, to finish up
a transect of S. caninervis growing in the dirt nearby, Dr. Fisher
picked up one of the many glittering rocks scattered nearby and, she
recalled, saw a brilliant green carpet of the moss underneath: ``And I
said, `Holy moly, there's moss under this rock.'''

Ms. Ekwealor was sure the green moss was a fluke. It had not rained in
Wrightwood for at least two weeks. But as she and Dr. Fisher flipped
over more rocks, they found several patches of strangely moist moss. Dr.
Fisher and Ms. Ekwealor published documentation of this relationship
last week in the journal
\href{https://journals.plos.org/plosone/article?id=10.1371/journal.pone.0235928}{PLoS
One}.

``I never thought to look under the rocks,'' said Brent Mishler, a
biologist at Berkeley who studies Syntrichia but was not involved with
the research*,* with an exaggerated face-palm over Zoom. (Dr. Mishler
has mentored Dr. Fisher and Ms. Ekwealor.)

Bunking under a rock can be luxurious. Astrobiologists have long studied
arctic and Antarctic
\href{https://www.nature.com/articles/431414a}{cyanobacteria} that grow
under translucent rocks to trap moisture while still being able to
photosynthesize. But no one has studied this survival strategy, used by
organisms known as hypoliths, in anything as large as moss.

``No one's really looked at this combination before,'' said Kimberley
Warren-Rhodes, a research scientist at the NASA Ames Research Center,
who peer-reviewed the study.

\includegraphics{https://static01.nyt.com/images/2020/07/28/science/00SCI-MOSS3/00SCI-MOSS3-articleLarge.jpg?quality=75\&auto=webp\&disable=upscale}

Last September, the researchers placed sensors underneath the pebbles to
measure how the microclimate changed with the seasons. Some sensors fell
prey to the middens of enterprising wood rats, but those that remained
found that the quartzite underbellies preserved approximately twice the
humidity of the surrounding area and buffered the temperature swings by
as much as 7 degrees Fahrenheit. ``It's a little quartz house,'' Ms.
Ekwealor said.

S. caninervis, a common moss in the Mojave Desert, spends most of the
year parched and brown --- in a state of suspended animation awaiting
the next rain. ``It is something only a mother can love,'' Dr. Mishler
said. But the mosses are long-lived; a single clump could easily be a
centenarian.

Although S. caninervis made up more than two-thirds of the hypolithic
moss at Wrightwood, the researchers identified another species, Tortula
inermis. That moss typically grows at lower, hotter elevations, but was
able to thrive at the Wrightwood site, seeming to rely on the quartz for
protection from the cold.

These quartzite oases, while common at Wrightwood, only emerge in what
Ms. Ekwealor called a ``goldilocks'' situation. If the quartz is too
tiny, it will be too easily windswept to let anything grow underneath.
If it is too large or opaque, not enough light will shine through for
photosynthesis. If it is too clear, it could become a miniature
greenhouse and capture even more heat. The quartz needs to be just
right: around an inch thick and milky enough to transmit up to 4 percent
of incident light.

But the vastness of the desert and the abundance of pebbles means that
serendipity can become commonplace, Ms. Ekwealor said: ``It's low
probability, but lots of opportunity,'' she said.

The study highlighted the importance of microenvironments that may be
invisible to the human eye, Ms. Ekwealor added. Dr. Warren-Rhodes noted
that hypolithic communities, tiny as they are, affect carbon cycling and
soil conservation.

After presenting this research at several conferences, Ms. Ekwealor said
she now receives sporadic texts from people identifying hypolithic
mosses across the country. ``I hope people start flipping rocks to see
what else is out there,'' she said. After a pause, she added, ``And
gently placing them back down again, so the moss can survive.''

\textbf{\emph{{[}}\href{http://on.fb.me/1paTQ1h}{\emph{Like the Science
Times page on Facebook.}}} ****** \emph{\textbar{} Sign up for the}
\textbf{\href{http://nyti.ms/1MbHaRU}{\emph{Science Times
newsletter.}}\emph{{]}}}

Advertisement

\protect\hyperlink{after-bottom}{Continue reading the main story}

\hypertarget{site-index}{%
\subsection{Site Index}\label{site-index}}

\hypertarget{site-information-navigation}{%
\subsection{Site Information
Navigation}\label{site-information-navigation}}

\begin{itemize}
\tightlist
\item
  \href{https://help.nytimes.com/hc/en-us/articles/115014792127-Copyright-notice}{©~2020~The
  New York Times Company}
\end{itemize}

\begin{itemize}
\tightlist
\item
  \href{https://www.nytco.com/}{NYTCo}
\item
  \href{https://help.nytimes.com/hc/en-us/articles/115015385887-Contact-Us}{Contact
  Us}
\item
  \href{https://www.nytco.com/careers/}{Work with us}
\item
  \href{https://nytmediakit.com/}{Advertise}
\item
  \href{http://www.tbrandstudio.com/}{T Brand Studio}
\item
  \href{https://www.nytimes.com/privacy/cookie-policy\#how-do-i-manage-trackers}{Your
  Ad Choices}
\item
  \href{https://www.nytimes.com/privacy}{Privacy}
\item
  \href{https://help.nytimes.com/hc/en-us/articles/115014893428-Terms-of-service}{Terms
  of Service}
\item
  \href{https://help.nytimes.com/hc/en-us/articles/115014893968-Terms-of-sale}{Terms
  of Sale}
\item
  \href{https://spiderbites.nytimes.com}{Site Map}
\item
  \href{https://help.nytimes.com/hc/en-us}{Help}
\item
  \href{https://www.nytimes.com/subscription?campaignId=37WXW}{Subscriptions}
\end{itemize}
