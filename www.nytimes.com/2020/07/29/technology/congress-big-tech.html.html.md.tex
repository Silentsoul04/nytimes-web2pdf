Sections

SEARCH

\protect\hyperlink{site-content}{Skip to
content}\protect\hyperlink{site-index}{Skip to site index}

\href{https://www.nytimes.com/section/technology}{Technology}

\href{https://myaccount.nytimes.com/auth/login?response_type=cookie\&client_id=vi}{}

\href{https://www.nytimes.com/section/todayspaper}{Today's Paper}

\href{/section/technology}{Technology}\textbar{}Congress Doesn't Get Big
Tech. By Design.

\url{https://nyti.ms/309BcvJ}

\begin{itemize}
\item
\item
\item
\item
\item
\end{itemize}

Advertisement

\protect\hyperlink{after-top}{Continue reading the main story}

Supported by

\protect\hyperlink{after-sponsor}{Continue reading the main story}

on tech

\hypertarget{congress-doesnt-get-big-tech-by-design}{%
\section{Congress Doesn't Get Big Tech. By
Design.}\label{congress-doesnt-get-big-tech-by-design}}

Members of Congress may say dumb things at the tech hearing, but it's
not necessarily their fault.

\includegraphics{https://static01.nyt.com/images/2020/07/29/business/29OnTech-congress-NL/merlin_136662141_d00edf4e-32e1-4792-838d-be6203b96ea3-articleLarge.jpg?quality=75\&auto=webp\&disable=upscale}

\href{https://www.nytimes.com/by/shira-ovide}{\includegraphics{https://static01.nyt.com/images/2020/03/18/reader-center/author-shira-ovide/author-shira-ovide-thumbLarge-v2.png}}

By \href{https://www.nytimes.com/by/shira-ovide}{Shira Ovide}

\begin{itemize}
\item
  July 29, 2020, 12:01 p.m. ET
\item
  \begin{itemize}
  \item
  \item
  \item
  \item
  \item
  \end{itemize}
\end{itemize}

\emph{This article is part of the On Tech newsletter. You can}
\href{https://www.nytimes.com/newsletters/signup/OT}{\emph{sign up
here}} \emph{to receive it weekdays.}

I'll make an easy prediction about
\href{https://www.nytimes.com/2020/07/28/technology/amazon-apple-facebook-google-antitrust-hearing.html}{Wednesday's
congressional hearing} into the power of big tech companies: Members of
Congress will say dumb things.

But please don't believe that these people are too old or too clueless
to exercise effective oversight of tech superpowers.

This idea, which is prevalent inside of tech companies, lets the tech
giants off the hook for what they do. It shows a smug superiority that
is not a good look. And it ignores that tech companies are built around
software that is designed not to be understood by outsiders.

\emph{(}\href{https://www.nytimes.com/live/2020/07/29/technology/tech-ceos-hearing-testimony}{\emph{Follow
The Times's live coverage of the hearing.}}\emph{)}

After Mark Zuckerberg's
\href{https://www.nytimes.com/2018/04/12/technology/mark-zuckerberg-testimony.html}{first
turns in the congressional hot seat} two years ago, people inside of
Facebook thought that
\href{https://www.wired.com/story/sigh-of-relief-inside-facebook/}{their
boss had completely dominated those old fogies}. I've heard this from
Facebook executives. Their conclusions have worried me.

Members of Congress were fairly blamed for not understanding Facebook,
but Zuckerberg didn't get enough blame for failing to make Facebook
understood. He
\href{https://www.nytimes.com/2018/04/10/technology/zuckerberg-elections-russia-data-privacy.html}{dodged},
\href{https://www.washingtonpost.com/news/fact-checker/wp/2018/04/13/fact-checking-mark-zuckerbergs-testimony-on-facebook-and-data-collection/}{occasionally
misled} and essentially
\href{https://www.bloomberg.com/opinion/articles/2018-04-12/mark-zuckerberg-refuses-to-admit-how-facebook-works}{tried
to say as little as possible} about how Facebook works. At points, he
didn't seem to know how Facebook worked, either.

Executives from Facebook, Google, Amazon and Apple at
\href{https://www.nytimes.com/2019/07/16/technology/big-tech-antitrust-hearing.html}{a
hearing last year} likewise seemed to intentionally deflect or dismiss
what were generally excellent questions from lawmakers. (Seriously, I
could have stared at C-SPAN for many more hours.) No one inside the big
tech companies should have felt like they ``won.''

To be fair, that is part of the theatrics of all congressional hearings.
Members of Congress grandstand and witnesses generally try to be
inoffensive or run out the clock.

\hypertarget{live-updates-big-tech-hearing}{%
\section{\texorpdfstring{\href{https://www.nytimes.com/live/2020/07/29/technology/tech-ceos-hearing-testimony?action=click\&pgtype=Article\&state=default\&region=MAIN_CONTENT_1\&context=storylines_live_updates}{Live
Updates: Big Tech
Hearing}}{Live Updates: Big Tech Hearing}}\label{live-updates-big-tech-hearing}}

\href{https://www.nytimes.com/live/2020/07/29/technology/tech-ceos-hearing-testimony?action=click\&pgtype=Article\&state=default\&region=MAIN_CONTENT_1\&context=storylines_live_updates\#what-ceos-said}{26m
ago}

\href{https://www.nytimes.com/live/2020/07/29/technology/tech-ceos-hearing-testimony?action=click\&pgtype=Article\&state=default\&region=MAIN_CONTENT_1\&context=storylines_live_updates\#what-ceos-said}{Here's
a tally of the C.E.O.s' catchphrases. Follow along as we update it
live.}

\href{https://www.nytimes.com/live/2020/07/29/technology/tech-ceos-hearing-testimony?action=click\&pgtype=Article\&state=default\&region=MAIN_CONTENT_1\&context=storylines_live_updates\#republicans-immediately-raise-bias-concerns-about-platforms}{38m
ago}

\href{https://www.nytimes.com/live/2020/07/29/technology/tech-ceos-hearing-testimony?action=click\&pgtype=Article\&state=default\&region=MAIN_CONTENT_1\&context=storylines_live_updates\#republicans-immediately-raise-bias-concerns-about-platforms}{Republicans
immediately raise bias concerns about platforms.}

\href{https://www.nytimes.com/live/2020/07/29/technology/tech-ceos-hearing-testimony?action=click\&pgtype=Article\&state=default\&region=MAIN_CONTENT_1\&context=storylines_live_updates\#lawmaker-americans-should-not-bow-before-the-emperors-of-the-online-economy}{1h
ago}

\href{https://www.nytimes.com/live/2020/07/29/technology/tech-ceos-hearing-testimony?action=click\&pgtype=Article\&state=default\&region=MAIN_CONTENT_1\&context=storylines_live_updates\#lawmaker-americans-should-not-bow-before-the-emperors-of-the-online-economy}{Lawmaker:
Americans should not `bow before the emperors of the online economy.'}

\href{https://www.nytimes.com/live/2020/07/29/technology/tech-ceos-hearing-testimony?action=click\&pgtype=Article\&state=default\&region=MAIN_CONTENT_1\&context=storylines_live_updates}{See
more updates}

Yet it's in everyone's interest to complete this set of hearings
\href{https://www.nytimes.com/2020/05/22/technology/google-antitrust.html}{and
effectively address these central questions:} Are these big technology
companies cheating to get a leg up over competitors? If so, does that
hurt all of us and what --- if anything --- should the government do
about it?

If members of Congress are confused about how to ask and answer these
questions, that's partly because big tech companies \emph{are}
confusing.

Few people on the outside can truly understand how Amazon influences the
prices of products we buy on its site or
\href{https://www.bloomberg.com/news/articles/2019-08-05/amazon-is-squeezing-sellers-that-offer-better-prices-on-walmart}{at
other retailers}; assess fears that Google
\href{https://themarkup.org/google-the-giant/2020/07/28/google-search-results-prioritize-google-products-over-competitors}{funnels
people to its own websites} or that Apple
\href{https://www.nytimes.com/interactive/2019/09/09/technology/apple-app-store-competition.html}{steers
people to its own apps}; or peer into Facebook's
\href{https://www.nytimes.com/2018/12/05/technology/facebook-emails-privacy-data.html}{strategy
to squash rivals in their cribs}. All of this is, by design, shrouded in
secrecy and mystery.

Even many of the big tech companies now say that there needs to be more
federal oversight and rules regarding areas like protecting elections
and what constitutes appropriate speech online.

That means everyone --- the tech companies, lawmakers and you and me ---
have a vested interest in getting under the hood of these big companies
and seeing how they work.

This is a worthy goal --- just as it was to assess the big banks after
the 2008 financial crisis. Those banks also thought Congress was too
clueless to question them effectively. Maybe so, but
\href{https://www.nytimes.com/2010/07/16/business/16regulate.html}{regulation
came anyway}.

\emph{What questions do you have about the hearing and the power of big
tech? Send them to
\href{mailto:ontech@nytimes.com}{\nolinkurl{ontech@nytimes.com}}, and
Shira will answer a selection in an upcoming newsletter. Please include
your full name and location.}

\begin{center}\rule{0.5\linewidth}{\linethickness}\end{center}

\hypertarget{dont-fall-for-these-distractions-at-the-hearing}{%
\subsection{Don't fall for these distractions at the
hearing}\label{dont-fall-for-these-distractions-at-the-hearing}}

Me again, taking another moment to talk about Wednesday's hearing ---
Sorry! Not sorry! --- to explain what it is NOT about.

\textbf{How big the tech companies are compared with the planet
Jupiter:} In his
\href{https://docs.house.gov/meetings/JU/JU05/20200729/110883/HHRG-116-JU05-Wstate-BezosJ-20200729.pdf}{prepared
testimony} for Wednesday's hearing, Amazon's Jeff Bezos cited
competition from the grocery delivery service Instacart and mentioned
the fast sales growth of Walmart's online shopping operation.

Sure, but online sales at those companies are a minuscule fraction of
Amazon's. There will be a lot of slicing and dicing of data for
misdirection like this. Please ignore.

The assessment of tech company power is not solely about their size or
that of rivals. It is also about their \emph{behavior}: Do big tech
companies tilt the game to their advantage in a way that creates less
competition?

\textbf{Whether these sites show political bias:} We'll hear a lot about
this today, because some conservatives and Republican politicians argue
that big tech companies habitually
\href{https://www.nytimes.com/2019/05/15/us/donald-trump-twitter-facebook-youtube.html}{squash
information} reflecting conservative perspectives.

There's little credible reporting to support this, but a root cause of
the concern is what I mentioned above: Outsiders can't see or assess the
software that determines what information we see online. Black boxes
naturally create suspicion.

How tech companies influence what information we're exposed to, and how
they fairly police what people say online, are complicated topics worthy
of debate. However, I'm not sure that there's a direct connection
between those topics and the central question at Wednesday's hearing: Do
big tech companies cheat to win?

\textbf{How many American jobs they create:} In
\href{https://docs.house.gov/meetings/JU/JU05/20200729/110883/HHRG-116-JU05-Wstate-PichaiS-20200729.pdf}{a
letter to Congress}, Google's chief executive touted a (delicious
sounding) brownie shop in New York that drums up business from buying
ads on Google. Bezos
\href{https://docs.house.gov/meetings/JU/JU05/20200729/110883/HHRG-116-JU05-Wstate-BezosJ-20200729.pdf}{talked
up} Amazon training programs to pay for warehouse workers to move into
higher-paying careers.

This is great! We want American companies to create jobs and contribute
to economic growth. But companies that create jobs and support small
businesses can still break the law by unfairly exercising their power
and influence.

\emph{If you don't already get this newsletter in your inbox,}
\href{https://www.nytimes.com/newsletters/signup/OT}{\emph{please sign
up here}}\emph{.}

\begin{center}\rule{0.5\linewidth}{\linethickness}\end{center}

\hypertarget{before-we-go-}{%
\subsection{Before we go \ldots{}}\label{before-we-go-}}

\begin{itemize}
\item
  \textbf{How to support alternatives to Big Tech:} My colleague Brian
  X. Chen tells us how we can
  \href{https://www.nytimes.com/2020/07/29/technology/personaltech/big-tech-power-how-to-fight.html}{help
  tech's little guys} if we're concerned about having choices. Brian
  suggests trying the search engine DuckDuckGo, the social network
  Mastodon and other alternatives to Big Tech products; advises us to
  buy used electronics to help repair shops and resellers; and asks us
  to consider paying for software we like from smaller companies rather
  than taking freebies from the tech giants.
\item
  \textbf{That coronavirus video was tailored to go wild:} My colleagues
  Sheera Frenkel and Davey Alba walked through
  \href{https://www.nytimes.com/2020/07/28/technology/virus-video-trump.html}{the
  stagecraft of a viral video} that promoted an unproven coronavirus
  treatment as a miracle cure. With ingredients including an
  official-looking setting, people in white medical coats and the
  ability to clip the video and share it on social media easily, the
  video had been designed to appeal to those who don't trust public
  health officials and want quick fixes to get past the pandemic.
\item
  \textbf{Can facial recognition technology be effective, unbiased and
  do more good than harm?} Those are questions raised by
  \href{https://www.reuters.com/investigates/special-report/usa-riteaid-software/}{this
  Reuters investigation} into the use of the technology at 200 Rite Aid
  drugstores in the United States.

  Facial recognition systems that were intended partly to notify store
  workers about potential shoplifters were more likely to be installed
  at stores in neighborhoods with a large share of lower-income or Black
  or Latino residents, and shoppers were not generally told that their
  images were being captured and analyzed. At times the facial
  recognition software also misidentified people. Rite Aid told Reuters
  it had suspended use of the cameras.
\end{itemize}

\hypertarget{hugs-to-this}{%
\subsubsection{Hugs to this}\label{hugs-to-this}}

This
\href{https://www.tiktok.com/@bizqueen/video/6852740756692569349}{dancing
duet of a woman and cat} is just plain weird. (Thanks to the Bloomberg
columnist \href{https://twitter.com/firstadopter}{Tae Kim} for bringing
this TikTok video into my life.)

\begin{center}\rule{0.5\linewidth}{\linethickness}\end{center}

\emph{We want to hear from you. Tell us what you think of this
newsletter and what else you'd like us to explore. You can reach us at}
\href{mailto:ontech@nytimes.com?subject=On\%20Tech\%20Feedback}{\emph{ontech@nytimes.com.}}
**

\emph{If you don't already get this newsletter in your inbox,}
\href{https://www.nytimes.com/newsletters/signup/OT}{\emph{please sign
up here}}\emph{.}

Advertisement

\protect\hyperlink{after-bottom}{Continue reading the main story}

\hypertarget{site-index}{%
\subsection{Site Index}\label{site-index}}

\hypertarget{site-information-navigation}{%
\subsection{Site Information
Navigation}\label{site-information-navigation}}

\begin{itemize}
\tightlist
\item
  \href{https://help.nytimes.com/hc/en-us/articles/115014792127-Copyright-notice}{©~2020~The
  New York Times Company}
\end{itemize}

\begin{itemize}
\tightlist
\item
  \href{https://www.nytco.com/}{NYTCo}
\item
  \href{https://help.nytimes.com/hc/en-us/articles/115015385887-Contact-Us}{Contact
  Us}
\item
  \href{https://www.nytco.com/careers/}{Work with us}
\item
  \href{https://nytmediakit.com/}{Advertise}
\item
  \href{http://www.tbrandstudio.com/}{T Brand Studio}
\item
  \href{https://www.nytimes.com/privacy/cookie-policy\#how-do-i-manage-trackers}{Your
  Ad Choices}
\item
  \href{https://www.nytimes.com/privacy}{Privacy}
\item
  \href{https://help.nytimes.com/hc/en-us/articles/115014893428-Terms-of-service}{Terms
  of Service}
\item
  \href{https://help.nytimes.com/hc/en-us/articles/115014893968-Terms-of-sale}{Terms
  of Sale}
\item
  \href{https://spiderbites.nytimes.com}{Site Map}
\item
  \href{https://help.nytimes.com/hc/en-us}{Help}
\item
  \href{https://www.nytimes.com/subscription?campaignId=37WXW}{Subscriptions}
\end{itemize}
