Sections

SEARCH

\protect\hyperlink{site-content}{Skip to
content}\protect\hyperlink{site-index}{Skip to site index}

\href{https://www.nytimes.com/section/us}{U.S.}

\href{https://myaccount.nytimes.com/auth/login?response_type=cookie\&client_id=vi}{}

\href{https://www.nytimes.com/section/todayspaper}{Today's Paper}

\href{/section/us}{U.S.}\textbar{}U.S. Surpasses 150,000 Coronavirus
Deaths, Far Eclipsing Projections

\url{https://nyti.ms/39BI9cf}

\begin{itemize}
\item
\item
\item
\item
\item
\end{itemize}

\href{https://www.nytimes.com/news-event/coronavirus?action=click\&pgtype=Article\&state=default\&region=TOP_BANNER\&context=storylines_menu}{The
Coronavirus Outbreak}

\begin{itemize}
\tightlist
\item
  live\href{https://www.nytimes.com/2020/08/04/world/coronavirus-cases.html?action=click\&pgtype=Article\&state=default\&region=TOP_BANNER\&context=storylines_menu}{Latest
  Updates}
\item
  \href{https://www.nytimes.com/interactive/2020/us/coronavirus-us-cases.html?action=click\&pgtype=Article\&state=default\&region=TOP_BANNER\&context=storylines_menu}{Maps
  and Cases}
\item
  \href{https://www.nytimes.com/interactive/2020/science/coronavirus-vaccine-tracker.html?action=click\&pgtype=Article\&state=default\&region=TOP_BANNER\&context=storylines_menu}{Vaccine
  Tracker}
\item
  \href{https://www.nytimes.com/2020/08/02/us/covid-college-reopening.html?action=click\&pgtype=Article\&state=default\&region=TOP_BANNER\&context=storylines_menu}{College
  Reopening}
\item
  \href{https://www.nytimes.com/live/2020/08/04/business/stock-market-today-coronavirus?action=click\&pgtype=Article\&state=default\&region=TOP_BANNER\&context=storylines_menu}{Economy}
\end{itemize}

Advertisement

\protect\hyperlink{after-top}{Continue reading the main story}

Supported by

\protect\hyperlink{after-sponsor}{Continue reading the main story}

\hypertarget{us-surpasses-150000-coronavirus-deaths-far-eclipsing-projections}{%
\section{U.S. Surpasses 150,000 Coronavirus Deaths, Far Eclipsing
Projections}\label{us-surpasses-150000-coronavirus-deaths-far-eclipsing-projections}}

The national toll shows how difficult predicting the virus --- or human
behavior --- can be. President Trump and leading experts have at times
said that deaths would be much lower.

\includegraphics{https://static01.nyt.com/images/2020/07/29/us/29virus-deaths02/merlin_174909924_eb82597b-1044-4ee8-8279-b54a8f91da7d-articleLarge.jpg?quality=75\&auto=webp\&disable=upscale}

By \href{https://www.nytimes.com/by/nicholas-bogel-burroughs}{Nicholas
Bogel-Burroughs}

\begin{itemize}
\item
  Published July 29, 2020Updated July 31, 2020
\item
  \begin{itemize}
  \item
  \item
  \item
  \item
  \item
  \end{itemize}
\end{itemize}

The United States' leading authority on infectious disease
\href{https://www.today.com/video/dr-anthony-fauci-virus-death-toll-may-be-more-like-60-000-than-100-000-to-200-000-81825861735}{expressed
hope in April} that no more than 60,000 people in the country would die
from the coronavirus. A revered research center predicted a few weeks
later that the figure would be just over 70,000 people by early August.
When the number of deaths shot up in May,
\href{https://www.nytimes.com/2020/05/03/us/politics/trump-coronavirus.html}{President
Trump said} that anywhere between 75,000 and 100,000 people could die.

On Wednesday, the nation's death toll surpassed 150,000.

That the figure, based on
\href{https://www.nytimes.com/interactive/2020/us/coronavirus-us-cases.html?action=click\&pgtype=Article\&state=default\&module=STYLN_pharmacy_components\&region=TOP_BANNER\&context=storylines_menu}{a
New York Times database}, has soared so soon and so far beyond those
estimates illustrates how difficult it can be to accurately forecast the
spread of the virus, or the way citizens and politicians will respond to
it.

``The aspect which is really impossible to predict is human behavior,''
said Virginia Pitzer, a professor of epidemiology at Yale. ``To what
extent are people going to socially distance themselves? To what extent
are politics going to influence whether you wear a mask? All of these
factors are impossible to factor in.''

Americans have rarely been as hungry for scientific predictions as they
have been this year. Charts of virus case counts fill social media
feeds; epidemiologists are all over television; Dr. Anthony S. Fauci has
become a household name.

But the statistical modelers who were trying to predict the spread of a
new virus began with very little solid data, experts said, so it was no
surprise that they have had to repeatedly revise their projections. The
revisions have generally been in one direction: up.

As of Wednesday evening, at least 150,909 people were known to have died
of the virus in the United States, out of more than 4.4 million reported
infections. And even these figures are likely to be undercounts, experts
say.

\includegraphics{https://static01.nyt.com/images/2020/07/29/us/29virus-deaths01/merlin_174547323_88a6ca17-fbcd-40b0-a2c2-0fe4eee14c97-articleLarge.jpg?quality=75\&auto=webp\&disable=upscale}

The Centers for Disease Control and Prevention estimate that in some
regions the number of people who have been infected
\href{https://www.nytimes.com/2020/07/21/health/coronavirus-infections-us.html}{could
be two to 13 times higher} than the tallies of reported cases. Experts
have said the official death toll probably
\href{https://www.nytimes.com/2020/04/05/us/coronavirus-deaths-undercount.html}{omits
many people whose deaths were virus-related,} especially early in the
pandemic. And more people will die each day as long as the virus
continues to spread.

Weekly averages of reported deaths in the United States had fallen
substantially since an early peak in mid-April, when the national death
toll was driven largely by a catastrophic surge in New York State. But
deaths began to climb again this month, and the nation is now reporting
about 1,000 deaths a day.

\hypertarget{latest-updates-global-coronavirus-outbreak}{%
\section{\texorpdfstring{\href{https://www.nytimes.com/2020/08/04/world/coronavirus-cases.html?action=click\&pgtype=Article\&state=default\&region=MAIN_CONTENT_1\&context=storylines_live_updates}{Latest
Updates: Global Coronavirus
Outbreak}}{Latest Updates: Global Coronavirus Outbreak}}\label{latest-updates-global-coronavirus-outbreak}}

Updated 2020-08-05T05:01:36.114Z

\begin{itemize}
\tightlist
\item
  \href{https://www.nytimes.com/2020/08/04/world/coronavirus-cases.html?action=click\&pgtype=Article\&state=default\&region=MAIN_CONTENT_1\&context=storylines_live_updates\#link-762df92}{As
  talks drag on, McConnell signals openness to jobless aid extension,
  and negotiators agree on a deadline.}
\item
  \href{https://www.nytimes.com/2020/08/04/world/coronavirus-cases.html?action=click\&pgtype=Article\&state=default\&region=MAIN_CONTENT_1\&context=storylines_live_updates\#link-1228a480}{Novavax
  sees encouraging results from two studies of its experimental
  vaccine.}
\item
  \href{https://www.nytimes.com/2020/08/04/world/coronavirus-cases.html?action=click\&pgtype=Article\&state=default\&region=MAIN_CONTENT_1\&context=storylines_live_updates\#link-794484ed}{Mississippians
  must now wear masks in public, governor says.}
\end{itemize}

\href{https://www.nytimes.com/2020/08/04/world/coronavirus-cases.html?action=click\&pgtype=Article\&state=default\&region=MAIN_CONTENT_1\&context=storylines_live_updates}{See
more updates}

More live coverage:
\href{https://www.nytimes.com/live/2020/08/04/business/stock-market-today-coronavirus?action=click\&pgtype=Article\&state=default\&region=MAIN_CONTENT_1\&context=storylines_live_updates}{Markets}

The current toll is being felt much more widely across many states,
especially in the South, while New York is down to reporting an average
of 16 deaths a day. Nearly 2,200 deaths have been reported in the past
week in Texas, the state with the highest recent
\href{https://www.nytimes.com/interactive/2020/us/coronavirus-us-cases.html}{death
toll relative to its population}, followed by Arizona and South
Carolina. Florida broke its daily record again on Wednesday, reporting
216 fatalities and bringing the state's overall total to 6,332.

``The mortality is going to march in lockstep with our transmission,''
said Dr. Sarah Fortune, the chair of immunology and infectious diseases
at the T.H. Chan School of Public Health at Harvard.

Exactly what percentage of people who get the virus die from it is not
yet clear. The World Health Organization's chief scientist, Dr. Soumya
Swaminathan, said last month that it was
\href{https://www.nytimes.com/2020/07/04/health/coronavirus-death-rate.html}{likely
to be about 0.6 percent}. If that rate proves accurate, it would mean a
vast majority of infections in the United States have gone unreported.

Dr. Fortune estimated that a mortality rate of 0.5 percent of all
coronavirus cases would be a ``best-case scenario,'' but that the death
rate could range up to 2 percent of cases, depending on how much the
virus reaches into the highest-risk environments, like nursing homes.

``We have to do better in terms of limiting transmission,'' Dr. Fortune
said. ``We have this terrible death toll because we have done a lousy
job at limiting transmission.''

How well Americans will adhere to measures meant to limit the spread of
the virus has been one of the hardest things to predict, experts said,
and may be partially to blame for the underestimates.

\href{https://www.nytimes.com/news-event/coronavirus?action=click\&pgtype=Article\&state=default\&region=MAIN_CONTENT_3\&context=storylines_faq}{}

\hypertarget{the-coronavirus-outbreak-}{%
\subsubsection{The Coronavirus Outbreak
›}\label{the-coronavirus-outbreak-}}

\hypertarget{frequently-asked-questions}{%
\paragraph{Frequently Asked
Questions}\label{frequently-asked-questions}}

Updated August 4, 2020

\begin{itemize}
\item ~
  \hypertarget{i-have-antibodies-am-i-now-immune}{%
  \paragraph{I have antibodies. Am I now
  immune?}\label{i-have-antibodies-am-i-now-immune}}

  \begin{itemize}
  \tightlist
  \item
    As of right
    now,\href{https://www.nytimes.com/2020/07/22/health/covid-antibodies-herd-immunity.html?action=click\&pgtype=Article\&state=default\&region=MAIN_CONTENT_3\&context=storylines_faq}{that
    seems likely, for at least several months.} There have been
    frightening accounts of people suffering what seems to be a second
    bout of Covid-19. But experts say these patients may have a
    drawn-out course of infection, with the virus taking a slow toll
    weeks to months after initial exposure. People infected with the
    coronavirus typically
    \href{https://www.nature.com/articles/s41586-020-2456-9}{produce}
    immune molecules called antibodies, which are
    \href{https://www.nytimes.com/2020/05/07/health/coronavirus-antibody-prevalence.html?action=click\&pgtype=Article\&state=default\&region=MAIN_CONTENT_3\&context=storylines_faq}{protective
    proteins made in response to an
    infection}\href{https://www.nytimes.com/2020/05/07/health/coronavirus-antibody-prevalence.html?action=click\&pgtype=Article\&state=default\&region=MAIN_CONTENT_3\&context=storylines_faq}{.
    These antibodies may} last in the body
    \href{https://www.nature.com/articles/s41591-020-0965-6}{only two to
    three months}, which may seem worrisome, but that's perfectly normal
    after an acute infection subsides, said Dr. Michael Mina, an
    immunologist at Harvard University. It may be possible to get the
    coronavirus again, but it's highly unlikely that it would be
    possible in a short window of time from initial infection or make
    people sicker the second time.
  \end{itemize}
\item ~
  \hypertarget{im-a-small-business-owner-can-i-get-relief}{%
  \paragraph{I'm a small-business owner. Can I get
  relief?}\label{im-a-small-business-owner-can-i-get-relief}}

  \begin{itemize}
  \tightlist
  \item
    The
    \href{https://www.nytimes.com/article/small-business-loans-stimulus-grants-freelancers-coronavirus.html?action=click\&pgtype=Article\&state=default\&region=MAIN_CONTENT_3\&context=storylines_faq}{stimulus
    bills enacted in March} offer help for the millions of American
    small businesses. Those eligible for aid are businesses and
    nonprofit organizations with fewer than 500 workers, including sole
    proprietorships, independent contractors and freelancers. Some
    larger companies in some industries are also eligible. The help
    being offered, which is being managed by the Small Business
    Administration, includes the Paycheck Protection Program and the
    Economic Injury Disaster Loan program. But lots of folks have
    \href{https://www.nytimes.com/interactive/2020/05/07/business/small-business-loans-coronavirus.html?action=click\&pgtype=Article\&state=default\&region=MAIN_CONTENT_3\&context=storylines_faq}{not
    yet seen payouts.} Even those who have received help are confused:
    The rules are draconian, and some are stuck sitting on
    \href{https://www.nytimes.com/2020/05/02/business/economy/loans-coronavirus-small-business.html?action=click\&pgtype=Article\&state=default\&region=MAIN_CONTENT_3\&context=storylines_faq}{money
    they don't know how to use.} Many small-business owners are getting
    less than they expected or
    \href{https://www.nytimes.com/2020/06/10/business/Small-business-loans-ppp.html?action=click\&pgtype=Article\&state=default\&region=MAIN_CONTENT_3\&context=storylines_faq}{not
    hearing anything at all.}
  \end{itemize}
\item ~
  \hypertarget{what-are-my-rights-if-i-am-worried-about-going-back-to-work}{%
  \paragraph{What are my rights if I am worried about going back to
  work?}\label{what-are-my-rights-if-i-am-worried-about-going-back-to-work}}

  \begin{itemize}
  \tightlist
  \item
    Employers have to provide
    \href{https://www.osha.gov/SLTC/covid-19/standards.html}{a safe
    workplace} with policies that protect everyone equally.
    \href{https://www.nytimes.com/article/coronavirus-money-unemployment.html?action=click\&pgtype=Article\&state=default\&region=MAIN_CONTENT_3\&context=storylines_faq}{And
    if one of your co-workers tests positive for the coronavirus, the
    C.D.C.} has said that
    \href{https://www.cdc.gov/coronavirus/2019-ncov/community/guidance-business-response.html}{employers
    should tell their employees} -\/- without giving you the sick
    employee's name -\/- that they may have been exposed to the virus.
  \end{itemize}
\item ~
  \hypertarget{should-i-refinance-my-mortgage}{%
  \paragraph{Should I refinance my
  mortgage?}\label{should-i-refinance-my-mortgage}}

  \begin{itemize}
  \tightlist
  \item
    \href{https://www.nytimes.com/article/coronavirus-money-unemployment.html?action=click\&pgtype=Article\&state=default\&region=MAIN_CONTENT_3\&context=storylines_faq}{It
    could be a good idea,} because mortgage rates have
    \href{https://www.nytimes.com/2020/07/16/business/mortgage-rates-below-3-percent.html?action=click\&pgtype=Article\&state=default\&region=MAIN_CONTENT_3\&context=storylines_faq}{never
    been lower.} Refinancing requests have pushed mortgage applications
    to some of the highest levels since 2008, so be prepared to get in
    line. But defaults are also up, so if you're thinking about buying a
    home, be aware that some lenders have tightened their standards.
  \end{itemize}
\item ~
  \hypertarget{what-is-school-going-to-look-like-in-september}{%
  \paragraph{What is school going to look like in
  September?}\label{what-is-school-going-to-look-like-in-september}}

  \begin{itemize}
  \tightlist
  \item
    It is unlikely that many schools will return to a normal schedule
    this fall, requiring the grind of
    \href{https://www.nytimes.com/2020/06/05/us/coronavirus-education-lost-learning.html?action=click\&pgtype=Article\&state=default\&region=MAIN_CONTENT_3\&context=storylines_faq}{online
    learning},
    \href{https://www.nytimes.com/2020/05/29/us/coronavirus-child-care-centers.html?action=click\&pgtype=Article\&state=default\&region=MAIN_CONTENT_3\&context=storylines_faq}{makeshift
    child care} and
    \href{https://www.nytimes.com/2020/06/03/business/economy/coronavirus-working-women.html?action=click\&pgtype=Article\&state=default\&region=MAIN_CONTENT_3\&context=storylines_faq}{stunted
    workdays} to continue. California's two largest public school
    districts --- Los Angeles and San Diego --- said on July 13, that
    \href{https://www.nytimes.com/2020/07/13/us/lausd-san-diego-school-reopening.html?action=click\&pgtype=Article\&state=default\&region=MAIN_CONTENT_3\&context=storylines_faq}{instruction
    will be remote-only in the fall}, citing concerns that surging
    coronavirus infections in their areas pose too dire a risk for
    students and teachers. Together, the two districts enroll some
    825,000 students. They are the largest in the country so far to
    abandon plans for even a partial physical return to classrooms when
    they reopen in August. For other districts, the solution won't be an
    all-or-nothing approach.
    \href{https://bioethics.jhu.edu/research-and-outreach/projects/eschool-initiative/school-policy-tracker/}{Many
    systems}, including the nation's largest, New York City, are
    devising
    \href{https://www.nytimes.com/2020/06/26/us/coronavirus-schools-reopen-fall.html?action=click\&pgtype=Article\&state=default\&region=MAIN_CONTENT_3\&context=storylines_faq}{hybrid
    plans} that involve spending some days in classrooms and other days
    online. There's no national policy on this yet, so check with your
    municipal school system regularly to see what is happening in your
    community.
  \end{itemize}
\end{itemize}

Dr. Fauci and Dr. Deborah L. Birx, the Trump administration's
coronavirus response coordinator, estimated in March that the virus
could kill between 100,000 and 240,000 people in the United States, even
with preventive measures. In early April, Dr. Fauci
\href{https://www.today.com/video/dr-anthony-fauci-virus-death-toll-may-be-more-like-60-000-than-100-000-to-200-000-81825861735}{said
on the ``Today'' show} that he thought deaths may never reach 100,000.

``Models are really only as good as the assumptions that you put into
the model, but when you start to see real data, you can modify that
model, and the real data are telling us that it is highly likely that
we're having a definite positive effect,'' Dr. Fauci said on the show,
later adding: ``It looks more like the 60,000 than the 100,000 to
200,000. But having said that, we'd better be careful that we don't say,
`OK, we're doing so well, we can pull back.'''

By May 1,
\href{https://www.nytimes.com/2020/04/20/us/coronavirus-us-hot-spots-reopening.html}{several
states} were
\href{https://www.nytimes.com/interactive/2020/us/states-reopen-map-coronavirus.html}{reopening
gyms, salons, restaurants and other businesses}. Mr. Trump, who has
\href{https://www.cnn.com/2020/05/04/politics/trump-rising-coronavirus-death-estimates/index.html}{given
a wide range of predictions for the ultimate death count}, said on May 3
that the virus might end up killing 100,000 people, after saying for
much of April that the virus would not kill more than 75,000.

The next day, the Institute for Health Metrics and Evaluation, whose
model is closely watched by the White House,
\href{http://www.healthdata.org/news-release/new-ihme-forecast-projects-nearly-135000-covid-19-deaths-us}{increased
its own projection}, warning that there would likely be about 135,000
deaths by early August. The institute's model, which includes a wide
range of possible scenarios, now
\href{https://covid19.healthdata.org/united-states-of-america}{projects
about 220,000 deaths by November}.

Adding to the difficulty of predicting human behavior, Professor Pitzer
said, is that public policy can be influenced by the models: seeing a
forecast may prompt officials to take actions that make the forecast
less likely to come true.

``Models are useful for playing out scenarios, but they're not really
meant to be accurate in generating long-term predictions,'' she said.
``They can be good at short-term forecasting --- what might happen in
the next couple of weeks. But longer term, knowing exactly what the
trajectory of the epidemic will be ---~ there are just too many
variables.''

Sarah Mervosh contributed reporting.

Advertisement

\protect\hyperlink{after-bottom}{Continue reading the main story}

\hypertarget{site-index}{%
\subsection{Site Index}\label{site-index}}

\hypertarget{site-information-navigation}{%
\subsection{Site Information
Navigation}\label{site-information-navigation}}

\begin{itemize}
\tightlist
\item
  \href{https://help.nytimes.com/hc/en-us/articles/115014792127-Copyright-notice}{©~2020~The
  New York Times Company}
\end{itemize}

\begin{itemize}
\tightlist
\item
  \href{https://www.nytco.com/}{NYTCo}
\item
  \href{https://help.nytimes.com/hc/en-us/articles/115015385887-Contact-Us}{Contact
  Us}
\item
  \href{https://www.nytco.com/careers/}{Work with us}
\item
  \href{https://nytmediakit.com/}{Advertise}
\item
  \href{http://www.tbrandstudio.com/}{T Brand Studio}
\item
  \href{https://www.nytimes.com/privacy/cookie-policy\#how-do-i-manage-trackers}{Your
  Ad Choices}
\item
  \href{https://www.nytimes.com/privacy}{Privacy}
\item
  \href{https://help.nytimes.com/hc/en-us/articles/115014893428-Terms-of-service}{Terms
  of Service}
\item
  \href{https://help.nytimes.com/hc/en-us/articles/115014893968-Terms-of-sale}{Terms
  of Sale}
\item
  \href{https://spiderbites.nytimes.com}{Site Map}
\item
  \href{https://help.nytimes.com/hc/en-us}{Help}
\item
  \href{https://www.nytimes.com/subscription?campaignId=37WXW}{Subscriptions}
\end{itemize}
