Sections

SEARCH

\protect\hyperlink{site-content}{Skip to
content}\protect\hyperlink{site-index}{Skip to site index}

\href{https://www.nytimes.com/section/politics}{Politics}

\href{https://myaccount.nytimes.com/auth/login?response_type=cookie\&client_id=vi}{}

\href{https://www.nytimes.com/section/todayspaper}{Today's Paper}

\href{/section/politics}{Politics}\textbar{}Ex-C.I.A. Chief Criticizes
Silence of Top Republicans on Russian Interference

\url{https://nyti.ms/3gnHhKA}

\begin{itemize}
\item
\item
\item
\item
\item
\end{itemize}

Advertisement

\protect\hyperlink{after-top}{Continue reading the main story}

Supported by

\protect\hyperlink{after-sponsor}{Continue reading the main story}

\hypertarget{ex-cia-chief-criticizes-silence-of-top-republicans-on-russian-interference}{%
\section{Ex-C.I.A. Chief Criticizes Silence of Top Republicans on
Russian
Interference}\label{ex-cia-chief-criticizes-silence-of-top-republicans-on-russian-interference}}

A new book by John Brennan is another salvo in the political debate over
the intelligence community's investigation of the election interference
campaign.

\includegraphics{https://static01.nyt.com/images/2020/07/29/us/politics/29dc-brennan/merlin_166188930_b4655867-0b29-4b35-9afa-30aaf082185f-articleLarge.jpg?quality=75\&auto=webp\&disable=upscale}

\href{https://www.nytimes.com/by/julian-e-barnes}{\includegraphics{https://static01.nyt.com/images/2019/12/13/reader-center/author-julian-barnes/author-julian-barnes-thumbLarge.png}}

By \href{https://www.nytimes.com/by/julian-e-barnes}{Julian E. Barnes}

\begin{itemize}
\item
  July 29, 2020
\item
  \begin{itemize}
  \item
  \item
  \item
  \item
  \item
  \end{itemize}
\end{itemize}

WASHINGTON --- In the final days of the Obama administration,
intelligence officials laid out evidence of Russia's 2016 election
interference campaign to congressional leaders, prompting Nancy Pelosi,
then the House minority leader, to press for Moscow to be punished.

``This can't happen again,'' she said. Two Republicans, Speaker Paul D.
Ryan and Senator Richard M. Burr, the chairman of the Senate
Intelligence Committee, concurred.

But two other Republicans --- Mitch McConnell, the Senate majority
leader, and Representative Devin Nunes, then the chairman of the House
Intelligence Committee --- remained silent, according to a new account
of the so-called Gang of Eight meeting by John O. Brennan, the former
C.I.A. director.

``I was not surprised that McConnell and Nunes, early and ardent
partisan defenders of Mr. Trump, were silent in the face of what
everyone else recognized was a clear national security threat,'' Mr.
Brennan wrote in his forthcoming book, ``Undaunted.'' The New York Times
obtained an excerpt from the memoir, which
\href{https://www.washingtonpost.com/national-security/ex-cia-director-brennan-writes-in-upcoming-memoir-that-trump-blocked-access-to-records-and-notes/2020/07/28/f70b833e-d0f1-11ea-9038-af089b63ac21_story.html}{The
Washington Post first reported on}.

Mr. Brennan's book, and the early excerpt, amount to salvos in the
continuing political tug of war over the 2016 Russian interference
campaign and the subsequent investigations of the Trump campaign's ties
to Moscow.

The next few months are shaping up to be a season of warring accounts
about the events of 2016 and the intelligence agencies' reports about
Russian interference. Mr. Brennan's book is scheduled to be published in
October by Celadon Books, and other former law enforcement officials who
investigated Russia's interference are writing books due out in
September.

Attorney General William P. Barr, who has harshly criticized the early
efforts to examine ties between the Trump campaign and Russia's
interference, declined this week to commit to waiting until after the
election to release any report by the criminal prosecutor he appointed
to look at the origins of the Russia investigation. That prosecutor,
John H. Durham, the U.S. attorney in Connecticut, is said to be
examining the role of Mr. Brennan in helping shape the Obama
administration's assessment of the Russian interference campaign,
according to current and former officials.

Many Republicans now view the early 2017 intelligence community
assessment, which Mr. Brennan was instrumental in helping craft, as
flawed, especially the view that President Vladimir V. Putin of Russia
favored Mr. Trump. They hope the Durham investigation will show that
concern over Russian support for Mr. Trump was overblown and the result
of political opposition to him.

Representatives of neither Mr. McConnell nor Mr. Nunes returned requests
for comment about the intelligence briefing.

Mr. Brennan portrayed himself in the excerpt as deeply skeptical of Mr.
Trump and his ability to keep secrets. He said that even before the
election he was convinced of Mr. Trump's ``dishonesty, unabashed
self-aggrandizement and demagogic rhetoric.''

The congressional briefing occurred a day after President Barack Obama
was told about the intelligence agencies' assessment and just hours
before officials briefed Mr. Trump, then the president-elect. He
expressed skepticism and doubts about the findings on Russian
interference.

Mr. Brennan wrote that he had decided ahead of time not to share details
with Mr. Trump about how the C.I.A. learned about Russian intentions or
interference efforts. Sources and methods of information-gathering are
among the intelligence agencies' most closely held material.

``I had serious doubts that Donald Trump would protect our nation's most
vital secrets,'' Mr. Brennan wrote.

During the briefing, Mr. Trump's questions ``revealed that he was
uninterested in finding out what the Russians had done or holding them
to account,'' Mr. Brennan wrote. Instead, Mr. Brennan said, Mr. Trump
seemed intent on challenging the intelligence and the agencies' judgment
that Russia had interfered in the election.

Mr. Brennan said in the book that he believed that Mr. Trump was seeking
to learn what the C.I.A. knew and how the agency had gathered the
intelligence. ``This deeply troubled me, as I worried about what he
might do with the information he was being given,'' Mr. Brennan wrote.

Soon after he stepped down as the C.I.A. director, Mr. Brennan began
criticizing Mr. Trump vociferously in appearances on MSNBC, where he
became a contributor, and on Twitter. In 2018, after Mr. Trump's joint
news conference with Mr. Putin where he
\href{https://www.nytimes.com/2018/07/16/world/europe/trump-putin-election-intelligence.html}{expressed
skepticism} about Russian interference in the election, Mr. Brennan
wrote that the president's comments were
``\href{https://twitter.com/johnbrennan/status/1018885971104985093?lang=en}{nothing
short of treasonous}.''

Mr. Brennan's book has been
\href{https://twitter.com/nick_shapiro/status/1288466966135500800}{cleared
for publication by the C.I.A.}, but he was not allowed to review any
classified materials while he researched it.

In 2018, the White House ordered Mr. Brennan's security clearance
stripped. Mr. Brennan wrote that his clearance was never revoked but
that the White House had also ordered intelligence agencies to block his
access to classified material, which the C.I.A. did.

Mr. Brennan, unlike previous directors who have written memoirs, was
able to review only redacted, declassified versions of his calendars and
other notes.

In his opening chapter, Mr. Brennan expressed disappointment that one of
his successors, Gina Haspel, did not respond to his letter to discuss
his access.

``So much for my fervent hope that interactions with my successors would
be unencumbered by Washington's partisan waters,'' Mr. Brennan wrote.

Advertisement

\protect\hyperlink{after-bottom}{Continue reading the main story}

\hypertarget{site-index}{%
\subsection{Site Index}\label{site-index}}

\hypertarget{site-information-navigation}{%
\subsection{Site Information
Navigation}\label{site-information-navigation}}

\begin{itemize}
\tightlist
\item
  \href{https://help.nytimes.com/hc/en-us/articles/115014792127-Copyright-notice}{©~2020~The
  New York Times Company}
\end{itemize}

\begin{itemize}
\tightlist
\item
  \href{https://www.nytco.com/}{NYTCo}
\item
  \href{https://help.nytimes.com/hc/en-us/articles/115015385887-Contact-Us}{Contact
  Us}
\item
  \href{https://www.nytco.com/careers/}{Work with us}
\item
  \href{https://nytmediakit.com/}{Advertise}
\item
  \href{http://www.tbrandstudio.com/}{T Brand Studio}
\item
  \href{https://www.nytimes.com/privacy/cookie-policy\#how-do-i-manage-trackers}{Your
  Ad Choices}
\item
  \href{https://www.nytimes.com/privacy}{Privacy}
\item
  \href{https://help.nytimes.com/hc/en-us/articles/115014893428-Terms-of-service}{Terms
  of Service}
\item
  \href{https://help.nytimes.com/hc/en-us/articles/115014893968-Terms-of-sale}{Terms
  of Sale}
\item
  \href{https://spiderbites.nytimes.com}{Site Map}
\item
  \href{https://help.nytimes.com/hc/en-us}{Help}
\item
  \href{https://www.nytimes.com/subscription?campaignId=37WXW}{Subscriptions}
\end{itemize}
