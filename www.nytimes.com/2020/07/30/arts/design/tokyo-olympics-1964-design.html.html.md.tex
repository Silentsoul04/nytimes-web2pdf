Sections

SEARCH

\protect\hyperlink{site-content}{Skip to
content}\protect\hyperlink{site-index}{Skip to site index}

\href{https://www.nytimes.com/section/arts/design}{Art \& Design}

\href{https://myaccount.nytimes.com/auth/login?response_type=cookie\&client_id=vi}{}

\href{https://www.nytimes.com/section/todayspaper}{Today's Paper}

\href{/section/arts/design}{Art \& Design}\textbar{}The 1964 Olympics
Certified a New Japan, in Steel and on the Screen

\url{https://nyti.ms/3hPSap2}

\begin{itemize}
\item
\item
\item
\item
\item
\end{itemize}

\href{https://www.nytimes.com/spotlight/at-home?action=click\&pgtype=Article\&state=default\&region=TOP_BANNER\&context=at_home_menu}{At
Home}

\begin{itemize}
\tightlist
\item
  \href{https://www.nytimes.com/2020/07/28/books/time-for-a-literary-road-trip.html?action=click\&pgtype=Article\&state=default\&region=TOP_BANNER\&context=at_home_menu}{Take:
  A Literary Road Trip}
\item
  \href{https://www.nytimes.com/2020/07/29/magazine/bored-with-your-home-cooking-some-smoky-eggplant-will-fix-that.html?action=click\&pgtype=Article\&state=default\&region=TOP_BANNER\&context=at_home_menu}{Cook:
  Smoky Eggplant}
\item
  \href{https://www.nytimes.com/2020/07/27/travel/moose-michigan-isle-royale.html?action=click\&pgtype=Article\&state=default\&region=TOP_BANNER\&context=at_home_menu}{Look
  Out: For Moose}
\item
  \href{https://www.nytimes.com/interactive/2020/at-home/even-more-reporters-editors-diaries-lists-recommendations.html?action=click\&pgtype=Article\&state=default\&region=TOP_BANNER\&context=at_home_menu}{Explore:
  Reporters' Obsessions}
\end{itemize}

Advertisement

\protect\hyperlink{after-top}{Continue reading the main story}

Supported by

\protect\hyperlink{after-sponsor}{Continue reading the main story}

Critic's Notebook

\hypertarget{the-1964-olympics-certified-a-new-japan-in-steel-and-on-the-screen}{%
\section{The 1964 Olympics Certified a New Japan, in Steel and on the
Screen}\label{the-1964-olympics-certified-a-new-japan-in-steel-and-on-the-screen}}

The world's elite athletes would have been in Tokyo right now if not for
the coronavirus pandemic. When they went half a century ago, they
discovered a capital transformed by design.

By \href{https://www.nytimes.com/by/jason-farago}{Jason Farago}

\begin{itemize}
\item
  July 30, 2020
\item
  \begin{itemize}
  \item
  \item
  \item
  \item
  \item
  \end{itemize}
\end{itemize}

\includegraphics{https://static01.nyt.com/images/2020/07/31/arts/30olympics-notebook6/30olympics-notebook5-02-articleLarge.jpg?quality=75\&auto=webp\&disable=upscale}

This weekend ought to have been the midway point of the Summer Olympics
in Tokyo, which would have gathered the world's leading runners,
jumpers, throwers, lifters and --- for the first time --- skateboarders
in the world's most populous city. May the Simone Biles fan club forgive
me, but the event I was most excited about was handball.

Not for the sport, but for the stadium: Handball matches were to take
place in the Yoyogi National Gymnasium, a landmark of Japanese modern
architecture designed by Kenzo Tange. The stadium is defined by its
massive, plunging roof, formed from two catenaries --- steel cables
stretched between concrete pillars, like a suspension bridge --- and the
perpendicular ribs that sweep down from those axes to the floor. Years
ago, biking through Yoyogi Park, I remember stopping dead before the
gymnasium's soldered roof panels, marveling at its canopies of steel. It
might have been the most glamorous venue of this year's Olympics, even
though it was built more than half a century ago.

Image

Another view of the Yoyogi National Gymnasium shows the draping of its
massive roof.~Credit...Keystone Features/Hulton Archive, via Getty
Images

Image

A marching band performing at the '64 Games.Credit...Art Rickerby/The
LIFE Picture Collection, via Getty Images

Image

Local residents at the Komazawa Olympic Park preparing for the opening
of the Olympic Games.Credit...Bettmann/Getty Images

The coronavirus pandemic has forced the Olympics's first postponement:
Tokyo 2020,
\href{https://www.nytimes.com/2020/03/25/world/asia/japan-olympics-delay.html}{its
name unchanged}, will now take place in July 2021
\href{https://www.nytimes.com/2020/07/19/sports/2021-tokyo-olympics-protocols.html}{if
it takes place at all}. Yet all around the Japanese capital is the
legacy of another Olympics: the 1964 Summer Games, which crowned Tokyo's
20-year transformation from a firebombed ruin to an ultramodern
megalopolis. (Actually, the ``summer'' Games were held in autumn;
organizers thought October in Tokyo would be smarter than
\href{https://www.nytimes.com/2019/10/10/sports/tokyo-braces-for-the-hottest-olympics-ever.html}{sweltering
July}.) Those first Tokyo Olympics served as a debutante ball for
democratic, postwar Japan, which reintroduced itself to the world not
only through sport but also through design.

The preparations turned Tokyo into a citywide construction site.
\href{https://www.japantimes.co.jp/sports/2014/10/10/olympics/olympic-construction-transformed-tokyo/}{The
author Robert Whiting}, who was stationed with the U.S. Air Force in
Tokyo in 1962, describes the pile drivers and jackhammers that delivered
an ``overwhelming assault on the senses.'' Pedestrians went about with
face masks and earplugs, and salarymen drank in bars protected by
dust-blocking plastic sheets. Japan was just a few years out from
becoming the world's second-largest economy, and the 1964 Olympics were
to be a pageant of economic revival and honor regained.

Trolleys went out, elevated highways came in. The city got a new sewer
system, a new port, two new subway lines, and serious pollution. Slums,
and their residents, were mercilessly cleared to make room for new
construction, some of it grand --- like the exquisite
\href{https://www.studionicholson.com/blogs/features/hotel-okura-tokyo\#:~:text=The\%20Okura\%20was\%20built\%20in,back\%20modernism\%20of\%20the\%20time.}{Hotel
Okura}, designed in 1962 by Yoshiro Taniguchi (father of the MoMA
architect Yoshio Taniguchi) --- and much forgettable. The new
shinkansen, or bullet train, hurried between Tokyo and Osaka for the
first time just one week before the opening ceremony. Not until 2008,
when the Games opened in booming Beijing, would an Olympics so
profoundly alter a city and a nation.

Image

Reshaping a city: An elevated expressway was constructed in the
Akasaka-mitsuke section of Tokyo, ahead of the Olympics. A bullet train,
between Tokyo and Osaka, was also among the infrastructure
changes.Credit...Keystone/Hulton Archive, via Getty Images

Tokyo had been awarded the Games once before; it was meant to host the
canceled 1940 Olympics, succeeding the Nazi spectacle in Berlin in 1936.
The architects and designers of the 1964 Games therefore had to satisfy
a clear ideological goal: This was to be a showcase of New Japan,
pacifist and forward-dawning, largely free of classical Japanese
aesthetics or traditional national symbols. No Fuji, no cherry blossoms,
and no calligraphy. And any expression of national pride had to be as
distanced as possible from the old imperial militarism.

Devising the look of Tokyo '64 fell to
\href{http://adcglobal.org/hall-of-fame/yusaku-kamekura/}{Yusaku
Kamekura}, the dean of Japanese graphic designers, who had imbibed
modern design from the Bauhaus-trained professors of Tokyo's Institute
of New Architecture and Industrial Arts. Where past Olympics posters had
relied on figurative, often explicitly Greco-Roman imagery, Kamekura
distilled Tokyo's ambitions to the simplest of emblems: the five
interlocking rings, all gold, topped by a huge red disc, the rising sun.

Kamekura's poster didn't just spurn western expectations of the
``exotic'' Orient for hard, clean modernity. More impressive than that,
it rebooted the Japanese flag --- which was all but banned during the
first years of American occupation --- as a symbol for a democratic
state. The same bold aesthetic would also characterize
Kamekura's\href{https://www.moma.org/collection/works/8783}{second (and,
for 1964, technically daunting) Olympics poster}, with a full-bleed,
split-second photograph of runners against a black background.

Image

Yusaku Kamekura's design for the Olympic poster conveyed Tokyo's
ambitions: the five interlocking rings, topped by a huge red disc, the
rising sun.Credit...Yusaku Kamekura, via The Museum of Modern Art, New
York

The main ceremonies and athletic events took place at a nothing-special
stadium that has since been demolished. In the Komazawa Olympic Park in
Setagaya, a control tower designed by
\href{https://www.timeout.com/tokyo/art/the-yoshinobu-ashihara-architectural-archives-dreaming-modernism}{Yoshinobu
Ashihara} took the form of a 165-foot-tall concrete tree; it's still
standing, though its brutalist candor has been softened with a
shellacking of white paint. It was, however, the somewhat smaller
stadium in Yoyogi, designed by Tange --- who would go on to build
Tokyo's towering city hall and its Sofia Coppola-approved Park Hyatt
Hotel --- that expressed in concrete what Kamekura and the other
designers did on paper.

In 1964 Tange's stadium hosted the swimming, diving and basketball
events, and its marriage of brawn and dynamism broadcast more loudly
than any other that Japan had been restored, even reborn. From the
outside, it looks like two misconjoined halves of a sliced pair,
rendered in steel and concrete, though its real innovation was the roof.
Its tensile structure elaborates on Eero Saarinen's recently completed
\href{https://sportsandrecreation.yale.edu/gallery/ingalls-rink-gallery}{hockey
rink at Yale University}, and, even more, the Philips Pavilion at the
Brussels World's Fair, designed in 1958 by his hero Le Corbusier.

More quietly, the gymnasium nods to Tange's most significant work up to
this point:
\href{https://www.nytimes.com/2016/05/28/world/asia/obama-hiroshima-japan.html}{the
cenotaph arch in Hiroshima}, another curve of reinforced concrete. In
Hiroshima, Tange's arcing concrete became a mausoleum for Japan's
darkest hour; in Tokyo, it enclosed a festival of new national life.
(The legacy of Hiroshima also suffused the opening ceremony, where the
sprinter Yoshinori Sakai --- born on Aug. 6, 1945, the day the first
atom bomb fell --- lit the caldron.)

Image

A landmark of Japanese modern architecture: Construction on the steel
struts curving up to a concrete support tower at Yoyogi National
Gymnasium, in March 1964.Credit...The Asahi Shimbun, via Getty Images

Image

The draping, plunging canopies inside the gymnasium.Credit...The Asahi
Shimbun, via Getty Images

Image

Kenzo Tange's ambitious designs for Yoyogi Gymnasium: He developed new
technology for reinforced concrete, and the building's suspension roof
was the world's largest at the time.Credit...via the Frances Loeb
Library/Harvard University Graduate School of Design

The 1964 Olympics were the first to be
\href{https://www.nytimes.com/1964/07/23/archives/tv-satellite-test-to-show-olympics-tokyo-games-to-be-seen-in-us.html}{broadcast
worldwide}, via the first geostationary satellite for commercial use,
and Japanese families with growing household budgets could even watch
the Games in color. Nevertheless, the most enduring images from Tokyo
'64 appeared in the cinema, in the director
\href{https://www.nytimes.com/1964/04/19/archives/shooting-an-oriental-olympiad.html}{Kon
Ichikawa}'s three-hour documentary
``\href{https://www.criterion.com/films/709-tokyo-olympiad}{Tokyo
Olympiad}.'' Shot in the wide CinemaScope format, in rich color, with
newfangled telephoto lenses, ``Tokyo Olympiad'' is, by several lengths
of the track, the greatest film ever made about the Olympics. (You can
stream it, along with much drearier movies of the Games from 1912 to
2012, on the
\href{https://www.criterionchannel.com/100-years-of-olympic-films-1912-2012}{Criterion
Channel}.)

Unlike Leni Riefenstahl's ``Olympia,'' which prefaced the Berlin Games
with Aryan athlete-gods in Greek drag, ``Tokyo Olympiad'' plunges us
into modernity from its opening sequence: a blazing white sun against a
red sky --- the Japanese flag, inverted --- smash-cuts into a wrecking
ball slamming into pylons. Building facades tumble to powder, bulldozers
haul away rubble. We see Tange's stadium in mist, then the torch relay,
and then crowds jostling to see the young foreigners arriving at Haneda
Airport. Inside the stadiums, the telephoto lens allowed Ishikawa to get
stunning close-ups of the sprinters' sweat and the swimmers' gooseflesh,
but just as often he shot nearly abstract sequences of fencers and
cyclists blurred into streams of color.

Image

Hurdlers compete, above, and~the sprinter Yoshinori Sakai lights the
caldron, below, at the 1964 Tokyo Olympics as seen in ``Tokyo
Olympiad.'' A new 4K digital restoration of the 1965 documentary film,
directed by Kon Ichikawa, is streaming on Criterion
Channel.~Credit...via the Criterion Collection

Image

Credit...via Criterion Collection

There are champions and record-breakers in ``Tokyo Olympiad,'' but they
share screen time with last-place finishers. Gold-medal matches get
intercut with overlooked details of attendants sweeping the triple jump
track, or shot put officials wheeling away the metal balls. The Japanese
Olympic Committee hated the film and commissioned another; nationalist
boosters called it unpatriotic or worse. But Ichikawa's distillation of
national ambition into abstract form was the hallmark of Tokyo '64, and
``Tokyo Olympiad'' went on to become Japan's biggest domestic box office
success, a record that would stand for four decades.

Whether they happen in 2021 or not at all, the upcoming Tokyo Games will
surely have a quieter cultural impact than their predecessor's. The
first logo for Tokyo 2020 was
\href{https://www.bbc.com/news/world-asia-34115750}{thrown out}, on
grounds of supposed plagiarism. The first stadium, too: Zaha Hadid's
initial design got dumped, and was replaced by a more serene and much
less expensive wood stadium, designed by the architect Kengo Kuma.

If Tange's steel and concrete expressed Japanese ambitions in 1964, now
it is natural materials that point to a vision of a future whose
challenges are as much ecological as economic. But Mr. Kuma, who
attended the 1964 Games as a child, credits Tange's swooping stadium as
the trigger for his own architectural career. ``Tange treated natural
light like a magician,''
\href{https://www.nytimes.com/2018/02/15/t-magazine/kengo-kuma-architect.html}{Mr.
Kuma told the Times two years ago}, reminiscing on his childhood
discovery of Yoyogi National Gymnasium. ``From that day, I wanted to be
an architect.''

Advertisement

\protect\hyperlink{after-bottom}{Continue reading the main story}

\hypertarget{site-index}{%
\subsection{Site Index}\label{site-index}}

\hypertarget{site-information-navigation}{%
\subsection{Site Information
Navigation}\label{site-information-navigation}}

\begin{itemize}
\tightlist
\item
  \href{https://help.nytimes.com/hc/en-us/articles/115014792127-Copyright-notice}{©~2020~The
  New York Times Company}
\end{itemize}

\begin{itemize}
\tightlist
\item
  \href{https://www.nytco.com/}{NYTCo}
\item
  \href{https://help.nytimes.com/hc/en-us/articles/115015385887-Contact-Us}{Contact
  Us}
\item
  \href{https://www.nytco.com/careers/}{Work with us}
\item
  \href{https://nytmediakit.com/}{Advertise}
\item
  \href{http://www.tbrandstudio.com/}{T Brand Studio}
\item
  \href{https://www.nytimes.com/privacy/cookie-policy\#how-do-i-manage-trackers}{Your
  Ad Choices}
\item
  \href{https://www.nytimes.com/privacy}{Privacy}
\item
  \href{https://help.nytimes.com/hc/en-us/articles/115014893428-Terms-of-service}{Terms
  of Service}
\item
  \href{https://help.nytimes.com/hc/en-us/articles/115014893968-Terms-of-sale}{Terms
  of Sale}
\item
  \href{https://spiderbites.nytimes.com}{Site Map}
\item
  \href{https://help.nytimes.com/hc/en-us}{Help}
\item
  \href{https://www.nytimes.com/subscription?campaignId=37WXW}{Subscriptions}
\end{itemize}
