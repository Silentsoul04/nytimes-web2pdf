Sections

SEARCH

\protect\hyperlink{site-content}{Skip to
content}\protect\hyperlink{site-index}{Skip to site index}

\href{https://www.nytimes.com/section/arts/television}{Television}

\href{https://myaccount.nytimes.com/auth/login?response_type=cookie\&client_id=vi}{}

\href{https://www.nytimes.com/section/todayspaper}{Today's Paper}

\href{/section/arts/television}{Television}\textbar{}The Reconciliation
Must Be Televised

\url{https://nyti.ms/3goRuXa}

\begin{itemize}
\item
\item
\item
\item
\item
\item
\end{itemize}

\href{https://www.nytimes.com/news-event/george-floyd-protests-minneapolis-new-york-los-angeles?action=click\&pgtype=Article\&state=default\&region=TOP_BANNER\&context=storylines_menu}{Race
and America}

\begin{itemize}
\tightlist
\item
  \href{https://www.nytimes.com/2020/07/26/us/protests-portland-seattle-trump.html?action=click\&pgtype=Article\&state=default\&region=TOP_BANNER\&context=storylines_menu}{Protesters
  Return to Other Cities}
\item
  \href{https://www.nytimes.com/2020/07/24/us/portland-oregon-protests-white-race.html?action=click\&pgtype=Article\&state=default\&region=TOP_BANNER\&context=storylines_menu}{Portland
  at the Center}
\item
  \href{https://www.nytimes.com/2020/07/23/podcasts/the-daily/portland-protests.html?action=click\&pgtype=Article\&state=default\&region=TOP_BANNER\&context=storylines_menu}{Podcast:
  Showdown in Portland}
\item
  \href{https://www.nytimes.com/interactive/2020/07/16/us/black-lives-matter-protests-louisville-breonna-taylor.html?action=click\&pgtype=Article\&state=default\&region=TOP_BANNER\&context=storylines_menu}{45
  Days in Louisville}
\end{itemize}

Advertisement

\protect\hyperlink{after-top}{Continue reading the main story}

Supported by

\protect\hyperlink{after-sponsor}{Continue reading the main story}

The Great Read

Critic's notebook

\hypertarget{the-reconciliation-must-be-televised}{%
\section{The Reconciliation Must Be
Televised}\label{the-reconciliation-must-be-televised}}

What is the next step as America confronts its racism? A broadcast
spectacle, our critic writes, that could look like court, a telethon,
therapy, an Oprah show --- and more.

\includegraphics{https://static01.nyt.com/images/2020/08/02/arts/02truth-WEB/02truth-WEB-articleLarge.jpg?quality=75\&auto=webp\&disable=upscale}

\href{https://www.nytimes.com/by/wesley-morris}{\includegraphics{https://static01.nyt.com/images/2018/06/13/multimedia/author-wesley-morris/author-wesley-morris-thumbLarge.jpg}}

By \href{https://www.nytimes.com/by/wesley-morris}{Wesley Morris}

\begin{itemize}
\item
  July 30, 2020
\item
  \begin{itemize}
  \item
  \item
  \item
  \item
  \item
  \item
  \end{itemize}
\end{itemize}

When someone wants to explain where the country's been since Memorial
Day, they refer to The Moment. ``The Moment,'' at first, seemed to name
a finite period, the killing of George Floyd on May 25, and the
\href{https://www.nytimes.com/2020/06/03/arts/george-floyd-video-racism.html}{moments
his death comprised}. ``The Moment'' then proved spongy quick, absorbing
the bewildering madness of the deaths of Ahmaud Arbery and Breonna
Taylor and expanding into more protests in more corners of the planet
than seemed fathomable. (The demonstrations took place during a
pandemic; The Moment had swelled inside a Moment.) It appealed to people
whose response to such Moments has tended to be less than vociferous ---
white people. White people marched and chanted. They ate tear gas and
pepper spray. White people said ``Black Lives Matter,'' ``systemic
racism'' and, occasionally, ``reparations.''

Questions arose about what The Moment was and what should be asked of
it. The Moment brought us new vision to see old wrongs and emboldened us
to raze and ruin them. The Moment reversed power. Mayors stood among
civilians, the police took a knee, a president had been absconded into a
bunker. This Moment was the sort that Black America had been waiting for
--- when the woke learned to walk, when the Confederate flag ceased
official operation as a security blanket, when even a beloved music trio
had to concede that
\href{https://www.nytimes.com/2020/06/25/arts/music/dixie-chicks-change-name.html}{``Dixie''
no longer becomes them}.

So here we are, still in this Moment, tasked to behold the changing of
names and the signaling of virtue. Waiting for meaningful legislative
reform, seizing matters with civilian hands in the meantime: recasting
jobs; reclaiming parks and pedestals and city streets, these local
reclamations, seemingly one public space at a time. The speed of change
in a country notoriously allergic to it feels like a spree, reckoning as
a marathon of ``Supermarket Sweep.'' We know The Moment is connected to
other moments yet there's a sense in our bones that it differs from
them. Who knows when such a Moment might come along again?

Before it vanishes, the centuries and conditions that produced it
warrant commemoration. They warrant further confrontation, reclamation
and connection. They warrant an event --- broadcast across the country,
over months, not days --- that squares the present with the past, that
explains The Moment to those who say they are, at last, awake to it.
This Moment of historic holding to account, of looking inward, deserves
a commensurate, totalizing event that explains what is being reckoned
with, demanded and hoped for, an experience that rubs between its
fingers the earth upon which all those toppled monuments had so brazenly
stood. The Moment warrants a depth of conversation the United States has
never had. It demands truth and reconciliation.

Other countries have undergone such commissions, tribunals and soul
searching --- among them, El Salvador, Rwanda, Peru, Germany, South
Africa. They recount staggering atrocity --- inconceivable corruption,
organized oppression, genocide. Of their participants, they compel
confession and vulnerability. Of their audience, they require fortitude,
a pillow to wail into, a strong stomach.

\begin{center}\rule{0.5\linewidth}{\linethickness}\end{center}

\hypertarget{a-truth-and-reconciliation-event-in-2020-would-help-make-up-for-150-years-of-missed-opportunities}{%
\subsection{\texorpdfstring{\emph{A truth and reconciliation event in
2020 would help make up for 150 years of missed
opportunities.}}{A truth and reconciliation event in 2020 would help make up for 150 years of missed opportunities.}}\label{a-truth-and-reconciliation-event-in-2020-would-help-make-up-for-150-years-of-missed-opportunities}}

\begin{center}\rule{0.5\linewidth}{\linethickness}\end{center}

This country has flirted with truth and reconciliation. Reconstruction
ended in 1877, a dozen years after the end of the Civil War. It was more
political action than ritual, a campaign of personhood and rights that
ended when racists intimidated it out of existence. In 1968, in the wake
of the racial conflagrations roiling American cities during the mid- to
late 1960s, Gov. Otto Kerner Jr. of Illinois presented the findings of
his so-called riot commission, whose politically moderate and racially
uniform makeup (two of its members were Black; there was one woman) was
strategically cast for ho-hum results. What it delivered to President
Lyndon B. Johnson was, instead, shockingly, comprehensively grim. The
United States, the commission concluded, is a hopelessly divided nation
that has locked its Black citizens in impoverishment and swallowed the
key, that good white folks were out-to-lunch and therefore as culpable
as the white supremacists were malignant.

The conclusions and recommendations were urgent, vast yet granular,
attentive and astringent. The report deduced that, among other things,
the roots of the violence demanded massive housing and police reform, a
serious political and economic commitment to social programs, and higher
taxes. But nothing meaningful came of it. The conclusions were too
overwhelming --- too indicting. Johnson seemed to take the findings
personally. Plus: the money required to confront them was being spent to
prolong the fight in Vietnam. So white America went the opposite
direction, electing Richard Nixon, who ran, in part, on a law-and-order
campaign. The wound festered.

When it was published as a book early in '68, the report became a best
seller. But it ought to have been part of a one-two punch. Part two
should have been a televised, multipart presentation of the commission's
intensive effort: its conclusions, considerable field work and
still-bracing historical contextualizing put before the public,
alongside the disgruntled, despondent, enraged, \emph{hurt} Black
Americans whose circumstances swell the report. The country watched the
cities burn but never met the human beings who lived in them. It didn't
spend days on end hearing Kerner and especially, perhaps, John V.
Lindsay, the mayor of New York and the commission's most popular member,
inveighing against the racism in our marrow. Johnson and Nixon were
essentially able to look the other way.

The nation had become consumed with news of the war. But there was
evident hunger to know more about the terrors at home. Nine years after
the Kerner Report, a century after Reconstruction's abandonment, we got
``Roots'' --- eight nights of generational magnum opus meant to inspire
as much as explain. It was far from the Kerner Report, set in Gambia and
the antebellum South, during the Civil War and its aftermath.

ABC aired ``Roots'' on consecutive nights as a hedge; the network, home
of ``Happy Days,'' had expected a dud, despite the Alex Haley novel it
was based on being a huge hit. The year before, during the bicentennial,
NBC had a ratings smash with the network-television debut of ``Gone With
the Wind.'' ``Roots'' wasn't perfumed with nostalgia. It was, for 1977,
a watershed retrospective, in which a Black family were the heroes, and
the dads from some of America's favorite shows --- ``The Brady Bunch,''
``The Waltons,'' ``Bonanza'' --- played racists. Scores of millions of
people beheld its 12 or so hours; the finale on ABC remains the third
highest-rated television episode ever. Which is to say that we once were
ready to go through something ugly together as a nation.

Neither the times nor the climes are, of course, what they were in '77.
For one thing, most of the country watched that series because there
wasn't much else on. A truth and reconciliation event in 2020 would help
make up for 150 years of missed opportunities. It should be broadcast
live and streamed the way impeachments and inaugurations are; the way
certain trials are. That would require more than just ABC's audacity,
however backhanded. It would need CBS's, NBC's and Fox's; CNN's, BET's
and the Weather Channel's. It would demand the platforms of Netflix,
HBO, Disney+, Hulu and Amazon. There would be no escaping this thing,
since there is no escape in the daily lives of many Americans. We've
marched for systemic reform. This event --- some of it recorded, some
broadcast live --- would tell the horror story of the system, draw
straight lines from slavery to right now and demand the system be
reformed.

\begin{center}\rule{0.5\linewidth}{\linethickness}\end{center}

\hypertarget{what-would-an-american-version-be-court-theater-a-hearing-a-telethon-therapy-tv-church-ken-burns-anna-deavere-smith-each-perhaps--and-more}{%
\subsection{\texorpdfstring{\emph{What would an American version be?
Court, theater, a hearing, a telethon, therapy, TV, church, Ken Burns,
Anna Deavere Smith? Each perhaps --- and
more.}}{What would an American version be? Court, theater, a hearing, a telethon, therapy, TV, church, Ken Burns, Anna Deavere Smith? Each perhaps --- and more.}}\label{what-would-an-american-version-be-court-theater-a-hearing-a-telethon-therapy-tv-church-ken-burns-anna-deavere-smith-each-perhaps--and-more}}

\begin{center}\rule{0.5\linewidth}{\linethickness}\end{center}

In South Africa, in 1996, a Truth and Reconciliation Commission arose
from an agreement to grant amnesty to those who confessed to crimes
against humanity committed during more than four decades of Apartheid.
The commission took statements from 22,000 victims and witnesses;
thousands of people applied for amnesty; and a kind of extralegal trial
ensued in which the perpetrators faced their victims.

Some of the hearings were broadcast on Sundays for two years in hourlong
episodes and some, on very few occasions, were live. Initially, the
government resisted televising them at all but relented to international
pressure. Deborah Hoffman and Frances Reid made a haunting documentary
of the proceedings, focused on a few cases. Released in 2000, it's
called
``\href{https://www.nytimes.com/2000/03/29/movies/film-review-following-south-africa-s-wrenching-road-to-truth.html}{Long
Night's Journey Into Day}''; and in it, you can see why such an event
would be difficult for live production. The hearings were unpredictable
and thorny. Not everyone looking for amnesty was necessarily contrite.
Racial exorcism proved elusive.

What would an American version be? Court, theater, a hearing, a
telethon, therapy, TV, church, Ken Burns, Anna Deavere Smith? Each
perhaps --- and more. Who would make it? I don't know. It could
certainly proceed in conjunction with the minds and imaginations of the
staff within the Smithsonian brain trust and Bryan Stevenson's Equal
Justice Initiative. Who has been keeping C-SPAN going all these many
decades? The production, however, is merely the second hurdle to clear.
The first would be convincing executives that it's worth doing in the
first place. Here's what to say about that: The entertainment industry
itself has more than a century of harm to atone for and ameliorate. Any
company that believes the solution to ``systemic racism'' is ``The
Help'' shouldn't mind a surrender of its airwaves.

Should this event be night after night of that scene in
``\href{https://www.nytimes.com/2016/12/22/movies/hidden-figures-review.html}{Hidden
Figures}'' in which Taraji P. Henson
\href{https://www.youtube.com/watch?v=9j6p7ajuh-E}{unloads on a giant
room} full of white men, including Kevin Costner, that she's always late
because her colored bathroom \emph{is a mile away from her desk}? No.
This wouldn't be an exercise in rage, self-pity or despair, not purely,
although the terrain will, by necessity, be despairing. It wouldn't be a
series of ``white fragility'' lectures, either. What's needed is a
broadcast that could include white Americans awakening to racism but
remains focused on the legacies of the racism itself. There might be
some of the emotional individual confrontation that put so many South
Africans through the wringer. The American version would dare to hold
the country to account and atone.

Would that then mean the duty for reconciliation resides with the
government? Would the commission just be Congress? I hope not --- it
should entail more than elected officials. A mandate for the event would
come as much from the public as from Washington. The power of Kerner's
outfit was that it went out and \emph{heard} people.

There's a blueprint for what I'm proposing, and it's basically sitting
in a vault.
\href{https://www.nytimes.com/2019/06/19/us/politics/slavery-reparations-hearing.html}{House
Bill H.R. 40}, as it's now called, was originally introduced by John
Conyers in 1989. He brought it up repeatedly until he left Congress in
2017. The Commission to Study and Develop Reparation Proposals for
African-Americans currently rests in the hands of Representative Sheila
Jackson Lee. What Conyers, who died in 2019, was asking the bill to do
seems perfectly reasonable --- ``address the fundamental injustice,
cruelty, brutality, and inhumanity of slavery in the United States and
the 13 American colonies between 1619 and 1865 and to establish a
commission to study and consider a national apology and proposal for
reparations for the institution of slavery.''

Why not start with that? The bill is simply calling for a conversation
about reparations. It doesn't demand a dollar be paid and only
insinuates that money is owed. Instead, it simply wants members of
Congress to \emph{talk} about what it would mean for the United States
government to close a wealth gap that it opened and, over centuries,
widened until inequality among the races appears irreconcilable. If
Congress refuses to take it up, Hollywood should adapt it.

\begin{center}\rule{0.5\linewidth}{\linethickness}\end{center}

\hypertarget{slavery-wouldnt-be-the-subject-of-this-televised-reckoning-racism-would}{%
\subsection{\texorpdfstring{\emph{Slavery wouldn't be the subject of
this televised reckoning. Racism
would.}}{Slavery wouldn't be the subject of this televised reckoning. Racism would.}}\label{slavery-wouldnt-be-the-subject-of-this-televised-reckoning-racism-would}}

\begin{center}\rule{0.5\linewidth}{\linethickness}\end{center}

The white people who bought, owned, traded, lashed and raped Black
people are long dead. Their descendants are among us. Slavery, however,
wouldn't be the subject of this televised reckoning. Racism would. A
crucial chunk of a truth and reconciliation broadcast would use the work
of scholars and thinkers like
\href{https://www.nytimes.com/2016/02/22/books/evicted-book-review-matthew-desmond.html}{Matthew
Desmond}, Ta-Nehisi Coates,
\href{https://www.nytimes.com/interactive/2020/06/24/magazine/reparations-slavery.html}{Nikole
Hannah-Jones}, Isabel Wilkerson and Richard Rothstein to enumerate the
means by which the country has prospered from the theft of land and the
strategic denial of housing.

It's both a logical framing and a literary one. A home is a transferable
asset. It is a refuge, a nest, a beacon of welcome, a source of dignity
--- the most basic of needs, and for many people over many too many
decades, outrageously elusive. ``What white Americans have never fully
understood --- but what the Negro can never forget,'' the Kerner Report
concluded back in 1968, ``is that white society is deeply implicated in
the ghetto. White institutions created it, white institutions maintain
it, and white society condones it.''

Weeks could be spent covering housing with, among other things, a series
of documentaries that highlight the many government and
government-backed programs designed to strengthen segregation and
bolster so-called ghettos. You could spend an entire night with the
story of Clyde Ross, the Chicago laborer who wound up as a housing
rights activist and whose travails Coates built his argument around in
``\href{https://www.theatlantic.com/magazine/archive/2014/06/the-case-for-reparations/361631/}{The
Case for Reparations}.''

Weeks more could be spent on law enforcement, telling the story of
America's police force, its roots in enslavement and how racism now
seems so inextricable from policing that calls for its abolition have
migrated from the ideological fringes. Enough police officers, lawyers,
families of the dead, legal scholars and people who are currently and
formerly incarcerated could testify to racism's criminal-justice
puppetry. The same goes for the ways in which nonwhite people are far
likelier to
\href{https://www.theatlantic.com/politics/archive/2018/02/the-trump-administration-finds-that-environmental-racism-is-real/554315/}{live
amid pollution} than white Americans; and the deep, unbreakable
hypocrisies that continue to keep Black children learning separately and
in substandard conditions.

There must be room for the testimony of young Black people whom nobody's
heard of, folks whose hopelessness and alienation, whose fragile
personal ambitions and self-belief, can be traced from here back to the
1980s and the 1960s, back to the disillusionments of the late 1870s
after the government foreclosed Reconstruction. They are my cousins, my
neighbors, my pals. They're between almost every line of the Kerner
Report. And no one is listening to them now.

This reckoning event would in part entail stories of the ways in which
the poison of racism has ruined lives and wrecked families, like the
Rushes of Lowndes County, a sparsely populated, desperately poor patch
of central Alabama. Two years ago, during a congressional hearing,
\href{https://www.facingsouth.org/2020/07/remembering-pamela-sue-rush-death-caused-structural-poverty}{Pamela
Sue Rush} discussed the devastating squalor to which she'd been
relegated for most of her life. Rush was enlisted to become an activist
against her own poverty and poor health care options. In July, she died
of Covid-19.
\href{https://www.pbs.org/newshour/show/in-alabama-racial-disparities-in-health-outcomes-predate-the-pandemic}{She
was 50}.

The country deserves to hear her family discuss her underlying
conditions and how they took hold on the land of the former slave
quarters that held her mobile home. Citizens of Lowndes could inform the
country of their lack of access to plumbing or basic sanitation
services, about their shouldering **** of a **** disproportionate share
of this pandemic. The Rosses could stand before the country and tell of
Clyde's losses, fights and gains. Then we'd hear from the officials and
schemers who neglected and bilked them. Their confessors would be the
likes of Oprah Winfrey, Gayle King, Terry Gross, Katie Couric, Trevor
Noah, Brian Lehrer, Cheryl Strayed and Connie Chung, people who excel at
listening, people whom Americans are used to listening to, people whose
ears seem connected directly to their hearts. The listening feels
important. So does the facilitation of dialogue. This makes someone like
Winfrey critical to the undertaking. She is a
\href{https://www.nytimes.com/2018/06/21/arts/design/oprah-winfrey-smithsonian-national-museum-of-african-american-history-and-culture.html}{pioneer
of televised reckoning} and remains a master facilitator.

In June, in the midst of the protests, Winfrey held a two-night,
existential video conference call that included Hannah-Jones; Atlanta's
mayor, Keisha Lance Bottoms; and the antipoverty activist the Rev.
William Barber II. Titled ``Where Do We Go From Here?'' it made for a
snapshot of a potential commission. For Apple TV+, Winfrey has just
begun holding conversations about our times with thought leaders and
others. She was made for The Moment. ``The Oprah Winfrey Show'' was a
25-year truth and reconciliation commission.

\begin{center}\rule{0.5\linewidth}{\linethickness}\end{center}

\hypertarget{there-is-no-shame-in-entertainment-intended-to-restore-heal-repair-reveal-reframe-to-midwife}{%
\subsection{\texorpdfstring{\emph{There is no shame in entertainment
intended to restore, heal, repair, reveal, reframe, to
midwife.}}{There is no shame in entertainment intended to restore, heal, repair, reveal, reframe, to midwife.}}\label{there-is-no-shame-in-entertainment-intended-to-restore-heal-repair-reveal-reframe-to-midwife}}

\begin{center}\rule{0.5\linewidth}{\linethickness}\end{center}

Over a series of weeks, the scope would zoom out and contract, telling
stories about the country in order to place the lives of individuals in
a national context. We've grown used to television that's both
expansionist and screen-pinching, macrocosmic and personal: ``The
Wire,'' ``Hamilton,'' ``O.J.: Made in America'' and the Michael Jordan
documentary ``The Last Dance.'' We've devoured ``Star Trek,'' ``Star
Wars,'' ``Game of Thrones,'' ``Harry Potter,'' the Boston Red Sox, the
Chicago Cubs, and 70 years of soap operas. Sagas are a food group.
Obviously, barriers exist to wanting and understanding this one. We
think we know it. We don't think we \emph{need} to know it. ``I have a
Black friend.'' ``I saw Ken Burns's `The Civil War.''' ``Obama.''

Every title on that list is, in its way**,** an entertainment, and so,
perhaps, is this event. It should be a spectacle. It shouldn't be
spectacular. Maybe some nights ought to feature gospel choirs and tribal
groups, music by Kendrick Lamar, Lila Downs, Pamyua, Gladys Knight and
Rhiannon Giddens. Maybe every installment should simply feature the
lustrous power of Bernice Johnson Reagon, Rutha Mae Harris and Bettie
Mae Fikes, the voices of the civil rights movement. Fikes is ``the Voice
of Selma'' and the most important living American singer without a
Wikipedia entry. She vowed that the singing she did at John Lewis's
funeral would be her last. Someone should beg her to reconsider.

Entertainment, here, would be a sobering virtue, catharsis rather than a
loophole. There would be serious room made for spiritual address and
cosmic redress; for acknowledgments of country and native land
stewardship; for many nights of Native Americans reinserting themselves
in the nation's narrative, troubling it, setting it right; for breath
work and silence that assists us through the heft of this undertaking.
There should be readings and dancing and photography and bands and
orchestras. **** There would be a place, as well, for comedy, some of
which would arise on its own, some of which might necessitate actual
comedians. Laughter helps.

There is no shame in entertainment intended to restore, heal, repair,
reveal, reframe, to midwife. A belief in that aspect of entertainment is
what once brought historic droves of us to ``Roots.'' We just didn't
know what to do once it was over. That finale ends with its formerly
enslaved family standing atop the hills of Lauderdale County, Tenn., as
though it were the beginning of ``The Sound of Music.''

Feels like a comfort now. But in 1977, the predominant response was a
deep sigh. The Center for Policy Research
\href{https://www.nytimes.com/1977/06/07/archives/blacks-and-whites-found-to-have-misapprehensions-on-impact-of-roots.html?searchResultPosition=9}{polled
500 Black Americans} and 500 white Americans and found that many people
were saddened by ``Roots.'' It was a Moment that ultimately withered.

This Moment didn't come cheaply. It should not be squandered. It should
be nationally witnessed and absorbed. Truth and reconciliation is a
death and a birth, accordingly arduous, tense, procedural, affirming,
painful. The outcome feels secondary to the process. The ritual is the
benefit. The Moment demands that we summon the courage to put ourselves
through it. At last.

\begin{center}\rule{0.5\linewidth}{\linethickness}\end{center}

Photo credits for photo illustration: Warner Bros. and Walt Disney
Television (``Roots''); ESPN Films (``O.J.: MADE in AMERICA''); Mandel
Ngan, via Agence France-Presse --- Getty Images and Harpo Productions
(``The Oprah Winfrey Show''); Colin Urquhart (``Long Night's Journey
Into Day''); HBO (``The Wire''); Sara Krulwich, via The New York Times
(``Hamilton''); 20th Century Fox (``Hidden Figures''); Associated Press
(``The Phil Donahue Show''

Advertisement

\protect\hyperlink{after-bottom}{Continue reading the main story}

\hypertarget{site-index}{%
\subsection{Site Index}\label{site-index}}

\hypertarget{site-information-navigation}{%
\subsection{Site Information
Navigation}\label{site-information-navigation}}

\begin{itemize}
\tightlist
\item
  \href{https://help.nytimes.com/hc/en-us/articles/115014792127-Copyright-notice}{©~2020~The
  New York Times Company}
\end{itemize}

\begin{itemize}
\tightlist
\item
  \href{https://www.nytco.com/}{NYTCo}
\item
  \href{https://help.nytimes.com/hc/en-us/articles/115015385887-Contact-Us}{Contact
  Us}
\item
  \href{https://www.nytco.com/careers/}{Work with us}
\item
  \href{https://nytmediakit.com/}{Advertise}
\item
  \href{http://www.tbrandstudio.com/}{T Brand Studio}
\item
  \href{https://www.nytimes.com/privacy/cookie-policy\#how-do-i-manage-trackers}{Your
  Ad Choices}
\item
  \href{https://www.nytimes.com/privacy}{Privacy}
\item
  \href{https://help.nytimes.com/hc/en-us/articles/115014893428-Terms-of-service}{Terms
  of Service}
\item
  \href{https://help.nytimes.com/hc/en-us/articles/115014893968-Terms-of-sale}{Terms
  of Sale}
\item
  \href{https://spiderbites.nytimes.com}{Site Map}
\item
  \href{https://help.nytimes.com/hc/en-us}{Help}
\item
  \href{https://www.nytimes.com/subscription?campaignId=37WXW}{Subscriptions}
\end{itemize}
