Sections

SEARCH

\protect\hyperlink{site-content}{Skip to
content}\protect\hyperlink{site-index}{Skip to site index}

\href{https://www.nytimes.com/section/books/review}{Book Review}

\href{https://myaccount.nytimes.com/auth/login?response_type=cookie\&client_id=vi}{}

\href{https://www.nytimes.com/section/todayspaper}{Today's Paper}

\href{/section/books/review}{Book Review}\textbar{}In `Memorial Drive' a
Poet Evokes Her Childhood and Confronts Her Mother's Murder

\url{https://nyti.ms/3352Iwt}

\begin{itemize}
\item
\item
\item
\item
\item
\end{itemize}

Advertisement

\protect\hyperlink{after-top}{Continue reading the main story}

Supported by

\protect\hyperlink{after-sponsor}{Continue reading the main story}

nonfiction

\hypertarget{in-memorial-drive-a-poet-evokes-her-childhood-and-confronts-her-mothers-murder}{%
\section{In `Memorial Drive' a Poet Evokes Her Childhood and Confronts
Her Mother's
Murder}\label{in-memorial-drive-a-poet-evokes-her-childhood-and-confronts-her-mothers-murder}}

\includegraphics{https://static01.nyt.com/images/2020/06/16/books/review/Laymon1/Laymon1-articleLarge.jpg?quality=75\&auto=webp\&disable=upscale}

Buy Book ▾

\begin{itemize}
\tightlist
\item
  \href{https://www.amazon.com/gp/search?index=books\&tag=NYTBSREV-20\&field-keywords=Memorial+Drive\%3A+A+Daughter\%27s+Memoir+Natastha+Trethewey}{Amazon}
\item
  \href{https://du-gae-books-dot-nyt-du-prd.appspot.com/buy?title=Memorial+Drive\%3A+A+Daughter\%27s+Memoir\&author=Natastha+Trethewey}{Apple
  Books}
\item
  \href{https://www.anrdoezrs.net/click-7990613-11819508?url=https\%3A\%2F\%2Fwww.barnesandnoble.com\%2Fs\%2FMemorial+Drive\%3A+A+Daughter\%27s+Memoir+Natastha+Trethewey}{Barnes
  and Noble}
\item
  \href{https://www.anrdoezrs.net/click-7990613-35140?url=https\%3A\%2F\%2Fwww.booksamillion.com\%2Fsearch\%3Fquery\%3DMemorial\%2BDrive\%253A\%2BA\%2BDaughter\%2527s\%2BMemoir\%2BNatastha\%2BTrethewey}{Books-A-Million}
\item
  \href{https://bookshop.org/books?keywords=Memorial+Drive\%3A+A+Daughter\%27s+Memoir}{Bookshop}
\item
  \href{https://www.indiebound.org/search/book?searchfor=Memorial+Drive\%3A+A+Daughter\%27s+Memoir+Natastha+Trethewey\&aff=NYT}{Indiebound}
\end{itemize}

When you purchase an independently reviewed book through our site, we
earn an affiliate commission.

By Kiese Makeba Laymon

\begin{itemize}
\item
  July 30, 2020
\item
  \begin{itemize}
  \item
  \item
  \item
  \item
  \item
  \end{itemize}
\end{itemize}

\textbf{MEMORIAL DRIVE}\\
\textbf{A Daughter's Memoir}\\
By Natasha Trethewey

``All art is a kind of confession, more or less oblique,'' James Baldwin
wrote in 1960. ``All artists, if they are to survive, are forced, at
last, to tell the whole story; to vomit the anguish up.'' After reading
Natasha Trethewey's memoir, ``Memorial Drive,'' I was stuck on how to
balance Baldwin's nuanced take on art and anguish with something the
scholar Imani Perry, herself the author of a
\href{https://www.nytimes.com/2019/09/28/books/review/breathe-imani-perry.html?searchResultPosition=3}{breath-filled
memoir} addressed to her sons, told me: ``You never tell all the secrets
when you're trying to get free.''

``Memorial Drive'' is, among so many other wondrous things, an
exploration of a Black mother and daughter trying to get free in a land
that conflates survival with freedom and womanhood with girlhood. It is
also the story of Trethewey's life before and after the day in 1985 when
her mother was murdered by her ex-husband, Trethewey's former
stepfather, in the parking lot of her apartment complex on Atlanta's
Memorial Drive.

A book that makes a reader feel as much as ``Memorial Drive'' does
cannot be written without an absolute mastery of varied modes of
discourse. Trethewey, a
\href{https://www.nytimes.com/2007/05/13/magazine/13wwln-Q4-t.html?searchResultPosition=5}{Pulitzer
Prize-winning poet}, deploys scenes of inventiveness and sensuality
(``she mistook the plants that had come to mean the backbending labor of
slaves and sharecroppers for the flowers that symbolize honor and
remembrance, the swords of gladiators, tall borders of pleasure
gardens''), grinding action verbs and alliteration (``great locomotives
lurching at the railroad switch''). There is comedic play where we
expect sorrowful melodrama. There is languish where we expect
deliverance. There is also documentary evidence from her mother's case
file, including transcripts of telephone conversations between her
mother, Gwen, and her ex-stepfather, Joel, in the days before her death.

\emph{{[} Read an excerpt from}
\href{https://www.nytimes.com/2020/07/30/books/review/memorial-drive-by-natasha-trethewey-an-excerpt.html}{\emph{``Memorial
Drive.''}} \emph{{]}}

In one of the book's most devastating and artful chapters, Trethewey
makes an unexpected but wholly necessary switch to the second person. A
fifth grader, young Tasha has overheard Joel beating Gwen. At school,
the teacher in whom she confides does nothing. Then Tasha hears Gwen
attempt to use her daughter's awareness of the violence to get Joel to
stop.

``You are ashamed,'' Trethewey writes, ``and you don't know why. The
need in the voice of your powerful, lovely mother is teaching you
something about the world of men and women, of dominance and submission.
\ldots{} You hear her desperate hope that his knowing \emph{you} know,
knowing \emph{you} listen, will put an end to the abuse. As if the fact
that you are a child, that you are only in the fifth grade, will change
anything at all. And now you know that there is nothing you can do.''

After Joel breaks into Tasha's diary, the ``you'' whom she addresses
changes: ``You stupid {[}expletive{]},'' she writes. ``Do you think I
don't know what you're doing? You wouldn't know I thought of you like
this if you weren't reading my diary.''

Trethewey calls this her first act of resistance, and in doing it she
inadvertently makes the stepfather who will eventually murder her mother
her ``first audience.''

\includegraphics{https://static01.nyt.com/images/2020/06/16/books/review/Laymon2/Laymon2-articleLarge.jpg?quality=75\&auto=webp\&disable=upscale}

Joel feels the force of Tasha's words. When she comes home from school
after joining the staff of the literary journal, announcing giddily,
``I'm going to be a writer!'' Joel shrugs and tells her she is ``not
gonna do any of that.''

Along with Tasha herself, we are stunned at Gwen's response: ``She. Will
do. WHATEVER. She wants.'' Tasha watches her mother's face at dinner
that night, imagining the inevitable bruises, ``calculating the price
she'll keep paying'' to save her daughter.

The morning of the murder, the police enter into evidence a handwritten
document on a yellow legal pad. Trethewey writes that it took 25 years
before she willed herself to read her mother's words. We read these
italicized pages almost immediately after those describing her mother's
vows to protect Tasha from Joel. Gwen's words are haunted and haunting:
*``*He told me he would be nice and let me choose the way I wanted to
die.''

What happens in most riveting literature is seldom located solely in
plot. I've not read an American memoir where more happens in the
assemblage of language than ``Memorial Drive.'' Trethewey's subtext has
subtext, much of it gendered, raced, playful and sincerely placed in the
lush literary distance between Mississippi --- where she spent her
early, happy childhood in her mother's hometown, tenderly evoked here
--- and Memorial Drive in Atlanta.

Trethewey's memoir is not the hardest book I have ever read. The poetry
holding the prose together, the innovativeness of the composition, make
such a claim impossible. ``Memorial Drive'' is, however, the hardest
book I could imagine writing. ``When I finally sit down to write the
part of our story I've most needed to avoid,'' Trethewey says toward the
end, ``when I force myself at last to read the evidence, all of it ---
the transcripts, witness accounts, the autopsy and official reports, the
A.D.A.'s statement, indications of police indifference --- I collapse on
the floor, keening as though I had just learned of my mother's death.''

We cannot simply watch what could be seen as traumatic spectacle ---
what Baldwin called ``anguish'' --- not if we want, as Imani Perry says,
``to get free.'' We owe more to ourselves, and more to Trethewey's
masterpiece. There is a deeply Southern echo in these pages that offers
us the opportunity to do more than marvel, more than pander to pathos,
more than pity Tasha, the child, and admire Natasha Trethewey, the
writer.

``Memorial Drive'' forces the reader to think about how the sublime
Southern conjurers of words, spaces, sounds and patterns protect
themselves from trauma when trauma may be, in part, what nudged them
down the dusty road to poetic mastery. I closed ``Memorial Drive''
asking myself how one psychologically survives the secrets we hide from
ourselves when our freedom depends not simply on extraction, but on the
obliteration of cliché --- the lazy reader's and lazy memory's truth.

The more virtuosic our ability to use language to probe, the harder it
becomes to protect ourselves from the secrets buried in our --- and our
nation's --- marrow. This is the conundrum and the blessing of the poet.
This is the conundrum and blessing of ``Memorial Drive.'' How do you not
vomit up all the anguish when artfully vomiting up all the anguish is
one way of getting free?

Advertisement

\protect\hyperlink{after-bottom}{Continue reading the main story}

\hypertarget{site-index}{%
\subsection{Site Index}\label{site-index}}

\hypertarget{site-information-navigation}{%
\subsection{Site Information
Navigation}\label{site-information-navigation}}

\begin{itemize}
\tightlist
\item
  \href{https://help.nytimes.com/hc/en-us/articles/115014792127-Copyright-notice}{©~2020~The
  New York Times Company}
\end{itemize}

\begin{itemize}
\tightlist
\item
  \href{https://www.nytco.com/}{NYTCo}
\item
  \href{https://help.nytimes.com/hc/en-us/articles/115015385887-Contact-Us}{Contact
  Us}
\item
  \href{https://www.nytco.com/careers/}{Work with us}
\item
  \href{https://nytmediakit.com/}{Advertise}
\item
  \href{http://www.tbrandstudio.com/}{T Brand Studio}
\item
  \href{https://www.nytimes.com/privacy/cookie-policy\#how-do-i-manage-trackers}{Your
  Ad Choices}
\item
  \href{https://www.nytimes.com/privacy}{Privacy}
\item
  \href{https://help.nytimes.com/hc/en-us/articles/115014893428-Terms-of-service}{Terms
  of Service}
\item
  \href{https://help.nytimes.com/hc/en-us/articles/115014893968-Terms-of-sale}{Terms
  of Sale}
\item
  \href{https://spiderbites.nytimes.com}{Site Map}
\item
  \href{https://help.nytimes.com/hc/en-us}{Help}
\item
  \href{https://www.nytimes.com/subscription?campaignId=37WXW}{Subscriptions}
\end{itemize}
