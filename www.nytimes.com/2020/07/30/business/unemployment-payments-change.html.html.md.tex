Sections

SEARCH

\protect\hyperlink{site-content}{Skip to
content}\protect\hyperlink{site-index}{Skip to site index}

\href{https://www.nytimes.com/section/business}{Business}

\href{https://myaccount.nytimes.com/auth/login?response_type=cookie\&client_id=vi}{}

\href{https://www.nytimes.com/section/todayspaper}{Today's Paper}

\href{/section/business}{Business}\textbar{}Why Changing Unemployment
Payments Could Take Months

\url{https://nyti.ms/3gdBQhz}

\begin{itemize}
\item
\item
\item
\item
\item
\end{itemize}

Advertisement

\protect\hyperlink{after-top}{Continue reading the main story}

Supported by

\protect\hyperlink{after-sponsor}{Continue reading the main story}

\hypertarget{why-changing-unemployment-payments-could-take-months}{%
\section{Why Changing Unemployment Payments Could Take
Months}\label{why-changing-unemployment-payments-could-take-months}}

Republicans want wage replacement instead of an extra \$600 per week in
unemployment benefits, but technical and political hurdles are in the
way.

\includegraphics{https://static01.nyt.com/images/2020/07/31/business/29DC-Virus-UI-print/merlin_174604200_c5c6c368-73a3-4721-99d6-a510cf7167bb-articleLarge.jpg?quality=75\&auto=webp\&disable=upscale}

\href{https://www.nytimes.com/by/jim-tankersley}{\includegraphics{https://static01.nyt.com/images/2018/10/19/multimedia/author-jim-tankersley/author-jim-tankersley-thumbLarge.png}}\href{https://www.nytimes.com/by/tara-siegel-bernard}{\includegraphics{https://static01.nyt.com/images/2019/01/18/multimedia/author-tara-siegel-bernard/author-tara-siegel-bernard-thumbLarge.png}}

By \href{https://www.nytimes.com/by/jim-tankersley}{Jim Tankersley} and
\href{https://www.nytimes.com/by/tara-siegel-bernard}{Tara Siegel
Bernard}

\begin{itemize}
\item
  July 30, 2020
\item
  \begin{itemize}
  \item
  \item
  \item
  \item
  \item
  \end{itemize}
\end{itemize}

WASHINGTON --- Republicans
\href{https://www.nytimes.com/2020/07/23/business/economy/unemployment-benefits.html}{want
to replace} a weekly bonus check for the unemployed with a new system
that offers 70 percent of the wages workers were earning before they
were laid off. Experts say it would be a difficult switch to pull off
and one that would disadvantage lower-wage workers.

There are 53 different unemployment systems across the United States and
its territories, all of them inundated with
\href{https://www.nytimes.com/interactive/2020/05/08/business/economy/april-jobs-report.html}{record
numbers of unemployment claims}, and they all have different ways of
calculating and handing out benefits.

As of now, they all dispense their normal unemployment checks, which
vary based on the state and how much a worker was earning over a certain
period before losing his or her job. For the last several months, states
have been
\href{https://www.nytimes.com/2020/07/29/business/economy/unemployment-benefits-coronavirus.html}{adding
\$600 per week} from the federal government on top of those benefits
because of the coronavirus
\href{https://www.nytimes.com/2020/07/21/business/economy/coronavirus-unemployment-benefits.html}{pandemic}.

Republicans want to transition the system to a uniform enhanced benefit
for every unemployed worker in the country --- one that equals 70
percent of what workers were earning immediately before they were laid
off. That would require states to implement a new way of calculating
past wages, and to adjust benefit checks accordingly, at a time when
they have been overwhelmed with the more straightforward task of
processing and paying out a deluge of unemployment claims.

It could take months to pull that switch off in every state.

``You're asking for a varying amount of changes in these state
governments,'' said Kathryn Anne Edwards, an economist at the RAND
Corporation who studies unemployment benefits. ``Some of them are going
to be faster than others. Because the story of unemployment benefits is
always, always going to be the story of differences between states.''

Complicating matters, the change would be most damaging to lower-wage
workers given 70 percent of their previous earnings would amount to a
meager payout. Thanks to the \$600 weekly supplement, many of those
workers have been receiving more than they were earning from their jobs
--- a data point that Republicans cite when arguing that the program is
too generous and discourages workers from seeking employment. But
economists say those payments have provided a vital financial cushion to
the unemployed at a moment when returning to work is still not an option
for many people.

\hypertarget{it-would-be-difficult-to-change-the-benefit-system}{%
\subsection{It would be difficult to change the benefit
system.}\label{it-would-be-difficult-to-change-the-benefit-system}}

\includegraphics{https://static01.nyt.com/images/2020/07/31/business/29jpDC-Virus-UI-print/merlin_173363955_397ceec6-31f9-4e9f-8d23-9f5f08062fd5-articleLarge.jpg?quality=75\&auto=webp\&disable=upscale}

\href{https://www.finance.senate.gov/imo/media/doc/SFC\%20CARES\%202.0\%20Legislative\%20Text.pdf}{Under
a bill} that Senator Charles Grassley, Republican of Iowa and the
finance committee chairman, released on Monday, the weekly \$600 check
would fall to \$200 through August and September. On Oct. 5, it would be
replaced by a formula that starts with the amount of state benefits a
worker would normally receive for unemployment and then adds federal
dollars to bring the total benefit to 70 percent of the worker's former
wages.

States would have the option of proposing an alternative system, or
continuing flat payments to each worker, that would allow for the
average benefit to match 70 percent of lost wages.

Such a structure would be far more cumbersome for state unemployment
offices than the current system. Part of the challenge is that each
state agency has its own benefit formula and maximum benefit amount.
That means each person will need to get an individual determination
about what their federal-level benefit will need to be so that their
total benefit package is equivalent to about 70 percent of their
pre-pandemic income.

In states with low maximum amounts --- like Arizona, at \$240 weekly ---
the federal benefit will need to be much higher. ``They will have to
figure out a way to set up the system to figure out the differences,''
said Michele Evermore, a senior policy analyst for social insurance at
the National Employment Law Project.

The National Association of State Workforce Agencies, a national group
representing state unemployment offices, said it expects states's
implementation schedules to vary widely, from four to twelve weeks or
more, according to an agency document that analyzed several policy
proposals.

``If such a policy solution is chosen, the effective date should be set
well in the future,'' the agency said in the document, ``with a
continuation of a flat amount until that future effective date.''

\hypertarget{self-employed-and-gig-workers-present-an-even-larger-challenge}{%
\subsection{Self-employed and gig workers present an even larger
challenge.}\label{self-employed-and-gig-workers-present-an-even-larger-challenge}}

There are other complications for self-employed people and others who
aren't typically eligible for benefits, including those with limited
work histories. Under the expanded system, many of these individuals can
collect checks through the so-called pandemic unemployment assistance
program.

But they do not necessarily have to submit proof of earnings to qualify
for that program's minimum benefit amount --- and receiving the minimum
makes them automatically eligible for the extra \$600 federal benefit.

So if the new federal benefit is based on actual earnings records, each
state would need to build a system to receive and analyze wage data from
self-employed people. ``The state may not have good documentation about
what they were earning,'' Ms. Evermore added, ``as the documentation
requirements to get the minimum P.U.A. are less stringent.''

Other jobless workers receiving benefits may not have any earnings
history at all. Workers who had job offers that were rescinded because
of the pandemic, for example, can still receive checks even though they
didn't have any income.

\hypertarget{archaic-computer-systems-arent-helping}{%
\subsection{Archaic computer systems aren't
helping.}\label{archaic-computer-systems-arent-helping}}

States already had trouble reprogramming their systems to deploy the
expanded benefits provided under expansion under the CARES Act. Many
states administering unemployment benefits are relying
on\href{https://www.nytimes.com/2020/04/17/nyregion/coronavirus-pandemic-unemployment-assistance-ny-delays.html}{archaic
systems}, which were quickly overwhelmed by the influx of claims. Some
are using aging mainframe computers programmed using a language called
COBOL, which is more than 50 years old, and some states,
like\href{https://www.nytimes.com/2020/04/04/nyregion/coronavirus-ny-unemployment-benefits.html}{Connecticut},
had to recruit retirees who knew how to program in the antiquated
language.

Only 16 states have fully modernized their unemployment insurance
systems, according to
recent\href{https://www.nelp.org/publication/from-disrepair-to-transformation-how-to-revive-unemployment-insurance-information-technology-infrastructure/}{testimony}
by Rebecca Dixon, executive director at the National Employment Law
Project, and many of those that did update their system still
experienced problems.

``I would be very surprised if a state could get a new system to pay a
percentage replacement up in two months,'' Ms. Evermore added, ``given
everything else they have to deal with right now.''

\hypertarget{low-wage-workers-lose-while-red-states-win}{%
\subsection{Low-wage workers lose while red states
win.}\label{low-wage-workers-lose-while-red-states-win}}

If states were somehow able to make the shift, it would carry the side
effect of subsidizing states with less generous unemployment benefits,
funded by lower taxes --- a set of states that is heavily Republican. It
is the opposite dynamic from another sticking point in the negotiations
over the next stimulus bill: Republicans have resisted sending direct
aid to states with large budget shortfalls amid the crisis, because they
say they do not want to subsidize high-tax Democratic states.

It would also be a particularly large blow to workers in Democratic
states, who would lose the most money per week out of their benefit
checks. The federal government would kick in significantly more support
to bring the benefit up to 70 percent for workers in a low-benefit state
like Arizona than in a high-benefit state like Washington.

Low-wage workers will be hurt the most. They have been receiving the
most, compared to previous earning, from the \$600 weekly federal
supplements. (That also means they are the workers Republicans fear are
being discouraged from returning to the workplace, because they've been
earning more from unemployment than from their former jobs.)

The move to reduce benefits via the formula change also comes as the
composition of America's unemployed is changing to include more nonwhite
workers, because employers are rehiring whites at a more rapid pace than
Black or Latino workers, continuing a trend in America after recessions.

``There is a racial element to this --- there is absolutely a racial
element,'' said Ms. Edwards, who favors extending the \$600 per week
enhancement, citing research showing it has buoyed consumer spending in
a sharp downturn while not deterring workers from taking jobs if offered
them. ``We are in an unprecedented level of unemployment right now, and
rather than focus on how to mitigate those scars, we're debating the
work ethic of the unemployed.''

\hypertarget{politics-may-be-the-biggest-impediment}{%
\subsection{Politics may be the biggest
impediment}\label{politics-may-be-the-biggest-impediment}}

Image

Senator Ron Wyden, at a Senate Finance Committee hearing, had supported
a plan to replace 100 percent of unemployed workers' wages.Credit...Pool
photo by Anna Moneymaker

Democrats, led by Senator Ron Wyden of Oregon, the party's top-ranking
member of the finance committee, have criticized the wage-replacement
proposal and called it unworkable. Earlier this year, though, such a
system was Democrats' goal --- in discussions with the Trump
administration over an economic rescue package in March, Mr. Wyden and
others pushed for an enhanced unemployment benefit that would replace
100 percent of workers' wages.

Labor Department officials told them such a plan was not workable for
states. The \$600 additional payment was selected as a compromise --- it
is the average gap between state unemployment benefits and a typical
unemployed worker's former pay. Because it is an average, the payment
has allowed millions of Americans to earn more from unemployment that
they were earning before being laid off.

Given the challenges involved with transitioning to a wage replacement
system, policy watchers expect Congress to ultimately agree to a \$400
per week compromise that splits the difference between what Democrats
and Republicans support.

Advertisement

\protect\hyperlink{after-bottom}{Continue reading the main story}

\hypertarget{site-index}{%
\subsection{Site Index}\label{site-index}}

\hypertarget{site-information-navigation}{%
\subsection{Site Information
Navigation}\label{site-information-navigation}}

\begin{itemize}
\tightlist
\item
  \href{https://help.nytimes.com/hc/en-us/articles/115014792127-Copyright-notice}{©~2020~The
  New York Times Company}
\end{itemize}

\begin{itemize}
\tightlist
\item
  \href{https://www.nytco.com/}{NYTCo}
\item
  \href{https://help.nytimes.com/hc/en-us/articles/115015385887-Contact-Us}{Contact
  Us}
\item
  \href{https://www.nytco.com/careers/}{Work with us}
\item
  \href{https://nytmediakit.com/}{Advertise}
\item
  \href{http://www.tbrandstudio.com/}{T Brand Studio}
\item
  \href{https://www.nytimes.com/privacy/cookie-policy\#how-do-i-manage-trackers}{Your
  Ad Choices}
\item
  \href{https://www.nytimes.com/privacy}{Privacy}
\item
  \href{https://help.nytimes.com/hc/en-us/articles/115014893428-Terms-of-service}{Terms
  of Service}
\item
  \href{https://help.nytimes.com/hc/en-us/articles/115014893968-Terms-of-sale}{Terms
  of Sale}
\item
  \href{https://spiderbites.nytimes.com}{Site Map}
\item
  \href{https://help.nytimes.com/hc/en-us}{Help}
\item
  \href{https://www.nytimes.com/subscription?campaignId=37WXW}{Subscriptions}
\end{itemize}
