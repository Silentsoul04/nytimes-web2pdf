Sections

SEARCH

\protect\hyperlink{site-content}{Skip to
content}\protect\hyperlink{site-index}{Skip to site index}

\href{https://www.nytimes.com/section/food/drinks}{Wine, Beer \&
Cocktails}

\href{https://myaccount.nytimes.com/auth/login?response_type=cookie\&client_id=vi}{}

\href{https://www.nytimes.com/section/todayspaper}{Today's Paper}

\href{/section/food/drinks}{Wine, Beer \& Cocktails}\textbar{}What Is a
Great Wine? Verdicchio di Matelica Has Some Ideas

\url{https://nyti.ms/30eHDhb}

\begin{itemize}
\item
\item
\item
\item
\item
\item
\end{itemize}

\href{https://www.nytimes.com/spotlight/at-home?action=click\&pgtype=Article\&state=default\&region=TOP_BANNER\&context=at_home_menu}{At
Home}

\begin{itemize}
\tightlist
\item
  \href{https://www.nytimes.com/2020/08/03/well/family/the-benefits-of-talking-to-strangers.html?action=click\&pgtype=Article\&state=default\&region=TOP_BANNER\&context=at_home_menu}{Talk:
  To Strangers}
\item
  \href{https://www.nytimes.com/2020/08/01/at-home/coronavirus-make-pizza-on-a-grill.html?action=click\&pgtype=Article\&state=default\&region=TOP_BANNER\&context=at_home_menu}{Make:
  Grilled Pizza}
\item
  \href{https://www.nytimes.com/2020/07/31/arts/television/goldbergs-abc-stream.html?action=click\&pgtype=Article\&state=default\&region=TOP_BANNER\&context=at_home_menu}{Watch:
  'The Goldbergs'}
\item
  \href{https://www.nytimes.com/interactive/2020/at-home/even-more-reporters-editors-diaries-lists-recommendations.html?action=click\&pgtype=Article\&state=default\&region=TOP_BANNER\&context=at_home_menu}{Explore:
  Reporters' Google Docs}
\end{itemize}

Advertisement

\protect\hyperlink{after-top}{Continue reading the main story}

Supported by

\protect\hyperlink{after-sponsor}{Continue reading the main story}

\href{/column/wine-school}{Wine School}

\hypertarget{what-is-a-great-wine-verdicchio-di-matelica-has-some-ideas}{%
\section{What Is a Great Wine? Verdicchio di Matelica Has Some
Ideas}\label{what-is-a-great-wine-verdicchio-di-matelica-has-some-ideas}}

\includegraphics{https://static01.nyt.com/images/2020/08/05/dining/05Wine-School/05Wine-School-articleLarge.jpg?quality=75\&auto=webp\&disable=upscale}

By \href{https://www.nytimes.com/by/eric-asimov}{Eric Asimov}

\begin{itemize}
\item
  July 30, 2020
\item
  \begin{itemize}
  \item
  \item
  \item
  \item
  \item
  \item
  \end{itemize}
\end{itemize}

Complexity is a good thing in a wine, right? It's a descriptive term
that is almost always used approvingly. You would not disparage a wine
by calling it complex.

Yet at times, complexity might be wasted on its audience. Whether
because of fatigue, distraction or life getting on your last nerve, a
complex wine may not always fit the moment.

This, in a nutshell, captures the paradox of wine evaluation. Without
context, bottles are rated on a universal scale of what makes a wine
good, which is weighted toward the ability to age and evolve, to express
complex aromas and flavors, to convey the character of the place in
which the grapes were grown and the culture of the people who made the
wine, to evoke contemplation.

These are all wonderful characteristics in a wine, and difficult to
achieve. A wine that could do all of these things would be considered
great, and few would argue.

Sometimes, though, the occasion calls for a different kind of great.
Instead, what's wanted is a bottle that refreshes, relaxes and perhaps
spurs conversation and intimacy. In a situation like this, the best
bottle may not be the one conventionally lauded. How do wine ratings and
evaluation square with the question of context?

We ask these sorts of questions frequently at Wine School, even if we
are not always able to answer them. The answers, after all, are not
necessarily as important as the questions.

I'm not referring to the simple sort of queries that are easily resolved
with a swipes of the smartphone: What are the soils and bedrock in the
vineyard? Was the wine aged in oak barrels? Let those cramming for the
wine exam recite such litanies of facts.

Siri can't tell you what greatness in wine means, for instance. This is
the sort of question we all have to consider for ourselves. Such a
question may better be left unresolved, maybe for a long time. Let it
reside in the mind to be pondered with many sorts of wines on all types
of occasions, in many differing moods.

Only through such consideration can each of us arrive at deciding for
ourselves what might be the best wine for the moment, regardless of what
the books, the apps or your know-it-all friends say.

It's all a matter of developing ease and confidence in one's taste,
maybe not of knowing the answers but of knowing which questions to ask.
Here at Wine School, we don't pretend to be gurus, rabbis or life
coaches, to use a currently popular term. But we do think our method of
trying many different wines with open minds in relaxed situations is as
foolproof as it is simple in achieving comfort with wine.

I started thinking about standards of greatness because of something one
reader,
\href{https://www.nytimes.com/2020/07/02/dining/drinks/wine-school-assignment-verdicchio-di-matelica.html\#commentsContainer\&permid=108235588}{Peter}
of Philadelphia, said about a bottle of Verdicchio di Matelica, our
subject over the last month. He consumed a bottle with a pesto dish,
made with basil from his own garden.

``It was what I think of as a typical Italian white wine,'' he wrote,
describing it as ``not particularly complicated, but who needs
complicated on a hot summer evening?''

I might take issue with the first part of what he said --- Verdicchio di
Matelica seems similar to other Italian whites we've tried, like
\href{https://www.nytimes.com/2018/04/26/dining/drinks/wine-school-etna-bianco-sicily.html}{Etna
Bianco},
\href{https://www.nytimes.com/2019/06/06/dining/drinks/wine-school-soave-classico.html}{Soave
Classico} and
\href{https://www.nytimes.com/2018/05/31/dining/drinks/wine-school-fiano.html}{Fiano
di Avellino}, but it is also very different. They are all dry, aromatic,
not overly oaked and have great acidity. But you could say this about
white wines from a lot of countries. And I do find these wines quite
distinct from one another.

I might even take issue with the second part, although I agree with the
sentiment. Who needs complicated on a hot summer evening?

But that led me to wonder about whether these verdicchios could properly
be described as uncomplicated. Could they actually be simple and complex
at the same time?

As usual, I suggested three bottles to try. They were
\href{https://www.bisci.it/en/}{Bisci} Verdicchio di Matelica 2018, the
one Peter drank;
\href{https://portovinoitaliano.com/producers/cantine-belisario/}{Cantine
Belisario} Verdicchio di Matelica Le Salse 2018 and
\href{https://www.collestefano.com/en/}{ColleStefano} Verdicchio di
Matelica 2019.

Verdicchio di Matelica is the lesser known of two major verdicchio
appellations in the Marche region, on the Adriatic coast of Italy inland
from the city of Ancona. The other, bigger and better known, is
Verdicchio dei Castelli di Jesi.

Verdicchio di Matelica is farther from the coast and generally at a
higher elevation, in the foothills of the
\href{https://www.britannica.com/place/Apennine-Range}{Apennine
Mountains}. The wines are often thought to be a bit weightier than those
from Castelli di Jesi, with more acidity and minerality, but not as
light and floral.

The Belisario Salse, the least expensive at \$15, was a striking wine,
incisive and lean, with laserlike acidity. It smelled like seashells and
crushed rocks, with a little almond flavoring thrown in. I wouldn't want
this as an aperitif, standing around at a gallery opening. Its raging
acidity demands food. I was craving clams on the half shell.

The Bisci, likewise, had that seashell minerality, but it was richer,
rounder and more herbal than the Salse. It was more forgiving and
flexible, and didn't require food in the same way. This you could
happily enjoy at a party.

The ColleStefano, I thought, was the most complete wine of the three,
though I don't mean to suggest that either of the others were lacking.
Citrus, herbs, almonds, seashells and stones, along with the richer
roundness of the Bisci, made for the most satisfying combination, for me
at least.

I thought back to Peter's point that these wines were uncomplicated.
Maybe now they were, but they seemed to have the elements of complexity
if they were given time to evolve. These all were young wines, and they
were entry-level bottles, as well. But I couldn't help feeling that over
time, the acidity in each would become more sedate, and the other
elements would become more expressive.

Some readers, in fact, drank older bottles. ``What a wine!'' said
\href{https://www.nytimes.com/2020/07/02/dining/drinks/wine-school-assignment-verdicchio-di-matelica.html\#commentsContainer\&permid=107985862}{Reynolds}
of Manhattan after drinking a 2010 Bisci Senex Riserva, made from
Bisci's oldest vines and aged for four years in concrete tanks. ``I
could see this improving for another decade.''

That bottle sounds as if it's on its way to greatness, if it hasn't
already arrived.
\href{https://www.nytimes.com/2020/07/02/dining/drinks/wine-school-assignment-verdicchio-di-matelica.html\#commentsContainer\&permid=107955758}{Dan
Barron} of Manhattan drank a 2013 Bisci, which he said became more
complex as it warmed up.

\href{https://www.nytimes.com/2020/07/02/dining/drinks/wine-school-assignment-verdicchio-di-matelica.html\#commentsContainer\&permid=108292784}{Martina
Mirandola Mullen} of New York tried a 2019 Bisci, and found plenty to
intrigue her in this very young wine. She said it begged for serious
food, suggesting
\href{https://ouritaliantable.com/a-modern-coniglio-in-porchetta/}{coniglio
in porchetta}, rabbit prepared in the style of
\href{https://cooking.nytimes.com/recipes/1017068-porchetta-pork-roast?action=click\&module=Collection\%20Page\%20Recipe\%20Card\&region=Project\%20Cooking\&pgType=collection\&rank=33}{porchetta}.

What is it about these wines? How can they can offer uncomplicated
refreshment, as Peter perceived, yet express more complex aromas and
flavors, too?

Perhaps their prices, just \$15 to \$18, liberate us to experience them
as we wish? If a \$100 chardonnay came off as delicious and
uncomplicated, I imagine anybody would be tremendously disappointed.
These, on the other hand, are great values, capable of a range of
pleasures. Dare we call them great wines?

Some readers would. ``These are great wines,''
\href{https://www.nytimes.com/2020/07/02/dining/drinks/wine-school-assignment-verdicchio-di-matelica.html\#commentsContainer\&permid=108034206}{Joe
Appel} of Portland, Maine, said flat out. Mr. Appel happens to be a
\href{https://twitter.com/joeyappel?lang=en}{wine writer} and winemaker.

\href{https://www.nytimes.com/2020/07/02/dining/drinks/wine-school-assignment-verdicchio-di-matelica.html\#commentsContainer\&permid=108072656}{Ferguson}
in Princeton appreciated the texture and liveliness of the wine. ``It
will leave you with enough energy to still do the dishes perhaps with
another half glass poised next to the sink,'' she said of the
ColleStefano.

Drinking the Salse gave
\href{https://www.nytimes.com/2020/07/02/dining/drinks/wine-school-assignment-verdicchio-di-matelica.html\#commentsContainer\&permid=108094105}{Martin
Schappeit} of Forest, Va., insight into a historic legend. ``Before the
dinner I was wondering why
\href{https://www.empson.com/territory/marche/verdicchio/}{Alaric the
Visigoth} had 40 donkeys loaded up with barrels of verdicchio,'' he
said. ``Now I know: They needed refreshment before they sacked Rome.''

In the end, I have to conclude that these are great wines. They each did
their jobs extraordinarily well, fulfilling the imperative of
refreshment, offering energy and intriguing texture as well as a bit of
complexity if you chose to look for it.

It's not so much the conventional definition. It's more a question of
fulfilling expectations. We often preach about choosing the right wine
for the occasion. For those expecting a simple white wine, these offer
those uncomplicated pleasures. For those wanting more, these wines come
with extras. That they are superb values can't be discounted.

But you don't have to answer the question of whether they are great or
of what constitutes greatness. Just keep the questions in mind.

\emph{Follow} \emph{\href{https://twitter.com/nytfood}{NYT Food on
Twitter}} \emph{and}
\emph{\href{https://www.instagram.com/nytcooking/}{NYT Cooking on
Instagram},}
\emph{\href{https://www.facebook.com/nytcooking/}{Facebook},}
\emph{\href{https://www.youtube.com/nytcooking}{YouTube}} \emph{and}
\emph{\href{https://www.pinterest.com/nytcooking/}{Pinterest}.}
\emph{\href{https://www.nytimes.com/newsletters/cooking}{Get regular
updates from NYT Cooking, with recipe suggestions, cooking tips and
shopping advice}.}

Advertisement

\protect\hyperlink{after-bottom}{Continue reading the main story}

\hypertarget{site-index}{%
\subsection{Site Index}\label{site-index}}

\hypertarget{site-information-navigation}{%
\subsection{Site Information
Navigation}\label{site-information-navigation}}

\begin{itemize}
\tightlist
\item
  \href{https://help.nytimes.com/hc/en-us/articles/115014792127-Copyright-notice}{©~2020~The
  New York Times Company}
\end{itemize}

\begin{itemize}
\tightlist
\item
  \href{https://www.nytco.com/}{NYTCo}
\item
  \href{https://help.nytimes.com/hc/en-us/articles/115015385887-Contact-Us}{Contact
  Us}
\item
  \href{https://www.nytco.com/careers/}{Work with us}
\item
  \href{https://nytmediakit.com/}{Advertise}
\item
  \href{http://www.tbrandstudio.com/}{T Brand Studio}
\item
  \href{https://www.nytimes.com/privacy/cookie-policy\#how-do-i-manage-trackers}{Your
  Ad Choices}
\item
  \href{https://www.nytimes.com/privacy}{Privacy}
\item
  \href{https://help.nytimes.com/hc/en-us/articles/115014893428-Terms-of-service}{Terms
  of Service}
\item
  \href{https://help.nytimes.com/hc/en-us/articles/115014893968-Terms-of-sale}{Terms
  of Sale}
\item
  \href{https://spiderbites.nytimes.com}{Site Map}
\item
  \href{https://help.nytimes.com/hc/en-us}{Help}
\item
  \href{https://www.nytimes.com/subscription?campaignId=37WXW}{Subscriptions}
\end{itemize}
