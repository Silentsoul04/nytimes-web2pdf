Sections

SEARCH

\protect\hyperlink{site-content}{Skip to
content}\protect\hyperlink{site-index}{Skip to site index}

\href{https://www.nytimes.com/section/health}{Health}

\href{https://myaccount.nytimes.com/auth/login?response_type=cookie\&client_id=vi}{}

\href{https://www.nytimes.com/section/todayspaper}{Today's Paper}

\href{/section/health}{Health}\textbar{}Johnson \& Johnson's Coronavirus
Vaccine Protects Monkeys, Study Finds

\url{https://nyti.ms/2Ep9G4X}

\begin{itemize}
\item
\item
\item
\item
\item
\end{itemize}

\href{https://www.nytimes.com/news-event/coronavirus?action=click\&pgtype=Article\&state=default\&region=TOP_BANNER\&context=storylines_menu}{The
Coronavirus Outbreak}

\begin{itemize}
\tightlist
\item
  live\href{https://www.nytimes.com/2020/08/02/world/coronavirus-updates.html?action=click\&pgtype=Article\&state=default\&region=TOP_BANNER\&context=storylines_menu}{Latest
  Updates}
\item
  \href{https://www.nytimes.com/interactive/2020/us/coronavirus-us-cases.html?action=click\&pgtype=Article\&state=default\&region=TOP_BANNER\&context=storylines_menu}{Maps
  and Cases}
\item
  \href{https://www.nytimes.com/interactive/2020/science/coronavirus-vaccine-tracker.html?action=click\&pgtype=Article\&state=default\&region=TOP_BANNER\&context=storylines_menu}{Vaccine
  Tracker}
\item
  \href{https://www.nytimes.com/interactive/2020/07/29/us/schools-reopening-coronavirus.html?action=click\&pgtype=Article\&state=default\&region=TOP_BANNER\&context=storylines_menu}{What
  School May Look Like}
\item
  \href{https://www.nytimes.com/live/2020/07/31/business/stock-market-today-coronavirus?action=click\&pgtype=Article\&state=default\&region=TOP_BANNER\&context=storylines_menu}{Economy}
\end{itemize}

Advertisement

\protect\hyperlink{after-top}{Continue reading the main story}

Supported by

\protect\hyperlink{after-sponsor}{Continue reading the main story}

\hypertarget{johnson--johnsons-coronavirus-vaccine-protects-monkeys-study-finds}{%
\section{Johnson \& Johnson's Coronavirus Vaccine Protects Monkeys,
Study
Finds}\label{johnson--johnsons-coronavirus-vaccine-protects-monkeys-study-finds}}

It's the second study in a week to report promising results in monkeys
for a vaccine candidate. But the real test will come with human trials
that are now underway.

\includegraphics{https://static01.nyt.com/images/2020/08/03/science/30VIRUS-MONKEYS1/30VIRUS-MONKEYS1-articleLarge.jpg?quality=75\&auto=webp\&disable=upscale}

\href{https://www.nytimes.com/by/carl-zimmer}{\includegraphics{https://static01.nyt.com/images/2018/06/12/multimedia/author-carl-zimmer/author-carl-zimmer-thumbLarge.png}}

By \href{https://www.nytimes.com/by/carl-zimmer}{Carl Zimmer}

\begin{itemize}
\item
  July 30, 2020
\item
  \begin{itemize}
  \item
  \item
  \item
  \item
  \item
  \end{itemize}
\end{itemize}

An experimental coronavirus vaccine developed by Johnson \& Johnson
protected monkeys from infection in a new study. It is the second
vaccine candidate to show promising results in monkeys this week.

The company recently began a clinical trial in Europe and the United
States to test its vaccine in people. It is one of
\href{https://www.nytimes.com/interactive/2020/science/coronavirus-vaccine-tracker.html}{more
than 30 human trials} for coronavirus vaccines underway across the
world. But until these trials are complete --- which will probably take
several months --- the monkey data offers the best clues to whether the
vaccines will work.

``This week has been good --- now we have two vaccines that work in
monkeys,'' said Angela Rasmussen, a virologist at Columbia University
who was not involved in the studies. ``It's nice to be upbeat for a
change.''

But she cautioned that the new results shouldn't be used to rush
large-scale trials in humans. ``We just can't take shortcuts,'' she
said.

Unlike many other vaccines in development that might require two
injections, the Johnson \& Johnson candidate shielded the monkeys with
just one dose, according to a
\href{https://www.nature.com/articles/s41586-020-2607-z}{study}
published on Thursday in Nature.

``It's a very reassuring level of protection we saw,'' said Dr. Dan
Barouch, a virologist at Beth Israel Deaconess Medical Center in Boston
and a co-author of the new study.

The study comes just two days after a similar one was published on a
\href{https://www.nytimes.com/2020/07/28/health/coronavirus-moderna-vaccine-monkeys.html}{vaccine
tested by Moderna}and the National Institutes of Health.

\hypertarget{latest-updates-global-coronavirus-outbreak}{%
\section{\texorpdfstring{\href{https://www.nytimes.com/2020/08/01/world/coronavirus-covid-19.html?action=click\&pgtype=Article\&state=default\&region=MAIN_CONTENT_1\&context=storylines_live_updates}{Latest
Updates: Global Coronavirus
Outbreak}}{Latest Updates: Global Coronavirus Outbreak}}\label{latest-updates-global-coronavirus-outbreak}}

Updated 2020-08-02T17:52:35.962Z

\begin{itemize}
\tightlist
\item
  \href{https://www.nytimes.com/2020/08/01/world/coronavirus-covid-19.html?action=click\&pgtype=Article\&state=default\&region=MAIN_CONTENT_1\&context=storylines_live_updates\#link-34047410}{The
  U.S. reels as July cases more than double the total of any other
  month.}
\item
  \href{https://www.nytimes.com/2020/08/01/world/coronavirus-covid-19.html?action=click\&pgtype=Article\&state=default\&region=MAIN_CONTENT_1\&context=storylines_live_updates\#link-780ec966}{Top
  U.S. officials work to break an impasse over the federal jobless
  benefit.}
\item
  \href{https://www.nytimes.com/2020/08/01/world/coronavirus-covid-19.html?action=click\&pgtype=Article\&state=default\&region=MAIN_CONTENT_1\&context=storylines_live_updates\#link-2bc8948}{Its
  outbreak untamed, Melbourne goes into even greater lockdown.}
\end{itemize}

\href{https://www.nytimes.com/2020/08/01/world/coronavirus-covid-19.html?action=click\&pgtype=Article\&state=default\&region=MAIN_CONTENT_1\&context=storylines_live_updates}{See
more updates}

More live coverage:
\href{https://www.nytimes.com/live/2020/07/31/business/stock-market-today-coronavirus?action=click\&pgtype=Article\&state=default\&region=MAIN_CONTENT_1\&context=storylines_live_updates}{Markets}

But the two vaccines work in very different ways.

The Moderna vaccine delivers a kind of genetic material called
``messenger RNA'' into cells.

The cells use the vaccine RNA to produce a protein found on the surface
of the coronavirus, called spike protein, which then hopefully prompts
an immune response.

RNA-based vaccines are being tested for a number of diseases, but none
have yet been licensed for use in people.

In the Moderna study, researchers vaccinated monkeys by giving them two
shots spaced over four weeks. A month later, they infected the animals
with the coronavirus. In some of the vaccinated monkeys, researchers
could not detect the virus in the nose or lungs. In others, the virus
replicated slowly before disappearing.

\href{https://www.nytimes.com/2020/07/27/health/moderna-vaccine-covid.html}{Moderna
began Phase 3 trials} of its mRNA vaccine on Monday, as did Pfizer,
which is testing its own mRNA vaccine.

\textbf{\emph{{[}}\href{http://on.fb.me/1paTQ1h}{\emph{Like the Science
Times page on Facebook.}}} ****** \emph{\textbar{} Sign up for the}
\textbf{\href{http://nyti.ms/1MbHaRU}{\emph{Science Times
newsletter.}}\emph{{]}}}

\href{https://www.nytimes.com/2020/07/17/health/coronavirus-vaccine-johnson-janssen.html}{The
Johnson \& Johnson vaccine}, in contrast, is based on a virus called
Ad26, which researchers have modified so that it carries the coronavirus
spike protein gene. The Ad26 virus can slip into human cells, but cannot
replicate once inside them. Its host cell then uses the spike gene to
make the coronavirus proteins.

This month, European regulators approved Johnson \& Johnson's Ad26
vaccine for Ebola. It was the first time this kind of virus-assisted
gene delivery was approved for any disease.

\includegraphics{https://static01.nyt.com/images/2020/08/03/science/30VIRUS-MONKEYS2/30VIRUS-MONKEYS2-articleLarge.jpg?quality=75\&auto=webp\&disable=upscale}

In March, Dr. Barouch and his colleagues designed seven variants of an
Ad26 vaccine for the coronavirus. They made tiny changes to the spike
gene to see whether they could get cells to make more copies of the
viral protein. They also tested variants that would make the spike
protein more stable, which might prompt a stronger immune response.

Based
\href{https://www.nytimes.com/2020/05/20/health/coronavirus-vaccine-harvard.html}{on
earlier research}, Dr. Barouch and his colleagues suspected that the
Ad26 vaccine would be very potent. They decided to run their experiment
using just one dose, to see whether that was enough to provide immunity.

\href{https://www.nytimes.com/news-event/coronavirus?action=click\&pgtype=Article\&state=default\&region=MAIN_CONTENT_3\&context=storylines_faq}{}

\hypertarget{the-coronavirus-outbreak-}{%
\subsubsection{The Coronavirus Outbreak
›}\label{the-coronavirus-outbreak-}}

\hypertarget{frequently-asked-questions}{%
\paragraph{Frequently Asked
Questions}\label{frequently-asked-questions}}

Updated July 27, 2020

\begin{itemize}
\item ~
  \hypertarget{should-i-refinance-my-mortgage}{%
  \paragraph{Should I refinance my
  mortgage?}\label{should-i-refinance-my-mortgage}}

  \begin{itemize}
  \tightlist
  \item
    \href{https://www.nytimes.com/article/coronavirus-money-unemployment.html?action=click\&pgtype=Article\&state=default\&region=MAIN_CONTENT_3\&context=storylines_faq}{It
    could be a good idea,} because mortgage rates have
    \href{https://www.nytimes.com/2020/07/16/business/mortgage-rates-below-3-percent.html?action=click\&pgtype=Article\&state=default\&region=MAIN_CONTENT_3\&context=storylines_faq}{never
    been lower.} Refinancing requests have pushed mortgage applications
    to some of the highest levels since 2008, so be prepared to get in
    line. But defaults are also up, so if you're thinking about buying a
    home, be aware that some lenders have tightened their standards.
  \end{itemize}
\item ~
  \hypertarget{what-is-school-going-to-look-like-in-september}{%
  \paragraph{What is school going to look like in
  September?}\label{what-is-school-going-to-look-like-in-september}}

  \begin{itemize}
  \tightlist
  \item
    It is unlikely that many schools will return to a normal schedule
    this fall, requiring the grind of
    \href{https://www.nytimes.com/2020/06/05/us/coronavirus-education-lost-learning.html?action=click\&pgtype=Article\&state=default\&region=MAIN_CONTENT_3\&context=storylines_faq}{online
    learning},
    \href{https://www.nytimes.com/2020/05/29/us/coronavirus-child-care-centers.html?action=click\&pgtype=Article\&state=default\&region=MAIN_CONTENT_3\&context=storylines_faq}{makeshift
    child care} and
    \href{https://www.nytimes.com/2020/06/03/business/economy/coronavirus-working-women.html?action=click\&pgtype=Article\&state=default\&region=MAIN_CONTENT_3\&context=storylines_faq}{stunted
    workdays} to continue. California's two largest public school
    districts --- Los Angeles and San Diego --- said on July 13, that
    \href{https://www.nytimes.com/2020/07/13/us/lausd-san-diego-school-reopening.html?action=click\&pgtype=Article\&state=default\&region=MAIN_CONTENT_3\&context=storylines_faq}{instruction
    will be remote-only in the fall}, citing concerns that surging
    coronavirus infections in their areas pose too dire a risk for
    students and teachers. Together, the two districts enroll some
    825,000 students. They are the largest in the country so far to
    abandon plans for even a partial physical return to classrooms when
    they reopen in August. For other districts, the solution won't be an
    all-or-nothing approach.
    \href{https://bioethics.jhu.edu/research-and-outreach/projects/eschool-initiative/school-policy-tracker/}{Many
    systems}, including the nation's largest, New York City, are
    devising
    \href{https://www.nytimes.com/2020/06/26/us/coronavirus-schools-reopen-fall.html?action=click\&pgtype=Article\&state=default\&region=MAIN_CONTENT_3\&context=storylines_faq}{hybrid
    plans} that involve spending some days in classrooms and other days
    online. There's no national policy on this yet, so check with your
    municipal school system regularly to see what is happening in your
    community.
  \end{itemize}
\item ~
  \hypertarget{is-the-coronavirus-airborne}{%
  \paragraph{Is the coronavirus
  airborne?}\label{is-the-coronavirus-airborne}}

  \begin{itemize}
  \tightlist
  \item
    The coronavirus
    \href{https://www.nytimes.com/2020/07/04/health/239-experts-with-one-big-claim-the-coronavirus-is-airborne.html?action=click\&pgtype=Article\&state=default\&region=MAIN_CONTENT_3\&context=storylines_faq}{can
    stay aloft for hours in tiny droplets in stagnant air}, infecting
    people as they inhale, mounting scientific evidence suggests. This
    risk is highest in crowded indoor spaces with poor ventilation, and
    may help explain super-spreading events reported in meatpacking
    plants, churches and restaurants.
    \href{https://www.nytimes.com/2020/07/06/health/coronavirus-airborne-aerosols.html?action=click\&pgtype=Article\&state=default\&region=MAIN_CONTENT_3\&context=storylines_faq}{It's
    unclear how often the virus is spread} via these tiny droplets, or
    aerosols, compared with larger droplets that are expelled when a
    sick person coughs or sneezes, or transmitted through contact with
    contaminated surfaces, said Linsey Marr, an aerosol expert at
    Virginia Tech. Aerosols are released even when a person without
    symptoms exhales, talks or sings, according to Dr. Marr and more
    than 200 other experts, who
    \href{https://academic.oup.com/cid/article/doi/10.1093/cid/ciaa939/5867798}{have
    outlined the evidence in an open letter to the World Health
    Organization}.
  \end{itemize}
\item ~
  \hypertarget{what-are-the-symptoms-of-coronavirus}{%
  \paragraph{What are the symptoms of
  coronavirus?}\label{what-are-the-symptoms-of-coronavirus}}

  \begin{itemize}
  \tightlist
  \item
    Common symptoms
    \href{https://www.nytimes.com/article/symptoms-coronavirus.html?action=click\&pgtype=Article\&state=default\&region=MAIN_CONTENT_3\&context=storylines_faq}{include
    fever, a dry cough, fatigue and difficulty breathing or shortness of
    breath.} Some of these symptoms overlap with those of the flu,
    making detection difficult, but runny noses and stuffy sinuses are
    less common.
    \href{https://www.nytimes.com/2020/04/27/health/coronavirus-symptoms-cdc.html?action=click\&pgtype=Article\&state=default\&region=MAIN_CONTENT_3\&context=storylines_faq}{The
    C.D.C. has also} added chills, muscle pain, sore throat, headache
    and a new loss of the sense of taste or smell as symptoms to look
    out for. Most people fall ill five to seven days after exposure, but
    symptoms may appear in as few as two days or as many as 14 days.
  \end{itemize}
\item ~
  \hypertarget{does-asymptomatic-transmission-of-covid-19-happen}{%
  \paragraph{Does asymptomatic transmission of Covid-19
  happen?}\label{does-asymptomatic-transmission-of-covid-19-happen}}

  \begin{itemize}
  \tightlist
  \item
    So far, the evidence seems to show it does. A widely cited
    \href{https://www.nature.com/articles/s41591-020-0869-5}{paper}
    published in April suggests that people are most infectious about
    two days before the onset of coronavirus symptoms and estimated that
    44 percent of new infections were a result of transmission from
    people who were not yet showing symptoms. Recently, a top expert at
    the World Health Organization stated that transmission of the
    coronavirus by people who did not have symptoms was ``very rare,''
    \href{https://www.nytimes.com/2020/06/09/world/coronavirus-updates.html?action=click\&pgtype=Article\&state=default\&region=MAIN_CONTENT_3\&context=storylines_faq\#link-1f302e21}{but
    she later walked back that statement.}
  \end{itemize}
\end{itemize}

After a single injection of the vaccine, they waited six weeks and then
infected the animals with the coronavirus. Six of the seven vaccine
variants offered monkeys partial protection against the coronavirus,
meaning that the virus replicated only at low levels in the animals.

The seventh version proved more powerful than the rest: Five out of six
monkeys that received it had no detectable viruses at all. The sixth had
only low levels in its nose.

``The fact that we could protect with a single shot in animal models was
quite a positive surprise to us,'' said Dr. Paul Stoffels, the chief
scientific officer of Johnson \& Johnson.

It was this best-performing vaccine that Johnson \& Johnson used last
week to begin its first human safety trial, a so-called Phase 1 trial.
If it goes well, the company hopes by September to enter Phase 3 trials,
which test not only whether the vaccine is safe, but also whether it
works.

The company plans on testing both single and double doses. Dr. Rasmussen
said that a vaccine that proved effective with a single dose would make
it far easier to treat the billions of people who need it.
``Theoretically, you would need less of it, so you give it to more
people more quickly,'' she said.

Inovio, a company developing a DNA-based vaccine, announced Thursday
that monkeys challenged four months after vaccination had a reduced load
of the virus in their nose and lungs. Their
\href{https://www.biorxiv.org/content/10.1101/2020.07.28.225649v1}{report}
has not yet been published in a scientific journal.

AstraZeneca and the University of Oxford have developed a vaccine based
on yet
\href{https://www.nytimes.com/2020/07/20/world/covid-coronavirus-vaccine.html}{another
type of modified virus}, called ChAdOx1. In May, they released promising
monkey data, which was also
\href{https://www.nature.com/articles/s41586-020-2608-y}{published} on
Thursday in Nature. The team is now running Phase 3 trials in people,
which could produce results by October.

``It's exciting to see the number of platforms that are showing promise
for a vaccine,'' said Stacey L. Schultz-Cherry, a virologist at St. Jude
Children's Research Hospital in Memphis who was not involved in any of
the trials.

Advertisement

\protect\hyperlink{after-bottom}{Continue reading the main story}

\hypertarget{site-index}{%
\subsection{Site Index}\label{site-index}}

\hypertarget{site-information-navigation}{%
\subsection{Site Information
Navigation}\label{site-information-navigation}}

\begin{itemize}
\tightlist
\item
  \href{https://help.nytimes.com/hc/en-us/articles/115014792127-Copyright-notice}{©~2020~The
  New York Times Company}
\end{itemize}

\begin{itemize}
\tightlist
\item
  \href{https://www.nytco.com/}{NYTCo}
\item
  \href{https://help.nytimes.com/hc/en-us/articles/115015385887-Contact-Us}{Contact
  Us}
\item
  \href{https://www.nytco.com/careers/}{Work with us}
\item
  \href{https://nytmediakit.com/}{Advertise}
\item
  \href{http://www.tbrandstudio.com/}{T Brand Studio}
\item
  \href{https://www.nytimes.com/privacy/cookie-policy\#how-do-i-manage-trackers}{Your
  Ad Choices}
\item
  \href{https://www.nytimes.com/privacy}{Privacy}
\item
  \href{https://help.nytimes.com/hc/en-us/articles/115014893428-Terms-of-service}{Terms
  of Service}
\item
  \href{https://help.nytimes.com/hc/en-us/articles/115014893968-Terms-of-sale}{Terms
  of Sale}
\item
  \href{https://spiderbites.nytimes.com}{Site Map}
\item
  \href{https://help.nytimes.com/hc/en-us}{Help}
\item
  \href{https://www.nytimes.com/subscription?campaignId=37WXW}{Subscriptions}
\end{itemize}
