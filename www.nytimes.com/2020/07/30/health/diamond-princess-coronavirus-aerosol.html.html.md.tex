Sections

SEARCH

\protect\hyperlink{site-content}{Skip to
content}\protect\hyperlink{site-index}{Skip to site index}

\href{https://www.nytimes.com/section/health}{Health}

\href{https://myaccount.nytimes.com/auth/login?response_type=cookie\&client_id=vi}{}

\href{https://www.nytimes.com/section/todayspaper}{Today's Paper}

\href{/section/health}{Health}\textbar{}Aboard the Diamond Princess, a
Case Study in Aerosol Transmission

\url{https://nyti.ms/2Pm9K85}

\begin{itemize}
\item
\item
\item
\item
\item
\end{itemize}

\href{https://www.nytimes.com/news-event/coronavirus?action=click\&pgtype=Article\&state=default\&region=TOP_BANNER\&context=storylines_menu}{The
Coronavirus Outbreak}

\begin{itemize}
\tightlist
\item
  live\href{https://www.nytimes.com/2020/08/04/world/coronavirus-covid-19.html?action=click\&pgtype=Article\&state=default\&region=TOP_BANNER\&context=storylines_menu}{Latest
  Updates}
\item
  \href{https://www.nytimes.com/interactive/2020/us/coronavirus-us-cases.html?action=click\&pgtype=Article\&state=default\&region=TOP_BANNER\&context=storylines_menu}{Maps
  and Cases}
\item
  \href{https://www.nytimes.com/interactive/2020/science/coronavirus-vaccine-tracker.html?action=click\&pgtype=Article\&state=default\&region=TOP_BANNER\&context=storylines_menu}{Vaccine
  Tracker}
\item
  \href{https://www.nytimes.com/2020/08/02/us/covid-college-reopening.html?action=click\&pgtype=Article\&state=default\&region=TOP_BANNER\&context=storylines_menu}{College
  Reopening}
\item
  \href{https://www.nytimes.com/live/2020/08/04/business/stock-market-today-coronavirus?action=click\&pgtype=Article\&state=default\&region=TOP_BANNER\&context=storylines_menu}{Economy}
\end{itemize}

Advertisement

\protect\hyperlink{after-top}{Continue reading the main story}

Supported by

\protect\hyperlink{after-sponsor}{Continue reading the main story}

\hypertarget{aboard-the-diamond-princess-a-case-study-in-aerosol-transmission}{%
\section{Aboard the Diamond Princess, a Case Study in Aerosol
Transmission}\label{aboard-the-diamond-princess-a-case-study-in-aerosol-transmission}}

A computer model of the cruise-ship outbreak found that the virus spread
most readily in microscopic droplets light enough to linger in the air.

\includegraphics{https://static01.nyt.com/images/2020/07/27/science/00VIRUS-DIAMOND1/merlin_169003674_e096c442-6d94-4240-9ceb-7cba7755451a-articleLarge.jpg?quality=75\&auto=webp\&disable=upscale}

By \href{https://www.nytimes.com/by/benedict-carey}{Benedict Carey} and
\href{https://www.nytimes.com/by/james-glanz}{James Glanz}

\begin{itemize}
\item
  July 30, 2020
\item
  \begin{itemize}
  \item
  \item
  \item
  \item
  \item
  \end{itemize}
\end{itemize}

In a year of endless viral outbreaks, the details of the Diamond
Princess tragedy seem like ancient history. On Jan. 20, one infected
passenger boarded the cruise ship; a month later, more than 700 of the
3,711 passengers and crew members had tested positive, with many falling
seriously ill. The invader moved as swiftly and invisibly as the
perpetrators on Agatha Christie's Orient Express, leaving doctors and
health officials with only fragmentary evidence to sift through.

Ever since, scientists have tried to pin down exactly how the
coronavirus spread throughout the ship. And for good reason: The Diamond
Princess' outbreak remains perhaps the most valuable case study
available of coronavirus transmission --- an experiment-in-a-bottle,
rich in data, as well as a dark warning for what was to come in much of
the world.

Now, researchers are beginning to use macroscopic tools --- computer
models, which have revealed patterns in the virus's global spread --- to
clarify the much smaller-scale questions that currently dominate public
discussions of safety: How, exactly, does the virus move through a
community, a building or a small group of people? Which modes of
transmission should concern us most, and how might we stop them?

In
\href{https://www.medrxiv.org/content/10.1101/2020.07.13.20153049v1}{a
new report}, a research team based at Harvard and the Illinois Institute
of Technology has tried to tease out the ways in which the virus passed
from person to person in the staterooms, corridors and common areas of
the Diamond Princess. It found that the virus spread most readily in
microscopic droplets that were light enough to float in the air, for
several minutes or much longer.

The new findings add to an escalating debate among doctors, scientists
and health officials about the primary routes of coronavirus
transmission. Earlier this month, after
\href{https://www.nytimes.com/2020/07/04/health/239-experts-with-one-big-claim-the-coronavirus-is-airborne.html}{pressure
from more than 200 scientists,} the World Health Organization
acknowledged that the virus could linger in the air indoors, potentially
causing new infections. Previously, it had emphasized only large
droplets, as from coughing, and infected surfaces as the primary drivers
of transmission. Many clinicians and epidemiologists continue to argue
that these routes are central to disease progression.

The new paper has been posted on a preprint server and submitted to a
journal; it has not yet been peer-reviewed, but it was shown by Times
reporters to nearly a dozen experts in aerosols and infectious disease.
The new findings, if confirmed, would have major implications for making
indoor spaces safer and choosing among a panoply of personal protective
gear.

For example, ventilation systems that ``turn over'' or replace the air
in a room or building as often as possible, preferably drawing on
external air to do so, should make indoor spaces healthier. But good
ventilation is not enough; the Diamond Princess was well ventilated and
the air did not recirculate, the researchers noted. So wearing
good-quality masks --- standard surgical masks, or cloth masks with
multiple layers rather than just one --- will most likely be needed as
well, even in well-ventilated spaces where people are keeping their
distance.

\hypertarget{latest-updates-global-coronavirus-outbreak}{%
\section{\texorpdfstring{\href{https://www.nytimes.com/2020/08/04/world/coronavirus-covid-19.html?action=click\&pgtype=Article\&state=default\&region=MAIN_CONTENT_1\&context=storylines_live_updates}{Latest
Updates: Global Coronavirus
Outbreak}}{Latest Updates: Global Coronavirus Outbreak}}\label{latest-updates-global-coronavirus-outbreak}}

Updated 2020-08-04T15:56:36.876Z

\begin{itemize}
\tightlist
\item
  \href{https://www.nytimes.com/2020/08/04/world/coronavirus-covid-19.html?action=click\&pgtype=Article\&state=default\&region=MAIN_CONTENT_1\&context=storylines_live_updates\#link-4d1eafa8}{N.Y.C.'s
  health commissioner resigns after clashing with the mayor over the
  virus.}
\item
  \href{https://www.nytimes.com/2020/08/04/world/coronavirus-covid-19.html?action=click\&pgtype=Article\&state=default\&region=MAIN_CONTENT_1\&context=storylines_live_updates\#link-6b644638}{`Long
  days, long nights': Washington prepares for a prolonged fight over
  virus relief.}
\item
  \href{https://www.nytimes.com/2020/08/04/world/coronavirus-covid-19.html?action=click\&pgtype=Article\&state=default\&region=MAIN_CONTENT_1\&context=storylines_live_updates\#link-7af9fca0}{Israel's
  rocky reopening of its schools may be a lesson for the U.S.}
\end{itemize}

\href{https://www.nytimes.com/2020/08/04/world/coronavirus-covid-19.html?action=click\&pgtype=Article\&state=default\&region=MAIN_CONTENT_1\&context=storylines_live_updates}{See
more updates}

More live coverage:
\href{https://www.nytimes.com/live/2020/08/04/business/stock-market-today-coronavirus?action=click\&pgtype=Article\&state=default\&region=MAIN_CONTENT_1\&context=storylines_live_updates}{Markets}

The computer modeling adds a new dimension of support to an accumulating
body of evidence implicating small, airborne droplets in multiple
outbreaks, including at
\href{https://www.nytimes.com/2020/04/20/health/airflow-coronavirus-restaurants.html}{a
Chinese restaurant}, a
\href{https://www.medrxiv.org/content/10.1101/2020.06.15.20132027v2}{choir
in Washington State}, as well as
\href{https://www.nature.com/articles/s41598-020-69286-3}{a recent
study} at a Nebraska hospital to which 13 passengers from the Diamond
Princess had been evacuated.

One researcher not involved in the new work, Julian Tang, an honorary
associate professor of respiratory sciences at the University of
Leicester in the United Kingdom, said the paper was ``the first attempt,
as far as I know, to formally compare the different routes of
coronavirus transmission, especially of short versus long-range
aerosols.''

He characterized the distances and the kinds of particles involved with
a simple analogy from everyday life: ``If you can smell what I had for
lunch, you're getting my air, and you can be getting virus particles as
well.''

Another researcher, Linsey Marr, a professor of civil and environmental
engineering at Virginia Tech who studies airborne transmission of
viruses, had a more vivid description of the finding: the ``garlic
breath'' effect.

``As you're close to someone, you smell that garlic breath,'' Dr. Marr
said. ``As you're farther away, you don't smell it.''

The ``garlic breath'' effect would suggest that powerful ventilation in
buildings --- primarily using outside air, or very well filtered ---
could reduce the transmission of the virus. The study found that small
particles also had some ability to spread it at longer distances,
presumably beyond the range of breath odor.

\includegraphics{https://static01.nyt.com/images/2020/07/27/science/00VIRUS-DIAMOND2/merlin_168906204_8b9319e1-c7c4-43da-9527-0cf68d0f1384-articleLarge.jpg?quality=75\&auto=webp\&disable=upscale}

From the start of the pandemic, scientists have grappled with the
mechanisms of coronavirus spread. Early on, surface transmission was
widely emphasized; larger droplets, which travel on more ballistic
trajectories, like a stone through the air, and strike mucus membranes
directly, are now favored by a number of researchers.

Other possibilities are candidates as well, said Dr. John Conly, an
infectious disease physician and infection control expert with the
University of Calgary in Canada who has done consulting with the World
Health Organization.

``We're getting surprises all the way along,'' Dr. Conly said. ``This
paper I find interesting, but it has a long way to go to be able to get
into a line of credibility, in my mind.''

Dr. George Rutherford, a professor of epidemiology at the University of
California, San Francisco, was equally skeptical. He said that, outside
of hospital settings, ``large droplets in my mind account for the vast
majority of cases. Aerosols transmission --- if you really run with
that, it creates lots of dissonance. Are there situations where it could
occur? Yeah maybe, but it's a tiny amount.''

Dr. Tang and other scientists strongly disagree. ``If I'm talking to an
infectious person for 15 or 20 minutes and inhaling some of their air,''
Dr. Tang said, ``isn't that a much simpler way to explain transmission
than touching an infected surface and touching your eyes? When you're
talking about an outbreak, like at a restaurant, that latter seems like
a torturous way to explain transmission.''

\href{https://www.nytimes.com/news-event/coronavirus?action=click\&pgtype=Article\&state=default\&region=MAIN_CONTENT_3\&context=storylines_faq}{}

\hypertarget{the-coronavirus-outbreak-}{%
\subsubsection{The Coronavirus Outbreak
›}\label{the-coronavirus-outbreak-}}

\hypertarget{frequently-asked-questions}{%
\paragraph{Frequently Asked
Questions}\label{frequently-asked-questions}}

Updated August 3, 2020

\begin{itemize}
\item ~
  \hypertarget{im-a-small-business-owner-can-i-get-relief}{%
  \paragraph{I'm a small-business owner. Can I get
  relief?}\label{im-a-small-business-owner-can-i-get-relief}}

  \begin{itemize}
  \tightlist
  \item
    The
    \href{https://www.nytimes.com/article/small-business-loans-stimulus-grants-freelancers-coronavirus.html?action=click\&pgtype=Article\&state=default\&region=MAIN_CONTENT_3\&context=storylines_faq}{stimulus
    bills enacted in March} offer help for the millions of American
    small businesses. Those eligible for aid are businesses and
    nonprofit organizations with fewer than 500 workers, including sole
    proprietorships, independent contractors and freelancers. Some
    larger companies in some industries are also eligible. The help
    being offered, which is being managed by the Small Business
    Administration, includes the Paycheck Protection Program and the
    Economic Injury Disaster Loan program. But lots of folks have
    \href{https://www.nytimes.com/interactive/2020/05/07/business/small-business-loans-coronavirus.html?action=click\&pgtype=Article\&state=default\&region=MAIN_CONTENT_3\&context=storylines_faq}{not
    yet seen payouts.} Even those who have received help are confused:
    The rules are draconian, and some are stuck sitting on
    \href{https://www.nytimes.com/2020/05/02/business/economy/loans-coronavirus-small-business.html?action=click\&pgtype=Article\&state=default\&region=MAIN_CONTENT_3\&context=storylines_faq}{money
    they don't know how to use.} Many small-business owners are getting
    less than they expected or
    \href{https://www.nytimes.com/2020/06/10/business/Small-business-loans-ppp.html?action=click\&pgtype=Article\&state=default\&region=MAIN_CONTENT_3\&context=storylines_faq}{not
    hearing anything at all.}
  \end{itemize}
\item ~
  \hypertarget{what-are-my-rights-if-i-am-worried-about-going-back-to-work}{%
  \paragraph{What are my rights if I am worried about going back to
  work?}\label{what-are-my-rights-if-i-am-worried-about-going-back-to-work}}

  \begin{itemize}
  \tightlist
  \item
    Employers have to provide
    \href{https://www.osha.gov/SLTC/covid-19/standards.html}{a safe
    workplace} with policies that protect everyone equally.
    \href{https://www.nytimes.com/article/coronavirus-money-unemployment.html?action=click\&pgtype=Article\&state=default\&region=MAIN_CONTENT_3\&context=storylines_faq}{And
    if one of your co-workers tests positive for the coronavirus, the
    C.D.C.} has said that
    \href{https://www.cdc.gov/coronavirus/2019-ncov/community/guidance-business-response.html}{employers
    should tell their employees} -\/- without giving you the sick
    employee's name -\/- that they may have been exposed to the virus.
  \end{itemize}
\item ~
  \hypertarget{should-i-refinance-my-mortgage}{%
  \paragraph{Should I refinance my
  mortgage?}\label{should-i-refinance-my-mortgage}}

  \begin{itemize}
  \tightlist
  \item
    \href{https://www.nytimes.com/article/coronavirus-money-unemployment.html?action=click\&pgtype=Article\&state=default\&region=MAIN_CONTENT_3\&context=storylines_faq}{It
    could be a good idea,} because mortgage rates have
    \href{https://www.nytimes.com/2020/07/16/business/mortgage-rates-below-3-percent.html?action=click\&pgtype=Article\&state=default\&region=MAIN_CONTENT_3\&context=storylines_faq}{never
    been lower.} Refinancing requests have pushed mortgage applications
    to some of the highest levels since 2008, so be prepared to get in
    line. But defaults are also up, so if you're thinking about buying a
    home, be aware that some lenders have tightened their standards.
  \end{itemize}
\item ~
  \hypertarget{what-is-school-going-to-look-like-in-september}{%
  \paragraph{What is school going to look like in
  September?}\label{what-is-school-going-to-look-like-in-september}}

  \begin{itemize}
  \tightlist
  \item
    It is unlikely that many schools will return to a normal schedule
    this fall, requiring the grind of
    \href{https://www.nytimes.com/2020/06/05/us/coronavirus-education-lost-learning.html?action=click\&pgtype=Article\&state=default\&region=MAIN_CONTENT_3\&context=storylines_faq}{online
    learning},
    \href{https://www.nytimes.com/2020/05/29/us/coronavirus-child-care-centers.html?action=click\&pgtype=Article\&state=default\&region=MAIN_CONTENT_3\&context=storylines_faq}{makeshift
    child care} and
    \href{https://www.nytimes.com/2020/06/03/business/economy/coronavirus-working-women.html?action=click\&pgtype=Article\&state=default\&region=MAIN_CONTENT_3\&context=storylines_faq}{stunted
    workdays} to continue. California's two largest public school
    districts --- Los Angeles and San Diego --- said on July 13, that
    \href{https://www.nytimes.com/2020/07/13/us/lausd-san-diego-school-reopening.html?action=click\&pgtype=Article\&state=default\&region=MAIN_CONTENT_3\&context=storylines_faq}{instruction
    will be remote-only in the fall}, citing concerns that surging
    coronavirus infections in their areas pose too dire a risk for
    students and teachers. Together, the two districts enroll some
    825,000 students. They are the largest in the country so far to
    abandon plans for even a partial physical return to classrooms when
    they reopen in August. For other districts, the solution won't be an
    all-or-nothing approach.
    \href{https://bioethics.jhu.edu/research-and-outreach/projects/eschool-initiative/school-policy-tracker/}{Many
    systems}, including the nation's largest, New York City, are
    devising
    \href{https://www.nytimes.com/2020/06/26/us/coronavirus-schools-reopen-fall.html?action=click\&pgtype=Article\&state=default\&region=MAIN_CONTENT_3\&context=storylines_faq}{hybrid
    plans} that involve spending some days in classrooms and other days
    online. There's no national policy on this yet, so check with your
    municipal school system regularly to see what is happening in your
    community.
  \end{itemize}
\item ~
  \hypertarget{is-the-coronavirus-airborne}{%
  \paragraph{Is the coronavirus
  airborne?}\label{is-the-coronavirus-airborne}}

  \begin{itemize}
  \tightlist
  \item
    The coronavirus
    \href{https://www.nytimes.com/2020/07/04/health/239-experts-with-one-big-claim-the-coronavirus-is-airborne.html?action=click\&pgtype=Article\&state=default\&region=MAIN_CONTENT_3\&context=storylines_faq}{can
    stay aloft for hours in tiny droplets in stagnant air}, infecting
    people as they inhale, mounting scientific evidence suggests. This
    risk is highest in crowded indoor spaces with poor ventilation, and
    may help explain super-spreading events reported in meatpacking
    plants, churches and restaurants.
    \href{https://www.nytimes.com/2020/07/06/health/coronavirus-airborne-aerosols.html?action=click\&pgtype=Article\&state=default\&region=MAIN_CONTENT_3\&context=storylines_faq}{It's
    unclear how often the virus is spread} via these tiny droplets, or
    aerosols, compared with larger droplets that are expelled when a
    sick person coughs or sneezes, or transmitted through contact with
    contaminated surfaces, said Linsey Marr, an aerosol expert at
    Virginia Tech. Aerosols are released even when a person without
    symptoms exhales, talks or sings, according to Dr. Marr and more
    than 200 other experts, who
    \href{https://academic.oup.com/cid/article/doi/10.1093/cid/ciaa939/5867798}{have
    outlined the evidence in an open letter to the World Health
    Organization}.
  \end{itemize}
\end{itemize}

In the new analysis, a team led by Parham Azimi, an indoor-air
researcher at Harvard's T.H. Chan School of Public Health, studied the
outbreak on the Diamond Princess, where physical spaces and infections
were well documented. It ran more than 20,000 simulations of how the
virus might have spread throughout the ship. Each simulation made a
variety of assumptions, about factors like patterns of social
interaction --- how much time people spent in their cabins, on deck or
in the cafeteria, on average --- and the amount of time the virus can
live on surfaces. Each also factored in varying contributions of
smaller, floating droplets, broadly defined as 10 microns or smaller;
and larger droplets, which fall more quickly and infect surfaces or
other people, by landing on their eyes, mouth or nose, say.

About 130 of those simulations reproduced, to some extent, what actually
happened on the Diamond Princess as the outbreak progressed. By
analyzing these most ``realistic'' scenarios, the research team
calculated the most likely contributions of each route of transmission.
The researchers concluded that the smaller droplets predominated, and
accounted for about 60 percent of new infections over all, both at close
range, within a few yards of an infectious person, and at greater
distances.

``Many people have argued that airborne transmission is happening, but
no one had numbers for it,'' Dr. Azimi said. ``What is the contribution
from these small droplets --- is it 5 percent, or 90 percent? In this
paper, we provide the first real estimates for what that number could
be, at least in the case of this cruise ship.''

The logic behind such transmission is straightforward, experts said.
When a person is speaking, he or she emits a cloud of droplets, the vast
majority of which are small enough to remain suspended in the air for a
few minutes or longer. Through inhalation, that cloud of small droplets
is more likely to reach a mucus membrane than larger ones soaring
ballistically.

The smaller droplets are also more likely to penetrate deeply into the
respiratory system, down to the lungs. It may take a much smaller viral
load --- fewer viruses --- to cause infection in the lungs than higher
up, such as in the throat. This, at least, is the case for other
respiratory viruses, like the flu.

Brent Stephens, an engineering professor at the Illinois Institute of
Technology in Chicago and a co-author on the paper, said the findings
were important in shaping, for example, measures that should be taken as
college students return to campus.

The first, he said, should be ``really enforcing mask policies.''
Another, he said, is to recognize that there is a ``huge variability in
mask quality,'' and material that actually stops small aerosols when
someone is breathing, speaking, coughing or sneezing is crucial.
Surgical masks are good, he said, but single-ply fabrics often are not.

As various transmission routes come into clearer focus, they will
provide specific guidelines on how to reopen schools, offices,
restaurants and other businesses.

``The value of this model is that it allows for recommendations and
guidance to be specific to each unique environment,'' said another
co-author, Joseph G. Allen, an expert in indoor air quality and an
assistant professor at Harvard's T.H. Chan School of Public Health.

Dr. Allen said those environments ranged from restaurants to dentist
offices. In each case, he said, there are low-cost solutions that
sharply improve ventilation and filtration --- most buildings fall well
short of optimal levels --- and in turn reduce the risks of airborne
infection.

``To me, this is an all-in moment,'' Dr. Allen said. ``We need better
ventilation and better filtration, across the board, in all our
buildings.''

\textbf{\emph{{[}}\href{http://on.fb.me/1paTQ1h}{\emph{Like the Science
Times page on Facebook.}}} ****** \emph{\textbar{} Sign up for the}
\textbf{\href{http://nyti.ms/1MbHaRU}{\emph{Science Times
newsletter.}}\emph{{]}}}

Advertisement

\protect\hyperlink{after-bottom}{Continue reading the main story}

\hypertarget{site-index}{%
\subsection{Site Index}\label{site-index}}

\hypertarget{site-information-navigation}{%
\subsection{Site Information
Navigation}\label{site-information-navigation}}

\begin{itemize}
\tightlist
\item
  \href{https://help.nytimes.com/hc/en-us/articles/115014792127-Copyright-notice}{©~2020~The
  New York Times Company}
\end{itemize}

\begin{itemize}
\tightlist
\item
  \href{https://www.nytco.com/}{NYTCo}
\item
  \href{https://help.nytimes.com/hc/en-us/articles/115015385887-Contact-Us}{Contact
  Us}
\item
  \href{https://www.nytco.com/careers/}{Work with us}
\item
  \href{https://nytmediakit.com/}{Advertise}
\item
  \href{http://www.tbrandstudio.com/}{T Brand Studio}
\item
  \href{https://www.nytimes.com/privacy/cookie-policy\#how-do-i-manage-trackers}{Your
  Ad Choices}
\item
  \href{https://www.nytimes.com/privacy}{Privacy}
\item
  \href{https://help.nytimes.com/hc/en-us/articles/115014893428-Terms-of-service}{Terms
  of Service}
\item
  \href{https://help.nytimes.com/hc/en-us/articles/115014893968-Terms-of-sale}{Terms
  of Sale}
\item
  \href{https://spiderbites.nytimes.com}{Site Map}
\item
  \href{https://help.nytimes.com/hc/en-us}{Help}
\item
  \href{https://www.nytimes.com/subscription?campaignId=37WXW}{Subscriptions}
\end{itemize}
