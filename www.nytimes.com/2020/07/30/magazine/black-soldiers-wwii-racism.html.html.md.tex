Sections

SEARCH

\protect\hyperlink{site-content}{Skip to
content}\protect\hyperlink{site-index}{Skip to site index}

\href{https://myaccount.nytimes.com/auth/login?response_type=cookie\&client_id=vi}{}

\href{https://www.nytimes.com/section/todayspaper}{Today's Paper}

Returning From War, Returning to Racism

\begin{itemize}
\item
\item
\item
\item
\item
\item
\end{itemize}

Advertisement

\protect\hyperlink{after-top}{Continue reading the main story}

Supported by

\protect\hyperlink{after-sponsor}{Continue reading the main story}

Beyond The World War II We Know

\hypertarget{returning-from-war-returning-to-racism}{%
\section{Returning From War, Returning to
Racism}\label{returning-from-war-returning-to-racism}}

After fighting overseas, Black soldiers faced violence and segregation
at home. Many, like Lewis W. Matthews, were forced to take menial jobs.
Although he managed to push through racism, that wasn't an option for
many.

\includegraphics{https://static01.nyt.com/images/2020/07/30/multimedia/30ww2-black-soldiers-returning/30ww2-black-soldiers-returning-articleLarge.jpg?quality=75\&auto=webp\&disable=upscale}

By Alexis Clark

\begin{itemize}
\item
  July 30, 2020
\item
  \begin{itemize}
  \item
  \item
  \item
  \item
  \item
  \item
  \end{itemize}
\end{itemize}

\emph{\emph{\emph{The latest article from
``}\href{https://www.nytimes.com/spotlight/beyond-wwii}{\emph{Beyond the
World War II We Know}}},'' a series by The Times that documents
lesser-known stories from the war, focuses on the racism and segregation
that Black soldiers faced upon their return.}**

His trip back home in May 1946 was much like the one going --- 30 days
of sailing between the South Pacific and Oakland, mostly spent below
deck in a separate area for Black soldiers.

After guarding the gasoline supply for Army vehicles and planes and
taking fire while on patrol in the Philippines, Lewis W. Matthews, then
a corporal in an all-Black unit, was no better off socially after World
War II than he'd been before joining the service. The Army was still
segregated, and so was much of the United States.

``I thought there would be a big change in that,'' said Matthews, now
93.

After the formal Japanese surrender on Sept. 2, 1945, Matthews
disembarked in Oakland and headed home to New York City to start a new
chapter in life as a veteran with an honorable discharge. But he, along
with the 1.2 million African-Americans who served, would discover that
another battle, the one for equality in the United States, raged on.

Black soldiers returning from the war found the same socioeconomic ills
and racist violence that they faced before. Despite their sacrifices
overseas, they still struggled to get hired for well-paying jobs,
encountered segregation and endured targeted brutality, especially while
wearing their military uniforms. Black veterans realized that being
treated as equals was still a matter society hadn't resolved.

``At the heart of it was a kind of nervousness and fear that many whites
had that returning Black veterans would upset the racial status quo,''
said Charissa Threat, a history professor at Chapman University, who has
written extensively on civilian-military relationships and race. ``They
saw images of Black soldiers coming from abroad from places like Germany
and England, where Black soldiers were intermingling with whites and had
a lot more freedom.''

To quell any expectation of social equality held by African-American
servicemen, mobs of whites engaged in unspeakable violence toward them.
A case from February 1946 involved Isaac Woodard, a Black veteran who
served in the Pacific theater. After getting into an argument with a bus
driver while traveling from Georgia to South Carolina, Woodard, in his
uniform, was ordered off the bus in a town now known as
Batesburg-Leesville, S.C., and
\href{https://www.nytimes.com/2019/02/08/us/sergeant-woodard-batesburg-south-carolina.html}{beaten
so badly} with a billy club by the local police chief that he was
permanently blinded.

In August of the same year, John C. Jones, a Black veteran, was lynched
in Minden, La., after he was accused of looking at a young white woman
through a window of her family's house. Two other Black veterans,
Richard Gordon and Alonza Brooks, were murdered in Marshall, Texas,
after a labor dispute with their employers.

The violence became so pervasive and brutal that civil rights activists
formed the National Emergency Committee Against Mob Violence in 1946. A
delegation representing the group met with President Harry S. Truman,
arguing for a federal anti-lynching law, but Southern Democrats shut
down Truman's attempt.

Hope came in the form of the G.I. Bill of Rights, a substantial piece of
social legislation that President Franklin D. Roosevelt signed into law
in 1944 to avert mass unemployment among returning veterans and a
postwar depression. Promoted as race-neutral, the G.I. Bill offered
veterans unemployment insurance, tuition assistance, job placement and
guaranteed loans for homes, farms or businesses.

On the face of it, the bill was transformative. During the war, the
N.A.A.C.P. and other civil rights groups encouraged Blacks to enlist in
the military so they could receive G.I. benefits. After the war,
however, the bill failed to propel Black servicemen into the middle
class in the numbers it did for white veterans. Discrimination toward
African-Americans found its way through loopholes in the legislation,
just as it did in everyday life.

``So much of the G.I. Bill involved deference to state and local
authorities,'' said Steven White, a political-science professor at
Syracuse University and the author of ``World War II and American Racial
Politics: Public Opinion, the Presidency and Civil Rights Advocacy.''
``Black southerners, even if they got benefits, they couldn't go to the
same colleges and universities. They couldn't get the same jobs.''

Despite coming out of the military fully trained as mechanics,
carpenters, welders or electricians, Black veterans encountered white
job counselors at local employment offices who refused to refer them for
skilled and semiskilled jobs.

``State employment agencies all across the country honored employer
requests for whites only for many jobs,'' said Richard Rothstein, author
of ``The Color of Law: A Forgotten History of How Our Government
Segregated America.''

Senator John Rankin, an openly racist Mississippi Democrat who helped
draft the G.I. Bill, made sure states controlled the distribution of
veteran benefits. According to the historian David H. Onkst
\href{https://www.jstor.org/stable/3789713?seq=1}{in his article}
``First a Negro \ldots{} Incidentally a Veteran,'' in October 1946, for
example, out of the 6,583 nonagricultural jobs filled in Mississippi by
G.I. Bill job counselors, 86 percent of the professional, skilled and
semiskilled positions went to whites, while 92 percent of the unskilled
and service-sector jobs went to Blacks.

\includegraphics{https://static01.nyt.com/images/2020/07/30/multimedia/30ww2-black-soldiers-returning-02/30ww2-black-soldiers-returning-02-articleLarge.jpg?quality=75\&auto=webp\&disable=upscale}

When Lewis Matthews returned home to his mother's apartment in the
Bronx, he made deliveries for an art supply store and juggled other
menial jobs. But, needing to earn a better living, he used his G.I.
tuition benefit to enroll in a training program in Newark that taught
him how to make denture molds. He then found a job for a denture
manufacturing business near Times Square.

``All they told me to do was mix plaster,'' said Matthews, who had been
trained to do more technical work with the molds, which would have paid
better. ``I said to hell with this damn job, and I left and went back to
the G.I. people,'' he said. ``I told them: `Hey, look, I can't make it
out here. They aren't paying me any money in that jive job.'''

Matthews, who had dropped out of high school to earn money for his
family before joining the service at 16, decided to get his high school
diploma and then enrolled at New York University, where he studied
business administration for the remaining three years on his G.I.
education benefit. ``I couldn't afford to pay for the rest of N.Y.U.,
but I read everything I could get my hands on concerning everything I
wanted to learn,'' he said.

Unlike Matthews, many Black veterans were denied access to a college
education, largely relegated to vocational programs. According to the
journalist and historian Edward Humes, in his article ``How the G.I.
Bill Shunted Blacks Into Vocational Training,'' 28 percent of white
veterans went to college on the G.I. Bill, compared with 12 percent of
Blacks. Of that number, upward of 90 percent of Black veterans attended
historically Black colleges and universities --- institutions mainly in
the South that were already underfunded with limited resources. In his
book ``When Affirmative Action Was White: An Untold History of Racial
Inequality in Twentieth-Century America,'' Ira Katznelson wrote that the
enrollment of veterans at historically Black colleges and universities
was 29,000 in 1940 and 73,000 in 1947. And 20,000 to 50,000 were turned
away because of limited capacity, Humes wrote.

After studying at N.Y.U. and working more menial jobs, Matthews heard a
radio announcement about selling life insurance. He became a general
agent for a number of years, but was denied a small-business loan from
the Department of Veterans Affairs when he wanted to open his own
insurance office. ``I know Black veterans who couldn't get loans and had
real problems,'' he said.

African-Americans were routinely denied mortgages, and Black veterans
were no exception. During the summer of 1947, Ebony magazine surveyed 13
cities in Mississippi and discovered that of the 3,229 V.A. home loans
given to veterans, two went to African-Americans. According to Humes, in
the postwar years, two out of three whites owned a home, whereas Black
homeownership stayed around 40 percent. And it wasn't just in the South.

``There were planned communities like Levittown in Long Island that
didn't allow Blacks,'' said Jeffrey Sammons, a history professor at
N.Y.U., whose research focuses on African-Americans in the military and
sports. Many of these communities were developed specifically for white
veterans.

``By even owning a house, you create equity, and that creates wealth for
the next generation,'' Threat said. ``African-Americans did not have the
opportunity to create a future generation of economic security.''

Rothstein put it more bluntly: ``The wealth gap was created by these
unconstitutional policies.''

Civil rights groups, frustrated by the lack of progress, continued to
press Truman on legislation for racial equality. Knowing that civil
rights legislation would stall in Congress, and with the reputation of
the United States as a great democratic nation being questioned as
racism continued to flourish during a nascent Cold War, on July 26,
1948, Truman signed two
\href{https://prologue.blogs.archives.gov/2014/05/19/executive-orders-9980-and-9981-ending-segregation-in-the-armed-forces-and-the-federal-workforce/\#:~:text=Pieces\%20of\%20History-,Executive\%20Orders\%209980\%20and\%209981\%3A\%20Ending\%20segregation\%20in\%20the,Forces\%20and\%20the\%20Federal\%20workforce\&text=Without\%20Congress's\%20blessing\%2C\%20the\%20executive,carry\%20the\%20force\%20of\%20law.}{Executive
Orders, 9980 and 9981}, desegregating the federal work force and armed
services --- practices that would take years to be fully carried out.

Matthews bought his first house in the Bushwick section of Brooklyn with
a V.A. mortgage. Matthews, a father of eight who has since lived in the
Bedford-Stuyvesant neighborhood of Brooklyn for decades with his second
wife, Dovie Ree Matthews, said he had no regrets about his service,
despite any racism he endured. He credits the G.I. Bill for his success.

``I'm so glad that I was a soldier,'' he said. ``I knew that I was
Black, and I knew that they discriminated against me. But I tried to
make the best of my situation.''

\begin{center}\rule{0.5\linewidth}{\linethickness}\end{center}

Alexis Clark **** is an adjunct professor at Columbia Journalism School
and author of ``Enemies in Love: A German POW, a Black Nurse and an
Unlikely Romance.''

Advertisement

\protect\hyperlink{after-bottom}{Continue reading the main story}

\hypertarget{site-index}{%
\subsection{Site Index}\label{site-index}}

\hypertarget{site-information-navigation}{%
\subsection{Site Information
Navigation}\label{site-information-navigation}}

\begin{itemize}
\tightlist
\item
  \href{https://help.nytimes.com/hc/en-us/articles/115014792127-Copyright-notice}{©~2020~The
  New York Times Company}
\end{itemize}

\begin{itemize}
\tightlist
\item
  \href{https://www.nytco.com/}{NYTCo}
\item
  \href{https://help.nytimes.com/hc/en-us/articles/115015385887-Contact-Us}{Contact
  Us}
\item
  \href{https://www.nytco.com/careers/}{Work with us}
\item
  \href{https://nytmediakit.com/}{Advertise}
\item
  \href{http://www.tbrandstudio.com/}{T Brand Studio}
\item
  \href{https://www.nytimes.com/privacy/cookie-policy\#how-do-i-manage-trackers}{Your
  Ad Choices}
\item
  \href{https://www.nytimes.com/privacy}{Privacy}
\item
  \href{https://help.nytimes.com/hc/en-us/articles/115014893428-Terms-of-service}{Terms
  of Service}
\item
  \href{https://help.nytimes.com/hc/en-us/articles/115014893968-Terms-of-sale}{Terms
  of Sale}
\item
  \href{https://spiderbites.nytimes.com}{Site Map}
\item
  \href{https://help.nytimes.com/hc/en-us}{Help}
\item
  \href{https://www.nytimes.com/subscription?campaignId=37WXW}{Subscriptions}
\end{itemize}
