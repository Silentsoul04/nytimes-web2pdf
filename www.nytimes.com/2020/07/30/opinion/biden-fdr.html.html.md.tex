\href{/section/opinion}{Opinion}\textbar{}The Future of American
Liberalism

\href{https://nyti.ms/39GPd7n}{https://nyti.ms/39GPd7n}

\begin{itemize}
\item
\item
\item
\item
\item
\item
\end{itemize}

\includegraphics{https://static01.nyt.com/images/2020/07/30/opinion/30brooksWeb/30brooksWeb-articleLarge.jpg?quality=75\&auto=webp\&disable=upscale}

Sections

\protect\hyperlink{site-content}{Skip to
content}\protect\hyperlink{site-index}{Skip to site index}

\href{/section/opinion}{Opinion}

\hypertarget{the-future-of-american-liberalism}{%
\section{The Future of American
Liberalism}\label{the-future-of-american-liberalism}}

What Biden can learn from F.D.R.

Franklin Roosevelt's New Deal brought balance to an economy that had
become wildly unbalanced. Credit...Bettmann Archive/Getty Images

Supported by

\protect\hyperlink{after-sponsor}{Continue reading the main story}

\href{https://www.nytimes.com/by/david-brooks}{\includegraphics{https://static01.nyt.com/images/2018/04/03/opinion/david-brooks/david-brooks-thumbLarge-v2.png}}

By \href{https://www.nytimes.com/by/david-brooks}{David Brooks}

Opinion Columnist

\begin{itemize}
\item
  July 30, 2020
\item
  \begin{itemize}
  \item
  \item
  \item
  \item
  \item
  \item
  \end{itemize}
\end{itemize}

The United States just endured its worst economic quarter in recorded
history. If this trend had continued for an entire year, American
economic output would have been down by about a third.

So I'm hoping Joe Biden and his team are reading up on Franklin
Roosevelt and the New Deal. The New Dealers succeeded in a moment like
this. Their experience offers some powerful lessons for Biden as he
campaigns and if he wins:

\hypertarget{offer-big-change-that-feels-familiar}{%
\subsection{Offer big change that feels
familiar.}\label{offer-big-change-that-feels-familiar}}

Economic and health calamities are experienced by most people as if they
were natural disasters and complete societal breakdowns. People feel
intense waves of fear about the future. They want a leader, like F.D.R.,
who demonstrates optimistic fearlessness.

They want one who, once in office, produces an intense burst of activity
that is both new but also offers people security and safety. During the
New Deal, Social Security gave seniors secure retirements. The Works
Progress Administration gave 8.5 million Americans secure jobs.

Biden's ``Build Back Better'' slogan is a perfect encapsulation of this
mood of simultaneously longing for the safety of the past while moving
to a brighter future.

\hypertarget{broadcast-pragmatism-not-ideology}{%
\subsection{Broadcast pragmatism, not
ideology.}\label{broadcast-pragmatism-not-ideology}}

New Dealers were willing to try anything that met the specific
emergencies of the moment. There was a strong anti-ideological bias in
the administration and a wanton willingness to experiment. For example,
Roosevelt's first instinct was to cut government spending in order to
reduce the deficit, until he flipped, realizing that it wouldn't work in
a depression.

``I really do not know what the basic principle of the New Deal is,''
one of his top advisers admitted. That pragmatism reassured the American
people, who didn't want a revolution; they wanted a recovery.

\hypertarget{even-in-a-crisis-of-capitalism-embrace-capitalism}{%
\subsection{Even in a crisis of capitalism, embrace
capitalism.}\label{even-in-a-crisis-of-capitalism-embrace-capitalism}}

Historian ** Richard Pells
\href{https://www.google.com/books/edition/Radical_Visions_and_American_Dreams/ENEKTEdhhtMC?hl=en\&gbpv=0}{notes}
that flagship progressive magazines like The Nation and The New Republic
did not endorse F.D.R. in 1932, but rather his socialist opponent,
Norman Thomas. As the New Deal succeeded, many progressive intellectuals
mobilized a barrage of criticism against it. By 1934 they were producing
books with titles like ``The Coming American Revolution'' and calling
for the creation of a new political party of the left.

They understood Roosevelt was a liberal capitalist, not a socialist. ``I
want to save our system, the capitalist system,'' he said at one point.
``My desire {[}is{]} to obviate revolution,'' he said at another. He was
seeking to save capitalism from the capitalists, who had concentrated
too much power in themselves. He was trying to reform capitalism to
preserve it.

\hypertarget{get-capitalism-moving}{%
\subsection{Get capitalism moving.}\label{get-capitalism-moving}}

The Reconstruction Finance Corporation, run by Jesse Jones, a Hoover
administration holdover, gave bankers incentives to take the capital
that had been sitting in their vaults and get it out into the community.
The Federal Housing Administration backed mortgages. As Louis Hyman of
Cornell
\href{https://www.theatlantic.com/ideas/archive/2019/03/surprising-truth-about-roosevelts-new-deal/584209/}{notes},
the F.H.A. induced more private lending in a few months than the Public
Works Administration spent during the entire decade. The New Deal was
more clever and diverse than just tax-and-spend liberalism.

\hypertarget{embrace-expertise}{%
\subsection{Embrace expertise.}\label{embrace-expertise}}

Huey Long, Father Coughlin and Francis Townsend were leading a populist
revolt that threatened to bring an era of bottom-up authoritarianism.
F.D.R. tried to co-opt them a bit, but mostly he just outperformed them
with talent. He staffed his administration with a very bright and
unabashedly ``brains trust'' array of lawyers, professors, economists
and social workers.

\hypertarget{look-for-imbalances}{%
\subsection{Look for imbalances.}\label{look-for-imbalances}}

Capitalist economies get out of whack from time to time. The New Deal
brought balance. It made it easier for workers to unionize and deal on
more equal terms with business. Wall Street was too powerful. The New
Deal reined it in.

\hypertarget{devolve-power-to-congress}{%
\subsection{Devolve power to
Congress.}\label{devolve-power-to-congress}}

Historian Ira Katznelson argues that too much attention is paid to
F.D.R., when the real action was in Congress. If you want to unleash a
torrent of action you have to let individual members of Congress drive
their own initiatives, not concentrate power in the White House or House
speaker's office.

The New Deal didn't produce an instant economic turnaround. But it did
show that democratic capitalism could still function. His enemies called
Roosevelt a socialist or a populist, but in reality it was Roosevelt who
defeated socialism and populism. In America at least, they were spent
forces by 1939.

F.D.R. also demonstrated that the most effective leaders in crisis are
often at the center of their party, not at left or right vanguard.
Abraham Lincoln took enormous heat from abolitionists. But he's the one
who defeated slavery. Theodore Roosevelt had a conservative disposition
and lagged behind many Progressives. But he's the one who led
Progressive reforms. F.D.R. was able to pass so much legislation
precisely because he was so shifting and pragmatic and did not turn
everything into a polarized war.

We're not going to have another Roosevelt. But in a time of crisis, in
an ideological age, he showed it's possible to get a lot done if you
turn down the ideological temperature, if you evade the culture war, if
you are willing to be positive and openly experimental.

That's the New Dealers' big lesson for Biden \& Company.

\emph{The Times is committed to publishing}
\href{https://www.nytimes.com/2019/01/31/opinion/letters/letters-to-editor-new-york-times-women.html}{\emph{a
diversity of letters}} \emph{to the editor. We'd like to hear what you
think about this or any of our articles. Here are some}
\href{https://help.nytimes.com/hc/en-us/articles/115014925288-How-to-submit-a-letter-to-the-editor}{\emph{tips}}\emph{.
And here's our email:}
\href{mailto:letters@nytimes.com}{\emph{letters@nytimes.com}}\emph{.}

\emph{Follow The New York Times Opinion section on}
\href{https://www.facebook.com/nytopinion}{\emph{Facebook}}\emph{,}
\href{http://twitter.com/NYTOpinion}{\emph{Twitter (@NYTopinion)}}
\emph{and}
\href{https://www.instagram.com/nytopinion/}{\emph{Instagram}}\emph{.}

Advertisement

\protect\hyperlink{after-bottom}{Continue reading the main story}

\hypertarget{site-index}{%
\subsection{Site Index}\label{site-index}}

\hypertarget{site-information-navigation}{%
\subsection{Site Information
Navigation}\label{site-information-navigation}}

\begin{itemize}
\tightlist
\item
  \href{https://help.nytimes.com/hc/en-us/articles/115014792127-Copyright-notice}{©~2020~The
  New York Times Company}
\end{itemize}

\begin{itemize}
\tightlist
\item
  \href{https://www.nytco.com/}{NYTCo}
\item
  \href{https://help.nytimes.com/hc/en-us/articles/115015385887-Contact-Us}{Contact
  Us}
\item
  \href{https://www.nytco.com/careers/}{Work with us}
\item
  \href{https://nytmediakit.com/}{Advertise}
\item
  \href{http://www.tbrandstudio.com/}{T Brand Studio}
\item
  \href{https://www.nytimes.com/privacy/cookie-policy\#how-do-i-manage-trackers}{Your
  Ad Choices}
\item
  \href{https://www.nytimes.com/privacy}{Privacy}
\item
  \href{https://help.nytimes.com/hc/en-us/articles/115014893428-Terms-of-service}{Terms
  of Service}
\item
  \href{https://help.nytimes.com/hc/en-us/articles/115014893968-Terms-of-sale}{Terms
  of Sale}
\item
  \href{https://spiderbites.nytimes.com}{Site Map}
\item
  \href{https://help.nytimes.com/hc/en-us}{Help}
\item
  \href{https://www.nytimes.com/subscription?campaignId=37WXW}{Subscriptions}
\end{itemize}
