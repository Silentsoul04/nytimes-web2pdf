Sections

SEARCH

\protect\hyperlink{site-content}{Skip to
content}\protect\hyperlink{site-index}{Skip to site index}

\href{https://myaccount.nytimes.com/auth/login?response_type=cookie\&client_id=vi}{}

\href{https://www.nytimes.com/section/todayspaper}{Today's Paper}

\href{/section/opinion}{Opinion}\textbar{}Yes, the Coronavirus Is in the
Air

\href{https://nyti.ms/3hUaaPb}{https://nyti.ms/3hUaaPb}

\begin{itemize}
\item
\item
\item
\item
\item
\end{itemize}

Advertisement

\protect\hyperlink{after-top}{Continue reading the main story}

\href{/section/opinion}{Opinion}

Supported by

\protect\hyperlink{after-sponsor}{Continue reading the main story}

\hypertarget{yes-the-coronavirus-is-in-the-air}{%
\section{Yes, the Coronavirus Is in the
Air}\label{yes-the-coronavirus-is-in-the-air}}

Transmission through aerosols matters --- and probably a lot more than
we've been able to prove yet.

By Linsey C. Marr

Dr. Marr is a professor of engineering.

\begin{itemize}
\item
  July 30, 2020
\item
  \begin{itemize}
  \item
  \item
  \item
  \item
  \item
  \end{itemize}
\end{itemize}

\includegraphics{https://static01.nyt.com/images/2020/07/30/opinion/30Marr/30Marr-articleLarge.jpg?quality=75\&auto=webp\&disable=upscale}

\href{https://www.nytimes.com/es/2020/08/01/espanol/opinion/coronavirus-aire.html}{Leer
en
español}\href{https://www.nytimes.com/pt/2020/08/05/opinion/international-world/coronavirus-esta-no-ar.html}{Ler
em português}

Finally. The World Health Organization has now formally recognized that
SARS-CoV-2, the virus that causes Covid-19,
\href{https://www.nytimes.com/2020/07/09/health/virus-aerosols-who.html}{is
airborne} and that it can be carried
\href{https://www.nature.com/articles/d41586-020-02058-1}{by tiny
aerosols}.

As we cough and sneeze, talk or just breathe, we
\href{https://www.sciencedirect.com/science/article/pii/S0021850211001200}{naturally
release} droplets (small particles of fluid) and aerosols (smaller
particles of fluid) into the air. Yet until earlier this month, the
W.H.O. --- like the U.S. Centers for Disease Control and Prevention or
Public Health England ---
\href{https://www.who.int/news-room/commentaries/detail/modes-of-transmission-of-virus-causing-covid-19-implications-for-ipc-precaution-recommendations}{had
warned mostly} about the transmission of the new coronavirus through
direct contact and droplets released at close range.

The organization had cautioned against aerosols only in rare
circumstances, such as after intubation and other
\href{https://www.who.int/publications/i/item/WHO-2019-nCoV-IPC-2020.4}{medical
procedures} involving infected patients in hospitals.

After
\href{https://www.nature.com/articles/d41586-020-00974-w\#ref-CR5}{several
months of pressure from scientists}, on July 9, the W.H.O. changed its
position --- going from denial to
\href{https://www.who.int/news-room/commentaries/detail/transmission-of-sars-cov-2-implications-for-infection-prevention-precautions}{grudging
partial acceptance}: ``Further studies are needed to determine whether
it is possible to detect viable SARS-CoV-2 in air samples from settings
where no procedures that generate aerosols are performed and what role
aerosols might play in transmission.''

I am a civil and environmental engineer who studies how viruses and
bacteria spread through the air --- as well as
\href{https://www.nytimes.com/2020/07/04/health/239-experts-with-one-big-claim-the-coronavirus-is-airborne.html}{one
of the 239 scientists} who signed
\href{https://academic.oup.com/cid/article/doi/10.1093/cid/ciaa939/5867798}{an
open letter} in late June pressing the W.H.O. to consider the risk of
airborne transmission more seriously.

A month later, I believe that the transmission of SARS-CoV-2 via
aerosols matters much more than has been officially acknowledged to
date.

In \href{https://www.nature.com/articles/s41598-020-69286-3}{a
peer-reviewed study} published in Nature on Wednesday, researchers at
the University of Nebraska Medical Center found that aerosols collected
in the hospital rooms of Covid-19 patients contained the coronavirus.

This confirms the results of
\href{https://www.medrxiv.org/content/10.1101/2020.05.31.20115154v1}{a
study} from late May (not peer-reviewed) in which Covid-19 patients were
found to release SARS-CoV-2 simply by exhaling --- without coughing or
even talking. The authors of that study said the finding implied that
airborne transmission ``plays a major role'' in spreading the virus.

Accepting these conclusions wouldn't much change what is currently being
recommended as best behavior. The strongest protection against
SARS-CoV-2, whether the virus is mostly contained in droplets or in
aerosols, essentially remains the same: Keep your distance and wear
masks.

Rather, the recent findings are an important reminder to also be
vigilant about opening windows and improving airflow indoors. And they
are further evidence that the quality of masks and their fit matter,
too.

The W.H.O. defines as a ``droplet''
\href{https://www.who.int/news-room/commentaries/detail/modes-of-transmission-of-virus-causing-covid-19-implications-for-ipc-precaution-recommendations}{a
particle larger than 5 microns} and has said that droplets don't travel
farther than one meter.

In fact, there is no neat and no meaningful cutoff point --- at 5
microns or any other size --- between droplets and aerosols: All are
tiny specks of liquid, their size ranging along a spectrum that goes
from very small to really microscopic.

(I am working with medical historians to track down the scientific basis
for the W.H.O.'s definition, and we have not found a sensible
explanation yet.)

Yes, droplets tend to fly through the air like mini cannonballs and they
fall to the ground rather quickly, while aerosols can float around for
many hours.

But basic physics also says that a 5-micron droplet takes about a
half-hour to drop to the floor from the mouth of an adult of average
height --- and during that time, the droplet can travel many meters on
an air current. Droplets expelled in coughs or sneezes also
\href{https://academic.oup.com/jid/advance-article/doi/10.1093/infdis/jiaa189/5820886}{travel
much farther than one meter}.

Here is another common misconception: To the (limited) extent that the
role of aerosols had been recognized so far, they were usually mentioned
as lingering in the air, suspended, and wafting away --- a long-distance
threat.

But before aerosols can get far, they must travel through the air that's
near: meaning that they are a hazard at close range, too. And all the
more so because, just like the smoke from a cigarette, aerosols are most
concentrated near the infected person (or smoker) and become diluted in
the air as they drift away.

\href{https://www.sciencedirect.com/science/article/abs/pii/S0360132320302183?via\%3Dihub}{A
peer-reviewed study} by scientists at the University of Hong Kong and
Zhejiang University, in Hangzhou, China, published in the journal
Building and Environment in June concluded, ``The smaller the exhaled
droplets, the more important the short-range airborne route.''

So what does this all mean exactly, practically?

Can you walk into an empty room and contract the virus if an infected
person, now gone, was there before you? Perhaps, but probably only if
the room is small and stuffy.

Can the virus waft up and down buildings via air ducts or pipes? Maybe,
though that hasn't been established.

More likely, the research suggests, aerosols matter in extremely mundane
scenarios.

Consider
\href{https://www.nytimes.com/2020/04/20/health/airflow-coronavirus-restaurants.html}{the
case of a restaurant in Guangzhou}, southern China, at the beginning of
the year, in which one diner infected with SARS-CoV-2 at one table
spread the virus to a total of nine people seated at their table and two
other tables.

Yuguo Li, a professor of engineering at the University of Hong Kong, and
colleagues
\href{https://www.medrxiv.org/content/10.1101/2020.04.16.20067728v1}{analyzed
video footage} from the restaurant and in a preprint (not peer reviewed)
published in April found no evidence of close contact between the
diners.

Droplets can't account for transmission in this case, at least not among
the people at the tables other than the infected person's: The droplets
would have fallen to the floor before reaching those tables.

But the three tables were in a poorly ventilated section of the
restaurant, and an air conditioning unit pushed air across them.
Notably, too, no staff member and none of the other diners in the
restaurant --- including at two tables just beyond the air conditioner's
airstream --- became infected.

Similarly, just one person is thought to have infected 52 of the other
60 people at
\href{https://www.nytimes.com/2020/05/12/health/coronavirus-choir.html}{a
choir rehearsal} in Skagit County, Wash., in March.

Several colleagues at various universities and I analyzed that event and
in
\href{https://www.medrxiv.org/content/10.1101/2020.06.15.20132027v2}{a
preprint (not peer-reviewed) published last month} concluded that
aerosols likely were the dominant means of transmission.

Attendees had used hand sanitizer and avoided hugs and handshakes,
limiting the potential for infection through direct contact or droplets.
On the other hand, the room was poorly ventilated, the rehearsal lasted
a long time (2.5 hours) and singing is known to produce aerosols and
\href{https://www.atsjournals.org/doi/abs/10.1164/arrd.1968.98.2.297}{facilitate
the spread of diseases like tuberculosis}.

What about the outbreak on the Diamond Princess cruise ship off Japan
early this year? Some 712 of the 3,711 people on board became infected.

Professor Li and others also
\href{https://www.medrxiv.org/content/10.1101/2020.04.09.20059113v1}{investigated
that case} and in a preprint (not peer reviewed) in April concluded that
transmission had not occurred between rooms after people were
quarantined: The ship's air-conditioning system did not spread the virus
over long distances.

The more likely cause of transmission, according to that study, appeared
to be close contact with infected people or contaminated objects before
the passengers and crew members were isolated. (The researchers did not
parse precisely what they meant by contact, or if that included droplets
or short-range aerosols.)

But
\href{https://www.medrxiv.org/content/10.1101/2020.07.13.20153049v1}{another,
recent, preprint} (not peer reviewed) about the Diamond Princess
concluded that ``aerosol inhalation was likely the dominant contributor
to Covid-19 transmission'' among the ship's passengers.

It might seem logical, or make intuitive sense, that larger droplets
would contain more virus than do smaller aerosols --- but they don't.

\href{https://www.thelancet.com/journals/lanres/article/PIIS2213-2600(20)30323-4/fulltext}{A
paper published this week} by The Lancet Respiratory Medicine that
analyzed the aerosols produced by the coughs and exhaled breaths of
patients with various respiratory infections found ``a predominance of
pathogens in small particles'' (under 5 microns). ``There is no
evidence,'' the study also concluded, ``that some pathogens are carried
only in large droplets.''

A
\href{https://www.medrxiv.org/content/10.1101/2020.07.13.20041632v2}{recent
preprint} (not peer reviewed) by researchers at the University of
Nebraska Medical Center found that viral samples retrieved from aerosols
emitted by Covid-19 patients were infectious.

Some scientists
\href{https://jamanetwork.com/journals/jama/fullarticle/2768396}{have
argued} that just because aerosols can contain SARS-CoV-2 does not in
itself prove that they can cause an infection and that if SARS-CoV-2
were primarily spread by aerosols, there would be more evidence of
long-range transmission.

I agree that long-range transmission by aerosols probably is not
significant, but I believe that, taken together, much of the evidence
gathered to date suggests that \emph{close-range} transmission by
aerosols is significant --- possibly very significant, and certainly
more significant than direct droplet spray.

The practical implications are plain:

\begin{itemize}
\tightlist
\item
  \textbf{Social distancing really is important}. It keeps us out of the
  most concentrated parts of other people's respiratory plumes. So stay
  away from one another by one or two meters at least --- though farther
  is safer.
\end{itemize}

\begin{itemize}
\tightlist
\item
  \textbf{Wear a mask.} Masks help block aerosols released by the
  wearer.
  \href{https://ucsf.app.box.com/s/blvolkp5z0mydzd82rjks4wyleagt036}{Scientific
  evidence} is also building that
  \href{https://www.nytimes.com/2020/07/27/health/coronavirus-mask-protection.html?campaign_id=154\&emc=edit_cb_20200727\&instance_id=20696\&nl=coronavirus-briefing\&regi_id=65413713\&segment_id=34503\&te=1\&user_id=bd32fbf008e5183a7928ed61c60669f7}{masks
  protect the wearer from breathing in aerosols} around them.
\end{itemize}

When it comes to masks, size \emph{does} matter.

The gold standard is a N95 or a KN95 respirator, which, if properly
fitted, filters out and prevents the wearer from breathing in at least
95 percent of small aerosols.

The efficacy of surgical masks against aerosols varies widely.

\href{https://pubmed.ncbi.nlm.nih.gov/23498357/}{One study from 2013}
found that surgical masks reduced exposure to flu viruses by between 10
percent and 98 percent (depending on the mask's design).

A recent paper found that surgical masks can completely block seasonal
coronaviruses
\href{https://www.nature.com/articles/s41591-020-0843-2}{from getting
into the air}.

To my knowledge, no similar study has been conducted for SARS-CoV-2 yet,
but these findings might apply to this virus as well since it is similar
to seasonal coronaviruses in size and structure.

My lab has been testing cloth masks on a mannequin, sucking in air
through its mouth at a realistic rate. We found that even a bandanna
loosely tied over its mouth and nose blocked half or more of aerosols
larger than 2 microns from entering the mannequin.

We also found that especially with very small aerosols --- smaller than
1 micron --- it is more effective to use a softer fabric (which is
easier to fit tightly over the face) than a stiffer fabric (which, even
if it is a better filter, tends to sit more awkwardly, creating gaps).

\begin{itemize}
\tightlist
\item
  \textbf{Avoid crowds.} The more people around you, the more likely
  someone among them will be infected. Especially avoid crowds indoors,
  where aerosols can accumulate.
\end{itemize}

\begin{itemize}
\tightlist
\item
  \textbf{Ventilation counts.} Open windows and doors. Adjust dampers in
  air-conditioning and heating systems. Upgrade the filters in those
  systems. Add portable air cleaners, or install germicidal ultraviolet
  technologies to remove or kill virus particles in the air.
\end{itemize}

It's not clear just how much this coronavirus is transmitted by aerosols
as opposed to droplets or via contact with contaminated surfaces. Then
again, we still don't know the answer to that question
\href{https://journals.plos.org/plospathogens/article?id=10.1371/journal.ppat.1008704}{even
for the flu}, which has been studied for decades.

But by now we do know this much: Aerosols matter in the transmission of
Covid-19 --- and probably even more so than we have yet been able to
prove.

Linsey C. Marr
(\href{https://twitter.com/linseymarr?lang=en}{@linseymarr}) is a
professor of civil and environmental engineering at Virginia Tech.

\emph{The Times is committed to publishing}
\href{https://www.nytimes.com/2019/01/31/opinion/letters/letters-to-editor-new-york-times-women.html}{\emph{a
diversity of letters}} \emph{to the editor. We'd like to hear what you
think about this or any of our articles. Here are some}
\href{https://help.nytimes.com/hc/en-us/articles/115014925288-How-to-submit-a-letter-to-the-editor}{\emph{tips}}\emph{.
And here's our email:}
\href{mailto:letters@nytimes.com}{\emph{letters@nytimes.com}}\emph{.}

\emph{Follow The New York Times Opinion section on}
\href{https://www.facebook.com/nytopinion}{\emph{Facebook}}\emph{,}
\href{http://twitter.com/NYTOpinion}{\emph{Twitter (@NYTopinion)}}
\emph{and}
\href{https://www.instagram.com/nytopinion/}{\emph{Instagram}}\emph{.}

Advertisement

\protect\hyperlink{after-bottom}{Continue reading the main story}

\hypertarget{site-index}{%
\subsection{Site Index}\label{site-index}}

\hypertarget{site-information-navigation}{%
\subsection{Site Information
Navigation}\label{site-information-navigation}}

\begin{itemize}
\tightlist
\item
  \href{https://help.nytimes.com/hc/en-us/articles/115014792127-Copyright-notice}{©~2020~The
  New York Times Company}
\end{itemize}

\begin{itemize}
\tightlist
\item
  \href{https://www.nytco.com/}{NYTCo}
\item
  \href{https://help.nytimes.com/hc/en-us/articles/115015385887-Contact-Us}{Contact
  Us}
\item
  \href{https://www.nytco.com/careers/}{Work with us}
\item
  \href{https://nytmediakit.com/}{Advertise}
\item
  \href{http://www.tbrandstudio.com/}{T Brand Studio}
\item
  \href{https://www.nytimes.com/privacy/cookie-policy\#how-do-i-manage-trackers}{Your
  Ad Choices}
\item
  \href{https://www.nytimes.com/privacy}{Privacy}
\item
  \href{https://help.nytimes.com/hc/en-us/articles/115014893428-Terms-of-service}{Terms
  of Service}
\item
  \href{https://help.nytimes.com/hc/en-us/articles/115014893968-Terms-of-sale}{Terms
  of Sale}
\item
  \href{https://spiderbites.nytimes.com}{Site Map}
\item
  \href{https://help.nytimes.com/hc/en-us}{Help}
\item
  \href{https://www.nytimes.com/subscription?campaignId=37WXW}{Subscriptions}
\end{itemize}
