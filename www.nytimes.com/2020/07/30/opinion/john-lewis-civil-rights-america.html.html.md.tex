Sections

SEARCH

\protect\hyperlink{site-content}{Skip to
content}\protect\hyperlink{site-index}{Skip to site index}

\href{/section/opinion}{Opinion}\textbar{}John Lewis: Together, You Can
Redeem the Soul of Our Nation

\url{https://nyti.ms/2P6qaku}

\begin{itemize}
\item
\item
\item
\item
\item
\item
\end{itemize}

\includegraphics{https://static01.nyt.com/newsgraphics/2020/07/14/op-header/c998714d96ba195218174d25716062b22597e147/h_14047247.jpg}
\includegraphics{https://static01.nyt.com/newsgraphics/2020/07/14/op-header/c998714d96ba195218174d25716062b22597e147/h_14047247.jpg}

 Opinion John Lewis

\hypertarget{together-you}{%
\section{Together, You}\label{together-you}}

Can Redeem the Soul\\
of Our Nation

\hypertarget{together-you-1}{%
\section{Together, You}\label{together-you-1}}

Can Redeem the Soul\\
of Our Nation

\hypertarget{though-i-am-gone-i-urge-you-to}{%
\subsection{Though I am gone, I urge you
to}\label{though-i-am-gone-i-urge-you-to}}

answer the highest calling of your heart and stand up for what you truly
believe.

\hypertarget{though-i-am-gone-i-urge}{%
\subsection{Though I am gone, I urge}\label{though-i-am-gone-i-urge}}

you to answer the highest calling\\
of your heart and stand\\
up for what you truly believe.

Supported by

\protect\hyperlink{after-sponsor}{Continue reading the main story}

\hypertarget{john-lewis-together-you-can-redeem-the-soul-of-our-nation}{%
\section{John Lewis: Together, You Can Redeem the Soul of Our
Nation}\label{john-lewis-together-you-can-redeem-the-soul-of-our-nation}}

Though I may not be here with you, I urge you to answer the highest
calling of your heart and stand up for what you truly believe.

By John Lewis

Mr. Lewis, the civil rights leader who died on July 17, wrote this essay
shortly before his death, to be published upon the day of his
\href{https://www.nytimes.com/2020/07/30/us/john-lewis-live-funeral.html}{funeral}.

\begin{itemize}
\item
  July 30, 2020
\item
  \begin{itemize}
  \item
  \item
  \item
  \item
  \item
  \item
  \end{itemize}
\end{itemize}

\href{https://www.nytimes.com/es/2020/07/30/espanol/opinion/john-lewis-derechos-civiles.html}{Leer
en español}

\textbf{W}hile my time here has now come to an end, I want you to know
that in the last days and hours of my life you inspired me. You filled
me with hope about the next chapter of the great American story when you
used your power to make a difference in our society. Millions of people
motivated simply by human compassion laid down the burdens of division.
Around the country and the world you set aside race, class, age,
language and nationality to demand respect for human dignity.

That is why I had to visit Black Lives Matter Plaza in Washington,
though I was admitted to the hospital the following day. I just had to
see and feel it for myself that, after many years of silent witness, the
truth is still marching on.

Emmett Till was my George Floyd. He was my Rayshard Brooks, Sandra Bland
and Breonna Taylor. He was 14 when he was killed, and I was only 15
years old at the time. I will never ever forget the moment when it
became so clear that he could easily have been me. In those days, fear
constrained us like an imaginary prison, and troubling thoughts of
potential brutality committed for no understandable reason were the
bars.

\hypertarget{listen-to-this-op-ed}{%
\subsubsection{Listen to This Op-Ed}\label{listen-to-this-op-ed}}

Audio Recording by Audm

Though I was surrounded by two loving parents, plenty of brothers,
sisters and cousins, their love could not protect me from the unholy
oppression waiting just outside that family circle. Unchecked,
unrestrained violence and government-sanctioned terror had the power to
turn a simple stroll to the store for some Skittles or an innocent
morning jog down a lonesome country road into a nightmare. If we are to
survive as one unified nation, we must discover what so readily takes
root in our hearts that could rob Mother Emanuel Church in South
Carolina of her brightest and best, shoot unwitting concertgoers in Las
Vegas and choke to death the hopes and dreams of a gifted violinist like
Elijah McClain.

Like so many young people today, I was searching for a way out, or some
might say a way in, and then I heard the voice of Dr. Martin Luther King
Jr. on an old radio. He was talking about the philosophy and discipline
of nonviolence. He said we are all complicit when we tolerate injustice.
He said it is not enough to say it will get better by and by. He said
each of us has a moral obligation to stand up, speak up and speak out.
When you see something that is not right, you must say something. You
must do something. Democracy is not a state. It is an act, and each
generation must do its part to help build what we called the Beloved
Community, a nation and world society at peace with itself.

Ordinary people with extraordinary vision can redeem the soul of America
by getting in what I call good trouble, necessary trouble. Voting and
participating in the democratic process are key. The vote is the most
powerful nonviolent change agent you have in a democratic society. You
must use it because it is not guaranteed. You can lose it.

You must also study and learn the lessons of history because humanity
has been involved in this soul-wrenching, existential struggle for a
very long time. People on every continent have stood in your shoes,
through decades and centuries before you. The truth does not change, and
that is why the answers worked out long ago can help you find solutions
to the challenges of our time. Continue to build union between movements
stretching across the globe because we must put away our willingness to
profit from the exploitation of others.

Though I may not be here with you, I urge you to answer the highest
calling of your heart and stand up for what you truly believe. In my
life I have done all I can to demonstrate that the way of peace, the way
of love and nonviolence is the more excellent way. Now it is your turn
to let freedom ring.

When historians pick up their pens to write the story of the 21st
century, let them say that it was your generation who laid down the
heavy burdens of hate at last and that peace finally triumphed over
violence, aggression and war. So I say to you, walk with the wind,
brothers and sisters, and let the spirit of peace and the power of
everlasting love be your guide.

\begin{center}\rule{0.5\linewidth}{\linethickness}\end{center}

John Lewis, the civil rights leader and congressman who died on July 17,
wrote this essay shortly before his death.

\emph{The Times is committed to publishing}
\href{https://slack-redir.net/link?url=https\%3A\%2F\%2Fwww.nytimes.com\%2F2019\%2F01\%2F31\%2Fopinion\%2Fletters\%2Fletters-to-editor-new-york-times-women.html}{\emph{a
diversity of letters}} \emph{to the editor. We'd like to hear what you
think about this or any of our articles. Here are some}
\href{https://slack-redir.net/link?url=https\%3A\%2F\%2Fhelp.nytimes.com\%2Fhc\%2Fen-us\%2Farticles\%2F115014925288-How-to-submit-a-letter-to-the-editor}{\emph{tips}}\emph{.
And here's our email:}
\href{mailto:letters@nytimes.com}{\emph{letters@nytimes.com}}\emph{.}

\emph{Follow The New York Times Opinion section on}
\href{https://slack-redir.net/link?url=https\%3A\%2F\%2Fwww.facebook.com\%2Fnytopinion}{\emph{Facebook}}\emph{,}
\href{https://slack-redir.net/link?url=http\%3A\%2F\%2Ftwitter.com\%2FNYTOpinion}{\emph{Twitter
(@NYTopinion)}} \emph{and}
\href{https://slack-redir.net/link?url=https\%3A\%2F\%2Fwww.instagram.com\%2Fnytopinion\%2F}{\emph{Instagram}}\emph{.}

\emph{Photograph of John Lewis by David Deal/Redux}

Advertisement

\protect\hyperlink{after-bottom}{Continue reading the main story}

\hypertarget{site-index}{%
\subsection{Site Index}\label{site-index}}

\hypertarget{site-information-navigation}{%
\subsection{Site Information
Navigation}\label{site-information-navigation}}

\begin{itemize}
\tightlist
\item
  \href{https://help.nytimes.com/hc/en-us/articles/115014792127-Copyright-notice}{©~2020~The
  New York Times Company}
\end{itemize}

\begin{itemize}
\tightlist
\item
  \href{https://www.nytco.com/}{NYTCo}
\item
  \href{https://help.nytimes.com/hc/en-us/articles/115015385887-Contact-Us}{Contact
  Us}
\item
  \href{https://www.nytco.com/careers/}{Work with us}
\item
  \href{https://nytmediakit.com/}{Advertise}
\item
  \href{http://www.tbrandstudio.com/}{T Brand Studio}
\item
  \href{https://www.nytimes.com/privacy/cookie-policy\#how-do-i-manage-trackers}{Your
  Ad Choices}
\item
  \href{https://www.nytimes.com/privacy}{Privacy}
\item
  \href{https://help.nytimes.com/hc/en-us/articles/115014893428-Terms-of-service}{Terms
  of Service}
\item
  \href{https://help.nytimes.com/hc/en-us/articles/115014893968-Terms-of-sale}{Terms
  of Sale}
\item
  \href{https://spiderbites.nytimes.com}{Site Map}
\item
  \href{https://help.nytimes.com/hc/en-us}{Help}
\item
  \href{https://www.nytimes.com/subscription?campaignId=37WXW}{Subscriptions}
\end{itemize}
