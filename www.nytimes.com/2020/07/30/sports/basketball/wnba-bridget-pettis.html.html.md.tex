Sections

SEARCH

\protect\hyperlink{site-content}{Skip to
content}\protect\hyperlink{site-index}{Skip to site index}

\href{https://www.nytimes.com/section/sports/basketball}{Pro Basketball}

\href{https://myaccount.nytimes.com/auth/login?response_type=cookie\&client_id=vi}{}

\href{https://www.nytimes.com/section/todayspaper}{Today's Paper}

\href{/section/sports/basketball}{Pro Basketball}\textbar{}A W.N.B.A.
Coach Heard a Call to Feed the Hungry. She's Answering It.

\url{https://nyti.ms/3fl4JHq}

\begin{itemize}
\item
\item
\item
\item
\item
\end{itemize}

Advertisement

\protect\hyperlink{after-top}{Continue reading the main story}

Supported by

\protect\hyperlink{after-sponsor}{Continue reading the main story}

\hypertarget{a-wnba-coach-heard-a-call-to-feed-the-hungry-shes-answering-it}{%
\section{A W.N.B.A. Coach Heard a Call to Feed the Hungry. She's
Answering
It.}\label{a-wnba-coach-heard-a-call-to-feed-the-hungry-shes-answering-it}}

Bridget Pettis, a former W.N.B.A. player and coach, is sitting out this
season and focusing on teaching her community in Phoenix about gardening
and healthy eating.

\includegraphics{https://static01.nyt.com/images/2020/07/14/sports/14sideline-pettis/14sideline-pettis-articleLarge.jpg?quality=75\&auto=webp\&disable=upscale}

By \href{https://www.nytimes.com/by/gillian-r--brassil}{Gillian R.
Brassil}

\begin{itemize}
\item
  July 30, 2020
\item
  \begin{itemize}
  \item
  \item
  \item
  \item
  \item
  \end{itemize}
\end{itemize}

\emph{People throughout the sports world, from athletes to arena staff
members, tell The New York Times how their lives have changed during the
coronavirus pandemic.}

Bridget Pettis said she has always chosen her next move based on what's
in her heart, from the basketball court to the coach's chair to her new
nonprofit geared toward growing healthy food for people who don't have
access to it.

``If I see someone hungry, I am to feed them,'' she said.

Before she decided to pivot her focus to food education, Pettis was an
assistant coach for the Chicago Sky in the W.N.B.A. Her career began in
1997 with the Phoenix Mercury, which selected her seventh over all in
the league's inaugural draft. She played guard for the Mercury and the
Indiana Fever before she switched to coaching. Pettis opted out of
joining the Sky for the
\href{https://www.nytimes.com/2020/07/23/sports/basketball/wnba-season-preview.html}{W.N.B.A.
season}, which began last week, citing concerns about the health and
safety precautions in the so-called bubble at IMG Academy in Bradenton,
Fla., as one reason for stepping aside.

Now, during her time away from the W, she's taking what she learned as a
teammate, coach and player to the garden and educating members of her
community in Phoenix about how to grow their own healthy food through
her months-old nonprofit, \href{https://www.projectrootsaz.org/}{Project
Roots}.

\includegraphics{https://static01.nyt.com/images/2020/08/01/sports/14sideline-pettis-print/merlin_159174729_6ba579c4-54ab-40b0-9bba-477b1011533f-articleLarge.jpg?quality=75\&auto=webp\&disable=upscale}

\emph{This interview has been condensed and lightly edited for clarity.}

\textbf{Q: What brought you into basketball?}

\textbf{Pettis}: I grew up in the inner city of East Chicago, Ind.; in
the projects, the basketball court was the thing that attracted
everybody. And I remember seeing all the boys out there playing, and one
day I just went out there and wanted to try it for myself, and it was
just love at first sight.

Just the challenge of it. When I first went out there, all the boys were
saying, ``Girls don't do this.'' So that motivated me.

\textbf{What brought you to the W.N.B.A.?}

At the time when the W.N.B.A. came about, I was already a four-year
professional in Europe. But I had heard about the W.N.B.A. and was just
so excited. I got selected through the Phoenix team and kind of went
from there.

\textbf{Do you have any words of advice for W.N.B.A. players who are
going to compete in the bubble? And do you have any advice for how to
effectively promote social change while competing?}

Now that my W.N.B.A. sisters are there, I would push for them to do
whatever their hearts are holding for them on the platform that is there
for them. I love them and want for them all to be safe.

Now I'm a woman of faith, so I don't know how God is going to work
through that ultimate change. I just know that we could do something,
make a shift. And I see that there is a strong attention to planting
conscious seeds in people of the messages that are being said, socially,
for us to change. Maybe they're going to do different fund-raisers and
to use those resources to make change. But the attention that they get,
I think that's a good idea.

\textbf{Could you tell me more about your decision to take this season
off as well as what you're going to be doing with your nonprofit?}

My decision was, I felt it was time to move. When I feel like it's time
to move, I speak with my heart and I do that. I have encouraged all the
people around me, all my life, to do that.

And I'm going to focus now on my nonprofit that I feel like it can help.
And I call it --- this is my ``growmotion'' instead of promotion --- to
get it out there, to grow food and make the difference of providing
healthy food. Being a part of and making accessibility of healthier ways
to provide food; removing food deserts from areas where our people are,
where people who are struggling financially are.

In this world today, everybody should be able to have food.

\textbf{When did you start gardening?}

For years since I had my house, I had been growing little things. All of
my life I have always wanted plants and flowers around me. It starts
with a tomato: You take your chances on tomatoes, and when you see a
tomato grow, it just kind of went from there. I became a part of a
community garden about three years ago and that's when I connected in
that area and got so much benefit from it.

\textbf{What do you eat mainly? During your athletic career and now?}

For the most part, I eat a lot of fruits and vegetables. I just keep it
simple. I eat a lot of the things that I would grow back in Phoenix ---
zucchini, squash, onions, garlic --- all those vegetables and different
fruits. We planted fruit trees, so we eat a lot of things that come off
the trees. Every now and then I still mess with some fish, but for the
most part, I eat the things that come out of the ground.

\textbf{What kind of struggles have you seen in your community, in
Arizona, during the pandemic with food accessibility and where help
lies? The government? Supporting more nonprofits?}

I'm definitely a believer of people coming out and supporting
nonprofits. This is my first year of being a part of a nonprofit. And I
know my intention and I know the drive that it takes and the work that
it takes to do something like this.

I've seen the change, the impact that it has made --- very fast --- and
I just think that this is a good way for us to take more control of what
it is we would like to be done. And not worrying about putting all of
our eggs in one basket for a governmental change. That's just not where
my heart is. My heart is really in the people and in the care of
ourselves.

\textbf{What do you like to do in your free time when you're not in the
garden or on the court?}

Now I'm helping others play. I train a lot of younger players. I work
out with players. My nephews are playing basketball right now, so I've
been working with them. That's kind of what I do in my downtime.

I'm also looking for more community gardens. I go out and see where
people are starting to garden and I like to take pictures and see what's
going on in the world, as far as the interests in growing food.

But most of my days, it's mostly me getting information and enjoying
life right now. It's been 23 years of basketball and I'm just enjoying
the fruits of that labor a little bit and relaxing, and giving back to
basketball in a different way with my family and in the community out
here in Gary, Ind.

\textbf{What are you going to miss the most about the W.N.B.A.?}

The teams. In our locker room, the relationships that we built in those
moments as teams. I'll miss that union that we've always had. It was
always special. Every team was always special.

\textbf{Do you think you'll go back to the W.N.B.A. in the future?}

I don't really know. I just kind of go where I'm at. So this is where
I'm at now.

Advertisement

\protect\hyperlink{after-bottom}{Continue reading the main story}

\hypertarget{site-index}{%
\subsection{Site Index}\label{site-index}}

\hypertarget{site-information-navigation}{%
\subsection{Site Information
Navigation}\label{site-information-navigation}}

\begin{itemize}
\tightlist
\item
  \href{https://help.nytimes.com/hc/en-us/articles/115014792127-Copyright-notice}{©~2020~The
  New York Times Company}
\end{itemize}

\begin{itemize}
\tightlist
\item
  \href{https://www.nytco.com/}{NYTCo}
\item
  \href{https://help.nytimes.com/hc/en-us/articles/115015385887-Contact-Us}{Contact
  Us}
\item
  \href{https://www.nytco.com/careers/}{Work with us}
\item
  \href{https://nytmediakit.com/}{Advertise}
\item
  \href{http://www.tbrandstudio.com/}{T Brand Studio}
\item
  \href{https://www.nytimes.com/privacy/cookie-policy\#how-do-i-manage-trackers}{Your
  Ad Choices}
\item
  \href{https://www.nytimes.com/privacy}{Privacy}
\item
  \href{https://help.nytimes.com/hc/en-us/articles/115014893428-Terms-of-service}{Terms
  of Service}
\item
  \href{https://help.nytimes.com/hc/en-us/articles/115014893968-Terms-of-sale}{Terms
  of Sale}
\item
  \href{https://spiderbites.nytimes.com}{Site Map}
\item
  \href{https://help.nytimes.com/hc/en-us}{Help}
\item
  \href{https://www.nytimes.com/subscription?campaignId=37WXW}{Subscriptions}
\end{itemize}
