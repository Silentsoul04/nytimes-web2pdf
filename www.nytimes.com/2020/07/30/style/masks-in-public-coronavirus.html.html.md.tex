Sections

SEARCH

\protect\hyperlink{site-content}{Skip to
content}\protect\hyperlink{site-index}{Skip to site index}

\href{https://www.nytimes.com/section/style}{Style}

\href{https://myaccount.nytimes.com/auth/login?response_type=cookie\&client_id=vi}{}

\href{https://www.nytimes.com/section/todayspaper}{Today's Paper}

\href{/section/style}{Style}\textbar{}How Can I (Kindly) Tell People to
Wear Masks in Public?

\url{https://nyti.ms/3ghyTfY}

\begin{itemize}
\item
\item
\item
\item
\item
\item
\end{itemize}

\href{https://www.nytimes.com/spotlight/at-home?action=click\&pgtype=Article\&state=default\&region=TOP_BANNER\&context=at_home_menu}{At
Home}

\begin{itemize}
\tightlist
\item
  \href{https://www.nytimes.com/2020/07/28/books/time-for-a-literary-road-trip.html?action=click\&pgtype=Article\&state=default\&region=TOP_BANNER\&context=at_home_menu}{Take:
  A Literary Road Trip}
\item
  \href{https://www.nytimes.com/2020/07/29/magazine/bored-with-your-home-cooking-some-smoky-eggplant-will-fix-that.html?action=click\&pgtype=Article\&state=default\&region=TOP_BANNER\&context=at_home_menu}{Cook:
  Smoky Eggplant}
\item
  \href{https://www.nytimes.com/2020/07/27/travel/moose-michigan-isle-royale.html?action=click\&pgtype=Article\&state=default\&region=TOP_BANNER\&context=at_home_menu}{Look
  Out: For Moose}
\item
  \href{https://www.nytimes.com/interactive/2020/at-home/even-more-reporters-editors-diaries-lists-recommendations.html?action=click\&pgtype=Article\&state=default\&region=TOP_BANNER\&context=at_home_menu}{Explore:
  Reporters' Obsessions}
\end{itemize}

Advertisement

\protect\hyperlink{after-top}{Continue reading the main story}

Supported by

\protect\hyperlink{after-sponsor}{Continue reading the main story}

Social Q's

\hypertarget{how-can-i-kindly-tell-people-to-wear-masks-in-public}{%
\section{How Can I (Kindly) Tell People to Wear Masks in
Public?}\label{how-can-i-kindly-tell-people-to-wear-masks-in-public}}

Our advice columnist tested a few possible lines. Here's how that went.

By \href{https://www.nytimes.com/by/philip-galanes}{Philip Galanes}

\begin{itemize}
\item
  July 30, 2020
\item
  \begin{itemize}
  \item
  \item
  \item
  \item
  \item
  \item
  \end{itemize}
\end{itemize}

\emph{I live in a small city that's in the process of reopening as new
Covid-19 cases remain relatively low. This is wonderful! Still, it
frightens me to see people in stores and on buses, for instance, who are
not wearing face masks as required by my state. I would never subject
another person to the virus, even accidentally, if I didn't know I was
infected or have any symptoms! Do you have a script for asking strangers
to put on their masks?}

M.M.

I wanted to road-test a script before suggesting one, so I tried my luck
with three strangers who were violating New York's mandate to wear masks
indoors and when social distancing is difficult outdoors. I struck out
every time! This surprised me. It also made me feel angry and helpless,
even though most people I saw were wearing masks.

The first person I asked was a fellow shopper at the market. ``Can you
put on a mask, please?'' I said, extra friendly. He ignored me, so I
asked again. ``I'm good,'' he said. This annoyed me. ``You're good
because I'm wearing a mask,'' I told him. ``Why not return the favor?''
He glared, and I moved on. There is no upside in a
\href{https://www.nytimes.com/2020/06/30/style/mask-america-freedom-coronavirus.html}{screaming
match with a person without a mask}. Hello, viral droplets!

The next day, I was in the dog park, where playful dogs can often bring
owners within six feet of each other. I pulled down my mask briefly, to
smile and ask a woman nearby to put on a mask. ``We're outdoors,'' she
said. ``But our dogs bring us close,'' I replied. I even offered her a
mask. She suggested I leave if I didn't feel safe.

My final attempt was at the post office, where I tried a firmer tone
with a man who pointed to a mask dangling beneath his chin, as if that
should appease me. ``Please pull it up,'' I said. ``It's the law.'' But
he refused. ``I won't be long,'' he said.

Now, you may have better luck than I did. But I'm not optimistic. No one
screamed at me, like in the nasty videos I see online, but no one
cooperated either. So, ask away.

But it may be more productive to report mask-less patrons to store
managers or bus drivers when possible. (They have more authority to
enforce the rules.) Or save your energy for trying to keep your distance
from selfish people without masks.

Image

Credit...Christoph Niemann

\hypertarget{that-was-an-heirloom}{%
\subsection{That Was an Heirloom!}\label{that-was-an-heirloom}}

\emph{My grandson dated a woman for five years. After three years, I
asked him if it was serious. He said it was. For Christmas that year, I
gave her an aquamarine necklace that belonged to my mother. I continued
giving her heirloom pieces for the next two holidays. Then they broke
up, to my surprise. I asked my grandson to retrieve our family
heirlooms, but his ex-girlfriend refused to return them. What can I do?}

NORA

I'm sorry you mistook a ``serious'' relationship for one that was
permanent (or might yield great-granddaughters). That doesn't always
happen, as you now know. The ex is under no obligation, other than a
sympathetic one, to return gifts that were freely given to her. Givers
retain no ownership in gifts.

But perhaps a call from you to the ex about the sentimental value of the
jewelry may help? When something similar happened in my family, my
mother agreed to buy back the heirlooms. (She was furious about it, but
she did it.) Is that possible? And next time, think twice before handing
over a tiara you intend to take back if circumstances change.

\hypertarget{no-longer-estranged-but-still-distant}{%
\subsection{No Longer Estranged, but Still
Distant}\label{no-longer-estranged-but-still-distant}}

\emph{My brother and I were estranged for 15 years. The pandemic helped
us break through our silence. Now, he has invited me to his 60th
birthday party in September, which would require a six-hour flight.
Obviously, I'm not getting on a plane now. How can I preserve our
relationship? (He's sensitive.)}

SISTER

I'm sorry the pandemic threw a wrench into your reconciliation with your
brother. Call him and say: ``I'm so happy we're talking again! I missed
you. If there was anyone I would get on a six-hour flight for, it's you.
But I can't do that safely now. I hope you'll understand.'' Then send
him a thoughtful gift, cross your fingers and keep talking. It's not as
if you have a sensible alternative, right?

\hypertarget{sign-our-petition}{%
\subsection{Sign Our Petition?}\label{sign-our-petition}}

\emph{For the last four years, since I was 9, I went to summer camp with
my older brother in August. We love it. This year, after making many
rules about masks and social distancing, our camp announced it would
reopen. But my parents aren't letting us go. They don't think it's safe.
Can we add your name to the list of people protesting our parents'
decision?}

AUSTIN

Permission denied, camper! I'm sorry you're disappointed. But I suspect
the low adult-to-kid ratio at camp would put too much pressure on you to
behave responsibly all the time. (And if my math is correct, you're only
13 or 14.) Instead, use the leverage of your parents' guilt to persuade
them to buy you some nice swag or adopt a dog.

\begin{center}\rule{0.5\linewidth}{\linethickness}\end{center}

For help with your awkward situation, send a question to
\href{mailto:SocialQ@nytimes.com}{\nolinkurl{SocialQ@nytimes.com}}, to
Philip Galanes on Facebook or
\href{https://twitter.com/SocialQPhilip}{@SocialQPhilip} on Twitter.

Advertisement

\protect\hyperlink{after-bottom}{Continue reading the main story}

\hypertarget{site-index}{%
\subsection{Site Index}\label{site-index}}

\hypertarget{site-information-navigation}{%
\subsection{Site Information
Navigation}\label{site-information-navigation}}

\begin{itemize}
\tightlist
\item
  \href{https://help.nytimes.com/hc/en-us/articles/115014792127-Copyright-notice}{©~2020~The
  New York Times Company}
\end{itemize}

\begin{itemize}
\tightlist
\item
  \href{https://www.nytco.com/}{NYTCo}
\item
  \href{https://help.nytimes.com/hc/en-us/articles/115015385887-Contact-Us}{Contact
  Us}
\item
  \href{https://www.nytco.com/careers/}{Work with us}
\item
  \href{https://nytmediakit.com/}{Advertise}
\item
  \href{http://www.tbrandstudio.com/}{T Brand Studio}
\item
  \href{https://www.nytimes.com/privacy/cookie-policy\#how-do-i-manage-trackers}{Your
  Ad Choices}
\item
  \href{https://www.nytimes.com/privacy}{Privacy}
\item
  \href{https://help.nytimes.com/hc/en-us/articles/115014893428-Terms-of-service}{Terms
  of Service}
\item
  \href{https://help.nytimes.com/hc/en-us/articles/115014893968-Terms-of-sale}{Terms
  of Sale}
\item
  \href{https://spiderbites.nytimes.com}{Site Map}
\item
  \href{https://help.nytimes.com/hc/en-us}{Help}
\item
  \href{https://www.nytimes.com/subscription?campaignId=37WXW}{Subscriptions}
\end{itemize}
