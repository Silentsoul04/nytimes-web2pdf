Sections

SEARCH

\protect\hyperlink{site-content}{Skip to
content}\protect\hyperlink{site-index}{Skip to site index}

\href{https://myaccount.nytimes.com/auth/login?response_type=cookie\&client_id=vi}{}

\href{https://www.nytimes.com/section/todayspaper}{Today's Paper}

\href{/section/upshot}{The Upshot}\textbar{}As Covid Has Become a
Red-State Problem, Too, Have Attitudes Changed?

\url{https://nyti.ms/33auFTw}

\begin{itemize}
\item
\item
\item
\item
\item
\item
\end{itemize}

\href{https://www.nytimes.com/news-event/coronavirus?action=click\&pgtype=Article\&state=default\&region=TOP_BANNER\&context=storylines_menu}{The
Coronavirus Outbreak}

\begin{itemize}
\tightlist
\item
  live\href{https://www.nytimes.com/2020/08/01/world/coronavirus-covid-19.html?action=click\&pgtype=Article\&state=default\&region=TOP_BANNER\&context=storylines_menu}{Latest
  Updates}
\item
  \href{https://www.nytimes.com/interactive/2020/us/coronavirus-us-cases.html?action=click\&pgtype=Article\&state=default\&region=TOP_BANNER\&context=storylines_menu}{Maps
  and Cases}
\item
  \href{https://www.nytimes.com/interactive/2020/science/coronavirus-vaccine-tracker.html?action=click\&pgtype=Article\&state=default\&region=TOP_BANNER\&context=storylines_menu}{Vaccine
  Tracker}
\item
  \href{https://www.nytimes.com/interactive/2020/07/29/us/schools-reopening-coronavirus.html?action=click\&pgtype=Article\&state=default\&region=TOP_BANNER\&context=storylines_menu}{What
  School May Look Like}
\item
  \href{https://www.nytimes.com/live/2020/07/31/business/stock-market-today-coronavirus?action=click\&pgtype=Article\&state=default\&region=TOP_BANNER\&context=storylines_menu}{Economy}
\end{itemize}

Advertisement

\protect\hyperlink{after-top}{Continue reading the main story}

Upshot

Supported by

\protect\hyperlink{after-sponsor}{Continue reading the main story}

\hypertarget{as-covid-has-become-a-red-state-problem-too-have-attitudes-changed}{%
\section{As Covid Has Become a Red-State Problem, Too, Have Attitudes
Changed?}\label{as-covid-has-become-a-red-state-problem-too-have-attitudes-changed}}

There's still a persistent partisan gap in the level of concern and in
mask wearing.

\href{https://www.nytimes.com/by/robert-gebeloff}{\includegraphics{https://static01.nyt.com/images/2018/06/12/multimedia/author-robert-gebeloff/author-robert-gebeloff-thumbLarge-v2.png}}

By \href{https://www.nytimes.com/by/robert-gebeloff}{Robert Gebeloff}

\begin{itemize}
\item
  July 30, 2020
\item
  \begin{itemize}
  \item
  \item
  \item
  \item
  \item
  \item
  \end{itemize}
\end{itemize}

With the coronavirus spreading into red America, policymakers there have
been forced to change course.

Some reopening plans have been scuttled; Republican leaders in
Washington have
\href{https://www.nytimes.com/2020/07/19/us/politics/republicans-contradict-trump-coronavirus.html}{broken
ranks}with President Trump on response policy; and the president himself
eventually declared that Americans should wear protective masks.

At the grass-roots level, however, it's not clear if the partisan
differences in how people view the pandemic have changed much at all.

There is some evidence of a narrowing on some fronts, but other survey
data suggests that Americans disagree about the severity of the crisis
as much today as they did two months ago, when people in Democratic
counties were taking the
\href{https://www.nytimes.com/2020/05/25/us/politics/coronavirus-red-blue-states.html}{brunt
of the damage}.

The New York Times recently published a detailed
\href{https://www.nytimes.com/interactive/2020/07/17/upshot/coronavirus-face-mask-map.html}{map}
on geographic variations in the use of face coverings, and here also
there is a partisan difference. People who identify as Democrats more
consistently report they use face coverings, regardless of whether they
live in a county with a high case rate, while for Republicans,
self-reported mask use goes up where the crisis is worse.

If the presence of the virus is driving these changes, the infection
data shows why.

Through late May, counties that supported President Trump in 2016
accounted for just 26 percent of reported cases and 21 percent of
deaths, despite making up 45 percent of the nation's population.

Over the past two months, however, Republican-leaning counties have
accounted for 43 percent of new cases and a third of deaths. Of the 100
counties that have seen the most case growth per capita over the past
two months, 71 of them backed Mr. Trump in the last election.

Moreover, at the state level, where policy is made, the relationship
also shifted. Survey data shows Republican governors are more likely to
be
\href{https://fivethirtyeight.com/features/americans-increasingly-dislike-how-republican-governors-are-handling-the-coronavirus-outbreak/}{taking
criticism.}

``New cases per capita were higher in redder states than in bluer states
in June and July, even after controlling for a wide range of
demographic, geographic and weather variables,'' said Jed Kolko, an
economist and Upshot contributor. ``Deaths per capita, too, were higher
in redder states than in bluer states in July.''

Still, it's less clear how these changes affected perceptions on an
individual level.

``Perhaps people who are skeptical about mask-wearing in the abstract
might change their minds when the virus is an immediate threat,'' he
said.

Matthew Gentzkow, a Stanford economist who studies partisan behavior,
led an
\href{http://web.stanford.edu/~gentzkow/research/social_distancing.pdf}{early
study} based on cellphone records that showed ``that areas with more
Republicans engage in less social distancing, controlling for other
factors including public policies, population density, and local Covid
cases and deaths.''

Two months later, he said, there are still differences, but not as
large.

``They're strongly significant in a statistical sense but modest in
magnitude,'' he said. ``People do appear to be behaving differently by
party, but it's not as dramatic as all Republicans are going to the
beach and having parties while Democrats stay home.''

It's still not hard to
\href{https://www.washingtontimes.com/news/2020/jul/15/rush-limbaugh-americans-should-adapt-to-coronaviru/}{find}
pandemic \href{https://www.youtube.com/watch?v=Lhx_XBUTOI0}{naysayers}
in conservative-leaning media. And several surveys show that on broader
questions, a large difference remains. Even as the virus spreads,
Republicans and Democrats differ in their concerns about the health
impacts.

Polling data from Civiqs shows Democrats are nearly three times as
likely to be concerned about a virus outbreak in their area. Republican
response to this question budged only slightly even as virus outbreaks
actually multiplied in Republican counties in June and July.

\href{https://www.nytimes.com/news-event/coronavirus?action=click\&pgtype=Article\&state=default\&region=MAIN_CONTENT_3\&context=storylines_faq}{}

\hypertarget{the-coronavirus-outbreak-}{%
\subsubsection{The Coronavirus Outbreak
›}\label{the-coronavirus-outbreak-}}

\hypertarget{frequently-asked-questions}{%
\paragraph{Frequently Asked
Questions}\label{frequently-asked-questions}}

Updated July 27, 2020

\begin{itemize}
\item ~
  \hypertarget{should-i-refinance-my-mortgage}{%
  \paragraph{Should I refinance my
  mortgage?}\label{should-i-refinance-my-mortgage}}

  \begin{itemize}
  \tightlist
  \item
    \href{https://www.nytimes.com/article/coronavirus-money-unemployment.html?action=click\&pgtype=Article\&state=default\&region=MAIN_CONTENT_3\&context=storylines_faq}{It
    could be a good idea,} because mortgage rates have
    \href{https://www.nytimes.com/2020/07/16/business/mortgage-rates-below-3-percent.html?action=click\&pgtype=Article\&state=default\&region=MAIN_CONTENT_3\&context=storylines_faq}{never
    been lower.} Refinancing requests have pushed mortgage applications
    to some of the highest levels since 2008, so be prepared to get in
    line. But defaults are also up, so if you're thinking about buying a
    home, be aware that some lenders have tightened their standards.
  \end{itemize}
\item ~
  \hypertarget{what-is-school-going-to-look-like-in-september}{%
  \paragraph{What is school going to look like in
  September?}\label{what-is-school-going-to-look-like-in-september}}

  \begin{itemize}
  \tightlist
  \item
    It is unlikely that many schools will return to a normal schedule
    this fall, requiring the grind of
    \href{https://www.nytimes.com/2020/06/05/us/coronavirus-education-lost-learning.html?action=click\&pgtype=Article\&state=default\&region=MAIN_CONTENT_3\&context=storylines_faq}{online
    learning},
    \href{https://www.nytimes.com/2020/05/29/us/coronavirus-child-care-centers.html?action=click\&pgtype=Article\&state=default\&region=MAIN_CONTENT_3\&context=storylines_faq}{makeshift
    child care} and
    \href{https://www.nytimes.com/2020/06/03/business/economy/coronavirus-working-women.html?action=click\&pgtype=Article\&state=default\&region=MAIN_CONTENT_3\&context=storylines_faq}{stunted
    workdays} to continue. California's two largest public school
    districts --- Los Angeles and San Diego --- said on July 13, that
    \href{https://www.nytimes.com/2020/07/13/us/lausd-san-diego-school-reopening.html?action=click\&pgtype=Article\&state=default\&region=MAIN_CONTENT_3\&context=storylines_faq}{instruction
    will be remote-only in the fall}, citing concerns that surging
    coronavirus infections in their areas pose too dire a risk for
    students and teachers. Together, the two districts enroll some
    825,000 students. They are the largest in the country so far to
    abandon plans for even a partial physical return to classrooms when
    they reopen in August. For other districts, the solution won't be an
    all-or-nothing approach.
    \href{https://bioethics.jhu.edu/research-and-outreach/projects/eschool-initiative/school-policy-tracker/}{Many
    systems}, including the nation's largest, New York City, are
    devising
    \href{https://www.nytimes.com/2020/06/26/us/coronavirus-schools-reopen-fall.html?action=click\&pgtype=Article\&state=default\&region=MAIN_CONTENT_3\&context=storylines_faq}{hybrid
    plans} that involve spending some days in classrooms and other days
    online. There's no national policy on this yet, so check with your
    municipal school system regularly to see what is happening in your
    community.
  \end{itemize}
\item ~
  \hypertarget{is-the-coronavirus-airborne}{%
  \paragraph{Is the coronavirus
  airborne?}\label{is-the-coronavirus-airborne}}

  \begin{itemize}
  \tightlist
  \item
    The coronavirus
    \href{https://www.nytimes.com/2020/07/04/health/239-experts-with-one-big-claim-the-coronavirus-is-airborne.html?action=click\&pgtype=Article\&state=default\&region=MAIN_CONTENT_3\&context=storylines_faq}{can
    stay aloft for hours in tiny droplets in stagnant air}, infecting
    people as they inhale, mounting scientific evidence suggests. This
    risk is highest in crowded indoor spaces with poor ventilation, and
    may help explain super-spreading events reported in meatpacking
    plants, churches and restaurants.
    \href{https://www.nytimes.com/2020/07/06/health/coronavirus-airborne-aerosols.html?action=click\&pgtype=Article\&state=default\&region=MAIN_CONTENT_3\&context=storylines_faq}{It's
    unclear how often the virus is spread} via these tiny droplets, or
    aerosols, compared with larger droplets that are expelled when a
    sick person coughs or sneezes, or transmitted through contact with
    contaminated surfaces, said Linsey Marr, an aerosol expert at
    Virginia Tech. Aerosols are released even when a person without
    symptoms exhales, talks or sings, according to Dr. Marr and more
    than 200 other experts, who
    \href{https://academic.oup.com/cid/article/doi/10.1093/cid/ciaa939/5867798}{have
    outlined the evidence in an open letter to the World Health
    Organization}.
  \end{itemize}
\item ~
  \hypertarget{what-are-the-symptoms-of-coronavirus}{%
  \paragraph{What are the symptoms of
  coronavirus?}\label{what-are-the-symptoms-of-coronavirus}}

  \begin{itemize}
  \tightlist
  \item
    Common symptoms
    \href{https://www.nytimes.com/article/symptoms-coronavirus.html?action=click\&pgtype=Article\&state=default\&region=MAIN_CONTENT_3\&context=storylines_faq}{include
    fever, a dry cough, fatigue and difficulty breathing or shortness of
    breath.} Some of these symptoms overlap with those of the flu,
    making detection difficult, but runny noses and stuffy sinuses are
    less common.
    \href{https://www.nytimes.com/2020/04/27/health/coronavirus-symptoms-cdc.html?action=click\&pgtype=Article\&state=default\&region=MAIN_CONTENT_3\&context=storylines_faq}{The
    C.D.C. has also} added chills, muscle pain, sore throat, headache
    and a new loss of the sense of taste or smell as symptoms to look
    out for. Most people fall ill five to seven days after exposure, but
    symptoms may appear in as few as two days or as many as 14 days.
  \end{itemize}
\item ~
  \hypertarget{does-asymptomatic-transmission-of-covid-19-happen}{%
  \paragraph{Does asymptomatic transmission of Covid-19
  happen?}\label{does-asymptomatic-transmission-of-covid-19-happen}}

  \begin{itemize}
  \tightlist
  \item
    So far, the evidence seems to show it does. A widely cited
    \href{https://www.nature.com/articles/s41591-020-0869-5}{paper}
    published in April suggests that people are most infectious about
    two days before the onset of coronavirus symptoms and estimated that
    44 percent of new infections were a result of transmission from
    people who were not yet showing symptoms. Recently, a top expert at
    the World Health Organization stated that transmission of the
    coronavirus by people who did not have symptoms was ``very rare,''
    \href{https://www.nytimes.com/2020/06/09/world/coronavirus-updates.html?action=click\&pgtype=Article\&state=default\&region=MAIN_CONTENT_3\&context=storylines_faq\#link-1f302e21}{but
    she later walked back that statement.}
  \end{itemize}
\end{itemize}

A
recent\href{https://www.pewresearch.org/fact-tank/2020/07/22/republicans-remain-far-less-likely-than-democrats-to-view-covid-19-as-a-major-threat-to-public-health/}{Pew
survey} found that Republicans and Democrats agreed on one aspect of the
pandemic --- that it has a severe impact on the economy. But when asked
whether the virus posed a major threat to their personal health, or the
health of their community, Democrats were twice as likely to say yes.

The authors concluded that the partisan divide over all was about the
same as it was in May, before cases and deaths surged into red counties.

The extent to which firsthand experience drives these ideological
differences among Americans, and how much the differences in behavior
contribute to the virus's spread, is hard to disentangle with much
precision.

But it's more clear that shifts at the policy level have not come out of
nowhere. They've aligned with the changing geography of the pandemic.

\begin{center}\rule{0.5\linewidth}{\linethickness}\end{center}

\emph{Josh Katz contributed reporting.}

Advertisement

\protect\hyperlink{after-bottom}{Continue reading the main story}

\hypertarget{site-index}{%
\subsection{Site Index}\label{site-index}}

\hypertarget{site-information-navigation}{%
\subsection{Site Information
Navigation}\label{site-information-navigation}}

\begin{itemize}
\tightlist
\item
  \href{https://help.nytimes.com/hc/en-us/articles/115014792127-Copyright-notice}{©~2020~The
  New York Times Company}
\end{itemize}

\begin{itemize}
\tightlist
\item
  \href{https://www.nytco.com/}{NYTCo}
\item
  \href{https://help.nytimes.com/hc/en-us/articles/115015385887-Contact-Us}{Contact
  Us}
\item
  \href{https://www.nytco.com/careers/}{Work with us}
\item
  \href{https://nytmediakit.com/}{Advertise}
\item
  \href{http://www.tbrandstudio.com/}{T Brand Studio}
\item
  \href{https://www.nytimes.com/privacy/cookie-policy\#how-do-i-manage-trackers}{Your
  Ad Choices}
\item
  \href{https://www.nytimes.com/privacy}{Privacy}
\item
  \href{https://help.nytimes.com/hc/en-us/articles/115014893428-Terms-of-service}{Terms
  of Service}
\item
  \href{https://help.nytimes.com/hc/en-us/articles/115014893968-Terms-of-sale}{Terms
  of Sale}
\item
  \href{https://spiderbites.nytimes.com}{Site Map}
\item
  \href{https://help.nytimes.com/hc/en-us}{Help}
\item
  \href{https://www.nytimes.com/subscription?campaignId=37WXW}{Subscriptions}
\end{itemize}
