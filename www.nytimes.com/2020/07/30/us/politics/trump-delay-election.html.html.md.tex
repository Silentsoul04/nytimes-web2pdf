Sections

SEARCH

\protect\hyperlink{site-content}{Skip to
content}\protect\hyperlink{site-index}{Skip to site index}

\href{https://www.nytimes.com/section/politics}{Politics}

\href{https://myaccount.nytimes.com/auth/login?response_type=cookie\&client_id=vi}{}

\href{https://www.nytimes.com/section/todayspaper}{Today's Paper}

\href{/section/politics}{Politics}\textbar{}Trump Attacks an Election He
Is at Risk of Losing

\url{https://nyti.ms/310PE8F}

\begin{itemize}
\item
\item
\item
\item
\item
\end{itemize}

\begin{itemize}
\item
  \href{https://www.nytimes.com/2020/07/31/us/elections/biden-vs-trump.html?action=click\&pgtype=Article\&state=default\&region=TOP_BANNER\&context=storylines_menu}{Election
  Updates}
\item
  \href{https://www.nytimes.com/article/biden-vice-president-2020.html?action=click\&pgtype=Article\&state=default\&region=TOP_BANNER\&context=storylines_menu}{Biden's
  V.P. Search}
\item
  \href{https://www.nytimes.com/interactive/2020/07/24/us/politics/trump-biden-campaign-donors.html?action=click\&pgtype=Article\&state=default\&region=TOP_BANNER\&context=storylines_menu}{Map
  of Donations}
\item
  \href{https://www.nytimes.com/interactive/2020/us/elections/delegate-count-primary-results.html?action=click\&pgtype=Article\&state=default\&region=TOP_BANNER\&context=storylines_menu}{Delegate
  Count}
\item
  \href{https://www.nytimes.com/interactive/2019/us/politics/2020-presidential-candidates.html?action=click\&pgtype=Article\&state=default\&region=TOP_BANNER\&context=storylines_menu}{The
  Candidates}
\item
  \href{https://www.nytimes.com/newsletters/politics?action=click\&pgtype=Article\&state=default\&region=TOP_BANNER\&context=storylines_menu}{Politics
  Newsletter}
\end{itemize}

Advertisement

\protect\hyperlink{after-top}{Continue reading the main story}

Supported by

\protect\hyperlink{after-sponsor}{Continue reading the main story}

news analysis

\hypertarget{trump-attacks-an-election-he-is-at-risk-of-losing}{%
\section{Trump Attacks an Election He Is at Risk of
Losing}\label{trump-attacks-an-election-he-is-at-risk-of-losing}}

Mr. Trump has become a heckler in his own government, failing to marshal
leaders in Washington to form a robust response to the health and
economic crises. Instead, he is raising doubts about holding the
election on time.

\includegraphics{https://static01.nyt.com/images/2020/07/30/us/politics/30TRUMP-ANALYSIS/30TRUMP-ANALYSIS-articleLarge-v2.jpg?quality=75\&auto=webp\&disable=upscale}

\href{https://www.nytimes.com/by/alexander-burns}{\includegraphics{https://static01.nyt.com/images/2018/09/25/multimedia/author-alexander-burns/author-alexander-burns-thumbLarge-v2.png}}

By \href{https://www.nytimes.com/by/alexander-burns}{Alexander Burns}

\begin{itemize}
\item
  July 30, 2020
\item
  \begin{itemize}
  \item
  \item
  \item
  \item
  \item
  \end{itemize}
\end{itemize}

For several years, it has been the stuff of his opponents' nightmares:
that President Trump, facing the prospect of defeat in the 2020
election, would declare by presidential edict that the vote had been
delayed or canceled.

Never mind that no president has that power, that the timing of federal
elections has been fixed since the 19th century and that the
Constitution sets an immovable expiration date on the president's term.
Given Mr. Trump's contempt for the legal limits on his office and his
oft-expressed admiration for foreign dictators, it hardly seemed
far-fetched to imagine he would at least attempt the gambit.

But when the moment came on Thursday, with Mr. Trump suggesting for the
first time that
\href{https://www.nytimes.com/2020/07/31/us/politics/trump-tweet-democracy.html}{the
election could be delayed}, his proposal appeared as impotent as it was
predictable --- less a stunning assertion of his authority than yet
another lament that his political prospects have dimmed amid a global
public-health crisis. Indeed, his
\href{https://www.nytimes.com/2020/07/31/us/politics/trump-tweet-democracy.html}{comments
on Twitter} came shortly after the Commerce Department reported that
American economic output contracted last quarter at the fastest rate in
recorded history, underscoring one of Mr. Trump's most severe
vulnerabilities as he pursues a second term.

Far from a strongman, Mr. Trump has lately become a heckler in his own
government, promoting medical conspiracy theories on social media,
playing no constructive role in either the management of the coronavirus
pandemic or the negotiation of an economic rescue plan in Congress ---
and complaining endlessly about the unfairness of it all.

``It will be a great embarrassment to the USA,'' Mr. Trump tweeted of
the election, asserting without evidence that mail-in voting would lead
to fraud. ``Delay the Election until people can properly, securely and
safely vote???''

The most powerful leaders in Congress immediately shot down the idea of
moving the election, including the top figures in Mr. Trump's own party.

``Never in the history of the country, through wars, depressions, and
the Civil War have we ever not had a federally scheduled election on
time, and we'll find a way to do that again this Nov. 3,'' Mitch
McConnell, the Senate Majority Leader, said
\href{https://twitter.com/MaxWinitz/status/1288875891985129480?s=20}{in
an interview with WNKY television} in Kentucky. ``We'll cope with
whatever the situation is and have the election on Nov. 3 as already
scheduled.''

\hypertarget{latest-updates-2020-election}{%
\section{\texorpdfstring{\href{https://www.nytimes.com/2020/07/31/us/elections/biden-vs-trump.html?action=click\&pgtype=Article\&state=default\&region=MAIN_CONTENT_1\&context=storylines_live_updates}{Latest
Updates: 2020
Election}}{Latest Updates: 2020 Election}}\label{latest-updates-2020-election}}

Updated 2020-08-01T01:26:45.732Z

\begin{itemize}
\tightlist
\item
  \href{https://www.nytimes.com/2020/07/31/us/elections/biden-vs-trump.html?action=click\&pgtype=Article\&state=default\&region=MAIN_CONTENT_1\&context=storylines_live_updates\#link-29fdff45}{Kamala
  Harris, a top vice-presidential contender, confronts double
  standards.}
\item
  \href{https://www.nytimes.com/2020/07/31/us/elections/biden-vs-trump.html?action=click\&pgtype=Article\&state=default\&region=MAIN_CONTENT_1\&context=storylines_live_updates\#link-13ec3d9c}{Karen
  Bass and Susan Rice are rising on Biden's vice-presidential
  shortlist.}
\item
  \href{https://www.nytimes.com/2020/07/31/us/elections/biden-vs-trump.html?action=click\&pgtype=Article\&state=default\&region=MAIN_CONTENT_1\&context=storylines_live_updates\#link-49e9a016}{Trump
  says Russian bounties to kill U.S. troops `never took place.'}
\end{itemize}

\href{https://www.nytimes.com/2020/07/31/us/elections/biden-vs-trump.html?action=click\&pgtype=Article\&state=default\&region=MAIN_CONTENT_1\&context=storylines_live_updates}{See
more updates}

Mr. Trump's tweet about delaying the election marked a phase of his
presidency defined not by the accumulation of executive power, but by an
abdication of presidential leadership on a national emergency.

Faced with the kind of economic wreckage besieging millions of
Americans, any other president would be shoulder-deep in the process of
marshaling his top lieutenants and leaders in Congress to form a robust
government response. Instead, Mr. Trump has been absent this week from
economic-relief talks, even as a crucial unemployment benefit is poised
to expire and the Federal Reserve chairman, Jerome H. Powell, warned
publicly that the country's recovery is lagging.

And any other president confronted with a virulent viral outbreak across
huge regions of the country would be at least trying to deliver a clear
and consistent message about public safety. Instead, Mr. Trump has
continued to promote a drug of no proven efficacy, hydroxychloroquine,
as a potential miracle cure, and to demand that schools and businesses
reopen quickly --- even as he has also claimed that it might be
impossible to hold a safe election.

William F. Weld, the former governor of Massachusetts who mounted a
largely symbolic challenge to Mr. Trump in the Republican primaries this
year, said on Thursday that the president's tweet was a sign that Mr.
Trump was panicked and unmoored. Though Mr. Weld has argued for years
that Mr. Trump had dictatorial impulses, he said Thursday that the
election-delay idea was ``not a legitimate threat.''

``So many dead and the economy in free fall --- and what's his reaction?
Delay the election,'' Mr. Weld said. ``It's a sign of a mind that's
having a great deal of difficulty coming to terms with reality.''

Mr. Trump has attacked the legitimacy of American elections before,
including the one in 2016 that made him president. Even after winning
the Electoral College that year, Mr. Trump cast doubt on the popular
vote and postulated baselessly that Hillary Clinton's substantial lead
in that metric had been tainted by illegal voting.

With that as precedent, there has never been much doubt --- certainly
among his opponents --- that Mr. Trump would attempt to undercut the
election if it appeared likely he would lose it. While Mr. Trump does
not have the power to shift the date of the election, there is ample
concern among Democrats that his appointees in Washington or his allies
in state governments could make a large-scale effort to snarl the
process of voting.

Given the extreme nature of Mr. Trump's suggestion, there was an odd
familiarity to the response it garnered from political leaders in both
parties. There was no immediate call to the barricades, or renewed push
from Democrats for presidential impeachment. Opposition leaders
expressed outrage, but most agreed, in public and private, that Mr.
Trump's outburst should be treated as a distress call rather than a real
statement of his governing intentions.

House Speaker Nancy Pelosi, the most powerful Democrat in government,
replied to Mr. Trump's tweet simply by posting on Twitter the language
from the Constitution stating that Congress, not the president, sets the
date of national elections. Representative Zoe Lofgren of California, a
Democrat who chairs the congressional committee that oversees elections,
suggested in no uncertain terms that Mr. Trump's tweet was another
symptom of his inability to master the coronavirus.

``Only Congress can change the date of our elections,'' Ms. Lofgren
said, ``and under no circumstances will we consider doing so to
accommodate the President's inept and haphazard response to the
coronavirus pandemic, or give credence to the lies and misinformation he
spreads regarding the manner in which Americans can safely and securely
cast their ballots.''

Republicans, who typically answer the president with a combination of
evasion or no comment, did not rush to become profiles in courage by
thundering against an out-of-control presidency, and some ducked the
issue entirely when confronted by reporters. But many others were blunt
in their rejection of Mr. Trump's position.

``Make no mistake: the election will happen in New Hampshire on November
3rd. End of story,'' Gov. Chris Sununu of New Hampshire, a Republican
who is up for re-election, said on Twitter.

Senator Marco Rubio of Florida said on Capitol Hill, ``Since 1845, we've
had an election on the first Tuesday after November first and we're
going to have one again this year.''

Representative Kevin McCarthy, the House minority leader and one of Mr.
Trump's staunchest allies in Congress, echoed that position, saying ``we
should go forward.''

Others were more equivocal, following a well-worn Republican playbook
for avoiding direct conflict with the president over his wilder
pronouncements. Secretary of State Mike Pompeo, asked in a Senate
hearing whether he believed it was legal for a president to delay an
election, said he was ``not going to enter a legal judgment on that on
the fly this morning.'' That would be an assessment, he said, for the
Justice Department.

Even Mr. Trump's campaign declined to turn his tweet into a rallying
cry, instead playing down the notion that it might have been a policy
prescription. Hogan Gidley, a spokesman for the campaign, said Mr. Trump
was ``just raising a question about the chaos Democrats have created
with their insistence on all mail-in voting'' --- an obviously false
paraphrase of the president's tweet, one that minimized the gravity of
what Mr. Trump had said.

The timing of Mr. Trump's tweet, as much as the content, highlighted the
extent to which he has become a loud but isolated figure in government,
and in the public life of the country. In addition to failing to devise
a credible national response to the coronavirus pandemic, he has not
played the traditional presidential role of calming the country in
moments of fear and soothing it in moments of grief.

Never was that more apparent than on Thursday, when Mr. Trump spent the
morning posting a combination of incendiary and pedestrian tweets, while
his three immediate predecessors ---~Barack Obama, George W. Bush and
Bill Clinton --- gathered in Atlanta for the funeral of John Lewis, the
congressman and civil rights hero.

As mourners assembled at the Ebenezer Baptist Church, Mr. Trump had
other matters on his mind, like hypothetical election fraud and, as it
happened, Italian food.

``Support Patio Pizza and its wonderful owner, Guy Caligiuri, in St.
James, Long Island (N.Y.).'' the president tweeted, referring to a
restaurateur who said he faced backlash for supporting Mr. Trump.
``Great Pizza!!!''

\hypertarget{our-2020-election-guide}{%
\section{Our 2020 Election Guide}\label{our-2020-election-guide}}

Updated July 31, 2020

\begin{itemize}
\item
  \begin{center}\rule{0.5\linewidth}{\linethickness}\end{center}

  \hypertarget{the-latest}{%
  \subsection{The Latest}\label{the-latest}}

  \begin{itemize}
  \tightlist
  \item
    President Trump's assault on the Postal Service is intersecting with
    his attacks on mail-in voting.
    \href{https://www.nytimes.com/2020/07/31/us/politics/trump-usps-mail-delays.html?action=click\&pgtype=Article\&state=default\&region=BELOW_MAIN_CONTENT\&context=storylines_guide}{Voting
    rights groups say it is a recipe for disaster.}
  \end{itemize}
\item
  \begin{center}\rule{0.5\linewidth}{\linethickness}\end{center}

  \hypertarget{bidens-vp-search}{%
  \subsection{Biden's V.P. Search}\label{bidens-vp-search}}

  \begin{itemize}
  \tightlist
  \item
    \href{https://www.nytimes.com/article/biden-vice-president-2020.html?action=click\&pgtype=Article\&state=default\&region=BELOW_MAIN_CONTENT\&context=storylines_guide}{Here
    are 13 women} who have been under consideration to be Joe Biden's
    running mate, and why each might be chosen --- and might not be.
  \end{itemize}
\item
  \begin{center}\rule{0.5\linewidth}{\linethickness}\end{center}

  \hypertarget{keep-up-with-our-coverage}{%
  \subsection{Keep Up With Our
  Coverage}\label{keep-up-with-our-coverage}}

  \begin{itemize}
  \tightlist
  \item
    Get an
    \href{https://www.nytimes.com/newsletters/politics?action=click\&pgtype=Article\&state=default\&region=BELOW_MAIN_CONTENT\&context=storylines_guide}{email}
    recapping the day's news
  \end{itemize}

  \begin{itemize}
  \tightlist
  \item
    Download our mobile app on
    \href{https://apps.apple.com/us/app/nytimes/id284862083?ls=1\&mat_click_id=5c79ae7455014fd1bd66b5610c05b8f2-20191112-16948\&referrer=mat_click_id\%3D5c79ae7455014fd1bd66b5610c05b8f2-20191112-16948\%26link_click_id\%3D722930677036718082}{iOS}
    and
    \href{http://a.localytics.com/android?id=com.nytimes.android\&referrer=utm_source\%3Dother_nyt_mobile_web\%26utm_medium\%3DWeb\%2520page\%26utm_term\%3DGeneral\%2520Mobile\%2520Page\%26utm_campaign\%3DNYT\%2520Mobile\%2520General\%2520Page}{Android}
    and turn on Breaking News and Politics alerts
  \end{itemize}
\end{itemize}

Advertisement

\protect\hyperlink{after-bottom}{Continue reading the main story}

\hypertarget{site-index}{%
\subsection{Site Index}\label{site-index}}

\hypertarget{site-information-navigation}{%
\subsection{Site Information
Navigation}\label{site-information-navigation}}

\begin{itemize}
\tightlist
\item
  \href{https://help.nytimes.com/hc/en-us/articles/115014792127-Copyright-notice}{©~2020~The
  New York Times Company}
\end{itemize}

\begin{itemize}
\tightlist
\item
  \href{https://www.nytco.com/}{NYTCo}
\item
  \href{https://help.nytimes.com/hc/en-us/articles/115015385887-Contact-Us}{Contact
  Us}
\item
  \href{https://www.nytco.com/careers/}{Work with us}
\item
  \href{https://nytmediakit.com/}{Advertise}
\item
  \href{http://www.tbrandstudio.com/}{T Brand Studio}
\item
  \href{https://www.nytimes.com/privacy/cookie-policy\#how-do-i-manage-trackers}{Your
  Ad Choices}
\item
  \href{https://www.nytimes.com/privacy}{Privacy}
\item
  \href{https://help.nytimes.com/hc/en-us/articles/115014893428-Terms-of-service}{Terms
  of Service}
\item
  \href{https://help.nytimes.com/hc/en-us/articles/115014893968-Terms-of-sale}{Terms
  of Sale}
\item
  \href{https://spiderbites.nytimes.com}{Site Map}
\item
  \href{https://help.nytimes.com/hc/en-us}{Help}
\item
  \href{https://www.nytimes.com/subscription?campaignId=37WXW}{Subscriptions}
\end{itemize}
