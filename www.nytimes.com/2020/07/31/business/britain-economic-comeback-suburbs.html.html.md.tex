Sections

SEARCH

\protect\hyperlink{site-content}{Skip to
content}\protect\hyperlink{site-index}{Skip to site index}

\href{https://www.nytimes.com/section/business}{Business}

\href{https://myaccount.nytimes.com/auth/login?response_type=cookie\&client_id=vi}{}

\href{https://www.nytimes.com/section/todayspaper}{Today's Paper}

\href{/section/business}{Business}\textbar{}In Britain, the Economic
Comeback Is in the Suburbs

\url{https://nyti.ms/2EzZ6YZ}

\begin{itemize}
\item
\item
\item
\item
\item
\end{itemize}

\href{https://www.nytimes.com/news-event/coronavirus?action=click\&pgtype=Article\&state=default\&region=TOP_BANNER\&context=storylines_menu}{The
Coronavirus Outbreak}

\begin{itemize}
\tightlist
\item
  live\href{https://www.nytimes.com/2020/08/01/world/coronavirus-covid-19.html?action=click\&pgtype=Article\&state=default\&region=TOP_BANNER\&context=storylines_menu}{Latest
  Updates}
\item
  \href{https://www.nytimes.com/interactive/2020/us/coronavirus-us-cases.html?action=click\&pgtype=Article\&state=default\&region=TOP_BANNER\&context=storylines_menu}{Maps
  and Cases}
\item
  \href{https://www.nytimes.com/interactive/2020/science/coronavirus-vaccine-tracker.html?action=click\&pgtype=Article\&state=default\&region=TOP_BANNER\&context=storylines_menu}{Vaccine
  Tracker}
\item
  \href{https://www.nytimes.com/interactive/2020/07/29/us/schools-reopening-coronavirus.html?action=click\&pgtype=Article\&state=default\&region=TOP_BANNER\&context=storylines_menu}{What
  School May Look Like}
\item
  \href{https://www.nytimes.com/live/2020/07/31/business/stock-market-today-coronavirus?action=click\&pgtype=Article\&state=default\&region=TOP_BANNER\&context=storylines_menu}{Economy}
\end{itemize}

Advertisement

\protect\hyperlink{after-top}{Continue reading the main story}

Supported by

\protect\hyperlink{after-sponsor}{Continue reading the main story}

\hypertarget{in-britain-the-economic-comeback-is-in-the-suburbs}{%
\section{In Britain, the Economic Comeback Is in the
Suburbs}\label{in-britain-the-economic-comeback-is-in-the-suburbs}}

Central London remains ``very, very quiet'' while shops and cafes
outside town centers are seeing a fragile recovery.

\includegraphics{https://static01.nyt.com/images/2020/07/30/business/00virus-uk-cities-1/merlin_174873252_01a5e146-783a-484c-a317-5c6cdff2819f-articleLarge.jpg?quality=75\&auto=webp\&disable=upscale}

By \href{https://www.nytimes.com/by/eshe-nelson}{Eshe Nelson}

\begin{itemize}
\item
  July 31, 2020
\item
  \begin{itemize}
  \item
  \item
  \item
  \item
  \item
  \end{itemize}
\end{itemize}

READING, England --- The British economy is facing its worst recession
since ``The Great Frost'' of 1709, a horrifically cold winter. Large
retailers are shutting stores, and inconsistent quarantine rules are
\href{https://www.nytimes.com/2020/07/30/world/europe/UK-deaths-coronavirus-europe.html}{raising
anxiety about a second pandemic wave}. And yet Summertown, a suburb
north of Oxford, has something to look forward to: Its main shopping
street is about to get a new bookstore.

Daunt Books, a prominent chain, is opening its ninth store this weekend
in Summertown. The suburb's last bookstore closed in 2018 after nearly
four decades. ``People are so delighted a shop is opening and not
closing,'' said Brett Wolstencroft, the manager of the bookseller.

About 60 miles away, in central London, the scene turns bleak.

Daunt's flagship store on Marylebone High Street, in an Edwardian
building with stained glass and parquet floors, is normally a popular
destination, drawing in travelers and locals alike. These days, it's
``very, very quiet'' for long stretches of the week, Mr. Wolstencroft
said.

Go further into central London, and the Daunt store on Cheapside, not
far from St. Paul's Cathedral, is doing even worse. ``There is nobody
there at the moment,'' Mr. Wolstencroft said. ``It's down to a trickle
of people.''

\includegraphics{https://static01.nyt.com/images/2020/07/30/business/00virus-uk-cities-2/merlin_174872859_f9a65ed7-a35f-4a96-bc31-c2471585b7e2-articleLarge.jpg?quality=75\&auto=webp\&disable=upscale}

Without tourists and office workers, city centers in Britain are
suffering steep economic losses from the measures put in place to
contain the spread of
\href{https://www.nytimes.com/news-event/coronavirus}{the coronavirus}.
Even though shops and restaurants have been
\href{https://www.nytimes.com/2020/06/23/world/europe/uk-coronavirus-reopening.html}{allowed
to reopen} since being
\href{https://www.nytimes.com/2020/03/24/world/europe/britain-coronavirus-lockdown.html}{ordered
shut in March}, foot traffic in central London was down 72 percent in
mid-July compared with last year, according to Springboard data on
retail activity. If the pandemic permanently alters the way many people
work, shop and travel, this slump will become entrenched and cities will
\href{https://www.nytimes.com/2020/07/21/business/economy/coronavirus-cities.html}{no
longer be the essential engines of growth} that they once were to
national economies.

It is a serious problem for Robin Baxter, the 27-year-old co-owner of
Hideaway Coffee in central London. The small coffee shop, situated in a
Soho courtyard, was dependent on nearby office workers before the
pandemic.

``We used to go through 30 kilos of coffee a week, and now we're going
through just under a kilo a day,'' Mr. Baxter said. The shop used to be
open from 8 a.m. to 5.30 p.m. Now it opens at 9 a.m. and closes once
they it does not have a customer come in for an hour --- usually around
three or four in the afternoon, he said.

Areas out of town, however, appear to be benefiting from people's desire
to meet and shop in less densely populated places closer to home. Mr.
Wolstencroft said the new store's suburban location in Summertown was an
advantage. The foot traffic at other Daunt stores in more residential
areas in North London have given him a reason for optimism. ``These feel
quite normal,'' he said.

London's recovery is lagging behind the rest of the country, according
to analysis from Fable Data, which uses transactions records from credit
card companies and banks to track spending patterns. In the past month
or so, spending in ``majority urban'' areas, particularly in central
London, has been weaker than spending in less densely populated urban
areas, such as suburbs and other towns away from the capital city
(``mixed urban'').

But over all the numbers remain down: Total spending was 23 percent
lower than last year, Fable's analysis showed. The recovery is only
\href{https://www.nytimes.com/2020/07/14/business/britains-economic-recovery-disappointed-in-may-as-businesses-started-to-reopen.html}{plodding
along}, and the first peak
\href{https://www.nytimes.com/live/2020/07/24/business/stock-market-updates-coronavirus\#british-retail-sales-see-a-v-shaped-recovery-but-it-may-not-last}{has
already subsided}.

This shaky upturn can be seen in Westbury-on-Trym, a suburb of Bristol
in southwest England, where Tiriel Lovejoy has just expanded his small
chain of specialty retail markets called Preserve Foods.

Image

Tiriel Lovejoy of Preserve Foods in Westbury-on-Trym, a suburb of
Bristol. His markets were insulated from the worst of the pandemic's
economic shock during lockdown.Credit...Suzie Howell for The New York
Times

``The lease was ready to be signed pretty much the day the country went
into lockdown,'' Mr. Lovejoy said. Other retailers he knew thought about
delaying expansion plans, but he took a gamble. ``I thought, `Well, this
Covid is temporary, and what we do is hopefully permanent.'''

Part of a budding group of zero-waste grocery stores, Preserve Foods
sells food by weight, encourages customers to buy only the minimum they
need, and avoids packaging. The two other stores are also on the
outskirts of Bristol's city center, and like other grocers and
supermarkets, they were insulated from the worst of the pandemic's
economic shock.

In fact, in the weeks before the government enforced a lockdown in
March, the original shops did two and a half times the sales as usual
Mr. Lovejoy said. And during the months when people were told not to go
outside except for essentials, sales were similar to a normal week, he
added. The biggest change was in the mix of what was sold:
\href{https://www.nytimes.com/2020/05/20/business/britain-flour-mills-baking.html}{lots
of flour}, few toiletries.

But it is unclear if that hum of activity will continue. The
surprisingly strong sales during the lockdown have started to dissipate,
and opening weekend in Westbury-on-Trym was quieter than Mr. Lovejoy had
hoped. And there are the small additional costs that add up; more credit
card transaction fees and disposable shopping bags. ``It's been hard,''
he said.

Image

Opening weekend in mid-July for the new Preserve Foods store was quieter
than Mr. Lovejoy had hoped.Credit...Suzie Howell for The New York Times

While Britain's city centers are comparatively empty, the suburbs are
not exactly booming. Even Mr. Wolstencroft of Daunt Books is not certain
how the Summertown store will do. ``It's probably a question of whether
people stay and browse,'' he said. ``There's an experiment about to
happen.''

Expensive government-funded wage protection programs, praised for
keeping households afloat, are being phased
out\href{https://www.nytimes.com/2020/07/10/business/economy/britain-jobs-plan-layoffs.html}{in
favor of incentives to get people spending} in the hospitality industry.
There are hopes that by reopening the economy, much of the recovery will
take care of itself. But that is putting many businesses to the test.

\href{https://cepr.org/active/publications/discussion_papers/dp.php?dpno=15101}{Research
suggests} people and businesses have taken a more cautious approach to
the pandemic than Britain's policymakers.

Image

Preserve Foods sells food by weight, encourages customers to buy only
the minimum they need, and avoids packaging.Credit...Suzie Howell for
The New York Times

Outside of London, even businesses fortunate enough to see a steady
return of customers are scaling down their ambitions. About 40 miles
west of the capital, everyone at Tutu's Ethiopian Table on a recent
Friday was sitting at tables outside. Although indoor dining in
restaurants has been allowed for weeks, Tutu Melaku does not want to
take any risks, regardless of government guidelines. She said she would
not allow customers to sit inside her cafe until October, at the
earliest.

Ms. Melaku, who was born in Ethiopia but has worked in Britain for the
past three decades, opened her cafe and restaurant last year in Palmer
Park, a public park in a largely residential neighborhood outside the
center of Reading, a large town of about 230,000. Over the course of a
year, she built up the business with music and quiz nights and other
events in addition to the traditional Ethiopian stews on her menu, such
as keye sega wot, served with injera bread. ``When that was all settled,
when I said, `That's it, I've done everything,' Covid arrived,'' Ms.
Melaku said. She shut the doors, and furloughed her staff members.

Image

The Bank area of central London, usually bustling, remains
quiet.Credit...Suzie Howell for The New York Times

Two months later, she reopened the cafe alone, offering a takeout
service that proved to be a success. And so on July 4, the first day
restaurants were allowed to serve diners on the premises, the cafe
opened for outdoor dining only, with a shorter five-hour day and a
smaller menu. ``We were busy all week,'' she said. ``We had more people
than before Covid.''

But despite the successful reopening of Tutu's Ethiopian Table, Ms.
Melaku is cautious about the future. The government's furlough program,
which has supported a third of Britain's labor force, now requires
employers to pay a portion of their workers' wages to keep them on the
program. The stricter terms of the program, which will end completely in
October, led Ms. Melaku to lay off her one full-time staff member in
July.

Even next year, she does not plan on opening the cafe in the evenings
again on a regular basis because she is concerned about the continuing
spread of the coronavirus and expects to have fewer customers. But Ms.
Melaku said this will allow her to save on essential costs, including
electricity. ``There is no need for me to open,'' she said.

Image

At the Daunt Books on Cheapside, London, ``it's down to a trickle of
people,'' said Brett Wolstencroft, the manager of the
bookseller.Credit...Suzie Howell for The New York Times

On the edge of Reading, Woodley, a suburb of about 34,000, is
experiencing a burst of activity. It has its own town center, made up of
an eclectic mix of shops and cafes around a tree garden that is still
waiting to be planted. It has come to life as people stick closer to
home.

Image

Daunt's flagship store on Marylebone High Street in 2019. These days,
the normally popular destination is ``very, very quiet'' for long
stretches of the week, Mr. Wolstencroft said.Credit...Suzie Howell for
The New York Times

The Saturday market is back to normal, according to Brian Fennelly, the
manager of the town center. ``Last Saturday's one was the busiest we've
had this year, even pre-the Covid lockdown,'' he said last week. A new
vegan market on July 19 was three times the size of the one in June, he
added.

But again, long-term success is not assured. Already Mr. Fennelly is
concerned about Christmas. Normally planning would be well underway by
now, but he is trying to delay any major decisions until October. This
year, the Christmas lights will still be turned on, he has ordered a
tree, but he expects most of the town's residents will be watching the
ceremony via a livestream.

Iliana Magra contributed reporting from London.

Advertisement

\protect\hyperlink{after-bottom}{Continue reading the main story}

\hypertarget{site-index}{%
\subsection{Site Index}\label{site-index}}

\hypertarget{site-information-navigation}{%
\subsection{Site Information
Navigation}\label{site-information-navigation}}

\begin{itemize}
\tightlist
\item
  \href{https://help.nytimes.com/hc/en-us/articles/115014792127-Copyright-notice}{©~2020~The
  New York Times Company}
\end{itemize}

\begin{itemize}
\tightlist
\item
  \href{https://www.nytco.com/}{NYTCo}
\item
  \href{https://help.nytimes.com/hc/en-us/articles/115015385887-Contact-Us}{Contact
  Us}
\item
  \href{https://www.nytco.com/careers/}{Work with us}
\item
  \href{https://nytmediakit.com/}{Advertise}
\item
  \href{http://www.tbrandstudio.com/}{T Brand Studio}
\item
  \href{https://www.nytimes.com/privacy/cookie-policy\#how-do-i-manage-trackers}{Your
  Ad Choices}
\item
  \href{https://www.nytimes.com/privacy}{Privacy}
\item
  \href{https://help.nytimes.com/hc/en-us/articles/115014893428-Terms-of-service}{Terms
  of Service}
\item
  \href{https://help.nytimes.com/hc/en-us/articles/115014893968-Terms-of-sale}{Terms
  of Sale}
\item
  \href{https://spiderbites.nytimes.com}{Site Map}
\item
  \href{https://help.nytimes.com/hc/en-us}{Help}
\item
  \href{https://www.nytimes.com/subscription?campaignId=37WXW}{Subscriptions}
\end{itemize}
