Sections

SEARCH

\protect\hyperlink{site-content}{Skip to
content}\protect\hyperlink{site-index}{Skip to site index}

\href{https://www.nytimes.com/section/fashion/weddings}{Love}

\href{https://myaccount.nytimes.com/auth/login?response_type=cookie\&client_id=vi}{}

\href{https://www.nytimes.com/section/todayspaper}{Today's Paper}

\href{/section/fashion/weddings}{Love}\textbar{}Their Very Own Loving
Story

\url{https://nyti.ms/30ftzEj}

\begin{itemize}
\item
\item
\item
\item
\item
\end{itemize}

\href{https://www.nytimes.com/news-event/coronavirus?action=click\&pgtype=Article\&state=default\&region=TOP_BANNER\&context=storylines_menu}{The
Coronavirus Outbreak}

\begin{itemize}
\tightlist
\item
  live\href{https://www.nytimes.com/2020/08/04/world/coronavirus-cases.html?action=click\&pgtype=Article\&state=default\&region=TOP_BANNER\&context=storylines_menu}{Latest
  Updates}
\item
  \href{https://www.nytimes.com/interactive/2020/us/coronavirus-us-cases.html?action=click\&pgtype=Article\&state=default\&region=TOP_BANNER\&context=storylines_menu}{Maps
  and Cases}
\item
  \href{https://www.nytimes.com/interactive/2020/science/coronavirus-vaccine-tracker.html?action=click\&pgtype=Article\&state=default\&region=TOP_BANNER\&context=storylines_menu}{Vaccine
  Tracker}
\item
  \href{https://www.nytimes.com/2020/08/02/us/covid-college-reopening.html?action=click\&pgtype=Article\&state=default\&region=TOP_BANNER\&context=storylines_menu}{College
  Reopening}
\item
  \href{https://www.nytimes.com/live/2020/08/04/business/stock-market-today-coronavirus?action=click\&pgtype=Article\&state=default\&region=TOP_BANNER\&context=storylines_menu}{Economy}
\end{itemize}

Advertisement

\protect\hyperlink{after-top}{Continue reading the main story}

Supported by

\protect\hyperlink{after-sponsor}{Continue reading the main story}

vows

\hypertarget{their-very-own-loving-story}{%
\section{Their Very Own Loving
Story}\label{their-very-own-loving-story}}

Renee Knake and Wallace B. Jefferson, both lawyers, are grateful to
Mildred Loving and Richard Loving for challenging interracial marriage
laws in the 1960s.

\includegraphics{https://static01.nyt.com/images/2020/08/02/fashion/00VOWS-MackinacIsland-03/00VOWS-MackinacIsland-03-articleLarge.jpg?quality=75\&auto=webp\&disable=upscale}

By Brianna Holt

\begin{itemize}
\item
  July 31, 2020
\item
  \begin{itemize}
  \item
  \item
  \item
  \item
  \item
  \end{itemize}
\end{itemize}

Before Renee Newman Knake and Wallace B. Jefferson, both lawyers, became
romantically involved, they met at various work events. The first time
was when Mr. Jefferson delivered a commencement address in 2011 at
Michigan State University College of Law, where Ms. Knake was then a law
professor, though it wasn't until years later that she was reminded of
that encounter.

The two also worked as volunteers for the American Bar Association, from
2014 to 2016. Ms. Knake, now 46, served as a reporter to the group's
Commission on the Future of Legal Services. Mr. Jefferson, who is 57 and
has served as chief justice of the Texas Supreme Court, was a
commissioner. They spoke on conference calls and attended meetings,
maintaining a platonic and professional connection. But during a dinner
toward the end of the commission's work, their relationship seemed to
change.

``Something just clicked and all of a sudden this person that I had been
aware of professionally --- it was suddenly like, wow,'' said Ms. Knake,
who divorced in 2013. ``I remember having a conversation with him at one
point and then thinking, `I don't want this conversation to end.'''

In the fall of 2016, Ms. Knake took a job as a law professor at the
University of Houston, about two and a half hours from Austin, Texas,
where Mr. Jefferson was living. The two continued to stay in contact and
met again when Ms. Knake invited Mr. Jefferson to attend her first
public lecture at the university in 2017. She was presenting research
that eventually would be turned into a book published this year,
``Shortlisted: Women in the Shadows of the Supreme Court.''

``He came and sat in the front row of this lecture hall that was filled
with students,'' Ms. Knake said. ``My parents traveled in and my kids
were there. It was this wonderful alignment of my professional world,
and my family, and this man who I was starting to fall in love with.''

\emph{{[}}\href{https://www.nytimes.com/newsletters/love-letter?module=inline}{\emph{Sign
up for Love Letter and always get the latest in Modern Love, weddings,
and relationships in the news by email.}}\emph{{]}}

\includegraphics{https://static01.nyt.com/images/2020/08/02/fashion/00VOWS-MackinacIsland/00VOWS-MackinacIsland-articleLarge.jpg?quality=75\&auto=webp\&disable=upscale}

Because the couple resided and worked in different cities, long distance
and commuting became undeniable choices. But both Mr. Jefferson and Ms.
Knake were willing to make it work. Between traveling together to
conferences and visits when Ms. Knake wasn't teaching, the couple saw
the long distance as an exciting part of their union, rather than an
issue.

In 2019, they solidified plans to live together for the first time. Ms.
Knake accepted a Fulbright scholar position at the Royal Melbourne
Institute of Technology in Melbourne, Australia. Her two children,
James, 15, and Grace, 12, along with Mr. Jefferson joined her for the
six-month stay. Before making the move, Mr. Jefferson proposed on Jan.
1, 2019 in front of her children on a FaceTime call.

The new year would start with greater commitment to each other and lay
out a foundation for an unexpected quarantine after the coronavirus
outbreak.

\hypertarget{latest-updates-global-coronavirus-outbreak}{%
\section{\texorpdfstring{\href{https://www.nytimes.com/2020/08/04/world/coronavirus-cases.html?action=click\&pgtype=Article\&state=default\&region=MAIN_CONTENT_1\&context=storylines_live_updates}{Latest
Updates: Global Coronavirus
Outbreak}}{Latest Updates: Global Coronavirus Outbreak}}\label{latest-updates-global-coronavirus-outbreak}}

Updated 2020-08-05T03:50:43.949Z

\begin{itemize}
\tightlist
\item
  \href{https://www.nytimes.com/2020/08/04/world/coronavirus-cases.html?action=click\&pgtype=Article\&state=default\&region=MAIN_CONTENT_1\&context=storylines_live_updates\#link-762df92}{As
  talks drag on, McConnell signals openness to jobless aid extension,
  and negotiators agree on a deadline.}
\item
  \href{https://www.nytimes.com/2020/08/04/world/coronavirus-cases.html?action=click\&pgtype=Article\&state=default\&region=MAIN_CONTENT_1\&context=storylines_live_updates\#link-1228a480}{Novavax
  sees encouraging results from two studies of its experimental
  vaccine.}
\item
  \href{https://www.nytimes.com/2020/08/04/world/coronavirus-cases.html?action=click\&pgtype=Article\&state=default\&region=MAIN_CONTENT_1\&context=storylines_live_updates\#link-794484ed}{Mississippians
  must now wear masks in public, governor says.}
\end{itemize}

\href{https://www.nytimes.com/2020/08/04/world/coronavirus-cases.html?action=click\&pgtype=Article\&state=default\&region=MAIN_CONTENT_1\&context=storylines_live_updates}{See
more updates}

More live coverage:
\href{https://www.nytimes.com/live/2020/08/04/business/stock-market-today-coronavirus?action=click\&pgtype=Article\&state=default\&region=MAIN_CONTENT_1\&context=storylines_live_updates}{Markets}

``And that was just really a magical time to sort of solidify this new
dimension to our love and our family,'' Ms. Knake said. ``The kids were
going to Australian schools, they didn't know anybody. We had to really
rely on each other. And looking back, it was really great preparation
for the four of us being in quarantine together since early March.''

``Blending families sometimes can be a real challenge,'' she continued.
``And for us, it has gone so smoothly in part, because of how this
transition evolved over the past couple of years. Being out of the
country together helped us prepare for being quarantined in a very small
condo in Austin and in East Lansing for the past few months.''

Quarantine has involved adopting a puppy, working from home, online
school, and of course, wedding planning. It's also erupted some very
important and in-depth conversations about race. As a social justice
movement unfolds in America, Ms. Knake admits there is a lot to learn
about her partner's experience.

``I am learning every single day alongside him, more and more about what
it means to be a Black man in the United States,'' she said. ``My eyes
were open to this continually before it was really in the headlines, as
it should be deservedly, but being in a relationship with him makes me
more aware.''

Coincidentally, Mr. Jefferson and Ms. Knake, who both had not dated
outside their race before, share birthdays with Mildred Loving and
Richard Loving. In 1967, the Lovings effectively challenged interracial
marriage laws in the United States forever, resulting in the
legalization of interracial marriage. It is because of their legacy that
Mr. Jefferson and Ms. Knake walked down the aisle on July 4 to legally
recognize their union.

Image

The couple were joined by three of their children: Ms. Knake's daughter,
Grace, and son, James, and Mr. Jefferson's son Samuel.Credit...Sydney
Shrewsbury

``When we saw their birthdays, we were like, `wow,' it made us feel even
more connected to them,'' Ms. Knake said. ``Because of that coincidence
we can retell our part of their story in a way that will help people
remember that it wasn't that long ago in this country that just loving
someone that wasn't the same race as you could not be made legitimate
through marriage.''

``I can't imagine what it would be like to be so profoundly and deeply
in love as I am with this man and to have to feel like I was committing
a crime,'' Ms. Knake added.

The couple paid homage to the Lovings during their ceremony by sharing
statements dedicated to their fight for interracial marriage, read by
Samuel Jefferson, one of Mr. Jefferson's three sons.

Michigan's governor, Gretchen Whitmer, officiated on the porch of the
governor's summer residence on Mackinac Island.

Ms. Whitmer is a friend to both Ms. Knake and Mr. Jefferson. When Ms.
Knake first moved to Michigan in 2005, Ms. Whitmer lived across the
street from her. As for Mr. Jefferson, he and the governor are Michigan
State alumni and quickly bonded after meeting through Ms. Knake.

``Over the many years I've known her, she's been an inspiration and a
source of strength for me through some really tough times, including
helping me recover from a brain aneurysm a decade ago. So it was
especially wonderful to have her officiate our wedding, a moment of
total happiness and joy,'' Ms. Knake said.

\href{https://www.nytimes.com/news-event/coronavirus?action=click\&pgtype=Article\&state=default\&region=MAIN_CONTENT_3\&context=storylines_faq}{}

\hypertarget{the-coronavirus-outbreak-}{%
\subsubsection{The Coronavirus Outbreak
›}\label{the-coronavirus-outbreak-}}

\hypertarget{frequently-asked-questions}{%
\paragraph{Frequently Asked
Questions}\label{frequently-asked-questions}}

Updated August 4, 2020

\begin{itemize}
\item ~
  \hypertarget{i-have-antibodies-am-i-now-immune}{%
  \paragraph{I have antibodies. Am I now
  immune?}\label{i-have-antibodies-am-i-now-immune}}

  \begin{itemize}
  \tightlist
  \item
    As of right
    now,\href{https://www.nytimes.com/2020/07/22/health/covid-antibodies-herd-immunity.html?action=click\&pgtype=Article\&state=default\&region=MAIN_CONTENT_3\&context=storylines_faq}{that
    seems likely, for at least several months.} There have been
    frightening accounts of people suffering what seems to be a second
    bout of Covid-19. But experts say these patients may have a
    drawn-out course of infection, with the virus taking a slow toll
    weeks to months after initial exposure. People infected with the
    coronavirus typically
    \href{https://www.nature.com/articles/s41586-020-2456-9}{produce}
    immune molecules called antibodies, which are
    \href{https://www.nytimes.com/2020/05/07/health/coronavirus-antibody-prevalence.html?action=click\&pgtype=Article\&state=default\&region=MAIN_CONTENT_3\&context=storylines_faq}{protective
    proteins made in response to an
    infection}\href{https://www.nytimes.com/2020/05/07/health/coronavirus-antibody-prevalence.html?action=click\&pgtype=Article\&state=default\&region=MAIN_CONTENT_3\&context=storylines_faq}{.
    These antibodies may} last in the body
    \href{https://www.nature.com/articles/s41591-020-0965-6}{only two to
    three months}, which may seem worrisome, but that's perfectly normal
    after an acute infection subsides, said Dr. Michael Mina, an
    immunologist at Harvard University. It may be possible to get the
    coronavirus again, but it's highly unlikely that it would be
    possible in a short window of time from initial infection or make
    people sicker the second time.
  \end{itemize}
\item ~
  \hypertarget{im-a-small-business-owner-can-i-get-relief}{%
  \paragraph{I'm a small-business owner. Can I get
  relief?}\label{im-a-small-business-owner-can-i-get-relief}}

  \begin{itemize}
  \tightlist
  \item
    The
    \href{https://www.nytimes.com/article/small-business-loans-stimulus-grants-freelancers-coronavirus.html?action=click\&pgtype=Article\&state=default\&region=MAIN_CONTENT_3\&context=storylines_faq}{stimulus
    bills enacted in March} offer help for the millions of American
    small businesses. Those eligible for aid are businesses and
    nonprofit organizations with fewer than 500 workers, including sole
    proprietorships, independent contractors and freelancers. Some
    larger companies in some industries are also eligible. The help
    being offered, which is being managed by the Small Business
    Administration, includes the Paycheck Protection Program and the
    Economic Injury Disaster Loan program. But lots of folks have
    \href{https://www.nytimes.com/interactive/2020/05/07/business/small-business-loans-coronavirus.html?action=click\&pgtype=Article\&state=default\&region=MAIN_CONTENT_3\&context=storylines_faq}{not
    yet seen payouts.} Even those who have received help are confused:
    The rules are draconian, and some are stuck sitting on
    \href{https://www.nytimes.com/2020/05/02/business/economy/loans-coronavirus-small-business.html?action=click\&pgtype=Article\&state=default\&region=MAIN_CONTENT_3\&context=storylines_faq}{money
    they don't know how to use.} Many small-business owners are getting
    less than they expected or
    \href{https://www.nytimes.com/2020/06/10/business/Small-business-loans-ppp.html?action=click\&pgtype=Article\&state=default\&region=MAIN_CONTENT_3\&context=storylines_faq}{not
    hearing anything at all.}
  \end{itemize}
\item ~
  \hypertarget{what-are-my-rights-if-i-am-worried-about-going-back-to-work}{%
  \paragraph{What are my rights if I am worried about going back to
  work?}\label{what-are-my-rights-if-i-am-worried-about-going-back-to-work}}

  \begin{itemize}
  \tightlist
  \item
    Employers have to provide
    \href{https://www.osha.gov/SLTC/covid-19/standards.html}{a safe
    workplace} with policies that protect everyone equally.
    \href{https://www.nytimes.com/article/coronavirus-money-unemployment.html?action=click\&pgtype=Article\&state=default\&region=MAIN_CONTENT_3\&context=storylines_faq}{And
    if one of your co-workers tests positive for the coronavirus, the
    C.D.C.} has said that
    \href{https://www.cdc.gov/coronavirus/2019-ncov/community/guidance-business-response.html}{employers
    should tell their employees} -\/- without giving you the sick
    employee's name -\/- that they may have been exposed to the virus.
  \end{itemize}
\item ~
  \hypertarget{should-i-refinance-my-mortgage}{%
  \paragraph{Should I refinance my
  mortgage?}\label{should-i-refinance-my-mortgage}}

  \begin{itemize}
  \tightlist
  \item
    \href{https://www.nytimes.com/article/coronavirus-money-unemployment.html?action=click\&pgtype=Article\&state=default\&region=MAIN_CONTENT_3\&context=storylines_faq}{It
    could be a good idea,} because mortgage rates have
    \href{https://www.nytimes.com/2020/07/16/business/mortgage-rates-below-3-percent.html?action=click\&pgtype=Article\&state=default\&region=MAIN_CONTENT_3\&context=storylines_faq}{never
    been lower.} Refinancing requests have pushed mortgage applications
    to some of the highest levels since 2008, so be prepared to get in
    line. But defaults are also up, so if you're thinking about buying a
    home, be aware that some lenders have tightened their standards.
  \end{itemize}
\item ~
  \hypertarget{what-is-school-going-to-look-like-in-september}{%
  \paragraph{What is school going to look like in
  September?}\label{what-is-school-going-to-look-like-in-september}}

  \begin{itemize}
  \tightlist
  \item
    It is unlikely that many schools will return to a normal schedule
    this fall, requiring the grind of
    \href{https://www.nytimes.com/2020/06/05/us/coronavirus-education-lost-learning.html?action=click\&pgtype=Article\&state=default\&region=MAIN_CONTENT_3\&context=storylines_faq}{online
    learning},
    \href{https://www.nytimes.com/2020/05/29/us/coronavirus-child-care-centers.html?action=click\&pgtype=Article\&state=default\&region=MAIN_CONTENT_3\&context=storylines_faq}{makeshift
    child care} and
    \href{https://www.nytimes.com/2020/06/03/business/economy/coronavirus-working-women.html?action=click\&pgtype=Article\&state=default\&region=MAIN_CONTENT_3\&context=storylines_faq}{stunted
    workdays} to continue. California's two largest public school
    districts --- Los Angeles and San Diego --- said on July 13, that
    \href{https://www.nytimes.com/2020/07/13/us/lausd-san-diego-school-reopening.html?action=click\&pgtype=Article\&state=default\&region=MAIN_CONTENT_3\&context=storylines_faq}{instruction
    will be remote-only in the fall}, citing concerns that surging
    coronavirus infections in their areas pose too dire a risk for
    students and teachers. Together, the two districts enroll some
    825,000 students. They are the largest in the country so far to
    abandon plans for even a partial physical return to classrooms when
    they reopen in August. For other districts, the solution won't be an
    all-or-nothing approach.
    \href{https://bioethics.jhu.edu/research-and-outreach/projects/eschool-initiative/school-policy-tracker/}{Many
    systems}, including the nation's largest, New York City, are
    devising
    \href{https://www.nytimes.com/2020/06/26/us/coronavirus-schools-reopen-fall.html?action=click\&pgtype=Article\&state=default\&region=MAIN_CONTENT_3\&context=storylines_faq}{hybrid
    plans} that involve spending some days in classrooms and other days
    online. There's no national policy on this yet, so check with your
    municipal school system regularly to see what is happening in your
    community.
  \end{itemize}
\end{itemize}

The couple were joined by three of their children, four family friends,
as well as Dr. Marc Mallory, the governor's husband. Nearly 150 other
guests streamed the ceremony via Zoom. Originally, the wedding was going
to take place inside the same residence on July 4, but coronavirus
precautions pushed the occasion online for guests.

In proper wedding-planning fashion, a rehearsal was held the evening
before. The outdoor lighting, Wi-Fi connection and video camera were all
tested. The celebration was organized by Wedfuly, a virtual wedding
company that partners with Zoom to host online celebrations.

As attendees entered the video conference, ``America the Beautiful,'' by
Ray Charles, ``A Sunday Kind of Love,'' by Etta James, and a handful of
other songs chosen by the bride and groom played in the background. A
slide show of photos featuring the couple and their family hovered over
the main screen. Grace Knake, Ms. Knake's daughter, shared an
Acknowledgment of Country, which pays respect to the traditional
landowners. This was followed by James Knake, Ms. Knake's son, reading
the poem ``Believer's Hymn for the Republic.''

At about 4:30 p.m., the ceremony took place and was followed by an
intermission, allowing guests to grab a drink for the toast. Seated in
front of a laptop, Mr. Jefferson and Ms. Knake cut the cake and received
seven toasts by friends and family. After, the couple popped into
individual chat rooms to speak with guests by group. Ms. Knake described
the event as far more moving and magical than expected. ``It was
probably the only time I ever enjoyed a Zoom call,'' she said.

The reception dinner was held outdoors on the patio of the Carriage
House at Iroquois Hotel. Their original plans for a larger reception
were postponed to 2021, including a boat cruise party on Mackinac Island
and a rooftop gathering for family and close friends in Austin.
Additionally, honeymoon festivities have been pushed to ``further
notice.'' The couple still plans to vacation in southern France once
travel bans are lifted between Europe and American visitors.

``I tell people coronavirus could have canceled pretty much everything
in my life, but it has not canceled my commitment to Wallace,'' Ms.
Knake said.

\begin{center}\rule{0.5\linewidth}{\linethickness}\end{center}

\hypertarget{on-this-day}{%
\subsubsection{\texorpdfstring{\textbf{On This
Day}}{On This Day}}\label{on-this-day}}

\textbf{When} July 4, 2020

\textbf{Where} The outdoor porch of the Michigan governor's summer
residence, Mackinac Island, Mich.

\textbf{Bands, Dresses, and Masks} The couple wore matching platinum
bands from Canturi that were picked while living abroad in Melbourne,
Australia. Ms. Knake wore a floor-length, off-the-shoulder white gown
designed by Chiara Boni La Petite Robe, for the ceremony. For the
dinner, she donned a knee-length white sleeveless dress by Nicole
Miller. Mr. Jefferson wore a tuxedo that he has had for years. Ms. Knake
wore a flamingo-finish mask bought on Etsy by ZhenLinen from Los
Angeles; Mr. Jefferson wore a Michigan State mask.

\textbf{Bachelorette Yoga} Ms. Knake and five of her close girlfriends
took a socially distanced, sunset yoga class in East Lansing, the Sunday
before the wedding. Six feet apart and outdoors, the gathering created a
safe and relaxing start to a busy wedding week.

\emph{Continue following our fashion and lifestyle coverage on Facebook
(}\href{https://www.facebook.com/nytimesstyles}{\emph{Styles}}
\emph{and} \href{https://www.facebook.com/modernlove}{\emph{Modern
Love}}\emph{), Twitter
(}\href{https://twitter.com/nytstyles}{\emph{Styles}}\emph{,}
\href{https://twitter.com/nytfashion}{\emph{Fashion}} \emph{and}
\href{https://twitter.com/nytimesvows}{\emph{Weddings}}\emph{) and}
\href{https://instagram.com/nytimesfashion}{\emph{Instagram}}\emph{.}

Advertisement

\protect\hyperlink{after-bottom}{Continue reading the main story}

\hypertarget{site-index}{%
\subsection{Site Index}\label{site-index}}

\hypertarget{site-information-navigation}{%
\subsection{Site Information
Navigation}\label{site-information-navigation}}

\begin{itemize}
\tightlist
\item
  \href{https://help.nytimes.com/hc/en-us/articles/115014792127-Copyright-notice}{©~2020~The
  New York Times Company}
\end{itemize}

\begin{itemize}
\tightlist
\item
  \href{https://www.nytco.com/}{NYTCo}
\item
  \href{https://help.nytimes.com/hc/en-us/articles/115015385887-Contact-Us}{Contact
  Us}
\item
  \href{https://www.nytco.com/careers/}{Work with us}
\item
  \href{https://nytmediakit.com/}{Advertise}
\item
  \href{http://www.tbrandstudio.com/}{T Brand Studio}
\item
  \href{https://www.nytimes.com/privacy/cookie-policy\#how-do-i-manage-trackers}{Your
  Ad Choices}
\item
  \href{https://www.nytimes.com/privacy}{Privacy}
\item
  \href{https://help.nytimes.com/hc/en-us/articles/115014893428-Terms-of-service}{Terms
  of Service}
\item
  \href{https://help.nytimes.com/hc/en-us/articles/115014893968-Terms-of-sale}{Terms
  of Sale}
\item
  \href{https://spiderbites.nytimes.com}{Site Map}
\item
  \href{https://help.nytimes.com/hc/en-us}{Help}
\item
  \href{https://www.nytimes.com/subscription?campaignId=37WXW}{Subscriptions}
\end{itemize}
