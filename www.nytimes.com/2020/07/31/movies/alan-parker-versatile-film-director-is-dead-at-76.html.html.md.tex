Sections

SEARCH

\protect\hyperlink{site-content}{Skip to
content}\protect\hyperlink{site-index}{Skip to site index}

\href{https://www.nytimes.com/section/movies}{Movies}

\href{https://myaccount.nytimes.com/auth/login?response_type=cookie\&client_id=vi}{}

\href{https://www.nytimes.com/section/todayspaper}{Today's Paper}

\href{/section/movies}{Movies}\textbar{}Alan Parker, Versatile Film
Director, Is Dead at 76

\url{https://nyti.ms/33d4r2B}

\begin{itemize}
\item
\item
\item
\item
\item
\end{itemize}

Advertisement

\protect\hyperlink{after-top}{Continue reading the main story}

Supported by

\protect\hyperlink{after-sponsor}{Continue reading the main story}

\hypertarget{alan-parker-versatile-film-director-is-dead-at-76}{%
\section{Alan Parker, Versatile Film Director, Is Dead at
76}\label{alan-parker-versatile-film-director-is-dead-at-76}}

``Midnight Express'' and ``Mississippi Burning'' brought him Oscar
nominations, and many of his other films, including ``Fame,'' were
acclaimed.

\includegraphics{https://static01.nyt.com/images/2020/08/01/obituaries/01Parker-obit1/31Parker6-articleLarge.jpg?quality=75\&auto=webp\&disable=upscale}

\href{https://www.nytimes.com/by/neil-genzlinger}{\includegraphics{https://static01.nyt.com/images/2018/06/13/multimedia/author-neil-genzlinger/author-neil-genzlinger-thumbLarge.jpg}}

By \href{https://www.nytimes.com/by/neil-genzlinger}{Neil Genzlinger}

\begin{itemize}
\item
  July 31, 2020
\item
  \begin{itemize}
  \item
  \item
  \item
  \item
  \item
  \end{itemize}
\end{itemize}

Alan Parker, who was nominated for the best-director Oscar for the 1978
film ``Midnight Express'' and again 10 years later for ``Mississippi
Burning,'' died on Friday in South London. He was 76.

His death followed a long, unspecified illness, a spokeswoman for the
British Film Institute said.

Mr. Parker directed a number of other well-regarded films, working in a
range of styles and genres. ``Fame'' (1980) was a musical about a
performing arts high school in New York. ``Birdy'' (1984) was based on a
William Wharton novel about a boy who had an erotic fascination with
avian life. ``Angel Heart'' (1987) was a sexy noir that flirted with an
X rating but ended up with an R. ``Angela's Ashes'' (1999) was based on
Frank McCourt's popular autobiography.

Music underpinned some of Mr. Parker's best-known work. His first
feature film was the gangster satire ``Bugsy Malone'' in 1976, in which
adolescents played the gangsters and Paul Williams songs punctuated the
action. Two years after ``Fame,'' he directed ``Pink Floyd: The Wall,''
an imagery-filled story about a British rock star that was written by
Roger Waters of the band Pink Floyd and based on the band's album of the
same name. In 1991 came ``The Commitments,'' a lighthearted tale about a
band in Dublin. In 1996 he directed the film version of the stage
musical ``Evita,'' with Madonna in the role of Eva Perón.

Madonna, he told The Mirror in 1996, wasn't the easiest person to work
with, but he found a way to get the best out of her.

``My secret was to let her moan to my assistants to get it out of her
system so that by the time she stepped in front of the camera she was
all complained out,'' he said.

The performance won her a Golden Globe.

\includegraphics{https://static01.nyt.com/images/2020/08/01/obituaries/01Parker-obit5/31Parker5-articleLarge.jpg?quality=75\&auto=webp\&disable=upscale}

Alan William Parker was born on Feb. 14, 1944, in the Islington district
of London. He started his career as a copywriter and then moved into
making television commercials.

``The only way anybody would give me a chance to say `Action!' and
`Cut!' was by doing commercials,'' he told The New York Times in 1980.
``That's how I learned the craft. I've done ridiculous things, like
re-create --- frame by frame --- `Brief Encounter' for Birds Eye Dinner
for One.''

That background, he said, gave him a certain disdain for the auteur
theory of filmmaking, which holds that the director is the main creative
force of a project.

``A film is never \emph{my} film,'' he said, ``because I'm part of a
talented lot of people.''

In the early 1970s, with hundreds of commercials under his belt, he
began moving into feature films, first as the screenwriter on a 1971
British movie, ``Melody.'' In 1974 he directed a BBC Television movie
called ``The Evacuees,'' about Jewish children being evacuated from
London during the Blitz in World War II.

Soon, though, Mr. Parker was thoroughly identified with films about
American subjects.

``Midnight Express,'' with a screenplay by Oliver Stone, is about an
American college student who is thrown into a Turkish prison on a drug
smuggling charge. ``Fame,'' about students at the High School of
Performing Arts in New York, brought Mr. Parker some criticism in his
home country, where, he said, people asked, ``Why don't you make a film
about London, about the Royal Academy of Dramatic Art?''

``The exciting thing about the High School of Performing Arts,'' he told
The Times in 1980, ``is that it has a social and ethnic mix that you
couldn't possibly find anywhere in the world, especially not England.''

Image

``Mississippi Burning'' is a fictionalized treatment of the real-life
case involving the murder of three civil rights workers in Mississippi
in 1964. Vincent Canby,
\href{https://www.nytimes.com/1988/12/09/movies/review-film-retracing-mississippi-s-agony-1964.html}{reviewing
it} in The Times in 1988, called it ``one of the toughest, straightest,
most effective fiction films yet made about bigotry and racial violence,
whether in this country or anywhere else in the world.''

Image

Mr. Parker and the cinematographer Peter Biziou filming ``Pink Floyd:
The Wall'' (1982), based on the album of the same name.

Some of Mr. Parker's films generated controversy. ``Midnight Express''
was accused of demonizing Turkey and its people. ``Angel Heart''
featured a steamy sex scene between Mickey Rourke and
\href{https://decider.com/2019/10/03/angel-heart-walter-chaw/}{Lisa
Bonet}, who was then best known for her role as Denise Huxtable on the
family-friendly sitcom ``The Cosby Show.'' ``Mississippi Burning,''
starring Gene Hackman and Willem Dafoe, was faulted for, among other
things, not having strong Black characters even though it was a
civil-rights-era story. ``Angela's Ashes'' was criticized as
misrepresenting Irish life.

``It would be nice to do a film that isn't controversial,'' Mr. Parker
\href{https://www.chicagotribune.com/news/ct-xpm-1991-08-18-9103010538-story.html}{told
The Chicago Tribune} just before the relatively benign ``The
Commitments'' was released, ``although I'm sure someone is bound to find
controversy in `The Commitments.'''

Mr. Parker received a lifetime achievement award from the Directors
Guild of Great Britain in 1998 and was knighted in 2002.

He is survived by his second wife, Lisa Moran-Parker; a son from their
marriage, Henry; four children from his marriage to Annie Inglis, Lucy,
Alexander, Jake and Nathan Parker; and seven grandchildren.

In a
\href{https://www.bfi.org.uk/news-opinion/news-bfi/interviews/alan-parker-interview}{2003
discussion} organized by the British Film Institute in conjunction with
the release of his final film, ``The Life of David Gale,'' about a
death-penalty opponent (Kevin Spacey) facing execution for murder, Mr.
Parker talked about the intuition and serendipity that play a part in
the director's art.

``It seems to me that a director's job is to look for wherever the magic
may be in any scene,'' he said, ``and sometimes it's not where you
think.''

``Sometimes the images in your head are better than what you end up
with,'' he added. ``Sometimes they're nowhere near as good as what
happens in front of you.''

Alex Marshall contributed reporting from London.

Advertisement

\protect\hyperlink{after-bottom}{Continue reading the main story}

\hypertarget{site-index}{%
\subsection{Site Index}\label{site-index}}

\hypertarget{site-information-navigation}{%
\subsection{Site Information
Navigation}\label{site-information-navigation}}

\begin{itemize}
\tightlist
\item
  \href{https://help.nytimes.com/hc/en-us/articles/115014792127-Copyright-notice}{©~2020~The
  New York Times Company}
\end{itemize}

\begin{itemize}
\tightlist
\item
  \href{https://www.nytco.com/}{NYTCo}
\item
  \href{https://help.nytimes.com/hc/en-us/articles/115015385887-Contact-Us}{Contact
  Us}
\item
  \href{https://www.nytco.com/careers/}{Work with us}
\item
  \href{https://nytmediakit.com/}{Advertise}
\item
  \href{http://www.tbrandstudio.com/}{T Brand Studio}
\item
  \href{https://www.nytimes.com/privacy/cookie-policy\#how-do-i-manage-trackers}{Your
  Ad Choices}
\item
  \href{https://www.nytimes.com/privacy}{Privacy}
\item
  \href{https://help.nytimes.com/hc/en-us/articles/115014893428-Terms-of-service}{Terms
  of Service}
\item
  \href{https://help.nytimes.com/hc/en-us/articles/115014893968-Terms-of-sale}{Terms
  of Sale}
\item
  \href{https://spiderbites.nytimes.com}{Site Map}
\item
  \href{https://help.nytimes.com/hc/en-us}{Help}
\item
  \href{https://www.nytimes.com/subscription?campaignId=37WXW}{Subscriptions}
\end{itemize}
