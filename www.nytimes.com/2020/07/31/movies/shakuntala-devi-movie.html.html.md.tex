Sections

SEARCH

\protect\hyperlink{site-content}{Skip to
content}\protect\hyperlink{site-index}{Skip to site index}

\href{https://www.nytimes.com/section/movies}{Movies}

\href{https://myaccount.nytimes.com/auth/login?response_type=cookie\&client_id=vi}{}

\href{https://www.nytimes.com/section/todayspaper}{Today's Paper}

\href{/section/movies}{Movies}\textbar{}5 Things to Know About
Shakuntala Devi

\url{https://nyti.ms/30hIkpY}

\begin{itemize}
\item
\item
\item
\item
\item
\end{itemize}

\href{https://www.nytimes.com/spotlight/at-home?action=click\&pgtype=Article\&state=default\&region=TOP_BANNER\&context=at_home_menu}{At
Home}

\begin{itemize}
\tightlist
\item
  \href{https://www.nytimes.com/2020/07/28/books/time-for-a-literary-road-trip.html?action=click\&pgtype=Article\&state=default\&region=TOP_BANNER\&context=at_home_menu}{Take:
  A Literary Road Trip}
\item
  \href{https://www.nytimes.com/2020/07/29/magazine/bored-with-your-home-cooking-some-smoky-eggplant-will-fix-that.html?action=click\&pgtype=Article\&state=default\&region=TOP_BANNER\&context=at_home_menu}{Cook:
  Smoky Eggplant}
\item
  \href{https://www.nytimes.com/2020/07/27/travel/moose-michigan-isle-royale.html?action=click\&pgtype=Article\&state=default\&region=TOP_BANNER\&context=at_home_menu}{Look
  Out: For Moose}
\item
  \href{https://www.nytimes.com/interactive/2020/at-home/even-more-reporters-editors-diaries-lists-recommendations.html?action=click\&pgtype=Article\&state=default\&region=TOP_BANNER\&context=at_home_menu}{Explore:
  Reporters' Obsessions}
\end{itemize}

Advertisement

\protect\hyperlink{after-top}{Continue reading the main story}

Supported by

\protect\hyperlink{after-sponsor}{Continue reading the main story}

\hypertarget{5-things-to-know-about-shakuntala-devi}{%
\section{5 Things to Know About Shakuntala
Devi}\label{5-things-to-know-about-shakuntala-devi}}

A film about the Indian mathematics genius is now streaming on Amazon
Prime Video. Here are five facts to get you more familiar.

\includegraphics{https://static01.nyt.com/images/2020/07/31/arts/31shakuntala-primer2/merlin_175165059_16547c37-e893-491f-95ef-35e15e153060-articleLarge.jpg?quality=75\&auto=webp\&disable=upscale}

By \href{https://www.nytimes.com/by/priya-arora}{Priya Arora}

\begin{itemize}
\item
  July 31, 2020
\item
  \begin{itemize}
  \item
  \item
  \item
  \item
  \item
  \end{itemize}
\end{itemize}

Shakuntala Devi (1929-2013) was best known as ``the human computer'' for
her ability to perform lengthy calculations in her head, swiftly. One
example of this, described in
\href{https://www.nytimes.com/2013/04/24/world/asia/shakuntala-devi-human-computer-dies-in-india-at-83.html}{her
New York Times obituary,} took place in 1977, at Southern Methodist
University in Dallas, where she extracted the 23rd root of a 201-digit
number in 50 seconds. It took a Univac computer 62 seconds to do the
same.

Now, her life story has inspired the Hindi-language film ``Shakuntala
Devi,''
\href{https://www.amazon.com/gp/video/detail/B08D71WWXD/ref=atv_dl_rdr?autoplay=1}{streaming
on Amazon Prime Video}. Starring the veteran Bollywood actress Vidya
Balan as Devi, the film is directed by Anu Menon and tells the story of
Devi's life from the perspective of her daughter, Anupama Banerji.
Played by Sanya Malhotra, Banerji was involved in the making of the
film.

Here are five facts about Devi you may not know.

\hypertarget{1-devi-holds-the-guinness-world-record-for-fastest-human-computation}{%
\subsubsection{\texorpdfstring{\textbf{1. Devi holds the}
\textbf{\href{https://www.guinnessworldrecords.com/world-records/67741-human-computation}{Guinness
World Record}} \textbf{for ``Fastest Human
Computation.''}}{1. Devi holds the Guinness World Record for ``Fastest Human Computation.''}}\label{1-devi-holds-the-guinness-world-record-for-fastest-human-computation}}

In 1980, she correctly multiplied two 13-digit numbers in just 28
seconds at Imperial College London. The feat, also included in
\href{https://www.nytimes.com/2013/04/24/world/asia/shakuntala-devi-human-computer-dies-in-india-at-83.html}{her
obituary}, earned her a place in the 1982 edition of the Guinness Book
of World Records. It was even more remarkable because it included the
time it took Devi to recite the 26-digit solution. (The numbers,
selected at random by a computer, were 7,686,369,774,870 and
2,465,099,745,779. The answer was 18,947,668,177,995,426,462,773,730.)

In one famous interview on the BBC in 1950
(\href{https://youtu.be/8q6ekdz30MA}{recreated in the biopic}), her
answer to a mathematical question was deemed incorrect, before the host
later acknowledged that in fact, the computer's answer was wrong and
Devi was right.

\hypertarget{2-she-was-an-ally-of-lgbtq-people}{%
\subsubsection{\texorpdfstring{\textbf{2. She was an ally of L.G.B.T.Q.
people.}}{2. She was an ally of L.G.B.T.Q. people.}}\label{2-she-was-an-ally-of-lgbtq-people}}

In 1960, Devi married Paritosh Banerji. They divorced years later, and
the 2001 documentary ``\href{https://youtu.be/jMvfkzgzK0c}{For Straights
Only}'' claimed the marriage fell apart because Banerji was gay. Devi
said in the documentary that she set out to learn more about the
challenges faced by L.G.B.T.Q. individuals to promote wider acceptance.
In 1977, she wrote ``The World of Homosexuals,'' which featured her
research findings, including interviews with same-sex couples in India
and abroad.

``It is not the individual whose sexual relations depart from the social
custom who is immoral --- but those are immoral who would penalize him
for being different,'' she wrote in the book.

\includegraphics{https://static01.nyt.com/images/2020/07/31/arts/31shakuntala-primer/merlin_68513848_a9557963-d00d-48b4-b02c-ab3ee74f5843-articleLarge.jpg?quality=75\&auto=webp\&disable=upscale}

\hypertarget{3-she-applied-her-mathematical-strength-to-a-pursuit-of-astrology}{%
\subsubsection{\texorpdfstring{\textbf{3. She applied her mathematical
strength to a pursuit of
astrology.}}{3. She applied her mathematical strength to a pursuit of astrology.}}\label{3-she-applied-her-mathematical-strength-to-a-pursuit-of-astrology}}

Perhaps because of her fascination with numbers, Devi tried her hand at
astrology, which is highly revered in Indian culture. ``Personal
Astrologer of Presidents, Prime Ministers, Royalty, Movie Stars and Top
Business Tycoons of the world is now available for Astrological
Consultations''
\href{https://www.infoqueenbee.com/2013/11/biography-of-human-computer-shakunthala.html}{a
newspaper ad} claimed at the time. She similarly
\href{https://www.nytimes.com/news/the-lives-they-lived/2013/12/21/shakuntala-devi/}{toured
the world}, according to a Times article, seeing up to 60 clients a day.
They would give her a date of birth, time of birth and birthplace, and
she would answer three questions about their lives. She also wrote a
book called ``Astrology for You.''

\hypertarget{4-she-wrote-a-novel}{%
\subsubsection{\texorpdfstring{\textbf{4. She wrote a
novel.}}{4. She wrote a novel.}}\label{4-she-wrote-a-novel}}

When Devi stopped touring the world doing shows featuring her arithmetic
prowess, she wrote several books on math and her techniques, including
``Puzzles to Puzzle You,'' ``Super Memory: It Can Be Yours'' and
``Mathability: Awaken the Math Genius in Your Child.'' But decades
prior, in 1976, Devi also wrote a crime thriller called
``\href{https://www.goodreads.com/book/show/18212970-perfect-murder}{Perfect
Murder}.'' Written entirely in the first-person, the story explores what
happens when a lawyer, motivated by greed, decides to kill his wife to
escape the marriage.

\hypertarget{5-she-once-tried-to-forge-a-path-into-politics}{%
\subsubsection{\texorpdfstring{\textbf{5. She once tried to forge a path
into
politics.}}{5. She once tried to forge a path into politics.}}\label{5-she-once-tried-to-forge-a-path-into-politics}}

In 1980, Devi ran for Parliament, the Lok Sabha, as an independent
candidate from two different localities --- Mumbai and Medak (in the
present-day state of Telangana). In Medak, her main opponent was the
former prime minister, Indira Gandhi, whom Devi had openly criticized.
Her fame, however, didn't translate into votes, and she finished ninth,
while Gandhi went on to win and became prime minister once again.

Advertisement

\protect\hyperlink{after-bottom}{Continue reading the main story}

\hypertarget{site-index}{%
\subsection{Site Index}\label{site-index}}

\hypertarget{site-information-navigation}{%
\subsection{Site Information
Navigation}\label{site-information-navigation}}

\begin{itemize}
\tightlist
\item
  \href{https://help.nytimes.com/hc/en-us/articles/115014792127-Copyright-notice}{©~2020~The
  New York Times Company}
\end{itemize}

\begin{itemize}
\tightlist
\item
  \href{https://www.nytco.com/}{NYTCo}
\item
  \href{https://help.nytimes.com/hc/en-us/articles/115015385887-Contact-Us}{Contact
  Us}
\item
  \href{https://www.nytco.com/careers/}{Work with us}
\item
  \href{https://nytmediakit.com/}{Advertise}
\item
  \href{http://www.tbrandstudio.com/}{T Brand Studio}
\item
  \href{https://www.nytimes.com/privacy/cookie-policy\#how-do-i-manage-trackers}{Your
  Ad Choices}
\item
  \href{https://www.nytimes.com/privacy}{Privacy}
\item
  \href{https://help.nytimes.com/hc/en-us/articles/115014893428-Terms-of-service}{Terms
  of Service}
\item
  \href{https://help.nytimes.com/hc/en-us/articles/115014893968-Terms-of-sale}{Terms
  of Sale}
\item
  \href{https://spiderbites.nytimes.com}{Site Map}
\item
  \href{https://help.nytimes.com/hc/en-us}{Help}
\item
  \href{https://www.nytimes.com/subscription?campaignId=37WXW}{Subscriptions}
\end{itemize}
