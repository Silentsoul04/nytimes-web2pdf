Sections

SEARCH

\protect\hyperlink{site-content}{Skip to
content}\protect\hyperlink{site-index}{Skip to site index}

\href{https://www.nytimes.com/section/nyregion}{New York}

\href{https://myaccount.nytimes.com/auth/login?response_type=cookie\&client_id=vi}{}

\href{https://www.nytimes.com/section/todayspaper}{Today's Paper}

\href{/section/nyregion}{New York}\textbar{}How an Urban Flower Farmer
Spends Her Sundays

\url{https://nyti.ms/3fnVrdL}

\begin{itemize}
\item
\item
\item
\item
\item
\end{itemize}

Advertisement

\protect\hyperlink{after-top}{Continue reading the main story}

Supported by

\protect\hyperlink{after-sponsor}{Continue reading the main story}

\hypertarget{how-an-urban-flower-farmer-spends-her-sundays}{%
\section{How an Urban Flower Farmer Spends Her
Sundays}\label{how-an-urban-flower-farmer-spends-her-sundays}}

On a quest for sunlight and soil, Christina Clum approached Brooklyn
residents about growing flowers in their yards. It worked.

\includegraphics{https://static01.nyt.com/images/2020/07/31/nyregion/31nyvirus-routine1/31nyvirus-routine1-articleLarge.jpg?quality=75\&auto=webp\&disable=upscale}

By Alyson Krueger

\begin{itemize}
\item
  July 31, 2020
\item
  \begin{itemize}
  \item
  \item
  \item
  \item
  \item
  \end{itemize}
\end{itemize}

Three years ago, Christina Clum left the corporate world to become an
urban flower farmer.

Her backyard in Kensington, Brooklyn, however, ``is the size of a
postage stamp,'' she said. ``And it doesn't get good light.'' So in
February 2018, she put the word out to other Brooklyn residents about
doing plantings in their yards. The exchange would be simple: They would
get to enjoy the flowers, and then she would cut them and sell them
through her company, \href{https://spryflowerfarm.com/}{Spry Flower
Farm}.

Ms. Clum, 51, settled on five yards. ``I had certain criteria,'' she
said. ``I didn't want to have to walk through someone's home, because it
would be weird and invasive.'' She needed sunlight and an outside water
source. Ms. Clum also made it clear that she wasn't a landscaper. ``Some
people still don't get that,'' she said.

``I have developed quite a fondness for my hosts and have attended
barbecues and plays in which they are involved,'' said Ms. Clum, who
visits her hosts' properties several times a week to dig up weeds, plant
new seeds, and water the flowers, which she sells through a subscription
service and to two stores. ``I think it definitely takes a certain type
of person to volunteer their yard and put their trust in strangers.''

Ms. Clum lives with her husband, Christopher Longworth, 51, who is an
architectural metal fabricator; their daughter Cora, 15; a dog, Ida Mae;
and three cats.

\textbf{BREAKFAST WITH IDA MAE} I get up early, around 6 a.m. If I was a
real farmer I would probably have to get up even earlier. I have one cup
of French-press coffee, and I will make myself a hearty breakfast,
because I will be outside for several hours. There are almost always
eggs, maybe some sautéed greens. On Sundays everyone in the house sleeps
in, and it's really quiet and lovely. My dog will get up. though, and if
she's staring at me, I will have to give her a walk before I leave. She
almost always wins.

\textbf{WEEDING AND WATERING} I try to visit two yards one day, three
yards the next day. Now that it's July, most everything is planted. When
I go to the yards, I am mostly weeding and watering and cutting. I do
what they call succession planting, which means planting seeds so there
is always something blooming. That way I will have flowers throughout
the season. I currently have black-eyed Susans, zinnias, cosmos,
snapdragons, and chocolate lace flowers blooming.

\includegraphics{https://static01.nyt.com/images/2020/07/31/nyregion/31nyvirus-routine2/31nyvirus-routine2-articleLarge.jpg?quality=75\&auto=webp\&disable=upscale}

\textbf{SOCIAL DISTANCING} I spend from two to four hours in each yard.
During the pandemic when everyone was at home, I saw people more often
than I did previously. Sometimes they would stay behind their door, and
I would talk to them. Sometimes they would come out and maintain a
distance. People are lonely and craving human interaction.

Image

Ms. Clum sells her flowers through a subscription service and to two
stores.Credit...Aundre Larrow for The New York Times

\textbf{BEAT THE HEAT} On Sundays I'm usually able to get to two houses
before the heat of the day comes. You don't want to cut flowers in the
heat because they wilt and can't recover. I drive between all my houses
and carry a backpack along with all my tools. I bring clipping scissors,
my kitchen knife, my hoe, planting seeds, and buckets that I put all the
cut flowers in. I also bring my handy-dandy garden hat, because I don't
want to get too much sun. I will wear masks whether I run into people or
not because I want to be responsible.

Image

``Now that it's July most everything is planted. When I go to the yards
I am mostly weeding and watering and cutting.''Credit...Aundre Larrow
for The New York Times

\textbf{HOMEWORK} On Sundays I cut flowers for two shops that I sell to,
\href{https://grdnbklyn.com/}{GRDN} and
\href{https://www.thankyouhaveagoodday.com/}{Thank You Have a Good Day},
both in Brooklyn. I typically come home and clean the flowers off, which
consists of cutting or breaking off leaves from the flowers. I'll
separate them out by type and arrange them into the numbers each shop
wants.

Image

A drop-off at GRDN, in Boerum Hill, Brooklyn.Credit...Aundre Larrow for
The New York Times

\textbf{FAMILY ERRAND} Each Sunday around noon my husband and daughter
and I go to the farmer's market on Cortelyou Road. It's a smaller
market, not a big one, so I find it to be more manageable. We spend
about an hour getting food for the weekly meals. We get fresh
vegetables, fresh fish, whatever looks good.

\textbf{DELIVERIES} Around 2 p.m. I'll take the cut flowers to the two
stores. One of them usually has a specific order for how much she wants
to get, and the other is less specific and will take what looks good. I
used to take my buckets of flowers on the subway, but now I am kind of
avoiding the subway because of coronavirus, so I drive.

Image

Credit...Aundre Larrow for The New York Times

\textbf{BACKYARD HANG} The rest of Sunday is pretty relaxed. We will sit
in the backyard and maybe have a glass of wine and discuss what is
happening in the world. We talk a lot about what will happen with school
next year or what is going on with the presidential election. We've kind
of stopped talking about the absurdity of what is going on with the
pandemic; it's too much in your face every day.

\textbf{FATIGUE SETS IN} I'll go to bed around 11, but sometimes I fall
asleep watching television before that. I feel the physical stress of
being a farmer. I am in positions when I am weeding that aren't normal
for the body. Or I'll stretch my legs in a contorted way to not knock
over flowers. It's not terrible though, and my body definitely loosens
up by the next day.

Sunday Routine readers can follow Christina Clum on Instagram
@clumbleweed or @spryflowerfarm.

Advertisement

\protect\hyperlink{after-bottom}{Continue reading the main story}

\hypertarget{site-index}{%
\subsection{Site Index}\label{site-index}}

\hypertarget{site-information-navigation}{%
\subsection{Site Information
Navigation}\label{site-information-navigation}}

\begin{itemize}
\tightlist
\item
  \href{https://help.nytimes.com/hc/en-us/articles/115014792127-Copyright-notice}{©~2020~The
  New York Times Company}
\end{itemize}

\begin{itemize}
\tightlist
\item
  \href{https://www.nytco.com/}{NYTCo}
\item
  \href{https://help.nytimes.com/hc/en-us/articles/115015385887-Contact-Us}{Contact
  Us}
\item
  \href{https://www.nytco.com/careers/}{Work with us}
\item
  \href{https://nytmediakit.com/}{Advertise}
\item
  \href{http://www.tbrandstudio.com/}{T Brand Studio}
\item
  \href{https://www.nytimes.com/privacy/cookie-policy\#how-do-i-manage-trackers}{Your
  Ad Choices}
\item
  \href{https://www.nytimes.com/privacy}{Privacy}
\item
  \href{https://help.nytimes.com/hc/en-us/articles/115014893428-Terms-of-service}{Terms
  of Service}
\item
  \href{https://help.nytimes.com/hc/en-us/articles/115014893968-Terms-of-sale}{Terms
  of Sale}
\item
  \href{https://spiderbites.nytimes.com}{Site Map}
\item
  \href{https://help.nytimes.com/hc/en-us}{Help}
\item
  \href{https://www.nytimes.com/subscription?campaignId=37WXW}{Subscriptions}
\end{itemize}
