Sections

SEARCH

\protect\hyperlink{site-content}{Skip to
content}\protect\hyperlink{site-index}{Skip to site index}

\href{https://www.nytimes.com/section/obituaries}{Obituaries}

\href{https://myaccount.nytimes.com/auth/login?response_type=cookie\&client_id=vi}{}

\href{https://www.nytimes.com/section/todayspaper}{Today's Paper}

\href{/section/obituaries}{Obituaries}\textbar{}Overlooked No More:
Roland Johnson, Who Fought to Shut Down Institutions for the Disabled

\url{https://nyti.ms/2BLiv8g}

\begin{itemize}
\item
\item
\item
\item
\item
\end{itemize}

Advertisement

\protect\hyperlink{after-top}{Continue reading the main story}

Supported by

\protect\hyperlink{after-sponsor}{Continue reading the main story}

\hypertarget{overlooked-no-more-roland-johnson-who-fought-to-shut-down-institutions-for-the-disabled}{%
\section{Overlooked No More: Roland Johnson, Who Fought to Shut Down
Institutions for the
Disabled}\label{overlooked-no-more-roland-johnson-who-fought-to-shut-down-institutions-for-the-disabled}}

He survived 13 years of neglect and abuse, including sexual assault, at
the notorious Pennhurst State School and Hospital outside Philadelphia
before emerging as a champion for the disabled.

\includegraphics{https://static01.nyt.com/images/2020/08/03/multimedia/03overlooked-johnson-01/00overlooked-johnson-01-articleLarge.jpg?quality=75\&auto=webp\&disable=upscale}

By Glenn Rifkin

\begin{itemize}
\item
  July 31, 2020
\item
  \begin{itemize}
  \item
  \item
  \item
  \item
  \item
  \end{itemize}
\end{itemize}

\emph{Overlooked is a series of obituaries about remarkable people whose
deaths, beginning in 1851, went unreported in The Times. This latest
installment is part of a series exploring how the Americans With
Disabilities Act has shaped modern life for disabled
people.}\href{https://www.nytimes.com/2020/07/10/reader-center/disability-america-questions.html}{\emph{Share
your stories}} \emph{or email us at
\href{mailto:ada@nytimes.com}{\nolinkurl{ada@nytimes.com}}.}

In 1958, when Roland Johnson was 12, his parents sent him to the
Pennhurst State School and Hospital outside Philadelphia. There he would
spend 13 tormented years living through the nightmare of
institutionalization that was commonplace in mid-20th-century America.

Terrified and confused, Roland, who had an intellectual disability,
quickly discovered the
\href{https://timesmachine.nytimes.com/timesmachine/1983/11/04/017104.html?pageNumber=38}{inhumane
realities of Pennhurst,} including neglect, beatings and sexual assault.
And as a Black child, he encountered the toxic racism roiling life both
outside and within the institution's walls.

``After that long ride up there, it was just horrible,'' Johnson wrote
of his arrival at Pennhurst in a posthumously published autobiography,
``Lost in a Desert World'' (2002, with Karl Williams). He described
himself as having been ``lost and lonely,'' as if ``in a desert world.''

``I thought I would be there forever,'' he added.

But Johnson did get out, and would see his family again. More
remarkably, he would survive a prolonged and difficult transition to the
outside world and emerge as a pioneering champion for the disabled.
Through speeches across the country and in courtroom testimony, he
played a significant part in the shutting down of Pennhurst in 1987. He
also assisted in the release of countless people from other state
institutions. By demonstrating that the developmentally disabled could
speak up for themselves, he was at the forefront of an emerging
self-advocacy movement that would take hold in the Philadelphia area in
the 1970s.

\includegraphics{https://static01.nyt.com/images/2020/08/03/multimedia/03overlooked-johnson-02/00overlooked-johnson-02-articleLarge.jpg?quality=75\&auto=webp\&disable=upscale}

As president of the Philadelphia chapter of Speaking for Ourselves, a
Pennsylvania organization that later expanded nationally, Johnson became
a spokesman and a mentor for others who had been institutionalized,
including Deborah Robinson, who succeeded Johnson as president.

``He was a strong and powerful speaker,'' Robinson said in an interview,
``who believed in people getting out of institutions, living in the
community and having their own voice.''

Johnson \href{https://www.youtube.com/watch?v=zFI7u6V_GvA}{began every
speech with his mantra}: ``Who's in control?'' He urged his audiences
not to feel trapped by others dictating every facet of their existence.
``The only way to break that barrier is to tell people that you are in
control over your own life and in your own ways,'' he declared.

When Johnson died on Aug. 29, 1994, at 48 after being trapped in a house
fire, he left an indelible legacy: his work on behalf of one of the most
disenfranchised segments of society. He became president of the board of
Speaking for Ourselves and a board member of
\href{https://www.sabeusa.org/}{Self Advocates Becoming Empowered}, a
national organization. When President George H.W. Bush
\href{https://www.nytimes.com/interactive/2020/us/disability-ADA-30-anniversary.html}{signed
the landmark Americans with Disabilities Act} on July 26, 1990, on the
South Lawn of the White House, Johnson was there, part of a delegation
that had arrived to witness that historic moment.

Image

Johnson in 1993 offering an award to President George H.W. Bush for his
work to ban discrimination against people with disabilities.Credit...AB
Historic/Alamy

``It is impossible to know the courage of a man who had slung at him the
worst labels and insults imaginable, who suffered abuse and neglect, and
who belonged to a group totally discounted by society,'' Nancy Thaler,
the former deputy secretary of the Pennsylvania Office of Developmental
Programs, wrote in an open letter after his death, ``but who
nevertheless stood up in public to speak for himself and his people.
Roland gave voice to the people. Roland made us listen. Roland changed
how we think about disabilities.''

\href{http://www.eoutcome.org/default.aspx?pg=326}{James W. Conroy}, a
medical sociologist who worked on the litigation that led to the closing
of Pennhurst, worked closely with Johnson in overseeing
\href{https://aspe.hhs.gov/basic-report/pennhurst-longitudinal-study-combined-report-five-years-research-and-analysis}{studies}
of what happens to people when they leave institutions.

``He motivated his friends and others at Speaking for Ourselves, and he
really pushed the movement toward freedom,'' Conroy said in a phone
interview. ``His was a fantastic contribution unlike any I've ever
seen.''

Roland Johnson was born in Philadelphia on Sept. 14, 1945, to Grace and
Roy Johnson. His father was an auto mechanic, his mother a housekeeper.
Roland's twin, Rosemary, died in infancy. With nine children,life was a
struggle for the Johnson family. Because both parents had to work, the
older children had to care for the younger ones.

When it became clear that Roland had been born with an intellectual
disability, his parents were urged to put the baby in an institution,
the norm at the time. But Roland's parents refused to do that and tried
to raise him at home.

``His family failed him,'' LaVerne Cheatham, his closest sibling, said
in an interview. ``It was a sad situation. All of us, including me,
didn't give him what he needed.'' She said of her mother, ``There wasn't
a day that she didn't worry about him.''

With public schools unable or unwilling to accommodate him, he stayed at
home. In his book, Johnson describes himself as having had an insatiable
appetite and a penchant for stealing food from stores and running away.
His mother, he wrote, ``didn't know how to handle me.''

To punish him, he said, she'd first heat a knife on a stove. ``Then she
put it on my hand and burnt me with it,'' he wrote. ``And then she had
an iron and she whipped me with the iron cord and made bruises all over
my back. I don't blame her for it --- I probably needed it, a licking.
My mother tried but she couldn't take it anymore.''

His parents turned to the Philadelphia children's court for help.
Instructed to send him to a state institution, they chose Pennhurst,
originally called the Eastern State Institution for the Feeble-Minded
and Epileptic when it opened in 1908.

``This is it for me,'' Johnson remembered thinking. ``I guess I will be
locked up in there, in a big cellar with locks.''

At Pennhurst he was traumatized by the emotional and physical abuse. He
was ridiculed: ``You're stupid. You're crazy. Dummy, Dopey, don't know
nothing.'' He witnessed patients being beaten by other patients with
broom handles and hid under the bed to avoid the same fate. He saw a
young patient drink a bottle of liquid Thorazine, an antipsychotic, and
die of an overdose. A young friend was strangled with a rope and left to
die in a filthy, rat-infested punishment ward. In his frustration and
anger, Johnson broke windows, for which he was locked in the punishment
ward and forced to scrub its walls and floors.

The sexual abuse began early on. ``All this stuff happened late at
night,'' he wrote, adding, ``They did awful things to me.'' From
multiple rapes, he said, he contracted sexually transmitted diseases.
Years after he left Pennhurst he learned that he was H.I.V. positive.

Because the institution was severely understaffed and overpopulated,
Johnson and others were forced to do laundry and maintenance and care
for the young children and babies. ``Nobody got paid,'' he wrote. ``They
would work, work, work.''

Pennhurst was once called ``the shame of the nation,'' according to
\href{http://www.preservepennhurst.org/}{Preserve Pennhurst}, a website
dedicated to preserving the lessons from its dark legacy.

In 1968, \href{https://www.broadcastpioneers.com/billbaldini.html}{Bill
Baldini}, a Philadelphia television news reporter, produced a six-part
exposé about Pennhurst called
``\href{http://www.preservepennhurst.org/default.aspx?pg=26}{Suffer the
Little Children}.'' Johnson was one of the children he interviewed.

``We ship them 25 miles out of town to an institution and forget them,
while they decay from neglect,'' Baldini said in the introduction to the
series. ``Zoos spend more on their wild animals than Pennsylvania spends
on its 2,800 patients at Pennhurst.''

The series resulted in lawsuits that led to Pennhurst's closing. Johnson
was released in 1971.

Afterward he stayed with his family, but the old tensions flared up
anew, and before long he moved out, rooming in boardinghouses and
holding low-paying jobs. At one boardinghouse he got into a fight with
another former patient and was arrested. ``The police threw me against
the wall and threw me in the paddy wagon, and it hurt my head,'' he
recalled in his book.

A bicycle accident and a series of illnesses landed him in a hospital.
He eventually joined a psychiatric day program, and his life began to
improve.

Johnson heard about Speaking for Ourselves in the early 1980s while
working as a janitor. He went to a conference and stood in the back to
observe. Surprising himself, he spoke up. ``We're tired of the old
system,'' he recalled saying. ``We need to make things change, to make
things happen.''

Mark Friedman, who helped found the organization, saw something in
Johnson.

``He found great camaraderie with other disabled people, who accepted
him and loved him,'' Mr. Friedman said in a phone interview. ``To this
day, people still talk about him and share stories and still look up to
Roland --- and it's been decades since he passed.''

Advertisement

\protect\hyperlink{after-bottom}{Continue reading the main story}

\hypertarget{site-index}{%
\subsection{Site Index}\label{site-index}}

\hypertarget{site-information-navigation}{%
\subsection{Site Information
Navigation}\label{site-information-navigation}}

\begin{itemize}
\tightlist
\item
  \href{https://help.nytimes.com/hc/en-us/articles/115014792127-Copyright-notice}{©~2020~The
  New York Times Company}
\end{itemize}

\begin{itemize}
\tightlist
\item
  \href{https://www.nytco.com/}{NYTCo}
\item
  \href{https://help.nytimes.com/hc/en-us/articles/115015385887-Contact-Us}{Contact
  Us}
\item
  \href{https://www.nytco.com/careers/}{Work with us}
\item
  \href{https://nytmediakit.com/}{Advertise}
\item
  \href{http://www.tbrandstudio.com/}{T Brand Studio}
\item
  \href{https://www.nytimes.com/privacy/cookie-policy\#how-do-i-manage-trackers}{Your
  Ad Choices}
\item
  \href{https://www.nytimes.com/privacy}{Privacy}
\item
  \href{https://help.nytimes.com/hc/en-us/articles/115014893428-Terms-of-service}{Terms
  of Service}
\item
  \href{https://help.nytimes.com/hc/en-us/articles/115014893968-Terms-of-sale}{Terms
  of Sale}
\item
  \href{https://spiderbites.nytimes.com}{Site Map}
\item
  \href{https://help.nytimes.com/hc/en-us}{Help}
\item
  \href{https://www.nytimes.com/subscription?campaignId=37WXW}{Subscriptions}
\end{itemize}
