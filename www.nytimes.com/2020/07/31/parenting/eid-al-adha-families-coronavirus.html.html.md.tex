Sections

SEARCH

\protect\hyperlink{site-content}{Skip to
content}\protect\hyperlink{site-index}{Skip to site index}

\href{https://www.nytimes.com/section/parenting}{Parenting}

\href{https://myaccount.nytimes.com/auth/login?response_type=cookie\&client_id=vi}{}

\href{https://www.nytimes.com/section/todayspaper}{Today's Paper}

\href{/section/parenting}{Parenting}\textbar{}How Families Are
Celebrating Eid al-Adha This Year

\url{https://nyti.ms/3ggvqyc}

\begin{itemize}
\item
\item
\item
\item
\item
\end{itemize}

\href{https://www.nytimes.com/spotlight/at-home?action=click\&pgtype=Article\&state=default\&region=TOP_BANNER\&context=at_home_menu}{At
Home}

\begin{itemize}
\tightlist
\item
  \href{https://www.nytimes.com/2020/07/28/books/time-for-a-literary-road-trip.html?action=click\&pgtype=Article\&state=default\&region=TOP_BANNER\&context=at_home_menu}{Take:
  A Literary Road Trip}
\item
  \href{https://www.nytimes.com/2020/07/29/magazine/bored-with-your-home-cooking-some-smoky-eggplant-will-fix-that.html?action=click\&pgtype=Article\&state=default\&region=TOP_BANNER\&context=at_home_menu}{Cook:
  Smoky Eggplant}
\item
  \href{https://www.nytimes.com/2020/07/27/travel/moose-michigan-isle-royale.html?action=click\&pgtype=Article\&state=default\&region=TOP_BANNER\&context=at_home_menu}{Look
  Out: For Moose}
\item
  \href{https://www.nytimes.com/interactive/2020/at-home/even-more-reporters-editors-diaries-lists-recommendations.html?action=click\&pgtype=Article\&state=default\&region=TOP_BANNER\&context=at_home_menu}{Explore:
  Reporters' Obsessions}
\end{itemize}

Advertisement

\protect\hyperlink{after-top}{Continue reading the main story}

Supported by

\protect\hyperlink{after-sponsor}{Continue reading the main story}

\hypertarget{how-families-are-celebrating-eid-al-adha-this-year}{%
\section{How Families Are Celebrating Eid al-Adha This
Year}\label{how-families-are-celebrating-eid-al-adha-this-year}}

You can make the festival feel special for kids, without having to leave
home.

\includegraphics{https://static01.nyt.com/images/2020/08/30/multimedia/30par-eid-al-adha-pandemic1/30par-eid-al-adha-pandemic1-articleLarge.jpg?quality=75\&auto=webp\&disable=upscale}

By Tasmiha Khan

\begin{itemize}
\item
  July 31, 2020
\item
  \begin{itemize}
  \item
  \item
  \item
  \item
  \item
  \end{itemize}
\end{itemize}

As the coronavirus pandemic continues to force changes to daily
activities and rituals around the world, Muslim families are finding new
ways to celebrate Eid al-Adha.

One of Islam's two main festivals, Eid al-Adha traditionally begins in
the morning with the Eid prayer at the nearest mosque or an open field
and continues with feasts, visits and an exchanging of gifts among
relatives and friends. But widespread coronavirus lockdowns mean that
those kinds of gatherings aren't feasible in many communities.

Afshan Malik, a development director with the nonprofit educational
organization Rabata, is still including gifts, decorations and special
treats typically reserved for this day, even if they are not able to
enjoy time with their community in Houston. She is encouraging her five
children, ages 3, 5, 7, 9 and 11, to learn how to pray the Eid prayer at
home: ``We want to make sure our family stays connected to the
sacredness and historical significance.''

Eid al-Adha, also known as the Festival of Sacrifice, commemorates
Ibrahim's willingness to sacrifice his first-born son, Ismael, as an act
of obedience to God. Muslims believe that as Ibrahim was about to
fulfill God's command, God offered an animal to be slaughtered in place
of Ismael. The revered holiday takes place after the completion of the
pilgrimage known as the hajj.

\includegraphics{https://static01.nyt.com/images/2020/08/30/multimedia/30par-eid-al-adha-pandemic7/merlin_175170807_6bdeab47-4515-48b2-a347-dc78f8200b7d-articleLarge.jpg?quality=75\&auto=webp\&disable=upscale}

As part of their observances, Malik and her family will use Rabata's
worship program called Pilgrims at Home, an alternative for those who
\href{https://www.nytimes.com/2020/06/23/world/middleeast/hajj-pilgrimage-canceled.html}{cannot
complete the hajj} this year because of the coronavirus.

``Being quarantined at home does have its drawbacks, but it is a time
where we can be more self-reflective and reach out in different ways,''
she said. ``We aim to pay tribute to the spirit of Eid by spending time
calling relatives or members of the community who may not have family
around, distributing food to neighbors, and fulfilling the spiritual
components of the day despite not being in a community space.''

Lail Hossain of Dallas normally hosts an open house brunch for more than
125 guests. Hossain and her husband prepare for three weeks for a feast
that includes traditional Bangladeshi food like paya (beef leg soup, a
Bangladeshi delicacy), eggs, mixed vegetables, saffron semolina halwa,
chicken puff pastry, and Mughlai paratha, a crispy pastry filled with
ground beef, eggs, sliced onion, cilantro and spices.

Image

Eid plates were set out before lunch.Credit...Cooper Neill for The New
York Times

To commemorate the story of Ibrahim's sacrifice, Hossain typically takes
her 9-year-old daughter, Rida, to a local meat shop where a third of the
meat she buys is distributed to the less fortunate, as Islamic law
outlines, with the rest of the meat used for their brunch.

But with coronavirus cases rising in Texas, Hossain, the founder of an
Islamic décor and gift business, decided instead to bake Eid
al-Adha-themed cookies in the shapes of a masjid, camel and lamb and
plans to share them with firefighters and other essential service
providers in the community. She's decorating her home with Eid garlands,
masjid-shaped lights and festive lanterns.

``We want to instill the love of Islam in my daughter and make the
Islamic festivals a real part of her life,'' Hossain said. ``We want to
create memories, and Covid-19 isn't going to stop us from doing that.''

Image

Rida's parents built a tent as a teaching tool to educate Rida on the
different rituals of Hajj, including the tradition of staying in the
``Tent City'' of Mina Valley.Credit...Cooper Neill for The New York
Times

Hossain is also making a Kaaba and decorating a tent in her home's game
room to teach Rida about the tent city in Mina, one of the spots that
Muslims visit while in hajj.

Dr. Fariha Rub, a hospitalist and mother of two from Naperville, Ill.,
plans to re-enact her pilgrimage from two years ago, when she decided
not to take her then 3-year-old son.

``Setting up a makeshift Kaaba and creating the various stations of hajj
will help to explain how the rites are performed,'' Dr. Rub said. ``We
have also built a play mosque,'' for our son, who is now 5, and
1-year-old daughter. ``This helps them still feel connected to the place
we frequently visited.''

Image

Cookies and cream puffs were put out for Eid.Credit...Cooper Neill for
The New York Times

Image

Lachcha shemai, a sweet traditional pudding, was prepared for the Eid
lunch.Credit...Cooper Neill for The New York Times

Dr. Rub said the hajj reignited her faith and led her to start a Muslim
sisterhood initiative, which is hosting a drive-through parade, complete
with goodie bags for kids, to capture the essence of Eid.

In spite of the pandemic, the eagerness to make Eid special for children
and their families echoes throughout the Muslim community.

``At its core, Eid al-Adha commemorates Ibrahim's willingness to put his
love for God before all else,'' said Dr. Mohammad Hussaini, a
pathologist and founder of Pureway.org, an Islamic spirituality website.
He is also a father of four kids, ages 19, 17, 14 and 1 month, from
Tampa.

Image

``We want to create memories, and Covid-19 isn't going to stop us from
doing that,'' Hossain said.Credit...Cooper Neill for The New York Times

Dr. Hussaini and his family are wearing masks and staying physically
distant from neighbors and friends, while still honoring traditions such
as special meals and gift exchanges. With many in the community avoiding
the typical large gatherings of the holiday, it has left more time to
focus on worship and spirituality. ``It is our connection to God that
uplifts the soul above all else, above the tempest doldrums of life,''
he said. ``This is Ibrahim's gift and his legacy.''

\begin{center}\rule{0.5\linewidth}{\linethickness}\end{center}

Tasmiha Khan is a journalist based in Illinois. Follow
her\href{https://twitter.com/CraftOurStory}{@CraftOurStory}.

Advertisement

\protect\hyperlink{after-bottom}{Continue reading the main story}

\hypertarget{site-index}{%
\subsection{Site Index}\label{site-index}}

\hypertarget{site-information-navigation}{%
\subsection{Site Information
Navigation}\label{site-information-navigation}}

\begin{itemize}
\tightlist
\item
  \href{https://help.nytimes.com/hc/en-us/articles/115014792127-Copyright-notice}{©~2020~The
  New York Times Company}
\end{itemize}

\begin{itemize}
\tightlist
\item
  \href{https://www.nytco.com/}{NYTCo}
\item
  \href{https://help.nytimes.com/hc/en-us/articles/115015385887-Contact-Us}{Contact
  Us}
\item
  \href{https://www.nytco.com/careers/}{Work with us}
\item
  \href{https://nytmediakit.com/}{Advertise}
\item
  \href{http://www.tbrandstudio.com/}{T Brand Studio}
\item
  \href{https://www.nytimes.com/privacy/cookie-policy\#how-do-i-manage-trackers}{Your
  Ad Choices}
\item
  \href{https://www.nytimes.com/privacy}{Privacy}
\item
  \href{https://help.nytimes.com/hc/en-us/articles/115014893428-Terms-of-service}{Terms
  of Service}
\item
  \href{https://help.nytimes.com/hc/en-us/articles/115014893968-Terms-of-sale}{Terms
  of Sale}
\item
  \href{https://spiderbites.nytimes.com}{Site Map}
\item
  \href{https://help.nytimes.com/hc/en-us}{Help}
\item
  \href{https://www.nytimes.com/subscription?campaignId=37WXW}{Subscriptions}
\end{itemize}
