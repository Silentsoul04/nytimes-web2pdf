Sections

SEARCH

\protect\hyperlink{site-content}{Skip to
content}\protect\hyperlink{site-index}{Skip to site index}

\href{https://www.nytimes.com/section/realestate}{Real Estate}

\href{https://myaccount.nytimes.com/auth/login?response_type=cookie\&client_id=vi}{}

\href{https://www.nytimes.com/section/todayspaper}{Today's Paper}

\href{/section/realestate}{Real Estate}\textbar{}A Rockaway Life

\url{https://nyti.ms/2DmtwgV}

\begin{itemize}
\item
\item
\item
\item
\item
\item
\end{itemize}

\href{https://www.nytimes.com/spotlight/at-home?action=click\&pgtype=Article\&state=default\&region=TOP_BANNER\&context=at_home_menu}{At
Home}

\begin{itemize}
\tightlist
\item
  \href{https://www.nytimes.com/2020/08/03/well/family/the-benefits-of-talking-to-strangers.html?action=click\&pgtype=Article\&state=default\&region=TOP_BANNER\&context=at_home_menu}{Talk:
  To Strangers}
\item
  \href{https://www.nytimes.com/2020/08/01/at-home/coronavirus-make-pizza-on-a-grill.html?action=click\&pgtype=Article\&state=default\&region=TOP_BANNER\&context=at_home_menu}{Make:
  Grilled Pizza}
\item
  \href{https://www.nytimes.com/2020/07/31/arts/television/goldbergs-abc-stream.html?action=click\&pgtype=Article\&state=default\&region=TOP_BANNER\&context=at_home_menu}{Watch:
  'The Goldbergs'}
\item
  \href{https://www.nytimes.com/interactive/2020/at-home/even-more-reporters-editors-diaries-lists-recommendations.html?action=click\&pgtype=Article\&state=default\&region=TOP_BANNER\&context=at_home_menu}{Explore:
  Reporters' Google Docs}
\end{itemize}

Advertisement

\protect\hyperlink{after-top}{Continue reading the main story}

Supported by

\protect\hyperlink{after-sponsor}{Continue reading the main story}

Domestic Lives

\hypertarget{a-rockaway-life}{%
\section{A Rockaway Life}\label{a-rockaway-life}}

Rockaway Beach has a land's-end bacchanalian spirit, but it is also a
place where you can get to know people just by being there.

\href{https://www.nytimes.com/slideshow/2020/07/31/realestate/an-urban-beach-life.html}{}

\hypertarget{an-urban-beach-life}{%
\subsection{An Urban Beach Life}\label{an-urban-beach-life}}

15 Photos

View Slide Show ›

\includegraphics{https://static01.nyt.com/images/2020/08/02/realestate/31DOMESTIC-ROCKAWAY-slide-OJH5/31DOMESTIC-ROCKAWAY-slide-OJH5-articleLarge.jpg?quality=75\&auto=webp\&disable=upscale}

Stefano Ukmar for The New York Times

\href{https://www.nytimes.com/by/diane-cardwell}{\includegraphics{https://static01.nyt.com/images/2009/12/28/timestopics/topics-cardwell-pic/topics-cardwell-pic-thumbLarge-v2.jpg}}

By \href{https://www.nytimes.com/by/diane-cardwell}{Diane Cardwell}

\begin{itemize}
\item
  July 31, 2020
\item
  \begin{itemize}
  \item
  \item
  \item
  \item
  \item
  \item
  \end{itemize}
\end{itemize}

The first morning that I woke up in
\href{https://www.nytimes.com/2016/06/05/realestate/rockaway-beach-city-life-with-sand-and-surf.html}{Rockaway
Beach} after moving from Brooklyn eight years ago, I walked barefoot
with my coffee to the boardwalk just because I could. It was April and
my feet were stinging from the cold by the time I got there, but the
smell of salt air, the warmth of the sun and the sight of cormorants
diving for fish in the slate blue waves more than made up for it.

``I can't believe I really get to live here,'' I remember thinking.

At the time, I was still reconstructing myself after a divorce and had
thought the Rockaways, which stretch beneath the southern edges of
Brooklyn and Queens, were too far away from my job in Midtown as a
reporter at The New York Times and my regular haunts in Lower Manhattan
and Brooklyn to live there full time. But as I found myself making
frequent pilgrimages on the A-train in a hapless, middle-aged quest to
learn to surf, the idea of having an urban beach life took hold.

It was a kind of carefree idyll I was after, a way to escape the
pressures and strictures of my workaday life without giving it up --- I
had a career and a mortgage to sustain. But, after almost a decade of
living in Rockaway --- and experiencing two disasters, first Hurricane
Sandy and now coronavirus --- I realize I ended up with something even
more important: membership in a real community I hadn't realized I was
looking for.

\includegraphics{https://static01.nyt.com/images/2020/08/02/realestate/31DOMESTIC-ROCKAWAY-slide-PGN7/31DOMESTIC-ROCKAWAY-slide-PGN7-articleLarge.jpg?quality=75\&auto=webp\&disable=upscale}

In my search for a beach home, I considered a few options --- a bungalow
in Far Rockaway where the eastern end of the peninsula hits Nassau
County; a boxy two-family house a few doors down from a friend's; a
clean, modern oceanfront condo across from the main surf break --- but
none quite fit my finances. Determined to put down roots, though, I
reached out to an agent I'd met.

QUEENS

NEW YorK

Jamaica

Bay

Wildlife

Refuge

BROOKLYN

nassau

BELT pkwy.

JAMAICA

BAY

ROCKAWAY BEACH

ROCKAWAY

PENINSULA

ATLANTIC OCEAN

2 miles

By The New York Times

It was late 2011 and sales activity was still nowhere near where it had
been during the bubble that preceded the financial crisis. In 2009, for
instance, the Rockaways logged 617 home sales with an average value of
\$425,259, according to data compiled by Jonathan Miller, president of
the appraisal firm Miller Samuel. Activity fell over the next two years,
reaching 433 sales with an average value of \$366,870 in 2010 and 360
sales with an average value of \$405,904 the year after.

There wasn't much on the market, but the agent showed me a place I fell
for on the spot: a two-story Dutch Colonial Revival built in 1913 as a
summer cottage that was listed for \$265,000. Not even 650 square feet
in all, it was tucked away on an alley that ran in from the sidewalk a
half block from the ocean and abutted a community garden.

Image

Rockaway Beach still feels like a small town --- in part because of how
close the houses are, like these on Beach 93rd Street between Holland
Avenue and Shore Front Parkway.Credit...Stefano Ukmar for The New York
Times

I was captivated by its location, architecture and old-school quirks,
like its gambrel roof and the unmilled tree trunks in the basement that
had once propped up the building. But I was also smitten with the way of
life it embodied, one that dated back to some of the earliest New
Yorkers.

Long before European colonists arrived, the Rockaway peninsula was
already a summer destination: the Lenape people who populated the area
would spend the warm months at the beach fishing and gathering oysters
before moving inland for fall and winter.

The peninsula eventually grew to become the city's premier resort,
attracting Gilded Age high society like Astors and Vanderbilts, literati
like Henry Wadsworth Longfellow, Washington Irving and Walt Whitman, and
regular folk looking for relief from the sweltering city. At its height,
millions of visitors thronged to the bathhouses and honky-tonk
establishments that lined the boardwalk and to the amusements that
stretched from the ocean to Jamaica Bay.

Those days are long gone, but that land's-end bacchanalian spirit still
pervades the place, especially in summer when the crowds of raucous
day-trippers arrive --- DFDs, those down-for-the-day, as they're known.
Over the years, I've learned to stock up and hide out at home with my
boyfriend on weekends, not venturing out for a meal on the boardwalk
until Monday --- T.G.I.M. we always say.

Image

Even in a pandemic, the beachfront concessions have opened, offering
visitors and residents alike new food options for the
summer.Credit...Stefano Ukmar for The New York Times

Even this year, and especially since swimmers have been allowed back
into the water, the influx of excited young people, many decked out in
colorful and revealing beachwear, has transformed the boardwalk ---
where the concessions have opened --- into a kaleidoscope of bare chests
and bottoms, tattoos, piercings and hair dyed all manner of colors.

As a year-rounder, I found it wasn't so difficult to penetrate the
social scene --- Rockaway is a place where you can get to know people
just by being there. I made friends at first with my neighbors and the
other regulars on the boardwalk checking out the waves at sunrise.
Later, I'd come to know people from the water, where surfers would offer
me advice in the waves and invitations to parties on land.

I saw that same generosity after Sandy, when surfers from all over
showed up in Rockaway with shovels to help dig out basements ---
including mine, which had been completely submerged --- and clear rubble
from the flood. And I've seen it since, in the regular beach cleanups,
mutual pandemic aid networks and still-too-frequent rescues of swimmers
drowning in our treacherous waters.

I've also come to anticipate the seasonal rhythms of a part of the city
that's particularly vulnerable to weather. Spring and summer bring not
only lush greenery and flowers to neighborhood streets and yards, and
vegetables to plots in the community gardens, but also new inhabitants
to the marine wilderness surrounding the peninsula.

I still marvel at the giant osprey nests and flocks of egrets and
multicolored ducks along the waterways of the Jamaica Bay Wildlife
Refuge. On the Atlantic side, it's seals resting in the sand, dolphins
playing in the waves, seabirds divebombing the ocean and humpback whales
erupting from the surface.

Fall often brings the big, powerful swells surfers more adept than I
look forward to all year, the product of hurricane-season storms. I
can't see the water from my house, but I can smell and hear it on those
days as the waves heave up off the sandbars and rumble onto the shore,
the briny scent of their spray wafting down my block.

The whole neighborhood seems to crackle with excitement. Streets fill
with surfers running toward the ocean, boards under their arms. Local
hangouts ring with amped-up chatter about waves made or lost --- ``Dude,
did you see that barrel? It was sick!'' --- or what the delicate
interplay of pressure systems, wind, tides and daylight will bring the
next day.

Winter, at least when I first arrived, brought a raw peacefulness with
the summer visitors gone, wind sweeping sand off the beach and the ocean
turning a foreboding charcoal gray. Long commutes on the A train in
darkness also heightened the sense of isolation from the convenience and
bustle of more heavily populated districts closer to the center of the
city.

Image

Luxury condos rise on Beach 116th Street in Rockaway Park near the
ocean.Credit...Stefano Ukmar for The New York Times

But even that's changed in the years since Sandy, as dilapidated
buildings and derelict lots give way to new homes and apartment
buildings and evermore year-round establishments. Luxury condo buildings
are springing up on Beach 116th Street, a commercial corridor in
Rockaway Park, spurring new food offerings and an Orangetheory Fitness.
On Beach 108th Street near the shore, the Rockaway Hotel, a boutique
outfit complete with pool, rooftop bar and spa, is planning to open in
August, with rooms starting at \$300 a night.

The Arverne by the Sea project, part of a vast urban renewal area
designated in the 1960s, is nearing the final phases of development,
bringing dozens of new apartments and businesses --- a wine shop, an
espresso bar, a cafe and a well-regarded barbecue joint --- to an
already burgeoning area.

There's a new eco-friendly affordable housing complex in Edgemere
designed to meet passive house standards for energy efficiency and with
features to help withstand flooding and storm damage. And a nearly \$300
million redevelopment plan for downtown Far Rockaway is underway. Though
home sales activity has never rebounded to its 2009 height, values have
generally been on the rise in recent years, with 509 sales in 2019 at an
average price of \$536,668.

Even so, Rockaway Beach still feels like a small town --- the kind of
neighborhood where you look around when the door of the bar or
restaurant opens because it might be someone you know. Many of those
places are operated by people who live in or have a longstanding
connection to the area, and they have scrambled to stay open while
adjusting to evolving Covid-19 regulations. That, along with the
relatively low population density and open shoreline, has made Rockaway
Beach a less difficult place to weather the pandemic than some others.

Image

The jetty at Beach 90th Street marks one of the city's legal surfing
beaches. Another is in Arverne near Beach 69th Street.Credit...Stefano
Ukmar for The New York Times

``I wish I was a teenager now in Rockaway because there's things to
do,'' said Rashida Voorhies, a co-owner of Sayra's Wine Bar, who grew up
in the neighborhood at a time when she said there wasn't much activity
beyond sitting on the beach jetty or playing in the sand.

``You can actually hang out in your community --- you don't have to get
on the train or the bus,'' she said. ``I heard a customer say, `I don't
leave Rockaway on the weekend.' Usually, it was the other way around.''

Diane Cardwell is the author of ``Rockaway: Surfing Headlong into a New
Life.''

For weekly email updates on residential real estate news,
\href{http://www.nytimes.com/newsletters/realestate/}{sign up here}.
Follow us on Twitter:
\href{https://twitter.com/nytrealestate}{@nytrealestate}.

Advertisement

\protect\hyperlink{after-bottom}{Continue reading the main story}

\hypertarget{site-index}{%
\subsection{Site Index}\label{site-index}}

\hypertarget{site-information-navigation}{%
\subsection{Site Information
Navigation}\label{site-information-navigation}}

\begin{itemize}
\tightlist
\item
  \href{https://help.nytimes.com/hc/en-us/articles/115014792127-Copyright-notice}{©~2020~The
  New York Times Company}
\end{itemize}

\begin{itemize}
\tightlist
\item
  \href{https://www.nytco.com/}{NYTCo}
\item
  \href{https://help.nytimes.com/hc/en-us/articles/115015385887-Contact-Us}{Contact
  Us}
\item
  \href{https://www.nytco.com/careers/}{Work with us}
\item
  \href{https://nytmediakit.com/}{Advertise}
\item
  \href{http://www.tbrandstudio.com/}{T Brand Studio}
\item
  \href{https://www.nytimes.com/privacy/cookie-policy\#how-do-i-manage-trackers}{Your
  Ad Choices}
\item
  \href{https://www.nytimes.com/privacy}{Privacy}
\item
  \href{https://help.nytimes.com/hc/en-us/articles/115014893428-Terms-of-service}{Terms
  of Service}
\item
  \href{https://help.nytimes.com/hc/en-us/articles/115014893968-Terms-of-sale}{Terms
  of Sale}
\item
  \href{https://spiderbites.nytimes.com}{Site Map}
\item
  \href{https://help.nytimes.com/hc/en-us}{Help}
\item
  \href{https://www.nytimes.com/subscription?campaignId=37WXW}{Subscriptions}
\end{itemize}
