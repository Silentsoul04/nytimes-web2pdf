Sections

SEARCH

\protect\hyperlink{site-content}{Skip to
content}\protect\hyperlink{site-index}{Skip to site index}

\href{https://www.nytimes.com/section/sports/soccer}{Soccer}

\href{https://myaccount.nytimes.com/auth/login?response_type=cookie\&client_id=vi}{}

\href{https://www.nytimes.com/section/todayspaper}{Today's Paper}

\href{/section/sports/soccer}{Soccer}\textbar{}Embracing the Value in
Scarcity

\url{https://nyti.ms/33dH35i}

\begin{itemize}
\item
\item
\item
\item
\item
\end{itemize}

Advertisement

\protect\hyperlink{after-top}{Continue reading the main story}

Supported by

\protect\hyperlink{after-sponsor}{Continue reading the main story}

Rory Smith On Soccer

\hypertarget{embracing-the-value-in-scarcity}{%
\section{Embracing the Value in
Scarcity}\label{embracing-the-value-in-scarcity}}

Major League Baseball is busy worrying that a 60-game season is not long
enough. But that hand-wringing might offer an insight into why elite
soccer is such a global phenomenon.

\includegraphics{https://static01.nyt.com/images/2020/07/31/sports/31rorynewsletter-cheslea/merlin_174969222_6eca17f3-ef34-4a6b-85a3-ecbaeeabbd63-articleLarge.jpg?quality=75\&auto=webp\&disable=upscale}

\href{https://www.nytimes.com/by/rory-smith}{\includegraphics{https://static01.nyt.com/images/2019/08/23/sports/Rory-Smith-better/Rory-Smith-thumbLarge.png}}

By \href{https://www.nytimes.com/by/rory-smith}{Rory Smith}

\begin{itemize}
\item
  July 31, 2020
\item
  \begin{itemize}
  \item
  \item
  \item
  \item
  \item
  \end{itemize}
\end{itemize}

\emph{You could have been enjoying this newsletter hours ago. If you
like it, sign up to get it directly, every Friday, at}
\href{https://www.nytimes.com/newsletters/rory-smith?smid=rd}{\emph{nytimes.com/rory}}\emph{.}

The feast is almost over. For six weeks, barely an evening has passed
without television schedules filled by soccer from one, or more, of the
world's major leagues. In England, almost every game was given its own
time slot. In Spain, they tried not to miss a day. In Italy, they played
late into the sweltering night.

This weekend, our appetites for European soccer long since sated, we
have the final few morsels: Saturday's F.A. Cup final between Arsenal
and Chelsea, and a round of fixtures to conclude the Serie A season. And
then, no sooner than the table is cleared, the next course appears: two
weeks in which the Champions League and the Europa League will be
decided in World Cup-style tournaments.

In the spring, as soccer plotted its way out of its pandemic shutdown,
the fear was that this would prove a watershed, a revolution not just
televised, but one available only to subscribers. Networks would not
willingly turn back the clock to a world in which some games could not
be shown live. Soccer's transformation from attendance event to
broadcast content would be complete.

If anything, though, the opposite is true. The Premier League, for one,
is inclined to reduce the number of games it broadcasts domestically
next season. It is likely to return to scheduling some games
simultaneously, meaning even outside the United Kingdom, fans will not
be able to watch every single game. Italy and Spain should be expected
to go the same way. (Germany, first to the finish line, maintained a
more traditional schedule.)

\includegraphics{https://static01.nyt.com/images/2020/07/31/sports/31rorynewsletter-psg/merlin_175118631_4630e8fd-908a-4359-a055-cf8d485b022b-articleLarge.jpg?quality=75\&auto=webp\&disable=upscale}

At first glance, the change seems counterintuitive. After all, the more
games that can be broadcast, the more television rights should be worth
to the leagues and their clubs. More significant, with no definitive
timeline on when fans might return to stadiums, it raises what is almost
an existential question: If a game is played in an empty venue and it is
not broadcast, what is its purpose? Soccer is an entertainment business.
Why would it happen with nobody to entertain?

But on a deeper level, it is entirely fitting, an (unwitting, most
likely) acknowledgment of one of the central ingredients that has made
soccer, the game, the most popular on the planet, and soccer, the elite
competition, one of the most remarkable cultural phenomena in history:
scarcity.

There is a scene in
``\href{https://www.youtube.com/watch?v=ftlfPb0gyxI}{The Simpsons},''
from back before it grew old and bloated, in which the residents of
Springfield go to watch a soccer game. After kickoff, the team in green
passes the ball to one another aimlessly. The crowd cheers. The team in
red just stands there, waiting. The green team keeps passing. The cheers
die down.

The joke, of course, is that soccer is dull, that nothing happens, and
that nobody ever scores: ``fast-kicking, low-scoring and ties? You
bet!'' as the show would later describe it. (We can safely assume that
Matt Groening was, at best, ambivalent toward the world's game.) But
that is a fundamental misunderstanding: The rarity of goals is not
soccer's shortcoming, but its magic.

``The excitement unleashed whenever the white bullet makes the net
ripple might appear mysterious or crazy,'' as the great Uruguayan author
and journalist Eduardo Galeano wrote, ``but remember that the miracle
does not happen very often.''

Image

Why are soccer's goal celebrations so intense? Because it's hard to
score.Credit...Marco Bertorello/Agence France-Presse --- Getty Images

Even the smallest, least significant goals have the power to make ``the
stadium forget it is made of concrete and break free of earth and fly
through the air,'' as Galeano put it. It is the breaking of a dam: a
pent-up force, waiting and waiting to be unleashed. Goals mean more,
mean everything, because you never know how long you might have to wait
for the next. It is here that the passion soccer can engender is rooted.
It is from here that soccer draws its power.

Among major team sports, that makes soccer unique. In their book ``The
Numbers Game,'' Chris Anderson and David Sally found that, in (American)
football, there is a score every nine minutes. In rugby, it is every 12
and a half. In hockey, every 22. In soccer, that rockets up to a goal
every 69 minutes. Soccer's ``genius lies in the way it makes fans and
players alike wait for their reward,'' they wrote. ``It is a sport of
delayed gratification.''

It is worth considering, though, that the value of rarity does not only
apply on the field. In the United States, Major League Baseball is
currently going through what we might term the ``Asterisk Debate.'' It
is a conversation that Europe experienced several months ago, though
with one central difference.

In soccer, the assertion was that a title awarded without completing the
season --- as happened in Belgium, France and Scotland --- should be
accompanied by an asterisk, because it was not won over a full,
traditional campaign. (In England, this was later amended to: a title
won with all of the games completed but some held behind closed doors
should have an asterisk because it was won by Liverpool).

In baseball, it is not quite the same. M.L.B. was merely in spring
training and had not started its regular season when the pandemic hit.
It is not awarding a World Series to the team with the best
regular-season record. The asterisk is because of the length of the
season: only 60 games before an expanded postseason, rather than the
traditional 162.

To the sport's devotees, that is simply not enough. Baseball Prospectus
\href{https://www.baseballprospectus.com/news/article/59712/baseball-therapy-how-legitimate-is-a-60-game-season/}{idly
wondered} if a season so short could be counted as legitimate, or
whether ``randomness is going to have a heavier thumb on the scale.''

Image

Major League Baseball teams will play only 60 regular-season games this
year, just a bit more than a major European soccer club does in a
season.~Credit...Elise Amendola/Associated Press

Sports Illustrated acknowledged that playing 162 games is ``arbitrary,''
but pointed out that ``seasons of that length are, at the very least,
large enough samples to separate the good teams from bad.'' The Houston
Chronicle was even more damning:
``\href{https://www.houstonchronicle.com/texas-sports-nation/brian-t-smith/article/Smith-60-games-an-MLB-sham-You-bet-your-asterisk-15347892.php}{A
60-game season isn't really baseball},'' one of its columnists declared.

It is a view shared by players, past and present. Christian Yelich of
the Milwaukee Brewers believes there ``will be an asterisk next to this
year,
\href{https://www.usatoday.com/story/sports/mlb/columnist/bob-nightengale/2020/07/22/2020-mlb-season-60-games-crazy-unconventional/5486001002/}{no
matter what happens}.'' Mike Stanton, a three-time World Series winner
with the Yankees, is sure that the ``teams that lose, they'll be the
ones going, `Well, it's not for real, they didn't play 162,
\href{https://www.nytimes.com/2020/06/23/sports/baseball/mlb-60-game-season.html\#:~:text=\%E2\%80\%9CThe\%20teams\%20that\%20lose\%2C\%20they,the\%20Yankees\%2C\%20said\%20on\%20Tuesday.}{they
didn't have the marathon}.'''

With the exception of the N.F.L., American major leagues are not sports
of scarcity. The baseball season is particularly unwieldy, but
basketball and hockey both have regular seasons of 82 games, plus a
playoff system that normally includes 16 teams.

A cynic might suggest that is because more games means more money in
ticket sales, but in reality, it seems like the sort of arrangement
where everyone wins: the owners, the players --- as the pay dispute
around baseball's return made clear, the players earn more the more
games there are --- and the fans. Who doesn't want to watch more of
their favorite sport?

Soccer, of course, runs contrary to that. Most league seasons play
somewhere between 30 and 46 games, plus knockout and continental
tournaments, and the sport as a whole is confident that a sample that
size is large enough to separate good teams from bad.

Randomness does have a stronger hand with fewer games, of course, but
that is part of the charm: It makes the experience more unpredictable,
more compelling. More important, scarcity heightens the meaning of ---
and therefore the emotion at stake in --- every game. Soccer's slim-line
calendar infuses every occasion with jeopardy. With every additional
game, that effect is diluted. If tonight's defeat can be set right
tomorrow, then perhaps it does not matter so much.

Soccer is turning away from the model it established to escape the
pandemic. The endless television buffet will, slowly but surely, be
removed. In doing so, the sport's major leagues will restore to their
televised fixtures a sense of occasion, something to be savored and
anticipated, something not to be missed. After all, you will have to
wait awhile for the next one.

Perhaps, though, baseball --- a sport that has wrestled with declining
attendance, aging fans, dwindling attention spans and an ever-less
certain place in the modern sports landscape --- could approach this
shortened season with an open mind.

Perhaps the answer is not
\href{https://www.nytimes.com/interactive/2020/07/21/sports/coronavirus-changes-baseball-nba-nfl.html}{tweaking
the rules or changing the games}, but in making them more rare. Maybe
each hit, each out, will seem to matter a little more. Maybe each game
will appear more decisive. Maybe the tension will be ratcheted up a
notch, and maybe the joy of victory and the despair of defeat will seem
heightened. Maybe, just maybe, it might come to seem that less can be
more.

\begin{center}\rule{0.5\linewidth}{\linethickness}\end{center}

\hypertarget{erling-haaland-will-be-taking-questions-he-likes}{%
\subsection{Erling Haaland Will Be Taking Questions (He
Likes)}\label{erling-haaland-will-be-taking-questions-he-likes}}

Image

Erling Haaland scored 13 goals in 15 matches for Dortmund this
year.Credit...Martin Meissner/Associated Press

The night before I met Erling Haaland had been the night of that
television interview. You may remember it, from back in the before
times. On his first appearance in the Champions League for Borussia
Dortmund, Haaland had scored twice against Paris St.-Germain. The second
had contained enough power to rattle the bones of Signal Iduna Park.

As he walked off the field, he was steered toward the waiting banks of
television cameras. Everyone, at that time, wanted a piece of Europe's
shooting star. His first appointment was with German television. The
reporter was polite and assiduous, asking questions in English,
simultaneously translating the answers into German for his audience.

It became clear that Haaland, pretty quickly, did not want to play ball.
He gave brief, matter-of-fact responses: not rude, not exactly, but
seemingly designed to highlight that he did not think much of the
questions. The reporter, gamely, persisted. At one point, Haaland rolled
his eyes and nodded his head, the universal gesture for ``what's with
this guy?''

The prospect of my sitting and asking him about his
\href{https://www.nytimes.com/2020/07/28/sports/soccer/erling-haaland-gio-reyna-thuram.html}{relationship
with his father}, the former Leeds United and Norway midfielder Alfie,
suddenly seemed a little more daunting. If Haaland was
\href{https://twitter.com/BBCSport/status/1263571516546969601}{not in
the mood to talk}, if he decided not to play ball, then it might be a
very long 30 minutes indeed. For both of us.

There is a risk in judging a player from what you see in
\href{https://www.youtube.com/watch?v=nqKH-_VzufQ\&feature=youtu.be}{a
postgame television interview}. There is a limit to what a reporter can
ask --- there is no time for an in-depth investigation into anyone's
psyche --- and a limit to what a player can say. They are still caked in
sweat. They are often still catching their breath. And much of what they
have just done, out on the field, is not immediately explicable to them;
it is, instead, a mixture of instinct and instruction internalized so
deeply that the two are indistinguishable.

Haaland was a reminder of that. He does, I think, have a prickly side:
closed questions tend to elicit a closed response. He will not do your
work for you. But once he settled down, and stretched those long legs
out on the table in front of him, and talked about his father, and what
it's like to be the Son Of Someone, the sentences grew longer, more
considered. He is surprisingly offbeat. He is a little quirky. A flash
interview after a game is no time to show that. But that does not mean
it is not there.

\hypertarget{correspondence}{%
\subsection{Correspondence}\label{correspondence}}

Image

Karim Benzema is regularly judged not for the player he is, but against
the players he isn't.Credit...Javier Soriana/Agence France-Presse ---
Getty Images

Many of you, it seems, have been nursing the same sense as me that Karim
Benzema has spent his career at Real Madrid not getting quite the credit
that he deserves. \textbf{Rick Burroughs} wrote to say that Benzema has
``often felt that he was this strange outlier, one that demanded respect
from the teams Real played and always made those around him better.''

Kudos to \textbf{Edward Baker}, too, for probably putting it best: He
has always liked Benzema, he wrote, ``and one of his great attractions
is that José Mourinho never did.''

But \textbf{Richard Whiddington} made an observation than warrants
exploration. The fallout from the rumbling incident with Mathieu
Valbuena, Benzema's erstwhile France teammate --- a tabloid affair
involving
\href{https://www.theguardian.com/football/2015/nov/04/karim-benzema-arrested-sex-tape-blackmail}{a
sex tape and accusations of blackmail} --- served to preclude Benzema
from ``permanently reaching superstar status: particularly losing his
place in the French side. How might it have been if he had been an
integral part in Russia?''

It is just a personal view, but there was a reason behind the choice not
to focus on the Valbuena allegations: Benzema's being overlooked and
underappreciated predates the scandal. That it has had an impact on his
public image is without question, though, and while the World Cup is no
longer where reputations are made, it is frequently where they are
reassessed, particularly as the twilight starts to descend.

Advertisement

\protect\hyperlink{after-bottom}{Continue reading the main story}

\hypertarget{site-index}{%
\subsection{Site Index}\label{site-index}}

\hypertarget{site-information-navigation}{%
\subsection{Site Information
Navigation}\label{site-information-navigation}}

\begin{itemize}
\tightlist
\item
  \href{https://help.nytimes.com/hc/en-us/articles/115014792127-Copyright-notice}{©~2020~The
  New York Times Company}
\end{itemize}

\begin{itemize}
\tightlist
\item
  \href{https://www.nytco.com/}{NYTCo}
\item
  \href{https://help.nytimes.com/hc/en-us/articles/115015385887-Contact-Us}{Contact
  Us}
\item
  \href{https://www.nytco.com/careers/}{Work with us}
\item
  \href{https://nytmediakit.com/}{Advertise}
\item
  \href{http://www.tbrandstudio.com/}{T Brand Studio}
\item
  \href{https://www.nytimes.com/privacy/cookie-policy\#how-do-i-manage-trackers}{Your
  Ad Choices}
\item
  \href{https://www.nytimes.com/privacy}{Privacy}
\item
  \href{https://help.nytimes.com/hc/en-us/articles/115014893428-Terms-of-service}{Terms
  of Service}
\item
  \href{https://help.nytimes.com/hc/en-us/articles/115014893968-Terms-of-sale}{Terms
  of Sale}
\item
  \href{https://spiderbites.nytimes.com}{Site Map}
\item
  \href{https://help.nytimes.com/hc/en-us}{Help}
\item
  \href{https://www.nytimes.com/subscription?campaignId=37WXW}{Subscriptions}
\end{itemize}
