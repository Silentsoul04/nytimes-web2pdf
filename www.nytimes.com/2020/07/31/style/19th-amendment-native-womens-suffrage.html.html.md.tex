Sections

SEARCH

\protect\hyperlink{site-content}{Skip to
content}\protect\hyperlink{site-index}{Skip to site index}

\href{https://www.nytimes.com/section/style}{Style}

\href{https://myaccount.nytimes.com/auth/login?response_type=cookie\&client_id=vi}{}

\href{https://www.nytimes.com/section/todayspaper}{Today's Paper}

\href{/section/style}{Style}\textbar{}In 1920, Native Women Sought the
Vote. Here's What's Next.

\href{https://nyti.ms/3fhofEH}{https://nyti.ms/3fhofEH}

\begin{itemize}
\item
\item
\item
\item
\item
\item
\end{itemize}

Advertisement

\protect\hyperlink{after-top}{Continue reading the main story}

Supported by

\protect\hyperlink{after-sponsor}{Continue reading the main story}

\hypertarget{in-1920-native-women-sought-the-vote-heres-whats-next}{%
\section{In 1920, Native Women Sought the Vote. Here's What's
Next.}\label{in-1920-native-women-sought-the-vote-heres-whats-next}}

The 19th Amendment did not bring the right to vote to all Native women,
but two experts in a conversation said it did usher in the possibility
of change.

\includegraphics{https://static01.nyt.com/images/2020/07/31/multimedia/31suffrage-nativewomen-07/31suffrage-nativewomen-07-articleLarge.jpg?quality=75\&auto=webp\&disable=upscale}

By Cathleen D. Cahill and Sarah Deer

\begin{itemize}
\item
  Published July 31, 2020Updated Aug. 7, 2020
\item
  \begin{itemize}
  \item
  \item
  \item
  \item
  \item
  \item
  \end{itemize}
\end{itemize}

Native women were highly visible in early 20th-century suffrage
activism. White suffragists, fascinated by Native matriarchal power,
invited Native women to speak at conferences, join parades and write for
their publications. Native suffragists took advantage of these
opportunities to speak about pressing issues in their communities ---
Native voting, land loss and treaty rights. But their stories have
largely been forgotten.

After the 19th Amendment was ratified on Aug. 18, 1920, and celebrated
by millions of women across the country, the Indigenous suffragist
Gertrude Simmons Bonnin, also known as Zitkala-Sa, a citizen of the
Yankton Sioux Tribe, reminded newly enfranchised white women that the
fight was far from over. ``The Indian woman rejoices with you,'' she
proclaimed to members of Alice Paul's National Woman's Party, but she
urged them to remember their Native sisters, many of whom lacked the
right to vote. Not only that, she explained, many were not United States
citizens, but legally wards of the government, without a political voice
to address the many problems facing their communities.

Bonnin and other Native suffragists would continue to remind audiences
that federal assimilation policy had attacked their communities and
cultures. Despite treaty promises, the U.S. dismantled tribal
governments, privatized tribally-held land and removed Native children
to boarding schools. Those devastating policies resulted in massive land
loss, poverty and poor health that reverberate through these communities
today.

Native suffragists' activism contributed to Congress passing the Snyder
Act of 1924, which extended U.S. citizenship to all Native people,
though in response many states enacted Jim Crow-like policies aimed at
disenfranchising them. The Native suffragists also aided the push for
the Wheeler-Howard Act of 1934, which stopped the breakup of tribal
lands and emphasized tribal self-governance.

\includegraphics{https://static01.nyt.com/images/2020/07/31/multimedia/31suffrage-nativewomen-01/31suffrage-nativewomen-01-articleLarge.jpg?quality=75\&auto=webp\&disable=upscale}

As the centennial of the 19th Amendment approaches, it is worth taking
up Bonnin's call to remember Native women and their full range of
political experiences. With this in mind, Prof. Cathleen D. Cahill, a
historian who has written about Native suffragists, joined Prof. Sarah
Deer, a scholar of Native law and a citizen of the Muscogee (Creek)
Nation, to talk about issues Native women face today. The conversation
has been edited for length and clarity.

\textbf{Cathleen D. Cahill}:

You have spent much of your career addressing the issue of violence
against Native women, including in your book ``The Beginning and End of
Rape.'' Native women have been calling attention to this kind of
violence for more than a century. \textbf{Why are Native women
especially vulnerable?}

\textbf{Sarah Deer:}

I'm a citizen of the Muscogee (Creek) Nation of Oklahoma, and I have
been working to address violence against Native women for over 25 years.
I started when I was 20 years old as a volunteer advocate for survivors
of sexual assault, and that experience inspired me to go to law school.
It was in federal Indian law classes that I began to understand the
reasons for the high rate of violence. Quite simply, the criminal legal
system in Indian Country is broken. What else could explain these
statistics: Over 84 percent of Native women have experienced violence in
their lifetime, and over 56 percent of Native women have experienced
sexual violence. This is data directly
\href{https://nij.ojp.gov/topics/articles/violence-against-american-indian-and-alaska-native-women-and-men}{from
the federal government} --- and these are probably low estimates.

To make matters worse, in 1978 the Supreme Court ruled that tribal
nations lack authority to prosecute non-Natives --- again, for any
crime. Many experts believe this is one of the reasons Native people
experience the highest rates of interracial violence in the nation. A
system that doesn't hold people accountable sends two messages: to
victims, it says ``don't bother to report,'' and to perpetrators, it
says ``keep victimizing people.''

\textbf{Cahill}: That's really awful. In the 1920s Gertrude Simmons
Bonnin drew similar connections between violence against Native women
and the fact that federal policies had dismantled tribal governments and
made Indian people ``wards'' without any political power. That seems
like such a long time ago, but \textbf{the July 9}
\textbf{\href{https://www.nytimes.com/2020/07/09/us/supreme-court-oklahoma-mcgirt-creek-nation.html}{Supreme
Court ruling}} \textbf{in McGirt v. Oklahoma demonstrates that the past
is so clearly present in Indian Country.} \textbf{Can you talk about the
ruling's ramifications?}

\textbf{Deer:} Indian law scholars are calling this the greatest win for
tribal governments in the last 50 years. It also hits close to home ---
it was a victory for my own tribal nation.

Our Nation signed a peace treaty with the United States in 1866 which
established specific boundaries for our reservation --- about 3 million
acres. The United States promised that this reservation would ``be
forever set apart as a home for said Creek Nation.'' Seems simple,
right?

Throughout the 20th century, though, the state of Oklahoma ignored the
treaty and gradually began exercising criminal and civil authority over
the reservation, denying its existence.

Image

An 1892 map of the Indian and Oklahoma territories showing the
boundaries of tribal reservations.~Soon after, the federal government
started the process of dividing the tribally-held land despite
resistance by tribal leaders.~Credit...Library of Congress

The Supreme Court's 5-4 decision, written by Justice Neil M. Gorsuch,
determined that the Creek reservation boundaries were never
disestablished; the reservation promised to the Creek people in 1866 is
still in full force.

Tribal issues don't fare well in the U.S. Supreme Court --- losing over
75 percent of the time --- so this was an unlikely win, and a tremendous
win; the legal reasoning in this decision will have far-reaching
implications for many different tribal nations who are attempting to
preserve land and resources. \textbf{Your research has looked into the
role of Native women in the American suffrage movement. I'd love to
learn more.}

\textbf{Cahill:} White feminists were inspired by the matriarchal
traditions of Native people. They especially looked to Haudenosaunee (or
Iroquois) women's power to appoint male political leadership, control
their property, and have custodial rights to their children --- those
were legal rights white women did not have. They wanted to hear more and
often invited Native women to speak at their meetings. This gave Native
activists a chance to educate their audiences and while they did proudly
talk about their traditions, they also insisted on talking about the
problems that faced ``the Indian woman of today,'' as Bonnin put it.

Image

Marie Louise Bottineau Baldwin, a~citizen of the Turtle Mountain Band of
Chippewa, posing with floral pattern beadwork on her wrists and behind
her to showcase Native women's artistry.~In 1914 she became one of the
first Native American women in the U.S. to graduate from law
school.~~Credit...Library of Congress

A good example of this is when organizers asked Marie Louise Bottineau
Baldwin, a citizen of the Turtle Mountain Band of Chippewa, to put
together a float for the 1913 suffrage parade in Washington. They wanted
the float to portray Native women as they were in the past, you know,
wearing buckskin with their hair in braids, that kind of thing. Baldwin
was deeply aware of the power of imagery in shaping public perceptions
of Native Americans, so she used her image strategically. She decided
not to organize the float, and instead marched with her classmates and
teachers from the Washington College of Law. I think she was making a
statement that Native women were modern New Women who were looking to
the future. She also thought it was important for Native people to study
law to protect their land and treaty rights. She was one of the first
Native woman to graduate from law school, in 1914. \textbf{You're also
an attorney (and a tribal court justice). What do you think is the role
of legal training for Native women in the 21st century?}

\textbf{Deer}: Access to legal education is a critical step to
strengthening tribal sovereignty. There are still relatively few Native
attorneys in the United States, but the numbers are increasing. There
are also only a handful of Native women law professors. Nonetheless,
Native people are actively litigating important questions of tribal
jurisdiction, land rights and criminal authority. Native women serve on
tribal courts, but there are also Native women who serve on state
benches. Diane Humetewa (Hopi) became the first Native woman appointed
to the federal bench in 2014. Some Native attorneys focus their work on
legislation like the Violence Against Women Act (VAWA) which contains
significant provisions that directly affect tribal justice systems.
Native women have also been leading the movements to address
environmental abuses and pipelines. At
\href{https://www.nytimes.com/2017/01/31/magazine/the-youth-group-that-launched-a-movement-at-standing-rock.html}{Standing
Rock Sioux Reservation}, in particular, women were doing most of the
organizing and decision-making in the fight over the Dakota Access
Pipeline.

Image

Rep. Sharice Davids (D-KS), a Ho-Chunk citizen, at a news conference on
March 10, 2020.Credit...Pete Marovich for The New York Times

Image

U.S. Rep. Deb Haaland (D-N.M.) of the Laguna Pueblo, at a press
conference on June 19, 2019 in Washington, D.C.Credit...Stefani
Reynolds/Getty Images

\textbf{Cahill:} One striking thing just in the past few years is the
\href{https://www.npr.org/2018/07/04/625425037/record-number-of-native-americans-running-for-office-in-midterms}{growing
number} of Native women running for state and federal offices. The
\href{https://www.nytimes.com/2018/11/07/us/elections/native-americans-congress-haaland-davids.html}{first
Native women} in Congress were just elected in 2018: Deb Haaland of the
Laguna Pueblo represents New Mexico and Sharice Davids, a Ho-Chunk
citizen, represents your state of Kansas. Native men have served in
Congress for well over a century, but they are the first Native women to
hold office in Washington. \textbf{What does it mean to have Native
women in Congress or other elected offices?}

\textbf{Deer:} Native women have served in state legislatures for many
years, but we are now seeing a critical mass of new Native women
politicians. Today, we have one Native woman in the Kansas House, and
another young Native woman is campaigning for the Kansas House as well.
In Minnesota, White Earth citizen Peggy Flanagan, became the first
Native women to be elected as a lieutenant governor in the United States
in 2018.

When Haaland and Davids were elected as the first two Native women in
Congress, it was seen as a tremendous victory for Native people. It
seems fitting that there were two women elected together. From my
perspective, being the ``first'' or ``only'' Native woman serving in
Congress could be a lonely experience. A ``partnership'' of two Native
women perhaps makes it easier to achieve great things in Congress. For
far too long, Congress has been passing laws to limit the power of
tribal governments without any tribal input. It is far past time for us
to have a seat at the table.

\textbf{Cahill}: Absolutely. And that is so important to remember when
we think about the anniversary of the suffrage amendment. For all
suffragists, getting the vote wasn't an end point: It was the
possibility for change that voting opened up. Native suffragists saw the
vote as a way to change the awful circumstances that faced Native
communities at the time. \textbf{One hundred years later, what's next
for Indigenous feminism?}

\textbf{Deer}: I'm still basking in the afterglow of the McGirt
decision, so I'm optimistic about the future for Native women and tribal
nations. I hope to see more Native women elected to public office --- at
all levels, tribal, state and national. We have been politically and
symbolically disenfranchised for too long. I'm so glad our issues are
getting more national attention.

\begin{center}\rule{0.5\linewidth}{\linethickness}\end{center}

Cathleen D. Cahill is an associate professor of history at Penn State
University and the author of the forthcoming book ``Recasting the Vote:
How Women of Color Transformed the Suffrage Movement.''

Sarah Deer is a citizen of the Muscogee (Creek) Nation and a professor
at the University of Kansas.

Advertisement

\protect\hyperlink{after-bottom}{Continue reading the main story}

\hypertarget{site-index}{%
\subsection{Site Index}\label{site-index}}

\hypertarget{site-information-navigation}{%
\subsection{Site Information
Navigation}\label{site-information-navigation}}

\begin{itemize}
\tightlist
\item
  \href{https://help.nytimes.com/hc/en-us/articles/115014792127-Copyright-notice}{©~2020~The
  New York Times Company}
\end{itemize}

\begin{itemize}
\tightlist
\item
  \href{https://www.nytco.com/}{NYTCo}
\item
  \href{https://help.nytimes.com/hc/en-us/articles/115015385887-Contact-Us}{Contact
  Us}
\item
  \href{https://www.nytco.com/careers/}{Work with us}
\item
  \href{https://nytmediakit.com/}{Advertise}
\item
  \href{http://www.tbrandstudio.com/}{T Brand Studio}
\item
  \href{https://www.nytimes.com/privacy/cookie-policy\#how-do-i-manage-trackers}{Your
  Ad Choices}
\item
  \href{https://www.nytimes.com/privacy}{Privacy}
\item
  \href{https://help.nytimes.com/hc/en-us/articles/115014893428-Terms-of-service}{Terms
  of Service}
\item
  \href{https://help.nytimes.com/hc/en-us/articles/115014893968-Terms-of-sale}{Terms
  of Sale}
\item
  \href{https://spiderbites.nytimes.com}{Site Map}
\item
  \href{https://help.nytimes.com/hc/en-us}{Help}
\item
  \href{https://www.nytimes.com/subscription?campaignId=37WXW}{Subscriptions}
\end{itemize}
