Sections

SEARCH

\protect\hyperlink{site-content}{Skip to
content}\protect\hyperlink{site-index}{Skip to site index}

\href{https://www.nytimes.com/section/style}{Style}

\href{https://myaccount.nytimes.com/auth/login?response_type=cookie\&client_id=vi}{}

\href{https://www.nytimes.com/section/todayspaper}{Today's Paper}

\href{/section/style}{Style}\textbar{}Spending My Tenderness on Animals

\url{https://nyti.ms/39Fgnvs}

\begin{itemize}
\item
\item
\item
\item
\item
\end{itemize}

Advertisement

\protect\hyperlink{after-top}{Continue reading the main story}

Supported by

\protect\hyperlink{after-sponsor}{Continue reading the main story}

Modern Love

\hypertarget{spending-my-tenderness-on-animals}{%
\section{Spending My Tenderness on
Animals}\label{spending-my-tenderness-on-animals}}

As a vulnerable girl at a remote commune, I sought solace from horses,
goats, even a bear cub. Today's dark times have sent me their way again.

\includegraphics{https://static01.nyt.com/images/2020/08/02/fashion/02MODERN-FARM/02MODERN-FARM-articleLarge.jpg?quality=75\&auto=webp\&disable=upscale}

By Bethany Groff Dorau

\begin{itemize}
\item
  July 31, 2020
\item
  \begin{itemize}
  \item
  \item
  \item
  \item
  \item
  \end{itemize}
\end{itemize}

When it all went to hell, we adopted sheep.

We were standing in the shed, five years into our marriage, and I just
wanted my husband to put the saddle on the rack. It wasn't going well.
First, James was not sure why I owned a saddle since I have never owned
a horse. Second, when I barked out instructions \textbf{---} ``Don't
step on the girth!'' and ``Do you see the martingale anywhere?'' --- he
looked at me as if I had just ordered breakfast in Japanese.

I relieved him of the saddle, found its accessories and popped it onto
its stand. ``Jesus,'' I said. ``It's not brain surgery.''

``You know I have only ever lived in a city,'' he said. ``I have no idea
what you're talking about half the time.''

After living in a (small) city for years, we had recently moved into my
ancestral farm in rural Massachusetts. James is a professional brewer.
What did he know about farms and animals?

Not much, it turned out.

In any marriage, there are moments when one looks at one's spouse and
thinks, ``I don't know you at all, do I?'' Sometimes this is charming, a
new facet to the jewel that is your beloved. In our case, I felt
profoundly betrayed.

I knew James was not a farmer. Even so, he's a stocky six-footer with a
long gray beard and a perpetual jet-black ponytail who wears work boots
and Carhartt. His hands are rough, his chest broad. He likes to take the
doors off his Jeep. Everything about this man screamed ruggedness and
hard work, and to me this will always mean that you know your way around
farmland and animals.

\emph{{[}}\href{https://www.nytimes.com/newsletters/love-letter}{\emph{Sign
up for Love Letter, our weekly email about Modern Love, weddings and
relationships.}}\emph{{]}}

My parents were not farmers in the beginning. They were frightened young
parents who heard a clarion call to run for the hills. My mother met my
father in 1970 in a prayer group organized by a group of born-again
Christians, followers of a charismatic leader named Sam Fife, founder of
a group called The Move.

Brother Sam's message was simple: Western society was corrupt and
disintegrating. Women and children did not know their place. Christians
worldwide were being persecuted, and it was going to get much worse.

When I was an infant, we emigrated to Canada and worked our way north to
a communal farm in northern British Columbia called Evergreen, which was
off the Alaska Highway at the end of a dirt track deep in the woods.
After an interminable, jolting drive, windrows of saskatoon bushes and
blueberries announced the beginning of the farm, then a potato field,
then the first glimpse of a long log house surrounded by white trailers.

I see it in my dreams. I was 7. Our first night in the Tabernacle, the
central building where we prayed, attended school and cooked and ate our
meals, a blonde-haired girl in a denim skirt hissed as she walked by me.
``Just what we need,'' she said. ``Another city slicker.''

I soon learned that I had been paid the ultimate insult by this girl,
and I would spend the next five years trying to prove her wrong.

The farm had been established by people who knew what they were doing,
but by the time we arrived, it was populated with well-intentioned
people like my parents, whose backgrounds in civil engineering and music
were useful to the group but not especially relevant to feeding a
family. The crops were thin, the animals thinner. We were trying to live
as far away from society as possible, and this meant little food for
people and even less for animals.

We children lived with our parents but spent our days in groups managed
by other adults. One of my first shifts was in the cow barn where I was
kicked into a pile of manure. I tried to regain my dignity by carrying
two five-gallon buckets of milk up to the separator. The milk sloshed
into my boots and my arms felt like they were tearing from my shoulders,
but I made it.

The blonde, a year older, tanned and wiry, trotted up behind me with her
two full pails, flashing me a nearly sincere smile.

The next day she put me on a horse, a stocky pinto mare, and told me I
would be a true horsewoman when I had fallen off 100 times. I kept
count. Falls 34 to 40 happened on one day. After each tumble, I limped
over and dragged myself back onto the horse, my blonde nemesis watching
astride the fence. I logged every fall, named the resulting scars on my
knees, forehead and shins.

The farm was a hard place for the vulnerable. I struggled to be tough
enough to survive not only accidents and falls but also sexual and
physical abuse. I lost what little faith I ever had in God and focused
on being physically strong, taking any dare, riding any horse. Whatever
softness I had was reserved for animals, whose suffering I could
alleviate in small ways.

I stole peanut butter from the buckets in the kitchen and fed a bear cub
whose mother we had killed and eaten (the image of the mother bear
haunts me to this day). I sneaked bread to the skinniest cows and
mourned the death of each chicken, goat and dog. I wrote their eulogies
on notebook paper and hid them in a coffee can.

We left Evergreen when I was 11 and returned, penniless, to
Massachusetts. I was angry, traumatized, feral. I took jobs in stables
and cow barns just to be near animals. My life began a slow, upward arc
that finds me now volunteering for animal rescues and working at a
historic farm.

Though I am a vegetarian, I came to it late and am moderate in my
advocacy. The pledges I made to the carcasses of skinned goats I loved
are faintly remembered. The blonde girl is now my friend on Facebook,
and we don't talk about Evergreen.

The killing of George Floyd during a worldwide pandemic, with children
in cages and people disrespecting the sick and dying, sent me into a
place I had not been since the darkest days of my youth.

The Massachusetts Society for the Prevention of Cruelty to Animals
called to ask if we could take three skinny sheep at the historic farm I
manage. I said no, worried that the staff and volunteers were already
overstretched. That afternoon I paced around the house, made a donation
to a civil rights advocacy group, read increasingly desperate pleas for
justice online and added my voice to them.

My husband came home from the brewery that night exhausted and
depressed. They had laid off their staff. He was running the canning
line alongside the owner.

I told him about the sheep, about how helpless I felt.

``Tell me what we need to bring them here,'' he said.

The next day he looked up ``sheep shelter'' on YouTube and began working
on a hoop house, a temporary home for the sheep so we could take them
immediately and begin work on a barn. They arrived the following week,
three toothless, skinny old ewes.

A week after that I got a call about a young goat. We added him to the
group, and within days, a small barn arrived on a flatbed truck. I threw
all my fear, frustration and hope into pounding fence posts, hauling
water, dispensing medication and ear scratches. James got up early to
chop carrots and apples for them. He sang them songs and ordered bells
from the Alps with their names engraved.

In June, the M.S.P.C.A. called again. A former carriage horse needed a
retirement home. He was massive --- 6 feet 2 at the shoulder --- and
needed to be placed with someone with ``draft horse experience.''

I hung up and cried, thinking of all the big, tired, kind-eyed horses
from my childhood, pulling plows and wagons and balers, dropping their
massive heads so I could rub their sweaty necks. James didn't skip a
beat when I told him about the carriage horse.

``Tell me what we need to do,'' he said, and we started building. James
was dirty and grumpy --- a city slicker trying to learn to manage all
the complex human and animal needs that suddenly had become his
responsibility.

The horse arrived a few days ago --- skinny, a little wary, magnificent.
We already had laid more than 100 fence posts and added five chickens
and two turkeys. These animals will never mean to him what they mean to
me --- the fulfillment of dozens of tearful promises I made decades ago.

For him, this is the fulfillment of just one promise: to count my scars,
to ask me how I got them and to love me as I am.

\href{https://bethanygroffdorau.com/}{Bethany Groff Dorau}, a writer and
historian in Massachusetts and regional administrator for Historic New
England, is the author of
``\href{https://bethanygroffdorau.com/product/a-newburyport-marine-in-world-war-one/}{A
Newburyport Marine in World War I}.''

Modern Love can be reached at
\href{mailto:modernlove@nytimes.com}{\nolinkurl{modernlove@nytimes.com}}.

Want more from Modern Love? Watch the
\href{https://www.nytimes.com/2019/09/12/style/modern-love-tv-show-trailer.html}{TV
series}; sign up for the
\href{https://www.nytimes.com/newsletters/love-letter}{newsletter}; or
listen to the
\href{https://www.nytimes.com/column/modern-love-podcast}{podcast} on
\href{https://itunes.apple.com/us/podcast/modern-love/id1065559535?mt=2\&version=meter+at+0\&module=meter-Links\&pgtype=article\&contentId=\&mediaId=\&referrer=\&priority=true\&action=click\&contentCollection=meter-links-click}{iTunes},
\href{https://open.spotify.com/show/03Er7mSPq9IEewOgbPD3vO}{Spotify} or
\href{https://play.google.com/music/listen?u=0\#/ps/Iktqjbkz7bychbnofblw32dik64}{Google
Play}. We also have swag at
\href{https://store.nytimes.com/collections/modern-love}{the NYT Store}
and a book,
``\href{https://www.penguinrandomhouse.com/books/623036/modern-love-revised-and-updated-by-edited-by-daniel-jones-with-contributions-by-andrew-rannells-ayelet-waldman-amy-krouse-rosenthal-veronica-chambers-and-more/}{Modern
Love: True Stories of Love, Loss, and Redemption}.''

Advertisement

\protect\hyperlink{after-bottom}{Continue reading the main story}

\hypertarget{site-index}{%
\subsection{Site Index}\label{site-index}}

\hypertarget{site-information-navigation}{%
\subsection{Site Information
Navigation}\label{site-information-navigation}}

\begin{itemize}
\tightlist
\item
  \href{https://help.nytimes.com/hc/en-us/articles/115014792127-Copyright-notice}{©~2020~The
  New York Times Company}
\end{itemize}

\begin{itemize}
\tightlist
\item
  \href{https://www.nytco.com/}{NYTCo}
\item
  \href{https://help.nytimes.com/hc/en-us/articles/115015385887-Contact-Us}{Contact
  Us}
\item
  \href{https://www.nytco.com/careers/}{Work with us}
\item
  \href{https://nytmediakit.com/}{Advertise}
\item
  \href{http://www.tbrandstudio.com/}{T Brand Studio}
\item
  \href{https://www.nytimes.com/privacy/cookie-policy\#how-do-i-manage-trackers}{Your
  Ad Choices}
\item
  \href{https://www.nytimes.com/privacy}{Privacy}
\item
  \href{https://help.nytimes.com/hc/en-us/articles/115014893428-Terms-of-service}{Terms
  of Service}
\item
  \href{https://help.nytimes.com/hc/en-us/articles/115014893968-Terms-of-sale}{Terms
  of Sale}
\item
  \href{https://spiderbites.nytimes.com}{Site Map}
\item
  \href{https://help.nytimes.com/hc/en-us}{Help}
\item
  \href{https://www.nytimes.com/subscription?campaignId=37WXW}{Subscriptions}
\end{itemize}
