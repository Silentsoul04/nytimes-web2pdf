Sections

SEARCH

\protect\hyperlink{site-content}{Skip to
content}\protect\hyperlink{site-index}{Skip to site index}

\href{https://www.nytimes.com/section/technology}{Technology}

\href{https://myaccount.nytimes.com/auth/login?response_type=cookie\&client_id=vi}{}

\href{https://www.nytimes.com/section/todayspaper}{Today's Paper}

\href{/section/technology}{Technology}\textbar{}I Tried to Live Without
the Tech Giants. It Was Impossible.

\url{https://nyti.ms/2CWI86R}

\begin{itemize}
\item
\item
\item
\item
\item
\item
\end{itemize}

Advertisement

\protect\hyperlink{after-top}{Continue reading the main story}

Supported by

\protect\hyperlink{after-sponsor}{Continue reading the main story}

\hypertarget{i-tried-to-live-without-the-tech-giants-it-was-impossible}{%
\section{I Tried to Live Without the Tech Giants. It Was
Impossible.}\label{i-tried-to-live-without-the-tech-giants-it-was-impossible}}

As lawmakers debate whether Apple, Google, Facebook, and Amazon are
monopolies, a reporter recalls her attempt to avoid interacting with the
companies.

\includegraphics{https://static01.nyt.com/images/2020/08/02/business/31Bigfive-illo/31Bigfive-illo-articleLarge.jpg?quality=75\&auto=webp\&disable=upscale}

\href{https://www.nytimes.com/by/kashmir-hill}{\includegraphics{https://static01.nyt.com/images/2020/07/24/business/author-hill-kashmir/author-hill-kashmir-thumbLarge-v2.png}}

By \href{https://www.nytimes.com/by/kashmir-hill}{Kashmir Hill}

\begin{itemize}
\item
  July 31, 2020
\item
  \begin{itemize}
  \item
  \item
  \item
  \item
  \item
  \item
  \end{itemize}
\end{itemize}

The chief executives of Amazon, Facebook, Google and Apple were called
before a House antitrust committee this week, ostensibly to answer
questions about whether they have too much power and whether that hurts
consumers.

The tech bosses, who appeared via videoconference, fended off questions
about being ``cyber barons,'' saying they have plenty of competition and
that consumers have other options for the services they offer.

But do they? Last year, in an effort to understand just how dependent we
are on these companies, I did
\href{https://gizmodo.com/c/goodbye-big-five}{an experiment} for the
tech news site Gizmodo to see how hard it would be to remove them from
my life.

To do that wasn't easy. From my years writing about digital privacy, I
knew these companies were in the background of many of our online
interactions. I worked with a technologist named Dhruv Mehrotra, who
designed a custom tool for me, a virtual private network that kept my
devices from sending data to or receiving data from the tech giants by
blocking the millions of internet addresses the companies controlled.

Then I blocked Amazon, Facebook, Google, Apple and Microsoft, one by one
--- and then all at once --- over six weeks. Amazon and Google were the
hardest companies to avoid by far.

Cutting Amazon from my life meant losing access to any site hosted by
Amazon Web Services, the internet's largest cloud provider. Many apps
and a large portion of the internet use Amazon's servers to host their
digital content, and much of the digital world became inaccessible when
I said goodbye to Amazon, including the Amazon Prime Video competitor
Netflix.

Amazon was difficult to avoid in the real world as well. When I ordered
a phone holder for my car from eBay, it arrived in Amazon's signature
packaging, because the seller used ``Fulfillment by Amazon,'' paying the
company to store and ship his product.

When I blocked Google, the entire internet slowed down for me, because
almost every site I visited was using Google to supply its fonts, run
its ads, track its users, or determine if its users were humans or bots.
While blocking Google, I couldn't sign into the data storage service
Dropbox because the site thought I wasn't a real person. Uber and Lyft
stopped working for me, because they were both dependent on Google Maps
for navigating the world. I discovered that Google Maps had
\href{https://i.kinja-img.com/gawker-media/image/upload/c_scale,f_auto,fl_progressive,pg_1,q_80,w_1600/zrwbegnvl9qrpiyaerqv.png}{a
de facto monopoly} on online maps. Even Google's
longtime\href{https://www.nytimes.com/2017/07/01/technology/yelp-google-european-union-antitrust.html}{critic
Yelp} used it to tell computer users where businesses could be found.

I came to think of Amazon and Google as the providers of the very
infrastructure of the internet, so embedded in the architecture of the
digital world that even their competitors had to rely on their services.

Facebook, Apple and Microsoft came with their own challenges. While
Facebook was less debilitating to block, I missed Instagram (which
Facebook owns) terribly, and I stopped getting news from my social
circle, like the birth of a good friend's child. ``I just assume that if
I post something on Facebook, everyone will know about it,'' she told me
when I called her weeks later to congratulate her. I tried out an
alternative called Mastodon, but a social network devoid of any of your
friends isn't much fun.

Apple was hard to leave because I had two Apple computers and an iPhone,
so I wound up getting some radical new hardware in order to keep
accessing the internet and making phone calls.

Apple and Google's Android software have a duopoly on the smartphone
market. Wanting to avoid both companies, I wound up getting a dumb phone
--- a Nokia 3310 on which I had to relearn the fine art of texting on
numerical phone keys --- and a laptop with a Linux operating system from
a company called Purism that is trying to create ``an ethical computing
environment,'' namely by helping its users avoid the tech giants.

\includegraphics{https://static01.nyt.com/images/2020/07/31/business/31bigfive1/merlin_175158504_76dcbef2-989a-4bba-b785-36d7b84132a3-articleLarge.jpg?quality=75\&auto=webp\&disable=upscale}

Yes, there are
\href{https://www.nytimes.com/2020/07/29/technology/personaltech/big-tech-power-how-to-fight.html}{alternatives}
for products and services offered by the tech giants, but they are
harder to find and to use.

Microsoft, which is not in the antitrust hot seat this time around but
\href{https://www.nytimes.com/2019/06/23/technology/antitrust-tech-microsoft-lessons.html}{knows
what it feels like}, was easy to block on the consumer level. As my
colleague
\href{https://www.nytimes.com/2018/11/29/technology/microsoft-apple-worth-how.html}{Steve
Lohr notes}, Microsoft is ``mainly a supplier of technology to business
customers'' these days.

But like Amazon, Microsoft has a cloud service, and so a few sites went
dark for me, as did two Microsoft-owned services I used frequently,
LinkedIn and Skype. Not being able to use tech giant-owned services I
love was a hazard of this experiment: As The
\href{https://www.wsj.com/articles/beware-the-big-tech-backlash-11545227197?mod=e2tw}{Wall
Street Journal}noted, the tech giants have bought more than 400
companies and start-ups over the last decade.

Critics of the big tech companies are often told, ``If you don't like
the company, don't use its products.'' My takeaway from the experiment
was that it's not possible to do that. It's not just the products and
services branded with the big tech giant's name. It's that these
companies control a thicket of more obscure products and services that
are hard to untangle from tools we rely on for everything we do, from
work to getting from point A to point B.

Many people called what I did ``digital veganism.''
\href{https://observer.com/2011/06/what-is-digital-veganism-cody-brown-explains-his-catchphrase/}{Digital
vegans} are deliberative about the hardware and software they use and
the data they consume and share, because information is power, and
increasingly a handful of companies seem to have it all.

There were two very different types of reaction to the story. Some
people said that it proved just how essential these companies are to the
American economy and how useful they are to consumers, meaning
regulators shouldn't interfere with them. Others, like Representative
Jerrold Nadler, Democrat of New York and ex officio member of the
House's antitrust committee, said at the time that the experiment was
proof of their monopolistic power.

``By virtue of controlling essential infrastructure, these companies
appear to have the ability to control access to markets,'' Mr. Nadler
said. ``In some basic ways, the problem is not unlike what we faced 130
years ago, when railroads transformed American life --- both enabling
farmers and producers to access new markets, but also creating a key
chokehold that the railroad monopolies could exploit.''

If I were still blocking the tech giants today, I wouldn't have been
able to watch this week's antitrust hearing online. C-SPAN streamed it
live via YouTube, which Google owns.

After the experiment was over, though, I went back to using the
companies' services again, because as it demonstrated, I didn't really
have any other choice.

Advertisement

\protect\hyperlink{after-bottom}{Continue reading the main story}

\hypertarget{site-index}{%
\subsection{Site Index}\label{site-index}}

\hypertarget{site-information-navigation}{%
\subsection{Site Information
Navigation}\label{site-information-navigation}}

\begin{itemize}
\tightlist
\item
  \href{https://help.nytimes.com/hc/en-us/articles/115014792127-Copyright-notice}{©~2020~The
  New York Times Company}
\end{itemize}

\begin{itemize}
\tightlist
\item
  \href{https://www.nytco.com/}{NYTCo}
\item
  \href{https://help.nytimes.com/hc/en-us/articles/115015385887-Contact-Us}{Contact
  Us}
\item
  \href{https://www.nytco.com/careers/}{Work with us}
\item
  \href{https://nytmediakit.com/}{Advertise}
\item
  \href{http://www.tbrandstudio.com/}{T Brand Studio}
\item
  \href{https://www.nytimes.com/privacy/cookie-policy\#how-do-i-manage-trackers}{Your
  Ad Choices}
\item
  \href{https://www.nytimes.com/privacy}{Privacy}
\item
  \href{https://help.nytimes.com/hc/en-us/articles/115014893428-Terms-of-service}{Terms
  of Service}
\item
  \href{https://help.nytimes.com/hc/en-us/articles/115014893968-Terms-of-sale}{Terms
  of Sale}
\item
  \href{https://spiderbites.nytimes.com}{Site Map}
\item
  \href{https://help.nytimes.com/hc/en-us}{Help}
\item
  \href{https://www.nytimes.com/subscription?campaignId=37WXW}{Subscriptions}
\end{itemize}
