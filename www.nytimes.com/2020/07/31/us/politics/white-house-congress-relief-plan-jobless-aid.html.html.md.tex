Sections

SEARCH

\protect\hyperlink{site-content}{Skip to
content}\protect\hyperlink{site-index}{Skip to site index}

\href{https://www.nytimes.com/section/politics}{Politics}

\href{https://myaccount.nytimes.com/auth/login?response_type=cookie\&client_id=vi}{}

\href{https://www.nytimes.com/section/todayspaper}{Today's Paper}

\href{/section/politics}{Politics}\textbar{}White House and Congress
Clash on Relief Plan as Jobless Aid Expires

\url{https://nyti.ms/3i1eQmt}

\begin{itemize}
\item
\item
\item
\item
\item
\end{itemize}

Advertisement

\protect\hyperlink{after-top}{Continue reading the main story}

Supported by

\protect\hyperlink{after-sponsor}{Continue reading the main story}

\hypertarget{white-house-and-congress-clash-on-relief-plan-as-jobless-aid-expires}{%
\section{White House and Congress Clash on Relief Plan as Jobless Aid
Expires}\label{white-house-and-congress-clash-on-relief-plan-as-jobless-aid-expires}}

President Trump and Democrats blamed each other for the lapse of
\$600-per-week federal unemployment benefits, risking further economic
pain and a voter backlash for failing to act.

\includegraphics{https://static01.nyt.com/images/2020/07/31/us/politics/31dc-virus-cong/merlin_175160493_fd20e84b-c6d8-43b4-8450-d1067875391b-articleLarge.jpg?quality=75\&auto=webp\&disable=upscale}

\href{https://www.nytimes.com/by/emily-cochrane}{\includegraphics{https://static01.nyt.com/images/2018/11/28/multimedia/author-emily-cochrane/author-emily-cochrane-thumbLarge-v3.png}}

By \href{https://www.nytimes.com/by/emily-cochrane}{Emily Cochrane}

\begin{itemize}
\item
  July 31, 2020
\item
  \begin{itemize}
  \item
  \item
  \item
  \item
  \item
  \end{itemize}
\end{itemize}

WASHINGTON --- The White House and Congress hurtled over an economic and
political cliff on Friday, failing to reach agreement to extend
\$600-per-week federal jobless aid payments for millions of Americans
before their midnight expiration and risking a backlash from voters amid
a pandemic-induced recession.

Republicans and Democrats heaped blame on each other for the breakdown,
even as they said they would continue to try to strike an agreement on a
recovery package that would restore the benefits, which have become a
critical lifeline for laid-off workers and the economy itself. But in
allowing the lapse, President Trump and congressional leaders in both
parties were treading on politically perilous ground three months before
the general election.

The relief package under discussion is almost certainly the last chance
Mr. Trump has to inject fresh stimulus into the battered economy before
he faces voters in November, with his political standing damaged amid
widespread public dissatisfaction over his handling of the pandemic.
Republicans, whose hold on the Senate majority is under threat, could
also pay a steep price for failing to deliver relief to struggling
Americans, after waiting weeks to present their own proposal and
ultimately endorsing a deep cut to the enhanced jobless benefits.

But the impasse also amounts to a calculated risk for Democrats, who
have pressed to extend the full \$600-a-week federal unemployment
payments through January as part of a sweeping \$3 trillion recovery
plan. They opted to reject Republicans' last-ditch proposals for a
short-term extension or a continuation of the aid at a lower rate,
taking the position that allowing the benefits to expire was better than
agreeing to an aid proposal that they considered far too stingy to
meeting the needs of a crippled economy and a continuing public health
crisis.

``We don't have shared values --- that's just the way it is,'' Speaker
Nancy Pelosi of California said. ``It's not bickering. It's standing our
ground.''

Mr. Trump, who has been largely absent from the talks on a recovery
package, savaged Democrats for the impasse, saying they were jockeying
for political advantage instead of working to find a compromise.

``Democrats do not care about the people of our country,'' he said at
the White House. ``It is a disgrace that they're not negotiating, but
they're only looking to play a political game.''

As the two sides traded insults from either side of Pennsylvania Avenue,
the economy hung in the balance. The pandemic has obliterated
\href{https://www.nytimes.com/2020/07/30/business/economy/q2-gdp-coronavirus-economy.html}{nearly
five years of growth}, according to the latest government figures, which
came as the tally of new claims for state unemployment benefits exceeded
one million for the 19th consecutive week. Economists have warned that a
failure to enact additional federal relief could further devastate
American families and permanently damage an already shuddering economy.

Still, a deal to produce such a package appeared no closer.

Senate Republicans waited until Monday --- days after workers in several
states
\href{https://www.nytimes.com/2020/07/21/business/economy/coronavirus-unemployment-benefits.html}{received
their last unemployment benefit payments} --- to unveil their \$1
trillion relief proposal, and even then remained deeply divided on its
contents, with many preferring no action at all.

Democrats, who pushed their \$3 trillion stimulus plan through the House
in May, have rejected Republicans' efforts to buy time for negotiations
with a short-term extension of the unemployment benefits. Sensing that
they have the upper hand in the talks given the likelihood of
substantial Republican defections, Democrats appear reluctant to offer
many concessions.

In the absence of much common ground, the two sides resorted to
finger-pointing in dueling news conferences on Friday. Ms. Pelosi
charged that the administration officials who had huddled in her Capitol
Hill suite the night before had no grasp of the gravity of the pandemic
and its mounting toll. Mark Meadows, the White House chief of staff,
blamed the Democrats for the impasse, saying they had rejected four
proposals that would have maintained the unemployment benefits for a
brief period.

\includegraphics{https://static01.nyt.com/images/2020/07/31/us/31dc-virus-cong2/merlin_175162092_a59dd000-245f-4533-ac29-cb96c216493e-articleLarge.jpg?quality=75\&auto=webp\&disable=upscale}

A weeklong extension of the benefit, Ms. Pelosi said, would only suffice
had there been a broader agreement within reach and an additional few
days were needed to cement and pass such a deal.

``What are we going to do in a week?'' she said. ``We anticipate that we
will have a bill, but we're not there yet.''

Despite the bitter talk, there were glimmers of efforts to break through
the logjam. Ms. Pelosi is set to host Senator Chuck Schumer of New York,
the minority leader, Mr. Meadows and Steven Mnuchin, the Treasury
secretary, for a rare meeting on Saturday morning in her office on
Capitol Hill, according to an aide familiar with the plans. The House
postponed the start of its monthlong August recess until a deal was
struck, and on Thursday, before the Senate adjourned until Monday, the
procedural wheels for passing legislation had begun to creak forward.

But the stalemate could already be taking its toll. The sudden reduction
in buying power that comes with the loss of the enhanced unemployment
insurance benefit is likely to cause the economic slowdown to worsen, a
blow that comes as the chair of the Federal Reserve, Jerome H. Powell,
warned that the recovery
\href{https://www.nytimes.com/2020/07/29/business/economy/federal-reserve-meeting-interest-rates.html}{underway}
in May and June was slipping and other economists offered dour
predictions of little job growth and stubbornly high unemployment.

``The danger is that the economy stalls, and that we have double-digit
unemployment for months and months and months until a vaccine arrives,''
said Michael R. Strain, an economist at the conservative American
Enterprise Institute. ``That creates a situation where the roots of
economic weakness grow stronger and grow deeper, and you start to have
systematic and structural problems in the economy, over and above the
need to partially close due to the coronavirus.''

On Capitol Hill, lawmakers who support another sweeping infusion of
federal money across the American economy, including many who are facing
difficult re-election contests in November, said that a failure to
strike an agreement would be unacceptable.

``Congress has to rise to the crisis --- it is too serious,'' said
Senator Susan Collins, Republican of Maine. ``If we can't work together
in a bipartisan, bicameral way, in the midst of a persistent pandemic
that is causing such harm to people's health and to the economic
stability, then we will have failed the American people.''

Jim Tankersley, Michael Crowley and Zach Montague contributed reporting.

Advertisement

\protect\hyperlink{after-bottom}{Continue reading the main story}

\hypertarget{site-index}{%
\subsection{Site Index}\label{site-index}}

\hypertarget{site-information-navigation}{%
\subsection{Site Information
Navigation}\label{site-information-navigation}}

\begin{itemize}
\tightlist
\item
  \href{https://help.nytimes.com/hc/en-us/articles/115014792127-Copyright-notice}{©~2020~The
  New York Times Company}
\end{itemize}

\begin{itemize}
\tightlist
\item
  \href{https://www.nytco.com/}{NYTCo}
\item
  \href{https://help.nytimes.com/hc/en-us/articles/115015385887-Contact-Us}{Contact
  Us}
\item
  \href{https://www.nytco.com/careers/}{Work with us}
\item
  \href{https://nytmediakit.com/}{Advertise}
\item
  \href{http://www.tbrandstudio.com/}{T Brand Studio}
\item
  \href{https://www.nytimes.com/privacy/cookie-policy\#how-do-i-manage-trackers}{Your
  Ad Choices}
\item
  \href{https://www.nytimes.com/privacy}{Privacy}
\item
  \href{https://help.nytimes.com/hc/en-us/articles/115014893428-Terms-of-service}{Terms
  of Service}
\item
  \href{https://help.nytimes.com/hc/en-us/articles/115014893968-Terms-of-sale}{Terms
  of Sale}
\item
  \href{https://spiderbites.nytimes.com}{Site Map}
\item
  \href{https://help.nytimes.com/hc/en-us}{Help}
\item
  \href{https://www.nytimes.com/subscription?campaignId=37WXW}{Subscriptions}
\end{itemize}
