Sections

SEARCH

\protect\hyperlink{site-content}{Skip to
content}\protect\hyperlink{site-index}{Skip to site index}

\href{https://www.nytimes.com/section/us}{U.S.}

\href{https://myaccount.nytimes.com/auth/login?response_type=cookie\&client_id=vi}{}

\href{https://www.nytimes.com/section/todayspaper}{Today's Paper}

\href{/section/us}{U.S.}\textbar{}Supreme Court Lets Trump Keep Building
His Border Wall

\url{https://nyti.ms/3gljU4v}

\begin{itemize}
\item
\item
\item
\item
\item
\end{itemize}

\begin{itemize}
\item
  \href{https://www.nytimes.com/2020/07/31/us/elections/biden-vs-trump.html?action=click\&pgtype=Article\&state=default\&region=TOP_BANNER\&context=storylines_menu}{Election
  Updates}
\item
  \href{https://www.nytimes.com/article/biden-vice-president-2020.html?action=click\&pgtype=Article\&state=default\&region=TOP_BANNER\&context=storylines_menu}{Biden's
  V.P. Search}
\item
  \href{https://www.nytimes.com/interactive/2020/07/24/us/politics/trump-biden-campaign-donors.html?action=click\&pgtype=Article\&state=default\&region=TOP_BANNER\&context=storylines_menu}{Map
  of Donations}
\item
  \href{https://www.nytimes.com/interactive/2020/us/elections/delegate-count-primary-results.html?action=click\&pgtype=Article\&state=default\&region=TOP_BANNER\&context=storylines_menu}{Delegate
  Count}
\item
  \href{https://www.nytimes.com/interactive/2019/us/politics/2020-presidential-candidates.html?action=click\&pgtype=Article\&state=default\&region=TOP_BANNER\&context=storylines_menu}{The
  Candidates}
\item
  \href{https://www.nytimes.com/newsletters/politics?action=click\&pgtype=Article\&state=default\&region=TOP_BANNER\&context=storylines_menu}{Politics
  Newsletter}
\end{itemize}

Advertisement

\protect\hyperlink{after-top}{Continue reading the main story}

Supported by

\protect\hyperlink{after-sponsor}{Continue reading the main story}

\hypertarget{supreme-court-lets-trump-keep-building-his-border-wall}{%
\section{Supreme Court Lets Trump Keep Building His Border
Wall}\label{supreme-court-lets-trump-keep-building-his-border-wall}}

The court refused to lift a year-old stay notwithstanding an appeals
court ruling that the construction was unlawful.

\includegraphics{https://static01.nyt.com/images/2020/07/31/us/politics/31dc-scotus/merlin_169246947_dae8850a-dea7-4044-9afb-212e6287d838-articleLarge.jpg?quality=75\&auto=webp\&disable=upscale}

\href{https://www.nytimes.com/by/adam-liptak}{\includegraphics{https://static01.nyt.com/images/2018/07/13/multimedia/author-adam-liptak/author-adam-liptak-thumbLarge-v3.png}}

By \href{https://www.nytimes.com/by/adam-liptak}{Adam Liptak}

\begin{itemize}
\item
  July 31, 2020
\item
  \begin{itemize}
  \item
  \item
  \item
  \item
  \item
  \end{itemize}
\end{itemize}

WASHINGTON --- The Supreme Court on Friday
\href{https://www.supremecourt.gov/opinions/19pdf/19a60_bqm1.pdf}{rejected
a request} from environmental groups to stop construction of President
Trump's border wall while the administration seeks review of an appeals
court loss.

The vote was 5 to 4, with the court's more conservative members in the
majority. Its brief order was unsigned and gave no reasons, which is
typical when the court acts on emergency applications.

In dissent, Justice Stephen G. Breyer wrote that he feared that the
court's action effectively decided the case before the justices even
considered whether to hear the administration's appeal. Justices Ruth
Bader Ginsburg, Sonia Sotomayor and Elena Kagan joined the dissent.

A year ago, by the same 5-to-4 vote, the court
\href{https://www.supremecourt.gov/opinions/18pdf/19a60_o75p.pdf}{allowed
the administration} to start using \$2.5 billion in Pentagon money to
build the wall on the southwestern border while the case moved forward
in the lower courts. In June, a divided three-judge panel of the United
States Court of Appeals for the Ninth Circuit, in San Francisco,
\href{https://cdn.ca9.uscourts.gov/datastore/opinions/2020/06/26/19-16102.pdf}{ruled
against the administration}, saying Congress had not authorized the
spending.

But the Supreme Court's earlier order allowing the construction remains
in place and will not expire until the court either denies the
administration's petition seeking review or agrees to hear the
administration's appeal and rules on it.

On July 22, the Sierra Club and the Southern Border Communities
Coalition, represented by the American Civil Liberties Union,
\href{https://www.supremecourt.gov/DocketPDF/19/19A60/148405/20200722140912601_2020.07.21\%20stay\%20lift\%20motion\%20FINAL.pdf}{asked
the justices} to lift their earlier stay. The alternative, the groups
said, was to allow the administration to run out the litigation clock
and complete construction in Arizona, California and New Mexico even in
the face of an appeals court ruling that the work was unlawful.

The administration's deadline for filing its appeal is 150 days from the
appeals court's judgment. As a practical matter, the environmental
groups told the court, the disputed portions of the wall would be built
before the administration had to file its petition seeking review, in
late November.

\hypertarget{latest-updates-2020-election}{%
\section{\texorpdfstring{\href{https://www.nytimes.com/2020/07/31/us/elections/biden-vs-trump.html?action=click\&pgtype=Article\&state=default\&region=MAIN_CONTENT_1\&context=storylines_live_updates}{Latest
Updates: 2020
Election}}{Latest Updates: 2020 Election}}\label{latest-updates-2020-election}}

Updated 2020-08-01T01:26:45.732Z

\begin{itemize}
\tightlist
\item
  \href{https://www.nytimes.com/2020/07/31/us/elections/biden-vs-trump.html?action=click\&pgtype=Article\&state=default\&region=MAIN_CONTENT_1\&context=storylines_live_updates\#link-29fdff45}{Kamala
  Harris, a top vice-presidential contender, confronts double
  standards.}
\item
  \href{https://www.nytimes.com/2020/07/31/us/elections/biden-vs-trump.html?action=click\&pgtype=Article\&state=default\&region=MAIN_CONTENT_1\&context=storylines_live_updates\#link-13ec3d9c}{Karen
  Bass and Susan Rice are rising on Biden's vice-presidential
  shortlist.}
\item
  \href{https://www.nytimes.com/2020/07/31/us/elections/biden-vs-trump.html?action=click\&pgtype=Article\&state=default\&region=MAIN_CONTENT_1\&context=storylines_live_updates\#link-49e9a016}{Trump
  says Russian bounties to kill U.S. troops `never took place.'}
\end{itemize}

\href{https://www.nytimes.com/2020/07/31/us/elections/biden-vs-trump.html?action=click\&pgtype=Article\&state=default\&region=MAIN_CONTENT_1\&context=storylines_live_updates}{See
more updates}

``A stay should be just that: a stay,'' the brief said. ``Not a victory
for the party that has lost ** before every court that has adjudicated
the wall's legality.''

\href{https://www.supremecourt.gov/DocketPDF/19/19A60/148956/20200729131434023_19A60\%20Sierra\%20Club\%20-\%20Govt\%20Opp\%20to\%20Motion\%20to\%20Lift\%20Stay\%20-\%20final.pdf}{In
response}, Jeffrey B. Wall, the acting solicitor general, said nothing
of significance had changed since the Supreme Court issued its stay last
year.

He added that the administration planned to file its petition by Aug. 7,
far ahead of the deadline. That would allow the justices to consider
whether to hear the case at their first private conference after their
summer break, on Sept. 29.

In a
\href{https://www.supremecourt.gov/DocketPDF/19/19A60/149072/20200730130006392_2020.07.30\%20stay\%20lift\%20reply\%20FINAL.pdf}{second
brief}, the environmental groups said that the quicker timeline
contemplated by the government was too slow and ``still means that they
will complete the very border wall construction in dispute before this
court can hear argument on the case, much less render a decision.''

``Only by lifting the stay will the court ensure that it will have the
opportunity to resolve plaintiffs' claims on the merits before the wall
is built,'' the brief said.

The Supreme Court's earlier order was unsigned and only one paragraph
long, but it indicated that the groups challenging the administration
may not have a legal right to do so. That suggested that the court's
conservative majority was likely to side with the administration in the
end.

The lawsuit arose from Mr. Trump's efforts to make good on a campaign
promise to build the barrier. In early 2019, he
\href{https://www.nytimes.com/2019/02/15/us/politics/national-emergency-trump.html?module=inline}{declared
a national emergency along the Mexican border}. The declaration followed
a two-month impasse with Congress over funding to build the wall, a
standoff that gave rise to the
\href{https://www.nytimes.com/interactive/2019/01/09/us/politics/longest-government-shutdown.html}{longest
partial government shutdown} in the nation's history.

After Congress appropriated only a fraction of what Mr. Trump had
sought, he announced that he would act unilaterally to spend billions
more.

Soon after, the environmental groups sued to stop the president's plan
to use money meant for military programs to build barriers along the
border in what he said was an effort to combat drug trafficking.

\href{https://www.fjc.gov/history/judges/gilliam-haywood-stirling-jr}{Judge
Haywood S. Gilliam Jr.}, of the United States District Court for the
Northern District of California, blocked the effort in
\href{https://assets.documentcloud.org/documents/6026005/California-Border-Wall-20190524.pdf}{a
pair} of
\href{https://assets.documentcloud.org/documents/6177180/Sierra-Club-Ruling.pdf}{decisions}
that said the statute the administration had relied on to justify the
transfer did not authorize it.

``The case is not about whether the challenged border barrier
construction plan is wise or unwise,''
\href{https://assets.documentcloud.org/documents/6026005/California-Border-Wall-20190524.pdf}{Judge
Gilliam wrote}. ``It is not about whether the plan is the right or wrong
policy response to existing conditions at the southern border of the
United States. Instead, this case presents strictly legal questions
regarding whether the proposed plan for funding border barrier
construction exceeds the executive branch's lawful authority.''

The Ninth Circuit affirmed Judge Gilliam's injunction, saying that ``the
Constitution delegates exclusively to Congress the power of the purse.''

``The executive branch lacked independent constitutional authority to
authorize the transfer of funds,'' Judge Sidney R. Thomas wrote for the
majority. ``Therefore, the transfer of funds here was unlawful.''

In their first brief, the environmental groups urged the Supreme Court
to act before it was too late.

``Although some of plaintiffs' injuries can be reversed by taking down
the unlawful construction,'' the brief said, ``much of the damage
defendants are inflicting on the borderlands will be beyond repair.''

In a statement after the court acted on Friday, Dror Ladin, a lawyer
with the A.C.L.U., said ``the fight continues.''

``We'll be back before the Supreme Court soon to put a stop to Trump's
xenophobic border wall once and for all,'' he said. ``The administration
has admitted that the wall can be taken down if we ultimately prevail,
and we will hold them to their word and seek the removal of every mile
of unlawful wall built.''

\hypertarget{our-2020-election-guide}{%
\section{Our 2020 Election Guide}\label{our-2020-election-guide}}

Updated July 31, 2020

\begin{itemize}
\item
  \begin{center}\rule{0.5\linewidth}{\linethickness}\end{center}

  \hypertarget{the-latest}{%
  \subsection{The Latest}\label{the-latest}}

  \begin{itemize}
  \tightlist
  \item
    President Trump's assault on the Postal Service is intersecting with
    his attacks on mail-in voting.
    \href{https://www.nytimes.com/2020/07/31/us/politics/trump-usps-mail-delays.html?action=click\&pgtype=Article\&state=default\&region=BELOW_MAIN_CONTENT\&context=storylines_guide}{Voting
    rights groups say it is a recipe for disaster.}
  \end{itemize}
\item
  \begin{center}\rule{0.5\linewidth}{\linethickness}\end{center}

  \hypertarget{bidens-vp-search}{%
  \subsection{Biden's V.P. Search}\label{bidens-vp-search}}

  \begin{itemize}
  \tightlist
  \item
    \href{https://www.nytimes.com/article/biden-vice-president-2020.html?action=click\&pgtype=Article\&state=default\&region=BELOW_MAIN_CONTENT\&context=storylines_guide}{Here
    are 13 women} who have been under consideration to be Joe Biden's
    running mate, and why each might be chosen --- and might not be.
  \end{itemize}
\item
  \begin{center}\rule{0.5\linewidth}{\linethickness}\end{center}

  \hypertarget{keep-up-with-our-coverage}{%
  \subsection{Keep Up With Our
  Coverage}\label{keep-up-with-our-coverage}}

  \begin{itemize}
  \tightlist
  \item
    Get an
    \href{https://www.nytimes.com/newsletters/politics?action=click\&pgtype=Article\&state=default\&region=BELOW_MAIN_CONTENT\&context=storylines_guide}{email}
    recapping the day's news
  \end{itemize}

  \begin{itemize}
  \tightlist
  \item
    Download our mobile app on
    \href{https://apps.apple.com/us/app/nytimes/id284862083?ls=1\&mat_click_id=5c79ae7455014fd1bd66b5610c05b8f2-20191112-16948\&referrer=mat_click_id\%3D5c79ae7455014fd1bd66b5610c05b8f2-20191112-16948\%26link_click_id\%3D722930677036718082}{iOS}
    and
    \href{http://a.localytics.com/android?id=com.nytimes.android\&referrer=utm_source\%3Dother_nyt_mobile_web\%26utm_medium\%3DWeb\%2520page\%26utm_term\%3DGeneral\%2520Mobile\%2520Page\%26utm_campaign\%3DNYT\%2520Mobile\%2520General\%2520Page}{Android}
    and turn on Breaking News and Politics alerts
  \end{itemize}
\end{itemize}

Advertisement

\protect\hyperlink{after-bottom}{Continue reading the main story}

\hypertarget{site-index}{%
\subsection{Site Index}\label{site-index}}

\hypertarget{site-information-navigation}{%
\subsection{Site Information
Navigation}\label{site-information-navigation}}

\begin{itemize}
\tightlist
\item
  \href{https://help.nytimes.com/hc/en-us/articles/115014792127-Copyright-notice}{©~2020~The
  New York Times Company}
\end{itemize}

\begin{itemize}
\tightlist
\item
  \href{https://www.nytco.com/}{NYTCo}
\item
  \href{https://help.nytimes.com/hc/en-us/articles/115015385887-Contact-Us}{Contact
  Us}
\item
  \href{https://www.nytco.com/careers/}{Work with us}
\item
  \href{https://nytmediakit.com/}{Advertise}
\item
  \href{http://www.tbrandstudio.com/}{T Brand Studio}
\item
  \href{https://www.nytimes.com/privacy/cookie-policy\#how-do-i-manage-trackers}{Your
  Ad Choices}
\item
  \href{https://www.nytimes.com/privacy}{Privacy}
\item
  \href{https://help.nytimes.com/hc/en-us/articles/115014893428-Terms-of-service}{Terms
  of Service}
\item
  \href{https://help.nytimes.com/hc/en-us/articles/115014893968-Terms-of-sale}{Terms
  of Sale}
\item
  \href{https://spiderbites.nytimes.com}{Site Map}
\item
  \href{https://help.nytimes.com/hc/en-us}{Help}
\item
  \href{https://www.nytimes.com/subscription?campaignId=37WXW}{Subscriptions}
\end{itemize}
