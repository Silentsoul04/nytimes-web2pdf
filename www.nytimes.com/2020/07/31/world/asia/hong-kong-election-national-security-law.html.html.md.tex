Sections

SEARCH

\protect\hyperlink{site-content}{Skip to
content}\protect\hyperlink{site-index}{Skip to site index}

\href{/section/world/asia}{Asia Pacific}\textbar{}With Security Law as a
Cudgel, Beijing Cracks Down on Hong Kong

\url{https://nyti.ms/3191FJn}

\begin{itemize}
\item
\item
\item
\item
\item
\item
\end{itemize}

\includegraphics{https://static01.nyt.com/images/2020/07/31/world/31hk-crackdown-top/merlin_175152162_1fa5b592-489b-4b41-86b8-76274f3a02cf-articleLarge.jpg?quality=75\&auto=webp\&disable=upscale}

\hypertarget{with-security-law-as-a-cudgel-beijing-cracks-down-on-hong-kong}{%
\section{With Security Law as a Cudgel, Beijing Cracks Down on Hong
Kong}\label{with-security-law-as-a-cudgel-beijing-cracks-down-on-hong-kong}}

The spirit and the letter of the new law has been used to crush Hong
Kong's opposition. In the latest blow to the pro-democracy movement, the
government postponed an election, citing the coronavirus.

The Chinese government has used the national security law to crush Hong
Kong's pro-democracy opposition with a ferocity that has surprised even
some pro-Beijing officials in the territory.Credit...Lam Yik Fei for The
New York Times

Supported by

\protect\hyperlink{after-sponsor}{Continue reading the main story}

By \href{https://www.nytimes.com/by/keith-bradsher}{Keith Bradsher},
Elaine Yu and \href{https://www.nytimes.com/by/steven-lee-myers}{Steven
Lee Myers}

\begin{itemize}
\item
  July 31, 2020
\item
  \begin{itemize}
  \item
  \item
  \item
  \item
  \item
  \item
  \end{itemize}
\end{itemize}

BEIJING --- For weeks, as Beijing quickly drafted and imposed a
stringent new national security law for Hong Kong, many in the territory
feared the rules would be used to intimidate the opposition, but hoped
they would not presage a broad crackdown.

Now those hopes have been dashed.
\href{https://www.nytimes.com/2020/07/15/world/asia/china-trump-hong-kong.html}{Brushing
aside international criticism and sanctions}, the Chinese government has
used the letter and spirit of the law to crush Hong Kong's pro-democracy
opposition with surprising ferocity.

In the last week alone, the authorities have
\href{https://www.nytimes.com/2020/07/28/world/asia/benny-tai-hong-kong-university.html}{ousted
a tenured law professor} at the University of Hong Kong who has been a
key figure in the city's democracy movement, and arrested
\href{https://www.nytimes.com/2020/07/29/world/asia/hong-kong-arrests-security-law.html}{four
young activists} on suspicion that they expressed support online for
independence. They have also barred a dozen candidates from running for
the legislature, using opposition to the security law as new ground for
disqualification.

On Friday, the authorities
\href{https://www.nytimes.com/2020/07/31/world/asia/hong-kong-election-delayed.html}{postponed
for a year} the election itself, which had been scheduled for Sept. 6.
While they cited the coronavirus pandemic as justification for the move,
it underscored Beijing's fears that pro-democracy candidates could
triumph.

The breadth and severity of the actions reflect Beijing's urgency to
smother opposition to its encroaching authority over the territory after
more than a year of political upheaval there.

``More will come,'' said Victoria Tin-bor Hui, a political scientist
from Hong Kong at University of Notre Dame.

\includegraphics{https://static01.nyt.com/images/2020/07/31/world/00hk-crackdown-2/merlin_175140693_f4a51514-6c19-4bf0-9818-6e2f77b472ee-articleLarge.jpg?quality=75\&auto=webp\&disable=upscale}

The aggressive consolidation of power mirrors China's broader moves to
flex its political, economic and military might as the world is
distracted by the pandemic.

Western nations have pushed back aggressively against Beijing's
measures, imposing sanctions and even suspending extradition agreements
with Hong Kong, but to no avail. In some ways, it appears to have
emboldened China, which blames the dissent in Hong Kong on foreign
interference.

``The people of Hong Kong deserve to have their voice represented by the
elected officials that they choose in those elections,'' Secretary of
State Mike Pompeo said in a radio interview on Thursday ahead of the
postponement of the election. ``If they destroy that, if they take that
down, it will be another marker that will simply prove that the Chinese
Communist Party has now made Hong Kong just another communist-run
city.''

Wang Wenbin, the chief spokesman of China's Ministry of Foreign Affairs,
said that Beijing was allowing the Hong Kong authorities to decide the
timing of the election. But he also insisted that Beijing would not be
dissuaded by any foreign countries from doing what it deems necessary in
Hong Kong.

``China is not afraid of intimidation by any external forces --- our
determination is unwavering and unshakable in safeguarding national
sovereignty, security, and development interests,'' he said.

Image

Police officers clashing with protesters at a Hong Kong shopping mall on
July 1.Credit...Lam Yik Fei for The New York Times

On the mainland, China thoroughly stifles political dissent. For the
authorities there, Hong Kong --- with its nominal political autonomy and
robust democracy movement --- has been a major irritant, especially
after huge protests openly and at times violently challenged Beijing's
control and even sovereignty over the city last year.

With its crackdown, Beijing is following the authoritarian playbook of
countries like Russia, holding elections but managing them so that they
cease to reflect genuine voter will. Russia's leader, Vladimir V. Putin,
recently
\href{https://www.nytimes.com/2020/07/01/world/europe/putin-referendum-vote-russia.html}{orchestrated}
a constitutional referendum to perpetuate his rule --- and then followed
it up with a series of arrests, hoping to smother discontent before it
could gain momentum.

``They are running short of confidence to face the people, to face the
people's choice, to face the people's demands,'' said Alvin Yeung, a
sitting pro-democracy lawmaker from the moderate Civic Party who was
disqualified from running on Thursday. ``It's fear.''

Years in the making, the national security law created a climate of fear
and uncertainty in only a matter of hours after it was imposed.

The day the law took effect, the police detained 10 protesters for
national security violations, including a young man on a motorcycle with
a Hong Kong liberation flag who collided with police officers. Tong
Ying-kit, who was hospitalized after the collision, was later the first
charged under the new law. He remains in custody.

Image

The newly created agency to enforce the law, the Office for Safeguarding
National Security, began working out of the Metropark Hotel in Causeway
Bay.Credit...Lam Yik Fei for The New York Times

The newly created agency to enforce the law, the Office for Safeguarding
National Security, soon took up residence in the Metropark Hotel in
Causeway Bay (a 4.5 rating on Tripadvisor.com) and surrounded it with
barricades in a physical manifestation of Beijing's growing
authoritarian footprint on the city.

The agency is headed by Zheng Yanxiong, a senior Communist Party
official dispatched from Guangdong, the neighboring province on the
mainland. He is best known for his
\href{https://www.nytimes.com/2011/12/31/world/asia/chinese-official-wang-yang-tests-new-political-approach.html}{hostility
to a short-lived democracy experiment} in a Guangdong village, Wukan,
nearly a decade ago.

On July 10, the Hong Kong police
\href{https://www.nytimes.com/2020/07/10/world/asia/hong-kong-police-raid-pollster.html}{raided}
an independent polling institute whose computers were being used by
democracy supporters for an unofficial primary to decide which
candidates would run for the legislature. Five days later, the police
arrested five activists, including a vice chairman of the territory's
Democratic Party, in connection with protests and violent clashes at
Hong Kong Polytechnic University last November.

The four young activists arrested on Wednesday were all former members
of Studentlocalism, a pro-independence group led by secondary school
students that ended its operations just before the security law took
effect. In the past, the group had typically distributed leaflets
supporting independence outside schools.

Regina Ip, a cabinet member and the leader of a small pro-Beijing
political party in the legislature, welcomed the arrest on Wednesday of
the four activists, who ranged in age from 16 to 21. She said that their
postings showed continued support for Hong Kong independence after the
law went into effect, although the police have not elaborated on what
the four specifically said.

Their arrest shows that the authorities are ``acting in accordance with
the law,'' she said.

Image

Protesters in Hong Kong held blank pages on July 3 to avoid running
afoul of the new security law.Credit...Lam Yik Fei for The New York
Times

On Friday evening, Chinese state television reported that the Hong Kong
police had issued warrants for the arrests of six democracy advocates
who are now overseas. They are wanted on charges of promoting secession
and colluding with foreign forces, according to the report --- crimes
that are punishable with life imprisonment under the security law. The
police declined to comment.

One of the six, Samuel Chu,
\href{https://twitter.com/samuelmchu/status/1289256588868370432}{said on
Twitter} that he had been an American citizen for 25 years.

Also on Friday, Hong Kong's secretary for justice said that David Leung,
the city's British-trained director of prosecutions, had submitted his
resignation. Pro-Beijing politicians and the police had accused Mr.
Leung of being too cautious about bringing charges against protesters,
though he had prosecuted some high-profile activists. Mr. Leung did not
issue a statement.

The spirit of the security law has been used to justify the dismissal of
the professor at the University of Hong Kong, Benny Tai. After the
rollout of the rules, Mr. Tai, who was convicted of public nuisance for
his role in protests in 2014, helped organize the recent primary vote
for the pro-democracy camp.

Beijing's Liaison Office in Hong Kong supported his removal, describing
it in a statement as ``a just act of punishing evil and promoting good
and conforming to the people's will.'' The security law was also invoked
this week for the disqualifications of candidates for the legislature.

The legislature cannot have seats ``for these unscrupulous individuals
who are plotting to destroy'' Hong Kong, the liaison office said. The
Hong Kong government said that candidates who objected ``in principle''
to Beijing's enactment of the law were violating the oath to uphold Hong
Kong's constitution.

The government also said it was unconstitutional to vow **** to block
its legislative proposals in order to pressure the administration. Some
opposition lawmakers had floated the idea of voting down the
government's budget. Under Hong Kong's mini-constitution, known as the
Basic Law, that could force the resignation of the chief executive,
Carrie Lam, and new elections.

Image

Hong Kong's chief executive, Carrie Lam, during a news conference on
Friday.Credit...Lam Yik Fei for The New York Times

The yearlong delay in the election now gives the authorities time to
disqualify more pro-democracy candidates from running and quash any
remaining momentum of the anti-government movement.

While the protests have largely quieted down since the law was imposed,
the opposition had been looking toward the election as a way to revive
their cause. The pro-democracy camp had been hoping for big gains in the
voting, following their
\href{https://www.nytimes.com/2019/11/25/world/asia/hong-kong-election-protests.html}{landslide
victory last fall in district elections}. With many of their most
prominent leaders now banned from running, their chances seem less
certain.

Mr. Yeung, the lawmaker, said the government had failed to show that the
election could not go ahead during the pandemic. He cited safety
measures adopted by South Korea and Singapore during recent elections.

``How on earth can they convince the rest of the world, including Hong
Kong people and the international community, that they have no other
ulterior motives other than public health concerns?'' he said.

Ms. Hui, from Notre Dame, compared Beijing's strategy to Sun Tzu's ``The
Art of War.'' Beijing, she said, has exploited advantages to defeat its
perceived enemies, not just with the security law in Hong Kong, but also
in the South China Sea, on China's border with India and in other
contested areas.

``The harshness of the law cannot be measured by the number of
arrests,'' she said, ``but by the deterrent effects on silencing anyone
who dares to dissent.''

Image

The government painted over protest messages at a ``Lennon Wall'' in the
Sai Wan Ho neighborhood in Hong Kong earlier this month.Credit...Lam Yik
Fei for The New York Times

Keith Bradsher reported from Beijing, Elaine Yu reported from Hong Kong
and Steven Lee Myers reported from Seoul.

Advertisement

\protect\hyperlink{after-bottom}{Continue reading the main story}

\hypertarget{site-index}{%
\subsection{Site Index}\label{site-index}}

\hypertarget{site-information-navigation}{%
\subsection{Site Information
Navigation}\label{site-information-navigation}}

\begin{itemize}
\tightlist
\item
  \href{https://help.nytimes.com/hc/en-us/articles/115014792127-Copyright-notice}{©~2020~The
  New York Times Company}
\end{itemize}

\begin{itemize}
\tightlist
\item
  \href{https://www.nytco.com/}{NYTCo}
\item
  \href{https://help.nytimes.com/hc/en-us/articles/115015385887-Contact-Us}{Contact
  Us}
\item
  \href{https://www.nytco.com/careers/}{Work with us}
\item
  \href{https://nytmediakit.com/}{Advertise}
\item
  \href{http://www.tbrandstudio.com/}{T Brand Studio}
\item
  \href{https://www.nytimes.com/privacy/cookie-policy\#how-do-i-manage-trackers}{Your
  Ad Choices}
\item
  \href{https://www.nytimes.com/privacy}{Privacy}
\item
  \href{https://help.nytimes.com/hc/en-us/articles/115014893428-Terms-of-service}{Terms
  of Service}
\item
  \href{https://help.nytimes.com/hc/en-us/articles/115014893968-Terms-of-sale}{Terms
  of Sale}
\item
  \href{https://spiderbites.nytimes.com}{Site Map}
\item
  \href{https://help.nytimes.com/hc/en-us}{Help}
\item
  \href{https://www.nytimes.com/subscription?campaignId=37WXW}{Subscriptions}
\end{itemize}
