Sections

SEARCH

\protect\hyperlink{site-content}{Skip to
content}\protect\hyperlink{site-index}{Skip to site index}

\href{https://www.nytimes.com/section/world/middleeast}{Middle East}

\href{https://myaccount.nytimes.com/auth/login?response_type=cookie\&client_id=vi}{}

\href{https://www.nytimes.com/section/todayspaper}{Today's Paper}

\href{/section/world/middleeast}{Middle East}\textbar{}Trump Officials
Reconsider Prosecuting ISIS `Beatles' Without Death Penalty

\url{https://nyti.ms/3fh9SjI}

\begin{itemize}
\item
\item
\item
\item
\item
\end{itemize}

Advertisement

\protect\hyperlink{after-top}{Continue reading the main story}

Supported by

\protect\hyperlink{after-sponsor}{Continue reading the main story}

\hypertarget{trump-officials-reconsider-prosecuting-isis-beatles-without-death-penalty}{%
\section{Trump Officials Reconsider Prosecuting ISIS `Beatles' Without
Death
Penalty}\label{trump-officials-reconsider-prosecuting-isis-beatles-without-death-penalty}}

American military officials in Iraq want two detainees taken off their
hands and a British court has blocked sharing evidence for any
death-penalty case. But other options are also getting a second look.

\includegraphics{https://static01.nyt.com/images/2020/07/31/us/politics/31dc-isis-beatles/merlin_136198725_e417b546-3fd9-4835-a71f-c60e3d42c320-articleLarge.jpg?quality=75\&auto=webp\&disable=upscale}

\href{https://www.nytimes.com/by/charlie-savage}{\includegraphics{https://static01.nyt.com/images/2018/06/12/multimedia/author-charlie-savage/author-charlie-savage-thumbLarge-v2.png}}\href{https://www.nytimes.com/by/eric-schmitt}{\includegraphics{https://static01.nyt.com/images/2018/06/12/multimedia/author-eric-schmitt/author-eric-schmitt-thumbLarge-v2.png}}

By \href{https://www.nytimes.com/by/charlie-savage}{Charlie Savage} and
\href{https://www.nytimes.com/by/eric-schmitt}{Eric Schmitt}

\begin{itemize}
\item
  July 31, 2020
\item
  \begin{itemize}
  \item
  \item
  \item
  \item
  \item
  \end{itemize}
\end{itemize}

WASHINGTON --- The Trump administration is trying again to find a way to
resolve the cases of two British Islamic State detainees who are
notorious for their roles in the torture and killing of Western
hostages, and who have been held in indefinite wartime detention by the
American military in Iraq since October, according to officials.

One option under renewed consideration is for the Justice Department to
drop its insistence that prosecutors be free to bring capital charges
against the men, half of a cell of Britons called the ``Beatles'' by
their captives because of their accents.

Since the men were captured in early 2018, when Jeff Sessions was
attorney general, the Justice Department has insisted that it be free to
seek their execution. But at an interagency National Security Council
meeting this week, Attorney General William P. Barr did not rule out
dropping that stance, officials said.

A chief obstacle to bringing the men to trial has been a need for
evidence held by the British government. Britain has abolished the death
penalty and a British court
\href{https://www.nytimes.com/2020/03/25/us/isis-beatles-death-penalty.html}{has
blocked it from cooperating} in capital charges. Litigation is slowly
continuing, but assurances that American prosecutors would not seek the
death penalty could swiftly make the evidence available.

Mr. Barr's indication that he is at least willing to consider changing
the department's position was
\href{https://www.washingtonpost.com/national-security/ag-barr-willing-to-consider-forgoing-death-penalty-to-secure-prosecution-of-isis-detainees-allegedly-involved-in-beheadings-of-american-hostages/2020/07/31/71e475f0-cdd4-11ea-91f1-28aca4d833a0_story.html?hpid=hp_hp-more-top-stories_hostages-205pm\%3Ahomepage\%2Fstory-ans}{first
reported by The Washington Post}, and an official familiar with internal
deliberations confirmed it.

But several officials stressed that this did not amount to a final
policy decision, and was just one of several previously rejected options
now being reconsidered.

\includegraphics{https://static01.nyt.com/images/2020/07/31/us/politics/31dc-isis-beatles2/merlin_175033842_b84422df-09b0-4c2a-9bc5-af90b59c3c18-articleLarge.jpg?quality=75\&auto=webp\&disable=upscale}

Other potential resolutions that are being reopened for discussion, they
said, include transferring custody of the men to the Iraqi government
for prosecution; taking them to the American military prison at
Guantánamo Bay, Cuba, for continued wartime detention without trial; and
revisiting whether the British evidence is truly crucial --- or whether
prosecutors might be able to mount a capital trial without it.

But since all of the other options bring their own problems and
complexities, the Justice Department's willingness to discuss the idea
of seeking life in prison instead of death was seen internally as
potentially significant.

The families of four of their American victims
\href{https://www.nytimes.com/2018/02/16/opinion/justice-isis-trial.html}{have
long said} it would be a mistake to send the detainees to Guantánamo Bay
or to seek the death penalty, and instead have advocated seeking justice
in a way that would not make the men into martyrs.

Representatives for the National Security Council and the Justice
Department declined to comment. The officials familiar with internal
deliberations spoke on the condition of anonymity.

The meeting was scheduled shortly after the four families published
\href{https://www.washingtonpost.com/opinions/2020/07/23/our-children-were-killed-by-islamic-state-members-they-must-face-trial/}{a
column in The Post} last week reiterating their call for the department
to move forward with prosecuting the men, expressing worries that they
could escape justice and restating their opposition to continuing to
hold them in long-term detention without trial.

``We implore the Trump administration: Please, for the sake of truth,
for the sake of justice, order these Islamic State suspects transferred
to the United States to face trial,'' they wrote.

The two men,
\href{https://www.washingtonpost.com/world/national-security/another-islamic-state-jailer-who-held-western-hostages-identified-as-londoner/2016/02/06/a0f11d28-cc10-11e5-ae11-57b6aeab993f_story.html?utm_term=.cdee9a6a6ebf}{Alexanda
Kotey} and
\href{https://www.washingtonpost.com/world/national-security/that-is-not-the-son-i-raised-how-a-british-citizen-became-one-of-the-most-notorious-members-of-isis/2016/05/23/6d66276c-1cfd-11e6-b6e0-c53b7ef63b45_story.html?utm_term=.d86c7802d277}{El
Shafee Elsheikh}, were part of a cell whose gruesome hostage beheadings
for Islamic State propaganda videos drew widespread attention in 2014.
Among their victims was James Foley, the American journalist who was
\href{https://www.nytimes.com/2014/08/20/world/middleeast/isis-james-foley-syria-execution.html?module=inline}{beheaded
that August.}

Another member of the cell, Mohammed Emwazi, or ``Jihadi John,'' is
believed to have killed Mr. Foley. Mr. Emwazi was later killed in a
drone strike. A fourth man, Aine Davis, has been
\href{https://www.theguardian.com/world/2017/may/09/british-jihadist-aine-davis-convicted-in-turkey-on-terror-charges}{imprisoned
in Turkey} on terrorism charges.

Mr. Kotey and Mr. Elsheikh were captured in Syria by a Kurdish militia,
which held them there with
\href{https://www.nytimes.com/2018/07/18/world/middleeast/islamic-state-detainees-syria-prisons.html?action=click\&module=RelatedCoverage\&pgtype=Article\&region=Footer}{numerous
other Islamic State detainees from Western countries that have refused
to take back their citizens.} The British government moved to strip them
of citizenship and made clear it did not want to take them back.

In part because Britain was not moving to solve the problem created by
its own citizens, the Trump administration was not willing to nod to its
legal system by offering an assurance that American prosecutors, if they
handled it, would not seek to impose the death penalty, which remains
legal in the United States.

After
\href{https://www.nytimes.com/2018/02/28/us/politics/britain-death-penalty-isis.html}{initial
reluctance}, the British government
m\href{https://www.nytimes.com/2019/01/18/us/politics/british-isis-prosecution-beatles.html}{oved
to share the evidence without such assurances}, and showed witness
statements and other material it had gathered about the two men to the
Justice Department. But testimony from British government officials
would also probably be necessary at any trial to make the evidence
admissible.

To block continued cooperation, Mr. Elsheikh's mother filed a lawsuit,
and won an initial ruling in March.

In the meantime, in October, the Turkish military
\href{https://www.nytimes.com/2019/10/09/world/middleeast/turkey-attacks-syria.html}{moved
into northern Syria} against the American-backed Kurds after getting a
green light from President Trump,
\href{https://www.nytimes.com/2019/10/07/us/politics/isis-prisons-detainees.html?}{calling
into question} the militia's ability to continue securely holding some
11,000 captured Islamic State fighters.

The American military
\href{https://www.nytimes.com/2019/10/09/us/politics/beatles-isis-us-custody.html}{took
custody of Mr. Kotey and Mr. Elsheikh} and took them to Iraq to ensure
they would remain locked up. But since then it has grown increasingly
impatient to hand them off.

Advertisement

\protect\hyperlink{after-bottom}{Continue reading the main story}

\hypertarget{site-index}{%
\subsection{Site Index}\label{site-index}}

\hypertarget{site-information-navigation}{%
\subsection{Site Information
Navigation}\label{site-information-navigation}}

\begin{itemize}
\tightlist
\item
  \href{https://help.nytimes.com/hc/en-us/articles/115014792127-Copyright-notice}{©~2020~The
  New York Times Company}
\end{itemize}

\begin{itemize}
\tightlist
\item
  \href{https://www.nytco.com/}{NYTCo}
\item
  \href{https://help.nytimes.com/hc/en-us/articles/115015385887-Contact-Us}{Contact
  Us}
\item
  \href{https://www.nytco.com/careers/}{Work with us}
\item
  \href{https://nytmediakit.com/}{Advertise}
\item
  \href{http://www.tbrandstudio.com/}{T Brand Studio}
\item
  \href{https://www.nytimes.com/privacy/cookie-policy\#how-do-i-manage-trackers}{Your
  Ad Choices}
\item
  \href{https://www.nytimes.com/privacy}{Privacy}
\item
  \href{https://help.nytimes.com/hc/en-us/articles/115014893428-Terms-of-service}{Terms
  of Service}
\item
  \href{https://help.nytimes.com/hc/en-us/articles/115014893968-Terms-of-sale}{Terms
  of Sale}
\item
  \href{https://spiderbites.nytimes.com}{Site Map}
\item
  \href{https://help.nytimes.com/hc/en-us}{Help}
\item
  \href{https://www.nytimes.com/subscription?campaignId=37WXW}{Subscriptions}
\end{itemize}
