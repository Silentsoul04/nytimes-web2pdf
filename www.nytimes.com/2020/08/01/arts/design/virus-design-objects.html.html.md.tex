Sections

SEARCH

\protect\hyperlink{site-content}{Skip to
content}\protect\hyperlink{site-index}{Skip to site index}

\href{https://www.nytimes.com/section/arts/design}{Art \& Design}

\href{https://myaccount.nytimes.com/auth/login?response_type=cookie\&client_id=vi}{}

\href{https://www.nytimes.com/section/todayspaper}{Today's Paper}

\href{/section/arts/design}{Art \& Design}\textbar{}The Strange Lives of
Objects in the Coronavirus Era

\url{https://nyti.ms/2BTepv2}

\begin{itemize}
\item
\item
\item
\item
\item
\end{itemize}

\href{https://www.nytimes.com/spotlight/at-home?action=click\&pgtype=Article\&state=default\&region=TOP_BANNER\&context=at_home_menu}{At
Home}

\begin{itemize}
\tightlist
\item
  \href{https://www.nytimes.com/2020/08/03/well/family/the-benefits-of-talking-to-strangers.html?action=click\&pgtype=Article\&state=default\&region=TOP_BANNER\&context=at_home_menu}{Talk:
  To Strangers}
\item
  \href{https://www.nytimes.com/2020/08/01/at-home/coronavirus-make-pizza-on-a-grill.html?action=click\&pgtype=Article\&state=default\&region=TOP_BANNER\&context=at_home_menu}{Make:
  Grilled Pizza}
\item
  \href{https://www.nytimes.com/2020/07/31/arts/television/goldbergs-abc-stream.html?action=click\&pgtype=Article\&state=default\&region=TOP_BANNER\&context=at_home_menu}{Watch:
  'The Goldbergs'}
\item
  \href{https://www.nytimes.com/interactive/2020/at-home/even-more-reporters-editors-diaries-lists-recommendations.html?action=click\&pgtype=Article\&state=default\&region=TOP_BANNER\&context=at_home_menu}{Explore:
  Reporters' Google Docs}
\end{itemize}

Advertisement

\protect\hyperlink{after-top}{Continue reading the main story}

Supported by

\protect\hyperlink{after-sponsor}{Continue reading the main story}

Surfacing

\hypertarget{the-strange-lives-of-objects-in-the-coronavirus-era}{%
\section{The Strange Lives of Objects in the Coronavirus
Era}\label{the-strange-lives-of-objects-in-the-coronavirus-era}}

The pandemic has inspired a flurry of new and novel items --- and given
ordinary ones new meanings.

By Sophie Haigney

Illustrations by Peter Arkle

\includegraphics{https://static01.nyt.com/images/2020/07/31/arts/31surfacing-virus-souvenirs3-13/31surfacing-virus-souvenirs3-13-articleLarge.png?quality=75\&auto=webp\&disable=upscale}

Plastic bubbles that hover over restaurant tables. Rods for contactless
elevator-button-pushing. Portable seats that attach to lampposts, for
shoppers waiting outside crowd-controlled stores. Dresses with skirts
that have a six-foot radius. Podlike enclosures to keep gym-goers
separate. A plastic sleeve that enables hugging at nursing homes. Masks
in every imaginable form.

A set of new objects has emerged in the last few months to address the
new reality of illness, lockdown, social distancing and social protest.
Some of these objects are wacky and unrealized --- speculative concepts
that may never see the light of day. Others, like cocktails-in-a-bag,
thermometers and all manner of partitions, are already circulating
widely. And some aren't new at all: familiar household items like
bottles of Lysol and rolls of toilet paper, which have taken on new
meaning and importance because of scarcity or sudden unusual needs.

Image

Image

Image

``I'm thinking a lot about what these objects are going to say about the
pandemic in the future,'' said Anna Talley, a master's student in the
history of design at the Victoria and Albert Museum and the Royal
College of Art. Talley and a fellow student, Fleur Elkerton, have
compiled an expansive online archive called
\href{https://designinquarantine.com/}{Design in Quarantine}. Some of
these objects are whimsical, or a little ridiculous, like an ultra-large
``distancing'' crown distributed by a German Burger King in May. Others
are the heartbreaking artifacts of illness and mass death, economic
collapse and crisis.

Image

``Objects can give us an insight into a time period that documents
cannot,'' said Alexandra Lord, chair of the medicine and science
division at the Smithsonian's
\href{https://americanhistory.si.edu}{National Museum of American
History}, who is helping to lead the museum's Covid-19 collecting task
force. As at many museums, curators there are engaging in what's called
\href{https://www.nytimes.com/2020/07/14/style/museums-coronavirus-protests-2020.html}{rapid
response collecting}, trying to gather material and objects even as the
crisis unfolds. The nature of the pandemic has made it difficult to
gather physical objects, but Lord and her colleagues have solicited
ideas and offers from the public. They are trying to determine what will
be crucial to future historians and viewers, even as the crisis
continues to unfold.

Image

Image

Image

``We as historians like to have hindsight, but we already know certain
objects like ventilators will be a crucial part of the story,'' Lord
said. Masks, too, have become
\href{https://www.nytimes.com/2020/05/15/style/the-hidden-language-of-masks-smithsonian.html}{symbols
of the crisis} in their myriad and already-evolving forms: hand-sewn,
N95, high-fashion, reusable, disposable.

At the \href{https://www.nyhistory.org}{New-York Historical Society},
historians have been collecting since mid-March, trying to gather things
that tell a specific story about the city's experience. They began
making a collecting wish list that included signs about store closures
in different languages, bottles from distilleries that were converted
into bottles for hand sanitizers, and the blanket of a baby born amid
the pandemic.

Image

Image

Image

``There's a white polo shirt that the governor tends to wear when he's
been doing his daily press briefings,'' Louise Mirrer, president and
chief executive of the New-York Historical Society, said in May, when
Gov. Andrew Cuomo was doing daily briefings. ``We'd like to have that,
and we will ask him for that.'' (As of publication time, it remains on
the wish list).

The New-York Historical Society is also seeking objects that illustrate
the personal toll of the pandemic --- some of which would be difficult
to collect now. ``There are some more sensitive objects that we'll ask
for later, like artifacts from people who have lost friends and
relatives,'' Mirrer said.

Image

Image

Image

Some ordinary objects have transformed into artifacts, either because of
the shadow of loss, or simply because of their newfound importance as
the crisis continues to shift. Some of the early fads of the pandemic
may already feel like relics of the past. ``Things from April seem old
already,'' said Donna Braden, senior curator at the
\href{https://www.thehenryford.org}{Henry Ford Museum}. ``It was almost
easier to identify those iconic objects early on, and now the crisis has
become so fragmented and so pervasive.''

The protests in June also marked a significant change, and a major
collecting event for history museums. The New-York Historical Society,
for instance, has collected a mural depicting George Floyd by the
artists Matt Adamson and Joaquin G that covered a boarded-up shoe store
in Soho. They've also collected protest signs and posters.

Image

Image

Image

Some objects exist at a kind overlap between the protests and the
pandemic, records that tell two narratives at once. ``At the Black Lives
Matter protests, many people are carrying signs that reference the fact
that Covid-19 is impacting communities of color disproportionately, and
that this is all part of this bigger story about systemic racism in the
U.S.,'' Lord said.

Some of the objects with which we've become familiar throughout the
pandemic have undergone changes or will have renewed meaning during
reopenings. ``Now there are also masks for kids who are going back to
school, these Crayola masks that are one for every day, then you put
them in a sealable package and wash them,'' Braden said.

Image

Image

Image

A number of the new designs and proposals might fall into the category
of what the architecture critic Kate Wagner describes as
``\href{https://mcmansionhell.com/post/618938984050147328/coronagrifting-a-design-phenomenon}{coronagrifting}'':
a trend defined by the emergence of ``cheap mockups of Covid-related
design `solutions''' **** that are substanceless but garner attention on
Instagram. Talley and Elkerton, of Design in Quarantine, are conscious
that some of the more outlandish designs in their archive might fall
into that category. ``We've been asked a bit about including quite
speculative and conceptual designs from design practices or designers
that can't be actualized and maybe are just responding to the pandemic
to get the publicity,'' Elkerton said. ``For a while we were wondering,
Are we actively promoting that by including these things? But we're just
trying to document what is happening in the design world, and the
`coronagrifting' projects are interesting in themselves.''

They've also become interested, Elkerton said, in ``failed designs.''
``As a historian, it's often more interesting to find out why something
doesn't work or take hold than what does,'' she said.

Image

Image

There is something both poignant and hopeful in these acts of
documentation and collection, in trying to look back at our current
crisis through the imagined lens of history. In collecting present
objects as artifacts of the future, we're imagining that future as a
kind of afterward --- a time and place where this is no longer ongoing,
and we can look back.

Image

Image

Image

As historians and curators begin to collect and document, many of us
have become engaged in a kind of self-archiving: documenting lockdowns
and sicknesses, saving newspaper articles and children's art projects,
building what amounts to pandemic collections. ``I find it really
interesting that people are becoming almost historians of their own
lives,'' Lord said.

We are by definition always living through history, but a crisis like
this brings it into relief: We sense the significance of this time for
future observers, and have the urge to preserve it.

\begin{center}\rule{0.5\linewidth}{\linethickness}\end{center}

\href{https://www.nytimes.com/series/surfacing}{Surfacing} is a biweekly
column that explores the intersection of art and life, produced by
Alicia DeSantis, Gabriel Gianordoli, Jolie Ruben and Josephine Sedgwick.

Advertisement

\protect\hyperlink{after-bottom}{Continue reading the main story}

\hypertarget{site-index}{%
\subsection{Site Index}\label{site-index}}

\hypertarget{site-information-navigation}{%
\subsection{Site Information
Navigation}\label{site-information-navigation}}

\begin{itemize}
\tightlist
\item
  \href{https://help.nytimes.com/hc/en-us/articles/115014792127-Copyright-notice}{©~2020~The
  New York Times Company}
\end{itemize}

\begin{itemize}
\tightlist
\item
  \href{https://www.nytco.com/}{NYTCo}
\item
  \href{https://help.nytimes.com/hc/en-us/articles/115015385887-Contact-Us}{Contact
  Us}
\item
  \href{https://www.nytco.com/careers/}{Work with us}
\item
  \href{https://nytmediakit.com/}{Advertise}
\item
  \href{http://www.tbrandstudio.com/}{T Brand Studio}
\item
  \href{https://www.nytimes.com/privacy/cookie-policy\#how-do-i-manage-trackers}{Your
  Ad Choices}
\item
  \href{https://www.nytimes.com/privacy}{Privacy}
\item
  \href{https://help.nytimes.com/hc/en-us/articles/115014893428-Terms-of-service}{Terms
  of Service}
\item
  \href{https://help.nytimes.com/hc/en-us/articles/115014893968-Terms-of-sale}{Terms
  of Sale}
\item
  \href{https://spiderbites.nytimes.com}{Site Map}
\item
  \href{https://help.nytimes.com/hc/en-us}{Help}
\item
  \href{https://www.nytimes.com/subscription?campaignId=37WXW}{Subscriptions}
\end{itemize}
