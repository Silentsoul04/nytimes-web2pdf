Sections

SEARCH

\protect\hyperlink{site-content}{Skip to
content}\protect\hyperlink{site-index}{Skip to site index}

\href{https://www.nytimes.com/spotlight/at-home}{At Home}

\href{https://myaccount.nytimes.com/auth/login?response_type=cookie\&client_id=vi}{}

\href{https://www.nytimes.com/section/todayspaper}{Today's Paper}

\href{/spotlight/at-home}{At Home}\textbar{}How to Stay Safer on Mass
Transit

\url{https://nyti.ms/30k8YP9}

\begin{itemize}
\item
\item
\item
\item
\item
\end{itemize}

\href{https://www.nytimes.com/spotlight/at-home?action=click\&pgtype=Article\&state=default\&region=TOP_BANNER\&context=at_home_menu}{At
Home}

\begin{itemize}
\tightlist
\item
  \href{https://www.nytimes.com/2020/08/03/well/family/the-benefits-of-talking-to-strangers.html?action=click\&pgtype=Article\&state=default\&region=TOP_BANNER\&context=at_home_menu}{Talk:
  To Strangers}
\item
  \href{https://www.nytimes.com/2020/08/01/at-home/coronavirus-make-pizza-on-a-grill.html?action=click\&pgtype=Article\&state=default\&region=TOP_BANNER\&context=at_home_menu}{Make:
  Grilled Pizza}
\item
  \href{https://www.nytimes.com/2020/07/31/arts/television/goldbergs-abc-stream.html?action=click\&pgtype=Article\&state=default\&region=TOP_BANNER\&context=at_home_menu}{Watch:
  'The Goldbergs'}
\item
  \href{https://www.nytimes.com/interactive/2020/at-home/even-more-reporters-editors-diaries-lists-recommendations.html?action=click\&pgtype=Article\&state=default\&region=TOP_BANNER\&context=at_home_menu}{Explore:
  Reporters' Google Docs}
\end{itemize}

Advertisement

\protect\hyperlink{after-top}{Continue reading the main story}

Supported by

\protect\hyperlink{after-sponsor}{Continue reading the main story}

\hypertarget{how-to-stay-safer-on-mass-transit}{%
\section{How to Stay Safer on Mass
Transit}\label{how-to-stay-safer-on-mass-transit}}

Timing your trip strategically, using contactless payments and not
eating onboard are some of the things to keep in mind.

\includegraphics{https://static01.nyt.com/images/2020/07/31/multimedia/31ah-transportation1/31ah-transportation1-articleLarge.jpg?quality=75\&auto=webp\&disable=upscale}

By \href{https://www.nytimes.com/by/katherine-cusumano}{Katherine
Cusumano}

\begin{itemize}
\item
  Published Aug. 1, 2020Updated Aug. 3, 2020
\item
  \begin{itemize}
  \item
  \item
  \item
  \item
  \item
  \end{itemize}
\end{itemize}

In cities across the country, ridership on public transportation has
dropped precipitously as people have stayed home to help prevent the
spread of the coronavirus. But for some, continuing to take mass transit
was never optional. Many essential workers who cannot work remotely or
don't drive have continued to ride buses, trains and ferries; they are
disproportionately people of color and the earners of lower incomes.

``The pandemic itself has changed the profile of who's using the
services and what they're using them for,'' said Brian Taylor, a
professor of urban planning and public policy at the University of
California, Los Angeles. ``It is mostly riders without other options who
are coming back to public transit so far'' --- that is, if they ever
stopped riding. (The school's Institute of Transportation Studies, which
Dr. Taylor directs, is studying the effects of the pandemic on
transportation, including on public transit ridership, operations and
finance.)

In some areas, ridership is now rebounding as businesses and workplaces
reopen: Last week in New York, subway ridership
\href{https://new.mta.info/coronavirus/ridership}{was down by} 70 to 80
percent --- but that's compared with as much as 93 percent in April. And
thanks to the Metropolitan Transportation Authority's new cleaning
protocols (and suspended service between 1 a.m. and 5 a.m. for
sanitizing), the trains sparkle.

``At the beginning, they were thought of as sort of virus trains,'' said
Sarah M. Kaufman, the associate director of the Rudin Center for
Transportation at New York University. That has been largely disproved;
in Paris and Tokyo, for example, the cities' crowded trains have not
been linked to outbreak clusters. (Transit workers, though, have
suffered a steep toll: In New York, 131 M.T.A. workers
\href{https://www.nytimes.com/interactive/2020/07/26/nyregion/nyc-covid-19-mta-transit-workers.html}{have}
died and more than 4,000 have tested positive for the virus. Some
employees
\href{https://www.nytimes.com/2020/04/08/nyregion/coronavirus-nyc-mta-subway.html}{have
cited} a lack of widespread mask-wearing and social distancing early in
the pandemic.)

So as traffic picks up again, on the streets and underground, what are
the best strategies to stay safe while commuting and making essential
trips? Here, a few experts weigh in.

\hypertarget{choose-your-method-wisely}{%
\subsubsection{Choose your method
wisely.}\label{choose-your-method-wisely}}

If you plan to go somewhere, evaluate which means of transportation
poses the least risk to yourself and others. ``The more that you can be
in open air and the farther you can be from other people and the less
likely that other people will be without a mask is the safest way to
go,'' said Robyn Gershon, a professor of epidemiology at New York
University focused on occupational and environmental safety. Dr. Gershon
and a team of scientists are working with TWU Local 100, a transit union
in New York that represents roughly 46,000 bus and subway workers, to
study the impact of the outbreak on its members.

Take into account how long you'll be waiting for your chosen vessel to
arrive, she explained, and whether the terminal or station is inside or
outside. You could get to the ferry dock early, for instance, to ensure
that you get a seat on the upper deck in the open air; even inside,
there's probably ample air circulation and space to
\href{https://twitter.com/vinbarone/status/1270470612071440386}{spread
out}. If you're riding the bus, try to sit near a window, and keep it
open. Don't do this on the subway, though: New York's underground
tunnels are ``full of steel dust and asbestos,'' Ms. Kaufman explained.
Choose the escalator or stairs over the elevator if you can.

Or travel by bicycle. The use of bike-share programs in New York and
Chicago has
\href{https://www.nytimes.com/2020/03/14/nyregion/coronavirus-nyc-bike-commute.html}{ballooned};
by June, Citi Bike had nearly
\href{https://d21xlh2maitm24.cloudfront.net/nyc/June-2020-Citi-Bike-Monthly-Report.pdf?mtime=20200722104600}{180,000
active users} --- across Manhattan, Brooklyn, Queens, the Bronx and
Jersey City --- and bikes became
\href{https://www.nytimes.com/2020/05/18/nyregion/bike-shortage-coronavirus.html}{difficult
to buy} throughout the country. ``My bike has been getting a lot more
miles than it ever has before,'' said Dr. Mirna Mohanraj, a pulmonary
and critical care specialist at Mount Sinai Morningside, who has been
riding all over Manhattan, including some morning trips to Central Park,
and into the Bronx and Brooklyn.

Most important, ``if anyone has any symptoms or thinks they're sick,
they should not take public transportation,'' said Dr. Georges Benjamin,
the executive director of the American Public Health Association.
Instead, they should stay home and get in touch with their doctor.

\includegraphics{https://static01.nyt.com/images/2020/07/31/multimedia/31ah-transportation2/31ah-transportation2-articleLarge.jpg?quality=75\&auto=webp\&disable=upscale}

\hypertarget{pack-well-but-dont-overdo-it}{%
\subsubsection{Pack well, but don't overdo
it.}\label{pack-well-but-dont-overdo-it}}

Don't leave home without a bottle of hand sanitizer that's at least 60
percent alcohol and disinfectant wipes to clean your phone, which is a
germ magnet. And anytime you're in close quarters with other people,
wear your mask, which ``protects you from them and them from you,'' Dr.
Benjamin said.

Some transportation agencies have made this easier by installing
sanitizer dispensers and offering masks. In San Francisco, Bay Area
Rapid Transit agents have distributed masks to riders
\href{https://twitter.com/SFBART/status/1280632009560174592}{at stations
across the city}, and in Portland, Ore., mask dispensers have been added
to TriMet buses and trains. The M.T.A. recently formed a volunteer
``mask force'' --- clad in unmissable yellow shirts --- who roam the
subways and buses handing out free masks.

Don't bring more than necessary: More than ever, Dr. Gershon said, you
don't want to leave your bag sitting on the floor, saddling you with yet
another thing you should disinfect.

\hypertarget{be-strategic-about-your-timing}{%
\subsubsection{Be strategic about your
timing.}\label{be-strategic-about-your-timing}}

In Boston, the Massachusetts Bay Transportation Authority has
\href{https://www.mbta.com/projects/crowding-information-riders}{introduced}
a real-time congestion tracker for more than 30 bus lines, with a simple
taxonomy (``not crowded,'' ``some crowding,'' ``crowded''), in an effort
to help riders make informed decisions about their travel times. The
M.T.A. is putting in effect a similar program:
\href{http://www.mta.info/press-release/nyc-transit/mta-announces-new-real-time-bus-ridership-tracker-web-and-app-0}{Onbuses},
sensors count the number of passengers, which is then communicated to
potential riders through the agency's app.

Regardless of whether your local transit network makes such data
available, you can attempt to avoid typically crowded times. Find out if
your employer will allow for more flexible hours so you can circumvent,
and not contribute to, the rush-hour crush. (During the 1918 flu
pandemic, the health commissioner
\href{https://www.reuters.com/article/us-health-coronavirus-usa-subway/as-in-1918-new-york-may-use-staggered-work-hours-to-keep-subway-safe-idUSKBN22W1D2}{directed}
New York businesses to stagger commutes by just 15 minutes to reduce
crowds on transit and at office buildings.)

Continue to work remotely if you can to reduce crowding for essential
workers and others who are obligated to commute. And if you're planning
to take public transit to run errands or socialize, or for any other
nonwork-related purpose, travel during off-peak hours.

Image

Credit...Mark Wickens for The New York Times

\hypertarget{avoid-touching-communal-surfaces}{%
\subsubsection{Avoid touching communal
surfaces.}\label{avoid-touching-communal-surfaces}}

Keep your hands off the subway poles and rails to the ferry deck or onto
the bus. Don't touch the turnstile as you move through it; stay away
from touch screens, keypads and elevator buttons. Make contactless
payments if they're offered, and skip the paper tickets.

Though surface contamination is
\href{https://www.nytimes.com/2020/05/22/health/cdc-coronavirus-touching-surfaces.html}{not}
the main way people contract Covid-19, Dr. Benjamin nevertheless
recommended washing your hands before departing on your journey and
again upon reaching your destination, in addition to sanitizing
frequently throughout. Think, too, about skipping the gloves, which can
pick up germs on one surface and spread them to another.

\hypertarget{follow-the-directions}{%
\subsubsection{Follow the directions.}\label{follow-the-directions}}

Take note of the decals on the floor and signs you may see shepherding
you through the station, an effort by some operators --- like the
Chicago Transit Authority --- to reduce the number of people crossing
paths and decrease crowding. Riders should be ``spreading ourselves out
still so we're not packed in like sardines,'' as Dr. Gershon put it,
including spacing out along the subway platform.

If you're driving onto the ferry, the Washington State Department of
Transportation, which manages the largest ferry system in the United
States, recommends remaining in your car for the duration of your trip.
And if you're boarding the bus, enter from the rear, to avoid shedding
respiratory droplets on the driver and other passengers. Many bus
systems, including those in Philadelphia and Minneapolis, have been
encouraging passengers to enter from the back for that precise reason.

\hypertarget{dont-eat-and-keep-the-volume-down}{%
\subsubsection{Don't eat, and keep the volume
down.}\label{dont-eat-and-keep-the-volume-down}}

It might already be an unspoken norm on public transit, but it's a good
public health practice, too: Don't eat onboard, as eating can carry
particles from a surface to your face. ``Once you've gotten into a
public setting, no matter how well sanitized you are, we're touching
things,'' Dr. Mohanraj said. ``You're risking putting whatever's on your
hand in contact with your mouth, your nose, your eyes.'' Besides, you'd
have to take off your mask.

Avoid extensive conversations, too; talking, and singing, sprays
aerosolized droplets that can carry virus particles.
\href{https://www.sciencemag.org/news/2020/05/japan-ends-its-covid-19-state-emergency}{In
Tokyo}, many riders were already accustomed to wearing masks and rarely
talking; these practices became universal with the pandemic's onset.

And most of all, respect your transit operator. Bus drivers, for
example, have been verbally and physically assaulted for enforcing mask
rules. ``It's just unconscionable,'' Dr. Gershon said. ``Nobody should
have to be afraid to go to work.''

Advertisement

\protect\hyperlink{after-bottom}{Continue reading the main story}

\hypertarget{site-index}{%
\subsection{Site Index}\label{site-index}}

\hypertarget{site-information-navigation}{%
\subsection{Site Information
Navigation}\label{site-information-navigation}}

\begin{itemize}
\tightlist
\item
  \href{https://help.nytimes.com/hc/en-us/articles/115014792127-Copyright-notice}{©~2020~The
  New York Times Company}
\end{itemize}

\begin{itemize}
\tightlist
\item
  \href{https://www.nytco.com/}{NYTCo}
\item
  \href{https://help.nytimes.com/hc/en-us/articles/115015385887-Contact-Us}{Contact
  Us}
\item
  \href{https://www.nytco.com/careers/}{Work with us}
\item
  \href{https://nytmediakit.com/}{Advertise}
\item
  \href{http://www.tbrandstudio.com/}{T Brand Studio}
\item
  \href{https://www.nytimes.com/privacy/cookie-policy\#how-do-i-manage-trackers}{Your
  Ad Choices}
\item
  \href{https://www.nytimes.com/privacy}{Privacy}
\item
  \href{https://help.nytimes.com/hc/en-us/articles/115014893428-Terms-of-service}{Terms
  of Service}
\item
  \href{https://help.nytimes.com/hc/en-us/articles/115014893968-Terms-of-sale}{Terms
  of Sale}
\item
  \href{https://spiderbites.nytimes.com}{Site Map}
\item
  \href{https://help.nytimes.com/hc/en-us}{Help}
\item
  \href{https://www.nytimes.com/subscription?campaignId=37WXW}{Subscriptions}
\end{itemize}
