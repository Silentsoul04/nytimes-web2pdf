Sections

SEARCH

\protect\hyperlink{site-content}{Skip to
content}\protect\hyperlink{site-index}{Skip to site index}

\href{https://www.nytimes.com/section/books/review}{Book Review}

\href{https://myaccount.nytimes.com/auth/login?response_type=cookie\&client_id=vi}{}

\href{https://www.nytimes.com/section/todayspaper}{Today's Paper}

\href{/section/books/review}{Book Review}\textbar{}For Two Teenagers on
Election Day, the Political Gets Personal

\url{https://nyti.ms/3hYeUTT}

\begin{itemize}
\item
\item
\item
\item
\item
\end{itemize}

Advertisement

\protect\hyperlink{after-top}{Continue reading the main story}

Supported by

\protect\hyperlink{after-sponsor}{Continue reading the main story}

\href{/column/childrens-books}{Children's Books}

\hypertarget{for-two-teenagers-on-election-day-the-political-gets-personal}{%
\section{For Two Teenagers on Election Day, the Political Gets
Personal}\label{for-two-teenagers-on-election-day-the-political-gets-personal}}

\includegraphics{https://static01.nyt.com/images/2020/08/09/books/review/09-BKS-YOON-KIDS/09-BKS_YOON_KIDS-articleLarge.jpg?quality=75\&auto=webp\&disable=upscale}

Buy Book ▾

\begin{itemize}
\tightlist
\item
  \href{https://www.amazon.com/gp/search?index=books\&tag=NYTBSREV-20\&field-keywords=The+Voting+Booth+Brandy+Colbert}{Amazon}
\item
  \href{https://du-gae-books-dot-nyt-du-prd.appspot.com/buy?title=The+Voting+Booth\&author=Brandy+Colbert}{Apple
  Books}
\item
  \href{https://www.anrdoezrs.net/click-7990613-11819508?url=https\%3A\%2F\%2Fwww.barnesandnoble.com\%2Fw\%2F\%3Fean\%3D9781368053297}{Barnes
  and Noble}
\item
  \href{https://www.anrdoezrs.net/click-7990613-35140?url=https\%3A\%2F\%2Fwww.booksamillion.com\%2Fp\%2FThe\%2BVoting\%2BBooth\%2FBrandy\%2BColbert\%2F9781368053297}{Books-A-Million}
\item
  \href{https://bookshop.org/a/3546/9781368053297}{Bookshop}
\item
  \href{https://www.indiebound.org/book/9781368053297?aff=NYT}{Indiebound}
\end{itemize}

When you purchase an independently reviewed book through our site, we
earn an affiliate commission.

By Nicola Yoon

\begin{itemize}
\item
  Aug. 1, 2020
\item
  \begin{itemize}
  \item
  \item
  \item
  \item
  \item
  \end{itemize}
\end{itemize}

I was an adult the first time I volunteered for a presidential campaign.
It was in 2008 for Barack Obama. As naïve as it sounds, I felt I was
helping America begin to fulfill its promise: to be a place where ``all
men are created equal.'' Fast-forward to today, when America has never
seemed farther away from keeping that promise. In a world with so much
injustice, how do we maintain hope and a willingness to fight, without
burning out, or simply giving in to the easier option of looking the
other way?

Marva Sheridan, the heroine of Brandy Colbert's wonderful new novel,
\textbf{THE VOTING BOOTH (Hyperion, 304 pp., \$18.99; ages 12 and up),}
would tell you that some people don't have the luxury of turning away.
Some fights are existential; to look the other way is to invite ruin or
worse.

In many ways, this novel is perfect for the times we're in. How better
to get young people involved in the voting process than with a book set
on Election Day featuring two incredibly charming main characters
exercising their civic duties while simultaneously falling in love?

The novel is told from alternating perspectives and takes place in a
single day. It's Nov. 3. Marva, who is Black, is thrilled to be voting
in her first election. She's spent the prior months canvassing,
text-banking and registering voters.

To say she's civic-minded would be an understatement. Marva has been
interested in politics from age 7, when she informed her second-grade
teacher that she wanted to become either secretary of state, an
environmental attorney or a Supreme Court justice.

Duke Crenshaw, the mixed-race son of a Black father and a white mother,
is not exactly Marva's opposite, but he's close. He does have a sense of
civic duty and intends to do his part by voting. But voting is
\emph{all} he intends. His main concern for the day is passing his
calculus test and drumming with his band --- hilariously named Drugstore
Sorrow --- in their first paying gig.

The pair meet just after Marva casts her vote. She hears Duke being told
he isn't on the list. Marva assumes this is a case of voter suppression,
but Duke realizes it may be that he's registered under his dad's address
in a different district. Marva makes it her responsibility to help Duke
find a way to vote before the polls close.

Mission set, the novel begins in earnest. Marva and Duke spend the day
getting to know each other and --- in some ways --- trying to convince
each other of their worldviews. Marva wants Duke to understand her
passionate activism. Duke wants Marva to understand that it's sometimes
OK to take a break from saving the world. As the day unfolds, the bond
between the two deepens. Duke helps Marva contend with her white
boyfriend's decision not to vote. Marva supports Duke as he confronts
his parents' overprotectiveness in the wake of his much more politically
active older brother's death. Both characters are smart and highly
opinionated, making for plenty of zippy and infectious dialogue.

In less skilled hands, this premise could easily have become didactic.
Fortunately, Colbert is deft at making the political feel truly
personal.

There's a pivotal moment in the book, after a traffic-stop encounter
with a Latina cop (Duke is in the passenger seat while Marva runs a
yellow light just as it's turning red), when Duke wonders how many times
he'll ``be so lucky,'' meaning how many times he'll escape such an
encounter without being killed. Marva tells him it's issues like these
--- systemic racial injustices against Black people --- that make her
care so much about the civic process. She talks of her heroes Bayard
Rustin, Diane Nash, Stokely Carmichael and Coretta Scott King, who all
risked their lives fighting against racial inequality. She, too, wants
to make a difference.

The book truly shines in moments like these. It makes us root for Marva
and all those among us who battle to make the world a better place. The
arc of the moral universe may bend toward justice, but someone needs to
apply the pressure.

Advertisement

\protect\hyperlink{after-bottom}{Continue reading the main story}

\hypertarget{site-index}{%
\subsection{Site Index}\label{site-index}}

\hypertarget{site-information-navigation}{%
\subsection{Site Information
Navigation}\label{site-information-navigation}}

\begin{itemize}
\tightlist
\item
  \href{https://help.nytimes.com/hc/en-us/articles/115014792127-Copyright-notice}{©~2020~The
  New York Times Company}
\end{itemize}

\begin{itemize}
\tightlist
\item
  \href{https://www.nytco.com/}{NYTCo}
\item
  \href{https://help.nytimes.com/hc/en-us/articles/115015385887-Contact-Us}{Contact
  Us}
\item
  \href{https://www.nytco.com/careers/}{Work with us}
\item
  \href{https://nytmediakit.com/}{Advertise}
\item
  \href{http://www.tbrandstudio.com/}{T Brand Studio}
\item
  \href{https://www.nytimes.com/privacy/cookie-policy\#how-do-i-manage-trackers}{Your
  Ad Choices}
\item
  \href{https://www.nytimes.com/privacy}{Privacy}
\item
  \href{https://help.nytimes.com/hc/en-us/articles/115014893428-Terms-of-service}{Terms
  of Service}
\item
  \href{https://help.nytimes.com/hc/en-us/articles/115014893968-Terms-of-sale}{Terms
  of Sale}
\item
  \href{https://spiderbites.nytimes.com}{Site Map}
\item
  \href{https://help.nytimes.com/hc/en-us}{Help}
\item
  \href{https://www.nytimes.com/subscription?campaignId=37WXW}{Subscriptions}
\end{itemize}
