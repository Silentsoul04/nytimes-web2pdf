Sections

SEARCH

\protect\hyperlink{site-content}{Skip to
content}\protect\hyperlink{site-index}{Skip to site index}

\href{https://www.nytimes.com/section/business/media}{Media}

\href{https://myaccount.nytimes.com/auth/login?response_type=cookie\&client_id=vi}{}

\href{https://www.nytimes.com/section/todayspaper}{Today's Paper}

\href{/section/business/media}{Media}\textbar{}More Than 1,000 Companies
Boycotted Facebook. Did It Work?

\url{https://nyti.ms/2D9Jyeb}

\begin{itemize}
\item
\item
\item
\item
\item
\end{itemize}

Advertisement

\protect\hyperlink{after-top}{Continue reading the main story}

Supported by

\protect\hyperlink{after-sponsor}{Continue reading the main story}

\hypertarget{more-than-1000-companies-boycotted-facebook-did-it-work}{%
\section{More Than 1,000 Companies Boycotted Facebook. Did It
Work?}\label{more-than-1000-companies-boycotted-facebook-did-it-work}}

Major advertisers on Facebook reduced their spending by millions of
dollars in July, but not enough to significantly damage the platform's
revenue.

\hypertarget{estimated-spending-of-facebooks-top-100-advertisers}{%
\subsection{Estimated spending of Facebook's top 100
advertisers}\label{estimated-spending-of-facebooks-top-100-advertisers}}

July Boycotters

Reduced Spenders

Other Top Advertisers

\$15

million

10

5

0

JAN

FEB

MAR

APR

MAY

JUNE

JULY

Other Top

Advertisers

July Boycotters

Reduced Spenders

\$10

million

5

0

MAR

MAY

JULY

JAN

Note: ``Reduced spenders'' are companies that did not officially
announce boycotts, but decreased their spending in July by at least 90
percent compared to June.

Source: Pathmatics

By Eleanor Lutz

By \href{https://www.nytimes.com/by/tiffany-hsu}{Tiffany Hsu} and
Eleanor Lutz

\begin{itemize}
\item
  Aug. 1, 2020
\item
  \begin{itemize}
  \item
  \item
  \item
  \item
  \item
  \end{itemize}
\end{itemize}

The advertiser boycott of Facebook took a toll on the social media
giant, but it may have caused more damage to the company's reputation
than to its bottom line.

The boycott, called \#StopHateForProfit by the civil rights groups that
organized it, urged companies to stop paying for ads on Facebook in July
to protest the platform's handling of hate speech and misinformation.
More than
\href{https://www.nytimes.com/2020/06/26/business/media/Facebook-advertising-boycott.html}{1,000
advertisers publicly joined}, out of a total pool of more than 9
million, while others quietly scaled back their spending.

The 100 advertisers that spent the most on Facebook in the first half of
the year spent \$221.4 million from July 1 through July 29, 12 percent
less than the \$251.4 million spent by the top 100 advertisers a year
earlier, according to estimates from the advertising analytics platform
\href{https://www.pathmatics.com/product/methodology}{Pathmatics}. Of
those 100, nine companies formally announced a pullback in paid
advertising, cutting their spending to \$507,500 from \$26.2 million.

Many of the companies that stayed away from Facebook said they planned
to return, and many are mom-and-pop enterprises and individuals that
depend on the platform for promotion. Mark Zuckerberg, Facebook's chief
executive, has emphasized the importance of small business, saying
\href{https://www.nytimes.com/live/2020/07/30/business/stock-market-today-coronavirus\#facebook-nearly-doubles-its-profit-but-warns-of-fallout-from-ad-boycottshttps://www.nytimes.com/2020/07/30/technology/tech-company-earnings-amazon-apple-facebook-google.html}{during
an earnings call} on Thursday that ``some seem to wrongly assume that
our business is dependent on a few large advertisers.''

Facebook said that the top 100 spenders contributed 16 percent of its
\$18.7 billion in revenue in the second quarter, which ended on June 30.
During the first three weeks of July, Facebook said, overall ad revenue
grew 10 percent over last year, a rate the company expects to continue
for the full quarter.

The boycott complicated planning for advertisers. The Kansas-based
digital agency DEG had ``a whirlwind of a month'' as its small to
midsize clients grappled with whether they could reach enough customers
without Facebook, said Quinn Sheek, its director of media and search.
Facebook and its subsidiary Instagram make up more than a third of
digital spending for DEG clients.

Of the 60 percent of DEG clients that joined the July boycott, four out
of five are planning to return to Facebook in August, with many having
``decided it's too much for them during a difficult economic time to
remain off,'' Ms. Sheek said. Still, the boycott helped amplify
discussion of toxic content on Facebook. The issue was raised in
\href{https://www.nytimes.com/2020/07/29/technology/big-tech-hearing-apple-amazon-facebook-google.html}{a
congressional hearing} this past week and in
\href{https://www.nytimes.com/2020/06/23/business/media/facebook-ad-boycott.html}{repeated
meetings} between ad industry representatives and Facebook leaders. In
the face of the pressure, Facebook released the results of
\href{https://www.nytimes.com/2020/07/08/technology/facebook-civil-rights-audit.html}{a
civil rights audit} last month and agreed to hire
\href{https://www.nytimes.com/2020/07/07/technology/facebook-ad-boycott-civil-rights.html}{a
civil rights executive}.

``What could really hurt Facebook is the long-term effect of its
perceived reputation and the association with being viewed as a
publisher of `hate speech' and other inappropriate content,'' Stephen
Hahn-Griffiths, the executive vice president of the public opinion
analysis company RepTrak, wrote
\href{https://www.reptrak.com/blog/what-companies-can-learn-from-facebooks-latest-reputation-challenge/}{in
a post last month}.

In addition to the prevalence of hate speech on the platform, its
critics have also focused on the company's treatment of
\href{https://www.nytimes.com/2018/12/18/technology/facebook-privacy.html}{user
privacy} and foreign
\href{https://www.nytimes.com/2017/11/01/us/politics/russia-2016-election-facebook.html}{election
interference}.

``You could argue that Facebook has a bloodied nose and two reputational
black eyes,'' Mr. Hahn-Griffiths wrote.

Sheryl Sandberg, Facebook's chief operating officer, said during the
company's earnings call that, like the boycott's organizers, ``we don't
want hate on our platforms, and we stand firmly against it.''

The ad industry was
\href{https://www.nytimes.com/2020/07/28/business/media/coronavirus-pandemic-advertising-industry.html}{already
in upheaval} when the boycott began, as businesses closed, layoffs swept
through the economy and homebound consumers slowed their shopping.
Before they reduced spending on Facebook in July, advertisers like
Microsoft, Starbucks, Unilever and Target took a temporary break from
the platform in June, as many companies were reacting to
pandemic-related marketing budget cuts and widespread protests over
racism and police brutality. Disney's spending on Facebook has mostly
trended downward since late March, according to Pathmatics.

Last month, large advertisers like Procter \& Gamble, Samsung, Walmart
and Geico sharply curtailed paid advertising on Facebook without joining
the official boycott, according to Pathmatics. Others, like Hershey and
Hulu, beefed up their spending on alternate platforms like Twitter and
YouTube.

\hypertarget{companies-that-did-not-announce-facebook-boycotts}{%
\subsection{Companies that did not announce Facebook
boycotts}\label{companies-that-did-not-announce-facebook-boycotts}}

Procter \& Gamble

Samsung

\$1,500,000

\$470,000

MAY

JUNE

JULY

MAY

JUNE

JULY

Geico

Walmart

\$220,000

\$190,000

MAY

JUNE

JULY

MAY

JUNE

JULY

Samsung

Procter \& Gamble

Walmart

Geico

\$470,000

\$1,500,000

\$220,000

\$190,000

JULY

JUNE

JUNE

JUNE

JUNE

JULY

MAY

MAY

JULY

MAY

JULY

MAY

Note: Spending amounts are estimates. Not all organizations that reduced
spending are shown.

Source: Pathmatics

By Eleanor Lutz

Companies like Beam Suntory and
\href{https://www.coca-colacompany.com/media-center/updated-statement-on-social-media-platform-pause}{Coca-Cola}
have vowed to continue pressuring Facebook, especially as the
presidential race heats up. On Thursday, the ice cream company Ben \&
Jerry's said it planned to keep withholding spending on product
promotions through the end of the year ``to send a message.''

The advertiser boycott ``was a warning shot, an opening salvo,'' said
Jonathan Greenblatt, the chief executive of the civil rights group the
Anti-Defamation League, which helped set up the ad boycott. Organizers
and other groups now plan to expand the boycott into Europe, to
\href{https://actions.sumofus.org/a/facebook-adblock-landing}{include
Facebook users}, and to address other concerns, like the presence of
\href{https://www.keepkidssafeonline.org/stop-hate-for-profit-facebook-pr}{child
sexual abuse} on the platform.

Half of the companies that work with the agency Allen \& Gerritsen in
Boston and Philadelphia participated in the boycott, said Derek Welch,
its vice president of media. Many felt it was important to ``do
something that is meaningful and tangible in a sea of brands putting out
very well-meaning statements,'' he said.

Mr. Welch said the agency's clients typically spend \$150,000 to
\$200,000 a month total on Facebook. Several plan to continue
boycotting.

``The big companies that have signed on have been great for visibility
and getting the word out,'' he said. ``But this is really all about
these small businesses in aggregate who are spending \$30,000 here or
\$50,000 there, whose decisions wouldn't normally make too much of a
difference.''

Advertisement

\protect\hyperlink{after-bottom}{Continue reading the main story}

\hypertarget{site-index}{%
\subsection{Site Index}\label{site-index}}

\hypertarget{site-information-navigation}{%
\subsection{Site Information
Navigation}\label{site-information-navigation}}

\begin{itemize}
\tightlist
\item
  \href{https://help.nytimes.com/hc/en-us/articles/115014792127-Copyright-notice}{©~2020~The
  New York Times Company}
\end{itemize}

\begin{itemize}
\tightlist
\item
  \href{https://www.nytco.com/}{NYTCo}
\item
  \href{https://help.nytimes.com/hc/en-us/articles/115015385887-Contact-Us}{Contact
  Us}
\item
  \href{https://www.nytco.com/careers/}{Work with us}
\item
  \href{https://nytmediakit.com/}{Advertise}
\item
  \href{http://www.tbrandstudio.com/}{T Brand Studio}
\item
  \href{https://www.nytimes.com/privacy/cookie-policy\#how-do-i-manage-trackers}{Your
  Ad Choices}
\item
  \href{https://www.nytimes.com/privacy}{Privacy}
\item
  \href{https://help.nytimes.com/hc/en-us/articles/115014893428-Terms-of-service}{Terms
  of Service}
\item
  \href{https://help.nytimes.com/hc/en-us/articles/115014893968-Terms-of-sale}{Terms
  of Sale}
\item
  \href{https://spiderbites.nytimes.com}{Site Map}
\item
  \href{https://help.nytimes.com/hc/en-us}{Help}
\item
  \href{https://www.nytimes.com/subscription?campaignId=37WXW}{Subscriptions}
\end{itemize}
