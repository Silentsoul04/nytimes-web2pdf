Sections

SEARCH

\protect\hyperlink{site-content}{Skip to
content}\protect\hyperlink{site-index}{Skip to site index}

\href{https://myaccount.nytimes.com/auth/login?response_type=cookie\&client_id=vi}{}

\href{https://www.nytimes.com/section/todayspaper}{Today's Paper}

\href{/section/opinion}{Opinion}\textbar{}Protecting the Rights of Those
With Disabilities

\url{https://nyti.ms/3164niJ}

\begin{itemize}
\item
\item
\item
\item
\item
\end{itemize}

Advertisement

\protect\hyperlink{after-top}{Continue reading the main story}

\href{/section/opinion}{Opinion}

Supported by

\protect\hyperlink{after-sponsor}{Continue reading the main story}

letters

\hypertarget{protecting-the-rights-of-those-with-disabilities}{%
\section{Protecting the Rights of Those With
Disabilities}\label{protecting-the-rights-of-those-with-disabilities}}

Readers discuss a series of articles about the impact of the Americans
With Disabilities Act, passed 30 years ago.

Aug. 1, 2020

\begin{itemize}
\item
\item
\item
\item
\item
\end{itemize}

\includegraphics{https://static01.nyt.com/images/2020/07/12/opinion/sunday/11disability/11disability-articleLarge.jpg?quality=75\&auto=webp\&disable=upscale}

\textbf{To the Editor:}

Re
``\href{https://www.nytimes.com/interactive/2020/us/disability-ADA-30-anniversary.html}{The
A.D.A. at 30: Beyond the Law's Promise}'' (special section, July 26):

When President George H.W. Bush signed the Americans With Disabilities
Act 30 years ago, it recognized the needs of millions of people of all
ages who had been overlooked in previous legislation. As one of them
myself, I found it gratifying to read your thoughtfully selected
articles commemorating that auspicious anniversary.

Unlike race, which cannot be altered, a temporary or permanent
disability, whether from injury, illness or aging, can be acquired by
anyone. Thus, we all have a stake in protecting the civil rights of
people with disabilities.

The remarkable advances in critical medical care and the aging trends in
society suggest that the ranks of this group will continue to increase
in the future. And it should not be forgotten that, in the words of
\href{https://www.nytimes.com/1989/11/05/obituaries/howard-rusk-88-dies-medical-pioneer.html}{Dr.
Howard Rusk}, though people may function within the limits of their
disability, they can still function to the hilt of their ability.

Stanley F. Wainapel\\
Bronx\\
\emph{The writer, who is blind, is clinical director of rehabilitation
medicine at Montefiore Medical Center.}

\textbf{To the Editor:}

I enjoyed reading your series examining the Americans With Disabilities
Act on its 30th anniversary, but I was disappointed to see no mention of
the person responsible for this historic piece of legislation, former
Senator Tom Harkin of Iowa. Put simply, the A.D.A. would likely not be
law today if it weren't for Senator Harkin. I know, because I was there.

While it's remembered as a bipartisan triumph, the A.D.A. was met with
tremendous pushback from the outset. Senator Harkin, whose brother was
deaf, authored what became the final bill and was its chief sponsor in
the Senate. The A.D.A. was personal for him, and the empathy and love
that drove Senator Harkin's tireless advocacy for the bill were
instrumental in securing key alliances across the aisle --- including
the Republican president, George H.W. Bush. Upon the passage of this
landmark legislation, Senator Harkin delivered his speech on the Senate
floor in sign language,
\href{https://disabilityvisibilityproject.com/2014/07/26/disability-history-senator-harkin-delivers-floor-speech-in-american-sign-language-upon-passage-of-the-ada-71390/}{a
first in the history of the Senate.}

Senator Harkin's masterful navigation of the A.D.A. was an example of
Washington at its best --- both parties coming together to make our
country a fairer, more just place. No examination of the A.D.A. is
complete without mentioning this man, whose leadership, vision and hard
work 30 years ago resulted in the passage of one of the most
consequential pieces of legislation in our nation's history.

Harry Reid\\
Las Vegas\\
\emph{The writer is a former Democratic senator from Nevada who served
as the Senate majority leader from 2007 to 2015.}

\textbf{To the Editor:}

Joseph Shapiro's excellent article
``\href{https://www.nytimes.com/2020/07/17/style/americans-with-disabilities-act.html}{Generational
Expectations}'' suggests that the generation of people with disabilities
that has come of age since the passage of the Americans With
Disabilities Act can be characterized by an identity of disability
pride. My colleague Alex Heckert and I have conducted research on this
population, as well as those of its older members. Our most striking
finding was the \emph{diversity} of disability identities.

Yes, pride was more common among our younger respondents, but so was
``typicality,'' an identity of ``fitting in'' and not participating in
disability rights activism. In general, disability pride is associated
with activism and is more common among those with lifelong disabilities
than among those who acquire their disabilities later in life.

Although the A.D.A. has increased accessibility, stigma has not
disappeared, and those who have been exposed to negative societal
attitudes often continue to have those attitudes after they become
disabled. Thus, while acknowledging the progress our society has made,
we should keep in mind the complexity of disability identity and
recognize that the need for attitudinal change must continue.

Rosalyn Benjamin Darling\\
Pittsboro, N.C.\\
\emph{The writer is professor emerita of sociology at Indiana University
of Pennsylvania.}

\textbf{To the Editor:}

Re
``\href{https://www.nytimes.com/2020/07/20/arts/disabilities-architecture-design.html}{Building
Accessibility,}'' by Michael Kimmelman:

I appreciate Mr. Kimmelman's belated recognition that when architecture
fails to consider the needs of people with disabilities, such projects
(like
\href{https://gothamist.com/news/new-41-million-hunters-point-library-has-one-major-flaw}{Hunters
Point Library}) are neither impressive nor --- often --- legal under the
Americans With Disabilities Act. I'd add that the A.D.A.'s impact is
largely due to the tireless work of those who enforce it through the
courts.

The A.D.A. has one primary enforcement mechanism: civil litigation.
There is no relevant regulatory board, and voluntary compliance is far
from universal. Although some litigants do abuse the system, the
discourse surrounding so-called drive-by lawsuits has had the
frustrating and often unfair result of giving A.D.A. litigation a bad
name. In fact, much of the progress made in disability rights would not
have happened without private litigants.

Private litigants have ensured that people with disabilities can use
\href{https://gothamist.com/news/nyc-agrees-to-make-all-sidewalk-curbs-accessible-to-the-disabled}{city
sidewalks},
\href{https://dralegal.org/case/eason-v-new-york-state-board-elections/}{vote}
\href{https://www.forbes.com/sites/peterslatin/2020/06/03/disabled-new-yorkers-can-vote--for-now/\#32c3ef4c5375}{privately}
and
\href{https://dralegal.org/press/landmark-decision-by-federal-appellate-court-vindicates-the-rights-of-voters-with-disabilities-in-new-york-city/}{independently},
safely \href{https://dralegal.org/case/metzler-v-kaiser/}{access}
\href{https://dralegal.org/case/sandra-lamb-v-nrad-medical-associates-et-al/}{health
care}
\href{https://dralegal.org/case/hinkle-et-al-v-kent-et-al/}{services},
\href{https://www.chicagotribune.com/la-me-ln-contra-costa-juvenile-education-20130808-story.html}{avoid
abusive solitary confinement in juvenile detention centers},
\href{https://www.nytimes.com/2013/11/08/nyregion/new-yorks-emergency-plans-violate-disabilities-act-judge-says.html}{evacuate
safely in an emergency},
\href{https://www.nytimes.com/2013/12/07/nyregion/wheelchair-settlement-poses-test-for-cab-industry.html}{hail
a cab}, access
\href{https://variety.com/2016/digital/news/netflix-audio-descriptions-blind-settlement-1201753569/}{audio
descriptions for streaming video services} and
\href{https://www.courthousenews.com/amc-movies-settles-class-action-blind/}{captioning
in movie theaters}, and so much more.

Thirty years later, this work is still far from over. The folly
\href{https://gothamist.com/news/lack-handicap-accessibility-flashy-new-hunters-point-library-sparks-lawsuit}{of
building a brand-new, \$41.5 million} inaccessible library underscores
why private litigants must continue bringing suits to enforce their
civil rights.

Andrea Kozak-Oxnard\\
New York\\
\emph{The writer is a staff attorney at Disability Rights Advocates.}

\textbf{To the Editor:}

I applaud The Times for devoting a special section to disabilities. But
it was disappointing not to see any coverage of the daily problems that
accompany hearing loss for the millions of people with the condition. I
would not claim that hearing loss has the same impact as some of the
disabilities discussed. Yet severe hearing loss can present many
challenges.

I lost my career as a performing musician. In my work as an educational
researcher at a university, it compromised my ability to function in
meetings. If I go to the doctor or to the hospital, I may miss important
communications, particularly when doctors and staff are masked. Hearing
loss complicates speaking on the phone, going to dinner in a restaurant,
using public transit and many other daily activities.

At the Hearing Loss Association of America, we celebrate the 30th
anniversary of the A.D.A. and the rights we have gained that have
improved our lives. But even those gains required a fight and would not
have been accomplished without the advocacy of our predecessor
organization, Self-Help for the Hard of Hearing, which led the struggle
to have hearing loss covered by the A.D.A.

Jon Taylor\\
New York\\
\emph{The writer is president of the Hearing Loss Association of
America, New York City chapter.}

\textbf{To the Editor:}

I very much appreciated Andrew Solomon's essay
``\href{https://www.nytimes.com/2020/07/10/style/invisible-disabilities.html}{Invisible
Disabilities}.'' I wholeheartedly agree that, at the 30th anniversary of
the A.D.A., we need to make the most vulnerable of us visible in order
to enforce the application of the A.D.A. to this population.

The A.D.A. itself, as Mr. Solomon describes, can too often be a ``blunt
tool,'' but for those with autism, learning differences, mental health
issues or neurological issues, there are often no instruments at all.

Because of the stigma commonly associated with invisible disabilities,
making those affected leery of disclosing them, it is even more
important that there be base-level support available. The history of the
A.D.A. has also taught us that the accommodations designed for one
community frequently support others. For example, ramps were created for
wheelchairs but also support those with strollers and walkers.

Similarly, as we develop guidelines for those with invisible
disabilities, we will notice benefits to a multitude of people who
think, feel and interact differently, either permanently or transiently.
Ultimately, taking into account every type of body and mind enhances the
future for all of us.

Wendy Ross\\
Philadelphia\\
\emph{The writer is the director of the Center for Autism and
Neurodiversity at Jefferson Health.}

Advertisement

\protect\hyperlink{after-bottom}{Continue reading the main story}

\hypertarget{site-index}{%
\subsection{Site Index}\label{site-index}}

\hypertarget{site-information-navigation}{%
\subsection{Site Information
Navigation}\label{site-information-navigation}}

\begin{itemize}
\tightlist
\item
  \href{https://help.nytimes.com/hc/en-us/articles/115014792127-Copyright-notice}{©~2020~The
  New York Times Company}
\end{itemize}

\begin{itemize}
\tightlist
\item
  \href{https://www.nytco.com/}{NYTCo}
\item
  \href{https://help.nytimes.com/hc/en-us/articles/115015385887-Contact-Us}{Contact
  Us}
\item
  \href{https://www.nytco.com/careers/}{Work with us}
\item
  \href{https://nytmediakit.com/}{Advertise}
\item
  \href{http://www.tbrandstudio.com/}{T Brand Studio}
\item
  \href{https://www.nytimes.com/privacy/cookie-policy\#how-do-i-manage-trackers}{Your
  Ad Choices}
\item
  \href{https://www.nytimes.com/privacy}{Privacy}
\item
  \href{https://help.nytimes.com/hc/en-us/articles/115014893428-Terms-of-service}{Terms
  of Service}
\item
  \href{https://help.nytimes.com/hc/en-us/articles/115014893968-Terms-of-sale}{Terms
  of Sale}
\item
  \href{https://spiderbites.nytimes.com}{Site Map}
\item
  \href{https://help.nytimes.com/hc/en-us}{Help}
\item
  \href{https://www.nytimes.com/subscription?campaignId=37WXW}{Subscriptions}
\end{itemize}
