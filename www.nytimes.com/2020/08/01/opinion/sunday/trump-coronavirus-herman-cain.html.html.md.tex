Sections

SEARCH

\protect\hyperlink{site-content}{Skip to
content}\protect\hyperlink{site-index}{Skip to site index}

\href{https://www.nytimes.com/section/opinion/sunday}{Sunday Review}

\href{https://myaccount.nytimes.com/auth/login?response_type=cookie\&client_id=vi}{}

\href{https://www.nytimes.com/section/todayspaper}{Today's Paper}

\href{/section/opinion/sunday}{Sunday Review}\textbar{}Double, Double,
Trump's Toil, Our Trouble

\href{https://nyti.ms/2PffLDt}{https://nyti.ms/2PffLDt}

\begin{itemize}
\item
\item
\item
\item
\item
\item
\end{itemize}

Advertisement

\protect\hyperlink{after-top}{Continue reading the main story}

\href{/section/opinion}{Opinion}

Supported by

\protect\hyperlink{after-sponsor}{Continue reading the main story}

\hypertarget{double-double-trumps-toil-our-trouble}{%
\section{Double, Double, Trump's Toil, Our
Trouble}\label{double-double-trumps-toil-our-trouble}}

Demon sperm meets alien D.N.A., as President Trump teeters.

\href{https://www.nytimes.com/by/maureen-dowd}{\includegraphics{https://static01.nyt.com/images/2018/04/02/opinion/maureen-dowd/maureen-dowd-thumbLarge.png}}

By \href{https://www.nytimes.com/by/maureen-dowd}{Maureen Dowd}

Opinion Columnist

\begin{itemize}
\item
  Aug. 1, 2020
\item
  \begin{itemize}
  \item
  \item
  \item
  \item
  \item
  \item
  \end{itemize}
\end{itemize}

\includegraphics{https://static01.nyt.com/images/2020/08/02/opinion/sunday/01Dowd/01Dowd-articleLarge.jpg?quality=75\&auto=webp\&disable=upscale}

WASHINGTON --- Macbeth has his doubts.

But his wife taunts him about his manliness until he bloodies his
country.

It's hard to believe, four centuries after Shakespeare, that the fear of
being unmanned is still so potent that it could wreck a country.

But it is. And it has.

Donald Trump's warped view of masculinity has warped this nation's
response to a deadly pandemic. And Trump doesn't even have a diabolical
Lady MacTrump whispering in his ear, goading him about being a man. He
goads himself, fueled by ghostly memories of his autocratic father.

As the Shakespeare scholar Stephen Greenblatt writes: ``The tyrant,
Macbeth and other plays suggest, is driven by a range of sexual
anxieties: a compulsive need to prove his manhood, dread of impotence, a
nagging apprehension that he will not be found sufficiently attractive
or powerful, a fear of failure. Hence the penchant for bullying, the
vicious misogyny, and the explosive violence. Hence, too, the
vulnerability to taunts. Especially those bearing a latent or explicit
sexual charge.''

Trump's fear of emasculation led to his de-mask-ulation. Instead of
cleaving to science and reason, he stuck with the old, corny Gordon
Gekko routine, putting concern for the stock market above all else.

Like Macbeth, the president made tragic errors of judgment and plunged
his country into a nightmare. Our trust in government is depleted, and
our relationships in the world are tattered. As Fintan O'Toole
\href{https://www.irishtimes.com/opinion/fintan-o-toole-donald-trump-has-destroyed-the-country-he-promised-to-make-great-again-1.4235928?mode=sample\&auth-failed=1\&pw-origin=https\%3A\%2F\%2Fwww.irishtimes.com\%2Fopinion\%2Ffintan-o-toole-donald-trump-has-destroyed-the-country-he-promised-to-make-great-again-1.4235928}{wrote}
in The Irish Times, the world has loved, hated and envied the United
States. But never before has it pitied us. Until now.

Trump has always said that the whole world is laughing at us because
it's taking advantage of us. That sound you're hearing is not laughter.

``He could be on his way to re-election now if he had done what many of
the governors did and followed science and public health advice and if
he had leveled with people about what the requirements were and why,''
says David Axelrod, the former Obama strategist. ``If he had done those
things, the country would have responded and been in a much better
place.

``But he didn't have the emotional capacity to do it. At a minimum, it's
Shakespearean. It's almost biblical.''

Even Trump's allies are baffled about why he can't fake a sense of
compassion and competency. He has made enough cheesy movie cameos ---
even one \href{https://www.youtube.com/watch?v=QVmAcULPMu4}{hawking}
cheese-stuffed Pizza Hut crust --- that he should know how to pretend to
be halfway human.

Now the president is threatening another crisis, tweeting that we might
have to delay the election because there could be mail-in voting fraud.

In his view, either he wins or the election is rigged. He's trying to
make mail-in ballots socially unacceptable the same way he made masks
socially unacceptable for the first five months of the plague.

The Washington Post
\href{https://www.washingtonpost.com/politics/postal-service-backlog-sparks-worries-that-ballot-delivery-could-be-delayed-in-november/2020/07/30/cb19f1f4-d1d0-11ea-8d32-1ebf4e9d8e0d_story.html}{reports}
that backlogs at the U.S. Postal Service are causing some employees
there to worry that the Trump lackey in charge, a top donor, is
intentionally gumming up the works just in time for the election. It is
astounding the corrupt lengths the administration seems willing to go to
--- destroying the Postal Service to win the election. Ben Franklin
would be incensed.

As Axelrod notes, ``Whatever happened back in the Bush v. Gore recount
days will seem like the Garden of Innocence compared to what's going to
happen now. Trump is not going to walk to the rostrum and say, `The
people have spoken and I accept their verdict.'''

Even Trump's closest allies in Congress, Mitch McConnell and Kevin
McCarthy, couldn't stomach the idea of postponing the election, and both
have swallowed a lot over the last three years. Trump's little trial
balloon blew up like the Hindenburg.

\href{https://www.nytimes.com/2020/07/30/us/politics/trump-delay-election.html}{Alexander
Burns wrote} in The New York Times that Trump was too pathetic to be a
tyrant: ``Far from a strongman, Mr. Trump has lately become a heckler in
his own government, promoting medical conspiracy theories on social
media, playing no constructive role in either the management of the
coronavirus pandemic or the negotiation of an economic rescue plan in
Congress --- and complaining endlessly about the unfairness of it all.''

Talk about unfair: The one thing holding the country together has been
the additional \$600 per week in unemployment benefits that has allowed
millions to pay the rent and fill the fridge. Republicans, though, are
so convinced that the few extra hundred dollars in jobless pay is
keeping people from work that they are loath to renew it. Unless
Congress gets it together soon and finds a way to extend the aid, the
country is going to be facing a catastrophe of homelessness and need
that makes these past few months look pleasant.

After the president began doing the coronavirus briefings again, he
tried a ``new'' tone, saying he was getting used to masks --- ``Think
about patriotism. Maybe it helps. It helps'' --- then face-planting by
offering good wishes to a past party pal and accused pedophile enabler,
Ghislaine Maxwell. But then things got really crazy as he defended a
retweet of a doctor who has promoted hydroxychloroquine as well as
declaimed on the existence of alien D.N.A. and demon sperm.

``I thought her voice was an important voice, but I know nothing about
her,'' he told reporters. (As he told Barstool sports: ``It's the
retweets that get you in trouble.'') He fell into more self-pity,
complaining about his ratings compared to those of Dr. Anthony Fauci:
``Nobody likes me. It can only be my personality, that's all.''

It has been clear for some time that Trump's Panglossian attitude toward
the virus was turning him into a public health menace.

But this week, the culture war over masks crystallized with the death of
Herman Cain. The former Republican presidential candidate, who dissed
masks and Covid restrictions, proudly tweeted a picture from the Trump
rally in Tulsa, surrounded by his fellow mask-less friends. ``Having a
fantastic time,'' he wrote. Nine days later, he tested positive for
corona. As we have learned, this virus often has the final say.

Right away, White House officials knew that this death would be laid at
Trump's feet. They began warning reporters that they should not
politicize Cain's death.

At the Friday White House briefing, asked if officials were concerned
that the 74-year-old Cain may have contracted the virus at the rally,
Kayleigh McEnany replied, ``We'll never know,'' and sanctimoniously
added, ``I will not politicize Herman Cain's passing.''

But it is undeniable that Trump politicized masks and set a lethal
example.

As
\href{https://www.nytimes.com/2020/07/30/us/politics/herman-cain-gop-coronavirus.html}{Jeremy
Peters wrote} in The Times, Republican officials all over the country
``have adopted a similar tone of skepticism and defiance, rejecting the
advice of public health officials and deferring instead to principles
they said were equally important: conservative values of economic
freedom and personal liberty.''

So conservatives are willing to embrace a new ethos? Give me liberty.
And death.

\emph{The Times is committed to publishing}
\href{https://www.nytimes.com/2019/01/31/opinion/letters/letters-to-editor-new-york-times-women.html}{\emph{a
diversity of letters}} \emph{to the editor. We'd like to hear what you
think about this or any of our articles. Here are some}
\href{https://help.nytimes.com/hc/en-us/articles/115014925288-How-to-submit-a-letter-to-the-editor}{\emph{tips}}\emph{.
And here's our email:}
\href{mailto:letters@nytimes.com}{\emph{letters@nytimes.com}}\emph{.}

\emph{Follow The New York Times Opinion section on}
\href{https://www.facebook.com/nytopinion}{\emph{Facebook}}\emph{,}
\href{http://twitter.com/NYTOpinion}{\emph{Twitter (@NYTopinion)}}
\emph{and}
\href{https://www.instagram.com/nytopinion/}{\emph{Instagram}}\emph{.}

Advertisement

\protect\hyperlink{after-bottom}{Continue reading the main story}

\hypertarget{site-index}{%
\subsection{Site Index}\label{site-index}}

\hypertarget{site-information-navigation}{%
\subsection{Site Information
Navigation}\label{site-information-navigation}}

\begin{itemize}
\tightlist
\item
  \href{https://help.nytimes.com/hc/en-us/articles/115014792127-Copyright-notice}{©~2020~The
  New York Times Company}
\end{itemize}

\begin{itemize}
\tightlist
\item
  \href{https://www.nytco.com/}{NYTCo}
\item
  \href{https://help.nytimes.com/hc/en-us/articles/115015385887-Contact-Us}{Contact
  Us}
\item
  \href{https://www.nytco.com/careers/}{Work with us}
\item
  \href{https://nytmediakit.com/}{Advertise}
\item
  \href{http://www.tbrandstudio.com/}{T Brand Studio}
\item
  \href{https://www.nytimes.com/privacy/cookie-policy\#how-do-i-manage-trackers}{Your
  Ad Choices}
\item
  \href{https://www.nytimes.com/privacy}{Privacy}
\item
  \href{https://help.nytimes.com/hc/en-us/articles/115014893428-Terms-of-service}{Terms
  of Service}
\item
  \href{https://help.nytimes.com/hc/en-us/articles/115014893968-Terms-of-sale}{Terms
  of Sale}
\item
  \href{https://spiderbites.nytimes.com}{Site Map}
\item
  \href{https://help.nytimes.com/hc/en-us}{Help}
\item
  \href{https://www.nytimes.com/subscription?campaignId=37WXW}{Subscriptions}
\end{itemize}
