Sections

SEARCH

\protect\hyperlink{site-content}{Skip to
content}\protect\hyperlink{site-index}{Skip to site index}

\href{https://www.nytimes.com/section/style}{Style}

\href{https://myaccount.nytimes.com/auth/login?response_type=cookie\&client_id=vi}{}

\href{https://www.nytimes.com/section/todayspaper}{Today's Paper}

\href{/section/style}{Style}\textbar{}The Little Fraught Schoolhouse

\url{https://nyti.ms/3i12VET}

\begin{itemize}
\item
\item
\item
\item
\item
\end{itemize}

\href{https://www.nytimes.com/spotlight/at-home?action=click\&pgtype=Article\&state=default\&region=TOP_BANNER\&context=at_home_menu}{At
Home}

\begin{itemize}
\tightlist
\item
  \href{https://www.nytimes.com/2020/08/03/well/family/the-benefits-of-talking-to-strangers.html?action=click\&pgtype=Article\&state=default\&region=TOP_BANNER\&context=at_home_menu}{Talk:
  To Strangers}
\item
  \href{https://www.nytimes.com/2020/08/01/at-home/coronavirus-make-pizza-on-a-grill.html?action=click\&pgtype=Article\&state=default\&region=TOP_BANNER\&context=at_home_menu}{Make:
  Grilled Pizza}
\item
  \href{https://www.nytimes.com/2020/07/31/arts/television/goldbergs-abc-stream.html?action=click\&pgtype=Article\&state=default\&region=TOP_BANNER\&context=at_home_menu}{Watch:
  'The Goldbergs'}
\item
  \href{https://www.nytimes.com/interactive/2020/at-home/even-more-reporters-editors-diaries-lists-recommendations.html?action=click\&pgtype=Article\&state=default\&region=TOP_BANNER\&context=at_home_menu}{Explore:
  Reporters' Google Docs}
\end{itemize}

Advertisement

\protect\hyperlink{after-top}{Continue reading the main story}

Supported by

\protect\hyperlink{after-sponsor}{Continue reading the main story}

\hypertarget{the-little-fraught-schoolhouse}{%
\section{The Little Fraught
Schoolhouse}\label{the-little-fraught-schoolhouse}}

New York parents of means are seeking less congested classrooms near
their second homes in the Hamptons and Hudson Valley. Not everyone is
happy about this.

\includegraphics{https://static01.nyt.com/images/2020/08/02/fashion/02HAMPTONS-SCHOOL1/02HAMPTONS-SCHOOL1-articleLarge.jpg?quality=75\&auto=webp\&disable=upscale}

\includegraphics{https://static01.nyt.com/images/2020/05/06/reader-center/alex-williams/alex-williams-thumbLarge-v2.jpg}

By Alex Williams

\begin{itemize}
\item
  Published Aug. 1, 2020Updated Aug. 4, 2020
\item
  \begin{itemize}
  \item
  \item
  \item
  \item
  \item
  \end{itemize}
\end{itemize}

Carlie Lawrence, 42, a fashion executive and mother of two from
Manhattan, was facing the same fear that
seemingly\href{https://www.nytimes.com/2020/07/30/nyregion/pod-schools-hastings-on-hudson.html}{every
parent} of school-age children has these days: what if schools don't
open in the fall?

Luckily for her, Ms. Lawrence and her husband, who works in finance, own
a shingled, four-bedroom house in the Hamptons, where schools, she
figured, may have a better chance at opening.

So they registered their 5-year old twin sons at the public elementary
school in East Hampton, and are planning to continue to live in their
weekend home full-time. Her sons, she thought, would benefit from the
fresh air, far from the stresses of the pandemic-stricken city.

``The decision is driven by knowing that schools in New York City will
not be able to offer a full-time, in-school option,'' Ms. Lawrence said,
``and visualizing what possibly will be a full year of two working
parents, working from home, while being cooped up all day in an
apartment with kids.''

The nagging uncertainties facing New York City schools this fall ---
Will they open? How crowded will they be? What precautions are being
taken? Will there be a resurgence of coronavirus? --- have parents
desperately searching for options. For those wealthy enough to have
second homes, one obvious-seeming option is to take their children out
of the city, and transfer them to schools in the country or near the
beach, where classes tend to be smaller and the air is, in theory,
cleaner.

No wonder public and private schools in second-home regions are seeing a
surge in enrollment. The influx is particularly notable in the Hamptons,
a region that usually enjoys a quiet off-season, with relatively few
year-round families with school-age children, according to local school
officials.

Andi O'Hearn, the head of advancement and operations for the
\href{https://www.ross.org/}{Ross School} in East Hampton, an elite K-12
private school founded by Courtney Sale Ross, the widow of Time Warner
chief
\href{https://www.nytimes.com/1992/12/21/obituaries/the-creator-of-time-warner-steven-j-ross-is-dead-at-65.html}{Steven
J. Ross}, said administrators there had received a record number of
applications for the fall.

``Due to the fear of Covid, there is a lot of uncertainty, and they're
trying to make the best decisions for their children,'' Ms. O'Hearn
said, referring to anxious parents from the city.

The nonprofit school, whose annual tuition ranges from \$22,700 to
\$45,000 for day students (boarding students pay \$74,000), features a
tennis academy and courses in social entrepreneurship, robotics and
figurative sculpture. More important, it plans to open full-time in the
fall, with 15 tents for outdoor learning on the 63-acre grounds, and
pupils separated into learning pods, organized by grade level or high
school section, to minimize exposure to other students.

So far, registrations for its elementary school grades have doubled, to
93 students from 57 last year, for the fall. ``I'm glad we're able to
accommodate them, although we're starting to really struggle,'' Ms.
O'Hearn said. ``We have already gone to waiting lists in six different
grades.''

Affluent parents in the Hamptons are also applying to
\href{https://www.avenues.org/}{Avenues: The World School}, a private
school based in the Chelsea neighborhood of Manhattan that caters to the
children of moguls and movie stars (Suri Cruise, the daughter of Tom
Cruise and Katie Holmes, was reportedly
\href{https://www.businessinsider.com/avenues-world-school-new-york-city-photos-tour-amenities-2019-4}{a
student there}).

It is opening a satellite campus in East Hampton this fall called
\href{https://studio.avenues.org/}{Avenues Studio Hamptons}, with room
for 60 students, grades 4 through 11, at a cost of \$48,000 per year.
The school received a dozen inquiries for every available space, and 45
students from the Manhattan campus are planning to transfer there this
fall, according to a school spokeswoman.

The exodus of wealthy families from the city comes at a moment that is
fraught socially and politically. The rich have access to multiple
educational options, including at-home
\href{https://www.nytimes.com/2020/07/22/parenting/school-pods-coronavirus.html}{``pandemic
pods''} with private tutors, or in the case of Avenues, personal mentors
and instruction in 50 languages, including Punjabi and Swahili. Children
from low-income homes, meanwhile, sometimes lack laptops and internet
access needed for remote learning, and may get less funding as the
children of affluent parents abandon the public school system.

And there are no guarantees that the Hamptons and other leafy enclaves
are immune from another coronavirus outbreak that can shut down the
schools.

Schools throughout New York State are still weighing options. Mayor Bill
de Blasio has announced that schools in New York City will open only if
the city sustains a coronavirus test positivity rate
\href{https://www.nytimes.com/2020/07/31/world/coronavirus-covid-19.html?action=click\&module=Top\%20Stories\&pgtype=Homepage\#link-22c71cd7}{below
three percent}, and is advocating a
``\href{https://www.nytimes.com/2020/07/08/nyregion/nyc-schools-reopening-plan.html}{blended
learning'' model}, combining virtual lessons and classrooms
\href{https://www.nytimes.com/2020/07/08/nyregion/nyc-schools-reopening-plan.html}{one
to three days} a week with staggered schedules to reduce density. City
Hall will not announce
\href{https://www.nbcnewyork.com/news/coronavirus/nyc-to-ask-state-for-extension-on-specific-back-to-school-plans/2540164/}{plans
for individual schools} until mid-August.

Gov. Andrew M. Cuomo will have the final say, adding that a decision for
school reopening will come in
\href{https://www.nbcnewyork.com/news/coronavirus/nyc-to-ask-state-for-extension-on-specific-back-to-school-plans/2540164/}{early
August}. ``We will not use our children as
\href{https://www.cnbc.com/video/2020/07/13/new-york-gov-cuomo-on-reopening-schools-were-not-going-to-use-our-children-as-guinea-pigs.html}{guinea
pigs},'' Mr. Cuomo said a few weeks ago.

But for some second homeowners, the decision is already made. If the
choice is between a crowded, poorly ventilated classroom in a dense
urban area, and an airy classroom in a bucolic setting, the choice is
clear.

``I think the biggest factor is everyone's mental well-being,'' said Ms.
Lawrence, the fashion executive. In East Hampton, her sons have access
to a backyard, a swimming pool, a tennis court, and a trampoline.
``We're in our own home, it's our little haven, and we're safe. So that
relaxed energy that we have has helped this be an easy time for them.''
(The family is also considering forming a private pandemic pod with
other Hamptons parents, in case schools there do not reopen).

It's not just private schools that are seeing a surge. Enrollment at the
Amagansett School, a public K-6 school in East Hampton, has reportedly
\href{https://www.easthamptonstar.com/education/2020716/amagansett-school-expects-enrollment-to-double}{doubled}
for the fall to about 150 students, after years of attrition because of
escalating real estate prices, said Peter Van Scoyoc, the town
supervisor of East Hampton.

``The district's enrollment had been in a steady decline for years, just
because it was getting more expensive for year-round families to live
there,'' Mr. Van Scoyoc said. ``Now, that's going the other way.''

\includegraphics{https://static01.nyt.com/images/2020/08/02/fashion/02HAMPTONSCHOOL2/02HAMPTONSCHOOL2-articleLarge.jpg?quality=75\&auto=webp\&disable=upscale}

While most schools still have capacity for newcomers, Mr. Van Scoyoc
said, it remains to be seen how year-round residents respond to a
potential influx. ``There's already a certain amount of tension every
summer as the population goes from 22,000 to maybe 120,000 on the Fourth
of July weekend,'' he said. ``And now there's concern among some
year-rounders, who have to be expecting heavier traffic and more demand
on services.''

More newcomers may be on their way. Cindy Scholz, a real estate agent
for Compass with clients in Manhattan and the Hamptons, said she has
been deluged with requests from young parents from New York City looking
to buy a house on the East End in time to enroll their children into
local schools.

``Their No. 1 priority is school districts, where before, that was never
a consideration,'' Ms. Scholz said. ``People understand that there might
be a fear of a second wave of coronavirus, and it does feel a lot more
normal outside the city than inside it,'' where
\href{https://www.nytimes.com/2020/07/26/nyregion/nyc-coronavirus-time-life-building.html}{deserted
office buildings}and uncrowded subways are a stark reminder of the
pandemic's effects.

Schools in the Hudson Valley are also seeing a rise in enrollment.
Administrators at the Rhinebeck Central School District, for example,
have seen a slight uptick in school registration, especially among New
Yorkers with second homes nearby.

And with the start of the school year fast approaching, the question of
where to enroll their children has become a hot topic among parents at
socially distanced backyard barbecues, pool parties and tennis courts
throughout the Hamptons, Hudson Valley, the Catskills and other places
where weekenders have found refuge.

For some of these parents, the choice is so wearying that they have
decided to pull their children out of the school system entirely and
home-school, which is undoubtedly easier when home is not a cramped
apartment, but a weekend house with guest bedrooms, home offices and
lots of outdoor space.

\href{https://www.glamour.com/story/noria-morales-what-it-costs-to-be-me}{Noria
Morales}, a founder of \href{https://www.thewonder.us/}{The Wonder}, a
club for families in Manhattan, and her husband, a restaurateur,
recently decided not to plunk down a small fortune for their children's
private school, Lycée\href{https://www.lfny.org/}{Francais de New York},
this fall.

Instead, they moved up to their weekend house on 176 rustic acres near
Elizaville, N.Y. ,in the Hudson Valley and plan to home-school their
children, using a portion of the money they saved on private school for
a part-time private tutor to help out.

``My kids are in second grade and fourth grade, so there's nothing I can
really do to set them back academically,'' Ms. Morales said. ``It's not
like I have to teach them algebra. Up here, you're surrounded by trees
and birds, and in my mind, I can encourage curiosity.''

But there are trade-offs, even for the wealthy. At small-town schools in
the Hamptons or the Hudson Valley, children tend to be cut off from the
racial, economic, and cultural diversity of city schools, not to the
mention the museums and other cultural institutions that help round out
their education.

Little surprise, then, that some parents with immense resources are
keeping their options open, while everyone else is waiting to see what
schools will do in the fall.

Alli McCartney, 44, a wealth manager for high-net-worth clients, was
planning to send her two children --- Maxwell, 10; and Luca, 8 --- to
the Dwight School, a private school in Manhattan, this fall. But since
she and her husband, Scott McCartney, have largely abandoned their West
Village dwelling for their weekend house in Montauk, N.Y., they have
also registered their children at the local public school, in case
Dwight is unable to reopen this fall. (so far, the school plans to).

``Just as I have been doing with clients for years, I am using a
three-pronged approach, asking myself three questions,'' Ms. McCartney
said. ``What is best for my children? What is best for my business? What
is best for my balance sheet?''

Others are taking a ``when life gives you lemons, make Limonata''
approach to the crisis.

Ms. McCartney said some of her Hamptons friends are exploring a form of
schools arbitrage, enrolling their children in public schools in the
Hamptons while waiting for slot to open at elite private schools in
Manhattan this spring.

``They're using this as a way to upgrade, and get into private schools
they never could get into otherwise,'' Ms. McCartney said.

Advertisement

\protect\hyperlink{after-bottom}{Continue reading the main story}

\hypertarget{site-index}{%
\subsection{Site Index}\label{site-index}}

\hypertarget{site-information-navigation}{%
\subsection{Site Information
Navigation}\label{site-information-navigation}}

\begin{itemize}
\tightlist
\item
  \href{https://help.nytimes.com/hc/en-us/articles/115014792127-Copyright-notice}{©~2020~The
  New York Times Company}
\end{itemize}

\begin{itemize}
\tightlist
\item
  \href{https://www.nytco.com/}{NYTCo}
\item
  \href{https://help.nytimes.com/hc/en-us/articles/115015385887-Contact-Us}{Contact
  Us}
\item
  \href{https://www.nytco.com/careers/}{Work with us}
\item
  \href{https://nytmediakit.com/}{Advertise}
\item
  \href{http://www.tbrandstudio.com/}{T Brand Studio}
\item
  \href{https://www.nytimes.com/privacy/cookie-policy\#how-do-i-manage-trackers}{Your
  Ad Choices}
\item
  \href{https://www.nytimes.com/privacy}{Privacy}
\item
  \href{https://help.nytimes.com/hc/en-us/articles/115014893428-Terms-of-service}{Terms
  of Service}
\item
  \href{https://help.nytimes.com/hc/en-us/articles/115014893968-Terms-of-sale}{Terms
  of Sale}
\item
  \href{https://spiderbites.nytimes.com}{Site Map}
\item
  \href{https://help.nytimes.com/hc/en-us}{Help}
\item
  \href{https://www.nytimes.com/subscription?campaignId=37WXW}{Subscriptions}
\end{itemize}
