Sections

SEARCH

\protect\hyperlink{site-content}{Skip to
content}\protect\hyperlink{site-index}{Skip to site index}

\href{https://www.nytimes.com/section/politics}{Politics}

\href{https://myaccount.nytimes.com/auth/login?response_type=cookie\&client_id=vi}{}

\href{https://www.nytimes.com/section/todayspaper}{Today's Paper}

\href{/section/politics}{Politics}\textbar{}Homeland Security Reassigns
Official Whose Office Compiled Intelligence on Journalists

\url{https://nyti.ms/3hQ9z0S}

\begin{itemize}
\item
\item
\item
\item
\item
\end{itemize}

\href{https://www.nytimes.com/news-event/george-floyd-protests-minneapolis-new-york-los-angeles?action=click\&pgtype=Article\&state=default\&region=TOP_BANNER\&context=storylines_menu}{Race
and America}

\begin{itemize}
\tightlist
\item
  \href{https://www.nytimes.com/2020/07/26/us/protests-portland-seattle-trump.html?action=click\&pgtype=Article\&state=default\&region=TOP_BANNER\&context=storylines_menu}{Protesters
  Return to Other Cities}
\item
  \href{https://www.nytimes.com/2020/07/24/us/portland-oregon-protests-white-race.html?action=click\&pgtype=Article\&state=default\&region=TOP_BANNER\&context=storylines_menu}{Portland
  at the Center}
\item
  \href{https://www.nytimes.com/2020/07/23/podcasts/the-daily/portland-protests.html?action=click\&pgtype=Article\&state=default\&region=TOP_BANNER\&context=storylines_menu}{Podcast:
  Showdown in Portland}
\item
  \href{https://www.nytimes.com/interactive/2020/07/16/us/black-lives-matter-protests-louisville-breonna-taylor.html?action=click\&pgtype=Article\&state=default\&region=TOP_BANNER\&context=storylines_menu}{45
  Days in Louisville}
\end{itemize}

Advertisement

\protect\hyperlink{after-top}{Continue reading the main story}

Supported by

\protect\hyperlink{after-sponsor}{Continue reading the main story}

\hypertarget{homeland-security-reassigns-official-whose-office-compiled-intelligence-on-journalists}{%
\section{Homeland Security Reassigns Official Whose Office Compiled
Intelligence on
Journalists}\label{homeland-security-reassigns-official-whose-office-compiled-intelligence-on-journalists}}

Brian Murphy's office compiled reports of protesters and journalists who
were covering the Trump administration's response to unrest in Portland,
Ore., last month.

\includegraphics{https://static01.nyt.com/images/2020/08/01/us/politics/01dc-murphy-search/merlin_175080105_075defb4-72fe-4589-90d0-601ba0ebac78-articleLarge.jpg?quality=75\&auto=webp\&disable=upscale}

\href{https://www.nytimes.com/by/zolan-kanno-youngs}{\includegraphics{https://static01.nyt.com/images/2019/12/13/reader-center/author-zolan-kanno-youngs/author-zolan-kanno-youngs-thumbLarge.png}}\href{https://www.nytimes.com/by/adam-goldman}{\includegraphics{https://static01.nyt.com/images/2018/07/12/multimedia/author-adam-goldman/author-adam-goldman-thumbLarge.png}}

By \href{https://www.nytimes.com/by/zolan-kanno-youngs}{Zolan
Kanno-Youngs} and \href{https://www.nytimes.com/by/adam-goldman}{Adam
Goldman}

\begin{itemize}
\item
  Aug. 1, 2020
\item
  \begin{itemize}
  \item
  \item
  \item
  \item
  \item
  \end{itemize}
\end{itemize}

WASHINGTON --- The head of the Department of Homeland Security's
intelligence branch was removed from his position after his office
compiled reports about protesters and journalists covering the Trump
administration's response to unrest in Portland, Ore., last month.

Brian Murphy, the acting under secretary for intelligence and analysis,
was reassigned to a new position in the department after his office
disseminated to the law enforcement community ``open-source intelligence
reports'' containing Twitter posts of journalists, noting they had
published leaked unclassified documents, according to an administration
official familiar with the matter. It was not clear what Mr. Murphy's
new position would be.

Chad F. Wolf, the acting secretary for the Department of Homeland
Security, made the decision on Friday after ordering the office to halt
the intelligence examination, the administration official said. Mr. Wolf
has also asked the Homeland Security's Office of Inspector General to
investigate any efforts by the intelligence branch to collect
information about protesters or journalists.

The ouster came after
\href{https://www.washingtonpost.com/national-security/dhs-compiled-intelligence-reports-on-journalists-who-published-leaked-documents/2020/07/30/5be5ec9e-d25b-11ea-9038-af089b63ac21_story.html}{The
Washington Post reported} that Mr. Murphy's office compiled reports that
in part targeted
\href{https://www.nytimes.com/2020/07/28/us/federal-agents-portland-seattle-protests.html}{The
New York Times's publishing} of an intelligence analysis indicating that
the Homeland Security Department had little understanding of the
situation in Portland when it deployed teams of tactical agents in
camouflage to face crowds of protesters.

In addition to summarizing the tweets of a Times reporter, Mike Baker,
the intelligence reports also included a tweet by Benjamin Wittes, the
editor in chief of Lawfare, a blog about law and national security, who
had shared an internal memo that warned Homeland Security officers not
to leak to the press.

The reports also included a tweet from Mr. Wittes that showed an email
from Mr. Murphy telling the intelligence officers to refer to
individuals attacking the federal courthouse in Portland as ``VIOLENT
ANTIFA ANARCHISTS.''

Mr. Murphy's conclusion about the motivations of the individuals in
Portland came just days after intelligence officers issued the memo
reported by The Times that said the agency had ``low confidence'' that
the attacks against the federal courthouse reflected a broader threat.

The issue prompted the Senate Intelligence Committee to send a letter to
Mr. Murphy questioning the intelligence-gathering effort of journalists
and protesters.

Separately, Representative Adam B. Schiff, Democrat of California and
the chairman of the House Intelligence Committee, said in a statement
Saturday that his committee had been ``conducting rigorous oversight of
the Office of Intelligence and Analysis, including actions by Acting
Undersecretary Murphy prior to his abrupt and apparent reassignment.

``In light of recent public reports, we are concerned that Murphy may
have provided incomplete and potentially misleading information to
Committee staff during our recent oversight engagement,'' Mr. Schiff
continued, adding that the committee would ``be expanding our oversight
even further in the coming days.''

The Department of Homeland Security has already faced widespread
backlash for the aggressive behavior of the tactical teams in Portland,
as well as investigations by the inspectors general for the Department
of Homeland Security and the Department of Justice.

Mr. Murphy, formerly with the F.B.I., led an office with the Homeland
Security Department charged with sharing information about potential
national security threats with federal, state and local law enforcement
agencies. Such a coordinating effort was one of the motivations in
creating the department after the Sept. 11, 2001, attacks.

In 2015, Mr. Murphy joined F.B.I. headquarters to work on an effort
known as Countering Violent Extremism, or C.V.E., after serving as an
assistant special agent in charge of counterterrorism in Chicago. Mr.
Murphy was known as an ambitious investigator who was once profiled in a
\href{https://www.esquire.com/news-politics/a2184/esq0307murphy/}{self-aggrandizing
article} about a terrorism case he had worked on. But some former agents
and Justice Department officials familiar with Mr. Murphy's work at the
time, who requested anonymity to discuss internal discussions at the
agencies, expressed concern about some C.V.E. proposals, his tendency to
ignore the rules and failure to coordinate his activities.

One agent at the time raised an alarm that Mr. Murphy wanted to prepare
materials for Chicago public schools without disclosing the F.B.I.'s
participation, according to an internal bureau document provided to The
New York Times. That would have violated F.B.I. policy requiring such
outreach to be public or overt.

Other former officials said that Mr. Murphy wanted to tap coaches,
therapists, social workers and religious leaders in several cities to
help steer people under the sway of Islamic extremism away from a
potentially violent future. That was not a bad idea, the former
officials said, but Mr. Murphy pushed internally to make those community
leaders sign memorandums of understanding with the F.B.I.

By doing so, Mr. Murphy would then have been able to track whether those
people in the program were headed down the wrong path again. That would
have essentially deputized community leaders to be arms of the bureau,
former F.B.I. and Justice Department officials said, a move that would
have only stoked existing concerns in the Muslim community that the
bureau was using outreach to spy on people. Officials eventually
scrapped Mr. Murphy's plan, calling it ill-conceived and legally
problematic.

One former official said that Mr. Murphy ``didn't have a good sense of
what the blowback would be.''

Advertisement

\protect\hyperlink{after-bottom}{Continue reading the main story}

\hypertarget{site-index}{%
\subsection{Site Index}\label{site-index}}

\hypertarget{site-information-navigation}{%
\subsection{Site Information
Navigation}\label{site-information-navigation}}

\begin{itemize}
\tightlist
\item
  \href{https://help.nytimes.com/hc/en-us/articles/115014792127-Copyright-notice}{©~2020~The
  New York Times Company}
\end{itemize}

\begin{itemize}
\tightlist
\item
  \href{https://www.nytco.com/}{NYTCo}
\item
  \href{https://help.nytimes.com/hc/en-us/articles/115015385887-Contact-Us}{Contact
  Us}
\item
  \href{https://www.nytco.com/careers/}{Work with us}
\item
  \href{https://nytmediakit.com/}{Advertise}
\item
  \href{http://www.tbrandstudio.com/}{T Brand Studio}
\item
  \href{https://www.nytimes.com/privacy/cookie-policy\#how-do-i-manage-trackers}{Your
  Ad Choices}
\item
  \href{https://www.nytimes.com/privacy}{Privacy}
\item
  \href{https://help.nytimes.com/hc/en-us/articles/115014893428-Terms-of-service}{Terms
  of Service}
\item
  \href{https://help.nytimes.com/hc/en-us/articles/115014893968-Terms-of-sale}{Terms
  of Sale}
\item
  \href{https://spiderbites.nytimes.com}{Site Map}
\item
  \href{https://help.nytimes.com/hc/en-us}{Help}
\item
  \href{https://www.nytimes.com/subscription?campaignId=37WXW}{Subscriptions}
\end{itemize}
