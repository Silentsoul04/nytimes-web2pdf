Sections

SEARCH

\protect\hyperlink{site-content}{Skip to
content}\protect\hyperlink{site-index}{Skip to site index}

\href{https://www.nytimes.com/section/politics}{Politics}

\href{https://myaccount.nytimes.com/auth/login?response_type=cookie\&client_id=vi}{}

\href{https://www.nytimes.com/section/todayspaper}{Today's Paper}

\href{/section/politics}{Politics}\textbar{}Law to Reduce Crack Cocaine
Sentences Leaves Some Imprisoned

\url{https://nyti.ms/2Xh0ZAd}

\begin{itemize}
\item
\item
\item
\item
\item
\end{itemize}

Advertisement

\protect\hyperlink{after-top}{Continue reading the main story}

Supported by

\protect\hyperlink{after-sponsor}{Continue reading the main story}

\hypertarget{law-to-reduce-crack-cocaine-sentences-leaves-some-imprisoned}{%
\section{Law to Reduce Crack Cocaine Sentences Leaves Some
Imprisoned}\label{law-to-reduce-crack-cocaine-sentences-leaves-some-imprisoned}}

Critics say the First Step Act is being applied too arbitrarily by
judges who are taking a hard line when it comes to revisiting nonviolent
drug sentences.

\includegraphics{https://static01.nyt.com/images/2020/08/01/us/politics/01dc-firststep/01dc-firststep-articleLarge.jpg?quality=75\&auto=webp\&disable=upscale}

By \href{https://www.nytimes.com/by/hailey-fuchs}{Hailey Fuchs}

\begin{itemize}
\item
  Aug. 1, 2020
\item
  \begin{itemize}
  \item
  \item
  \item
  \item
  \item
  \end{itemize}
\end{itemize}

WASHINGTON ---~Lazelle Maxwell, 48, is nearly 12 years into a 30-year
sentence for a nonviolent crack cocaine charge, a penalty exacerbated by
previous run-ins with law enforcement that led to his designation as a
career offender.

Three years into remission after a diagnosis of prostate cancer, Mr.
Maxwell has no major disciplinary infractions on his prison record. He
spends most of his days behind bars caring for an elderly, partly
paralyzed inmate at a low-security federal penitentiary in Butner, N.C.

More than a year and a half ago, he learned that he might be eligible
for a reduction in his sentence under
\href{https://www.nytimes.com/2018/12/14/us/politics/jared-kushner-criminal-justice-bill.html}{the
First Step Act}, a bipartisan bill President Trump signed in December
2018 that has been lauded as the most consequential criminal justice
legislation in a generation. Mr. Maxwell sent a hopeful one-page note to
the judge who sentenced him, asking for a lawyer so he could apply.

The judge rejected his request. Mr. Maxwell never had a chance to plead
his case.

``It really just knocked all the breath out of me, for real,'' Mr.
Maxwell said from an office in the Butner prison.

He later learned that the judge, Danny C. Reeves of the Eastern District
of Kentucky, rarely if ever grants motions for resentencing in First
Step Act cases.

By and large, the First Step Act has met its goal of reducing federal
sentences for nonviolent drug offenders, addressing a longstanding
disparity in which crack cocaine convictions in particular led to far
harsher penalties than other drug offenses and disproportionately
increased imprisonment of Black men.

Thousands of inmates across the country, predominantly people of color,
have been released or resentenced under a provision of the new law that
allowed changes to the sentencing provisions to be applied
retroactively. As of January, 2,387 inmates had their sentences reduced
under the provision that allows some crack cocaine offenders to be
resentenced, out of 2,660 that the United States Sentencing Commission
estimated in May 2018 were eligible.

But the law gives judges discretion in reducing sentences, leaving some
inmates like Mr. Maxwell without much recourse when their applications
are rejected. In those cases, activists and defense lawyers worry that
the First Step Act gives too much authority to judges to determine who
does and does not deserve early release.

``It's like the luck of the draw,'' said Sarah Ryan, a professor at
Wesleyan University who has analyzed hundreds of First Step Act
resentencing cases. ``You've got people sitting in prison during a
pandemic, and it's not supposed to come down to who your judge is. It's
supposed to come down to the law.''

The simple enactment of the bill was no guarantee for inmates. This
provision of the bill did not mandate that the judges must resentence
eligible offenders; Congress specified that ``nothing in this section
shall be construed to require a court to reduce any sentence.''

\includegraphics{https://static01.nyt.com/images/2020/08/01/us/politics/01dc-firststep02/merlin_148388541_459411ff-5200-4530-ad87-666685051a04-articleLarge.jpg?quality=75\&auto=webp\&disable=upscale}

The U.S. Sentencing Commission does not track rejected applications for
First Step Act resentencing, so it is impossible to know how many
eligible inmates have had their petitions denied. In many cases, inmates
who ask to be resentenced are blatantly ineligible, often because they
were not charged for a crack cocaine offense. In others, eligible
inmates do not have lawyers to represent them when their requests are
denied.

Patrick Estell Jones, the first federal inmate to die of the
coronavirus, was sentenced to 27 years in prison for a nonviolent drug
charge because he lived within 1,000 feet of a junior college. Mr. Jones
died this spring, a month after a judge denied his First Step
application, citing his criminal history.

The appeals courts have freed some inmates whose federal trial judges
declined to resentence them. In the Western District of North Carolina,
Brooks Tyrone Chambers applied in May 2019 to reduce his almost 22-year
sentence for a crack cocaine offense after he was wrongly labeled a
career offender.

The district court, where four of the five judges are former
prosecutors, maintained that he still qualified as a career offender and
denied his application. The U.S. Court of Appeals for the Fourth Circuit
issued a split ruling overturning that decision. The government, which
contests the vast majority of First Step applications, decided not to
appeal, and Mr. Chambers walked free in late July.

In another case, the First Circuit Court of Appeals overturned a
decision from a judge in New Hampshire, who found that an inmate was
ineligible for the First Step Act because he had been caught with too
little crack cocaine.

The section of the act that governs resentencing for crack cocaine
convictions is just four sentences long. It made retroactive the 2010
Fair Sentencing Act, which reduced sentencing disparities between crack
and powder cocaine. Courts have been relatively slow to determine some
of the ambiguities of the act, including whether to consider behavior
behind bars or other concurrent charges as factors in the decision.

Many public defenders --- who handle most of these applications --- in
the toughest districts declined to speak on the record for fear of
upsetting the judges who oversee their cases.

Parks Small, a federal public defender in Columbia, S.C., said an
imperfect First Step Act was still better than nothing, calling the bill
a ``godsend'' for many inmates. He added that judges varied as to the
importance they placed on the original offense or the inmate's behavior
behind bars.

``You give it to different judges, they're going to come up with
different opinions,'' Mr. Small said. ``It's frustrating.''

That discretion --- critics would say arbitrariness --- came into play
in Mr. Maxwell's case.

Complicating matters, the Eastern District of Kentucky is one of two
districts in the country without a public defender's office, and First
Step Act applicants are not guaranteed a lawyer under the law. Mr.
Maxwell got a break when Shon Hopwood, a lawyer who had spent years in
federal prison for armed robbery, decided to take up his case.

After Judge Reeves rejected his initial request last year, Mr. Maxwell
received some good news in January when the Sixth Circuit found that the
judge could not construe Mr. Maxwell's letter as a motion for
resentencing. He would be allowed a chance to demonstrate why he
deserved to be released early.

Still, despite his pleas that he be reassigned to a new judge by the
circuit, Judge Reeves would be the one to reconsider his request.

In June, he again rejected Mr. Maxwell's request.

Image

The low-security federal penitentiary in Butner, N.C., where Mr. Maxwell
is serving his sentence.Credit...Sara D. Davis/Getty Images

``While Maxwell is eligible for a sentence reduction, he is not entitled
to it,'' Judge Reeves wrote. He ruled that because Mr. Maxwell was
sentenced above his mandatory minimum, the court did not believe he
deserved fewer years behind bars. He cited Mr. Maxwell's criminal record
during his 20s, which included several bank robberies and a charge for
fleeing from police.

``The defendant's long pattern of criminal conduct exhibits a danger to
the public and a lack of respect for the law,'' the judge wrote.

Judge Reeves is one of two members of the U.S. Sentencing Commission,
the agency instructed to study and develop sentencing policies for
federal courts.

When Mr. Maxwell was sentenced, the charge for fleeing from police was
considered violent; this would not be true today. If he were sentenced
in 2020, he most likely would not be considered a career offender, which
would lower the recommended guidelines for his sentence.

Among dozens of First Step Act cases reviewed by The New York Times,
Judge Reeves also denied resentencing to crack cocaine offenders,
compassionate release to vulnerable inmates, or requests for early
release for ``good time'' served.

He ruled that defendants who were sentenced after Mr. Trump signed the
First Step Act into law were not eligible for some of its benefits so
long as their guilty pleas were entered before the date of enactment. In
at least two cases, inmates referred to him by the U.S. Sentencing
Commission as ``potentially eligible'' for a sentence reduction were
denied.

``What he's really doing is thumbing his nose at Congress, who said
these punishments were too harsh,'' said Mr. Hopwood, Mr. Maxwell's
lawyer who helped develop the First Step Act. ``It's not just enough to
give the power back to judges to resentence because you have judges like
Judge Reeves who just won't do it.''

Mr. Maxwell maintains that he has changed. He has taken many courses to
better himself behind bars and is one shy of his associate degree ---
the bout of prostate cancer has put his graduation on hold while he is
stationed at a medical prison.

His only chance to leave prison before his 60s appears to be a grant of
clemency from the president, but his application has remained pending
since 2016. Mr. Trump has only granted 36 applications for clemency,
most recently
\href{https://www.nytimes.com/2020/07/10/us/politics/trump-roger-stone-clemency.html}{his
longtime friend and campaign adviser Roger J. Stone Jr}.

Mr. Maxwell says his years behind bars have given him a sense of
clarity. When he gets out, he wants to start a heating and cooling
business.

``I never gave myself really a chance to move forward in life,'' he
said.

Advertisement

\protect\hyperlink{after-bottom}{Continue reading the main story}

\hypertarget{site-index}{%
\subsection{Site Index}\label{site-index}}

\hypertarget{site-information-navigation}{%
\subsection{Site Information
Navigation}\label{site-information-navigation}}

\begin{itemize}
\tightlist
\item
  \href{https://help.nytimes.com/hc/en-us/articles/115014792127-Copyright-notice}{©~2020~The
  New York Times Company}
\end{itemize}

\begin{itemize}
\tightlist
\item
  \href{https://www.nytco.com/}{NYTCo}
\item
  \href{https://help.nytimes.com/hc/en-us/articles/115015385887-Contact-Us}{Contact
  Us}
\item
  \href{https://www.nytco.com/careers/}{Work with us}
\item
  \href{https://nytmediakit.com/}{Advertise}
\item
  \href{http://www.tbrandstudio.com/}{T Brand Studio}
\item
  \href{https://www.nytimes.com/privacy/cookie-policy\#how-do-i-manage-trackers}{Your
  Ad Choices}
\item
  \href{https://www.nytimes.com/privacy}{Privacy}
\item
  \href{https://help.nytimes.com/hc/en-us/articles/115014893428-Terms-of-service}{Terms
  of Service}
\item
  \href{https://help.nytimes.com/hc/en-us/articles/115014893968-Terms-of-sale}{Terms
  of Sale}
\item
  \href{https://spiderbites.nytimes.com}{Site Map}
\item
  \href{https://help.nytimes.com/hc/en-us}{Help}
\item
  \href{https://www.nytimes.com/subscription?campaignId=37WXW}{Subscriptions}
\end{itemize}
