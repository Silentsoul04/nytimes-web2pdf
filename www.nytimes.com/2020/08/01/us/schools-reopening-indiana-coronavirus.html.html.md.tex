Sections

SEARCH

\protect\hyperlink{site-content}{Skip to
content}\protect\hyperlink{site-index}{Skip to site index}

\href{https://www.nytimes.com/section/us}{U.S.}

\href{https://myaccount.nytimes.com/auth/login?response_type=cookie\&client_id=vi}{}

\href{https://www.nytimes.com/section/todayspaper}{Today's Paper}

\href{/section/us}{U.S.}\textbar{}A School Reopens, and the Coronavirus
Creeps In

\url{https://nyti.ms/30jsYkE}

\begin{itemize}
\item
\item
\item
\item
\item
\end{itemize}

\href{https://www.nytimes.com/news-event/coronavirus?action=click\&pgtype=Article\&state=default\&region=TOP_BANNER\&context=storylines_menu}{The
Coronavirus Outbreak}

\begin{itemize}
\tightlist
\item
  live\href{https://www.nytimes.com/2020/08/04/world/coronavirus-cases.html?action=click\&pgtype=Article\&state=default\&region=TOP_BANNER\&context=storylines_menu}{Latest
  Updates}
\item
  \href{https://www.nytimes.com/interactive/2020/us/coronavirus-us-cases.html?action=click\&pgtype=Article\&state=default\&region=TOP_BANNER\&context=storylines_menu}{Maps
  and Cases}
\item
  \href{https://www.nytimes.com/interactive/2020/science/coronavirus-vaccine-tracker.html?action=click\&pgtype=Article\&state=default\&region=TOP_BANNER\&context=storylines_menu}{Vaccine
  Tracker}
\item
  \href{https://www.nytimes.com/2020/08/02/us/covid-college-reopening.html?action=click\&pgtype=Article\&state=default\&region=TOP_BANNER\&context=storylines_menu}{College
  Reopening}
\item
  \href{https://www.nytimes.com/live/2020/08/04/business/stock-market-today-coronavirus?action=click\&pgtype=Article\&state=default\&region=TOP_BANNER\&context=storylines_menu}{Economy}
\end{itemize}

Advertisement

\protect\hyperlink{after-top}{Continue reading the main story}

Supported by

\protect\hyperlink{after-sponsor}{Continue reading the main story}

\hypertarget{a-school-reopens-and-the-coronavirus-creeps-in}{%
\section{A School Reopens, and the Coronavirus Creeps
In}\label{a-school-reopens-and-the-coronavirus-creeps-in}}

As more schools abandon plans for in-person classes, one that opened in
Indiana this week had to quarantine students within hours.

\includegraphics{https://static01.nyt.com/images/2020/08/01/us/01virus-schools01/01virus-schools01-articleLarge.jpg?quality=75\&auto=webp\&disable=upscale}

By \href{https://www.nytimes.com/by/eliza-shapiro}{Eliza Shapiro},
Giulia McDonnell Nieto del Rio and
\href{https://www.nytimes.com/by/shawn-hubler}{Shawn Hubler}

\begin{itemize}
\item
  Aug. 1, 2020
\item
  \begin{itemize}
  \item
  \item
  \item
  \item
  \item
  \end{itemize}
\end{itemize}

One of the first
\href{https://www.nytimes.com/2020/08/03/us/school-closing-coronavirus.html}{school
districts in the country to reopen} its doors during the coronavirus
pandemic did not even make it a day before being forced to grapple with
the issue facing every system actively trying to get students into
classrooms: What happens when someone comes to school infected?

Just hours into the first day of classes on Thursday, a call from the
county health department notified Greenfield Central Junior High School
in Indiana that a student who had walked the halls and sat in various
classrooms had tested positive for the coronavirus.

Administrators began an emergency protocol, isolating the student and
ordering everyone who had come into close contact with the person,
including other students, to quarantine for 14 days. It is unclear
whether the student infected anyone else.

``We knew it was a when, not if,'' said Harold E. Olin, superintendent
of the Greenfield-Central Community School Corporation, but were ``very
shocked it was on Day 1.''

To avoid the same scenario, hundreds of districts across the country
that were once planning to reopen their classrooms, many on a part-time
basis, have
\href{https://www.nytimes.com/2020/07/13/us/lausd-san-diego-school-reopening.html}{reversed
course} in recent weeks as infections have spiked in many states.

Those
\href{https://www.nytimes.com/2020/07/14/us/coronavirus-schools-fall.html}{that
do still reopen} are having to prepare for the near-certain likelihood
of quarantines and abrupt shutdowns when students and staff members test
positive.

Of the nation's 25 largest school districts, all but six have announced
they will start remotely, although some in places like Florida and Texas
are hoping to open classrooms after a few weeks if infection rates go
down,
\href{https://www.nytimes.com/2020/07/29/us/teacher-union-school-reopening-coronavirus.html}{over
strong objections from teachers' unions}.

\hypertarget{latest-updates-global-coronavirus-outbreak}{%
\section{\texorpdfstring{\href{https://www.nytimes.com/2020/08/04/world/coronavirus-cases.html?action=click\&pgtype=Article\&state=default\&region=MAIN_CONTENT_1\&context=storylines_live_updates}{Latest
Updates: Global Coronavirus
Outbreak}}{Latest Updates: Global Coronavirus Outbreak}}\label{latest-updates-global-coronavirus-outbreak}}

Updated 2020-08-04T20:32:50.085Z

\begin{itemize}
\tightlist
\item
  \href{https://www.nytimes.com/2020/08/04/world/coronavirus-cases.html?action=click\&pgtype=Article\&state=default\&region=MAIN_CONTENT_1\&context=storylines_live_updates\#link-1228a480}{Novavax
  sees encouraging results from two studies of its experimental
  vaccine.}
\item
  \href{https://www.nytimes.com/2020/08/04/world/coronavirus-cases.html?action=click\&pgtype=Article\&state=default\&region=MAIN_CONTENT_1\&context=storylines_live_updates\#link-4825b93}{Public
  and private schools in Maryland and elsewhere are divided over
  in-person instruction.}
\item
  \href{https://www.nytimes.com/2020/08/04/world/coronavirus-cases.html?action=click\&pgtype=Article\&state=default\&region=MAIN_CONTENT_1\&context=storylines_live_updates\#link-50f7386d}{The
  United Nations calls on policymakers to `plan thoroughly for school
  reopenings.'}
\end{itemize}

\href{https://www.nytimes.com/2020/08/04/world/coronavirus-cases.html?action=click\&pgtype=Article\&state=default\&region=MAIN_CONTENT_1\&context=storylines_live_updates}{See
more updates}

More live coverage:
\href{https://www.nytimes.com/live/2020/08/04/business/stock-market-today-coronavirus?action=click\&pgtype=Article\&state=default\&region=MAIN_CONTENT_1\&context=storylines_live_updates}{Markets}

More than 80 percent of California residents live in counties where test
positivity rates and hospitalizations are too high for school buildings
to open under state rules issued last month. And schools in Alexandria,
Va., said on Friday that they would teach remotely, tipping the entire
Washington-Baltimore metro area, with more than one million children,
into virtual learning for the fall.

In March, when schools across America abruptly shuttered, it seemed
unimaginable that educators and students would not return to school come
fall, as they have in many other parts of the world. Now, with the virus
continuing to rage, tens of millions of students will start the year
remotely, and it has become increasingly clear that only a small
percentage of children are likely to see the inside of a school building
before the year ends.

``There's no good answer,'' Mark Henry, superintendent of the
Cypress-Fairbanks Independent School District near Houston,
\href{https://www.cfisd.net/en/board-trustees/board-meetings-agendas}{told
trustees} at a recent special meeting in which they voted to postpone
the district's hybrid reopening until September. ``If there was a good
answer, if there were an easy answer,'' he said, ``we would lay it out
for you and everybody would be happy.''

Anywhere that schools do reopen --- outside of a portion of the
Northeast where the virus is largely under control --- is likely to see
positive test results quickly, as in Indiana.

\href{https://www.nytimes.com/interactive/2020/07/31/us/coronavirus-school-reopening-risk.html}{A
New York Times analysis found} that in many districts in the Sun Belt,
at least five people infected with the coronavirus would be expected to
arrive at a school of about 500 students and staff members during the
first week if it reopened today.

To deal with that likelihood, many schools and some states have enacted
contact tracing and quarantine protocols, with differing thresholds at
which they would close classrooms or buildings.

Because of the low infection rate locally, New York City, the largest
district in the country,
\href{https://www.nytimes.com/2020/07/08/nyregion/nyc-schools-reopening-plan.html}{plans
to reopen schools on a hybrid model on Sept. 10}, with students
attending in-person classes one to three days a week. Yet even there,
the system might have to quickly close if the citywide infection rate
ticks up even modestly.

On Friday, Mayor Bill de Blasio laid out a
\href{https://www.nytimes.com/2020/07/30/world/coronavirus-covid-19.html\#link-177c5cda}{plan
for responding to positive cases} that would mean many of the city's
1,800 public schools would most likely have individual classrooms or
even entire buildings closed at certain points.

\includegraphics{https://static01.nyt.com/images/2020/08/01/us/01virus-schools02/merlin_175171986_2619d0ab-c908-4b60-9244-75cf7b71442a-articleLarge.jpg?quality=75\&auto=webp\&disable=upscale}

One or two confirmed cases in a single classroom would require those
classes to close for 14 days, with all students and staff members
ordered to quarantine. The rest of the school would continue to operate,
but if two or more people in different classrooms in the same school
tested positive, the entire building would close for an investigation,
and might not reopen for two weeks depending on the results.

In California, where schools in two-thirds of the state have been barred
from reopening in person for now, state guidelines call for a school to
close for at least 14 days if more than 5 percent of its students,
faculty and staff test positive over a two-week period.

\href{https://www.nytimes.com/news-event/coronavirus?action=click\&pgtype=Article\&state=default\&region=MAIN_CONTENT_3\&context=storylines_faq}{}

\hypertarget{the-coronavirus-outbreak-}{%
\subsubsection{The Coronavirus Outbreak
›}\label{the-coronavirus-outbreak-}}

\hypertarget{frequently-asked-questions}{%
\paragraph{Frequently Asked
Questions}\label{frequently-asked-questions}}

Updated August 4, 2020

\begin{itemize}
\item ~
  \hypertarget{i-have-antibodies-am-i-now-immune}{%
  \paragraph{I have antibodies. Am I now
  immune?}\label{i-have-antibodies-am-i-now-immune}}

  \begin{itemize}
  \tightlist
  \item
    As of right
    now,\href{https://www.nytimes.com/2020/07/22/health/covid-antibodies-herd-immunity.html?action=click\&pgtype=Article\&state=default\&region=MAIN_CONTENT_3\&context=storylines_faq}{that
    seems likely, for at least several months.} There have been
    frightening accounts of people suffering what seems to be a second
    bout of Covid-19. But experts say these patients may have a
    drawn-out course of infection, with the virus taking a slow toll
    weeks to months after initial exposure. People infected with the
    coronavirus typically
    \href{https://www.nature.com/articles/s41586-020-2456-9}{produce}
    immune molecules called antibodies, which are
    \href{https://www.nytimes.com/2020/05/07/health/coronavirus-antibody-prevalence.html?action=click\&pgtype=Article\&state=default\&region=MAIN_CONTENT_3\&context=storylines_faq}{protective
    proteins made in response to an
    infection}\href{https://www.nytimes.com/2020/05/07/health/coronavirus-antibody-prevalence.html?action=click\&pgtype=Article\&state=default\&region=MAIN_CONTENT_3\&context=storylines_faq}{.
    These antibodies may} last in the body
    \href{https://www.nature.com/articles/s41591-020-0965-6}{only two to
    three months}, which may seem worrisome, but that's perfectly normal
    after an acute infection subsides, said Dr. Michael Mina, an
    immunologist at Harvard University. It may be possible to get the
    coronavirus again, but it's highly unlikely that it would be
    possible in a short window of time from initial infection or make
    people sicker the second time.
  \end{itemize}
\item ~
  \hypertarget{im-a-small-business-owner-can-i-get-relief}{%
  \paragraph{I'm a small-business owner. Can I get
  relief?}\label{im-a-small-business-owner-can-i-get-relief}}

  \begin{itemize}
  \tightlist
  \item
    The
    \href{https://www.nytimes.com/article/small-business-loans-stimulus-grants-freelancers-coronavirus.html?action=click\&pgtype=Article\&state=default\&region=MAIN_CONTENT_3\&context=storylines_faq}{stimulus
    bills enacted in March} offer help for the millions of American
    small businesses. Those eligible for aid are businesses and
    nonprofit organizations with fewer than 500 workers, including sole
    proprietorships, independent contractors and freelancers. Some
    larger companies in some industries are also eligible. The help
    being offered, which is being managed by the Small Business
    Administration, includes the Paycheck Protection Program and the
    Economic Injury Disaster Loan program. But lots of folks have
    \href{https://www.nytimes.com/interactive/2020/05/07/business/small-business-loans-coronavirus.html?action=click\&pgtype=Article\&state=default\&region=MAIN_CONTENT_3\&context=storylines_faq}{not
    yet seen payouts.} Even those who have received help are confused:
    The rules are draconian, and some are stuck sitting on
    \href{https://www.nytimes.com/2020/05/02/business/economy/loans-coronavirus-small-business.html?action=click\&pgtype=Article\&state=default\&region=MAIN_CONTENT_3\&context=storylines_faq}{money
    they don't know how to use.} Many small-business owners are getting
    less than they expected or
    \href{https://www.nytimes.com/2020/06/10/business/Small-business-loans-ppp.html?action=click\&pgtype=Article\&state=default\&region=MAIN_CONTENT_3\&context=storylines_faq}{not
    hearing anything at all.}
  \end{itemize}
\item ~
  \hypertarget{what-are-my-rights-if-i-am-worried-about-going-back-to-work}{%
  \paragraph{What are my rights if I am worried about going back to
  work?}\label{what-are-my-rights-if-i-am-worried-about-going-back-to-work}}

  \begin{itemize}
  \tightlist
  \item
    Employers have to provide
    \href{https://www.osha.gov/SLTC/covid-19/standards.html}{a safe
    workplace} with policies that protect everyone equally.
    \href{https://www.nytimes.com/article/coronavirus-money-unemployment.html?action=click\&pgtype=Article\&state=default\&region=MAIN_CONTENT_3\&context=storylines_faq}{And
    if one of your co-workers tests positive for the coronavirus, the
    C.D.C.} has said that
    \href{https://www.cdc.gov/coronavirus/2019-ncov/community/guidance-business-response.html}{employers
    should tell their employees} -\/- without giving you the sick
    employee's name -\/- that they may have been exposed to the virus.
  \end{itemize}
\item ~
  \hypertarget{should-i-refinance-my-mortgage}{%
  \paragraph{Should I refinance my
  mortgage?}\label{should-i-refinance-my-mortgage}}

  \begin{itemize}
  \tightlist
  \item
    \href{https://www.nytimes.com/article/coronavirus-money-unemployment.html?action=click\&pgtype=Article\&state=default\&region=MAIN_CONTENT_3\&context=storylines_faq}{It
    could be a good idea,} because mortgage rates have
    \href{https://www.nytimes.com/2020/07/16/business/mortgage-rates-below-3-percent.html?action=click\&pgtype=Article\&state=default\&region=MAIN_CONTENT_3\&context=storylines_faq}{never
    been lower.} Refinancing requests have pushed mortgage applications
    to some of the highest levels since 2008, so be prepared to get in
    line. But defaults are also up, so if you're thinking about buying a
    home, be aware that some lenders have tightened their standards.
  \end{itemize}
\item ~
  \hypertarget{what-is-school-going-to-look-like-in-september}{%
  \paragraph{What is school going to look like in
  September?}\label{what-is-school-going-to-look-like-in-september}}

  \begin{itemize}
  \tightlist
  \item
    It is unlikely that many schools will return to a normal schedule
    this fall, requiring the grind of
    \href{https://www.nytimes.com/2020/06/05/us/coronavirus-education-lost-learning.html?action=click\&pgtype=Article\&state=default\&region=MAIN_CONTENT_3\&context=storylines_faq}{online
    learning},
    \href{https://www.nytimes.com/2020/05/29/us/coronavirus-child-care-centers.html?action=click\&pgtype=Article\&state=default\&region=MAIN_CONTENT_3\&context=storylines_faq}{makeshift
    child care} and
    \href{https://www.nytimes.com/2020/06/03/business/economy/coronavirus-working-women.html?action=click\&pgtype=Article\&state=default\&region=MAIN_CONTENT_3\&context=storylines_faq}{stunted
    workdays} to continue. California's two largest public school
    districts --- Los Angeles and San Diego --- said on July 13, that
    \href{https://www.nytimes.com/2020/07/13/us/lausd-san-diego-school-reopening.html?action=click\&pgtype=Article\&state=default\&region=MAIN_CONTENT_3\&context=storylines_faq}{instruction
    will be remote-only in the fall}, citing concerns that surging
    coronavirus infections in their areas pose too dire a risk for
    students and teachers. Together, the two districts enroll some
    825,000 students. They are the largest in the country so far to
    abandon plans for even a partial physical return to classrooms when
    they reopen in August. For other districts, the solution won't be an
    all-or-nothing approach.
    \href{https://bioethics.jhu.edu/research-and-outreach/projects/eschool-initiative/school-policy-tracker/}{Many
    systems}, including the nation's largest, New York City, are
    devising
    \href{https://www.nytimes.com/2020/06/26/us/coronavirus-schools-reopen-fall.html?action=click\&pgtype=Article\&state=default\&region=MAIN_CONTENT_3\&context=storylines_faq}{hybrid
    plans} that involve spending some days in classrooms and other days
    online. There's no national policy on this yet, so check with your
    municipal school system regularly to see what is happening in your
    community.
  \end{itemize}
\end{itemize}

Chicago, the nation's third-largest school district, has proposed a
hybrid system for reopening that would put students into 15-member pods
that can be quarantined if one member tests positive. School buildings
should close if the city averages more than 400 new cases a week or 200
cases a day, the plan states, with other worrying factors like low
hospital capacity or a sudden spike in cases taken into account.

In Indiana, where the middle school student tested positive on Thursday
in Greenfield, an Indianapolis suburb of 23,000 people, the virus began
to spike in mid-June, and the caseload has remained relatively high.
This week, Indianapolis opted to start the school year online.

The Greenfield-Central Community School Corporation, with eight schools
and 4,400 students, gave families
\href{https://www.gcsc.k12.in.us/wp-content/uploads/2020/07/Greenfield-Central-School-Opening-Procedures-Final-7-8-2020.pdf}{the
option of in-person or remote learning}. At Greenfield Central Junior
High School, which the student with the positive test attends, about 15
percent of the 700 enrolled students opted for remote learning, said Mr.
Olin, the superintendent.

``It was overwhelming that our families wanted us to return,'' he said,
adding that families needed to be responsible and not send students to
school if they were displaying symptoms or awaiting test results.
Students are also required to wear masks except when they are eating or
for physical education outside, he said --- and as far as he knew, the
student who tested positive was doing so.

Anyone who was within six feet of the student for more than 15 minutes
on Thursday was instructed to isolate themselves for two weeks, Mr. Olin
said. He would not give a specific number of people who were affected at
the school, but he said no teachers or staff members were identified as
close contacts, and therefore none have been told to quarantine.

``It really doesn't change my opinion about whether we should start or
not,'' Mr. Olin said. ``If we get down the road and realize that we need
to make some adjustments, we're not opposed to that.''

He said that the district did not have a specific threshold for when it
would close a school, but that it would likely do so if absences reached
20 percent. The state has not provided specific guidance to schools on
when they should shut their doors, he said.

Some teachers in the district said the positive case on the first day
confirmed their fears about returning.

``I most definitely felt like we were not ready,'' said Russell Wiley, a
history teacher at nearby Greenfield-Central High School. ``Really, our
whole state's not ready. We don't have the virus under control. It's
just kind of like pretending like it's not there.''

One father whose daughter goes to the middle school with the positive
case said he felt conflicted about his three children attending classes
in person. Few people in the community are wearing masks, said the
father, who asked not to be named because he worried that his family
would face backlash.

``I have all these concerns,'' the father said. But he has to commute at
least an hour to work every day, so remote learning was not a good
option for his family.

``It's just a mess,'' he said. ``I don't know what the answers are.''

Advertisement

\protect\hyperlink{after-bottom}{Continue reading the main story}

\hypertarget{site-index}{%
\subsection{Site Index}\label{site-index}}

\hypertarget{site-information-navigation}{%
\subsection{Site Information
Navigation}\label{site-information-navigation}}

\begin{itemize}
\tightlist
\item
  \href{https://help.nytimes.com/hc/en-us/articles/115014792127-Copyright-notice}{©~2020~The
  New York Times Company}
\end{itemize}

\begin{itemize}
\tightlist
\item
  \href{https://www.nytco.com/}{NYTCo}
\item
  \href{https://help.nytimes.com/hc/en-us/articles/115015385887-Contact-Us}{Contact
  Us}
\item
  \href{https://www.nytco.com/careers/}{Work with us}
\item
  \href{https://nytmediakit.com/}{Advertise}
\item
  \href{http://www.tbrandstudio.com/}{T Brand Studio}
\item
  \href{https://www.nytimes.com/privacy/cookie-policy\#how-do-i-manage-trackers}{Your
  Ad Choices}
\item
  \href{https://www.nytimes.com/privacy}{Privacy}
\item
  \href{https://help.nytimes.com/hc/en-us/articles/115014893428-Terms-of-service}{Terms
  of Service}
\item
  \href{https://help.nytimes.com/hc/en-us/articles/115014893968-Terms-of-sale}{Terms
  of Sale}
\item
  \href{https://spiderbites.nytimes.com}{Site Map}
\item
  \href{https://help.nytimes.com/hc/en-us}{Help}
\item
  \href{https://www.nytimes.com/subscription?campaignId=37WXW}{Subscriptions}
\end{itemize}
