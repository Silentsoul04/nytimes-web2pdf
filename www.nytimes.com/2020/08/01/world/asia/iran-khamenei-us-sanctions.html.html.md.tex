Sections

SEARCH

\protect\hyperlink{site-content}{Skip to
content}\protect\hyperlink{site-index}{Skip to site index}

\href{https://www.nytimes.com/section/world/asia}{Asia Pacific}

\href{https://myaccount.nytimes.com/auth/login?response_type=cookie\&client_id=vi}{}

\href{https://www.nytimes.com/section/todayspaper}{Today's Paper}

\href{/section/world/asia}{Asia Pacific}\textbar{}Iran Will Expand
Nuclear Program and Won't Talk to U.S., Ayatollah Says

\url{https://nyti.ms/33e9X54}

\begin{itemize}
\item
\item
\item
\item
\item
\end{itemize}

Advertisement

\protect\hyperlink{after-top}{Continue reading the main story}

Supported by

\protect\hyperlink{after-sponsor}{Continue reading the main story}

\hypertarget{iran-will-expand-nuclear-program-and-wont-talk-to-us-ayatollah-says}{%
\section{Iran Will Expand Nuclear Program and Won't Talk to U.S.,
Ayatollah
Says}\label{iran-will-expand-nuclear-program-and-wont-talk-to-us-ayatollah-says}}

In a televised speech, Ayatollah Ali Khamenei, Iran's supreme leader,
said that negotiating with Washington over his country's nuclear program
would only help President Trump get re-elected.

\includegraphics{https://static01.nyt.com/images/2020/08/01/world/01iran01/01iran01-videoSixteenByNineJumbo1600.jpg}

By \href{https://www.nytimes.com/by/farnaz-fassihi}{Farnaz Fassihi}

\begin{itemize}
\item
  Aug. 1, 2020
\item
  \begin{itemize}
  \item
  \item
  \item
  \item
  \item
  \end{itemize}
\end{itemize}

Iran's supreme leader, Ayatollah Ali Khamenei, has said in a televised
address that Iran will expand its nuclear program and will not negotiate
with the United States, doubling down on his defiance of the Trump
administration's
\href{https://www.nytimes.com/2019/06/14/us/politics/us-iran.html}{``maximum
pressure'' policy}.

In a Friday speech for the Eid al-Adha holiday, Ayatollah Khamenei said
that entering talks with Washington over Iran's nuclear program,
\href{https://www.nytimes.com/2020/06/05/world/middleeast/trump-iran-nuclear.html}{as
President Trump has urged Tehran to do}, would only improve Mr. Trump's
chances of being re-elected in November. That, the ayatollah said, was
Mr. Trump's reason for suggesting such talks in the first place.

``He is going to benefit from negotiations,'' Ayatollah Khamenei said.
``This old man who is in charge in America apparently used negotiations
with North Korea as propaganda,'' he added --- a reference to
\href{https://www.nytimes.com/2020/04/19/world/asia/north-korea-denies-nice-note-trump.html}{Mr.
Trump's high-profile nuclear diplomacy} on another front, which to date
has been mostly fruitless.

Ayatollah Khamenei also said that Iran would maintain its close
alliances with
\href{https://www.nytimes.com/2016/11/20/world/middleeast/iran-saudi-proxy-war.html}{militia
groups in the region that it uses as proxies}, defying another demand
from the Trump administration.

The Iranian leader was not the first to connect the possibility of talks
with the United States to the presidential election. Last month, Mr.
Trump said on Twitter that Iran could make a better deal if it did so
before November. ``Don't wait until after U.S. Election to make the Big
deal,''
\href{https://twitter.com/realDonaldTrump/status/1268774841810911232}{he
wrote}. ``I'm going to win. You'll make a better deal now!''

The United States has continued to tighten sanctions on Iran over its
nuclear program, which have had a crippling effect on the Middle Eastern
country's economy. On Thursday, Secretary of State Mike Pompeo said that
the State Department would expand the sanctions to cover 22 materials
believed to be used in Iran's nuclear, military and ballistic missile
programs.

\includegraphics{https://static01.nyt.com/images/2020/08/01/world/01iran02/merlin_172491219_465d493c-0556-4f0a-a7c0-c330b6ed06dd-articleLarge.jpg?quality=75\&auto=webp\&disable=upscale}

Ayatollah Khamenei said that Iran would not try to negotiate its way out
of the sanctions and that it would be better off relying on its own
industrial development. He said the Americans were targeting his
country's economy in the hope that Iranians would rise up against their
government, which the ayatollah dismissed as ``pipe dreams.''

Mr. Khamenei said that developing the nuclear program was an absolute
necessity for Iran's future. He dismissed the 2015 nuclear deal between
Iran and several world powers, which Mr. Trump
\href{https://www.nytimes.com/2018/05/08/world/middleeast/trump-iran-nuclear-deal.html}{abandoned
in 2018}, as ``very damaging,'' saying that Iran had suffered economic
setbacks because of it.

Iran has insisted that its nuclear program is meant exclusively for
peaceful purposes, but the United States and other countries believe it
is pursuing the capacity to build a nuclear weapon.

The Iranian foreign minister, Javad Zarif, who was in charge of the
negotiations for Iran, said as recently as last month in Parliament that
the negotiating team had Ayatollah Khamenei's full support and blessing
to reach a deal.

The ayatollah, who recently directed his closest economic advisers to
\href{https://www.nytimes.com/2020/07/11/world/asia/china-iran-trade-military-deal.html}{cement
a 25-year military and economic partnership with China}, said in his
speech that European countries involved in the nuclear deal were
unreliable, and that
\href{https://www.nytimes.com/2020/01/15/world/europe/europe-iran-nuclear-deal.html}{their
attempts to salvage the pact} --- such as creating a secure financial
channel so that Iran could maintain a limited amount of trade --- were
``useless games.''

Some Iranian officials and analysts have said that Iran's strategy was
to wait out the remainder of Mr. Trump's term in hopes of a Democratic
victory that could revive the deal, which was reached under President
Barack Obama.

Image

President Trump in 2018 after signing the proclamation to withdraw the
United States from the Iran nuclear deal.Credit...Doug Mills/The New
York Times

``Khamenei has always believed that accommodating to one U.S. demand
would bring about another demand and another,'' said Sina Azodi, a
nonresident fellow at the Atlantic Council in Washington. ``For him,
every solution would bring about another problem.''

But analysts, entrepreneurs and businessmen inside Iran have warned that
the economy risks collapse if the current situation continues.

Since the United States pulled out of the nuclear deal in May 2018,
Iran's currency has dropped sharply and inflation has surged. The
government said it faced a budget deficit of nearly 30 percent this
fiscal year. Oil sales have plummeted from 2.5 million barrels a day to
about 300,000, nearly eliminating Iran from the global crude oil market.

Advertisement

\protect\hyperlink{after-bottom}{Continue reading the main story}

\hypertarget{site-index}{%
\subsection{Site Index}\label{site-index}}

\hypertarget{site-information-navigation}{%
\subsection{Site Information
Navigation}\label{site-information-navigation}}

\begin{itemize}
\tightlist
\item
  \href{https://help.nytimes.com/hc/en-us/articles/115014792127-Copyright-notice}{©~2020~The
  New York Times Company}
\end{itemize}

\begin{itemize}
\tightlist
\item
  \href{https://www.nytco.com/}{NYTCo}
\item
  \href{https://help.nytimes.com/hc/en-us/articles/115015385887-Contact-Us}{Contact
  Us}
\item
  \href{https://www.nytco.com/careers/}{Work with us}
\item
  \href{https://nytmediakit.com/}{Advertise}
\item
  \href{http://www.tbrandstudio.com/}{T Brand Studio}
\item
  \href{https://www.nytimes.com/privacy/cookie-policy\#how-do-i-manage-trackers}{Your
  Ad Choices}
\item
  \href{https://www.nytimes.com/privacy}{Privacy}
\item
  \href{https://help.nytimes.com/hc/en-us/articles/115014893428-Terms-of-service}{Terms
  of Service}
\item
  \href{https://help.nytimes.com/hc/en-us/articles/115014893968-Terms-of-sale}{Terms
  of Sale}
\item
  \href{https://spiderbites.nytimes.com}{Site Map}
\item
  \href{https://help.nytimes.com/hc/en-us}{Help}
\item
  \href{https://www.nytimes.com/subscription?campaignId=37WXW}{Subscriptions}
\end{itemize}
