Sections

SEARCH

\protect\hyperlink{site-content}{Skip to
content}\protect\hyperlink{site-index}{Skip to site index}

\href{https://www.nytimes.com/section/world/middleeast}{Middle East}

\href{https://myaccount.nytimes.com/auth/login?response_type=cookie\&client_id=vi}{}

\href{https://www.nytimes.com/section/todayspaper}{Today's Paper}

\href{/section/world/middleeast}{Middle East}\textbar{}Iran Says It
Detained Leader of California-Based Exile Group

\url{https://nyti.ms/31fBr86}

\begin{itemize}
\item
\item
\item
\item
\item
\end{itemize}

Advertisement

\protect\hyperlink{after-top}{Continue reading the main story}

Supported by

\protect\hyperlink{after-sponsor}{Continue reading the main story}

\hypertarget{iran-says-it-detained-leader-of-california-based-exile-group}{%
\section{Iran Says It Detained Leader of California-Based Exile
Group}\label{iran-says-it-detained-leader-of-california-based-exile-group}}

The detained man, Jamshid Sharmahd of the Kingdom Assembly of Iran, is
accused of planning attacks on Iran, including a 2008 one that killed 14
and injured 200.

\includegraphics{https://static01.nyt.com/images/2020/08/01/world/01iran/merlin_151946559_26a9eb63-bf68-422c-a5ec-90c993481dc5-articleLarge.jpg?quality=75\&auto=webp\&disable=upscale}

By The Associated Press

\begin{itemize}
\item
  Aug. 1, 2020
\item
  \begin{itemize}
  \item
  \item
  \item
  \item
  \item
  \end{itemize}
\end{itemize}

TEHRAN --- Iran on Saturday said it had detained an Iranian-American
leader of a little-known, California-based opposition group for
allegedly planning a 2008 attack on a mosque that killed 14 people and
wounded over 200 others.

Iran's Intelligence Ministry also asserted that the detained man,
Jamshid Sharmahd of the Kingdom Assembly of Iran, planned more attacks
around the Islamic Republic amid heightened tensions between Tehran and
the United States.

Mr. Sharmahd's reported arrest comes as relations between the U.S. and
Iran remain inflamed in the wake of President Donald Trump's 2018
decision to withdraw America from the 2015 multinational nuclear deal.
In January, \href{https://apnews.com/5597ff0f046a67805cc233d5933a53ed}{a
U.S. drone strike killed a top Iranian general in Baghdad}. Iran
responded by
\href{https://apnews.com/add7a702258b4419d796aa5f48e577fc}{launching a
ballistic missile attack on U.S. soldiers in Iraq} that injured dozens.

Iran accused Mr. Sharmahd, 65, of running Tondar, or ``Thunder'' in
Farsi, a militant wing of the Kingdom Assembly of Iran. The
circumstances of his capture are unclear; the Intelligence Ministry
called it a ``complex operation,'' without elaborating. It published a
purported picture of Mr. Sharmahd, blindfolded, on its website.

Iran's intelligence minister, Mahmoud Alavi, later appeared on state TV,
saying Mr. Sharmahd had been arrested in Iran.

Requests for comment sent by email to the Kingdom Assembly of Iran,
based in Glendora, Calif., were not immediately answered and a telephone
number for the group no longer worked.

According to
\href{https://www.state.gov/outlaw-regime-a-chronicle-of-irans-destructive-activities/}{a
recent report published by the U.S. State Department}, Mr. Sharmahd had
been targeted for assassination. The U.S. State Department did not
immediately respond to a request for comment.

Iranian state television broadcast a report on Mr. Sharmahd's arrest,
linking him
\href{https://www.nytimes.com/2008/04/13/world/middleeast/13shiraz.html}{to
the 2008 bombing of the Shohada mosque in Shiraz.} It also said his
group was also behind a 2010 bombing at Ayatollah Ruhollah Khomeini's
mausoleum in Tehran that wounded several people.

The state TV report also alleged, without providing evidence, that
Tondar plotted attacks on a dam and planned to use cyanide bombs at
Tehran's annual book fair.

State TV aired footage of Mr. Sharmahd interspersed with footage from
the moment of the 2008 explosion at the Shiraz mosque. Mr. Sharmahd's
face appeared swollen and the style of the footage resembled that used
in one of the
\href{https://apnews.com/2f5e336cb7f96c2829a98a522f705855}{more than 350
coerced confessions that a rights group says the broadcaster has aired}
over the last decade.

The Intelligence Ministry has not said what charges Mr. Sharmahd will
face. Prisoners earlier accused in the same attack were sentenced to
death and executed.

The Kingdom Assembly of Iran, known in Farsi as Anjoman-e Padeshahi-e
Iran, and Tondar seek to restore Iran's monarchy, which ended when the
fatally ill Shah Mohammed Reza Pahlavi fled the country in 1979 just
before its Islamic Revolution. The California-based group's founder
disappeared in the mid-2000s.

Iranian intelligence operatives in the past have used family members and
other tricks to lure targets back to Iran or to friendly countries to be
captured. An alleged Iranian government operative who is accused of
trying to hire a hit man to kill Mr. Sharmahd disappeared in 2010 before
facing trial in California, likely having returned to Iran.

According to a 2010 U.S. diplomatic cable from London, later published
by WikiLeaks, a Voice of America commentator said that same operative
had earlier been in contact with him. The British antiterror police
later warned the commentator that he ``had been targeted by the Iranian
regime,'' the cable said.

The two cases represented ``a clear escalation in the regime's attempts
to intimidate critics outside its borders, and could have a chilling
effect on journalists, academics and others in the West who until
recently felt little physical threat from the regime,'' the cable said.

Mr. Sharmahd last appeared in an online livestream video on Dec. 29,
according to his group's website, speaking in Farsi while sitting in a
black chair in front of a black background.

``We are not only seeking the liberation of the homeland, but we are
also moving toward a special direction, and that is to be Iranian,'' Mr.
Sharmahd said at one point in the video. ``Because we have heard that
once upon a time some people were living in the region who were able to
build an empire.''

The Kingdom Assembly is overshadowed by other exiled opposition groups.
But Iran reportedly brought up the group multiple times while
negotiating the terms of the 2015 deal, in which Tehran agreed to limit
its enrichment of uranium in exchange for the lifting of economic
sanctions.

A spokesman for Iran's foreign ministry, Abbas Mousavi, reacted to the
news of Mr. Sharmahd's detention by criticizing the U.S. for allowing
him and other militant opponents to live in America.

The U.S. ``must be responsible for supporting terrorist groups which are
inside of this country and carry out and lead terrorist acts against the
Iranian people,'' state TV quoted Mr. Mousavi as saying.

A statement attributed to Tondar claimed the assassination of an Iranian
nuclear scientist in 2010 by a remote-control bomb, though the group
later said it wasn't responsible. Suspicion long has fallen on Israel
for a string of assassinations targeting scientists amid concerns about
Iran's nuclear program, which the West fears could be used to develop a
nuclear bomb. Iran has long maintained its program is for peaceful
purposes.

Advertisement

\protect\hyperlink{after-bottom}{Continue reading the main story}

\hypertarget{site-index}{%
\subsection{Site Index}\label{site-index}}

\hypertarget{site-information-navigation}{%
\subsection{Site Information
Navigation}\label{site-information-navigation}}

\begin{itemize}
\tightlist
\item
  \href{https://help.nytimes.com/hc/en-us/articles/115014792127-Copyright-notice}{©~2020~The
  New York Times Company}
\end{itemize}

\begin{itemize}
\tightlist
\item
  \href{https://www.nytco.com/}{NYTCo}
\item
  \href{https://help.nytimes.com/hc/en-us/articles/115015385887-Contact-Us}{Contact
  Us}
\item
  \href{https://www.nytco.com/careers/}{Work with us}
\item
  \href{https://nytmediakit.com/}{Advertise}
\item
  \href{http://www.tbrandstudio.com/}{T Brand Studio}
\item
  \href{https://www.nytimes.com/privacy/cookie-policy\#how-do-i-manage-trackers}{Your
  Ad Choices}
\item
  \href{https://www.nytimes.com/privacy}{Privacy}
\item
  \href{https://help.nytimes.com/hc/en-us/articles/115014893428-Terms-of-service}{Terms
  of Service}
\item
  \href{https://help.nytimes.com/hc/en-us/articles/115014893968-Terms-of-sale}{Terms
  of Sale}
\item
  \href{https://spiderbites.nytimes.com}{Site Map}
\item
  \href{https://help.nytimes.com/hc/en-us}{Help}
\item
  \href{https://www.nytimes.com/subscription?campaignId=37WXW}{Subscriptions}
\end{itemize}
