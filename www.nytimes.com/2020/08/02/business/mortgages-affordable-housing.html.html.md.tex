Sections

SEARCH

\protect\hyperlink{site-content}{Skip to
content}\protect\hyperlink{site-index}{Skip to site index}

\href{https://www.nytimes.com/section/business}{Business}

\href{https://myaccount.nytimes.com/auth/login?response_type=cookie\&client_id=vi}{}

\href{https://www.nytimes.com/section/todayspaper}{Today's Paper}

\href{/section/business}{Business}\textbar{}Where a Little Mortgage Goes
a Long Way

\url{https://nyti.ms/318WEQZ}

\begin{itemize}
\item
\item
\item
\item
\item
\end{itemize}

Advertisement

\protect\hyperlink{after-top}{Continue reading the main story}

Supported by

\protect\hyperlink{after-sponsor}{Continue reading the main story}

\hypertarget{where-a-little-mortgage-goes-a-long-way}{%
\section{Where a Little Mortgage Goes a Long
Way}\label{where-a-little-mortgage-goes-a-long-way}}

Affordable homes can be hard to buy because lenders don't make much
money on small loans. But programs to encourage homeownership can help
buyers build wealth.

\includegraphics{https://static01.nyt.com/images/2020/07/31/business/31smallmortgages1/merlin_174909417_6801b90c-d375-4a5d-b3af-ca9f1e0da20e-articleLarge.jpg?quality=75\&auto=webp\&disable=upscale}

\href{https://www.nytimes.com/by/matthew-goldstein}{\includegraphics{https://static01.nyt.com/images/2018/11/06/multimedia/author-matthew-goldstein/author-matthew-goldstein-thumbLarge.png}}

By \href{https://www.nytimes.com/by/matthew-goldstein}{Matthew
Goldstein}

\begin{itemize}
\item
  Aug. 2, 2020
\item
  \begin{itemize}
  \item
  \item
  \item
  \item
  \item
  \end{itemize}
\end{itemize}

The Shawnee neighborhood in Louisville, Ky., is a paradox: The houses
are affordable, but they can be difficult to buy. The prices are so low
that most banks and lenders will not bother writing mortgages for them.

That was the problem facing Christopher T. Smith when he moved back to
Shawnee, a historically Black neighborhood along the Ohio River, where
his mother still lives in the house where he grew up.

He and his wife, Gloria, did not expect to buy in an area where houses
are more often scooped up by speculators who can pay in cash. ``We were
just looking to rent,'' said Mr. Smith, who works as a hospital
housekeeper and a part-time gardener.

But then the broker who was showing them rentals mentioned that a local
credit union had begun offering so-called small-dollar mortgages ---
loans of less than \$100,000 that are not lucrative enough for most
lenders to make.

The Smiths qualified and closed on their \$86,000 home in October.
``There's nothing like owning your own home,'' Mr. Smith said. ``If I
want to paint it, I can.''

Small-dollar mortgages open a path to homeownership for those who
otherwise would be shut out, particularly Black and Hispanic borrowers.
But they are not popular among lenders. Last year, mortgages for
\$100,000 or less accounted for just 10 percent of loans used to buy a
single-family home or a condominium in the United States, according to
Attom Data, a housing data company. That share is down from 17 percent
in 2014.

\includegraphics{https://static01.nyt.com/images/2020/08/03/business/00JPsmallmortgages2-print/00smallmortgages2-articleLarge.jpg?quality=75\&auto=webp\&disable=upscale}

A new program in Louisville --- the MicroMortgage Marketplace project,
which officially
\href{https://www.newswire.com/news/micromortgage-marketplace-launches-in-support-of-small-dollar-home-21180593}{started
two weeks ago} --- is trying to help other potential buyers like the
Smiths. Its goal is to become a demonstration project that can be
replicated in other cities where modest homes are plentiful but the
mortgages to buy them are in short supply.

Tamika Jackson, the real estate agent who helped the Smiths buy their
home with a small-dollar mortgage, is already lining up potential
customers for the new program, which is being coordinated by the Urban
Institute, a Washington think tank.

``The banks don't think it is worth their while to make these loans,''
she said, adding that there are ``a lot of people who are paying rent
who'd like to be homeowners.''

Homeownership is a crucial part of a family's ability to
\href{https://www.nytimes.com/2020/06/09/your-money/race-income-equality.html}{build
wealth}: A home is the largest asset for most American families, and the
value it can gain over decades can be tapped during retirement or left
to the next generation. But the share of Black households that own homes
has only inched upward over the last 50 years, and the continuing
homeownership gap is one of the
\href{https://www.brookings.edu/blog/up-front/2020/02/27/examining-the-black-white-wealth-gap/}{main
reasons} the net worth of white households far exceeds that of Black
families.

``We are trying to help people who have the hardest time getting access
to homeownership,'' said Alanna McCargo, vice president for housing
finance policy at the Urban Institute. ``There hasn't been any kind of
mandate from the federal government for banks to do small-dollar
lending.''

Similar programs have been set up or explored elsewhere. In Detroit,
where there were just under 1,700 mortgages in the entire city last
year, about half were small-dollar mortgages, according to Attom Data.
\href{https://www.nytimes.com/2017/11/04/business/detroit-housing.html}{Some
of the efforts to spur lending}there have come from a variety of
programs aimed at providing low-cost financing for first-time home
buyers and even grants to fix up dilapidated homes.

And in November, federal bank regulators and the Federal Reserve Bank of
Chicago sponsored a forum in South Bend, Ind., to explore ways to
\href{https://www.occ.gov/news-events/events/files/event-cra-small-dollar-mortgage-strategies-11192019.html}{spur
more small-dollar mortgage} lending under the Community Reinvestment
Act.

The MicroMortgage Marketplace program --- still in its infancy, with
just three applicants, none of whom have yet bought a home --- has been
in the works since last year. But it is taking place largely in a city
where issues of racial equality have been front and center after the
death of
\href{https://www.nytimes.com/article/breonna-taylor-police.html}{Breonna
Taylor}, a 26-year-old Black emergency room technician in Louisville who
was shot and killed by the police in March. Ms. Taylor's killing has
been invoked by
\href{https://www.nytimes.com/news-event/george-floyd-protests-minneapolis-new-york-los-angeles}{protesters
around the country} who have gathered to demonstrate against police
brutality and demand broader social changes.

Image

Low-priced homes in neighborhoods like Shawnee can be scooped by
speculators who can pay in cash, then rent them out.Credit...Michael
Blackshire for The New York Times

Image

Mr. Smith was referred to a local lender, Park Community Credit Union,
by the agent who had been showing him rentals.Credit...Michael
Blackshire for The New York Times

Ms. McCargo, of the Urban Institute, said she did not believe that banks
were intentionally avoiding making mortgages to Black residents. But she
said the communities hit hardest were ``historically redlined
communities'' with high concentrations of Black or Hispanic borrowers.

Ms. McCargo was referring to the
\href{https://www.nytimes.com/2017/08/24/upshot/how-redlinings-racist-effects-lasted-for-decades.html}{illegal
and notorious practice} in which banks drew lines around largely Black
communities to denote places where they would not make mortgages. Today,
banks may not make loans in poorer communities because small-dollar
mortgages require the same research as larger mortgages.

``The bottom line is the economics often don't pencil out,'' said Steve
O'Connor, a senior vice president with the Mortgage Bankers Association
who focuses on affordable housing issues. ``There are risks involved.
There are compliance risk and market risk.'' He added, the ``fixed cost
often exceeds the revenue on the loan.''

The result is a market dynamic that perpetuates renting and promotes
risky behaviors by those desperate to buy.

When borrowers cannot buy,
\href{https://www.nytimes.com/2019/03/25/business/fox-news-clayton-morris-indianapolis.html}{speculators}
--- often flush with cash --- can easily buy up modestly priced homes on
the cheap and then rent them out.
\href{https://www.nytimes.com/2017/11/04/business/detroit-housing.html}{Mortgage
deserts} also give rise to
\href{https://www.nytimes.com/series/the-housing-trap}{predatory housing
practices}, in which would-be home buyers are lured into rent-to-own
arrangements or contract-for-deed sales, where evictions are common.

In Louisville, a city of 625,000, the overall number of small-dollar
loans last year was somewhat higher than the national average. Roughly
18 percent of the 9,800 mortgages made in the city were for \$100,000 or
less, according to Attom Data. Those mortgages tended to be made by
local organizations. The Kentucky Housing Corporation, a state-sponsored
provider of affordable housing, made the most small-dollar loans, with
224. The next-biggest lender was the Republic Bank \& Trust Company, a
Louisville-based bank, with 93.

Image

A block of the Shawnee neighborhood in Louisville. A new program in the
city is trying to help people obtain home mortgages for amounts less
than \$100,000, which most banks won't do.Credit...Michael Blackshire
for The New York Times

Park Community Credit Union, which made Mr. Smith's mortgage, wrote 35;
JPMorgan Chase --- the nation's biggest bank --- made 29.

The pilot project --- which the Urban Institute is coordinating with the
Homeownership Council of America and Fahe, a regional community
development financial institution --- is being funded with a \$300,000
grant from Access Ventures, an investment firm, and additional financial
backing from Fahe. Organizers hope to finance as many as 50 mortgages in
Louisville and communities on the other side of the Ohio River in
southern Indiana.

The program will mainly serve first-time home buyers with credits scores
as low as 640 --- which most lenders consider a below-average rating.
Buyers, who must be employed full time, can borrow up to \$100,000 and
can finance the entire purchase price if they want, without paying for
mortgage insurance.

That flexibility comes at a price: The loans carry a 4.5 percent
interest rate. The \href{http://www.freddiemac.com/pmms/}{average rate}
on a conventional 30-year fixed mortgage is about 3 percent.

Fahe, a nonprofit organization that focuses on providing mortgages to
residents of the Appalachian region, aspires to build the pilot project
into something bigger. The organization, which is a licensed lender in
16 states, hopes the demonstration project will attract financial
support and backing from more traditional banks

``Profit is important to us, too, but mission is more important,'' said
Laura Meadows, Fahe's executive vice president for lending.
``Scalability is something we are going to look at.''

Antoinette Hines, 44, who works as a counselor for troubled teens, is
one of the first to apply for a mortgage under the pilot project. Ms.
Hines, who was married in July, is looking to buy the \$75,000 home she
has rented for the last six years. If the deal goes through, she said,
the monthly payment on her mortgage would be several hundred dollars
less than she pays in rent.

Before Ms. Jackson told her about the new loan program, Ms. Hines said,
she sought out a bank for mortgage. ``They said they won't make a loan
for that small of an amount,'' Ms. Hines said.

One challenge the project faces is finding brokers like Ms. Jackson who
are willing to work with buyers looking for modest homes. Like lenders,
brokers who work on commission have an incentive to seek more lucrative
sales.

But Ms. Jackson, who owns her firm, said the intangible rewards made it
worth the effort.

``I get fulfillment out of it,'' she said.

Advertisement

\protect\hyperlink{after-bottom}{Continue reading the main story}

\hypertarget{site-index}{%
\subsection{Site Index}\label{site-index}}

\hypertarget{site-information-navigation}{%
\subsection{Site Information
Navigation}\label{site-information-navigation}}

\begin{itemize}
\tightlist
\item
  \href{https://help.nytimes.com/hc/en-us/articles/115014792127-Copyright-notice}{©~2020~The
  New York Times Company}
\end{itemize}

\begin{itemize}
\tightlist
\item
  \href{https://www.nytco.com/}{NYTCo}
\item
  \href{https://help.nytimes.com/hc/en-us/articles/115015385887-Contact-Us}{Contact
  Us}
\item
  \href{https://www.nytco.com/careers/}{Work with us}
\item
  \href{https://nytmediakit.com/}{Advertise}
\item
  \href{http://www.tbrandstudio.com/}{T Brand Studio}
\item
  \href{https://www.nytimes.com/privacy/cookie-policy\#how-do-i-manage-trackers}{Your
  Ad Choices}
\item
  \href{https://www.nytimes.com/privacy}{Privacy}
\item
  \href{https://help.nytimes.com/hc/en-us/articles/115014893428-Terms-of-service}{Terms
  of Service}
\item
  \href{https://help.nytimes.com/hc/en-us/articles/115014893968-Terms-of-sale}{Terms
  of Sale}
\item
  \href{https://spiderbites.nytimes.com}{Site Map}
\item
  \href{https://help.nytimes.com/hc/en-us}{Help}
\item
  \href{https://www.nytimes.com/subscription?campaignId=37WXW}{Subscriptions}
\end{itemize}
