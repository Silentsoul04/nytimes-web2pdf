Sections

SEARCH

\protect\hyperlink{site-content}{Skip to
content}\protect\hyperlink{site-index}{Skip to site index}

\href{https://www.nytimes.com/section/us}{U.S.}

\href{https://myaccount.nytimes.com/auth/login?response_type=cookie\&client_id=vi}{}

\href{https://www.nytimes.com/section/todayspaper}{Today's Paper}

\href{/section/us}{U.S.}\textbar{}U.S. Identifies Some of the Mysterious
Seeds Mailed From China

\url{https://nyti.ms/3i1BsDi}

\begin{itemize}
\item
\item
\item
\item
\item
\end{itemize}

Advertisement

\protect\hyperlink{after-top}{Continue reading the main story}

Supported by

\protect\hyperlink{after-sponsor}{Continue reading the main story}

\hypertarget{us-identifies-some-of-the-mysterious-seeds-mailed-from-china}{%
\section{U.S. Identifies Some of the Mysterious Seeds Mailed From
China}\label{us-identifies-some-of-the-mysterious-seeds-mailed-from-china}}

The 14 varieties identified include common ones, such as hibiscus,
morning glory and lavender. Still, experts warned recipients not to
plant them.

\includegraphics{https://static01.nyt.com/images/2020/08/03/multimedia/03xp-seeds/merlin_175213314_648d3f4c-1f6e-4788-9e08-818f0f116ff1-articleLarge.jpg?quality=75\&auto=webp\&disable=upscale}

By \href{https://www.nytimes.com/by/allyson-waller}{Allyson Waller}

\begin{itemize}
\item
  Aug. 2, 2020
\item
  \begin{itemize}
  \item
  \item
  \item
  \item
  \item
  \end{itemize}
\end{itemize}

\href{https://www.nytimes.com/es/2020/08/03/espanol/estados-unidos/semillas-correo-china.html}{Leer
en español}

A federal agency said it had identified 14 types of plants from
\href{https://www.nytimes.com/2020/07/26/us/seeds-from-china-mail.html}{unsolicited
packages of seeds} that appeared to have been mailed from China,
revealing a ``mix of ornamental, fruit and vegetable, herb and weed
species.''

Among the plant species botanists have identified so far: cabbage,
hibiscus, lavender, mint, morning glory, mustard, rose, rosemary and
sage,
\href{https://www.aphis.usda.gov/publications/plant_health/faq-unsolicited-seeds.pdf}{according
to the U.S. Department of Agriculture Animal and Plant Health Inspection
Service.}

``This is just a subset of the samples we've collected so far,'' Osama
El-Lissy, deputy administrator for the service's plant protection and
quarantine,
\href{https://www.usda.gov/media/radio/daily-newsline/2020-07-29/actuality-unsolicited-seeds-are-several-plant-species}{said
this past week}.

Last month, a number of states reported that residents were getting
packages of seeds they did not order.

All 50 states have since issued warnings about the unsolicited packages
and the inspection service said it had been sent packets from at least
22 states.

Doyle Crenshaw of Booneville, Ark., said he had planted some of the
unsolicited seeds he got.

``I told my wife, `They don't look like any flower seed I had ever
seen,''' he said on Sunday.

Mr. Crenshaw said he had ordered blue zinnia seeds from Amazon, but when
he got the package about two months ago, it contained the blue zinnia
seeds as well as seed packets he did not order.

The package label read ``studded earrings'' and ``China,'' he said.

``It's a really pretty plant,'' he said, describing what grew from the
unsolicited seeds. ``It looks like a giant squash plant.''

A representative from Amazon could not be immediately reached on Sunday.

Mr. Crenshaw said he called the Arkansas Department of Agriculture and
officials were set to come this week to dig up the plant that grew from
the unsolicited seeds. He also plans to have them collect another
unsolicited package he received --- but has not opened --- that was
labeled to say it contained beads.

After receiving these packages, he said he and his wife will from now on
order their seeds locally.

The
\href{https://www.aphis.usda.gov/aphis/newsroom/stakeholder-info/sa_by_date/sa-2020/sa-07/seeds-china}{federal
inspection agency said} evidence indicates the packages are part of a
``brushing scam'' in which sellers send unsolicited items in hopes of
increasing sales.

Although the risk is low for some nefarious outcome, like introducing an
exotic species in the United States or some form of biological warfare,
recipients of the mailings should not plant the seeds, said Art Gover, a
plant science researcher at Penn State University.

These seeds can be troublesome because they can introduce problematic
weeds and diseases, he said.

Lisa Delissio, a professor of biology at Salem State University in
Massachusetts, said if any of the unidentified seeds turned out to be
invasive species, they could displace native plants and compete for
resources and cause harm to the environment, agriculture or human
health.

Bernd Blossey, a professor in the department of natural resources at
Cornell University in Ithaca, N.Y., said he received a few calls from
worried recipients of the seed packets.

``Obviously planting rosemary or thyme in your garden isn't something
that will endanger our environment,'' he said. ``But there may be other
things in there that have not been identified yet. Any time you gain
something unknown, my suggestion is burning them, not even throwing them
in the trash.''

Gardeners have been responsible for introducing invasive plant species
in the past, and nurturing them with a green thumb, including the
butterfly bush, Japanese knotweed and some ornamental grasses, Professor
Blossey said.

``Who knows who's behind it or what's behind it?'' he said. ``I think
there may be more to the story.''

Marie Fazio and Christina Morales contributed reporting.

Advertisement

\protect\hyperlink{after-bottom}{Continue reading the main story}

\hypertarget{site-index}{%
\subsection{Site Index}\label{site-index}}

\hypertarget{site-information-navigation}{%
\subsection{Site Information
Navigation}\label{site-information-navigation}}

\begin{itemize}
\tightlist
\item
  \href{https://help.nytimes.com/hc/en-us/articles/115014792127-Copyright-notice}{©~2020~The
  New York Times Company}
\end{itemize}

\begin{itemize}
\tightlist
\item
  \href{https://www.nytco.com/}{NYTCo}
\item
  \href{https://help.nytimes.com/hc/en-us/articles/115015385887-Contact-Us}{Contact
  Us}
\item
  \href{https://www.nytco.com/careers/}{Work with us}
\item
  \href{https://nytmediakit.com/}{Advertise}
\item
  \href{http://www.tbrandstudio.com/}{T Brand Studio}
\item
  \href{https://www.nytimes.com/privacy/cookie-policy\#how-do-i-manage-trackers}{Your
  Ad Choices}
\item
  \href{https://www.nytimes.com/privacy}{Privacy}
\item
  \href{https://help.nytimes.com/hc/en-us/articles/115014893428-Terms-of-service}{Terms
  of Service}
\item
  \href{https://help.nytimes.com/hc/en-us/articles/115014893968-Terms-of-sale}{Terms
  of Sale}
\item
  \href{https://spiderbites.nytimes.com}{Site Map}
\item
  \href{https://help.nytimes.com/hc/en-us}{Help}
\item
  \href{https://www.nytimes.com/subscription?campaignId=37WXW}{Subscriptions}
\end{itemize}
