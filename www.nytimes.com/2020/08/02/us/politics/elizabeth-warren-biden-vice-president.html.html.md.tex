Sections

SEARCH

\protect\hyperlink{site-content}{Skip to
content}\protect\hyperlink{site-index}{Skip to site index}

\href{https://www.nytimes.com/section/politics}{Politics}

\href{https://myaccount.nytimes.com/auth/login?response_type=cookie\&client_id=vi}{}

\href{https://www.nytimes.com/section/todayspaper}{Today's Paper}

\href{/section/politics}{Politics}\textbar{}Elizabeth Warren's Evolution
on Race Brought Her Here

\url{https://nyti.ms/30jHfxS}

\begin{itemize}
\item
\item
\item
\item
\item
\end{itemize}

\begin{itemize}
\item
  \href{https://www.nytimes.com/2020/07/31/us/elections/biden-vs-trump.html?action=click\&pgtype=Article\&state=default\&region=TOP_BANNER\&context=storylines_menu}{Election
  Updates}
\item
  \href{https://www.nytimes.com/article/biden-vice-president-2020.html?action=click\&pgtype=Article\&state=default\&region=TOP_BANNER\&context=storylines_menu}{Biden's
  V.P. Search}
\item
  \href{https://www.nytimes.com/interactive/2020/07/24/us/politics/trump-biden-campaign-donors.html?action=click\&pgtype=Article\&state=default\&region=TOP_BANNER\&context=storylines_menu}{Map
  of Donations}
\item
  \href{https://www.nytimes.com/interactive/2020/us/elections/delegate-count-primary-results.html?action=click\&pgtype=Article\&state=default\&region=TOP_BANNER\&context=storylines_menu}{Delegate
  Count}
\item
  \href{https://www.nytimes.com/interactive/2019/us/politics/2020-presidential-candidates.html?action=click\&pgtype=Article\&state=default\&region=TOP_BANNER\&context=storylines_menu}{The
  Candidates}
\item
  \href{https://www.nytimes.com/newsletters/politics?action=click\&pgtype=Article\&state=default\&region=TOP_BANNER\&context=storylines_menu}{Politics
  Newsletter}
\end{itemize}

Advertisement

\protect\hyperlink{after-top}{Continue reading the main story}

Supported by

\protect\hyperlink{after-sponsor}{Continue reading the main story}

\hypertarget{elizabeth-warrens-evolution-on-race-brought-her-here}{%
\section{Elizabeth Warren's Evolution on Race Brought Her
Here}\label{elizabeth-warrens-evolution-on-race-brought-her-here}}

Ms. Warren, a racially progressive politician, is one of a handful of
white women still under serious consideration to become Joe Biden's
running mate.

\includegraphics{https://static01.nyt.com/images/2020/08/03/us/politics/03warren-race1/merlin_169475661_56e5eb41-8da5-4976-93c0-02227d8ec7cd-articleLarge.jpg?quality=75\&auto=webp\&disable=upscale}

\href{https://www.nytimes.com/by/lisa-lerer}{\includegraphics{https://static01.nyt.com/images/2018/09/11/us/politics/author-lisa-lerer/lisa-lerer-headshot-thumbLarge.png}}\href{https://www.nytimes.com/by/sydney-ember}{\includegraphics{https://static01.nyt.com/images/2018/06/12/multimedia/author-sydney-ember/author-sydney-ember-thumbLarge.png}}

By \href{https://www.nytimes.com/by/lisa-lerer}{Lisa Lerer} and
\href{https://www.nytimes.com/by/sydney-ember}{Sydney Ember}

\begin{itemize}
\item
  Published Aug. 2, 2020Updated Aug. 3, 2020, 12:32 a.m. ET
\item
  \begin{itemize}
  \item
  \item
  \item
  \item
  \item
  \end{itemize}
\end{itemize}

When Liz Herring arrived at George Washington University as a freshman
in 1966, she entered a capital city in the throes of the civil rights
movement. Just three years after a quarter-million people had crowded
the National Mall to hear the Rev. Dr. Martin Luther King Jr., Congress
was debating civil rights legislation as violent protests continued
across the country.

Yet, little of that political unrest reached Kappa Alpha Theta, the
all-white sorority the future senator from Massachusetts would soon
pledge. Yearbook photos show Ms. Herring in a group of smiling young
women, corsages pinned to their white dresses, hair perfectly flipped up
at the ends.

The young Ms. Herring, who fought her mother to attend college away from
her conservative hometown, went to rush parties and meetings, charity
events and the annual ``goat show,'' a sketch comedy performance for all
of the Greek organizations, where a master of ceremonies defended
sororities as a ``unifying force'' for the school. No Black woman had
ever been offered acceptance into any of the sororities on campus.

More than half a century later, the young college coed, who now goes by
Senator Elizabeth Warren, led the charge in Congress to require the
Pentagon to rename bases that honor Confederate military leaders. She
spent much of her time on the campaign trail during the Democratic
primary campaign talking about the racial wealth gap and systemic
discrimination, and
\href{https://www.nytimes.com/2019/06/10/us/politics/elizabeth-warren-2020-policies-platform.html}{proposing
plans} on housing, maternal mortality, child care and other issues,
which had an explicit focus on racial justice.

She has emerged, according to activists and organizers, as one of the
most racially progressive white politicians in the country.

She's also one of a handful of white women still under serious
consideration to become
\href{https://www.nytimes.com/interactive/2020/us/elections/joe-biden.html}{Joseph
R. Biden Jr.}'s running mate, at a time when some Democratic leaders are
pushing for more racial representation on their ticket.

``She did the work and continues to do the work,'' said Angela Peoples,
the director of Black Womxn For, who recently co-wrote
\href{https://www.washingtonpost.com/outlook/2020/07/15/biden-black-women-warren-running-mate/}{an
op-ed} urging Mr. Biden to select Ms. Warren as his running mate over
several Black women. ``That's the model that I would love to see other
Democrats follow.''

In many ways, Ms. Warren's evolution on issues of race is a preview of
the journey many white liberals are on now. In the past decade,
Democrats have been moving steadily to the left on racial equality and
criminal justice. That shift became a leap after the death of George
Floyd in police custody in late May, with majorities of Democratic
voters now expressing support for the Black Lives Matter movement.

Ms. Warren wasn't always outspoken on the specific cause of racial
justice. For much of her academic career and even after she entered
politics, she remained most vocal on the central cause of her career,
economic inequality as it affects all Americans. Her most politically
defining misstep was over an issue of race, when she took a DNA test to
demonstrate her purported Native American heritage and a backlash
followed.

Allies say her awakening traces the arc of much of her life, with the
beginnings of a worldview coalescing when she was a student at Rutgers
Law School in Newark, where racial unrest several years earlier had
turned the institution into a hub of civil rights activism. As a law
professor, her work on bankruptcy illuminated the systemic barriers
Black Americans face and helped convince Ms. Warren that race was
intimately intertwined with inequality. As a presidential candidate, she
made tackling racial disparities a central part of her mission.

Some Black strategists and officials attribute Ms. Warren's changing
focus to political opportunism, saying she started speaking about racial
justice only as she began expanding her national profile. Her embrace of
issues of race and equality during the primary campaign failed to
resonate with many Black voters, even as prominent racial justice
activists showered her with support.

Ms. Warren declined to comment for this article.

``Her evolution is great, but her evolution is one of convenience,''
said Bakari Sellers, a former South Carolina State legislator and a
supporter of Senator Kamala Harris, a rival for the vice-presidential
nomination. ``A lot of people find stuff when you're running for
president.''

Yet as Democrats cast their eyes toward winning back the White House,
some activists see Ms. Warren's journey --- from a segregated high
school in Oklahoma City to racial justice fighter --- as a political
template in a country that is shifting rapidly on issues of racial
equity. In her life, there is a way to understand the journey of some
other white Democrats, who may find their views on race shifting far
from those they learned in their youth.

\includegraphics{https://static01.nyt.com/images/2020/07/31/us/politics/00Warren-race2/merlin_156446115_bb7787d1-e959-4ba7-864f-12d31614c1e0-articleLarge.jpg?quality=75\&auto=webp\&disable=upscale}

\hypertarget{not-one-person-of-color-anywhere}{%
\subsection{Not `one person of color'
anywhere}\label{not-one-person-of-color-anywhere}}

As a student at Northwest Classen High School, Ms. Warren's world was an
overwhelmingly white one. Located in an affluent area of Oklahoma City,
the school was an embodiment of the kind of segregation created by
decades of discriminatory housing practices.

\hypertarget{latest-updates-2020-election}{%
\section{\texorpdfstring{\href{https://www.nytimes.com/2020/07/31/us/elections/biden-vs-trump.html?action=click\&pgtype=Article\&state=default\&region=MAIN_CONTENT_1\&context=storylines_live_updates}{Latest
Updates: 2020
Election}}{Latest Updates: 2020 Election}}\label{latest-updates-2020-election}}

Updated 2020-08-01T01:26:45.732Z

\begin{itemize}
\tightlist
\item
  \href{https://www.nytimes.com/2020/07/31/us/elections/biden-vs-trump.html?action=click\&pgtype=Article\&state=default\&region=MAIN_CONTENT_1\&context=storylines_live_updates\#link-29fdff45}{Kamala
  Harris, a top vice-presidential contender, confronts double
  standards.}
\item
  \href{https://www.nytimes.com/2020/07/31/us/elections/biden-vs-trump.html?action=click\&pgtype=Article\&state=default\&region=MAIN_CONTENT_1\&context=storylines_live_updates\#link-13ec3d9c}{Karen
  Bass and Susan Rice are rising on Biden's vice-presidential
  shortlist.}
\item
  \href{https://www.nytimes.com/2020/07/31/us/elections/biden-vs-trump.html?action=click\&pgtype=Article\&state=default\&region=MAIN_CONTENT_1\&context=storylines_live_updates\#link-49e9a016}{Trump
  says Russian bounties to kill U.S. troops `never took place.'}
\end{itemize}

\href{https://www.nytimes.com/2020/07/31/us/elections/biden-vs-trump.html?action=click\&pgtype=Article\&state=default\&region=MAIN_CONTENT_1\&context=storylines_live_updates}{See
more updates}

Of the thousands of students, only a handful were Black, according to
former students and teachers. The first few Black faculty members,
including
\href{https://okdemocrats.org/oklahoma-black-history-heroes-clara-luper/}{Clara
Luper}, a noted local civil rights activist, wouldn't arrive until two
years after Ms. Warren graduated. In a speech years
\href{https://www.youtube.com/watch?v=dBMem0grMv8\&t=18s.}{later}, Ms.
Luper recalled protests outside her classroom window and boys chanting
racial slurs at her in the hall.

After Ms. Warren's father lost his job, her family struggled to stay in
the district so their children could attend the school, considered one
of the academically strongest in the area. Friends described Ms. Warren
as conservative at the time, and don't recall spending much time
discussing civil rights, even as protests, sit-ins and integration
efforts roiled her still largely segregated city throughout her high
school years.

Dr. Katrina Cochran, a childhood friend who would go on to become a
psychologist, said that Ms. Warren had been deeply conscious of the
stigma then associated with having a mother who worked outside the home
and that she had displayed an interest in economic inequality that would
define her career. But the topic of race didn't often come up between
the two girls.

``It was so clearly segregated,'' Dr. Cochran said, of their high
school. ``I look back on it now, and there wasn't one person of color
that I recall anywhere, except in the janitorial or kitchen staff.
That's how we grew up.''

Ms. Warren left Oklahoma City for George Washington University eager to
expand her horizons beyond the confines of her upbringing.

``I had never seen a ballet, never been to a museum and never ridden in
a taxi,'' Ms. Warren recalled in her 2014 memoir. ``I'd never had a
debate partner who was Black, never known anyone from Asia, and never
had a roommate of any kind.''

As an older cousin also had, a young Ms. Warren found her way into a
sorority, pledging the Gamma Kappa chapter of Kappa Alpha Theta. The
university had been officially desegregated in 1954, when it began
admitting Black students, but the sororities on campus remained a
bastion of discrimination.

Sorority life on campuses today often remains divided by race, and
historically Black Greek organizations, founded more than a century ago
during legal segregation, can be places where Black women seek
sisterhood. (Ms. Harris joined Alpha Kappa Alpha, the oldest
historically Black sorority, as an undergraduate at Howard University
and has spoken about how meaningful it was.)

The first Black Greek group didn't come to George Washington until 1975.
In the late 1960s, Black students were permitted to rush sororities and
fraternities but were never accepted. The girls were greeted with a
smile, according to their accounts in the student newspaper at the time,
but then rejected --- some repeatedly. Many were not informed of
specific requirements, including a recommendation letter attesting to
their ``moral character'' from someone in their hometown.

Greek life on campus was sheltered and exclusionary, a culture reflected
in the ``goat show'' that took place the year after Ms. Warren pledged
her sorority. At the event that year, in another sorority's performance,
three students
appeared\href{https://archive.org/details/gwu_cherry_tree_1968/page/n203/mode/2up}{onstage
in K.K.K. hoods} in a skit they said was intended as political satire.
Ms. Warren believes she did not attend the show, according to her staff,
because her debate team was traveling to competitions out of state the
same weekend.

School administrators tacitly condoned segregated Greek life, even as
the newly formed Black Student Union made desegregating sororities a top
priority.

``You know there's not very much we can do,'' Nan Webster, the president
of the Panhellenic Council, the governing body of sororities, told a
Black rushee, according to a 1968 report in the student newspaper.

When a Black woman tried to join Ms. Warren's chapter, her membership
was voted down by a few sorority sisters who were ``clearly of the
Southern attitude,'' said Carol Cushing, a former Kappa.

``To my recollection, no one talked about race or civil rights,''
recalled Ms. Cushing. ``As far I know, every single sorority was totally
white.''

In the spring of 1968, 200 students marched on campus to demand more
rights for Black students. By that fall, the sorority was ordered by the
university to insert a nondiscrimination clause in its bylaws. Ms.
Warren would not be there to see those changes: In the fall of 1968, she
married her high school boyfriend and transferred to the University of
Houston.

Image

Ms. Warren with her parents and her daughter, Amelia, at her graduation
from Rutgers Law School in 1976.Credit...Elizabeth Warren Campaign, via
Associated Press

\hypertarget{that-was-liz}{%
\subsection{`That was Liz'}\label{that-was-liz}}

The young Rutgers law student made his case to other members of the law
review.

Shouldn't the all-white organization include some students of color?

``It certainly hit me at that meeting that there wasn't one person of
color on the law review,'' recalled Louis Raveson, the student who had
broached the subject. ``I thought and said to my colleagues, `This is
not OK.'''

Mr. Raveson proposed reserving some spots for nonwhite members. ``I
recall very clearly there was only one person who supported that,'' he
said. ``And that was Liz.''

By the time Ms. Warren began her legal studies at Rutgers Law School in
the fall of 1973, she was married, a former teacher and a mother. She
was focused on balancing her studies with caring for her young daughter
and was not as involved in civil rights activism even as she was
becoming more aware of racial inequality around her.

But on campus, the environment was changing. Six years earlier, racial
tension in Newark had exploded into days of
\href{https://www.nytimes.com/2017/07/11/nyregion/newark-riots-50-years.html}{rioting
and rebellion}. In response, the law school --- which came to be known
informally as the People's Electric Law School --- created legal clinics
to assist the city's Black residents and formed a minority student
program to increase the diversity of its student body.

``Discussions about race were everywhere at Newark and at Rutgers at
that time,'' said Mr. Raveson, who is now a professor at the law school.
``I have to think that being at Rutgers and being in Newark must have
had a profound effect on Liz.''

At the University of Houston Law Center, where she was hired as an
assistant professor in 1978, she largely focused on trying to get
tenure, said John Mixon, a retired University of Houston law professor
and a colleague of Ms. Warren's.

``Her growth at our law school was more in the direction of trying to
find an academic theme to work with than it was in civil rights or
anything of that sort,'' he said.

Her academic portfolio broadened, however, as she began delving deeper
into her academic research on consumer bankruptcy, colleagues said.

Dissatisfied with the conventional narrative --- that people who went
bankrupt were victims of their own poor economic choices --- she set out
to determine why people went bankrupt by analyzing data and visiting
courthouses to uncover the individual stories behind the filings. What
she found surprised her: Many families who were going bankrupt were
middle class.

And she and two colleagues at the University of Texas, Jay L. Westbrook
and Teresa A. Sullivan, made another discovery through their research
that would come to shape her views on systemic inequality. ``We found
some real evidence that there were disparate impacts on people by ZIP
code that implicated race,'' Mr. Westbrook said.

Stephen Burbank, a colleague of Ms. Warren's at the University of
Pennsylvania law school who was involved in her hiring there in 1987,
saw the effect of that work.

``I believe that finding out what was happening to people, including
minorities, was very, very influential in the development of all sorts
of her views and policy positions,'' he said.

Image

Ms. Warren and her husband, Bruce, walked through a Black Lives Matter
protest near the White House in June.Credit...Erin Schaff/The New York
Times

\hypertarget{reflecting-the-diversity-of-america}{%
\subsection{Reflecting the `diversity of
America'}\label{reflecting-the-diversity-of-america}}

When Ms. Warren arrived at Harvard Law School in the 1990s, the school
was undergoing something of an evolution. Students were battling the
institution over racial and cultural diversity on the faculty and had
even recently
\href{https://www.nytimes.com/1992/03/06/archives/battling-harvard-law-over-diversity.html}{sued
the school}, contending that its hiring practices were discriminatory.

On campus, Ms. Warren was a popular and demanding teacher. David
Wilkins, a former colleague of hers, recalled that she joined the
admissions committee --- generally considered unglamorous --- where she
pushed to make Harvard a law school that ``reflected the diversity of
America.''

She also became a mentor to young female law students. One woman,
Chrystin Ondersma, who is now a law professor at Rutgers, said she had
applied to Harvard with the goal of studying critical race theory and
gender studies, and she remembered meeting with Ms. Warren to discuss
her interests. ``If you really care about gender justice and racial
justice, then you really need to focus on bankruptcy and commercial
law,'' Ms. Warren responded.

It was in those years at Harvard that Ms. Warren's reputation as an
expert on the intersection of race and economics grew. She also
\href{https://www.nytimes.com/2019/08/25/us/politics/elizabeth-warren-republican-history.html}{switched
her political party affiliation}, in 1996, from Republican to Democrat.

As Congress debated bankruptcy legislation, Ms. Warren became a pro bono
adviser for Wade Henderson, then-head of the N.A.A.C.P.'s Washington
office and a fellow graduate of Rutgers Law School. Even then, Mr.
Henderson was impressed by Ms. Warren's understanding of the role race
plays in economic inequality, he recalled.

When she testified before the Senate Judiciary Committee about the bill
in 1999, Ms. Warren argued that Black and Hispanic homeowners would be
disproportionally harmed by the legislation.

``She already had a sensitivity to those issues that had been honed in
other places,'' said Mr. Henderson, the former president of the
Leadership Conference on Civil and Human Rights.

In 2004, Ms. Warren was invited to speak at a symposium on critical race
theory at Washington \& Lee University in Lexington, Va. The symposium's
organizer, Dorothy A. Brown, an expert on race and tax, wanted to have a
symposium ``that looked at areas not normally associated with systemic
racism,'' she said, like corporate law and bankruptcy.

When Ms. Warren agreed to come, Ms. Brown said, ``I was over the moon.''
She recalled in particular that Ms. Warren had spoken about how Black
college graduates were more likely to file for bankruptcy, because of
the student debt they carried.

``When she presented, she freaked everybody out with her research,'' Ms.
Brown said. ``She blew us all away.''

Ms. Warren would publish an academic paper that fall, ``The Economics of
Race: When Making It to the Middle Is Not Enough,'' in a volume
connected to the symposium.

In her own paper, Ms. Brown wrote that the volume ``makes a genuine
contribution to the literature by creating the space for scholars who
have not previously written about or explored issues of race to do so.''

The observation came with an accompanying footnote: ``See, e.g.,
Elizabeth Warren.''

Kitty Bennett contributed research.

\hypertarget{our-2020-election-guide}{%
\section{Our 2020 Election Guide}\label{our-2020-election-guide}}

Updated July 31, 2020

\begin{itemize}
\item
  \begin{center}\rule{0.5\linewidth}{\linethickness}\end{center}

  \hypertarget{the-latest}{%
  \subsection{The Latest}\label{the-latest}}

  \begin{itemize}
  \tightlist
  \item
    President Trump's assault on the Postal Service is intersecting with
    his attacks on mail-in voting.
    \href{https://www.nytimes.com/2020/07/31/us/politics/trump-usps-mail-delays.html?action=click\&pgtype=Article\&state=default\&region=BELOW_MAIN_CONTENT\&context=storylines_guide}{Voting
    rights groups say it is a recipe for disaster.}
  \end{itemize}
\item
  \begin{center}\rule{0.5\linewidth}{\linethickness}\end{center}

  \hypertarget{bidens-vp-search}{%
  \subsection{Biden's V.P. Search}\label{bidens-vp-search}}

  \begin{itemize}
  \tightlist
  \item
    \href{https://www.nytimes.com/article/biden-vice-president-2020.html?action=click\&pgtype=Article\&state=default\&region=BELOW_MAIN_CONTENT\&context=storylines_guide}{Here
    are 13 women} who have been under consideration to be Joe Biden's
    running mate, and why each might be chosen --- and might not be.
  \end{itemize}
\item
  \begin{center}\rule{0.5\linewidth}{\linethickness}\end{center}

  \hypertarget{keep-up-with-our-coverage}{%
  \subsection{Keep Up With Our
  Coverage}\label{keep-up-with-our-coverage}}

  \begin{itemize}
  \tightlist
  \item
    Get an
    \href{https://www.nytimes.com/newsletters/politics?action=click\&pgtype=Article\&state=default\&region=BELOW_MAIN_CONTENT\&context=storylines_guide}{email}
    recapping the day's news
  \end{itemize}

  \begin{itemize}
  \tightlist
  \item
    Download our mobile app on
    \href{https://apps.apple.com/us/app/nytimes/id284862083?ls=1\&mat_click_id=5c79ae7455014fd1bd66b5610c05b8f2-20191112-16948\&referrer=mat_click_id\%3D5c79ae7455014fd1bd66b5610c05b8f2-20191112-16948\%26link_click_id\%3D722930677036718082}{iOS}
    and
    \href{http://a.localytics.com/android?id=com.nytimes.android\&referrer=utm_source\%3Dother_nyt_mobile_web\%26utm_medium\%3DWeb\%2520page\%26utm_term\%3DGeneral\%2520Mobile\%2520Page\%26utm_campaign\%3DNYT\%2520Mobile\%2520General\%2520Page}{Android}
    and turn on Breaking News and Politics alerts
  \end{itemize}
\end{itemize}

Advertisement

\protect\hyperlink{after-bottom}{Continue reading the main story}

\hypertarget{site-index}{%
\subsection{Site Index}\label{site-index}}

\hypertarget{site-information-navigation}{%
\subsection{Site Information
Navigation}\label{site-information-navigation}}

\begin{itemize}
\tightlist
\item
  \href{https://help.nytimes.com/hc/en-us/articles/115014792127-Copyright-notice}{©~2020~The
  New York Times Company}
\end{itemize}

\begin{itemize}
\tightlist
\item
  \href{https://www.nytco.com/}{NYTCo}
\item
  \href{https://help.nytimes.com/hc/en-us/articles/115015385887-Contact-Us}{Contact
  Us}
\item
  \href{https://www.nytco.com/careers/}{Work with us}
\item
  \href{https://nytmediakit.com/}{Advertise}
\item
  \href{http://www.tbrandstudio.com/}{T Brand Studio}
\item
  \href{https://www.nytimes.com/privacy/cookie-policy\#how-do-i-manage-trackers}{Your
  Ad Choices}
\item
  \href{https://www.nytimes.com/privacy}{Privacy}
\item
  \href{https://help.nytimes.com/hc/en-us/articles/115014893428-Terms-of-service}{Terms
  of Service}
\item
  \href{https://help.nytimes.com/hc/en-us/articles/115014893968-Terms-of-sale}{Terms
  of Sale}
\item
  \href{https://spiderbites.nytimes.com}{Site Map}
\item
  \href{https://help.nytimes.com/hc/en-us}{Help}
\item
  \href{https://www.nytimes.com/subscription?campaignId=37WXW}{Subscriptions}
\end{itemize}
