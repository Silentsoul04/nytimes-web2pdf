Sections

SEARCH

\protect\hyperlink{site-content}{Skip to
content}\protect\hyperlink{site-index}{Skip to site index}

\href{https://www.nytimes.com/section/politics}{Politics}

\href{https://myaccount.nytimes.com/auth/login?response_type=cookie\&client_id=vi}{}

\href{https://www.nytimes.com/section/todayspaper}{Today's Paper}

\href{/section/politics}{Politics}\textbar{}Kayleigh McEnany Heckles the
Press. Is That All?

\url{https://nyti.ms/3i0XaqO}

\begin{itemize}
\item
\item
\item
\item
\item
\end{itemize}

Advertisement

\protect\hyperlink{after-top}{Continue reading the main story}

Supported by

\protect\hyperlink{after-sponsor}{Continue reading the main story}

White HOUse MEMO

\hypertarget{kayleigh-mcenany-heckles-the-press-is-that-all}{%
\section{Kayleigh McEnany Heckles the Press. Is That
All?}\label{kayleigh-mcenany-heckles-the-press-is-that-all}}

President Trump does not always watch her briefings, and even his allies
say she risks being known more for ``hitting the press with a
two-by-four'' than advancing his priorities.

\includegraphics{https://static01.nyt.com/images/2020/08/02/us/politics/02dc-kayleigh-pix1/02dc-kayleigh-pix1-articleLarge.jpg?quality=75\&auto=webp\&disable=upscale}

\href{https://www.nytimes.com/by/katie-rogers}{\includegraphics{https://static01.nyt.com/images/2018/06/12/multimedia/author-katie-rogers/author-katie-rogers-thumbLarge-v2.png}}\href{https://www.nytimes.com/by/maggie-haberman}{\includegraphics{https://static01.nyt.com/images/2018/07/12/multimedia/author-maggie-haberman/author-maggie-haberman-thumbLarge.png}}

By \href{https://www.nytimes.com/by/katie-rogers}{Katie Rogers} and
\href{https://www.nytimes.com/by/maggie-haberman}{Maggie Haberman}

\begin{itemize}
\item
  Aug. 2, 2020
\item
  \begin{itemize}
  \item
  \item
  \item
  \item
  \item
  \end{itemize}
\end{itemize}

WASHINGTON --- For the first half of his term, President Trump treated
the White House press briefing as must-see television. From his small
dining room off the Oval Office, he kept close watch over his first two
press secretaries as they battled with journalists, defended his
performance and often tried to rewrite history.

Kayleigh McEnany, the fourth person to hold the job since Mr. Trump took
office, does all of those things. The difference now is that the
president, who once considered the White House briefings to be
appointment television, does not always watch.

Just over 13 weeks from Election Day, Mr. Trump is back to serving as
his own primary spokesman, putting his faith in himself to pull out of a
deep polling hole and make the case for why voters should choose to give
him four more years despite his mishandling of the
\href{https://www.nytimes.com/news-event/coronavirus}{coronavirus
pandemic} and a
\href{https://www.nytimes.com/2020/07/30/business/economy/q2-gdp-coronavirus-economy.html}{brutal
recession}.

Whether he is helping himself is subject to debate. But his decision to
put himself front and center on a near-daily basis has left Ms. McEnany
in a distinctly secondary role. Not only has her audience of one failed
to watch a number of her briefings in recent weeks --- a senior
administration official suggested on Friday that the president was busy
with other matters --- she has also encountered flagging interest from
television networks; only Fox News regularly carries her briefings live,
and at least one network has declined a request to have her appear on
one of its shows.

All of that is leaving it increasingly unclear what purpose Ms. McEnany
is filling beyond berating the news media from the briefing room podium.

When Ms. McEnany, 32,
\href{https://www.nytimes.com/2020/04/07/us/politics/kayleigh-mcenany-stephanie-grisham-trump.html}{assumed
the role in April}, Mr. Trump was delivering two-hour briefings on the
coronavirus pandemic. At the time, questions were raised internally
about whether Ms. McEnany should start formalized briefings or continue
defending the president through television appearances --- she initially
made a name for herself in his world by defending him on CNN, a network
where he has few allies.

In April, her new colleagues also wondered how Ms. McEnany, an operative
known for overtly partisan and often fact-free defenses of the
president, would fare in the White House, which is
\href{https://www.nytimes.com/2020/07/16/us/politics/trump-goya-ivanka.html}{supposed
to separate official business from campaign operations} --- at least in
theory.

If that was ever a concern for Ms. McEnany or anyone else on staff, it
never showed. Working with Mark Meadows, the White House chief of staff,
and several of the officials he brought with him when he joined the
administration in March, Ms. McEnany reinstated the briefings, which had
been stopped under her predecessor, Stephanie Grisham. Ms. McEnany
infused them with campaign-style talking points and edited videos of
news clips, some of which were stripped of context.

Instead of bringing the public health experts involved in the
coronavirus response with her to the podium so that reporters could ask
them questions, she spoke to them herself beforehand.

Early in Ms. McEnany's tenure, she was praised by conservative news
media for her attacks on the mainstream press. But since then, she has
so far struggled to make the briefings compelling enough --- or credible
enough --- to refocus the attention on what the administration hopes to
highlight instead of the pandemic. She recently resorted to playing a
video of protesters in Portland, Ore., that she accused the news media
of ignoring. The video
\href{https://twitter.com/alexnazaryan/status/1286718175506178048?ref_src=twsrc\%5Etfw\%7Ctwcamp\%5Etweetembed\%7Ctwterm\%5E1286718175506178048\%7Ctwgr\%5E\&ref_url=https\%3A\%2F\%2Fwww.businessinsider.com\%2Ffox-news-cut-away-from-disturbing-portland-video-wh-briefing-2020-7}{contained
explicit language}, causing Fox News, which faithfully airs the
briefings, to cut away.

``We were not expecting that video, and our management here at Fox News
has decided we will cut away at this time,'' the host, Harris Faulkner,
said.

Ms. McEnany is hardly the first Trump press secretary to criticize the
news media, or to say things from the podium that are untrue.

Sean Spicer, the first to hold the job under Mr. Trump, started out by
trying to persuade incredulous reporters and the nation that the
Inauguration Day crowd size set an attendance record.
(\href{https://www.nytimes.com/2017/01/22/us/politics/president-trump-inauguration-crowd-white-house.html}{It
didn't}.) Sarah Huckabee Sanders
\href{https://www.theguardian.com/us-news/2018/aug/03/sanders-trump-acosta-media-enemy}{refused}
to disavow Mr. Trump's claim that journalists were the enemies of the
people. And Ms. Grisham,
\href{https://www.nytimes.com/2020/01/10/business/media/stephanie-grisham-trump-press-secretary.html}{who
never held a briefing}, preferred using Twitter to call out journalists
by name.

But for all that, Ms. McEnany's predecessors also understood the value
in working with reporters, even when it was not something Mr. Trump
encouraged.

Jonathan Karl of ABC News, the former president of the White House
Correspondents' Association, criticized Ms. McEnany's tendency to fill
the press briefings with ``head-spinning contradictions'' and her lack
of interest in clarifying Mr. Trump's decisions, particularly on matters
of race.

``The White House press secretary serves at the pleasure of a president
but is also a public servant whose salary is paid by taxpayers,'' Mr.
Karl
\href{https://www.washingtonpost.com/opinions/its-the-duty-of-the-white-house-press-secretary-to-hold-briefings-but-not-like-this/2020/07/10/1a61ae78-c2cc-11ea-b178-bb7b05b94af1_story.html}{wrote
in a Washington Post op-ed} in July. ``Denying reality and using the
White House podium for purely political purposes is a violation of
public trust.''

Ms. McEnany has taken the adversarial posture to a new level. At one
point, she insinuated that reporters were adverse to religion by saying
journalists were ``desperately'' opposed to houses of worship reopening
during the pandemic.

Ari Fleischer, who served as press secretary during the administration
of President George W. Bush, praised Ms. McEnany's performance but said
she had demonstrated little interest in forging a relationship with the
press. He said her tendency of ``hitting the press with a two-by-four''
risked overshadowing any substantive contributions.

``The press secretaries have always
\href{https://www.nytimes.com/2001/01/22/us/public-lives-careful-steps-took-press-secretary-to-the-white-house.html}{served
two masters}, as the old adage goes,'' Mr. Fleischer, a Fox News
contributor, said. ``But these days, the president is way bigger than
the press in reality of how the briefing works.''

Mike McCurry, who served as press secretary under President Bill
Clinton, said that Ms. McEnany only seemed to be interested in serving
as a partisan booster for the president.

``There's got to be a balance between providing information and content
that is useful to the public while advancing whatever the president's
political objectives are,'' Mr. McCurry said. ``You kind of have to keep
both of those goals in mind.''

The White House disputed the idea that Ms. McEnany did not work with
reporters, but officials said she spent most of her time with the
president, maintaining that is what she should be doing as his chief
spokeswoman. Officials said that no cable television networks had
declined to have Ms. McEnany appear, but that she almost exclusively
appeared on Fox News.

``Kayleigh is a skilled, savvy communicator who delivers policy
positions while vocally defending President Trump against biased,
negative media coverage --- which she is not afraid to expose when
journalists attempt to distort the record,'' Judd Deere, a White House
spokesman, said in a statement about his boss.

Questioning news coverage has become a central part of her regular
briefings. Ms. McEnany and members of the White House communications
staff spend much of the mornings on the days she briefs in preparation
for her to take the podium, assembling talking points and counterattacks
that are then stuffed into an oversized binder. The final product is
assembled for her by Julia Hahn, a former Breitbart News employee who
now works on the White House press and communications staff.

\includegraphics{https://static01.nyt.com/images/2020/08/02/us/politics/02dc-kayleigh-pix2/merlin_174798384_5e8585d7-a727-4a98-869b-232fd7a8fc1a-articleLarge.jpg?quality=75\&auto=webp\&disable=upscale}

A recent photograph of the innards of the binder --- full of tabs
containing ready-made talking points on topics such as ``HATE,''
``LIES,'' ``MASKS'' and ``WINS'' --- offered a lens not into the
problems the administration faces, but
\href{https://www.washingtonpost.com/politics/2020/07/17/heres-what-kayleigh-mcenany-would-like-talk-about/}{how
Mr. Trump and his aides would prefer to recast them}.

On Friday, Ms. McEnany's fingers trailed through the binder tabs as she
received a question on elections in Hong Kong. ``I do have an answer for
you,'' she said, before reading from a prepared statement that condemned
officials there for
\href{https://www.nytimes.com/2020/07/31/world/asia/hong-kong-election-delayed.html}{postponing
an election}.

In the same briefing, she defended Mr. Trump's suggestion that the
November election be postponed, in part because he had directed her to
focus part of her briefing on the topic, a senior administration
official said.

At other times, Ms. McEnany and Mr. Trump have not been in sync. Last
week, the president appeared confused at one of his own briefings when
he was told that Ms. McEnany had shared with reporters on live
television that he was being tested for the coronavirus several times a
day.

``I don't know about more than one,'' Mr. Trump said. ``I do probably on
average a test every two days, three days, and I don't know of any time
I've taken two in one day, but I could see that happening.''

That moment starkly illustrated the limitation of press briefings under
Mr. Trump, a president who has no interest in letting them serve in a
traditional capacity as an organizing device to make sure his most
senior officials are in step with him.

``That has always been the value of a press briefing,'' Mr. McCurry
said. ``It is really a way to get a handle on the information and
content that the public needs to know, and that the government has an
obligation to disclose. I'm not sure that attitude prevails in this
White House.''

Advertisement

\protect\hyperlink{after-bottom}{Continue reading the main story}

\hypertarget{site-index}{%
\subsection{Site Index}\label{site-index}}

\hypertarget{site-information-navigation}{%
\subsection{Site Information
Navigation}\label{site-information-navigation}}

\begin{itemize}
\tightlist
\item
  \href{https://help.nytimes.com/hc/en-us/articles/115014792127-Copyright-notice}{©~2020~The
  New York Times Company}
\end{itemize}

\begin{itemize}
\tightlist
\item
  \href{https://www.nytco.com/}{NYTCo}
\item
  \href{https://help.nytimes.com/hc/en-us/articles/115015385887-Contact-Us}{Contact
  Us}
\item
  \href{https://www.nytco.com/careers/}{Work with us}
\item
  \href{https://nytmediakit.com/}{Advertise}
\item
  \href{http://www.tbrandstudio.com/}{T Brand Studio}
\item
  \href{https://www.nytimes.com/privacy/cookie-policy\#how-do-i-manage-trackers}{Your
  Ad Choices}
\item
  \href{https://www.nytimes.com/privacy}{Privacy}
\item
  \href{https://help.nytimes.com/hc/en-us/articles/115014893428-Terms-of-service}{Terms
  of Service}
\item
  \href{https://help.nytimes.com/hc/en-us/articles/115014893968-Terms-of-sale}{Terms
  of Sale}
\item
  \href{https://spiderbites.nytimes.com}{Site Map}
\item
  \href{https://help.nytimes.com/hc/en-us}{Help}
\item
  \href{https://www.nytimes.com/subscription?campaignId=37WXW}{Subscriptions}
\end{itemize}
