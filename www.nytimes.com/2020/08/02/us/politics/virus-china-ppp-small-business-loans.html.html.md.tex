Sections

SEARCH

\protect\hyperlink{site-content}{Skip to
content}\protect\hyperlink{site-index}{Skip to site index}

\href{https://www.nytimes.com/section/politics}{Politics}

\href{https://myaccount.nytimes.com/auth/login?response_type=cookie\&client_id=vi}{}

\href{https://www.nytimes.com/section/todayspaper}{Today's Paper}

\href{/section/politics}{Politics}\textbar{}U.S. Small Business Bailout
Money Flowed to Chinese-Owned Companies

\url{https://nyti.ms/2DqhkeX}

\begin{itemize}
\item
\item
\item
\item
\item
\end{itemize}

\href{https://www.nytimes.com/news-event/coronavirus?action=click\&pgtype=Article\&state=default\&region=TOP_BANNER\&context=storylines_menu}{The
Coronavirus Outbreak}

\begin{itemize}
\tightlist
\item
  live\href{https://www.nytimes.com/2020/08/03/world/coronavirus-covid-19.html?action=click\&pgtype=Article\&state=default\&region=TOP_BANNER\&context=storylines_menu}{Latest
  Updates}
\item
  \href{https://www.nytimes.com/interactive/2020/us/coronavirus-us-cases.html?action=click\&pgtype=Article\&state=default\&region=TOP_BANNER\&context=storylines_menu}{Maps
  and Cases}
\item
  \href{https://www.nytimes.com/interactive/2020/science/coronavirus-vaccine-tracker.html?action=click\&pgtype=Article\&state=default\&region=TOP_BANNER\&context=storylines_menu}{Vaccine
  Tracker}
\item
  \href{https://www.nytimes.com/2020/08/02/us/covid-college-reopening.html?action=click\&pgtype=Article\&state=default\&region=TOP_BANNER\&context=storylines_menu}{College
  Reopening}
\item
  \href{https://www.nytimes.com/live/2020/08/03/business/stock-market-today-coronavirus?action=click\&pgtype=Article\&state=default\&region=TOP_BANNER\&context=storylines_menu}{Economy}
\end{itemize}

Advertisement

\protect\hyperlink{after-top}{Continue reading the main story}

Supported by

\protect\hyperlink{after-sponsor}{Continue reading the main story}

\hypertarget{us-small-business-bailout-money-flowed-to-chinese-owned-companies}{%
\section{U.S. Small Business Bailout Money Flowed to Chinese-Owned
Companies}\label{us-small-business-bailout-money-flowed-to-chinese-owned-companies}}

Millions of dollars of Paycheck Protection Program loans went to
China-backed businesses in critical sectors, a study found.

\includegraphics{https://static01.nyt.com/images/2020/08/03/us/politics/02JPdc-pppchina-print/merlin_171898629_58bf8b69-8779-4aaa-acce-8e43ad81a471-articleLarge.jpg?quality=75\&auto=webp\&disable=upscale}

\href{https://www.nytimes.com/by/alan-rappeport}{\includegraphics{https://static01.nyt.com/images/2018/06/12/multimedia/author-alan-rappeport/author-alan-rappeport-thumbLarge-v2.png}}

By \href{https://www.nytimes.com/by/alan-rappeport}{Alan Rappeport}

\begin{itemize}
\item
  Aug. 2, 2020
\item
  \begin{itemize}
  \item
  \item
  \item
  \item
  \item
  \end{itemize}
\end{itemize}

\href{https://cn.nytimes.com/usa/20200803/virus-china-ppp-small-business-loans/}{阅读简体中文版}\href{https://cn.nytimes.com/usa/20200803/virus-china-ppp-small-business-loans/zh-hant/}{閱讀繁體中文版}

WASHINGTON --- President Trump has blamed China for the
\href{https://www.nytimes.com/news-event/coronavirus}{coronavirus
pandemic} and the ensuing
\href{https://www.nytimes.com/2020/07/30/business/economy/q2-gdp-coronavirus-economy.html}{economic
crisis}, but as the White House looks to stabilize small businesses in
the United States, the rescue effort has had an unintended beneficiary:
Chinese companies.

Millions of dollars of American taxpayer money have flowed to China from
the \$660 billion Paycheck Protection Program that was created in March
to be a lifeline for struggling small businesses in the United States.
But because the economic relief legislation allowed American
subsidiaries of foreign firms to receive the loans, a substantial chunk
of the money went to America's biggest economic rival, a new analysis
shows.

According to \href{https://www.horizonadvisory.org/paycheckprotection}{a
review} of
\href{https://www.nytimes.com/2020/07/06/us/ppp-small-business-loans.html}{publicly
available loan data} by the strategy consulting firm Horizon Advisory,
\$192 million to \$419 million has gone to more than 125 companies that
Chinese entities own or invest in. Many of the loans were quite sizable;
at least 32 Chinese companies received loans worth more than \$1
million, with those totaling as much as \$180 million.

``The extent and nature of P.R.C.-owned, -invested and -connected
entities among the P.P.P. loan recipients indicate that without
appropriate policy guardrails, U.S. tax dollars intended for relief,
recovery and growth of the U.S. economy --- and small businesses in
particular --- risk supporting foreign competitors, namely China,''
wrote Emily de La Bruyère and Nathan Picarsic, the co-founders of
Horizon Advisory, referring to the People's Republic of China.

The report acknowledges that the participation of these companies in the
lending program most likely saved an unspecified number of jobs based in
the United States, but it also suggests that many of the businesses
probably had access to other forms of capital from public or private
markets to support their American operations. The Treasury Department
has estimated that the overall program has kept 50 million workers
employed in the United States.

The revelation that Chinese-backed companies were helped by American tax
dollars shows the deep ties that remain between American and Chinese
businesses even as
\href{https://www.nytimes.com/2020/07/25/world/asia/us-china-trump-xi.html}{relations
between the countries have deteriorated} in recent months. Mr. Trump has
regularly vented his anger at China and accused it of spreading a virus
that has left the once-thriving United States economy in tatters.

``It's China's fault,'' he said on Friday. ``China should be paying for
it, and maybe they will.''

The virus has emboldened the administration's hawks in their calls for
both punishing China and decoupling the world's two largest economies,
saying Beijing poses a national security threat. The administration in
recent weeks has
\href{https://www.nytimes.com/2020/07/31/us/politics/sanctions-china-xinjiang-uighurs.html}{sanctioned
Chinese officials} accused of facilitating human rights violations in
the Xinjiang region and
\href{https://www.nytimes.com/2020/07/20/business/economy/china-sanctions-uighurs-labor.html}{banned
more Chinese technology companies} from buying American technology and
components.

\hypertarget{latest-updates-global-coronavirus-outbreak}{%
\section{\texorpdfstring{\href{https://www.nytimes.com/2020/08/03/world/coronavirus-covid-19.html?action=click\&pgtype=Article\&state=default\&region=MAIN_CONTENT_1\&context=storylines_live_updates}{Latest
Updates: Global Coronavirus
Outbreak}}{Latest Updates: Global Coronavirus Outbreak}}\label{latest-updates-global-coronavirus-outbreak}}

Updated 2020-08-04T04:02:32.475Z

\begin{itemize}
\tightlist
\item
  \href{https://www.nytimes.com/2020/08/03/world/coronavirus-covid-19.html?action=click\&pgtype=Article\&state=default\&region=MAIN_CONTENT_1\&context=storylines_live_updates\#link-4547638f}{Fauci
  defends Birx after she is criticized by Trump.}
\item
  \href{https://www.nytimes.com/2020/08/03/world/coronavirus-covid-19.html?action=click\&pgtype=Article\&state=default\&region=MAIN_CONTENT_1\&context=storylines_live_updates\#link-15e7f995}{Trump
  derides Democrats as lawmakers and administration officials try to
  break stimulus impasse.}
\item
  \href{https://www.nytimes.com/2020/08/03/world/coronavirus-covid-19.html?action=click\&pgtype=Article\&state=default\&region=MAIN_CONTENT_1\&context=storylines_live_updates\#link-e5a2cda}{The
  deadline for 2020 census counting has been moved up by a month.}
\end{itemize}

\href{https://www.nytimes.com/2020/08/03/world/coronavirus-covid-19.html?action=click\&pgtype=Article\&state=default\&region=MAIN_CONTENT_1\&context=storylines_live_updates}{See
more updates}

More live coverage:
\href{https://www.nytimes.com/live/2020/08/03/business/stock-market-today-coronavirus?action=click\&pgtype=Article\&state=default\&region=MAIN_CONTENT_1\&context=storylines_live_updates}{Markets}

The White House is also
\href{https://www.nytimes.com/2020/07/31/technology/tiktok-microsoft.html}{nearing
a decision} that could force ByteDance, a Chinese firm, to
\href{https://www.nytimes.com/2020/08/01/technology/tiktok-sale-trump-ban.html}{sell
the U.S. operations of the social media app TikTok} over national
security concerns.

``As far as TikTok is concerned, we're
\href{https://www.nytimes.com/2020/08/01/technology/tiktok-trump-microsoft-bytedance-china-ban.html}{banning
them} from the United States,'' the president said on Friday.

But the administration's aggressive approach to China did not stop
companies with ties to Beijing from benefiting from one of the main
programs intended to prop up the United States economy.

The Paycheck Protection Program, which was created as part of a \$2.2
trillion relief package signed in March, was devised to help small
businesses with fewer than 500 workers cover payrolls and overhead
expenses while much of the economy was shuttered. When big publicly
traded companies that had access to other forms of capital took out
P.P.P. loans, the Trump administration publicly shamed them and Treasury
Secretary Steven Mnuchin urged them to repay the money, saying they
could
\href{https://www.nytimes.com/2020/04/28/us/politics/coronavirus-treasury-payment-protection-program.html}{face
criminal liability} if they did not qualify for the loans.

But the rules of the program also allowed American subsidiaries of
foreign-owned companies to apply for and receive loans.

The Horizon Advisory report, which analyzed public filings and loan data
that was released by the Treasury Department in June, does not claim to
be an exhaustive account of the more than five million loans that were
initiated through the program. The data released by the Treasury
Department, for instance, shared loan amounts in ranges only for
businesses that borrowed more than \$150,000, and information for
private firms that took smaller loans was released only in aggregate.

Among the companies highlighted in the report were Continental Aerospace
Technologies, which received a loan of up to \$10 million, and Aviage
Systems, which received a loan of up to \$350,000. The companies are
owned by Aviation Industry Corporation of China, a state-owned
conglomerate that the Department of Defense classified this year as a
Chinese military company.

\href{https://www.nytimes.com/news-event/coronavirus?action=click\&pgtype=Article\&state=default\&region=MAIN_CONTENT_3\&context=storylines_faq}{}

\hypertarget{the-coronavirus-outbreak-}{%
\subsubsection{The Coronavirus Outbreak
›}\label{the-coronavirus-outbreak-}}

\hypertarget{frequently-asked-questions}{%
\paragraph{Frequently Asked
Questions}\label{frequently-asked-questions}}

Updated August 3, 2020

\begin{itemize}
\item ~
  \hypertarget{im-a-small-business-owner-can-i-get-relief}{%
  \paragraph{I'm a small-business owner. Can I get
  relief?}\label{im-a-small-business-owner-can-i-get-relief}}

  \begin{itemize}
  \tightlist
  \item
    The
    \href{https://www.nytimes.com/article/small-business-loans-stimulus-grants-freelancers-coronavirus.html?action=click\&pgtype=Article\&state=default\&region=MAIN_CONTENT_3\&context=storylines_faq}{stimulus
    bills enacted in March} offer help for the millions of American
    small businesses. Those eligible for aid are businesses and
    nonprofit organizations with fewer than 500 workers, including sole
    proprietorships, independent contractors and freelancers. Some
    larger companies in some industries are also eligible. The help
    being offered, which is being managed by the Small Business
    Administration, includes the Paycheck Protection Program and the
    Economic Injury Disaster Loan program. But lots of folks have
    \href{https://www.nytimes.com/interactive/2020/05/07/business/small-business-loans-coronavirus.html?action=click\&pgtype=Article\&state=default\&region=MAIN_CONTENT_3\&context=storylines_faq}{not
    yet seen payouts.} Even those who have received help are confused:
    The rules are draconian, and some are stuck sitting on
    \href{https://www.nytimes.com/2020/05/02/business/economy/loans-coronavirus-small-business.html?action=click\&pgtype=Article\&state=default\&region=MAIN_CONTENT_3\&context=storylines_faq}{money
    they don't know how to use.} Many small-business owners are getting
    less than they expected or
    \href{https://www.nytimes.com/2020/06/10/business/Small-business-loans-ppp.html?action=click\&pgtype=Article\&state=default\&region=MAIN_CONTENT_3\&context=storylines_faq}{not
    hearing anything at all.}
  \end{itemize}
\item ~
  \hypertarget{what-are-my-rights-if-i-am-worried-about-going-back-to-work}{%
  \paragraph{What are my rights if I am worried about going back to
  work?}\label{what-are-my-rights-if-i-am-worried-about-going-back-to-work}}

  \begin{itemize}
  \tightlist
  \item
    Employers have to provide
    \href{https://www.osha.gov/SLTC/covid-19/standards.html}{a safe
    workplace} with policies that protect everyone equally.
    \href{https://www.nytimes.com/article/coronavirus-money-unemployment.html?action=click\&pgtype=Article\&state=default\&region=MAIN_CONTENT_3\&context=storylines_faq}{And
    if one of your co-workers tests positive for the coronavirus, the
    C.D.C.} has said that
    \href{https://www.cdc.gov/coronavirus/2019-ncov/community/guidance-business-response.html}{employers
    should tell their employees} -\/- without giving you the sick
    employee's name -\/- that they may have been exposed to the virus.
  \end{itemize}
\item ~
  \hypertarget{should-i-refinance-my-mortgage}{%
  \paragraph{Should I refinance my
  mortgage?}\label{should-i-refinance-my-mortgage}}

  \begin{itemize}
  \tightlist
  \item
    \href{https://www.nytimes.com/article/coronavirus-money-unemployment.html?action=click\&pgtype=Article\&state=default\&region=MAIN_CONTENT_3\&context=storylines_faq}{It
    could be a good idea,} because mortgage rates have
    \href{https://www.nytimes.com/2020/07/16/business/mortgage-rates-below-3-percent.html?action=click\&pgtype=Article\&state=default\&region=MAIN_CONTENT_3\&context=storylines_faq}{never
    been lower.} Refinancing requests have pushed mortgage applications
    to some of the highest levels since 2008, so be prepared to get in
    line. But defaults are also up, so if you're thinking about buying a
    home, be aware that some lenders have tightened their standards.
  \end{itemize}
\item ~
  \hypertarget{what-is-school-going-to-look-like-in-september}{%
  \paragraph{What is school going to look like in
  September?}\label{what-is-school-going-to-look-like-in-september}}

  \begin{itemize}
  \tightlist
  \item
    It is unlikely that many schools will return to a normal schedule
    this fall, requiring the grind of
    \href{https://www.nytimes.com/2020/06/05/us/coronavirus-education-lost-learning.html?action=click\&pgtype=Article\&state=default\&region=MAIN_CONTENT_3\&context=storylines_faq}{online
    learning},
    \href{https://www.nytimes.com/2020/05/29/us/coronavirus-child-care-centers.html?action=click\&pgtype=Article\&state=default\&region=MAIN_CONTENT_3\&context=storylines_faq}{makeshift
    child care} and
    \href{https://www.nytimes.com/2020/06/03/business/economy/coronavirus-working-women.html?action=click\&pgtype=Article\&state=default\&region=MAIN_CONTENT_3\&context=storylines_faq}{stunted
    workdays} to continue. California's two largest public school
    districts --- Los Angeles and San Diego --- said on July 13, that
    \href{https://www.nytimes.com/2020/07/13/us/lausd-san-diego-school-reopening.html?action=click\&pgtype=Article\&state=default\&region=MAIN_CONTENT_3\&context=storylines_faq}{instruction
    will be remote-only in the fall}, citing concerns that surging
    coronavirus infections in their areas pose too dire a risk for
    students and teachers. Together, the two districts enroll some
    825,000 students. They are the largest in the country so far to
    abandon plans for even a partial physical return to classrooms when
    they reopen in August. For other districts, the solution won't be an
    all-or-nothing approach.
    \href{https://bioethics.jhu.edu/research-and-outreach/projects/eschool-initiative/school-policy-tracker/}{Many
    systems}, including the nation's largest, New York City, are
    devising
    \href{https://www.nytimes.com/2020/06/26/us/coronavirus-schools-reopen-fall.html?action=click\&pgtype=Article\&state=default\&region=MAIN_CONTENT_3\&context=storylines_faq}{hybrid
    plans} that involve spending some days in classrooms and other days
    online. There's no national policy on this yet, so check with your
    municipal school system regularly to see what is happening in your
    community.
  \end{itemize}
\item ~
  \hypertarget{is-the-coronavirus-airborne}{%
  \paragraph{Is the coronavirus
  airborne?}\label{is-the-coronavirus-airborne}}

  \begin{itemize}
  \tightlist
  \item
    The coronavirus
    \href{https://www.nytimes.com/2020/07/04/health/239-experts-with-one-big-claim-the-coronavirus-is-airborne.html?action=click\&pgtype=Article\&state=default\&region=MAIN_CONTENT_3\&context=storylines_faq}{can
    stay aloft for hours in tiny droplets in stagnant air}, infecting
    people as they inhale, mounting scientific evidence suggests. This
    risk is highest in crowded indoor spaces with poor ventilation, and
    may help explain super-spreading events reported in meatpacking
    plants, churches and restaurants.
    \href{https://www.nytimes.com/2020/07/06/health/coronavirus-airborne-aerosols.html?action=click\&pgtype=Article\&state=default\&region=MAIN_CONTENT_3\&context=storylines_faq}{It's
    unclear how often the virus is spread} via these tiny droplets, or
    aerosols, compared with larger droplets that are expelled when a
    sick person coughs or sneezes, or transmitted through contact with
    contaminated surfaces, said Linsey Marr, an aerosol expert at
    Virginia Tech. Aerosols are released even when a person without
    symptoms exhales, talks or sings, according to Dr. Marr and more
    than 200 other experts, who
    \href{https://academic.oup.com/cid/article/doi/10.1093/cid/ciaa939/5867798}{have
    outlined the evidence in an open letter to the World Health
    Organization}.
  \end{itemize}
\end{itemize}

HNA Group North America LLC and HNA Training Center NY, subsidiaries of
China's HNA Group, both received loans of up to \$1 million. HNA Group
specializes in real estate, aviation and financial services transactions
and is part of the Fortune Global 500.

And BGI Americas Corporation, which is a subsidiary of the Chinese
gene-testing giant BGI Group, took a loan of up to \$1 million. The
company, which is building a gene bank in Xinjiang, did return the
money, however, after
\href{https://www.axios.com/chinese-biotech-us-subsidiary-ppp-loan-8134a863-3779-46c4-9190-ceca54ba00ca.html}{Axios
reported on the loan}.

Larger loans went to businesses that spanned critical sectors such as
pharmaceuticals, defense, advanced manufacturing, electric cars and
information technology. In each case, the United States was indirectly
funding the kinds of corporations whose owners the Trump administration
regularly accuses of intellectual property theft.

For example, Dendreon Pharmaceuticals, a California-based biotech
company, received a loan worth \$5 million to \$10 million. It is owned
by Nanjing Xinbai, a Chinese state-invested company whose controlling
shareholder has close ties to the Communist Party.

Money from the Paycheck Protection Program has also made its way to the
financial technology sector. Citcon USA LLC, a Silicon Valley mobile
payments firm, received \$150,000 to \$350,000 in loan money. ZhenFund
is a major investor; the company connects American companies to Chinese
payment platforms such as Alipay and WeChat, which
\href{https://www.nytimes.com/2020/07/15/technology/tiktok-washington-lobbyist.html}{could
also face restrictions} from the Trump administration.

Lawmakers and the Trump administration are in the process of negotiating
how to recalibrate the small business lending program in an upcoming
economic relief package. A provision of the bill proposed last week by
Senate Republicans would make businesses that are partially owned by
Chinese companies or that have a Chinese resident on the board of
directors ineligible for the next round of loan money. It is uncertain
whether such a provision might make it to a final bill.

A Treasury spokeswoman noted that the Small Business Administration
might review any of the loans administered through the program and deny
forgiveness if it turned out that the borrower was not eligible or
misrepresented their business in the loan application.

The White House had no comment on the loans.

Advertisement

\protect\hyperlink{after-bottom}{Continue reading the main story}

\hypertarget{site-index}{%
\subsection{Site Index}\label{site-index}}

\hypertarget{site-information-navigation}{%
\subsection{Site Information
Navigation}\label{site-information-navigation}}

\begin{itemize}
\tightlist
\item
  \href{https://help.nytimes.com/hc/en-us/articles/115014792127-Copyright-notice}{©~2020~The
  New York Times Company}
\end{itemize}

\begin{itemize}
\tightlist
\item
  \href{https://www.nytco.com/}{NYTCo}
\item
  \href{https://help.nytimes.com/hc/en-us/articles/115015385887-Contact-Us}{Contact
  Us}
\item
  \href{https://www.nytco.com/careers/}{Work with us}
\item
  \href{https://nytmediakit.com/}{Advertise}
\item
  \href{http://www.tbrandstudio.com/}{T Brand Studio}
\item
  \href{https://www.nytimes.com/privacy/cookie-policy\#how-do-i-manage-trackers}{Your
  Ad Choices}
\item
  \href{https://www.nytimes.com/privacy}{Privacy}
\item
  \href{https://help.nytimes.com/hc/en-us/articles/115014893428-Terms-of-service}{Terms
  of Service}
\item
  \href{https://help.nytimes.com/hc/en-us/articles/115014893968-Terms-of-sale}{Terms
  of Sale}
\item
  \href{https://spiderbites.nytimes.com}{Site Map}
\item
  \href{https://help.nytimes.com/hc/en-us}{Help}
\item
  \href{https://www.nytimes.com/subscription?campaignId=37WXW}{Subscriptions}
\end{itemize}
