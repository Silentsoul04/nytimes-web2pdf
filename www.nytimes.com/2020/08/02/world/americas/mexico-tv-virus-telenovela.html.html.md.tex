Sections

SEARCH

\protect\hyperlink{site-content}{Skip to
content}\protect\hyperlink{site-index}{Skip to site index}

\href{https://www.nytimes.com/section/world/americas}{Americas}

\href{https://myaccount.nytimes.com/auth/login?response_type=cookie\&client_id=vi}{}

\href{https://www.nytimes.com/section/todayspaper}{Today's Paper}

\href{/section/world/americas}{Americas}\textbar{}Less Sex, More
Viewers: Pandemic Boosts Mexico's Flagging Telenovelas

\url{https://nyti.ms/3fiBzsd}

\begin{itemize}
\item
\item
\item
\item
\item
\end{itemize}

\href{https://www.nytimes.com/news-event/coronavirus?action=click\&pgtype=Article\&state=default\&region=TOP_BANNER\&context=storylines_menu}{The
Coronavirus Outbreak}

\begin{itemize}
\tightlist
\item
  live\href{https://www.nytimes.com/2020/08/04/world/coronavirus-cases.html?action=click\&pgtype=Article\&state=default\&region=TOP_BANNER\&context=storylines_menu}{Latest
  Updates}
\item
  \href{https://www.nytimes.com/interactive/2020/us/coronavirus-us-cases.html?action=click\&pgtype=Article\&state=default\&region=TOP_BANNER\&context=storylines_menu}{Maps
  and Cases}
\item
  \href{https://www.nytimes.com/interactive/2020/science/coronavirus-vaccine-tracker.html?action=click\&pgtype=Article\&state=default\&region=TOP_BANNER\&context=storylines_menu}{Vaccine
  Tracker}
\item
  \href{https://www.nytimes.com/2020/08/02/us/covid-college-reopening.html?action=click\&pgtype=Article\&state=default\&region=TOP_BANNER\&context=storylines_menu}{College
  Reopening}
\item
  \href{https://www.nytimes.com/live/2020/08/04/business/stock-market-today-coronavirus?action=click\&pgtype=Article\&state=default\&region=TOP_BANNER\&context=storylines_menu}{Economy}
\end{itemize}

Advertisement

\protect\hyperlink{after-top}{Continue reading the main story}

Supported by

\protect\hyperlink{after-sponsor}{Continue reading the main story}

\hypertarget{less-sex-more-viewers-pandemic-boosts-mexicos-flagging-telenovelas}{%
\section{Less Sex, More Viewers: Pandemic Boosts Mexico's Flagging
Telenovelas}\label{less-sex-more-viewers-pandemic-boosts-mexicos-flagging-telenovelas}}

Dismissed as too old-fashioned to compete with streaming services, the
melodramas have seen viewership surge from Mexicans seeking the familiar
in a distressing time.

\includegraphics{https://static01.nyt.com/images/2020/07/23/world/00mexico-melodrama/merlin_173965713_f45fb63a-5152-433a-b9e9-4cfa786e4780-articleLarge.jpg?quality=75\&auto=webp\&disable=upscale}

\href{https://www.nytimes.com/by/natalie-kitroeff}{\includegraphics{https://static01.nyt.com/images/2019/03/01/multimedia/author-natalie-kitroeff/author-natalie-kitroeff-thumbLarge.png}}

By \href{https://www.nytimes.com/by/natalie-kitroeff}{Natalie Kitroeff}

\begin{itemize}
\item
  Aug. 2, 2020
\item
  \begin{itemize}
  \item
  \item
  \item
  \item
  \item
  \end{itemize}
\end{itemize}

\href{https://www.nytimes.com/es/2020/08/02/espanol/america-latina/televisa-rosa-de-guadalupe-netflix.html}{Leer
en español}

MEXICO CITY --- Mexico's love affair with melodrama was over.

After decades of reigning supreme over prime time slots, telenovelas,
the country's iconic soap operas, were losing viewers. Industry
executives declared them obsolete, too corny and simplistic to compete
with higher-brow, higher-budget shows.

Now, thanks partly to the pandemic, the telenovela is roaring back.

Confined to their homes, millions of Mexicans have devoted their
evenings to the traditional melodramas and other kitschy classics,
finding in the familiar faces and guaranteed happy endings a balm for
anxieties raised by a health crisis that has left at least 43,000 dead
and millions unemployed.

``There's no fear, no horror, no misery,'' said Enrique Millán, 75, of
the telenovelas that claimed his undivided attention after the pandemic
put soccer on pause. ``I can imagine what's going to happen at the end
of each episode. There's no stress.''

\includegraphics{https://static01.nyt.com/images/2020/07/28/world/00mexico-melodrama6/merlin_174535812_5c6ee301-bd88-4ec0-b4dc-48a5b3416a9a-articleLarge.jpg?quality=75\&auto=webp\&disable=upscale}

Ratings for the shows have soared in recent months, reviving a genre
that shaped generations of Mexicans and became one of the nation's most
important cultural exports.

The onset of a global economic downturn has made such programming more
attractive by default. Telenovelas air on broadcast channels, making
them more accessible than Netflix or premium channels for the average
Mexican family.

But their draw also comes from a specific brand of uncomplicated
storytelling that eases the boredom of life in quarantine while calming
fears and delivering the emotional intimacy that daily interactions have
lost to the virus.

``I turn on the television, time goes by and you don't feel like you're
doing nothing,'' said Minerva Becerril, who watches telenovelas and
other melodramas every evening with her 90-year-old mother in her house
on the outskirts of Mexico City. ``It brings a moment of calm and you
watch love scenes, which I like because I'm a romantic.''

Ms. Becerril began her evenings with ``Te Doy La Vida'' (``I Give You
Life''), a novella that features a love triangle, and then turned to
``La Rosa de Guadalupe'' (``The Rose of Guadalupe''), a drama with
religious undertones. She sometimes tunes into ``Destilando Amor''
(``Distilling Love''), but doesn't like ``Rubí,'' a reboot of a 2004
soap based on a short story she read in a comic book from the 1960s.
``The version in the magazine was better,'' she said.

The resurgence of melodramas in Mexico has been a boon to Televisa, a
onetime media monopoly that has taken a beating from streaming services
and other competitors in recent years.

Image

Photos of Televisa's stars line the hallways of its offices. A line down
the center reminds people to keep their distance.Credit...Meghan
Dhaliwal for The New York Times

During the second quarter, 6.6 million people watched Televisa's
flagship channel during prime time each evening, when telenovelas and
other melodramas air, up from around five million during the same period
in 2019, according to the network. Ratings for the channel increased
twice as much as overall TV viewership in Mexico from May to June.

Based on Nielsen ratings, Televisa estimates that more than 10 million
people watched the finale of ``Te Doy La Vida," which became the
most-watched episode of a telenovela on the network since 2016.

\hypertarget{latest-updates-global-coronavirus-outbreak}{%
\section{\texorpdfstring{\href{https://www.nytimes.com/2020/08/04/world/coronavirus-cases.html?action=click\&pgtype=Article\&state=default\&region=MAIN_CONTENT_1\&context=storylines_live_updates}{Latest
Updates: Global Coronavirus
Outbreak}}{Latest Updates: Global Coronavirus Outbreak}}\label{latest-updates-global-coronavirus-outbreak}}

Updated 2020-08-05T07:58:24.076Z

\begin{itemize}
\tightlist
\item
  \href{https://www.nytimes.com/2020/08/04/world/coronavirus-cases.html?action=click\&pgtype=Article\&state=default\&region=MAIN_CONTENT_1\&context=storylines_live_updates\#link-762df92}{As
  talks drag on, McConnell signals openness to jobless aid extension,
  and negotiators agree on a deadline.}
\item
  \href{https://www.nytimes.com/2020/08/04/world/coronavirus-cases.html?action=click\&pgtype=Article\&state=default\&region=MAIN_CONTENT_1\&context=storylines_live_updates\#link-1228a480}{Novavax
  sees encouraging results from two studies of its experimental
  vaccine.}
\item
  \href{https://www.nytimes.com/2020/08/04/world/coronavirus-cases.html?action=click\&pgtype=Article\&state=default\&region=MAIN_CONTENT_1\&context=storylines_live_updates\#link-794484ed}{Mississippians
  must now wear masks in public, governor says.}
\end{itemize}

\href{https://www.nytimes.com/2020/08/04/world/coronavirus-cases.html?action=click\&pgtype=Article\&state=default\&region=MAIN_CONTENT_1\&context=storylines_live_updates}{See
more updates}

More live coverage:
\href{https://www.nytimes.com/live/2020/08/04/business/stock-market-today-coronavirus?action=click\&pgtype=Article\&state=default\&region=MAIN_CONTENT_1\&context=storylines_live_updates}{Markets}

``Suddenly the ratings are going up,'' said Isaac Lee, a former
executive at Televisa and Univision. ``Nobody knows if this is a moment,
a flick, a trend or if the telenovela is back.''

When Mr. Lee became head of content at Televisa in 2017, the network was
in crisis. Incomes had been rising and internet access spreading across
Mexico for decades, luring people away from the signature melodramas
that had been Televisa's bread and butter for half a century.

Industry executives wanted more action, more violence and bigger budgets
--- the ingredients that seemed to explain the success of dramas about
drug traffickers on Telemundo and series like ``Narcos'' on Netflix.

Mr. Lee began binge-watching all of its programming and soon realized
what should have been obvious: He wasn't the target audience. And
neither were the other company executives who had been making decisions
about the shows.

``I decided not to watch the content,'' he said, ``because I knew that I
would screw it up.''

Image

Minerva Becerril, left, watches telenovelas with her mother, Gorgonia
Becerril Rocha, because they offer ``a moment of calm.''Credit...Meghan
Dhaliwal for The New York Times

After many conversations with viewers, it became clear that melodrama
just needed a makeover, he said. Televisa began to modernize its
telenovelas, toning down the face slapping and operatic baritones in
favor of characters who talked in normal voices about real problems.

Their North Star was ``La Rosa de Guadalupe," a decade-old Televisa
drama that had long been underestimated by the network's own executives.

``La Rosa de Guadalupe'' is not a telenovela, with established
characters and conflicts, but it is the pinnacle of melodrama. Each
hourlong episode tells a self-contained story that always follows the
same arc: People encounter problems and pray for help to the Virgin of
Guadalupe. A white rose appears, a saintly wind blows over their faces
and soon their troubles are over.

What the show had that the network's soaps did not was cultural
currency. The themes ``La Rosa de Guadalupe'' addresses are often ripped
from the headlines, like the episode devoted to a family separated by
deportation from the United States, or the one about teenagers who were
consuming liquor by
\href{https://www.ncbi.nlm.nih.gov/pmc/articles/PMC4009175/}{pouring it
into their eye sockets} --- a dangerous prank that was making the rounds
on social media.

The drama was also attracting a surprising following among young
Mexicans --- though many swore that they, unlike their grandmothers,
were watching ironically, to make fun of the far-fetched story lines.
TikTok, Twitter and YouTube are full of memes and videos ridiculing the
show.

Image

A scene from an episode of ``La Rosa de Guadalupe'' in which a character
fights to reunite with her family after being deported from the United
States to Mexico.Credit...Televisa

``We think it's absurd,'' said Héctor Ortega, 22, who created the
Twitter account Out of Context Rosa, where he posts short clips of the
show's most exaggerated moments. ``I don't even watch the program. I
just saw all the memes and the impact that it has on my generation,
which isn't exactly the target market.''

\href{https://www.nytimes.com/news-event/coronavirus?action=click\&pgtype=Article\&state=default\&region=MAIN_CONTENT_3\&context=storylines_faq}{}

\hypertarget{the-coronavirus-outbreak-}{%
\subsubsection{The Coronavirus Outbreak
›}\label{the-coronavirus-outbreak-}}

\hypertarget{frequently-asked-questions}{%
\paragraph{Frequently Asked
Questions}\label{frequently-asked-questions}}

Updated August 4, 2020

\begin{itemize}
\item ~
  \hypertarget{i-have-antibodies-am-i-now-immune}{%
  \paragraph{I have antibodies. Am I now
  immune?}\label{i-have-antibodies-am-i-now-immune}}

  \begin{itemize}
  \tightlist
  \item
    As of right
    now,\href{https://www.nytimes.com/2020/07/22/health/covid-antibodies-herd-immunity.html?action=click\&pgtype=Article\&state=default\&region=MAIN_CONTENT_3\&context=storylines_faq}{that
    seems likely, for at least several months.} There have been
    frightening accounts of people suffering what seems to be a second
    bout of Covid-19. But experts say these patients may have a
    drawn-out course of infection, with the virus taking a slow toll
    weeks to months after initial exposure. People infected with the
    coronavirus typically
    \href{https://www.nature.com/articles/s41586-020-2456-9}{produce}
    immune molecules called antibodies, which are
    \href{https://www.nytimes.com/2020/05/07/health/coronavirus-antibody-prevalence.html?action=click\&pgtype=Article\&state=default\&region=MAIN_CONTENT_3\&context=storylines_faq}{protective
    proteins made in response to an
    infection}\href{https://www.nytimes.com/2020/05/07/health/coronavirus-antibody-prevalence.html?action=click\&pgtype=Article\&state=default\&region=MAIN_CONTENT_3\&context=storylines_faq}{.
    These antibodies may} last in the body
    \href{https://www.nature.com/articles/s41591-020-0965-6}{only two to
    three months}, which may seem worrisome, but that's perfectly normal
    after an acute infection subsides, said Dr. Michael Mina, an
    immunologist at Harvard University. It may be possible to get the
    coronavirus again, but it's highly unlikely that it would be
    possible in a short window of time from initial infection or make
    people sicker the second time.
  \end{itemize}
\item ~
  \hypertarget{im-a-small-business-owner-can-i-get-relief}{%
  \paragraph{I'm a small-business owner. Can I get
  relief?}\label{im-a-small-business-owner-can-i-get-relief}}

  \begin{itemize}
  \tightlist
  \item
    The
    \href{https://www.nytimes.com/article/small-business-loans-stimulus-grants-freelancers-coronavirus.html?action=click\&pgtype=Article\&state=default\&region=MAIN_CONTENT_3\&context=storylines_faq}{stimulus
    bills enacted in March} offer help for the millions of American
    small businesses. Those eligible for aid are businesses and
    nonprofit organizations with fewer than 500 workers, including sole
    proprietorships, independent contractors and freelancers. Some
    larger companies in some industries are also eligible. The help
    being offered, which is being managed by the Small Business
    Administration, includes the Paycheck Protection Program and the
    Economic Injury Disaster Loan program. But lots of folks have
    \href{https://www.nytimes.com/interactive/2020/05/07/business/small-business-loans-coronavirus.html?action=click\&pgtype=Article\&state=default\&region=MAIN_CONTENT_3\&context=storylines_faq}{not
    yet seen payouts.} Even those who have received help are confused:
    The rules are draconian, and some are stuck sitting on
    \href{https://www.nytimes.com/2020/05/02/business/economy/loans-coronavirus-small-business.html?action=click\&pgtype=Article\&state=default\&region=MAIN_CONTENT_3\&context=storylines_faq}{money
    they don't know how to use.} Many small-business owners are getting
    less than they expected or
    \href{https://www.nytimes.com/2020/06/10/business/Small-business-loans-ppp.html?action=click\&pgtype=Article\&state=default\&region=MAIN_CONTENT_3\&context=storylines_faq}{not
    hearing anything at all.}
  \end{itemize}
\item ~
  \hypertarget{what-are-my-rights-if-i-am-worried-about-going-back-to-work}{%
  \paragraph{What are my rights if I am worried about going back to
  work?}\label{what-are-my-rights-if-i-am-worried-about-going-back-to-work}}

  \begin{itemize}
  \tightlist
  \item
    Employers have to provide
    \href{https://www.osha.gov/SLTC/covid-19/standards.html}{a safe
    workplace} with policies that protect everyone equally.
    \href{https://www.nytimes.com/article/coronavirus-money-unemployment.html?action=click\&pgtype=Article\&state=default\&region=MAIN_CONTENT_3\&context=storylines_faq}{And
    if one of your co-workers tests positive for the coronavirus, the
    C.D.C.} has said that
    \href{https://www.cdc.gov/coronavirus/2019-ncov/community/guidance-business-response.html}{employers
    should tell their employees} -\/- without giving you the sick
    employee's name -\/- that they may have been exposed to the virus.
  \end{itemize}
\item ~
  \hypertarget{should-i-refinance-my-mortgage}{%
  \paragraph{Should I refinance my
  mortgage?}\label{should-i-refinance-my-mortgage}}

  \begin{itemize}
  \tightlist
  \item
    \href{https://www.nytimes.com/article/coronavirus-money-unemployment.html?action=click\&pgtype=Article\&state=default\&region=MAIN_CONTENT_3\&context=storylines_faq}{It
    could be a good idea,} because mortgage rates have
    \href{https://www.nytimes.com/2020/07/16/business/mortgage-rates-below-3-percent.html?action=click\&pgtype=Article\&state=default\&region=MAIN_CONTENT_3\&context=storylines_faq}{never
    been lower.} Refinancing requests have pushed mortgage applications
    to some of the highest levels since 2008, so be prepared to get in
    line. But defaults are also up, so if you're thinking about buying a
    home, be aware that some lenders have tightened their standards.
  \end{itemize}
\item ~
  \hypertarget{what-is-school-going-to-look-like-in-september}{%
  \paragraph{What is school going to look like in
  September?}\label{what-is-school-going-to-look-like-in-september}}

  \begin{itemize}
  \tightlist
  \item
    It is unlikely that many schools will return to a normal schedule
    this fall, requiring the grind of
    \href{https://www.nytimes.com/2020/06/05/us/coronavirus-education-lost-learning.html?action=click\&pgtype=Article\&state=default\&region=MAIN_CONTENT_3\&context=storylines_faq}{online
    learning},
    \href{https://www.nytimes.com/2020/05/29/us/coronavirus-child-care-centers.html?action=click\&pgtype=Article\&state=default\&region=MAIN_CONTENT_3\&context=storylines_faq}{makeshift
    child care} and
    \href{https://www.nytimes.com/2020/06/03/business/economy/coronavirus-working-women.html?action=click\&pgtype=Article\&state=default\&region=MAIN_CONTENT_3\&context=storylines_faq}{stunted
    workdays} to continue. California's two largest public school
    districts --- Los Angeles and San Diego --- said on July 13, that
    \href{https://www.nytimes.com/2020/07/13/us/lausd-san-diego-school-reopening.html?action=click\&pgtype=Article\&state=default\&region=MAIN_CONTENT_3\&context=storylines_faq}{instruction
    will be remote-only in the fall}, citing concerns that surging
    coronavirus infections in their areas pose too dire a risk for
    students and teachers. Together, the two districts enroll some
    825,000 students. They are the largest in the country so far to
    abandon plans for even a partial physical return to classrooms when
    they reopen in August. For other districts, the solution won't be an
    all-or-nothing approach.
    \href{https://bioethics.jhu.edu/research-and-outreach/projects/eschool-initiative/school-policy-tracker/}{Many
    systems}, including the nation's largest, New York City, are
    devising
    \href{https://www.nytimes.com/2020/06/26/us/coronavirus-schools-reopen-fall.html?action=click\&pgtype=Article\&state=default\&region=MAIN_CONTENT_3\&context=storylines_faq}{hybrid
    plans} that involve spending some days in classrooms and other days
    online. There's no national policy on this yet, so check with your
    municipal school system regularly to see what is happening in your
    community.
  \end{itemize}
\end{itemize}

Of course, many of the haters turn out to be loyal viewers. ``La Rosa de
Guadalupe'' has seen huge growth in its younger audience in recent
months, especially among male viewers aged 13-31, whose numbers have
increased by about 40 percent compared with last year.

It is unclear, even to Televisa executives, whether the success can last
through a pandemic that has taken physical displays of affection out of
the contact sport that is a telenovela.

``There are no kisses, no hugs, no caresses, no scenes in bed,'' said
Miguel Ángel Herros, the executive producer of ``La Rosa de Guadalupe.''

Any touching is ``hands only, and conversations happen at this
distance,'' he said, gesturing at the roughly 10 feet between his desk
and his assistant.

Mr. Herros, 80, is filming for shorter periods, in locations that leave
ample space for his crew. Actors have their temperatures taken when they
arrive on set and rehearse with masks and face shields. And the network
already had to send one actress, from the soap ``Te Doy La Vida,'' into
quarantine after she tested positive for coronavirus.

Image

Miguel Ángel Herros, executive producer of ``La Rosa de
Guadalupe.''Credit...Meghan Dhaliwal for The New York Times

But Mr. Herros doesn't view the epidemic as a threat. ``La Rosa de
Guadalupe" stopped filming only briefly during the pandemic, on the
orders of the city government, but quickly picked back up.

``I come to the office every day,'' said Mr. Herros, sitting in an
office adorned with religious iconography in the middle of Televisa's
expansive headquarters in San Ángel, just south of Mexico City's center.
``We haven't stopped since March.''

For the time being, at least, Televisa has some advantages over
streamers in Mexico. The company occupies more than a million square
feet in Mexico City, where actors and crews can be kept in tightly
controlled environments to contain the spread of the virus.

And when it comes to dishing comfort food to an anxious audience,
there's no match for the old-fashioned melodrama.

``Unlike Netflix, we give people certainty,'' said Carlos Mercado, the
show's creator and head writer. ``You know what you're going to see on
``La Rosa de Guadalupe,'' even if you want to make fun of it.''

Advertisement

\protect\hyperlink{after-bottom}{Continue reading the main story}

\hypertarget{site-index}{%
\subsection{Site Index}\label{site-index}}

\hypertarget{site-information-navigation}{%
\subsection{Site Information
Navigation}\label{site-information-navigation}}

\begin{itemize}
\tightlist
\item
  \href{https://help.nytimes.com/hc/en-us/articles/115014792127-Copyright-notice}{©~2020~The
  New York Times Company}
\end{itemize}

\begin{itemize}
\tightlist
\item
  \href{https://www.nytco.com/}{NYTCo}
\item
  \href{https://help.nytimes.com/hc/en-us/articles/115015385887-Contact-Us}{Contact
  Us}
\item
  \href{https://www.nytco.com/careers/}{Work with us}
\item
  \href{https://nytmediakit.com/}{Advertise}
\item
  \href{http://www.tbrandstudio.com/}{T Brand Studio}
\item
  \href{https://www.nytimes.com/privacy/cookie-policy\#how-do-i-manage-trackers}{Your
  Ad Choices}
\item
  \href{https://www.nytimes.com/privacy}{Privacy}
\item
  \href{https://help.nytimes.com/hc/en-us/articles/115014893428-Terms-of-service}{Terms
  of Service}
\item
  \href{https://help.nytimes.com/hc/en-us/articles/115014893968-Terms-of-sale}{Terms
  of Sale}
\item
  \href{https://spiderbites.nytimes.com}{Site Map}
\item
  \href{https://help.nytimes.com/hc/en-us}{Help}
\item
  \href{https://www.nytimes.com/subscription?campaignId=37WXW}{Subscriptions}
\end{itemize}
