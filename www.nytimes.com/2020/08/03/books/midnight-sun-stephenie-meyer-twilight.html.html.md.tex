Sections

SEARCH

\protect\hyperlink{site-content}{Skip to
content}\protect\hyperlink{site-index}{Skip to site index}

\href{https://www.nytimes.com/section/books}{Books}

\href{https://myaccount.nytimes.com/auth/login?response_type=cookie\&client_id=vi}{}

\href{https://www.nytimes.com/section/todayspaper}{Today's Paper}

\href{/section/books}{Books}\textbar{}Stephenie Meyer Is Telling
Edward's Story, Even if It Makes Her Anxious

\url{https://nyti.ms/33kG7Mr}

\begin{itemize}
\item
\item
\item
\item
\item
\item
\end{itemize}

Advertisement

\protect\hyperlink{after-top}{Continue reading the main story}

Supported by

\protect\hyperlink{after-sponsor}{Continue reading the main story}

\hypertarget{stephenie-meyer-is-telling-edwards-story-even-if-it-makes-her-anxious}{%
\section{Stephenie Meyer Is Telling Edward's Story, Even if It Makes Her
Anxious}\label{stephenie-meyer-is-telling-edwards-story-even-if-it-makes-her-anxious}}

The best-selling author talks about her latest book, ``Midnight Sun,''
which retells ``Twilight'' from the vampire's perspective. Why now?
``Because I finished it,'' she says.

\includegraphics{https://static01.nyt.com/images/2020/08/03/books/03Meyer/03Meyer-articleLarge.jpg?quality=75\&auto=webp\&disable=upscale}

\href{https://www.nytimes.com/by/concepcion-de-leon}{\includegraphics{https://static01.nyt.com/images/2018/07/16/multimedia/author-concepcion-de-leon/author-concepcion-de-leon-thumbLarge.png}}

By \href{https://www.nytimes.com/by/concepcion-de-leon}{Concepción de
León}

\begin{itemize}
\item
  Aug. 3, 2020
\item
  \begin{itemize}
  \item
  \item
  \item
  \item
  \item
  \item
  \end{itemize}
\end{itemize}

When Stephenie Meyer decided this year
\href{https://www.nytimes.com/2020/05/04/books/stephenie-meyer-midnight-sun-twilight.html}{to
release ``Midnight Sun,''} a retelling of her best-selling ``Twilight''
novel from the vampire's point of view, she thought: ``No one can
possibly care about it anymore.''

She put the book on hold after several chapters leaked online in 2008.
Now, more than a decade later, her legions of fans will finally be able
to read it. She had hoped for a low-key release, but when she announced
the publication date in May, so many of them flocked to her website that
it quickly crashed.

``That's really flattering but also nerve-racking,'' Meyer said in an
interview last month. ``I'm pretty sure people aren't going to get
exactly what they think they're getting. Because of all the time that's
passed, they've built up in their minds what they thought it was going
to be, and so no one can live up to those kinds of expectations.''

The Twilight saga, which follows teenage Bella Swan's romance with
Edward Cullen, a century-old vampire, turned into a multimillion-dollar
brand following the first book's release in 2005, producing five movies
and millions of devotees around the world, many of whom have been
clamoring for ``Midnight Sun.''

\emph{{[} ``Midnight Sun'' is one of our most anticipated books of
August.}
\href{https://www.nytimes.com/2020/07/30/books/new-august-books.html}{\emph{See
the full list}}\emph{. {]}}

\includegraphics{https://static01.nyt.com/images/2020/08/03/books/review/03Meyer2/03Meyer2-articleLarge.jpg?quality=75\&auto=webp\&disable=upscale}

Maren Abercrombie and Emily Mensing, who host the podcast ``Remember
Twilight?,'' are two such fans. ``I feel like we kind of manifested its
release, honestly,'' Mensing said.

After reading the version of the novel that leaked in 2008, Abercrombie
said she was eager for Meyer to release the finished one. ``Bella is
fine, everybody else is just fine, but to me, the most interesting
character in `Twilight' is Edward,'' she said. ``All I ever wanted was
the entire `Midnight Sun.'''

Ahead of its publication, Meyer talked about the stress of releasing a
book during a pandemic, what readers can expect and why they shouldn't
be waiting for another one in Edward's voice. These are edited excerpts
from the conversation.

\textbf{Why did you decide to publish the book now?}

Because I finished it. The reason it wasn't published earlier was
because it was not done, and when I did see the light at the end of the
tunnel --- when I saw that I was actually going to be able to finish it
--- I started the publication process right away, because I knew there
were people who had been waiting really kindly and patiently, but also
anxiously, for quite a while.

And then Covid-19 happens. And so do we put it out still? It became
quickly obvious that there wasn't a real end in sight with Covid. I am
really excited when I have a book to read right now, because there's not
much else that's exciting. I hoped people would feel the same way.

\textbf{What happened back then? Why did you decide to postpone the book
indefinitely?}

I don't know exactly what happened, which is one of the reasons it shook
me. I don't think there was any bad intent. I think people made copies
instead of returning it to me when they had been asked to read it. But
that wasn't as scary. It was when I thought that maybe someone was
reading things on my computer that I was more frightened by it.

And at the time it was hard, because no one wants to have a rough draft
be out there for criticism. You know you can make it better. You're
literally just throwing things out of your brain onto the page at that
point. It was so long ago --- it was a hiccup, I would say.

The real reason the book took so long to write is because this was just
a huge, pain-in-the-butt book to write. With some of my books, it was
like they were writing themselves, and I was just working to keep up
with dictation. That kind of writing is fun and exciting. This was like,
every single word was a struggle.

\textbf{What do readers have to look forward to in this new
installment?}

The things that I enjoy most about it are --- I liked not being the
human being. I like that experience, stripping away your humanity and
getting to be something other.

I think the part that people won't expect is: Edward is a very anxious
character. Writing him made me more anxious, and that's one of the
reasons it was hard to be in that story. His anxiety combined with mine
was potent. He starts off fairly confident, but boy does he get broken
down by the end. Bella really breaks him into pieces. I think he comes
across in ``Twilight'' being very strong and so super sure of himself,
when that never was really actually the case.

Image

From left, Taylor Lautner, Kristen Stewart, Stephenie Meyer and Robert
Pattinson at the premiere of ``The Twilight Saga: Breaking Dawn - Part
2,'' which starred the three actors, in 2012.Credit...Christopher
Polk/Getty Images

\textbf{Without giving away any spoilers, is there anything you can say
about what readers will learn about Edward or what new perspectives
they'll gain on moments in the ``Twilight'' book?}

I mean, it's difficult to spoil this book, because spoiler: Edward falls
in love with Bella. That's all known, so it's difficult to spoil it.

The stuff they're going to get that's new is, like I said, the inhuman
point of view and then the time away. The best parts to write, hands
down, were the times that Bella was not present, and I wasn't locked
into a certain set of dialogues and actions. That was when I felt he
could be more himself.

Some people are going to like some characters more, and they're going to
like some characters less, because not only is he spending time with
them that way, but he is reading their minds all the time. It's a reflex
reaction for him, he can't control it, so you get, not just a picture of
people, you get the full story all the time, which is kind of
overwhelming. I think you get a sense of how overwhelming it would be to
constantly have people's voices in your head.

\textbf{Do you plan to write the whole series through Edward's eyes?}

No. Not at all. This is it for Edward. Writing from his point of view
makes me extra anxious. And the experience of writing this book was not
a super pleasant one. So no, I wouldn't want to do that --- especially
given that ``New Moon'' would just be a nightmare of depression and
emptiness. I think this gives you enough of a sense of what it's like to
be Edward that you could go and look at the other books and you would
know what's going on in his head.

\textbf{A lot has changed in the world since the first book was
published in 2005, including the \#MeToo movement, which has cast a new
light on a lot of our most beloved cultural institutions. Have you
thought about how Bella and Edward's relationship might be perceived
differently today, almost two decades later?}

I've had feedback from the very beginning with people who reacted to
some things and didn't like them at all, which I absolutely can see. I
don't know if ``Midnight Sun'' will make that better or worse for them.

I feel like you get the sense of him from the perspective of him not
being someone who follows human rules. And the worst of it isn't that,
you could say, he spies on her. Really he's just like a very curious
animal who doesn't think of it that way. But really the real problem is
that he's murdered a ton of people --- that's the worst thing, right,
that you're a murderer many times over.

And again, that comes from the fact that this is a fiction book that's
not even set in a realistic world. It's fantasy, and so you have this
character who's not human and who isn't part of the social things that
we do. He's different. That doesn't change the fact that for somebody
who experienced something terrible that this might feel horrible for
them, and that I feel bad about, because for me it's just a fantasy that
doesn't exist. It hasn't been my experience, and so it just feels like
this totally other world.

\textbf{What has the book launch process been like?}

Pretty insane and stressful. I like to have everything planned out and
know in advance, this is what I'm going to have to do, this is where I'm
going to go, and I can plan it for, seriously, six months ahead, and
then I'm happy. Now we don't know what we're doing. We've had plans and
we get excited, and then the plans fall apart. I guess we're going to be
doing a lot of virtual stuff. That's probably fun, but to me it just
doesn't feel like enough. The fans are so excited to do something, to do
anything, and we can't really give them that. That's a little
frustrating.

\textbf{You wrote on}
\textbf{\href{https://stepheniemeyer.com/updates}{your blog}}
\textbf{that books are your main source of escape right now. What have
you been reading?}

The last one that I really loved --- they're pretty short, you'll tear
through them --- her name is Martha Wells, and it's a sci-fi series
called the
\href{https://www.nytimes.com/2020/05/26/books/review/docile-murderbot-otaku-hotspur-shorefall.html}{Murderbot
Diaries}.
``\href{https://www.nytimes.com/2018/11/30/books/review/martha-wells-exit-strategy.html}{All
Systems Red}'' is the first one. It's great. It's about this cyborg who
is neither male nor female, who is supposed to be under control but they
have free agency, but they just use it to watch TV basically, and the
poor murderbot has huge social anxiety, can't look anybody in the face,
just wants to be left alone to watch their shows. I really identified
with the character {[}\emph{laughs}{]}.

\textbf{What do you plan on writing next?}

I have, like, three candidates right now. I work on them occasionally.
When ``Midnight Sun'' is out and that's passed, then I'll see which one
is pulling me in. I'd like to do something in \emph{fantasy} fantasy,
where you have to have a map in the beginning of the book, but we'll see
if that's the one that gets picked.

\emph{Follow New York Times Books on}
\href{https://www.facebook.com/nytbooks/}{\emph{Facebook}}\emph{,}
\href{https://twitter.com/nytimesbooks}{\emph{Twitter}} \emph{and}
\href{https://www.instagram.com/nytbooks/}{\emph{Instagram}}\emph{, sign
up for}
\href{https://www.nytimes.com/newsletters/books-review}{\emph{our
newsletter}} \emph{or}
\href{https://www.nytimes.com/interactive/2017/books/books-calendar.html}{\emph{our
literary calendar}}\emph{. And listen to us on the}
\href{https://www.nytimes.com/column/book-review-podcast}{\emph{Book
Review podcast}}\emph{.}

Advertisement

\protect\hyperlink{after-bottom}{Continue reading the main story}

\hypertarget{site-index}{%
\subsection{Site Index}\label{site-index}}

\hypertarget{site-information-navigation}{%
\subsection{Site Information
Navigation}\label{site-information-navigation}}

\begin{itemize}
\tightlist
\item
  \href{https://help.nytimes.com/hc/en-us/articles/115014792127-Copyright-notice}{©~2020~The
  New York Times Company}
\end{itemize}

\begin{itemize}
\tightlist
\item
  \href{https://www.nytco.com/}{NYTCo}
\item
  \href{https://help.nytimes.com/hc/en-us/articles/115015385887-Contact-Us}{Contact
  Us}
\item
  \href{https://www.nytco.com/careers/}{Work with us}
\item
  \href{https://nytmediakit.com/}{Advertise}
\item
  \href{http://www.tbrandstudio.com/}{T Brand Studio}
\item
  \href{https://www.nytimes.com/privacy/cookie-policy\#how-do-i-manage-trackers}{Your
  Ad Choices}
\item
  \href{https://www.nytimes.com/privacy}{Privacy}
\item
  \href{https://help.nytimes.com/hc/en-us/articles/115014893428-Terms-of-service}{Terms
  of Service}
\item
  \href{https://help.nytimes.com/hc/en-us/articles/115014893968-Terms-of-sale}{Terms
  of Sale}
\item
  \href{https://spiderbites.nytimes.com}{Site Map}
\item
  \href{https://help.nytimes.com/hc/en-us}{Help}
\item
  \href{https://www.nytimes.com/subscription?campaignId=37WXW}{Subscriptions}
\end{itemize}
