Sections

SEARCH

\protect\hyperlink{site-content}{Skip to
content}\protect\hyperlink{site-index}{Skip to site index}

\href{https://www.nytimes.com/section/books/review}{Book Review}

\href{https://myaccount.nytimes.com/auth/login?response_type=cookie\&client_id=vi}{}

\href{https://www.nytimes.com/section/todayspaper}{Today's Paper}

\href{/section/books/review}{Book Review}\textbar{}Welcome to Another
Novel Set in Brooklyn. This One Is Different.

\url{https://nyti.ms/3k6Xg2g}

\begin{itemize}
\item
\item
\item
\item
\item
\end{itemize}

Advertisement

\protect\hyperlink{after-top}{Continue reading the main story}

Supported by

\protect\hyperlink{after-sponsor}{Continue reading the main story}

Fiction

\hypertarget{welcome-to-another-novel-set-in-brooklyn-this-one-is-different}{%
\section{Welcome to Another Novel Set in Brooklyn. This One Is
Different.}\label{welcome-to-another-novel-set-in-brooklyn-this-one-is-different}}

\includegraphics{https://static01.nyt.com/images/2020/07/01/books/review/Waldman1/Waldman1-articleLarge.jpg?quality=75\&auto=webp\&disable=upscale}

Buy Book ▾

\begin{itemize}
\tightlist
\item
  \href{https://www.amazon.com/gp/search?index=books\&tag=NYTBSREV-20\&field-keywords=Kings+County+David+Goodwillie}{Amazon}
\item
  \href{https://du-gae-books-dot-nyt-du-prd.appspot.com/buy?title=Kings+County\&author=David+Goodwillie}{Apple
  Books}
\item
  \href{https://www.anrdoezrs.net/click-7990613-11819508?url=https\%3A\%2F\%2Fwww.barnesandnoble.com\%2Fw\%2F\%3Fean\%3D9781501192135}{Barnes
  and Noble}
\item
  \href{https://www.anrdoezrs.net/click-7990613-35140?url=https\%3A\%2F\%2Fwww.booksamillion.com\%2Fp\%2FKings\%2BCounty\%2FDavid\%2BGoodwillie\%2F9781501192135}{Books-A-Million}
\item
  \href{https://bookshop.org/a/3546/9781501192135}{Bookshop}
\item
  \href{https://www.indiebound.org/book/9781501192135?aff=NYT}{Indiebound}
\end{itemize}

When you purchase an independently reviewed book through our site, we
earn an affiliate commission.

By Adelle Waldman

\begin{itemize}
\item
  Aug. 3, 2020
\item
  \begin{itemize}
  \item
  \item
  \item
  \item
  \item
  \end{itemize}
\end{itemize}

\textbf{KINGS COUNTY}\\
By David Goodwillie

Can a person with a tattoo have a soul? To judge from a broad swath of
contemporary fiction, the answer would seem to be no --- at least if the
tattooed person in question is young, lives in a place like Los Angeles
or Austin or Brooklyn and works in the arts. In that case, the character
is clearly a member of the species
``\href{https://www.nytimes.com/2010/11/14/books/review/Greif-t.html}{hipster},''
almost always written about ironically, portrayed as too vain and
ridiculous to be taken seriously.

It's refreshing, then, that David Goodwillie's very good new novel,
``Kings County,'' depicts such people with genuine, unmitigated sympathy
and good-fellowship, as if, in spite of their fashionable lifestyles,
they are as fully human as anyone else.

His characters either live or lived in the Kings County of the book's
title, a place commonly known as Brooklyn. Specifically, they live (or
lived) in Williamsburg, in the early 2000s and up through the Occupy
Wall Street movement in the fall of 2011. But remarkably enough they are
more concerned about being kind to one another than following the latest
culinary or sartorial trends. (They're mostly too broke for artisanal
anything.) And like the characters at the center of Goodwillie's smart
debut novel,
``\href{https://www.nytimes.com/2010/05/09/books/review/Watrous-t.html}{American
Subversive}'' --- about a disillusioned Manhattan writer who gets
wrapped up with a group of radical environmentalists --- the youngish
people who populate ``Kings County'' are thoughtful and appealing.

At the center of the new book is Audrey, a 32-year-old ``artist
liaison'' for an indie record company. Audrey arrived in New York via
bus from a Florida trailer park. In her early 20s at the time, she came
ostensibly to find work as an actress but really to see the world. Once
in Williamsburg, armed with the first of many waitressing gigs, she
turned out to be less committed to acting than she was to drinking,
smoking, hanging out with her best friend, a fellow waitress named
Sarah, and sleeping around. But Audrey had good taste in music, and she
became well known and well liked enough around the Brooklyn music scene
to land the job --- the ``rock and roll prom queen of the North Side,''
one character calls her.

Her crowd is made up largely of people like her. As a blue-blooded
banker named Chris puts it: ``What was often said of the indie crowd ---
they had hidden trust funds; they were faux-contrarians --- could not be
said of Audrey and Sarah's circle, most of whom balanced multiple jobs
and artistic pursuits with a deft sleight of hand. (And anyway, so what
if someone came from money but wore white bucks or striped jumpsuits or
bangs down past her eyes? Why did limo liberals get such a bad rap when
the alternative was the tedious redundancy of limo conservatism?)''

By the time we meet her, Audrey has sowed her wild oats. She lives with
her boyfriend, Theo, a book editor who was laid off from his publishing
job in the wake of the Great Recession. Like Audrey, Theo does not come
from the moneyed classes. He fled a depressed industrial town in
Massachusetts where his father and brother worked for a long time at the
last remaining vestige of industry, an AT\&T/Lucent plant, until it too
shut down. In high school and then college --- which Theo attended on a
football scholarship --- he fell hard for literature. But his tastes
were not sufficiently commercial for the world of publishing; hence his
failure to thrive and ultimate layoff.

It's been a long time since I've encountered a character like Theo in
contemporary fiction. His bookishness and uncompromising, unabashedly
serious taste make life harder on a practical level, but these qualities
are also treated as something to be respected, even admired, rather than
mocked as snobby or elitist. In his sincerity, Theo is a character more
in the mold of Thomas Wolfe than Tom Wolfe. This is part of what
attracts Audrey. But it's not just her. Everyone agrees: On first
meeting,, Theo may seem``quiet, oafish, socially inept,'' but he is a
good guy, a person of ``substance and deliberation,'' as Chris puts it.

\includegraphics{https://static01.nyt.com/images/2020/07/01/books/review/Waldman2/Waldman2-articleLarge.jpg?quality=75\&auto=webp\&disable=upscale}

Theo and Audrey live in Bushwick --- ``the Edison labs of emerging
style'' --- in a loft building where the stairs have been made
impassable by ``a large, heavily stained couch wedged between the
second- and third-floor landings.'' But they're in love and they're
happy, mostly. Theo's new job, as a literary scout for a film company,
makes him anxious. He has yet to find even one novel fit for adaptation;
he worries about being fired and what that will do to the couple's
already precarious finances, as well as to his self-esteem.

This is the state of things when Audrey learns that an old friend, a
strange but charismatic drug dealer named Fender, may have killed
himself. Theo doesn't know that Audrey and Fender and a few others share
a secret, from the time before she met Theo. The revelation of this
hidden chapter of Audrey's past --- and its present-day consequences, as
Audrey comes to suspect that Fender didn't commit suicide --- becomes
the engine of the novel's plot. It makes for a suspenseful read. After
the first chapter or two, the pages of ``Kings County'' begin to turn
quickly.

But suspense plots have certain requirements, some of which conflict
with or simply crowd out the quieter imperatives of character-driven
fiction. In a mystery novel, for example, the characters' relationships
generally evolve in tandem with the plot, becoming strained as the
mystery ratchets up in intensity and then resolving on cue. ``Kings
County'' hews pretty closely to this formula, wrapping everything up a
little too neatly.

On the other hand, Goodwillie's characters are so likable --- so sincere
in their affections and so decent in their moral decision-making, in
spite of their decadent lifestyles --- that it's hard to begrudge them
their pat resolutions. Even Chris, the banker --- that is, a type of
person less likely to be granted full humanity than a hipster --- turns
out to be somewhat appealing. ``Exasperatingly superficial and
surprisingly genuine,'' as Audrey describes him. When he watches the
Occupy Wall Street protests from his office window, he thinks
endearingly --- without ire --- that no one likes to be reminded of his
own worst qualities. But Chris also proves to be a good friend, even
when it means taking real risks.

Goodwillie is also a stylish writer, smart and witty without being a
show-off. He's great at minor moments, like this one: As Audrey ``tied
her hair up in a knot, her principal tattoo, a western scene rendered in
black ink, became visible on her left shoulder blade.'' I love the
phrase ``principal tattoo'' as well as what it conveys about Audrey.

The tattoo is an image of two cowboys riding into the distance. What it
lacks in originality, it makes up for in size, covering the entire top
half of Audrey's left arm. ``Commitment-wise, it was hard to criticize a
half sleeve,'' Goodwillie observes. The same might be said of Audrey and
Theo generally. It's not their originality or their coolness that makes
them appealing --- it's something else, a willingness to go all-in that
transcends where they live or how they dress.

Advertisement

\protect\hyperlink{after-bottom}{Continue reading the main story}

\hypertarget{site-index}{%
\subsection{Site Index}\label{site-index}}

\hypertarget{site-information-navigation}{%
\subsection{Site Information
Navigation}\label{site-information-navigation}}

\begin{itemize}
\tightlist
\item
  \href{https://help.nytimes.com/hc/en-us/articles/115014792127-Copyright-notice}{©~2020~The
  New York Times Company}
\end{itemize}

\begin{itemize}
\tightlist
\item
  \href{https://www.nytco.com/}{NYTCo}
\item
  \href{https://help.nytimes.com/hc/en-us/articles/115015385887-Contact-Us}{Contact
  Us}
\item
  \href{https://www.nytco.com/careers/}{Work with us}
\item
  \href{https://nytmediakit.com/}{Advertise}
\item
  \href{http://www.tbrandstudio.com/}{T Brand Studio}
\item
  \href{https://www.nytimes.com/privacy/cookie-policy\#how-do-i-manage-trackers}{Your
  Ad Choices}
\item
  \href{https://www.nytimes.com/privacy}{Privacy}
\item
  \href{https://help.nytimes.com/hc/en-us/articles/115014893428-Terms-of-service}{Terms
  of Service}
\item
  \href{https://help.nytimes.com/hc/en-us/articles/115014893968-Terms-of-sale}{Terms
  of Sale}
\item
  \href{https://spiderbites.nytimes.com}{Site Map}
\item
  \href{https://help.nytimes.com/hc/en-us}{Help}
\item
  \href{https://www.nytimes.com/subscription?campaignId=37WXW}{Subscriptions}
\end{itemize}
