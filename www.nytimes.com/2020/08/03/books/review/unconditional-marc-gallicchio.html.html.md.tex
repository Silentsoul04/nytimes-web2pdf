Sections

SEARCH

\protect\hyperlink{site-content}{Skip to
content}\protect\hyperlink{site-index}{Skip to site index}

\href{https://www.nytimes.com/section/books/review}{Book Review}

\href{https://myaccount.nytimes.com/auth/login?response_type=cookie\&client_id=vi}{}

\href{https://www.nytimes.com/section/todayspaper}{Today's Paper}

\href{/section/books/review}{Book Review}\textbar{}Why the U.S. Dropped
Atomic Bombs on Japan

\begin{itemize}
\item
\item
\item
\item
\item
\end{itemize}

Advertisement

\protect\hyperlink{after-top}{Continue reading the main story}

Supported by

\protect\hyperlink{after-sponsor}{Continue reading the main story}

nonfiction

\hypertarget{why-the-us-dropped-atomic-bombs-on-japan}{%
\section{Why the U.S. Dropped Atomic Bombs on
Japan}\label{why-the-us-dropped-atomic-bombs-on-japan}}

\includegraphics{https://static01.nyt.com/images/2020/07/27/books/review/Samuels/merlin_97837742_8ecc8eab-deb9-41c2-990b-cb9479152b9c-articleLarge.jpg?quality=75\&auto=webp\&disable=upscale}

Buy Book ▾

\begin{itemize}
\tightlist
\item
  \href{https://www.amazon.com/gp/search?index=books\&tag=NYTBSREV-20\&field-keywords=Unconditional\%3A+The+Japanese+Surrender+in+World+War+II+Marc+Gallicchio}{Amazon}
\item
  \href{https://du-gae-books-dot-nyt-du-prd.appspot.com/buy?title=Unconditional\%3A+The+Japanese+Surrender+in+World+War+II\&author=Marc+Gallicchio}{Apple
  Books}
\item
  \href{https://www.anrdoezrs.net/click-7990613-11819508?url=https\%3A\%2F\%2Fwww.barnesandnoble.com\%2Fs\%2FUnconditional\%3A+The+Japanese+Surrender+in+World+War+II+Marc+Gallicchio}{Barnes
  and Noble}
\item
  \href{https://www.anrdoezrs.net/click-7990613-35140?url=https\%3A\%2F\%2Fwww.booksamillion.com\%2Fsearch\%3Fquery\%3DUnconditional\%253A\%2BThe\%2BJapanese\%2BSurrender\%2Bin\%2BWorld\%2BWar\%2BII\%2BMarc\%2BGallicchio}{Books-A-Million}
\item
  \href{https://bookshop.org/books?keywords=Unconditional\%3A+The+Japanese+Surrender+in+World+War+II}{Bookshop}
\item
  \href{https://www.indiebound.org/search/book?searchfor=Unconditional\%3A+The+Japanese+Surrender+in+World+War+II+Marc+Gallicchio\&aff=NYT}{Indiebound}
\end{itemize}

When you purchase an independently reviewed book through our site, we
earn an affiliate commission.

By Richard J. Samuels

\begin{itemize}
\item
  Aug. 3, 2020
\item
  \begin{itemize}
  \item
  \item
  \item
  \item
  \item
  \end{itemize}
\end{itemize}

\textbf{UNCONDITIONAL}\\
\textbf{The Japanese Surrender in World War II}\\
By Marc Gallicchio

Every August, newspapers are dotted with stories of Hiroshima and
Nagasaki, accompanied by a well-picked-over --- but never resolved ---
debate over whether atomic bombs were needed to end the Asia-Pacific war
on American terms. What is left to learn 75 years (and with so much
spilled ink) later? For
\href{https://www1.villanova.edu/villanova/artsci/history/facstaff/biodetail.html?mail=marc.gallicchio@villanova.edu\&xsl=bio_long}{Mark
Gallicchio}, the answer is in the domestic politics of the United States
and Japan, which drive a narrative that unwinds less like a debate than
a geopolitical thriller.

``Unconditional'' offers a fresh perspective on how the decision to
insist on ``unconditional surrender'' was not simply a choice between
pressing the Japanese into submission or negotiating an end to the
conflict. It also traces ideological battle lines that remained visible
well into the atomic age as the enemy shifted from Tokyo to Moscow.

President Harry Truman believed unconditional surrender would keep the
Soviet Union involved while reassuring American voters and soldiers that
their sacrifices in a total war would be compensated by total victory.
Disarming enemy militaries was the start; consolidating democracy abroad
was the goal. Only by refusing to deal with dictators could Germany and
Japan be redesigned root to branch.

But Truman faced powerful opposition from the Republican establishment,
including the former president Herbert Hoover and Henry Luce, whose
Time/Life media empire presaged Fox News today. Republicans fought
Truman on two fronts: First, they sought to undo New Deal social and
economic reforms; second, they argued that giving Japan a respectable
way out of the conflict would save lives and, at the same time, block
Soviet ambitions in Asia. Conservatives believed the left in the United
States was more determined to use unconditional surrender to destroy
Japanese feudalism than to confront Soviet ambitions --- future manna
from heaven for postwar redbaiters like Senator Joseph McCarthy.

Gallicchio characterizes conciliatory State Department ``Japan hands''
as dupes of cosmopolitan Japanese who persuaded them that Japan's
emperor was actually a progressive who would help America build a
stable, anti-Communist East Asia. But New Deal Democrats believed these
experts did not know what they did not know about Japan. And prefiguring
neoconservatives of a later era, they insisted that only the deposition
of the emperor --- as part of a full transformation of the country's
political culture --- would usher Japan into a peaceful postwar
community of nations.

The left-wing journalist\href{http://www.ifstone.org}{I.}
\href{http://www.ifstone.org}{F. Stone} joined the fray. He railed
against ``reactionaries'' who he said were determined to stir a red
scare to roll back reform in America, purge progressive officials and
deliver a conditional unconditional surrender to their friends in Tokyo.
Gallicchio, the author of several books of military history, sorts out
these players --- and many others --- with great clarity, noting that
Truman played coyly with both sides as the war shifted decisively in the
Allies' favor.

Convinced that the Japanese would not surrender short of a final,
decisive battle --- or (once the A-bomb was available) a final
incendiary event --- Truman was unwilling to suggest American resolve
was weakening. He used the Potsdam Declaration of July to remind the
Japanese that only more devastation awaited if they held out. He
understood that imperial cooperation would ease the difficult task of
disarming 5.5 million Japanese soldiers --- and he ultimately spared
Hirohito --- but he would not guarantee the emperor's status before the
end of the war.

Japan's leaders felt little urgency. The imperial military had amassed
an astonishing number of troops for a desperate homeland defense, while
politicians fantasized about a Soviet-brokered peace. Lacking a
guarantee of his safety, the emperor supported the effort to reach out
to Moscow and busied himself with protecting sacred relics. Even after
the first A-bomb incinerated Hiroshima, he asked the government to seek
Allied concessions, underscoring Gallicchio's claim that Japanese
officials ``seemed uncertain of what they were doing.''

With the Red Army suddenly deep into Manchuria, Japanese leaders were
weighing evaporating options when the second bomb incinerated Nagasaki.
What had been chimeric was now clearly delusional.

The emperor finally intervened. Overruling his generals, he broadcast a
decree Gallicchio sardonically calls ``almost comically evasive''
because it omitted the words ``surrender'' and ``defeat.'' While many
Japanese were confused and saddened, they accepted the emperor's most
famous edict to ``endure the unendurable.'' Some military officers,
though, committed suicide after a failed mutiny on what has become known
as ``Japan's longest day.''

Gallicchio deftly recounts how debate about Truman's decision persisted
well after the surrender. In Japan, aggressive reforms early in the
occupation were opposed by the same Western-educated Japanese who had
influenced America's Japan hands. These elites were keen on defanging
the Japanese military, but tried to block land, labor and electoral
change.

``Unconditional'' documents how conservatives back home targeted New
Dealers within the occupation as Communist sympathizers and hatched
revisionist histories of Truman's motives, exaggerating the emperor's
antimilitarism. Their revisionism was replaced by a New Left brand in
the 1960s. Truman, some now argued, instigated the Cold War by trying to
intimidate the Soviet Union with America's nuclear might.

In 1995, a half-century after the war, the debate was reignited when
curators at the Smithsonian Institution tried unsuccessfully to use this
account of United States aggression to
\href{https://www.atomicheritage.org/history/controversy-over-enola-gay-exhibition}{frame
an exhibition} in which the Enola Gay, the plane that dropped the A-bomb
on Hiroshima, was the leading artifact. ``Unconditional'' is a sharp
reminder of the power, imperfection and politicization of historical
narrative --- and of the way debates can continue long after history's
witnesses have left the stage.

Advertisement

\protect\hyperlink{after-bottom}{Continue reading the main story}

\hypertarget{site-index}{%
\subsection{Site Index}\label{site-index}}

\hypertarget{site-information-navigation}{%
\subsection{Site Information
Navigation}\label{site-information-navigation}}

\begin{itemize}
\tightlist
\item
  \href{https://help.nytimes.com/hc/en-us/articles/115014792127-Copyright-notice}{©~2020~The
  New York Times Company}
\end{itemize}

\begin{itemize}
\tightlist
\item
  \href{https://www.nytco.com/}{NYTCo}
\item
  \href{https://help.nytimes.com/hc/en-us/articles/115015385887-Contact-Us}{Contact
  Us}
\item
  \href{https://www.nytco.com/careers/}{Work with us}
\item
  \href{https://nytmediakit.com/}{Advertise}
\item
  \href{http://www.tbrandstudio.com/}{T Brand Studio}
\item
  \href{https://www.nytimes.com/privacy/cookie-policy\#how-do-i-manage-trackers}{Your
  Ad Choices}
\item
  \href{https://www.nytimes.com/privacy}{Privacy}
\item
  \href{https://help.nytimes.com/hc/en-us/articles/115014893428-Terms-of-service}{Terms
  of Service}
\item
  \href{https://help.nytimes.com/hc/en-us/articles/115014893968-Terms-of-sale}{Terms
  of Sale}
\item
  \href{https://spiderbites.nytimes.com}{Site Map}
\item
  \href{https://help.nytimes.com/hc/en-us}{Help}
\item
  \href{https://www.nytimes.com/subscription?campaignId=37WXW}{Subscriptions}
\end{itemize}
