Sections

SEARCH

\protect\hyperlink{site-content}{Skip to
content}\protect\hyperlink{site-index}{Skip to site index}

\href{https://www.nytimes.com/section/books/review}{Book Review}

\href{https://myaccount.nytimes.com/auth/login?response_type=cookie\&client_id=vi}{}

\href{https://www.nytimes.com/section/todayspaper}{Today's Paper}

\href{/section/books/review}{Book Review}\textbar{}Morgan Jerkins Heads
Down South in Search of Her Black Identity

\url{https://nyti.ms/31crcBh}

\begin{itemize}
\item
\item
\item
\item
\item
\end{itemize}

Advertisement

\protect\hyperlink{after-top}{Continue reading the main story}

Supported by

\protect\hyperlink{after-sponsor}{Continue reading the main story}

Nonfiction

\hypertarget{morgan-jerkins-heads-down-south-in-search-of-her-black-identity}{%
\section{Morgan Jerkins Heads Down South in Search of Her Black
Identity}\label{morgan-jerkins-heads-down-south-in-search-of-her-black-identity}}

\includegraphics{https://static01.nyt.com/images/2020/07/29/books/review/Hirsch1/Hirsch1-articleLarge.jpg?quality=75\&auto=webp\&disable=upscale}

Buy Book ▾

\begin{itemize}
\tightlist
\item
  \href{https://www.amazon.com/gp/search?index=books\&tag=NYTBSREV-20\&field-keywords=Wandering+in+Strange+Lands\%3A+A+Daughter+of+the+Great+Migration+Reclaims+Her+Roots+Morgan+Jerkins}{Amazon}
\item
  \href{https://du-gae-books-dot-nyt-du-prd.appspot.com/buy?title=Wandering+in+Strange+Lands\%3A+A+Daughter+of+the+Great+Migration+Reclaims+Her+Roots\&author=Morgan+Jerkins}{Apple
  Books}
\item
  \href{https://www.anrdoezrs.net/click-7990613-11819508?url=https\%3A\%2F\%2Fwww.barnesandnoble.com\%2Fw\%2F\%3Fean\%3D9780062873040}{Barnes
  and Noble}
\item
  \href{https://www.anrdoezrs.net/click-7990613-35140?url=https\%3A\%2F\%2Fwww.booksamillion.com\%2Fp\%2FWandering\%2Bin\%2BStrange\%2BLands\%253A\%2BA\%2BDaughter\%2Bof\%2Bthe\%2BGreat\%2BMigration\%2BReclaims\%2BHer\%2BRoots\%2FMorgan\%2BJerkins\%2F9780062873040}{Books-A-Million}
\item
  \href{https://bookshop.org/a/3546/9780062873040}{Bookshop}
\item
  \href{https://www.indiebound.org/book/9780062873040?aff=NYT}{Indiebound}
\end{itemize}

When you purchase an independently reviewed book through our site, we
earn an affiliate commission.

By Afua Hirsch

\begin{itemize}
\item
  Published Aug. 3, 2020Updated Aug. 4, 2020, 6:12 a.m. ET
\item
  \begin{itemize}
  \item
  \item
  \item
  \item
  \item
  \end{itemize}
\end{itemize}

\textbf{WANDERING IN STRANGE LANDS}\\
\textbf{A Daughter of the Great Migration Reclaims Her Roots}\\
By Morgan Jerkins

At times I have wondered how helpful introspection is when it comes to
defining Black identity. As minorities in highly racialized societies,
we have never had the luxury of \emph{not} thinking about our Blackness.

It's whiteness that has enjoyed the toxic combination of being both
weaponized and yet invisible. This --- the obvious reasoning goes --- is
the identity that should command our attention now.

But Morgan Jerkins's latest book, ``Wandering in Strange Lands,'' is a
mesmerizing reminder that this divide between Black and white is a false
binary. On the premise of reconnecting her Northern identity to its
Southern roots, Jerkins embarks on a journey that is anything but
direct, or simple. Instead the story of her personal heritage, and its
erasure within her own family, reveals the reductive power of the white
gaze to flatten the complexities of Black lineage. ``I existed in that
Black-white binary,'' Jerkins writes, ``because it was easier.''

\emph{{[} Read an excerpt from}
\href{https://www.nytimes.com/2020/08/04/books/review/wandering-in-strange-lands-by-morgan-jerkins-an-excerpt.html}{\emph{``Wandering
in Strange Lands.''}} \emph{{]}}

Jerkins divides her heritage up geographically: the Lowcountry of
Georgia and South Carolina, Creole Louisiana, Oklahoma and finally Los
Angeles, revealing their distinct but overlapping phenomena of
enslavement, emancipation, multiculturalism and migration. In each
corner of the country she seeks to unveil her ancestors' secrets with
the help of local historians and activists, who in turn share their own.

Hers is a journey that exists at the crossroads of so much contemporary
analysis of the African-American experience. A backward trail through
Isabel Wilkerson's
``\href{https://www.nytimes.com/2010/08/31/books/31book.html}{The Warmth
of Other Suns}''; the story of white passing in Brit Bennett's novel
``\href{https://www.nytimes.com/2020/05/26/books/review-vanishing-half-brit-bennett.html}{The
Vanishing Half}''; the pain and power of water as it carries Black
people both toward and away from slavery in Ta-Nehisi Coates's
``\href{https://www.nytimes.com/2019/09/24/books/review/water-dancer-ta-nehisi-coates.html}{The
Water Dancer}''; the ingenuity of traditional African rootwork and
healing practices in Jesmyn Ward's
``\href{https://www.nytimes.com/2017/09/05/books/review-sing-unburied-sing-jesmyn-ward.html}{Sing,
Unburied, Sing}.''

Image

The question that hovers over this work is an ancient one: How much
Africa is there still in African American identity?

Like these other masterly recent works, ``Wandering in Strange Lands''
is in many ways a quintessentially American story, one that posits the
South as a motherland where, as Beyoncé recently declared, one's ``roots
ain't watered down.''

Yet it's the African continent --- whose presence is woven subtly
throughout her prose --- that becomes Jerkins's unmistakable ancestral
hinterland. From the cultural and spiritual practices of the
Lowcountry's Gullah Geechee communities, to Creole superstitions whose
traces Jerkins detects even in her New Jersey upbringing, the question
that hovers over this work is an ancient one: How much Africa is there
still in African-American identity?

It's a question Black Americans have and will continue to ask
themselves, but as Jerkins reminds us, the parallels between American
and other Black diasporic experiences are unmistakable. She could have
drawn them out more, though: how the systematic degrading of oral
traditions dislocates memory; the uneasy juxtaposition of non-Western
spirituality and Christianity; the abandonment of local foods according
to Eurocentric notions of nutrition. All of the above have harmed Black
people wherever white colonization took place.

As has the glamorization of plantations, across the American South and
the Caribbean, as synonymous with luxury white housing and tourism,
rather than as sites of crimes against humanity. This is one of the many
profound injustices Jerkins describes powerfully yet accessibly. Her
writing has a light touch as it takes on subjects like land
dispossession, punitive taxation, a lack of public services, and
environmental contamination, blending them seamlessly with the tastes of
couche-couche, chitlins and crawfish étouffée.

The tone of the book feels as meandering as its subject matter, verging
on repetitive at times; but Jerkins herself confesses her task is
Sisyphean. She has a gift for turning circular stories of identity into
something conclusive: a ``disentanglement of Black ethnic identity as it
twists and turns under the powers and laws of white supremacy.''

Her task is also courageous. Jerkins approaches territory that is taboo
even in Black circles: the complexities of caste and colorism within
Creole culture, the denial of Black claims to citizenship in Native
nations, even the fraught question of whether it was possible for sex
between master and slave to be consensual.

Jerkins makes plain that denying space for Black identities in history
is itself a legacy as American as its original sins of racism and
enslavement. By exploring the truth of that past with such integrity,
this memoir enriches our future.

Advertisement

\protect\hyperlink{after-bottom}{Continue reading the main story}

\hypertarget{site-index}{%
\subsection{Site Index}\label{site-index}}

\hypertarget{site-information-navigation}{%
\subsection{Site Information
Navigation}\label{site-information-navigation}}

\begin{itemize}
\tightlist
\item
  \href{https://help.nytimes.com/hc/en-us/articles/115014792127-Copyright-notice}{©~2020~The
  New York Times Company}
\end{itemize}

\begin{itemize}
\tightlist
\item
  \href{https://www.nytco.com/}{NYTCo}
\item
  \href{https://help.nytimes.com/hc/en-us/articles/115015385887-Contact-Us}{Contact
  Us}
\item
  \href{https://www.nytco.com/careers/}{Work with us}
\item
  \href{https://nytmediakit.com/}{Advertise}
\item
  \href{http://www.tbrandstudio.com/}{T Brand Studio}
\item
  \href{https://www.nytimes.com/privacy/cookie-policy\#how-do-i-manage-trackers}{Your
  Ad Choices}
\item
  \href{https://www.nytimes.com/privacy}{Privacy}
\item
  \href{https://help.nytimes.com/hc/en-us/articles/115014893428-Terms-of-service}{Terms
  of Service}
\item
  \href{https://help.nytimes.com/hc/en-us/articles/115014893968-Terms-of-sale}{Terms
  of Sale}
\item
  \href{https://spiderbites.nytimes.com}{Site Map}
\item
  \href{https://help.nytimes.com/hc/en-us}{Help}
\item
  \href{https://www.nytimes.com/subscription?campaignId=37WXW}{Subscriptions}
\end{itemize}
