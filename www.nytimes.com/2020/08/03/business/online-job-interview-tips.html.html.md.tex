Sections

SEARCH

\protect\hyperlink{site-content}{Skip to
content}\protect\hyperlink{site-index}{Skip to site index}

\href{https://www.nytimes.com/section/business}{Business}

\href{https://myaccount.nytimes.com/auth/login?response_type=cookie\&client_id=vi}{}

\href{https://www.nytimes.com/section/todayspaper}{Today's Paper}

\href{/section/business}{Business}\textbar{}How to Ace an Online Job
Interview

\url{https://nyti.ms/2Xm9r17}

\begin{itemize}
\item
\item
\item
\item
\item
\end{itemize}

\href{https://www.nytimes.com/spotlight/at-home?action=click\&pgtype=Article\&state=default\&region=TOP_BANNER\&context=at_home_menu}{At
Home}

\begin{itemize}
\tightlist
\item
  \href{https://www.nytimes.com/2020/08/03/well/family/the-benefits-of-talking-to-strangers.html?action=click\&pgtype=Article\&state=default\&region=TOP_BANNER\&context=at_home_menu}{Talk:
  To Strangers}
\item
  \href{https://www.nytimes.com/2020/08/01/at-home/coronavirus-make-pizza-on-a-grill.html?action=click\&pgtype=Article\&state=default\&region=TOP_BANNER\&context=at_home_menu}{Make:
  Grilled Pizza}
\item
  \href{https://www.nytimes.com/2020/07/31/arts/television/goldbergs-abc-stream.html?action=click\&pgtype=Article\&state=default\&region=TOP_BANNER\&context=at_home_menu}{Watch:
  'The Goldbergs'}
\item
  \href{https://www.nytimes.com/interactive/2020/at-home/even-more-reporters-editors-diaries-lists-recommendations.html?action=click\&pgtype=Article\&state=default\&region=TOP_BANNER\&context=at_home_menu}{Explore:
  Reporters' Google Docs}
\end{itemize}

Advertisement

\protect\hyperlink{after-top}{Continue reading the main story}

Supported by

\protect\hyperlink{after-sponsor}{Continue reading the main story}

\hypertarget{how-to-ace-an-online-job-interview}{%
\section{How to Ace an Online Job
Interview}\label{how-to-ace-an-online-job-interview}}

A handful of classic techniques and some tips unique to the
work-from-home era can help you land that next job.

\includegraphics{https://static01.nyt.com/images/2020/08/01/business/01virus-interview-illo/31virus-interview-illo-articleLarge.jpg?quality=75\&auto=webp\&disable=upscale}

By Julie Weed

\begin{itemize}
\item
  Aug. 3, 2020
\item
  \begin{itemize}
  \item
  \item
  \item
  \item
  \item
  \end{itemize}
\end{itemize}

The in-person job interview went away when offices emptied this spring
because of \href{https://www.nytimes.com/news-event/coronavirus}{the
coronavirus pandemic}. On the plus side, no more flying out to company
headquarters and staying at a hotel, just to spend a day of meetings in
an uncomfortable suit and then heading right back home. On the downside,
common technical snafus and fewer body language clues can make the
online process feel fraught. To successfully make the jump to
\href{https://www.nytimes.com/2020/06/21/business/work-home-coronavirus.html}{team
member} from virtual job seeker, brush up on classic interview
techniques and adapt them to the new world of internet interviews.

\textbf{Research the Company and Your Interviewer}

Interview basics still apply, so start by learning about the company,
delving deeply into its website, related news coverage and employee
reviews like those on Glassdoor or Indeed. Know why you want to work
there, because you are sure to be asked.

To research publicly traded companies, Amelia Ransom, senior director of
engagement and diversity at the tax compliance software company
\href{https://www.avalara.com/us/en/index.html}{Avalara}, suggests
delving into their online 10-K forms, which summarize annual
performance, paying close attention to the key challenges a company is
facing in the ``Risk Factors'' section.

``Connect how hiring you can help them solve those challenges,'' she
said.

Check out your interviewer's LinkedIn profile, to understand his or her
background and perhaps find things in common. Make sure that your own
LinkedIn profile is up to date and that you've asked past managers to
post a recommendation in case your interviewer is checking you out, too.

\textbf{Set the Scene}

For video interviews, make sure your lighting, camera angle, outfit and
background all help you look polished. Best bets for lighting are
sunshine from a window that's facing you, a lamp bouncing light off a
wall that reflects softly, computer screen clip-on lights or an
inexpensive ring light.
\href{https://www.nytimes.com/wirecutter/blog/video-call-lighting-tips/}{The
New York Times Wirecutter site} provides a video with more details.

Place your computer's camera at eye level or slightly above and tilted
down (a stack of books underneath can help). Wear a professional-looking
top that makes you feel confident.

Virtual backgrounds can be tricky, so it's best to find a clean
uncluttered space, with nothing to distract the interviewer. Shut the
door in case someone walks by.

``Do the best with what you have,'' Ms. Ransom said, ``but don't worry
too much about it.''

Recruiters understand the limitations of home-based interviews. ``Don't
beat yourself up'' if your child wanders by looking for a snack or the
dog bursts in, she said. The interviewer is sitting at home ``dealing
with the same things.''

\textbf{Double-Check the Tech}

Technical difficulties are understandable, but do all you can to avoid
them, said Eliot Kaplan, a former vice present of talent acquisition at
Hearst Magazines who is now a career coach. Start by ensuring your
\href{https://www.nytimes.com/wirecutter/blog/make-wi-fi-suck-less-working-from-home/}{Wi-Fi
is as strong and reliable as possible}. That might mean setting up your
video call in the part of your home that gets the best reception, asking
housemates to stay off the network during your interview or even paying
for better Wi-Fi for a few months while you are job hunting.

Make sure your laptop is fully charged. Keep your cellphone by your side
(on ``do not disturb'') with the interviewer's phone number handy in
case you need a backup communication method. Close other apps on your
computer so you are not distracted by pop-ups. Double-check what will be
in sight, because video software programs differ in how they crop web
camera views.

\textbf{Practice Your Answers and Your Presence}

Think ahead about common questions and how you will answer (without
sounding too rehearsed). So-called behavioral questions are in vogue:
asking for examples from your experience, like a time when you overcame
an obstacle, led a team or creatively solved a problem. It's important
to answer concisely and listen closely, especially on a phone interview
because you can't see the interviewer's responses and other visual cues,
said Karen Amatangelo-Block, a talent acquisition executive at a global
hotel company and a private coach. ``You'll definitely lose them after
five to seven minutes.''

Practice your posture as well, Ms. Amatangelo-Block said, because it's
important to communicate that you are engaged in the video conversation
and excited about the opportunity. A tip she learned from newscasters is
to ``sit on the edge of your seat,'' which helps you to sit up straight.
Pull your shoulders back to convey confidence, she said.

Even phone interviews should be conducted this way. ``If you don't think
about your presence,'' Ms. Amatangelo-Block said, ``you'll be more
likely to start slouching, feel less engaged and be more likely to
ramble.''

Set up a video call with a friend to check on setting, posture and to
practice questions.

\textbf{Convey Your Value}

Think of the three things about yourself that you can bring to the job
that are not on your résumé, Ms. Ransom said, and communicate those.
``Maybe you are going for an engineering job but are also a great public
speaker.'' As an interviewer, Ms. Ransom said, she wants to know the
candidate beyond the résumé page and understand ``their motivations and
communication style, their personality: How will they expand the company
culture?''

Some of the qualities that companies have traditionally looked for ---
adaptability, flexibility, showing up as a self-starter and an
independent worker --- are more important than ever in a work-from-home
world in which the boss isn't around to see what you are doing, Mr.
Kaplan said. One way to demonstrate those qualities in the interview is
to talk about what you've done during the pandemic.

``If you've used the extra time at home to pick up a new skill or take
on extra work responsibilities to help out your team, let the recruiter
know,'' he said. If you relearned 10th-grade geometry to help your high
schooler pass a math class, that's impressive, too.

\textbf{Questions for the Interviewer?}

Interviewers often conclude by asking, ``Do you have any questions for
me?'' Let your curiosity shine through and ask something that will help
you decide if the position will be a good match for you, Ms. Ransom
said. ``Asking something like `Tell me how you got to where you are'
feels like a template question'' and won't help your decision-making
process, she said.

\textbf{After You Hang Up}

Always send your interviewer a thank-you email and make it as specific
as possible, mentioning a topic you discussed or something that inspired
you. If you don't have the interviewer's contact information, send the
email to your recruiter and ask her or him to pass it along.

\textbf{Each Experience Helps Prepare You for the Next}

Emily Chang, a recent graduate of Duke University, interviewed with 10
companies by phone or video before recently accepting an offer to work
as a researcher for Rubius Therapeutics, a cell therapy biotechnology
company in Cambridge, Mass. She said she was nervous when she began
interviewing and after each interview would ``think of something that
could have been done better and file it away for the next time.''

Ms. Chang said signing into the interview web link 10 or 15 minutes in
advance to make sure it was working, and to take some time to collect
her thoughts, helped her avoid feeling rushed. She also placed a glass
of water just off camera and set up slips of paper she could glance down
at with notes she had prepared, such as how her skills matched the job
requirements.

After being interrupted a few times, Ms. Chang started letting others in
her household know when she would be interviewing so they would be
quiet. Specifically, ``I had to ask my dad not to play the piano,'' she
said.

Advertisement

\protect\hyperlink{after-bottom}{Continue reading the main story}

\hypertarget{site-index}{%
\subsection{Site Index}\label{site-index}}

\hypertarget{site-information-navigation}{%
\subsection{Site Information
Navigation}\label{site-information-navigation}}

\begin{itemize}
\tightlist
\item
  \href{https://help.nytimes.com/hc/en-us/articles/115014792127-Copyright-notice}{©~2020~The
  New York Times Company}
\end{itemize}

\begin{itemize}
\tightlist
\item
  \href{https://www.nytco.com/}{NYTCo}
\item
  \href{https://help.nytimes.com/hc/en-us/articles/115015385887-Contact-Us}{Contact
  Us}
\item
  \href{https://www.nytco.com/careers/}{Work with us}
\item
  \href{https://nytmediakit.com/}{Advertise}
\item
  \href{http://www.tbrandstudio.com/}{T Brand Studio}
\item
  \href{https://www.nytimes.com/privacy/cookie-policy\#how-do-i-manage-trackers}{Your
  Ad Choices}
\item
  \href{https://www.nytimes.com/privacy}{Privacy}
\item
  \href{https://help.nytimes.com/hc/en-us/articles/115014893428-Terms-of-service}{Terms
  of Service}
\item
  \href{https://help.nytimes.com/hc/en-us/articles/115014893968-Terms-of-sale}{Terms
  of Sale}
\item
  \href{https://spiderbites.nytimes.com}{Site Map}
\item
  \href{https://help.nytimes.com/hc/en-us}{Help}
\item
  \href{https://www.nytimes.com/subscription?campaignId=37WXW}{Subscriptions}
\end{itemize}
