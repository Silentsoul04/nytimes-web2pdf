Sections

SEARCH

\protect\hyperlink{site-content}{Skip to
content}\protect\hyperlink{site-index}{Skip to site index}

\href{https://www.nytimes.com/section/nyregion}{New York}

\href{https://myaccount.nytimes.com/auth/login?response_type=cookie\&client_id=vi}{}

\href{https://www.nytimes.com/section/todayspaper}{Today's Paper}

\href{/section/nyregion}{New York}\textbar{}Facebook Bets Big on Future
of N.Y.C., and Offices, With New Lease

\url{https://nyti.ms/39Uy4XL}

\begin{itemize}
\item
\item
\item
\item
\item
\end{itemize}

Advertisement

\protect\hyperlink{after-top}{Continue reading the main story}

Supported by

\protect\hyperlink{after-sponsor}{Continue reading the main story}

\hypertarget{facebook-bets-big-on-future-of-nyc-and-offices-with-new-lease}{%
\section{Facebook Bets Big on Future of N.Y.C., and Offices, With New
Lease}\label{facebook-bets-big-on-future-of-nyc-and-offices-with-new-lease}}

Despite the pandemic, the social media giant leased all the office space
in the former main post office at Penn Station in Midtown.

\includegraphics{https://static01.nyt.com/images/2020/08/03/nyregion/03nyfacebook-1/03nyfacebook-1-articleLarge.jpg?quality=75\&auto=webp\&disable=upscale}

\href{https://www.nytimes.com/by/matthew-haag}{\includegraphics{https://static01.nyt.com/images/2018/06/14/multimedia/author-matthew-haag/author-matthew-haag-thumbLarge.jpg}}

By \href{https://www.nytimes.com/by/matthew-haag}{Matthew Haag}

\begin{itemize}
\item
  Aug. 3, 2020
\item
  \begin{itemize}
  \item
  \item
  \item
  \item
  \item
  \end{itemize}
\end{itemize}

Facebook on Monday agreed to lease all the office space in the mammoth
107-year-old James A. Farley Building in Midtown Manhattan, cementing
New York City as a growing global technology hub and reaffirming a major
corporation's commitment to an office-centric urban culture despite the
pandemic.

With the 730,000-square-foot lease, Facebook
\href{https://www.nytimes.com/2020/01/05/nyregion/nyc-tech-facebook-amazon-google.html}{has
acquired more than 2.2 million square feet} of office space in the city
for thousands of employees in less than a year, all of it on Manhattan's
West Side between Pennsylvania Station and the Hudson River.

Apple, Amazon and Google all lease space in the same area, an emerging
tech corridor.

The timing of the deal's announcement was somewhat of a surprise because
Facebook, which had expressed interest in the Farley Building for
months, has given most of its employees the
\href{https://www.nytimes.com/2020/05/21/technology/facebook-remote-work-coronavirus.html}{option
of working from home} during the pandemic. Even after the pandemic
subsides, Facebook has said that within the next 10 years up to half of
its roughly 52,200 employees across the country would work from home.

New York's economy has been cratered by the outbreak, and even as the
virus has been contained and the city is slowly reopening, many
companies have told their employees not to return to their offices until
early next year if not later.

Much of Manhattan's business district remains a virtual ghost town with
only a small fraction of workers filling office towers.

But Facebook has more than 4,000 employees in its offices in Manhattan
now, up from about 2,900 employees at the beginning of the year.

The company's new office spaces in Manhattan, at the Farley Building and
further west at Hudson Yards, could allow Facebook to move another 8,500
workers to the city. The deal at Hudson Yards, signed late last year,
includes 1.5 million square feet in three buildings.

``Vornado's and Facebook's investment in New York and commitment to
further putting down roots here --- even in the midst of a global
pandemic --- is a signal to the world that our brightest days are still
ahead and we are open for business,'' said Gov. Andrew M. Cuomo in a
statement. ``This public-private partnership fortifies New York as an
international center of innovation.''

A Facebook spokeswoman said it was too soon to estimate how many
employees will end up at the Manhattan properties, given the
uncertainties of the outbreak.

``Facebook first joined New York's vibrant business and tech community
in 2007,'' the spokeswoman, Jamila Reeves, said. ``Since that time,
we've continuously grown and expanded our presence throughout the city.
The Farley Building will further anchor our New York footprint and
create a dedicated hub for our tech and engineering teams.''

The Farley Building, most of which was built in 1913, is on Eighth
Avenue across from Penn Station and Madison Square Garden. A
long-awaited, large-scale renovation of the building is expected to be
completed by the end of the year.

Over the past two years, the rapid growth of technology firms, both
those from the West Coast and start-ups in the city, has turned a broad
area of Manhattan into a vibrant tech hub.

Late last year, Amazon, which has continued to expand despite backing
out of a plan to build a massive campus in Long Island City, Queens, in
the face of strong community opposition, added 350,000 square feet in a
building on 10th Avenue near Hudson Yards. It is enough space to bring
its work force in New York City to more than 8,000 people.

Just south along the Hudson River, Google has built an enormous campus
that spreads across several buildings in the Chelsea neighborhood.

Just before the pandemic, Apple signed a lease for 220,000 square feet
at 11 Penn Plaza, a 1923 Art Deco tower a block from the Farley Building
that is also owned by Vornado. It was Apple's first expansion in New
York City outside its office in the Flatiron district.

Julie Samuels, the executive director of Tech: NYC, a nonprofit industry
group, said that Facebook's decision on Monday was a vote of confidence
in the future of New York and its growing tech industry.

``It's great news that affirms what we've always known: even facing
economic uncertainty and a global pandemic,'' Ms. Samuels said, ``New
York is overflowing with the creativity and potential that will drive
the growth of the next generation of technology companies.''

Advertisement

\protect\hyperlink{after-bottom}{Continue reading the main story}

\hypertarget{site-index}{%
\subsection{Site Index}\label{site-index}}

\hypertarget{site-information-navigation}{%
\subsection{Site Information
Navigation}\label{site-information-navigation}}

\begin{itemize}
\tightlist
\item
  \href{https://help.nytimes.com/hc/en-us/articles/115014792127-Copyright-notice}{©~2020~The
  New York Times Company}
\end{itemize}

\begin{itemize}
\tightlist
\item
  \href{https://www.nytco.com/}{NYTCo}
\item
  \href{https://help.nytimes.com/hc/en-us/articles/115015385887-Contact-Us}{Contact
  Us}
\item
  \href{https://www.nytco.com/careers/}{Work with us}
\item
  \href{https://nytmediakit.com/}{Advertise}
\item
  \href{http://www.tbrandstudio.com/}{T Brand Studio}
\item
  \href{https://www.nytimes.com/privacy/cookie-policy\#how-do-i-manage-trackers}{Your
  Ad Choices}
\item
  \href{https://www.nytimes.com/privacy}{Privacy}
\item
  \href{https://help.nytimes.com/hc/en-us/articles/115014893428-Terms-of-service}{Terms
  of Service}
\item
  \href{https://help.nytimes.com/hc/en-us/articles/115014893968-Terms-of-sale}{Terms
  of Sale}
\item
  \href{https://spiderbites.nytimes.com}{Site Map}
\item
  \href{https://help.nytimes.com/hc/en-us}{Help}
\item
  \href{https://www.nytimes.com/subscription?campaignId=37WXW}{Subscriptions}
\end{itemize}
