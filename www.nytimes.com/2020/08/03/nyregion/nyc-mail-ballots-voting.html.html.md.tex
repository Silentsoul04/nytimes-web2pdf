Sections

SEARCH

\protect\hyperlink{site-content}{Skip to
content}\protect\hyperlink{site-index}{Skip to site index}

\href{https://www.nytimes.com/section/nyregion}{New York}

\href{https://myaccount.nytimes.com/auth/login?response_type=cookie\&client_id=vi}{}

\href{https://www.nytimes.com/section/todayspaper}{Today's Paper}

\href{/section/nyregion}{New York}\textbar{}Why the Botched N.Y.C.
Primary Has Become the November Nightmare

\url{https://nyti.ms/2XlVlgx}

\begin{itemize}
\item
\item
\item
\item
\item
\item
\end{itemize}

\begin{itemize}
\item
  \href{https://www.nytimes.com/interactive/2020/08/04/us/elections/results-arizona-kansas-michigan-missouri-primaries.html?action=click\&pgtype=Article\&state=default\&region=TOP_BANNER\&context=storylines_menu}{Latest
  Results}
\item
  \href{https://www.nytimes.com/article/biden-vice-president-2020.html?action=click\&pgtype=Article\&state=default\&region=TOP_BANNER\&context=storylines_menu}{Biden's
  V.P. Search}
\item
  \href{https://www.nytimes.com/interactive/2020/07/24/us/politics/trump-biden-campaign-donors.html?action=click\&pgtype=Article\&state=default\&region=TOP_BANNER\&context=storylines_menu}{Map
  of Donations}
\item
  \href{https://www.nytimes.com/interactive/2020/us/elections/delegate-count-primary-results.html?action=click\&pgtype=Article\&state=default\&region=TOP_BANNER\&context=storylines_menu}{Delegate
  Count}
\item
  \href{https://www.nytimes.com/interactive/2019/us/politics/2020-presidential-candidates.html?action=click\&pgtype=Article\&state=default\&region=TOP_BANNER\&context=storylines_menu}{The
  Candidates}
\item
  \href{https://www.nytimes.com/newsletters/politics?action=click\&pgtype=Article\&state=default\&region=TOP_BANNER\&context=storylines_menu}{Politics
  Newsletter}
\end{itemize}

Advertisement

\protect\hyperlink{after-top}{Continue reading the main story}

Supported by

\protect\hyperlink{after-sponsor}{Continue reading the main story}

\hypertarget{why-the-botched-nyc-primary-has-become-the-november-nightmare}{%
\section{Why the Botched N.Y.C. Primary Has Become the November
Nightmare}\label{why-the-botched-nyc-primary-has-become-the-november-nightmare}}

Nearly six weeks later, two congressional races remain undecided, and
officials are trading blame over the mishandling of tens of thousands of
mail-in ballots.

\includegraphics{https://static01.nyt.com/images/2020/07/31/nyregion/00nyvoting-02/00nyvoting-02-articleLarge.jpg?quality=75\&auto=webp\&disable=upscale}

\href{https://www.nytimes.com/by/jesse-mckinley}{\includegraphics{https://static01.nyt.com/images/2018/02/20/multimedia/author-jesse-mckinley/author-jesse-mckinley-thumbLarge.jpg}}

By \href{https://www.nytimes.com/by/jesse-mckinley}{Jesse McKinley}

\begin{itemize}
\item
  Published Aug. 3, 2020Updated Aug. 4, 2020
\item
  \begin{itemize}
  \item
  \item
  \item
  \item
  \item
  \item
  \end{itemize}
\end{itemize}

Election officials in New York City widely distributed mail-in ballots
for the primary on June 23, which featured dozens of hard-fought races.
The officials had hoped to make voting much easier, but they did not
seem prepared for the response: more than 10 times the number of
absentee ballots received in recent elections in the city.

Now, nearly six weeks later, two closely watched congressional races
remain undecided, and major delays in counting a deluge of 400,000
\href{https://www.nytimes.com/2020/08/03/us/politics/trump-mail-in-voting.html}{mail-in
ballots} and other problems are being cited as examples of the
challenges facing the nation as it looks toward conducting the November
general election during the pandemic.

Gov. Andrew M. Cuomo and other officials are trading blame for the
botched counting in the city, and the Postal Service is coming under
criticism over whether it is equipped to handle the sharp increase in
absentee ballots.

Election lawyers said one area of concern in New York City was that
mail-in ballots have prepaid return envelopes. The Postal Service
apparently had difficulty processing some of them correctly and, as a
result, an unknown number of votes --- perhaps thousands --- may have
been wrongfully disqualified because of a lack of a postmark.

Thousands more ballots in the city were discarded by election officials
for minor errors, or not even sent to voters until the day before the
primary, making it all but impossible for the ballots to be returned in
time.

In recent days, President Trump has also jumped into the fray,
repeatedly citing the primary in New York City for his unfounded claims
that mail-in voting is susceptible to fraud. There is no evidence that
the primary results were tainted by criminal malfeasance, according to a
wide array of election officials and representatives of campaigns.

Still, candidates and political analysts are warning that government
officials at all levels need to take urgent action to avoid a nightmare
in November.

``This election is a canary in the coal mine,'' said Suraj Patel, a
Democrat running for Congress in a district that includes parts of
Manhattan, Brooklyn and Queens, who has filed a federal lawsuit over the
primary.

Mr. Patel trails the incumbent, Representative Carolyn B. Maloney, by
some 3,700 votes, though more than 12,000 ballots have been
disqualified, including about 1,200 that were missing postmarks, he
said.

He is among the plaintiffs in a lawsuit filed in July that is asking a
federal court to order election officials to count disqualified ballots.
The lawsuit included testimony that election officials had mailed out
more than 34,000 ballots one day before the June 23 primary.

\hypertarget{latest-updates-2020-election}{%
\section{\texorpdfstring{\href{https://www.nytimes.com/2020/08/04/us/elections/primary-election-michigan-arizona-kansas.html?action=click\&pgtype=Article\&state=default\&region=MAIN_CONTENT_1\&context=storylines_live_updates}{Latest
Updates: 2020
Election}}{Latest Updates: 2020 Election}}\label{latest-updates-2020-election}}

Updated 2020-08-05T03:23:56.561Z

\begin{itemize}
\tightlist
\item
  \href{https://www.nytimes.com/2020/08/04/us/elections/primary-election-michigan-arizona-kansas.html?action=click\&pgtype=Article\&state=default\&region=MAIN_CONTENT_1\&context=storylines_live_updates\#link-3924dd44}{Two
  G.O.P. Senate primaries offer --- what else? --- a test of loyalty to
  Trump.}
\item
  \href{https://www.nytimes.com/2020/08/04/us/elections/primary-election-michigan-arizona-kansas.html?action=click\&pgtype=Article\&state=default\&region=MAIN_CONTENT_1\&context=storylines_live_updates\#link-62a8e06b}{The
  military-style uniforms of federal agents who responded to the unrest
  in Portland will be replaced.}
\item
  \href{https://www.nytimes.com/2020/08/04/us/elections/primary-election-michigan-arizona-kansas.html?action=click\&pgtype=Article\&state=default\&region=MAIN_CONTENT_1\&context=storylines_live_updates\#link-32b39e33}{President
  Trump is suddenly a big supporter of mail-in voting --- in Florida.}
\end{itemize}

\href{https://www.nytimes.com/2020/08/04/us/elections/primary-election-michigan-arizona-kansas.html?action=click\&pgtype=Article\&state=default\&region=MAIN_CONTENT_1\&context=storylines_live_updates}{See
more updates}

A winner has also not been declared in a congressional district in the
Bronx, where Ritchie Torres, a Democratic city councilman, holds a
comfortable lead over several other contenders.

Other states and localities had vote-by-mail primaries during the
pandemic, with some scattered reports of problems --- though nothing on
the scale of New York City's weekslong process. Even before the
outbreak, the city's Board of Elections had a reputation as a troubled
agency that ran elections
\href{https://www.nytimes.com/2018/11/07/nyregion/voting-problems-nyc-.html}{rife
with problems}.

New York City election officials insisted last week that
\href{https://www.nytimes.com/2020/07/17/nyregion/election-absentee-ballots-primary.html}{they
were doing their best} under the extraordinary circumstances.

They pointed out the difficulties in protecting election workers from
the coronavirus, and cited
\href{https://www.nysenate.gov/legislation/laws/ELN/9-209}{state laws}
requiring the disqualification of ballots for various small errors ---
including missing signatures on ballot envelopes or envelopes sealed
with tape --- for contributing to the high number of invalidated
ballots.

Election officials also said the changing plans for the state's
presidential primary --- it was
\href{https://www.nytimes.com/2020/04/27/us/politics/democratic-primary-canceled-coronavirus.html}{initially
canceled} before being reinstated by the courts --- had delayed the
process of sending out absentee ballots.

The city's Board of Elections is not expected to certify the vote until
Tuesday. Thirteen weeks later, on Nov. 3, the state and city could face
another crush of absentee ballots.

Frederic M. Umane, the board's secretary, defended the handling of the
election, calling the ballot-counting process ``slow, but accurate and
open.''

He said the board's operation was greatly affected by the outbreak.

``Our staff was decimated by Covid,'' Mr. Umane said, noting the board
had about 350 permanent workers and other temporary workers, many of
whom were sick and could not work before the primary.

Mr. Umane said the board might need hundreds more workers for the
November election.

Primaries were conducted across the state, but New York City seemed to
encounter the biggest problems, in part because it had many closely
contested races and substantial voter participation.

Mr. Cuomo, a third-term Democrat, acknowledged last week that the
primary was flawed, likening mail-in voting to other ``systems that we
were working on but were not ready,'' such as remote learning and
telemedicine, and suggesting the problem lay at a local level.

``We did have --- not we --- boards of elections had operational issues,
some better, some worse, and they have to learn from them,'' Mr. Cuomo
said. ``And we want to get the lessons and make the system better and
make it better for November.''

A person familiar with the internal operation of the city's Board of
Elections, but not authorized to speak on the record, said that having
to increase the number of mail-in ballots had caused enormous struggles
at the agency.

``Imagine saying, `I'm having a dinner party for 10 people,' and then
they say, `No, it's 100 people,''' the person said. ``It's a very deep
learning curve.''

The person added that the board made missteps along the way, including
not hiring enough people to count the absentee ballots. Even the vendors
hired to produce the ballots seemed overwhelmed.

In comments on Saturday, Mr. Cuomo said his administration had offered
help to local election boards, including ``personnel to do counting,''
though no boards seemed to take the state up on its offer. He also noted
that some boards did not start counting ballots until the second week of
July. ``Well, what was that?'' he said.

Mr. Trump has repeatedly referred to the New York primary over the last
two weeks, warning that the ``same thing would happen, but on massive
scale'' across the country on Nov. 3.

The president returned to the topic on Thursday as means of justifying
his suggestion that the general election might need to be postponed, a
trial balloon that was
\href{https://www.nytimes.com/2020/07/31/us/politics/trump-tweet-democracy.html}{widely
panned by even his fellow Republicans}.

New York State lawmakers said they had responded to the problems last
month by approving a roster of fixes
\href{https://www.nysenate.gov/newsroom/press-releases/senate-majority-advances-automatic-voter-registration-system-strengthens}{to
the vote-by-mail system}, though it was not clear if Mr. Cuomo would
sign the bills. Among other changes, the legislation would allow the
counting of ballots received shortly after the election
\href{https://www.nysenate.gov/legislation/bills/2019/s8799/amendment/a}{without
postmarks} and would require officials to
\href{https://www.nysenate.gov/legislation/bills/2019/s8370/amendment/b}{notify
voters of small errors} in their ballot envelopes.

Election experts pointed to an array of causes for the issues in the
primary: In late April, as the toll from the coronavirus mounted,
\href{https://www.governor.ny.gov/news/amid-ongoing-covid-19-pandemic-governor-cuomo-issues-executive-order-make-sure-every-new-yorker}{Mr.
Cuomo ordered a wide expansion of absentee voting}, sending every New
Yorker eligible to vote in the primary an application for an absentee
ballot.

\includegraphics{https://static01.nyt.com/images/2020/07/31/nyregion/00nyvoting4/00nyvoting4-articleLarge.jpg?quality=75\&auto=webp\&disable=upscale}

While the intention may have been to encourage voting, the
infrastructure lagged: Until a wave of changes approved in 2019, New
York
\href{https://www.nytimes.com/2019/01/10/nyregion/voting-reform-election-ny.html}{had
been behind other states in adopting measures like early voting}.

``The state has long had some of the strictest rules when it comes to
being able to cast an absentee ballot, and it wasn't built to support
the increased volume,'' said Lawrence Norden, director of the Election
Reform Program at the Brennan Center for Justice.

The counting of absentee ballots is more labor intensive than machine
counts of in-person votes, which in the past had made up more than 90
percent of New York's election returns. Jerry H. Goldfeder, a veteran
election lawyer, said the board did not have enough money to hire
workers to process absentee ballots.

``They could have asked for money and hired more staff, because they
knew in advance they were going to get an avalanche of absentee
ballots,'' Mr. Goldfeder said. ``There's nothing magical about that.''

In all, the New York City Board of Elections sent more than 750,000
ballots with prepaid return envelopes, and some 400,000 were mailed
back. Postage on prepaid envelopes costs less than a stamp and is
charged to the payer only if used.

Prepaid envelopes are not typically postmarked by the post office's
sorting systems, though the Postal Service recommends that ballot
envelopes use a special bar code to help identify them. Officials say
they make every effort to identify ballots and assure a postmark, a
critical element in determining if ballots were sent by the Election Day
deadline.

That means going so far as to use human ``gatekeepers'' to backstop the
Postal Service's massive computerized sorting systems, who pull ballot
envelopes out, one at a time, and feed them through a cancellation
machine to assure a postmark.

Image

The city's Board of Elections sent over 750,000 mail-in ballots, and
about 400,000 were returned.Credit...Victor J. Blue for The New York
Times

But it was far from foolproof: Michael Calabrese, a manager at the
Postal Service's
\href{https://www.uspsoig.gov/document/new-york-morgan-processing-and-distribution-center-efficiency}{Manhattan
processing plant}, could only confirm that extra gatekeepers were on
hand to locate stray ballots on Election Day itself.

Even so, some unmarked ballots got through.

``It's not a 100 percent process,'' he said, under questioning from
Judge Analisa Torres of Federal District Court in Manhattan. ``It's not
something we could normally do, but in order to capture and read those
ballots, we did that.''

The postal agency defended its performance, but also acknowledged that
``some ballots may not have been postmarked.'' It said it would take
``action to resolve the issue going forward.''

``We continue to work with the secretary of state and all New York
boards of election and look forward to a successful general election,''
Xavier C. Hernandez, a Postal Service spokesman, said.

Bruce Gyory, a Democratic political consultant, said the state and city
needed to drastically increase election staff for November. ``This is
logistics,'' Mr. Gyory said. ``It isn't rocket science.''

He added that such steps could make it more difficult for Mr. Trump to
cite problems in New York to dispute the results of the general
election.

``He is trying to create doubt,'' Mr. Gyory said. ``Because he knows
he's going to lose the election if things don't change.''

Jeffery C. Mays, Luis Ferré-Sadurní and Emma G. Fitzsimmons contributed
reporting.

\hypertarget{our-2020-election-guide}{%
\section{Our 2020 Election Guide}\label{our-2020-election-guide}}

Updated Aug. 4, 2020

\begin{itemize}
\item
  \begin{center}\rule{0.5\linewidth}{\linethickness}\end{center}

  \hypertarget{the-latest}{%
  \subsection{The Latest}\label{the-latest}}

  \begin{itemize}
  \tightlist
  \item
    Kris Kobach, a polarizing figure in Kansas politics,
    \href{https://www.nytimes.com/2020/08/04/us/politics/kobach-tlaib.html?action=click\&pgtype=Article\&state=default\&region=BELOW_MAIN_CONTENT\&context=storylines_guide}{lost
    the Senate primary there}, relieving G.O.P. officials who feared he
    could jeopardize a safe seat.
  \end{itemize}
\item
  \begin{center}\rule{0.5\linewidth}{\linethickness}\end{center}

  \hypertarget{bidens-vp-search}{%
  \subsection{Biden's V.P. Search}\label{bidens-vp-search}}

  \begin{itemize}
  \tightlist
  \item
    \href{https://www.nytimes.com/article/biden-vice-president-2020.html?action=click\&pgtype=Article\&state=default\&region=BELOW_MAIN_CONTENT\&context=storylines_guide}{Here
    are 13 women} who have been under consideration to be Joe Biden's
    running mate, and why each might be chosen --- and might not be.
  \end{itemize}
\item
  \begin{center}\rule{0.5\linewidth}{\linethickness}\end{center}

  \hypertarget{keep-up-with-our-coverage}{%
  \subsection{Keep Up With Our
  Coverage}\label{keep-up-with-our-coverage}}

  \begin{itemize}
  \tightlist
  \item
    Get an
    \href{https://www.nytimes.com/newsletters/politics?action=click\&pgtype=Article\&state=default\&region=BELOW_MAIN_CONTENT\&context=storylines_guide}{email}
    recapping the day's news
  \end{itemize}

  \begin{itemize}
  \tightlist
  \item
    Download our mobile app on
    \href{https://apps.apple.com/us/app/nytimes/id284862083?ls=1\&mat_click_id=5c79ae7455014fd1bd66b5610c05b8f2-20191112-16948\&referrer=mat_click_id\%3D5c79ae7455014fd1bd66b5610c05b8f2-20191112-16948\%26link_click_id\%3D722930677036718082}{iOS}
    and
    \href{http://a.localytics.com/android?id=com.nytimes.android\&referrer=utm_source\%3Dother_nyt_mobile_web\%26utm_medium\%3DWeb\%2520page\%26utm_term\%3DGeneral\%2520Mobile\%2520Page\%26utm_campaign\%3DNYT\%2520Mobile\%2520General\%2520Page}{Android}
    and turn on Breaking News and Politics alerts
  \end{itemize}
\end{itemize}

Advertisement

\protect\hyperlink{after-bottom}{Continue reading the main story}

\hypertarget{site-index}{%
\subsection{Site Index}\label{site-index}}

\hypertarget{site-information-navigation}{%
\subsection{Site Information
Navigation}\label{site-information-navigation}}

\begin{itemize}
\tightlist
\item
  \href{https://help.nytimes.com/hc/en-us/articles/115014792127-Copyright-notice}{©~2020~The
  New York Times Company}
\end{itemize}

\begin{itemize}
\tightlist
\item
  \href{https://www.nytco.com/}{NYTCo}
\item
  \href{https://help.nytimes.com/hc/en-us/articles/115015385887-Contact-Us}{Contact
  Us}
\item
  \href{https://www.nytco.com/careers/}{Work with us}
\item
  \href{https://nytmediakit.com/}{Advertise}
\item
  \href{http://www.tbrandstudio.com/}{T Brand Studio}
\item
  \href{https://www.nytimes.com/privacy/cookie-policy\#how-do-i-manage-trackers}{Your
  Ad Choices}
\item
  \href{https://www.nytimes.com/privacy}{Privacy}
\item
  \href{https://help.nytimes.com/hc/en-us/articles/115014893428-Terms-of-service}{Terms
  of Service}
\item
  \href{https://help.nytimes.com/hc/en-us/articles/115014893968-Terms-of-sale}{Terms
  of Sale}
\item
  \href{https://spiderbites.nytimes.com}{Site Map}
\item
  \href{https://help.nytimes.com/hc/en-us}{Help}
\item
  \href{https://www.nytimes.com/subscription?campaignId=37WXW}{Subscriptions}
\end{itemize}
