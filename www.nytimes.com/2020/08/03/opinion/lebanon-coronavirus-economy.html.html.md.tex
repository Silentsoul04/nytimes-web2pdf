Sections

SEARCH

\protect\hyperlink{site-content}{Skip to
content}\protect\hyperlink{site-index}{Skip to site index}

\href{https://myaccount.nytimes.com/auth/login?response_type=cookie\&client_id=vi}{}

\href{https://www.nytimes.com/section/todayspaper}{Today's Paper}

\href{/section/opinion}{Opinion}\textbar{}We Lebanese Thought We Could
Survive Anything. We Were Wrong.

\href{https://nyti.ms/2D4bZKF}{https://nyti.ms/2D4bZKF}

\begin{itemize}
\item
\item
\item
\item
\item
\end{itemize}

Advertisement

\protect\hyperlink{after-top}{Continue reading the main story}

\href{/section/opinion}{Opinion}

Supported by

\protect\hyperlink{after-sponsor}{Continue reading the main story}

\hypertarget{we-lebanese-thought-we-could-survive-anything-we-were-wrong}{%
\section{We Lebanese Thought We Could Survive Anything. We Were
Wrong.}\label{we-lebanese-thought-we-could-survive-anything-we-were-wrong}}

The myth of their resilience helped the Lebanese function despite a
miserably corrupt and inept state. No longer.

By Lina Mounzer

Ms. Mounzer is a Lebanese writer and translator.

\begin{itemize}
\item
  Aug. 3, 2020
\item
  \begin{itemize}
  \item
  \item
  \item
  \item
  \item
  \end{itemize}
\end{itemize}

\includegraphics{https://static01.nyt.com/images/2020/08/04/opinion/04mounzer/merlin_174682458_fd595377-8ee0-40fa-bc2c-e2a4eb6433dc-articleLarge.jpg?quality=75\&auto=webp\&disable=upscale}

BEIRUT --- Anyone who knows Lebanon has heard this: The Lebanese are
resilient. A reputation earned by weathering an outsize list of
challenges over the years: a 15-year civil war, political tensions with
Syria, wars with Israel and a collapsing public service infrastructure.

Economic collapse, exacerbated by a coronavirus lockdown in March, is
the latest disaster to befall the Lebanese people. Years of financial
maneuvering and debt mismanagement by the government, the banks and the
central bank led to a depletion of the country's foreign currency
reserves, which caused depreciation of the Lebanese pound --- long
pegged to the dollar --- by
\href{https://www.forbes.com/sites/tatianakoffman/2020/07/09/lebanons-currency-crisis-paves-the-way-to-a-new-future/\#5ae751526a17}{more
than 80 percent} since October.

People have
\href{https://en.annahar.com/article/1187657-lebanons-banking-crisis-risks-wiping-life-savings-of-thousands}{lost
their life savings overnight}, and prices of food and basic goods ****
have inflated
\href{https://www.reuters.com/article/us-emerging-inflation-graphic/lebanon-follows-venezuela-into-hyperinflation-wilderness-idUSKCN24O20J}{over
50 percent for the third month} in a row.

Lebanon's economy is based on services; the country produces little and
imports a lot. The plummeting value of the Lebanese lira and the lack of
access to dollars means many stores can't afford to remain open.
Salaries that haven't been slashed outright have lost their spending
power. Tourism, once one of the pillars of the economy, has been
severely hit in recent months from the worsening political and economic
situation and the enforced pandemic lockdown.

Nearly 1,000 restaurants have been forced to close, and 25,000 people
have
\href{https://www.aljazeera.com/ajimpact/lebanons-restaurants-sinking-coronavirus-lockdown-200506095822226.html}{lost
their jobs} in the restaurant sector alone. Remittances, which once
brought enough dollars into the country to help offset the budget
deficit, have been on the wane for some years now because of lower oil
prices in the Gulf, where many Lebanese work. Data for 2020 is still not
in, but remittances are
\href{https://www.middleeasteye.net/news/coronavirus-egypt-lebanon-jordan-remittance-economy}{set
to dwindle more} as a lack of trust in the banks and travel restrictions
mean money cannot find its way into the country either through bank or
in-person transfers.

The cumulative effect of these factors is that much of the middle class
--- about
\href{https://foreignpolicy.com/2020/05/21/lebanon-coronavirus-middle-class-poverty/}{3.25
million people}, or 65 percent of the country --- has
\href{https://carnegie-mec.org/diwan/82348}{slipped} into poverty. The
poor are now destitute, near starvation. Once again, people must
scramble to find solutions for the government's ills.

Facebook
\href{https://www.nytimes.com/2020/07/12/world/middleeast/beirut-lebanon-economic-crisis.html}{barter
groups} have sprung up, with people seeking to exchange whatever they
have, in order to secure what they so desperately need: fancy dresses
for baby formula; a set of drinking glasses for a bag of rice. One man
held up a pharmacy for diapers. Another mugged someone at knife point
only to return,
\href{https://www.france24.com/en/20200715-hunger-crimes-on-the-rise-in-crisis-hit-lebanon}{sobbing
and apologizing} that he is unable to feed his family, that he's only
doing this to survive.

While the mass protests that erupted on Oct. 17, 2019 --- against the
corruption of the ruling elite and the sectarian system they uphold ---
succeeded in toppling the government, the new government appointed in
its place has done little to alleviate or address worsening living
conditions.

Misery is now palpable across the country, in the rows of shuttered
shops, in the garbage piling up in different neighborhoods as basic
services are disrupted, and in the
\href{https://www.washingtonpost.com/world/middle_east/lebanons-rising-power-cuts-add-to-gloom-of-economic-crisis/2020/07/07/95498db4-c05f-11ea-8908-68a2b9eae9e0_story.html}{darkness
of the nighttime streets} of Beirut as electricity cuts soar to 20 hours
a day.

It is hard to imagine that this is the same Beirut that has seen so many
\href{https://www.telegraph.co.uk/news/worldnews/middleeast/lebanon/11748872/War-is-a-million-miles-away-when-the-Lebanese-begin-to-party.html}{odes
to the unparalleled revelry of its bar scene} --- the favorite subject
and haunt of many a Western journalist --- written in every major
newspaper.

Those articles repeatedly peddled the same myth that we are so eager to
believe about ourselves: The Lebanese know how to dance while the
bullets fly, how to repurpose even
\href{https://www.washingtonpost.com/lifestyle/travel/interior-designer-sarah-a-abdallah-brings-a-global-perspective-to-hotel-lobbies-and-restaurants/2018/02/09/8ef88344-f72a-11e7-a9e3-ab18ce41436a_story.html}{war
bunkers into nightclubs}, how to find a way around every shortage,
because the Lebanese are resourceful and resilient. It is an idea that
has served both as consolation for difficult living conditions and a
point of pride for how efficiently we manage them.

But now it has become clear that there is nothing truly resilient about
Lebanon except its politicians and ancient warlords, who refuse to step
down, even after their profiteering has bankrupted the country and its
people. And while Lebanese banks have long been
\href{https://www.worldfinance.com/strategy/a-resilient-banking-sector-is-allowing-lebanons-economy-to-endure-regional-strife}{deemed
resilient by economists}, the current situation reveals that this
designation belongs only to the bankers themselves.

Even the International Monetary Fund, the so-called ``lender of last
resort,'' cannot bring the banks to agree on
\href{https://www.bloomberg.com/news/articles/2020-06-25/lebanon-tallies-up-its-debt-problem-give-or-take-a-few-billion}{the
audit of financial losses} required to unlock a loan. Two high-ranking
officials have so far resigned over the stalled negotiations. One of
them, Alain Bifani, a former director general of the Finance Ministry,
revealed that bankers smuggled close to
\href{https://www.nasdaq.com/articles/lebanons-ex-finance-chief-says-banks-smuggled-\%246bln-out-report-2020-07-13}{\$6
billion outside} the country while preventing small depositors from
withdrawing so much as \$100.

Christine Tohme is the curator and founder of
\href{https://ashkalalwan.org/}{Ashkal Alwan}, an organization that
supports local art and cultural production. She started operating it out
of her car in 1993, two years after the end of the civil war. Over the
last 27 years, despite a lack of state funding, the organization ran a
prestigious art study program that sees applicants from across the
world. This year, the program was forced to end early when funds meant
for students were seized by the banks. ``We all worked so hard to make
things better, and look where we are now,'' said Ms. Tohme. ``I am so
sad about not being able to dream or see into a future. I am
exhausted.''

That exhaustion is heavy in the voices and faces of everyone I
encounter. Perhaps resilience has always been the lie we have been fed
and that we continue to tell ourselves in order to keep functioning
under a state so corrupt it cannot provide a bare minimum of public or
social services.

A state that thinks so little of its people that it bickered over the
pittance of financial aid it promised to hungry families during the
pandemic until devaluation reduced the worth of the aid that each family
was getting from \$180 to about \$50. And then it was never distributed.

But the protests late last year were evidence that refusing to accept
resilience is also a refusal to accept the conditions that made it
necessary for us to rely on the idea in the first place. A refusal to
find private solutions to problems that should rightfully be solved by
the state. If things seem bleak now, at least there is this: We have
finally come to recognize that a myth is poor consolation for a
half-lived life, no matter how attractive that myth might be.

Lina Mounzer is a Lebanese writer and translator.

\emph{The Times is committed to publishing}
\href{https://www.nytimes.com/2019/01/31/opinion/letters/letters-to-editor-new-york-times-women.html}{\emph{a
diversity of letters}} \emph{to the editor. We'd like to hear what you
think about this or any of our articles. Here are some}
\href{https://help.nytimes.com/hc/en-us/articles/115014925288-How-to-submit-a-letter-to-the-editor}{\emph{tips}}\emph{.
And here's our email:}
\href{mailto:letters@nytimes.com}{\emph{letters@nytimes.com}}\emph{.}

\emph{Follow The New York Times Opinion section on}
\href{https://www.facebook.com/nytopinion}{\emph{Facebook}}\emph{,}
\href{http://twitter.com/NYTOpinion}{\emph{Twitter (@NYTopinion)}}
\emph{and}
\href{https://www.instagram.com/nytopinion/}{\emph{Instagram}}\emph{.}

Advertisement

\protect\hyperlink{after-bottom}{Continue reading the main story}

\hypertarget{site-index}{%
\subsection{Site Index}\label{site-index}}

\hypertarget{site-information-navigation}{%
\subsection{Site Information
Navigation}\label{site-information-navigation}}

\begin{itemize}
\tightlist
\item
  \href{https://help.nytimes.com/hc/en-us/articles/115014792127-Copyright-notice}{©~2020~The
  New York Times Company}
\end{itemize}

\begin{itemize}
\tightlist
\item
  \href{https://www.nytco.com/}{NYTCo}
\item
  \href{https://help.nytimes.com/hc/en-us/articles/115015385887-Contact-Us}{Contact
  Us}
\item
  \href{https://www.nytco.com/careers/}{Work with us}
\item
  \href{https://nytmediakit.com/}{Advertise}
\item
  \href{http://www.tbrandstudio.com/}{T Brand Studio}
\item
  \href{https://www.nytimes.com/privacy/cookie-policy\#how-do-i-manage-trackers}{Your
  Ad Choices}
\item
  \href{https://www.nytimes.com/privacy}{Privacy}
\item
  \href{https://help.nytimes.com/hc/en-us/articles/115014893428-Terms-of-service}{Terms
  of Service}
\item
  \href{https://help.nytimes.com/hc/en-us/articles/115014893968-Terms-of-sale}{Terms
  of Sale}
\item
  \href{https://spiderbites.nytimes.com}{Site Map}
\item
  \href{https://help.nytimes.com/hc/en-us}{Help}
\item
  \href{https://www.nytimes.com/subscription?campaignId=37WXW}{Subscriptions}
\end{itemize}
