Sections

SEARCH

\protect\hyperlink{site-content}{Skip to
content}\protect\hyperlink{site-index}{Skip to site index}

\href{https://myaccount.nytimes.com/auth/login?response_type=cookie\&client_id=vi}{}

\href{https://www.nytimes.com/section/todayspaper}{Today's Paper}

\href{/section/opinion}{Opinion}\textbar{}The Good News About What Human
Genius Can Still Do

\href{https://nyti.ms/3i8d1UP}{https://nyti.ms/3i8d1UP}

\begin{itemize}
\item
\item
\item
\item
\item
\end{itemize}

\href{https://www.nytimes.com/2020/08/02/science/spacex-astronauts-splashdown.html?action=click\&pgtype=Article\&state=default\&region=TOP_BANNER\&context=storylines_menu}{SpaceX's
Astronaut Trip}

\begin{itemize}
\tightlist
\item
  \href{https://www.nytimes.com/2020/08/02/science/spacex-astronauts-splashdown.html?action=click\&pgtype=Article\&state=default\&region=TOP_BANNER\&context=storylines_menu}{`Thanks
  for Flying SpaceX'}
\item
  \href{https://www.nytimes.com/2020/05/26/science/spacex-launch-nasa.html?action=click\&pgtype=Article\&state=default\&region=TOP_BANNER\&context=storylines_menu}{Why
  NASA Picked SpaceX}
\item
  \href{https://www.nytimes.com/interactive/2020/05/26/science/spacex-nasa.html?action=click\&pgtype=Article\&state=default\&region=TOP_BANNER\&context=storylines_menu}{Inside
  the Capsule}
\item
  \href{https://www.nytimes.com/2020/05/27/science/bob-behnken-doug-hurley.html?action=click\&pgtype=Article\&state=default\&region=TOP_BANNER\&context=storylines_menu}{Meet
  the Astronauts}
\end{itemize}

Advertisement

\protect\hyperlink{after-top}{Continue reading the main story}

\href{/section/opinion}{Opinion}

Supported by

\protect\hyperlink{after-sponsor}{Continue reading the main story}

\hypertarget{the-good-news-about-what-human-genius-can-still-do}{%
\section{The Good News About What Human Genius Can Still
Do}\label{the-good-news-about-what-human-genius-can-still-do}}

There's something uniquely compelling about our need to learn what we
can about the universe.

By
\href{https://www.nytimes.com/interactive/opinion/editorialboard.html}{The
Editorial Board}

The editorial board is a group of opinion journalists whose views are
informed by expertise, research, debate and certain longstanding ****
\href{https://www.nytimes.com/interactive/2018/opinion/editorialboard.html}{values}.
It is separate from the newsroom.

\begin{itemize}
\item
  Aug. 3, 2020
\item
  \begin{itemize}
  \item
  \item
  \item
  \item
  \item
  \end{itemize}
\end{itemize}

\includegraphics{https://static01.nyt.com/images/2020/08/03/opinion/03wonder-edit/03wonder-edit-articleLarge.jpg?quality=75\&auto=webp\&disable=upscale}

We may never solve the puzzle of Stonehenge, and, if we do, it probably
won't change our lives. Nor will much change if the latest American
robot zooming toward Mars discovers that eons ago microorganisms swam in
long vanished lakes. Yet how gloriously enthralling the search!

The past week has provided a remarkable demonstration that despite the
coronavirus and its global economic wreckage, humankind persists in
devoting extraordinary talent, science and treasure to decoding the
mysteries of our universe. NASA
\href{https://www.nytimes.com/2020/07/30/science/nasa-mars-launch.html?searchResultPosition=3}{launched
a car-size rover} --- carrying a little helicopter --- to Mars, Elon
Musk's SpaceX brought the first privately developed manned space capsule
\href{https://www.nytimes.com/video/us/100000007269118/spacex-splash-down.html?searchResultPosition=2}{safely
back from space}, and on a totally different front, archaeologists
\href{https://www.nytimes.com/2020/07/29/science/stonehenge-archaeology-sarsens.html?searchResultPosition=1}{chipped
away another unknown} from the enigma of Stonehenge.

Certainly some good news on what human genius can do provided a welcome
diversion from all the harrowing news about the pandemic and toxic
politics. There's something uniquely compelling about our need to learn
what we can about the universe, and to do that not for some practical or
commercial reason, but simply out of curiosity, out of wonder.

``\href{https://www.nytimes.com/1923/03/18/archives/climbing-mount-everest-is-work-for-supermen-a-member-of-former.html}{Because
it's there}'' was the explanation famously offered by George Mallory for
persevering in his ultimately fatal efforts to scale Mount Everest,
providing a timeless refrain for any endeavor that we cannot justify by
logic, gain or value --- like answering questions about Stonehenge or
exploring Mars. The latter, admittedly, has been justified by some as a
search for a potential haven should we succeed in destroying our own
planet. Mr. Musk, who has set getting a human to Mars as a future
ambition for the Dragon capsule that successfully splashed down Sunday
after taking two astronauts to the International Space Station and
returning them home, told an interviewer that
``\href{https://slate.com/technology/2015/04/elon-musk-and-mars-spacex-ceo-and-our-multiplanet-species.html}{humans
need to be a multiplanet species}.''

It would be a stretch to suggest that survival of our kind is really a
serious consideration for spending billions on missions to Mars, of
which there were fully three in July --- one each from
\href{https://www.nature.com/articles/d41586-020-02187-7}{China} and the
\href{https://www.cnn.com/2020/07/19/middleeast/uae-mars-hope-launch-intl-hnk-scn-scli/index.html}{United
Arab Emirates} in addition to NASA's. The American mission is by far the
most sophisticated: The rover, named Perseverance, which will reach Mars
sometime in February, will study a former lake bed and assemble a pile
of rock samples for a future mission to bring back in search of any
evidence that there may once have been life on the Red Planet. More
immediately exciting is the helicopter piggybacked on the rover, which,
if it works, will make the first powered flight on another planet.

The Stonehenge news may not be on the same order of scientific gee-whiz,
nor anywhere near as costly, but efforts to solve the enigma of that
cluster of giant stones raised in southern England by a prehistoric
people are far older than efforts to colonize Mars, so resolving
something as elemental as where the megaliths came from amounted to a
grand achievement. How it was determined, moreover, involves that
combination of luck, science and old-fashioned detective work that makes
for a great mystery story: The chance return of a cylindrical chunk cut
out of one of the massive rocks 60 years ago enabled geologists to
finally narrow the source of the megaliths to a site 15 miles north of
Stonehenge.

That doesn't really do much to answer the big question --- why druids
created the ring on a field there --- but it will lead to the actual
quarries and perhaps to an explanation on how our ancestors dressed and
moved 20-ton sandstone blocks before they invented the wheel. For
hundreds of years, the leading explanation was the one offered by a
12th-century Welsh cleric who declared that Merlin the wizard persuaded
his king to send 15,000 men --- other variants had Merlin assign the
task to the Devil --- to bring a mighty stone construction from Ireland
to build a proper burial ground for his warriors. On the cemetery for
the elite he was probably right. Other modern theories suggest
Stonehenge was designed as a sacred place of healing, or music, or
observations of the sun, or all the above.

Does it matter? Archaeologists are likely to have a prepared script
along the lines of forgetting history and repeating the past, or getting
right what the winners who write history got wrong. Space scientists
will have their own set of justifications for the astronomical costs of
their projects (the Perseverance mission went way over its \$1.5 billion
budget, and that was before the helicopter was thrown in), which usually
focus on scientific discovery, technological spinoffs, potential
economic benefit and national security.

But press a bit harder, and the logical justifications tend to give way
to variations on ``because it's there.'' The former NASA administrator
Michael Griffin suggested that things like space exploration are
motivated by elemental human traits like competitiveness, curiosity and
a longing to leave behind something worthy and good. In a similar vein,
one archaeologist proposed that archaeology is important for the same
reason that art, literature, philosophy and history are important,
because people have a basic need to know where we came from. More
simply,
\href{https://pahistoricpreservation.com/why-is-archaeology-important/}{she
wrote}, it's important ``because we feel it is.''

In the end, these are all different ways of saying the same thing. The
world is full of wonder, and there's no need to justify its hold on us.

\emph{The Times is committed to publishing}
\href{https://www.nytimes.com/2019/01/31/opinion/letters/letters-to-editor-new-york-times-women.html}{\emph{a
diversity of letters}} \emph{to the editor. We'd like to hear what you
think about this or any of our articles. Here are some}
\href{https://help.nytimes.com/hc/en-us/articles/115014925288-How-to-submit-a-letter-to-the-editor}{\emph{tips}}\emph{.
And here's our email:}
\href{mailto:letters@nytimes.com}{\emph{letters@nytimes.com}}\emph{.}

\emph{Follow The New York Times Opinion section on}
\href{https://www.facebook.com/nytopinion}{\emph{Facebook}}\emph{,}
\href{http://twitter.com/NYTOpinion}{\emph{Twitter (@NYTopinion)}}
\emph{and}
\href{https://www.instagram.com/nytopinion/}{\emph{Instagram}}\emph{.}

Advertisement

\protect\hyperlink{after-bottom}{Continue reading the main story}

\hypertarget{site-index}{%
\subsection{Site Index}\label{site-index}}

\hypertarget{site-information-navigation}{%
\subsection{Site Information
Navigation}\label{site-information-navigation}}

\begin{itemize}
\tightlist
\item
  \href{https://help.nytimes.com/hc/en-us/articles/115014792127-Copyright-notice}{©~2020~The
  New York Times Company}
\end{itemize}

\begin{itemize}
\tightlist
\item
  \href{https://www.nytco.com/}{NYTCo}
\item
  \href{https://help.nytimes.com/hc/en-us/articles/115015385887-Contact-Us}{Contact
  Us}
\item
  \href{https://www.nytco.com/careers/}{Work with us}
\item
  \href{https://nytmediakit.com/}{Advertise}
\item
  \href{http://www.tbrandstudio.com/}{T Brand Studio}
\item
  \href{https://www.nytimes.com/privacy/cookie-policy\#how-do-i-manage-trackers}{Your
  Ad Choices}
\item
  \href{https://www.nytimes.com/privacy}{Privacy}
\item
  \href{https://help.nytimes.com/hc/en-us/articles/115014893428-Terms-of-service}{Terms
  of Service}
\item
  \href{https://help.nytimes.com/hc/en-us/articles/115014893968-Terms-of-sale}{Terms
  of Sale}
\item
  \href{https://spiderbites.nytimes.com}{Site Map}
\item
  \href{https://help.nytimes.com/hc/en-us}{Help}
\item
  \href{https://www.nytimes.com/subscription?campaignId=37WXW}{Subscriptions}
\end{itemize}
