Sections

SEARCH

\protect\hyperlink{site-content}{Skip to
content}\protect\hyperlink{site-index}{Skip to site index}

\href{https://www.nytimes.com/section/parenting}{Parenting}

\href{https://myaccount.nytimes.com/auth/login?response_type=cookie\&client_id=vi}{}

\href{https://www.nytimes.com/section/todayspaper}{Today's Paper}

\href{/section/parenting}{Parenting}\textbar{}How to Diversify Your Toy
Box

\url{https://nyti.ms/2PlTXpz}

\begin{itemize}
\item
\item
\item
\item
\item
\end{itemize}

Advertisement

\protect\hyperlink{after-top}{Continue reading the main story}

Supported by

\protect\hyperlink{after-sponsor}{Continue reading the main story}

\hypertarget{how-to-diversify-your-toy-box}{%
\section{How to Diversify Your Toy
Box}\label{how-to-diversify-your-toy-box}}

Ways to create a multiracial and multicultural spectrum in a child's
activities and toys.

\includegraphics{https://static01.nyt.com/images/2020/08/03/autossell/diversetoys-still/diversetoys-still-threeByTwoMediumAt2X.jpg}

By Shanicia Boswell

\begin{itemize}
\item
  Aug. 3, 2020
\item
  \begin{itemize}
  \item
  \item
  \item
  \item
  \item
  \end{itemize}
\end{itemize}

Becoming a mother made me reflect on my own childhood and how I was
raised to view diversity. I grew up in South Georgia at a time when
dolls of my complexion were not easy to find. So I played with mostly
white dolls. The features I observed in Ken and Barbie were exactly what
I saw around me: the familiar aspect of white people. My doctor was
white. My teachers were mostly white. Police officers and bank tellers
all had skin that did not look like my own.

My friends' toy chests overflowed with white ballerina Barbies and
blonde, pigtailed Cabbage Patch Kids. Their bookshelves were filled with
stories about happy white children who spent their days gallivanting
with big red dogs. As a Black child in the 1990s, I knew that my skin
color was different, and I often wondered why I didn't see more people
who looked like me in my cartoon shows or books or reflected in my
dolls.

Imagine how children's views of themselves and others could change by
having dolls of diverse races and ethnicities, not just white Barbies
and G.I. Joes.

``Diversity through play is achieved by having a variety of toys, music,
clothes, books, food, and art that highlight and celebrate similarities,
as well as differences,'' said Ann-Louise Lockhart, Psy.D.\emph{,} a
pediatric psychologist and parent coach.

Research has shown children become aware of racial differences
\href{https://ncela.ed.gov/files/rcd/BE020476/Play_and_Cultural_Diversity.pdf?origin=publication_detail\#:~:text=The\%20dramatic\%20play\%20area\%20may,reflect\%20different\%20styles\%20of\%20living.}{as
early as 3 years old}, and children often mimic their home life
dynamics, classrooms and daily interactions with their toys. They
``learn, interact, grow, and construct their understanding of their
world through play,'' Dr. Lockhart said. ``They learn about social and
family relationships through play, including how to relate to and
interact with others.''

When encouraging diversity through play, parents and caregivers can
explore activities and provide toys that help kids acknowledge
differences and embrace similarities, and let them know that seeing
color is OK. I select each doll and book for my 7-year-old daughter with
intention. I want her toys to reflect the real world, in all its diverse
glory. If you are a white parent of young children, ask yourself: How
many of your children's toys and books are multiracial? And for
multicultural families and families of color, take note of how your
children are being represented in action figures, dolls and other toys.
Since diversity is not just about race, consider how your toy, game and
book selections might also include people with nonconforming gender
identities, different abilities, and a range of faiths and ages.

With more kids learning from home this school year, away from other
children, and more awareness of systemic racism and injustice in the
U.S., parents are becoming more conscious of how important teaching
diversity is. Here are some other ways to diversify play in your child's
life.

\hypertarget{dolls-and-action-figures}{%
\subsection{Dolls and action figures}\label{dolls-and-action-figures}}

By collecting dolls across the color spectrum, we help children create a
self concept that acknowledges their own differences and helps them see
that others are unique, too. Self concepts, Dr. Lockhart explained, are
the statements we make about ourselves, which also include ``the images
and ideas we have in our mind about ourselves.''

``When children see themselves represented in their toys, books, movies,
music, food, and artwork, it shapes their self concept'' Dr. Lockhart
said.

Yelitsa Jean-Charles, the founder and chief executive of
\href{https://healthyrootsdolls.com/}{Healthy Roots Dolls}, said she
created her doll company to represent today's children. She said she was
inspired by the 1940s Doll Test by the psychologists Kenneth B. and
Mamie P. Clark. The researchers showed identical dolls, except for their
skin color, to 253 Black children between the ages of 3 and 7 to test
children's racial perceptions. They found that the majority of children
attributed positive characteristics to the white dolls and negative
characteristics to the Black dolls. The findings suggested how Black
children felt inferior and had lost their self esteem during a time of
segregation, prejudice and discrimination.

With her toy company, Jean-Charles has normalized Black dolls.
Jean-Charles said that between 20 to 40 percent of her customers are
non-Black parents with children of color who specifically purchase her
dolls to have conversations about diversity with their children or to
have dolls that represent their children in complexion and hair texture.

``I created these dolls because I never had a doll that looked like me
growing up,'' Jean-Charles said.

In addition to dolls, parents can also include multicultural play
figures by both small toy companies, like
\href{https://blacktoystore.com/product-category/action-figures/}{The
Black Toy Store}, and large toy manufactures, like
\href{https://corporate.mattel.com/news/mattel-unveils-special-edition-thankyouheroes-collection-from-fisher-price-xae-to-honor-today-s-heroes}{Mattel}
or
\href{https://education.lego.com/en-us/products/people-by-lego-education/45030\#product}{Lego
Duplo}.

\hypertarget{games-and-imaginative-play}{%
\subsection{Games and imaginative
play}\label{games-and-imaginative-play}}

Reflecting on my childhood, I can see how my imaginative play
perpetuated a racial bias. In fantasy games, like playing ``teacher'' or
``doctor,'' I remember my white friends always holding a position of
power; they were the teacher or the doctor. The Black students, myself
included, played the role of the student or patient. As a parent now, I
have had conversations with my daughter about this. I encourage her to
be the leader in her imaginative play with friends.

Dr. Lockhart said that through play, children can experiment with their
views of the world, then dismantle that world, build new worlds, and try
them out again.

Even though children are limited in playing with others because of the
pandemic, there are still ways to use
\href{https://www.browntoybox.com/shop/all/}{games},
\href{https://puzzlehuddle.com/}{puzzles} and imaginative to promote
diversity. One study suggested
\href{https://www.nytimes.com/2020/06/09/parenting/childrens-books-black-characters.html}{reading
books} that focus on diversity, equality and social justice and then
\href{https://pdfs.semanticscholar.org/33ff/b7ab64d021e34f5a1ec1ee975152e479f1d2.pdf}{acting
out scenes}, discussing with parents how kids would respond to certain
situations. This allows children to put themselves in someone else's
shoes and practice empathy as well.

Role-playing can help create safe spaces of child-led conversations
where your children feel comfortable to ask questions and explore topics
about race and culture.

``To create space for our children to have conversations related to race
and diversity, we have to be willing to do our own work so that we can
come to the conversation comfortable enough to hold whatever our
children bring,'' said Kira Hudson Banks, Ph.D, a psychologist and
educator of more than 20 years.

``We have to also be willing to trust the intelligence of our children.
We know they understand the dynamics of oppression by preschool and they
start to judge by race, then it is incumbent upon us to realize that
they are capable of starting to understand and interrupt some of those
stereotypes.''

\hypertarget{visual-arts}{%
\subsection{Visual arts}\label{visual-arts}}

With drawing or painting, parents can incorporate examples and materials
from artists with backgrounds different from their own. Introduce
artists who use bright and intentional colors or work for social causes.
Consider
\href{https://www.nytimes.com/2018/02/12/arts/obama-portraitists-bonded-the-everyday-and-the-extraordinary.html}{the
artists Kehinde Wiley and Amy Sherald}, who were commissioned to create
portraits of former President Barack Obama and first lady Michelle Obama
for the National Portrait Gallery. Or show work from
\href{https://www.nytimes.com/2010/10/24/arts/design/24muniz.html}{Vik
Muniz}, a mixed-media artist from Brazil, who creates art from everyday
materials. His photographs can prompt conversations about recycling as
well as what it was like growing up in Brazil, which challenged his
views on waste in America. Children can then design their own artistic
pieces having been inspired by a more diverse art world.

\hypertarget{dance-and-music}{%
\subsection{Dance and music}\label{dance-and-music}}

Dance is another way to introduce diversity into home-based playtime and
learning. Caregivers can show videos of dancers, like Misty Copeland and
Michaela DePrince. Or play music from different parts of the world. Have
fun with this: Spin a globe, pick a location and look up children's
music for that particular region. If you can, carve out 20 minutes of
dance each day with your children and discuss what you like about the
music you are dancing to.

\hypertarget{languages}{%
\subsection{Languages}\label{languages}}

Learning a few key words of another language can assist in language
barriers between children. It could be as simple as teaching one word,
like ``hello,'' in multiple languages. Word flash cards with pictures in
English and a second language can be a fun way to incorporate learning a
new language into play time. For children, apps like
\href{https://www.duolingo.com/}{Duolingo} and
\href{https://mangolanguages.com/}{Mango}, is an interactive way to
expand the knowledge of basic linguistics for communication.

Patricia Nunley, Ed.D., a professor for early childhood development at
the City College of San Francisco, urges parents to move with caution
here. ``Language is directly tied to culture.'' she said. ``Parents must
group language in with other concepts of learning to properly utilize
it. Children have to find the appreciation of culture before truly
appreciating another language.''

For example, if I teach my daughter to say ``hello'' in Japanese, or
``Kon-ni-chi-wa,'' but I neglect to explain to her the formal greetings
of bowing, showing respect to elders, or the etiquette of greeting, I am
not fully connecting the Japanese culture to the word.

Diversity is more than speaking another language or teaching about other
races. It includes, but is not limited to, gender identity, individuals
with disabilities and people of different faiths. Age also plays a major
role in how we teach our children.

As parents, we must make a conscious effort to explore playtime in a way
that does not only reflect our family dynamic but positively reflects
our melting pot of cultures around the world. We must uplift the voices
of those who do not share our complexion.

The process of diversifying play should not feel stressful. Parents
should enjoy learning alongside children. Children should feel dignified
and honored during the process of learning diversity. They should be
empowered and know that seeing color is OK. In acknowledging our
differences, we learn to embrace our similarities.

Shanicia Boswell is the founder of
\href{https://blackmomsblog.com/}{Black Moms Blog}, a collaborative
blogging community that focuses on parenting, culture, and lifestyle
from a Black mom's point of view. Her views on parenting have been
featured on OWN network, HLN/CNN, Thrive, March of Dimes and Procter \&
Gamble.

Advertisement

\protect\hyperlink{after-bottom}{Continue reading the main story}

\hypertarget{site-index}{%
\subsection{Site Index}\label{site-index}}

\hypertarget{site-information-navigation}{%
\subsection{Site Information
Navigation}\label{site-information-navigation}}

\begin{itemize}
\tightlist
\item
  \href{https://help.nytimes.com/hc/en-us/articles/115014792127-Copyright-notice}{©~2020~The
  New York Times Company}
\end{itemize}

\begin{itemize}
\tightlist
\item
  \href{https://www.nytco.com/}{NYTCo}
\item
  \href{https://help.nytimes.com/hc/en-us/articles/115015385887-Contact-Us}{Contact
  Us}
\item
  \href{https://www.nytco.com/careers/}{Work with us}
\item
  \href{https://nytmediakit.com/}{Advertise}
\item
  \href{http://www.tbrandstudio.com/}{T Brand Studio}
\item
  \href{https://www.nytimes.com/privacy/cookie-policy\#how-do-i-manage-trackers}{Your
  Ad Choices}
\item
  \href{https://www.nytimes.com/privacy}{Privacy}
\item
  \href{https://help.nytimes.com/hc/en-us/articles/115014893428-Terms-of-service}{Terms
  of Service}
\item
  \href{https://help.nytimes.com/hc/en-us/articles/115014893968-Terms-of-sale}{Terms
  of Sale}
\item
  \href{https://spiderbites.nytimes.com}{Site Map}
\item
  \href{https://help.nytimes.com/hc/en-us}{Help}
\item
  \href{https://www.nytimes.com/subscription?campaignId=37WXW}{Subscriptions}
\end{itemize}
