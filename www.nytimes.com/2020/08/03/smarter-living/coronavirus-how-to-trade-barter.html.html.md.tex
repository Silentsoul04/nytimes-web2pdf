Sections

SEARCH

\protect\hyperlink{site-content}{Skip to
content}\protect\hyperlink{site-index}{Skip to site index}

\href{https://www.nytimes.com/section/smarter-living}{Smarter Living}

\href{https://myaccount.nytimes.com/auth/login?response_type=cookie\&client_id=vi}{}

\href{https://www.nytimes.com/section/todayspaper}{Today's Paper}

\href{/section/smarter-living}{Smarter Living}\textbar{}Can't Find It at
the Store?

\url{https://nyti.ms/3k7ygYn}

\begin{itemize}
\item
\item
\item
\item
\item
\end{itemize}

\href{https://www.nytimes.com/spotlight/at-home?action=click\&pgtype=Article\&state=default\&region=TOP_BANNER\&context=at_home_menu}{At
Home}

\begin{itemize}
\tightlist
\item
  \href{https://www.nytimes.com/2020/08/03/well/family/the-benefits-of-talking-to-strangers.html?action=click\&pgtype=Article\&state=default\&region=TOP_BANNER\&context=at_home_menu}{Talk:
  To Strangers}
\item
  \href{https://www.nytimes.com/2020/08/01/at-home/coronavirus-make-pizza-on-a-grill.html?action=click\&pgtype=Article\&state=default\&region=TOP_BANNER\&context=at_home_menu}{Make:
  Grilled Pizza}
\item
  \href{https://www.nytimes.com/2020/07/31/arts/television/goldbergs-abc-stream.html?action=click\&pgtype=Article\&state=default\&region=TOP_BANNER\&context=at_home_menu}{Watch:
  'The Goldbergs'}
\item
  \href{https://www.nytimes.com/interactive/2020/at-home/even-more-reporters-editors-diaries-lists-recommendations.html?action=click\&pgtype=Article\&state=default\&region=TOP_BANNER\&context=at_home_menu}{Explore:
  Reporters' Google Docs}
\end{itemize}

Advertisement

\protect\hyperlink{after-top}{Continue reading the main story}

Supported by

\protect\hyperlink{after-sponsor}{Continue reading the main story}

Aug. 3, 2020

\includegraphics{https://static01.nyt.com/images/2020/08/03/smarter-living/00sl-barter-1/00sl-barter-1-articleLarge.jpg?quality=75\&auto=webp\&disable=upscale}

\hypertarget{cant-find-it-at-the-store}{%
\section{Can't Find It at the Store?}\label{cant-find-it-at-the-store}}

Image

\hypertarget{try-bartering}{%
\subsection{Try Bartering}\label{try-bartering}}

Image

Credit...Fran Caballero

\hypertarget{heres-how-to-do-it-right}{%
\subsection{Here's How to Do It Right}\label{heres-how-to-do-it-right}}

\hypertarget{and-fairly}{%
\subsection{(and Fairly)}\label{and-fairly}}

By A.C. Shilton

When the coronavirus pandemic hit, Kate Coffman, who works at Alliance
Brewing Company in Knoxville, Tenn., knew cash was going to be tight for
a while. Although her brewery has stayed open for takeout, she has
worked fewer hours and business is down. When a friend, Angie Ervin,
started a local bartering group on Facebook, Ms. Coffman was eager to
join.

``It kind of blew up overnight,'' Ms. Coffman said.

Soon, Ms. Ervin was begging Ms. Coffman to help her manage the group's
page. Since it started on April 5, the group, Knoxville Barters, has
swelled to more than 2,200 members.

If it feels that we've zagged from bean-counting every restaurant bill
over Venmo to casually trading yeast for cut flowers without any thought
to the monetary value of these items, it's because, well, we have.
Bartering hasn't been this widespread since its days at the
elementary-school lunch table. Front Porch Forum, a hyperlocal social
network in Vermont and parts of New York that has long been a hub of
bartering, has seen an 83 percent increase in new-member sign-ups this
year over the same period last year, said Michael Wood-Lewis, who
co-founded the site with his wife, Valerie, as a neighborhood listserv
back in 2000. While Front Porch Forum is a way for neighbors to connect
on a range of things, recently, appeals for swapping eggs for rhubarb or
chicken wire for day lily bulbs have increased, Mr. Wood-Lewis said.

\hypertarget{enter-the-modern-barter-economy}{%
\subsection{Enter the modern barter
economy}\label{enter-the-modern-barter-economy}}

It's hard to know exactly how much bartering is happening in the United
States, said Brian Burke, an economic anthropologist who teaches in the
Sustainable Development Department at Appalachian State University in
Boone, N.C. That's because bartering is set up to be --- by design ---
outside our traditional cash-based economy.

``People often rely on Craigslist offerings as an indicator, though it's
a pretty weak one, especially in a time like this when people are
actually creating new online exchanges and mutual aid systems,'' he
said. In fact, the recent racial justice protests have probably
increased the amount of bartering happening throughout the United
States, although those participating in it may not have even realized
it. Take Ms. Coffman, for example. She wanted to participate in
Knoxville's Black Lives Matter protests, but she didn't feel safe
because of the pandemic. So she sewed masks and gave them to friends who
were marching. In essence, she traded her time and expertise as a
seamstress for her friends' willingness to be on the front lines of the
protests.

If your introduction to bartering came this spring when your grocery
store was low on supplies and you swapped rice for toilet paper, know
that bartering doesn't have to stop now that grocery stores are mostly
full again. In fact, Mr. Wood-Lewis said he thinks bartering is a
powerful community builder. In a time when many of us have limited
contact with the outside world, a friend repaying a small debt feels
like a connection tethering us to an otherwise just out-of-reach
community. So, as the economy slowly reopens consider continuing with
those neighborly trades --- who knows how far the recent
\href{https://www.nytimes.com/interactive/2020/us/states-reopen-map-coronavirus.html}{spate
of opening-reversals} will go.

Still, the rise in bartering is not emblematic of a total failure of the
capitalistic marketplace, according to Frederick Wherry, an economic
sociologist at Princeton University in Princeton, N.J. Instead, what's
probably happening is that we're reaching for any little bit of humanity
we can grasp.

``This is a way of being in community with other folks,'' Dr. Wherry
said.

\hypertarget{consider-why-youre-bartering}{%
\subsection{Consider why you're
bartering}\label{consider-why-youre-bartering}}

Dr. Burke studied bartering markets in Medellín, Colombia, for his
doctoral dissertation. He recalled a telling moment.

One day, someone brought a high-quality gas stove to the market. ``They
were trying to figure out how they wanted to trade it,'' he said. ``And
everybody had an opinion, right? So, some people said, `Oh, you should
just trade it for whoever gives you the best deal, or whoever is going
to give you the most useful stuff.'''

Others at the market, however, felt that this wasn't what bartering was
about.

``They argued that bartering is not about generating profit,'' Dr. Burke
said. ``It's about being part of an economic community.'' With that in
mind, the owners of the stove ended up trading it with a woman who was
getting it for her grandmother, who lived in the country and was using
an ancient wood stove for cooking.

``So, what they prioritized was not profit, but what they thought of as
the most just use of this product,'' Dr. Burke said.

Got masks to trade? Consider offering them first to friends deemed
essential workers, even if they might not have as much to offer you in
return.

\hypertarget{understand-our-lockdown-experiences-are-not-the-same}{%
\subsection{Understand our lockdown experiences are not the
same}\label{understand-our-lockdown-experiences-are-not-the-same}}

``One of the critiques of these systems is that they become kind of
little niche systems for middle-class people who want to experiment with
something different,'' Dr. Burke said. For people dealing with financial
uncertainty, there's less wiggle room for a bad trade while bartering.
``If they're going to invest time and resources in something like this,
they're going to have confidence that it's going to pay off,'' he said.

Before offering a trade to someone who you know is struggling, consider
what may be on the line for them and weigh what you ask for in return
carefully. Furthermore, don't be offended if they decline to barter with
you. Even if cash is scarce, buying from a store may feel more secure
for them, since they know exactly what they're getting --- and can make
returns without hassle.

\hypertarget{our-time-however-is-the-same}{%
\subsection{Our time, however, is the
same}\label{our-time-however-is-the-same}}

Back in 1980, Edgar Cahn, one of the leading advocates of the War on
Poverty, came up with timebanking. The premise is simple: Everyone's
time is of equal value. We all get the same number of hours in a day,
and so members of a timebank barter in hours. ``If it {[}a volunteer
task{]} takes me two hours, no matter what it is, I get two hours of
credit in my time bank,'' says Judith Lasker, an emeritus professor of
sociology at Lehigh University in Bethlehem, Pa., who wrote a book on
timebanking called ``Equal Time, Equal Value: Community Currencies and
Time Banking in the USA.''

In her research, Dr. Lasker found that participants enjoyed being part
of the organization and feeling as if their time was valued. Dr. Lasker
thinks that a moment like this --- when many people don't have enough
work, and some have too much --- could be the perfect moment for time
trading. Those out of work can bank their time by offering services like
plumbing, cooking, yard work or whatever talent they have. Those
juggling full-time parenting and employment can pay those hours back
later.

\hypertarget{just-gift-it}{%
\subsection{Just gift it}\label{just-gift-it}}

There's no shortage of research showing that generosity makes us feel
good. However, gifts are tricky, because there's almost always a feeling
of needing to reciprocate, says Beth Notar, a professor of anthropology
at Trinity College in Hartford, Conn.

Recently, her husband left two cans of semi-hard-to-get beer on a
neighbor's doorstep. A few days later, a single can of an even rarer
beer arrived on their doorstep. ``Our neighbor met my husband's gift
`challenge' so to speak, and maybe even upped the bar,'' she said.

Furthermore, whether you realize it or not, sometimes we use gifts to
establish social status. If you do want to give a gift, in these strange
and hard times, the kind gift giver will say upfront that no reciprocity
--- not even a thank-you note --- is needed, though don't be surprised
if your friend finds a way to reciprocate anyway.

\hypertarget{be-kind-if-things-dont-work-out}{%
\subsection{Be kind if things don't work
out}\label{be-kind-if-things-dont-work-out}}

If you barter with something now for something promised in the future,
that agreement may fall through. Things happen and people are going to
be particularly overwhelmed --- both financially and emotionally --- by
this crisis for quite some time, Dr. Wherry said.

And when it comes to getting a debt repaid, ``sometimes kindness can be
the quickest route to repayment,'' Dr. Wherry added. ``If you feel like
you're being investigated, all the walls go up,'' he said. But when
someone shows faith in your ability to get out of the mess you're in,
that helps. ``It's like, this person has faith in me, and I'm not going
to let them down,'' Dr. Wherry said. Finally, consider just letting the
barter be a gift.

``Keep in mind that letting things go means that you have an informal
credit,'' said Dr. Wherry, adding, not only will you be building good
will for the future, but you'll also feel good about having the ability
to let things go.

Advertisement

\protect\hyperlink{after-bottom}{Continue reading the main story}

\hypertarget{site-index}{%
\subsection{Site Index}\label{site-index}}

\hypertarget{site-information-navigation}{%
\subsection{Site Information
Navigation}\label{site-information-navigation}}

\begin{itemize}
\tightlist
\item
  \href{https://help.nytimes.com/hc/en-us/articles/115014792127-Copyright-notice}{©~2020~The
  New York Times Company}
\end{itemize}

\begin{itemize}
\tightlist
\item
  \href{https://www.nytco.com/}{NYTCo}
\item
  \href{https://help.nytimes.com/hc/en-us/articles/115015385887-Contact-Us}{Contact
  Us}
\item
  \href{https://www.nytco.com/careers/}{Work with us}
\item
  \href{https://nytmediakit.com/}{Advertise}
\item
  \href{http://www.tbrandstudio.com/}{T Brand Studio}
\item
  \href{https://www.nytimes.com/privacy/cookie-policy\#how-do-i-manage-trackers}{Your
  Ad Choices}
\item
  \href{https://www.nytimes.com/privacy}{Privacy}
\item
  \href{https://help.nytimes.com/hc/en-us/articles/115014893428-Terms-of-service}{Terms
  of Service}
\item
  \href{https://help.nytimes.com/hc/en-us/articles/115014893968-Terms-of-sale}{Terms
  of Sale}
\item
  \href{https://spiderbites.nytimes.com}{Site Map}
\item
  \href{https://help.nytimes.com/hc/en-us}{Help}
\item
  \href{https://www.nytimes.com/subscription?campaignId=37WXW}{Subscriptions}
\end{itemize}
