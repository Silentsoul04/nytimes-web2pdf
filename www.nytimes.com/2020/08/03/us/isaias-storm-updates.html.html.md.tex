Sections

SEARCH

\protect\hyperlink{site-content}{Skip to
content}\protect\hyperlink{site-index}{Skip to site index}

\href{https://www.nytimes.com/section/us}{U.S.}

\href{https://myaccount.nytimes.com/auth/login?response_type=cookie\&client_id=vi}{}

\href{https://www.nytimes.com/section/todayspaper}{Today's Paper}

\href{/section/us}{U.S.}\textbar{}Hurricane Makes Landfall in North
Carolina

\url{https://nyti.ms/39PGv6Z}

\begin{itemize}
\item
\item
\item
\item
\item
\item
\end{itemize}

Advertisement

\protect\hyperlink{after-top}{Continue reading the main story}

Supported by

\protect\hyperlink{after-sponsor}{Continue reading the main story}

\hypertarget{hurricane-makes-landfall-in-north-carolina}{%
\section{Hurricane Makes Landfall in North
Carolina}\label{hurricane-makes-landfall-in-north-carolina}}

Isaias regained strength into a Category 1 hurricane on Monday night,
bringing heavy winds and rain to the East Coast. Flash flooding and
tornadoes are possible.

Published Aug. 3, 2020Updated Aug. 4, 2020, 10:12 a.m. ET

\begin{itemize}
\item
\item
\item
\item
\item
\item
\end{itemize}

\emph{This briefing has ended.}
\href{https://www.nytimes.com/2020/08/04/us/isaias-storm-updates.html}{\emph{Click
here for live coverage of Isaias}}\emph{.}

\hypertarget{heres-what-you-need-to-know}{%
\subsubsection{Here's what you need to
know:}\label{heres-what-you-need-to-know}}

\begin{itemize}
\tightlist
\item
  \protect\hyperlink{link-34a2c843}{Isaias makes landfall in North
  Carolina after becoming a hurricane again.}
\item
  \protect\hyperlink{link-f7e08eb}{The Carolinas face the dual threat of
  the storm and the virus.}
\item
  \protect\hyperlink{link-64deafd3}{Isaias has tended to fluctuate, and
  so have the forecasts.}
\item
  \protect\hyperlink{link-536a9c31}{Emergency managers worry about
  communication during a pandemic.}
\item
  \protect\hyperlink{link-52f7dcb}{The Northeast can expect a soaking,
  too.}
\item
  \protect\hyperlink{link-303d84ae}{The head of Puerto Rico's power
  utility is resigning after widespread outages.}
\end{itemize}

\includegraphics{https://static01.nyt.com/images/2020/08/02/us/02isaias-briefing-lead2/02isaias-briefing-lead2-videoSixteenByNine3000.jpg}

\hypertarget{isaias-makes-landfall-in-north-carolina-after-becoming-a-hurricane-again}{%
\subsection{Isaias makes landfall in North Carolina after becoming a
hurricane
again.}\label{isaias-makes-landfall-in-north-carolina-after-becoming-a-hurricane-again}}

\href{https://www.nytimes.com/2020/08/04/nyregion/isaias-tropical-storm-nyc.html}{Hurricane
Isaias} made landfall in southern North Carolina late Monday night,
hours after strengthening into a Category 1 hurricane. Forecasters
warned of heavy rainfall and powerful winds as the storm travels up the
East Coast, and flash floods, storm surges and even tornadoes are
possible, the National Hurricane Center said.

The storm made landfall around 11:10 p.m. Eastern time near Ocean Isle
Beach, N.C., about halfway between Myrtle Beach, S.C., and Wilmington,
N.C. It has sustained maximum winds of 75 m.p.h., and a hurricane
warning was issued from the South Santee River in South Carolina to Surf
City, N.C.

Officials have told residents in the storm's projected path to prepare
themselves, and businesses are concerned about how much damage it will
bring.

``It's a wait-and-see game,'' said Jay Slevin, the manager of a pizzeria
about a mile from the shore in Myrtle Beach, S.C., southwest of where
Isaias made landfall.

\href{https://www.nytimes.com/interactive/2020/07/31/us/hurricane-isaias-tracker-map.html}{}

\includegraphics{https://static01.nyt.com/images/2020/07/31/us/hurricane-isaias-tracker-map-promo-1596209917104/hurricane-isaias-tracker-map-promo-1596209917104-articleLarge-v10.jpg}

\hypertarget{isaias-tracking-map}{%
\subsection{Isaias Tracking Map}\label{isaias-tracking-map}}

Follow the storm's path as it moves north along the Atlantic Coast.

The storm, the ninth to be named in what has become a busy hurricane
season, has come at a time when many people in the Southeast are already
beleaguered by the coronavirus outbreak. Officials in the region are
juggling the response to a storm with a pandemic, and business owners
are wary of being dealt yet another crippling blow.

Isaias, which is written as Isaías in Spanish and pronounced
ees-ah-EE-ahs, clobbered the Bahamas with hurricane conditions over the
weekend after hitting parts of Puerto Rico and the Dominican Republic.
Over the weekend, Isaias buffeted Florida's eastern edge with heavy
rainfall and powerful winds, yet it failed to deliver the punch that
state officials had feared.

\hypertarget{the-carolinas-face-the-dual-threat-of-the-storm-and-the-virus}{%
\subsection{The Carolinas face the dual threat of the storm and the
virus.}\label{the-carolinas-face-the-dual-threat-of-the-storm-and-the-virus}}

\includegraphics{https://static01.nyt.com/images/2020/08/03/autossell/30-north-carolina-gov/30-north-carolina-gov-videoSixteenByNine3000.png}

The center of Isaias hit the North Carolina coast on Monday, and is
expected to drive inland overnight, according to the National Hurricane
Center.

Rainfall will range from three to six inches in most areas, with a few
areas getting up to eight inches --- enough to produce flash flooding.
Widespread power outages are also expected.

To try to enforce social distancing, shelters in North Carolina will
give each evacuee 115 square feet of space, Gov. Roy Cooper of North
Carolina said. He encouraged people to evacuate to the homes of family
or friends, or to a hotel, if they can afford to, to keep shelters from
becoming crowded.

\hypertarget{live-updates-isaias}{%
\section{\texorpdfstring{\href{https://www.nytimes.com/2020/08/04/us/isaias-storm-updates.html?action=click\&pgtype=Article\&state=default\&region=MAIN_CONTENT_1\&context=storylines_live_updates}{Live
Updates: Isaias}}{Live Updates: Isaias}}\label{live-updates-isaias}}

Updated 2020-08-04T19:32:28.621Z

\begin{itemize}
\tightlist
\item
  \href{https://www.nytimes.com/2020/08/04/us/isaias-storm-updates.html?action=click\&pgtype=Article\&state=default\&region=MAIN_CONTENT_1\&context=storylines_live_updates\#link-362830dd}{Isaias
  is bringing the threat of tornadoes as it barrels north.}
\item
  \href{https://www.nytimes.com/2020/08/04/us/isaias-storm-updates.html?action=click\&pgtype=Article\&state=default\&region=MAIN_CONTENT_1\&context=storylines_live_updates\#link-7961bdbc}{At
  least two people were killed by a tornado in North Carolina.}
\item
  \href{https://www.nytimes.com/2020/08/04/us/isaias-storm-updates.html?action=click\&pgtype=Article\&state=default\&region=MAIN_CONTENT_1\&context=storylines_live_updates\#link-34e5d4e4}{The
  storm is knocking out power over wide areas.}
\end{itemize}

\href{https://www.nytimes.com/2020/08/04/us/isaias-storm-updates.html?action=click\&pgtype=Article\&state=default\&region=MAIN_CONTENT_1\&context=storylines_live_updates}{See
more updates}

``I know that North Carolinians have had to dig deep in recent months to
tap into our strength and resilience during the pandemic, and that
hasn't been easy,'' he said. ``But with this storm on the way, we have
to dig a little deeper.''

\href{https://www.nytimes.com/interactive/2020/us/north-carolina-coronavirus-cases.html}{}

\includegraphics{https://static01.nyt.com/images/2020/03/29/us/north-carolina-coronavirus-cases-promo-1585539326617/north-carolina-coronavirus-cases-promo-1585539326617-articleLarge-v118.png}

\hypertarget{north-carolina-coronavirus-map-and-case-count}{%
\subsection{North Carolina Coronavirus Map and Case
Count}\label{north-carolina-coronavirus-map-and-case-count}}

A detailed county map shows the extent of the coronavirus outbreak, with
tables of the number of cases by county.

In South Carolina, Myrtle Beach was expected to see the brunt of the
storm on Monday night, when the rain will increase and the risk of flash
floods will be greatest. There could also be a storm surge of three to
five feet, and a possibility of tornadoes.

Even before the storm hit, a swimmer was reported missing at Myrtle
Beach. A witness said they had seen a swimmer in distress around 8 p.m.
Sunday, and despite crews searching in the water and using helicopters,
the swimmer had not been found by Monday morning, when it became too
dangerous for crews to remain in the water.

Gov. Henry McMaster of South Carolina said on Friday that he had no
plans to call for evacuations. But North Carolina has declared a state
of emergency.

\hypertarget{isaias-has-tended-to-fluctuate-and-so-have-the-forecasts}{%
\subsection{Isaias has tended to fluctuate, and so have the
forecasts.}\label{isaias-has-tended-to-fluctuate-and-so-have-the-forecasts}}

\includegraphics{https://static01.nyt.com/images/2020/08/03/us/03isiaias-briefing-flucture2/merlin_175275501_94f3760a-2c73-4c54-82fb-7f32eb732268-articleLarge.jpg?quality=75\&auto=webp\&disable=upscale}

\emph{Why have predictions for Isaias seemed so changeable?}
\emph{\textbf{Adam Sobel}}*, an atmospheric scientist, professor and
director of the Initiative on Extreme Weather and Climate at Columbia
University, explains.*

Isaias has been a tricky storm since it formed. Actually, it was tricky
even before it formed, when forecasts benefited from a practice that the
National Hurricane Center began three years ago.

The center's meteorologists have always looked for weather systems in
the Atlantic that could become tropical cyclones. But before 2017, they
did not start issuing advisories about likely tracks and intensities
until the storms actually formed. That left a big hole in the center's
warning system: The public heard days in advance about storms that
developed far out to sea, but got much less notice for those forming
close to shore.

The center patched that hole by starting to flag ``potential tropical
cyclones'' that could reach land within 48 hours, even though they were
still just an idea in the minds of forecasters. Now the public gets the
word earlier, though less definitively.

\textbf{A storm is born:} When a low-pressure system that was dithering
over the tropical Atlantic last week posed a threat to Puerto Rico and
the island of Hispaniola, the center designated it Potential Tropical
Cyclone Nine and started issuing forecasts and warnings. The system
formed into Isaias, but it was far from clear yet what its future held.

Isaias weakened while passing over the mountainous Dominican Republic,
as storms generally do, but it strengthened more quickly than expected
afterward, and by the time it reached the Bahamas on Friday it was a
Category 1 hurricane.

At that point, the forecast track threatened nearly the whole Eastern
Seaboard of the United States, from South Florida to Maine. The storm
could have affected almost anywhere, everywhere or nowhere along that
track, as far as we could tell.

\textbf{A near miss:} Florida, the closest potential target, braced for
a hurricane, but as the weekend progressed, it gradually became clear
that the storm would only graze the state as a ragged tropical storm
that seemed likely to stay that way until landfall in the Carolinas.

But Isaias reorganized yet again, reaching hurricane strength again on
Monday night before making landfall in North Carolina.

\textbf{A helping hand:} At the same time, the wind shear that used to
look as though it would diminish the storm may now sustain it. The
hurricane center noted for the first time on Monday morning that ``an
unusually strong winter-type Jetstream'' would produce ``strong
baroclinic forcing'' --- meteorologist-speak for what drives nontropical
storms like nor'easters --- and would ``produce very strong wind gusts
along the Mid-Atlantic states.''

That is why, a day before Isaias is expected to reach New York City, we
now have a forecast for hurricane-strength gusts in the area, with the
potential for widespread power outages and other problems that were not
on the radar, literally or figuratively, until today.

The forecasts issued for Isaias and other tricky storms these days are
amazingly good by historical standards, much better than a few decades
ago. But as they raise our expectations and turn ``unknown unknowns''
into ``known unknowns,'' they can still confuse and disorient us.

\hypertarget{emergency-managers-worry-about-communication-during-a-pandemic}{%
\subsection{Emergency managers worry about communication during a
pandemic.}\label{emergency-managers-worry-about-communication-during-a-pandemic}}

Image

Filling sandbags in Virginia Beach on Monday in preparation for
Isaias.Credit...Stephen M. Katz/The Virginian-Pilot, via Associated
Press

W. Craig Fugate, a former administrator of the Federal Emergency
Management Agency, said his biggest concern this hurricane season is
that coastal residents will stay home to avoid the coronavirus even if
they face a real storm surge risk.

``We often talk about evacuations, and we don't really clarify why we're
evacuating,'' he said. ``People drown. And we don't say that.''

``Covid is scary,'' added Mr. Fugate, a Florida resident who once ran
the state's division of emergency management. ``For a lot of people,
they're thinking, `You know, evacuation, maybe that's not so critical.'
We need to be clear and precise: The reason we do evacuations is
drownings.''

Gov. Ron DeSantis of Florida said on Monday that the brush from Isaias
gave officials a trial run for how to deal with sick evacuees. In Palm
Beach County, for example, people who arrived at a shelter with a recent
positive coronavirus test result or with a high temperature were sent to
a nearby hotel instead.

``They had a safe place to stay until the storm passed,'' the governor
said.

On a more positive note, Mr. Fugate said, virus contagion fears could
also keep people who do not need to evacuate off the roads.

``The fewer people that are not in evacuation zones that leave, the
better for people who do need to leave,'' he said.

And he offered this advice: ``Wear a mask. Pack masks. If you're
evacuating, take masks with you. If you're out shopping: Wear a mask.''

\hypertarget{the-northeast-can-expect-a-soaking-too}{%
\subsection{The Northeast can expect a soaking,
too.}\label{the-northeast-can-expect-a-soaking-too}}

Much of the East Coast of the United States will get a soaking,
forecasters say. The National Hurricane Center said on Monday that
tropical storm warnings and watches were in effect all way up the
Eastern Seaboard, including Martha's Vineyard, Mass., and Stonington,
Maine.

With three to six inches expected across the eastern Carolinas and
Virginia and isolated areas getting up to eight inches, significant
flash floods and urban flooding is can be expected through the middle of
the week, and widespread minor to moderate river flooding is possible in
the region. The rain could be at its heaviest in the Chesapeake Bay
region of Maryland, forecasters said, with as much as seven inches
falling there in just eight hours.

``People don't realize it, but in the Mid-Atlantic and a lot of areas,
flooding actually causes the most loss of life and damage,'' said Jeremy
Geiger, a meteorologist with the National Weather Service. ``So be aware
of where you live, and what's going on.''

Heavy rainfall in northeast New Jersey, New York City and the lower
Hudson Valley was expected to begin late Monday night, building into
heavier downpours by Tuesday afternoon and evening, according to Matthew
Wunsch, a meteorologist with the National Weather Service. Emergency
management officials in New York City said the storm might bring three
to six inches of rain in some areas.

Winds are expected to pick up on Tuesday afternoon, he said. Sustained
winds could be between 30 to 45 m.p.h., with gusts up to 65 m.p.h.

Tuesday night could bring the possibility of flooding along the southern
coast of Long Island and the New Jersey coastline near New York City,
Mr. Wunsch said. He said coastal flooding was expected to coincide with
high tide, which is between 10 p.m. and 1 a.m. on Tuesday, bringing an
additional one to two feet of storm surge. New York City said that it
would close all city-run beaches to swimming on Tuesday, though surfing
will be permitted in certain areas, officials said.

Gov. Andrew Cuomo said on Sunday that the state was deploying high-water
vehicles, pumps and generators to areas that might be affected by the
storm.

Storm surge could also bring high water into Lower Manhattan, according
to the New York City Emergency Management Department, and officials are
deploying sand bags and other barriers in the area.

\hypertarget{the-head-of-puerto-ricos-power-utility-is-resigning-after-widespread-outages}{%
\subsection{The head of Puerto Rico's power utility is resigning after
widespread
outages.}\label{the-head-of-puerto-ricos-power-utility-is-resigning-after-widespread-outages}}

Image

Power lines were down in Mayaguez, P.R., after Isaias moved through the
area last week.Credit...Ricardo Arduengo/Agence France-Presse --- Getty
Images

The resignation of José Ortiz, the executive director of the Puerto Rico
Electric Power Authority, known as PREPA, will be effective on
Wednesday, the utility's governing board said in a statement on Monday.

The statement praised Mr. Ortiz for his work over the past two years but
did not mention that tens of thousands of PREPA customers were left
without electricity after Isaias barreled past Puerto Rico late last
week. The outages exposed the persistent weakness of the island's power
grid, which had fallen into disrepair even before Hurricane Maria
devastated it in 2017.

Last week, a blackout unrelated to Isaias began before the storm hit and
left more than 300,000 of the utility's 1.5 million customers without
power. Another 400,000 customers lost electricity after the storm.

Mr. Ortiz was appointed in July 2018 as the utility struggled to recover
from bankruptcy and Hurricane Maria. He said on Monday that at the time
of his hiring, he had committed to the job for two years.

Reporting was contributed by Michael Gold, Rebecca Halleck, Patricia
Mazzei, Rick Rojas, Lucy Tompkins and Mihir Zaveri.

Advertisement

\protect\hyperlink{after-bottom}{Continue reading the main story}

\hypertarget{site-index}{%
\subsection{Site Index}\label{site-index}}

\hypertarget{site-information-navigation}{%
\subsection{Site Information
Navigation}\label{site-information-navigation}}

\begin{itemize}
\tightlist
\item
  \href{https://help.nytimes.com/hc/en-us/articles/115014792127-Copyright-notice}{©~2020~The
  New York Times Company}
\end{itemize}

\begin{itemize}
\tightlist
\item
  \href{https://www.nytco.com/}{NYTCo}
\item
  \href{https://help.nytimes.com/hc/en-us/articles/115015385887-Contact-Us}{Contact
  Us}
\item
  \href{https://www.nytco.com/careers/}{Work with us}
\item
  \href{https://nytmediakit.com/}{Advertise}
\item
  \href{http://www.tbrandstudio.com/}{T Brand Studio}
\item
  \href{https://www.nytimes.com/privacy/cookie-policy\#how-do-i-manage-trackers}{Your
  Ad Choices}
\item
  \href{https://www.nytimes.com/privacy}{Privacy}
\item
  \href{https://help.nytimes.com/hc/en-us/articles/115014893428-Terms-of-service}{Terms
  of Service}
\item
  \href{https://help.nytimes.com/hc/en-us/articles/115014893968-Terms-of-sale}{Terms
  of Sale}
\item
  \href{https://spiderbites.nytimes.com}{Site Map}
\item
  \href{https://help.nytimes.com/hc/en-us}{Help}
\item
  \href{https://www.nytimes.com/subscription?campaignId=37WXW}{Subscriptions}
\end{itemize}
