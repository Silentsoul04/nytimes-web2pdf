Sections

SEARCH

\protect\hyperlink{site-content}{Skip to
content}\protect\hyperlink{site-index}{Skip to site index}

\href{https://www.nytimes.com/section/us}{U.S.}

\href{https://myaccount.nytimes.com/auth/login?response_type=cookie\&client_id=vi}{}

\href{https://www.nytimes.com/section/todayspaper}{Today's Paper}

\href{/section/us}{U.S.}\textbar{}The Mask Slackers of 1918

\url{https://nyti.ms/2XpK4vD}

\begin{itemize}
\item
\item
\item
\item
\item
\item
\end{itemize}

\href{https://www.nytimes.com/news-event/coronavirus?action=click\&pgtype=Article\&state=default\&region=TOP_BANNER\&context=storylines_menu}{The
Coronavirus Outbreak}

\begin{itemize}
\tightlist
\item
  live\href{https://www.nytimes.com/2020/08/04/world/coronavirus-cases.html?action=click\&pgtype=Article\&state=default\&region=TOP_BANNER\&context=storylines_menu}{Latest
  Updates}
\item
  \href{https://www.nytimes.com/interactive/2020/us/coronavirus-us-cases.html?action=click\&pgtype=Article\&state=default\&region=TOP_BANNER\&context=storylines_menu}{Maps
  and Cases}
\item
  \href{https://www.nytimes.com/interactive/2020/science/coronavirus-vaccine-tracker.html?action=click\&pgtype=Article\&state=default\&region=TOP_BANNER\&context=storylines_menu}{Vaccine
  Tracker}
\item
  \href{https://www.nytimes.com/2020/08/02/us/covid-college-reopening.html?action=click\&pgtype=Article\&state=default\&region=TOP_BANNER\&context=storylines_menu}{College
  Reopening}
\item
  \href{https://www.nytimes.com/live/2020/08/04/business/stock-market-today-coronavirus?action=click\&pgtype=Article\&state=default\&region=TOP_BANNER\&context=storylines_menu}{Economy}
\end{itemize}

Advertisement

\protect\hyperlink{after-top}{Continue reading the main story}

Supported by

\protect\hyperlink{after-sponsor}{Continue reading the main story}

\hypertarget{the-mask-slackers-of-1918}{%
\section{The Mask Slackers of 1918}\label{the-mask-slackers-of-1918}}

As the influenza pandemic swept across the United States in 1918 and
1919, masks took a role in political and cultural wars.

\includegraphics{https://static01.nyt.com/images/2020/07/16/multimedia/00xp-1918masks-03/00xp-1918masks-03-articleLarge.jpg?quality=75\&auto=webp\&disable=upscale}

\href{https://www.nytimes.com/by/christine-hauser}{\includegraphics{https://static01.nyt.com/images/2018/02/16/multimedia/author-christine-hauser/author-christine-hauser-thumbLarge.jpg}}

By \href{https://www.nytimes.com/by/christine-hauser}{Christine Hauser}

\begin{itemize}
\item
  Published Aug. 3, 2020Updated Aug. 4, 2020, 9:16 a.m. ET
\item
  \begin{itemize}
  \item
  \item
  \item
  \item
  \item
  \item
  \end{itemize}
\end{itemize}

The masks were called muzzles, germ shields and dirt traps. They gave
people a ``pig-like snout.'' Some people snipped holes in their masks to
smoke cigars. Others fastened them to dogs in mockery. Bandits used them
to rob banks.

More than a century ago, as the 1918 influenza pandemic raged in the
United States, masks of gauze and cheesecloth became the facial front
lines in the battle against the virus. But as they have now, the masks
also stoked political division. Then, as now, medical authorities urged
the wearing of masks to help slow the spread of disease. And then, as
now, some people resisted.

In 1918 and 1919, as bars, saloons, restaurants, theaters and schools
were closed, masks became a scapegoat, a symbol of government overreach,
inspiring protests, petitions and defiant bare-face gatherings. All the
while, thousands of Americans were dying in a deadly pandemic.

\hypertarget{1918-the-infection-spreads}{%
\subsection{1918: The infection
spreads.}\label{1918-the-infection-spreads}}

The first infections were identified in March, at an Army base in
Kansas, where 100 soldiers were infected. Within a week, the
\href{https://www.cdc.gov/flu/pandemic-resources/1918-commemoration/pandemic-timeline-1918.htm}{number
of flu cases grew fivefold,} and soon the disease was taking hold across
the country, prompting some cities to impose quarantines and mask orders
to contain it.

By the fall of 1918, seven cities --- San Francisco, Seattle, Oakland,
Sacramento, Denver, Indianapolis and Pasadena, Calif. --- had put in
effect mandatory face mask laws, said
\href{http://chm.med.umich.edu/about/howard-markel-m-d-ph-d/}{Dr. Howard
Markel}, a historian of epidemics and the author of
``\href{https://jhupbooks.press.jhu.edu/title/quarantine}{Quarantine!}''

Organized resistance to mask wearing was not common, Dr. Markel said,
but it was present. ``There were flare-ups, there were scuffles and
there were occasional groups, like the Anti-Mask League,'' he said,
``but that is the exception rather than the rule.''

At the forefront of the safety measures was San Francisco, where a man
returning from a trip to Chicago apparently carried the virus home,
according to
\href{http://www.influenzaarchive.org/cities/city-sanfrancisco.html}{archives
about the pandemic at the University of Michigan.}

By the end of October, there were more than 60,000 cases statewide, with
7,000 of them in San Francisco. It soon became known as the ``masked
city.''

\includegraphics{https://static01.nyt.com/images/2020/07/16/multimedia/00xp-1918masks-06/00xp-1918masks-06-articleLarge.jpg?quality=75\&auto=webp\&disable=upscale}

``The Mask Ordinance,'' signed by Mayor James Rolph on Oct. 22, made San
Francisco the first American city to require face coverings, which had
to be four layers thick.

\hypertarget{masks-that-looked-like-slabs-of-ravioli}{%
\subsection{Masks that looked like `slabs of
ravioli'}\label{masks-that-looked-like-slabs-of-ravioli}}

Resisters complained about appearance, comfort and freedom, even after
the flu
\href{https://www.cdc.gov/flu/pandemic-resources/1918-commemoration/pandemic-timeline-1918.htm}{killed
an estimated 195,000 Americans in October alone.}

Alma Whitaker, writing in The Los Angeles Times on Oct. 22, 1918,
reviewed masks' impact on society and celebrity, saying famous people
shunned them because it was ``so horrid'' to go unrecognized.

``The big restaurants are the funniest sights, with all the waiters and
diners masked, the latter just raising their screen to pop in a mouthful
of food,'' she wrote.

\hypertarget{latest-updates-global-coronavirus-outbreak}{%
\section{\texorpdfstring{\href{https://www.nytimes.com/2020/08/04/world/coronavirus-cases.html?action=click\&pgtype=Article\&state=default\&region=MAIN_CONTENT_1\&context=storylines_live_updates}{Latest
Updates: Global Coronavirus
Outbreak}}{Latest Updates: Global Coronavirus Outbreak}}\label{latest-updates-global-coronavirus-outbreak}}

Updated 2020-08-04T18:14:55.559Z

\begin{itemize}
\tightlist
\item
  \href{https://www.nytimes.com/2020/08/04/world/coronavirus-cases.html?action=click\&pgtype=Article\&state=default\&region=MAIN_CONTENT_1\&context=storylines_live_updates\#link-4d1eafa8}{N.Y.C.'s
  health commissioner resigns after clashing with the mayor over the
  virus.}
\item
  \href{https://www.nytimes.com/2020/08/04/world/coronavirus-cases.html?action=click\&pgtype=Article\&state=default\&region=MAIN_CONTENT_1\&context=storylines_live_updates\#link-18bf040e}{Public
  and private schools are dividing over in-person instruction in
  Maryland and elsewhere.}
\item
  \href{https://www.nytimes.com/2020/08/04/world/coronavirus-cases.html?action=click\&pgtype=Article\&state=default\&region=MAIN_CONTENT_1\&context=storylines_live_updates\#link-6b644638}{`Long
  days, long nights': Washington prepares for a prolonged fight over
  virus relief.}
\end{itemize}

\href{https://www.nytimes.com/2020/08/04/world/coronavirus-cases.html?action=click\&pgtype=Article\&state=default\&region=MAIN_CONTENT_1\&context=storylines_live_updates}{See
more updates}

More live coverage:
\href{https://www.nytimes.com/live/2020/08/04/business/stock-market-today-coronavirus?action=click\&pgtype=Article\&state=default\&region=MAIN_CONTENT_1\&context=storylines_live_updates}{Markets}

When Ms. Whitaker herself declined to wear one, she was ``forcibly
taken'' to the Red Cross as a ``slacker,'' and ordered to make one and
put it on.

Image

A policeman wearing a flu mask talking to a couple, one masked, one
not.Credit...Hamilton Henry Dobbin, via California State Library

The San Francisco Chronicle said the simplest type of mask was of folded
gauze affixed with elastic or tape. The police went for gauze masks,
which resembled an unflattering ``nine ordinary slabs of ravioli
arranged in a square.''

There was room for creativity. Some of the coverings were ``fearsome
looking machines'' that lent a ``pig-like aspect'' to the wearer's face.

\hypertarget{mask-court}{%
\subsection{Mask court}\label{mask-court}}

The penalty for violators was \$5 to \$10, or 10 days' imprisonment.

On Nov. 9, 1,000 people were arrested, The San Francisco Chronicle
reported. City prisons swelled to standing room only; police shifts and
court sessions were added to help manage.

``Where is your mask?'' Judge Mathew Brady asked offenders at the Hall
of Justice, where sessions dragged into night. Some gave fake names,
said they just wanted to light a cigar or that they hated following
laws.

Jail terms of 8 hours to 10 days were given out. Those who could not pay
\$5 were jailed for 48 hours.

Image

Rail commuters wearing white protective masks, one with the additional
message ``wear a mask or go to jail,'' during the 1918 influenza
pandemic in California.Credit...Vintage Space/Alamy

\hypertarget{the-mask-slacker-of-san-francisco-is-shot}{%
\subsection{The `mask slacker' of San Francisco is
shot.}\label{the-mask-slacker-of-san-francisco-is-shot}}

On Oct. 28, a blacksmith named James Wisser stood on Powell and Market
streets in front of a drugstore, urging a crowd to dispose of their
masks, which he described as ``bunk.''

A health inspector, Henry D. Miller, led him to the drugstore to buy a
mask.

At the door, Mr. Wisser struck Mr. Miller with a sack of silver dollars
and knocked him to the ground, The San Francisco Chronicle reported.
While being ``pummeled,'' Mr. Miller, 62, fired four times with a
revolver. Passers-by ``scurried for cover,'' The Associated Press said.

Mr. Wisser was injured, as were two bystanders. He was charged with
disturbing the peace, resisting an officer and assault. The inspector
was charged with assault with a deadly weapon.

\hypertarget{in-los-angeles-to-mask-or-not-to-mask}{%
\subsection{In Los Angeles, `To Mask or Not to
Mask.'}\label{in-los-angeles-to-mask-or-not-to-mask}}

That was the headline for a report published in The Los Angeles Times
when city officials met in November to decide whether to require
residents to wear ``germ scarers'' or ``flu-scarers.''

Public feedback was invited. Some supported masks so theaters, churches
and schools could operate. Opponents said masks were ``mere dirt and
dust traps and do more harm than good.''

``I have seen some persons wearing their masks for a while hanging about
their necks, and then apply them to their faces, forgetting that they
might have picked up germs while dangling about their clothes,'' Dr.
E.W. Fleming said in a Los Angeles Times report.

An ear, nose and throat specialist, Dr. John J. Kyle, said: ``I saw a
woman in a restaurant today with a mask on. She was in ordinary street
clothes, and every now and then she raised her hand to her face and
fussed with the mask.''

\hypertarget{in-illinois-the-right-to-choose-and-to-reject}{%
\subsection{In Illinois, the right to choose, and to
reject.}\label{in-illinois-the-right-to-choose-and-to-reject}}

Suffragists fighting for the right to vote made a gesture that rejected
covering their mouths at a time when their voices were crucial.

At the annual convention of the Illinois Equal Suffrage Association, in
October 1918, they set chairs four feet apart, closed doors to the
public and limited attendance to 100 delegates, the Chicago Daily
Tribune reported.

Image

Headlines from newspapers in Chicago.Credit...Chicago History
Museum/Getty Images

But the women ``showed their scorn'' for masks, it said. It's unclear
why.

Allison K. Lange, an associate history professor at Wentworth Institute
of Technology, said one reason could have been that they wanted to keep
a highly visible profile.

``Suffragists wanted to make sure their leaders were familiar political
figures,'' Dr. Lange said.

\hypertarget{four-weeks-of-muzzled-misery}{%
\subsection{`Four weeks of muzzled
misery'}\label{four-weeks-of-muzzled-misery}}

San Francisco's mask ordinance expired after four weeks at noon on Nov.
21. The city celebrated, and church bells tolled.

A ``delinquent'' bent on blowing his nose tore his mask off so quickly
that it ``nearly ruptured his ear,'' The San Francisco Chronicle
reported. He and others stomped on their masks in the street. As a
police officer watched, it dawned on him that ``his vigil over the masks
was done.''

Waiters, barkeeps and others bared their faces. Drinks were on the
house. Ice cream shops handed out treats. The sidewalks were strewn with
gauze, the ``relics of a torturous month,'' The Chronicle said.

\href{https://www.nytimes.com/news-event/coronavirus?action=click\&pgtype=Article\&state=default\&region=MAIN_CONTENT_3\&context=storylines_faq}{}

\hypertarget{the-coronavirus-outbreak-}{%
\subsubsection{The Coronavirus Outbreak
›}\label{the-coronavirus-outbreak-}}

\hypertarget{frequently-asked-questions}{%
\paragraph{Frequently Asked
Questions}\label{frequently-asked-questions}}

Updated August 4, 2020

\begin{itemize}
\item ~
  \hypertarget{i-have-antibodies-am-i-now-immune}{%
  \paragraph{I have antibodies. Am I now
  immune?}\label{i-have-antibodies-am-i-now-immune}}

  \begin{itemize}
  \tightlist
  \item
    As of right
    now,\href{https://www.nytimes.com/2020/07/22/health/covid-antibodies-herd-immunity.html?action=click\&pgtype=Article\&state=default\&region=MAIN_CONTENT_3\&context=storylines_faq}{that
    seems likely, for at least several months.} There have been
    frightening accounts of people suffering what seems to be a second
    bout of Covid-19. But experts say these patients may have a
    drawn-out course of infection, with the virus taking a slow toll
    weeks to months after initial exposure. People infected with the
    coronavirus typically
    \href{https://www.nature.com/articles/s41586-020-2456-9}{produce}
    immune molecules called antibodies, which are
    \href{https://www.nytimes.com/2020/05/07/health/coronavirus-antibody-prevalence.html?action=click\&pgtype=Article\&state=default\&region=MAIN_CONTENT_3\&context=storylines_faq}{protective
    proteins made in response to an
    infection}\href{https://www.nytimes.com/2020/05/07/health/coronavirus-antibody-prevalence.html?action=click\&pgtype=Article\&state=default\&region=MAIN_CONTENT_3\&context=storylines_faq}{.
    These antibodies may} last in the body
    \href{https://www.nature.com/articles/s41591-020-0965-6}{only two to
    three months}, which may seem worrisome, but that's perfectly normal
    after an acute infection subsides, said Dr. Michael Mina, an
    immunologist at Harvard University. It may be possible to get the
    coronavirus again, but it's highly unlikely that it would be
    possible in a short window of time from initial infection or make
    people sicker the second time.
  \end{itemize}
\item ~
  \hypertarget{im-a-small-business-owner-can-i-get-relief}{%
  \paragraph{I'm a small-business owner. Can I get
  relief?}\label{im-a-small-business-owner-can-i-get-relief}}

  \begin{itemize}
  \tightlist
  \item
    The
    \href{https://www.nytimes.com/article/small-business-loans-stimulus-grants-freelancers-coronavirus.html?action=click\&pgtype=Article\&state=default\&region=MAIN_CONTENT_3\&context=storylines_faq}{stimulus
    bills enacted in March} offer help for the millions of American
    small businesses. Those eligible for aid are businesses and
    nonprofit organizations with fewer than 500 workers, including sole
    proprietorships, independent contractors and freelancers. Some
    larger companies in some industries are also eligible. The help
    being offered, which is being managed by the Small Business
    Administration, includes the Paycheck Protection Program and the
    Economic Injury Disaster Loan program. But lots of folks have
    \href{https://www.nytimes.com/interactive/2020/05/07/business/small-business-loans-coronavirus.html?action=click\&pgtype=Article\&state=default\&region=MAIN_CONTENT_3\&context=storylines_faq}{not
    yet seen payouts.} Even those who have received help are confused:
    The rules are draconian, and some are stuck sitting on
    \href{https://www.nytimes.com/2020/05/02/business/economy/loans-coronavirus-small-business.html?action=click\&pgtype=Article\&state=default\&region=MAIN_CONTENT_3\&context=storylines_faq}{money
    they don't know how to use.} Many small-business owners are getting
    less than they expected or
    \href{https://www.nytimes.com/2020/06/10/business/Small-business-loans-ppp.html?action=click\&pgtype=Article\&state=default\&region=MAIN_CONTENT_3\&context=storylines_faq}{not
    hearing anything at all.}
  \end{itemize}
\item ~
  \hypertarget{what-are-my-rights-if-i-am-worried-about-going-back-to-work}{%
  \paragraph{What are my rights if I am worried about going back to
  work?}\label{what-are-my-rights-if-i-am-worried-about-going-back-to-work}}

  \begin{itemize}
  \tightlist
  \item
    Employers have to provide
    \href{https://www.osha.gov/SLTC/covid-19/standards.html}{a safe
    workplace} with policies that protect everyone equally.
    \href{https://www.nytimes.com/article/coronavirus-money-unemployment.html?action=click\&pgtype=Article\&state=default\&region=MAIN_CONTENT_3\&context=storylines_faq}{And
    if one of your co-workers tests positive for the coronavirus, the
    C.D.C.} has said that
    \href{https://www.cdc.gov/coronavirus/2019-ncov/community/guidance-business-response.html}{employers
    should tell their employees} -\/- without giving you the sick
    employee's name -\/- that they may have been exposed to the virus.
  \end{itemize}
\item ~
  \hypertarget{should-i-refinance-my-mortgage}{%
  \paragraph{Should I refinance my
  mortgage?}\label{should-i-refinance-my-mortgage}}

  \begin{itemize}
  \tightlist
  \item
    \href{https://www.nytimes.com/article/coronavirus-money-unemployment.html?action=click\&pgtype=Article\&state=default\&region=MAIN_CONTENT_3\&context=storylines_faq}{It
    could be a good idea,} because mortgage rates have
    \href{https://www.nytimes.com/2020/07/16/business/mortgage-rates-below-3-percent.html?action=click\&pgtype=Article\&state=default\&region=MAIN_CONTENT_3\&context=storylines_faq}{never
    been lower.} Refinancing requests have pushed mortgage applications
    to some of the highest levels since 2008, so be prepared to get in
    line. But defaults are also up, so if you're thinking about buying a
    home, be aware that some lenders have tightened their standards.
  \end{itemize}
\item ~
  \hypertarget{what-is-school-going-to-look-like-in-september}{%
  \paragraph{What is school going to look like in
  September?}\label{what-is-school-going-to-look-like-in-september}}

  \begin{itemize}
  \tightlist
  \item
    It is unlikely that many schools will return to a normal schedule
    this fall, requiring the grind of
    \href{https://www.nytimes.com/2020/06/05/us/coronavirus-education-lost-learning.html?action=click\&pgtype=Article\&state=default\&region=MAIN_CONTENT_3\&context=storylines_faq}{online
    learning},
    \href{https://www.nytimes.com/2020/05/29/us/coronavirus-child-care-centers.html?action=click\&pgtype=Article\&state=default\&region=MAIN_CONTENT_3\&context=storylines_faq}{makeshift
    child care} and
    \href{https://www.nytimes.com/2020/06/03/business/economy/coronavirus-working-women.html?action=click\&pgtype=Article\&state=default\&region=MAIN_CONTENT_3\&context=storylines_faq}{stunted
    workdays} to continue. California's two largest public school
    districts --- Los Angeles and San Diego --- said on July 13, that
    \href{https://www.nytimes.com/2020/07/13/us/lausd-san-diego-school-reopening.html?action=click\&pgtype=Article\&state=default\&region=MAIN_CONTENT_3\&context=storylines_faq}{instruction
    will be remote-only in the fall}, citing concerns that surging
    coronavirus infections in their areas pose too dire a risk for
    students and teachers. Together, the two districts enroll some
    825,000 students. They are the largest in the country so far to
    abandon plans for even a partial physical return to classrooms when
    they reopen in August. For other districts, the solution won't be an
    all-or-nothing approach.
    \href{https://bioethics.jhu.edu/research-and-outreach/projects/eschool-initiative/school-policy-tracker/}{Many
    systems}, including the nation's largest, New York City, are
    devising
    \href{https://www.nytimes.com/2020/06/26/us/coronavirus-schools-reopen-fall.html?action=click\&pgtype=Article\&state=default\&region=MAIN_CONTENT_3\&context=storylines_faq}{hybrid
    plans} that involve spending some days in classrooms and other days
    online. There's no national policy on this yet, so check with your
    municipal school system regularly to see what is happening in your
    community.
  \end{itemize}
\end{itemize}

The spread had been halted. But a second wave was on the horizon.

By December, the San Francisco Board of Supervisors was again proposing
a mask requirement, meeting with testy opposition.

Image

Police Court Officials of San Francisco holding a session in the open,
as a precaution against the spreading influenza epidemic in late
November of 1918.Credit...National Archives

Around the end of the year, a bomb was defused outside the office of San
Francisco's chief health officer, Dr. William C. Hassler. ``Things were
violent and aggressive, but it was because people were losing money,''
said Brian Dolan, a medical historian at the University of California,
San Francisco. ``It wasn't about a constitutional issue; it was a money
issue.''

By the end of 1918, the death toll from influenza had reached at least
244,681, mostly in the last four months, according to government
\href{https://www.cdc.gov/nchs/data/vsushistorical/mortstatsh_1918.pdf}{statistics.}

\hypertarget{1919-a-new-year}{%
\subsection{1919: A new year}\label{1919-a-new-year}}

In January, Pasadena's city commission passed a mask ordinance. The
police grudgingly enforced it, cracking down on cigar smokers and
passengers in cars. Sixty people were arrested on the first day, The Los
Angeles Times reported on Jan. 22, in an article titled ``Pasadena
Snorts Under Masks.''

``It is the most unpopular law ever placed on the Pasadena records,''
W.S. McIntyre, the chief of police, told the paper. ``We are cursed from
all sides.''

Some mocked the rule by stretching gauze across car vents or dog snouts.
Cigar vendors said they lost customers, though enterprising aficionados
cut a hole in the cloth. (They were still arrested.) Barbers lost
shaving business. Merchants complained traffic dropped as more people
stayed home.

Petitions were circulated at cigar stands. Arrests rose, even of the
powerful. Ernest May, the president of Security National Bank of
Pasadena, and five ``prominent'' guests were rounded up at the Maryland
Hotel one Sunday.

They had masks on, but not covering their faces.

\hypertarget{the-anti-mask-league}{%
\subsection{The Anti-Mask League.}\label{the-anti-mask-league}}

As the contagion moved into its second year, so did the skepticism.

On Dec. 17, 1918, the San Francisco Board of Supervisors reinstituted
the mask ordinance after deaths started to climb, a trend that spilled
over into the new year with
\href{https://www.cdc.gov/flu/pandemic-resources/1918-commemoration/pandemic-timeline-1918.htm}{1,800
flu cases and 101 deaths reported there in the first five days of
January.}

That board's decision led to the creation of the Anti-Mask League, a
sign that resistance to masks was resurfacing as cities tried to
reimpose orders to wear them when infections returned.

The league was led by a woman, E.J. Harrington, a lawyer, social
activist and political opponent of the mayor. About a half-dozen other
women filled its top ranks. Eight men also joined, some of them
representing unions, along with two members of the board of supervisors
who had voted against masks.

``The masks turned into a political symbol,'' Dr. Dolan said.

Image

A call to protest by the Anti-Mask League in The San Francisco
Chronicle, on Jan. 25, 1919.Credit...UC Berkley

On Jan. 25, the league held its first organizational meeting, open to
the public at the Dreamland Rink, where they united behind demands for
the repeal of the mask ordinance and for the resignations of the mayor
and health officials.

Their objections included lack of scientific evidence that masks worked
and the idea that forcing people to wear the coverings was
unconstitutional.

On Jan. 27, the league protested at a Board of Supervisors meeting, but
the mayor held his ground. There were hisses and cries of ``freedom and
liberty,'' \href{https://escholarship.org/uc/item/5q91q53r}{Dr. Dolan
wrote in his paper on the epidemic.}

Repeal came a few days later on Feb. 1, when Mayor Rolph cited a
downturn in infections.

But a third wave of flu rolled in late that year. The final death toll
reached an estimated 675,000 nationwide, or 30 for every 1,000 people in
San Francisco, making it one of the worst-hit cities in America.

Dr. Dolan said the story of the Anti-Mask League, which has drawn
renewed interest now in 2020, demonstrates the disconnect between
individual choice and universal compliance.

That sentiment echoes through the century from the voice of a San
Francisco railway worker named Frank Cocciniglia.

Arrested on Kearny Street in January, Mr. Cocciniglia told the judge
that he ``was not disposed to do anything not in harmony with his
feelings,'' according to a Los Angeles Times report.

He was sentenced to five days in jail.

``That suits me,'' Mr. Cocciniglia said as he left the stand. ``I won't
have to wear a mask there.''

Alain Delaqueriere contributed research.

Advertisement

\protect\hyperlink{after-bottom}{Continue reading the main story}

\hypertarget{site-index}{%
\subsection{Site Index}\label{site-index}}

\hypertarget{site-information-navigation}{%
\subsection{Site Information
Navigation}\label{site-information-navigation}}

\begin{itemize}
\tightlist
\item
  \href{https://help.nytimes.com/hc/en-us/articles/115014792127-Copyright-notice}{©~2020~The
  New York Times Company}
\end{itemize}

\begin{itemize}
\tightlist
\item
  \href{https://www.nytco.com/}{NYTCo}
\item
  \href{https://help.nytimes.com/hc/en-us/articles/115015385887-Contact-Us}{Contact
  Us}
\item
  \href{https://www.nytco.com/careers/}{Work with us}
\item
  \href{https://nytmediakit.com/}{Advertise}
\item
  \href{http://www.tbrandstudio.com/}{T Brand Studio}
\item
  \href{https://www.nytimes.com/privacy/cookie-policy\#how-do-i-manage-trackers}{Your
  Ad Choices}
\item
  \href{https://www.nytimes.com/privacy}{Privacy}
\item
  \href{https://help.nytimes.com/hc/en-us/articles/115014893428-Terms-of-service}{Terms
  of Service}
\item
  \href{https://help.nytimes.com/hc/en-us/articles/115014893968-Terms-of-sale}{Terms
  of Sale}
\item
  \href{https://spiderbites.nytimes.com}{Site Map}
\item
  \href{https://help.nytimes.com/hc/en-us}{Help}
\item
  \href{https://www.nytimes.com/subscription?campaignId=37WXW}{Subscriptions}
\end{itemize}
