Sections

SEARCH

\protect\hyperlink{site-content}{Skip to
content}\protect\hyperlink{site-index}{Skip to site index}

\href{https://www.nytimes.com/section/world/europe}{Europe}

\href{https://myaccount.nytimes.com/auth/login?response_type=cookie\&client_id=vi}{}

\href{https://www.nytimes.com/section/todayspaper}{Today's Paper}

\href{/section/world/europe}{Europe}\textbar{}Juan Carlos, Spain's
Former King, Quits Country Amid Multiple Investigations

\url{https://nyti.ms/2XogIxA}

\begin{itemize}
\item
\item
\item
\item
\item
\end{itemize}

Advertisement

\protect\hyperlink{after-top}{Continue reading the main story}

Supported by

\protect\hyperlink{after-sponsor}{Continue reading the main story}

\hypertarget{juan-carlos-spains-former-king-quits-country-amid-multiple-investigations}{%
\section{Juan Carlos, Spain's Former King, Quits Country Amid Multiple
Investigations}\label{juan-carlos-spains-former-king-quits-country-amid-multiple-investigations}}

The former king's departure, which comes as he faces financial
inquiries, may fuel Spain's political and social debate over the future
of the monarchy.

\includegraphics{https://static01.nyt.com/images/2020/08/03/world/03spain-royal/merlin_155832780_07433a85-f437-4a7d-8551-168d51627ea6-articleLarge.jpg?quality=75\&auto=webp\&disable=upscale}

\href{https://www.nytimes.com/by/raphael-minder}{\includegraphics{https://static01.nyt.com/images/2018/10/15/multimedia/author-raphael-minder/author-raphael-minder-thumbLarge.png}}

By \href{https://www.nytimes.com/by/raphael-minder}{Raphael Minder}

\begin{itemize}
\item
  Aug. 3, 2020
\item
  \begin{itemize}
  \item
  \item
  \item
  \item
  \item
  \end{itemize}
\end{itemize}

The former king of Spain, Juan Carlos, announced on Monday he was
abandoning his country amid a Spanish court investigation into a
lucrative business contract and a separate money laundering and tax
evasion probe in Switzerland.

In a
\href{https://www.casareal.es/EN/AreaPrensa/Paginas/area_prensa_comunicados_interior.aspx?data=113}{letter}
released by the royal household and addressed to King Felipe VI, his son
and Spain's current monarch, Juan Carlos said that his decision to leave
amid the multiple inquiries was taken ``with the same eagerness to serve
Spain that inspired my reign.''

In a brief response, King Felipe thanked his father and stressed the
importance of his legacy, while at the same time underlining Spain's
adherence to the rule of law and the values enshrined in its
Constitution.

In 2014, King Juan Carlos
\href{https://www.nytimes.com/2014/06/12/world/europe/spanish-lawmakers-clear-way-for-kings-abdication.html}{abdicated}
in favor of his son, amid health problems and a series of personal
scandals that were already undermining the reputation of Spain's
monarchy.

But six years later, Juan Carlos, 82, has continued to cast a long and
troublesome shadow over the royal household, as his mounting judicial
problems have also been accompanied by a steady flow of embarrassing
media stories about his past lifestyle and personal wealth.

His departure could fuel Spain's political and social debate over the
monarchy, with Spain's left-wing deputy prime minister, Pablo Iglesias,
describing the decision of Juan Carlos to leave as a ``flight abroad,''
which he called ``unworthy of a former head of state.''

Juan Carlos's troubles come as the integrity of other members of
European royal households has been questioned.

Federal prosecutors in New York have accused
\href{https://www.nytimes.com/2020/06/08/nyregion/jeffrey-epstein-prince-andrew.html}{Prince
Andrew} of Britain of refusing to assist in their investigation into
allegations of sex trafficking and other crimes by the financier Jeffrey
Epstein. In January, King Albert II, the former monarch of Belgium, was
\href{https://www.nytimes.com/2020/01/28/world/europe/belgium-king-albert-delphine-boel.html}{forced
to acknowledge fathering a child} after years of lawsuits.

Juan Carlos left his residency in the Zarzuela royal palace, on the
outskirts of Madrid, but he did not say to which foreign country he
would relocate, nor did he discuss what his decision would mean for his
wife, Queen Sofia.

The departure of Juan Carlos follows recent efforts by his son to
distance himself from his father's activities. In March, King Felipe VI
renounced his
\href{https://www.nytimes.com/2020/03/15/world/europe/king-felipe-juan-carlos-spain.html}{personal
inheritance} from his father, and also stripped his father of his
stipend, amid a money laundering investigation launched by a Swiss
prosecutor and focused on two separate offshore foundations.

\includegraphics{https://static01.nyt.com/images/2020/08/03/world/03spain-royal2/merlin_175267443_7160d1d8-20d4-44c8-b896-cdd6ae81ccd6-articleLarge.jpg?quality=75\&auto=webp\&disable=upscale}

One of the foundations being investigated, Zagatka, was registered in
Liechtenstein and set up by Álvaro de Orleans-Borbón, a cousin of Juan
Carlos. Prosecutors are trying to determine how the foundation came to
accumulate its sizable wealth and why it moved money between undeclared
bank accounts.

The other foundation, the Panama-based Lucum, received \$100 million
from Saudi Arabia. Prosecutors are now trying to establish whether this
money was somehow connected to the awarding of a contract that Spanish
companies won to build
\href{https://www.nytimes.com/2019/03/14/reader-center/saudi-arabia-high-speed-train-medina-mecca.html}{a
high-speed rail link} between the Saudi cities of Medina and Mecca.

As part of the probe into a Swiss private bank account held by Lucum, a
Geneva prosecutor, Yves Bertossa, has questioned a lawyer and a fund
manager who had links to Juan Carlos, as well as a former companion,
Corinna zu Sayn-Wittgenstein. King Juan Carlos has not himself been
placed under investigation in Switzerland, but his son, King Felipe, cut
ties with Lucum after finding out that he had been named as a
beneficiary of the fund.

Spain then launched its own investigation, in part based on information
shared by the Swiss prosecutor about the large payment made by Saudi
Arabia to Lucum.

It is unclear what the former monarch's departure will mean for his
legal problems. The chances of Juan Carlos appearing in court in Madrid
were slim, as he continues to benefit from legal immunity.

Spain's government said on Monday that it respected the decision of Juan
Carlos to leave the country. But the monarchy has also been a point of
divergence between the two parties that came into office in January to
form Spain's first coalition government.

The junior coalition partner, Unidas Podemos, wants Spain to become a
republic, a view that is also supported by some smaller left-wing or
regional political parties.

Mr. Iglesias, the deputy prime minister and leader of Unidas Podemos, on
Monday urged Juan Carlos to answer to the Spanish judiciary and the
Spanish people.

But the Socialist Party of Prime Minister Pedro Sánchez recently joined
right-leaning parties to vote down in Parliament a proposal from Unidas
Podemos to investigate Juan Carlos's wealth.

Juan Carlos was praised as a key participant in Spain's return to
democracy, having come to the throne in 1975, two days after the death
of the country's authoritarian leader, Gen. Francisco Franco. In 1981,
the king helped stop a military coup by making a televised broadcast in
which he ordered soldiers to return to their barracks.

``I think we should separate what happened in the first decades of the
reign of Juan Carlos from what has been going on recently, but of course
people tend to remember more the recent events,'' said Carmen Enríquez,
who has written books about the Spanish royal family.

Advertisement

\protect\hyperlink{after-bottom}{Continue reading the main story}

\hypertarget{site-index}{%
\subsection{Site Index}\label{site-index}}

\hypertarget{site-information-navigation}{%
\subsection{Site Information
Navigation}\label{site-information-navigation}}

\begin{itemize}
\tightlist
\item
  \href{https://help.nytimes.com/hc/en-us/articles/115014792127-Copyright-notice}{©~2020~The
  New York Times Company}
\end{itemize}

\begin{itemize}
\tightlist
\item
  \href{https://www.nytco.com/}{NYTCo}
\item
  \href{https://help.nytimes.com/hc/en-us/articles/115015385887-Contact-Us}{Contact
  Us}
\item
  \href{https://www.nytco.com/careers/}{Work with us}
\item
  \href{https://nytmediakit.com/}{Advertise}
\item
  \href{http://www.tbrandstudio.com/}{T Brand Studio}
\item
  \href{https://www.nytimes.com/privacy/cookie-policy\#how-do-i-manage-trackers}{Your
  Ad Choices}
\item
  \href{https://www.nytimes.com/privacy}{Privacy}
\item
  \href{https://help.nytimes.com/hc/en-us/articles/115014893428-Terms-of-service}{Terms
  of Service}
\item
  \href{https://help.nytimes.com/hc/en-us/articles/115014893968-Terms-of-sale}{Terms
  of Sale}
\item
  \href{https://spiderbites.nytimes.com}{Site Map}
\item
  \href{https://help.nytimes.com/hc/en-us}{Help}
\item
  \href{https://www.nytimes.com/subscription?campaignId=37WXW}{Subscriptions}
\end{itemize}
