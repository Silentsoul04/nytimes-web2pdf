Sections

SEARCH

\protect\hyperlink{site-content}{Skip to
content}\protect\hyperlink{site-index}{Skip to site index}

\href{https://www.nytimes.com/section/books/review}{Book Review}

\href{https://myaccount.nytimes.com/auth/login?response_type=cookie\&client_id=vi}{}

\href{https://www.nytimes.com/section/todayspaper}{Today's Paper}

\href{/section/books/review}{Book Review}\textbar{}Edmund White's
High-Octane Saga of Twin Sisters and 1950s Texas

\url{https://nyti.ms/2EOj5Dw}

\begin{itemize}
\item
\item
\item
\item
\item
\end{itemize}

Advertisement

\protect\hyperlink{after-top}{Continue reading the main story}

Supported by

\protect\hyperlink{after-sponsor}{Continue reading the main story}

Fiction

\hypertarget{edmund-whites-high-octane-saga-of-twin-sisters-and-1950s-texas}{%
\section{Edmund White's High-Octane Saga of Twin Sisters and 1950s
Texas}\label{edmund-whites-high-octane-saga-of-twin-sisters-and-1950s-texas}}

\includegraphics{https://static01.nyt.com/images/2020/07/14/books/review/Bird1/Bird1-articleLarge.jpg?quality=75\&auto=webp\&disable=upscale}

Buy Book ▾

\begin{itemize}
\tightlist
\item
  \href{https://www.amazon.com/gp/search?index=books\&tag=NYTBSREV-20\&field-keywords=A+Saint+From+Texas+Edmund+White}{Amazon}
\item
  \href{https://du-gae-books-dot-nyt-du-prd.appspot.com/buy?title=A+Saint+From+Texas\&author=Edmund+White}{Apple
  Books}
\item
  \href{https://www.anrdoezrs.net/click-7990613-11819508?url=https\%3A\%2F\%2Fwww.barnesandnoble.com\%2Fw\%2F\%3Fean\%3D9781635572551}{Barnes
  and Noble}
\item
  \href{https://www.anrdoezrs.net/click-7990613-35140?url=https\%3A\%2F\%2Fwww.booksamillion.com\%2Fp\%2FA\%2BSaint\%2BFrom\%2BTexas\%2FEdmund\%2BWhite\%2F9781635572551}{Books-A-Million}
\item
  \href{https://bookshop.org/a/3546/9781635572551}{Bookshop}
\item
  \href{https://www.indiebound.org/book/9781635572551?aff=NYT}{Indiebound}
\end{itemize}

When you purchase an independently reviewed book through our site, we
earn an affiliate commission.

By Sarah Bird

\begin{itemize}
\item
  Aug. 4, 2020, 5:00 a.m. ET
\item
  \begin{itemize}
  \item
  \item
  \item
  \item
  \item
  \end{itemize}
\end{itemize}

\textbf{A SAINT FROM TEXAS}\\
By Edmund White

Rules are made to be broken by the brilliant. It is Edmund White's
narrative brilliance that allows him to break the inviolable First
Commandment of writing classes, ``Show, don't tell,'' and give us the
divinely well-told tale of identical twins who set out to answer the
question: Can Texas be transcended?

Born in the late 1930s on a desolate scrap of East Texas soon to gush
forth a fortune in oil, Yvonne and Yvette --- ``Why-Von'' and
``Why-Vet'' in the original Texan --- are identical in appearance only.
Temperamentally, our narrator, bubbly Yvonne, and her bookish twin,
Yvette, are polar opposites.

A social-climbing stepmother and a doltish Babbitt of a father use the
newly minted petrodollars to grease their entry into Dallas nouveau
riche society. Determined to escape the gaucheries of the family's
exclusive Turtle Creek enclave, Yvonne ascends from cheerleader to top
majorette baton twirler to debutante to sorority girl. She attains her
ultimate goal, member of the French aristocracy, as the bartered bride
--- ``my title and taste for your fortune'' --- of the odious Baron
Adhéaume de Courcy, ``whose family goes back to the First Crusade.''

Yvette, on the other hand, has her sights set much, \emph{much} higher.
She yearns for nobility of the spirit rather than of the family coat of
arms sort. While Yvonne is preoccupied with Tab Hunter, Peter Pan
collars and how to keep a boy she doesn't like from ``slow-dancing up
her leg,'' Yvette is reading Plato and mortifying her flesh until she's
so thin that her ``monthlies'' stop. When she confides her desire to
lead a life that will land her ``in immortal, loving arms,'' Yvonne
realizes that her twin has ``a crush on God.'' Her cap firmly set on
Jesus, Yvette, pausing only to perform her first miracle, makes her way
to a convent in Jericó, Colombia, where, indeed, she becomes a bride of
Christ.

\includegraphics{https://static01.nyt.com/images/2020/07/14/books/review/Bird2/Bird2-articleLarge.jpg?quality=75\&auto=webp\&disable=upscale}

White's miracle is how he manages to deliver an epic --- told from
Yvonne's perspective and a scattering of letters from Yvette --- that
covers five decades, several precisely observed cultures and a host of
indelible characters in a little under 300 pages. The same story, in
less skilled hands, could have easily lumbered in at twice the length.

White's tale is exactly like a stroll through Le Jardin des Tuileries
--- if the garden had been planted with land mines instead of tulips.
Blackmail, infidelity, incest, sadomasochism, assassination, death and
murder are just a few of White's I.E.D.s. All are nestled innocently in
placid passages detailing, for example, the bric-a-brac that Adhéaume is
squandering Yvonne's money on --- a \$15,000 Louis XV \emph{commode};
silver replicas of the furniture at Versailles; an André Charles Boulle
desk of ``rare woods and gold fittings'' that costs more than two
downtown Dallas blocks. To say nothing of the army of craftsmen the
feckless baron employs to meticulously restore the family's medieval
estate.

The reader, while happily distracted by remodeling minutiae or, say, a
vivid description of a meeting of the Knights of Malta, ``full of
backslapping, heavy teasing and faces dilated with drink-broken
capillaries,'' will blithely stumble upon one of White's booby-traps and
\emph{kablooey:} A shocking new plot point explodes, vaulting the story
forward, soaring over what might typically have been endless pages of
setup, along with the sweaty palms, welling tears and lurching hearts
employed to ``show'' a character's emotional state. Instead, White pays
us the ultimate compliment of assuming we can connect the dots on our
own.

The rocket fuel that propels these abrupt plot twists past the slightest
suspicion of implausibility is the author's trademark narrative
virtuosity and high-octane erudition. It is not surprising that White, a
renowned Francophile and author of biographies
of\href{https://www.nytimes.com/1993/11/07/books/the-high-priest-of-apostasy.html}{Jean
Genet},
\href{https://www.nytimes.com/1999/01/10/books/biography-the-short-form.html}{Marcel
Proust}and
\href{https://www.nytimes.com/2008/10/12/books/review/Hell-t.html}{Arthur
Rimbaud}, can capture French culture with ironclad authority in such
throwaway aperçus as ``the eternal politeness of the French, which often
concealed a barb in its silky tail.'' His characterization of University
of Texas sorority girls, on the other hand --- ``a starter tan,''
``Delta Delta Delta. May I help you help you help you.'' ``Cute shoes!''
``No dark meat in the chicken salad, plenty of mayonnaise, no weird
curry powder'' --- is so unexpectedly convincing that you might find
yourself checking Wikipedia just to make sure that the author never
pledged.

Always an anthropologically acute observer of cultural footprints and
foibles, White reserves his sharpest satirical barbs for the most
deserving targets: the French aristocracy, racists, frat boys, social
climbers, fortune hunters and ``terrible Texas Baptists'' with their
``shallow, bigoted, self-satisfied religion!'' Yvette's world, by
contrast, is portrayed without the slightest prick of irony. In her,
White crafts a pure-hearted, cleareyed seeker who struggles with doubt.
``It occurred to me,'' the nun writes to her sister, ``that the
religious life was all hocus-pocus --- designed to protect the rich,
harbor lazy, gluttonous nuns and monks, supply fresh-faced boys for
priests to groom and sodomize, drug the living and tranquilize the
dying.'' As White chronicles her quest for ``rebirth in God's love'' in
a remote Colombian village, his gaze remains profoundly compassionate.
After stops for involvements, both chaste and otherwise, with the
soon-to-be martyred Bishop Romero and a winsome Filipina nun, Yvette
ultimately finds her path to salvation in the conviction that ``it's my
job as a Christian to find what's salvageable in every person.''

The end of White's sumptuous novel is bittersweet. Yvonne surveys the
emotional wreckage of her life and comes to the doleful realization
that, though she and her beloved twin might have escaped 1950s-era
Texas, the damage inflicted by their father has traveled with them both.
``People,'' she concludes, ``can summon up only the love that was
bequeathed them.''

Advertisement

\protect\hyperlink{after-bottom}{Continue reading the main story}

\hypertarget{site-index}{%
\subsection{Site Index}\label{site-index}}

\hypertarget{site-information-navigation}{%
\subsection{Site Information
Navigation}\label{site-information-navigation}}

\begin{itemize}
\tightlist
\item
  \href{https://help.nytimes.com/hc/en-us/articles/115014792127-Copyright-notice}{©~2020~The
  New York Times Company}
\end{itemize}

\begin{itemize}
\tightlist
\item
  \href{https://www.nytco.com/}{NYTCo}
\item
  \href{https://help.nytimes.com/hc/en-us/articles/115015385887-Contact-Us}{Contact
  Us}
\item
  \href{https://www.nytco.com/careers/}{Work with us}
\item
  \href{https://nytmediakit.com/}{Advertise}
\item
  \href{http://www.tbrandstudio.com/}{T Brand Studio}
\item
  \href{https://www.nytimes.com/privacy/cookie-policy\#how-do-i-manage-trackers}{Your
  Ad Choices}
\item
  \href{https://www.nytimes.com/privacy}{Privacy}
\item
  \href{https://help.nytimes.com/hc/en-us/articles/115014893428-Terms-of-service}{Terms
  of Service}
\item
  \href{https://help.nytimes.com/hc/en-us/articles/115014893968-Terms-of-sale}{Terms
  of Sale}
\item
  \href{https://spiderbites.nytimes.com}{Site Map}
\item
  \href{https://help.nytimes.com/hc/en-us}{Help}
\item
  \href{https://www.nytimes.com/subscription?campaignId=37WXW}{Subscriptions}
\end{itemize}
