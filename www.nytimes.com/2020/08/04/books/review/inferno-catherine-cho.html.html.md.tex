Sections

SEARCH

\protect\hyperlink{site-content}{Skip to
content}\protect\hyperlink{site-index}{Skip to site index}

\href{https://www.nytimes.com/section/books/review}{Book Review}

\href{https://myaccount.nytimes.com/auth/login?response_type=cookie\&client_id=vi}{}

\href{https://www.nytimes.com/section/todayspaper}{Today's Paper}

\href{/section/books/review}{Book Review}\textbar{}A New Mother
Chronicles Her Journey to Hell and Back

\url{https://nyti.ms/3fsRjZU}

\begin{itemize}
\item
\item
\item
\item
\item
\end{itemize}

Advertisement

\protect\hyperlink{after-top}{Continue reading the main story}

Supported by

\protect\hyperlink{after-sponsor}{Continue reading the main story}

Nonfiction

\hypertarget{a-new-mother-chronicles-her-journey-to-hell-and-back}{%
\section{A New Mother Chronicles Her Journey to Hell and
Back}\label{a-new-mother-chronicles-her-journey-to-hell-and-back}}

\includegraphics{https://static01.nyt.com/images/2020/07/13/books/review/Brooks1/Brooks1-popup.jpg?quality=75\&auto=webp\&disable=upscale}

Buy Book ▾

\begin{itemize}
\tightlist
\item
  \href{https://www.amazon.com/gp/search?index=books\&tag=NYTBSREV-20\&field-keywords=Inferno\%3A+A+Memoir+of+Motherhood+and+Madness+Catherine+Cho}{Amazon}
\item
  \href{https://du-gae-books-dot-nyt-du-prd.appspot.com/buy?title=Inferno\%3A+A+Memoir+of+Motherhood+and+Madness\&author=Catherine+Cho}{Apple
  Books}
\item
  \href{https://www.anrdoezrs.net/click-7990613-11819508?url=https\%3A\%2F\%2Fwww.barnesandnoble.com\%2Fw\%2F\%3Fean\%3D9781250623713}{Barnes
  and Noble}
\item
  \href{https://www.anrdoezrs.net/click-7990613-35140?url=https\%3A\%2F\%2Fwww.booksamillion.com\%2Fp\%2FInferno\%253A\%2BA\%2BMemoir\%2Bof\%2BMotherhood\%2Band\%2BMadness\%2FCatherine\%2BCho\%2F9781250623713}{Books-A-Million}
\item
  \href{https://bookshop.org/a/3546/9781250623713}{Bookshop}
\item
  \href{https://www.indiebound.org/book/9781250623713?aff=NYT}{Indiebound}
\end{itemize}

When you purchase an independently reviewed book through our site, we
earn an affiliate commission.

By Kim Brooks

\begin{itemize}
\item
  Aug. 4, 2020
\item
  \begin{itemize}
  \item
  \item
  \item
  \item
  \item
  \end{itemize}
\end{itemize}

By the time
\href{https://www.nytimes.com/2001/09/08/us/despair-plagued-mother-held-in-children-s-deaths.html}{Andrea
Yates} drowned her children, she believed that Satan was inside her, and
that the only way to protect her daughter and four sons from a similar
fate was to kill them and send them to paradise. In the wake of Yates's
trial --- and in the trials of other women who hurt or neglect their
children during bouts of postpartum psychosis --- coverage has tended to
dwell on the least useful question: How could any sane woman kill her
kids? A better question, and the one explored in Catherine Cho's
captivating first book, ``Inferno,'' would inquire about the factors
(biological, cultural and environmental) that make some women vulnerable
to episodes of acute, severe mental illness in the period after they
become mothers.

Cho's title refers to the perceived hell in which the author finds
herself a couple of months after her son is born, a hell that the reader
quickly learns is the inpatient unit of a mental hospital. The book
begins just as Cho is starting to recover from psychosis, struggling to
remember who she is: ``I write the words I can call myself. I am a
daughter. A sister. A wife. Those words come easily. I can remember
them. I stare at the page. And then I write MOTHER. The word looks
strange. Next to the others, it stands separate.''

The narrative toggles back and forth between Cho's recovery in the
hospital and the months preceding her breakdown. Before her pregnancy,
she strove to be an obedient daughter, a protective sister, a desirable
girlfriend (even to a man who abused her), and finally a loving and
devoted wife to her kind and doting husband. Moving in and out of these
relationships, Cho nonetheless maintains a strong sense of self and a
curiosity about the world. Something, however, changes after she gives
birth to her son, Cato: ``I'd thought I would reclaim my body after
birth, but instead, it was now a tool, something to sustain life.
\ldots{} In the blur of those hours, I stopped thinking of myself as
having a name; I was a body. I had no identity, I was just a number on
the marker board and a set of vitals.''

Cho pushes past this disorientation after childbirth and feels well
enough to suggest she and her husband take their newborn from London
(where they live) to the United States, where they will introduce him to
friends and family on a cross-country tour. As they travel, Cho sleeps
less and less. Their plan is to conclude the trip at her in-laws' home
in New Jersey, where they will celebrate her son's 100th day, according
to Korean tradition.

\includegraphics{https://static01.nyt.com/images/2020/07/13/books/review/Brooks2/Brooks2-articleLarge.jpg?quality=75\&auto=webp\&disable=upscale}

In her in-laws' home, under their loving but anxious gaze, Cho begins to
sense something is not right. Her insomnia worsens. She is unable to set
limits with her husband's parents, who lob a barrage of questions about
their grandson's well-being. ``Why did you let so many people hold the
baby in California? \ldots{} Why was Cato so big? \ldots{} Why wasn't he
rolling yet? \ldots{} Why did we hold him in a wrap? \ldots{} Each
comment and criticism, although kindly meant, struck at me like
pinpricks of a needle. Was I such a terrible mother? Was I doing
everything wrong?''

The intensity of the first-person perspective here gives this section
the claustrophobic dread of a psychological thriller. Cho conveys how an
atmosphere of constant anxiety and judgment slowly loosens her grip on
what is real and what is imagined. The cameras installed in her in-laws'
house contribute to her panic and fear. Are they watching her? Is she
being watched? Her husband tries to help, but by the time he intervenes
it is too late. We see Cho slipping, losing touch. Eventually, she looks
down at her son and sees him staring back with ``devils' eyes.'' It's
not hard to understand how Cho became a prisoner in her own mind; the
only question is if and how she'll find her way out.

``Inferno'' is a disturbing and masterfully told memoir, but it's also
an important one that pushes back against powerful taboos. We still
don't like to talk about postpartum mental illness, or the fact that,
when a mother becomes ill and doesn't have a support system or access to
mental health care, the emotional damage to both her and her children
can reverberate across generations.

This culture of silence is the only topic I wish Cho had expounded on at
greater length. She recounts how, during her recovery, she read
obsessively about postpartum psychosis and joined a forum of other women
who had experienced it. I would have loved to hear more about these
women, about how shame perpetuated their trauma. Discussions of severe
mental illness in mothers continue to induce discomfort and judgment in
those who have never experienced it, and embarrassment and shame in
those who have. The persistence of such stigmas makes memoirs like Cho's
all the more courageous.

Advertisement

\protect\hyperlink{after-bottom}{Continue reading the main story}

\hypertarget{site-index}{%
\subsection{Site Index}\label{site-index}}

\hypertarget{site-information-navigation}{%
\subsection{Site Information
Navigation}\label{site-information-navigation}}

\begin{itemize}
\tightlist
\item
  \href{https://help.nytimes.com/hc/en-us/articles/115014792127-Copyright-notice}{©~2020~The
  New York Times Company}
\end{itemize}

\begin{itemize}
\tightlist
\item
  \href{https://www.nytco.com/}{NYTCo}
\item
  \href{https://help.nytimes.com/hc/en-us/articles/115015385887-Contact-Us}{Contact
  Us}
\item
  \href{https://www.nytco.com/careers/}{Work with us}
\item
  \href{https://nytmediakit.com/}{Advertise}
\item
  \href{http://www.tbrandstudio.com/}{T Brand Studio}
\item
  \href{https://www.nytimes.com/privacy/cookie-policy\#how-do-i-manage-trackers}{Your
  Ad Choices}
\item
  \href{https://www.nytimes.com/privacy}{Privacy}
\item
  \href{https://help.nytimes.com/hc/en-us/articles/115014893428-Terms-of-service}{Terms
  of Service}
\item
  \href{https://help.nytimes.com/hc/en-us/articles/115014893968-Terms-of-sale}{Terms
  of Sale}
\item
  \href{https://spiderbites.nytimes.com}{Site Map}
\item
  \href{https://help.nytimes.com/hc/en-us}{Help}
\item
  \href{https://www.nytimes.com/subscription?campaignId=37WXW}{Subscriptions}
\end{itemize}
