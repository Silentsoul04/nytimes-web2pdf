Sections

SEARCH

\protect\hyperlink{site-content}{Skip to
content}\protect\hyperlink{site-index}{Skip to site index}

\href{https://www.nytimes.com/section/books/review}{Book Review}

\href{https://myaccount.nytimes.com/auth/login?response_type=cookie\&client_id=vi}{}

\href{https://www.nytimes.com/section/todayspaper}{Today's Paper}

\href{/section/books/review}{Book Review}\textbar{}`Wandering in Strange
Lands,' by Morgan Jerkins: An Excerpt

\url{https://nyti.ms/30tmv77}

\begin{itemize}
\item
\item
\item
\item
\item
\end{itemize}

Advertisement

\protect\hyperlink{after-top}{Continue reading the main story}

Supported by

\protect\hyperlink{after-sponsor}{Continue reading the main story}

\hypertarget{wandering-in-strange-lands-by-morgan-jerkins-an-excerpt}{%
\section{`Wandering in Strange Lands,' by Morgan Jerkins: An
Excerpt}\label{wandering-in-strange-lands-by-morgan-jerkins-an-excerpt}}

Buy Book ▾

\begin{itemize}
\tightlist
\item
  \href{https://www.amazon.com/gp/search?index=books\&tag=NYTBSREV-20\&field-keywords=Wandering+in+Strange+Lands\%3A+A+Daughter+of+the+Great+Migration+Reclaims+Her+Roots+Morgan+Jerkins}{Amazon}
\item
  \href{https://du-gae-books-dot-nyt-du-prd.appspot.com/buy?title=Wandering+in+Strange+Lands\%3A+A+Daughter+of+the+Great+Migration+Reclaims+Her+Roots\&author=Morgan+Jerkins}{Apple
  Books}
\item
  \href{https://www.anrdoezrs.net/click-7990613-11819508?url=https\%3A\%2F\%2Fwww.barnesandnoble.com\%2Fw\%2F\%3Fean\%3D9780062873040}{Barnes
  and Noble}
\item
  \href{https://www.anrdoezrs.net/click-7990613-35140?url=https\%3A\%2F\%2Fwww.booksamillion.com\%2Fp\%2FWandering\%2Bin\%2BStrange\%2BLands\%253A\%2BA\%2BDaughter\%2Bof\%2Bthe\%2BGreat\%2BMigration\%2BReclaims\%2BHer\%2BRoots\%2FMorgan\%2BJerkins\%2F9780062873040}{Books-A-Million}
\item
  \href{https://bookshop.org/a/3546/9780062873040}{Bookshop}
\item
  \href{https://www.indiebound.org/book/9780062873040?aff=NYT}{Indiebound}
\end{itemize}

When you purchase an independently reviewed book through our site, we
earn an affiliate commission.

Aug. 4, 2020

\begin{itemize}
\item
\item
\item
\item
\item
\end{itemize}

In 1869 the Gullah Geechee people owned half of Beaufort County of South
Carolina. Since then, they've lost fourteen million acres. The best
example of the rapid marginalization of Gullah Geechee people is on
Hilton Head Island, one of the most lucrative places in the South.

In 2018, the \emph{Conde Nast Traveler} ranked Hilton Head Island as the
best island resort area in the United States for the second year in a
row, and for good reason. The island is full of world-class resorts,
delectable cuisine, and sandy beaches, and it generally has idyllic
weather. But beyond the optics of a wonderful vacation spot, there's a
grim side. Although Gullah Geechee people are spread out all across the
island, their biggest concentration is in the north end. Hilton Head
Island was once one large plantation. After the Civil War, freed black
people bought land from the United States government and settled into
what was the first self-governed town of formerly enslaved black people
in the country: Mitchelville. Other pieces of land were sold to (white)
speculators or previous plantation owners, who settled on the south end.
In 1956, Charles Fraser, the son of a well-off family who made their
fortune in timber, started the ``modern'' era in Hilton Head by creating
the first resort, Sea Pines Plantation, on the south end, constructing a
bridge to connect to the mainland, and bringing air-conditioning to the
island. But over time, developers wanted to expand to the north, as
well, and were ultimately successful. The lack of legal protection for
those who had heirs property, the rising land taxes, and the exodus of
African Americans leaving the South in general as a part of the Great
Migration, all allowed Fraser to succeed in converting most of Hilton
Head into a prime resort community.

\emph{{[} Return to the review of}
\href{https://www.nytimes.com/2020/08/03/books/review/wandering-in-strange-lands-morgan-jerkins.html}{\emph{``Wandering
in Strange Lands.''}} \emph{{]}}

Over 80 percent of ``early black landowners'' from the post-- Civil War
period and later did not have wills or clear titles. These landowners
simply passed down their acres to their descendants or relatives as what
is called heirs property. But families who have heirs property cannot
get mortgages, do home repairs, apply for state or federal aid, get
conservation funding, or take out loans available through the US
Department of Agriculture. Heirs property is one of the biggest issues
when it comes to black land preservation and cultural heritage on Hilton
Head Island.

I perused the internet to find Hilton Head locals who were outspoken
about the effects of business expansion into predominantly Gullah
Geechee communities, and found Taiwan Scott. Coincidentally, Taiwan---or
Tai, for short---was born in New Jersey like me. But unlike me, Tai has
traveled regularly to his grandmother's birthplace in Hilton Head and
decided as an adult to move there to help preserve Gullah Geechee
heritage. Unlike any other person I'd met from the Lowcountry, Scott was
a real estate agent and therefore was willing to divulge his personal
and professional stakes in Hilton Head. At the time that I was scheduled
to meet him, Scott was in the midst of an ongoing battle with the town
to run a business on his own property, and he believed the local media
was not adequately covering the story.

I drove away from Savannah via I-95 North and then 278 through Jasper
County and Bluffton to Hilton Head, and to this day, it is one of the
most beautiful drives I've ever been on. But it was in Hilton Head where
I soon learned how beautiful landscapes masked black carnage that was
simplified and mocked at every turn. I saw the word plantation so much
that I was starting to get a headache: Plantation Cafe \& Grill,
Plantation Cafe \& Deli, Plantation Shopping Center, Paper \& Party
Plantation, Plantation Drive, Plantation Road, Plantation Club,
Plantation Animal Hospital, Plantation Interiors, Plantation Cabinetry,
Plantation Station Inc. . . . With every road I passed, there was
another indication of a perverse symmetry between leisure and slavery.

On the island's north end, I met Taiwan Scott on his property at 15
Marshland Road. Tai has invested time poring over archaeological studies
on the island to match the locations of praise houses and cemeteries to
new buildings and developments. He was stunned that anyone got clearance
to construct, because these places hold such cultural and spiritual
significance.

Tai's plan was to have a food-truck-style restaurant and a shop where
fruit, vegetables, seafood, jewelry, and sweetgrass baskets would be
sold. The food truck and kitchen were already DHEC (South Carolina
Department of Health and Environmental Control) approved. Initially, he
was told by the local government that they did not want food trucks on
the island, but they wanted to see his development concept. His initial
development concept would have cost around \$50,000. After the town
government's less-than-enthusiastic response to his plan, the concept
escalated to nearly twice that amount because he was told that he was in
a flood area and everything had to be elevated. He would also need to
include a wrap-around deck. With the comments from the town government
in hand, Tai began to read their rules and regulations book, and found a
section that would allow his food truck without any of the
flood-elevation requirements.

After sitting down with local officials, he was given the OK to proceed.
During the design-review process, he was told that his building was too
orange, despite the color being a cedar natural tone with a transparent
stain. On a trip around town, he spotted a building within Shelter Cove
Harbor, a newly developed upscale waterfront community, with bright
orange awnings over the door and windows. Afterward, Tai consulted his
white next-door neighbor, who owns a successful honey business. This
neighbor showed up to a town meeting and went on record supporting Tai
and his business plan.

\emph{{[} Return to the review of}
\href{https://www.nytimes.com/2020/08/03/books/review/wandering-in-strange-lands-morgan-jerkins.html}{\emph{``Wandering
in Strange Lands.''}} \emph{{]}}

I wondered why Scott didn't know what happened in the town meeting;
these kinds of meetings usually have someone recording what happens.
``They didn't have minutes?''

``No minutes. I did a freedom-of-information request. They said they
weren't obligated to take any minutes to this meeting . . . because it
wasn't an official public meeting.''

``Oh, OK. So once it was discussed, it wasn't open to the public.''
``They had a closed door meeting . . .''

``. . . that they didn't invite you to.'' ``. . . about my
establishment.''

What has happened to Tai and his business has caused a shock wave among
the native islanders, since Tai spread the word about what's going on.
Some elders have encouraged him to stop pressing the local government to
make things right. Others' hopes are dashed for any economic mobility.

His persistence has come with a cost. He fears for his wife and
children. In 2017, a white pickup truck often circled around his home.
Banana trees that grew on his property were dug up, and the security
camera on his Marshland lot was smashed.

Ideally, Scott would like Gullah people to keep their heirs property but
if they cannot and are forced to sell, he wants them to get the best
offer available. Hilton Head is continuing to develop, and therefore the
property taxes are going to steadily rise. He said to me, ``I had one
client---seventeen thousand dollars a year in taxes. I mean how can they
afford to keep that? So they're forced to sell it.''

To give me a more well-rounded sense of this divided place, Tai took me
on a tour around the island. Within a half hour or so, I was able to
distinguish between Gullah land, with its mobile park homes and weeds
growing wildly on the lawns, and the plantations---or gated
communities---where a pass is required for entry and we could only see
an entrance sign with a long trail to the security booth behind it. In
2016, 77 percent of the Hilton Head population was white and under 7
percent was black. Over 10 percent of the population was in poverty, but
of those in poverty, 30.6 percent were black, whereas only 5.4 percent
were white.

In the 1860s, there were twenty-four plantations on Hilton Head Island.
The rebranding of plantations as gated communities appeals to the white
imagination. An adjunct professor of anthropology at the University of
Tennessee--Knoxville, Melissa Hargrove wrote in her PhD dissertation on
this spatial segregation. ``For the Gullah, this practice has translated
into a reinvention of history that denies the collective memories
intimately linking them to these recently appropriated spaces.''

One of these collective memories is of the way the Gullah people honor
the dead. Alex Brown, chairman of the Hilton Head Town Planning
Commission, whose family has been in Hilton Head for eight generations,
knows of Gullah burial grounds within three plantations: Hilton Head,
Sea Pines, and Indigo Run. About five years ago, one of Brown's closest
friends passed away and was set to be buried in Indigo Run. Because that
friend was a motorcyclist, Brown and his social circle decided to ride
motorcycles to the funeral in his honor. However, the bikes weren't
allowed. Brown isn't sure whether this restriction was discriminatory,
but demonstrates that the gated communities make rules independent of
the town.

Taiwan Scott's ancestors and countless others are buried in Sea Pines
Plantation where South Carolina's annual RBC Heritage PGA tournament is
held, a renowned event that Taiwan aches to see. He's even been a part
of protest during this event to call attention to Gullah displacement.
With regard to preservation, the state of South Carolina, SC Code of
Laws 27--43--210 states:

This law grants family members and descendants limited access to graves
on private property. It requires owners of cemeteries on private
property to provide reasonable access to family members and descendants
of those buried in the cemetery. The law requires the person wanting
access to the cemetery to submit a written request to the property
owner.

Descendants of someone who's buried in a cemetery on private property
must rely on the goodwill of a property owner and petition said person,
whether they're natives or visitors, to pay respects to their own
people. Furthermore, one may have to pay a fee, as is the case with Sea
Pines Plantation. There is a three-dollar fee, like the entrance fee to
a state park, for locals to visit their deceased relatives. That
alongside having to explain themselves to security guards makes it hard
for the Gullah people to maintain their connection to the land.

Tish Lynn is director of communications and outreach for the Center for
Heirs Property Preservation, a Charleston-based non-profit organization
working to help heirs retain their land. In a phone interview, she said,
``All of these developments that have become gated communities on both
Hilton Head and along the coast of South Carolina have cut them {[}the
Gullah Geechee{]} off from their traditions, their culture, and their
way of life. It's more than losing land. It's heritage and culture as we
know it. It's the loss of access to water as a means of transportation,
to fish and oyster, and make a sustainable living.''

I thought of my family and how suburbs were supposed to be the dream. If
we could have been living in a gated community, that would have been
even better. But now I was seeing that the planning of gated
communities, especially in the South, was at the expense of black
people, their ancestors' bodies, and their customs. I, the Northerner,
was face-to-face with a man who lived in the North but came back to the
South to fight for his people. Both of us were effects of migration.
Arguably, if more Gullah Geechee people had stayed in Hilton Head, Tai
would not have faced so many challenges with his business venture,
because there would have been more of them to claim their stakes in the
land. Arguably, if my grandfather and his father had kept returning to
Georgia with their children after moving up north, then maybe I would've
never felt that my connection to the South had been severed. But this
day felt like a meeting in the middle. No matter where we are along the
coast, we are a vulnerable people, prone to cultural erasure and
amnesia.

\emph{{[} Return to the review of}
\href{https://www.nytimes.com/2020/08/03/books/review/wandering-in-strange-lands-morgan-jerkins.html}{\emph{``Wandering
in Strange Lands.''}} \emph{{]}}

Advertisement

\protect\hyperlink{after-bottom}{Continue reading the main story}

\hypertarget{site-index}{%
\subsection{Site Index}\label{site-index}}

\hypertarget{site-information-navigation}{%
\subsection{Site Information
Navigation}\label{site-information-navigation}}

\begin{itemize}
\tightlist
\item
  \href{https://help.nytimes.com/hc/en-us/articles/115014792127-Copyright-notice}{©~2020~The
  New York Times Company}
\end{itemize}

\begin{itemize}
\tightlist
\item
  \href{https://www.nytco.com/}{NYTCo}
\item
  \href{https://help.nytimes.com/hc/en-us/articles/115015385887-Contact-Us}{Contact
  Us}
\item
  \href{https://www.nytco.com/careers/}{Work with us}
\item
  \href{https://nytmediakit.com/}{Advertise}
\item
  \href{http://www.tbrandstudio.com/}{T Brand Studio}
\item
  \href{https://www.nytimes.com/privacy/cookie-policy\#how-do-i-manage-trackers}{Your
  Ad Choices}
\item
  \href{https://www.nytimes.com/privacy}{Privacy}
\item
  \href{https://help.nytimes.com/hc/en-us/articles/115014893428-Terms-of-service}{Terms
  of Service}
\item
  \href{https://help.nytimes.com/hc/en-us/articles/115014893968-Terms-of-sale}{Terms
  of Sale}
\item
  \href{https://spiderbites.nytimes.com}{Site Map}
\item
  \href{https://help.nytimes.com/hc/en-us}{Help}
\item
  \href{https://www.nytimes.com/subscription?campaignId=37WXW}{Subscriptions}
\end{itemize}
