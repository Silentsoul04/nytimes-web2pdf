Sections

SEARCH

\protect\hyperlink{site-content}{Skip to
content}\protect\hyperlink{site-index}{Skip to site index}

\href{https://www.nytimes.com/section/business}{Business}

\href{https://myaccount.nytimes.com/auth/login?response_type=cookie\&client_id=vi}{}

\href{https://www.nytimes.com/section/todayspaper}{Today's Paper}

\href{/section/business}{Business}\textbar{}Argentina secures a deal to
restructure \$65 billion in debt.

\url{https://nyti.ms/2DwfTMf}

\begin{itemize}
\item
\item
\item
\item
\item
\end{itemize}

Advertisement

\protect\hyperlink{after-top}{Continue reading the main story}

Supported by

\protect\hyperlink{after-sponsor}{Continue reading the main story}

\hypertarget{argentina-secures-a-deal-to-restructure-65-billion-in-debt}{%
\section{Argentina secures a deal to restructure \$65 billion in
debt.}\label{argentina-secures-a-deal-to-restructure-65-billion-in-debt}}

\includegraphics{https://static01.nyt.com/images/2020/08/04/business/04markets-brf-argentina/04markets-brf-argentina-articleLarge.jpg?quality=75\&auto=webp\&disable=upscale}

By \href{https://www.nytimes.com/by/eshe-nelson}{Eshe Nelson} and Daniel
Politi

\begin{itemize}
\item
  Aug. 4, 2020
\item
  \begin{itemize}
  \item
  \item
  \item
  \item
  \item
  \end{itemize}
\end{itemize}

After months of negotiations, Argentina has reached a deal with its
creditors, including the asset managers \textbf{BlackRock} and
\textbf{Greylock Capital Management}, to restructure about \$65 billion
in foreign debt,
\href{https://www.economia.gob.ar/en/argentina-and-three-creditor-groups-reach-a-deal-on-debt-restructuring/}{the
economy ministry said} on Tuesday.

The deal between Argentina and its largest creditors would grant
``significant'' debt relief to the country, the ministry said. In May,
Argentina missed a bond payment and entered into its ninth default since
the country's independence and third in the last 20 years.

Under the agreement, creditors will be paid 54.8 cents on the dollar,
according to an Argentine government official who has taken part in the
negotiations who spoke on condition of anonymity to discuss private
negotiations.

Argentina would offer creditors new bonds in exchange for defaulted debt
and unpaid interest. Investors who hold euro-denominated and Swiss
franc-denominated bonds would also be able to swap these for new U.S.
dollar-denominated bonds. The agreement also includes changes to payment
dates.

A range of prominent people, including Pope Francis, Senator Elizabeth
Warren and the Nobel-winning economist Joseph E. Stiglitz,
\href{https://www.nytimes.com/2020/05/22/world/americas/argentina-default.html}{called
on Argentina's bondholders} to come to a favorable agreement quickly
with the cash-strapped nation.

The International Monetary Fund has forecast that Argentina's economy
will fall nearly 10 percent this year, deepening a yearslong economic
crisis which has been compounded by the coronavirus pandemic and
extremely high levels of inflation.

BlackRock, the world's largest asset manger, had
\href{https://www.nytimes.com/2020/07/31/business/argentina-debt.html}{rejected
an earlier proposal} from Argentina and pushed other creditors to also
hold out for a better deal. The government and its creditors --- the Ad
Hoc Group of Argentine Bondholders, the Argentina Creditor Committee and
the Exchange Bondholder Group and other investors --- were only three
pennies on the dollar apart on their proposed terms.

The country has also been in talks with the I.M.F.,
after\href{https://qz.com/1274875/how-argentina-went-from-selling-100-year-bonds-to-an-imf-rescue-in-a-matter-of-months/}{restoring
ties to the institution with a 2018 bailout}. The fund said Argentina's
debt pile was unsustainable as its currency plummeted and economic
output kept falling. Martín Guzmán, Argentina's economy minister, had
told local media that he
would\href{https://www.pagina12.com.ar/282488-deuda-la-ultima-oferta-y-el-mensaje-de-martin-guzman}{look
to the I.M.F}. for an alternative way out of its debt crisis if the
talks with private creditors failed.

Advertisement

\protect\hyperlink{after-bottom}{Continue reading the main story}

\hypertarget{site-index}{%
\subsection{Site Index}\label{site-index}}

\hypertarget{site-information-navigation}{%
\subsection{Site Information
Navigation}\label{site-information-navigation}}

\begin{itemize}
\tightlist
\item
  \href{https://help.nytimes.com/hc/en-us/articles/115014792127-Copyright-notice}{©~2020~The
  New York Times Company}
\end{itemize}

\begin{itemize}
\tightlist
\item
  \href{https://www.nytco.com/}{NYTCo}
\item
  \href{https://help.nytimes.com/hc/en-us/articles/115015385887-Contact-Us}{Contact
  Us}
\item
  \href{https://www.nytco.com/careers/}{Work with us}
\item
  \href{https://nytmediakit.com/}{Advertise}
\item
  \href{http://www.tbrandstudio.com/}{T Brand Studio}
\item
  \href{https://www.nytimes.com/privacy/cookie-policy\#how-do-i-manage-trackers}{Your
  Ad Choices}
\item
  \href{https://www.nytimes.com/privacy}{Privacy}
\item
  \href{https://help.nytimes.com/hc/en-us/articles/115014893428-Terms-of-service}{Terms
  of Service}
\item
  \href{https://help.nytimes.com/hc/en-us/articles/115014893968-Terms-of-sale}{Terms
  of Sale}
\item
  \href{https://spiderbites.nytimes.com}{Site Map}
\item
  \href{https://help.nytimes.com/hc/en-us}{Help}
\item
  \href{https://www.nytimes.com/subscription?campaignId=37WXW}{Subscriptions}
\end{itemize}
