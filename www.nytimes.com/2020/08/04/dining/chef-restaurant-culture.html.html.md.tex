\href{/section/food}{Food}\textbar{}Twilight of the Imperial Chef

\url{https://nyti.ms/3k6RPQV}

\begin{itemize}
\item
\item
\item
\item
\item
\item
\end{itemize}

\includegraphics{https://static01.nyt.com/images/2020/08/05/dining/05Chefs-Centric-Cover-Illo/05Chefs-Centric-Cover-Illo-articleLarge.jpg?quality=75\&auto=webp\&disable=upscale}

Sections

\protect\hyperlink{site-content}{Skip to
content}\protect\hyperlink{site-index}{Skip to site index}

The Great Readcritic's Notebook

\hypertarget{twilight-of-the-imperial-chef}{%
\section{Twilight of the Imperial
Chef}\label{twilight-of-the-imperial-chef}}

For decades, the notion of the lone genius in the kitchen has fostered
culinary creativity --- and restaurants marred by abuse and unfairness.
This may be the time for change.

Credit...Ryan Garcia

Supported by

\protect\hyperlink{after-sponsor}{Continue reading the main story}

By \href{https://www.nytimes.com/by/tejal-rao}{Tejal Rao}

\begin{itemize}
\item
  Aug. 4, 2020
\item
  \begin{itemize}
  \item
  \item
  \item
  \item
  \item
  \item
  \end{itemize}
\end{itemize}

Picture a great restaurant, the chef up at dawn, dusting hand-milled
flour on a butcher's block. The chef under a spotlight, tweezing chive
blossoms in the chaos of the pass, or fanning the wood fire under a row
of shimmering, trussed birds.

The chef is in sharp focus, but everything else --- everyone else --- is
an inconsequential blur.

I don't need to describe the chef to you. He is a man, probably. A
genius, definitely. Let's say this genius is volatile, meticulous,
impenetrable, charming, camera-ready. He doesn't just manage the staff
behind a great restaurant. He \emph{is} the great restaurant.

For decades, the chef has been cast as the star at the center of the
kitchen. In the same way the auteur theory in film frames the director
as the author of a movie's creative vision, the chef has been considered
entirely responsible for the restaurant's success. Everyone else ---
line cooks, servers, dishwashers, even diners --- is background, there
to support that vision.

This way of thinking has informed the industry's culture at every level.
But the power of the chef-auteur as an idea is fading, and as restaurant
workers organize and speak up about abusive workplaces, toxic bosses and
inequities in pay and benefits, it's clear that the restaurant industry
has to change.

The elevation of the chef to front and center is relatively new. Until
about 40 years ago, chefs were considered unglamorous, trolls of the
stove, hidden behind the kitchen's swinging doors.

With a few exceptions, they weren't thought of as artists, or
visionaries. They couldn't generally aspire to magazine covers, or amass
devoted, cultlike, international followings. They did not get book
deals, or discuss their inspirations in interviews, or star in
documentaries, or hire publicists to make horrific scandals disappear.

In his 2018 book,
``\href{https://www.barnesandnoble.com/w/chefs-drugs-and-rock-roll-andrew-friedman/1126512983}{Chefs,
Drugs and Rock \& Roll},'' Andrew Friedman documents the mythologizing
of chefs, and their rise from obscurity. He writes that before the 1970s
and '80s, chefs were ``anonymous workhorses,'' in many cases not only
unknown, but thought of as interchangeable.

Image

Wolfgang Puck's Spago, in Los Angeles, was an early example of the
chef-driven restaurant in the United States.Credit...Larry Davis/Los
Angeles Times via Getty Images

The 1970s kicked off a shift, changing the way chefs were perceived in
the United States. As
\href{https://www.nytimes.com/2012/10/31/dining/wolfgang-puck-the-original-celebrity-chef-is-still-keeping-busy.html}{Wolfgang
Puck} built a reputation for innovation in the kitchen at Ma Maison, and
went on to open Spago, he helped usher in an era of American dining when
chefs became names --- big names --- known to the public outside the
restaurant business.

As chefs inched toward auteurship, they were finally recognized for
grueling, previously undervalued labor. They were also given more room
to reimagine dishes and menus, to tinker with how restaurants worked,
and who they were for. They made restaurants infinitely more exciting
place to dine, and to work.

By the time I started cooking in restaurant kitchens, in the mid-2000s,
willingly vanishing into the militaristic brigade system, the chef's
status as an auteur was beyond question, and the deeply embarrassing
phrase ``food is the new rock'' was tossed around with almost no sense
of irony.

One chef I worked for shared photocopied pages of Ferran and Albert
Adrià's cookbooks, in Spanish, so the staff could study the ratios and
techniques used in the famous kitchen of
\href{https://www.nytimes.com/2010/09/22/dining/reviews/22pour.html}{El
Bulli}. It was thrilling, and many of us experimented with blowing
isomalt sugar sculptures or setting hot jellies.

\includegraphics{https://static01.nyt.com/images/2020/08/05/dining/03Chefs5/03Chefs5-articleLarge.jpg?quality=75\&auto=webp\&disable=upscale}

That iconic photo of Marco Pierre White looking young and angry and
sleepless and beautiful in his chef whites was a talisman for several
cooks I knew.

It appeared in his influential 1990 book,
``\href{https://www.nytimes.com/2015/04/08/dining/marco-pierre-white-white-heat-a-game-changer-revisited.html}{White
Heat},'' which showed what was possible when an ambitious, brilliant
young chef achieved total power: Mr. White wrote about his habit of
putting cooks inside trash cans to punish them, among other forms of
intimidation.

``\href{https://www.harpercollins.com/products/kitchen-confidential-updated-ed-anthony-bourdain}{Kitchen
Confidential},'' by Anthony Bourdain, was also canon. Throughout his
career, Mr. Bourdain called for attention and respect for immigrants,
undocumented workers and the many underpaid, overlooked roles essential
to a restaurant.

But he was also a celebrity, and he upheld a romantic ideal of cheffing
as the kind of brutal, impossibly demanding, but ultimately meaningful
work that exalted misfits, drawing them together with a sense of purpose
--- at least, for the duration of dinner service.

This complicated, shared understanding of restaurant kitchens was often
used to justify the work and the hours, and the unreasonable
expectations in service of excellence and glory. It also explained away
the gross, systemic deficiencies of the business, and normalized abusive
work cultures.

Image

Jean-Georges Vongerichten (without toque) preparing dishes at Restaurant
Lafayette in 1988, with, as the original caption put it, ``the
restaurant's staff.''Credit...Ruby Washington/The New York Times

In his 2019 memoir,
``\href{https://wwnorton.com/books/9780393608489}{JGV: My Life in 12
Recipes},'' the chef
\href{https://www.nytimes.com/2020/01/14/dining/jean-georges-vongerichten.html}{Jean
Georges Vongerichten} writes about the culture he fostered in the late
1980s at Restaurant Lafayette, which received
\href{https://www.nytimes.com/1988/04/22/arts/restaurants-067888.html}{a
three-star review} from Bryan Miller in The New York Times.

The restaurant's longtime dishwasher, referred to as ``Sam'' in the
book, had been working at the hotel for 20 years, and took a 45-minute
break while a critic was in the house. Mr. Vongerichten, who took the
dishwasher's place at the sink during that time, was furious. As his
sous-chef held the walk-in door shut, trapping Sam inside, Mr.
Vongerichten pummeled him.

``I'm not proud of it,'' Mr. Vongerichten writes. After the dishwasher
went to security to report the abuse, the kitchen closed ranks.
``Everyone in the kitchen knew what happened,'' he adds. ``But nobody
said a word.''

Mr. Vongerichten went on to find
\href{https://www.nytimes.com/2019/10/17/magazine/jean-georges-restaurants.html}{international
renown} and open 38 restaurants all over the world. As of last fall, The
\href{https://www.jean-georges.com/restaurants/united-states}{Jean-Georges
restaurant group} managed 5,000 employees; its 2018 sales totaled \$350
million.

As chefs built big restaurant businesses, often referred to as empires,
they became powerful brands, capable of obscuring abuse, assault and
discrimination. And if they continued to make money for their investors,
they often maintained their power --- as in the case of Mario Batali.

Image

After years of acclaim, presiding over a culinary empire, Mario Batali
left his restaurants because of a series of sexual assault
accusations.Credit...Fred R. Conrad/The New York Times

Mr. Batali became one of the country's most high-profile chefs and
restaurateurs, opening popular restaurants, hosting shows on ABC and the
Food Network, publishing a series of popular cookbooks, and playing a
central role in
\href{https://www.nytimes.com/2020/07/21/dining/bill-buford-dirt-book-chicken-recipe.html}{Bill
Buford}'s vivid book
``\href{https://www.penguinrandomhouse.com/books/20949/heat-by-bill-buford/}{Heat},''
published in 2007.

But in 2017, several women spoke up about Mr. Batali's pattern of
\href{https://www.nytimes.com/2017/12/11/dining/mario-batali-sexual-misconduct.html}{sexual
harassment and assault}. It wasn't until 2019 that he
\href{https://www.nytimes.com/2019/03/06/dining/mario-batali-bastianich-restaurants.html}{divested}
from the Bastianich \& Batali Hospitality Group, and stopped profiting
from the restaurants he'd established. In the same way, the chef April
Bloomfield severed her partnership with the restaurateur
\href{https://www.nytimes.com/2017/12/12/dining/ken-friedman-sexual-harassment.html}{Ken
Friedman} in 2018, after he was
\href{https://www.nytimes.com/2017/12/12/dining/ken-friedman-sexual-harassment.html}{accused
of sexual harassment}, and she
\href{https://www.nytimes.com/2018/10/16/dining/april-bloomfield-spotted-pig-ken-friedman.html}{conceded
in an interview} that she hadn't done enough to end the abuse.

The writer Meghan McCarron recently
\href{https://www.eater.com/2019/11/7/20953914/jessica-koslow-gabriela-camara-restaurant-onda-opening}{described}
the lasting power of auteur theory --- a way of thinking about
restaurants that has come at a cost both hard to measure and impossible
to ignore.

``In the food world's under-examined version of this theory, singular
visionaries are still seen as the sole architects of a restaurant's
greatness,'' Ms. McCarron wrote.

The idea of a chef-auteur is tenacious, and sly --- it limits the
narrative, and it sustains itself. Look at the homogeneity among major
industry best-of lists from organizations like the
\href{https://www.jamesbeard.org/}{James Beard Foundation},
\href{https://guide.michelin.com/en/article/news-and-views/michelin-nordic-guide-2020-stars-and-awards-announced}{Michelin}
and the \href{https://www.theworlds50best.com/}{World's 50 Best
Restaurants}.

White male chefs who already fit neatly into the stereotype of the
auteur are overrepresented, praised for a highly specific approach to
fine dining, then rewarded with more investment and opportunities to
replicate that same approach.

So many alternative kinds of food businesses are never considered for
awards or investments. They don't fit into the chef-auteur framework,
and in some cases have no desire to do so --- community farms with food
stalls, roving trucks, collaborative projects, temporary projects, or
family restaurants where three different cooks take turns in the
kitchen, depending on their child care schedules.

But for so many, it's already too late. They've been excluded from the
narrative, over and over again, to serve the idea of the auteur. They've
been subject to abuse. They've been paid unfairly. Many have dropped out
of the business altogether.

The pandemic has exposed the fragility and inequity of the restaurant
industry, disproportionately affecting Black people, people of color,
restaurant workers and those who keep the food chain running in the
nation's factories and farms. Bolstered by the power of the \#MeToo and
Black Lives Matter movements, workers are speaking up. The model for the
industry, as it exists now, has to change.

In a recent
\href{https://aliciakennedy.substack.com/p/on-restaurants}{newsletter},
Alicia Kennedy, a writer based in Puerto Rico, declared that the chef,
as an ego, had become irrelevant. ``What's next?'' she asked. And as
reports of
\href{https://www.washingtonpost.com/news/voraciously/wp/2020/07/13/after-la-cafe-sqirl-sold-moldy-jam-its-owner-cited-a-mycologist-to-defend-it-but-he-doesnt-approve/}{moldy
food} and allegations of
\href{https://thelandmag.com/beyond-moldy-jam-the-inside-story-of-what-went-wrong-at-sqirl/}{poor
conditions} for cooks at \href{https://sqirlla.com/}{Sqirl} surfaced
this summer, the Los Angeles writer Tien Nguyen asked another urgent
\href{https://tien.substack.com/p/what-would-a-food-media-that-de-centers}{question}:
What would food journalism look like if it centered on rank-and-file
workers instead of chefs?

It's hard but necessary to imagine these answers. And as workers
unionize at places like \href{https://twitter.com/TartineUnion}{Tartine}
in San Francisco and
\href{https://nwlaborpress.org/wp-content/uploads/2020/03/VoodooDoughnutsUnionAnnouncement.pdf}{Voodoo
Doughnut} in Portland, Ore., they're claiming power, demanding better
conditions and pushing toward newer, fairer models.

Other workers are pointing to the gap between how restaurants are
perceived and how they're run, as in Chicago, where more than 20
employees of Fat Rice
\href{https://www.nytimes.com/2020/06/16/dining/fat-rice-chicago-abe-conlon-racism.html}{challenged
their employer's social-media claim} that it supported racial justice.

Menus are collaborative, to some degree or another. Chefs lead that
work, perhaps assigning tests, approving new dishes, or tasting them,
editing them, and in most cases making the final decisions that shape
the way the food comes to the table. But in some cases dozens of other
cooks could be involved in the process.

Restaurants are the work of teams, kitchens full of cooks and
dishwashers coordinating with dining rooms full of servers, runners and
bartenders. Each role, each day, plays a part in a restaurant's success.

One of my last fancy dinners before the pandemic shut down dining rooms
in Los Angeles was at \href{https://www.thebazaar.com/somni/}{Somni}, a
small horseshoe bar inside the SLS Beverly Hills hotel owned by
\href{https://www.nytimes.com/2017/10/30/dining/jose-andres-puerto-rico.html}{José
Andrés}. The chef, Aitor Zabala, printed out a menu that credited
everyone working dinner service.

The porters on duty that night were Josue Rodriguez and Mario Alarcon.
The detailed chocolate work was by Ivonne Cerdas and Lindsey Newman.
About a dozen more cooks had worked on the exuberant, fast-flowing
27-course meal, and each one was listed, like the cast and crew on a
playbill.

When I asked him in an email about the design, Mr. Zabala replied that
he wanted the whole team to feel connected to the restaurant, and
responsible for its experience. He explained that it's part of why meals
at Somni include a service charge, and why all employees both contribute
to service and share in those earnings.

A menu is just a menu, but I found this one a tiny, eloquent gesture,
urging diners to consider the restaurant as a whole --- a collective ---
with so many people at work beyond the chef.

\emph{Follow} \href{https://twitter.com/nytfood}{\emph{NYT Food on
Twitter}} \emph{and}
\href{https://www.instagram.com/nytcooking/}{\emph{NYT Cooking on
Instagram}}\emph{,}
\href{https://www.facebook.com/nytcooking/}{\emph{Facebook}}\emph{,}
\href{https://www.youtube.com/nytcooking}{\emph{YouTube}} \emph{and}
\href{https://www.pinterest.com/nytcooking/}{\emph{Pinterest}}\emph{.}
\href{https://www.nytimes.com/newsletters/cooking}{\emph{Get regular
updates from NYT Cooking, with recipe suggestions, cooking tips and
shopping advice}}\emph{.}

Advertisement

\protect\hyperlink{after-bottom}{Continue reading the main story}

\hypertarget{site-index}{%
\subsection{Site Index}\label{site-index}}

\hypertarget{site-information-navigation}{%
\subsection{Site Information
Navigation}\label{site-information-navigation}}

\begin{itemize}
\tightlist
\item
  \href{https://help.nytimes.com/hc/en-us/articles/115014792127-Copyright-notice}{©~2020~The
  New York Times Company}
\end{itemize}

\begin{itemize}
\tightlist
\item
  \href{https://www.nytco.com/}{NYTCo}
\item
  \href{https://help.nytimes.com/hc/en-us/articles/115015385887-Contact-Us}{Contact
  Us}
\item
  \href{https://www.nytco.com/careers/}{Work with us}
\item
  \href{https://nytmediakit.com/}{Advertise}
\item
  \href{http://www.tbrandstudio.com/}{T Brand Studio}
\item
  \href{https://www.nytimes.com/privacy/cookie-policy\#how-do-i-manage-trackers}{Your
  Ad Choices}
\item
  \href{https://www.nytimes.com/privacy}{Privacy}
\item
  \href{https://help.nytimes.com/hc/en-us/articles/115014893428-Terms-of-service}{Terms
  of Service}
\item
  \href{https://help.nytimes.com/hc/en-us/articles/115014893968-Terms-of-sale}{Terms
  of Sale}
\item
  \href{https://spiderbites.nytimes.com}{Site Map}
\item
  \href{https://help.nytimes.com/hc/en-us}{Help}
\item
  \href{https://www.nytimes.com/subscription?campaignId=37WXW}{Subscriptions}
\end{itemize}
