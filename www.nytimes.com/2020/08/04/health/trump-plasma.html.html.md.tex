Sections

SEARCH

\protect\hyperlink{site-content}{Skip to
content}\protect\hyperlink{site-index}{Skip to site index}

\href{https://www.nytimes.com/section/health}{Health}

\href{https://myaccount.nytimes.com/auth/login?response_type=cookie\&client_id=vi}{}

\href{https://www.nytimes.com/section/todayspaper}{Today's Paper}

\href{/section/health}{Health}\textbar{}As Trump Praises Plasma,
Researchers Struggle to Finish Critical Studies

\url{https://nyti.ms/33qT4nT}

\begin{itemize}
\item
\item
\item
\item
\item
\end{itemize}

\href{https://www.nytimes.com/news-event/coronavirus?action=click\&pgtype=Article\&state=default\&region=TOP_BANNER\&context=storylines_menu}{The
Coronavirus Outbreak}

\begin{itemize}
\tightlist
\item
  live\href{https://www.nytimes.com/2020/08/04/world/coronavirus-cases.html?action=click\&pgtype=Article\&state=default\&region=TOP_BANNER\&context=storylines_menu}{Latest
  Updates}
\item
  \href{https://www.nytimes.com/interactive/2020/us/coronavirus-us-cases.html?action=click\&pgtype=Article\&state=default\&region=TOP_BANNER\&context=storylines_menu}{Maps
  and Cases}
\item
  \href{https://www.nytimes.com/interactive/2020/science/coronavirus-vaccine-tracker.html?action=click\&pgtype=Article\&state=default\&region=TOP_BANNER\&context=storylines_menu}{Vaccine
  Tracker}
\item
  \href{https://www.nytimes.com/2020/08/02/us/covid-college-reopening.html?action=click\&pgtype=Article\&state=default\&region=TOP_BANNER\&context=storylines_menu}{College
  Reopening}
\item
  \href{https://www.nytimes.com/live/2020/08/04/business/stock-market-today-coronavirus?action=click\&pgtype=Article\&state=default\&region=TOP_BANNER\&context=storylines_menu}{Economy}
\end{itemize}

Advertisement

\protect\hyperlink{after-top}{Continue reading the main story}

Supported by

\protect\hyperlink{after-sponsor}{Continue reading the main story}

\hypertarget{as-trump-praises-plasma-researchers-struggle-to-finish-critical-studies}{%
\section{As Trump Praises Plasma, Researchers Struggle to Finish
Critical
Studies}\label{as-trump-praises-plasma-researchers-struggle-to-finish-critical-studies}}

Thousands of Covid-19 patients have been treated with blood plasma
outside of rigorous clinical trials --- hampering research that would
have shown whether the therapy worked.

\includegraphics{https://static01.nyt.com/images/2020/07/31/science/31VIRUS-PLASMA1/31VIRUS-PLASMA1-articleLarge.jpg?quality=75\&auto=webp\&disable=upscale}

By \href{https://www.nytimes.com/by/katie-thomas}{Katie Thomas} and
\href{https://www.nytimes.com/by/noah-weiland}{Noah Weiland}

\begin{itemize}
\item
  Aug. 4, 2020
\item
  \begin{itemize}
  \item
  \item
  \item
  \item
  \item
  \end{itemize}
\end{itemize}

An American Airlines flight took off from La Guardia Airport in New York
last Wednesday morning, carrying 100 pouches of
\href{https://www.nytimes.com/2020/08/04/health/trump-plasma.html}{blood
plasma donated by Covid-19} survivors for delivery to Rio de Janeiro.
American scientists are hoping Covid-19 patients in Brazil will help
them answer a century-old question: Can this golden serum, loaded with
antibodies against a pathogen, actually heal the sick?

The truth is that no one knows if it works.

Since April, the Trump administration has funneled \$48 million into a
program with the Mayo Clinic, allowing more than 53,000 Covid-19
patients to get plasma infusions. Doctors and hospitals desperate to
save the sickest patients have been eager to try a therapy that is safe
and might work. Tens of thousands more people are now enrolled to get
the treatment that's been trumpeted by everyone from the president to
the actor Dwayne Johnson, better known as The Rock.

President Trump on Monday promoted its promise: ``You had something very
special. You had something that knocked it out. So we want to be able to
use it,'' he said, calling on Covid-19 survivors to donate their plasma,
which he called a ``beautiful ingredient.''

But the unexpected demand for plasma~has inadvertently undercut the
research that could prove that it works. The only way to get convincing
evidence is with a clinical trial that compares outcomes for patients
who are randomly assigned to get the treatment with those who are given
a placebo. Many patients and their doctors --- knowing they could get
the treatment under the government program --- have been unwilling to
join clinical trials that might provide them with a placebo instead of
the plasma.

The trials have also been stymied by the waning of the virus outbreak in
many cities, complicating researchers' ability to recruit sick people.
One of those clinical trials, at Columbia University, sputtered to a
halt after the outbreak subsided in New York. One of its leaders, Dr. W.
Ian Lipkin, looked for hospitals in other hot spots in the United States
to continue the work. But he found few takers.

``Without a randomized control trial, it's very difficult to be certain
that what you have is meaningful,'' he said.

As of last week, just 67 people had enrolled in the Columbia study ---
too few to form sound statistical conclusions. In a last-ditch effort,
Dr. Lipkin's team shipped the plasma to Brazil, where the epidemic is
still raging.

Now, at the height of a public health crisis, the government's push to
distribute an unproven treatment to desperately ill patients as quickly
as possible could come at the cost of completing clinical trials that
would potentially benefit millions around the world by determining
whether those treatments actually work.

In a statement, a spokeswoman for the Food and Drug Administration said
that the expanded access program was meant to bridge the gap until
trials could get underway and ``was never intended to substitute for
randomized clinical trials, which are critically important for the
demonstration of efficacy.''

The F.D.A. is preparing an emergency authorization to use the treatment,
according to scientists who have been briefed on the plans. The policy
would ease the clerical burden on hospitals to get clearance for
transfusions, further hampering clinical trials, researchers said. An
F.D.A. spokeswoman declined to comment on whether such an authorization
was in the works.

The move would mean the F.D.A. is ``yielding to political pressure,''
said Dr. Luciana Borio, who oversaw public health preparedness for the
National Security Council under Mr. Trump and who was acting chief
scientist at the F.D.A. under President Barack Obama.

``I'm not as concerned about the political leaders having a misguided
approach to science,'' she said. ``What I'm really concerned about is
scientists having a misguided approach to science.''

On Monday, four former F.D.A. commissioners --- including Dr. Scott
Gottlieb, who served under Mr. Trump ---
\href{https://www.washingtonpost.com/opinions/2020/08/03/4-former-fda-commissioners-blood-plasma-might-be-covid-19-treatment-we-need/}{called
for more rigorous clinical trials} to evaluate whether plasma is an
effective treatment for the coronavirus. ``If this is going to work, we
need to do it right,'' they wrote.

\includegraphics{https://static01.nyt.com/images/2020/07/31/science/31VIRUS-PLASMA2/31VIRUS-PLASMA2-articleLarge.jpg?quality=75\&auto=webp\&disable=upscale}

Convalescent plasma, the pale yellow liquid that's left after blood is
stripped of its red and white cells, has been used
\href{https://www.ncbi.nlm.nih.gov/pmc/articles/PMC4781783/}{since the
1890s} to treat infectious diseases, including the flu, SARS and Ebola.
Scientists believe it may work by giving sick patients the antibodies of
those who have recovered from the infection.

\hypertarget{latest-updates-global-coronavirus-outbreak}{%
\section{\texorpdfstring{\href{https://www.nytimes.com/2020/08/04/world/coronavirus-cases.html?action=click\&pgtype=Article\&state=default\&region=MAIN_CONTENT_1\&context=storylines_live_updates}{Latest
Updates: Global Coronavirus
Outbreak}}{Latest Updates: Global Coronavirus Outbreak}}\label{latest-updates-global-coronavirus-outbreak}}

Updated 2020-08-05T07:58:24.076Z

\begin{itemize}
\tightlist
\item
  \href{https://www.nytimes.com/2020/08/04/world/coronavirus-cases.html?action=click\&pgtype=Article\&state=default\&region=MAIN_CONTENT_1\&context=storylines_live_updates\#link-762df92}{As
  talks drag on, McConnell signals openness to jobless aid extension,
  and negotiators agree on a deadline.}
\item
  \href{https://www.nytimes.com/2020/08/04/world/coronavirus-cases.html?action=click\&pgtype=Article\&state=default\&region=MAIN_CONTENT_1\&context=storylines_live_updates\#link-1228a480}{Novavax
  sees encouraging results from two studies of its experimental
  vaccine.}
\item
  \href{https://www.nytimes.com/2020/08/04/world/coronavirus-cases.html?action=click\&pgtype=Article\&state=default\&region=MAIN_CONTENT_1\&context=storylines_live_updates\#link-794484ed}{Mississippians
  must now wear masks in public, governor says.}
\end{itemize}

\href{https://www.nytimes.com/2020/08/04/world/coronavirus-cases.html?action=click\&pgtype=Article\&state=default\&region=MAIN_CONTENT_1\&context=storylines_live_updates}{See
more updates}

More live coverage:
\href{https://www.nytimes.com/live/2020/08/04/business/stock-market-today-coronavirus?action=click\&pgtype=Article\&state=default\&region=MAIN_CONTENT_1\&context=storylines_live_updates}{Markets}

Plasma's potential benefits are also promoted on conservative talk
shows, as was hydroxychloroquine, a treatment for malaria that was
\href{https://www.nytimes.com/2020/03/20/health/coronavirus-chloroquine-trump.html}{enthusiastically
embraced by Mr. Trump} but had not been found effective against the
coronavirus in recent clinical trials.

Unlike hydroxychloroquine, which has potentially harmful side effects,
plasma was seen as safe and top medical researchers had enthusiastically
set out to study it as American hospitals filled with Covid-19 patients.

``We are in a medical crisis --- we don't have alternatives,'' said Dr.
Arturo Casadevall, a microbiologist at Johns Hopkins University who is
the chair of the National Covid-19 Convalescent Plasma Project, a
consortium coordinating research into the therapy.

But at least 10 randomized, placebo-controlled trials in the United
States have enrolled only a few hundred people. And now, seven months
into the health crisis, some scientists say the F.D.A.'s program has
undermined their efforts to get answers about plasma's utility.

``I've seen other people describe it as liquid gold,'' said Dr. Richard
Kaufman, medical director of the transfusion service at the Brigham and
Women's Hospital in Boston, where he is the principal investigator of a
trial that had
\href{https://clinicaltrials.gov/ct2/show/NCT04361253?term=convalescent+plasma\&type=Intr\&cond=COVID\&intr=randomized\&draw=3\&rank=11}{intended
to enroll 220 patients} but has enrolled only one. ``I would say I have
a lot of uncertainty at this point.''

Image

Drawing blood from a horse to produce anti-diphtheria serum at the Paris
Pasteur Institute in 1892.Credit...Pictorial Press/Alamy

\hypertarget{a-century-of-experimentation}{%
\subsection{A century of
experimentation}\label{a-century-of-experimentation}}

Antibodies have been tapped to heal the sick
\href{https://academic.oup.com/cid/article/21/1/150/402600}{since at
least the 1890s}, when doctors used the serum of animals to treat
diphtheria, a dangerous bacterial disease. Convalescent plasma was used
during the 1918 flu pandemic, and so-called serum therapy became a
treatment for everything from pneumonia to measles. In 1925, teams of
sled dogs
\href{http://www.bbc.com/earth/story/20161014-in-1925-a-remote-town-was-saved-from-lethal-disease-by-dogs}{traveled
hundreds of miles} over ice to deliver serum to the Alaskan town Nome,
which was battling a diphtheria outbreak.

Although it fell out of favor in the 1940s with the discovery of
antibiotics, convalescent plasma is often the first tool that doctors
use when they are desperate to treat an emerging epidemic.

So when the coronavirus began spreading this year, doctors in Wuhan,
China, as well as in Iran and Italy turned to the old standby.

Dr. Casadevall became one of its earliest U.S. backers,
\href{https://www.wsj.com/articles/how-a-boys-blood-stopped-an-outbreak-11582847330}{writing
an opinion piece in The Wall Street Journal in February} and calling
colleagues from his Baltimore living room to encourage its study.

By late March, as deaths from the virus rose,
\href{https://www.nytimes.com/2020/03/26/health/plasma-coronavirus-treatment.html}{Mount
Sinai Hospital in New York} and
\href{https://www.houstonmethodist.org/blog/articles/2020/mar/coronavirus-blood-transfusion-therapy-may-offer-promise-for-critically-ill-patients/}{Houston
Methodist in Texas} began transfusing patients with plasma.

As the outbreak spread across the United States, calls grew to expand
distribution of plasma. But hospitals could use the plasma on a limited
number of patients only if they received emergency approval from the
F.D.A. Every day beginning in March, the agency heard from the doctors
of hundreds of patients asking for permission to try plasma, according
to a spokeswoman for the agency.

A loosely organized group of doctors, including Dr. Casadevall, began
pushing for a more coordinated approach. On April 3, the F.D.A. and the
Mayo Clinic
\href{https://www.fda.gov/news-events/press-announcements/coronavirus-covid-19-update-fda-coordinates-national-effort-develop-blood-related-therapies-covid-19}{opened}
the ``expanded access'' program, using plasma donated through the
American Red Cross.

Extracting plasma is cumbersome. It begins much like a blood donation,
with a needle inserted into a vein. The blood is drawn into a machine
with a centrifuge, which filters the plasma and returns the rest of the
blood to the body. The plasma must be stored at freezing temperatures.
It cannot be mass produced.

Dr. R. Scott Wright of the Mayo Clinic, who is helping to run its plasma
program, said he was an early advocate for conducting randomized trials
of convalescent plasma. But the mechanics of setting up large studies
were complicated by early shortages of plasma, coordination via Zoom and
the difficulty of predicting where the virus would spread to next.

Image

Discarded parts from an apheresis kit, which separates plasma from other
components of blood, after use with a recovered coronavirus patient who
donated blood in Seattle.Credit...Lindsey Wasson/Reuters

Still, researchers at major medical centers in the Northeast began
setting up studies. Dr. Mila B. Ortigoza, an infectious disease
specialist at NYU Langone Health, started a trial with colleagues at
Montefiore Medical Center in just weeks, enrolling its first patient on
April 17 and condensing years of work into days. But by the time it got
started, the pandemic was easing.

``The curve got squashed here in New York,'' said Dr. Elliott
Bennett-Guerrero, the leader of another randomized trial of plasma at
Stony Brook Medicine on Long Island. He said the hospital had enrolled
only about 80 of the
\href{https://clinicaltrials.gov/ct2/show/NCT04344535?term=convalescent+plasma\&type=Intr\&cond=COVID\&intr=randomized\&draw=3\&rank=20}{500
planned participants}. The trial is now stalled.

And the Mayo Clinic's expanded access program exploded.

``We initially thought that we would enroll 3,000 people,'' said Dr.
Michael Joyner, the scientist leading the effort. Dr. Casadevall was so
inundated with inquiries from patients' families asking about plasma
that he removed his personal email from the Johns Hopkins website.

By June, 20,000 people had received plasma, and the program released
\href{https://els-jbs-prod-cdn.jbs.elsevierhealth.com/pb/assets/raw/Health\%20Advance/journals/jmcp/jmcp_ft95_6_8.pdf}{a
promising report} on the method's safety. But there was no control group
for comparison, so the study could not evaluate whether the treatment
did any good.

\href{https://www.nytimes.com/interactive/2020/science/coronavirus-drugs-treatments.html}{}

\includegraphics{https://static01.nyt.com/images/2020/07/14/us/coronavirus-drugs-treatments-promo-1594761806092/coronavirus-drugs-treatments-promo-1594761806092-articleLarge-v12.png}

\hypertarget{coronavirus-drug-and-treatment-tracker}{%
\subsection{Coronavirus Drug and Treatment
Tracker}\label{coronavirus-drug-and-treatment-tracker}}

An updated list of potential treatments for Covid-19.

And yet, the treatment is now more popular than ever. Alex M. Azar II,
the secretary for health and human services, told governors on a call on
Monday that demand for plasma was outstripping supply.

Randomized trials outside the United States have not been able to prove
plasma's effectiveness, either. A trial at seven medical centers in
Wuhan, the likely ground zero for the virus,
\href{https://jamanetwork.com/journals/jama/article-abstract/2766943}{concluded
that convalescent plasma} did not significantly improve patients'
recovery time.

\href{https://www.nytimes.com/news-event/coronavirus?action=click\&pgtype=Article\&state=default\&region=MAIN_CONTENT_3\&context=storylines_faq}{}

\hypertarget{the-coronavirus-outbreak-}{%
\subsubsection{The Coronavirus Outbreak
›}\label{the-coronavirus-outbreak-}}

\hypertarget{frequently-asked-questions}{%
\paragraph{Frequently Asked
Questions}\label{frequently-asked-questions}}

Updated August 4, 2020

\begin{itemize}
\item ~
  \hypertarget{i-have-antibodies-am-i-now-immune}{%
  \paragraph{I have antibodies. Am I now
  immune?}\label{i-have-antibodies-am-i-now-immune}}

  \begin{itemize}
  \tightlist
  \item
    As of right
    now,\href{https://www.nytimes.com/2020/07/22/health/covid-antibodies-herd-immunity.html?action=click\&pgtype=Article\&state=default\&region=MAIN_CONTENT_3\&context=storylines_faq}{that
    seems likely, for at least several months.} There have been
    frightening accounts of people suffering what seems to be a second
    bout of Covid-19. But experts say these patients may have a
    drawn-out course of infection, with the virus taking a slow toll
    weeks to months after initial exposure. People infected with the
    coronavirus typically
    \href{https://www.nature.com/articles/s41586-020-2456-9}{produce}
    immune molecules called antibodies, which are
    \href{https://www.nytimes.com/2020/05/07/health/coronavirus-antibody-prevalence.html?action=click\&pgtype=Article\&state=default\&region=MAIN_CONTENT_3\&context=storylines_faq}{protective
    proteins made in response to an
    infection}\href{https://www.nytimes.com/2020/05/07/health/coronavirus-antibody-prevalence.html?action=click\&pgtype=Article\&state=default\&region=MAIN_CONTENT_3\&context=storylines_faq}{.
    These antibodies may} last in the body
    \href{https://www.nature.com/articles/s41591-020-0965-6}{only two to
    three months}, which may seem worrisome, but that's perfectly normal
    after an acute infection subsides, said Dr. Michael Mina, an
    immunologist at Harvard University. It may be possible to get the
    coronavirus again, but it's highly unlikely that it would be
    possible in a short window of time from initial infection or make
    people sicker the second time.
  \end{itemize}
\item ~
  \hypertarget{im-a-small-business-owner-can-i-get-relief}{%
  \paragraph{I'm a small-business owner. Can I get
  relief?}\label{im-a-small-business-owner-can-i-get-relief}}

  \begin{itemize}
  \tightlist
  \item
    The
    \href{https://www.nytimes.com/article/small-business-loans-stimulus-grants-freelancers-coronavirus.html?action=click\&pgtype=Article\&state=default\&region=MAIN_CONTENT_3\&context=storylines_faq}{stimulus
    bills enacted in March} offer help for the millions of American
    small businesses. Those eligible for aid are businesses and
    nonprofit organizations with fewer than 500 workers, including sole
    proprietorships, independent contractors and freelancers. Some
    larger companies in some industries are also eligible. The help
    being offered, which is being managed by the Small Business
    Administration, includes the Paycheck Protection Program and the
    Economic Injury Disaster Loan program. But lots of folks have
    \href{https://www.nytimes.com/interactive/2020/05/07/business/small-business-loans-coronavirus.html?action=click\&pgtype=Article\&state=default\&region=MAIN_CONTENT_3\&context=storylines_faq}{not
    yet seen payouts.} Even those who have received help are confused:
    The rules are draconian, and some are stuck sitting on
    \href{https://www.nytimes.com/2020/05/02/business/economy/loans-coronavirus-small-business.html?action=click\&pgtype=Article\&state=default\&region=MAIN_CONTENT_3\&context=storylines_faq}{money
    they don't know how to use.} Many small-business owners are getting
    less than they expected or
    \href{https://www.nytimes.com/2020/06/10/business/Small-business-loans-ppp.html?action=click\&pgtype=Article\&state=default\&region=MAIN_CONTENT_3\&context=storylines_faq}{not
    hearing anything at all.}
  \end{itemize}
\item ~
  \hypertarget{what-are-my-rights-if-i-am-worried-about-going-back-to-work}{%
  \paragraph{What are my rights if I am worried about going back to
  work?}\label{what-are-my-rights-if-i-am-worried-about-going-back-to-work}}

  \begin{itemize}
  \tightlist
  \item
    Employers have to provide
    \href{https://www.osha.gov/SLTC/covid-19/standards.html}{a safe
    workplace} with policies that protect everyone equally.
    \href{https://www.nytimes.com/article/coronavirus-money-unemployment.html?action=click\&pgtype=Article\&state=default\&region=MAIN_CONTENT_3\&context=storylines_faq}{And
    if one of your co-workers tests positive for the coronavirus, the
    C.D.C.} has said that
    \href{https://www.cdc.gov/coronavirus/2019-ncov/community/guidance-business-response.html}{employers
    should tell their employees} -\/- without giving you the sick
    employee's name -\/- that they may have been exposed to the virus.
  \end{itemize}
\item ~
  \hypertarget{should-i-refinance-my-mortgage}{%
  \paragraph{Should I refinance my
  mortgage?}\label{should-i-refinance-my-mortgage}}

  \begin{itemize}
  \tightlist
  \item
    \href{https://www.nytimes.com/article/coronavirus-money-unemployment.html?action=click\&pgtype=Article\&state=default\&region=MAIN_CONTENT_3\&context=storylines_faq}{It
    could be a good idea,} because mortgage rates have
    \href{https://www.nytimes.com/2020/07/16/business/mortgage-rates-below-3-percent.html?action=click\&pgtype=Article\&state=default\&region=MAIN_CONTENT_3\&context=storylines_faq}{never
    been lower.} Refinancing requests have pushed mortgage applications
    to some of the highest levels since 2008, so be prepared to get in
    line. But defaults are also up, so if you're thinking about buying a
    home, be aware that some lenders have tightened their standards.
  \end{itemize}
\item ~
  \hypertarget{what-is-school-going-to-look-like-in-september}{%
  \paragraph{What is school going to look like in
  September?}\label{what-is-school-going-to-look-like-in-september}}

  \begin{itemize}
  \tightlist
  \item
    It is unlikely that many schools will return to a normal schedule
    this fall, requiring the grind of
    \href{https://www.nytimes.com/2020/06/05/us/coronavirus-education-lost-learning.html?action=click\&pgtype=Article\&state=default\&region=MAIN_CONTENT_3\&context=storylines_faq}{online
    learning},
    \href{https://www.nytimes.com/2020/05/29/us/coronavirus-child-care-centers.html?action=click\&pgtype=Article\&state=default\&region=MAIN_CONTENT_3\&context=storylines_faq}{makeshift
    child care} and
    \href{https://www.nytimes.com/2020/06/03/business/economy/coronavirus-working-women.html?action=click\&pgtype=Article\&state=default\&region=MAIN_CONTENT_3\&context=storylines_faq}{stunted
    workdays} to continue. California's two largest public school
    districts --- Los Angeles and San Diego --- said on July 13, that
    \href{https://www.nytimes.com/2020/07/13/us/lausd-san-diego-school-reopening.html?action=click\&pgtype=Article\&state=default\&region=MAIN_CONTENT_3\&context=storylines_faq}{instruction
    will be remote-only in the fall}, citing concerns that surging
    coronavirus infections in their areas pose too dire a risk for
    students and teachers. Together, the two districts enroll some
    825,000 students. They are the largest in the country so far to
    abandon plans for even a partial physical return to classrooms when
    they reopen in August. For other districts, the solution won't be an
    all-or-nothing approach.
    \href{https://bioethics.jhu.edu/research-and-outreach/projects/eschool-initiative/school-policy-tracker/}{Many
    systems}, including the nation's largest, New York City, are
    devising
    \href{https://www.nytimes.com/2020/06/26/us/coronavirus-schools-reopen-fall.html?action=click\&pgtype=Article\&state=default\&region=MAIN_CONTENT_3\&context=storylines_faq}{hybrid
    plans} that involve spending some days in classrooms and other days
    online. There's no national policy on this yet, so check with your
    municipal school system regularly to see what is happening in your
    community.
  \end{itemize}
\end{itemize}

As in the U.S. trials, the Wuhan study had trouble recruiting
participants and concluded early with just 103 volunteers. An
\href{https://www.medrxiv.org/content/10.1101/2020.07.29.20162917v1.full.pdf}{analysis
recently conducted} by researchers, including Drs. Joyner and
Casadevall, found that several overseas studies hinted that plasma was
effective, but not all of them were randomized.

\hypertarget{an-opening-for-president-trump}{%
\subsection{An Opening for President
Trump}\label{an-opening-for-president-trump}}

The Trump administration has framed convalescent plasma as a rare bright
spot in the pandemic.

Eager to present his administration as marching toward a ``cure,'' Mr.
Trump has mentioned plasma alongside remdesivir and dexamethasone, two
coronavirus treatments that have been shown to be effective in
randomized trials.

Dr. Deborah L. Birx, the leader of the White House's coronavirus task
force, at one point pushed for the federal government to secure 500,000
bags of plasma to store for a possible wave of infections in the fall,
according to a senior administration official. She also pushed for
plasma transfusions in nursing homes, the official said.

When asked about these claims, a task force official said that Dr. Birx
wanted to move quickly to capitalize on the period of time after a
person is infected, when their plasma contains higher antibody levels.
Dr. Birx said she wanted clinical trials to include vulnerable people in
nursing homes, the official added.

Dr. Stephen M. Hahn, the F.D.A. commissioner, began discussing the
benefits of plasma at White House briefings in March. In interviews and
congressional testimony since then, he has presented it as one of the
few therapeutics the agency can publicly endorse.

Last week, he said the F.D.A. was ``encouraged by the early promising
data that we've seen'' and that it was ``studying these data to
determine, ultimately, the safety and efficacy of this product.''

But he added that if plasma ``doesn't turn out to be the treatment we
think it might be, remember that your donations still count with the
American Blood Centers and the American Red Cross.''

Image

Donated convalescent plasma at Inova Blood Services, a blood bank in
Dulles, Va.Credit...Alex Edelman/Agence France-Presse --- Getty Images

The treatment has the backing of celebrities like the songwriter Dolly
Parton,
\href{https://clinicaltrials.gov/ct2/show/NCT04362176?term=convalescent+plasma\&type=Intr\&cond=COVID\&intr=randomized\&draw=2\&rank=9}{who
is financing} a randomized trial at Vanderbilt University Medical Center
in Nashville, and Mr. Johnson, who recorded
\href{https://www.youtube.com/watch?v=DYpmJcAdp2E}{a video} pleading
with survivors to donate blood.

``The plasma that's in your blood can literally save lives,'' he says in
the message. ``But we have to act fast.''

Dr. Joyner said that Mayo researchers were preparing to publish a more
detailed analysis of the data they had collected through the access
program. But he said even he was not sure of the F.D.A.'s plans for the
expanded access program.

There is also an effort underway, led by New York medical institutions,
to
\href{https://med.nyu.edu/departments-institutes/population-health/divisions-sections-centers/biostatistics/research/continuous-monitoring-pooled-international-trials-convalescent-plasma-covid19-hospitalized-patients}{pool
the data of unfinished trials}, a strategy encouraged by Dr. Francis
Collins, the director of the National Institutes of Health. ``It is a
really, really powerful approach,'' said Dr. Liise-anne Pirofski, the
chief of infectious diseases at Albert Einstein College of Medicine and
Montefiore Medical Center who is leading the clinical trial with Dr.
Ortigoza.

But that idea, said Dr. Kaufman of Brigham and Women's Hospital, is less
than ideal. He said he did not plan to participate. ``I worry about
combining partially finished studies that really may be different,'' he
said.

Some of the trial investigators, like Dr. Lipkin, are finding new sites
where they hope to complete their work. NYU Langone is expanding to
hospitals in Connecticut, Florida and Texas. And researchers at Johns
Hopkins have begun two trials of convalescent plasma in people who are
not yet sick enough to be hospitalized, testing the theory that the
treatment might work best earlier in the infection.

Dr. Casadevall said that he still believed randomized trials were the
``only way we're going to know whether it works or not,'' but that they
should not be put in opposition to Mayo's program.

``These things can always be second-guessed afterwards,'' he said. ``But
given the likelihood that it would work and given the history of safety,
it was worth trying it as a compassionate use. Maybe you can do this in
an emergency and still walk out with efficacy data.''

Dr. Lipkin said that, in retrospect, he might have played a role in
shaping plasma's fame, unknowingly undermining his own trial. In March,
he
\href{https://twitter.com/loudobbs/status/1237892870331191297?lang=en}{appeared
on television shows like ``Lou Dobbs Tonight}'' on Fox News, where he
extolled its potential benefits --- a move that, he speculates, could
have led administration officials to move more quickly to expand access.

``I share some responsibility for this,'' he said. ``I think there are
all kinds of arguments one can make based on history, having a
precedent. But that's not a substitute for rigorous science.''

Sheri Fink contributed reporting from Houston. Katie Thomas reported
from Chicago, and Noah Weiland from Washington.

Image

Plasma taken from a coronavirus patient at the Hospital de Clínicas in
Buenos Aires.Credit...Juan Ignacio Roncoroni/EPA, via Shutterstock

Advertisement

\protect\hyperlink{after-bottom}{Continue reading the main story}

\hypertarget{site-index}{%
\subsection{Site Index}\label{site-index}}

\hypertarget{site-information-navigation}{%
\subsection{Site Information
Navigation}\label{site-information-navigation}}

\begin{itemize}
\tightlist
\item
  \href{https://help.nytimes.com/hc/en-us/articles/115014792127-Copyright-notice}{©~2020~The
  New York Times Company}
\end{itemize}

\begin{itemize}
\tightlist
\item
  \href{https://www.nytco.com/}{NYTCo}
\item
  \href{https://help.nytimes.com/hc/en-us/articles/115015385887-Contact-Us}{Contact
  Us}
\item
  \href{https://www.nytco.com/careers/}{Work with us}
\item
  \href{https://nytmediakit.com/}{Advertise}
\item
  \href{http://www.tbrandstudio.com/}{T Brand Studio}
\item
  \href{https://www.nytimes.com/privacy/cookie-policy\#how-do-i-manage-trackers}{Your
  Ad Choices}
\item
  \href{https://www.nytimes.com/privacy}{Privacy}
\item
  \href{https://help.nytimes.com/hc/en-us/articles/115014893428-Terms-of-service}{Terms
  of Service}
\item
  \href{https://help.nytimes.com/hc/en-us/articles/115014893968-Terms-of-sale}{Terms
  of Sale}
\item
  \href{https://spiderbites.nytimes.com}{Site Map}
\item
  \href{https://help.nytimes.com/hc/en-us}{Help}
\item
  \href{https://www.nytimes.com/subscription?campaignId=37WXW}{Subscriptions}
\end{itemize}
