Sections

SEARCH

\protect\hyperlink{site-content}{Skip to
content}\protect\hyperlink{site-index}{Skip to site index}

\href{https://www.nytimes.com/section/nyregion}{New York}

\href{https://myaccount.nytimes.com/auth/login?response_type=cookie\&client_id=vi}{}

\href{https://www.nytimes.com/section/todayspaper}{Today's Paper}

\href{/section/nyregion}{New York}\textbar{}After 6 Weeks, Victors Are
Declared in 2 N.Y. Congressional Primaries

\href{https://nyti.ms/3gpzQ5M}{https://nyti.ms/3gpzQ5M}

\begin{itemize}
\item
\item
\item
\item
\item
\item
\end{itemize}

Advertisement

\protect\hyperlink{after-top}{Continue reading the main story}

Supported by

\protect\hyperlink{after-sponsor}{Continue reading the main story}

\hypertarget{after-6-weeks-victors-are-declared-in-2-ny-congressional-primaries}{%
\section{After 6 Weeks, Victors Are Declared in 2 N.Y. Congressional
Primaries}\label{after-6-weeks-victors-are-declared-in-2-ny-congressional-primaries}}

Representative Carolyn Maloney and Councilman Ritchie Torres won in New
York City after a Democratic primary that raised concerns about mail-in
voting.

\includegraphics{https://static01.nyt.com/images/2020/08/04/nyregion/04nyprimaries1/merlin_170409351_7017604d-3c83-428f-9845-8523c22d0ed3-articleLarge.jpg?quality=75\&auto=webp\&disable=upscale}

\href{https://www.nytimes.com/by/jesse-mckinley}{\includegraphics{https://static01.nyt.com/images/2018/02/20/multimedia/author-jesse-mckinley/author-jesse-mckinley-thumbLarge.jpg}}\href{https://www.nytimes.com/by/shane-goldmacher}{\includegraphics{https://static01.nyt.com/images/2018/07/27/multimedia/author-shane-goldmacher/author-shane-goldmacher-thumbLarge.png}}\href{https://www.nytimes.com/by/matt-stevens}{\includegraphics{https://static01.nyt.com/images/2019/04/03/multimedia/author-matt-stevens/author-matt-stevens-thumbLarge.png}}

By \href{https://www.nytimes.com/by/jesse-mckinley}{Jesse McKinley},
\href{https://www.nytimes.com/by/shane-goldmacher}{Shane Goldmacher} and
\href{https://www.nytimes.com/by/matt-stevens}{Matt Stevens}

\begin{itemize}
\item
  Aug. 4, 2020
\item
  \begin{itemize}
  \item
  \item
  \item
  \item
  \item
  \item
  \end{itemize}
\end{itemize}

After six weeks of delays, the New York City Board of Elections
confirmed results in a pair of congressional races on Tuesday evening,
delivering victories to a longtime Democratic incumbent and a young city
lawmaker who could be a trailblazer for gay and African-American rights.

In the South Bronx, Ritchie Torres, a 32-year-old New York City
councilman, won a 12-way Democratic primary for a soon-to-be-open House
seat, continuing a dramatic remaking of the New York congressional
delegation.

Just to the south, Representative Carolyn B. Maloney, 74, narrowly
brushed back a primary challenge from
\href{https://www.nytimes.com/2018/06/21/nyregion/congress-primaries-democrats-midterm-ny.html}{Suraj
Patel}, 36. The longtime incumbent just managed to sidestep a wave of
youthful progressivism that has tilted New York's congressional
delegation to the left.

The
\href{https://www.nytimes.com/2020/08/03/nyregion/nyc-mail-ballots-voting.html}{extensive
delays} in declaring winners had stirred significant concerns about the
problems facing officials trying to conduct elections during the
coronavirus outbreak. New York City's handling of the primary has been
used as fodder by President Trump to raise questions about whether the
nation is ready for the general election in November.

The primary was held on June 23, though the outbreak had caused a huge
expansion in the use of vote-by-mail, which led to a torrent of absentee
ballots sent to the Board of Elections in New York City.

The outbreak has prompted states across the nation to consider expanding
mail-in voting for the general election in November, as public health
officials worry that convening at polling locations may spread the
disease. In New York City, officials were left counting more than
400,000 absentee ballots, more than 10 times the usual number in a
primary.

\includegraphics{https://static01.nyt.com/images/2020/08/04/nyregion/04nyprimaries2/04nyprimaries2-articleLarge.jpg?quality=75\&auto=webp\&disable=upscale}

The resulting backlog drew the derision of President Trump, who used the
long delays as a reason to cast aspersions on voting-by-mail systems
nationwide. It also led to bickering between Gov. Andrew M. Cuomo and
other officials.

Although city elections officials certified the results, they did not
release updated vote totals; The Associated Press, which typically
declares election results, reported the results on Wednesday.

Mr. Torres, who identifies as Afro-Latino, would most likely be one of
the first two openly gay Black or Latino members of Congress; the other
is Mondaire Jones, a 33-year old lawyer
\href{https://www.nytimes.com/2020/07/14/nyregion/mondaire-jones-house-primary.html}{who
defeated} another crowded field seeking to fill the seat in the Hudson
Valley being vacated by Nita Lowey, the first woman to chair the House
Appropriations Committee.

Both Ms. Maloney's seat in the 12th Congressional District, which
includes parts of Manhattan, Queens and Brooklyn, and the 15th
Congressional District in the Bronx, where Mr. Torres won his primary,
are solidly Democratic, making both candidates overwhelming favorites to
win in November.

The incumbent in the 15th, Representative José E. Serrano, has served in
Congress for three decades, and some Democrats were already threatening
to run against him in the primary before he
\href{https://www.nytimes.com/2019/03/25/nyregion/jose-serrano-parkinsons-retire.html}{announced
his retirement} ahead of 2020, citing the effects of Parkinson's
disease.

Among the candidates Mr. Torres finished ahead of were Michael Blake, a
state assemblyman and vice chair of the Democratic National Committee;
Samelys López, a community organizer who had the backing of some key
progressives, including Representative Alexandria Ocasio-Cortez; Ydanis
Rodriguez, a city councilman; and Melissa Mark-Viverito, a former City
Council speaker.

But it was the chance to topple the Rev. Rubén Díaz Sr., Mr. Torres had
said in an interview before the election, that would be especially
sweet, representing ``poetic justice that is long overdue.''

Looking back to his first City Council race seven years ago, he said,
``I ran in a state of fear because of the homophobic political culture
that Rubén Díaz Sr. has spent his life cultivating in the Bronx.''

Mr. Díaz, a 77-year-old Pentecostal minister who has served for two
decades in the State Senate and the City Council, was under pressure
last year from Council colleagues demanding that he resign for saying
that legislative body was ``controlled by the homosexual community.''

Ms. Maloney's race against Mr. Patel was also marred by a dispute that
wound up in federal court, where a judge in Manhattan
\href{https://www.nytimes.com/2020/08/03/nyregion/nyc-congress-carolyn-maloney-ballots.html}{ruled
late Monday} that elections officials must count at least 1,000 disputed
mail-in ballots in their race, even though they did not have accurate or
extant postmarks.

Even if Mr. Patel were to capture all those votes, it would not be
enough for him to overtake Ms. Maloney. Nonetheless, Mr. Patel, a lawyer
and business teacher who had framed the Democratic primary as a ``change
election'' and himself as a change agent, has not conceded.

``The Board of Elections has preliminarily certified our race without a
final vote tally,'' he said on Wednesday. ``The Democratic process does
not end when it becomes politically inconvenient.''

Ms. Maloney has served in Congress since 1993 and
\href{https://www.nytimes.com/2019/11/20/us/politics/carolyn-maloney-oversight-committee.html}{became
the first woman ever to lead the powerful Oversight and Reform
Committee} in November when she was elected to succeed Representative
Elijah E. Cummings of Maryland, who had
\href{https://www.nytimes.com/2019/10/17/us/politics/elijah-cummings-death-illness.html}{died
a month earlier}.

In a statement released on Tuesday night, Ms. Maloney framed the primary
as a ``historic election, with historic turnout and participation, and a
historic wait time for results.''

Advertisement

\protect\hyperlink{after-bottom}{Continue reading the main story}

\hypertarget{site-index}{%
\subsection{Site Index}\label{site-index}}

\hypertarget{site-information-navigation}{%
\subsection{Site Information
Navigation}\label{site-information-navigation}}

\begin{itemize}
\tightlist
\item
  \href{https://help.nytimes.com/hc/en-us/articles/115014792127-Copyright-notice}{©~2020~The
  New York Times Company}
\end{itemize}

\begin{itemize}
\tightlist
\item
  \href{https://www.nytco.com/}{NYTCo}
\item
  \href{https://help.nytimes.com/hc/en-us/articles/115015385887-Contact-Us}{Contact
  Us}
\item
  \href{https://www.nytco.com/careers/}{Work with us}
\item
  \href{https://nytmediakit.com/}{Advertise}
\item
  \href{http://www.tbrandstudio.com/}{T Brand Studio}
\item
  \href{https://www.nytimes.com/privacy/cookie-policy\#how-do-i-manage-trackers}{Your
  Ad Choices}
\item
  \href{https://www.nytimes.com/privacy}{Privacy}
\item
  \href{https://help.nytimes.com/hc/en-us/articles/115014893428-Terms-of-service}{Terms
  of Service}
\item
  \href{https://help.nytimes.com/hc/en-us/articles/115014893968-Terms-of-sale}{Terms
  of Sale}
\item
  \href{https://spiderbites.nytimes.com}{Site Map}
\item
  \href{https://help.nytimes.com/hc/en-us}{Help}
\item
  \href{https://www.nytimes.com/subscription?campaignId=37WXW}{Subscriptions}
\end{itemize}
