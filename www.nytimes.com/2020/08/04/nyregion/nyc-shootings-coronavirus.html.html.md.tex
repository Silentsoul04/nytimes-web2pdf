Sections

SEARCH

\protect\hyperlink{site-content}{Skip to
content}\protect\hyperlink{site-index}{Skip to site index}

\href{https://www.nytimes.com/section/nyregion}{New York}

\href{https://myaccount.nytimes.com/auth/login?response_type=cookie\&client_id=vi}{}

\href{https://www.nytimes.com/section/todayspaper}{Today's Paper}

\href{/section/nyregion}{New York}\textbar{}The Mayor Blames the Virus
for Shootings. Here's What Crime Data Shows.

\url{https://nyti.ms/33oVLGI}

\begin{itemize}
\item
\item
\item
\item
\item
\item
\end{itemize}

Advertisement

\protect\hyperlink{after-top}{Continue reading the main story}

Supported by

\protect\hyperlink{after-sponsor}{Continue reading the main story}

\hypertarget{the-mayor-blames-the-virus-for-shootings-heres-what-crime-data-shows}{%
\section{The Mayor Blames the Virus for Shootings. Here's What Crime
Data
Shows.}\label{the-mayor-blames-the-virus-for-shootings-heres-what-crime-data-shows}}

Mr. de Blasio has pointed to court delays and bail reform to explain the
surge in gun violence. But the N.Y.P.D.'s own numbers tell a different
story.

\includegraphics{https://static01.nyt.com/images/2020/08/03/nyregion/00POLICEDATA/merlin_174653094_a9ca17cb-5305-4a9c-907b-1b2caaadcd37-articleLarge.jpg?quality=75\&auto=webp\&disable=upscale}

By \href{https://www.nytimes.com/by/alan-feuer}{Alan Feuer}

\begin{itemize}
\item
  Aug. 4, 2020
\item
  \begin{itemize}
  \item
  \item
  \item
  \item
  \item
  \item
  \end{itemize}
\end{itemize}

In the past few weeks, Mayor Bill de Blasio and his police commissioner,
Dermot F. Shea, have tied the steep rise in shootings in New York City
to a breakdown in the criminal justice system that they contend has
allowed criminals back out on the streets.

The mayor and commissioner have cited a range of causes that they have
portrayed as outside their control: the pandemic and the George Floyd
protests, as well as measures approved by the State Legislature,
including one that eliminated cash bail for many defendants.

But a confidential analysis of police data, conducted by city officials
but not released to the public, offers little if any evidence to back up
their claims. In fact, the analysis, obtained by The New York Times,
suggests the state's new bail law and the mass release of inmates from
city jails in recent months because of the coronavirus outbreak played
almost no role in the spike in shootings.

Of the 1,500 inmates let out of Rikers from March 16 to April 30, only
seven had been rearrested on a weapons charge by mid-July, according to
the confidential analysis.

Nearly 2,000 people who in July had open gun cases were allowed to go
home to await trial, but only about 40 of those defendants were arrested
on another weapons charge while they were out, the analysis said.

Instead, the analysis points to a different possible reason for the wave
of shootings: The number of arrests for gun crimes has plummeted.

While murders and shootings have surged, reports of other major crimes
have actually fallen in recent months. Still, the spike in gun violence
has stirred deep fears that the city might be sliding back to an era of
random violence on the streets. Recent shooting victims have included a
\href{https://www.nytimes.com/2020/07/27/nyregion/nyc-shootings-weekend.html?searchResultPosition=1}{two
teenagers going to play basketball} and
a\href{https://www.nytimes.com/2020/07/13/nyregion/Davell-Gardner-brooklyn-shooting.html?searchResultPosition=9}{baby
boy.}

New York City is not alone.
\href{https://www.nytimes.com/2020/07/05/us/chicago-shootings.html?searchResultPosition=6}{Shootings
have skyrocketed in major cities}across the country, and that surge has
led to intense political fights over whether efforts to rein in the
police, including the Defund the Police movement touched off by the
killing of George Floyd, are playing a role.

On Sunday, another 19 people were shot in New York City, one fatally.
Through the first seven months of this year, shootings were up 72
percent over the same period last year and murders rose 30 percent, even
as reports of other violent crimes like rape, assault and robbery fell.

The police say feuds between street gangs are behind most of the
incidents, and so far detectives have been unable to make enough arrests
to stop reprisals. The pandemic and the need to divert investigators to
cover widespread protests have set back investigations, police officials
said.

In recent days, Mr. de Blasio has been particularly critical of the
courts, saying that the lack of trials because of the pandemic and the
inability of prosecutors to push cases forward with indictments were ``a
huge piece'' of the spike in violent crime.

``The bottom line is our criminal justice system needs to get back to
full strength,'' Mr. de Blasio said. ``Our courts not only need to
reopen, they need to reopen as fully and as quickly as possible.''

But prosecutors, court officials and defense lawyers have pushed back
against that theory.

Lawrence Marks, the state's chief administrative judge, told the NY1
cable news station that the mayor's attacks on the courts were ``false,
misleading and irresponsible.''

Judge Marks countered that the rise in violent crime was more likely a
result of the sharp drop in gun arrests in recent months, a position
that the department's own data seems to buttress.

In mid-May, gun arrests citywide began to drop precipitously, the city
analysis of police data shows. During the week of May 24, there were 113
gun arrests. During the week of June 7, there were 71 such arrests. By
the week of June 28, there were only 22.

Over the same period, the data shows, shootings started rising. During
the week of May 24, there were 23 shootings; in the week of June 7,
there were 40. In the week of June 28, the number of shootings spiked to
63.

At a new conference on Tuesday, Mr. de Blasio said the city had deployed
more officers to troubled precincts, and gun arrests were beginning to
rise again. During the week ending on July 27, arrests for firearms
climbed up back up to 54, the police said.

The confidential analysis that was obtained by The Times was prepared by
city officials with Police Department data and shared with the city's
district attorneys' offices. It was provided to The Times by an official
who wanted to counter the mayor's narrative, but wished to remain
anonymous because the report was not intended to be released.

On Tuesday, Mr. de Blasio denied that he had ever blamed the rise in
violent crime solely on the changes the pandemic had wrought on the
courts. He said he had always cited a ``perfect storm'' of causes,
including the virus's devastating effect on the city's economy.

Still, he said that the slowdown in the courts and the lack of trials
had affected the behavior of some people, sending a signal that there
would not be consequences for their acts. ``And it affects the ability
to ensure that someone who should not be on the street, isn't,'' he
added.

``You cannot take away all the underpinnings of normal life and expect
the same outcome, and then when you don't have all of the pieces of the
criminal justice system working, that does affect the reality,'' he
said.

But the city's own analysis suggests the bail law, which allows many
defendants accused of nonviolent crimes to be released before trial
without posting bail, had little to do with the rise in violence. It
notes that shooting incidents stayed relatively stable for more than
four months after the legislation was passed.

The analysis also indicates that the courts are processing gun crimes at
close to the same rate as before the pandemic. According to the Police
Department's data, there were 2,181 unresolved gun cases in July ---
slightly fewer than the 2,285 gun cases that were open in December 2019.

Similarly, the courts handled 642 gun and murder arraignments from
October 2019 to December 2019. Between April and June of this year, they
handled 819 gun and murder arraignments --- and all of them were
conducted remotely by video.

``The way we are processing arrests has not changed at all,'' said Cyrus
R. Vance, Jr., the Manhattan district attorney. ``In May, the volume and
severity of the arrests we were handling was the same as it was in
January. We're open.''

Darcel D. Clark, the Bronx district attorney, said that her office's
complaint room, where new crimes are charged, was ``running at full
strength'' --- albeit virtually.

``It is wrong to say that the district attorneys are not prosecuting or
that the court system is not functioning,'' Ms. Clark said.

Court records in Brooklyn, which has seen some of the worst gun violence
in recent weeks, suggest there is little sign that the release of people
from jail was driving most of the shootings there.

From June 15 to July 15, according to court records, the Brooklyn
district attorney's office opened a total of five prosecutions of
defendants for shootings or for homicides with guns.

None of the defendants, an analysis of the records showed, had been
released from Rikers Island because of the pandemic or had been sent
back home on a separate case under the new, more lenient bail law. Nor
were they free because of the slowdown in court proceedings, the records
showed.

Still, Michael LiPetri, the Police Department's chief of crime control
strategies, said that the virus's effects on the criminal justice system
were being felt on the streets.

Early in the pandemic, Chief LiPetri said, many suspects arrested on gun
charges who in the past would have been asked to post bail were instead
released without bail to stem the spread of disease in jail.

So far this year, he said, 40 percent of all gun suspects were released
on their own recognizance, compared to only 25 percent last year, and
about 35 percent had bail set, compared to 55 percent last year.

The large number of people being sent home to await trial, even with a
serious gun charge, he said, had created a permissive atmosphere,
especially among gang members who the police believe are driving the
wave of shootings.

``When people get arrested and then get out, their crew members start
feeling comfortable carrying firearms,'' he said.

Chief LiPetri acknowledged the number of gun arrests had dropped off,
saying that the force was stretched thin because of the pandemic and the
need to redeploy people to cover protests.

In the past month, he said, the department has started moving robbery
detectives to work on violent crime and has shifted more than 300
officers in administrative positions to precincts with high numbers of
shootings.

``We were stretched --- even an agency as big as the N.Y.P.D.,'' he
said.

Advertisement

\protect\hyperlink{after-bottom}{Continue reading the main story}

\hypertarget{site-index}{%
\subsection{Site Index}\label{site-index}}

\hypertarget{site-information-navigation}{%
\subsection{Site Information
Navigation}\label{site-information-navigation}}

\begin{itemize}
\tightlist
\item
  \href{https://help.nytimes.com/hc/en-us/articles/115014792127-Copyright-notice}{©~2020~The
  New York Times Company}
\end{itemize}

\begin{itemize}
\tightlist
\item
  \href{https://www.nytco.com/}{NYTCo}
\item
  \href{https://help.nytimes.com/hc/en-us/articles/115015385887-Contact-Us}{Contact
  Us}
\item
  \href{https://www.nytco.com/careers/}{Work with us}
\item
  \href{https://nytmediakit.com/}{Advertise}
\item
  \href{http://www.tbrandstudio.com/}{T Brand Studio}
\item
  \href{https://www.nytimes.com/privacy/cookie-policy\#how-do-i-manage-trackers}{Your
  Ad Choices}
\item
  \href{https://www.nytimes.com/privacy}{Privacy}
\item
  \href{https://help.nytimes.com/hc/en-us/articles/115014893428-Terms-of-service}{Terms
  of Service}
\item
  \href{https://help.nytimes.com/hc/en-us/articles/115014893968-Terms-of-sale}{Terms
  of Sale}
\item
  \href{https://spiderbites.nytimes.com}{Site Map}
\item
  \href{https://help.nytimes.com/hc/en-us}{Help}
\item
  \href{https://www.nytimes.com/subscription?campaignId=37WXW}{Subscriptions}
\end{itemize}
