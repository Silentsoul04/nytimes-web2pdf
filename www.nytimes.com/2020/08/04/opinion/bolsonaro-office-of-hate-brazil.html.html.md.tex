Sections

SEARCH

\protect\hyperlink{site-content}{Skip to
content}\protect\hyperlink{site-index}{Skip to site index}

\href{https://myaccount.nytimes.com/auth/login?response_type=cookie\&client_id=vi}{}

\href{https://www.nytimes.com/section/todayspaper}{Today's Paper}

\href{/section/opinion}{Opinion}\textbar{}Brazil's Troll Army Moves Into
the Streets

\href{https://nyti.ms/3i4scOD}{https://nyti.ms/3i4scOD}

\begin{itemize}
\item
\item
\item
\item
\item
\item
\end{itemize}

Advertisement

\protect\hyperlink{after-top}{Continue reading the main story}

\href{/section/opinion}{Opinion}

Supported by

\protect\hyperlink{after-sponsor}{Continue reading the main story}

\hypertarget{brazils-troll-army-moves-into-the-streets}{%
\section{Brazil's Troll Army Moves Into the
Streets}\label{brazils-troll-army-moves-into-the-streets}}

President Jair Bolsonaro and his allies have seeded online hatred
against the institutions that defend democracy. Now the outrage is
spilling beyond the internet.

By Patrícia Campos Mello

Ms. Campos Mello is a Brazilian journalist.

\begin{itemize}
\item
  Aug. 4, 2020
\item
  \begin{itemize}
  \item
  \item
  \item
  \item
  \item
  \item
  \end{itemize}
\end{itemize}

\href{https://www.nytimes.com/pt/2020/08/04/opinion/international-world/bolsonaro-gabinete-do-odio.html}{Ler
em
português}\href{https://www.nytimes.com/es/2020/08/04/espanol/opinion/bolsonaro-oficina-odio-brasil.html}{Leer
en español}

\includegraphics{https://static01.nyt.com/images/2020/08/05/opinion/05campos/04campos-articleLarge.jpg?quality=75\&auto=webp\&disable=upscale}

SÃO PAULO, Brazil --- On June 13, members of ``Brazil's 300,'' a militia
of radical far-right supporters of President Jair Bolsonaro,
\href{https://www1.folha.uol.com.br/poder/2020/05/sara-winter-xinga-moraes-diz-querer-trocar-socos-com-ele-e-promete-inferniza-lo.shtml}{launched
fireworks in the direction} of the Supreme Federal Court building in
Brasília, simulating a bombing. ``Get ready, Supreme {[}Court{]} bandits
\ldots{} you are leading the country to communism,'' one of the leaders,
\href{https://www.metropoles.com/brasil/video-bolsonaristas-lancam-fogos-de-artificio-em-predio-do-stf}{who
broadcast the protest live, said}. ``It's over, damn it!,'' another
protester said, echoing the words the president
\href{https://www.youtube.com/watch?v=I2bZoC8FHJI}{had used} to condemn
an investigation by the Supreme Court against some of his supporters,
who are engaged in disinformation campaigns and threats against the
justices.

Where did this hatred of Brazil's highest court come from?

In the months leading up to the fireworks incident, thousands of social
media accounts, many of them fakes linked to supporters of Mr. Bolsonaro
or far-right bloggers, posted
\href{https://www1.folha.uol.com.br/poder/2020/05/sara-winter-xinga-moraes-diz-querer-trocar-socos-com-ele-e-promete-inferniza-lo.shtml}{threats}
against the Supreme Court justices. They called for the court to be
abolished, or for a return to a military dictatorship. One of the
president's followers even talked of
\href{https://g1.globo.com/politica/noticia/2020/06/17/moraes-vota-pela-legalidade-e-continuidade-do-inquerito-das-fake-news.ghtml}{killing
and dismembering} the justices and their families. It was only a matter
of time before the animosity spilled into the street.

This toxic environment has been fomented by what Brazilians call the
``office of hate,'' an operation run by advisers to the president, who
support a network of pro-Bolsonaro blogs and social media accounts that
spread fake news and attack journalists, politicians, artists and media
outlets that are critical of the president. The office of hate does not
have an official title or budget --- but its work is subsidized with
taxpayer money. It's unclear how many people work for this office, or
who they are. In fact, Mr. Bolsonaro and his allies deny that it exists.
But the seeds of hatred and division it is sowing threaten to undo our
democracy.

The Bolsonaro administration is currently facing three investigations
directly linked to this hate machine. One Supreme Court inquiry is
investigating attacks on members of the court financed by business
leaders and disseminated by the pro-Bolsonaro network, while another is
examining the financing of demonstrations calling for the closing of
Congress and the judiciary. Four inquiries in the Superior Electoral
Court are looking into the use of mass-messaging disinformation and
defamation campaigns through WhatsApp during the 2018 election campaign,
which was allegedly funded by business leaders.

Image

A supporter of Brazilian President Jair Bolsonaro throwing a water
balloon at a banner reading ``Target Shooting,'' with photos of
Alexandre de Moraes, minister of the Brazilian Supreme Court, and other
National Congress and Supreme Court officials during a motorcade and
protest over lockdown measures amidst the coronavirus in
Brasilia.Credit...Andressa Anholete/Getty Images

On July 8, Facebook removed
\href{https://www1.folha.uol.com.br/poder/2020/07/facebook-remove-contas-falsas-ligadas-aos-bolsonaros-e-ao-gabinete-da-presidencia.shtml}{dozens
of accounts}, some used by employees of Mr. Bolsonaro and his sons.
\href{https://elpais.com/internacional/2020-07-10/facebook-rompe-la-oficina-del-odio-una-red-de-88-cuentas-de-apoyo-a-jair-bolsonaro.html}{Tércio
Arnaud Tomaz}, a special adviser to Mr. Bolsonaro, who is believed to
run the office of hate,
\href{https://apnews.com/0c58cccec2004bf250c8dab38172cbc9}{administered
some of the accounts}.

I am sadly all too familiar with the office of hate. For the past two
years, I have been covering disinformation and politics. I also became
one of its targets in 2018, when I exposed in the newspaper Folha de São
Paulo that business leaders had been paying for the dissemination of
millions of fake messages via WhatsApp to influence the presidential
election that year.

As a result, I have faced a violent onslaught of crude threats and
personal attacks. Trolls and even politicians have shared memes where my
face appears in pornographic montages in which I am referred to as a
prostitute. People send me messages that say I should be raped. I am
suing Mr. Bolsonaro, his son Eduardo, and a pro-Bolsonaro blogger for
\href{http://www.fundamedios.us/incidentes/patriciacampos-demanda-jairbolsonaro-ofensas-periodista/}{moral
damages} for repeatedly stating or implying that I offer sex in exchange
for scoops.

I am not alone. Many respected female journalists in Brazil have also
been the target of misogynistic attacks. The press, along with the
courts and Congress, is one of the last barriers containing the
president. But I'm not sure for how much longer we will be able to
resist Mr. Bolsonaro and his followers. The increasingly aggressive
rhetoric and actions on the part of the president, his children, and
allies serve as a green light for pro-Bolsonaro militias to progress
from insults to injury.

On May 25, journalists were subjected to a vicious torrent of abuse from
his supporters at the presidential residence in Brasília.
\href{https://twitter.com/folha/status/1264913877399212034}{Footage
taken that day shows} reporters being called extortionists and crooks.
One woman is seen shouting: ``Scum! Trash! Rats! Bolsonaro until 2050!
Rotten press! Communists!''

Image

Supporters of Brazil's President Jair Bolsonaro, yelling at journalists,
calling them ``trash'' and ``coup plotters,'' after the president's
departure from his official residence of Alvorada palace in Brasilia,
Brazil, in May.Credit...Eraldo Peres/Associated Press

Journalists, of course, are not the only ones being targeted. Over the
last year, the office of hate has pitted Brazilians against one another,
and against those who have served as checks and balances against Mr.
Bolsonaro's authoritarian rise. It has eroded their trust in the
institutions designed to protect the county's democracy.

The group was behind a smear campaign that labeled Sergio Moro, the lead
judge of Brazil's landmark
\href{https://www.nytimes.com/2017/09/18/opinion/brazil-corruption-car-wash.html?searchResultPosition=1}{Car
Wash corruption investigation} and the former star justice minister, as
a ``traitor'' and ``communist.'' Mr. Moro
\href{https://www.nytimes.com/2020/04/24/world/americas/brazil-bolsonaro-moro.html}{resigned
in protest} in April, and denounced the president's meddling in a
Federal Police investigation to shield his sons and allies from criminal
investigations. Shortly after he quit, memes threatening Mr. Moro
flooded social media from fake accounts.

With the spread of the coronavirus,
\href{https://www.bbc.com/news/53361876}{fake news and bogus cures}
began to proliferate on social media, often via federal lawmakers with
hundreds of thousands of followers. Mr. Bolsonaro has
\href{https://www.hrw.org/news/2020/04/10/brazil-bolsonaro-sabotages-anti-covid-19-efforts}{thwarted
social distancing guidelines} put in place by governors. Accounts linked
to advisers like Mr. Arnaud Tomaz claimed that the reaction to Covid-19
\href{https://www.bbc.com/portuguese/brasil-53353594}{was exaggerated}
and that hydroxychloroquine, the antimalarial drug heavily
\href{https://www.nytimes.com/2020/06/13/world/americas/virus-brazil-bolsonaro-chloroquine.html}{promoted}
by Mr. Bolsonaro as a coronavirus cure, could kill the virus.

In April, the government created the ``Scoreboard of Life,'' on
\href{https://www.facebook.com/minsaude/posts/3549590468392877}{Facebook}
and
\href{https://twitter.com/secomvc/status/1257836970518200323}{Twitter},
which logged only the number of patients who have recovered. Then in
June, the Ministry of Health removed the total number of confirmed
Covid-19 cases and deaths since the pandemic's onset. Instead, a chart
showed only the cases and deaths reported in the previous 24 hours. The
Supreme Court later
\href{https://www.nytimes.com/2020/06/19/world/coronavirus-live-updates.html}{ordered}
the government to stop concealing data.

But the coronavirus is not deterred by political agendas. The
``\href{https://www.cnn.com/2020/05/23/americas/brazil-coronavirus-hospitals-intl/index.html}{little
flu},'' as Mr. Bolsonaro has referred to the virus that
\href{https://www.reuters.com/article/us-health-coronavirus-brazil-bolsonaro/brazilian-president-bolsonaro-says-he-has-mold-in-lungs-idUSKCN24V3SH}{he}
and his
\href{https://time.com/5874061/bolsonaro-wife-coronavirus/}{wife}
contracted in July, has already killed
\href{https://www.nytimes.com/interactive/2020/world/americas/brazil-coronavirus-cases.html}{more
than 94,000 Brazilians} --- the
\href{https://coronavirus.jhu.edu/map.html}{second-highest} death toll
in the world. The president's fake news campaign has sent thousands of
people to an early grave.

Beyond smear and disinformation campaigns, the office of hate's purpose
is far more nefarious: to weaken Brazil's democratic institutions.
Investigations by the prosecutor general revealed that some
pro-Bolsonaro legislators are spending cabinet funds on marketing
agencies that use social media to promote protests against the Supreme
Court and Congress, and in favor of military intervention in politics.

This incitement is intended to convince supporters that Supreme Court
justices are dictators, and that the press and Congress are preventing
the president from governing, and are plotting a coup. He may be laying
the groundwork to justify a military intervention on his behalf. And in
a young democracy like Brazil, institutions can be more fragile than
they appear.

Though Mr. Bolsonaro was democratically elected, he has professed
admiration for the military regime that ruled Brazil from 1964 to 1985.
Long before he ran for president,
\href{https://www.youtube.com/watch?v=qIDyw9QKIvw\&t=577s}{he said a
civil war would do the job} that the military regime didn't. He also
said he would shut down Congress if he were president. During the ****
2018 presidential elections,
\href{https://congressoemfoco.uol.com.br/especial/noticias/fas-usam-imagem-de-torturador-para-promover-bolsonaro/}{his
sons and followers printed T-shirts with the face of Col. Carlos Alberto
Brilhante Ustra}, the dictatorship's master torturer --- a figure
celebrated by the president.

Mr. Bolsonaro has tried to make good on his vision. In an effort to
bypass Congress, he has signed a record number of executive orders and
bills designed to do away with the independence of public universities,
which he describes as dens of communism; restrict access to information,
weaken unions and newspapers. He has threatened to disobey the
judiciary's rulings.

Image

A protest in Rio de Janeiro against the court, in June.Credit...Bruna
Prado/Getty Images

He has said that he wants to arm the entire population, so that people
can defend themselves against the ``dictatorship'' of the Supreme Court
and governors. ``I want everybody to have weapons because an armed
population will never be turned into slaves,'' he said during a cabinet
meeting in May. He later signed an executive order making it easier to
import guns and increasing the amount of ammunition a person can buy in
a year. In any functioning democracy, all this would amount to a call
for insurrection.

The president and his cronies would like nothing more than to silence
all of those who shine light on the darkest recesses of his government.

This incitement is intended to convince supporters that Supreme Court
justices are dictators, and that the press and Congress are preventing
the president from governing and plotting a coup. Attacks such as the
one against the Supreme Court and the aggression against a
photojournalist in a protest against Congress and in favor of military
coup are a sign that the office of hate is somehow succeeding in its
call to insurrection.

Last Wednesday, two men in a car outfitted with speakers showed up
\href{https://esportes.yahoo.com/noticias/aliados-jair-bolsonaro-atacam-casa-felipe-neto-010129218.html}{outside
the home of Felipe Neto}, an actor, writer and extremely popular YouTube
star. They accused Mr. Neto of destroying the ``most important
institution of all, which is the family,'' in an effort to intimidate
the actor, writer and popular YouTuber. One of the men who threatened
him had participated in the fireworks shooting at the Supreme Court in
Brasilía carried out by Brazil's 300. Days earlier, Mr. Neto called Mr.
Bolsonaro ``the worst pandemic president'' in
\href{https://www.nytimes.com/2020/07/15/opinion/coronavirus-covid-brazil-bolsonaro.html}{a
video that ran in The New York Times Opinion section}. The attack is yet
another example of how the vitriol propagated by office of hate is
increasingly extending beyond the internet*.*

\includegraphics{https://static01.nyt.com/images/2020/07/16/autossell/15op-brazil-thumb-print/15op-brazil-thumb-videoSixteenByNineJumbo1600.jpg}

If there is any hope for our young democracy, we must remain vigilant
and continue to hold this government accountable. It's not just lives of
Brazilians that are at stake, but the very institutions --- Congress,
the judiciary, academia and the media --- that for the time being have
managed to forestall the rise of authoritarianism.

Patrícia Campos Mello (@camposmello) is a journalist at the Brazilian
newspaper Folha de São Paulo and the author of the forthcoming ``Máquina
do ódio,'' about disinformation campaigns and Bolsonaro. This article
was translated by Erin Goodman from the Portuguese.

\emph{The Times is committed to publishing}
\href{https://www.nytimes.com/2019/01/31/opinion/letters/letters-to-editor-new-york-times-women.html}{\emph{a
diversity of letters}} \emph{to the editor. We'd like to hear what you
think about this or any of our articles. Here are some}
\href{https://help.nytimes.com/hc/en-us/articles/115014925288-How-to-submit-a-letter-to-the-editor}{\emph{tips}}\emph{.
And here's our email:}
\href{mailto:letters@nytimes.com}{\emph{letters@nytimes.com}}\emph{.}

\emph{Follow The New York Times Opinion section on}
\href{https://www.facebook.com/nytopinion}{\emph{Facebook}}\emph{,}
\href{http://twitter.com/NYTOpinion}{\emph{Twitter (@NYTopinion)}}
\emph{and}
\href{https://www.instagram.com/nytopinion/}{\emph{Instagram}}\emph{.}

Advertisement

\protect\hyperlink{after-bottom}{Continue reading the main story}

\hypertarget{site-index}{%
\subsection{Site Index}\label{site-index}}

\hypertarget{site-information-navigation}{%
\subsection{Site Information
Navigation}\label{site-information-navigation}}

\begin{itemize}
\tightlist
\item
  \href{https://help.nytimes.com/hc/en-us/articles/115014792127-Copyright-notice}{©~2020~The
  New York Times Company}
\end{itemize}

\begin{itemize}
\tightlist
\item
  \href{https://www.nytco.com/}{NYTCo}
\item
  \href{https://help.nytimes.com/hc/en-us/articles/115015385887-Contact-Us}{Contact
  Us}
\item
  \href{https://www.nytco.com/careers/}{Work with us}
\item
  \href{https://nytmediakit.com/}{Advertise}
\item
  \href{http://www.tbrandstudio.com/}{T Brand Studio}
\item
  \href{https://www.nytimes.com/privacy/cookie-policy\#how-do-i-manage-trackers}{Your
  Ad Choices}
\item
  \href{https://www.nytimes.com/privacy}{Privacy}
\item
  \href{https://help.nytimes.com/hc/en-us/articles/115014893428-Terms-of-service}{Terms
  of Service}
\item
  \href{https://help.nytimes.com/hc/en-us/articles/115014893968-Terms-of-sale}{Terms
  of Sale}
\item
  \href{https://spiderbites.nytimes.com}{Site Map}
\item
  \href{https://help.nytimes.com/hc/en-us}{Help}
\item
  \href{https://www.nytimes.com/subscription?campaignId=37WXW}{Subscriptions}
\end{itemize}
