Sections

SEARCH

\protect\hyperlink{site-content}{Skip to
content}\protect\hyperlink{site-index}{Skip to site index}

\href{https://myaccount.nytimes.com/auth/login?response_type=cookie\&client_id=vi}{}

\href{https://www.nytimes.com/section/todayspaper}{Today's Paper}

\href{/section/opinion}{Opinion}\textbar{}Don't Believe the Lie That
Voting Is All You Can Do

\url{https://nyti.ms/3fqA3EG}

\begin{itemize}
\item
\item
\item
\item
\item
\end{itemize}

Advertisement

\protect\hyperlink{after-top}{Continue reading the main story}

\href{/section/opinion}{Opinion}

Supported by

\protect\hyperlink{after-sponsor}{Continue reading the main story}

\hypertarget{dont-believe-the-lie-that-voting-is-all-you-can-do}{%
\section{Don't Believe the Lie That Voting Is All You Can
Do}\label{dont-believe-the-lie-that-voting-is-all-you-can-do}}

Stop minimizing the work of movements.

By Daniel Hunter

Mr. Hunter is the author of ``Building a Movement to End the New Jim
Crow.''

\begin{itemize}
\item
  Aug. 4, 2020
\item
  \begin{itemize}
  \item
  \item
  \item
  \item
  \item
  \end{itemize}
\end{itemize}

\includegraphics{https://static01.nyt.com/images/2020/08/05/opinion/30Hunter1/30Hunter1-articleLarge.jpg?quality=75\&auto=webp\&disable=upscale}

The Black Lives Matter movement has had significant wins in recent
months. Municipalities have removed statues of racists, corporations
have changed branding that reinforced racial stereotypes,
\href{https://www.nytimes.com/2020/06/12/us/schools-police-resource-officers.html}{schools
have cut ties with police forces} and
\href{https://www.nytimes.com/2020/06/08/us/unrest-defund-police.html}{cities
have reduced police funding}.

But too often, politicians, celebrities and community leaders who
applaud the protesters for these victories are quick to follow up by
asserting, like Mayor Keisha Lance Bottoms of Atlanta, that voting
``\href{https://www.nytimes.com/2020/06/03/opinion/police-protests-atlanta-keisha-bottoms.html}{would
be the most effective response, the deepest payback}'' for George
Floyd's death --- or that there is ``no greater form of protest'' than
voting, as Lisa Deeley, chair of the Philadelphia City Commissioners,
put it.

I've led movements for most of my adult life and have heard similar
misguided refrains far too many times. The truth is voting is an
honorable act that many movements use as a tactic. But the popular
message that it's the only \emph{real} source of power misleads the
public about how social change happens and stifles the energy required
to bring about the change we need.

Instead of suggesting that participation in movements is inferior to
voting, people with influence should educate themselves and the public
about the often hidden role of social movements in achieving change in
this country.

Movements led to the abolition of slavery, brought Jim Crow to its knees
and won child labor laws, the minimum wage, the Clean Water Act and
more. African-Americans and women wouldn't even have the right to vote
if it weren't for people taking action.

Those victories weren't just the results of elections. They came from
the work of activists to change social conditions. Where voting changes
the players on the battlefield, social movements alter the very terrain
on which the battle is being fought.

``Movement work is the thing that enables any of the legal and policy
change to be successful,''
\href{https://www.gq.com/story/chase-strangio-aclu-lgbtq-legal-victory}{Chase
Strangio}\textbf{,} a lawyer who won the recent Supreme Court ruling
\href{https://www.nytimes.com/2020/06/15/us/gay-transgender-workers-supreme-court.html}{protecting
L.G.B.T. rights}, explained in an interview with GQ. He noted that
Justice Neil Gorsuch, who wrote the majority opinion, had initially
worried that protecting transgender people might result in social
upheaval. But less than a year later, his mind had been changed.

``On some level, I have to believe that in eight months, he learned
something from watching what was going on in the world,'' he said. ``And
that is a testament not to our briefs and not to the legal movement, but
to the organizing movement.''

A common misconception about movements --- like the mythic story that
\href{https://genprogress.org/the-myth-of-rosa-parks/}{Rosa Parks's
refusal} to move to the back of the bus spontaneously sparked the civil
rights movement --- is that they ``just happen.''

Yes, George Floyd's brutal murder, a flagrantly racist president and the
pent-up emotions of a pandemic motivated people to take to the streets
to demand racial justice. But social movements never emerge just because
conditions are bad.

Bill Moyer, a movement strategist, wrote about this dynamic in his
``\href{https://www.indybay.org/olduploads/movement_action_plan.pdf}{Movement
Action Plan}.'' He noted that the partial meltdown of the
\href{https://www.nytimes.com/1979/04/16/archives/three-mile-island-notes-from-a-nightmare-three-mile-island-a.html}{Three
Mile Island nuclear power plant} in 1979 became a rallying point for
people concerned about the dangers of nuclear power. Yet Michigan's
Enrico Fermi plant had been closer to a full meltdown in 1966 and didn't
lead to soul-searching or a social crisis. The difference was that in
the intervening years, organizers had worked to seed local groups, build
national networks, hone responses to the pronuclear lobby and develop
alternative policy platforms.

The current movement has done all those things, spurred largely by the
2014 protests in Ferguson, Mo., over the killing of Michael Brown. It
grew into a network of dozens of local Black Lives Matter chapters
across the United States and Canada. Groups like
\href{http://agendatobuildblackfutures.org/wp-content/uploads/2016/01/BYP_AgendaBlackFutures_booklet_web.pdf}{Black
Youth Project 100} and
\href{https://m4bl.org/policy-platforms/}{Movement for Black Lives}
built comprehensive policy platforms, leading to radical,
\href{https://mavenroundtable.io/rinkusen/politics/why-defundthepolice-is-genius-strategy-ZNTk0AGz3kOxAwIqj2BTEg}{ground-shaking
demands} like ``defund the police.'' As Jessica Byrd, a leader in
Movement for Black Lives, said in a recent interview with Time,
``\href{https://time.com/5847506/time-100-talks-black-lives-matter/}{Movement
made this moment different}.''

If one isn't aware of this work, it's easy to assume that after this
phase of street protests ends, the movement will be gone and it will be
time to turn to the ``real'' work of voting to fulfill our civic duty.

But people who understand movements know that voting is not the end ---
it's one part of the process. Movements amplify complex questions that
otherwise get simplified to sound bites in elections. Questions like:
Does society really need armed police answering mental health crises?
Can the police be reformed while still armed with military-grade
weapons? What are practical alternatives to police systems? ** By
changing ** people's views, movements apply pressure to decision makers.

Contrary to popular belief, movements shouldn't be measured by whether
the preferred candidates get into office, nor are they undermined by
short-term failures to cobble together national legislation.

A better yardstick for a movement is the public's perception of the
problem, a growing certainty that current policies don't work --- and
ultimately people's commitment to embracing alternatives.

After all, the 1960s student sit-ins against segregation did not
immediately result in legislative wins. Even after the peak event of the
March on Washington, it took another year for the 1964 Civil Rights Act
to become law.

It's tempting to think that reform will rain down if we elect the right
leaders. Yet most of us know through experience that voting is no magic
bullet. Regardless of who wins the election in November, anyone seeking
justice knows there's an enormous amount of work ahead of us. Movements
provide an avenue to do that work.

So yes, I'll vote --- and help turn out the vote. But I'll never believe
the lie that that's the best or only thing I can do to change this
country.

Daniel Hunter is the associate director of global trainings at
\href{http://trainings.350.org/}{350.org}, a strategist with Sunrise
Movement and the author of
``\href{http://www.newjimcroworganizing.org/}{Building a Movement to End
the New Jim Crow}.''

\emph{The Times is committed to publishing}
\href{https://www.nytimes.com/2019/01/31/opinion/letters/letters-to-editor-new-york-times-women.html}{\emph{a
diversity of letters}} \emph{to the editor. We'd like to hear what you
think about this or any of our articles. Here are some}
\href{https://help.nytimes.com/hc/en-us/articles/115014925288-How-to-submit-a-letter-to-the-editor}{\emph{tips}}\emph{.
And here's our email:}
\href{mailto:letters@nytimes.com}{\emph{letters@nytimes.com}}\emph{.}

\emph{Follow The New York Times Opinion section on}
\href{https://www.facebook.com/nytopinion}{\emph{Facebook}}\emph{,}
\href{http://twitter.com/NYTOpinion}{\emph{Twitter (@NYTopinion)}}
\emph{and}
\href{https://www.instagram.com/nytopinion/}{\emph{Instagram}}\emph{.}

Advertisement

\protect\hyperlink{after-bottom}{Continue reading the main story}

\hypertarget{site-index}{%
\subsection{Site Index}\label{site-index}}

\hypertarget{site-information-navigation}{%
\subsection{Site Information
Navigation}\label{site-information-navigation}}

\begin{itemize}
\tightlist
\item
  \href{https://help.nytimes.com/hc/en-us/articles/115014792127-Copyright-notice}{©~2020~The
  New York Times Company}
\end{itemize}

\begin{itemize}
\tightlist
\item
  \href{https://www.nytco.com/}{NYTCo}
\item
  \href{https://help.nytimes.com/hc/en-us/articles/115015385887-Contact-Us}{Contact
  Us}
\item
  \href{https://www.nytco.com/careers/}{Work with us}
\item
  \href{https://nytmediakit.com/}{Advertise}
\item
  \href{http://www.tbrandstudio.com/}{T Brand Studio}
\item
  \href{https://www.nytimes.com/privacy/cookie-policy\#how-do-i-manage-trackers}{Your
  Ad Choices}
\item
  \href{https://www.nytimes.com/privacy}{Privacy}
\item
  \href{https://help.nytimes.com/hc/en-us/articles/115014893428-Terms-of-service}{Terms
  of Service}
\item
  \href{https://help.nytimes.com/hc/en-us/articles/115014893968-Terms-of-sale}{Terms
  of Sale}
\item
  \href{https://spiderbites.nytimes.com}{Site Map}
\item
  \href{https://help.nytimes.com/hc/en-us}{Help}
\item
  \href{https://www.nytimes.com/subscription?campaignId=37WXW}{Subscriptions}
\end{itemize}
