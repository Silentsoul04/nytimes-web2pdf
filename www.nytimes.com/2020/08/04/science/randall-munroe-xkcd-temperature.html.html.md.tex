Sections

SEARCH

\protect\hyperlink{site-content}{Skip to
content}\protect\hyperlink{site-index}{Skip to site index}

\href{https://www.nytimes.com/section/science}{Science}

\href{https://myaccount.nytimes.com/auth/login?response_type=cookie\&client_id=vi}{}

\href{https://www.nytimes.com/section/todayspaper}{Today's Paper}

\href{/section/science}{Science}\textbar{}How Hot Is Too Hot?

\url{https://nyti.ms/2DgnMpa}

\begin{itemize}
\item
\item
\item
\item
\item
\item
\end{itemize}

Advertisement

\protect\hyperlink{after-top}{Continue reading the main story}

Supported by

\protect\hyperlink{after-sponsor}{Continue reading the main story}

good question

\hypertarget{how-hot-is-too-hot}{%
\section{How Hot Is Too Hot?}\label{how-hot-is-too-hot}}

The human body can survive at surprisingly high temperatures, so long as
you're prepared to sweat.

By Randall Munroe

\begin{itemize}
\item
  Aug. 4, 2020
\item
  \begin{itemize}
  \item
  \item
  \item
  \item
  \item
  \item
  \end{itemize}
\end{itemize}

\emph{\textbf{``What is the hottest `room' temperature at which a human
body can, by sweating, keep itself cool enough to avoid health
damage?''}}

--- Matt B., Etterzhausen, Germany

It's not the heat, it's the humidity.

Your body avoids overheating by taking advantage of a bit of physics:
When water evaporates from a surface, it leaves the surface cooler. When
your body gets too hot, it pumps water onto your skin and lets it
evaporate, carrying away heat. This effect can actually lower the
temperature of your skin to below the air temperature. This allows
humans to survive in places where the air temperature is as high as
human body temperature --- as long as we keep drinking water to produce
more sweat.

\includegraphics{https://static01.nyt.com/images/2020/07/30/science/00SCI-MUNROE-ROOM4/merlin_175107396_f26297b9-d16c-4d17-ae29-978f2c9095ef-articleLarge.jpg?quality=75\&auto=webp\&disable=upscale}

Sweating works best in dry air.

If there's a lot of moisture in the air, then evaporation slows to a
crawl, because water condenses onto your skin almost as fast as the
moisture evaporates off it. When you feel sticky from sweat pooling on
your skin, it means your body is struggling to evaporate water fast
enough to keep you cool.

Image

I asked Zachary Schlader, a researcher at Indiana University who studies
how our bodies handle
\href{https://journals.physiology.org/doi/abs/10.1152/ajpregu.00018.2020}{extreme
heat}, about the hottest temperature a normal human could tolerate under
ideal conditions. He sent me a
\href{https://www.sciencedirect.com/science/article/abs/pii/S0003687014001355}{2014
study} by Ollie Jay, of the Thermal Ergonomics Laboratory at the
University of Sydney, and colleagues. The study found that a person who
is at rest, wearing minimal clothing, in a very dry room --- 10 percent
relative humidity --- and drinking water constantly could probably avoid
overheating in temperatures as high as 115 degrees Fahrenheit.

The limiting factor for our heat tolerance is sweat --- how quickly we
can produce it and how quickly it evaporates. If you kept your skin wet
with a steady spray of water, and sat in front of a powerful fan, you
could increase the evaporation rate and keep your skin cool in even
higher temperatures.

Image

But even if you do everything you can to increase sweat as fast as
possible, there is a limit to how cool you can make a surface through
evaporation. This limit is called the wet-bulb temperature, and it
depends on both temperature and humidity. Its precise value can be found
using various \href{https://www.weather.gov/epz/wxcalc_rh}{calculators}.

Models of human thermoregulation like the one in the 2014 paper don't
usually cover such extreme conditions, but I tried adjusting their
formulas to approximate what would happen under extreme evaporation and
high wind. The results suggested that, with the help of a pool of water
and a powerful fan, a human could conceivably tolerate heat of up to 140
degrees Fahrenheit in air with 10 percent humidity.

That seemed awfully high, so I ran the number by Dr. Schlader. ``Doing
some rough calculations, I come up with a similar number,'' he said.
``Honestly, I was surprised.'' But, he added, these models are likely
not reliable at such extremes. ``I would interpret such findings with
caution.''

Personally, my advice would be to avoid any room where the thermostat
has a setting labeled ``potentially survivable, under some
circumstances, according to theoretical calculations.''

Image

\textbf{\emph{{[}}\href{http://on.fb.me/1paTQ1h}{\emph{Like the Science
Times page on Facebook.}}} ****** \emph{\textbar{} Sign up for the}
\textbf{\href{http://nyti.ms/1MbHaRU}{\emph{Science Times
newsletter.}}\emph{{]}}}

Advertisement

\protect\hyperlink{after-bottom}{Continue reading the main story}

\hypertarget{site-index}{%
\subsection{Site Index}\label{site-index}}

\hypertarget{site-information-navigation}{%
\subsection{Site Information
Navigation}\label{site-information-navigation}}

\begin{itemize}
\tightlist
\item
  \href{https://help.nytimes.com/hc/en-us/articles/115014792127-Copyright-notice}{©~2020~The
  New York Times Company}
\end{itemize}

\begin{itemize}
\tightlist
\item
  \href{https://www.nytco.com/}{NYTCo}
\item
  \href{https://help.nytimes.com/hc/en-us/articles/115015385887-Contact-Us}{Contact
  Us}
\item
  \href{https://www.nytco.com/careers/}{Work with us}
\item
  \href{https://nytmediakit.com/}{Advertise}
\item
  \href{http://www.tbrandstudio.com/}{T Brand Studio}
\item
  \href{https://www.nytimes.com/privacy/cookie-policy\#how-do-i-manage-trackers}{Your
  Ad Choices}
\item
  \href{https://www.nytimes.com/privacy}{Privacy}
\item
  \href{https://help.nytimes.com/hc/en-us/articles/115014893428-Terms-of-service}{Terms
  of Service}
\item
  \href{https://help.nytimes.com/hc/en-us/articles/115014893968-Terms-of-sale}{Terms
  of Sale}
\item
  \href{https://spiderbites.nytimes.com}{Site Map}
\item
  \href{https://help.nytimes.com/hc/en-us}{Help}
\item
  \href{https://www.nytimes.com/subscription?campaignId=37WXW}{Subscriptions}
\end{itemize}
