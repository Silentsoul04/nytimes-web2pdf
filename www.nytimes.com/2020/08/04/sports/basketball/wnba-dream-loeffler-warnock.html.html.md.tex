Sections

SEARCH

\protect\hyperlink{site-content}{Skip to
content}\protect\hyperlink{site-index}{Skip to site index}

\href{https://www.nytimes.com/section/sports/basketball}{Pro Basketball}

\href{https://myaccount.nytimes.com/auth/login?response_type=cookie\&client_id=vi}{}

\href{https://www.nytimes.com/section/todayspaper}{Today's Paper}

\href{/section/sports/basketball}{Pro Basketball}\textbar{}W.N.B.A.
Players Escalate Protest of Anti-B.L.M. Team Owner

\url{https://nyti.ms/39WMLcZ}

\begin{itemize}
\item
\item
\item
\item
\item
\end{itemize}

\href{https://www.nytimes.com/news-event/george-floyd-protests-minneapolis-new-york-los-angeles?action=click\&pgtype=Article\&state=default\&region=TOP_BANNER\&context=storylines_menu}{Race
and America}

\begin{itemize}
\tightlist
\item
  \href{https://www.nytimes.com/interactive/2020/07/03/us/george-floyd-protests-crowd-size.html?action=click\&pgtype=Article\&state=default\&region=TOP_BANNER\&context=storylines_menu}{Black
  Lives Matter Movement}
\item
  \href{https://www.nytimes.com/interactive/2020/06/28/us/i-cant-breathe-police-arrest.html?action=click\&pgtype=Article\&state=default\&region=TOP_BANNER\&context=storylines_menu}{History
  of `I Can't Breathe'}
\item
  \href{https://www.nytimes.com/interactive/2020/06/10/upshot/black-lives-matter-attitudes.html?action=click\&pgtype=Article\&state=default\&region=TOP_BANNER\&context=storylines_menu}{How
  Public Opinion Shifted}
\item
  \href{https://www.nytimes.com/interactive/2020/07/16/us/black-lives-matter-protests-louisville-breonna-taylor.html?action=click\&pgtype=Article\&state=default\&region=TOP_BANNER\&context=storylines_menu}{45
  Days in Louisville}
\end{itemize}

Advertisement

\protect\hyperlink{after-top}{Continue reading the main story}

Supported by

\protect\hyperlink{after-sponsor}{Continue reading the main story}

\hypertarget{wnba-players-escalate-protest-of-anti-blm-team-owner}{%
\section{W.N.B.A. Players Escalate Protest of Anti-B.L.M. Team
Owner}\label{wnba-players-escalate-protest-of-anti-blm-team-owner}}

Led by the members of the Atlanta Dream, players have begun wearing
T-shirts supporting a political opponent of Senator Kelly Loeffler, who
co-owns the Dream and has spoken against the Black Lives Matter
movement.

\includegraphics{https://static01.nyt.com/images/2020/08/04/sports/04wnba-dream-1/merlin_175030845_bef46721-096f-4eb5-b04e-b1574eafa348-articleLarge.jpg?quality=75\&auto=webp\&disable=upscale}

By \href{https://www.nytimes.com/by/sopan-deb}{Sopan Deb}

\begin{itemize}
\item
  Aug. 4, 2020
\item
  \begin{itemize}
  \item
  \item
  \item
  \item
  \item
  \end{itemize}
\end{itemize}

Players for the Atlanta Dream and other teams across the W.N.B.A. have
begun a public show of defiance by wearing T-shirts endorsing the
Democratic opponent of the Dream's co-owner Senator Kelly Loeffler,
Republican of Georgia, who is in a tightly contested race for her seat
and has spoken disparagingly of the Black Lives Matter movement.

Images of players, including the nine-time All-Star Diana Taurasi,
wearing the shirts endorsing Dr. Raphael G. Warnock flooded social media
on Tuesday ahead of a nationally televised matchup between Atlanta and
the Phoenix Mercury.

Across the chest of the black T-shirts were two words ``Vote Warnock,''
a reference to the Atlanta pastor who is one of the top Democrats
running against Loeffler
\href{https://www.nytimes.com/2020/07/09/us/politics/kelly-loeffler-georgia-senate-arizona.html}{in
a special election in November}.

It was the latest escalation in a conflict that has roiled the W.N.B.A.
in recent weeks. Loeffler, who owns 49 percent of the team, has publicly
and frequently derided the league for dedicating its season
\href{https://www.nytimes.com/2020/07/09/sports/basketball/kelly-loeffler-atlanta-dream-protests.html}{to
the Black Lives Matter movement}, provoking sharp criticism from some of
the league's most high-profile figures. The players' union has
\href{https://twitter.com/TheWNBPA?ref_src=twsrc\%5Egoogle\%7Ctwcamp\%5Eserp\%7Ctwgr\%5Eauthor}{called
for her ouster}, but W.N.B.A. Commissioner Cathy Engelbert
\href{https://www.youtube.com/watch?v=rVaI_-G8Nd0}{told CNN in mid-July}
that Loeffler would not be forced to sell the team.

\includegraphics{https://static01.nyt.com/images/2020/08/04/sports/04wnba-dream-2/merlin_175303587_da12dca8-b08d-4526-b9f5-517e38fc3b07-articleLarge.jpg?quality=75\&auto=webp\&disable=upscale}

Elizabeth Williams, who has played for the Dream since 2016, said in an
interview on Monday that the players plan to ``vocally support'' Warnock
in the coming weeks, and that players have had ``several'' conversations
with him.

``When we realized what our owner was doing and how she was kind of
using us and the Black Lives Matter movement for her political gain, we
felt like we didn't want to feel kind of lost as the pawns in this,''
Williams said.

``This is just more proof that the out of control cancel culture wants
to shut out anyone who disagrees with them,'' Loeffler said in a
statement late Tuesday. ``It's clear that the league is more concerned
with playing politics than basketball, and I stand by what I wrote in
June.''

Warnock, in an email, said he was ``honored and humbled by the
overwhelming support from the W.N.B.A. players.''

``This movement gives us the opportunity to fight for what we believe
in, and I stand by all athletes promoting social justice on and off the
court,'' Warnock said. ``Senator Loeffler and those like her who seek to
silence and dismiss others when they speak up for justice have planted
themselves on the wrong side of history.''

Williams said the idea to publicly endorse Warnock came from Sue Bird,
the 11-time All-Star guard for the Seattle Storm. Both Williams and Bird
are executives in the players' union. Top players also consulted with
Stacey Abrams, the Democrat who lost a close race for the Georgia
governor's seat in 2018 to Brian Kemp, a Republican. Abrams joined
\href{https://wnbpa.com/about/board/}{the players' union board of
advocates} last summer.

In planning the demonstration, the players thought it was important that
the T-shirts debuted during a game that was nationally televised.
Tuesday's game between the Dream and Mercury was scheduled to be
broadcast on ESPN 2. Williams said that the Dream players will continue
to wear the T-shirts in upcoming games, and other teams have agreed to
do so, too.

``We can't really do anything about her ownership,'' Williams said,
referring to Loeffler. ``That's not something we can control. We can
control who we vote for.''

The teams' coaches were made aware of the demonstration, Williams said,
but she was unsure whether Mary Brock, the philanthropist who owns the
other 51 percent of the Dream, was told ahead of time. Brock thus far
has not publicly addressed the conflict between Loeffler and the
W.N.B.A. players.

This level of public protest --- players on a team openly campaigning
against their own owner --- is virtually unheard of in professional
sports. But it is not out of character for players in the W.N.B.A., a
league that has frequently shown a willingness to tackle social justice
issues publicly. In 2016, W.N.B.A. players were among the first
professional athletes in the United States
\href{https://www.nytimes.com/2016/07/11/sports/basketball/liberty-show-solidarity-with-black-lives-matter-in-rare-public-stance.html}{to
demonstrate against police brutality}, also with T-shirts. The W.N.B.A.
initially fined those players before rescinding the fines.

It was not immediately clear whether the league would take action
against the players for Tuesday's demonstration. Representatives for the
league did not immediately return a request for comment.

Recently, the league has waded deeper into politics, in large part
because of its players' willingness to do the same. In 2018, the
W.N.B.A.
\href{https://www.wnba.com/news/wnba-take-a-seat-take-a-stand-women-girls-2018-season/}{partnered
with Planned Parenthood}for an initiative called ``Take a Seat, Take a
Stand,'' which sent a portion of ticket proceeds to multiple groups
including Planned Parenthood. Since then, W.N.B.A. players have often
been spotted at the front lines of demonstrations, especially recently
after the deaths of George Floyd and Breonna Taylor, who were killed by
the police.

The initial response to Loeffler, who is not involved in the day-to-day
operations of the team, came after she wrote a letter in early July to
Engelbert, saying, ``I adamantly oppose the Black Lives Matter political
movement, which has advocated for the defunding of police.'' Loeffler
also accused the movement of promoting ``violence and destruction across
the country.''

One of the most high-profile rebukes came
\href{https://twitter.com/moniquebillings/status/1281713112949850113?s=21}{from
Dream players}, who issued a collective statement on July 10: ``It is
not extreme to demand change after centuries of inequality. This is not
a political statement. This is a statement of humanity.''

Loeffler has since made a cause celebre of criticizing the W.N.B.A.'s
commitment to social justice, frequently conducting interviews painting
herself as a victim of so-called ``cancel culture'' as a result of the
backlash from players.

In her Senate race, Loeffler is fighting to remain in the seat she was
appointed to late last year after Johnny Isakson stepped down because of
health problems. To win, a candidate must get 50 percent of the vote. If
no candidate reaches that mark, the top two candidates will have a
runoff, which is widely expected to be the outcome. The other top
candidate in the race is Representative Doug Collins, a Republican and
ardent supporter of President Trump.

Collins and the W.N.B.A. players have one thing in common: They want
Loeffler to sell her stake in the team. But Collins's reasoning was
different. On July 7, he issued a statement that said Loeffler should
``get out of the liberal agenda advocacy business,'' in reference to the
Dream.

Asked if she was bothered that the W.N.B.A. was not forcing Loeffler to
sell her stake, Williams said: ``Honestly, I think that she wants the
league to push her out. She wants that to be part of this statement that
she's making that, `Oh, Black Lives Matter is divisive. They pushed me
out because they feel differently, blah blah blah.'

``So I think, ironically, playing this kind of wait and see game is more
frustrating for Kelly than it is for us,'' Williams said.

Advertisement

\protect\hyperlink{after-bottom}{Continue reading the main story}

\hypertarget{site-index}{%
\subsection{Site Index}\label{site-index}}

\hypertarget{site-information-navigation}{%
\subsection{Site Information
Navigation}\label{site-information-navigation}}

\begin{itemize}
\tightlist
\item
  \href{https://help.nytimes.com/hc/en-us/articles/115014792127-Copyright-notice}{©~2020~The
  New York Times Company}
\end{itemize}

\begin{itemize}
\tightlist
\item
  \href{https://www.nytco.com/}{NYTCo}
\item
  \href{https://help.nytimes.com/hc/en-us/articles/115015385887-Contact-Us}{Contact
  Us}
\item
  \href{https://www.nytco.com/careers/}{Work with us}
\item
  \href{https://nytmediakit.com/}{Advertise}
\item
  \href{http://www.tbrandstudio.com/}{T Brand Studio}
\item
  \href{https://www.nytimes.com/privacy/cookie-policy\#how-do-i-manage-trackers}{Your
  Ad Choices}
\item
  \href{https://www.nytimes.com/privacy}{Privacy}
\item
  \href{https://help.nytimes.com/hc/en-us/articles/115014893428-Terms-of-service}{Terms
  of Service}
\item
  \href{https://help.nytimes.com/hc/en-us/articles/115014893968-Terms-of-sale}{Terms
  of Sale}
\item
  \href{https://spiderbites.nytimes.com}{Site Map}
\item
  \href{https://help.nytimes.com/hc/en-us}{Help}
\item
  \href{https://www.nytimes.com/subscription?campaignId=37WXW}{Subscriptions}
\end{itemize}
