Sections

SEARCH

\protect\hyperlink{site-content}{Skip to
content}\protect\hyperlink{site-index}{Skip to site index}

\href{https://www.nytimes.com/section/technology}{Technology}

\href{https://myaccount.nytimes.com/auth/login?response_type=cookie\&client_id=vi}{}

\href{https://www.nytimes.com/section/todayspaper}{Today's Paper}

\href{/section/technology}{Technology}\textbar{}Star Technologist Who
Crossed Google Sentenced to 18 Months in Prison

\url{https://nyti.ms/2DlSey3}

\begin{itemize}
\item
\item
\item
\item
\item
\end{itemize}

Advertisement

\protect\hyperlink{after-top}{Continue reading the main story}

Supported by

\protect\hyperlink{after-sponsor}{Continue reading the main story}

\hypertarget{star-technologist-who-crossed-google-sentenced-to-18-months-in-prison}{%
\section{Star Technologist Who Crossed Google Sentenced to 18 Months in
Prison}\label{star-technologist-who-crossed-google-sentenced-to-18-months-in-prison}}

Anthony Levandowski, a onetime star Silicon Valley engineer of
self-driving cars, had pleaded guilty to stealing trade secrets.

\includegraphics{https://static01.nyt.com/images/2020/08/04/business/04levandowski2/merlin_161433504_890e9630-5bd1-47f0-b55c-504c6980e87a-articleLarge.jpg?quality=75\&auto=webp\&disable=upscale}

By \href{https://www.nytimes.com/by/kate-conger}{Kate Conger}

\begin{itemize}
\item
  Aug. 4, 2020
\item
  \begin{itemize}
  \item
  \item
  \item
  \item
  \item
  \end{itemize}
\end{itemize}

OAKLAND, Calif. --- Anthony Levandowski, a
\href{https://www.nytimes.com/2016/05/17/technology/want-to-buy-a-self-driving-car-trucks-may-come-first.html}{pioneer
of self-driving car technology} in Silicon Valley, had once been feted
by companies such as Google and Uber for his engineering expertise.

But on Tuesday, Mr. Levandowski's fall from grace was capped when he was
sentenced to 18 months in prison for
\href{https://www.nytimes.com/2020/03/19/technology/levandowski-uber-google-plea.html}{stealing
self-driving car trade secrets from Google}. He will not be required to
serve his sentence until the coronavirus pandemic subsides, a federal
judge ordered.

Mr. Levandowski, 40, also agreed to pay \$756,499 to Waymo, a
self-driving business spun out of Google, as restitution. He had
\href{https://www.nytimes.com/2020/03/04/technology/anthony-levandowski-google-uber.html}{filed
for bankruptcy} in March, saying he had \$50 million to \$100 million in
personal assets. He will also be required to pay a fine of \$95,000.

``Today marks the end of three and a half long years and the beginning
of another long road ahead,'' Mr. Levandowski said in a statement. ``I'm
thankful to my family and friends for their continued love and support
during this difficult time.''

Over the past few years, Mr. Levandowski had become a Silicon Valley
cautionary tale. He had initially earned millions of dollars for his
work at Google and had a close relationship with Larry Page, a Google
founder, but that changed when Mr. Levandowski left the search giant. As
part of his departure, he took some of Google's self-driving talent with
him to
\href{https://www.nytimes.com/2016/05/17/technology/want-to-buy-a-self-driving-car-trucks-may-come-first.html}{found
Otto, a self-driving truck start-up}.

Mr. Levandowski sold Otto to Uber for more than \$600 million in 2016.
At Uber, Mr. Levandowski was highly valued by the ride-hailing giant's
chief executive at the time, Travis Kalanick, who also wanted to develop
a fleet of self-driving cars.

But in 2017, Waymo --- newly spun out of Google --- sued Uber for theft
of trade secrets and identified Mr. Levandowski for taking years of
autonomous vehicle research to bolster Uber's self-driving program. Uber
and
\href{https://www.nytimes.com/2018/02/09/technology/uber-waymo-lawsuit-driverless.html}{Waymo
eventually settle}d, with Uber handing Waymo roughly \$245 million in
Uber stock. Uber also agreed not to infringe upon Waymo's intellectual
property.

The settlement between the companies did not mean Mr. Levandowski's
troubles were over.

Last August, federal prosecutors
\href{https://www.nytimes.com/2019/08/27/technology/google-trade-secrets-levandowski.html}{charged
him with 33 counts of theft and attempted theft} of trade secrets from
Google. Before resigning from his job at the search giant, Mr.
Levandowski had downloaded thousands of files related to the company's
development of self-driving cars, the Justice Department said.

In March, Mr. Levandowski pleaded guilty to one count of trade secret
theft in an agreement with federal prosecutors to drop the remaining
charges, according to a court filing at the time. The plea carried a
maximum sentence of 10 years in prison and a maximum fine of \$250,000.

In a legal filing last week, lawyers for Mr. Levandowski said he was
willing to take on community service by educating other tech company
employees and encouraging them not to take files from their workplaces.

``He proposes to offer himself as an object lesson in `what not to do,'
by candidly sharing the story of his misdeeds,'' Mr. Levandowski's
lawyers wrote in the filing. ``His message is clear: taking a trade
secret to the next venture is a `life-altering terrible decision,' never
worth it.''

But Waymo argued that Mr. Levandowski had still not taken responsibility
for his actions and had not returned the stolen files. In a victim
statement, the company asked that Mr. Levandowski face a ``substantial
period of incarceration.''

``His misconduct was enormously disruptive and harmful to Waymo,
constituted a betrayal, and the financial effects would likely have been
even more severe had it gone undetected,'' wrote Leo Cunningham, a
lawyer representing Waymo.

Ismail Ramsey and Miles Ehrlich, lawyers for Mr. Levandowski, thanked
the judge in the case for allowing Mr. Levandowski to stay out of prison
during the pandemic.

``Anthony deeply regrets his past decisions and, while we are saddened
that he will to have to spend time in prison, Anthony remains committed
to his life's mission of building innovative technologies to improve
people's lives,'' they said in a statement.

Advertisement

\protect\hyperlink{after-bottom}{Continue reading the main story}

\hypertarget{site-index}{%
\subsection{Site Index}\label{site-index}}

\hypertarget{site-information-navigation}{%
\subsection{Site Information
Navigation}\label{site-information-navigation}}

\begin{itemize}
\tightlist
\item
  \href{https://help.nytimes.com/hc/en-us/articles/115014792127-Copyright-notice}{©~2020~The
  New York Times Company}
\end{itemize}

\begin{itemize}
\tightlist
\item
  \href{https://www.nytco.com/}{NYTCo}
\item
  \href{https://help.nytimes.com/hc/en-us/articles/115015385887-Contact-Us}{Contact
  Us}
\item
  \href{https://www.nytco.com/careers/}{Work with us}
\item
  \href{https://nytmediakit.com/}{Advertise}
\item
  \href{http://www.tbrandstudio.com/}{T Brand Studio}
\item
  \href{https://www.nytimes.com/privacy/cookie-policy\#how-do-i-manage-trackers}{Your
  Ad Choices}
\item
  \href{https://www.nytimes.com/privacy}{Privacy}
\item
  \href{https://help.nytimes.com/hc/en-us/articles/115014893428-Terms-of-service}{Terms
  of Service}
\item
  \href{https://help.nytimes.com/hc/en-us/articles/115014893968-Terms-of-sale}{Terms
  of Sale}
\item
  \href{https://spiderbites.nytimes.com}{Site Map}
\item
  \href{https://help.nytimes.com/hc/en-us}{Help}
\item
  \href{https://www.nytimes.com/subscription?campaignId=37WXW}{Subscriptions}
\end{itemize}
