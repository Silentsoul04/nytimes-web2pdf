Sections

SEARCH

\protect\hyperlink{site-content}{Skip to
content}\protect\hyperlink{site-index}{Skip to site index}

\href{https://www.nytimes.com/news-event/2020-election}{Elections}

\href{https://myaccount.nytimes.com/auth/login?response_type=cookie\&client_id=vi}{}

\href{https://www.nytimes.com/section/todayspaper}{Today's Paper}

\href{/news-event/2020-election}{Elections}\textbar{}Live 2020 Election
Updates: Kris Kobach and Rashida Tlaib Are on the Primary Ballot

\url{https://nyti.ms/2DgrAXu}

\begin{itemize}
\item
\item
\item
\item
\item
\item
\end{itemize}

\begin{itemize}
\item
  \href{https://www.nytimes.com/2020/08/04/us/elections/primary-election-michigan-arizona-kansas.html?action=click\&pgtype=Article\&state=default\&region=TOP_BANNER\&context=storylines_menu}{Election
  Updates}
\item
  \href{https://www.nytimes.com/article/biden-vice-president-2020.html?action=click\&pgtype=Article\&state=default\&region=TOP_BANNER\&context=storylines_menu}{Biden's
  V.P. Search}
\item
  \href{https://www.nytimes.com/interactive/2020/07/24/us/politics/trump-biden-campaign-donors.html?action=click\&pgtype=Article\&state=default\&region=TOP_BANNER\&context=storylines_menu}{Map
  of Donations}
\item
  \href{https://www.nytimes.com/interactive/2020/us/elections/delegate-count-primary-results.html?action=click\&pgtype=Article\&state=default\&region=TOP_BANNER\&context=storylines_menu}{Delegate
  Count}
\item
  \href{https://www.nytimes.com/interactive/2019/us/politics/2020-presidential-candidates.html?action=click\&pgtype=Article\&state=default\&region=TOP_BANNER\&context=storylines_menu}{The
  Candidates}
\item
  \href{https://www.nytimes.com/newsletters/politics?action=click\&pgtype=Article\&state=default\&region=TOP_BANNER\&context=storylines_menu}{Politics
  Newsletter}
\end{itemize}

Advertisement

\protect\hyperlink{after-top}{Continue reading the main story}

Supported by

\protect\hyperlink{after-sponsor}{Continue reading the main story}

LIVE UPDATES

Updated~

Aug. 4, 2020, 2:55 p.m. ET

Aug. 4, 2020, 2:55 p.m. ET

\hypertarget{live-2020-election-updates-kris-kobach-and-rashida-tlaib-are-on-the-primary-ballot}{%
\section{Live 2020 Election Updates: Kris Kobach and Rashida Tlaib Are
on the Primary
Ballot}\label{live-2020-election-updates-kris-kobach-and-rashida-tlaib-are-on-the-primary-ballot}}

President Trump downplayed the legacy of John Lewis in an interview. In
St. Louis, Cori Bush, a progressive activist, will try again to unseat
William Lacy Clay, a 20-year incumbent.

Right Now

After weeks of disparaging voting by mail, President Trump urged voters
to request mail ballots in Florida, where some polls show him trailing.

\hypertarget{heres-what-you-need-to-know}{%
\subsubsection{Here's what you need to
know:}\label{heres-what-you-need-to-know}}

\begin{itemize}
\tightlist
\item
  \protect\hyperlink{link-3924dd44}{Two G.O.P. Senate primaries offer
  --- what else? --- a test of loyalty to Trump.}
\item
  \protect\hyperlink{link-32b39e33}{President Trump is suddenly a big
  supporter of mail-in voting --- in Florida.}
\item
  \protect\hyperlink{link-6d019753}{Election experts warn Congress about
  widespread disenfranchisement of voters of color in November.}
\item
  \protect\hyperlink{link-12734b84}{Trump tried to praise Yosemite
  National Park. It didn't go well.}
\item
  \protect\hyperlink{link-318dd837}{Texas' lieutenant governor
  correlates a spike in gun sales with a Trump victory.}
\item
  \protect\hyperlink{link-6d917025}{Rashida Tlaib, a fierce Trump
  critic, faces a stiff Democratic challenge in Michigan.}
\item
  \protect\hyperlink{link-7c97d7ae}{Why Kris Kobach's run in Kansas has
  Republicans nervous.}
\end{itemize}

\includegraphics{https://static01.nyt.com/images/2020/08/04/us/04elections-briefing-primary2/merlin_175293699_a2f1f7bf-7a06-466b-86c0-94971f14b424-articleLarge.jpg?quality=75\&auto=webp\&disable=upscale}

\subsection{}

Two G.O.P. Senate primaries offer --- what else? --- a test of loyalty
to Trump.

There are, as
\href{https://www.nytimes.com/2004/07/27/politics/campaign/senator-john-edwardss-remarks-to-the-democratic-national.html}{a
once promising Democrat} memorably said, two Americas.

In one,
\href{https://www.nytimes.com/interactive/2020/us/elections/donald-trump.html}{President
Trump} is sagging in the polls, his ineffectual handling of the
coronavirus and incendiary response to racial justice protests having
alienated a large segment of the country.

Yet in the other America, the one where Republican primary voters will
go to the polls this week in Kansas and Tennessee, Mr. Trump is not
toxic at all. In fact, his endorsement amounts to a political seal of
good housekeeping.

The Senate primaries in those states for seats held by two retiring
Republicans have, like G.O.P. contests elsewhere, evolved into tests of
who's most supportive of Mr. Trump and critical of his intraparty
opponents. Never mind that the president could be on his way out of
office when the would-be senators are sworn in next year.

In Tennessee, which holds its primary on Thursday, Bill Hagerty, a
private equity executive who served as finance chairman for Mitt
Romney's 2012 campaign, is running on his endorsement from the
president. Mr. Hagerty helped exorcise his ties to Mr. Romney by
supporting Mr. Trump in the 2016 general election and serving as his
ambassador to Japan.

His upstart challenger, Manny Sethi, an orthopedist who teaches at
Vanderbilt University, is trying to upset Mr. Hagerty by pledging his
loyalty to Mr. Trump and relentlessly highlighting Mr. Hagerty's
relationship with Mr. Romney, now a first-term Utah senator. Mr. Romney,
the only Senate Republican
\href{https://www.nytimes.com/2020/02/05/us/politics/romney-trump-impeachment.html}{who
voted to remove Mr. Trump from office during impeachment}, is now
something of a dirty word in Republican primaries.

The president has not endorsed a candidate in Kansas, where
establishment-aligned Republicans are
\href{https://www.nytimes.com/2020/08/03/us/politics/kris-kobach-kansas-senate-primary.html}{petrified
that the polarizing former secretary of state Kris Kobach} may claim the
nomination and imperil an otherwise safe Republican Senate seat.

But Mr. Kobach and two of his main rivals --- Representative Roger
Marshall and the plumbing executive Bob Hamilton --- are all embracing
Mr. Trump in their advertising and vowing to support his policies.

Senator Mitch McConnell, the majority leader, and a few other
Republicans in Washington had urged Mr. Trump to throw his support to
Mr. Marshall. But that was before Senator Ted Cruz used an Air Force One
flight last week to remind the president that Mr. Marshall supported
former Gov. John Kasich of Ohio in the 2016 presidential primary.

Mr. Kasich is one of the most vocal Trump critics in the G.O.P. In fact,
going further than Mr. Romney, he's expected to speak at the Democratic
National Convention and endorse
\href{https://www.nytimes.com/interactive/2020/us/elections/joe-biden.html}{Joseph
R. Biden Jr.} this month.

Elsewhere on Tuesday, Representative Rashida Tlaib is
\href{https://www.nytimes.com/2020/08/03/us/politics/michigan-primary-rashida-tlaib-brenda-jones.html}{facing
a tough Democratic primary} in Michigan, a bitter House battle in St.
Louis is unfolding between a Justice Democrats-backed upstart and a
longtime Democratic incumbent, and Joe Arpaio, the 88-year-old
immigration hard-liner and former sheriff of Maricopa County,
\href{https://www.nytimes.com/2020/08/02/us/politics/arizona-election-joe-arpaio.html}{is
asking Arizona voters to return him to that office}.

\hypertarget{-1}{%
\subsection{}\label{-1}}

President Trump is suddenly a big supporter of mail-in voting --- in
Florida.

President Trump, who only a day ago
\href{https://www.nytimes.com/2020/08/03/us/politics/trump-mail-in-voting.html}{suggested
he could restrict voting by mail} through executive fiat, reversed
course on Tuesday, urging his supporters in Florida to request mail-in
ballots as
\href{https://www.cnn.com/2020/07/26/politics/trump-trails-in-must-win-florida/index.html}{some
polls show him trailing} in the
\href{https://www.nytimes.com/2020/08/02/us/florida-hurricane-isaias-coronavirus.html}{pandemic-racked
battleground state.}

``Whether you call it Vote by Mail or Absentee Voting, in Florida the
election system is Safe and Secure, Tried and True,''
\href{https://twitter.com/realDonaldTrump/status/1290692768675901440}{Mr.
Trump wrote on Twitter.} ``Florida's Voting system has been cleaned up
(we defeated Democrats attempts at change), so in Florida I encourage
all to request a Ballot \& Vote by Mail!''

Mr. Trump's caustic public denunciation of the mail-in voting system has
chafed many Republicans who see his statements as likely to cost him
support among older voters, many of whom may rely on mail-in ballots to
avoid going to polling stations during the pandemic. Senator Rick Scott,
Republican of Florida, and other party leaders, have said mail-in voting
can work as long
\href{https://thehill.com/homenews/senate/499352-gop-senator-you-can-do-mail-in-voting-with-laws-in-place-to-limit-fraud}{as
anti-fraud measures are in place}.

Campaign officials have recently warned the president that efforts to
block mail-in voting in Florida, with its large population of older
voters, could have especially dire consequences, according to two
Republicans familiar with the situation.

Kayleigh McEnany, the White House press secretary, said Wednesday that
the president's tweet about Florida mail-in voting was meant to
encourage voters to file ``absentee ballots,'' not to expand the
practice more broadly; Democrats have said that is a distinction without
a difference.

Mr. Trump was asked on Monday if he has the authority to draft an
executive order to prevent states from widening the use of alternative
voting methods despite the fact that the Constitution vests control of
voting procedures with the states.

``I have the right to do it,'' Mr. Trump told reporters. ``We haven't
got there yet, but we'll see what happens.''

For months, the president has been
\href{https://www.nytimes.com/article/mail-in-voting-explained.html}{claiming
without evidence} that voting by mail will encourage widespread voter
fraud, even as Democrats and some Republicans press to loosen
restrictions in order to protect voters from the coronavirus.

Mr. Trump has, increasingly, taken aim at officials in Nevada, which
enacted universal mail-in balloting this week. He threatened to sue to
block the measure on Monday.

Gov. Steve Sisolak, a Democrat, has said fraud concerns are unfounded
and failing to expand voting would disenfranchise thousands of Nevada
residents, many of them older and low-income voters.

\hypertarget{-2}{%
\subsection{}\label{-2}}

Election experts warn Congress about widespread disenfranchisement of
voters of color in November.

Image

Voters lined up to cast their ballots in Union City, Ga., in June. The
Georgia primary was plagued with long lines, with some voters waiting
hours to cast their ballots.Credit...Dustin Chambers/Reuters

Election integrity experts told Congress on Tuesday that without an
immediate, substantial infusion of federal funds to help administer
November's general election, many voters --- particularly Black and
other minority voters --- could be disenfranchised.

``With less than three months until the November election, Congress must
act now so states have enough time to make the necessary changes and
plans, recruit and train workers, buy equipment, and do outreach to the
public about new voting processes,'' Sylvia Albert, the director of
voting and elections at Common Cause, said in written testimony to a
subcommittee of the House Committee on Homeland Security.

Primary elections around the country this year --- plagued with long
lines, polling place closures and high rates of ballot rejection ---
have exposed ``significant barriers to voting for certain individuals,
especially Black and brown voters,'' Ms. Albert said in her written
testimony.

\hypertarget{latest-updates-2020-election}{%
\section{\texorpdfstring{\href{https://www.nytimes.com/2020/08/04/us/elections/primary-election-michigan-arizona-kansas.html?action=click\&pgtype=Article\&state=default\&region=MAIN_CONTENT_1\&context=storylines_live_updates}{Latest
Updates: 2020
Election}}{Latest Updates: 2020 Election}}\label{latest-updates-2020-election}}

Updated 2020-08-04T18:55:19.561Z

\begin{itemize}
\tightlist
\item
  \href{https://www.nytimes.com/2020/08/04/us/elections/primary-election-michigan-arizona-kansas.html?action=click\&pgtype=Article\&state=default\&region=MAIN_CONTENT_1\&context=storylines_live_updates\#link-3924dd44}{Two
  G.O.P. Senate primaries offer --- what else? --- a test of loyalty to
  Trump.}
\item
  \href{https://www.nytimes.com/2020/08/04/us/elections/primary-election-michigan-arizona-kansas.html?action=click\&pgtype=Article\&state=default\&region=MAIN_CONTENT_1\&context=storylines_live_updates\#link-32b39e33}{President
  Trump is suddenly a big supporter of mail-in voting --- in Florida.}
\item
  \href{https://www.nytimes.com/2020/08/04/us/elections/primary-election-michigan-arizona-kansas.html?action=click\&pgtype=Article\&state=default\&region=MAIN_CONTENT_1\&context=storylines_live_updates\#link-6d019753}{Election
  experts warn Congress about widespread disenfranchisement of voters of
  color in November.}
\end{itemize}

\href{https://www.nytimes.com/2020/08/04/us/elections/primary-election-michigan-arizona-kansas.html?action=click\&pgtype=Article\&state=default\&region=MAIN_CONTENT_1\&context=storylines_live_updates}{See
more updates}

``Longstanding disparities, including long lines, the ballot rejection
rates, particularly of Black and brown communities, are now exacerbated
by the Covid-19 pandemic,'' she said Tuesday. ``The chasm of those with
access is growing larger. Voters of color are on the losing end.''

Several states, including Georgia, Pennsylvania and Indiana, saw voting
machine glitches and other failures that contributed to long waits.

``Without proper funding, the problems seen in previous elections are
going to be just the tip of the iceberg this November,'' she testified.

David Levine, an elections integrity fellow with the Alliance for
Securing Democracy, told the panel that many states and counties lacked
the resources needed to offer alternatives for safe and secure elections
amid the pandemic, like ``robust voting by mail, early voting, and
Election Day options.''

The stimulus law enacted in March provided \$400 million to states for
administering elections, but one study said the need was 10 times that.
In May, the Democratic-controlled House passed another pandemic relief
bill that would provide \$3.6 billion in additional election funding,
but that plan has run into a brick wall in the Senate, where there is
little appetite among Republicans for such spending.

The dispute is one of the issues fueling a stalemate between the White
House and congressional Democrats on a sweeping economic recovery
package. Mr. Trump has cast doubt on the idea of expanding mail-in
voting to make it safer for Americans to cast ballots during the
pandemic, saying that it would lead to widespread fraud. But there is no
evidence that the practice leads to higher incidence of voter fraud.

\hypertarget{-3}{%
\subsection{}\label{-3}}

Trump tried to praise Yosemite National Park. It didn't go well.

\includegraphics{https://static01.nyt.com/images/2020/08/04/us/04elections-briefing-yosemite/04elections-briefing-yosemite-videoSixteenByNine3000.jpg}

A routine bill-signing turned into an embarrassing blooper for Mr. Trump
on Tuesday.

At a signing of the Great American Outdoors Act, the president appeared
not to recognize the word ``Yosemite'' in his prepared text, pronouncing
it instead ``Yo Semites.''

The latest verbal miscue by Mr. Trump came as he and his campaign have
repeatedly seized on verbal gaffes by his Democratic opponent, former
Vice President Joseph R. Biden Jr., and sought to portray Mr. Biden as
senile.

And Yosemite National Park in California, which Mr. Trump was trying to
reference, is a federal treasure that presidents have often highlighted.

Before he signed the act into law, Mr. Trump talked about young
Americans looking at ``the breathtaking beauty of the Grand Canyon,''
before moving on in his script.

``When they gaze upon Yo Semites --- Yo Seminites --- towering sequoias,
their love of country grows stronger,'' Mr. Trump said.

The original sponsor of the bill was Representative John Lewis, the late
civil rights icon from Georgia,
\href{https://www.nytimes.com/2020/08/04/us/politics/trump-john-lewis-axios.html}{of
whom Mr. Trump was dismissive} in an interview with ``Axios on HBO.''

\hypertarget{-4}{%
\subsection{}\label{-4}}

Texas' lieutenant governor correlates a spike in gun sales with a Trump
victory.

Image

Lt. Gov. Dan Patrick of Texas has been an unwavering ally of the
president.Credit...Nick Wagner/Austin American-Statesman, via Associated
Press

Gun sales skyrocketed after the police killing of George Floyd in May,
and the lieutenant governor of Texas pointing to the increase as a sign
of a victory for Mr. Trump in the November election.

Lt. Gov. Dan Patrick, who is the chairman of Mr. Trump's re-election
campaign in Texas, on Tuesday cited gun sales as a measure of the
president's political fortunes.

``Guns sales are a leading indicator pointing to Trump's electoral
success in November,'' Mr. Patrick said
\href{https://www.scribd.com/document/471381594/PR-20-08-04}{in a
statement}. ``The millions of Americans who are buying guns will not
vote for a candidate who threatens their Second Amendment rights and
promises confiscation policies.''

In Texas, where
\href{https://projects.fivethirtyeight.com/polls/texas/}{polls show a
tight race} between Mr. Trump and Mr. Biden, Mr. Patrick has been an
unwavering ally of the president. Last month, Mr. Patrick drew attention
when he said during a
\href{https://video.foxnews.com/v/6168445167001\#sp=show-clips}{Fox News
interview} that Dr. Anthony S. Fauci, the nation's top infectious
disease expert, ``doesn't know what he's talking about.'' His comments
came as Texas became a hot spot for coronavirus cases.

Mr. Patrick's comments on gun ownership came one day after Texas and the
nation paused to remember the 23 people who were killed in an
\href{https://www.nytimes.com/2019/08/09/us/el-paso-suspect-confession.html}{Aug.
3, 2019, massacre at a Walmart in El Paso}. The suspect confessed that
he had been targeting Mexicans, the authorities said.

Citing F.B.I. statistics, Mr. Patrick said that more than 2.3 million
guns were sold in June --- a more than 145 percent increase compared to
last year --- and that a record 3.9 million background checks were
conducted.

Much the way Mr. Trump and the president's leading surrogates have done,
Mr. Patrick appeared to be trying to
\href{https://www.nytimes.com/2020/07/30/upshot/trump-suburban-voters.html}{stoke
fears among suburban voters and Mr. Trump's conservative base over the
unrest} in many of the country's cities since Mr. Floyd's death in late
May.

``Democrat-led cities across the nation, including Portland, Seattle and
now Austin, are surrendering the streets to lawlessness and mob rule,''
he said. ``With their `defund the police' rhetoric escalating, they have
sent a clear message that protecting and defending their residents is a
low priority.''

\hypertarget{-5}{%
\subsection{}\label{-5}}

Rashida Tlaib, a fierce Trump critic, faces a stiff Democratic challenge
in Michigan.

Image

Representative Rashida Tlaib, seen at an event in Highland Park, Mich.,
faces a primary challenge from Brenda Jones, the Detroit City Council
president whom she defeated two years ago.Credit...Sylvia Jarrus for The
New York Times

Representative Rashida Tlaib marked her first day in Congress in January
2019 with
\href{https://www.nytimes.com/2019/01/04/us/politics/tlaib-impeach-trump.html}{an
expletive-fueled call to impeach Mr. Trump}. It made her an instant
Democratic star as she became part of the four-member Squad, the group
of progressive Democratic women who were elected to the House in 2018
and have come to embody the vanguard of the party's liberal grass-roots
energy.

While popular with the Democratic base, Ms. Tlaib's broadside didn't win
her many friends in the party's House leadership, which held off
impeachment proceedings for months before allowing them to begin last
fall.

It also fueled whispers back home in Detroit that Ms. Tlaib, a
Palestinian-American who was one of the first two Muslim women elected
to Congress, was more eager to advance a national profile and
fund-raising network than she was in representing her predominantly
Black district.

Now the most endangered member of the group, Ms. Tlaib faces a rematch
with Brenda Jones, the Detroit City Council president she defeated in
the 2018 primary. (In a separate contest held the same day, Ms. Jones
eked out a two-point victory over Ms. Tlaib in a special primary
election to serve out the remainder of the long-serving Representative
John Conyers's term after
\href{https://www.nytimes.com/2017/12/05/us/politics/john-conyers-election.html}{his
abrupt resignation}.)

Ms. Jones has now regrouped to challenge Ms. Tlaib, arguing that she has
become too preoccupied with national issues.

Ms. Tlaib has raised far more money, \$3 million, than Ms. Jones, who
posted just \$165,000 in her latest Federal Election Commission report.

And while she has emerged as one of the Democratic Party's most muscular
disrupters, Ms. Tlaib retained the endorsement last week of its most
significant establishment figure: Speaker Nancy Pelosi.

More than 1.4 million Michigan voters already cast absentee ballots and
fewer people were expected to vote in person at polling places. Election
officials were expecting a late night and perhaps delayed results
because of the surge in those ballots, and the fact that local clerks
could not begin processing or tabulating them until the polls opened
this morning.

Although Michigan sent 35 extra elections workers to Detroit on Tuesday,
a few polling places opened late after election workers did not show up
for shifts.

``This is in direct relation to the pandemic and in direct relation to
the high number of absentee ballots,'' said Jake Rollow, a spokesman for
Secretary of State Jocelyn Benson. ``And we've got fewer election
workers able to serve because traditionally so many election workers are
older and more susceptible to the virus.''

\hypertarget{-6}{%
\subsection{}\label{-6}}

Why Kris Kobach's run in Kansas has Republicans nervous.

Image

Kris Kobach, left, a Republican candidate for Senate in Kansas, is an
incendiary figure known for hard-line views on immigration and voting
rights.Credit...John Hanna/Associated Press

Kansas has not sent a Democrat to the U.S. Senate since the 1930s. But
for months, Republicans in Washington and in the state
\href{https://www.nytimes.com/2020/05/30/us/politics/kansas-senate-kobach.html}{have
feared} that if Mr. Kobach wins the party's Senate primary, a
traditionally safe seat will be endangered --- and so will the Senate
majority.

Mr. Kobach, the former Kansas secretary of state, is an incendiary
figure in state politics, known for his hard-line views on immigration
and voting rights, and his 2018 loss in the governor's race to Laura
Kelly, a Democrat.

Top Senate Republicans, long wary of Mr. Kobach and his 2018
performance, have pleaded with Mr. Trump to endorse Representative Roger
Marshall, whom they see as the strongest general election candidate in a
crowded field. But the president has remained on the sidelines,
\href{https://www.nytimes.com/2020/07/30/us/politics/kansas-senate-kobach-trump.html}{stoking
tensions} between Senate Republicans and the White House.

The winner of Tuesday's primary is expected to face Barbara Bollier, a
Democratic state senator who was until recently a Republican herself.

A statewide race remains a challenge in Kansas for any Democrat,
regardless of the Republican nominee. But as Mr. Trump's faltering
approval ratings have endangered Republican candidates in down-ballot
races across the country, there is a growing sense that the outcome even
in deep-red Kansas is no sure bet.

The results on Tuesday may help determine just how competitive the state
is come November.

\hypertarget{-7}{%
\subsection{}\label{-7}}

In St. Louis, a progressive challenger will try again to unseat a
longtime congressman.

Image

Cori Bush, a Democratic congressional candidate and progressive
activist, cast her ballot in St Louis on Tuesday.Credit...Michael B.
Thomas/Getty Images

An increasingly bitter Democratic primary in St. Louis
\href{https://www.nytimes.com/2020/08/02/us/politics/cori-bush-william-lacy-clay-missouri.html}{between
the activist Cori Bush and Representative William Lacy Clay}, a 20-year
incumbent with the party establishment's full backing, will be one of
the most significant tests this summer of the power of the resurgent
progressive wing of the party.

Mr. Clay and his father, a founding member of the Congressional Black
Caucus, have held the seat in Congress for more than 50 years, and the
congressman had routinely sailed to re-election until Ms. Bush
challenged him in 2018.

He
\href{https://www.nytimes.com/elections/results/missouri-house-district-1-primary-election}{prevailed
then by about 20 points}, and this time around, Mr. Clay wants to make a
show of blunting the progressive movement against primary incumbents
like himself.

If Mr. Clay loses, he would be the first Black congressman to fall to a
challenger backed by the Justice Democrats, a progressive national group
that helped fuel the rise of Representative Alexandria Ocasio-Cortez of
New York. So far, the long-serving House members, like Eliot L. Engel
and Joseph Crowley in New York, who Justice Democrats-backed candidates
have succeeded in upsetting have all been white, in many cases
representing racially diverse districts.

Ms. Bush, an activist who jumped into the political arena after the
police shooting of Michael Brown six years ago, responded in a recent
interview with The Times to Mr. Clay calling her a ``prop'' for the
Justice Democrats.

``I had no title, no name, came out of the Ferguson uprising and people
know who I am across the world,'' Ms. Bush said. ``Not because I took
money from some group --- none of that. It is because I stayed true to a
message of change for real people.''

\hypertarget{-8}{%
\subsection{}\label{-8}}

Joe Arpaio is testing the patience of Arizona Republicans.

Image

Joe Arpaio is trying to win back his old job as the sheriff in Maricopa
County, Ariz.Credit...Adriana Zehbrauskas for The New York Times

Four years and two electoral defeats since he last held office, Joe
Arpaio is asking Republicans in Maricopa County, Ariz.,
\href{https://www.nytimes.com/2020/08/02/us/politics/arizona-election-joe-arpaio.html}{to
return him to his former role as sheriff of the state's largest
jurisdiction}.

But Mr. Arpaio, 88, is no longer the towering local figure who had
forced inmates to wear pink underwear, castigated illegal immigration
and who, well after former President Barack Obama left office, continued
insisting the 44th president wasn't born in Hawaii.

Mr. Arpaio lost a 2016 re-election bid to Paul Penzone, a Democrat, then
finished a distant third in the state's 2018 Senate primary, winning
just one of the state's 15 counties. His comeback bid, like his Senate
race, is fueled more by his name recognition and repeated attempts to
tie himself to Mr. Trump than it is by any sense of how he would run the
sheriff's office.

Now, Mr. Arpaio is involved in a three-way race that includes Jerry
Sheridan, his former chief deputy. But it is Mr. Penzone, the Democrat,
who has the support of the state's Republican establishment figures, who
find Mr. Arpaio's antics generally embarrassing and hurtful to Arizona's
business climate.

What Republican voters in Phoenix and its suburbs will decide Tuesday is
whether the best way to help Mr. Trump carry a key battleground state is
by putting one of his most enthusiastic supporters on the ballot --- or
if the party's fortunes can be improved without Mr. Arpaio, a candidate
guaranteed to mobilize the state's ascendant Latino population.

\hypertarget{-9}{%
\subsection{}\label{-9}}

Trump downplays John Lewis's accomplishments and nurses a grudge: `He
didn't come to my inauguration.'

Image

Representative John Lewis stands on the Edmund Pettus Bridge in Selma,
Ala., in 2018. Activists are pushing to rename the bridge after Mr.
Lewis, a civil rights icon.Credit...Albert Cesare/The Montgomery
Advertiser, via Associated Press

\href{https://www.nytimes.com/2020/08/04/us/politics/trump-john-lewis-axios.html}{Mr.
Trump played down the accomplishments of Representative John Lewis}, the
recently deceased civil rights icon, and criticized him for not
attending the Trump inauguration in January 2017.

The comments from Mr. Trump, in an interview with
\href{https://www.axios.com/trump-john-lewis-inauguration-1adc0747-51b8-4990-a7d8-29290e990dc5.html}{``Axios
on HBO''} that aired Monday night, were unsurprising, given his penchant
for grievance. But they were nonetheless stunning for the degree to
which Mr. Trump refused to view Mr. Lewis's life and legacy in terms
beyond how it related to Mr. Trump himself.

``I never met John Lewis, actually,'' Mr. Trump said. ``He didn't come
to my inauguration. He didn't come to my State of the Union speeches,
and that's OK. That's his right.''

Asked to reflect on Mr. Lewis's contributions to the civil rights
movement, Mr. Trump instead talked up his own record.

``Again, nobody has done more for Black Americans than I have,'' he
said. ``He should have come. I think he made a big mistake.''

Mr. Trump declined to say whether he found Mr. Lewis's life story
``impressive.'' He seemed indifferent to renaming the Edmund Pettus
Bridge in Selma, Ala., after the congressman. The bridge, named after a
former Confederate general, Grand Dragon in the K.K.K. and senator, was
the site of a turning point in the civil rights movement that became
known as Bloody Sunday.

On that day, March 7, 1965, Mr. Lewis suffered a cracked skull during a
march across the bridge when a state trooper clubbed him and beat him to
the ground. The moment was a defining one in his life and in the civil
rights movement. Mr. Trump, in the Axios interview, suggested there
``were many others also'' whose work should be praised.

\hypertarget{-10}{%
\subsection{}\label{-10}}

The census will end early, a move many say is political.

Image

A census outreach event in Dallas in June. The Census Bureau said Monday
it would end its count by Sept. 30, a month earlier than
planned.Credit...Tony Gutierrez/Associated Press

Abruptly reversing its stated schedule, the Census Bureau confirmed late
Monday that it would end its count of the nation's 330 million residents
by Sept. 30, a month earlier than it had stated only this spring.

The four-week acceleration sounds small, but census experts have said it
would wreak havoc with efforts to reach the very hardest-to-count
households --- immigrants, minorities, young people and others --- that
have long been flagged as most likely to be missed in this year's tally.

Critics of the sped-up schedule pounced on the announcement, casting it
as an unvarnished attempt by the administration to twist the nation's
population count to exclude groups that, by and large, tended to support
Democrats.

``This is a whole systemic attack on the census for political gain,''
Julie Menin, the census director for New York City, said in an
interview. ``There's an intentional attempt here to basically steal the
census --- to politicize this census to gain Republican seats across the
country.''

The bureau has offered no explanation for the change
\href{https://www.census.gov/newsroom/press-releases/2020/delivering-complete-accurate-count.html}{posted
on its website}. But outside experts said the explanation was clearly
rooted in politics --- in particular, in a demand by Mr. Trump last
month to exclude undocumented immigrants from the population totals that
are used every 10 years to reallocate House seats among the states.

The House Committee on Oversight and Reform said on Tuesday that it
wanted to question eight census officials about the change.
Representative Carolyn B. Maloney, Democrat of New York and the
committee's chairwoman, called the development
\href{https://oversight.house.gov/sites/democrats.oversight.house.gov/files/2020-08-04.CBM\%20to\%20Dillingham\%20re\%20Transcribed\%20Interviews.pdf}{``alarming''
in a letter to the census director} and said it was a further attempt to
``rush and politicize'' the census.

Slammed by the pandemic, the Census Bureau had said earlier that it
wanted to delay its final delivery of population totals to April 2021,
rather than the statutory deadline of December 31. The speedup announced
late Monday reverses that request and assures that the totals will be
delivered to the White House by year's end --- before any new president
or Congress might take office.

That gives the White House its best opportunity to act on Mr. Trump's
effort to remove undocumented immigrants from the reapportionment
totals.

The announcement on Monday by the Census Bureau speeds up the last
counts of some 60 million households that have failed to respond to
requests to turn in census forms. The pandemic-delayed schedule called
for that count to be completed by October 31. The plan announced on
Monday,
\href{https://www.nytimes.com/2020/07/28/us/trump-census.html}{which had
been reported last week}, will move that deadline up by one month, to
September 30.

Reporting was contributed by Luke Broadwater, Reid J. Epstein, Nicholas
Fandos, Manny Fernandez, Katie Glueck, Kathleen Gray, Maggie Haberman,
Annie Karni, Jonathan Martin, Jesse McKinley, Giovanni Russonello, Neil
Vigdor and Michael Wines.

\hypertarget{our-2020-election-guide}{%
\section{Our 2020 Election Guide}\label{our-2020-election-guide}}

Updated Aug. 4, 2020

\begin{itemize}
\item
  \begin{center}\rule{0.5\linewidth}{\linethickness}\end{center}

  \hypertarget{the-latest}{%
  \subsection{The Latest}\label{the-latest}}

  \begin{itemize}
  \tightlist
  \item
    Five states are holding primary elections Tuesday, with voters in
    Arizona, Kansas, Michigan, Missouri and Washington State choosing
    nominees for Congress and local offices.
    \href{https://www.nytimes.com/2020/08/04/us/elections/primary-election-michigan-arizona-kansas.html?action=click\&pgtype=Article\&state=default\&region=BELOW_MAIN_CONTENT\&context=storylines_guide}{Follow
    live election updates here.}
  \end{itemize}
\item
  \begin{center}\rule{0.5\linewidth}{\linethickness}\end{center}

  \hypertarget{bidens-vp-search}{%
  \subsection{Biden's V.P. Search}\label{bidens-vp-search}}

  \begin{itemize}
  \tightlist
  \item
    \href{https://www.nytimes.com/article/biden-vice-president-2020.html?action=click\&pgtype=Article\&state=default\&region=BELOW_MAIN_CONTENT\&context=storylines_guide}{Here
    are 13 women} who have been under consideration to be Joe Biden's
    running mate, and why each might be chosen --- and might not be.
  \end{itemize}
\item
  \begin{center}\rule{0.5\linewidth}{\linethickness}\end{center}

  \hypertarget{keep-up-with-our-coverage}{%
  \subsection{Keep Up With Our
  Coverage}\label{keep-up-with-our-coverage}}

  \begin{itemize}
  \tightlist
  \item
    Get an
    \href{https://www.nytimes.com/newsletters/politics?action=click\&pgtype=Article\&state=default\&region=BELOW_MAIN_CONTENT\&context=storylines_guide}{email}
    recapping the day's news
  \end{itemize}

  \begin{itemize}
  \tightlist
  \item
    Download our mobile app on
    \href{https://apps.apple.com/us/app/nytimes/id284862083?ls=1\&mat_click_id=5c79ae7455014fd1bd66b5610c05b8f2-20191112-16948\&referrer=mat_click_id\%3D5c79ae7455014fd1bd66b5610c05b8f2-20191112-16948\%26link_click_id\%3D722930677036718082}{iOS}
    and
    \href{http://a.localytics.com/android?id=com.nytimes.android\&referrer=utm_source\%3Dother_nyt_mobile_web\%26utm_medium\%3DWeb\%2520page\%26utm_term\%3DGeneral\%2520Mobile\%2520Page\%26utm_campaign\%3DNYT\%2520Mobile\%2520General\%2520Page}{Android}
    and turn on Breaking News and Politics alerts
  \end{itemize}
\end{itemize}

Advertisement

\protect\hyperlink{after-bottom}{Continue reading the main story}

\hypertarget{site-index}{%
\subsection{Site Index}\label{site-index}}

\hypertarget{site-information-navigation}{%
\subsection{Site Information
Navigation}\label{site-information-navigation}}

\begin{itemize}
\tightlist
\item
  \href{https://help.nytimes.com/hc/en-us/articles/115014792127-Copyright-notice}{©~2020~The
  New York Times Company}
\end{itemize}

\begin{itemize}
\tightlist
\item
  \href{https://www.nytco.com/}{NYTCo}
\item
  \href{https://help.nytimes.com/hc/en-us/articles/115015385887-Contact-Us}{Contact
  Us}
\item
  \href{https://www.nytco.com/careers/}{Work with us}
\item
  \href{https://nytmediakit.com/}{Advertise}
\item
  \href{http://www.tbrandstudio.com/}{T Brand Studio}
\item
  \href{https://www.nytimes.com/privacy/cookie-policy\#how-do-i-manage-trackers}{Your
  Ad Choices}
\item
  \href{https://www.nytimes.com/privacy}{Privacy}
\item
  \href{https://help.nytimes.com/hc/en-us/articles/115014893428-Terms-of-service}{Terms
  of Service}
\item
  \href{https://help.nytimes.com/hc/en-us/articles/115014893968-Terms-of-sale}{Terms
  of Sale}
\item
  \href{https://spiderbites.nytimes.com}{Site Map}
\item
  \href{https://help.nytimes.com/hc/en-us}{Help}
\item
  \href{https://www.nytimes.com/subscription?campaignId=37WXW}{Subscriptions}
\end{itemize}
