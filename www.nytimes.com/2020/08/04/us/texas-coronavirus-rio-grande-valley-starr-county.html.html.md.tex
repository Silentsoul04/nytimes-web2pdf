Sections

SEARCH

\protect\hyperlink{site-content}{Skip to
content}\protect\hyperlink{site-index}{Skip to site index}

\href{/section/us}{U.S.}\textbar{}`Like a Horror Movie': A Small Border
Hospital Battles the Coronavirus

\url{https://nyti.ms/3k7pji5}

\begin{itemize}
\item
\item
\item
\item
\item
\end{itemize}

\href{https://www.nytimes.com/news-event/coronavirus?action=click\&pgtype=Article\&state=default\&region=TOP_BANNER\&context=storylines_menu}{The
Coronavirus Outbreak}

\begin{itemize}
\tightlist
\item
  live\href{https://www.nytimes.com/2020/08/04/world/coronavirus-cases.html?action=click\&pgtype=Article\&state=default\&region=TOP_BANNER\&context=storylines_menu}{Latest
  Updates}
\item
  \href{https://www.nytimes.com/interactive/2020/us/coronavirus-us-cases.html?action=click\&pgtype=Article\&state=default\&region=TOP_BANNER\&context=storylines_menu}{Maps
  and Cases}
\item
  \href{https://www.nytimes.com/interactive/2020/science/coronavirus-vaccine-tracker.html?action=click\&pgtype=Article\&state=default\&region=TOP_BANNER\&context=storylines_menu}{Vaccine
  Tracker}
\item
  \href{https://www.nytimes.com/2020/08/02/us/covid-college-reopening.html?action=click\&pgtype=Article\&state=default\&region=TOP_BANNER\&context=storylines_menu}{College
  Reopening}
\item
  \href{https://www.nytimes.com/live/2020/08/04/business/stock-market-today-coronavirus?action=click\&pgtype=Article\&state=default\&region=TOP_BANNER\&context=storylines_menu}{Economy}
\end{itemize}

\includegraphics{https://static01.nyt.com/images/2020/08/04/us/04virus-ruraltexas/merlin_175264749_b521b6a7-6869-418b-bba4-07ec27cae42c-articleLarge.jpg?quality=75\&auto=webp\&disable=upscale}

\hypertarget{like-a-horror-movie-a-small-border-hospital-battles-the-coronavirus}{%
\section{`Like a Horror Movie': A Small Border Hospital Battles the
Coronavirus}\label{like-a-horror-movie-a-small-border-hospital-battles-the-coronavirus}}

The hospital in Starr County, Texas, had no I.C.U. at all, and only one
doctor on duty for each shift. Then the coronavirus began surging
through the Rio Grande Valley.

Credit...Christopher Lee for The New York Times

Supported by

\protect\hyperlink{after-sponsor}{Continue reading the main story}

By \href{https://www.nytimes.com/by/edgar-sandoval}{Edgar Sandoval}

\begin{itemize}
\item
  Aug. 4, 2020
\item
  \begin{itemize}
  \item
  \item
  \item
  \item
  \item
  \end{itemize}
\end{itemize}

RIO GRANDE CITY, Texas --- A tense rescue scene has been unfolding for
weeks outside a small rural hospital on the Mexican border that has been
the first line of defense against one of the most voracious coronavirus
outbreaks in the country.

Nearly every day, a crew at Starr County Memorial Hospital prepares a
patient whom its doctors are unable to help, loads the gurney into a
helicopter and stands back as the aircraft roars into the country sky.

``Very, very unfortunately, of all of the patients we have transferred,
none have come back alive,'' said Dr. Jose Vazquez, the health authority
in Starr County, a remote section of the Rio Grande Valley in Texas that
before the coronavirus outbreak did not have a single I.C.U. bed.

There was a time not long ago when the pace was a lot less frantic at
Starr County Memorial, whose 45 beds were once sufficient for the
roughly 65,000 people spread out along the border near Tamaulipas,
Mexico. The county is dotted with tiny villages, long stretches of open
road, cattle ranches and the occasional small town.

On an average day before the outbreak, a handful of doctors and nurses
treated patients at the hospital for ``infections, pneumonia, heart
conditions, and roughly, that's it,'' said Joseph Panlilio, one of the
hospital's head nurses.

But the new wave of coronavirus infections has been as swift as it has
been merciless, with more than 2,110 cases in the county and nearly 70
deaths that are suspected of being linked to Covid-19, local officials
said.

Nearby counties in the valley are also battling surges in infections,
but Starr County lacks the medical staffing and facilities of its more
populated neighbors. On a good day, about 12 full-time doctors serve the
entire county.

``To say we are overwhelmed, it's an understatement,'' said Dr. Cruz
Alberto Bernal, who until recently was the only doctor on duty during
his shifts at the hospital.

Facing an overwhelming number of cases, the hospital said in July that
it would convene an ethics committee to help make difficult decisions
about which patients to treat, which to medevac to better-equipped
hospitals, and which to send home to die.

``The time of rationing medical care is a time that we all have feared
from the beginning, but it looks like we are getting to that point
now,'' Dr. Vazquez
\href{https://www.themonitor.com/2020/07/19/starr-county-form-ethics-committee-responsible-virus-resources/}{said
last month}.

Ultimately, Dr. Vazquez said in an interview later, the final decision
rests with the next of kin, in a region where close-knit families may
prefer to take a terminally ill patient home, rather than leave a loved
one to die alone in a hospital room. ``We are not deciding who lives or
dies,'' he said. ``We are not creating death panels.''

\hypertarget{latest-updates-global-coronavirus-outbreak}{%
\section{\texorpdfstring{\href{https://www.nytimes.com/2020/08/04/world/coronavirus-cases.html?action=click\&pgtype=Article\&state=default\&region=MAIN_CONTENT_1\&context=storylines_live_updates}{Latest
Updates: Global Coronavirus
Outbreak}}{Latest Updates: Global Coronavirus Outbreak}}\label{latest-updates-global-coronavirus-outbreak}}

Updated 2020-08-05T07:58:24.076Z

\begin{itemize}
\tightlist
\item
  \href{https://www.nytimes.com/2020/08/04/world/coronavirus-cases.html?action=click\&pgtype=Article\&state=default\&region=MAIN_CONTENT_1\&context=storylines_live_updates\#link-762df92}{As
  talks drag on, McConnell signals openness to jobless aid extension,
  and negotiators agree on a deadline.}
\item
  \href{https://www.nytimes.com/2020/08/04/world/coronavirus-cases.html?action=click\&pgtype=Article\&state=default\&region=MAIN_CONTENT_1\&context=storylines_live_updates\#link-1228a480}{Novavax
  sees encouraging results from two studies of its experimental
  vaccine.}
\item
  \href{https://www.nytimes.com/2020/08/04/world/coronavirus-cases.html?action=click\&pgtype=Article\&state=default\&region=MAIN_CONTENT_1\&context=storylines_live_updates\#link-794484ed}{Mississippians
  must now wear masks in public, governor says.}
\end{itemize}

\href{https://www.nytimes.com/2020/08/04/world/coronavirus-cases.html?action=click\&pgtype=Article\&state=default\&region=MAIN_CONTENT_1\&context=storylines_live_updates}{See
more updates}

More live coverage:
\href{https://www.nytimes.com/live/2020/08/04/business/stock-market-today-coronavirus?action=click\&pgtype=Article\&state=default\&region=MAIN_CONTENT_1\&context=storylines_live_updates}{Markets}

Hospital officials said the community needed to understand that so small
a facility could not treat everyone on its own.

``We were not built for a situation like this,'' said Eloy Vera, the
county judge.

\includegraphics{https://static01.nyt.com/images/2020/08/04/us/04virus-ruraltexas1/merlin_175264935_010ab7eb-3978-42c3-9574-bee41dd37630-articleLarge.jpg?quality=75\&auto=webp\&disable=upscale}

Starr County,
\href{https://www.usatoday.com/story/money/2019/01/25/poorest-counties-in-the-us-median-household-income/38870175/}{one
of the poorest in the nation}, is not alone. A
\href{https://www.healthaffairs.org/doi/10.1377/hlthaff.2020.00581}{study
published this week in the journal Health Affairs}, warning of a stark
disparity in the availability of critical care facilities in the midst
of the pandemic, found that nearly half of the nation's communities with
a median income of \$35,000 or less had no intensive care beds at all,
compared with 3 percent of the highest-income communities.

``Unfortunately, there will be a lot of unnecessary suffering and deaths
from Covid-19 because of the lack of I.C.U. capacity in these low-income
areas,'' said Genevieve Kanter, an assistant professor at the University
of Pennsylvania Perelman School of Medicine and one of the study's
authors.

As cases climbed in Starr County and the hospital struggled, it began
transporting a handful of its most severe cases by helicopter and
ambulance to bigger hospitals in Lubbock, Dallas, Houston and San
Antonio, and even across the state line in Oklahoma.

Image

The Covid-19 ward at Starr County Memorial Hospital.Credit...Christopher
Lee for The New York Times

The crush shows no signs of abating. On a recent afternoon, doctors and
nurses rushed in and out of the clinic's improvised Covid-19 unit,
roughly the size of one and a half tractor-trailers.

It was put together behind a makeshift wall of plywood, heavy plastic
and duct tape to separate coronavirus patients from those in the rest of
the hospital.

Doctors and nurses, most of them wearing several layers of protective
gear, fanned themselves desperately during their rounds. Any hint of
cool air sputtering from the hospital's overburdened air-conditioning
system was quickly overcome by the unforgiving Texas heat seeping
through the walls.

``The A.C. is working,'' Dr. Bernal said. ``It's overworked, just like
us.''

Most of the patients in the Covid-19 unit were older, and were grappling
with pre-existing conditions including obesity, hypertension and heart
conditions.

Roger Garcia, 38, said his mother, Martha Ramos de Garcia, 65, had
contracted the virus in late July while she was undergoing chemotherapy
for breast cancer.

For a while, the family held off sending her to Starr County Memorial,
he said. ``We knew it was a small hospital. They don't have enough of
everything.''

But one blistering day, she was unable to breathe on her own, and a
responding paramedic only needed a quick look. ``Se ve malita'' --- she
looks a bit sick, he told Mr. Garcia, his words minimizing what they all
knew was a grave situation. His mother died a few days later in the
makeshift Covid-19 unit.

``They tried, but couldn't save her,'' Mr. Garcia said. ``It feels like
a horror movie. People are dying everywhere.''

Image

Staff in the COVID-19 ward move equipment.Credit...Christopher Lee for
The New York Times

Residents are still trying to understand how the situation became so
serious as suddenly as it did.

The surge was slow to arrive. After neighboring counties began reporting
an explosion of infections in the spring, 21 days passed before a single
case was detected in Starr County, Dr. Vazquez said.

Image

Dr. Cruz Bernal said he never expected the pace at Starr County Memorial
Hospital to become as busy and harried as it has during the
pandemic.Credit...Christopher Lee for The New York Times

But when the state reopened its economy in May, the virus began
spreading rapidly through nearby Hidalgo and Cameron Counties, fueled by
poverty and chronic disease.
\href{https://www.nytimes.com/2020/07/14/us/coronavirus-texas-rio-grande-valley-border.html}{Large
family outbreaks} occurred as soon as people were allowed to leave their
homes freely, health officials said.

Image

Starr County Memorial Hospital serves an area of rural Texas near the
Mexican border.Credit...Christopher Lee for The New York Times

Fear spread through the communities along the border. ``We're in a
crisis,'' said Roel Ruiz, 57, a construction worker who was strolling
along the river last week wearing an N-95 face mask in the sweltering
heat.

\href{https://www.nytimes.com/news-event/coronavirus?action=click\&pgtype=Article\&state=default\&region=MAIN_CONTENT_3\&context=storylines_faq}{}

\hypertarget{the-coronavirus-outbreak-}{%
\subsubsection{The Coronavirus Outbreak
›}\label{the-coronavirus-outbreak-}}

\hypertarget{frequently-asked-questions}{%
\paragraph{Frequently Asked
Questions}\label{frequently-asked-questions}}

Updated August 4, 2020

\begin{itemize}
\item ~
  \hypertarget{i-have-antibodies-am-i-now-immune}{%
  \paragraph{I have antibodies. Am I now
  immune?}\label{i-have-antibodies-am-i-now-immune}}

  \begin{itemize}
  \tightlist
  \item
    As of right
    now,\href{https://www.nytimes.com/2020/07/22/health/covid-antibodies-herd-immunity.html?action=click\&pgtype=Article\&state=default\&region=MAIN_CONTENT_3\&context=storylines_faq}{that
    seems likely, for at least several months.} There have been
    frightening accounts of people suffering what seems to be a second
    bout of Covid-19. But experts say these patients may have a
    drawn-out course of infection, with the virus taking a slow toll
    weeks to months after initial exposure. People infected with the
    coronavirus typically
    \href{https://www.nature.com/articles/s41586-020-2456-9}{produce}
    immune molecules called antibodies, which are
    \href{https://www.nytimes.com/2020/05/07/health/coronavirus-antibody-prevalence.html?action=click\&pgtype=Article\&state=default\&region=MAIN_CONTENT_3\&context=storylines_faq}{protective
    proteins made in response to an
    infection}\href{https://www.nytimes.com/2020/05/07/health/coronavirus-antibody-prevalence.html?action=click\&pgtype=Article\&state=default\&region=MAIN_CONTENT_3\&context=storylines_faq}{.
    These antibodies may} last in the body
    \href{https://www.nature.com/articles/s41591-020-0965-6}{only two to
    three months}, which may seem worrisome, but that's perfectly normal
    after an acute infection subsides, said Dr. Michael Mina, an
    immunologist at Harvard University. It may be possible to get the
    coronavirus again, but it's highly unlikely that it would be
    possible in a short window of time from initial infection or make
    people sicker the second time.
  \end{itemize}
\item ~
  \hypertarget{im-a-small-business-owner-can-i-get-relief}{%
  \paragraph{I'm a small-business owner. Can I get
  relief?}\label{im-a-small-business-owner-can-i-get-relief}}

  \begin{itemize}
  \tightlist
  \item
    The
    \href{https://www.nytimes.com/article/small-business-loans-stimulus-grants-freelancers-coronavirus.html?action=click\&pgtype=Article\&state=default\&region=MAIN_CONTENT_3\&context=storylines_faq}{stimulus
    bills enacted in March} offer help for the millions of American
    small businesses. Those eligible for aid are businesses and
    nonprofit organizations with fewer than 500 workers, including sole
    proprietorships, independent contractors and freelancers. Some
    larger companies in some industries are also eligible. The help
    being offered, which is being managed by the Small Business
    Administration, includes the Paycheck Protection Program and the
    Economic Injury Disaster Loan program. But lots of folks have
    \href{https://www.nytimes.com/interactive/2020/05/07/business/small-business-loans-coronavirus.html?action=click\&pgtype=Article\&state=default\&region=MAIN_CONTENT_3\&context=storylines_faq}{not
    yet seen payouts.} Even those who have received help are confused:
    The rules are draconian, and some are stuck sitting on
    \href{https://www.nytimes.com/2020/05/02/business/economy/loans-coronavirus-small-business.html?action=click\&pgtype=Article\&state=default\&region=MAIN_CONTENT_3\&context=storylines_faq}{money
    they don't know how to use.} Many small-business owners are getting
    less than they expected or
    \href{https://www.nytimes.com/2020/06/10/business/Small-business-loans-ppp.html?action=click\&pgtype=Article\&state=default\&region=MAIN_CONTENT_3\&context=storylines_faq}{not
    hearing anything at all.}
  \end{itemize}
\item ~
  \hypertarget{what-are-my-rights-if-i-am-worried-about-going-back-to-work}{%
  \paragraph{What are my rights if I am worried about going back to
  work?}\label{what-are-my-rights-if-i-am-worried-about-going-back-to-work}}

  \begin{itemize}
  \tightlist
  \item
    Employers have to provide
    \href{https://www.osha.gov/SLTC/covid-19/standards.html}{a safe
    workplace} with policies that protect everyone equally.
    \href{https://www.nytimes.com/article/coronavirus-money-unemployment.html?action=click\&pgtype=Article\&state=default\&region=MAIN_CONTENT_3\&context=storylines_faq}{And
    if one of your co-workers tests positive for the coronavirus, the
    C.D.C.} has said that
    \href{https://www.cdc.gov/coronavirus/2019-ncov/community/guidance-business-response.html}{employers
    should tell their employees} -\/- without giving you the sick
    employee's name -\/- that they may have been exposed to the virus.
  \end{itemize}
\item ~
  \hypertarget{should-i-refinance-my-mortgage}{%
  \paragraph{Should I refinance my
  mortgage?}\label{should-i-refinance-my-mortgage}}

  \begin{itemize}
  \tightlist
  \item
    \href{https://www.nytimes.com/article/coronavirus-money-unemployment.html?action=click\&pgtype=Article\&state=default\&region=MAIN_CONTENT_3\&context=storylines_faq}{It
    could be a good idea,} because mortgage rates have
    \href{https://www.nytimes.com/2020/07/16/business/mortgage-rates-below-3-percent.html?action=click\&pgtype=Article\&state=default\&region=MAIN_CONTENT_3\&context=storylines_faq}{never
    been lower.} Refinancing requests have pushed mortgage applications
    to some of the highest levels since 2008, so be prepared to get in
    line. But defaults are also up, so if you're thinking about buying a
    home, be aware that some lenders have tightened their standards.
  \end{itemize}
\item ~
  \hypertarget{what-is-school-going-to-look-like-in-september}{%
  \paragraph{What is school going to look like in
  September?}\label{what-is-school-going-to-look-like-in-september}}

  \begin{itemize}
  \tightlist
  \item
    It is unlikely that many schools will return to a normal schedule
    this fall, requiring the grind of
    \href{https://www.nytimes.com/2020/06/05/us/coronavirus-education-lost-learning.html?action=click\&pgtype=Article\&state=default\&region=MAIN_CONTENT_3\&context=storylines_faq}{online
    learning},
    \href{https://www.nytimes.com/2020/05/29/us/coronavirus-child-care-centers.html?action=click\&pgtype=Article\&state=default\&region=MAIN_CONTENT_3\&context=storylines_faq}{makeshift
    child care} and
    \href{https://www.nytimes.com/2020/06/03/business/economy/coronavirus-working-women.html?action=click\&pgtype=Article\&state=default\&region=MAIN_CONTENT_3\&context=storylines_faq}{stunted
    workdays} to continue. California's two largest public school
    districts --- Los Angeles and San Diego --- said on July 13, that
    \href{https://www.nytimes.com/2020/07/13/us/lausd-san-diego-school-reopening.html?action=click\&pgtype=Article\&state=default\&region=MAIN_CONTENT_3\&context=storylines_faq}{instruction
    will be remote-only in the fall}, citing concerns that surging
    coronavirus infections in their areas pose too dire a risk for
    students and teachers. Together, the two districts enroll some
    825,000 students. They are the largest in the country so far to
    abandon plans for even a partial physical return to classrooms when
    they reopen in August. For other districts, the solution won't be an
    all-or-nothing approach.
    \href{https://bioethics.jhu.edu/research-and-outreach/projects/eschool-initiative/school-policy-tracker/}{Many
    systems}, including the nation's largest, New York City, are
    devising
    \href{https://www.nytimes.com/2020/06/26/us/coronavirus-schools-reopen-fall.html?action=click\&pgtype=Article\&state=default\&region=MAIN_CONTENT_3\&context=storylines_faq}{hybrid
    plans} that involve spending some days in classrooms and other days
    online. There's no national policy on this yet, so check with your
    municipal school system regularly to see what is happening in your
    community.
  \end{itemize}
\end{itemize}

The coronavirus, he said, was everywhere.

``I'm not sick. Neither is my family. But I'm afraid it's a matter of
time.''

Image

An improvised partition isolates the Covid-19 unit from the rest of the
hospital.Credit...Christopher Lee for The New York Times

Aid from the state and from the Navy has staved off some of the
casualties that might otherwise have occurred, hospital officials said.
Medical specialists and much-needed medical supplies and equipment ---
including ventilators, oxygen support and IV pumps --- were welcomed
with a huge sense of relief by the small hospital staff.

``We can use any help we can get,'' Mr. Vera said.

As is the case at most hospitals during a pandemic, visitors are not
allowed. But that did not stop dozens of local residents from flocking
to the hospital grounds last week, peering in at patients through the
windows. They resembled Victorian suitors defying orders to stay away
from a forbidden love, throwing air kisses and heart hand signs their
way.

Inside, the doctors made their way among the roughly 30 patients in the
ward --- it was filled to capacity --- walking in and out of quaint
rooms adorned with country flowered curtains.

Dr. Bernal, who graduated from medical school three years ago, said that
when he took the job at the rural hospital, he never thought he would be
facing the pace of a big-city facility.

``Before the pandemic, I was signing three to four death certificates a
year,'' he said. ``These days I have been signing at least six a week.
And that's just me.''

A nurse wearing a face shield delivered news that the doctor had dreaded
but expected --- a 72-year-old woman, already suffering from severe
obesity, had succumbed to the virus moments earlier.

Image

Medical workers confer in the Covid-19 ward.~Credit...Christopher Lee
for The New York Times

Not far away, his colleagues were trying to save the lives of several
other patients who were fading fast.

Mr. Panlilio, the nurse in charge, watched closely as three other nurses
wrapped bandages around the knees of a woman in her 60s who was
connected to a ventilator. Her treatment at the hospital, he said, had
run its course.

``She needs a higher level of care than we can provide,'' he said. ``We
need to open her throat and clear her airways. We simply don't have the
necessary tools to do that here.''

He ordered an air transfer to a bigger hospital in another city ---~
anywhere that would take her, he said.

Wasting no time, another nurse stabbed a phone's keypad with her finger.
``We are trying to make it happen as soon as possible,'' Mr. Panlilio
said.

But there was no answer.

Mr. Panlilio stood and watched as his colleague tried another phone
number, and another, and another.

Advertisement

\protect\hyperlink{after-bottom}{Continue reading the main story}

\hypertarget{site-index}{%
\subsection{Site Index}\label{site-index}}

\hypertarget{site-information-navigation}{%
\subsection{Site Information
Navigation}\label{site-information-navigation}}

\begin{itemize}
\tightlist
\item
  \href{https://help.nytimes.com/hc/en-us/articles/115014792127-Copyright-notice}{©~2020~The
  New York Times Company}
\end{itemize}

\begin{itemize}
\tightlist
\item
  \href{https://www.nytco.com/}{NYTCo}
\item
  \href{https://help.nytimes.com/hc/en-us/articles/115015385887-Contact-Us}{Contact
  Us}
\item
  \href{https://www.nytco.com/careers/}{Work with us}
\item
  \href{https://nytmediakit.com/}{Advertise}
\item
  \href{http://www.tbrandstudio.com/}{T Brand Studio}
\item
  \href{https://www.nytimes.com/privacy/cookie-policy\#how-do-i-manage-trackers}{Your
  Ad Choices}
\item
  \href{https://www.nytimes.com/privacy}{Privacy}
\item
  \href{https://help.nytimes.com/hc/en-us/articles/115014893428-Terms-of-service}{Terms
  of Service}
\item
  \href{https://help.nytimes.com/hc/en-us/articles/115014893968-Terms-of-sale}{Terms
  of Sale}
\item
  \href{https://spiderbites.nytimes.com}{Site Map}
\item
  \href{https://help.nytimes.com/hc/en-us}{Help}
\item
  \href{https://www.nytimes.com/subscription?campaignId=37WXW}{Subscriptions}
\end{itemize}
