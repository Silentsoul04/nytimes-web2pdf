Sections

SEARCH

\protect\hyperlink{site-content}{Skip to
content}\protect\hyperlink{site-index}{Skip to site index}

\href{https://www.nytimes.com/section/well/family}{Family}

\href{https://myaccount.nytimes.com/auth/login?response_type=cookie\&client_id=vi}{}

\href{https://www.nytimes.com/section/todayspaper}{Today's Paper}

\href{/section/well/family}{Family}\textbar{}Why Parents Should Pause
Before Oversharing Online

\href{https://nyti.ms/31hYzT7}{https://nyti.ms/31hYzT7}

\begin{itemize}
\item
\item
\item
\item
\item
\item
\end{itemize}

Advertisement

\protect\hyperlink{after-top}{Continue reading the main story}

Supported by

\protect\hyperlink{after-sponsor}{Continue reading the main story}

\hypertarget{why-parents-should-pause-before-oversharing-online}{%
\section{Why Parents Should Pause Before Oversharing
Online}\label{why-parents-should-pause-before-oversharing-online}}

As social media comes of age, will we regret all the information we
revealed about our families during its early years?

\includegraphics{https://static01.nyt.com/images/2020/08/04/well/04sharenting/04sharenting-articleLarge.jpg?quality=75\&auto=webp\&disable=upscale}

By Stacey Steinberg

\begin{itemize}
\item
  Aug. 4, 2020
\item
  \begin{itemize}
  \item
  \item
  \item
  \item
  \item
  \item
  \end{itemize}
\end{itemize}

As a mother of three, a photographer and a children's rights lawyer, my
roles as memory keeper and memory revealer are constantly in flux.

I joined Facebook in 2008 when my first child was a year old. His life,
along with the lives of my other two children, have since been the
highlight reels of my newsfeed, which in the beginning contained more
milestones than I was recording in their baby books. A few years ago, I
began to question whether I was unwittingly putting my children's
privacy in jeopardy and whether their life stories were really mine to
tell. I reversed course and changed how I share online.

Parents of children under the age of 18 are the first to raise kids
entirely alongside our newsfeeds. As social media comes of age, will we
regret all the information we revealed about our families during its
early years?

\hypertarget{a-conflict-of-interest}{%
\subsection{A Conflict of Interest}\label{a-conflict-of-interest}}

Studying children's privacy on social media fed both my personal
conflicts and my professional passions, so six years ago, I delved deep
into the work of studying the intersection of a child's right to privacy
and a parent's right to share.

What I quickly learned was that the law does not give us much guidance
when it comes to how we use social media as families. Societal norms
encour­age us to use restraint before publicly sharing personal
informa­tion about our friends and family. But nothing stops us as
parents from sharing our child's stories with the virtual world.

While there are laws that protect American children's privacy in certain
contexts --- such as \href{https://www.hhs.gov/hipaa/index.html}{HIPAA}
for health care,
\href{https://www2.ed.gov/policy/gen/guid/fpco/ferpa/index.html}{FERPA}
for education and
\href{https://www.ftc.gov/enforcement/rules/rulemaking-regulatory-reform-proceedings/childrens-online-privacy-protection-rule}{COPPA}
for the online privacy of children under 13 --- they \emph{do not} have
a right to privacy
\href{https://papers.ssrn.com/sol3/papers.cfm?abstract_id=1746540}{``}\href{https://papers.ssrn.com/sol3/papers.cfm?abstract_id=1746540}{\emph{from}}
\href{https://papers.ssrn.com/sol3/papers.cfm?abstract_id=1746540}{their
parents,''} except in the most limited of circumstances.

Most other countries guarantee a child the right to privacy through an
international agreement called the United Nations Convention on the
Rights of the Child. The United States signed the agreement, but it is
the only United Nations member country not to have ratified it, which
means it is not law or formal policy here. Additionally, doctrines like
\href{https://www.washingtonpost.com/news/parenting/wp/2018/07/11/how-europes-right-to-be-forgotten-could-protect-kids-online-privacy-in-the-u-s/?noredirect=on}{the
Right to Be Forgotten} might offer children in the European Union
remedies for their parents' oversharing once they come of age.

In addition to the risk that children will be mortified by what their
parents have posted about them, there is the chance that peers might
come across certain posts and use them as fodder for bullying. And there
are possible risks from strangers as well. For example, one study by
Barclays suggested that by the year 2030,
\href{https://www.bbc.com/news/education-44153754\#:~:text=\%22Sharenting\%22\%20\%2D\%20where\%20parents\%20share,with\%20so\%20much\%20online\%20sharing.}{parental
sharing of their children's data will result in over seven million
incidents of identity fraud.}

We know that data collectors can collect our personal information from
social media posts. When we share online, these same collectors may be
building digital dossiers on our children.

Sharing online could also lead to image theft by pedophiles. Bath and
beach pictures could be prime targets, but other images could be wrongly
appropriated as well. A pedophile could potentially take any image of a
child, use computer technology to morph it with a separate nude or
sexual image of an adult, and share it as child pornography. While it is
difficult to know with any confidence the frequency of such occurrences,
the impact can be devastating, said Mary Anne Franks, a professor at the
University of Miami School of Law and an adviser on online privacy to
legislators and the technology industry.

\hypertarget{why-parents-share}{%
\subsection{Why Parents Share}\label{why-parents-share}}

I have seen firsthand the power that sharing on social media has on my
life and on the lives of my children and community. I have advocated on
behalf of issues that are important to my family, sharing our own
experiences facing
\href{https://www.gainesville.com/news/20190507/students-talk-about-anti-semitic-bullying-in-alachua-county-schools}{anti-Semitism}
in the hopes of changing school curriculum, for example. I've
\href{https://www.law.ufl.edu/law-news/law-professors-artistic-turn}{photographed
families} whose lives have been
\href{https://www.washingtonpost.com/news/parenting/wp/2016/09/20/families-of-children-with-cancer-ask-please-dont-look-away-heres-how-you-can-help/}{touched
by childhood cancer} who shared their children's story alongside my
pictures to help fund medical research and the costs associated with
treatment, raise awareness of rare conditions, and create supportive
communities that allow children to feel connected to others during long
hospital stays.

In our current era of social distancing, social media has become a
primary way to
\href{https://www.washingtonpost.com/lifestyle/2020/04/06/this-may-be-time-harness-power-social-media-family/}{stay
in touch}. I have watched as friends shared stories of medical
challenges, employment discrimination and racial injustice, and I have
learned from the power of their narratives. While I am still working on
remembering to put the phone down more frequently, I can also appreciate
the appeal of having a community at my fingertips.

When we share openly, others similar­ly situated gain support and
knowledge. As a result, we deeply connect with one another and recognize
the rich diversity in society.

Social media might also offer tools to help us become better parents. A
\href{https://www.pewresearch.org/internet/2020/07/28/parents-attitudes-and-experiences-related-to-digital-technology/}{Pew
Research study} out last week reported that 82 percent of parents who
use social media post about their children online. Many of these parents
turn to the internet and social media to get advice about screen time,
with ``40 percent of parents who use the internet getting advice from
parenting websites or blogs, 29 percent of parents who use social media
turning to social media sites and 19 percent of internet-using parents
getting information from online message boards.''

When I started this work, I expected to walk away from the research
never wanting to share again. That did not happen --- I'm still on
Facebook. I found that despite its drawbacks, social media has added
valuable connections to my life.

What has changed for me is that the conversations around sharing my
story, and sharing my children's stories, have become more nuanced.
Getting to know the families I met through my
\href{https://www.washingtonpost.com/news/parenting/wp/2016/09/20/families-of-children-with-cancer-ask-please-dont-look-away-heres-how-you-can-help/}{photog­raphy
project} constantly reminded me of the power of vulnerability. Working
as a child abuse prosecutor reminded me of the dangers lurking past a
parent's newsfeed. Being the parent of a grow­ing teen reminded me that
if I do not teach my children to exer­cise restraint online, they will
have a harder time learning how to respect others' privacy in digital
spaces.

\hypertarget{finding-a-balance}{%
\subsection{Finding a Balance}\label{finding-a-balance}}

Most parents do not overshare online because they are malicious; they
simply have not fully considered the significance of their child's
digital footprint. Well-informed parents are best suited to making
sharing decisions on behalf of their children. There are ways we can
share smarter on social media and do a better job of protecting our
children's privacy in a no-privacy world.

We can be mindful of the audience with which we share and appreciate how
data brokers might
\href{https://journals.sagepub.com/doi/pdf/10.1177/0743915619858290?casa_token=jLmLV-iBYhkAAAAA\%3Aj7WZWr2WWuc5BbRn262FPWcMtc5qdErO-fTpqp97T7c59Fpnx6AmrdeQZACuMAv-Eh9MgreG8uBUXQ\&}{try
to take advantage of our drive for connection}. We can also use social
media as a tool to talk to our kids about the importance of consent and
about the risks we all face by oversharing online.

We can ask our kids before sharing their pictures with friends and
family on social media. Even young kids benefit from being heard and
understood. At the same time, parents benefit from connecting with
others online. Balancing these competing interests can be challenging.

Should parents share videos of their child having a tantrum in a closed
Facebook group with the hopes of gaining support from other parents of
young children? What about a teenager doing a goofy dance in the living
room that she thought was private but her mom thinks is adorable to
share? Parents need to think deeply about their children's online
privacy and safety needs, because until they get older, we are the ones
responsible for protecting their digital footprints.

When our kids see us step out of the moment and into our newsfeeds to
share a picture, instead of waiting until later, they take note. We need
to think about how we want to model online sharing, so that our kids can
follow our example when they are old enough to start using Instagram,
TikTok or whatever the next popular platform may be.

\href{https://www.law.ufl.edu/faculty/stacey-steinberg}{\emph{Stacey
Steinberg}} \emph{is a legal-skills professor at the University of
Florida Levin College of Law and the author of the new book
``}\href{https://www.amazon.com/Growing-Up-Shared-Media-No-Privacy/dp/1492698105}{\emph{Growing
Up Shared}}\emph{.'' Portions of this essay are adapted from her book.}

\begin{center}\rule{0.5\linewidth}{\linethickness}\end{center}

Advertisement

\protect\hyperlink{after-bottom}{Continue reading the main story}

\hypertarget{site-index}{%
\subsection{Site Index}\label{site-index}}

\hypertarget{site-information-navigation}{%
\subsection{Site Information
Navigation}\label{site-information-navigation}}

\begin{itemize}
\tightlist
\item
  \href{https://help.nytimes.com/hc/en-us/articles/115014792127-Copyright-notice}{©~2020~The
  New York Times Company}
\end{itemize}

\begin{itemize}
\tightlist
\item
  \href{https://www.nytco.com/}{NYTCo}
\item
  \href{https://help.nytimes.com/hc/en-us/articles/115015385887-Contact-Us}{Contact
  Us}
\item
  \href{https://www.nytco.com/careers/}{Work with us}
\item
  \href{https://nytmediakit.com/}{Advertise}
\item
  \href{http://www.tbrandstudio.com/}{T Brand Studio}
\item
  \href{https://www.nytimes.com/privacy/cookie-policy\#how-do-i-manage-trackers}{Your
  Ad Choices}
\item
  \href{https://www.nytimes.com/privacy}{Privacy}
\item
  \href{https://help.nytimes.com/hc/en-us/articles/115014893428-Terms-of-service}{Terms
  of Service}
\item
  \href{https://help.nytimes.com/hc/en-us/articles/115014893968-Terms-of-sale}{Terms
  of Sale}
\item
  \href{https://spiderbites.nytimes.com}{Site Map}
\item
  \href{https://help.nytimes.com/hc/en-us}{Help}
\item
  \href{https://www.nytimes.com/subscription?campaignId=37WXW}{Subscriptions}
\end{itemize}
