Sections

SEARCH

\protect\hyperlink{site-content}{Skip to
content}\protect\hyperlink{site-index}{Skip to site index}

\href{https://www.nytimes.com/section/world/americas}{Americas}

\href{https://myaccount.nytimes.com/auth/login?response_type=cookie\&client_id=vi}{}

\href{https://www.nytimes.com/section/todayspaper}{Today's Paper}

\href{/section/world/americas}{Americas}\textbar{}Colombia Supreme Court
Orders Ex-President Álvaro Uribe Detained

\url{https://nyti.ms/2DCSGHX}

\begin{itemize}
\item
\item
\item
\item
\item
\end{itemize}

Advertisement

\protect\hyperlink{after-top}{Continue reading the main story}

Supported by

\protect\hyperlink{after-sponsor}{Continue reading the main story}

\hypertarget{colombia-supreme-court-orders-ex-president-uxe1lvaro-uribe-detained}{%
\section{Colombia Supreme Court Orders Ex-President Álvaro Uribe
Detained}\label{colombia-supreme-court-orders-ex-president-uxe1lvaro-uribe-detained}}

A decision to put Mr. Uribe under house arrest as a fraud and bribery
investigation unfolds could be a turning point in a nation used to
seeing powerful politicians avoid prosecution.

\includegraphics{https://static01.nyt.com/images/2020/08/05/world/05uribe-print/merlin_138461055_e8f62ffd-95e1-4b71-8a94-dda7b8b50083-articleLarge.jpg?quality=75\&auto=webp\&disable=upscale}

By \href{https://www.nytimes.com/by/julie-turkewitz}{Julie Turkewitz}

\begin{itemize}
\item
  Aug. 4, 2020
\item
  \begin{itemize}
  \item
  \item
  \item
  \item
  \item
  \end{itemize}
\end{itemize}

\href{https://www.nytimes.com/es/2020/08/04/espanol/america-latina/alvaro-uribe-detencion-colombia.html}{Leer
en español}

BOGOTÁ, Colombia --- Colombia's Supreme Court ordered on Tuesday the
detention of a former president and longtime giant of Colombian
politics, Álvaro Uribe, amid an investigation into whether he committed
acts of fraud, bribery and witness tampering.

The decision is a landmark in a nation accustomed to back door deals
between politicians who were rarely called to answer for their actions
in court.

While some other nations in Latin America have tackled corruption
aggressively in recent years, sometimes prosecuting presidents, Colombia
has infrequently indicted major political players.

Widely viewed as the most powerful Colombian politician of the last two
decades, Mr. Uribe had been the subject of investigation for years, but
this is the closest he has come to facing a panel of judges. His ability
to avoid prosecution had led many Colombians to call him the ``Teflon
president.''

The court order has the potential to upend the political landscape in
Colombia. And it makes him the first president in modern Colombian
history to face detention.

He will spend the time under house arrest, the court said. While this is
far less severe than time in prison, Mr. Uribe said he was dreading the
confinement.

``Being deprived of my freedom causes me deep sadness,'' Mr. Uribe
\href{https://twitter.com/AlvaroUribeVel/status/1290712262504779784}{wrote}
on Twitter on Tuesday, ``for my wife, for my family and for the
Colombians who still believe that I have done something good for the
country.''

Mr. Uribe was president from 2002 to 2010, and continues to wield
outsize power from his seat as senator. The current president, Iván
Duque, was little known before Mr. Uribe backed him --- and he won
election in 2018 with a promise to restore Mr. Uribe's legacy.

Mr. Uribe's standing in Colombia makes his detention ``really something
significant for our country,'' signaling a possible shift toward forcing
previously untouchable politicians to answer for alleged crimes, said
Francisco Bernate, a law professor at the Universidad del Rosario in
Bogotá, the capital.

His detention threatens to further polarize Colombian politics,
heightening conflict between Mr. Uribe's allies and his opponents over
the former president's legacy.

On Tuesday, Mr. Duque attacked his own judicial system for pursuing his
mentor,
\href{https://twitter.com/IvanDuque/status/1290755832330813442}{denouncing}
the fact that Mr. Uribe would not be allowed to remain free pending the
resolution of his case --- something that criminals and guerrillas have
been allowed to do, he noted.

``It hurts, as a Colombian,'' Mr. Duque said, that ``an exemplary public
servant, who has occupied the highest post in the state, is not allowed
to defend himself in liberty, with the presumption of innocence.''

Prosecuting judges have not yet brought formal charges against Mr.
Uribe, but the Colombian justice system allows judges to detain him
pending an indictment if they believe he is a flight risk or could
tamper with evidence. He could be held for up to a year as the
investigation moves along.

The case stems from an investigation that the Supreme Court started in
2018. The court's judges are examining whether Mr. Uribe tried to
influence the testimony of an alleged paramilitary member, Juan
Guillermo Monsalve, pushing Mr. Monsalve to retract statements in which
he linked Mr. Uribe to the creation of paramilitary groups.

Mr. Uribe has denied a connection to paramilitary groups, instead saying
he fought against them. He has also denied asking anyone to obstruct
justice.

If found guilty, Mr. Uribe could face approximately six to eight years
in prison, Mr. Bernate said, though it is likely he would spend the time
under house arrest instead.

On Tuesday, as darkness fell, Colombians throughout the capital, Bogotá,
leaned out their windows to shout and bang pots in celebration of Mr.
Uribe's detention.

But in Medellín, an Uribe stronghold, hundreds of supporters gathered to
show their support. ``He gave us security like no other president did,''
said Catalina Pozada, 42, who credited the former president for forcing
one of the country's guerrilla groups to halt kidnappings and highway
blockades.

The court's decision could also affect the current president, Mr. Duque,
whose popularity sagged during his first year in office, until he got a
bump for his handling of the pandemic. His supporters on the right may
turn against him for not doing more to keep his mentor free, while
critics on the left may use Mr. Uribe's detention to taint Mr. Duque and
associating him with criminals.

Mr. Duque defended his mentor on Tuesday, saying the former president
embodied ``honorability.'' Speaking on a national radio station, Mr.
Duque said the idea that Mr. Uribe would be aligned with paramilitary
groups was ``absurd.''

The case is one of several investigations in the Supreme Court into Mr.
Uribe's conduct over the years.

The investigation came about after Mr. Uribe accused a political
opponent, Senator Iván Cepeda, of manipulating witnesses against him,
prompting an investigation into Mr. Cepeda. That inquiry was closed in
2018, and the court decided instead to proceed with the investigation
into Mr. Uribe for allegedly bribing a witness and procedural fraud.

``This is an important shift toward strengthening democracy,'' Mr.
Cepeda said. ``Colombia has been a country with monarchic tendencies in
which certain political figures are untouchable. Well, here there cannot
be anyone above the constitution, above the law and above justice.''

Jenny Carolina González contributed reporting from Bogotá and Megan
Janetsky contributed reporting from Medellín.

Advertisement

\protect\hyperlink{after-bottom}{Continue reading the main story}

\hypertarget{site-index}{%
\subsection{Site Index}\label{site-index}}

\hypertarget{site-information-navigation}{%
\subsection{Site Information
Navigation}\label{site-information-navigation}}

\begin{itemize}
\tightlist
\item
  \href{https://help.nytimes.com/hc/en-us/articles/115014792127-Copyright-notice}{©~2020~The
  New York Times Company}
\end{itemize}

\begin{itemize}
\tightlist
\item
  \href{https://www.nytco.com/}{NYTCo}
\item
  \href{https://help.nytimes.com/hc/en-us/articles/115015385887-Contact-Us}{Contact
  Us}
\item
  \href{https://www.nytco.com/careers/}{Work with us}
\item
  \href{https://nytmediakit.com/}{Advertise}
\item
  \href{http://www.tbrandstudio.com/}{T Brand Studio}
\item
  \href{https://www.nytimes.com/privacy/cookie-policy\#how-do-i-manage-trackers}{Your
  Ad Choices}
\item
  \href{https://www.nytimes.com/privacy}{Privacy}
\item
  \href{https://help.nytimes.com/hc/en-us/articles/115014893428-Terms-of-service}{Terms
  of Service}
\item
  \href{https://help.nytimes.com/hc/en-us/articles/115014893968-Terms-of-sale}{Terms
  of Sale}
\item
  \href{https://spiderbites.nytimes.com}{Site Map}
\item
  \href{https://help.nytimes.com/hc/en-us}{Help}
\item
  \href{https://www.nytimes.com/subscription?campaignId=37WXW}{Subscriptions}
\end{itemize}
