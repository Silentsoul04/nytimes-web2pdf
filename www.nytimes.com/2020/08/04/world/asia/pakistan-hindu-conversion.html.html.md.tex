Sections

SEARCH

\protect\hyperlink{site-content}{Skip to
content}\protect\hyperlink{site-index}{Skip to site index}

\href{https://www.nytimes.com/section/world/asia}{Asia Pacific}

\href{https://myaccount.nytimes.com/auth/login?response_type=cookie\&client_id=vi}{}

\href{https://www.nytimes.com/section/todayspaper}{Today's Paper}

\href{/section/world/asia}{Asia Pacific}\textbar{}Poor and Desperate,
Pakistani Hindus Accept Islam to Get By

\url{https://nyti.ms/3fsfhnW}

\begin{itemize}
\item
\item
\item
\item
\item
\end{itemize}

\href{https://www.nytimes.com/news-event/coronavirus?action=click\&pgtype=Article\&state=default\&region=TOP_BANNER\&context=storylines_menu}{The
Coronavirus Outbreak}

\begin{itemize}
\tightlist
\item
  live\href{https://www.nytimes.com/2020/08/04/world/coronavirus-cases.html?action=click\&pgtype=Article\&state=default\&region=TOP_BANNER\&context=storylines_menu}{Latest
  Updates}
\item
  \href{https://www.nytimes.com/interactive/2020/us/coronavirus-us-cases.html?action=click\&pgtype=Article\&state=default\&region=TOP_BANNER\&context=storylines_menu}{Maps
  and Cases}
\item
  \href{https://www.nytimes.com/interactive/2020/science/coronavirus-vaccine-tracker.html?action=click\&pgtype=Article\&state=default\&region=TOP_BANNER\&context=storylines_menu}{Vaccine
  Tracker}
\item
  \href{https://www.nytimes.com/2020/08/02/us/covid-college-reopening.html?action=click\&pgtype=Article\&state=default\&region=TOP_BANNER\&context=storylines_menu}{College
  Reopening}
\item
  \href{https://www.nytimes.com/live/2020/08/04/business/stock-market-today-coronavirus?action=click\&pgtype=Article\&state=default\&region=TOP_BANNER\&context=storylines_menu}{Economy}
\end{itemize}

Advertisement

\protect\hyperlink{after-top}{Continue reading the main story}

Supported by

\protect\hyperlink{after-sponsor}{Continue reading the main story}

\hypertarget{poor-and-desperate-pakistani-hindus-accept-islam-to-get-by}{%
\section{Poor and Desperate, Pakistani Hindus Accept Islam to Get
By}\label{poor-and-desperate-pakistani-hindus-accept-islam-to-get-by}}

Drawn by jobs or land offered by Muslim groups, some Hindus, facing
discrimination and a virus-ravaged economy, are essentially converting
to survive.

\includegraphics{https://static01.nyt.com/images/2020/07/16/world/00pakistan-hindus1/merlin_157508949_89bb1b8d-fea7-4148-b5d5-5cf478681272-articleLarge.jpg?quality=75\&auto=webp\&disable=upscale}

By \href{https://www.nytimes.com/by/maria-abi-habib}{Maria Abi-Habib}
and Zia ur-Rehman

\begin{itemize}
\item
  Aug. 4, 2020
\item
  \begin{itemize}
  \item
  \item
  \item
  \item
  \item
  \end{itemize}
\end{itemize}

The Hindus performed the prayer rituals awkwardly in supplication to
their new, single god, as they prepared to leave their many deities
behind them. Their lips stumbled over Arabic phrases that, once recited,
would seal their conversion to Islam. The last words uttered, the men
and boys were then circumcised.

Dozens of Hindu families converted in June in the Badin district of
Sindh Province in southern Pakistan. Video clips of the ceremony went
viral across the country, delighting hard-line Muslims and weighing on
Pakistan's dwindling Hindu minority.

The mass ceremony was the latest in what is a growing number of such
conversions to Pakistan's majority Muslim faith in recent years ---
although precise data is scarce. Some of these conversions are
voluntary, some not.

News outlets in India, Pakistan's majority-Hindu neighbor and archrival,
were quick to denounce the conversions as forced. But what is happening
is more subtle. Desperation, religious and political leaders on both
sides of the debate say, has often been the driving force behind their
change of religion.

Treated as second-class citizens, the Hindus of Pakistan are often
systemically discriminated against in every walk of life --- housing,
jobs, access to government welfare. While minorities have long been
drawn to convert in order to join the majority and escape discrimination
and sectarian violence, Hindu community leaders say that the recent
uptick in conversions has also been motivated by newfound economic
pressures.

``What we are seeking is social status, nothing else,'' said Muhammad
Aslam Sheikh, whose name was Sawan Bheel until June, when he converted
in Badin with his family. The ceremony in Badin was notable for its
size, involving just over 100 people.

``These conversions,'' he added, ``are becoming very common in poor
Hindu communities.''

Proselytizing Muslim clerics and charity groups add to the faith's
allure, offering incentives of jobs or land to impoverished minority
members only if they convert.

With Pakistan's economy
\href{https://www.nytimes.com/2020/06/15/world/asia/pakistan-coronavirus-hospitals.html}{on
the brink of collapse in the wake of the coronavirus pandemic}, the
pressures on the country's minorities, often its poorest people, have
increased. The economy will contract by 1.3 percent in the 2020 fiscal
year because of the pandemic, the World Bank predicts. And up to 18
million of Pakistan's 74 million jobs may be lost.

Mr. Sheikh and his family hope to find financial support from wealthy
Muslims or from Islamic charities that have cropped up in recent years,
which focus on drawing more people to Islam.

``There is nothing wrong with that,'' Mr. Sheikh said. ``Everyone helps
the people of their faith.''

As Mr. Sheikh sees it, there is nothing left for Pakistan's more
affluent Hindus to give to help the people of their own faith. That is
because there are so few Hindus left.

\includegraphics{https://static01.nyt.com/images/2020/07/16/world/00pakistan-hindus3/merlin_145028607_f5c156cb-2232-4746-b5fd-ef0aa7e83ac8-articleLarge.jpg?quality=75\&auto=webp\&disable=upscale}

At independence in 1947,
\href{https://theprint.in/opinion/modi-critics-decry-india-mistreating-minorities-but-cant-whitewash-pak-islamisation/355536/}{Hindus
composed 20.5 percent of the population} of the areas that now form
Pakistan. In the following decades, the percentage shrank rapidly, and
by 1998 --- the last government census to classify people by religion
--- Hindus were just 1.6 percent of Pakistan's population. Most
estimates say it has further dwindled in the past two decades.

Once a melting pot of religions, Sindh Province, where the conversion
ceremony took place, has seen minority members flee to other countries
in droves in recent decades. Many face harsh discrimination, as well as
the specter of violence --- and the risk of being accused of blasphemy,
a capital crime --- if they speak out against it.

AFGHANISTAN

PAKISTAN

IRAN

INDIA

Sindh

Karachi

BADIN district

200 miles

By The New York Times

``The dehumanization of minorities coupled with these very scary times
we are living in --- a weak economy and now the pandemic --- we may see
a raft of people converting to Islam to stave off violence or hunger or
just to live to see another day,'' said Farahnaz Ispahani, a former
Pakistani lawmaker who is now a senior fellow at the Religious Freedom
Institute, a research group in Washington.

\hypertarget{latest-updates-global-coronavirus-outbreak}{%
\section{\texorpdfstring{\href{https://www.nytimes.com/2020/08/04/world/coronavirus-cases.html?action=click\&pgtype=Article\&state=default\&region=MAIN_CONTENT_1\&context=storylines_live_updates}{Latest
Updates: Global Coronavirus
Outbreak}}{Latest Updates: Global Coronavirus Outbreak}}\label{latest-updates-global-coronavirus-outbreak}}

Updated 2020-08-05T07:58:24.076Z

\begin{itemize}
\tightlist
\item
  \href{https://www.nytimes.com/2020/08/04/world/coronavirus-cases.html?action=click\&pgtype=Article\&state=default\&region=MAIN_CONTENT_1\&context=storylines_live_updates\#link-762df92}{As
  talks drag on, McConnell signals openness to jobless aid extension,
  and negotiators agree on a deadline.}
\item
  \href{https://www.nytimes.com/2020/08/04/world/coronavirus-cases.html?action=click\&pgtype=Article\&state=default\&region=MAIN_CONTENT_1\&context=storylines_live_updates\#link-1228a480}{Novavax
  sees encouraging results from two studies of its experimental
  vaccine.}
\item
  \href{https://www.nytimes.com/2020/08/04/world/coronavirus-cases.html?action=click\&pgtype=Article\&state=default\&region=MAIN_CONTENT_1\&context=storylines_live_updates\#link-794484ed}{Mississippians
  must now wear masks in public, governor says.}
\end{itemize}

\href{https://www.nytimes.com/2020/08/04/world/coronavirus-cases.html?action=click\&pgtype=Article\&state=default\&region=MAIN_CONTENT_1\&context=storylines_live_updates}{See
more updates}

More live coverage:
\href{https://www.nytimes.com/live/2020/08/04/business/stock-market-today-coronavirus?action=click\&pgtype=Article\&state=default\&region=MAIN_CONTENT_1\&context=storylines_live_updates}{Markets}

Ms. Ispahani recalled the devastating floods of 2010 in Sindh Province,
which left thousands homeless and with little to eat. Hindus were not
allowed to sit with Muslims at soup kitchens, she said. And when
government aid was handed out, Hindus received less of it than their
Muslim peers did, she said.

``Will they be converting with their hearts and souls?'' Ms. Ispahani
said. ``I don't think so.''

The further economic devastation caused by the pandemic may spur more
sectarian violence, and that may intensify the pressure on minorities to
convert, Ms. Ispahani and others worry.

Murtaza Wahab, an adviser to the chief minister of Sindh, was among
several government officials who said they could not address Ms.
Ispahani's accusation that Hindus received less aid after the floods, as
it happened before they took office.

``The Hindu community is an important part of our society and we believe
that people from all faiths should live together without issue,'' Mr.
Wahab said.

\href{https://www.nytimes.com/2012/03/26/world/asia/pakistani-hindus-say-womans-conversion-to-islam-was-coerced.html}{Forced
conversions} of Hindu girls and women to Islam through kidnapping and
coerced marriages occur throughout Pakistan. But Hindu rights groups are
also troubled by the seemingly voluntary conversions, saying they take
place under such economic duress that they are tantamount to a forced
conversion anyway.

``Overall, religious minorities do not feel safe in Pakistan,'' said Lal
Chand Mahli, a Pakistani Hindu lawmaker with the ruling party, who is a
member of a parliamentary committee to protect minorities from forced
conversions. ``But poor Hindus are the most vulnerable among them. They
are extremely poor and illiterate, and Muslim mosques, charities and
traders exploit them easily and lure them to convert to Islam. A lot of
money is involved in it.''

Clerics like Muhammad Naeem were at the forefront of an effort to
convert more Hindus. (Mr. Naeem, who was 62, died of cardiac arrest two
weeks after he was interviewed in June).

Mr. Naeem said he had overseen more than 450 conversions over the past
two years at Jamia Binoria, his seminary in Karachi. Most of the
converts were low-caste Hindus from Sindh Province, he said.

``We have not been forcing them to convert,'' Mr. Naeem said. ``In fact,
people come to us because they want to escape discrimination attached
with their caste and change their socioeconomic status.''

Demand was so great, he added, that his seminary had set up a separate
department to guide the new converts and provide counsel in legal or
financial matters.

\href{https://www.nytimes.com/news-event/coronavirus?action=click\&pgtype=Article\&state=default\&region=MAIN_CONTENT_3\&context=storylines_faq}{}

\hypertarget{the-coronavirus-outbreak-}{%
\subsubsection{The Coronavirus Outbreak
›}\label{the-coronavirus-outbreak-}}

\hypertarget{frequently-asked-questions}{%
\paragraph{Frequently Asked
Questions}\label{frequently-asked-questions}}

Updated August 4, 2020

\begin{itemize}
\item ~
  \hypertarget{i-have-antibodies-am-i-now-immune}{%
  \paragraph{I have antibodies. Am I now
  immune?}\label{i-have-antibodies-am-i-now-immune}}

  \begin{itemize}
  \tightlist
  \item
    As of right
    now,\href{https://www.nytimes.com/2020/07/22/health/covid-antibodies-herd-immunity.html?action=click\&pgtype=Article\&state=default\&region=MAIN_CONTENT_3\&context=storylines_faq}{that
    seems likely, for at least several months.} There have been
    frightening accounts of people suffering what seems to be a second
    bout of Covid-19. But experts say these patients may have a
    drawn-out course of infection, with the virus taking a slow toll
    weeks to months after initial exposure. People infected with the
    coronavirus typically
    \href{https://www.nature.com/articles/s41586-020-2456-9}{produce}
    immune molecules called antibodies, which are
    \href{https://www.nytimes.com/2020/05/07/health/coronavirus-antibody-prevalence.html?action=click\&pgtype=Article\&state=default\&region=MAIN_CONTENT_3\&context=storylines_faq}{protective
    proteins made in response to an
    infection}\href{https://www.nytimes.com/2020/05/07/health/coronavirus-antibody-prevalence.html?action=click\&pgtype=Article\&state=default\&region=MAIN_CONTENT_3\&context=storylines_faq}{.
    These antibodies may} last in the body
    \href{https://www.nature.com/articles/s41591-020-0965-6}{only two to
    three months}, which may seem worrisome, but that's perfectly normal
    after an acute infection subsides, said Dr. Michael Mina, an
    immunologist at Harvard University. It may be possible to get the
    coronavirus again, but it's highly unlikely that it would be
    possible in a short window of time from initial infection or make
    people sicker the second time.
  \end{itemize}
\item ~
  \hypertarget{im-a-small-business-owner-can-i-get-relief}{%
  \paragraph{I'm a small-business owner. Can I get
  relief?}\label{im-a-small-business-owner-can-i-get-relief}}

  \begin{itemize}
  \tightlist
  \item
    The
    \href{https://www.nytimes.com/article/small-business-loans-stimulus-grants-freelancers-coronavirus.html?action=click\&pgtype=Article\&state=default\&region=MAIN_CONTENT_3\&context=storylines_faq}{stimulus
    bills enacted in March} offer help for the millions of American
    small businesses. Those eligible for aid are businesses and
    nonprofit organizations with fewer than 500 workers, including sole
    proprietorships, independent contractors and freelancers. Some
    larger companies in some industries are also eligible. The help
    being offered, which is being managed by the Small Business
    Administration, includes the Paycheck Protection Program and the
    Economic Injury Disaster Loan program. But lots of folks have
    \href{https://www.nytimes.com/interactive/2020/05/07/business/small-business-loans-coronavirus.html?action=click\&pgtype=Article\&state=default\&region=MAIN_CONTENT_3\&context=storylines_faq}{not
    yet seen payouts.} Even those who have received help are confused:
    The rules are draconian, and some are stuck sitting on
    \href{https://www.nytimes.com/2020/05/02/business/economy/loans-coronavirus-small-business.html?action=click\&pgtype=Article\&state=default\&region=MAIN_CONTENT_3\&context=storylines_faq}{money
    they don't know how to use.} Many small-business owners are getting
    less than they expected or
    \href{https://www.nytimes.com/2020/06/10/business/Small-business-loans-ppp.html?action=click\&pgtype=Article\&state=default\&region=MAIN_CONTENT_3\&context=storylines_faq}{not
    hearing anything at all.}
  \end{itemize}
\item ~
  \hypertarget{what-are-my-rights-if-i-am-worried-about-going-back-to-work}{%
  \paragraph{What are my rights if I am worried about going back to
  work?}\label{what-are-my-rights-if-i-am-worried-about-going-back-to-work}}

  \begin{itemize}
  \tightlist
  \item
    Employers have to provide
    \href{https://www.osha.gov/SLTC/covid-19/standards.html}{a safe
    workplace} with policies that protect everyone equally.
    \href{https://www.nytimes.com/article/coronavirus-money-unemployment.html?action=click\&pgtype=Article\&state=default\&region=MAIN_CONTENT_3\&context=storylines_faq}{And
    if one of your co-workers tests positive for the coronavirus, the
    C.D.C.} has said that
    \href{https://www.cdc.gov/coronavirus/2019-ncov/community/guidance-business-response.html}{employers
    should tell their employees} -\/- without giving you the sick
    employee's name -\/- that they may have been exposed to the virus.
  \end{itemize}
\item ~
  \hypertarget{should-i-refinance-my-mortgage}{%
  \paragraph{Should I refinance my
  mortgage?}\label{should-i-refinance-my-mortgage}}

  \begin{itemize}
  \tightlist
  \item
    \href{https://www.nytimes.com/article/coronavirus-money-unemployment.html?action=click\&pgtype=Article\&state=default\&region=MAIN_CONTENT_3\&context=storylines_faq}{It
    could be a good idea,} because mortgage rates have
    \href{https://www.nytimes.com/2020/07/16/business/mortgage-rates-below-3-percent.html?action=click\&pgtype=Article\&state=default\&region=MAIN_CONTENT_3\&context=storylines_faq}{never
    been lower.} Refinancing requests have pushed mortgage applications
    to some of the highest levels since 2008, so be prepared to get in
    line. But defaults are also up, so if you're thinking about buying a
    home, be aware that some lenders have tightened their standards.
  \end{itemize}
\item ~
  \hypertarget{what-is-school-going-to-look-like-in-september}{%
  \paragraph{What is school going to look like in
  September?}\label{what-is-school-going-to-look-like-in-september}}

  \begin{itemize}
  \tightlist
  \item
    It is unlikely that many schools will return to a normal schedule
    this fall, requiring the grind of
    \href{https://www.nytimes.com/2020/06/05/us/coronavirus-education-lost-learning.html?action=click\&pgtype=Article\&state=default\&region=MAIN_CONTENT_3\&context=storylines_faq}{online
    learning},
    \href{https://www.nytimes.com/2020/05/29/us/coronavirus-child-care-centers.html?action=click\&pgtype=Article\&state=default\&region=MAIN_CONTENT_3\&context=storylines_faq}{makeshift
    child care} and
    \href{https://www.nytimes.com/2020/06/03/business/economy/coronavirus-working-women.html?action=click\&pgtype=Article\&state=default\&region=MAIN_CONTENT_3\&context=storylines_faq}{stunted
    workdays} to continue. California's two largest public school
    districts --- Los Angeles and San Diego --- said on July 13, that
    \href{https://www.nytimes.com/2020/07/13/us/lausd-san-diego-school-reopening.html?action=click\&pgtype=Article\&state=default\&region=MAIN_CONTENT_3\&context=storylines_faq}{instruction
    will be remote-only in the fall}, citing concerns that surging
    coronavirus infections in their areas pose too dire a risk for
    students and teachers. Together, the two districts enroll some
    825,000 students. They are the largest in the country so far to
    abandon plans for even a partial physical return to classrooms when
    they reopen in August. For other districts, the solution won't be an
    all-or-nothing approach.
    \href{https://bioethics.jhu.edu/research-and-outreach/projects/eschool-initiative/school-policy-tracker/}{Many
    systems}, including the nation's largest, New York City, are
    devising
    \href{https://www.nytimes.com/2020/06/26/us/coronavirus-schools-reopen-fall.html?action=click\&pgtype=Article\&state=default\&region=MAIN_CONTENT_3\&context=storylines_faq}{hybrid
    plans} that involve spending some days in classrooms and other days
    online. There's no national policy on this yet, so check with your
    municipal school system regularly to see what is happening in your
    community.
  \end{itemize}
\end{itemize}

On a recent afternoon, the call to prayer echoed through a cluster of
newly erected tents in Matli, a barren patch of Sindh. A group of
Karachi's wealthy Muslim merchants bought the land last year for dozens
of families who had converted from Hinduism.

At a new mosque adjacent to the tents, Muhammad Ali --- who was known by
his Hindu name, Rajesh, before converting last year alongside 205 others
--- performed ablutions before praying.

Last year, his entire family had decided to convert to Islam when Mr.
Naeem, the cleric, offered to free them from the bonded labor in which
they were trapped, living and working as indentured servants because of
unpaid debt. Mr. Ali is originally from the Bheel caste, one of the
lowest in Hinduism.

``We have found a sense of equality and brotherhood in Islam, and
therefore we converted to it,'' Mr. Ali said.

Image

Shri Krishna Temple in Mithi, in Sindh Province. Once a melting pot of
religions, Sindh has seen religious minorities flee in droves in recent
decades.Credit...Rizwan Tabassum/Agence France-Presse --- Getty Images

Lower-caste Pakistani Hindus are often the victims of bonded labor. It
was outlawed in 1992, but the practice is still prevalent. The Global
Slavery Index
\href{https://www.globalslaveryindex.org/2018/data/maps/\#prevalence}{estimates
that just over three million Pakistanis} live in debt servitude.

Landlords trap poor Hindus into such bondage by providing loans that
they know can never be repaid. They and their families are then forced
to work off the debt. The women are often sexually abused, rights groups
say.

Mr. Naeem's seminary had rescued several Hindus --- including Mr. Ali
and his family --- from bonded labor by paying off their debts in
exchange for their conversions to Islam.

When Mr. Ali and his family converted, Mr. Naeem and a group of rich
Muslim traders had given them a piece of land and helped them find work,
considering it an Islamic responsibility to help them.

``Those who make efforts to spread the message and bring the non-Muslims
into the fold of Islam will be blessed in the hereafter,'' Mr. Naeem
said.

Advertisement

\protect\hyperlink{after-bottom}{Continue reading the main story}

\hypertarget{site-index}{%
\subsection{Site Index}\label{site-index}}

\hypertarget{site-information-navigation}{%
\subsection{Site Information
Navigation}\label{site-information-navigation}}

\begin{itemize}
\tightlist
\item
  \href{https://help.nytimes.com/hc/en-us/articles/115014792127-Copyright-notice}{©~2020~The
  New York Times Company}
\end{itemize}

\begin{itemize}
\tightlist
\item
  \href{https://www.nytco.com/}{NYTCo}
\item
  \href{https://help.nytimes.com/hc/en-us/articles/115015385887-Contact-Us}{Contact
  Us}
\item
  \href{https://www.nytco.com/careers/}{Work with us}
\item
  \href{https://nytmediakit.com/}{Advertise}
\item
  \href{http://www.tbrandstudio.com/}{T Brand Studio}
\item
  \href{https://www.nytimes.com/privacy/cookie-policy\#how-do-i-manage-trackers}{Your
  Ad Choices}
\item
  \href{https://www.nytimes.com/privacy}{Privacy}
\item
  \href{https://help.nytimes.com/hc/en-us/articles/115014893428-Terms-of-service}{Terms
  of Service}
\item
  \href{https://help.nytimes.com/hc/en-us/articles/115014893968-Terms-of-sale}{Terms
  of Sale}
\item
  \href{https://spiderbites.nytimes.com}{Site Map}
\item
  \href{https://help.nytimes.com/hc/en-us}{Help}
\item
  \href{https://www.nytimes.com/subscription?campaignId=37WXW}{Subscriptions}
\end{itemize}
