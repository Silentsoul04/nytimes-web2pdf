Sections

SEARCH

\protect\hyperlink{site-content}{Skip to
content}\protect\hyperlink{site-index}{Skip to site index}

\href{https://www.nytimes.com/section/world/middleeast}{Middle East}

\href{https://myaccount.nytimes.com/auth/login?response_type=cookie\&client_id=vi}{}

\href{https://www.nytimes.com/section/todayspaper}{Today's Paper}

\href{/section/world/middleeast}{Middle East}\textbar{}Trump's Syria
Sanctions `Cannot Solve the Problem,' Critics Say

\url{https://nyti.ms/33qvckv}

\begin{itemize}
\item
\item
\item
\item
\item
\end{itemize}

\href{https://www.nytimes.com/news-event/coronavirus?action=click\&pgtype=Article\&state=default\&region=TOP_BANNER\&context=storylines_menu}{The
Coronavirus Outbreak}

\begin{itemize}
\tightlist
\item
  live\href{https://www.nytimes.com/2020/08/04/world/coronavirus-cases.html?action=click\&pgtype=Article\&state=default\&region=TOP_BANNER\&context=storylines_menu}{Latest
  Updates}
\item
  \href{https://www.nytimes.com/interactive/2020/us/coronavirus-us-cases.html?action=click\&pgtype=Article\&state=default\&region=TOP_BANNER\&context=storylines_menu}{Maps
  and Cases}
\item
  \href{https://www.nytimes.com/interactive/2020/science/coronavirus-vaccine-tracker.html?action=click\&pgtype=Article\&state=default\&region=TOP_BANNER\&context=storylines_menu}{Vaccine
  Tracker}
\item
  \href{https://www.nytimes.com/2020/08/02/us/covid-college-reopening.html?action=click\&pgtype=Article\&state=default\&region=TOP_BANNER\&context=storylines_menu}{College
  Reopening}
\item
  \href{https://www.nytimes.com/live/2020/08/04/business/stock-market-today-coronavirus?action=click\&pgtype=Article\&state=default\&region=TOP_BANNER\&context=storylines_menu}{Economy}
\end{itemize}

Advertisement

\protect\hyperlink{after-top}{Continue reading the main story}

Supported by

\protect\hyperlink{after-sponsor}{Continue reading the main story}

\hypertarget{trumps-syria-sanctions-cannot-solve-the-problem-critics-say}{%
\section{Trump's Syria Sanctions `Cannot Solve the Problem,' Critics
Say}\label{trumps-syria-sanctions-cannot-solve-the-problem-critics-say}}

Without a broader diplomatic effort, the newest and toughest penalties
will worsen a humanitarian crisis without forcing a leadership change,
experts say.

\includegraphics{https://static01.nyt.com/images/2020/08/03/us/politics/03dc-syria-sanctions/merlin_170451123_c94169bd-de3a-4340-90d7-06525b9de1dd-articleLarge.jpg?quality=75\&auto=webp\&disable=upscale}

\href{https://www.nytimes.com/by/pranshu-verma}{\includegraphics{https://static01.nyt.com/images/2020/07/07/reader-center/author-pranshu-verma/author-pranshu-verma-thumbLarge.png}}\href{https://www.nytimes.com/by/vivian-yee}{\includegraphics{https://static01.nyt.com/images/2018/02/20/multimedia/author-vivian-yee/author-vivian-yee-thumbLarge-v2.png}}

By \href{https://www.nytimes.com/by/pranshu-verma}{Pranshu Verma} and
\href{https://www.nytimes.com/by/vivian-yee}{Vivian Yee}

\begin{itemize}
\item
  Aug. 4, 2020
\item
  \begin{itemize}
  \item
  \item
  \item
  \item
  \item
  \end{itemize}
\end{itemize}

WASHINGTON --- The Trump administration has embarked on its toughest
round of economic sanctions against Syria, intending to break President
Bashar al-Assad's reign over the country and stop a civil war that has
claimed over half a million lives.

Secretary of State Mike Pompeo says the administration will not end the
pressure campaign on Mr. al-Assad and his backers until he agrees to a
United Nations resolution prompting peace talks and a transition of
power.

But diplomatic and humanitarian assistance experts are wary of the
strategy, saying economic sanctions alone, no matter how punishing, will
do little to bring Mr. al-Assad to the negotiating table and will only
worsen the humanitarian crisis in Syria, which has been exacerbated by a
\href{https://www.nytimes.com/2020/06/15/world/middleeast/syria-economy-assad-makhlouf.html}{collapsing
economy}.

Critics also warn that the Trump administration will waste the broadened
sanctioning power Congress gave it if diplomatic outreach to the Syrian
government and its allies does not accompany economic punishment.

``Sanctions alone cannot solve the problem,'' said
\href{https://www.mofo.com/people/john-smith.html}{John E. Smith}, the
former director of the Treasury Department's Office of Foreign Assets
Control. ``It's difficult to see what else the U.S. government is doing
in Syria other than putting a bunch of Syrian regime insiders on a list
that they don't really care one way or the other they are on.''

The newest round of sanctions on Syria comes as the
\href{https://www.nytimes.com/2019/12/16/us/politics/us-syria-sanctions-war-crimes.html}{Caesar
Syria Civilian Protection Act}, which President Trump signed into law in
December, went into effect in mid-June.

The legislation --- named after a military photographer, code-named
Caesar, who leaked over 50,000 photos of torture and other atrocities
taking place in Mr. al-Assad's prisons --- is stronger than previous
sanctions, which started in 1979 when Washington first
\href{https://www.state.gov/state-sponsors-of-terrorism/}{declared
Syria} a state sponsor of terrorism.

It allows the United States to freeze the assets of any person or
business dealing with al-Assad's government, regardless of nationality.
It also targets Russia and Iran, Mr. al-Assad's primary backers.

The law also punishes individuals and corporations anywhere in the world
dealing with three crucial sectors of the Syrian economy: the domestic
oil industry, the Syrian military and engineering or construction
businesses operating in government-held regions.

``Congress is giving companies around the world a choice,'' Mr. Smith
said. ``You can go do business in Syria if you would like, but if you
do, you risk being cut off from the almighty dollar and the U.S.
financial system.''

Since June, over 40 elites have been sanctioned, including Mr.
al-Assad's wife and his oldest son, other members of his extended family
and senior military leaders. Business people crucial to the
reconstruction of government-held Syria have also been targeted.

\hypertarget{latest-updates-global-coronavirus-outbreak}{%
\section{\texorpdfstring{\href{https://www.nytimes.com/2020/08/04/world/coronavirus-cases.html?action=click\&pgtype=Article\&state=default\&region=MAIN_CONTENT_1\&context=storylines_live_updates}{Latest
Updates: Global Coronavirus
Outbreak}}{Latest Updates: Global Coronavirus Outbreak}}\label{latest-updates-global-coronavirus-outbreak}}

Updated 2020-08-05T07:58:24.076Z

\begin{itemize}
\tightlist
\item
  \href{https://www.nytimes.com/2020/08/04/world/coronavirus-cases.html?action=click\&pgtype=Article\&state=default\&region=MAIN_CONTENT_1\&context=storylines_live_updates\#link-762df92}{As
  talks drag on, McConnell signals openness to jobless aid extension,
  and negotiators agree on a deadline.}
\item
  \href{https://www.nytimes.com/2020/08/04/world/coronavirus-cases.html?action=click\&pgtype=Article\&state=default\&region=MAIN_CONTENT_1\&context=storylines_live_updates\#link-1228a480}{Novavax
  sees encouraging results from two studies of its experimental
  vaccine.}
\item
  \href{https://www.nytimes.com/2020/08/04/world/coronavirus-cases.html?action=click\&pgtype=Article\&state=default\&region=MAIN_CONTENT_1\&context=storylines_live_updates\#link-794484ed}{Mississippians
  must now wear masks in public, governor says.}
\end{itemize}

\href{https://www.nytimes.com/2020/08/04/world/coronavirus-cases.html?action=click\&pgtype=Article\&state=default\&region=MAIN_CONTENT_1\&context=storylines_live_updates}{See
more updates}

More live coverage:
\href{https://www.nytimes.com/live/2020/08/04/business/stock-market-today-coronavirus?action=click\&pgtype=Article\&state=default\&region=MAIN_CONTENT_1\&context=storylines_live_updates}{Markets}

Experts note that sanctions have caused widespread concern in the
country. Companies interested in rebuilding Syria's cities and
countryside --- which could require \$250 to \$400 billion to
reconstruct --- may be scared away from doing business in the region,
stalling Syria's path to recovery.

Ahead of the United States announcing its first round of Caesar Act
sanctions in mid-June, Syria devalued its currency by 44 percent.

But Trump administration officials say their efforts, billed as a
``sustained campaign of economic and political pressure,'' has just
started, and they expect many more actions to come.

``This will continue to be the `summer of Caesar,''' said
\href{https://www.state.gov/biographies/joel-d-rayburn/}{Joel D.
Rayburn}, the State Department's special envoy for Syria. ``There will
be no end to them until the Syrian regime and its allies accede.''

The sanctions arrive at a time when Mr. al-Assad has nearly won Syria's
nine-year civil war, and finds his economy crumbling.

\includegraphics{https://static01.nyt.com/images/2020/08/03/us/politics/03dc-syria-sanctions2/merlin_153682044_c2ddd686-4c38-435c-87da-3026b051bfbb-articleLarge.jpg?quality=75\&auto=webp\&disable=upscale}

The currency is nearly worthless, making basic commodities unaffordable
to large swaths of the population. Protests against poor living
conditions have erupted in parts of the country. The
\href{https://www.nytimes.com/interactive/2020/world/coronavirus-maps.html}{coronavirus}
is also taking hold. And Syria's main trading partner, Lebanon, is
dealing with an economic meltdown that has spillover effects into its
own economy.

Former government officials agree that the sanctions, piled on top of
the current state of Syria's economy, could have a devastating impact on
the humanitarian situation in a country where nearly 80 percent of the
people live in poverty.

Others note that Mr. Trump's
\href{https://www.nytimes.com/2019/11/15/us/politics/trump-iran-sanctions.html}{increasing
reliance} on sanctions against repressive governments like Iran and
North Korea has done little to change behaviors of the ruling class,
whose members often find ways to evade the punitive measures or pass
down the punishing effects onto their citizens.

``The regime elites continue to flourish, they continue to get luxury
goods, they continue to do their shopping trips,'' Mr. Smith said. ``It
is generally the people of the jurisdiction that pay the ultimate
penalty from the poverty that is inflicted on that government.''

There is little question that American sanctions, writ large, have made
life harder for many ordinary Syrians.

Business and factory owners describe the headache of importing and
exporting goods using channels outside the American banking system and
of losing international customers and suppliers who do not want to run
afoul of the sanctions.

Cheaper but inferior Syrian- and Iranian-made products have replaced
other imported ones at supermarkets. Iran, struggling with its own
American sanctions, cannot throw Syria a major lifeline.

But the causes of Syria's economic crisis go far beyond the sanctions,
including a civil war that has decimated its cities, factories,
infrastructure and hospitals.

A shopkeeper from the al-Midan neighborhood of Damascus blamed runaway
inflation and soaring food prices on corrupt dealings between the
government and Assad cronies.

``The corruption and government checkpoints and looting are making our
lives and our business harder than the American sanctions,'' said the
shopkeeper, Abu Muhammad, 60.

Democratic and Republican lawmakers have put provisions into the Caesar
Act ensuring humanitarian organizations are still able to provide food
and aid to Syrians.

But humanitarian workers operating in the country note that despite best
intentions of lawmakers, the reality on the ground will be different.

They note medicine is already becoming harder to bring into the country.
Insurance companies are telling aid organizations they will not cover
certain procedures. A.T.M.'s have shut down, causing relief workers to
waste precious time standing in line to withdraw salaries.

This may not all be directly because of the sanctions, aid experts said,
but the vast nature of the United States' sanctions efforts is scaring
companies away from the region, even though they may be legally allowed
to operate.

\href{https://www.nytimes.com/news-event/coronavirus?action=click\&pgtype=Article\&state=default\&region=MAIN_CONTENT_3\&context=storylines_faq}{}

\hypertarget{the-coronavirus-outbreak-}{%
\subsubsection{The Coronavirus Outbreak
›}\label{the-coronavirus-outbreak-}}

\hypertarget{frequently-asked-questions}{%
\paragraph{Frequently Asked
Questions}\label{frequently-asked-questions}}

Updated August 4, 2020

\begin{itemize}
\item ~
  \hypertarget{i-have-antibodies-am-i-now-immune}{%
  \paragraph{I have antibodies. Am I now
  immune?}\label{i-have-antibodies-am-i-now-immune}}

  \begin{itemize}
  \tightlist
  \item
    As of right
    now,\href{https://www.nytimes.com/2020/07/22/health/covid-antibodies-herd-immunity.html?action=click\&pgtype=Article\&state=default\&region=MAIN_CONTENT_3\&context=storylines_faq}{that
    seems likely, for at least several months.} There have been
    frightening accounts of people suffering what seems to be a second
    bout of Covid-19. But experts say these patients may have a
    drawn-out course of infection, with the virus taking a slow toll
    weeks to months after initial exposure. People infected with the
    coronavirus typically
    \href{https://www.nature.com/articles/s41586-020-2456-9}{produce}
    immune molecules called antibodies, which are
    \href{https://www.nytimes.com/2020/05/07/health/coronavirus-antibody-prevalence.html?action=click\&pgtype=Article\&state=default\&region=MAIN_CONTENT_3\&context=storylines_faq}{protective
    proteins made in response to an
    infection}\href{https://www.nytimes.com/2020/05/07/health/coronavirus-antibody-prevalence.html?action=click\&pgtype=Article\&state=default\&region=MAIN_CONTENT_3\&context=storylines_faq}{.
    These antibodies may} last in the body
    \href{https://www.nature.com/articles/s41591-020-0965-6}{only two to
    three months}, which may seem worrisome, but that's perfectly normal
    after an acute infection subsides, said Dr. Michael Mina, an
    immunologist at Harvard University. It may be possible to get the
    coronavirus again, but it's highly unlikely that it would be
    possible in a short window of time from initial infection or make
    people sicker the second time.
  \end{itemize}
\item ~
  \hypertarget{im-a-small-business-owner-can-i-get-relief}{%
  \paragraph{I'm a small-business owner. Can I get
  relief?}\label{im-a-small-business-owner-can-i-get-relief}}

  \begin{itemize}
  \tightlist
  \item
    The
    \href{https://www.nytimes.com/article/small-business-loans-stimulus-grants-freelancers-coronavirus.html?action=click\&pgtype=Article\&state=default\&region=MAIN_CONTENT_3\&context=storylines_faq}{stimulus
    bills enacted in March} offer help for the millions of American
    small businesses. Those eligible for aid are businesses and
    nonprofit organizations with fewer than 500 workers, including sole
    proprietorships, independent contractors and freelancers. Some
    larger companies in some industries are also eligible. The help
    being offered, which is being managed by the Small Business
    Administration, includes the Paycheck Protection Program and the
    Economic Injury Disaster Loan program. But lots of folks have
    \href{https://www.nytimes.com/interactive/2020/05/07/business/small-business-loans-coronavirus.html?action=click\&pgtype=Article\&state=default\&region=MAIN_CONTENT_3\&context=storylines_faq}{not
    yet seen payouts.} Even those who have received help are confused:
    The rules are draconian, and some are stuck sitting on
    \href{https://www.nytimes.com/2020/05/02/business/economy/loans-coronavirus-small-business.html?action=click\&pgtype=Article\&state=default\&region=MAIN_CONTENT_3\&context=storylines_faq}{money
    they don't know how to use.} Many small-business owners are getting
    less than they expected or
    \href{https://www.nytimes.com/2020/06/10/business/Small-business-loans-ppp.html?action=click\&pgtype=Article\&state=default\&region=MAIN_CONTENT_3\&context=storylines_faq}{not
    hearing anything at all.}
  \end{itemize}
\item ~
  \hypertarget{what-are-my-rights-if-i-am-worried-about-going-back-to-work}{%
  \paragraph{What are my rights if I am worried about going back to
  work?}\label{what-are-my-rights-if-i-am-worried-about-going-back-to-work}}

  \begin{itemize}
  \tightlist
  \item
    Employers have to provide
    \href{https://www.osha.gov/SLTC/covid-19/standards.html}{a safe
    workplace} with policies that protect everyone equally.
    \href{https://www.nytimes.com/article/coronavirus-money-unemployment.html?action=click\&pgtype=Article\&state=default\&region=MAIN_CONTENT_3\&context=storylines_faq}{And
    if one of your co-workers tests positive for the coronavirus, the
    C.D.C.} has said that
    \href{https://www.cdc.gov/coronavirus/2019-ncov/community/guidance-business-response.html}{employers
    should tell their employees} -\/- without giving you the sick
    employee's name -\/- that they may have been exposed to the virus.
  \end{itemize}
\item ~
  \hypertarget{should-i-refinance-my-mortgage}{%
  \paragraph{Should I refinance my
  mortgage?}\label{should-i-refinance-my-mortgage}}

  \begin{itemize}
  \tightlist
  \item
    \href{https://www.nytimes.com/article/coronavirus-money-unemployment.html?action=click\&pgtype=Article\&state=default\&region=MAIN_CONTENT_3\&context=storylines_faq}{It
    could be a good idea,} because mortgage rates have
    \href{https://www.nytimes.com/2020/07/16/business/mortgage-rates-below-3-percent.html?action=click\&pgtype=Article\&state=default\&region=MAIN_CONTENT_3\&context=storylines_faq}{never
    been lower.} Refinancing requests have pushed mortgage applications
    to some of the highest levels since 2008, so be prepared to get in
    line. But defaults are also up, so if you're thinking about buying a
    home, be aware that some lenders have tightened their standards.
  \end{itemize}
\item ~
  \hypertarget{what-is-school-going-to-look-like-in-september}{%
  \paragraph{What is school going to look like in
  September?}\label{what-is-school-going-to-look-like-in-september}}

  \begin{itemize}
  \tightlist
  \item
    It is unlikely that many schools will return to a normal schedule
    this fall, requiring the grind of
    \href{https://www.nytimes.com/2020/06/05/us/coronavirus-education-lost-learning.html?action=click\&pgtype=Article\&state=default\&region=MAIN_CONTENT_3\&context=storylines_faq}{online
    learning},
    \href{https://www.nytimes.com/2020/05/29/us/coronavirus-child-care-centers.html?action=click\&pgtype=Article\&state=default\&region=MAIN_CONTENT_3\&context=storylines_faq}{makeshift
    child care} and
    \href{https://www.nytimes.com/2020/06/03/business/economy/coronavirus-working-women.html?action=click\&pgtype=Article\&state=default\&region=MAIN_CONTENT_3\&context=storylines_faq}{stunted
    workdays} to continue. California's two largest public school
    districts --- Los Angeles and San Diego --- said on July 13, that
    \href{https://www.nytimes.com/2020/07/13/us/lausd-san-diego-school-reopening.html?action=click\&pgtype=Article\&state=default\&region=MAIN_CONTENT_3\&context=storylines_faq}{instruction
    will be remote-only in the fall}, citing concerns that surging
    coronavirus infections in their areas pose too dire a risk for
    students and teachers. Together, the two districts enroll some
    825,000 students. They are the largest in the country so far to
    abandon plans for even a partial physical return to classrooms when
    they reopen in August. For other districts, the solution won't be an
    all-or-nothing approach.
    \href{https://bioethics.jhu.edu/research-and-outreach/projects/eschool-initiative/school-policy-tracker/}{Many
    systems}, including the nation's largest, New York City, are
    devising
    \href{https://www.nytimes.com/2020/06/26/us/coronavirus-schools-reopen-fall.html?action=click\&pgtype=Article\&state=default\&region=MAIN_CONTENT_3\&context=storylines_faq}{hybrid
    plans} that involve spending some days in classrooms and other days
    online. There's no national policy on this yet, so check with your
    municipal school system regularly to see what is happening in your
    community.
  \end{itemize}
\end{itemize}

``It's a double-edged sword,'' said
\href{https://www.mei.edu/experts/basma-alloush}{Basma Alloush}, policy
and advocacy adviser for the Norwegian Refugee Council, a humanitarian
organization that operates in Syria. ``If the U.S. is using such
sweeping, vast sanctions to yield some kind of political goal, they're
not paying enough attention to the unintended consequences.''

She added: ``With the U.S. adding this additional pressure on ordinary
people that have gone through hell and back, they really need to do a
lot more to demonstrate how exactly they're going to be protecting these
civilians.''

To hear the Syrian government tell it, American sanctions are the root
of all of Syria's economic problems. Strident propaganda to that effect
fills state television and the mouths of government loyalists.

``The moment the United States of America lifts the sanctions and stops
the Caesar Act, the prices of goods, commodities and foodstuffs will
drop at least 50 percent,'' Mohammed Samer al-Khalil, the economic
minister, said at a talk on July 15.

Defiance is a common response. A government employee in Damascus, Abu
Nidal, noted that Syrians had already survived punishing American
sanctions from the 1980s on without succumbing to what he called
``American hegemony.''

``These measures will never ever change the peoples' loyalty and support
for President Bashar al-Assad and his war against the terrorist groups
and to retake every inch of Syria,'' added Abu Nidal, 50, who, like most
Syrians interviewed, asked to be identified by a nickname because he
feared repercussions for speaking to a foreign journalist.

Mr. al-Assad's government, like the aid groups, insists that the
sanctions snarl the process of importing medicines, vaccines and medical
equipment. But verifying those claims has proved difficult in a country
with little transparency and an interest in vilifying the United States.

During a
\href{https://www.nytimes.com/2019/08/20/world/middleeast/syria-recovery-aleppo-douma.html}{rare
visit} to Syria by New York Times journalists last year, the government
did not allow them to visit a hospital or interview doctors about what
officials said were medical shortages caused by the sanctions that
preceded the Caesar Act.

A physician at a government hospital in Damascus, interviewed this week,
however, said he had not seen any gaps recently.

``The Syrian health minister keeps saying the Caesar Act is badly
affecting the health sector,'' said Dr. Muhammad, 45, who asked to be
identified by his first name to avoid punishment for speaking, ``but I
haven't noticed it.''

Experts note a major goal for these sanctions is to restrict the flow of
capital from Russia and Iran into Syria. The rationale is that if Mr.
al-Assad's primary backers feel the financial pain of sanctions, they
could be persuaded into helping negotiate a peace deal.

But absent a diplomatic strategy to accompany economic sanctions,
critics worry the Trump administration will not get Mr. al-Assad's
allies to change the status quo.

``As Russians will tell you, they've been sanctioned themselves --- it's
not a game changer to them,'' said
\href{https://www.wilsoncenter.org/person/alexander-bick}{Alexander
Bick}, who was director for Syria in President Barack Obama's National
Security Council. ``Ultimately, sanctions are a tool in a diplomatic
process.''

He added that any change in Syrian leadership would be ``embarrassing''
to Russia: ``It would undermine Putin's message --- which has been, `I
stand by my clients' --- and would undermine Russia's broader goal to
prevent America from changing regimes at will.''

Some scholars say the Caesar Act sanctions could be leverage to achieve
small, specific goals, such as the release of political prisoners.
Others think they could have a greater effect --- with the broad
jurisdiction Congress has provided the Trump administration helping to
achieve peace in a region that has been plagued by a brutal civil war.

But they note that only thirteen businesses and individuals that have
been sanctioned since June are actually under the new law. Most are
sanctioned under executive orders signed by Mr. Trump in October 2019.

Some question why Russian individuals and entities have not been
sanctioned, given their outsized role in the conflict.

``Does this fall into that huge area of uncertainty about how willing
the Trump White House is to take actions against Russia?'' said
\href{https://www.brookings.edu/experts/steven-heydemann/}{Steven
Heydemann}, a senior fellow at the Brookings Institution. ``It risks
calling the credibility of the administration's commitment to the
sanctions into question.''

Pranshu Verma reported from Washington, and Vivian Yee from Beirut. An
employee of The New York Times contributed reporting from Damascus.

Advertisement

\protect\hyperlink{after-bottom}{Continue reading the main story}

\hypertarget{site-index}{%
\subsection{Site Index}\label{site-index}}

\hypertarget{site-information-navigation}{%
\subsection{Site Information
Navigation}\label{site-information-navigation}}

\begin{itemize}
\tightlist
\item
  \href{https://help.nytimes.com/hc/en-us/articles/115014792127-Copyright-notice}{©~2020~The
  New York Times Company}
\end{itemize}

\begin{itemize}
\tightlist
\item
  \href{https://www.nytco.com/}{NYTCo}
\item
  \href{https://help.nytimes.com/hc/en-us/articles/115015385887-Contact-Us}{Contact
  Us}
\item
  \href{https://www.nytco.com/careers/}{Work with us}
\item
  \href{https://nytmediakit.com/}{Advertise}
\item
  \href{http://www.tbrandstudio.com/}{T Brand Studio}
\item
  \href{https://www.nytimes.com/privacy/cookie-policy\#how-do-i-manage-trackers}{Your
  Ad Choices}
\item
  \href{https://www.nytimes.com/privacy}{Privacy}
\item
  \href{https://help.nytimes.com/hc/en-us/articles/115014893428-Terms-of-service}{Terms
  of Service}
\item
  \href{https://help.nytimes.com/hc/en-us/articles/115014893968-Terms-of-sale}{Terms
  of Sale}
\item
  \href{https://spiderbites.nytimes.com}{Site Map}
\item
  \href{https://help.nytimes.com/hc/en-us}{Help}
\item
  \href{https://www.nytimes.com/subscription?campaignId=37WXW}{Subscriptions}
\end{itemize}
