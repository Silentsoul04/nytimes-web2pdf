Sections

SEARCH

\protect\hyperlink{site-content}{Skip to
content}\protect\hyperlink{site-index}{Skip to site index}

\href{https://www.nytimes.com/section/your-money}{Your Money}

\href{https://myaccount.nytimes.com/auth/login?response_type=cookie\&client_id=vi}{}

\href{https://www.nytimes.com/section/todayspaper}{Today's Paper}

\href{/section/your-money}{Your Money}\textbar{}Interest Rates Are Low,
but Loans Are Harder to Get. Here's Why.

\url{https://nyti.ms/3k8BOK5}

\begin{itemize}
\item
\item
\item
\item
\item
\end{itemize}

\href{https://www.nytimes.com/news-event/coronavirus?action=click\&pgtype=Article\&state=default\&region=TOP_BANNER\&context=storylines_menu}{The
Coronavirus Outbreak}

\begin{itemize}
\tightlist
\item
  live\href{https://www.nytimes.com/2020/08/04/world/coronavirus-cases.html?action=click\&pgtype=Article\&state=default\&region=TOP_BANNER\&context=storylines_menu}{Latest
  Updates}
\item
  \href{https://www.nytimes.com/interactive/2020/us/coronavirus-us-cases.html?action=click\&pgtype=Article\&state=default\&region=TOP_BANNER\&context=storylines_menu}{Maps
  and Cases}
\item
  \href{https://www.nytimes.com/interactive/2020/science/coronavirus-vaccine-tracker.html?action=click\&pgtype=Article\&state=default\&region=TOP_BANNER\&context=storylines_menu}{Vaccine
  Tracker}
\item
  \href{https://www.nytimes.com/2020/08/02/us/covid-college-reopening.html?action=click\&pgtype=Article\&state=default\&region=TOP_BANNER\&context=storylines_menu}{College
  Reopening}
\item
  \href{https://www.nytimes.com/live/2020/08/04/business/stock-market-today-coronavirus?action=click\&pgtype=Article\&state=default\&region=TOP_BANNER\&context=storylines_menu}{Economy}
\end{itemize}

Advertisement

\protect\hyperlink{after-top}{Continue reading the main story}

Supported by

\protect\hyperlink{after-sponsor}{Continue reading the main story}

\hypertarget{interest-rates-are-low-but-loans-are-harder-to-get-heres-why}{%
\section{Interest Rates Are Low, but Loans Are Harder to Get. Here's
Why.}\label{interest-rates-are-low-but-loans-are-harder-to-get-heres-why}}

Banks have tightened standards, becoming more choosy about their
borrowers and asking a lot of questions.

\includegraphics{https://static01.nyt.com/images/2020/08/04/business/04borrow1/04borrow1-articleLarge-v2.jpg?quality=75\&auto=webp\&disable=upscale}

\href{https://www.nytimes.com/by/tara-siegel-bernard}{\includegraphics{https://static01.nyt.com/images/2019/01/18/multimedia/author-tara-siegel-bernard/author-tara-siegel-bernard-thumbLarge.png}}

By \href{https://www.nytimes.com/by/tara-siegel-bernard}{Tara Siegel
Bernard}

\begin{itemize}
\item
  Aug. 4, 2020
\item
  \begin{itemize}
  \item
  \item
  \item
  \item
  \item
  \end{itemize}
\end{itemize}

As public school teachers, Tori Smith and her husband have careers that
should survive the coronavirus economy, but their mortgage lender wasn't
taking any chances.

It told them that they would have to put down more money to keep the
interest rate they wanted, then dialed back what it was willing to lend
them. And Ms. Smith said it had checked their employment status several
times during the approval process --- and again a few days before the
couple closed on their home in Zebulon, N.C., last month.

Ms. Smith said she had never gotten a straight answer about the new
requirements, but she ventured a guess. ``I felt like we had to bring
more just because of Covid,'' she said.

The economic crisis caused by the pandemic has driven
\href{https://www.nytimes.com/2020/07/16/business/mortgage-rates-below-3-percent.html}{interest
rates to rock-bottom levels}, meaning there has hardly been a better
time to borrow. But with tens of millions of people out of work and
coronavirus infections surging in many parts of the country, qualifying
for a loan --- from mortgages to auto loans --- has become more trying,
even for well-positioned borrowers.

Lenders that have
\href{https://www.nytimes.com/2020/07/14/business/big-banks-quarterly-results.html}{set
aside billion of dollars} for future defaults have also tightened their
standards, often requiring higher credit scores, heftier down payments
and more documentation. Some, such as Wells Fargo and Chase, have
temporarily eliminated home equity lines of credit, while Wells Fargo
also stopped cash-out refinancing.

It's not unusual for lenders to tighten the credit reins during a
downturn, but the current situation has made it especially challenging
for them to get an accurate read on consumers' financial health.
Borrowers have been able to pause mortgages, halt student loan payments
and delay paying their tax bills, while millions of households have
received an extra \$600 weekly in unemployment benefits. Those forms of
government support could be masking an underlying condition.

``It makes it hard for a lender to understand what the consumer's true
state of credit quality is and their ability to pay back a loan,'' said
Peter Maynard, senior vice president of global data and analytics at the
Equifax credit bureau.

\hypertarget{latest-updates-global-coronavirus-outbreak}{%
\section{\texorpdfstring{\href{https://www.nytimes.com/2020/08/04/world/coronavirus-cases.html?action=click\&pgtype=Article\&state=default\&region=MAIN_CONTENT_1\&context=storylines_live_updates}{Latest
Updates: Global Coronavirus
Outbreak}}{Latest Updates: Global Coronavirus Outbreak}}\label{latest-updates-global-coronavirus-outbreak}}

Updated 2020-08-05T07:58:24.076Z

\begin{itemize}
\tightlist
\item
  \href{https://www.nytimes.com/2020/08/04/world/coronavirus-cases.html?action=click\&pgtype=Article\&state=default\&region=MAIN_CONTENT_1\&context=storylines_live_updates\#link-762df92}{As
  talks drag on, McConnell signals openness to jobless aid extension,
  and negotiators agree on a deadline.}
\item
  \href{https://www.nytimes.com/2020/08/04/world/coronavirus-cases.html?action=click\&pgtype=Article\&state=default\&region=MAIN_CONTENT_1\&context=storylines_live_updates\#link-1228a480}{Novavax
  sees encouraging results from two studies of its experimental
  vaccine.}
\item
  \href{https://www.nytimes.com/2020/08/04/world/coronavirus-cases.html?action=click\&pgtype=Article\&state=default\&region=MAIN_CONTENT_1\&context=storylines_live_updates\#link-794484ed}{Mississippians
  must now wear masks in public, governor says.}
\end{itemize}

\href{https://www.nytimes.com/2020/08/04/world/coronavirus-cases.html?action=click\&pgtype=Article\&state=default\&region=MAIN_CONTENT_1\&context=storylines_live_updates}{See
more updates}

More live coverage:
\href{https://www.nytimes.com/live/2020/08/04/business/stock-market-today-coronavirus?action=click\&pgtype=Article\&state=default\&region=MAIN_CONTENT_1\&context=storylines_live_updates}{Markets}

Credit card companies, for example, mailed out 57 million offers to
consumers in June, a historic low and down from 272 million a year
earlier, according to Mintel, a research firm that has been tracking the
offers since 1999. Some banks have stopped offering the types of cards
that attract people who may be focused on paying down debt, such as
BankAmericard, Mintel found.

Issuers are also being careful with cards belonging to current
customers, said Mark Miller, associate director of insights for payments
at Mintel.

``Some dormant accounts are being closed,'' he said. ``So if they have a
credit card sitting in a drawer, those accounts are at risk of being
closed, and credit lines with a \$10,000 limit may eventually be knocked
down to \$8,000.''

For auto loans, borrowers with lower credit scores and thin credit
histories face more rigorous requirements and less generous terms,
including shorter loan periods.

``Subprime borrowers are not getting loans as readily as they were
pre-pandemic or a year ago,'' said Jonathan Smoke, chief economist at
Cox Automotive, referring to consumers with credit scores below 620.

Interest rates for new and used vehicles remain low --- below 4 percent
at many banks and credit unions --- but only for more qualified
borrowers, said Greg McBride, chief financial analyst at BankRate.com.

``Good credit and a down payment are required to get the best rates,
with weaker credit increasingly sidelined --- particularly for
older-model used car purchases,'' he said.

Ford Motor said it hadn't tightened standards on loans through its
financing unit, but last month it introduced a program to make wary
borrowers more comfortable. Those who buy or lease a car through Ford's
financing unit before Sept. 30 can return it within a year if they lose
their jobs. Ford said it would reduce the customer's balance by the
vehicle's book value, and then waive up to an additional \$15,000.

If that measure is meant to stoke demand, no such program is necessary
for home buyers.

For the first time in nearly half a century of tracking, 30-year
fixed-rate mortgages averaged about 2.98 percent, according to
\href{https://freddiemac.gcs-web.com/news-releases/news-release-details/mortgage-rates-fall-below-three-percent}{Freddie
Mac}. The mortgage industry made \$865 billion in loans during the
second quarter, the highest amount since 2003, when quarterly
originations twice topped \$1 trillion, according to Inside Mortgage
Finance, a trade publication.

And that's with lenders being picky about their customers and particular
about their requirements. JPMorgan Chase, for example, will make
mortgages to new customers only with credit scores of 700 or more (up
from 640) and down payments of 20 percent or higher. USAA has
temporarily stopped writing jumbo loans, which are mortgages that are
generally too large to be backed by the federal government, among other
products. Bank of America said it had also tightened its underwriting,
but declined to provide details.

Ms. Smith and her husband, Philip Ellis, had hoped to go through a
first-time homebuyer program at Wells Fargo that would require them to
put down 3 percent. They even sat through a required educational course.
But two weeks before closing on their \$205,000 home, their lending
officer said they needed to put down 5 percent to keep their rate.

\href{https://www.nytimes.com/news-event/coronavirus?action=click\&pgtype=Article\&state=default\&region=MAIN_CONTENT_3\&context=storylines_faq}{}

\hypertarget{the-coronavirus-outbreak-}{%
\subsubsection{The Coronavirus Outbreak
›}\label{the-coronavirus-outbreak-}}

\hypertarget{frequently-asked-questions}{%
\paragraph{Frequently Asked
Questions}\label{frequently-asked-questions}}

Updated August 4, 2020

\begin{itemize}
\item ~
  \hypertarget{i-have-antibodies-am-i-now-immune}{%
  \paragraph{I have antibodies. Am I now
  immune?}\label{i-have-antibodies-am-i-now-immune}}

  \begin{itemize}
  \tightlist
  \item
    As of right
    now,\href{https://www.nytimes.com/2020/07/22/health/covid-antibodies-herd-immunity.html?action=click\&pgtype=Article\&state=default\&region=MAIN_CONTENT_3\&context=storylines_faq}{that
    seems likely, for at least several months.} There have been
    frightening accounts of people suffering what seems to be a second
    bout of Covid-19. But experts say these patients may have a
    drawn-out course of infection, with the virus taking a slow toll
    weeks to months after initial exposure. People infected with the
    coronavirus typically
    \href{https://www.nature.com/articles/s41586-020-2456-9}{produce}
    immune molecules called antibodies, which are
    \href{https://www.nytimes.com/2020/05/07/health/coronavirus-antibody-prevalence.html?action=click\&pgtype=Article\&state=default\&region=MAIN_CONTENT_3\&context=storylines_faq}{protective
    proteins made in response to an
    infection}\href{https://www.nytimes.com/2020/05/07/health/coronavirus-antibody-prevalence.html?action=click\&pgtype=Article\&state=default\&region=MAIN_CONTENT_3\&context=storylines_faq}{.
    These antibodies may} last in the body
    \href{https://www.nature.com/articles/s41591-020-0965-6}{only two to
    three months}, which may seem worrisome, but that's perfectly normal
    after an acute infection subsides, said Dr. Michael Mina, an
    immunologist at Harvard University. It may be possible to get the
    coronavirus again, but it's highly unlikely that it would be
    possible in a short window of time from initial infection or make
    people sicker the second time.
  \end{itemize}
\item ~
  \hypertarget{im-a-small-business-owner-can-i-get-relief}{%
  \paragraph{I'm a small-business owner. Can I get
  relief?}\label{im-a-small-business-owner-can-i-get-relief}}

  \begin{itemize}
  \tightlist
  \item
    The
    \href{https://www.nytimes.com/article/small-business-loans-stimulus-grants-freelancers-coronavirus.html?action=click\&pgtype=Article\&state=default\&region=MAIN_CONTENT_3\&context=storylines_faq}{stimulus
    bills enacted in March} offer help for the millions of American
    small businesses. Those eligible for aid are businesses and
    nonprofit organizations with fewer than 500 workers, including sole
    proprietorships, independent contractors and freelancers. Some
    larger companies in some industries are also eligible. The help
    being offered, which is being managed by the Small Business
    Administration, includes the Paycheck Protection Program and the
    Economic Injury Disaster Loan program. But lots of folks have
    \href{https://www.nytimes.com/interactive/2020/05/07/business/small-business-loans-coronavirus.html?action=click\&pgtype=Article\&state=default\&region=MAIN_CONTENT_3\&context=storylines_faq}{not
    yet seen payouts.} Even those who have received help are confused:
    The rules are draconian, and some are stuck sitting on
    \href{https://www.nytimes.com/2020/05/02/business/economy/loans-coronavirus-small-business.html?action=click\&pgtype=Article\&state=default\&region=MAIN_CONTENT_3\&context=storylines_faq}{money
    they don't know how to use.} Many small-business owners are getting
    less than they expected or
    \href{https://www.nytimes.com/2020/06/10/business/Small-business-loans-ppp.html?action=click\&pgtype=Article\&state=default\&region=MAIN_CONTENT_3\&context=storylines_faq}{not
    hearing anything at all.}
  \end{itemize}
\item ~
  \hypertarget{what-are-my-rights-if-i-am-worried-about-going-back-to-work}{%
  \paragraph{What are my rights if I am worried about going back to
  work?}\label{what-are-my-rights-if-i-am-worried-about-going-back-to-work}}

  \begin{itemize}
  \tightlist
  \item
    Employers have to provide
    \href{https://www.osha.gov/SLTC/covid-19/standards.html}{a safe
    workplace} with policies that protect everyone equally.
    \href{https://www.nytimes.com/article/coronavirus-money-unemployment.html?action=click\&pgtype=Article\&state=default\&region=MAIN_CONTENT_3\&context=storylines_faq}{And
    if one of your co-workers tests positive for the coronavirus, the
    C.D.C.} has said that
    \href{https://www.cdc.gov/coronavirus/2019-ncov/community/guidance-business-response.html}{employers
    should tell their employees} -\/- without giving you the sick
    employee's name -\/- that they may have been exposed to the virus.
  \end{itemize}
\item ~
  \hypertarget{should-i-refinance-my-mortgage}{%
  \paragraph{Should I refinance my
  mortgage?}\label{should-i-refinance-my-mortgage}}

  \begin{itemize}
  \tightlist
  \item
    \href{https://www.nytimes.com/article/coronavirus-money-unemployment.html?action=click\&pgtype=Article\&state=default\&region=MAIN_CONTENT_3\&context=storylines_faq}{It
    could be a good idea,} because mortgage rates have
    \href{https://www.nytimes.com/2020/07/16/business/mortgage-rates-below-3-percent.html?action=click\&pgtype=Article\&state=default\&region=MAIN_CONTENT_3\&context=storylines_faq}{never
    been lower.} Refinancing requests have pushed mortgage applications
    to some of the highest levels since 2008, so be prepared to get in
    line. But defaults are also up, so if you're thinking about buying a
    home, be aware that some lenders have tightened their standards.
  \end{itemize}
\item ~
  \hypertarget{what-is-school-going-to-look-like-in-september}{%
  \paragraph{What is school going to look like in
  September?}\label{what-is-school-going-to-look-like-in-september}}

  \begin{itemize}
  \tightlist
  \item
    It is unlikely that many schools will return to a normal schedule
    this fall, requiring the grind of
    \href{https://www.nytimes.com/2020/06/05/us/coronavirus-education-lost-learning.html?action=click\&pgtype=Article\&state=default\&region=MAIN_CONTENT_3\&context=storylines_faq}{online
    learning},
    \href{https://www.nytimes.com/2020/05/29/us/coronavirus-child-care-centers.html?action=click\&pgtype=Article\&state=default\&region=MAIN_CONTENT_3\&context=storylines_faq}{makeshift
    child care} and
    \href{https://www.nytimes.com/2020/06/03/business/economy/coronavirus-working-women.html?action=click\&pgtype=Article\&state=default\&region=MAIN_CONTENT_3\&context=storylines_faq}{stunted
    workdays} to continue. California's two largest public school
    districts --- Los Angeles and San Diego --- said on July 13, that
    \href{https://www.nytimes.com/2020/07/13/us/lausd-san-diego-school-reopening.html?action=click\&pgtype=Article\&state=default\&region=MAIN_CONTENT_3\&context=storylines_faq}{instruction
    will be remote-only in the fall}, citing concerns that surging
    coronavirus infections in their areas pose too dire a risk for
    students and teachers. Together, the two districts enroll some
    825,000 students. They are the largest in the country so far to
    abandon plans for even a partial physical return to classrooms when
    they reopen in August. For other districts, the solution won't be an
    all-or-nothing approach.
    \href{https://bioethics.jhu.edu/research-and-outreach/projects/eschool-initiative/school-policy-tracker/}{Many
    systems}, including the nation's largest, New York City, are
    devising
    \href{https://www.nytimes.com/2020/06/26/us/coronavirus-schools-reopen-fall.html?action=click\&pgtype=Article\&state=default\&region=MAIN_CONTENT_3\&context=storylines_faq}{hybrid
    plans} that involve spending some days in classrooms and other days
    online. There's no national policy on this yet, so check with your
    municipal school system regularly to see what is happening in your
    community.
  \end{itemize}
\end{itemize}

A week later, Ms. Smith said, they learned their loan was for less than
what they had been preapproved for --- and they needed to come up with
an additional \$4,000. In the end, their down payment and closing costs
exceeded \$14,000 --- about 45 percent more than they had anticipated.

The couple, who had married in April, used money recovered from their
canceled wedding reception. Ms. Smith said they were also lucky to have
the support of their families, who fed and sheltered them so they could
save every penny. But the stability of their jobs was also most likely a
crucial factor.

``I think our ability to secure the loan was due to us both being
schoolteachers and having a contract for employment already for the
following year,'' she said.

Wells Fargo said it hadn't increased its credit score requirements, but
it has raised down-payment minimums on certain loans not backed by the
government because it had to suspend most interior appraisals of homes
during the pandemic. Even under normal circumstances, there are a
variety of situations in which borrowers may be asked to raise their
down payment or obtain a better rate by doing so, a company spokesman
said.

Some lenders also want to know more about borrowers' other possible
sources of cash.

When Chris Eberle, a technology executive, and his wife were locking in
their jumbo mortgage for a new home in Palo Alto, Calif., their lender,
a California mortgage bank, wanted to know not only how much they had in
their retirement accounts but how easy it was to get at that money.

``They wanted, account by account, details on the withdrawal and loan
options,'' Mr. Eberle said.

And they, too, had to put down more than they had planned. Before the
crisis, a jumbo loan could be had with 10 percent down. Mr. Eberle said
they had to put down 20 percent --- and found a cheaper house to make it
easier.

Other borrowers, including the self-employed, are being asked to provide
more detailed proof of their earnings --- at least when they're getting
a loan that will be backed by
\href{https://singlefamily.fanniemae.com/media/22316/display?_ga=2.27818620.1955261175.1593457830-553926868.1583167746\&_gac=1.52708252.1591220517.EAIaIQobChMIyYH3gc7m6QIVlovICh2LUgv0EAMYASAAEgJkPvD_BwE}{Fannie
Mae} or
\href{https://guide.freddiemac.com/app/guide/bulletin/2020-19}{Freddie
Mac}.

``Employment and income verification for self-employed borrowers is now
multiple times more detailed as it previously was,'' said Ted Rood, a
loan officer in St. Louis who lends nationally.

Income verification is also more rigorous across the board, and Mr. Rood
said he was required to do two verifications over the phone. It makes
sense, he said: He had just prepared a loan for a married couple --- a
gym owner whose income had suffered and his wife, a speech therapist
with a seemingly more stable position because she was able to work with
clients remotely.

``We were set to close on a Monday in early June,'' said Mr. Rood, who
was working at Bayshore Mortgage Funding, which is based in Timonium,
Md., at the time. But when the loan processor called the wife's employer
the Friday before, the processor learned that the woman had been laid
off.

The lender withdrew the loan.

Advertisement

\protect\hyperlink{after-bottom}{Continue reading the main story}

\hypertarget{site-index}{%
\subsection{Site Index}\label{site-index}}

\hypertarget{site-information-navigation}{%
\subsection{Site Information
Navigation}\label{site-information-navigation}}

\begin{itemize}
\tightlist
\item
  \href{https://help.nytimes.com/hc/en-us/articles/115014792127-Copyright-notice}{©~2020~The
  New York Times Company}
\end{itemize}

\begin{itemize}
\tightlist
\item
  \href{https://www.nytco.com/}{NYTCo}
\item
  \href{https://help.nytimes.com/hc/en-us/articles/115015385887-Contact-Us}{Contact
  Us}
\item
  \href{https://www.nytco.com/careers/}{Work with us}
\item
  \href{https://nytmediakit.com/}{Advertise}
\item
  \href{http://www.tbrandstudio.com/}{T Brand Studio}
\item
  \href{https://www.nytimes.com/privacy/cookie-policy\#how-do-i-manage-trackers}{Your
  Ad Choices}
\item
  \href{https://www.nytimes.com/privacy}{Privacy}
\item
  \href{https://help.nytimes.com/hc/en-us/articles/115014893428-Terms-of-service}{Terms
  of Service}
\item
  \href{https://help.nytimes.com/hc/en-us/articles/115014893968-Terms-of-sale}{Terms
  of Sale}
\item
  \href{https://spiderbites.nytimes.com}{Site Map}
\item
  \href{https://help.nytimes.com/hc/en-us}{Help}
\item
  \href{https://www.nytimes.com/subscription?campaignId=37WXW}{Subscriptions}
\end{itemize}
