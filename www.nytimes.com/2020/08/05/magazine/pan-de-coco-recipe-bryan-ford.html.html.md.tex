Sections

SEARCH

\protect\hyperlink{site-content}{Skip to
content}\protect\hyperlink{site-index}{Skip to site index}

\href{https://myaccount.nytimes.com/auth/login?response_type=cookie\&client_id=vi}{}

\href{https://www.nytimes.com/section/todayspaper}{Today's Paper}

These Rolls Will Change the Way You See Sourdough

\href{https://nyti.ms/31ig5a5}{https://nyti.ms/31ig5a5}

\begin{itemize}
\item
\item
\item
\item
\item
\end{itemize}

\href{https://www.nytimes.com/spotlight/at-home?action=click\&pgtype=Article\&state=default\&region=TOP_BANNER\&context=at_home_menu}{At
Home}

\begin{itemize}
\tightlist
\item
  \href{https://www.nytimes.com/2020/08/04/arts/television/sam-jay-netflix-special.html?action=click\&pgtype=Article\&state=default\&region=TOP_BANNER\&context=at_home_menu}{Watch:
  Sam Jay}
\item
  \href{https://www.nytimes.com/interactive/2020/at-home/even-more-reporters-editors-diaries-lists-recommendations.html?action=click\&pgtype=Article\&state=default\&region=TOP_BANNER\&context=at_home_menu}{Peruse:
  Reporters' Google Docs}
\item
  \href{https://www.nytimes.com/2020/08/04/dining/colombian-empanadas-carlos-gaviria.html?action=click\&pgtype=Article\&state=default\&region=TOP_BANNER\&context=at_home_menu}{Make:
  Empanadas}
\item
  \href{https://www.nytimes.com/2020/08/06/arts/design/street-art-nyc-george-floyd.html?action=click\&pgtype=Article\&state=default\&region=TOP_BANNER\&context=at_home_menu}{Explore:
  N.Y.C. Street Art}
\end{itemize}

Advertisement

\protect\hyperlink{after-top}{Continue reading the main story}

Supported by

\protect\hyperlink{after-sponsor}{Continue reading the main story}

\href{/column/magazine-eat}{Eat}

\hypertarget{these-rolls-will-change-the-way-you-see-sourdough}{%
\section{These Rolls Will Change the Way You See
Sourdough}\label{these-rolls-will-change-the-way-you-see-sourdough}}

\includegraphics{https://static01.nyt.com/images/2020/08/09/magazine/09mag-eat/09mag-eat-articleLarge.jpg?quality=75\&auto=webp\&disable=upscale}

By \href{https://www.nytimes.com/by/tejal-rao}{Tejal Rao}

\begin{itemize}
\item
  Aug. 5, 2020
\item
  \begin{itemize}
  \item
  \item
  \item
  \item
  \item
  \end{itemize}
\end{itemize}

It's not that Bryan Ford didn't love those tall, quintessentially
crusty, flour-dusted, rustic French and Italian sourdough loaves --- the
kind you've seen cross-sectioned and shot from every angle on bread
blogs and in cookbooks and on Instagram. The kind an algorithm may have
even directed you toward with a far higher frequency since the pandemic
pushed more home cooks to care for sourdough starters. He loved those
breads! But Ford, \href{https://www.instagram.com/artisanbryan/}{a
Honduran-American baker from New Orleans}, also wondered why other
breads weren't valued in the same way, and why other doughs, especially
doughs that predated the sourdough fad, and that went through their own
processes of wild fermentations in home kitchens all over the world,
were left out of the conversation.

The slow, overnight fermentation of \emph{dhokla} leads to an airy,
tangy batter, steamed so it's very tender. But until I read
\href{https://www.artisanbryan.com/}{Ford's work}, it never even
occurred to me to refer to it, or to so many of the other everyday
fermented batters I grew up making and eating as ``sourdoughs.'' I
couldn't explain why, though maybe it's because their techniques and
their visual languages were just so different from the wide, open crumbs
and crackling edges fetishized on Instagram. And maybe, if I'm being
honest, it's because I didn't think I was allowed --- sourdough,
intentionally or not, has an exclusive Eurocentric definition as crusty
bread built with a starter. As a result, many of the world's great
fermented breads, from \emph{injera} to \emph{dosa}, are often left out.
But Ford didn't wait for anyone's permission to expand sourdough's
definition.

Ford grew up in New Orleans, a child of Honduran immigrants. Once a
week, sometimes more often, his father picked up a bag full of \emph{pan
de coco} from a Honduran bodega, and Ford would grab one or two of the
dense little rolls from the bag and run off to eat them to tide himself
over before dinner. His parents sat on the porch to have theirs, dipping
them in coffee, talking. Honduran \emph{pan de coco}, traditionally made
with coconut milk and some whole-wheat flour, might be used to soak up
soup or sauce with a meal, or eaten plain as a snack. ``It's such a
beautiful thing,'' Ford said, ``and for me, that is good bread.''

In 2018, when he was working as a baker in Miami, he changed the way he
thought about sourdoughs, expanding it to include fermentations from all
over the world and applying the word to breads that had most likely
benefited from natural leavening in warm kitchens in the past. ``Before,
I'd been posting rustic loaves, baguettes, crumb shots, all the
same-looking thing,'' Ford said. ``I was getting trapped in that
mentality.'' But when his mother came to visit, Ford baked \emph{pan de
coco} with a sourdough starter instead of yeast, complicating the
bread's flavors and changing its texture. He documented the process with
just as much care as he had before. ``People saw I was proud to be
Honduran. My following grew.'' Soon, Ford started posting his bread
recipes in English and Spanish, fielding questions from home bakers all
over the world about doughs not rising properly, about caring for a
healthy starter and about the abyss between their homemade loaves and
the ones posted by professional bakers on Instagram.

Ford's cookbook, \href{https://www.artisanbryan.com/cookbook}{``New
World Sourdough,''} published in June, is full of deep expertise that
answers many of these questions, but it's also an unusually warm,
friendly invitation to making sourdough bread, a subgenre of the baking
world that isn't known for being so inclusive and approachable. In the
introduction, Ford writes that he wants bakers to change their
expectations of bread. ``I really really mean it,'' Ford said. ``People
get into baking bread with an idea of what it's supposed to be, but when
you lose those expectations, you can make a roti or naan or
\emph{semita}, and you can appreciate it just as much.''

The first recipe I made from the book was a version of Ford's \emph{pan
de coco}, sweet, mottled brown with cocoa powder and chocolate chips,
the tin greased with coconut oil. As it baked, it filled my kitchen with
the rich smell of coconut. It came out of the oven airy, pulling apart
with threads that let out puffs of steam, smearing my fingertips with
melted chocolate. But it cooled to a more dense and wholesome texture. I
wasn't sure if I got it right, if my starter was in a good place when I
used it and if this bread was the way Ford intended it to be. Had I
failed? I wanted to show him a photo of the bread, the way it looked
when it was risen, the way it looked when I tore it open and ate it,
standing in front of the oven. But I didn't need to. ``When someone
messages me about a failure, I'll always ask, well did you share it? Did
you like it, did your friends and family like it? OK, then be proud of
making delicious bread!''

Recipe:
\href{https://cooking.nytimes.com/recipes/1021315-choco-pan-de-coco}{Choco
Pan de Coco}

Advertisement

\protect\hyperlink{after-bottom}{Continue reading the main story}

\hypertarget{site-index}{%
\subsection{Site Index}\label{site-index}}

\hypertarget{site-information-navigation}{%
\subsection{Site Information
Navigation}\label{site-information-navigation}}

\begin{itemize}
\tightlist
\item
  \href{https://help.nytimes.com/hc/en-us/articles/115014792127-Copyright-notice}{©~2020~The
  New York Times Company}
\end{itemize}

\begin{itemize}
\tightlist
\item
  \href{https://www.nytco.com/}{NYTCo}
\item
  \href{https://help.nytimes.com/hc/en-us/articles/115015385887-Contact-Us}{Contact
  Us}
\item
  \href{https://www.nytco.com/careers/}{Work with us}
\item
  \href{https://nytmediakit.com/}{Advertise}
\item
  \href{http://www.tbrandstudio.com/}{T Brand Studio}
\item
  \href{https://www.nytimes.com/privacy/cookie-policy\#how-do-i-manage-trackers}{Your
  Ad Choices}
\item
  \href{https://www.nytimes.com/privacy}{Privacy}
\item
  \href{https://help.nytimes.com/hc/en-us/articles/115014893428-Terms-of-service}{Terms
  of Service}
\item
  \href{https://help.nytimes.com/hc/en-us/articles/115014893968-Terms-of-sale}{Terms
  of Sale}
\item
  \href{https://spiderbites.nytimes.com}{Site Map}
\item
  \href{https://help.nytimes.com/hc/en-us}{Help}
\item
  \href{https://www.nytimes.com/subscription?campaignId=37WXW}{Subscriptions}
\end{itemize}
