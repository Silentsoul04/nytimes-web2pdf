Sections

SEARCH

\protect\hyperlink{site-content}{Skip to
content}\protect\hyperlink{site-index}{Skip to site index}

\href{https://myaccount.nytimes.com/auth/login?response_type=cookie\&client_id=vi}{}

\href{https://www.nytimes.com/section/todayspaper}{Today's Paper}

\href{/section/opinion}{Opinion}\textbar{}Why Did Lebanon Let a
Bomb-in-Waiting Sit in a Warehouse for 6 Years?

\href{https://nyti.ms/3fuIXAQ}{https://nyti.ms/3fuIXAQ}

\begin{itemize}
\item
\item
\item
\item
\item
\end{itemize}

Beirut Explosion

\begin{itemize}
\tightlist
\item
  \href{https://www.nytimes.com/2020/08/05/world/middleeast/beirut-explosion-what-happened.html?action=click\&pgtype=Article\&state=default\&region=TOP_BANNER\&context=storylines_menu}{What
  We Know}
\item
  \href{https://www.nytimes.com/2020/08/05/video/beirut-explosion-footage.html?action=click\&pgtype=Article\&state=default\&region=TOP_BANNER\&context=storylines_menu}{Footage
  of the Blast}
\item
  \href{https://www.nytimes.com/2020/08/05/world/middleeast/beirut-explosion-ammonium-nitrate.html?action=click\&pgtype=Article\&state=default\&region=TOP_BANNER\&context=storylines_menu}{What
  is Ammonium Nitrate?}
\item
  \href{https://www.nytimes.com/interactive/2020/08/04/world/middleeast/beirut-explosion-damage.html?action=click\&pgtype=Article\&state=default\&region=TOP_BANNER\&context=storylines_menu}{Mapping
  the Damage}
\end{itemize}

Advertisement

\protect\hyperlink{after-top}{Continue reading the main story}

\href{/section/opinion}{Opinion}

Supported by

\protect\hyperlink{after-sponsor}{Continue reading the main story}

\hypertarget{why-did-lebanon-let-a-bomb-in-waiting-sit-in-a-warehouse-for-6-years}{%
\section{Why Did Lebanon Let a Bomb-in-Waiting Sit in a Warehouse for 6
Years?}\label{why-did-lebanon-let-a-bomb-in-waiting-sit-in-a-warehouse-for-6-years}}

Yesterday's explosion, which destroyed Beirut's port, much of the city
and countless lives, was the result of business as usual.

By Faysal Itani

Mr. Itani is a political analyst.

\begin{itemize}
\item
  Aug. 5, 2020
\item
  \begin{itemize}
  \item
  \item
  \item
  \item
  \item
  \end{itemize}
\end{itemize}

\includegraphics{https://static01.nyt.com/images/2020/09/04/opinion/04-Itani-secondimage/04-Itani-secondimage-articleLarge.jpg?quality=75\&auto=webp\&disable=upscale}

My first summer job was at the port of Beirut. It was the late '90s and
I was just a teenager. I spent muggy months entering shipping data as
part of an ambitious new program to move the port from analog to digital
log keeping. It was as unglamorous as you would expect from a
bottom-rung job in the bowels of a Middle East bureaucracy. But despite
the heat and the monotony, there was optimism.

The port was critical infrastructure in an economy rejuvenating after 15
years of civil war. Digital log keeping was part of the future --- and
an attempt to introduce much-needed order and transparency to a
recovering public sector. This was, after all, the same port that had
been rendered unusable during the civil war by sunken vessels and
unexploded ordnance, save for one area controlled by a militia.

The Lebanon that emerged from that rubble is gone, gradually choked by a
cynical political class. Yesterday, it was finished off. The
\href{https://www.nytimes.com/2020/08/05/world/middleeast/beirut-explosion.html}{port
of Beirut was blown up in an explosion} that killed at least 100 people
(and counting), wounded more than 4,000 and destroyed blocks of the
city. Lebanon now faces a new type of catastrophe for which decades of
war and political instability were poor preparation.

By all appearances the port disaster did not involve the usual suspects
--- Hezbollah, Israel, jihadist terrorism or the government of
neighboring Syria. The truth seems to be both duller and more
disturbing: Decades of rot at every level of Lebanon's institutions
destroyed Beirut's port, much of the city, and far too many lives. It is
precisely the banality behind the explosion that captures the particular
punishment and humiliation heaped on Lebanon.

So far, Lebanese officials are in agreement about what happened, though
it's likely that more than one ``official'' account will emerge. After
all, this is Lebanon, a country deeply divided by politics, religion and
history. But here is what we know as of now, according to reporting by
credible Lebanese media: Some 2,750 tons of ammonium nitrate unloaded
from a disabled vessel in 2014 had been stored in a port warehouse. Then
yesterday, a welding accident ignited nearby fireworks --- which caused
the ammonium nitrate to explode.

Ports are prime real estate for political, criminal and militia
factions. Multiple security agencies with different levels of competence
(and different political allegiances) control various aspects of their
operations. And recruitment in the civilian bureaucracy is dictated by
political or sectarian quotas. There is a pervasive culture of
negligence, petty corruption and blame-shifting endemic to the Lebanese
bureaucracy, all overseen by a political class defined by its
incompetence and contempt for the public good.

It's unclear what combination of these elements let a bomb-in-waiting
sit in a warehouse for almost six years, moved fireworks next to it and
allowed irresponsible work practices to be carried out nearby. But the
catastrophe, while exceptionally severe, is the result of business as
usual in Lebanon. The country is
\href{https://www.nytimes.com/reuters/2020/08/04/world/middleeast/04reuters-lebanon-security-blast-timeline.html}{familiar
with explosions}, and it is just as familiar with disasters caused by
failures of public services: a garbage crisis that
\href{https://www.nytimes.com/2015/07/28/world/middleeast/lebanese-seethe-stinking-garbage-piles-grow-beirut.html}{dates
back to 2015}, an environmental
\href{https://www.nytimes.com/2019/10/19/opinion/international-world/lebanon-is-on-fire.html}{catastrophe}
in 2019 and power outages this year that last up to 20 hours a day.

The consequences of yesterday's explosion will be even more serious than
the immediate casualties and property damage. The main grain silo, which
holds some
\href{https://www.spglobal.com/platts/en/market-insights/latest-news/agriculture/080420-explosion-at-port-of-beirut-damages-grain-silos-terminal-reports}{85
percent} of the country's cereals, was destroyed. Even more, the port
will no longer be able to receive goods. Lebanon imports
\href{https://www.forbes.com/sites/tatianakoffman/2020/07/09/lebanons-currency-crisis-paves-the-way-to-a-new-future/\#69ede21a6a17}{80
percent} of what it consumes, including 90 percent of its wheat, which
is used to make the bread that is the staple of most people's diets.
About 60 percent of those imports come through the port of Beirut. Or,
at least, they did.

The timing couldn't be worse. An economic crisis has devastated Lebanon
for several months. The country's currency has collapsed, a problem that
is itself a result of years of mismanagement and corruption. Hundreds of
thousands of people can no longer buy fuel, food and medicine. As
Lebanese have seen their savings wiped out and their purchasing power
disappear, a new vocabulary appeared among even my optimistic Lebanese
friends and family. To describe the country, they began using words like
``doomed'' and ``hopeless.''

And the coronavirus crisis has placed greater pressure on the health
sector. After yesterday's explosion, hospital staff were reportedly
treating injuries in streets and parking lots. The explosion may well
put Lebanon on the path to a food and health catastrophe not seen in the
worst of its wars.

Lebanon's political class should be on guard in the weeks ahead: Shock
will inevitably turn to anger. But I fear that old habits die hard.
These politicians are well practiced in shifting the blame. I don't
expect many --- if any --- high-level resignations or admissions of
responsibility.

Will there be a revolution? An uprising of anger? Any revolutionary
impulse has to compete with tribal, sectarian and ideological
affiliations. For that matter, so do the facts: Even if a single
official version of the port incident is presented (and even if it is
true), some will not believe it. Paradoxically, our distrust of our
politicians makes it harder to unite against them.

These are real obstacles. Yet there has never been more urgency for
reform and accountability, beyond the likely scapegoating of midlevel
officials. It is difficult to imagine such a concerted, sustained
national movement because it has never materialized. But hunger and a
collapse in health care may change that.

Lebanon --- and the Lebanese --- will need a rapid influx of external
aid to stave off a critical food shortage and public health catastrophe.
It seems to be coming, from countries across the Middle East and around
the world. But this will not arrest the country's decline. Emergency aid
will only magnify public humiliation and helplessness. Yesterday's
explosion made clear that Lebanon is no longer a country where decent
people can live secure and fulfilling lives.

As I watched videos of Beirut engulfed in smoke and checked in with my
friends and family, I found myself thinking for the first time in a
while of that summer when I worked at the port. The digitization project
was completed, but parties who disliked the transparency it brought
found ways to work around it.

Today, it's irrelevant, of course. The port is destroyed. As for the
Lebanese, they will be far more consumed by survival than progress.

Faysal Itani is a deputy director at the Center for Global Policy and
adjunct professor of Middle East politics at Georgetown University.

\emph{The Times is committed to publishing}
\href{https://www.nytimes.com/2019/01/31/opinion/letters/letters-to-editor-new-york-times-women.html}{\emph{a
diversity of letters}} \emph{to the editor. We'd like to hear what you
think about this or any of our articles. Here are some}
\href{https://help.nytimes.com/hc/en-us/articles/115014925288-How-to-submit-a-letter-to-the-editor}{\emph{tips}}\emph{.
And here's our email:}
\href{mailto:letters@nytimes.com}{\emph{letters@nytimes.com}}\emph{.}

\emph{Follow The New York Times Opinion section on}
\href{https://www.facebook.com/nytopinion}{\emph{Facebook}}\emph{,}
\href{http://twitter.com/NYTOpinion}{\emph{Twitter (@NYTopinion)}}
\emph{and}
\href{https://www.instagram.com/nytopinion/}{\emph{Instagram}}\emph{.}

Advertisement

\protect\hyperlink{after-bottom}{Continue reading the main story}

\hypertarget{site-index}{%
\subsection{Site Index}\label{site-index}}

\hypertarget{site-information-navigation}{%
\subsection{Site Information
Navigation}\label{site-information-navigation}}

\begin{itemize}
\tightlist
\item
  \href{https://help.nytimes.com/hc/en-us/articles/115014792127-Copyright-notice}{©~2020~The
  New York Times Company}
\end{itemize}

\begin{itemize}
\tightlist
\item
  \href{https://www.nytco.com/}{NYTCo}
\item
  \href{https://help.nytimes.com/hc/en-us/articles/115015385887-Contact-Us}{Contact
  Us}
\item
  \href{https://www.nytco.com/careers/}{Work with us}
\item
  \href{https://nytmediakit.com/}{Advertise}
\item
  \href{http://www.tbrandstudio.com/}{T Brand Studio}
\item
  \href{https://www.nytimes.com/privacy/cookie-policy\#how-do-i-manage-trackers}{Your
  Ad Choices}
\item
  \href{https://www.nytimes.com/privacy}{Privacy}
\item
  \href{https://help.nytimes.com/hc/en-us/articles/115014893428-Terms-of-service}{Terms
  of Service}
\item
  \href{https://help.nytimes.com/hc/en-us/articles/115014893968-Terms-of-sale}{Terms
  of Sale}
\item
  \href{https://spiderbites.nytimes.com}{Site Map}
\item
  \href{https://help.nytimes.com/hc/en-us}{Help}
\item
  \href{https://www.nytimes.com/subscription?campaignId=37WXW}{Subscriptions}
\end{itemize}
