Sections

SEARCH

\protect\hyperlink{site-content}{Skip to
content}\protect\hyperlink{site-index}{Skip to site index}

\href{/section/opinion}{Opinion}\textbar{}It Sounded Like the World
Itself Was Breaking Open

\href{https://nyti.ms/33rYK17}{https://nyti.ms/33rYK17}

\begin{itemize}
\item
\item
\item
\item
\item
\end{itemize}

Beirut Explosion

\begin{itemize}
\tightlist
\item
  \href{https://www.nytimes.com/2020/08/05/world/middleeast/beirut-explosion-what-happened.html?action=click\&pgtype=Article\&state=default\&region=TOP_BANNER\&context=storylines_menu}{What
  We Know}
\item
  \href{https://www.nytimes.com/2020/08/05/video/beirut-explosion-footage.html?action=click\&pgtype=Article\&state=default\&region=TOP_BANNER\&context=storylines_menu}{Footage
  of the Blast}
\item
  \href{https://www.nytimes.com/2020/08/05/world/middleeast/beirut-explosion-ammonium-nitrate.html?action=click\&pgtype=Article\&state=default\&region=TOP_BANNER\&context=storylines_menu}{What
  is Ammonium Nitrate?}
\item
  \href{https://www.nytimes.com/interactive/2020/08/04/world/middleeast/beirut-explosion-damage.html?action=click\&pgtype=Article\&state=default\&region=TOP_BANNER\&context=storylines_menu}{Mapping
  the Damage}
\end{itemize}

\includegraphics{https://static01.nyt.com/images/2020/08/05/opinion/05mounzer1-Sub/merlin_175324842_fa957286-8d41-42be-978c-0f52c4b8df90-articleLarge.jpg?quality=75\&auto=webp\&disable=upscale}

\href{/section/opinion}{Opinion}

\hypertarget{it-sounded-like-the-world-itself-was-breaking-open}{%
\section{It Sounded Like the World Itself Was Breaking
Open}\label{it-sounded-like-the-world-itself-was-breaking-open}}

After the explosions at the Beirut port, an immense rage rises against
the corrupt and incompetent political class that has ruled for so long.

The aftermath of the massive explosion at the port of Beirut, in the
heart of the Lebanese capital.Credit...Agence France-Presse --- Getty
Images

Supported by

\protect\hyperlink{after-sponsor}{Continue reading the main story}

By Lina Mounzer

Ms. Mounzer is a Lebanese writer and translator.

\begin{itemize}
\item
  Aug. 5, 2020
\item
  \begin{itemize}
  \item
  \item
  \item
  \item
  \item
  \end{itemize}
\end{itemize}

\href{https://cn.nytimes.com/opinion/20200806/beirut-port-explosions/}{阅读简体中文版}\href{https://cn.nytimes.com/opinion/20200806/beirut-port-explosions/zh-hant/}{閱讀繁體中文版}

BEIRUT, Lebanon --- It began as a rumble. A deep bass rattling through
the building. And then a roar for seven, eight, nine seconds, an
eternity. A sound that could be made only by the world itself breaking
open. I was certain it was an earthquake.

My husband rushed from the balcony to our bedroom. Waves of pressure
rolled over us; we crouched and clutched at one another. Glass broke,
doors blew open, objects shattered. From the street rose screams and
oaths. And terrified exhortations: ``\emph{Ya Muhammad! Ya Muhammad}!''

``What was it?'' I asked, when I could breathe again.

``\emph{Infijar,''} he responded.
\href{https://www.nytimes.com/2020/08/07/world/middleeast/lebanon-explosion-ship.html}{\emph{Explosion}}\emph{.}
A word we have used far too often in this country. Thinking the blast
had come from underneath our building, I went to the balcony to survey
the damage: The ground glittered with glass as far as the eye could see.

My hands shook as I scrolled through my phone, trying to call or text
friends, checking Twitter to see what happened. The internet connection
went in and out of service. My husband coaxed me back inside. ``Get away
from the windows,'' he said. ``Put proper shoes on! We might have to
run.''

\includegraphics{https://static01.nyt.com/images/2020/08/05/opinion/05mounzer2/merlin_175336185_b78a4c2c-9c38-4c64-961d-79d55ad2b0bf-articleLarge.jpg?quality=75\&auto=webp\&disable=upscale}

Messages poured in on various WhatsApp groups.

``We're OK, all our glass is broken but we're fine.''

``Has anyone heard from H?''

``I spoke to him, he's fine, but his house isn't. He said he won't be
able to answer for a while.''

``The newborn kittens at my mom's house all died! From the pressure I
think but my parents are OK.''

We didn't know what had actually happened, but the reports seemed
certain about the location: the
\href{https://www.nytimes.com/2020/08/07/world/middleeast/lebanon-explosion-ship.html}{Beirut}
port. From our bedroom balcony, I saw a thick plume of pink smoke rising
in the cloudless sky. Speculation was rampant: Israeli warplanes! A
Hezbollah weapons cache! A suicide attack! A fireworks depot on fire!
The truth, which came in bits and pieces over the long and terrible
evening, turned out to be far worse.

Image

A woman rests in front of a damaged building the day after the massive
explosion at the port in Beirut.Credit...Nabil Mounzer/EPA, via
Shutterstock

Lebanon has been pushed into
\href{https://www.nytimes.com/2020/08/03/opinion/lebanon-coronavirus-economy.html}{a
full blown economic collapse} by the corruption and cronyism of the
warlords and influential families who have commanded the seats of power
in government since the end of our 15-year civil war in 1990. Our
currency has devalued over 80 percent. Stories of destitution abound.
Yet I couldn't imagine how spectacular and lethal the incompetence of
the Lebanese state could be.

The explosion turned out to be 2,750 tons of ammonium nitrate, which had
been confiscated from a ship and stored in a hangar at the port since
2014 without proper safety measures.

Customs officials
\href{https://www.nytimes.com/2020/08/05/world/middleeast/beirut-lebanon-explosion.html}{sent
numerous letters} to the courts, seeking guidance on how to dispose of
the material. The judiciary never responded. The chemicals sat in the
hangar until the inevitable happened.

Image

An injured man sits next to a restaurant in the partially destroyed
neighborhood of Mar Mikhael in Beirut.Credit...Patrick Baz/Agence
France-Presse --- Getty Images

I slowly processed the magnitude of it. There were photos of the people
still missing; the homes shattered, books and clothes and furniture
underfoot. The neighborhoods of Gemmayze,
\href{https://www.nytimes.com/interactive/2020/08/04/world/middleeast/beirut-explosion-damage.html}{Mar
Mikhael and Geitawi}, coveted for their red-roofed, century-old houses
overlooking the port from the east, all nearly flattened. One friend
narrowly missed being decapitated. Another friend, seven months
pregnant, was briefly buried under debris.

My friend's father was waiting for his wife in the hallway of a hospital
near the port when the explosion hit. The ceiling collapsed on him. He
came to his senses surrounded by bodies buried under the rubble. He
wished he could see his wife one last time. And then someone pulled him
out. Fortunately, my friend's mother too was unscathed.

Image

Hospital damaged from the explosion in Beirut.Credit...Hassan
Ammar/Associated Press

There were messages from friends, colleagues, acquaintances from all
over the world. The news had traveled far and fast, another measure of
its horror.

My husband spoke to his uncle in New York; I surveyed the damage in the
kitchen. Glassware had flown out of the cupboards. I pulled out the
broom and began sweeping. The night was filled with the dissonant music
of broken glass and in the distance, sirens.

Growing up in Lebanon taught me that an explosion resonates across time,
that the shock reverberates forward into your life, and the pressure
reconfigures the landscape of the mind. I know that it comes to shape
everything you think you deserve from the world. The people of Beirut
have been shaped by the bombs that reconfigured this country.

We haven't even begun to assess the damage that this bomb has done to
us, to our city. At least
\href{https://www.nytimes.com/2020/08/05/world/middleeast/beirut-lebanon-explosion.html}{135
dead and 5,000} injured. And then there is the loss of the port, a
lifeline for a country that imports nearly everything it consumes. We
were already facing food shortages. The explosion took out two massive
grain silos; wheat spilled into the rubble and the ash.

This is not some lamentable accident. ``I can't stress this enough but
the international community must respond to this as a war crime and not
an accidental tragedy,'' the
\href{https://twitter.com/sysh/status/1290980055544475650}{Lebanese-Palestinian
author Saleem Haddad} wrote on Twitter.

In 1989, when I was 10, during the final and deadliest phase of the
Lebanese civil war**,** we were huddled with our neighbors in a
vestibule on the fourth floor of our building when a shell screeched
into the floor below us and exploded. I thought that was the loudest
sound I had ever heard in my life. Our upstairs neighbor was screaming;
our downstairs neighbor's face was gray with concrete dust.

We referred to that phase of the civil war as the ``Aoun war,'' after
Michel Aoun, the general who commandeered the Lebanese Army like his own
militia, decimating West Beirut in his bid to oust the Syrians from
Lebanon.

Mr. Aoun is now
\href{https://www.nytimes.com/2016/11/01/world/middleeast/michel-aoun-lebanon-president.html}{our
octogenarian president}, allied with Hezbollah and Syria. That is how
vile and opportunistic and immortal our warlords are. I use him as an
example not because he is the worst among them --- that is a tough
competition. I mention Mr. Aoun to remind myself how long we have been
at the mercy of the same people and their pernicious ambitions.

Beneath the rubble, beneath the sadness, an immense rage has begun to
boil. Lebanese blood has been spilled for so long. After the war, the
criminals all granted themselves amnesty. This time, it won't be theirs
for the taking.

Image

A man at the scene of the explosion in Beirut.Credit...Ibrahim
Amro/Agence France-Presse --- Getty Images

Lina Mounzer is a Lebanese writer and translator.

\emph{The Times is committed to publishing}
\href{https://www.nytimes.com/2019/01/31/opinion/letters/letters-to-editor-new-york-times-women.html}{\emph{a
diversity of letters}} \emph{to the editor. We'd like to hear what you
think about this or any of our articles. Here are some}
\href{https://help.nytimes.com/hc/en-us/articles/115014925288-How-to-submit-a-letter-to-the-editor}{\emph{tips}}\emph{.
And here's our email:}
\href{mailto:letters@nytimes.com}{\emph{letters@nytimes.com}}\emph{.}

\emph{Follow The New York Times Opinion section on}
\href{https://www.facebook.com/nytopinion}{\emph{Facebook}}\emph{,}
\href{http://twitter.com/NYTOpinion}{\emph{Twitter (@NYTopinion)}}
\emph{and}
\href{https://www.instagram.com/nytopinion/}{\emph{Instagram}}\emph{.}

Advertisement

\protect\hyperlink{after-bottom}{Continue reading the main story}

\hypertarget{site-index}{%
\subsection{Site Index}\label{site-index}}

\hypertarget{site-information-navigation}{%
\subsection{Site Information
Navigation}\label{site-information-navigation}}

\begin{itemize}
\tightlist
\item
  \href{https://help.nytimes.com/hc/en-us/articles/115014792127-Copyright-notice}{©~2020~The
  New York Times Company}
\end{itemize}

\begin{itemize}
\tightlist
\item
  \href{https://www.nytco.com/}{NYTCo}
\item
  \href{https://help.nytimes.com/hc/en-us/articles/115015385887-Contact-Us}{Contact
  Us}
\item
  \href{https://www.nytco.com/careers/}{Work with us}
\item
  \href{https://nytmediakit.com/}{Advertise}
\item
  \href{http://www.tbrandstudio.com/}{T Brand Studio}
\item
  \href{https://www.nytimes.com/privacy/cookie-policy\#how-do-i-manage-trackers}{Your
  Ad Choices}
\item
  \href{https://www.nytimes.com/privacy}{Privacy}
\item
  \href{https://help.nytimes.com/hc/en-us/articles/115014893428-Terms-of-service}{Terms
  of Service}
\item
  \href{https://help.nytimes.com/hc/en-us/articles/115014893968-Terms-of-sale}{Terms
  of Sale}
\item
  \href{https://spiderbites.nytimes.com}{Site Map}
\item
  \href{https://help.nytimes.com/hc/en-us}{Help}
\item
  \href{https://www.nytimes.com/subscription?campaignId=37WXW}{Subscriptions}
\end{itemize}
