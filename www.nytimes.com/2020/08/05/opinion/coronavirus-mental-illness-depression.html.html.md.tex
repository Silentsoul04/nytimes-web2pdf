Sections

SEARCH

\protect\hyperlink{site-content}{Skip to
content}\protect\hyperlink{site-index}{Skip to site index}

\href{https://myaccount.nytimes.com/auth/login?response_type=cookie\&client_id=vi}{}

\href{https://www.nytimes.com/section/todayspaper}{Today's Paper}

\href{/section/opinion}{Opinion}\textbar{}We've Hit a Pandemic Wall

\href{https://nyti.ms/2C0xviT}{https://nyti.ms/2C0xviT}

\begin{itemize}
\item
\item
\item
\item
\item
\item
\end{itemize}

\href{https://www.nytimes.com/spotlight/at-home?action=click\&pgtype=Article\&state=default\&region=TOP_BANNER\&context=at_home_menu}{At
Home}

\begin{itemize}
\tightlist
\item
  \href{https://www.nytimes.com/2020/08/04/arts/television/sam-jay-netflix-special.html?action=click\&pgtype=Article\&state=default\&region=TOP_BANNER\&context=at_home_menu}{Watch:
  Sam Jay}
\item
  \href{https://www.nytimes.com/interactive/2020/at-home/even-more-reporters-editors-diaries-lists-recommendations.html?action=click\&pgtype=Article\&state=default\&region=TOP_BANNER\&context=at_home_menu}{Peruse:
  Reporters' Google Docs}
\item
  \href{https://www.nytimes.com/2020/08/04/dining/colombian-empanadas-carlos-gaviria.html?action=click\&pgtype=Article\&state=default\&region=TOP_BANNER\&context=at_home_menu}{Make:
  Empanadas}
\item
  \href{https://www.nytimes.com/2020/08/06/arts/design/street-art-nyc-george-floyd.html?action=click\&pgtype=Article\&state=default\&region=TOP_BANNER\&context=at_home_menu}{Explore:
  N.Y.C. Street Art}
\end{itemize}

Advertisement

\protect\hyperlink{after-top}{Continue reading the main story}

\href{/section/opinion}{Opinion}

Supported by

\protect\hyperlink{after-sponsor}{Continue reading the main story}

\hypertarget{weve-hit-a-pandemic-wall}{%
\section{We've Hit a Pandemic Wall}\label{weve-hit-a-pandemic-wall}}

New data show that Americans are suffering from record levels of mental
distress.

\href{https://www.nytimes.com/by/jennifer-senior}{\includegraphics{https://static01.nyt.com/images/2018/10/26/opinion/jennifer-senior/jennifer-senior-thumbLarge.png}}

By \href{https://www.nytimes.com/by/jennifer-senior}{Jennifer Senior}

Opinion Columnist

\begin{itemize}
\item
  Aug. 5, 2020
\item
  \begin{itemize}
  \item
  \item
  \item
  \item
  \item
  \item
  \end{itemize}
\end{itemize}

\includegraphics{https://static01.nyt.com/images/2020/08/08/opinion/08senior2/05senior2-articleLarge.jpg?quality=75\&auto=webp\&disable=upscale}

\href{https://www.nytimes.com/es/2020/08/07/espanol/opinion/ansiedad-coronavirus.html}{Leer
en español}

I am trying to think of when I first realized we'd all run smack into a
wall.

Was it two weeks ago, when a friend, ordinarily a paragon of wifely
discretion, started a phone conversation with a boffo rant about her
husband?

Was it when I looked at my own spouse --- one week later, this probably
was --- and calmly told him that each and every one of my problems was
his fault?

(They were not.)

Or maybe it was when I was scrolling through Twitter and saw
\href{https://twitter.com/amandastern/status/1284639637252845570}{a
tweet} from the author Amanda Stern, single and living in Brooklyn, who
noted it had been 137 days since she'd given or received a hug? ``Hello,
I am depressed'' were its last four words.

Whatever this is, it is real --- and quantifiable, and extends far
beyond my own meager solar system of colleagues and pals and dearly
beloveds. Call it pandemic fatigue; call it the summer poop-out; call it
whatever you wish. Any label, at this point, would probably be too
trivializing, belying what is in fact a far deeper problem. We are not,
as a nation, all right.

Let's start with the numbers. According to the National Center for
Health Statistics, roughly
\href{https://www.cdc.gov/nchs/data/nhis/earlyrelease/ERmentalhealth-508.pdf}{one
in 12 American adults} reported symptoms of an anxiety disorder at this
time last year; now it's more than
\href{https://www.cdc.gov/nchs/covid19/pulse/mental-health.htm}{one in
three}. Last week, the Kaiser Family Foundation
\href{https://www.kff.org/coronavirus-covid-19/report/kff-health-tracking-poll-july-2020/}{released
a tracking poll} showing that for the first time, a majority of American
adults --- 53 percent --- believes that the pandemic is taking a toll on
their mental health.

This number climbs to 68 percent if you look solely at
African-Americans. The disproportionate toll the pandemic has taken on
Black lives and livelihoods --- made possible by centuries of structural
disparities, compounded by the corrosive psychological effect of
everyday racism --- is appearing, starkly, in our mental health data.

``Even during so-called better times, Black adults are more likely to
report persistent symptoms of emotional distress,'' Hope Hill, a
clinical psychologist and associate professor in the psychology
department at Howard University, told me. ``So when I hear about that
fifteen-point difference, it's upsetting, but it's not surprising, given
the impact of long-term, race-based trauma and inequality.''

But even the luckiest among us haven't been spared. According to the
Kaiser Family Foundation, 36 percent of Americans report that
coronavirus-related worry is interfering with their sleep. Eighteen
percent say they're more easily losing their tempers. Thirty-two percent
say it has made them overeat or under-eat.

I'm solidly in the former category. Turns out the extra ten extra pounds
around my middle have moved in and unpacked, though I'd initially hoped
they were **** on a month-to-month lease.

So. How to account for this national slide into a sulfurous pit of
distress?

The most obvious answer is that the coronavirus is still claiming
hundreds of lives a day in the United States, whipping its way through
the South and heaving to the surface once again in the West. This is
true, and on its face is awful enough. But I suspect it's more than
that.

America's prodigious infection rates are also a testament to our own
national failure --- and therefore a source of existential ghastliness,
of sheer perversity: Why on earth were so many of us sacrificing so much
in these past four and a half months --- our livelihoods, our social
connections, our safety, our children's schooling, our attendance at
birthdays and anniversaries and funerals --- if it all came to naught?
At this point, weren't we expecting some form of relief, a resumption of
something like life?

\includegraphics{https://static01.nyt.com/images/2020/08/05/opinion/05senior1/05senior1-articleLarge.jpg?quality=75\&auto=webp\&disable=upscale}

``People often think of trauma as a discrete event --- a fire, getting
mugged,'' said Daphne de Marneffe, author of an excellent book about
marriage called
``\href{https://www.thecut.com/2018/01/daphne-de-marneff-on-the-rough-patch.html}{The
Rough Patch}'' and one of the most astute psychologists I know. ``But
what it's really about is helplessness, about being on the receiving end
of forces you can't control. Which is what we have now. It's like we're
in an endless car ride with a drunk at the wheel. No one knows when the
pain will stop.''

Nor, I would add, do any of us know what life will look like once this
pandemic has truly subsided. Will the economy remain in tatters? (One
word for you:
\href{https://www.washingtonpost.com/business/2020/08/04/grocery-prices-unemployed/?hpid=hp_hp-top-table-main_foodinflation-645pm\%3Ahomepage\%2Fstory-ans}{inflation}.)
**** Will our city centers be whistling, broken conch shells, gritty and
empty at their cores? (Lord, I hope not.) Will President Trump be
re-elected, transforming democracy as we've known it into an eerie
photonegative of itself?

In her own therapeutic practice, de Marneffe has noticed that families
with pre-existing tensions and frailties are doing much worse: The
pandemic has only provided more opportunities for struggling couples to
communicate poorly, roll their eyes and project rotten motives onto one
another. (``And marriage is already a hotbed of scapegoating,'' she
noted.) Parents who were barely limping along, praying for school to
start, are now brimming with despair and ruing their lack of
imagination**:** How are they supposed to make it through another
semester of remote schooling?

``Those of us who are average parents rely on structure,'' she told me.
``We \emph{need} school.''

I recently thumbed through ``The Plague,'' to see if Albert Camus had
intuited anything about the rhythms of human suffering in conditions of
fear, disease and constraint. Naturally, he had. It was on April 16 that
Dr. Rieux first felt the squish of a dead rat beneath his feet on his
landing; it was in mid-August that the plague ``had swallowed up
everything and everyone,'' with the prevailing emotion being ``the sense
of exile and of deprivation, with all the crosscurrents of revolt and
fear set up by these.'' Those returning from quarantine started setting
fire to their homes, convinced the plague had settled into their walls.

Camus chose, in other words, to make the four-month mark get pretty
freaky in Oran. That's more or less what happened here. If only we knew
how it ended.

\emph{The Times is committed to publishing}
\href{https://www.nytimes.com/2019/01/31/opinion/letters/letters-to-editor-new-york-times-women.html}{\emph{a
diversity of letters}} \emph{to the editor. We'd like to hear what you
think about this or any of our articles. Here are some}
\href{https://help.nytimes.com/hc/en-us/articles/115014925288-How-to-submit-a-letter-to-the-editor}{\emph{tips}}\emph{.
And here's our email:}
\href{mailto:letters@nytimes.com}{\emph{letters@nytimes.com}}\emph{.}

\emph{Follow The New York Times Opinion section on}
\href{https://www.facebook.com/nytopinion}{\emph{Facebook}}\emph{,}
\href{http://twitter.com/NYTOpinion}{\emph{Twitter (@NYTopinion)}}
\emph{and}
\href{https://www.instagram.com/nytopinion/}{\emph{Instagram}}\emph{.}

Advertisement

\protect\hyperlink{after-bottom}{Continue reading the main story}

\hypertarget{site-index}{%
\subsection{Site Index}\label{site-index}}

\hypertarget{site-information-navigation}{%
\subsection{Site Information
Navigation}\label{site-information-navigation}}

\begin{itemize}
\tightlist
\item
  \href{https://help.nytimes.com/hc/en-us/articles/115014792127-Copyright-notice}{©~2020~The
  New York Times Company}
\end{itemize}

\begin{itemize}
\tightlist
\item
  \href{https://www.nytco.com/}{NYTCo}
\item
  \href{https://help.nytimes.com/hc/en-us/articles/115015385887-Contact-Us}{Contact
  Us}
\item
  \href{https://www.nytco.com/careers/}{Work with us}
\item
  \href{https://nytmediakit.com/}{Advertise}
\item
  \href{http://www.tbrandstudio.com/}{T Brand Studio}
\item
  \href{https://www.nytimes.com/privacy/cookie-policy\#how-do-i-manage-trackers}{Your
  Ad Choices}
\item
  \href{https://www.nytimes.com/privacy}{Privacy}
\item
  \href{https://help.nytimes.com/hc/en-us/articles/115014893428-Terms-of-service}{Terms
  of Service}
\item
  \href{https://help.nytimes.com/hc/en-us/articles/115014893968-Terms-of-sale}{Terms
  of Sale}
\item
  \href{https://spiderbites.nytimes.com}{Site Map}
\item
  \href{https://help.nytimes.com/hc/en-us}{Help}
\item
  \href{https://www.nytimes.com/subscription?campaignId=37WXW}{Subscriptions}
\end{itemize}
