Sections

SEARCH

\protect\hyperlink{site-content}{Skip to
content}\protect\hyperlink{site-index}{Skip to site index}

\href{https://www.nytimes.com/section/world/europe}{Europe}

\href{https://myaccount.nytimes.com/auth/login?response_type=cookie\&client_id=vi}{}

\href{https://www.nytimes.com/section/todayspaper}{Today's Paper}

\href{/section/world/europe}{Europe}\textbar{}Welcome Back to Germany.
Now Take Your Free Virus Test.

\href{https://nyti.ms/3ftlNL5}{https://nyti.ms/3ftlNL5}

\begin{itemize}
\item
\item
\item
\item
\item
\end{itemize}

\href{https://www.nytimes.com/news-event/coronavirus?action=click\&pgtype=Article\&state=default\&region=TOP_BANNER\&context=storylines_menu}{The
Coronavirus Outbreak}

\begin{itemize}
\tightlist
\item
  live\href{https://www.nytimes.com/2020/08/08/world/coronavirus-updates.html?action=click\&pgtype=Article\&state=default\&region=TOP_BANNER\&context=storylines_menu}{Latest
  Updates}
\item
  \href{https://www.nytimes.com/interactive/2020/us/coronavirus-us-cases.html?action=click\&pgtype=Article\&state=default\&region=TOP_BANNER\&context=storylines_menu}{Maps
  and Cases}
\item
  \href{https://www.nytimes.com/interactive/2020/science/coronavirus-vaccine-tracker.html?action=click\&pgtype=Article\&state=default\&region=TOP_BANNER\&context=storylines_menu}{Vaccine
  Tracker}
\item
  \href{https://www.nytimes.com/interactive/2020/world/coronavirus-tips-advice.html?action=click\&pgtype=Article\&state=default\&region=TOP_BANNER\&context=storylines_menu}{F.A.Q.}
\item
  \href{https://www.nytimes.com/live/2020/08/07/business/stock-market-today-coronavirus?action=click\&pgtype=Article\&state=default\&region=TOP_BANNER\&context=storylines_menu}{Markets
  \& Economy}
\end{itemize}

Advertisement

\protect\hyperlink{after-top}{Continue reading the main story}

Supported by

\protect\hyperlink{after-sponsor}{Continue reading the main story}

\hypertarget{welcome-back-to-germany-now-take-your-free-virus-test}{%
\section{Welcome Back to Germany. Now Take Your Free Virus
Test.}\label{welcome-back-to-germany-now-take-your-free-virus-test}}

The country's capacity to make testing efficient, affordable and
available has distinguished it. Now, to head off a potential second
wave, it's testing anyone returning from a ``hot zone'' on entry.

\includegraphics{https://static01.nyt.com/images/2020/07/31/world/00germany-test04/merlin_174933735_795e0bdb-6733-4d2c-8a83-79abd2479eef-articleLarge.jpg?quality=75\&auto=webp\&disable=upscale}

\href{https://www.nytimes.com/by/melissa-eddy}{\includegraphics{https://static01.nyt.com/images/2018/10/09/multimedia/author-melissa-eddy/author-melissa-eddy-thumbLarge.png}}

By \href{https://www.nytimes.com/by/melissa-eddy}{Melissa Eddy}

\begin{itemize}
\item
  Aug. 5, 2020
\item
  \begin{itemize}
  \item
  \item
  \item
  \item
  \item
  \end{itemize}
\end{itemize}

BERLIN --- When she returned to Germany last week from a vacation in
Serbia, one of the first things Snjezana Kirstein did was to stop at a
pop-up coronavirus testing center at Berlin's Tegel Aiport.

Whereas such tests can be hard to find in the United States, with
\href{https://www.nytimes.com/2020/06/29/upshot/coronavirus-tests-unpredictable-prices.html}{unpredictable
costs} and results two weeks in coming, Ms. Kirstein was on her way in a
matter of minutes after having her nose and throat swabbed. She expected
an answer in 24 to 48 hours. The test was not only swift, it was free.

``I think it is super,'' Ms. Kirstein said. ``It was so easy to find,
and best of all, it didn't cost me a thing.''

As of Saturday, Germany will require that same simple test for all
citizens or residents, like Ms. Kirstein, and other travelers who enter
the country from coronavirus ``hot spots,'' again making it a leader in
using testing as a firewall against the spread of the virus.

``I am very aware that this is an intrusion in personal freedom,'' Jens
Spahn, Germany's health minister, said at a news conference Thursday.
``But freedom always comes with responsibility for myself and for
others.''

As Europe reopens, cases have begun ticking up nearly everywhere, to a
greater or lesser extent, leaving countries in a constant, seesaw battle
to tamp down outbreaks before they undo months of hard-won progress made
during costly lockdowns this spring.

Germany is no exception. This week it recorded 1,045 new coronavirus
infections in a single day, part of a rising trend that has begun to
worry officials as people return from trips abroad during the summer
vacation season.

One of the biggest concerns in Germany and across the continent is that
travelers will carry the virus with them. Until now, Germany, like other
countries, has relied on quarantining newly arriving travelers. But such
measures are not always enforced, or strictly followed.

Since the start of the pandemic, Germany has made
\href{https://www.nytimes.com/2020/04/18/world/europe/with-broad-random-tests-for-antibodies-germany-seeks-path-out-of-lockdown.html?searchResultPosition=29}{testing
a primary tool in its battle} against the virus. Now it is turning to
that approach again to head off a potential second wave of infections.
Its capacity to make testing efficient, affordable and available has
distinguished it among industrialized nations.

Unlike the United States or
\href{https://www.nytimes.com/2020/04/02/world/europe/uk-coronavirus-testing.html?action=click\&module=RelatedLinks\&pgtype=Article}{Britain},
both of which allowed their public health agencies to keep tight control
over standards for tests and discouraged private clinics, labs or
companies from developing their own, Germany disseminated a blueprint
for a test as soon as it had one.

\includegraphics{https://static01.nyt.com/images/2020/07/31/world/00germany-test03/merlin_174081960_c3c1150d-6dbe-4ac3-b3b3-73bb9735f25b-articleLarge.jpg?quality=75\&auto=webp\&disable=upscale}

In January, doctors at the Charité research hospital in Berlin developed
one of the world's first diagnostic tests for the new coronavirus. They
quickly made it available to the country's public hospitals and research
laboratories, as well as a nationwide network of about 200 privately
owned labs. Everyone focused on making test kits.

The World Health Organization later approved the kits and distributed
instructions for developing them worldwide. At that time, the
\href{https://www.nytimes.com/2020/03/28/us/testing-coronavirus-pandemic.html}{Centers
for Disease Control and Prevention were still struggling} to develop
test kits in the United States.

\hypertarget{latest-updates-the-coronavirus-outbreak}{%
\section{\texorpdfstring{\href{https://www.nytimes.com/2020/08/07/world/covid-19-news.html?action=click\&pgtype=Article\&state=default\&region=MAIN_CONTENT_1\&context=storylines_live_updates}{Latest
Updates: The Coronavirus
Outbreak}}{Latest Updates: The Coronavirus Outbreak}}\label{latest-updates-the-coronavirus-outbreak}}

Updated 2020-08-08T12:04:28.992Z

\begin{itemize}
\tightlist
\item
  \href{https://www.nytimes.com/2020/08/07/world/covid-19-news.html?action=click\&pgtype=Article\&state=default\&region=MAIN_CONTENT_1\&context=storylines_live_updates\#link-1f86d03a}{As
  the U.S. relief talks falter again, Trump says he is prepared to act
  on his own.}
\item
  \href{https://www.nytimes.com/2020/08/07/world/covid-19-news.html?action=click\&pgtype=Article\&state=default\&region=MAIN_CONTENT_1\&context=storylines_live_updates\#link-3f64a70a}{Cuomo
  says N.Y. schools can reopen in-person but leaves it up to districts
  to determine if, when and how.}
\item
  \href{https://www.nytimes.com/2020/08/07/world/covid-19-news.html?action=click\&pgtype=Article\&state=default\&region=MAIN_CONTENT_1\&context=storylines_live_updates\#link-14e70066}{Thousands
  of cases went unreported in California when a computer server failed.}
\end{itemize}

\href{https://www.nytimes.com/2020/08/07/world/covid-19-news.html?action=click\&pgtype=Article\&state=default\&region=MAIN_CONTENT_1\&context=storylines_live_updates}{See
more updates}

More live coverage:
\href{https://www.nytimes.com/live/2020/08/07/business/stock-market-today-coronavirus?action=click\&pgtype=Article\&state=default\&region=MAIN_CONTENT_1\&context=storylines_live_updates}{Markets}

While nearly 90 percent of Germany's population of 83 million is covered
by public insurance, people can also opt for private insurance that
either competes with the public system, or provides top-up coverage.

By May, the German authorities required all insurance companies to cover
the cost of testing. Initially it was limited to those with symptoms,
but today it is available to just about everyone, with the government
agreeing to help cover the costs.

``We have had a very important principle since the beginning of the
pandemic --- that testing in Germany is not a question of money,'' Mr.
Spahn said. ``That is a big difference to many other countries that have
failed to test enough over many months.''

Early on, the government also ordered hospitals and labs to reduce all
but necessary treatments to free resources to process coronavirus tests,
earmarking millions to support the effort and ensuring speedy results.

These measures allowed Germany to ramp up testing at a pace that
prevented the country's hospitals --- well equipped in any case, with
one of the world's highest ratios of intensive care beds per capita and
a centralized system to transfer patients between them --- from becoming
overwhelmed.

By March, Germany was one of the highest testers per capita in the
world. Other countries have since overtaken it, in part because Germany
has been so effective in targeting tests and following up on positive
results to isolate those infected and shut down chains of infection
before they get out of control.

So while Germany can carry out as many as 1.2 million tests per week,
according the health ministry, it is only currently testing half that
many people. That gives it the bandwidth to easily expand testing to
incoming travelers, while still maintaining readiness should a big
second wave arrive.

The United States has a capacity of 4.5 million tests per week,
according to the Rockefeller Foundation, but its population is also
roughly four times the size of Germany's, and results can take five to
14 days.

The United States also lacks comprehensive contract-tracing, another
factor that has left it struggling to contain the spread of the virus.
Germany has had fewer than 10,000 deaths attributed to the virus, the
United States more than 150,000.

Those awaiting test results in Germany are required to self-quarantine,
and are released from the obligation immediately if the result is
negative. They must also leave contact information in case they have a
positive result and need to be traced.

Image

The beach in Calella, Spain, one of the countries that Germany has
designated a hot spot and will require tests for
travelers.~~Credit...Albert Gea/Reuters

The speed of the results, then, is critical to the containment effort.
People are more likely to abide isolation for two days than for two
weeks, and even if they don't they still have less time to
\href{https://www.theatlantic.com/politics/archive/2020/03/rand-paul-coronavirus-test-reckless/608593/}{wander
around and potentially expose others}. Germany's shorter turnaround also
reduces the time that healthy people are taken out of productive roles
in society.

Those who refuse a test must remain in quarantine for two weeks, unless
they can provide a negative test less than 72 hours old. Under current
E.U. travel restrictions, foreigners from outside the bloc are allowed
to enter the country only if they have the right to live and work in
Germany, with few exceptions.

Germany's wide availability of testing has not made everyone happy,
however, and not all agree on the benefits.

\href{https://www.nytimes.com/news-event/coronavirus?action=click\&pgtype=Article\&state=default\&region=MAIN_CONTENT_3\&context=storylines_faq}{}

\hypertarget{the-coronavirus-outbreak-}{%
\subsubsection{The Coronavirus Outbreak
›}\label{the-coronavirus-outbreak-}}

\hypertarget{frequently-asked-questions}{%
\paragraph{Frequently Asked
Questions}\label{frequently-asked-questions}}

Updated August 6, 2020

\begin{itemize}
\item ~
  \hypertarget{why-are-bars-linked-to-outbreaks}{%
  \paragraph{Why are bars linked to
  outbreaks?}\label{why-are-bars-linked-to-outbreaks}}

  \begin{itemize}
  \tightlist
  \item
    Think about a bar. Alcohol is flowing. It can be loud, but it's
    definitely intimate, and you often need to lean in close to hear
    your friend. And strangers have way, way fewer reservations about
    coming up to people in a bar. That's sort of the point of a bar.
    Feeling good and close to strangers. It's no surprise, then, that
    \href{https://www.nytimes.com/2020/07/02/us/coronavirus-bars.html?action=click\&pgtype=Article\&state=default\&region=MAIN_CONTENT_3\&context=storylines_faq}{bars
    have been linked to outbreaks in several states.} Louisiana health
    officials have tied
    \href{https://www.nytimes.com/2020/06/22/us/new-coronavirus-phase.html?action=click\&pgtype=Article\&state=default\&region=MAIN_CONTENT_3\&context=storylines_faq}{at
    least 100 coronavirus cases} to bars in the Tigerland nightlife
    district in Baton Rouge. Minnesota has traced 328 recent cases to
    bars across the state.
    \href{https://www.boisestatepublicradio.org/post/bars-large-venues-close-ada-county-after-surge-coronavirus-prompts-rollback\#stream/0}{In
    Idaho}, health officials shut down bars in Ada County after
    reporting clusters of infections among young adults who had visited
    several bars in downtown Boise. Governors in
    \href{https://www.nytimes.com/2020/07/01/us/california-coronavirus-reopening.html?action=click\&pgtype=Article\&state=default\&region=MAIN_CONTENT_3\&context=storylines_faq}{California},
    \href{https://www.nytimes.com/2020/06/14/us/coronavirus-united-states.html?action=click\&pgtype=Article\&state=default\&region=MAIN_CONTENT_3\&context=storylines_faq}{Texas
    and Arizona}, where coronavirus cases are soaring, have ordered
    hundreds of newly reopened bars to shut down. Less than two weeks
    after Colorado's bars reopened at limited capacity, Gov. Jared Polis
    \href{https://www.denverpost.com/2020/06/30/colorado-bars-closed-coronavirus/}{ordered
    them to close}.
  \end{itemize}
\item ~
  \hypertarget{i-have-antibodies-am-i-now-immune}{%
  \paragraph{I have antibodies. Am I now
  immune?}\label{i-have-antibodies-am-i-now-immune}}

  \begin{itemize}
  \tightlist
  \item
    As of right now,
    \href{https://www.nytimes.com/2020/07/22/health/covid-antibodies-herd-immunity.html?action=click\&pgtype=Article\&state=default\&region=MAIN_CONTENT_3\&context=storylines_faq}{that
    seems likely, for at least several months.} There have been
    frightening accounts of people suffering what seems to be a second
    bout of Covid-19. But experts say these patients may have a
    drawn-out course of infection, with the virus taking a slow toll
    weeks to months after initial exposure. People infected with the
    coronavirus typically
    \href{https://www.nature.com/articles/s41586-020-2456-9}{produce}
    immune molecules called antibodies, which are
    \href{https://www.nytimes.com/2020/05/07/health/coronavirus-antibody-prevalence.html?action=click\&pgtype=Article\&state=default\&region=MAIN_CONTENT_3\&context=storylines_faq}{protective
    proteins made in response to an
    infection}\href{https://www.nytimes.com/2020/05/07/health/coronavirus-antibody-prevalence.html?action=click\&pgtype=Article\&state=default\&region=MAIN_CONTENT_3\&context=storylines_faq}{.
    These antibodies may} last in the body
    \href{https://www.nature.com/articles/s41591-020-0965-6}{only two to
    three months}, which may seem worrisome, but that's perfectly normal
    after an acute infection subsides, said Dr. Michael Mina, an
    immunologist at Harvard University. It may be possible to get the
    coronavirus again, but it's highly unlikely that it would be
    possible in a short window of time from initial infection or make
    people sicker the second time.
  \end{itemize}
\item ~
  \hypertarget{im-a-small-business-owner-can-i-get-relief}{%
  \paragraph{I'm a small-business owner. Can I get
  relief?}\label{im-a-small-business-owner-can-i-get-relief}}

  \begin{itemize}
  \tightlist
  \item
    The
    \href{https://www.nytimes.com/article/small-business-loans-stimulus-grants-freelancers-coronavirus.html?action=click\&pgtype=Article\&state=default\&region=MAIN_CONTENT_3\&context=storylines_faq}{stimulus
    bills enacted in March} offer help for the millions of American
    small businesses. Those eligible for aid are businesses and
    nonprofit organizations with fewer than 500 workers, including sole
    proprietorships, independent contractors and freelancers. Some
    larger companies in some industries are also eligible. The help
    being offered, which is being managed by the Small Business
    Administration, includes the Paycheck Protection Program and the
    Economic Injury Disaster Loan program. But lots of folks have
    \href{https://www.nytimes.com/interactive/2020/05/07/business/small-business-loans-coronavirus.html?action=click\&pgtype=Article\&state=default\&region=MAIN_CONTENT_3\&context=storylines_faq}{not
    yet seen payouts.} Even those who have received help are confused:
    The rules are draconian, and some are stuck sitting on
    \href{https://www.nytimes.com/2020/05/02/business/economy/loans-coronavirus-small-business.html?action=click\&pgtype=Article\&state=default\&region=MAIN_CONTENT_3\&context=storylines_faq}{money
    they don't know how to use.} Many small-business owners are getting
    less than they expected or
    \href{https://www.nytimes.com/2020/06/10/business/Small-business-loans-ppp.html?action=click\&pgtype=Article\&state=default\&region=MAIN_CONTENT_3\&context=storylines_faq}{not
    hearing anything at all.}
  \end{itemize}
\item ~
  \hypertarget{what-are-my-rights-if-i-am-worried-about-going-back-to-work}{%
  \paragraph{What are my rights if I am worried about going back to
  work?}\label{what-are-my-rights-if-i-am-worried-about-going-back-to-work}}

  \begin{itemize}
  \tightlist
  \item
    Employers have to provide
    \href{https://www.osha.gov/SLTC/covid-19/standards.html}{a safe
    workplace} with policies that protect everyone equally.
    \href{https://www.nytimes.com/article/coronavirus-money-unemployment.html?action=click\&pgtype=Article\&state=default\&region=MAIN_CONTENT_3\&context=storylines_faq}{And
    if one of your co-workers tests positive for the coronavirus, the
    C.D.C.} has said that
    \href{https://www.cdc.gov/coronavirus/2019-ncov/community/guidance-business-response.html}{employers
    should tell their employees} -\/- without giving you the sick
    employee's name -\/- that they may have been exposed to the virus.
  \end{itemize}
\item ~
  \hypertarget{what-is-school-going-to-look-like-in-september}{%
  \paragraph{What is school going to look like in
  September?}\label{what-is-school-going-to-look-like-in-september}}

  \begin{itemize}
  \tightlist
  \item
    It is unlikely that many schools will return to a normal schedule
    this fall, requiring the grind of
    \href{https://www.nytimes.com/2020/06/05/us/coronavirus-education-lost-learning.html?action=click\&pgtype=Article\&state=default\&region=MAIN_CONTENT_3\&context=storylines_faq}{online
    learning},
    \href{https://www.nytimes.com/2020/05/29/us/coronavirus-child-care-centers.html?action=click\&pgtype=Article\&state=default\&region=MAIN_CONTENT_3\&context=storylines_faq}{makeshift
    child care} and
    \href{https://www.nytimes.com/2020/06/03/business/economy/coronavirus-working-women.html?action=click\&pgtype=Article\&state=default\&region=MAIN_CONTENT_3\&context=storylines_faq}{stunted
    workdays} to continue. California's two largest public school
    districts --- Los Angeles and San Diego --- said on July 13, that
    \href{https://www.nytimes.com/2020/07/13/us/lausd-san-diego-school-reopening.html?action=click\&pgtype=Article\&state=default\&region=MAIN_CONTENT_3\&context=storylines_faq}{instruction
    will be remote-only in the fall}, citing concerns that surging
    coronavirus infections in their areas pose too dire a risk for
    students and teachers. Together, the two districts enroll some
    825,000 students. They are the largest in the country so far to
    abandon plans for even a partial physical return to classrooms when
    they reopen in August. For other districts, the solution won't be an
    all-or-nothing approach.
    \href{https://bioethics.jhu.edu/research-and-outreach/projects/eschool-initiative/school-policy-tracker/}{Many
    systems}, including the nation's largest, New York City, are
    devising
    \href{https://www.nytimes.com/2020/06/26/us/coronavirus-schools-reopen-fall.html?action=click\&pgtype=Article\&state=default\&region=MAIN_CONTENT_3\&context=storylines_faq}{hybrid
    plans} that involve spending some days in classrooms and other days
    online. There's no national policy on this yet, so check with your
    municipal school system regularly to see what is happening in your
    community.
  \end{itemize}
\end{itemize}

Some experts warn that it could overwhelm laboratories and threaten the
country's readiness to deal with a resurgence of the virus when colder
temperatures push people back indoors.

``It is questionable whether the general testing of travel returnees
offers an appropriate balance between benefit and expense,'' said Dr.
Michael Müller, head of the Association of Medical Labs, which
represents more than 200 labs across the country.

Others have questioned whether it is fair to saddle taxpayers with the
burden of paying for the tests for those who willingly risk traveling
outside Europe, despite warnings by government health authorities.

The requirement for travelers includes those coming from
\href{https://www.rki.de/DE/Content/InfAZ/N/Neuartiges_Coronavirus/Risikogebiete_neu.html}{130
countries} and regions, including the United States and three districts
in Spain, which the German authorities consider high-risk for spread of
the virus.

But Mr. Spahn rejected the idea that only the wealthy were traveling,
citing people with families in Turkey or elsewhere in Europe whom they
visited.

``I know the saying, `Whoever can afford a ski trip can pay for their
broken leg,' but when you think that idea through, then especially in a
pandemic, it compromises solidarity in society,'' Mr. Spahn said. ``So
testing will remain cost-free.''

While Germans are normally quick to decry any encroachment on their
personal privacy, the threat of the virus returning appears to frighten
them more, and there has been little pushback to the proposed
requirement for travelers.

On the second day after a testing station was set up at Tegel Airport,
dozens of people pushing luggage trolleys stacked with suitcases or
young children in strollers waited patiently for their turn to submit a
voluntary test.

``It is an unbelievable hassle to set it up and organize it, but if it
gives passengers back a sense of security when traveling, then it is
worth it,'' said Hannes Stefan Hönemann, spokesman for the airport's
operator.

The requirement for travelers include those coming over land as well if
they have visited a hot zone. Health authorities in the southern state
of Bavaria, the gateway for people returning by car from vacations on
the Mediterranean coasts, set up test centers at three highway rest
stops near the border, as well as at airports and major train stations.

Image

A testing station for travelers near Ruhpolding, southern Germany, last
week. Health authorities in the southern state of Bavaria set up test
centers at three highway rest stops near the border.Credit...Christof
Stache/Agence France-Presse --- Getty Images

Over the weekend, before the requirement went into effect this week,
about 18,000 people were tested voluntarily, they said.

Bavaria made free testing available to all residents on July 1. The
government invested 200 million euros, about \$235 million, to expand
laboratory capacity, both public and private, as well as personnel and
working hours. Today it has the capacity for about 27,000 tests a day,
state health minister, Melanie Huml, said.

``Test, test, test is the name of the game in Bavaria,'' she said. ``Our
goal is to recognize infections as quickly as possible to stop chains of
infection as early as early as possible.''

She estimates they can test as many as 2,000 travelers each day, the
same capacity as expected in Berlin's airports. Those who test positive
will be alerted immediately, as will the public local health office, so
it can follow up on care and contact tracing.

``Corona is not over and does not forgive any lack of vigilance,'' Ms.
Huml said. ``We have to be careful to prevent a second wave from
creeping up on us.''

Advertisement

\protect\hyperlink{after-bottom}{Continue reading the main story}

\hypertarget{site-index}{%
\subsection{Site Index}\label{site-index}}

\hypertarget{site-information-navigation}{%
\subsection{Site Information
Navigation}\label{site-information-navigation}}

\begin{itemize}
\tightlist
\item
  \href{https://help.nytimes.com/hc/en-us/articles/115014792127-Copyright-notice}{©~2020~The
  New York Times Company}
\end{itemize}

\begin{itemize}
\tightlist
\item
  \href{https://www.nytco.com/}{NYTCo}
\item
  \href{https://help.nytimes.com/hc/en-us/articles/115015385887-Contact-Us}{Contact
  Us}
\item
  \href{https://www.nytco.com/careers/}{Work with us}
\item
  \href{https://nytmediakit.com/}{Advertise}
\item
  \href{http://www.tbrandstudio.com/}{T Brand Studio}
\item
  \href{https://www.nytimes.com/privacy/cookie-policy\#how-do-i-manage-trackers}{Your
  Ad Choices}
\item
  \href{https://www.nytimes.com/privacy}{Privacy}
\item
  \href{https://help.nytimes.com/hc/en-us/articles/115014893428-Terms-of-service}{Terms
  of Service}
\item
  \href{https://help.nytimes.com/hc/en-us/articles/115014893968-Terms-of-sale}{Terms
  of Sale}
\item
  \href{https://spiderbites.nytimes.com}{Site Map}
\item
  \href{https://help.nytimes.com/hc/en-us}{Help}
\item
  \href{https://www.nytimes.com/subscription?campaignId=37WXW}{Subscriptions}
\end{itemize}
