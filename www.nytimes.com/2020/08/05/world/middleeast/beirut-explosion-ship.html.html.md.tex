Sections

SEARCH

\protect\hyperlink{site-content}{Skip to
content}\protect\hyperlink{site-index}{Skip to site index}

\href{https://www.nytimes.com/section/world/middleeast}{Middle East}

\href{https://myaccount.nytimes.com/auth/login?response_type=cookie\&client_id=vi}{}

\href{https://www.nytimes.com/section/todayspaper}{Today's Paper}

\href{/section/world/middleeast}{Middle East}\textbar{}Blame for Beirut
Explosion Begins With a Leaky, Troubled Ship

\href{https://nyti.ms/31gXTNO}{https://nyti.ms/31gXTNO}

\begin{itemize}
\item
\item
\item
\item
\item
\item
\end{itemize}

Beirut Explosion

\begin{itemize}
\tightlist
\item
  \href{https://www.nytimes.com/2020/08/05/world/middleeast/beirut-explosion-what-happened.html?action=click\&pgtype=Article\&state=default\&region=TOP_BANNER\&context=storylines_menu}{What
  We Know}
\item
  \href{https://www.nytimes.com/2020/08/05/video/beirut-explosion-footage.html?action=click\&pgtype=Article\&state=default\&region=TOP_BANNER\&context=storylines_menu}{Footage
  of the Blast}
\item
  \href{https://www.nytimes.com/2020/08/05/world/middleeast/beirut-explosion-ammonium-nitrate.html?action=click\&pgtype=Article\&state=default\&region=TOP_BANNER\&context=storylines_menu}{What
  is Ammonium Nitrate?}
\item
  \href{https://www.nytimes.com/interactive/2020/08/04/world/middleeast/beirut-explosion-damage.html?action=click\&pgtype=Article\&state=default\&region=TOP_BANNER\&context=storylines_menu}{Mapping
  the Damage}
\end{itemize}

Advertisement

\protect\hyperlink{after-top}{Continue reading the main story}

Supported by

\protect\hyperlink{after-sponsor}{Continue reading the main story}

\hypertarget{blame-for-beirut-explosion-begins-with-a-leaky-troubled-ship}{%
\section{Blame for Beirut Explosion Begins With a Leaky, Troubled
Ship}\label{blame-for-beirut-explosion-begins-with-a-leaky-troubled-ship}}

The bleak tale of chronic negligence started over six years ago, when an
indebted vessel and its volatile cargo pulled into port. It ended on
Tuesday in a giant explosion.

\includegraphics{https://static01.nyt.com/images/2020/08/05/world/05lebanon-ship/05lebanon-ship-articleLarge.jpg?quality=75\&auto=webp\&disable=upscale}

\href{https://www.nytimes.com/by/declan-walsh}{\includegraphics{https://static01.nyt.com/images/2018/10/15/multimedia/author-declan-walsh/author-declan-walsh-thumbLarge-v3.png}}\href{https://www.nytimes.com/by/andrew-higgins}{\includegraphics{https://static01.nyt.com/images/2018/10/10/multimedia/author-andrew-higgins/author-andrew-higgins-thumbLarge.png}}

By \href{https://www.nytimes.com/by/declan-walsh}{Declan Walsh} and
\href{https://www.nytimes.com/by/andrew-higgins}{Andrew Higgins}

\begin{itemize}
\item
  Aug. 5, 2020
\item
  \begin{itemize}
  \item
  \item
  \item
  \item
  \item
  \item
  \end{itemize}
\end{itemize}

CAIRO --- The countdown to catastrophe in
\href{https://www.nytimes.com/2020/08/06/world/middleeast/beirut-explosion-bride-video.html}{Beirut}
started more than six years ago when a troubled, Russian-leased cargo
ship made an unscheduled stop at the city's port.

The ship was trailed by debts, crewed by disgruntled sailors and dogged
by a small hole in its hull that meant water had to be constantly pumped
out. And it carried a volatile cargo, more than 2,000 tons of ammonium
nitrate, a combustible material used to make fertilizers --- and bombs
--- that was destined for Mozambique.

The ship, the Rhosus, never made it. Embroiled in a financial and
diplomatic dispute, it was abandoned by the Russian businessman who had
leased it. And the ammonium nitrate was transferred to a dockside
warehouse in Beirut, where it would languish for years, until Tuesday,
when Lebanese officials said it exploded,
\href{https://www.nytimes.com/2020/08/04/world/middleeast/lebanon-explosion.html}{sending
a shock wave} that killed more than 130 people and wounded another
5,000.

The story of the ship and its deadly cargo, which emerged on Wednesday
in accounts from Lebanon, Russia and Ukraine, offered a bleak tale about
how legal battles, financial wrangling and, apparently, chronic
negligence, set the stage for
\href{https://www.nytimes.com/interactive/2020/08/04/world/middleeast/beirut-explosion-damage.html}{a
horrific accident that devastated} one of the Middle East's most fondly
regarded cities.

``I was horrified,'' said Boris Prokoshev, the ship's 70-year-old
retired Russian captain, about the accident, speaking in a phone
interview from Sochi, Russia, a Black Sea resort town just up the coast
from where the ammonium nitrate began its journey to Beirut in 2013.

In Lebanon, public rage focused on the negligence of authorities who
were aware of the danger posed by the storage of 2,750 tons of ammonium
nitrate in a warehouse on the Beirut docks, yet failed to act.

Senior customs officials wrote to the Lebanese courts at least six times
from 2014 to 2017, seeking guidance on how to dispose of the ammonium
nitrate, according to public records
\href{https://twitter.com/SalimAoun/status/1290799728960180227}{posted
to social media} by a Lebanese lawmaker, Salim Aoun.

``In view of the serious danger posed by keeping this shipment in the
warehouses in an inappropriate climate,'' Shafik Marei, the director of
Lebanese customs, wrote in May 2016, ``we repeat our request to demand
the maritime agency to re-export the materials immediately.''

The customs officials proposed a number of solutions, including donating
the ammonium nitrate to the Lebanese Army, or selling it to the
privately owned Lebanese Explosives Company. Mr. Marei sent a second,
similar letter a year later. The judiciary failed to respond to any of
his pleas, the records suggested.

Lebanese judicial officials could not be reached for comment on
Wednesday.

\includegraphics{https://static01.nyt.com/images/2020/08/05/world/05lebanon-ship2/merlin_175300623_0ce6e984-75a5-4eee-a718-91551fd89708-articleLarge.jpg?quality=75\&auto=webp\&disable=upscale}

The Rhosus, which flew the flag of Moldova, arrived in Beirut in
November 2013, two months after it left the Black Sea port of Batumi, in
Georgia. The ship was leased by Igor Grechushkin, a Russian businessman
living in Cyprus.

Mr. Prokoshev, the captain, joined the ship in Turkey after a mutiny
over unpaid wages by a previous crew. Mr. Grechushkin had been paid \$1
million to transport the high-density ammonium nitrate to the port of
Beira in Mozambique, the captain said.

The ammonium nitrate was purchased by the International Bank of
Mozambique for Fábrica de Explosivos de Moçambique, a firm that makes
commercial explosives, according to Baroudi and Partners, a Lebanese law
firm representing the ship's crew, in a statement issued on Wednesday.

Mr. Grechushkin, who was in Cyprus at the time and communicating by
telephone, told the captain he didn't have enough money to pay for
passage through the Suez Canal. So he sent the ship to Beirut to earn
some cash by taking on an additional cargo of heavy machinery.

But in Beirut, the machinery would not fit into the ship, which was
about 30 or 40 years old, the captain said.

Then Lebanese officials found the ship unseaworthy and impounded the
vessel for failing to pay the port docking fees and other charges. When
the ship's suppliers tried to contact Mr. Grechushkin for payment for
fuel, food and other essentials, he could not be reached, having
apparently abandoned the ship he had leased.

Six crew members returned home, but Lebanese officials forced the
captain and three Ukrainian crew members to remain on board until the
debt issue was solved. Lebanese immigration restrictions prevented the
crew from leaving the ship, and they struggled to obtain food and other
supplies, according to their lawyers.

Mr. Prokoshev, the captain, said Lebanese port officials took pity on
the hungry crew and provided food. But, he added, they didn't show any
concern about the ship's highly dangerous cargo. ``They just wanted the
money we owed,'' he said.

Their plight attracted attention back in Ukraine, where news accounts
described the stranded crew as ``hostages,'' trapped aboard an abandoned
ship.

The captain, a Russian citizen, appealed to the Russian Embassy in
Lebanon for help, but got only snippy comments like, ``Do you expect
President Putin to send special forces to get you out,'' he recalled.

Increasingly desperate, Mr. Prokoshev sold some of the ship's fuel and
used the proceeds to hire a legal team, and these lawyers also warned
the Lebanese authorities that the ship was in danger ``of sinking or
blowing up at any moment,'' according to the law firm's statement.

A Lebanese judge ordered the release of the crew on compassionate
grounds in August 2014, and Mr. Grechushkin, having resurfaced, paid for
their passage back to Ukraine.

Mr. Grechushkin could not be reached for comment on Wednesday.

The crew's departure left the Lebanese authorities in charge of the
ship's deadly cargo, which was moved to a storage facility known as
Hangar 12, where it remained until the explosion on Tuesday.

Image

Carrying a wounded person from the port of Beirut on
Tuesday.~Credit...Anwar Amro/Agence France-Presse --- Getty Images

\href{https://www.nytimes.com/2020/08/05/world/middleeast/beirut-explosion-ammonium-nitrate.html}{Ammonium
nitrate}, when mixed with fuel, creates a powerful explosive commonly
used in construction and mining. But it has also been used to make
explosive devices deployed by terrorists such as the 1995 Oklahoma
bomber, Timothy McVeigh, and the Islamic State.

Sales of ammonium nitrate are regulated in the United States, and many
European countries require it to be mixed with other substances to make
it less potent.

The general manager of Beirut's port, Hassan Koraytem, said in an
interview that customs and security officials made repeated requests to
Lebanon's courts to have the volatile material moved. ``But nothing
happened,'' he said.

``We were told the cargo would be sold in an auction,'' he added. ``But
the auction never happened and the judiciary never acted.''

Mr. Koraytem, who has been in charge of the port for 17 years, said that
when he first heard the blast on Tuesday, he figured it might be an air
attack.

He had ``no idea'' what caused the initial fire at the storage facility
that preceded the second, far larger blast, he said. Four of his
employees died in the explosion. ``This is not the time to blame,'' he
said. ``We are living a national catastrophe.''

But for many Lebanese, the story is another sign of the chronic
mismanagement of a ruling class that steered the country into a
punishing economic crisis this year.

Mr. Prokshev, who said he is still owed \$60,000 in wages, placed the
fault with Mr. Grechushkin, and with Lebanese officials, who insisted on
first impounding the boat, and then on keeping the ammonium nitrate in
the port ``instead of spreading it on their fields.''

``They could have had very good crops instead of a huge explosion,'' he
said.

As for the Rhosus, Mr. Prokoshev learned from friends who sailed to
Beirut that it had sunk in the harbor in 2015 or 2016, after taking
water on board, he said.

His only surprise on hearing this, he added, was that it had not gone
down sooner.

Declan Walsh reported from Cairo, and Andrew Higgins from Moscow.
Reporting was contributed by Hwaida Saad and Ben Hubbard in Beirut, Nada
Rashwan in Cairo and Christiaan Triebert in New York.

Advertisement

\protect\hyperlink{after-bottom}{Continue reading the main story}

\hypertarget{site-index}{%
\subsection{Site Index}\label{site-index}}

\hypertarget{site-information-navigation}{%
\subsection{Site Information
Navigation}\label{site-information-navigation}}

\begin{itemize}
\tightlist
\item
  \href{https://help.nytimes.com/hc/en-us/articles/115014792127-Copyright-notice}{©~2020~The
  New York Times Company}
\end{itemize}

\begin{itemize}
\tightlist
\item
  \href{https://www.nytco.com/}{NYTCo}
\item
  \href{https://help.nytimes.com/hc/en-us/articles/115015385887-Contact-Us}{Contact
  Us}
\item
  \href{https://www.nytco.com/careers/}{Work with us}
\item
  \href{https://nytmediakit.com/}{Advertise}
\item
  \href{http://www.tbrandstudio.com/}{T Brand Studio}
\item
  \href{https://www.nytimes.com/privacy/cookie-policy\#how-do-i-manage-trackers}{Your
  Ad Choices}
\item
  \href{https://www.nytimes.com/privacy}{Privacy}
\item
  \href{https://help.nytimes.com/hc/en-us/articles/115014893428-Terms-of-service}{Terms
  of Service}
\item
  \href{https://help.nytimes.com/hc/en-us/articles/115014893968-Terms-of-sale}{Terms
  of Sale}
\item
  \href{https://spiderbites.nytimes.com}{Site Map}
\item
  \href{https://help.nytimes.com/hc/en-us}{Help}
\item
  \href{https://www.nytimes.com/subscription?campaignId=37WXW}{Subscriptions}
\end{itemize}
