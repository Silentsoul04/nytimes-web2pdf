Sections

SEARCH

\protect\hyperlink{site-content}{Skip to
content}\protect\hyperlink{site-index}{Skip to site index}

\href{https://myaccount.nytimes.com/auth/login?response_type=cookie\&client_id=vi}{}

\href{https://www.nytimes.com/section/todayspaper}{Today's Paper}

The Hiroshima Pilot Who Became a Symbol of Antinuclear Protest

\href{https://nyti.ms/39XjO0G}{https://nyti.ms/39XjO0G}

\begin{itemize}
\item
\item
\item
\item
\item
\item
\end{itemize}

Advertisement

\protect\hyperlink{after-top}{Continue reading the main story}

Supported by

\protect\hyperlink{after-sponsor}{Continue reading the main story}

Beyond the World War II We Know

\hypertarget{the-hiroshima-pilot-who-became-a-symbol-of-antinuclear-protest}{%
\section{The Hiroshima Pilot Who Became a Symbol of Antinuclear
Protest}\label{the-hiroshima-pilot-who-became-a-symbol-of-antinuclear-protest}}

Claude Eatherly spent years punishing himself for his role in the first
atomic bombing. His remorse made him an international celebrity.

\includegraphics{https://static01.nyt.com/images/2020/08/06/multimedia/06ww2-bombing-eatherly-01/06ww2-bombing-eatherly-01-articleLarge.jpg?quality=75\&auto=webp\&disable=upscale}

By Anne I. Harrington

\begin{itemize}
\item
  Aug. 6, 2020
\item
  \begin{itemize}
  \item
  \item
  \item
  \item
  \item
  \item
  \end{itemize}
\end{itemize}

\emph{\emph{\emph{The latest article from
``}\href{https://www.nytimes.com/spotlight/beyond-wwii}{\emph{Beyond the
World War II We Know}}},'' a series from The Times that documents
lesser-known stories from the war, looks at Claude Eatherly, an American
pilot involved in the atomic bombing of Hiroshima. After years of being
arrested for petty crimes, he became a high-profile antinuclear
activist.}**

The B-29 bomber banked hard to avoid the blast. The explosion lit the
plane's interior with a brilliant flash, so bright that some of the
aviators momentarily thought they had been blinded. More than one noted
a strange metallic taste in his mouth. A loud clap broke around them as
the first of three shock waves hit, causing the plane's aluminum body to
vibrate violently. Looking down, they saw the fireball unfurling.

The American airmen who flew the mission to drop the
\href{https://www.nytimes.com/2020/08/05/world/asia/hiroshima-japan-75th-anniversary.html}{atomic
bomb on the Japanese city of Hiroshima} on Aug. 6, 1945, were witnessing
a man-made cataclysm unlike anything seen in the previous history of
human warfare. They watched as fire swallowed the city whole: ``It was
like no ordinary fire,'' a crew member later recalled. ``It contained a
dozen colors, all of them blindingly bright.'' Just when it appeared
that the explosion was subsiding, ``a kind of mushroom spurted out of
the top and traveled up, up to what some say was a distance of 60,000 or
70,000 feet.''

The atomic bomb was the most ferociously deadly weapon ever created by
human ingenuity --- a technology that multiplied the power of these few
men and planes to a degree out of all comprehensible scale. In Hiroshima
alone, some 70,000 people were killed instantly --- a horrific deed fit
for gods or monsters --- but overhead in their plane the airmen were
normal men in human bodies, no more able than anyone else to fully
comprehend or bear responsibility for the mission they had been chosen
to execute.

\includegraphics{https://static01.nyt.com/images/2020/08/06/multimedia/06ww2-bombing-eatherly-02/merlin_175232370_7ba3457f-8138-408c-a201-7165cb507fbb-articleLarge.jpg?quality=75\&auto=webp\&disable=upscale}

In the ensuing decades, only one of the 90 servicemen who flew the
atomic bombing missions, Maj. Claude Eatherly, came forward to publicly
declare that he felt remorse for what he had done. Eatherly, then an
outgoing 26-year-old Texan, piloted the advance weather plane tasked
with assessing target visibility over Hiroshima, giving the go ahead to
drop the bomb that day. His role in the bombing would haunt him for the
rest of his life.

{[}\href{https://www.nytimes.com/newsletters/at-war?module=inline}{\emph{Sign
up for the At War newsletter}} \emph{for more about World War II.}{]}

The discrepancy between the tremendous power of humanity's inventions
and the limited ability of any single person to comprehend, let alone
control the moral and practical implications of that power, is what
Günther Anders, the postwar German-Jewish philosopher and antinuclear
activist, called ``the Promethean gap.'' Prometheus is a character from
Greek mythology who stole fire from the gods and gave it to humans. With
fire, humans were launched on the road to evermore powerful inventions
--- a cascade of technological advances that would also unleash new
forms of death, destruction and exploitation. In the Greek myth, the
gods punished Prometheus with eternal torment.

For Anders, the U.S. service members tasked with dropping the atomic
bombs on Hiroshima and Nagasaki were the prime example of people caught
in the Promethean gap. On the one hand, these U.S. servicemen were cogs
in the atomic machine. They were couriers sent to deliver a deadly
message about U.S. capability and commitment to winning the war. If one
of them were to decline the assignment, someone else would have stepped
up to fill his shoes. Under these circumstances, it was possible to be
``guiltlessly guilty.'' On the other hand, as participants in and
witnesses to the violence, these men came closer to connecting with the
physical consequences of and responsibility for their actions than any
others.

Once their initial sense of astonishment subsided, most of the airmen
reconciled themselves to the bombings by focusing on their affiliation
to their fellow American servicemen, whose lives they may have saved by
obviating a need for a ground invasion of Japan. Others simply distanced
themselves from the morality of the decision entirely. Col. Paul Tibbets
Jr., who commanded the Army Air Forces unit tasked with delivering the
atomic bombs and piloted the plane that dropped the bomb on Hiroshima,
defended his actions until his dying days. ``I made up my mind then that
the morality of dropping that bomb was not my business,'' he told an
interviewer in 1989. ``I have never lost a night's sleep on the deal.''

Unlike Tibbets, Eatherly reported suffering from nightmares about the
bombings, and his guilt drove him into a spiral of self sabotage. In
April 1957, Newsweek ran an article: ``Hero in Handcuffs,'' which
reported that Eatherly was in a jail cell in Fort Worth after breaking
into two post offices in rural Texas. It described a tattered postwar
life: Eatherly had been in and out of psychiatric treatment at a V.A.
hospital in Waco, had served time in a New Orleans jail for forging a
check and had been involved in a series of stickups at small-town
grocery stores. But his crimes were so poorly executed --- at least once
he fled the scene, leaving the money behind --- that his psychiatrist
and one of his defense attorneys separately reached the conclusion that
Eatherly must have intended to get caught. At his trial for the
post-office burglaries, Eatherly's psychiatrist testified that his
patient suffered from a guilt complex stemming above all from his role
in the bombing of Hiroshima. In carrying out these petty crimes, what
Eatherly actually wanted was punishment. A jury found him ``not guilty
by reason of insanity'' and he was released.

Image

The B-29 Superfortress crew that flew over Japan and radioed that the
weather appeared clear before the Enola Gay dropped the atomic bomb on
Hiroshima. Eatherly is standing in the center of the back
row.Credit...U.S. Air Force

Eatherly's guilt fascinated Anders because it provided him with a
glimmer of hope for humanity --- a path forward for nuclear peace
activists through the Promethean gap. For Eatherly, his dutiful service
and the standard justification that the atomic bombings saved lives by
ending the war, were not enough to quiet his conscience. In 1959, Anders
wrote to Eatherly and they struck up a correspondence. Anders was eager
to co-opt the pilot's story in the service of generating political will
to eliminate nuclear weapons, casting Eatherly as ``a symbol of the
future.'' For his part, Eatherly quickly developed the hope that Anders
would provide the platform that he lacked. ``Through writers like
yourself,'' Eatherly wrote in one of his first letters to Anders,
``someone will \ldots{} give a message that will influence the world
toward a reconciliation and peace. You may be the man, if I can be of
any help to you, count on me.''

In a 1961 interview with reporter Ronnie Dugger, Eatherly explained that
he was not convinced by the orthodox explanation about the atomic bomb
as a war winning weapon; the Japanese were putting up so little
resistance by early August that Eatherly believed the war would have
ended even without the nuclear devastation. Logically, he knew that if
it had not been him, it would have been someone else to give the go
ahead to drop the bomb. Yet, he still appeared to feel, and suffer
under, the enormity of his role in the atomic bombings. Anders saw in
Eatherly's behavior a person attempting, in his own way, to be held
accountable for his actions rather than finding ways to disclaim or
reject responsibility.

With Anders' encouragement, Eatherly sent a message to the people of
Hiroshima. ``I told them I was the Major that gave the `go ahead' to
destroy Hiroshima, that I was unable to forget the act, and that the
guilt of the act has caused me great suffering,'' Eatherly reported to
Anders. ``I asked them to forgive me.'' Thirty ``girls of Hiroshima,''
young hibakusha, or atomic bomb victims, left alive but scarred by the
blast, responded. ``We have learned to feel towards you a
fellow-feeling,'' they wrote, ``thinking that you are also a victim of
war like us.''

Image

Claude Eatherly speaking with a news reporter in a Dallas city jail
after he was arrested for attempted armed robbery in March
1959.Credit...FK/Associated Press

By the 1960s, Eatherly became something of a cause célèbre, especially
in war-ravaged Europe and Asia where his remorse fulfilled a deep-seated
desire for compassion. In 1961, Anders published his correspondence with
Eatherly, complete with a preface from renowned British philosopher,
mathematician and antinuclear activist Bertrand Russell. In 1962,
Eatherly was one of four people given ``Hiroshima Awards'' for
``outstanding contributions to world peace'' at a major peace
demonstration in New York. NBC made a TV drama based on his story. The
British papers embraced him as a symbol of antinuclear protest. Poets
described his plight in verse.

The more visible Eatherly became as a symbol of peace and disarmament,
the more heated the debate was about the sincerity of his experiences
and feelings. Journalists wrote detailed books and articles examining
his claims and motives. ``Is it possible,'' investigative reporter
William Bradford Huie asked in his 1965 book, ``The Hiroshima Pilot,''
``that Eatherly feigned guilt to attract attention and perhaps profit?''
Instead of guilt, Huie suggested an inferiority complex. '' The truth,''
he concluded, ``seems to be that when Claude Eatherly began evidencing
mental illness and `turning to crime', he was a disappointed and
immature man who thought he had been overlooked. \ldots{} Instead of
being the Hero of Hiroshima, he was a man who was disappointed at having
been left out of the attack on Hiroshima.'' Eatherly himself was
silenced by throat cancer and died in 1978, at the age of 59, in a
veterans hospital in Houston.

We are now living in the 75th year of the atomic age. Eatherly's
experiences deepen our understanding of the human dimensions of what it
means to undertake enormous acts of wartime violence. His remorse
highlighted the ethical quandary of winning a righteous war that
nevertheless cost such a huge toll in human lives. Passing judgment on
whether he was a hero for speaking out about his suffering, or a
malingerer out to capitalize on his wartime experiences became a way to
stake a claim within the debate about nuclear weapons. Eatherly was
aware of his predicament, and it grew into a larger question on the use
of nuclear weapons that would go on to outlast him and his legacy: ``I
have been having such difficulty in getting society to recognize the
fact of my guilt, which I have long since realized,'' Eatherly lamented
to Anders in one of his letters. ``The truth is that society simply
\emph{cannot} accept the fact of my guilt without at the same time
recognizing its own far deeper guilt.''

\textbf{Anne I. Harrington} is an associate professor in the department
of politics and international relations at Cardiff University. Her
research focuses on the causes of nuclear proliferation and the
international politics of nuclear weapons.

Advertisement

\protect\hyperlink{after-bottom}{Continue reading the main story}

\hypertarget{site-index}{%
\subsection{Site Index}\label{site-index}}

\hypertarget{site-information-navigation}{%
\subsection{Site Information
Navigation}\label{site-information-navigation}}

\begin{itemize}
\tightlist
\item
  \href{https://help.nytimes.com/hc/en-us/articles/115014792127-Copyright-notice}{©~2020~The
  New York Times Company}
\end{itemize}

\begin{itemize}
\tightlist
\item
  \href{https://www.nytco.com/}{NYTCo}
\item
  \href{https://help.nytimes.com/hc/en-us/articles/115015385887-Contact-Us}{Contact
  Us}
\item
  \href{https://www.nytco.com/careers/}{Work with us}
\item
  \href{https://nytmediakit.com/}{Advertise}
\item
  \href{http://www.tbrandstudio.com/}{T Brand Studio}
\item
  \href{https://www.nytimes.com/privacy/cookie-policy\#how-do-i-manage-trackers}{Your
  Ad Choices}
\item
  \href{https://www.nytimes.com/privacy}{Privacy}
\item
  \href{https://help.nytimes.com/hc/en-us/articles/115014893428-Terms-of-service}{Terms
  of Service}
\item
  \href{https://help.nytimes.com/hc/en-us/articles/115014893968-Terms-of-sale}{Terms
  of Sale}
\item
  \href{https://spiderbites.nytimes.com}{Site Map}
\item
  \href{https://help.nytimes.com/hc/en-us}{Help}
\item
  \href{https://www.nytimes.com/subscription?campaignId=37WXW}{Subscriptions}
\end{itemize}
