Sections

SEARCH

\protect\hyperlink{site-content}{Skip to
content}\protect\hyperlink{site-index}{Skip to site index}

\href{https://myaccount.nytimes.com/auth/login?response_type=cookie\&client_id=vi}{}

\href{https://www.nytimes.com/section/todayspaper}{Today's Paper}

\href{/section/opinion}{Opinion}\textbar{}Martin Peretz, on
Israeli-Palestinian Peace

\href{https://nyti.ms/3kiTRgB}{https://nyti.ms/3kiTRgB}

\begin{itemize}
\item
\item
\item
\item
\item
\end{itemize}

Advertisement

\protect\hyperlink{after-top}{Continue reading the main story}

\href{/section/opinion}{Opinion}

Supported by

\protect\hyperlink{after-sponsor}{Continue reading the main story}

letters

\hypertarget{martin-peretz-on-israeli-palestinian-peace}{%
\section{Martin Peretz, on Israeli-Palestinian
Peace}\label{martin-peretz-on-israeli-palestinian-peace}}

The former New Republic editor writes about the two-state solution.
Also: Donald Trump and Joe McCarthy; the theater industry, in dire
straits.

Aug. 6, 2020

\begin{itemize}
\item
\item
\item
\item
\item
\end{itemize}

\hypertarget{more-from-our-inbox}{%
\subsubsection{More from our inbox:}\label{more-from-our-inbox}}

\begin{itemize}
\tightlist
\item
  \protect\hyperlink{link-1ec0a03a}{Joe McCarthy's Biographer Sees an
  Echo in Trump}
\item
  \protect\hyperlink{link-64e228fc}{Help for Theater Workers}
\end{itemize}

\includegraphics{https://static01.nyt.com/images/2020/07/31/opinion/31cohenWeb/merlin_168152001_a0c3adaf-6181-41d2-9c1d-fb79543f7c1e-articleLarge.jpg?quality=75\&auto=webp\&disable=upscale}

\textbf{To the Editor:}

Re
``\href{https://www.nytimes.com/2020/07/31/opinion/israeli-palestinian-peace.html}{The
Less Impossible Israeli-Palestinian Peace}'' (column, nytimes.com, July
31):

Roger Cohen rightly supports a two-state solution against its
detractors. But the column doesn't mention Israel's detailed peace
offers and increased Israeli public support for a two-state solution
while ignoring the Palestinian leadership's refusal to deal.

Israelis have made or agreed to four substantive peace offers or
frameworks in the last 20 years that the Palestinian leadership outright
refused: Ehud Barak's peace offer in 2000; the Taba peace framework in
2001; Ehud Olmert's peace offer in 2008, and the 2014 Kerry framework,
which Benjamin Netanyahu's government quietly supported.

What's more, Israeli support for a two-state solution has grown
steadily. Twenty-four years ago, when the Israeli Labor Party put in its
platform that it did not oppose a Palestinian state, it was the first
mainstream Israeli political party to take this stance. Today, support
for a Palestinian state as part of a comprehensive peace and mutual
recognition is a consensus position across the Israeli political
spectrum, excepting the settler parties.

Many people on both sides want a two-state solution. But neither Israeli
politics nor its public sentiments are the reasons we lack one. The
reason is simpler: If the price of Palestinian statehood is dealing with
Israel, the Palestinian leadership would prefer not to have a state.

Martin Peretz\\
New York\\
\emph{The writer was for 40 years the editor of The New Republic.}

\hypertarget{joe-mccarthys-biographer-sees-an-echo-in-trump}{%
\subsection{Joe McCarthy's Biographer Sees an Echo in
Trump}\label{joe-mccarthys-biographer-sees-an-echo-in-trump}}

Image

Joseph McCarthy in 1950.Credit...Bettmann Archive/Getty Images

\textbf{To the Editor:}

President Trump is proving again why he is America's conspirator in
chief. First he dispatched agents to Portland, Ore., because its leaders
had ``lost control of the anarchists and agitators.'' Then he warned
that federal forces might be needed in Chicago, where violence is
``worse than Afghanistan.'' All the while, his TV ads are laying out a
false narrative, alleging that Democratic mayors are letting protesters
sow bedlam.

Terrifying, yes, but hardly original. Mr. Trump is once more pirating
the playbook of this country's all-time grand conspirator, ``Low Blow''
Joe McCarthy. Just as the crusading senator sprayed a hose of Communist
red on all he vilified, so the rabble-rousing president is painting his
perceived enemies as a treasonable blue.

Seventy years ago, McCarthy showed how far he'd go when he put in his
cross hairs Gen. George C. Marshall, mastermind of Allied military
operations during World War II. The senator placed the general at the
epicenter of ``a conspiracy so immense and an infamy so black as to
dwarf any previous such venture in the history of man.''

Conspiracy theories like the one involving Marshall are a central reason
that 63 years after his death Joe McCarthy's name remains an ism that
stands for smear mongering and guilt by association. Are you listening,
President Trump?

Larry Tye\\
Cotuit, Mass.\\
\emph{The writer is the author of ``Demagogue: The Life and Long Shadow
of Senator Joe McCarthy.''}

\hypertarget{help-for-theater-workers}{%
\subsection{Help for Theater Workers}\label{help-for-theater-workers}}

Image

A street in London's West End, where theaters have been shut since late
March.Credit...Dominic Lipinski/Press Association, via Associated Press

\textbf{To the Editor:}

Re
``\href{https://www.nytimes.com/2020/07/07/theater/theater-bailout-britain.html?searchResultPosition=1}{Britain
Saves Its Arts as U.S. Theater Sags}'' (Critic's Notebook, July 8):

There's no question that the American theater industry needs relief. We
were among the first to lose our jobs, and will be the last to go back
to work. The arts have already suffered \$9 billion in losses. There
will be more losses to come.

But there will not be an industry to save unless attention is paid to
the people who make theater possible: actors, stage managers,
stagehands, musicians and more.

Because the Senate has not yet acted, the \$600 weekly Pandemic
Unemployment Compensation payment has expired. One Equity member here in
New York, married to a union stagehand, told me that they saw \$110,000
in salary evaporate over a few days in March as their shows were
canceled. They have a young child. The additional \$600 is the only
thing keeping them afloat.

Because the industry is largely shut down, arts union health funds are
going without contributions. The longer the industry is closed, the more
workers are at risk for losing their health care.

Far from being ``complacent and uncoordinated,'' Equity called for
immediate relief when the closings began in March, and partnered with
other unions and employer groups to ensure that the CARES Act's
unemployment provisions included freelance arts and entertainment
workers.

We are taking that same approach now: to renew the expired unemployment
provisions, to ensure that arts workers do not lose their health
insurance, and to urge Congress to be bolder on arts funding.

Kate Shindle\\
New York\\
\emph{The writer is president of the Actors' Equity Association.}

Advertisement

\protect\hyperlink{after-bottom}{Continue reading the main story}

\hypertarget{site-index}{%
\subsection{Site Index}\label{site-index}}

\hypertarget{site-information-navigation}{%
\subsection{Site Information
Navigation}\label{site-information-navigation}}

\begin{itemize}
\tightlist
\item
  \href{https://help.nytimes.com/hc/en-us/articles/115014792127-Copyright-notice}{©~2020~The
  New York Times Company}
\end{itemize}

\begin{itemize}
\tightlist
\item
  \href{https://www.nytco.com/}{NYTCo}
\item
  \href{https://help.nytimes.com/hc/en-us/articles/115015385887-Contact-Us}{Contact
  Us}
\item
  \href{https://www.nytco.com/careers/}{Work with us}
\item
  \href{https://nytmediakit.com/}{Advertise}
\item
  \href{http://www.tbrandstudio.com/}{T Brand Studio}
\item
  \href{https://www.nytimes.com/privacy/cookie-policy\#how-do-i-manage-trackers}{Your
  Ad Choices}
\item
  \href{https://www.nytimes.com/privacy}{Privacy}
\item
  \href{https://help.nytimes.com/hc/en-us/articles/115014893428-Terms-of-service}{Terms
  of Service}
\item
  \href{https://help.nytimes.com/hc/en-us/articles/115014893968-Terms-of-sale}{Terms
  of Sale}
\item
  \href{https://spiderbites.nytimes.com}{Site Map}
\item
  \href{https://help.nytimes.com/hc/en-us}{Help}
\item
  \href{https://www.nytimes.com/subscription?campaignId=37WXW}{Subscriptions}
\end{itemize}
