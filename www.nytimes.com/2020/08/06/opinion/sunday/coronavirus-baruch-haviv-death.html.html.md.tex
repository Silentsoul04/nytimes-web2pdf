\href{/section/opinion/sunday}{Sunday Review}\textbar{}Could This Be the
Last Time We See Our Dad?

\href{https://nyti.ms/33CUZ8V}{https://nyti.ms/33CUZ8V}

\begin{itemize}
\item
\item
\item
\item
\item
\item
\end{itemize}

\includegraphics{https://static01.nyt.com/images/2020/08/09/opinion/00ron9/00ron9-articleLarge.jpg?quality=75\&auto=webp\&disable=upscale}

Sections

\protect\hyperlink{site-content}{Skip to
content}\protect\hyperlink{site-index}{Skip to site index}

\href{/section/opinion}{Opinion}

\hypertarget{could-this-be-the-last-time-we-see-our-dad}{%
\section{Could This Be the Last Time We See Our
Dad?}\label{could-this-be-the-last-time-we-see-our-dad}}

Zoom has become a link between loved ones. But is what we see onscreen
real life?

Talia Haviv taking a photograph of Zoom call details to send to her
siblings in order to see their father, Barry Haviv.Credit...

Supported by

\protect\hyperlink{after-sponsor}{Continue reading the main story}

Photographs and Text by Ron Haviv

Mr. Haviv is a photojournalist.

\begin{itemize}
\item
  Aug. 6, 2020
\item
  \begin{itemize}
  \item
  \item
  \item
  \item
  \item
  \item
  \end{itemize}
\end{itemize}

New York City's streets and restaurants were still packed in early
March, as Gov. Andrew Cuomo and Mayor Bill de Blasio argued over
methodology, logistics and the economic repercussions of whether to shut
the city down to curb the spread of the coronavirus.

The government
\href{https://www.politico.com/news/2020/03/30/coronavirus-masks-trump-administration-156327}{dismissed}
the benefits of wearing masks. But my sisters, Tamar, Talia and Elana,
and I knew that our father, Barry Haviv, an 82-year-old stroke survivor
who lived in Midtown, needed protection. We got him N95 masks and begged
him not to leave his apartment.

\includegraphics{https://static01.nyt.com/images/2020/08/05/opinion/05ron1b/05ron1b-articleLarge.jpg?quality=75\&auto=webp\&disable=upscale}

Our father first came down with a cough and a fever, and then, on April
8, he was taken to the hospital after collapsing. The city's health care
system was already overwhelmed, so he was sent back home. The next day,
he received a diagnosis of Covid-19.

Ten days after our father became ill, he was admitted to Mount Sinai
West in Manhattan, a short walk from where two of my siblings and I
live. And yet, he may as well have been in another state. Suddenly we
had become one of the many families for whom Zoom and caring strangers
became a link for loved ones. It was surreal to communicate with him in
this way.

Image

During a Zoom call joined by her brother, Ron, Mr. Haviv's daughter
Elana felt as if her heart and spirit were there in the hospital room in
Manhattan, rather than in a room in New Mexico. Invisible, nonexistent,
and intangible --- her Zoom room.

It was hard to hear him over the roar of air purifiers and oxygen
machines, and it was nearly impossible to make out what his doctors were
saying through their double masks and face shields. We strove to set up
routines and waited for his body to heal. Amazed when he seemed closer
to himself and crushed when he wasn't.

Image

Talia, a health care worker, was the designated member of the family to
be in the hospital.

Image

Mr. Haviv on a Zoom call.

Because our father lapsed into a contentious state that affected his
care, the hospital agreed to allow one family member to visit in person.
My sister Talia, who had recovered from Covid-19, stayed with him. The
hospital provided a mounted iPad linked to Zoom, which enabled us to
spend hours together every day, Talia in person and the rest of us in a
digital cocoon.

Image

Talia spent more than four hours a day helping the medical team care for
her father.

Image

A medical chart in Mr. Haviv's room.

Image

The family drew strength from the video calls.

Image

Talia taking a break while staff members tended to her father.

When we'd end our daily video call, we wondered if it would be the last
time we would see our father. Each day that Talia was able to return in
person was another day we felt we had more of an understanding of this
disease and control over what was happening.

Image

Almost every day doctors provided video briefings to Mr. Haviv's
children.

Image

``We are amazed with your father's response,'' the doctors would say.
The next day, they'd tell us it was ``time for hospice,'' and then back
again.

But was what we saw onscreen real life? One day we were told that our
father was dying, which came as a shock. From where I was standing,
things did not appear so dire. After all, he responded to our voices and
reacted when we played songs from his favorite operas and Broadway
musicals.

Our father hated hospitals. Months after he suffered a stroke four years
ago, he flew overseas on his own to visit friends. He had defied medical
predictions in the past, and we were reluctant to accept this latest
one.

Image

There was enormous pressure to determine the best course of action for
Mr. Haviv's care.

Image

One of his daughters, Tamar, would sometimes sign into Zoom late at
night, to try to feel a little closer and make sure he knew that he was
not alone.

Image

Talia relaying information about Mr. Haviv's condition. There was no joy
on this roller coaster.

The next day another doctor told us that he was going to make it. But
each day would bring a new health issue. Did we want to intubate if
needed? Resuscitate? Some decisions he had already made in his living
will; others were left to us to figure out. The pressure was immense.

We had another conversation about taking him home to die. Then within
hours, we were told he was accepted into a plasma trial. Again we went
from discussing taking him home to die to trying something new.

Image

In the end, after our father was hospitalized for five weeks and even
tested negative for the coronavirus, a doctor told us there was nothing
left that the staff could do for him, and asked, did we want to take him
home? We listened to the digital transmission, unable to fully accept
what was being said. I couldn't help but wonder what would have happened
if he had been given plasma a week earlier. We still held out hope that
once home, away from the noise and medical interruptions, he could
improve.

Image

After Mr. Haviv was discharged from the hospital, Talia had some time
with him at home.

I have spent my career shining a light on how the decisions leaders make
inflict harm on innocent people. My father's last days in a way mirrored
what I documented in Iraq, the Balkans and other war zones. He shouldn't
have died --- he and too many others are gone because of our
\href{https://www.nytimes.com/2020/04/11/us/politics/coronavirus-trump-response.html}{government's
failure}.

Image

Barry Haviv passed away on May 24.

I can't imagine how our family would have coped if my sister hadn't been
allowed to visit. What the hospital did for us, which would have been
normal at other times, was extraordinary during this time. For our
father to have had the chance to have his daughter nearby, and his
children to have had more time with him, albeit through a screen, is
irreplaceable. With
\href{https://www.nytimes.com/interactive/2020/us/coronavirus-us-cases.html?action=click\&module=Top\%20Stories\&pgtype=Homepage}{infections
spiking} across the country, I hope that other families will be allowed
the same experience for themselves and their loved ones.

Ron Haviv (\href{https://twitter.com/ronhaviv?lang=en}{@ronhaviv}), a
photographer and filmmaker, is the co-founder of the
\href{https://viiphoto.com/}{VII Agency} and
\href{https://theviifoundation.org/}{VII Foundation}.

\emph{The Times is committed to publishing}
\href{https://www.nytimes.com/2019/01/31/opinion/letters/letters-to-editor-new-york-times-women.html}{\emph{a
diversity of letters}} \emph{to the editor. We'd like to hear what you
think about this or any of our articles. Here are some}
\href{https://help.nytimes.com/hc/en-us/articles/115014925288-How-to-submit-a-letter-to-the-editor}{\emph{tips}}\emph{.
And here's our email:}
\href{mailto:letters@nytimes.com}{\emph{letters@nytimes.com}}\emph{.}

\emph{Follow The New York Times Opinion section on}
\href{https://www.facebook.com/nytopinion}{\emph{Facebook}}\emph{,}
\href{http://twitter.com/NYTOpinion}{\emph{Twitter (@NYTopinion)}}
\emph{and}
\href{https://www.instagram.com/nytopinion/}{\emph{Instagram}}\emph{.}

Advertisement

\protect\hyperlink{after-bottom}{Continue reading the main story}

\hypertarget{site-index}{%
\subsection{Site Index}\label{site-index}}

\hypertarget{site-information-navigation}{%
\subsection{Site Information
Navigation}\label{site-information-navigation}}

\begin{itemize}
\tightlist
\item
  \href{https://help.nytimes.com/hc/en-us/articles/115014792127-Copyright-notice}{©~2020~The
  New York Times Company}
\end{itemize}

\begin{itemize}
\tightlist
\item
  \href{https://www.nytco.com/}{NYTCo}
\item
  \href{https://help.nytimes.com/hc/en-us/articles/115015385887-Contact-Us}{Contact
  Us}
\item
  \href{https://www.nytco.com/careers/}{Work with us}
\item
  \href{https://nytmediakit.com/}{Advertise}
\item
  \href{http://www.tbrandstudio.com/}{T Brand Studio}
\item
  \href{https://www.nytimes.com/privacy/cookie-policy\#how-do-i-manage-trackers}{Your
  Ad Choices}
\item
  \href{https://www.nytimes.com/privacy}{Privacy}
\item
  \href{https://help.nytimes.com/hc/en-us/articles/115014893428-Terms-of-service}{Terms
  of Service}
\item
  \href{https://help.nytimes.com/hc/en-us/articles/115014893968-Terms-of-sale}{Terms
  of Sale}
\item
  \href{https://spiderbites.nytimes.com}{Site Map}
\item
  \href{https://help.nytimes.com/hc/en-us}{Help}
\item
  \href{https://www.nytimes.com/subscription?campaignId=37WXW}{Subscriptions}
\end{itemize}
