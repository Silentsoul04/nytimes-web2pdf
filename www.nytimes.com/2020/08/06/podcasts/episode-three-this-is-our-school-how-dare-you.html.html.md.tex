Sections

SEARCH

\protect\hyperlink{site-content}{Skip to
content}\protect\hyperlink{site-index}{Skip to site index}

\href{https://www.nytimes.com/spotlight/podcasts}{Podcasts}

\href{https://myaccount.nytimes.com/auth/login?response_type=cookie\&client_id=vi}{}

\href{https://www.nytimes.com/section/todayspaper}{Today's Paper}

\href{/spotlight/podcasts}{Podcasts}\textbar{}Episode Three: `This Is
Our School, How Dare You?'

\url{https://nyti.ms/2Dm4RcM}

\begin{itemize}
\item
\item
\item
\item
\item
\end{itemize}

\href{https://www.nytimes.com/spotlight/at-home?action=click\&pgtype=Article\&state=default\&region=TOP_BANNER\&context=at_home_menu}{At
Home}

\begin{itemize}
\tightlist
\item
  \href{https://www.nytimes.com/2020/08/14/dining/lobster-salad-recipe.html?action=click\&pgtype=Article\&state=default\&region=TOP_BANNER\&context=at_home_menu}{Make:
  Lobster Salad}
\item
  \href{https://www.nytimes.com/2020/08/15/at-home/coronavirus-at-home-quick-exercises.html?action=click\&pgtype=Article\&state=default\&region=TOP_BANNER\&context=at_home_menu}{Sneak
  In: Exercise}
\item
  \href{https://www.nytimes.com/interactive/2020/at-home/even-more-reporters-editors-diaries-lists-recommendations.html?action=click\&pgtype=Article\&state=default\&region=TOP_BANNER\&context=at_home_menu}{See:
  Reporters' Obsessions}
\item
  \href{https://www.nytimes.com/2020/08/15/at-home/coronavirus-fall-patio-furniture.html?action=click\&pgtype=Article\&state=default\&region=TOP_BANNER\&context=at_home_menu}{Deck
  Out: Your Porch}
\end{itemize}

Advertisement

\protect\hyperlink{after-top}{Continue reading the main story}

transcript

Back to Nice White Parents

bars

0:00/46:55

-46:55

transcript

\hypertarget{episode-three-this-is-our-school-how-dare-you}{%
\subsection{Episode Three: `This Is Our School, How Dare
You?'}\label{episode-three-this-is-our-school-how-dare-you}}

\hypertarget{reported-by-chana-joffe-walt-produced-by-julie-snyder-edited-by-sarah-koenig-neil-drumming-and-ira-glass-editorial-consulting-by-eve-l-ewing-and-rachel-lissy-and-sound-mix-by-stowe-nelson}{%
\subsubsection{Reported by Chana Joffe-Walt; produced by Julie Snyder;
edited by Sarah Koenig, Neil Drumming and Ira Glass; editorial
consulting by Eve L. Ewing and Rachel Lissy; and sound mix by Stowe
Nelson}\label{reported-by-chana-joffe-walt-produced-by-julie-snyder-edited-by-sarah-koenig-neil-drumming-and-ira-glass-editorial-consulting-by-eve-l-ewing-and-rachel-lissy-and-sound-mix-by-stowe-nelson}}

\hypertarget{how-white-parents-can-shape-a-school-even-when-they-arent-there}{%
\paragraph{How white parents can shape a school, even when they aren't
there.}\label{how-white-parents-can-shape-a-school-even-when-they-arent-there}}

Thursday, August 6th, 2020

\begin{itemize}
\tightlist
\item
  announcer\\
  ``Nice White Parents'' is brought to you by Serial Productions, a New
  York Times Company.
\end{itemize}

chana joffe-walt

I.S. 293 opened in 1968. Renee Flowers was part of the first generation
of students to walk in the door.

renee flowers

And you're talking about the building on Court Street. I went to school
there. It was nice. And it was brand new. It was nice.

\begin{itemize}
\item
  chana joffe-walt\\
  Were you nervous about going?
\item
  renee flowers\\
  Because all your friends --- all your friends from the neighborhood
  was there.
\end{itemize}

chana joffe-walt

The Gowanus neighborhood where Renee grew up, and still lives --- the
housing project's three blocks away from the school. Renee went to I.S.
293, graduated. And she kept going back to the building to play
handball, to vote, to attend graduations. Renee coaches the neighborhood
drill team, and they'd perform at the school. For years, she'd regularly
go watch the basketball tournament. Renee is in her 60s. She's just
retired from the post office. The school has been a fixture for most of
her life. She knows every part of the building.

\begin{itemize}
\tightlist
\item
  renee flowers\\
  Actually, if you go in on the Baltic Street entrance, the school
  safety sitting. Then when you walk actually into the building, the
  auditorium is right to your right.
\end{itemize}

chana joffe-walt

As we we're talking, she closes her eyes. She can see it.

\begin{itemize}
\tightlist
\item
  renee flowers\\
  Walk up a little more and turn left. And another left, the gym is
  right there. I know exactly where everything is --- the cafeteria ---
\end{itemize}

chana joffe-walt

Renee has this stack of old I.S. 293 yearbooks in her apartment --- even
the years she wasn't a student there.

She'll take the yearbooks out for Gowanus Old Timers Day every August.
Renee is Black. She has never wondered why I.S. 293 is located on Court
Street; why they all had to walk to the edge of the white neighborhood
to get to school. She never heard about the battle over where the
building would be located or the white parents who wanted a fringe
school. Renee just knew the school was theirs. Imani Gayle Gillison told
me the same thing --- 293 was ours.

\begin{itemize}
\tightlist
\item
  imani gayle gillison\\
  White folks were going to 29 or somewhere else. I don't where they
  went really, but they weren't even at 293. So we didn't even see them.
\end{itemize}

chana joffe-walt

Imani was very eager to talk about 293, which I found charming because
she didn't even go there. She says she was one of the only Gowanus kids
whose parents entered a Catholic school, and she's never forgotten it.
She was so jealous of her brothers, all her friends at I.S. 293. They
called it I.S. or Nathan Hale Junior High School. The kids would all
walk home together in a big group. And Imani remembers seeing them in
their green Nathan Hale sweaters, hearing them sing the Nathan Hale
school song.

\begin{itemize}
\item
  imani gayle gillison\\
  There was a pride in their school. They would sometimes be singing it
  on the way home and stuff.
\item
  chana joffe-walt\\
  Really?
\item
  imani gayle gillison\\
  Yeah.
\item
  chana joffe-walt\\
  Little 10-year-old boys?
\item
  imani gayle gillison\\
  {[}LAUGHS{]} Yeah, really. That was their anthem. (SINGING) in the
  Continental Army was a soldier of renown. Nathan Hale, his name was
  known to be. He was captured by the British in a lone, lone England
  town, so he died for his own country. So he died for his own country
  to keep his ---
\end{itemize}

chana joffe-walt

From Serial Productions, I'm Chana Joffe-Walt. This is ``Nice White
Parents.'' We're telling the story of one public school building to see
if it's possible to create a school that is equal and integrated. This
episode --- what if we dropped the integrated part?

{[}music{]}

I was talking to an academic recently, a sociologist and writer who
studies education, a Black woman named Eve Ewing. I was telling her what
I was working on. And at some point in the conversation, she asked me,
why are you so obsessed with integration? It threw me. I guess I'm
obsessed with integration because it feels like an obvious goal. It's
the best way to equalize schools, empirically in terms of test scores
and outcomes. But also, segregation is antithetical to the American
promise --- life, liberty. Segregation is anathema to all of that. It's
caste. But after seeing what happened at SIS, the year the new white
families came in, and after learning about how the school was founded on
a false ideal of integration, how unreliable white families were, how
they paid no attention to the actual voices and needs of families of
color, I don't know. Why expend energy chasing white people who don't
actually want to participate or don't even show up? Maybe it's better to
set aside integration entirely and focus instead on the kids who do show
up. For decades after it opened, I.S. 293 was largely a segregated
school. There weren't any white parents pushing their wispy ideas of
integration. The school was pretty much left alone. I'd seen what
happens when nice white parents came inside the building. Was it better
when they stayed out?

To start, I should say that I.S. 293 was not an experiment in Black
self-governance. There were schools like that opening all over the
country --- schools founded on the premise that you didn't need white
families to get a good education. Integration was not the answer. These
schools focused on Black power. They developed afrocentric curricula and
insisted on people of color in leadership positions. I.S. 293 was not
that. It was a pretty average 1970s public school. The principal was
white. The teachers were almost all white. The local community school
board, also white. The kids were Black and brown. There were always some
white kids at 293, but they were a small minority. I wanted to know ---
was I.S. 293 a good school back then? There wasn't much in the official
record --- some math and writing scores that weren't great. But aside
from that, there was curiously little written about the school. Most
schools show up here and there in the archive or in news reports --- not
293. which could mean everything was going just fine. Or it could mean
the school was falling apart. I found names of some I.S. 293 alumni in
Renee's yearbooks.

\begin{itemize}
\item
  chana joffe-walt\\
  Was it a good school?
\item
  speaker\\
  I don't really remember. I think I was one of the first ones to wear
  pants.
\item
  chana joffe-walt\\
  You were one of the first girls to wear pants?
\item
  speaker\\
  Yeah, underneath a skirt. {[}LAUGHTER{]}
\end{itemize}

chana joffe-walt

The pants have nothing to do with how the school was. It's just what she
members of this time in her life. She wanted to wear pants. I had a lot
of conversations like this. People had fond memories of I.S. 293. People
had sad memories. But mostly, they had very specific memories.

\begin{itemize}
\tightlist
\item
  rspeaker\\
  Jomar Brandon --- yeah, he was like the basketball superstar back
  then. I just thought he was the cutest thing ever.
\end{itemize}

chana joffe-walt

I heard about the song that was on repeat the summer before 7th grade,
the lighting in the basement, Mr. Barringer, the scary dean, whom
everyone called big head Barringer. One I.S. 293 graduate told me what
she remembered was a teacher who used her long nails to eat pumpkin
seeds in class. I met Sheila Saunders at a barbecue by the Gowanus
Houses. She went to 293, so did all of her siblings, friends, nearly
everyone else at this barbecue.

\begin{itemize}
\item
  chana joffe-walt\\
  Was 293 a good school?
\item
  sheila saunders\\
  293 was --- I would say it was good because I had nothing to compare
  it to. That was the local school that we had to go to. So what do we
  use to rate it?
\end{itemize}

chana joffe-walt

Was I.S. 293 a good school during this period of time? The more I asked
it, I recognized what a modern-day question that is. This is the way we
talk about public schools now --- good schools and bad schools. At I.S.
293, there was no school choice. Every neighborhood was zoned to its
designated middle school. Apart from the white families, most everyone
from the community was there --- middle class, working class, poor kids,
Black kids, hispanic, dorky, goofy, arty kids. Everyone went. I.S. 293
wasn't good or bad; it was just school. And then something happened at
I.S. 293. When I was looking through the Board of Ed archives, the year
1984 stood out. It's the year I.S. 293 starts showing up in the records.
That year, a few parents from the school began asking the district
superintendent for an investigation into the local community school
board. One person, writing on behalf of the 293 Parent Association,
suggests an investigation is critical because the board is planning in
secret to harm the school. The board, this person says, is, quote,
``controlled by people who are out for the real estate interest and have
little regard for minorities.'' Another parent writes, they're unhappy
with the local school board because it has, quote, ``ceased to act in
the best interest of our children.'' I couldn't really understand from
the archive exactly what these parents we're talking about. The first
person I thought to ask was Dolores Hadden Smith. So many 293 alumni
mentioned Ms. Smith. Ms. Smith worked at 293 longer than anyone else I
talked to --- 42 years.

\begin{itemize}
\tightlist
\item
  dolores hadden smith\\
  Everybody's name --- I knew everybody's name. They said, Ms. Smith was
  my teacher. And the kids --- in the projects would say, Ms. Smith was
  everybody's teacher. I know them by name. I have their addresses.
\end{itemize}

chana joffe-walt

She grew up in the neighborhood. Her mom worked at the school, her
sister, her brother.

\begin{itemize}
\tightlist
\item
  dolores hadden smith\\
  I called their parents right from the classroom. Our community, we
  were like one big family.
\end{itemize}

chana joffe-walt

Ms. Smith told me everything was fine. And then in the `80s, they
started messing with us, started messing with who they sent to our
school.

\begin{itemize}
\item
  dolores hadden smith\\
  I can't say why. It's just that you noticed it. It was blatant. You
  could see it for yourself. Nobody had to tell you that.
\item
  chana joffe-walt\\
  How could you see it? What made it possible?
\item
  dolores hadden smith\\
  Because the children they were putting in there were lower-functioning
  children. They weren't at the top, the creme de la creme. They wasn't
  those children anymore.
\end{itemize}

chana joffe-walt

There's a local news article from 1987 where the principal of 293 says
the same thing. He accuses the district of, quote, ``skimming off the
high-achieving students from his school,'' specifically poaching white
students. Ms. Smith says that just started disappearing.

\begin{itemize}
\item
  dolores hadden smith\\
  And they were offered, behind the curtain, other options that you
  could go to the places that maybe some of the other children weren't
  afforded the chance to go.
\item
  chana joffe-walt\\
  So there would be options for white kids that seemed like they were
  happening in ---
\item
  dolores hayden-smith\\
  Well, I know they offered some of the children positions that they
  could take, as opposed to come into our building. I know the Caucasian
  kids were offered other things. They started encouraging them to go
  other places, as opposed to coming to our school. And they're behind
  closed doors. And they never come out and just say it.
\end{itemize}

{[}music{]}

chana joffe-walt

OK. So there was something happening behind closed doors, and the local
community school board was part of it. I took these claims to Norm
Fruchter. He was on the local school board around this time; although, I
figured it was unlikely he'd say, why, yes, we did have a secret plot to
steal 293's high-achievers and white kids. And yet, that is basically
what he said.

\begin{itemize}
\tightlist
\item
  norm fruchter\\
  There was a lot of trepidation, particularly at the middle school
  level, as to whether white parents would stay.
\end{itemize}

chana joffe-walt

Norm says white parents had left his district in the 1970s. They left
the public schools entirely or moved out of the city. Black families
were also leaving in large numbers, but the school board was completely
preoccupied by the white flight. Norm says board members saw a decline
in white students as a serious threat.

\begin{itemize}
\tightlist
\item
  norm fruchter\\
  They equated that with school quality. If you lost white students,
  your achievement levels would go down, right? Your schools would be
  less attractive places for teachers to come into because when they
  thought teachers, they thought white teachers and a whole bunch of
  spillover effects would happen --- what the graduation rates would
  look like.
\end{itemize}

chana joffe-walt

Their solution? A gifted program.

\begin{itemize}
\item
  norm fruchter\\
  The district started the program explicitly to maintain a white
  population.
\item
  chana joffe-walt\\
  That was the explicit goal?
\item
  norm fruchter\\
  That was explicit because the unspoken assumption of the
  administration in our district and every district was that if you had
  a gifted program, it would attract white parents.
\end{itemize}

chana joffe-walt

To get into gifted programs, you had to take a test. Gifted kids would
be taught in separate classrooms. They opened gifted programs in select
elementary schools. And a new gifted program opened in a different
middle school, a school called M.S. 51. This is part of what the people
at I.S. 293 were seeing. Their strongest students were being siphoned
off. White parents, even when they were not inside 293, were beginning
to change the school.

\begin{itemize}
\tightlist
\item
  norm fruchter\\
  Because what you were creating was a predominantly white track within
  the schools, and their kids would get in, no matter what kind of
  testing you used.
\end{itemize}

{[}music{]}

Parents who were committed to getting their kids in the gifted program
could do it.

\begin{itemize}
\item
  chana joffe-walt\\
  White parents?
\item
  norm fruchter\\
  Yeah.
\item
  chana joffe-walt\\
  And what about non-white parents who were committed to getting their
  kids into the gifted program?
\item
  norm fruchter\\
  Well, what you had to also deal with there was a fair amount of bias
  in the testing administration.
\end{itemize}

chana joffe-walt

Norm says there were kids of color who were clearly qualified, but were
not in the gifted program. And he says this was because the questions
were biased, and the people administering the tests were sometimes
biased. He also says parents were hiring their own psychologists to test
their children and paying for test prep. But also, there was another
reason Black and Latino kids were not in the gifted program.

\begin{itemize}
\tightlist
\item
  nadine jackson\\
  I was a nerd --- yes. I was an honor student.
\end{itemize}

chana joffe-walt

Nadine Jackson might have been one of those kids who would have
qualified as gifted. She was a student at I.S. 293, a Black kid from the
Gowanus Projects.

\begin{itemize}
\item
  nadine jackson\\
  I was never absent --- math honor roll all the time. I was on the
  dean's list. I mean, I was that nerdy child. I've always wanted to be
  a professional. I've always wanted to be someone of importance.
\item
  chana joffe-walt\\
  You've always wanted to be someone of importance?
\item
  nadine jackson\\
  Always, always. I wanted to be an actress or a teacher.
\end{itemize}

chana joffe-walt

Nadine was not kept out of the gifted program because of bias or lack of
test prep. She simply had never heard of the program. She went to the
school everyone else went to. She started seventh grade at I.S. 293 in
1993. And Nadine was eager to jump in --- ready to be delivered to
importance with hard work which she put in. Nadine studied computer
technology. She played first clarinet in the band. She played Whitney
Houston, ``I Have Nothing'' on clarinet over and over. In her first
year, the I.S. 293 band went to perform at another middle school nearby
--- M.S. 51, the school with the gifted program. When Nadine arrived
there, she walked right into an experience a lot of kids have when they
leave their school and enter a world of wealthier kids.

\begin{itemize}
\tightlist
\item
  nadine jackson\\
  We were amazed at just the way they operate was completely different.
  They had a huge orchestra there. We had a small one here. And we were
  just amazed how they would just outshine us. I mean, they have better
  resources. They have better equipment. They have better instruments.
  Everything was top-notch. And us, it was more like second-class hand
  me downs.
\end{itemize}

chana joffe-walt

M.S. 51 and I.S. 293 were in the same school district. They were
governed by the same local community school board, and they were a mile
and half away from each other. When I asked Nadine the same question I
had asked previous graduates of I.S. 293, what was the school like, she
described the feeling of being trapped. She told me, it was normal at
293 to have 42 kids in a class. She said teachers came and went
frequently in the middle of the school year. She had six or seven social
studies teachers in one year. I was skeptical about the numbers, but I
looked into it. And all of this seems entirely plausible for those
years. There was a recession. School budgets were decimated. In 1991,
New York City proposed 250 million dollars in educational spending cuts.
In 1992, 600 million. School programs are being cut mid-year; class
sizes ballooned; teachers were moved around, a lot. At the same time,
Nadine's district was supporting gifted programs, bussing white kids out
of their zoned schools, hiring separate teachers, administering special
tests, running an entirely separate educational track. At M.S. 51, the
gifted middle school, there were not 40 plus kids in a class; there were
30. The school was written up in a book from the 1990s called ``New York
City's Best Public Middle Schools.'' It describes the school's leaders
as masters at developing faculty, talent and enthusiasm. The M.S. 51
principal is quoted saying, ``when we started the gifted program, we got
parents who were more involved, more inquisitive.'' He then goes on to
say, ``the gifted program shifted his whole educational approach. It
made him recognize that children in early adolescence need close contact
with nurturing adults.'' And he began to hire teachers who he saw as,
quote, ``warm and comforting.'' I.S. 293 and M.S. 51 were both public
middle schools. But that day she visited M.S. 51, Nadine felt like this
school --- this is the school that's preparing kids to be someone of
importance.

\begin{itemize}
\item
  nadine jackson\\
  The education system is better.

  The way they talked is different. They were so smart. The children
  there, they were taking their Regents at a very early stage.
\end{itemize}

chana joffe-walt

Regents are state tests kids normally take in high school.

\begin{itemize}
\tightlist
\item
  nadine jackson\\
  Then it's like, oh, my goodness, the way that students carry
  themselves was different, as if they knew something that we didn't
  know. Like, they had a secret we didn't know of. And when were we
  going to find out?
\end{itemize}

chana joffe-walt

After Nadine's out at M.S. 51, she says it made her see her own school
differently. I.S. 293, to her, looked like a school for chumps.

\begin{itemize}
\tightlist
\item
  nadine jackson\\
  This is where we all went. That's what we knew. That's what our
  parents knew. It really makes you wonder, do we even have a chance?
  You're tying to figure out who you are. How do I fit in to society?
  Where do I put myself? That was hard. It made me feel dumb in a sense.
  I didn't know anything.
\end{itemize}

chana joffe-walt

In the 1980s when the district started grading specialized programs at
other schools, I.S. 293 parents fought back. But Norm Fruchter, the
school board member, told me once the gifted programs were in place,
they were there to stay. The board was serving a constituency of white
parents who believed their kids deserved a program to serve their unique
needs. And he says, those parents wielded tremendous power.

\begin{itemize}
\item
  norm fruchter\\
  There were huge pitch fights in the school board meetings whenever we
  put a resolution on the agenda to change the gifted program. They
  could mobilize 500 people for a meeting. So you could fill an
  elementary school auditorium with gifted program parents, or, as we
  used to say the district, gifted parents, as if somehow the ---
\item
  chana joffe-walt\\
  The giftedness got passed up toward them?
\item
  norm fruchter\\
  Yeah. And they called themselves that as well. And one of the many
  things they argued was that it was important to maintain the white
  population in the gifted program in order to have some semblance of
  integration in the schools, and that there were benefits that would
  flow from the gifted program to the rest of the school.
\item
  chana joffe-walt\\
  Who argued that? The parents?
\item
  norm fruchter\\
  Yes.
\item
  chana joffe-walt\\
  The gifted parents?
\item
  norm fruchter\\
  Yeah. Yeah. {[}LAUGHS{]}
\end{itemize}

chana joffe-walt

They argued that the gifted program, designed to serve white families,
was actually an integration program, when, in fact, it was a separate
track in the school that kept Black and brown kids from resources from
special programs, which is what segregation was designed to do --- to
separate. This was its latest adaptation, and it wasn't the last. That's
after the break.

1994, about halfway through I.S. 293's 60-year history, and here's where
things stood --- 293 did not have any white parents messing with things
inside of the building. But white families in the district were drawing
resources away from I.S. 293 by creating specialty gifted programs in
other schools. I.S. 293 was separate and increasingly unequal. And that
is when Judi Aronson enters the scene --- a woman who is not connected
to I.S. 293, but was about to be. History is about to repeat itself.

\begin{itemize}
\tightlist
\item
  judi aronson\\
  My daughter was in third or fourth grade, and I felt that there was
  not a viable middle school for her.
\end{itemize}

chana joffe-walt

At the same time, Nadine, the nerdy honor roll student was starting
junior high school at 293, Judi had a daughter who was finishing
elementary school. Judi's daughter was zoned for M.S. 51, the school
with the gifted program, but Judi wasn't excited about that school.

\begin{itemize}
\tightlist
\item
  judi aronson\\
  It was a big school, very traditional, not a very exciting curriculum,
  fairly segregated because it had a segregated gifted program that was
  mostly white. And then the kids of color were in the mainstream at
  that time. And we wanted something a little different. We wanted
  another option for our kids.
\end{itemize}

chana joffe-walt

Judi had been a special ed teacher in a public school. Then she left the
classroom and started working at the Teachers Union, the UFT. Later, she
became a school principal and a superintendent. So she'd spent a lot of
time thinking about schools --- what makes a school successful. And
she'd begun to imagine what it would look like to build something
better.

\begin{itemize}
\tightlist
\item
  judi aronson\\
  I had this idea. I was born in Hungary. And then I lived in Vienna,
  and I grew up in Montreal. And I lived in Brooklyn for the last 46
  years. And we've traveled a lot, my husband and I. And I'm a firm
  believer that you learn so much about the world through other people,
  through talking to them through a variety of cultures. So the idea
  behind the school was that kids would have exchanges.
\end{itemize}

chana joffe-walt

Judi got a group of parents together, a planning committee.

\begin{itemize}
\item
  judi aronson\\
  They wanted something, a school that was diverse, that was
  child-centered, that had a progressive, innovative curriculum, small,
  student-centered, all the buzz words --- excellent teachers, not a
  large school where kids would learn a second language, not the way
  they learn it now, but a lot better, where they would learn about
  different cultures, all those ideas.
\item
  chana joffe-walt\\
  How much was diversity a part of it?
\item
  judi aronson\\
  I think it was very, very much a part of it. And I'm thinking of our
  planning committee. I don't think it was very diverse now looking back
  on it.
\item
  chana joffe-walt\\
  Why did you guys want the school to be diverse? why? was that central
  to what you were doing?
\item
  judi aronson\\
  Well, we all stayed in the city for a reason. And we didn't want --- I
  mean, one of the reasons that we didn't like 51 is that segregation of
  the gifted kids being all white and the rest of the school being
  children of color. So we want it diverse, and I wanted my kids to
  really be accepting of everyone.
\end{itemize}

{[}music{]}

chana joffe-walt

The planning committee put together a 13-page proposal for a new school
called the Brooklyn School for Global Citizenship. The local community
school board approve it; although, somewhere in the process they dropped
the citizenship part --- too controversial --- and it became the
Brooklyn School for Global Studies. OK. So, why am I telling you about
the School for Global Studies? Because this brand new school needed a
building --- the community school board surveyed its options and chose a
spot. The School for Global Studies would be located in the basement of
I.S. 293.

\begin{itemize}
\tightlist
\item
  nadine jackson\\
  One day it was like, you're going to get another school in your
  building. And we were like, how is that possible? Where? How? We only
  have three floors, and it's barely enough for us.
\end{itemize}

chana joffe-walt

Nadine was in eighth grade when this happened, September 1994, and she
was not into this idea.

\begin{itemize}
\tightlist
\item
  nadine jackson\\
  You want to put a new school in, and yet you have 43 kids in a
  classroom. Why? How about you make these classrooms a little bit
  smaller and get more teachers in before you put it in new school? I
  mean, I'm a big advocate of let's fix the problem first before you
  want to add onto things.
\end{itemize}

chana joffe-walt

An article in the New York Times proclaimed, a miracle of a school
opened its doors this fall in Brooklyn, thanks to determined parents
who've created with the new principal called, quote, ``the Taj Mahal of
education.'' Global Studies had class sizes as small as 18 kids. The
curriculum included trips to museums. The students went outdoors to
learn, measured shadows for math. They dug in soil for science
experiments. The students at 293 saw all of that, as they went about
their days at the not Taj Mahal of education, and they were pissed.

\begin{itemize}
\tightlist
\item
  nadine jackson\\
  You're in our lunchroom. You're in our gym. You're in our school yard.
  And it was like, where did these people come from? Where did the
  school come from? How was this even possible? This is our school. This
  is our neighborhood. How dare you?
\end{itemize}

chana joffe-walt

Whenever Nadine or the other 293 kids walked by the global studies
classes, they'd make sure to bang on the classroom doors. And the 293
teachers and staff, school security officers, the custodian, the
principal, they didn't welcome the new school for global cities either.
I heard stories from this time about the staff from Global Studies
asking to put up student work in the hallways and being told by the
long-time 293 custodian, you can't. That's a fire hazard. Global Studies
wanted to use the auditorium for a performance --- sorry, it's occupied.
And I heard this story from the Global Studies principal, a guy named
Larry Abrams, who'd been hired to lead what, to him, sounded like such
an exciting new school, and then he showed up to work.

\begin{itemize}
\item
  larry abrams\\
  The first day there --- or not the first day, the first week, the two
  school cops came down, put me in handcuffs. I said, what? But they
  were joking. They were going to arrest me because I was taking over
  space in the building.
\item
  chana joffe-walt\\
  {[}LAUGHS{]}
\item
  larry abrams\\
  {[}CHUCKLES{]} And now I think about it, it's pretty funny, but ---
\item
  chana joffe-walt\\
  Wait, wait, wait. The school security came down and were like, you're
  ---
\item
  larry abrams\\
  The police department --- you're under arrest. {[}LAUGHS{]}
\item
  chana joffe-walt\\
  This is like the first week of the school?
\item
  larry abrams\\
  Yeah. I forgot --- the first or the second week. I mean, but
  obviously, we weren't welcomed in the place, and it was going to be a
  battle.
\end{itemize}

chana joffe-walt

Remember how I said history repeats itself?

\begin{itemize}
\item
  chana joffe-walt\\
  Oh, your kid didn't go to Global Studies.
\item
  judi aronson\\
  No, no because it took forever.
\end{itemize}

chana joffe-walt

Judi Aronson did not end up sending her daughter to Global because by
the time it opened, her daughter was already in middle school. When her
younger son was old enough for middle school, a couple of years later,
she didn't send him either.

\begin{itemize}
\item
  judi aronson\\
  So I sent him to a new small school in Sheepshead Bay.
\item
  chana joffe-walt\\
  Oh, wow. Wait, you sent him to a small school that was not the small
  school that you made.
\item
  judi aronson\\
  No. Not the small school that I made --- no.
\item
  chana joffe-walt\\
  So your kids didn't even get to go to the school that you created.
\item
  judi aronson\\
  No, no. And it had a lot of rough --- don't ask.
\item
  chana joffe-walt\\
  I am going to ask you about that.
\item
  judi aronson\\
  Oh, my god, the school ran into a lot of problems. There were too many
  challenges. The kids were difficult the teachers had issues. None of
  us sent our kids there.
\end{itemize}

chana joffe-walt

This is not entirely true. I did speak with one parent from the planning
committee who sent her son to Global Studies. Although, she said when
they showed up in September, it looked to her like he was the only white
boy in the school. She said he had a good experience there. Judi decided
what was best for her kids was something else.

{[}music{]}

In an effort to appease white parents, the school district had once
again made a choice that sidelined 293. White parents had said jump, so
the district jumped. And now they were left trying to fill the school
for Global Studies, a school that had no obvious constituency. Most of
the parents who created it didn't send their kids, and the neighborhood
kids already had a school --- I.S. 293. This meant, to fill Global
Studies, the district had to find kids who weren't happy at their
schools, or kids whose schools weren't unhappy with them. Or they had to
bank on families randomly applying to a school they'd never heard of.

\begin{itemize}
\item
  judi aronson\\
  It's one thing if a student says, I want to go to this school because
  this is what I'm passionate about. OK? But that did not happen. So it
  became a place where they placed kids that were difficult. They were
  challenging --- very, very challenging.
\item
  chana joffe-walt\\
  They were acting out when they showed up?
\item
  judi aronson\\
  Yes.
\item
  chana joffe-walt\\
  Well, they were in a school that wasn't designed for them.
\item
  judi aronson\\
  That's true, 100\% true.
\item
  chana joffe-walt\\
  That had this whole vision that had nothing to do with the kids who
  were there.
\item
  judi aronson\\
  Yeah, Yeah. Yep.
\item
  chana joffe-walt\\
  Did you feel bad about that?
\item
  judi aronson\\
  Yes. I mean, yes. Yes, I did --- that we had these great ideas and not
  everything came to fruition. Yes, we opened up a school, but it wasn't
  exactly everything we thought it would be.
\end{itemize}

chana joffe-walt

Within six or seven years, most of the original Global Studies staff had
left, including the principal. Within a decade, nobody knew why the
school was called Global Studies in the first place. Global Studies
became a regular segregated public school, which shared a building with
another segregated public school.

{[}music{]}

In my experience, schools are immune to long-term memory. They get new
principals, new names, a new generation of parents. And they're
populated by children who have no reason to care about what came before
--- clean slate every September. This, I believe, is also what makes it
possible for us to keep repeating the same story. We constantly reset
the clock and move forward. When we look to diagnose the problems of our
public schools, we look at what is in front of us right now. We look
forward. Nobody looks backwards to history. And so the question is not
how do we stop white families from hoarding all the resources. Instead,
the question is, what's going on with the Black kids? This became the
question driving the next era at I.S. 293, the latest era of school
reform --- the mid 1990s right up to today, a time when business people
and American presidents and tech company billionaires committed
themselves to solving the problem of failing public schools. Basically,
it's everything you've heard about schools in the last two decades ---
charter schools, No Child Left Behind, and accountability, the
achievement gap, Race to the Top, these were data-driven initiatives.
They assessed the educational landscape and identified schools that were
failing --- teachers who were not getting results --- children who were
not performing. At I.S. 293, this meant a flurry of new programs that
came and went, sometimes in rapid succession. First, I.S. 293 a grant
from RJR Nabisco to break itself up into small academies --- smaller
schools within the building that would focus on different specialties. A
long-time 293 teacher, Carmen Sanchez, told me, after that, everything
just started changing. The staff turnover was dizzying.

\begin{itemize}
\tightlist
\item
  carmen sanchez\\
  All of a sudden, these people appear, and they are going to be the
  directors, not principals, directors of this math academy and academy
  in music.
\end{itemize}

chana joffe-walt

Ms. Sanchez says one of them came in to run the place, and she opened
her staff meeting by promising to fire everyone.

\begin{itemize}
\item
  carmen sanchez\\
  She lasted maybe nine months. She was gone. People just went --- I
  mean, it was amazing. That was just as revolving door of principals or
  directors, and they just left.
\item
  dolores hadden smith\\
  They keep changing over what kind of school you're in --- a science
  program school. Or you're in a this. What are we, you know?
\end{itemize}

chana joffe-walt

Ms. Smith, Delores Hadden Smith, was in her third decade working at the
school when this started changing names. They were The Mathematics
Academy, The Academy for Performing and Fine Arts, The School for
Integrated Learning Through the Arts. Teachers left. New staff came in,
new initiatives. They needed to be smaller, more specialized. They
needed more science. They needed a trade. They needed to be a 6 through
12 school, middle and high school. Ms. Smith says this was confusing for
the parents, especially the parents in her community, the Gowanus
community, parents who went to 293 and knew it as 293. Now, they were
asking Ms. Smith, what happened to 293? The School for Integrated
Learning Through the Arts, what's that mean?

\begin{itemize}
\tightlist
\item
  dolores hadden smith\\
  Well, I'm not sending my child there. I don't want my child to go to a
  performing arts school. I want my child to go get academics. But we
  gave both. But they made it like it was a tap dance school. And they
  said, I don't want my kids to go to a tap dance school. I want my kids
  go where they can get an education. Well, they thought we wasn't
  teaching education because we are performing arts in our schools also?
  That was just a feature, one of the many features that we did. But the
  parents didn't get it.
\end{itemize}

chana joffe-walt

By this time, public school admissions allowed more choice about where
parents sent their kids. So some of these local parents started choosing
other schools. 293 was losing students, which meant they were losing
money. A new principal came in --- and an assistant principal named Jeff
Chetirko. By that point, 293 had been renamed The School for
International Studies. But not even assistant principal Chetirko knew
why. Prospective parents would ask him, why should I send my kids here?
What does international mean?

\begin{itemize}
\item
  jeff chetirko\\
  So I remember just having this horrible response --- would be like,
  oh, yeah, our students come from all over the world, and that's really
  what it's about. It's about our diversity, which is kind of bull. But
  that's what I would sell because it didn't sound like we really spoke
  a lot about it in the curriculum. And eventually ---
\item
  chana joffe-walt\\
  You did you have students from all over the world, right?
\item
  jeff chetirko\\
  That's true.
\item
  chana joffe-walt\\
  I mean, you had students from maybe the Caribbean, from Yemen.
\item
  jeff chetirko\\
  Yeah, towards the end, I think we had more from Caribbean. Or if we
  had that one student, we would be like, yeah, they're from all over
  the world. {[}LAUGHS{]} You just make stuff up because you're just
  trying to sell it.
\end{itemize}

chana joffe-walt

By 2003, SIS had low enrollment and terrible test scores. The state put
it on a failing schools list, the dreaded SURR list, Schools Under
Registration Review. Being on a failing schools list made it harder to
sell the school to prospective families, but it did mean SIS got a chunk
of money to turn things around. They bought new reading programs, an
academic intervention program. They doubled periods for reading and
math. During this time, the leadership was stable --- less teacher
turnover. The school was less chaotic. The test scores stabilized. Jeff
Chetirko says they were feeling good about where things were headed.
Still, they had to compete for students. So he hired a marketing firm to
help draw families in.

\begin{itemize}
\tightlist
\item
  jeff chetirko\\
  I remember meeting the guy a couple of times. He had some good ideas.
  I don't really remember what came out of it. It didn't. We hung up
  signs outside the door, just tried to have a different look. But those
  banners, I think that came out of it.
\end{itemize}

chana joffe-walt

He says the marketing idea didn't attract any local families into the
school. Instead, it attracted the attention of the New York Post, which
found out the school was trying to market itself, as it had been told
to, and wrote a snarky article about it. The headline read, ``Lousy
Brooklyn Public School Wants to Hire a Press Agent to Enhance Appeal.''
it goes on to say, quote, ``If they build a buzz, the kids will come.
That's the thinking at a mediocre Brooklyn Public school with grandiose
aspirations.'' The article ends with the list of suggested marketing
slogans for the school. It's mean spirited and racist. Having trouble
with English? So is we. The School for International Studies, the best
six years of your life. Jaguar pride --- where you can go from state
champs to state pen. The Jaguars --- we score baskets; we just can't
count them. Jeff says everyone at the school read it. He distinctly
remembers the feeling.

\begin{itemize}
\tightlist
\item
  jeff chetirko\\
  It's horrible because if you're publicly going to put us on a SURR
  list, what do you think you're doing to that school? So now if we have
  to hire somebody to kind of get us off of that, that perception of
  this school's a failing school, and then to get this newspaper
  article, it just deflates everything. It just really sucks.
  {[}LAUGHS{]} There's no other way to say it. You get that pit feeling
  in your stomach. And you're just like, ugh, what's going on? Or what's
  going to happen next? I think everybody is always nervous about what
  happens next. And then afterwards, you just get super furious.
\end{itemize}

chana joffe-walt

Here is what happened next.

\begin{itemize}
\tightlist
\item
  archived recording\\
  Hi. Hi, How are you? I'm OK. You waited patiently. Oh, my god. Yes.
\end{itemize}

chana joffe-walt

Six years later, I'm standing in a sweaty school gym at a middle school
fair for parents. It's 2017, two years after that gala thrown by the
French Embassy for SIS. A couple dozen schools are here with information
tables. The table for the school for International Studies is mobbed.
There's a line of parents waiting to get a chance to talk with someone
from the school. A mother named Anissa is near the very back of the
school.

\begin{itemize}
\item
  chana joffe-walt\\
  What have you heard about the International School?
\item
  anissa\\
  I heard it's a hot ticket. Everybody wants to get in there.
\end{itemize}

chana joffe-walt

After 40 years of being neglected, messed with by the school board,
after losing students and losing money, losing the building, being
blamed and publicly mocked, SIS was suddenly the hot ticket, as if
history had been wiped away. Parents asked the SIS admissions director,
can their kids get priority if they have good grades? Extracurriculars?
Does attendance count? They want to know if it helps their chances if
they show up for a tour.

\begin{itemize}
\tightlist
\item
  archived recording\\
  Yes. I open access to tours tomorrow at three o'clock.
\end{itemize}

chana joffe-walt

They want to know, will you have enough space for all these people?

\begin{itemize}
\tightlist
\item
  archived recording\\
  Oh, I don't think I've got enough space. For next year, we're only
  accepting 140 sixth graders.
\end{itemize}

chana joffe-walt

Three years earlier, SIS had 30 sixth graders. What changed? The
admissions director is the same. Most of the staff is the same. The
building is the same. The test scores are still pretty low. There's an
IB program now in French. But the biggest change between the era of
being ignored and punished and the era of being celebrated and
oversubscribed is that white kids arrived. That's what's different, nine
times as many white students.

{[}music{]}

I.S. 293 was a mostly segregated school for decades. And still, it was
subject to the whims of white parents. Nice white parents shape public
schools even in our absence because public schools are maniacally loyal
to white families even when that loyalty is rarely returned back to the
public schools. Just the very idea of us, the threat of our displeasure,
warps the whole system. So separate is still not equal because the power
sits with white parents no matter where we are in the system. I think
the only way you equalize schools is by recognizing this fact and trying
wherever possible to suppress the power of white parents. Since no one's
forcing us to give up power. We white parents are going to have to do it
voluntarily. Which, yeah, how's that going to happen? That's next time
on ``Nice White Parents.''

``Nice White Parents'' is produced by Julie Snyder and me, with editing
on this episode from Sarah Koenig and Ira Glass. Neil Drumming is our
Managing Editor. Eve Ewing and Rachel Lissy are our editorial
consultants. Fact-checking and research by Ben Phelan, with additional
research from Lilly Sullivan. Archival research by Rebecca Kent. Music
supervision and mixing by Stowe Nelson, with production help from Aviva
DeKornfeld. Our Director of Operations is Seth Lind. Julie Whitaker is
our Digital Manager. Finance management by Cassie Howley, and Production
management by Frances Swanson. Original music for ``Nice White Parents''
is by The Bad Plus, with additional music written and performed by Matt
McGinley. The music you're hearing right now. is the Nathan Hale trilogy
performed by the Nathan Hale middle school 293 Concert Band. I benefited
from the memories and expertise of many people for this episode: Special
thanks to Charles Jones, Leanna Stiefel, Allison Roda, Ujju Aggarwal,
Clara Hemphill, Steven Schneps, Michael Rebell, Jeffrey Henig, Megan
Thompkins-Strange, Jeffrey Snyder, Dawn Meconi, Maura Walz, Coleen
Mingo, Neil Friedman, Jeff Tripp, Karl Rusnak, Lenny Garcia, Cindy
Black, Arthur Bargonetti, Heather Lewis, Theirry Rafuir, Kevin Davidson
and Afrah Omar.

``Nice White Parents'' is produced by Serial Productions, a New York
Times Company.

\href{https://www.nytimes.com/column/nice-white-parents}{\includegraphics{https://static01.nyt.com/images/2020/07/21/podcasts/nice-white-parents-album-art/nice-white-parents-album-art-square320.jpg}Nice
White Parents}

\hypertarget{episode-three-this-is-our-school-how-dare-you-1}{%
\section{Episode Three: `This Is Our School, How Dare
You?'}\label{episode-three-this-is-our-school-how-dare-you-1}}

\hypertarget{how-white-parents-can-shape-a-school-even-when-they-arent-there-1}{%
\subsection{How white parents can shape a school, even when they aren't
there.}\label{how-white-parents-can-shape-a-school-even-when-they-arent-there-1}}

Reported by Chana Joffe-Walt; produced by Julie Snyder; edited by Sarah
Koenig, Neil Drumming and Ira Glass; editorial consulting by Eve L.
Ewing and Rachel Lissy; and sound mix by Stowe Nelson

Transcript

transcript

Back to Nice White Parents

bars

0:00/46:55

-0:00

transcript

\hypertarget{episode-three-this-is-our-school-how-dare-you-2}{%
\subsection{Episode Three: `This Is Our School, How Dare
You?'}\label{episode-three-this-is-our-school-how-dare-you-2}}

\hypertarget{reported-by-chana-joffe-walt-produced-by-julie-snyder-edited-by-sarah-koenig-neil-drumming-and-ira-glass-editorial-consulting-by-eve-l-ewing-and-rachel-lissy-and-sound-mix-by-stowe-nelson-1}{%
\subsubsection{Reported by Chana Joffe-Walt; produced by Julie Snyder;
edited by Sarah Koenig, Neil Drumming and Ira Glass; editorial
consulting by Eve L. Ewing and Rachel Lissy; and sound mix by Stowe
Nelson}\label{reported-by-chana-joffe-walt-produced-by-julie-snyder-edited-by-sarah-koenig-neil-drumming-and-ira-glass-editorial-consulting-by-eve-l-ewing-and-rachel-lissy-and-sound-mix-by-stowe-nelson-1}}

\hypertarget{how-white-parents-can-shape-a-school-even-when-they-arent-there-2}{%
\paragraph{How white parents can shape a school, even when they aren't
there.}\label{how-white-parents-can-shape-a-school-even-when-they-arent-there-2}}

Thursday, August 6th, 2020

\begin{itemize}
\tightlist
\item
  announcer\\
  ``Nice White Parents'' is brought to you by Serial Productions, a New
  York Times Company.
\end{itemize}

chana joffe-walt

I.S. 293 opened in 1968. Renee Flowers was part of the first generation
of students to walk in the door.

renee flowers

And you're talking about the building on Court Street. I went to school
there. It was nice. And it was brand new. It was nice.

\begin{itemize}
\item
  chana joffe-walt\\
  Were you nervous about going?
\item
  renee flowers\\
  Because all your friends --- all your friends from the neighborhood
  was there.
\end{itemize}

chana joffe-walt

The Gowanus neighborhood where Renee grew up, and still lives --- the
housing project's three blocks away from the school. Renee went to I.S.
293, graduated. And she kept going back to the building to play
handball, to vote, to attend graduations. Renee coaches the neighborhood
drill team, and they'd perform at the school. For years, she'd regularly
go watch the basketball tournament. Renee is in her 60s. She's just
retired from the post office. The school has been a fixture for most of
her life. She knows every part of the building.

\begin{itemize}
\tightlist
\item
  renee flowers\\
  Actually, if you go in on the Baltic Street entrance, the school
  safety sitting. Then when you walk actually into the building, the
  auditorium is right to your right.
\end{itemize}

chana joffe-walt

As we we're talking, she closes her eyes. She can see it.

\begin{itemize}
\tightlist
\item
  renee flowers\\
  Walk up a little more and turn left. And another left, the gym is
  right there. I know exactly where everything is --- the cafeteria ---
\end{itemize}

chana joffe-walt

Renee has this stack of old I.S. 293 yearbooks in her apartment --- even
the years she wasn't a student there.

She'll take the yearbooks out for Gowanus Old Timers Day every August.
Renee is Black. She has never wondered why I.S. 293 is located on Court
Street; why they all had to walk to the edge of the white neighborhood
to get to school. She never heard about the battle over where the
building would be located or the white parents who wanted a fringe
school. Renee just knew the school was theirs. Imani Gayle Gillison told
me the same thing --- 293 was ours.

\begin{itemize}
\tightlist
\item
  imani gayle gillison\\
  White folks were going to 29 or somewhere else. I don't where they
  went really, but they weren't even at 293. So we didn't even see them.
\end{itemize}

chana joffe-walt

Imani was very eager to talk about 293, which I found charming because
she didn't even go there. She says she was one of the only Gowanus kids
whose parents entered a Catholic school, and she's never forgotten it.
She was so jealous of her brothers, all her friends at I.S. 293. They
called it I.S. or Nathan Hale Junior High School. The kids would all
walk home together in a big group. And Imani remembers seeing them in
their green Nathan Hale sweaters, hearing them sing the Nathan Hale
school song.

\begin{itemize}
\item
  imani gayle gillison\\
  There was a pride in their school. They would sometimes be singing it
  on the way home and stuff.
\item
  chana joffe-walt\\
  Really?
\item
  imani gayle gillison\\
  Yeah.
\item
  chana joffe-walt\\
  Little 10-year-old boys?
\item
  imani gayle gillison\\
  {[}LAUGHS{]} Yeah, really. That was their anthem. (SINGING) in the
  Continental Army was a soldier of renown. Nathan Hale, his name was
  known to be. He was captured by the British in a lone, lone England
  town, so he died for his own country. So he died for his own country
  to keep his ---
\end{itemize}

chana joffe-walt

From Serial Productions, I'm Chana Joffe-Walt. This is ``Nice White
Parents.'' We're telling the story of one public school building to see
if it's possible to create a school that is equal and integrated. This
episode --- what if we dropped the integrated part?

{[}music{]}

I was talking to an academic recently, a sociologist and writer who
studies education, a Black woman named Eve Ewing. I was telling her what
I was working on. And at some point in the conversation, she asked me,
why are you so obsessed with integration? It threw me. I guess I'm
obsessed with integration because it feels like an obvious goal. It's
the best way to equalize schools, empirically in terms of test scores
and outcomes. But also, segregation is antithetical to the American
promise --- life, liberty. Segregation is anathema to all of that. It's
caste. But after seeing what happened at SIS, the year the new white
families came in, and after learning about how the school was founded on
a false ideal of integration, how unreliable white families were, how
they paid no attention to the actual voices and needs of families of
color, I don't know. Why expend energy chasing white people who don't
actually want to participate or don't even show up? Maybe it's better to
set aside integration entirely and focus instead on the kids who do show
up. For decades after it opened, I.S. 293 was largely a segregated
school. There weren't any white parents pushing their wispy ideas of
integration. The school was pretty much left alone. I'd seen what
happens when nice white parents came inside the building. Was it better
when they stayed out?

To start, I should say that I.S. 293 was not an experiment in Black
self-governance. There were schools like that opening all over the
country --- schools founded on the premise that you didn't need white
families to get a good education. Integration was not the answer. These
schools focused on Black power. They developed afrocentric curricula and
insisted on people of color in leadership positions. I.S. 293 was not
that. It was a pretty average 1970s public school. The principal was
white. The teachers were almost all white. The local community school
board, also white. The kids were Black and brown. There were always some
white kids at 293, but they were a small minority. I wanted to know ---
was I.S. 293 a good school back then? There wasn't much in the official
record --- some math and writing scores that weren't great. But aside
from that, there was curiously little written about the school. Most
schools show up here and there in the archive or in news reports --- not
293. which could mean everything was going just fine. Or it could mean
the school was falling apart. I found names of some I.S. 293 alumni in
Renee's yearbooks.

\begin{itemize}
\item
  chana joffe-walt\\
  Was it a good school?
\item
  speaker\\
  I don't really remember. I think I was one of the first ones to wear
  pants.
\item
  chana joffe-walt\\
  You were one of the first girls to wear pants?
\item
  speaker\\
  Yeah, underneath a skirt. {[}LAUGHTER{]}
\end{itemize}

chana joffe-walt

The pants have nothing to do with how the school was. It's just what she
members of this time in her life. She wanted to wear pants. I had a lot
of conversations like this. People had fond memories of I.S. 293. People
had sad memories. But mostly, they had very specific memories.

\begin{itemize}
\tightlist
\item
  rspeaker\\
  Jomar Brandon --- yeah, he was like the basketball superstar back
  then. I just thought he was the cutest thing ever.
\end{itemize}

chana joffe-walt

I heard about the song that was on repeat the summer before 7th grade,
the lighting in the basement, Mr. Barringer, the scary dean, whom
everyone called big head Barringer. One I.S. 293 graduate told me what
she remembered was a teacher who used her long nails to eat pumpkin
seeds in class. I met Sheila Saunders at a barbecue by the Gowanus
Houses. She went to 293, so did all of her siblings, friends, nearly
everyone else at this barbecue.

\begin{itemize}
\item
  chana joffe-walt\\
  Was 293 a good school?
\item
  sheila saunders\\
  293 was --- I would say it was good because I had nothing to compare
  it to. That was the local school that we had to go to. So what do we
  use to rate it?
\end{itemize}

chana joffe-walt

Was I.S. 293 a good school during this period of time? The more I asked
it, I recognized what a modern-day question that is. This is the way we
talk about public schools now --- good schools and bad schools. At I.S.
293, there was no school choice. Every neighborhood was zoned to its
designated middle school. Apart from the white families, most everyone
from the community was there --- middle class, working class, poor kids,
Black kids, hispanic, dorky, goofy, arty kids. Everyone went. I.S. 293
wasn't good or bad; it was just school. And then something happened at
I.S. 293. When I was looking through the Board of Ed archives, the year
1984 stood out. It's the year I.S. 293 starts showing up in the records.
That year, a few parents from the school began asking the district
superintendent for an investigation into the local community school
board. One person, writing on behalf of the 293 Parent Association,
suggests an investigation is critical because the board is planning in
secret to harm the school. The board, this person says, is, quote,
``controlled by people who are out for the real estate interest and have
little regard for minorities.'' Another parent writes, they're unhappy
with the local school board because it has, quote, ``ceased to act in
the best interest of our children.'' I couldn't really understand from
the archive exactly what these parents we're talking about. The first
person I thought to ask was Dolores Hadden Smith. So many 293 alumni
mentioned Ms. Smith. Ms. Smith worked at 293 longer than anyone else I
talked to --- 42 years.

\begin{itemize}
\tightlist
\item
  dolores hadden smith\\
  Everybody's name --- I knew everybody's name. They said, Ms. Smith was
  my teacher. And the kids --- in the projects would say, Ms. Smith was
  everybody's teacher. I know them by name. I have their addresses.
\end{itemize}

chana joffe-walt

She grew up in the neighborhood. Her mom worked at the school, her
sister, her brother.

\begin{itemize}
\tightlist
\item
  dolores hadden smith\\
  I called their parents right from the classroom. Our community, we
  were like one big family.
\end{itemize}

chana joffe-walt

Ms. Smith told me everything was fine. And then in the `80s, they
started messing with us, started messing with who they sent to our
school.

\begin{itemize}
\item
  dolores hadden smith\\
  I can't say why. It's just that you noticed it. It was blatant. You
  could see it for yourself. Nobody had to tell you that.
\item
  chana joffe-walt\\
  How could you see it? What made it possible?
\item
  dolores hadden smith\\
  Because the children they were putting in there were lower-functioning
  children. They weren't at the top, the creme de la creme. They wasn't
  those children anymore.
\end{itemize}

chana joffe-walt

There's a local news article from 1987 where the principal of 293 says
the same thing. He accuses the district of, quote, ``skimming off the
high-achieving students from his school,'' specifically poaching white
students. Ms. Smith says that just started disappearing.

\begin{itemize}
\item
  dolores hadden smith\\
  And they were offered, behind the curtain, other options that you
  could go to the places that maybe some of the other children weren't
  afforded the chance to go.
\item
  chana joffe-walt\\
  So there would be options for white kids that seemed like they were
  happening in ---
\item
  dolores hayden-smith\\
  Well, I know they offered some of the children positions that they
  could take, as opposed to come into our building. I know the Caucasian
  kids were offered other things. They started encouraging them to go
  other places, as opposed to coming to our school. And they're behind
  closed doors. And they never come out and just say it.
\end{itemize}

{[}music{]}

chana joffe-walt

OK. So there was something happening behind closed doors, and the local
community school board was part of it. I took these claims to Norm
Fruchter. He was on the local school board around this time; although, I
figured it was unlikely he'd say, why, yes, we did have a secret plot to
steal 293's high-achievers and white kids. And yet, that is basically
what he said.

\begin{itemize}
\tightlist
\item
  norm fruchter\\
  There was a lot of trepidation, particularly at the middle school
  level, as to whether white parents would stay.
\end{itemize}

chana joffe-walt

Norm says white parents had left his district in the 1970s. They left
the public schools entirely or moved out of the city. Black families
were also leaving in large numbers, but the school board was completely
preoccupied by the white flight. Norm says board members saw a decline
in white students as a serious threat.

\begin{itemize}
\tightlist
\item
  norm fruchter\\
  They equated that with school quality. If you lost white students,
  your achievement levels would go down, right? Your schools would be
  less attractive places for teachers to come into because when they
  thought teachers, they thought white teachers and a whole bunch of
  spillover effects would happen --- what the graduation rates would
  look like.
\end{itemize}

chana joffe-walt

Their solution? A gifted program.

\begin{itemize}
\item
  norm fruchter\\
  The district started the program explicitly to maintain a white
  population.
\item
  chana joffe-walt\\
  That was the explicit goal?
\item
  norm fruchter\\
  That was explicit because the unspoken assumption of the
  administration in our district and every district was that if you had
  a gifted program, it would attract white parents.
\end{itemize}

chana joffe-walt

To get into gifted programs, you had to take a test. Gifted kids would
be taught in separate classrooms. They opened gifted programs in select
elementary schools. And a new gifted program opened in a different
middle school, a school called M.S. 51. This is part of what the people
at I.S. 293 were seeing. Their strongest students were being siphoned
off. White parents, even when they were not inside 293, were beginning
to change the school.

\begin{itemize}
\tightlist
\item
  norm fruchter\\
  Because what you were creating was a predominantly white track within
  the schools, and their kids would get in, no matter what kind of
  testing you used.
\end{itemize}

{[}music{]}

Parents who were committed to getting their kids in the gifted program
could do it.

\begin{itemize}
\item
  chana joffe-walt\\
  White parents?
\item
  norm fruchter\\
  Yeah.
\item
  chana joffe-walt\\
  And what about non-white parents who were committed to getting their
  kids into the gifted program?
\item
  norm fruchter\\
  Well, what you had to also deal with there was a fair amount of bias
  in the testing administration.
\end{itemize}

chana joffe-walt

Norm says there were kids of color who were clearly qualified, but were
not in the gifted program. And he says this was because the questions
were biased, and the people administering the tests were sometimes
biased. He also says parents were hiring their own psychologists to test
their children and paying for test prep. But also, there was another
reason Black and Latino kids were not in the gifted program.

\begin{itemize}
\tightlist
\item
  nadine jackson\\
  I was a nerd --- yes. I was an honor student.
\end{itemize}

chana joffe-walt

Nadine Jackson might have been one of those kids who would have
qualified as gifted. She was a student at I.S. 293, a Black kid from the
Gowanus Projects.

\begin{itemize}
\item
  nadine jackson\\
  I was never absent --- math honor roll all the time. I was on the
  dean's list. I mean, I was that nerdy child. I've always wanted to be
  a professional. I've always wanted to be someone of importance.
\item
  chana joffe-walt\\
  You've always wanted to be someone of importance?
\item
  nadine jackson\\
  Always, always. I wanted to be an actress or a teacher.
\end{itemize}

chana joffe-walt

Nadine was not kept out of the gifted program because of bias or lack of
test prep. She simply had never heard of the program. She went to the
school everyone else went to. She started seventh grade at I.S. 293 in
1993. And Nadine was eager to jump in --- ready to be delivered to
importance with hard work which she put in. Nadine studied computer
technology. She played first clarinet in the band. She played Whitney
Houston, ``I Have Nothing'' on clarinet over and over. In her first
year, the I.S. 293 band went to perform at another middle school nearby
--- M.S. 51, the school with the gifted program. When Nadine arrived
there, she walked right into an experience a lot of kids have when they
leave their school and enter a world of wealthier kids.

\begin{itemize}
\tightlist
\item
  nadine jackson\\
  We were amazed at just the way they operate was completely different.
  They had a huge orchestra there. We had a small one here. And we were
  just amazed how they would just outshine us. I mean, they have better
  resources. They have better equipment. They have better instruments.
  Everything was top-notch. And us, it was more like second-class hand
  me downs.
\end{itemize}

chana joffe-walt

M.S. 51 and I.S. 293 were in the same school district. They were
governed by the same local community school board, and they were a mile
and half away from each other. When I asked Nadine the same question I
had asked previous graduates of I.S. 293, what was the school like, she
described the feeling of being trapped. She told me, it was normal at
293 to have 42 kids in a class. She said teachers came and went
frequently in the middle of the school year. She had six or seven social
studies teachers in one year. I was skeptical about the numbers, but I
looked into it. And all of this seems entirely plausible for those
years. There was a recession. School budgets were decimated. In 1991,
New York City proposed 250 million dollars in educational spending cuts.
In 1992, 600 million. School programs are being cut mid-year; class
sizes ballooned; teachers were moved around, a lot. At the same time,
Nadine's district was supporting gifted programs, bussing white kids out
of their zoned schools, hiring separate teachers, administering special
tests, running an entirely separate educational track. At M.S. 51, the
gifted middle school, there were not 40 plus kids in a class; there were
30. The school was written up in a book from the 1990s called ``New York
City's Best Public Middle Schools.'' It describes the school's leaders
as masters at developing faculty, talent and enthusiasm. The M.S. 51
principal is quoted saying, ``when we started the gifted program, we got
parents who were more involved, more inquisitive.'' He then goes on to
say, ``the gifted program shifted his whole educational approach. It
made him recognize that children in early adolescence need close contact
with nurturing adults.'' And he began to hire teachers who he saw as,
quote, ``warm and comforting.'' I.S. 293 and M.S. 51 were both public
middle schools. But that day she visited M.S. 51, Nadine felt like this
school --- this is the school that's preparing kids to be someone of
importance.

\begin{itemize}
\item
  nadine jackson\\
  The education system is better.

  The way they talked is different. They were so smart. The children
  there, they were taking their Regents at a very early stage.
\end{itemize}

chana joffe-walt

Regents are state tests kids normally take in high school.

\begin{itemize}
\tightlist
\item
  nadine jackson\\
  Then it's like, oh, my goodness, the way that students carry
  themselves was different, as if they knew something that we didn't
  know. Like, they had a secret we didn't know of. And when were we
  going to find out?
\end{itemize}

chana joffe-walt

After Nadine's out at M.S. 51, she says it made her see her own school
differently. I.S. 293, to her, looked like a school for chumps.

\begin{itemize}
\tightlist
\item
  nadine jackson\\
  This is where we all went. That's what we knew. That's what our
  parents knew. It really makes you wonder, do we even have a chance?
  You're tying to figure out who you are. How do I fit in to society?
  Where do I put myself? That was hard. It made me feel dumb in a sense.
  I didn't know anything.
\end{itemize}

chana joffe-walt

In the 1980s when the district started grading specialized programs at
other schools, I.S. 293 parents fought back. But Norm Fruchter, the
school board member, told me once the gifted programs were in place,
they were there to stay. The board was serving a constituency of white
parents who believed their kids deserved a program to serve their unique
needs. And he says, those parents wielded tremendous power.

\begin{itemize}
\item
  norm fruchter\\
  There were huge pitch fights in the school board meetings whenever we
  put a resolution on the agenda to change the gifted program. They
  could mobilize 500 people for a meeting. So you could fill an
  elementary school auditorium with gifted program parents, or, as we
  used to say the district, gifted parents, as if somehow the ---
\item
  chana joffe-walt\\
  The giftedness got passed up toward them?
\item
  norm fruchter\\
  Yeah. And they called themselves that as well. And one of the many
  things they argued was that it was important to maintain the white
  population in the gifted program in order to have some semblance of
  integration in the schools, and that there were benefits that would
  flow from the gifted program to the rest of the school.
\item
  chana joffe-walt\\
  Who argued that? The parents?
\item
  norm fruchter\\
  Yes.
\item
  chana joffe-walt\\
  The gifted parents?
\item
  norm fruchter\\
  Yeah. Yeah. {[}LAUGHS{]}
\end{itemize}

chana joffe-walt

They argued that the gifted program, designed to serve white families,
was actually an integration program, when, in fact, it was a separate
track in the school that kept Black and brown kids from resources from
special programs, which is what segregation was designed to do --- to
separate. This was its latest adaptation, and it wasn't the last. That's
after the break.

1994, about halfway through I.S. 293's 60-year history, and here's where
things stood --- 293 did not have any white parents messing with things
inside of the building. But white families in the district were drawing
resources away from I.S. 293 by creating specialty gifted programs in
other schools. I.S. 293 was separate and increasingly unequal. And that
is when Judi Aronson enters the scene --- a woman who is not connected
to I.S. 293, but was about to be. History is about to repeat itself.

\begin{itemize}
\tightlist
\item
  judi aronson\\
  My daughter was in third or fourth grade, and I felt that there was
  not a viable middle school for her.
\end{itemize}

chana joffe-walt

At the same time, Nadine, the nerdy honor roll student was starting
junior high school at 293, Judi had a daughter who was finishing
elementary school. Judi's daughter was zoned for M.S. 51, the school
with the gifted program, but Judi wasn't excited about that school.

\begin{itemize}
\tightlist
\item
  judi aronson\\
  It was a big school, very traditional, not a very exciting curriculum,
  fairly segregated because it had a segregated gifted program that was
  mostly white. And then the kids of color were in the mainstream at
  that time. And we wanted something a little different. We wanted
  another option for our kids.
\end{itemize}

chana joffe-walt

Judi had been a special ed teacher in a public school. Then she left the
classroom and started working at the Teachers Union, the UFT. Later, she
became a school principal and a superintendent. So she'd spent a lot of
time thinking about schools --- what makes a school successful. And
she'd begun to imagine what it would look like to build something
better.

\begin{itemize}
\tightlist
\item
  judi aronson\\
  I had this idea. I was born in Hungary. And then I lived in Vienna,
  and I grew up in Montreal. And I lived in Brooklyn for the last 46
  years. And we've traveled a lot, my husband and I. And I'm a firm
  believer that you learn so much about the world through other people,
  through talking to them through a variety of cultures. So the idea
  behind the school was that kids would have exchanges.
\end{itemize}

chana joffe-walt

Judi got a group of parents together, a planning committee.

\begin{itemize}
\item
  judi aronson\\
  They wanted something, a school that was diverse, that was
  child-centered, that had a progressive, innovative curriculum, small,
  student-centered, all the buzz words --- excellent teachers, not a
  large school where kids would learn a second language, not the way
  they learn it now, but a lot better, where they would learn about
  different cultures, all those ideas.
\item
  chana joffe-walt\\
  How much was diversity a part of it?
\item
  judi aronson\\
  I think it was very, very much a part of it. And I'm thinking of our
  planning committee. I don't think it was very diverse now looking back
  on it.
\item
  chana joffe-walt\\
  Why did you guys want the school to be diverse? why? was that central
  to what you were doing?
\item
  judi aronson\\
  Well, we all stayed in the city for a reason. And we didn't want --- I
  mean, one of the reasons that we didn't like 51 is that segregation of
  the gifted kids being all white and the rest of the school being
  children of color. So we want it diverse, and I wanted my kids to
  really be accepting of everyone.
\end{itemize}

{[}music{]}

chana joffe-walt

The planning committee put together a 13-page proposal for a new school
called the Brooklyn School for Global Citizenship. The local community
school board approve it; although, somewhere in the process they dropped
the citizenship part --- too controversial --- and it became the
Brooklyn School for Global Studies. OK. So, why am I telling you about
the School for Global Studies? Because this brand new school needed a
building --- the community school board surveyed its options and chose a
spot. The School for Global Studies would be located in the basement of
I.S. 293.

\begin{itemize}
\tightlist
\item
  nadine jackson\\
  One day it was like, you're going to get another school in your
  building. And we were like, how is that possible? Where? How? We only
  have three floors, and it's barely enough for us.
\end{itemize}

chana joffe-walt

Nadine was in eighth grade when this happened, September 1994, and she
was not into this idea.

\begin{itemize}
\tightlist
\item
  nadine jackson\\
  You want to put a new school in, and yet you have 43 kids in a
  classroom. Why? How about you make these classrooms a little bit
  smaller and get more teachers in before you put it in new school? I
  mean, I'm a big advocate of let's fix the problem first before you
  want to add onto things.
\end{itemize}

chana joffe-walt

An article in the New York Times proclaimed, a miracle of a school
opened its doors this fall in Brooklyn, thanks to determined parents
who've created with the new principal called, quote, ``the Taj Mahal of
education.'' Global Studies had class sizes as small as 18 kids. The
curriculum included trips to museums. The students went outdoors to
learn, measured shadows for math. They dug in soil for science
experiments. The students at 293 saw all of that, as they went about
their days at the not Taj Mahal of education, and they were pissed.

\begin{itemize}
\tightlist
\item
  nadine jackson\\
  You're in our lunchroom. You're in our gym. You're in our school yard.
  And it was like, where did these people come from? Where did the
  school come from? How was this even possible? This is our school. This
  is our neighborhood. How dare you?
\end{itemize}

chana joffe-walt

Whenever Nadine or the other 293 kids walked by the global studies
classes, they'd make sure to bang on the classroom doors. And the 293
teachers and staff, school security officers, the custodian, the
principal, they didn't welcome the new school for global cities either.
I heard stories from this time about the staff from Global Studies
asking to put up student work in the hallways and being told by the
long-time 293 custodian, you can't. That's a fire hazard. Global Studies
wanted to use the auditorium for a performance --- sorry, it's occupied.
And I heard this story from the Global Studies principal, a guy named
Larry Abrams, who'd been hired to lead what, to him, sounded like such
an exciting new school, and then he showed up to work.

\begin{itemize}
\item
  larry abrams\\
  The first day there --- or not the first day, the first week, the two
  school cops came down, put me in handcuffs. I said, what? But they
  were joking. They were going to arrest me because I was taking over
  space in the building.
\item
  chana joffe-walt\\
  {[}LAUGHS{]}
\item
  larry abrams\\
  {[}CHUCKLES{]} And now I think about it, it's pretty funny, but ---
\item
  chana joffe-walt\\
  Wait, wait, wait. The school security came down and were like, you're
  ---
\item
  larry abrams\\
  The police department --- you're under arrest. {[}LAUGHS{]}
\item
  chana joffe-walt\\
  This is like the first week of the school?
\item
  larry abrams\\
  Yeah. I forgot --- the first or the second week. I mean, but
  obviously, we weren't welcomed in the place, and it was going to be a
  battle.
\end{itemize}

chana joffe-walt

Remember how I said history repeats itself?

\begin{itemize}
\item
  chana joffe-walt\\
  Oh, your kid didn't go to Global Studies.
\item
  judi aronson\\
  No, no because it took forever.
\end{itemize}

chana joffe-walt

Judi Aronson did not end up sending her daughter to Global because by
the time it opened, her daughter was already in middle school. When her
younger son was old enough for middle school, a couple of years later,
she didn't send him either.

\begin{itemize}
\item
  judi aronson\\
  So I sent him to a new small school in Sheepshead Bay.
\item
  chana joffe-walt\\
  Oh, wow. Wait, you sent him to a small school that was not the small
  school that you made.
\item
  judi aronson\\
  No. Not the small school that I made --- no.
\item
  chana joffe-walt\\
  So your kids didn't even get to go to the school that you created.
\item
  judi aronson\\
  No, no. And it had a lot of rough --- don't ask.
\item
  chana joffe-walt\\
  I am going to ask you about that.
\item
  judi aronson\\
  Oh, my god, the school ran into a lot of problems. There were too many
  challenges. The kids were difficult the teachers had issues. None of
  us sent our kids there.
\end{itemize}

chana joffe-walt

This is not entirely true. I did speak with one parent from the planning
committee who sent her son to Global Studies. Although, she said when
they showed up in September, it looked to her like he was the only white
boy in the school. She said he had a good experience there. Judi decided
what was best for her kids was something else.

{[}music{]}

In an effort to appease white parents, the school district had once
again made a choice that sidelined 293. White parents had said jump, so
the district jumped. And now they were left trying to fill the school
for Global Studies, a school that had no obvious constituency. Most of
the parents who created it didn't send their kids, and the neighborhood
kids already had a school --- I.S. 293. This meant, to fill Global
Studies, the district had to find kids who weren't happy at their
schools, or kids whose schools weren't unhappy with them. Or they had to
bank on families randomly applying to a school they'd never heard of.

\begin{itemize}
\item
  judi aronson\\
  It's one thing if a student says, I want to go to this school because
  this is what I'm passionate about. OK? But that did not happen. So it
  became a place where they placed kids that were difficult. They were
  challenging --- very, very challenging.
\item
  chana joffe-walt\\
  They were acting out when they showed up?
\item
  judi aronson\\
  Yes.
\item
  chana joffe-walt\\
  Well, they were in a school that wasn't designed for them.
\item
  judi aronson\\
  That's true, 100\% true.
\item
  chana joffe-walt\\
  That had this whole vision that had nothing to do with the kids who
  were there.
\item
  judi aronson\\
  Yeah, Yeah. Yep.
\item
  chana joffe-walt\\
  Did you feel bad about that?
\item
  judi aronson\\
  Yes. I mean, yes. Yes, I did --- that we had these great ideas and not
  everything came to fruition. Yes, we opened up a school, but it wasn't
  exactly everything we thought it would be.
\end{itemize}

chana joffe-walt

Within six or seven years, most of the original Global Studies staff had
left, including the principal. Within a decade, nobody knew why the
school was called Global Studies in the first place. Global Studies
became a regular segregated public school, which shared a building with
another segregated public school.

{[}music{]}

In my experience, schools are immune to long-term memory. They get new
principals, new names, a new generation of parents. And they're
populated by children who have no reason to care about what came before
--- clean slate every September. This, I believe, is also what makes it
possible for us to keep repeating the same story. We constantly reset
the clock and move forward. When we look to diagnose the problems of our
public schools, we look at what is in front of us right now. We look
forward. Nobody looks backwards to history. And so the question is not
how do we stop white families from hoarding all the resources. Instead,
the question is, what's going on with the Black kids? This became the
question driving the next era at I.S. 293, the latest era of school
reform --- the mid 1990s right up to today, a time when business people
and American presidents and tech company billionaires committed
themselves to solving the problem of failing public schools. Basically,
it's everything you've heard about schools in the last two decades ---
charter schools, No Child Left Behind, and accountability, the
achievement gap, Race to the Top, these were data-driven initiatives.
They assessed the educational landscape and identified schools that were
failing --- teachers who were not getting results --- children who were
not performing. At I.S. 293, this meant a flurry of new programs that
came and went, sometimes in rapid succession. First, I.S. 293 a grant
from RJR Nabisco to break itself up into small academies --- smaller
schools within the building that would focus on different specialties. A
long-time 293 teacher, Carmen Sanchez, told me, after that, everything
just started changing. The staff turnover was dizzying.

\begin{itemize}
\tightlist
\item
  carmen sanchez\\
  All of a sudden, these people appear, and they are going to be the
  directors, not principals, directors of this math academy and academy
  in music.
\end{itemize}

chana joffe-walt

Ms. Sanchez says one of them came in to run the place, and she opened
her staff meeting by promising to fire everyone.

\begin{itemize}
\item
  carmen sanchez\\
  She lasted maybe nine months. She was gone. People just went --- I
  mean, it was amazing. That was just as revolving door of principals or
  directors, and they just left.
\item
  dolores hadden smith\\
  They keep changing over what kind of school you're in --- a science
  program school. Or you're in a this. What are we, you know?
\end{itemize}

chana joffe-walt

Ms. Smith, Delores Hadden Smith, was in her third decade working at the
school when this started changing names. They were The Mathematics
Academy, The Academy for Performing and Fine Arts, The School for
Integrated Learning Through the Arts. Teachers left. New staff came in,
new initiatives. They needed to be smaller, more specialized. They
needed more science. They needed a trade. They needed to be a 6 through
12 school, middle and high school. Ms. Smith says this was confusing for
the parents, especially the parents in her community, the Gowanus
community, parents who went to 293 and knew it as 293. Now, they were
asking Ms. Smith, what happened to 293? The School for Integrated
Learning Through the Arts, what's that mean?

\begin{itemize}
\tightlist
\item
  dolores hadden smith\\
  Well, I'm not sending my child there. I don't want my child to go to a
  performing arts school. I want my child to go get academics. But we
  gave both. But they made it like it was a tap dance school. And they
  said, I don't want my kids to go to a tap dance school. I want my kids
  go where they can get an education. Well, they thought we wasn't
  teaching education because we are performing arts in our schools also?
  That was just a feature, one of the many features that we did. But the
  parents didn't get it.
\end{itemize}

chana joffe-walt

By this time, public school admissions allowed more choice about where
parents sent their kids. So some of these local parents started choosing
other schools. 293 was losing students, which meant they were losing
money. A new principal came in --- and an assistant principal named Jeff
Chetirko. By that point, 293 had been renamed The School for
International Studies. But not even assistant principal Chetirko knew
why. Prospective parents would ask him, why should I send my kids here?
What does international mean?

\begin{itemize}
\item
  jeff chetirko\\
  So I remember just having this horrible response --- would be like,
  oh, yeah, our students come from all over the world, and that's really
  what it's about. It's about our diversity, which is kind of bull. But
  that's what I would sell because it didn't sound like we really spoke
  a lot about it in the curriculum. And eventually ---
\item
  chana joffe-walt\\
  You did you have students from all over the world, right?
\item
  jeff chetirko\\
  That's true.
\item
  chana joffe-walt\\
  I mean, you had students from maybe the Caribbean, from Yemen.
\item
  jeff chetirko\\
  Yeah, towards the end, I think we had more from Caribbean. Or if we
  had that one student, we would be like, yeah, they're from all over
  the world. {[}LAUGHS{]} You just make stuff up because you're just
  trying to sell it.
\end{itemize}

chana joffe-walt

By 2003, SIS had low enrollment and terrible test scores. The state put
it on a failing schools list, the dreaded SURR list, Schools Under
Registration Review. Being on a failing schools list made it harder to
sell the school to prospective families, but it did mean SIS got a chunk
of money to turn things around. They bought new reading programs, an
academic intervention program. They doubled periods for reading and
math. During this time, the leadership was stable --- less teacher
turnover. The school was less chaotic. The test scores stabilized. Jeff
Chetirko says they were feeling good about where things were headed.
Still, they had to compete for students. So he hired a marketing firm to
help draw families in.

\begin{itemize}
\tightlist
\item
  jeff chetirko\\
  I remember meeting the guy a couple of times. He had some good ideas.
  I don't really remember what came out of it. It didn't. We hung up
  signs outside the door, just tried to have a different look. But those
  banners, I think that came out of it.
\end{itemize}

chana joffe-walt

He says the marketing idea didn't attract any local families into the
school. Instead, it attracted the attention of the New York Post, which
found out the school was trying to market itself, as it had been told
to, and wrote a snarky article about it. The headline read, ``Lousy
Brooklyn Public School Wants to Hire a Press Agent to Enhance Appeal.''
it goes on to say, quote, ``If they build a buzz, the kids will come.
That's the thinking at a mediocre Brooklyn Public school with grandiose
aspirations.'' The article ends with the list of suggested marketing
slogans for the school. It's mean spirited and racist. Having trouble
with English? So is we. The School for International Studies, the best
six years of your life. Jaguar pride --- where you can go from state
champs to state pen. The Jaguars --- we score baskets; we just can't
count them. Jeff says everyone at the school read it. He distinctly
remembers the feeling.

\begin{itemize}
\tightlist
\item
  jeff chetirko\\
  It's horrible because if you're publicly going to put us on a SURR
  list, what do you think you're doing to that school? So now if we have
  to hire somebody to kind of get us off of that, that perception of
  this school's a failing school, and then to get this newspaper
  article, it just deflates everything. It just really sucks.
  {[}LAUGHS{]} There's no other way to say it. You get that pit feeling
  in your stomach. And you're just like, ugh, what's going on? Or what's
  going to happen next? I think everybody is always nervous about what
  happens next. And then afterwards, you just get super furious.
\end{itemize}

chana joffe-walt

Here is what happened next.

\begin{itemize}
\tightlist
\item
  archived recording\\
  Hi. Hi, How are you? I'm OK. You waited patiently. Oh, my god. Yes.
\end{itemize}

chana joffe-walt

Six years later, I'm standing in a sweaty school gym at a middle school
fair for parents. It's 2017, two years after that gala thrown by the
French Embassy for SIS. A couple dozen schools are here with information
tables. The table for the school for International Studies is mobbed.
There's a line of parents waiting to get a chance to talk with someone
from the school. A mother named Anissa is near the very back of the
school.

\begin{itemize}
\item
  chana joffe-walt\\
  What have you heard about the International School?
\item
  anissa\\
  I heard it's a hot ticket. Everybody wants to get in there.
\end{itemize}

chana joffe-walt

After 40 years of being neglected, messed with by the school board,
after losing students and losing money, losing the building, being
blamed and publicly mocked, SIS was suddenly the hot ticket, as if
history had been wiped away. Parents asked the SIS admissions director,
can their kids get priority if they have good grades? Extracurriculars?
Does attendance count? They want to know if it helps their chances if
they show up for a tour.

\begin{itemize}
\tightlist
\item
  archived recording\\
  Yes. I open access to tours tomorrow at three o'clock.
\end{itemize}

chana joffe-walt

They want to know, will you have enough space for all these people?

\begin{itemize}
\tightlist
\item
  archived recording\\
  Oh, I don't think I've got enough space. For next year, we're only
  accepting 140 sixth graders.
\end{itemize}

chana joffe-walt

Three years earlier, SIS had 30 sixth graders. What changed? The
admissions director is the same. Most of the staff is the same. The
building is the same. The test scores are still pretty low. There's an
IB program now in French. But the biggest change between the era of
being ignored and punished and the era of being celebrated and
oversubscribed is that white kids arrived. That's what's different, nine
times as many white students.

{[}music{]}

I.S. 293 was a mostly segregated school for decades. And still, it was
subject to the whims of white parents. Nice white parents shape public
schools even in our absence because public schools are maniacally loyal
to white families even when that loyalty is rarely returned back to the
public schools. Just the very idea of us, the threat of our displeasure,
warps the whole system. So separate is still not equal because the power
sits with white parents no matter where we are in the system. I think
the only way you equalize schools is by recognizing this fact and trying
wherever possible to suppress the power of white parents. Since no one's
forcing us to give up power. We white parents are going to have to do it
voluntarily. Which, yeah, how's that going to happen? That's next time
on ``Nice White Parents.''

``Nice White Parents'' is produced by Julie Snyder and me, with editing
on this episode from Sarah Koenig and Ira Glass. Neil Drumming is our
Managing Editor. Eve Ewing and Rachel Lissy are our editorial
consultants. Fact-checking and research by Ben Phelan, with additional
research from Lilly Sullivan. Archival research by Rebecca Kent. Music
supervision and mixing by Stowe Nelson, with production help from Aviva
DeKornfeld. Our Director of Operations is Seth Lind. Julie Whitaker is
our Digital Manager. Finance management by Cassie Howley, and Production
management by Frances Swanson. Original music for ``Nice White Parents''
is by The Bad Plus, with additional music written and performed by Matt
McGinley. The music you're hearing right now. is the Nathan Hale trilogy
performed by the Nathan Hale middle school 293 Concert Band. I benefited
from the memories and expertise of many people for this episode: Special
thanks to Charles Jones, Leanna Stiefel, Allison Roda, Ujju Aggarwal,
Clara Hemphill, Steven Schneps, Michael Rebell, Jeffrey Henig, Megan
Thompkins-Strange, Jeffrey Snyder, Dawn Meconi, Maura Walz, Coleen
Mingo, Neil Friedman, Jeff Tripp, Karl Rusnak, Lenny Garcia, Cindy
Black, Arthur Bargonetti, Heather Lewis, Theirry Rafuir, Kevin Davidson
and Afrah Omar.

``Nice White Parents'' is produced by Serial Productions, a New York
Times Company.

Previous

More episodes ofNice White Parents

\href{https://www.nytimes.com/2020/08/13/podcasts/nice-white-parents-school.html?action=click\&module=audio-series-bar\&region=header\&pgtype=Article}{\includegraphics{https://static01.nyt.com/images/2020/07/30/podcasts/30nwp-art/nice-white-parents-album-art-thumbLarge.jpg}}

August 13, 2020~~•~ 50:38Episode Four: `Here's Another Fun Thing You Can
Do'

\href{https://www.nytimes.com/2020/08/06/podcasts/episode-three-this-is-our-school-how-dare-you.html?action=click\&module=audio-series-bar\&region=header\&pgtype=Article}{\includegraphics{https://static01.nyt.com/images/2020/07/30/podcasts/30nwp-art/nice-white-parents-album-art-thumbLarge.jpg}}

August 6, 2020~~•~ 46:55Episode Three: `This Is Our School, How Dare
You?'

\href{https://www.nytimes.com/2020/07/30/podcasts/nice-white-parents-serial-2.html?action=click\&module=audio-series-bar\&region=header\&pgtype=Article}{\includegraphics{https://static01.nyt.com/images/2020/07/30/podcasts/30nwp-art/nice-white-parents-album-art-thumbLarge.jpg}}

July 30, 2020~~•~ 53:37Episode Two: `I Still Believe in It'

\href{https://www.nytimes.com/2020/07/30/podcasts/nice-white-parents-serial.html?action=click\&module=audio-series-bar\&region=header\&pgtype=Article}{\includegraphics{https://static01.nyt.com/images/2020/07/30/podcasts/30nwp-art/nice-white-parents-album-art-thumbLarge.jpg}}

July 30, 2020~~•~ 1:02:23Episode One: The Book of Statuses

\href{https://www.nytimes.com/2020/07/23/podcasts/nice-white-parents-serial.html?action=click\&module=audio-series-bar\&region=header\&pgtype=Article}{\includegraphics{https://static01.nyt.com/images/2020/07/21/podcasts/nice-white-parents-album-art/nice-white-parents-album-art-thumbLarge.jpg}}

July 23, 2020~~•~ 2:49Introducing: Nice White Parents

\href{https://www.nytimes.com/column/nice-white-parents}{See All
Episodes ofNice White Parents}

Next

Published Aug. 6, 2020Updated Aug. 10, 2020

\begin{itemize}
\item
\item
\item
\item
\item
\end{itemize}

``Nice White Parents'' is a new podcast from Serial Productions, a New
York Times Company, about the 60-year relationship between white parents
and the public school down the block.

\textbf{Listen to the show on your mobile device:}
\textbf{\href{https://podcasts.apple.com/us/podcast/nice-white-parents/id1524080195}{Via
Apple Podcasts}} \textbf{\textbar{}}
\textbf{\href{https://open.spotify.com/show/7oBSLCZFCgpdCaBjIG8mLV?si=YcEPLD3xT2ejXmpQz-tRpw}{Via
Spotify}} \textbf{\textbar{}}
\textbf{\href{https://podcasts.google.com/feed/aHR0cHM6Ly9yc3MuYXJ0MTkuY29tL25pY2Utd2hpdGUtcGFyZW50cw}{Via
Google}}

Chana Joffe-Walt explores how white parents can shape a school --- even
when they aren't there.

She traces the history of I.S. 293, now the Boerum Hill School for
International Studies, from the 1980s through the modern education
reforms of the 2000s. In the process, Chana talks to alumni who loved
their school and never questioned why it was on the edge of a white
neighborhood. To them, it was just where everyone went. But she also
speaks to some who watched the school change over the years and
questioned whether a local community school board was secretly plotting
against 293.

``Nice White Parents'' was reported by Chana Joffe-Walt; produced by
Julie Snyder; edited by Sarah Koenig, Neil Drumming and Ira Glass.
Editorial consulting was done by Eve L. Ewing and Rachel Lissy; and
sound mix by Stowe Nelson.

The original score for ``Nice White Parents'' was written and performed
by the jazz group The Bad Plus. The band consists of Reid Anderson, the
bassist; Orrin Evans, the pianist; and Dave King, the drummer.
Additional music from Matt McGinley.

Special thanks to Sam Dolnick, Julie Whitaker, Seth Lind, Julia Simon
and Lauren Jackson.

Advertisement

\protect\hyperlink{after-bottom}{Continue reading the main story}

\hypertarget{site-index}{%
\subsection{Site Index}\label{site-index}}

\hypertarget{site-information-navigation}{%
\subsection{Site Information
Navigation}\label{site-information-navigation}}

\begin{itemize}
\tightlist
\item
  \href{https://help.nytimes.com/hc/en-us/articles/115014792127-Copyright-notice}{©~2020~The
  New York Times Company}
\end{itemize}

\begin{itemize}
\tightlist
\item
  \href{https://www.nytco.com/}{NYTCo}
\item
  \href{https://help.nytimes.com/hc/en-us/articles/115015385887-Contact-Us}{Contact
  Us}
\item
  \href{https://www.nytco.com/careers/}{Work with us}
\item
  \href{https://nytmediakit.com/}{Advertise}
\item
  \href{http://www.tbrandstudio.com/}{T Brand Studio}
\item
  \href{https://www.nytimes.com/privacy/cookie-policy\#how-do-i-manage-trackers}{Your
  Ad Choices}
\item
  \href{https://www.nytimes.com/privacy}{Privacy}
\item
  \href{https://help.nytimes.com/hc/en-us/articles/115014893428-Terms-of-service}{Terms
  of Service}
\item
  \href{https://help.nytimes.com/hc/en-us/articles/115014893968-Terms-of-sale}{Terms
  of Sale}
\item
  \href{https://spiderbites.nytimes.com}{Site Map}
\item
  \href{https://help.nytimes.com/hc/en-us}{Help}
\item
  \href{https://www.nytimes.com/subscription?campaignId=37WXW}{Subscriptions}
\end{itemize}
