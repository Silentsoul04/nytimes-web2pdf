Sections

SEARCH

\protect\hyperlink{site-content}{Skip to
content}\protect\hyperlink{site-index}{Skip to site index}

\href{/section/us}{U.S.}\textbar{}The Unique U.S. Failure to Control the
Virus

\href{https://nyti.ms/3fElddE}{https://nyti.ms/3fElddE}

\begin{itemize}
\item
\item
\item
\item
\item
\item
\end{itemize}

\href{https://www.nytimes.com/news-event/coronavirus?action=click\&pgtype=Article\&state=default\&region=TOP_BANNER\&context=storylines_menu}{The
Coronavirus Outbreak}

\begin{itemize}
\tightlist
\item
  live\href{https://www.nytimes.com/2020/08/08/world/coronavirus-updates.html?action=click\&pgtype=Article\&state=default\&region=TOP_BANNER\&context=storylines_menu}{Latest
  Updates}
\item
  \href{https://www.nytimes.com/interactive/2020/us/coronavirus-us-cases.html?action=click\&pgtype=Article\&state=default\&region=TOP_BANNER\&context=storylines_menu}{Maps
  and Cases}
\item
  \href{https://www.nytimes.com/interactive/2020/science/coronavirus-vaccine-tracker.html?action=click\&pgtype=Article\&state=default\&region=TOP_BANNER\&context=storylines_menu}{Vaccine
  Tracker}
\item
  \href{https://www.nytimes.com/interactive/2020/world/coronavirus-tips-advice.html?action=click\&pgtype=Article\&state=default\&region=TOP_BANNER\&context=storylines_menu}{F.A.Q.}
\item
  \href{https://www.nytimes.com/live/2020/08/07/business/stock-market-today-coronavirus?action=click\&pgtype=Article\&state=default\&region=TOP_BANNER\&context=storylines_menu}{Markets
  \& Economy}
\end{itemize}

\includegraphics{https://static01.nyt.com/images/2020/08/07/us/07VIRUS-STANDSALONE-P1/merlin_175302816_430501a4-06cb-4fe2-90fb-9feb9be002bd-articleLarge.jpg?quality=75\&auto=webp\&disable=upscale}

\hypertarget{the-unique-us-failure-to-control-the-virus}{%
\section{The Unique U.S. Failure to Control the
Virus}\label{the-unique-us-failure-to-control-the-virus}}

Slowing the coronavirus has been especially difficult for the United
States because of its tradition of prioritizing individualism and
missteps by the Trump administration.

A socially distant view of the Chicago River. Normal activities may be
more difficult in the United States than in any other affluent
country.Credit...Daniel Acker for The New York Times

Supported by

\protect\hyperlink{after-sponsor}{Continue reading the main story}

By \href{https://www.nytimes.com/by/david-leonhardt}{David Leonhardt}

Graphics by \href{https://www.nytimes.com/by/lauren-leatherby}{Lauren
Leatherby}

\begin{itemize}
\item
  Published Aug. 6, 2020Updated Aug. 8, 2020, 10:52 a.m. ET
\item
  \begin{itemize}
  \item
  \item
  \item
  \item
  \item
  \item
  \end{itemize}
\end{itemize}

Nearly every country has struggled to contain the coronavirus and made
mistakes along the way.

China committed the first major failure,
\href{https://www.nytimes.com/2020/02/01/world/asia/china-coronavirus.html}{silencing
doctors} who tried to raise alarms about the virus and allowing it to
escape from Wuhan. Much of Europe went next,
\href{https://www.nytimes.com/2020/03/21/world/europe/italy-coronavirus-center-lessons.html}{failing
to avoid} enormous outbreaks. Today, many countries --- Japan, Canada,
France, Australia and more --- are coping with
\href{https://www.nytimes.com/interactive/2020/world/coronavirus-maps.html}{new
increases in cases} after reopening parts of society.

Yet even with all of these problems, one country stands alone, as the
only affluent nation to have suffered a severe, sustained outbreak
\href{https://www.nytimes.com/interactive/2020/us/coronavirus-us-cases.html}{for
more than four months}: the United States.

\begin{itemize}
\item
\item
\item
\item
\end{itemize}

Over the past month, about 1.9 million Americans have tested positive
for the virus.

That's more than five times as many as in all of Europe, Canada, Japan,
South Korea and Australia, combined.

Even though some of these countries saw worrying new outbreaks over the
past month, including 50,000 new cases in Spain ...

... the outbreaks still pale in comparison to those in the United
States. Florida, with a population less than half of Spain, has reported
nearly 300,000 cases in the same period.

When it comes to the virus, the United States has come to resemble not
the wealthy and powerful countries to which it is often compared but
instead far poorer countries, like Brazil, Peru and South Africa, or
those with large migrant populations, like Bahrain and Oman.

As in several of those other countries, the toll of the virus in the
United States has fallen disproportionately on poorer people and groups
that have long suffered discrimination. Black and Latino residents of
the United States have contracted the virus at
\href{https://www.nytimes.com/interactive/2020/07/05/us/coronavirus-latinos-african-americans-cdc-data.html}{roughly
three times as high of a rate} as white residents.

How did this happen? The New York Times set out to reconstruct the
unique failure of the United States, through numerous interviews with
scientists and public health experts around the world. The reporting
points to two central themes.

First, the United States faced longstanding challenges in confronting a
major pandemic. It is a large country at the nexus of the global
economy, with a tradition of prioritizing individualism over government
restrictions. That tradition is one reason the United States suffers
from an unequal health care system that has long produced worse medical
outcomes --- including
\href{https://www.cia.gov/library/publications/the-world-factbook/rankorder/2091rank.html}{higher
infant mortality} and
\href{https://data.worldbank.org/indicator/SH.STA.DIAB.ZS}{diabetes
rates} and
\href{https://data.worldbank.org/indicator/SP.DYN.LE00.IN}{lower life
expectancy} --- than in most other rich countries.

``As an American, I think there is a lot of good to be said about our
libertarian tradition,'' Dr. Jared Baeten, an epidemiologist and vice
dean at the University of Washington School of Public Health, said.
``But this is the consequence --- we don't succeed as well as a
collective.''

The second major theme is one that public health experts often find
uncomfortable to discuss because many try to steer clear of partisan
politics. But many agree that the poor results in the United States stem
in substantial measure from
\href{https://www.nytimes.com/2020/07/18/us/politics/trump-coronavirus-response-failure-leadership.html}{the
performance of the Trump administration}.

In no other high-income country --- and in only a few countries, period
--- have political leaders departed from expert advice as frequently and
significantly as the Trump administration. President Trump
\href{https://www.factcheck.org/2020/03/trumps-statements-about-the-coronavirus/}{has
said} the virus was not serious; predicted it would disappear; spent
weeks questioning the need for masks; encouraged states to reopen even
with large and growing caseloads; and promoted medical disinformation.

In recent days, Mr. Trump has continued the theme, offering a torrent of
\href{https://www.tampabay.com/news/health/2020/08/04/politifact-fact-checking-22-claims-from-donald-trumps-axios-interview/}{misleading
statistics} in his public appearances that make the situation sound
\href{https://www.nytimes.com/2020/08/04/us/politics/coronavirus-trump-data-briefing.html}{less
dire than it is}.

Some Republican governors have followed his lead and also played down
the virus, while others have largely followed the science. Democratic
governors have more reliably heeded scientific advice, but their
performance in containing the virus has been uneven.

``In many of the countries that have been very successful they had a
much crisper strategic direction and really had a vision,'' said Caitlin
Rivers, an epidemiologist at the Johns Hopkins Center for Health
Security, who wrote
\href{https://www.aei.org/research-products/report/national-coronavirus-response-a-road-map-to-reopening/}{a
guide to reopening safely} for the American Enterprise Institute, a
conservative research group. ``I'm not sure we ever really had a plan or
a strategy --- or at least it wasn't public.''

Together, the national skepticism toward collective action and the Trump
administration's scattered response to the virus have contributed to
several specific failures and missed opportunities, Times reporting
shows:

\begin{itemize}
\item
  a lack of effective travel restrictions;
\item
  repeated breakdowns in testing;
\item
  confusing advice about masks;
\item
  a misunderstanding of the relationship between the virus and the
  economy;
\item
  and inconsistent messages from public officials.
\end{itemize}

Already, the American death toll is of a different order of magnitude
than in most other countries. With only 4 percent of the world's
population, the United States has accounted for
\href{https://www.nytimes.com/interactive/2020/world/coronavirus-maps.html}{22
percent of coronavirus deaths}. Canada, a rich country that neighbors
the United States, has a per capita death rate about half as large. And
these gaps may worsen in coming weeks, given the lag between new cases
and deaths.

\hypertarget{daily-deaths-per-million-in-wealthy-countries}{%
\subsubsection{Daily deaths per million in wealthy
countries}\label{daily-deaths-per-million-in-wealthy-countries}}

Source: New York Times database from state and local governments.
Includes all countries with a G.D.P. per capita of more than \$25,000
and a population of at least 10 million people.

For many Americans who survive the virus or do not contract it, the
future will bring other problems. Many schools will struggle to open.
And the normal activities of life --- family visits, social gatherings,
restaurant meals, sporting events --- may be more difficult in the
United States than in any other affluent country.

\hypertarget{a-travel-policy-that-fell-short}{%
\subsection{A travel policy that fell
short}\label{a-travel-policy-that-fell-short}}

In retrospect, one of Mr. Trump's first policy responses to the virus
appears to have been one of his most promising.

On Jan. 31, his administration announced that it was
\href{https://www.nytimes.com/2020/01/31/business/china-travel-coronavirus.html}{restricting
entry to the United States from China}: Many foreign nationals --- be
they citizens of China or other countries --- would not be allowed into
the United States if they had been to China in the previous two weeks.

It was still early in the spread of the virus. The first cases in Wuhan,
China, had been diagnosed about a month before, and the first announced
case in the United States had come on Jan. 21. In announcing the new
travel policy, Alex M. Azar II, the secretary of health and human
services, declared that the virus posed ``a public health emergency.''
Mr. Trump
\href{https://twitter.com/realdonaldtrump/status/1249519742093864961}{described}
the policy as his ``China ban.''

After the Trump administration acted, several other countries quickly
announced
\href{https://www.nytimes.com/2020/02/01/world/asia/china-coronavirus-us-australia.html}{their
own restrictions} on travel from China, including Japan, Vietnam and
Australia.

But it quickly became clear that the United States' policy was full of
holes. It did not apply to immediate family members of American citizens
and permanent residents returning from China, for example. In the two
months after the policy went into place, almost 40,000 people arrived in
the United States
\href{https://www.nytimes.com/2020/04/04/us/coronavirus-china-travel-restrictions.html}{on
direct flights from China}.

Even more important, the policy failed to take into account that the
virus had spread well beyond China by early February. Later data would
show that many infected people arriving in the United States
\href{https://www.nytimes.com/2020/04/08/science/new-york-coronavirus-cases-europe-genomes.html}{came
from Europe}. (The Trump administration did not
\href{https://www.nytimes.com/2020/03/11/us/politics/anthony-fauci-coronavirus.html}{restrict
travel from Europe} until March and exempted Britain from that ban
despite a high infection rate there.)

The administration's policy also did little to create quarantines for
people who entered the United States and may have had the virus.

\hypertarget{latest-updates-the-coronavirus-outbreak}{%
\section{\texorpdfstring{\href{https://www.nytimes.com/2020/08/07/world/covid-19-news.html?action=click\&pgtype=Article\&state=default\&region=MAIN_CONTENT_1\&context=storylines_live_updates}{Latest
Updates: The Coronavirus
Outbreak}}{Latest Updates: The Coronavirus Outbreak}}\label{latest-updates-the-coronavirus-outbreak}}

Updated 2020-08-08T12:04:28.992Z

\begin{itemize}
\tightlist
\item
  \href{https://www.nytimes.com/2020/08/07/world/covid-19-news.html?action=click\&pgtype=Article\&state=default\&region=MAIN_CONTENT_1\&context=storylines_live_updates\#link-1f86d03a}{As
  the U.S. relief talks falter again, Trump says he is prepared to act
  on his own.}
\item
  \href{https://www.nytimes.com/2020/08/07/world/covid-19-news.html?action=click\&pgtype=Article\&state=default\&region=MAIN_CONTENT_1\&context=storylines_live_updates\#link-3f64a70a}{Cuomo
  says N.Y. schools can reopen in-person but leaves it up to districts
  to determine if, when and how.}
\item
  \href{https://www.nytimes.com/2020/08/07/world/covid-19-news.html?action=click\&pgtype=Article\&state=default\&region=MAIN_CONTENT_1\&context=storylines_live_updates\#link-14e70066}{Thousands
  of cases went unreported in California when a computer server failed.}
\end{itemize}

\href{https://www.nytimes.com/2020/08/07/world/covid-19-news.html?action=click\&pgtype=Article\&state=default\&region=MAIN_CONTENT_1\&context=storylines_live_updates}{See
more updates}

More live coverage:
\href{https://www.nytimes.com/live/2020/08/07/business/stock-market-today-coronavirus?action=click\&pgtype=Article\&state=default\&region=MAIN_CONTENT_1\&context=storylines_live_updates}{Markets}

Authorities in some other places took a far more rigorous approach to
travel restrictions.

South Korea, Hong Kong and Taiwan largely restricted entry to residents
returning home. Those residents then had to quarantine for two weeks
upon arrival, with the government keeping close tabs to ensure they did
not leave their home or hotel. South Korea and Hong Kong also tested for
the virus at the airport and transferred anyone who was positive to a
government facility.

Australia offers a telling comparison. Like the United States, it is
separated from China by an ocean and is run by a conservative leader ---
Scott Morrison, the prime minister. Unlike the United States, it put
travel restrictions at the center of its virus response.

Australian officials noticed in March that the travel restrictions they
had announced on Feb. 1 were not preventing the virus from spreading. So
they went further.

On March 27, Mr. Morrison announced that Australia would no longer trust
travelers
\href{https://www.abc.net.au/news/2020-03-27/pm-announces-new-quarantine-measures-international-travellers/12097390?nw=0}{to
isolate themselves voluntarily}. The country would instead mandate that
everyone arriving from overseas, including Australian citizens, spend
two weeks quarantined in a hotel.

The protocols were strict. As people arrived at an airport, the
authorities transported them directly to hotels nearby. People were not
even allowed to leave their hotel to exercise. The Australian military
helped enforce the rules.

Around the same time, several Australian states with minor outbreaks
shut their own borders to keep out Australians from regions with higher
rates of infection. That hardening of internal boundaries had not
happened since the 1918 flu pandemic, said Ian Mackay, a virologist in
Queensland, one of the first states to block entry from other areas.

The United States, by comparison, imposed few travel restrictions,
either for foreigners or American citizens. Individual states did little
to enforce the rules they did impose.

``People need a bit more than a suggestion to look after their own
health,'' said Dr. Mackay, who has been working with Australian
officials on their pandemic response. ``They need guidelines, they need
rules --- and they need to be enforced.''

Travel restrictions and quarantines were central to the success in
controlling the virus in South Korea, Hong Kong, Taiwan and Australia,
\href{https://www.nytimes.com/2020/04/24/world/australia/new-zealand-coronavirus.html}{as
well as New Zealand}, many epidemiologists believe. In Australia, the
number of new cases per day fell more than 90 percent in April. It
remained near zero through May and early June, even as the virus surged
across much of the United States.

In the past six weeks, Australia has begun to have a resurgence ---
which itself points to the importance of travel rules. The latest
outbreak stems in large part from
\href{https://www.theage.com.au/national/victoria/how-hotel-quarantine-let-covid-19-out-of-the-bag-in-victoria-20200703-p558og.html}{problems
with the quarantine in the city of Melbourne}. Compared with other parts
of Australia, Melbourne relied more on private security contractors who
employed temporary workers --- some of whom lacked training and failed
to follow guidelines --- to enforce quarantines at local hotels.
Officials
\href{https://www.smh.com.au/national/victoria/numbers-are-too-high-melbourne-faces-stage-four-lockdown-within-days-20200801-p55hm3.html}{have
responded} by banning out-of-state travel again and imposing new
lockdowns.

Still, the tolls in Australia and the United States remain vastly
different. Fewer than 300 Australians have died of complications from
Covid-19, the illness caused by the virus. If the United States had the
same per capita death rate, about 3,300 Americans would have died,
rather than 158,000.

Enacting tough travel restrictions in the United States would not have
been easy. It is more integrated into the global economy than Australia
is, has a tradition of local policy decisions and borders two other
large countries. But there is a good chance that a different version of
Mr. Trump's restrictions --- one with fewer holes and stronger
quarantines --- would have meaningfully slowed the virus's spread.

Traditionally, public health experts had not seen travel restrictions as
central to fighting a pandemic, given their economic costs and the
availability of other options, like testing, quarantining and contact
tracing, Dr. Baeten, the University of Washington epidemiologist, said.
But he added that travel restrictions had been successful enough in
fighting the coronavirus around the world that those views may need to
be revisited.

``Travel,'' he said, ``is the hallmark of the spread of this virus
around the world.''

\includegraphics{https://static01.nyt.com/images/2020/08/07/us/06VIRUS-STANDSALONE-testing/merlin_175041549_f4134f61-d95b-4dc5-af99-dfff4e1d68c9-articleLarge.jpg?quality=75\&auto=webp\&disable=upscale}

\hypertarget{the-double-testing-failure}{%
\subsection{The double testing
failure}\label{the-double-testing-failure}}

On Jan. 16, nearly a week before the first announced case of the
coronavirus in the United States, a German hospital made an
announcement. Its researchers had developed a test for the virus, which
they described
\href{https://www.charite.de/en/service/press_reports/artikel/detail/researchers_develop_first_diagnostic_test_for_novel_coronavirus_in_china/}{as
the world's first}.

The researchers posted the formula for the test online and said they
expected that countries with strong public health systems would soon be
able to produce their own tests. ``We're more concerned about labs in
countries where it's not that easy to transport samples, or staff aren't
trained that thoroughly, or if there is a large number of patients who
have to be tested,'' Dr. Christian Drosten, the director of the
Institute for Virology at the hospital, known as Charité, in Berlin.

It turned out, however, that the testing problems would not be limited
to less-developed countries.

In the United States, the Centers for Disease Control and Prevention
developed their own test four days after the German lab did. C.D.C.
officials claimed that the American test would be more accurate than the
German one, by using three genetic sequences to detect the virus rather
than two. The federal government quickly began distributing the American
test to state officials.

But the test had a flaw. The third genetic sequence produced
inconclusive results, so the C.D.C. told state labs to pause their work.
In meetings of the White House's coronavirus task force, Dr. Robert R.
Redfield, the C.D.C. director, played down the problem and said it would
soon be solved.

Instead, it took
\href{https://www.nytimes.com/2020/03/28/us/testing-coronavirus-pandemic.html}{weeks
to fix}. During that time, the United States had to restrict testing to
people who had clear reason to think they had the virus. All the while,
the virus was quietly spreading.

By early March, with the testing delays still unresolved, the New York
region became a global center of the virus --- without people realizing
it until weeks later. More widespread testing could have made a major
difference, experts said, leading to earlier lockdowns and social
distancing and ultimately less sickness and death.

``You can't stop it if you can't see it,'' Dr. Bruce Aylward, a senior
adviser to the director general at the World Health Organization, said.

While the C.D.C. was struggling to solve its testing flaws, Germany was
\href{https://www.nytimes.com/2020/04/04/world/europe/germany-coronavirus-death-rate.html}{rapidly
building up its ability to test}. Chancellor Angela Merkel, a chemist by
training, and other political leaders were watching the virus sweep
across northern Italy, not far from southern Germany, and pushed for a
big expansion of testing.

By the time the virus became a problem in Germany, labs around the
country had thousands of test kits ready to use. From the beginning, the
government covered the cost of the tests. American laboratories often
charge patients
\href{https://www.nytimes.com/2020/06/16/upshot/coronavirus-test-cost-varies-widely.html}{about
\$100 for a test}.

Without free tests, Dr. Hendrik Streeck, director of the Institute of
Virology at the University Hospital Bonn, said at the time, ``a young
person with no health insurance and an itchy throat is unlikely to go to
the doctor and therefore risks infecting more people.''

Germany was soon far ahead of other countries in testing. It was able to
diagnose asymptomatic cases, trace the contacts of new patients and
isolate people before they could spread the virus. The country has still
suffered a significant outbreak. But it has had many fewer cases per
capita than Italy, Spain, France, Britain or Canada --- and about
one-fifth the rate of the United States.

The United States eventually made up ground on tests. In recent weeks,
it has been conducting more per capita than any other country, according
to Johns Hopkins researchers.

But now there is a new problem: The virus has grown even more rapidly
than testing capacity. In recent weeks, Americans have often
\href{https://www.nytimes.com/2020/07/06/us/coronavirus-test-shortage.html}{had
to wait in long lines}, sometimes in scorching heat, to be tested.

One measure of the continuing troubles with testing is the percentage of
tests that come back positive. In a country that has the virus under
control,
\href{https://www.nytimes.com/interactive/2020/us/coronavirus-testing.html}{fewer
than 5 percent of tests come back positive}, according to World Health
Organization guidelines. Many countries have reached that benchmark. The
United States, even with the large recent volume of tests, has not.

\hypertarget{percent-of-coronavirus-tests-that-come-back-positive}{%
\subsubsection{Percent of coronavirus tests that come back
positive}\label{percent-of-coronavirus-tests-that-come-back-positive}}

Seven-day averages. Source: Our World in Data. Includes all countries
with a G.D.P. per capita of more than \$25,000 and a population of at
least 10 million people.

``We do have a lot of testing,'' Ms. Rivers, the Johns Hopkins
epidemiologist, said. ``The problem is we also have a lot of cases.''

The huge demand for tests has overwhelmed medical laboratories, and many
need days --- or even up to two weeks --- to produce results. ``That
really is not useful for public health and medical management,'' Ms.
Rivers added. While people are waiting for their results, many are also
spreading the virus.

In Belgium recently, test results have typically come back in 48 to 72
hours. In Germany and Greece, it is two days. In France, the wait is
often 24 hours.

\hypertarget{the-double-mask-failure}{%
\subsection{The double mask failure}\label{the-double-mask-failure}}

For the first few months of the pandemic, public health experts could
not agree on a consistent message about masks. Some said masks reduced
the spread of the virus. Many experts, however, discouraged the use of
masks, saying --- somewhat contradictorily --- that their benefits were
modest and that they should be reserved for medical workers.

``We don't generally recommend the wearing of masks in public by
otherwise well individuals because it has not been up to now associated
with any particular benefit,'' Dr. Michael Ryan, a World Health
Organization official,
\href{https://www.who.int/docs/default-source/coronaviruse/transcripts/who-audio-emergencies-coronavirus-press-conference-full-30mar2020.pdf?sfvrsn=6b68bc4a_2}{said
at a March 30 news conference}.

His colleague Dr. Maria Van Kerkhove explained that it was important to
``prioritize the use of masks for those who need them most.''

The conflicting advice, echoed by the C.D.C. and others, led to
relatively little mask wearing in many countries early in the pandemic.
But several Asian countries were exceptions, partly because they had a
tradition of mask wearing to avoid sickness or minimize the effects of
pollution.

By January, mask wearing in Japan was widespread, as it often had been
during a typical flu season. Masks also quickly became the norm in much
of
\href{https://www.deseret.com/opinion/2020/7/11/21320058/masks-covid-19-media-utah-governor-inside-the-newsroom-the-culture-south-korea}{South
Korea},
\href{https://www.nytimes.com/2020/07/16/world/asia/coronavirus-thailand-photos.html?searchResultPosition=2}{Thailand},
\href{https://saigoneer.com/saigon-culture/18648-already-ubiquitous,-face-masks-become-the-symbol-of-vietnam-s-new-normal}{Vietnam},
\href{https://www.ncbi.nlm.nih.gov/pmc/articles/PMC7270822/}{Taiwan} and
\href{https://www.sciencemag.org/news/2020/03/not-wearing-masks-protect-against-coronavirus-big-mistake-top-chinese-scientist-says}{China}.

In the following months, scientists around the world began to report two
strands of evidence that both pointed to the importance of masks:
Research showed that the virus could be transmitted through
\href{https://www.nytimes.com/2020/07/04/health/239-experts-with-one-big-claim-the-coronavirus-is-airborne.html}{droplets
that hang in the air}, and several studies found that the virus spread
less frequently in places where people were wearing masks.

On one cruise ship that gave passengers masks after somebody got sick,
for example,
\href{https://www.nytimes.com/2020/07/27/health/coronavirus-mask-protection.html}{many
fewer people became ill} than on a different cruise where people did not
wear masks.

Consistent with that evidence was Asia's success in holding down the
number of cases (after China's initial failure to do so). In South
Korea, the per capita death rate is about one-eightieth as large as in
the United States; Japan, despite being slow to enact social distancing,
has a death rate about one-sixtieth as large.

``We should have
\href{https://www.ucsf.edu/news/2020/06/417906/still-confused-about-masks-heres-science-behind-how-face-masks-prevent}{told
people to wear cloth masks} right off the bat,'' Dr. George Rutherford
of the University of California, San Francisco, said.

In many countries, officials reacted to the emerging evidence with a
clear message: Wear a mask.

Prime Minister Justin Trudeau of Canada
\href{https://www.deccanherald.com/international/justin-trudeau-puts-on-mask-canadians-urged-to-do-same-840092.html}{began
wearing one in May}. During a visit to an elementary school, President
Emmanuel Macron of France wore
\href{https://apnews.com/fe461ecbffacb67c747a3234300c90de}{a French-made
blue mask} that complemented his suit and tie. Zuzana Caputova, the
president of Slovakia, created a social media sensation by wearing
\href{https://twitter.com/pfarqeu/status/1265154039190151168}{a
fuchsia-colored mask} that matched her dress.

In the United States, however, masks did not become a fashion symbol.
They became a political symbol.

Mr. Trump avoided wearing one in public for months. He poked fun at a
reporter who wore one to a news conference, asking the reporter
\href{https://www.cbsnews.com/video/trump-mocks-those-wearing-face-masks-calling-it-politically-correct/}{to
take it off} and saying that wearing one was ``politically correct.'' He
described former Vice President Joseph R. Biden Jr.'s decision to wear
one outdoors as ``very unusual.''

Many other Republicans and conservative news outlets,
\href{https://www.mediamatters.org/fox-news/foxs-ongoing-attacks-masks-are-public-health-menace}{like
Fox News}, echoed his position. Mask wearing, as a result, became yet
another partisan divide in a highly polarized country.

Throughout much of the Northeast and the West Coast, more than 80
percent of people wore masks when within six feet of someone else. In
more conservative areas, like the Southeast, the share was
\href{https://www.nytimes.com/interactive/2020/07/17/upshot/coronavirus-face-mask-map.html}{closer
to 50 percent}.

A March survey found that
\href{https://www.vox.com/science-and-health/2020/3/31/21199271/coronavirus-in-us-trump-republicans-democrats-survey-epistemic-crisis}{partisanship
was the biggest predictor} of whether Americans regularly wore masks ---
bigger than their age or whether they lived in a region with a high
number of virus cases. In many of the places where people adopted a
hostile view of masks, including Texas and the Southeast, the number of
virus cases began to soar this spring.

Image

President Trump avoided wearing a mask in public for months after health
experts said it was important.Credit...Anna Moneymaker for The New York
Times

\hypertarget{the-first-rule-of-virus-economics}{%
\subsection{The first rule of virus
economics}\label{the-first-rule-of-virus-economics}}

Throughout March and April, Gov. Brian Kemp of Georgia and staff members
held long meetings inside a conference room at the State Capitol in
Atlanta. They ordered takeout lunches from local restaurants like the
Varsity and held two daily conference calls with the public health
department, the National Guard and other officials.

\href{https://www.nytimes.com/news-event/coronavirus?action=click\&pgtype=Article\&state=default\&region=MAIN_CONTENT_3\&context=storylines_faq}{}

\hypertarget{the-coronavirus-outbreak-}{%
\subsubsection{The Coronavirus Outbreak
›}\label{the-coronavirus-outbreak-}}

\hypertarget{frequently-asked-questions}{%
\paragraph{Frequently Asked
Questions}\label{frequently-asked-questions}}

Updated August 6, 2020

\begin{itemize}
\item ~
  \hypertarget{why-are-bars-linked-to-outbreaks}{%
  \paragraph{Why are bars linked to
  outbreaks?}\label{why-are-bars-linked-to-outbreaks}}

  \begin{itemize}
  \tightlist
  \item
    Think about a bar. Alcohol is flowing. It can be loud, but it's
    definitely intimate, and you often need to lean in close to hear
    your friend. And strangers have way, way fewer reservations about
    coming up to people in a bar. That's sort of the point of a bar.
    Feeling good and close to strangers. It's no surprise, then, that
    \href{https://www.nytimes.com/2020/07/02/us/coronavirus-bars.html?action=click\&pgtype=Article\&state=default\&region=MAIN_CONTENT_3\&context=storylines_faq}{bars
    have been linked to outbreaks in several states.} Louisiana health
    officials have tied
    \href{https://www.nytimes.com/2020/06/22/us/new-coronavirus-phase.html?action=click\&pgtype=Article\&state=default\&region=MAIN_CONTENT_3\&context=storylines_faq}{at
    least 100 coronavirus cases} to bars in the Tigerland nightlife
    district in Baton Rouge. Minnesota has traced 328 recent cases to
    bars across the state.
    \href{https://www.boisestatepublicradio.org/post/bars-large-venues-close-ada-county-after-surge-coronavirus-prompts-rollback\#stream/0}{In
    Idaho}, health officials shut down bars in Ada County after
    reporting clusters of infections among young adults who had visited
    several bars in downtown Boise. Governors in
    \href{https://www.nytimes.com/2020/07/01/us/california-coronavirus-reopening.html?action=click\&pgtype=Article\&state=default\&region=MAIN_CONTENT_3\&context=storylines_faq}{California},
    \href{https://www.nytimes.com/2020/06/14/us/coronavirus-united-states.html?action=click\&pgtype=Article\&state=default\&region=MAIN_CONTENT_3\&context=storylines_faq}{Texas
    and Arizona}, where coronavirus cases are soaring, have ordered
    hundreds of newly reopened bars to shut down. Less than two weeks
    after Colorado's bars reopened at limited capacity, Gov. Jared Polis
    \href{https://www.denverpost.com/2020/06/30/colorado-bars-closed-coronavirus/}{ordered
    them to close}.
  \end{itemize}
\item ~
  \hypertarget{i-have-antibodies-am-i-now-immune}{%
  \paragraph{I have antibodies. Am I now
  immune?}\label{i-have-antibodies-am-i-now-immune}}

  \begin{itemize}
  \tightlist
  \item
    As of right now,
    \href{https://www.nytimes.com/2020/07/22/health/covid-antibodies-herd-immunity.html?action=click\&pgtype=Article\&state=default\&region=MAIN_CONTENT_3\&context=storylines_faq}{that
    seems likely, for at least several months.} There have been
    frightening accounts of people suffering what seems to be a second
    bout of Covid-19. But experts say these patients may have a
    drawn-out course of infection, with the virus taking a slow toll
    weeks to months after initial exposure. People infected with the
    coronavirus typically
    \href{https://www.nature.com/articles/s41586-020-2456-9}{produce}
    immune molecules called antibodies, which are
    \href{https://www.nytimes.com/2020/05/07/health/coronavirus-antibody-prevalence.html?action=click\&pgtype=Article\&state=default\&region=MAIN_CONTENT_3\&context=storylines_faq}{protective
    proteins made in response to an
    infection}\href{https://www.nytimes.com/2020/05/07/health/coronavirus-antibody-prevalence.html?action=click\&pgtype=Article\&state=default\&region=MAIN_CONTENT_3\&context=storylines_faq}{.
    These antibodies may} last in the body
    \href{https://www.nature.com/articles/s41591-020-0965-6}{only two to
    three months}, which may seem worrisome, but that's perfectly normal
    after an acute infection subsides, said Dr. Michael Mina, an
    immunologist at Harvard University. It may be possible to get the
    coronavirus again, but it's highly unlikely that it would be
    possible in a short window of time from initial infection or make
    people sicker the second time.
  \end{itemize}
\item ~
  \hypertarget{im-a-small-business-owner-can-i-get-relief}{%
  \paragraph{I'm a small-business owner. Can I get
  relief?}\label{im-a-small-business-owner-can-i-get-relief}}

  \begin{itemize}
  \tightlist
  \item
    The
    \href{https://www.nytimes.com/article/small-business-loans-stimulus-grants-freelancers-coronavirus.html?action=click\&pgtype=Article\&state=default\&region=MAIN_CONTENT_3\&context=storylines_faq}{stimulus
    bills enacted in March} offer help for the millions of American
    small businesses. Those eligible for aid are businesses and
    nonprofit organizations with fewer than 500 workers, including sole
    proprietorships, independent contractors and freelancers. Some
    larger companies in some industries are also eligible. The help
    being offered, which is being managed by the Small Business
    Administration, includes the Paycheck Protection Program and the
    Economic Injury Disaster Loan program. But lots of folks have
    \href{https://www.nytimes.com/interactive/2020/05/07/business/small-business-loans-coronavirus.html?action=click\&pgtype=Article\&state=default\&region=MAIN_CONTENT_3\&context=storylines_faq}{not
    yet seen payouts.} Even those who have received help are confused:
    The rules are draconian, and some are stuck sitting on
    \href{https://www.nytimes.com/2020/05/02/business/economy/loans-coronavirus-small-business.html?action=click\&pgtype=Article\&state=default\&region=MAIN_CONTENT_3\&context=storylines_faq}{money
    they don't know how to use.} Many small-business owners are getting
    less than they expected or
    \href{https://www.nytimes.com/2020/06/10/business/Small-business-loans-ppp.html?action=click\&pgtype=Article\&state=default\&region=MAIN_CONTENT_3\&context=storylines_faq}{not
    hearing anything at all.}
  \end{itemize}
\item ~
  \hypertarget{what-are-my-rights-if-i-am-worried-about-going-back-to-work}{%
  \paragraph{What are my rights if I am worried about going back to
  work?}\label{what-are-my-rights-if-i-am-worried-about-going-back-to-work}}

  \begin{itemize}
  \tightlist
  \item
    Employers have to provide
    \href{https://www.osha.gov/SLTC/covid-19/standards.html}{a safe
    workplace} with policies that protect everyone equally.
    \href{https://www.nytimes.com/article/coronavirus-money-unemployment.html?action=click\&pgtype=Article\&state=default\&region=MAIN_CONTENT_3\&context=storylines_faq}{And
    if one of your co-workers tests positive for the coronavirus, the
    C.D.C.} has said that
    \href{https://www.cdc.gov/coronavirus/2019-ncov/community/guidance-business-response.html}{employers
    should tell their employees} -\/- without giving you the sick
    employee's name -\/- that they may have been exposed to the virus.
  \end{itemize}
\item ~
  \hypertarget{what-is-school-going-to-look-like-in-september}{%
  \paragraph{What is school going to look like in
  September?}\label{what-is-school-going-to-look-like-in-september}}

  \begin{itemize}
  \tightlist
  \item
    It is unlikely that many schools will return to a normal schedule
    this fall, requiring the grind of
    \href{https://www.nytimes.com/2020/06/05/us/coronavirus-education-lost-learning.html?action=click\&pgtype=Article\&state=default\&region=MAIN_CONTENT_3\&context=storylines_faq}{online
    learning},
    \href{https://www.nytimes.com/2020/05/29/us/coronavirus-child-care-centers.html?action=click\&pgtype=Article\&state=default\&region=MAIN_CONTENT_3\&context=storylines_faq}{makeshift
    child care} and
    \href{https://www.nytimes.com/2020/06/03/business/economy/coronavirus-working-women.html?action=click\&pgtype=Article\&state=default\&region=MAIN_CONTENT_3\&context=storylines_faq}{stunted
    workdays} to continue. California's two largest public school
    districts --- Los Angeles and San Diego --- said on July 13, that
    \href{https://www.nytimes.com/2020/07/13/us/lausd-san-diego-school-reopening.html?action=click\&pgtype=Article\&state=default\&region=MAIN_CONTENT_3\&context=storylines_faq}{instruction
    will be remote-only in the fall}, citing concerns that surging
    coronavirus infections in their areas pose too dire a risk for
    students and teachers. Together, the two districts enroll some
    825,000 students. They are the largest in the country so far to
    abandon plans for even a partial physical return to classrooms when
    they reopen in August. For other districts, the solution won't be an
    all-or-nothing approach.
    \href{https://bioethics.jhu.edu/research-and-outreach/projects/eschool-initiative/school-policy-tracker/}{Many
    systems}, including the nation's largest, New York City, are
    devising
    \href{https://www.nytimes.com/2020/06/26/us/coronavirus-schools-reopen-fall.html?action=click\&pgtype=Article\&state=default\&region=MAIN_CONTENT_3\&context=storylines_faq}{hybrid
    plans} that involve spending some days in classrooms and other days
    online. There's no national policy on this yet, so check with your
    municipal school system regularly to see what is happening in your
    community.
  \end{itemize}
\end{itemize}

One of the main subjects of the meetings was when to end Georgia's
lockdown and reopen the state's economy. By late April, Mr. Kemp decided
that it was time.

Georgia
\href{https://www.nytimes.com/interactive/2020/05/07/us/coronavirus-states-reopen-criteria.html}{had
not met} the reopening criteria laid out by the Trump administration
(and many outside health experts considered those criteria too lax). The
state was reporting about 700 new cases a day, more than when it shut
down on April 3.

Nonetheless, Mr. Kemp went ahead. He said that Georgia's economy
\href{https://gov.georgia.gov/press-releases/2020-04-20/gov-kemp-updates-georgians-covid-19}{could
not wait any longer}, and it became one of the first states to reopen.

``I don't give a damn about politics right now,'' he said at an April 20
news conference announcing the reopening. He went on to describe
business owners with employees at home who were ``going broke, worried
about whether they can feed their children, make the mortgage payment.''

Four days later, across Georgia, barbers returned to their chairs,
\href{https://www.nytimes.com/2020/04/24/us/coronavirus-georgia-oklahoma-alaska-reopen.html}{wearing
face masks and latex gloves}. Gyms and bowling alleys were allowed to
reopen, followed by restaurants on April 27. The stay-at-home order
expired at 11:59 p.m. on April 30.

Mr. Kemp's decision was part of a pattern: Across the United States,
caseloads were typically much higher when the economy reopened than in
other countries.

\hypertarget{the-united-states-reopened-with-more-cases}{%
\subsubsection{The United States reopened with more
cases}\label{the-united-states-reopened-with-more-cases}}

\hypertarget{other-countries-relaxed-their-restrictions-to-americas-current-level-with-far-fewer-cases-per-million}{%
\paragraph{Other countries relaxed their restrictions to America's
current level with far fewer cases per
million.}\label{other-countries-relaxed-their-restrictions-to-americas-current-level-with-far-fewer-cases-per-million}}

Source: Oxford Covid-19 Government Response Tracker, New York Times
database from state and local governments. Includes all countries with a
G.D.P. per capita of more than \$25,000 that have a population of at
least 10 million people. Japan and Sweden never reached a high enough
stringency level to be included.

As the United States endured weeks of closed stores and rising
unemployment this spring, many politicians --- particularly Republicans,
like Mr. Kemp --- argued that there was an unavoidable trade-off between
public health and economic health. And if crushing the virus meant
ruining the economy, maybe the side effects of the treatment were worse
than the disease.

Dan Patrick, the Republican lieutenant governor of Texas, put the case
most bluntly, and became an object of scorn, especially from the
political left, for doing so. ``There are more important things than
living,''
\href{https://www.nbcnews.com/news/us-news/texas-lt-gov-dan-patrick-reopening-economy-more-important-things-n1188911}{Mr.
Patrick said in a television interview} the same week that Mr. Kemp
reopened Georgia.

It may have been an inartful line, but Mr. Patrick's full argument was
not wholly dismissive of human life. He was instead suggesting that the
human costs of shutting down the economy --- the losses of jobs and
income and the associated damages to living standards and people's
health --- were greater than the costs of a virus that kills only a
small percentage of people who get it.

``We are crushing the economy,'' he said, citing the damage to his own
children and grandchildren. ``We've got to take some risks and get back
in the game and get this country back up and running.''

The trouble with the argument, epidemiologists and economists agree, was
that public health and the economy's health were not really in conflict.

Early in the pandemic, Austan Goolsbee, a University of Chicago
economist and former Obama administration official, proposed what he
called
\href{https://twitter.com/austan_goolsbee/status/1241888950710669313?lang=en}{the
first rule of virus economics}: ``The best way to fix the economy is to
get control of the virus,'' he said. Until the virus was under control,
many people would be afraid to resume normal life and the economy would
not function normally.

The events of the last few months have borne out Mr. Goolsbee's
prediction. Even before states announced shutdown orders in the spring,
many families
\href{https://www.nytimes.com/interactive/2020/04/11/business/economy/coronavirus-us-economy-spending.html}{began
sharply reducing their spending}. They were responding to their own
worries about the virus, not any official government policy.

And the end of lockdowns, like Georgia's, did not fix the economy's
problems. It instead led to a brief increase in spending and hiring that
soon faded.

In the weeks after states reopened, the virus began surging. Those that
opened earliest tended to have worse outbreaks, according to a Times
analysis. The Southeast fared especially badly.

\hypertarget{states-that-reopened-earlier-are-seeing-bigger-outbreaks}{%
\subsubsection{States that reopened earlier are seeing bigger
outbreaks}\label{states-that-reopened-earlier-are-seeing-bigger-outbreaks}}

⟵ Reopened later Reopened earlier ⟶ //x Axis //yAxis 50 days since
reopening 70 90 110 100 200 300 400Avg. new cases per million now
Alabama Arizona California Florida Georgia Louisiana Michigan
Mississippi Missouri Nevada New Jersey New York Oklahoma Texas

In June and July, Georgia reported more than 125,000 new virus cases,
turning it into one of the globe's new hot spots. That was more new
cases than Canada, France, Germany, Italy, Japan and Australia combined
during that time frame.

Americans, frightened by the virus's resurgence, responded by visiting
restaurants and stores less often. The number of Americans filing new
claims for unemployment benefits
\href{https://www.nytimes.com/2020/07/23/business/economy/unemployment-economy-coronavirus.html}{has
stopped falling}. The economy's brief recovery in April and May seems to
have petered out in June and July.

In large parts of the United States, officials chose to reopen before
medical experts thought it wise, in an attempt to put people back to
work and spark the economy. Instead, the United States sparked a huge
new virus outbreak --- and the economy did not seem to benefit.

``Politicians are not in control,'' Mr. Goolsbee said. ``They got all
the illness and still didn't fix their economies.''

The situation is different in the European Union and other regions that
have had more success reducing new virus cases. Their economies have
begun showing
\href{https://www.nytimes.com/2020/07/31/business/europe-economy-recovery-coronavirus.html}{some
promising signs}, albeit tentative ones. In Germany, retail sales and
industrial production
\href{https://www.wsj.com/articles/germanys-economy-suffers-biggest-contraction-on-record-but-green-shoots-emerge-11596101866}{have
risen}, and the most recent unemployment rate was 6.4 percent. In the
United States, it was 11.1 percent.

Image

After a monthslong lockdown, New York had one of the country's lowest
rates of virus spread by June.Credit...Hiroko Masuike/The New York Times

\hypertarget{the-message-is-the-response}{%
\subsection{The message is the
response}\label{the-message-is-the-response}}

The United States has not performed uniquely poorly on every measure of
the virus response.

Mask wearing is more common than throughout much of Scandinavia and
Australia, according to surveys
\href{https://www.imperial.ac.uk/media/imperial-college/institute-of-global-health-innovation/ICL-YouGov-Covid-19-Behavour-Tracker_Global_FaceMask_20200609_VF.pdf}{by
YouGov and Imperial College London}. The total death rate is still
higher in Spain, Italy and Britain.

But there is one way --- in addition to the scale of the continuing
outbreaks and deaths --- that the United States stands apart: In no
other high-income country have the messages from political leaders been
nearly so mixed and confusing.

These messages, in turn, have been amplified by television stations and
websites friendly to the Republican Party, especially Fox News
\href{https://www.washingtonpost.com/lifestyle/media/sinclair-yanked-a-pandemic-conspiracy-theory-program-but-it-has-stayed-in-line-with-trump-on-coronavirus/2020/07/31/5d90a296-d021-11ea-8c55-61e7fa5e82ab_story.html}{and
the Sinclair Broadcast Group}, which operates almost 200 local stations.
To anybody listening to the country's politicians or watching these
television stations, it would have been difficult to know how to respond
to the virus.

Mr. Trump's comments, in particular, have regularly contradicted the
views of scientists and medical experts.

The day after the first American case was diagnosed, he said, ``We have
it totally under control.'' In late February, he said: ``It's going to
disappear. One day --- it's like a miracle --- it will disappear.''
Later, he incorrectly stated that any American who wanted a test could
get one. On July 28, he
\href{https://www.nytimes.com/2020/07/28/us/politics/trump-nobody-likes-me-walks-out-briefing.html}{falsely
proclaimed} that ``large portions of our country'' were ``corona-free.''

He has also
\href{https://www.theatlantic.com/politics/archive/2020/07/trumps-lies-about-coronavirus/608647/}{promoted
medical misinformation} about the virus. In March, Mr. Trump called it
``very mild'' and suggested it was less deadly than the common flu. He
has encouraged Americans to treat it with the antimalarial drug
hydroxychloroquine, despite a lack of evidence about its effectiveness
and concerns about its safety. At one White House briefing, he mused
aloud about injecting people with disinfectant to treat the virus.

These comments have helped create
\href{https://www.pewresearch.org/politics/2020/06/25/republicans-democrats-move-even-further-apart-in-coronavirus-concerns/}{a
large partisan divide} in the country, with Republican-leaning voters
less willing to wear masks or remain socially distant. Some
Democratic-leaning voters and less political Americans, in turn, have
decided that if everybody is not taking the virus seriously, they will
not either. State leaders from both parties have sometimes created so
many exceptions about which workplaces can continue operating normally
that their stay-at-home orders have had only modest effects.

``It doesn't seem we have had the same unity of purpose that I would
have expected,'' Ms. Rivers, the Johns Hopkins epidemiologist, said.
``You need everyone to come together to accomplish something big.''

Across much of Europe and Asia, as well as in Canada, Australia and
elsewhere, leaders have delivered a consistent message: The world is
facing a deadly virus, and only careful, consistent action will protect
people.

Many of those leaders have then pursued aggressive action. Mr. Trump and
his top aides, by contrast, persuaded themselves in April that the virus
was fading. They have also
\href{https://www.nytimes.com/2020/07/18/us/politics/trump-coronavirus-response-failure-leadership.html}{declined
to design a national strategy} for testing or other virus responses,
leading to a chaotic mix of state policies.

``If you had to summarize our approach, it's really poor federal
leadership --- disorganization and denial,'' said Andy Slavitt, who ran
Medicare and Medicaid from 2015 to 2017. ``Watch Angela Merkel. Watch
\href{https://nymag.com/intelligencer/2020/03/angela-merkel-nails-coronavirus-speech-unlike-trump.html}{how
she communicates} with the public. Watch how Jacinda Ardern in New
Zealand
\href{https://theconversation.com/three-reasons-why-jacinda-arderns-coronavirus-response-has-been-a-masterclass-in-crisis-leadership-135541}{does
it}. They're very clear. They're very consistent about what the most
important priorities are.''

New York --- both the city and the state --- offers a useful case study.
Like much of Europe, New York responded too slowly to the first wave of
the virus. As late as March 15, Mayor Bill de Blasio encouraged people
\href{https://nymag.com/intelligencer/2020/03/bill-de-blasio-had-his-worst-week-as-new-york-city-mayor.html}{to
go to their neighborhood bar}.

Soon, the city and state were overwhelmed. Ambulances wailed day and
night. Hospitals filled to the breaking point. Gov. Andrew M. Cuomo ---
a Democrat, like Mr. de Blasio --- was slow to protect nursing home
residents, and
\href{https://www.nytimes.com/2020/07/23/nyregion/nursing-homes-deaths-cuomo.html}{thousands
died}. Earlier action in New York could have saved a significant number
of lives, epidemiologists say.

By late March, however, New York's leaders understood the threat, and
they reversed course.

They insisted that people stay home. They repeated the message every
day, often on television. When other states began reopening, New York
did not. ``You look at the states that opened fast without metrics,
without guardrails, it's a boomerang,''
\href{https://www.governor.ny.gov/news/video-audio-photos-rush-transcript-governor-cuomo-announces-state-expanding-covid-19-testing}{Mr.
Cuomo said on June 4}.

The lockdowns and the consistent messages had a big effect. By June, New
York and surrounding states had
\href{https://www.nytimes.com/2020/07/22/us/coronavirus-northeast-governors.html}{some
of the lowest rates} of virus spread in the country. Across much of the
Southeast, Southwest and West Coast, on the other hand, the pandemic was
raging.

Many experts now say that the most disappointing part of the country's
failure is that the outcome was avoidable.

What may not have been avoidable was the initial surge of the virus: The
world's success in containing previous viruses, like SARS, had lulled
many people into thinking a devastating pandemic was unlikely. That
complacency helps explains China's early mistakes, as well as the
terrible death tolls in the New York region, Italy, Spain, Belgium,
Britain and other parts of Europe.

But these countries and dozens more --- as well as New York --- have
since shown that keeping the virus in check is feasible.

For all of the continuing uncertainty about how this new coronavirus is
transmitted and how it affects the human body, much has become clear. It
often spreads indoors, with close human contact. Talking, singing,
sneezing and coughing play a major role in transmission. Masks reduce
the risk. Restarting normal activity almost always leads to new cases
that require quick action --- testing, tracing of patients and
quarantining --- to keep the virus in check.

When countries and cities have heeded these lessons, they have rapidly
reduced the spread of the virus and been able to move back, gingerly,
toward normal life. In South Korea, fans have been able
\href{https://time.com/5871901/south-korea-baseball-coronavirus/}{to
attend baseball games} in recent weeks. In Denmark, Italy and other
parts of Europe, children have
\href{https://www.sciencemag.org/news/2020/07/school-openings-across-globe-suggest-ways-keep-coronavirus-bay-despite-outbreaks}{returned
to school}.

In the United States, the virus continues to overwhelm daily life.

``This isn't actually rocket science,'' said Dr. Thomas R. Frieden, who
ran the New York City health department and the C.D.C. for a combined 15
years. ``We know what to do, and we're not doing it.''

Contributing reporting were Damien Cave, J. David Goodman, Sarah
Mervosh, Monika Pronczuk and Motoko Rich.

Advertisement

\protect\hyperlink{after-bottom}{Continue reading the main story}

\hypertarget{site-index}{%
\subsection{Site Index}\label{site-index}}

\hypertarget{site-information-navigation}{%
\subsection{Site Information
Navigation}\label{site-information-navigation}}

\begin{itemize}
\tightlist
\item
  \href{https://help.nytimes.com/hc/en-us/articles/115014792127-Copyright-notice}{©~2020~The
  New York Times Company}
\end{itemize}

\begin{itemize}
\tightlist
\item
  \href{https://www.nytco.com/}{NYTCo}
\item
  \href{https://help.nytimes.com/hc/en-us/articles/115015385887-Contact-Us}{Contact
  Us}
\item
  \href{https://www.nytco.com/careers/}{Work with us}
\item
  \href{https://nytmediakit.com/}{Advertise}
\item
  \href{http://www.tbrandstudio.com/}{T Brand Studio}
\item
  \href{https://www.nytimes.com/privacy/cookie-policy\#how-do-i-manage-trackers}{Your
  Ad Choices}
\item
  \href{https://www.nytimes.com/privacy}{Privacy}
\item
  \href{https://help.nytimes.com/hc/en-us/articles/115014893428-Terms-of-service}{Terms
  of Service}
\item
  \href{https://help.nytimes.com/hc/en-us/articles/115014893968-Terms-of-sale}{Terms
  of Sale}
\item
  \href{https://spiderbites.nytimes.com}{Site Map}
\item
  \href{https://help.nytimes.com/hc/en-us}{Help}
\item
  \href{https://www.nytimes.com/subscription?campaignId=37WXW}{Subscriptions}
\end{itemize}
