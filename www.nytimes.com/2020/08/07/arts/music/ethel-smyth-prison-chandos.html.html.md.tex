Sections

SEARCH

\protect\hyperlink{site-content}{Skip to
content}\protect\hyperlink{site-index}{Skip to site index}

\href{https://www.nytimes.com/section/arts/music}{Music}

\href{https://myaccount.nytimes.com/auth/login?response_type=cookie\&client_id=vi}{}

\href{https://www.nytimes.com/section/todayspaper}{Today's Paper}

\href{/section/arts/music}{Music}\textbar{}Ethel Smyth, a Composer Long
Unheard, Is Recorded Anew

\href{https://nyti.ms/3in0zRb}{https://nyti.ms/3in0zRb}

\begin{itemize}
\item
\item
\item
\item
\item
\end{itemize}

Advertisement

\protect\hyperlink{after-top}{Continue reading the main story}

Supported by

\protect\hyperlink{after-sponsor}{Continue reading the main story}

My Favorite Page

\hypertarget{ethel-smyth-a-composer-long-unheard-is-recorded-anew}{%
\section{Ethel Smyth, a Composer Long Unheard, Is Recorded
Anew}\label{ethel-smyth-a-composer-long-unheard-is-recorded-anew}}

``The Prison'' --- her last major piece, premiered in 1931 --- comes out
on disc for the first time.

\includegraphics{https://static01.nyt.com/images/2020/08/08/arts/07smyth-1/07smyth-1-articleLarge.jpg?quality=75\&auto=webp\&disable=upscale}

By David Allen

\begin{itemize}
\item
  Aug. 7, 2020
\item
  \begin{itemize}
  \item
  \item
  \item
  \item
  \item
  \end{itemize}
\end{itemize}

``The exact worth of my music will probably not be known till naught
remains of the writer but sexless dots and lines on ruled paper,'' Ethel
Smyth wrote in 1928.

Long after her death, in 1944, she is finally being proven right. One of
numerous
\href{https://www.nytimes.com/interactive/2016/12/02/arts/music/01womencomposers.html}{female
composers} of the past now coming to
\href{https://www.nytimes.com/2019/08/28/arts/music/clara-schumann.html}{fresh},
\href{https://www.nytimes.com/2018/02/09/arts/music/florence-price-arkansas-symphony-concerto.html}{deserved}
\href{https://www.nytimes.com/2020/06/07/arts/music/amy-beach-takacs-ohlsson.html}{prominence},
Smyth was born in England in 1858 and moved to Leipzig, Germany, at 19,
training in the circle around Brahms.

She became the first woman to have a work
\href{https://www.nytimes.com/2016/02/19/arts/design/ethel-m-smyth-opera-composer-met-a-chorus-of-critical-disdain-in-1903.html}{performed}
by the Metropolitan Opera, in 1903, before she joined the militant wing
of British suffragists. When the conductor Thomas Beecham visited her at
Holloway Prison in London, where she spent three weeks in 1912 for
throwing rocks at a politician's house, he found inmates singing her
anthem, \href{https://www.youtube.com/watch?v=qTYv4wT8g4E}{``March of
the Women,''} while she conducted with a toothbrush.

\includegraphics{https://static01.nyt.com/images/2020/08/07/arts/07smyth-2/07smyth-2-articleLarge.jpg?quality=75\&auto=webp\&disable=upscale}

It was an experience that surely fed into Smyth's ``The Prison,'' first
performed in 1931 and, since she became progressively deaf, her last
major piece. It has now been recorded for the first time for
\href{https://www.chandos.net/products/catalogue/CHAN\%25205279}{Chandos},
the label which has already released Smyth works including her
\href{https://open.spotify.com/album/0g11cYcXjkCurVlymY8Jup?si=o-j4j4MeRmeNP6EV7CjS3Q}{Mass}
and
\href{https://open.spotify.com/album/4UOW8vU6ZmGF44jo6GjLkl?si=OBCppwafT1yiVG8j6G2pvw}{Serenade}.

\hypertarget{the-prisoner-awakes}{%
\subsubsection{``The prisoner awakes''}\label{the-prisoner-awakes}}

Chandos Records

An hourlong vocal symphony, ``The Prison'' recasts a text by Smyth's
frequent collaborator, the philosopher Henry Bennett Brewster, as the
tale of an innocent prisoner in solitary confinement (here the
bass-baritone \href{http://www.dashonburton.com/}{Dashon Burton})
reconciling himself to death in a dialogue with his soul (the soprano
\href{https://sarahbrailey.com/}{Sarah Brailey}).

\href{http://www.jamesblachly.com/}{James Blachly}, who conducts the
\href{https://experientialorchestra.com/}{Experiential Orchestra and
Chorus} on the recording, spoke over Zoom last month about the music,
its importance and his favorite page of the score, which he edited for
performance and publication. Here are edited excerpts from the
conversation.

Image

James Blachly, the conductor on a new recording of Smyth's ``The
Prison,'' chose the ending of Part One as his favorite page of the
score.Credit...via Musikproduktion Höflich

\textbf{You write in your booklet notes that you were initially hesitant
about conducting this music. Why was that, and what converted you?}

I have to confess that I had this sense that if I hadn't heard of her,
then she must not be very good. I had heard her name, but it was in
passing, more as a historical figure or as a novelty --- you know, the
rare female composer. Certainly the critical responses to her various
operas, especially, were misogynistic.

There are other reasons as well. Her publisher was Universal in Vienna,
and in 1939 she pulled the copyright from them, so her works were
obscured after her death. But I would say, more than anything, it was
this relegation of her to a historical figure, looking to her political
activity as the only thing to focus on.

The conversion moment really was a conversion moment. It was the first
time we did a reading of ``The Prison,'' in 2016. That downbeat, that
first sonority, I felt these chills go up and down my spine, and the
room seemed to just open up with this music that had been trapped, and
was getting released into the world again. It had never been heard in
the United States with an orchestra --- a colleague of mine, Mark
Shapiro, had
\href{https://ceciliachorusny.org/updates-contact/2018/4/13/conductors-note-the-prison-by-dame-ethel-smyth-and-requiem-by-wa-mozart}{conducted}
it with piano --- and so much of the piece is the orchestration.

\hypertarget{opening-bars}{%
\subsubsection{Opening Bars}\label{opening-bars}}

Chandos Records

\textbf{It's hard to talk about composers whose music isn't performed
much without talking about composers we might know better, but can you
describe Smyth's characteristic sound?}

I do think there are aspects that can be compared to other composers.
The opening to me sounds Wagnerian. It's rich, it's dark, and she was of
course a very German composer, in her education in Leipzig but also just
the way she's drawn, musically. But I have to say, as I conducted this
piece it became more and more clear to me that she has a unique musical
voice, and that to try to describe her in terms of other composers often
does a disservice. Finally having this recording means that I can just
say: Listen to the recording.

\textbf{Why do we need Ethel Smyth's music today?}

People will be drawn to her because of who she was. Her life story is
worthy of a feature-length film: her strength of character; her belief
in the integrity and worth of her music in the face of all sorts of
hardship; the way that she went and played her music for the conductor
Bruno Walter; showed her music to Brahms anonymously, so that he would
take it seriously. All of that, and who knows how she would identify now
in terms of sexuality; there is ample discussion of that by scholars.

But I can say that her music is needed today, on its own merits, and
because it seems to tie together so much else: so many other composers
that we know and love who she had direct contact with, and the
conductors who championed her. ** I was astonished at first to learn of
this list, but Thomas Beecham, first and foremost; Bruno Walter; Arthur
Nikisch; Adrian Boult. We need Smyth because she enriches our sense of
what music has been, and she enriches our sense of what music can be.

\textbf{What is ``The Prison'' about?}

It's a summary of her entire career. It's a farewell to her
compositional career. She knew that she was going deaf. There's a real
sense of making peace with that, and also reconciling herself to the
death of her closest creative companion, Henry Bennett Brewster. Anyone
who has lost someone can find a deep sense of peace through this work.
It's about love and life and loss and self-worth, and the essence of the
philosophy is about freeing oneself from the shackles of self. ``Who
doesn't have a prison?'' the prisoner writes at one point.

\hypertarget{end-of-part-one}{%
\subsubsection{End of Part One}\label{end-of-part-one}}

Chandos Records

\textbf{You picked the end of the first part as your favorite page. What
does it tell us?}

There is a profound sense of peace at this moment in the work. A lot has
been worked out and expressed in a gnarly, intricate fugue, and we come
to this dreamlike passage, an aria for the prisoner. The idea is that we
do live forever, whether we wish to or not --- that we will reappear in
another form. There's this sense of calm, and the D major chord is rich
and lush. This is almost more of an ending than the ending, which gives
a sense of indefiniteness.

\textbf{Does the fact that you have been able to perform this music with
such a large cast, then record it, show that classical music is slowly
changing for the better?}

I could sense even in the past decade that the general interest in music
that is yet uncelebrated is growing substantially. There has been a
problematic part of the classical music world which shuts out anything
that is not already granted access to this exclusive club. I hadn't set
out to be the next
\href{https://www.nytimes.com/2015/07/28/arts/music/review-the-wreckers-a-tale-of-piracy-love-and-betrayal.html}{Leon
Botstein}; I'm not trying to dedicate my career to unknown works. I just
happened to fall in love with this piece of music. But this has set me
on the course to question why and how we label certain music as
masterpieces --- and to realize that I can help shift people's
perception, that I have a role to play.

Advertisement

\protect\hyperlink{after-bottom}{Continue reading the main story}

\hypertarget{site-index}{%
\subsection{Site Index}\label{site-index}}

\hypertarget{site-information-navigation}{%
\subsection{Site Information
Navigation}\label{site-information-navigation}}

\begin{itemize}
\tightlist
\item
  \href{https://help.nytimes.com/hc/en-us/articles/115014792127-Copyright-notice}{©~2020~The
  New York Times Company}
\end{itemize}

\begin{itemize}
\tightlist
\item
  \href{https://www.nytco.com/}{NYTCo}
\item
  \href{https://help.nytimes.com/hc/en-us/articles/115015385887-Contact-Us}{Contact
  Us}
\item
  \href{https://www.nytco.com/careers/}{Work with us}
\item
  \href{https://nytmediakit.com/}{Advertise}
\item
  \href{http://www.tbrandstudio.com/}{T Brand Studio}
\item
  \href{https://www.nytimes.com/privacy/cookie-policy\#how-do-i-manage-trackers}{Your
  Ad Choices}
\item
  \href{https://www.nytimes.com/privacy}{Privacy}
\item
  \href{https://help.nytimes.com/hc/en-us/articles/115014893428-Terms-of-service}{Terms
  of Service}
\item
  \href{https://help.nytimes.com/hc/en-us/articles/115014893968-Terms-of-sale}{Terms
  of Sale}
\item
  \href{https://spiderbites.nytimes.com}{Site Map}
\item
  \href{https://help.nytimes.com/hc/en-us}{Help}
\item
  \href{https://www.nytimes.com/subscription?campaignId=37WXW}{Subscriptions}
\end{itemize}
