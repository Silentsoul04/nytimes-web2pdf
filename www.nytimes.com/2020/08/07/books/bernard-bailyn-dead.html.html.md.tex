Sections

SEARCH

\protect\hyperlink{site-content}{Skip to
content}\protect\hyperlink{site-index}{Skip to site index}

\href{https://www.nytimes.com/section/books}{Books}

\href{https://myaccount.nytimes.com/auth/login?response_type=cookie\&client_id=vi}{}

\href{https://www.nytimes.com/section/todayspaper}{Today's Paper}

\href{/section/books}{Books}\textbar{}Bernard Bailyn, Eminent Historian
of Early America, Dies at 97

\href{https://nyti.ms/3a7PksS}{https://nyti.ms/3a7PksS}

\begin{itemize}
\item
\item
\item
\item
\item
\end{itemize}

Advertisement

\protect\hyperlink{after-top}{Continue reading the main story}

Supported by

\protect\hyperlink{after-sponsor}{Continue reading the main story}

\hypertarget{bernard-bailyn-eminent-historian-of-early-america-dies-at-97}{%
\section{Bernard Bailyn, Eminent Historian of Early America, Dies at
97}\label{bernard-bailyn-eminent-historian-of-early-america-dies-at-97}}

On topic after topic he shifted the direction of scholarly inquiry,
winning two Pulitzers and a Bancroft Prize for his innovative research
and groundbreaking works.

\includegraphics{https://static01.nyt.com/images/2020/08/08/obituaries/08bailyn-obit1/merlin_162594921_2ad250ee-65df-4182-bdbb-7931e61faf82-articleLarge.jpg?quality=75\&auto=webp\&disable=upscale}

By \href{https://www.nytimes.com/by/renwick-mclean}{Renwick McLean} and
\href{https://www.nytimes.com/by/jennifer-schuessler}{Jennifer
Schuessler}

\begin{itemize}
\item
  Published Aug. 7, 2020Updated Aug. 8, 2020, 8:58 p.m. ET
\item
  \begin{itemize}
  \item
  \item
  \item
  \item
  \item
  \end{itemize}
\end{itemize}

Bernard Bailyn, a Harvard scholar whose award-winning books on early
American history reshaped the study of the origins of the American
Revolution, died on Friday at his home in Belmont, Mass., a suburb of
Boston. He was 97.

The cause was heart failure, said his wife,
\href{https://mitsloan.mit.edu/faculty/directory/lotte-bailyn}{Lotte
Bailyn}, a professor of management emerita at the M.I.T. Sloan School of
Management.

Though his name may not ring a bell with the legions of readers who
devour best-selling books on the founding of America, few historians
since World War II have left an imprint on that field of study that
rivals Professor Bailyn's. From the beginning, his work was innovative.
He was among the first historians to mine statistics from historical
records with a computer. And his insights and interpretations, notably
in his classic 1967 work, ``The Ideological Origins of the American
Revolution,'' could be groundbreaking.

On topic after topic, in more than 20 books that he wrote or edited, he
shifted the direction of scholarly inquiry, in the process winning two
Pulitzer Prizes, a National Book Award, a Bancroft Prize (the most
prestigious award given to scholars of American history) and, in 2011,
the National Humanities Medal, presented in a White House ceremony by
President Barack Obama. And as a professor at Harvard for more than a
half-century, he seeded many of the nation's top university history
departments with his acolytes.

``He has transformed the field of early American history as much as any
single person could,'' Gordon S. Wood, a historian at Brown University
and a former student of Professor Bailyn's, said in an interview for
this obituary in 2008. ``He transformed the history of education. He
turned over our entire interpretation of the Revolution. He changed the
way we think about immigration. Almost every single thing he did had a
profound impact on the field.''

When Professor Bailyn entered graduate school in 1946, the field of
colonial history was viewed by many as a backwater. Almost from the
beginning, he brought methodological rigor and startlingly fresh
interpretive questions to that endeavor.

Early in his career, he and his wife, while studying colonial-era
shipping, entered statistics from Massachusetts shipping records into a
primitive computer and found that Boston had one of the largest merchant
fleets in the British Empire in the early 1700s, indicating a
surprisingly vibrant and self-reliant economy. The resulting work,
``Massachusetts Shipping, 1697-1714: A Statistical Study'' (1959), was
one of the first historical works to include data analyzed by a
computer.

In other studies, Professor Bailyn examined specific social groups, like
New England merchants --- whose moneymaking, he argued, was as important
to understanding the country's origins as their Puritan religion --- and
the Virginia gentry.

He remains best known for ``The Ideological Origins of the American
Revolution,'' published in 1967. It began as a bibliographical essay on
hundreds of colonial pamphlets published between 1750 and 1776, which he
had been charged with preparing for publication. But it grew into a
sweeping study that changed the course of debate about the nation's
founding.

The book, which won both a Pulitzer and the Bancroft Prize, challenged
the then-dominant view of Progressive Era historians like Charles Beard,
who saw the founders' revolutionary rhetoric as a mask for economic
interests.

For Professor Bailyn, the pamphlets revealed a striking pattern. In his
view, though the colonists opposed taxes, restrictions on trade and
other economic measures, and were frustrated with their subordinate
status in British society, it was a fundamental distrust of government
power that led them to throw off the colonial yoke.

The colonists had inherited this ideology from opposition politicians
and writers in England, he argued. But it became particularly potent in
the relative isolation of the American colonies, where unpopular
policies enacted an ocean away were interpreted as signs of a corrupt
conspiracy to deny colonists their freedom.

The impact of Professor Bailyn's book reverberated far beyond colonial
history. The historian Forrest McDonald wrote in The New York Times Book
Review in 1990 that in the two decades after ``Ideological Origins'' was
published, ``ideological interpretation of the whole sweep of American
history from the 1760s to the 1840s expanded into a veritable cottage
industry.''

The book drew readers from beyond the scholarly world. A 1971 article in
the The Times about Daniel Ellsberg, the leaker of the Pentagon Papers,
described him pulling a copy of ``Ideological Origins'' the Bailyn book
out of his briefcase and being moved almost to tears as he read from it.

\includegraphics{https://static01.nyt.com/images/2020/08/08/obituaries/08bailyn-obit4/08bailyn-obit4-articleLarge.jpg?quality=75\&auto=webp\&disable=upscale}

Today, as debate over the origins and meaning of the American Revolution
remains contentious, the book remains on syllabuses, drawing engagement
even from younger scholars who might otherwise dismiss decades-old
historical works as outmoded.

``Most of the books published in the decades after `Ideological Origins'
responded to it in some way --- often by challenging its arguments,''
the historian Mary Beth Norton, a former Bailyn student,
\href{https://harvardpress.typepad.com/hup_publicity/2017/04/bernard-bailyns-ideological-origins-at-fifty.html\#norton}{wrote
in 2017} in one of a number of round tables marking the book's 50th
anniversary. ``That is a remarkable achievement for a book published
half a century ago.''

Professor Bailyn was known not just for rigorous scholarship but also
for his elegant prose. **** For him, ``a kind of literary imagination''
was essential to the historian's craft.

``Like a novelist,'' he wrote, the historian must conjure ``a
nonexistent, an impalpable world in all its living comprehension, and
yet do this within the constraints of verifiable facts.''

Though he stressed the importance of narrative, he did not write to
popularize history, and rarely gave interviews. But he wrote not just
for scholars but also for his ``better students'' ---non-scholars, as he
put it in one of those rare interviews, in 1994, with ``an active
interest in history who would be sufficiently interested to read some
detailed material.''

Within the profession, Professor Bailyn was a frequent critic of
overspecialization, abstraction and politicized ``presentism'' --- that
is, interpreting past events in terms of modern thinking and values. For
him, it was essential to respect the strangeness and pastness of the
past, and to see it, as much as possible, on its own terms.

``The establishment, in some significant degree, of a realistic
understanding of the past, free of myths, wish fulfillments and partisan
delusions, is essential for social sanity,'' he said in a 1995 lecture.

Bernard Bailyn --- Bud to his friends --- was born on Sept. 10, 1922, in
Hartford, Conn., to Charles and Esther (Schloss) Bailyn. His father was
a dentist, his mother a homemaker.

In 1940, he entered Williams College in Massachusetts, where he majored
in English and dabbled in philosophy. He earned a bachelor's degree in
1945, after he had been drafted into the Army.

Growing up, he later recalled, he had not much been engaged by history.
But while serving in the Signal Corps, he studied the German language
and social geography. After the war, he enrolled in graduate school at
Harvard.

At the time, Harvard was still a redoubt of the old WASP establishment.
Professor Bailyn, who was Jewish, later recalled how one of his
professors, the eminent scholar
\href{https://www.nytimes.com/1976/05/16/archives/adm-morison-88-historian-is-dead-samuel-eliot-morison-historian-is.html\#:~:text=Samuel\%20Eliot\%20Morison\%2C\%20the\%20undisputed,in\%20Maine\%20in\%20the\%20summer.}{Samuel
Eliot Morison}, had taken little interest in him, and repeatedly
confused him with a member of the Harvard Yacht Club.

By Professor Bailyn's account, he fell into colonial history almost
accidentally, driven mainly by a desire to examine, as he put it, ``the
connections between a distant past and an emerging modernity.''

He earned his Ph.D. in 1953 and joined the Harvard faculty. He was
famous for his vivid lectures and heady if not intimidating graduate
seminar, where he would punctuate wayward discussion with what the
historian Jack N. Rakove recalled as ``the most famous of his questions,
`So what?'''

The book on the **** Revolution cemented his reputation, but Professor
Bailyn continued to explore new territory and new genres. In 1975, he
published ``The Ordeal of Thomas Hutchinson,'' a biography of the last
colonial governor of Massachusetts.

Image

Professor Bailyn spoke at the Harvard convocation in 1986, before guests
that included the Prince of Wales.Credit...Frank O'Brien/The Boston
Globe, via Getty Images

The book, which won the National Book Award, was an attempt to explore,
as he said, ``the origins of the Revolution as experienced by the
losers.'' But it was read by some as a defense of the establishment ---
or even, some suggested, of Richard M. Nixon, who had resigned the
presidency the year before the book was published --- at a time of
political upheaval at Harvard and across the country.

Professor Bailyn, who once described himself as ``not very political,''
cheerfully scoffed at the idea that he would be using Hutchinson to make
a modern-day point. But he did allow that he had come to feel sympathy
for Hutchinson, whom he described as ``that rather stiff, intelligent,
highly literate, uncorrupted, honest, upright provincial
merchant-turned-judge and politician.''

In more recent decades, as interest in the experiences of women,
African-Americans and other marginalized groups exploded among
historians, Professor Bailyn's name was sometimes invoked as
``pejorative shorthand for an outmoded view of the past that celebrates
elites,'' as the historian Kenneth Owen put it in 2017.

For his part, Professor Bailyn often spoke against what he called the
``fashionable'' tendency to excoriate the American founders, whom
\href{https://www.neh.gov/news/press-release/1998-03-23}{he called}, for
all their faults, ``one of the most creative groups in history.''

``They gave us the foundations of our public life,'' he told an
interviewer in 2010. ``Their world was very different from ours, but,
more than any other country, we live with their world and with what they
achieved.''

Professor Bailyn won a second Pulitzer in 1987, for ``Voyagers to the
West,'' the first volume of a series called ``The Peopling of British
North America,'' which traces the journeys of the nearly 10,000 Britons
who were known to have emigrated to America from 1773 to 1776 and
explores the processes by which the colonies became a distinctly
American society.

A second volume, ``The Barbarous Years,'' published in 2013, chronicles
the chaotic, violent decades between the founding of Jamestown in 1607
and the 1675 conflict known as King Philip's War, which effectively
pushed Native Americans out of New England.

``The Barbarous Years'' was a finalist for the Pulitzer, but, like
``Voyagers,'' it drew strong criticism from fellow historians for what
they saw as inadequate or dismissive treatment of nonwhite people.

Professor Bailyn pressed on. In 1995, four years after officially
retiring, he established the
\href{https://sites.fas.harvard.edu/~atlantic/index.htm}{International
Seminar on the History of the Atlantic World}, an annual Harvard
gathering of young scholars from around the world that is credited with
helping to pioneer the now-vast field of Atlantic history.

Image

President Barack Obama presented Professor Bailyn with a National
Humanities Medal in 2011.Credit...Pablo Martinez Monsivais/Associated
Press

In addition to his wife, Professor Bailyn is survived by two sons,
Charles, an astronomy professor at Yale, and John, a linguistics
professor at Stony Brook University on Long Island; and two
granddaughters.

For all the grand sweep of his interpretations, Professor Bailyn could
seem at his most exuberant when digging into the fine-grained
particularities of sources, puzzling over the historical ``anomalies''
--- a favorite Bailyn word --- that they reveal.

In 2020, he published ``Illuminating History: A Retrospective of Seven
Decades,'' an intellectual self-portrait that eschews conventional
memoir in favor of a series of essays exploring some ``small, strange,
obscure documents and individuals'' that had captured his imagination.

In an epilogue, he cautioned, as he often did, against imposing our own
sense of certainty on the confusion of the past as it was actually
experienced by those who lived it.

``The fact --- the inescapable fact --- is that we know how it all came
out,'' he wrote, ``and they did not.''

Advertisement

\protect\hyperlink{after-bottom}{Continue reading the main story}

\hypertarget{site-index}{%
\subsection{Site Index}\label{site-index}}

\hypertarget{site-information-navigation}{%
\subsection{Site Information
Navigation}\label{site-information-navigation}}

\begin{itemize}
\tightlist
\item
  \href{https://help.nytimes.com/hc/en-us/articles/115014792127-Copyright-notice}{©~2020~The
  New York Times Company}
\end{itemize}

\begin{itemize}
\tightlist
\item
  \href{https://www.nytco.com/}{NYTCo}
\item
  \href{https://help.nytimes.com/hc/en-us/articles/115015385887-Contact-Us}{Contact
  Us}
\item
  \href{https://www.nytco.com/careers/}{Work with us}
\item
  \href{https://nytmediakit.com/}{Advertise}
\item
  \href{http://www.tbrandstudio.com/}{T Brand Studio}
\item
  \href{https://www.nytimes.com/privacy/cookie-policy\#how-do-i-manage-trackers}{Your
  Ad Choices}
\item
  \href{https://www.nytimes.com/privacy}{Privacy}
\item
  \href{https://help.nytimes.com/hc/en-us/articles/115014893428-Terms-of-service}{Terms
  of Service}
\item
  \href{https://help.nytimes.com/hc/en-us/articles/115014893968-Terms-of-sale}{Terms
  of Sale}
\item
  \href{https://spiderbites.nytimes.com}{Site Map}
\item
  \href{https://help.nytimes.com/hc/en-us}{Help}
\item
  \href{https://www.nytimes.com/subscription?campaignId=37WXW}{Subscriptions}
\end{itemize}
