Sections

SEARCH

\protect\hyperlink{site-content}{Skip to
content}\protect\hyperlink{site-index}{Skip to site index}

\href{https://www.nytimes.com/section/business/economy}{Economy}

\href{https://myaccount.nytimes.com/auth/login?response_type=cookie\&client_id=vi}{}

\href{https://www.nytimes.com/section/todayspaper}{Today's Paper}

\href{/section/business/economy}{Economy}\textbar{}Job Growth Slowed in
July, Signaling a Loss of Economic Momentum

\href{https://nyti.ms/30Dfcd9}{https://nyti.ms/30Dfcd9}

\begin{itemize}
\item
\item
\item
\item
\item
\end{itemize}

\href{https://www.nytimes.com/news-event/coronavirus?action=click\&pgtype=Article\&state=default\&region=TOP_BANNER\&context=storylines_menu}{The
Coronavirus Outbreak}

\begin{itemize}
\tightlist
\item
  live\href{https://www.nytimes.com/2020/08/08/world/coronavirus-updates.html?action=click\&pgtype=Article\&state=default\&region=TOP_BANNER\&context=storylines_menu}{Latest
  Updates}
\item
  \href{https://www.nytimes.com/interactive/2020/us/coronavirus-us-cases.html?action=click\&pgtype=Article\&state=default\&region=TOP_BANNER\&context=storylines_menu}{Maps
  and Cases}
\item
  \href{https://www.nytimes.com/interactive/2020/science/coronavirus-vaccine-tracker.html?action=click\&pgtype=Article\&state=default\&region=TOP_BANNER\&context=storylines_menu}{Vaccine
  Tracker}
\item
  \href{https://www.nytimes.com/interactive/2020/world/coronavirus-tips-advice.html?action=click\&pgtype=Article\&state=default\&region=TOP_BANNER\&context=storylines_menu}{F.A.Q.}
\item
  \href{https://www.nytimes.com/live/2020/08/07/business/stock-market-today-coronavirus?action=click\&pgtype=Article\&state=default\&region=TOP_BANNER\&context=storylines_menu}{Markets
  \& Economy}
\end{itemize}

Advertisement

\protect\hyperlink{after-top}{Continue reading the main story}

Supported by

\protect\hyperlink{after-sponsor}{Continue reading the main story}

\hypertarget{job-growth-slowed-in-july-signaling-a-loss-of-economic-momentum}{%
\section{Job Growth Slowed in July, Signaling a Loss of Economic
Momentum}\label{job-growth-slowed-in-july-signaling-a-loss-of-economic-momentum}}

The ranks of the employed grew by 1.8 million, a drop from the pace of
the previous two months, as renewed business closings hampered the
recovery.

\hypertarget{jobs-remain-far-below-pre-pandemic-levels}{%
\subsubsection{Jobs remain far below pre-pandemic
levels}\label{jobs-remain-far-below-pre-pandemic-levels}}

\hypertarget{cumulative-change-in-jobs-since-july-2016}{%
\paragraph{Cumulative change in jobs since July
2016}\label{cumulative-change-in-jobs-since-july-2016}}

By Allison McCann·Data is seasonally adjusted.·Source: Bureau of Labor
Statistics

By \href{https://www.nytimes.com/by/nelson-d-schwartz}{Nelson D.
Schwartz} and Gillian Friedman

\begin{itemize}
\item
  Aug. 7, 2020
\item
  \begin{itemize}
  \item
  \item
  \item
  \item
  \item
  \end{itemize}
\end{itemize}

The American economy slowed in July as the pace of hiring eased from the
robust rate of the previous two months, a victim of waning momentum and
the resurgence of the coronavirus in many parts of the country.

Employers added 1.8 million jobs, well below the 4.8 million jump in
payrolls in June, the Labor Department reported, after virus-related
restrictions caused some businesses to close for a second time. The
unemployment rate fell to 10.2 percent.

Hours after the report underscored the slowing recovery, talks between
administration officials and congressional Democrats on how to pump more
aid into the economy were on the verge of collapse. On Friday night,
President Trump threatened to bypass Congress and act on his own ---
though his power to do so was unclear.

Prominent among the unresolved issues were a revival of the government's
\$600-a-week supplement to unemployment aid, a lifeline for millions of
jobless workers until it expired at the end of last month, and a
possible extension of
\href{https://www.nytimes.com/2020/08/07/business/economy/housing-economy-eviction-renters.html}{an
eviction moratorium covering many of the nation's tenants}.

Even with July's gains, fewer than half of the 22 million jobs lost in
March and April have been restored. And economists warn that the rest of
the lost ground will be a challenge to regain.

``The easy hiring that was done in May and June has been exhausted,''
said Michelle Meyer, head of U.S. economics at Bank of America. ``With
many companies not running at full capacity, it becomes harder to get
that incremental worker back in.''

Over all, the job market reflects the crosswinds buffeting the economy
less than 100 days before the presidential election. Retailers continue
to file for bankruptcy, while airlines and hotels operate at a small
fraction of capacity. Some companies are calling back laid-off
employees, even as other employers continue to shed workers.

The longer the crisis goes on, the greater the toll for businesses,
especially smaller ones.

``We're going to start to see a lot of small businesses fall by the
wayside, a lot of people who are unemployed become chronically
unemployed,'' said Kenneth S. Rogoff, a Harvard University economist who
has written extensively on financial and economic crises. ``We're in
very, very dangerous territory.''

And underscoring the prevalence of what economists term ``churn'' in the
labor market, some jobless Americans have secured work only to find
themselves out of a job for a second time.

\hypertarget{latest-updates-the-coronavirus-outbreak-and-the-economy}{%
\section{\texorpdfstring{\href{https://www.nytimes.com/live/2020/08/07/business/stock-market-today-coronavirus?action=click\&pgtype=Article\&state=default\&region=MAIN_CONTENT_1\&context=storylines_live_updates}{Latest
Updates: The Coronavirus Outbreak and the
Economy}}{Latest Updates: The Coronavirus Outbreak and the Economy}}\label{latest-updates-the-coronavirus-outbreak-and-the-economy}}

\href{https://www.nytimes.com/live/2020/08/07/business/stock-market-today-coronavirus?action=click\&pgtype=Article\&state=default\&region=MAIN_CONTENT_1\&context=storylines_live_updates\#wealthy-families-are-throwing-a-lifeline-to-distressed-businesses}{25h
ago}

\href{https://www.nytimes.com/live/2020/08/07/business/stock-market-today-coronavirus?action=click\&pgtype=Article\&state=default\&region=MAIN_CONTENT_1\&context=storylines_live_updates\#wealthy-families-are-throwing-a-lifeline-to-distressed-businesses}{Wealthy
families are throwing a lifeline to distressed businesses.}

\href{https://www.nytimes.com/live/2020/08/07/business/stock-market-today-coronavirus?action=click\&pgtype=Article\&state=default\&region=MAIN_CONTENT_1\&context=storylines_live_updates\#the-publisher-of-the-onion-jezebel-and-other-websites-lays-off-15-employees}{26h
ago}

\href{https://www.nytimes.com/live/2020/08/07/business/stock-market-today-coronavirus?action=click\&pgtype=Article\&state=default\&region=MAIN_CONTENT_1\&context=storylines_live_updates\#the-publisher-of-the-onion-jezebel-and-other-websites-lays-off-15-employees}{The
publisher of The Onion, Jezebel and other websites lays off 15
employees.}

\href{https://www.nytimes.com/live/2020/08/07/business/stock-market-today-coronavirus?action=click\&pgtype=Article\&state=default\&region=MAIN_CONTENT_1\&context=storylines_live_updates\#canada-outlines-its-response-to-the-new-us-aluminum-tariff}{31h
ago}

\href{https://www.nytimes.com/live/2020/08/07/business/stock-market-today-coronavirus?action=click\&pgtype=Article\&state=default\&region=MAIN_CONTENT_1\&context=storylines_live_updates\#canada-outlines-its-response-to-the-new-us-aluminum-tariff}{Canada
outlines its response to the new U.S. aluminum tariff.}

\href{https://www.nytimes.com/live/2020/08/07/business/stock-market-today-coronavirus?action=click\&pgtype=Article\&state=default\&region=MAIN_CONTENT_1\&context=storylines_live_updates}{See
more updates}

More live coverage:
\href{https://www.nytimes.com/2020/08/07/world/covid-19-news.html?action=click\&pgtype=Article\&state=default\&region=MAIN_CONTENT_1\&context=storylines_live_updates}{Global}

After coronavirus-related lockdowns forced dining establishments in New
York to close in March, Hannah Lane, 24, was laid off as a server in a
popular Gramercy Park restaurant where she made about \$60,000 a year.

\includegraphics{https://static01.nyt.com/images/2020/08/07/business/07markets-brf-layoff/merlin_175402275_5dfa702d-9008-470c-aff0-0e3b61e03357-articleLarge.jpg?quality=75\&auto=webp\&disable=upscale}

She applied for unemployment benefits, but had to wait two months for
the payments to begin. Then, in early July, as New York allowed
restaurants to open for indoor dining, Ms. Lane was recalled to her job.

``I went back into work, clocked in, went back on payroll, the whole
nine yards,'' she said.

She had spent just one day there when Gov. Andrew M. Cuomo reversed
course and prohibited dining inside restaurants. Ms. Lane was laid off
again, and found herself back on unemployment and looking for work.

The leisure and hospitality industry was hit hard in the downturn and
faces new restrictions on bars and indoor dining in states like
California, Florida and Texas.

Last month, it added 592,000 jobs, or one-third of the net gain for the
economy over all.

\hypertarget{industries-are-rebounding-but-none-have-fully-recovered}{%
\subsubsection{Industries are rebounding, but none have fully
recovered}\label{industries-are-rebounding-but-none-have-fully-recovered}}

\hypertarget{cumulative-change-in-jobs-since-july-2016-by-industry}{%
\paragraph{Cumulative change in jobs since July 2016, by
industry}\label{cumulative-change-in-jobs-since-july-2016-by-industry}}

By Allison McCann·Data is seasonally adjusted.·Source: Bureau of Labor
Statistics

July's job growth in the industry followed a jump of 3.4 million in May
and June, seasonally adjusted, but employment in the field is still 4.3
million below where it was in February.

The retail industry, another hard-hit sector that has seen numerous
bankruptcies in recent months, added 258,000 jobs last month.

The pandemic's toll on jobs in those categories has hit lower-paid
workers especially hard, including millions who depend on tips. For big
increases in hiring at restaurants and bars, employees may need to wait
until indoor dining is again permitted in states like New York ---
something unlikely to occur until a vaccine is found.

While the survey of households released on Friday counted 16.3 million
Americans as unemployed, the Labor Department has reported that over 30
million are receiving some sort of unemployment benefit.

The household survey does not count people as unemployed if they have
given up the search for work and are not considered part of the labor
force. There are also differences between the Labor Department's
definition of unemployment and state requirements for benefits.

Image

A barista on the job this week in Kansas City, Mo., at a coffee spot
that had laid off much of its staff early in the
pandemic.Credit...Christopher Smith for The New York Times

Image

American Freight in Independence, Mo., was looking for workers this
week. Even as some businesses hire, layoffs persist.Credit...Christopher
Smith for The New York Times

However it is measured, the pandemic's economic pain has not been
distributed evenly.

The seasonally adjusted unemployment rate for Black adults
\href{https://www.bls.gov/news.release/empsit.t02.htm}{was 14.6 percent}
in July, down slightly from 15.4 percent the month before and a little
more than two percentage points from its peak in May --- but still more
than double its 5.8 percent rate in February.

Joblessness for white workers eased to 9.2 percent in July. While that
rate is up sharply from 3.1 percent in February, it has fallen about
five percentage points from its April peak.

Unemployment among other minority groups also remains elevated. The rate
for Hispanic
\href{https://www.bls.gov/news.release/empsit.t03.htm}{workers} was at
12.9 percent, up from 4.4 percent before the crisis. Asian workers ---
who before the downturn had the lowest jobless rate of any demographic
group --- posted a 12 percent unemployment rate in July.

\hypertarget{black-men-continue-to-have-the-highest-rate-of-unemployment}{%
\subsubsection{Black men continue to have the highest rate of
unemployment}\label{black-men-continue-to-have-the-highest-rate-of-unemployment}}

\hypertarget{unemployment-rates-by-race-for-men-women-and-over-all}{%
\paragraph{Unemployment rates by race for men, women and over
all}\label{unemployment-rates-by-race-for-men-women-and-over-all}}

Black

Hispanic

Asian

White

By Allison McCann·Rates are seasonally adjusted except those for Asian
men and women.·Source: Bureau of Labor Statistics

Policymakers have noted the differing impact. ``The rise in joblessness
has been especially severe for lower-wage workers, for women, and for
African-Americans and Hispanics,'' Jerome H. Powell, the Federal Reserve
chair, said at a
\href{https://www.federalreserve.gov/mediacenter/files/FOMCpresconf20200729.pdf}{news
conference} in late July. ``This reversal of economic fortune has
upended many lives and created great uncertainty about the future.''

For some workers, securing a position has meant accepting lower pay.

When the pandemic hit, David Espy was a safety manager overseeing the
construction of a resort hotel at Walt Disney World in Florida. But in
mid-March, when virus-related shutdowns forced entertainment venues to
close, Mr. Espy lost his job.

After being unemployed for one month, Mr. Espy, 59, was hired by a
consulting company called Safety, Solutions and Supply. Before the
pandemic, he was making \$125,000 a year. Now, he earns \$75,000.

The new job does not pay him enough to cover his expenses, including two
car loans and the mortgage on his house in Valrico, Fla., where he lives
with his wife and a 20-year-old son. To make ends meet, he is spending
\$2,000 of his savings each month.

``I would call myself underemployed,'' he said. ``I'm working at a
reduced rate just to pay my bills.''

Image

``I'm working at a reduced rate just to pay my bills,'' said David Espy,
who took a job paying far less than the one he lost at the start of the
pandemic.Credit...Zack Wittman for The New York Times

Even as some employers recall laid-off workers, others are concluding
they can no longer stay in business. That has caused financial and
emotional damage for owners and employees alike.

For Jackie Anscher, the closing of the boutique fitness studio where she
taught spinning classes in Long Beach, N.Y., until March meant more than
the loss of a job. It was the end of something she was passionate about
and halted the deep connections she had built with clients.

``I miss it like I've lost a limb,'' she said. ``What started as an
exercise class encompassed so much more. I'm a therapist on a bike. I'm
sure a lot of people can relate to the emotional loss.''

Ms. Anscher, who taught eight to 10 classes a week, said her financial
situation was stable because of her husband's job. But there is nowhere
to go to keep teaching as gyms remain closed.

``This was a forced retirement,'' Ms. Anscher, 58, said. ``I'm not ready
to retire. I'm waiting to see how I can pick up the pieces.''

Stephanie Horowitz, the studio's owner, didn't think the moratorium on
classes would be the end of her business, Ocean Ride, when it was
imposed in March. She offered spinning classes over the internet, she
said, ``but it never took off the way we needed it to.''

By mid-July, the financial drain was too great, and she decided to shut
down after seven years. Some of the bikes have been sold, and Ms.
Horowitz has been cleaning out the space on the South Shore of Long
Island, a few blocks from the Atlantic. Seven part-time workers,
including Ms. Anscher, have lost their jobs.

``We were a staple in the community, and we had a good run,'' Ms.
Horowitz, 40, said. ``It's emotional. We had just bought new bikes last
year. Who knows what the future holds for any of us?''

Jeanna Smialek and Ben Casselman contributed reporting.

Advertisement

\protect\hyperlink{after-bottom}{Continue reading the main story}

\hypertarget{site-index}{%
\subsection{Site Index}\label{site-index}}

\hypertarget{site-information-navigation}{%
\subsection{Site Information
Navigation}\label{site-information-navigation}}

\begin{itemize}
\tightlist
\item
  \href{https://help.nytimes.com/hc/en-us/articles/115014792127-Copyright-notice}{©~2020~The
  New York Times Company}
\end{itemize}

\begin{itemize}
\tightlist
\item
  \href{https://www.nytco.com/}{NYTCo}
\item
  \href{https://help.nytimes.com/hc/en-us/articles/115015385887-Contact-Us}{Contact
  Us}
\item
  \href{https://www.nytco.com/careers/}{Work with us}
\item
  \href{https://nytmediakit.com/}{Advertise}
\item
  \href{http://www.tbrandstudio.com/}{T Brand Studio}
\item
  \href{https://www.nytimes.com/privacy/cookie-policy\#how-do-i-manage-trackers}{Your
  Ad Choices}
\item
  \href{https://www.nytimes.com/privacy}{Privacy}
\item
  \href{https://help.nytimes.com/hc/en-us/articles/115014893428-Terms-of-service}{Terms
  of Service}
\item
  \href{https://help.nytimes.com/hc/en-us/articles/115014893968-Terms-of-sale}{Terms
  of Sale}
\item
  \href{https://spiderbites.nytimes.com}{Site Map}
\item
  \href{https://help.nytimes.com/hc/en-us}{Help}
\item
  \href{https://www.nytimes.com/subscription?campaignId=37WXW}{Subscriptions}
\end{itemize}
