Sections

SEARCH

\protect\hyperlink{site-content}{Skip to
content}\protect\hyperlink{site-index}{Skip to site index}

\href{https://www.nytimes.com/section/business}{Business}

\href{https://myaccount.nytimes.com/auth/login?response_type=cookie\&client_id=vi}{}

\href{https://www.nytimes.com/section/todayspaper}{Today's Paper}

\href{/section/business}{Business}\textbar{}Targeting WeChat, Trump
Takes Aim at China's Bridge to the World

\href{https://nyti.ms/33xiOzb}{https://nyti.ms/33xiOzb}

\begin{itemize}
\item
\item
\item
\item
\item
\item
\end{itemize}

Advertisement

\protect\hyperlink{after-top}{Continue reading the main story}

Supported by

\protect\hyperlink{after-sponsor}{Continue reading the main story}

\hypertarget{targeting-wechat-trump-takes-aim-at-chinas-bridge-to-the-world}{%
\section{Targeting WeChat, Trump Takes Aim at China's Bridge to the
World}\label{targeting-wechat-trump-takes-aim-at-chinas-bridge-to-the-world}}

The all-purpose app, which the administration is restricting along with
TikTok, is how many Chinese living abroad stay in touch with one
another, and with people back home.

\includegraphics{https://static01.nyt.com/images/2020/08/08/world/07JPchina-wechat1-print/merlin_139155753_efcaf9ee-1c9a-4a78-9a29-81cfc90402bf-articleLarge.jpg?quality=75\&auto=webp\&disable=upscale}

\href{https://www.nytimes.com/by/paul-mozur}{\includegraphics{https://static01.nyt.com/images/2018/10/15/multimedia/author-paul-mozur/author-paul-mozur-thumbLarge.png}}\href{https://www.nytimes.com/by/raymond-zhong}{\includegraphics{https://static01.nyt.com/images/2018/10/15/multimedia/author-raymond-zhong/author-raymond-zhong-thumbLarge.png}}

By \href{https://www.nytimes.com/by/paul-mozur}{Paul Mozur} and
\href{https://www.nytimes.com/by/raymond-zhong}{Raymond Zhong}

\begin{itemize}
\item
  Aug. 7, 2020
\item
  \begin{itemize}
  \item
  \item
  \item
  \item
  \item
  \item
  \end{itemize}
\end{itemize}

TAIPEI, Taiwan --- In China,
\href{https://www.nytimes.com/video/technology/100000004574648/china-internet-wechat.html}{WeChat
does more} than any app rightfully should. People use it to talk, shop,
share photos, pay bills, get their news and send money.

With much of the Chinese internet
\href{https://www.nytimes.com/2014/09/22/business/international/china-clamps-down-on-web-pinching-companies-like-google.html}{locked
behind a wall} of filters and censors, the country's everything app is
also one of the few digital bridges connecting China to the rest of the
world. It is the way exchange students talk to their families,
immigrants keep up with relatives and much of the Chinese diaspora swaps
memes, gossip and videos.

Now, that bridge is threatening to crumble.

Late Thursday, the Trump administration
\href{https://www.nytimes.com/2020/08/06/technology/trump-wechat-tiktok-china.html?smid=tw-share}{issued
an executive order} that could pull China's most important app from
Apple and Google stores across the world and prevent American companies
from doing business with its parent company, Tencent. Light on details,
\href{https://www.whitehouse.gov/presidential-actions/executive-order-addressing-threat-posed-wechat/}{the
decree} could prove cosmetic, crushing or something in between.

If enforced strongly when it takes effect in 45 days, the order will
take dead aim at China's single most
\href{https://www.nytimes.com/2017/07/16/business/china-cash-smartphone-payments.html}{groundbreaking
internet product}, which
\href{https://mp.weixin.qq.com/s/IH09J0CGOJqaYHFWXC4fNw}{1.2 billion
people} use every month. An effective ban on the app in the United
States would cut short millions of conversations between investors,
business partners, family members and friends. The threat alone will
likely start a new chapter in the
\href{https://www.nytimes.com/2020/07/14/world/asia/cold-war-china-us.html}{deepening
standoff} between China and the United States over the future of
technology.

Taken together with
\href{https://www.whitehouse.gov/presidential-actions/executive-order-addressing-threat-posed-tiktok/}{Thursday's
twin order} against the
\href{https://www.nytimes.com/2020/08/03/technology/tiktok-bytedance-us-china.html}{Chinese-owned
video app TikTok}, the move against WeChat marks a shift in the American
approach to the Great Firewall, which for years has kept companies like
Facebook and Google from operating in China. Restricting WeChat and
TikTok could be the first steps in an eye-for-an-eye reprisal.

While TikTok may be the fad of the moment in the United States, WeChat
is far more important in China. A digital bedrock of daily life, WeChat
emerged as a tool for the Chinese authorities to impose social controls.
Within China, the app is heavily censored and monitored by a newly
empowered force of
\href{https://www.nytimes.com/2020/03/16/business/china-coronavirus-internet-police.html}{internet
police}.

\includegraphics{https://static01.nyt.com/images/2020/08/08/world/07JPchina-wechat2-print/merlin_174797511_7a75afea-4af3-4125-8b52-d97198a1b86f-articleLarge.jpg?quality=75\&auto=webp\&disable=upscale}

Outside China's borders, the app has become a key conduit for the spread
of
\href{https://www.nytimes.com/2018/03/02/technology/china-technology-censorship-borders-expansion.html}{Beijing's
propaganda}. Chinese security forces have also regularly used WeChat to
intimidate and silence members of the Chinese diaspora, including
minority Uighurs
\href{https://www.nytimes.com/2019/08/15/podcasts/the-daily/china-xinjiang-uighur-detention.html}{seeking
to raise awareness} of harsh crackdowns in their homeland in western
China.

``The downside of this executive order is that it's addressing these
concerns by taking steps that also make it harder to directly
communicate with ordinary people in China,'' said Sheena Greitens, an
associate professor at the University of Texas at Austin.

``It puts this administration's policy into conflict with another one of
its stated goals: to maintain openness and friendly connections with the
Chinese people,'' she added.

While WeChat and its owner have long straddled the
\href{https://www.nytimes.com/2017/09/17/technology/facebook-government-regulations.html}{uncomfortable
divides} that separate China's internet from the world, they have rarely
come under such direct scrutiny from the United States.

Created as the copycat brainchild of a Tencent engineer, Allen Zhang,
WeChat mostly failed to catch on in overseas markets, even as the
company spent hundreds of millions in marketing dollars to compete with
WhatsApp. The app's reliance on other Chinese apps in the
\href{https://www.nytimes.com/2016/08/10/technology/china-homegrown-internet-companies-rest-of-the-world.html}{isolated
Chinese internet ecosystem} probably hurt its chances, even as its
innovations transformed life within China.

Outside China, it has mainly been a tether for the Chinese diaspora to
their homeland.

May Han, a Chinese-born American, moved to the United States with her
family when she was 9. Lonely when she first arrived, Ms. Han was
encouraged by her parents to use another Tencent chat service, QQ, to
keep up with her elementary school friends in China. They also hoped it
would help her remember Chinese.

Eventually she made the jump to WeChat, where she still whiles away her
online days chatting with about 350 friends and relatives, many of them
in China. Now an environmental science major at the University of
California, San Diego, Ms. Han said WeChat had become the cultural glue
that held together much of her Chinese community.

Image

Representatives of Tencent, WeChat's parent company, at the Global
Mobile Internet Conference in Beijing in 2017.~Credit...Mark
Schiefelbein/Associated Press

``If we can't use WeChat, our connections to China will decrease or even
vanish,'' she said. ``Most of us have got used to using WeChat,
especially older generations. Changing an app is not easy for them; it
means changing their lifestyle.''

Some of her friends, she said, had already begun posting links to Line,
a messaging app popular in Japan, in case they were forced to switch. To
Ms. Han, the order seemed un-American.

``Trump is violating our rights to connect with our families and
friends. If WeChat is really banned, the executive order seems rather
unconstitutional --- it violates the First Amendment,'' she said. ``It
may sound exaggerated here, but I do hope WeChat won't be blocked.''

The order could end up restricting a variety of dealings between
Americans and Tencent.

American companies could, for instance, be barred from advertising on
WeChat, cutting them off from a key channel for reaching China's vast
consumer market. Tencent could be prohibited from distributing WeChat
through Apple's and Google's app stores, which could leave users unable
to receive software updates, or unable to use the app entirely.

Apple and Google did not respond to requests for comment.

The White House order could even prevent Tencent from buying American
equipment for the servers from which it operates WeChat. If the company
uses those same servers to run other internet products and services, a
wider swath of its business could be affected, said David Dai, an
analyst in Hong Kong with the investment research firm Sanford C.
Bernstein.

This would be the ``worst-case scenario'' for Tencent, Mr. Dai wrote in
a research note on Friday.

Tencent, which has a market capitalization well above \$600 billion,
said on Friday that it was reviewing the executive order ``to get a full
understanding.'' The company's shares fell almost 6 percent in Friday
trading on the Hong Kong Stock Exchange.

\href{https://newsroom.tiktok.com/en-us/tiktok-responds}{TikTok said} it
was ``shocked'' by the White House order, which it said had been issued
``without any due process.''

At a daily news briefing on Friday, the Chinese Ministry of Foreign
Affairs spokesman Wang Wenbin called the order a ``nakedly hegemonic
act,'' saying that ``on the pretext of national security, the U.S.
frequently abuses national power and unreasonably suppresses relevant
enterprises.''

Tencent's own products may have struggled to break through in Western
countries. But it has built up a wide-ranging, if low-key, presence in
the United States through investments and partnerships --- all of which
could be affected if the White House order results in a broad ban on
working with Tencent.

Some of the company's most significant overseas forays have been in
video games, which account for much of its worldwide revenue. Tencent
owns Riot Games, the developer behind League of Legends, and a large
share of Epic Games, which makes Fortnite. The company's film unit,
Tencent Pictures, has been involved in Hollywood blockbusters including
``Wonder Woman'' and the most recent ``Terminator'' movie.

Image

An ad for WeChat at Hong Kong's international airport in
2017.Credit...Richard A. Brooks/Agence France-Presse --- Getty Images

Tencent has also taken stakes in companies with less direct connections
to its own businesses, including the electric carmaker Tesla and the
social media company Snap. It has even invested in
\href{https://weibo.com/6969784101/J1y7JuKuM?from=page_1006066969784101_profile\&wvr=6\&mod=weibotime}{the
Chinese operations of Tim Hortons}, the Canadian coffee chain, to aid in
the company's expansion in China.

As Tencent's global WeChat expansion foundered, the company tried to buy
WhatsApp but was beaten out by Facebook. If Tencent had succeeded, it
may well have looked more like ByteDance, the other Chinese internet
company in the cross hairs of the Trump administration. ByteDance's
best-known app, TikTok, got a big boost with its
\href{https://www.nytimes.com/2017/11/10/business/dealbook/musically-sold-app-video.html}{takeover
of Musical.ly}, a short-video app built by Chinese entrepreneurs that
had found success in Europe and the United States.

Both companies' workarounds functioned only because Washington did not
follow Beijing's censorship cues. That may now be changing, though Yaqiu
Wang, a researcher with Human Rights Watch, said the Trump
administration's executive orders looked puny compared with Beijing's
Great Firewall. While they raise free speech questions, she said, the
concerns about WeChat's role in democracies are very real.

``For many overseas Chinese, the popularity and multifunctionality of
WeChat has made apps popular outside of China unnecessary,'' she said.

``That means the Chinese government is able to control a significant
portion of the information overseas Chinese receive, even outside its
borders,'' she added. ``This could have real domestic political
implications, as many members of the Chinese diaspora are voters of the
countries they reside in and are, or can be, politically mobilized.''

Lin Qiqing contributed research.

Advertisement

\protect\hyperlink{after-bottom}{Continue reading the main story}

\hypertarget{site-index}{%
\subsection{Site Index}\label{site-index}}

\hypertarget{site-information-navigation}{%
\subsection{Site Information
Navigation}\label{site-information-navigation}}

\begin{itemize}
\tightlist
\item
  \href{https://help.nytimes.com/hc/en-us/articles/115014792127-Copyright-notice}{©~2020~The
  New York Times Company}
\end{itemize}

\begin{itemize}
\tightlist
\item
  \href{https://www.nytco.com/}{NYTCo}
\item
  \href{https://help.nytimes.com/hc/en-us/articles/115015385887-Contact-Us}{Contact
  Us}
\item
  \href{https://www.nytco.com/careers/}{Work with us}
\item
  \href{https://nytmediakit.com/}{Advertise}
\item
  \href{http://www.tbrandstudio.com/}{T Brand Studio}
\item
  \href{https://www.nytimes.com/privacy/cookie-policy\#how-do-i-manage-trackers}{Your
  Ad Choices}
\item
  \href{https://www.nytimes.com/privacy}{Privacy}
\item
  \href{https://help.nytimes.com/hc/en-us/articles/115014893428-Terms-of-service}{Terms
  of Service}
\item
  \href{https://help.nytimes.com/hc/en-us/articles/115014893968-Terms-of-sale}{Terms
  of Sale}
\item
  \href{https://spiderbites.nytimes.com}{Site Map}
\item
  \href{https://help.nytimes.com/hc/en-us}{Help}
\item
  \href{https://www.nytimes.com/subscription?campaignId=37WXW}{Subscriptions}
\end{itemize}
