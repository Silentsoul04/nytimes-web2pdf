Sections

SEARCH

\protect\hyperlink{site-content}{Skip to
content}\protect\hyperlink{site-index}{Skip to site index}

\href{https://myaccount.nytimes.com/auth/login?response_type=cookie\&client_id=vi}{}

\href{https://www.nytimes.com/section/todayspaper}{Today's Paper}

\href{/section/opinion}{Opinion}\textbar{}Beirut on the Potomac

\href{https://nyti.ms/30FkIMm}{https://nyti.ms/30FkIMm}

\begin{itemize}
\item
\item
\item
\item
\item
\item
\end{itemize}

Beirut Explosion

\begin{itemize}
\tightlist
\item
  \href{https://www.nytimes.com/2020/08/05/world/middleeast/beirut-explosion-what-happened.html?action=click\&pgtype=Article\&state=default\&region=TOP_BANNER\&context=storylines_menu}{What
  We Know}
\item
  \href{https://www.nytimes.com/2020/08/05/video/beirut-explosion-footage.html?action=click\&pgtype=Article\&state=default\&region=TOP_BANNER\&context=storylines_menu}{Footage
  of the Blast}
\item
  \href{https://www.nytimes.com/2020/08/05/world/middleeast/beirut-explosion-ammonium-nitrate.html?action=click\&pgtype=Article\&state=default\&region=TOP_BANNER\&context=storylines_menu}{What
  is Ammonium Nitrate?}
\item
  \href{https://www.nytimes.com/interactive/2020/08/04/world/middleeast/beirut-explosion-damage.html?action=click\&pgtype=Article\&state=default\&region=TOP_BANNER\&context=storylines_menu}{Mapping
  the Damage}
\end{itemize}

Advertisement

\protect\hyperlink{after-top}{Continue reading the main story}

\href{/section/opinion}{Opinion}

Supported by

\protect\hyperlink{after-sponsor}{Continue reading the main story}

\hypertarget{beirut-on-the-potomac}{%
\section{Beirut on the Potomac}\label{beirut-on-the-potomac}}

The American spirit gets a Lebanese makeover.

\href{https://www.nytimes.com/by/roger-cohen}{\includegraphics{https://static01.nyt.com/images/2014/11/01/opinion/cohen-circular/cohen-circular-thumbLarge-v6.png}}

By \href{https://www.nytimes.com/by/roger-cohen}{Roger Cohen}

Opinion Columnist

\begin{itemize}
\item
  Aug. 7, 2020
\item
  \begin{itemize}
  \item
  \item
  \item
  \item
  \item
  \item
  \end{itemize}
\end{itemize}

\includegraphics{https://static01.nyt.com/images/2020/08/07/opinion/07cohen1/merlin_175379832_19ae0c1a-a0eb-424e-aa22-2a639048fa25-articleLarge.jpg?quality=75\&auto=webp\&disable=upscale}

MILAN --- A society at the point of detonation suffers from internal
rot. This was true, for example, of the Austro-Hungarian and Ottoman
Empires in 1914. I am interested in two contemporary cases: the spark
that blew up 2,750 tons of ammonium nitrate
\href{https://www.nytimes.com/2020/08/05/world/middleeast/beirut-explosion-ship.html}{stored}
since 2014 in a warehouse at Beirut's port, and the pathogen about
one-thousandth the width of an eyelash that has killed at least 160,500
Americans and infected more than 4.9 million as President Trump has
flailed.

That Lebanon was rotten to the core is scarcely news. Think of all that
unprotected industrial explosive, whose detonation was so devastating,
as the ultimate expression of Lebanon's ``malign neglect,'' in the
phrase of my colleague Robert Worth, a former Times Beirut
correspondent,
\href{https://www.theglobeandmail.com/opinion/article-the-rot-at-the-heart-of-lebanon-has-finally-and-grimly-detonated/}{writing
in} The Globe and Mail.

The protests I witnessed last year in Beirut were an expression of rage
at the country's chronic corruption, sectarian fiefs and endemic waste.
The end of the civil war in 1990 locked in a power-sharing system that
looks to a millennial generation like a license to loot with impunity.
Since the protests the Lebanese pound has collapsed; the country's clan
leaders have clung to power; some people have survived through
bartering. And lo and behold, BOOM!

\includegraphics{https://static01.nyt.com/images/2020/08/07/opinion/07cohenNew/merlin_175299348_4f37fa45-7ea9-4f2c-957b-86d2c77aa635-articleLarge.jpg?quality=75\&auto=webp\&disable=upscale}

The discovery this year of Beirut-on-the-Potomac is more surprising,
even if America's malignant state has been evident for some time. The
virus highlighted and compounded a sickly national condition.

As other developed countries contained the pandemic, the United States
became the pariah nation of dysfunctional government, laughable
leadership, tribal confrontation and anti-scientific claptrap. Its own
sectarian fiefs, evident in the war of masked believers and unmasked
virus deniers, made a coherent response to Covid-19, impossible. The
United States, Trump's ``greatest, most exceptional, and most virtuous
nation in the history of the world,'' detonated into a free-for-all.

I write now from Italy, a nation often held up (somewhat unjustly) as
the land par excellence of governmental dysfunction, where the
coronavirus struck hard in March and
\href{https://www.nytimes.com/2020/07/31/world/europe/italy-coronavirus-reopening.html}{where
it seems to now be under control}. I write as the American stranger,
object of curiosity in this summer sans American tourists. Italy
demonstrates that coherent policy, science and a measure of discipline
can counter the pandemic. They are alien to Trump's America, which
elicits a pained Italian bewilderment.

The United States does not have two armies, as Lebanon does with the
official armed forces and Hezbollah. But like Lebanon, it has fractured.
The national unity evident at 9/11 has evaporated. The nation's
infrastructure is battered, its health system inadequate, its racial
tensions acute. Its leader, Trump, placed his own fortune above the
national interest when he wasted two months downplaying the virus in the
\href{https://www.nytimes.com/2020/04/17/opinion/trump-coronavirus.html}{belief
that a Dow at 30,000} would guarantee his re-election. That is what
Lebanon's clan leaders have always done: put their own financial
interests first.

Image

Protesters marching in Atlanta in June, over the fatal police shooting
of a young Black man four days earlier.Credit...Joe Raedle/Getty Images

In Lebanon the collective interest --- say, in having a functioning
electricity grid --- loses out to individual selfishness. Public money
lines personal pockets.

In the past American individualism, a source of economic vitality, could
be subsumed into collective determination at times of crisis. The virus
proved the exception because American self-reliance has metastasized
into narcissistic self-obsession, in the image of Trump. The American
spirit got a Lebanese makeover.

Democracy in Lebanon is flawed by nepotism and religious division. In
the United States, special interests and the power of the wealthy have
warped representative democracy to the point that it fails in its
essential task. How fair is the representation? How democratic is the
elective process? Ever less so, as things stand. This is part of the
rotting of the body politic.

The United States is not Lebanon, far from it. But it is ripe for
detonation, the more so because that is what Trump seeks.

When I was in Lebanon last October, I headed south toward the
borderlands controlled by Hezbollah, near the Israeli border. Plastic
bottles and bags, ineradicable detritus, skittered in the wind. On the
hard shoulder, cars careened the wrong way up the freeway. My driver
muttered in disgust. In Lebanon everyone goes freelance. They have no
choice. Governance is an exercise in crony deals. Credible tales of
seeping sewage kept the beaches I passed deserted. The further south I
went, into the hills, the more I saw the yellow flags of Hezbollah and
its slogans. ``Hezbollah always victorious!''

In June, after months confined in New York, I drove south toward
Dixieland. I was reminded of American vastness.
\href{https://www.nytimes.com/2020/06/26/opinion/let-freedom-ring-from-georgia.html}{I
crisscrossed rural Georgia} and saw a different flag, the Confederate
flag, here and there; and I drove on a stretch of highway named for
Jefferson Davis, the president of the Confederate States of America; and
I saw Confederate monuments that spoke of states' rights, but never of
slavery, and claimed the lost cause was somehow not lost; and I listened
to Americans whose language and values suggested a culture war so
intense as to shred any shared national lexicon.

Lebanese fracture is not American fracture. My southward journeys were
not really comparable. The United States has powerful institutions. Its
civil war left ``government of the people, by the people, for the
people'' alive. But vigilance is needed if, on Nov. 3, Trump's America
is not to go BOOM.

\emph{The Times is committed to publishing}
\href{https://www.nytimes.com/2019/01/31/opinion/letters/letters-to-editor-new-york-times-women.html}{\emph{a
diversity of letters}} \emph{to the editor. We'd like to hear what you
think about this or any of our articles. Here are some}
\href{https://help.nytimes.com/hc/en-us/articles/115014925288-How-to-submit-a-letter-to-the-editor}{\emph{tips}}\emph{.
And here's our email:}
\href{mailto:letters@nytimes.com}{\emph{letters@nytimes.com}}\emph{.}

\emph{Follow The New York Times Opinion section on}
\href{https://www.facebook.com/nytopinion}{\emph{Facebook}}\emph{,}
\href{http://twitter.com/NYTOpinion}{\emph{Twitter (@NYTopinion)}}
\emph{and}
\href{https://www.instagram.com/nytopinion/}{\emph{Instagram}}\emph{.}

Advertisement

\protect\hyperlink{after-bottom}{Continue reading the main story}

\hypertarget{site-index}{%
\subsection{Site Index}\label{site-index}}

\hypertarget{site-information-navigation}{%
\subsection{Site Information
Navigation}\label{site-information-navigation}}

\begin{itemize}
\tightlist
\item
  \href{https://help.nytimes.com/hc/en-us/articles/115014792127-Copyright-notice}{©~2020~The
  New York Times Company}
\end{itemize}

\begin{itemize}
\tightlist
\item
  \href{https://www.nytco.com/}{NYTCo}
\item
  \href{https://help.nytimes.com/hc/en-us/articles/115015385887-Contact-Us}{Contact
  Us}
\item
  \href{https://www.nytco.com/careers/}{Work with us}
\item
  \href{https://nytmediakit.com/}{Advertise}
\item
  \href{http://www.tbrandstudio.com/}{T Brand Studio}
\item
  \href{https://www.nytimes.com/privacy/cookie-policy\#how-do-i-manage-trackers}{Your
  Ad Choices}
\item
  \href{https://www.nytimes.com/privacy}{Privacy}
\item
  \href{https://help.nytimes.com/hc/en-us/articles/115014893428-Terms-of-service}{Terms
  of Service}
\item
  \href{https://help.nytimes.com/hc/en-us/articles/115014893968-Terms-of-sale}{Terms
  of Sale}
\item
  \href{https://spiderbites.nytimes.com}{Site Map}
\item
  \href{https://help.nytimes.com/hc/en-us}{Help}
\item
  \href{https://www.nytimes.com/subscription?campaignId=37WXW}{Subscriptions}
\end{itemize}
