Sections

SEARCH

\protect\hyperlink{site-content}{Skip to
content}\protect\hyperlink{site-index}{Skip to site index}

\href{https://myaccount.nytimes.com/auth/login?response_type=cookie\&client_id=vi}{}

\href{https://www.nytimes.com/section/todayspaper}{Today's Paper}

\href{/section/opinion}{Opinion}\textbar{}What Sonia Sotomayor Told a
10-Year-Old Girl

\href{https://nyti.ms/2CafGxU}{https://nyti.ms/2CafGxU}

\begin{itemize}
\item
\item
\item
\item
\item
\end{itemize}

Advertisement

\protect\hyperlink{after-top}{Continue reading the main story}

\href{/section/opinion}{Opinion}

Supported by

\protect\hyperlink{after-sponsor}{Continue reading the main story}

\hypertarget{what-sonia-sotomayor-told-a-10-year-old-girl}{%
\section{What Sonia Sotomayor Told a 10-Year-Old
Girl}\label{what-sonia-sotomayor-told-a-10-year-old-girl}}

Reflecting on the words of a Supreme Court justice and women's path to
political equality.

\includegraphics{https://static01.nyt.com/images/2019/11/08/opinion/jorge-ramos/jorege-ramos-thumbLarge.png}

By Jorge Ramos

Mr. Ramos is a contributing Opinion writer and an anchor for the
Univision network.

\begin{itemize}
\item
  Aug. 7, 2020
\item
  \begin{itemize}
  \item
  \item
  \item
  \item
  \item
  \end{itemize}
\end{itemize}

\includegraphics{https://static01.nyt.com/images/2020/08/08/opinion/08Ramos/07Ramos-articleLarge.jpg?quality=75\&auto=webp\&disable=upscale}

\href{https://www.nytimes.com/es/2020/08/07/espanol/opinion/presidenta-estados-unidos.html}{Leer
en español}

MIAMI --- Last February, before social distancing became a reality for
us all, I was able to interview Justice Sonia Sotomayor for my
\href{https://bit.ly/2P3Ylta}{``Contrapoder'' podcast}. Among our small
audience was Sophie McLoud, 10. The girl had a question for Justice
Sotomayor, the first Latina Supreme Court justice in American history.
``Do you think a girl like me could become president of the United
States?'' Sophie asked. I'll share the amazing answer that Justice
Sotomayor gave Sophie later. For now, let's focus on the latest news.

In the next few days, Joe Biden, the former vice president and now
presumptive Democratic presidential nominee, is expected to announce his
running mate. The possibility that a Black woman may fill the
vice-presidential slot on the Democratic ticket for the first time ever
adds to the excitement, especially since that woman could plausibly
become president of the United States.

Hopes for a woman to hold one of the two highest offices of the
executive branch have been long held. I still remember when, in 1984 in
Los Angeles, I interviewed Geraldine Ferraro, the first female
vice-presidential nominee of a major political party. On that occasion
we took a picture together; in it, she held her fist high.

Ms. Ferraro and then-Democratic presidential candidate Walter Mondale
lost that election to Ronald Reagan. But I remember her as a warrior.

Against the background of Hillary Clinton's defeat in the 2016
presidential race, it is hard to understand how one of the richest and
most powerful countries in the world has never elected a woman to the
White House. Other countries in the Western Hemisphere --- Nicaragua,
Panama, Chile, Argentina, Brazil and Costa Rica --- have had women as
presidents. Not the United States.

Although women serve in top government positions, as is the case with
the speaker of the House of Representatives, Nancy Pelosi, they occupy
only 101, or 23 percent, of voting seats in the House. On a global
scale, \href{https://data.ipu.org/women-ranking?month=6\&year=2020}{the
country ranks 83rd} in terms of female representation in national
legislatures, according to the Inter-Parliamentary Union, the
Geneva-based international organization of parliaments.

So what can we do to achieve political equality? ``You need laws and you
need structures that lead the way to gender equality,'' said
\href{https://www.cnn.com/videos/tv/2020/02/07/exp-gps-0209-marin-on-gender-equality-in-usa.cnn}{Prime
Minister Sanna Marin of Finland}, the second-youngest head of government
in the world, in a CNN interview. ``It just doesn't happen by itself.''
In Finland, for example, the
\href{https://thl.fi/en/web/gender-equality/gender-equality-in-finland/decision-making/gender-quotas}{law
requires} that the proportion of men and women serving in certain
governmental, municipal and intermunicipal bodies be equal to at least
40 percent for both groups.

In the United States we don't have such a law, but finally adopting the
Equal Rights Amendment,
\href{https://thewatchdogonline.com/the-equal-rights-amendment-is-almost-there-29291}{introduced
in Congress in 1923}, could go a long way toward solving our problems.
``Equality of rights under the law shall not be denied or abridged by
the United States or by any state on account of sex,'' the proposed
amendment states, in a simple and clear fashion. Could a new Congress in
2021 help remove the legal hurdles that have stood on the way of the
E.R.A.'s ratification for nearly a century?

On that note, let's go back to my interview with Justice Sotomayor,
which took place earlier this year in Miami. She had just published
``Just Ask! Be Different, Be Brave, Be You,'' a children's book on how
differences make people stronger. We talked about the experiences that
inspired her writing, her struggles with diabetes and how to confront
our fears.

\includegraphics{https://static01.nyt.com/images/2020/08/07/opinion/07ramos/merlin_160075995_2a30cfc6-8748-4e59-97fd-fda92063ef9f-articleLarge.jpg?quality=75\&auto=webp\&disable=upscale}

``When I was nominated to the Supreme Court I was really scared,''
Justice Sotomayor told me in Spanish. ``This is a huge job. But who
lives life free of fear? I have often told myself, `I don't want to do
this job.''' I wasn't sure I could get it right. And I was very, very
close to saying no to the president of the United States. But some
friends heard that I was having second thoughts, and one of them told
me: `Hey, Sonia, stop thinking about you. This is not about you. This is
about all those little girls who will see you in that role.'''

Girls like 10-year-old Sophie, who was listening intently. At the end of
the interview, as the adults looked on, she approached Justice Sotomayor
to ask if she, a Latina, could one day be president of the United
States. Justice Sotomayor hugged her and replied, ``Yes, yes.'' She then
went on to give the child a true life lesson.

``First of all, a girl like you should always dream big,'' Justice
Sotomayor told Sophie.

``Second, never let anyone say that you can't do it. And the minute they
say that, you should do as I have done myself and say: `You are telling
me I can't do it? Well, I'll show you I can.'

``Third, you have to study, study and study. That's the only way you can
achieve what you want in life. Education is the key to the future.

``And fourth, you have to work very hard. In life no one will give you
anything for free. You must earn every single thing in this life. It is
by studying and working hard that you will become president of the
United States.''

Before saying goodbye, Justice Sotomayor hugged Sophie once again. ``I
hope to be alive when you become president,'' the justice said, before
expressing her wish to be the one to administer the oath of office to
her.

I hope to be there for that occasion. But for that to happen, good
intentions and hard work won't be enough. I get why the idea of quotas
isn't very popular in the United States, a country that takes pride in
presenting itself as a meritocracy. But the reality is that if we don't
set gender quotas the way Finland did, putting an end to prejudice and
current inequalities will be hard. We need a sense of urgency and new
rules that reflect our outrage.

Latina women face a double burden. That's why when a Latina like Justice
Sotomayor reaches one of the most important institutional positions in
the country, when young dreamers achieve changes in the laws and when
there is another new Hispanic senator or governor, they open the way for
those who come after them.

Sophie may someday be the first Latina president of the United States; I
don't doubt it. But before that happens, many other girls like her will
need to pave the way. And like Ms. Marin said, ``It just doesn't happen
by itself.''

\emph{The Times is committed to publishing}
\href{https://www.nytimes.com/2019/01/31/opinion/letters/letters-to-editor-new-york-times-women.html}{\emph{a
diversity of letters}} \emph{to the editor. We'd like to hear what you
think about this or any of our articles. Here are some}
\href{https://help.nytimes.com/hc/en-us/articles/115014925288-How-to-submit-a-letter-to-the-editor}{\emph{tips}}\emph{.
And here's our email:}
\href{mailto:letters@nytimes.com}{\emph{letters@nytimes.com}}\emph{.}

\emph{Follow The New York Times Opinion section on}
\href{https://www.facebook.com/nytopinion}{\emph{Facebook}}\emph{,}
\href{http://twitter.com/NYTOpinion}{\emph{Twitter (@NYTopinion)}}
\emph{and}
\href{https://www.instagram.com/nytopinion/}{\emph{Instagram}}\emph{.}

Advertisement

\protect\hyperlink{after-bottom}{Continue reading the main story}

\hypertarget{site-index}{%
\subsection{Site Index}\label{site-index}}

\hypertarget{site-information-navigation}{%
\subsection{Site Information
Navigation}\label{site-information-navigation}}

\begin{itemize}
\tightlist
\item
  \href{https://help.nytimes.com/hc/en-us/articles/115014792127-Copyright-notice}{©~2020~The
  New York Times Company}
\end{itemize}

\begin{itemize}
\tightlist
\item
  \href{https://www.nytco.com/}{NYTCo}
\item
  \href{https://help.nytimes.com/hc/en-us/articles/115015385887-Contact-Us}{Contact
  Us}
\item
  \href{https://www.nytco.com/careers/}{Work with us}
\item
  \href{https://nytmediakit.com/}{Advertise}
\item
  \href{http://www.tbrandstudio.com/}{T Brand Studio}
\item
  \href{https://www.nytimes.com/privacy/cookie-policy\#how-do-i-manage-trackers}{Your
  Ad Choices}
\item
  \href{https://www.nytimes.com/privacy}{Privacy}
\item
  \href{https://help.nytimes.com/hc/en-us/articles/115014893428-Terms-of-service}{Terms
  of Service}
\item
  \href{https://help.nytimes.com/hc/en-us/articles/115014893968-Terms-of-sale}{Terms
  of Sale}
\item
  \href{https://spiderbites.nytimes.com}{Site Map}
\item
  \href{https://help.nytimes.com/hc/en-us}{Help}
\item
  \href{https://www.nytimes.com/subscription?campaignId=37WXW}{Subscriptions}
\end{itemize}
