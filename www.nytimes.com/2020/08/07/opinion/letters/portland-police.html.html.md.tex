Sections

SEARCH

\protect\hyperlink{site-content}{Skip to
content}\protect\hyperlink{site-index}{Skip to site index}

\href{https://myaccount.nytimes.com/auth/login?response_type=cookie\&client_id=vi}{}

\href{https://www.nytimes.com/section/todayspaper}{Today's Paper}

\href{/section/opinion}{Opinion}\textbar{}Police Violence in Portland
Protests

\href{https://nyti.ms/33BeJK7}{https://nyti.ms/33BeJK7}

\begin{itemize}
\item
\item
\item
\item
\item
\end{itemize}

Advertisement

\protect\hyperlink{after-top}{Continue reading the main story}

\href{/section/opinion}{Opinion}

Supported by

\protect\hyperlink{after-sponsor}{Continue reading the main story}

letters

\hypertarget{police-violence-in-portland-protests}{%
\section{Police Violence in Portland
Protests}\label{police-violence-in-portland-protests}}

The A.C.L.U. of Oregon says the police ``are making a mockery of the
First Amendment.'' Also: Planned Parenthood and Black women; a free pass
for Russia; universal internet; weekends in a pandemic.

Aug. 7, 2020

\begin{itemize}
\item
\item
\item
\item
\item
\end{itemize}

\hypertarget{more-from-our-inbox}{%
\subsubsection{More from our inbox:}\label{more-from-our-inbox}}

\begin{itemize}
\tightlist
\item
  \protect\hyperlink{link-eafaf65}{Black Women's Reproductive Rights}
\item
  \protect\hyperlink{link-3084cc09}{Betraying U.S. Forces: A Republican
  Habit}
\item
  \protect\hyperlink{link-2254537e}{Internet for All}
\item
  \protect\hyperlink{link-510bdec2}{Ah, the Weekend \ldots?}
\end{itemize}

\includegraphics{https://static01.nyt.com/images/2020/08/03/opinion/03Lovell2/03Lovell2-articleLarge.jpg?quality=75\&auto=webp\&disable=upscale}

\textbf{To the Editor:}

``\href{https://www.nytimes.com/2020/08/03/opinion/portland-protests-police-chief.html}{Violence
Is Not an Answer,''} by Chuck Lovell (Op-Ed, Aug. 4), is an insult to
the thousands of Portlanders who have been indiscriminately gassed,
beaten and shot with crowd-control weapons by the Portland, Ore., police
night after night, simply for protesting racist police violence.

Mr. Lovell, the chief of the Police Bureau in Portland, claims that ``as
police officers, our duty is to uphold the rights of anyone to assemble
peacefully and engage in free speech.'' But the police in Portland are
making a mockery of the First Amendment by using excessive force,
violence and intimidation to suppress free speech in the Black Lives
Matter movement.

The A.C.L.U. of Oregon had to
\href{https://www.aclu.org/press-releases/aclu-sues-federal-agents-portland/}{sue}
federal agents and police in Portland for violently attacking
journalists and legal observers, as well as
\href{https://www.aclu.org/press-releases/aclu-sues-feds-portland-police-attacking-medics-protests}{medics}
tending to the very community members whom the police harmed.

Chief Lovell is right that violence is not the answer. That includes
police violence. If the police don't want us to take their resources
(``defund the police''), then the Portland police and Mayor Ted Wheeler
need to use our community's resources to heed their own advice, take
accountability for their abuses, respect the Constitution and ensure
that Black Lives Matter in Portland.

Kelly Simon\\
Portland, Ore.\\
\emph{The writer is interim legal director of the A.C.L.U. of Oregon.}

\hypertarget{black-womens-reproductive-rights}{%
\subsection{Black Women's Reproductive
Rights}\label{black-womens-reproductive-rights}}

Image

Margaret Sanger, who founded the American Birth Control League in 1921,
speaks before a Senate committee to advocate for federal birth-control
legislation in Washington in 1934.Credit...Associated Press

\textbf{To the Editor:}

Re
``\href{https://www.nytimes.com/2020/07/25/opinion/sunday/abortion-racism-margaret-sanger.html}{The
Ghost of Margaret Sanger}'' (column, July 26):

Planned Parenthood has long denounced Margaret Sanger's eugenicist
beliefs, recognizing the need to engage in anti-racist work as a
104-year-old institution. Covid-19 leaves no confusion about the effects
of systemic racism. Yet Ross Douthat conflates the disparate impact of
public policy on Black communities with the fundamental right of Black
women to control our own bodies. Birthrates do not equal power, unless
you're a white supremacist.

Black women know reproductive control began at the auction block, when
our ancestors' forced reproduction was the engine that drove the
American economy. Whether we're attacked for having children and needing
support, or for having an abortion, we're damned if we do and damned if
we don't.

Mr. Douthat insinuates that his argument may produce ``intersectional
dilemmas no doctrine can resolve.'' Our experiences are intersectional,
but there's no dilemma: Our bodies are our own, and we won't apologize
for it.

Alexis McGill Johnson\\
New York\\
\emph{The writer is president and chief executive of the Planned
Parenthood Federation of America.}

\hypertarget{betraying-us-forces-a-republican-habit}{%
\subsection{Betraying U.S. Forces: A Republican
Habit}\label{betraying-us-forces-a-republican-habit}}

Image

~~Credit...Jorge Silva/Reuters

\textbf{To the Editor:}

Re
``\href{https://www.nytimes.com/2020/07/29/us/politics/trump-putin-bounties.html}{Trump
Says He and Putin Didn't Talk About Bounties}'' (news article, July 30):

President Trump has betrayed American forces by giving Vladimir Putin a
free pass on his reported bounty payments to Taliban-linked militants
for American lives. Mr. Trump's acquiescence, if motivated by his
personal goals, such as Russian electoral interference on his behalf, is
treasonous, but not unique. Leaders of the modern Republican Party have
repeatedly sought electoral advantage by disregarding the lives of
American soldiers and government officials.

In 1968 Richard Nixon's team
\href{https://www.nytimes.com/2017/01/02/us/politics/nixon-tried-to-spoil-johnsons-vietnam-peace-talks-in-68-notes-show.html}{sabotaged
talks} that might have ended the Vietnam War, because the Democrats
would have gotten credit and might have cost Nixon a close election.
Similarly, Ronald Reagan's campaign and leading Republicans
\href{https://www.nytimes.com/2019/12/29/world/middleeast/shah-iran-chase-papers.html}{worked
to thwart a deal} for the release of American hostages in Iran in order
to avoid an ``October surprise'' that would have helped President Jimmy
Carter's re-election bid.

Why have G.O.P. leaders shown themselves willing to betray American
troops and public servants for partisan gain? The answer seems to lie in
a worldview that rejects the principle of a loyal opposition and, quite
conceivably, democracy itself. Can they be held to account?

Daniel Lieberfeld\\
Pittsburgh\\
\emph{The writer is a retired professor of history and politics at
Duquesne University.}

\hypertarget{internet-for-all}{%
\subsection{Internet for All}\label{internet-for-all}}

Image

~~Credit...Ruth Fremson/The New York Times

\textbf{To the Editor:}

Re
``\href{https://www.nytimes.com/2020/07/18/opinion/sunday/broadband-internet-access-civil-rights.html?searchResultPosition=2}{The
Limits of Broadband}'' (editorial, July 19), about how many ``Americans
sheltering from Covid-19 are discovering the limitations of the
country's cobbled-together broadband service'':

Your editorial correctly declared that high-speed internet connections
are a ``civil rights issue'' and that service is ``often unavailable or
too expensive in rural communities and low-income neighborhoods.''

As the president of Midtel, an upstate New York telecommunications
company that serves underserved rural areas, I can say from experience
that smart government policies are a key to bridging the digital divide.

New York's current policies send mixed messages, discouraging the
infrastructure investments necessary to make full connectivity a reality
today and into the future.

My company has received more than \$15.5 million in state grants to
replace our copper network with fiber, enabling us to bring customers
fast, reliable and affordable high-speed service. But the state is
taxing fiber in the Department of Transportation right-of-way, a space
traditional utilities get to use free. This added tax, which we legally
cannot pass on to our customers, makes already expensive projects
cost-prohibitive.

If New York is indeed serious about closing the digital divide and
enabling all New Yorkers to prosper in the new normal, it must enable
the industry to make the goal of broadband for all a reality while
paving the way for next-generation connectivity.

Jim Becker\\
Middleburgh, N.Y.

\hypertarget{ah-the-weekend-}{%
\subsection{Ah, the Weekend \ldots?}\label{ah-the-weekend-}}

Image

Danielle Brooks as Beatrice and Grantham Coleman as Benedick in last
year's Shakespeare in the Park production of ``Much Ado About Nothing,''
which is streaming on PBS's website.Credit...Sara Krulwich/The New York
Times

\textbf{To the Editor:}

I appreciate your Aug. 7 Weekend Arts article
``\href{https://www.nytimes.com/2020/08/06/arts/things-to-do-weekend-coronavirus.html?searchResultPosition=1}{6
Things to Do This Weekend}.''

I do have a question, though: Remind me what a weekend is?

Marc Chafetz\\
Washington

Advertisement

\protect\hyperlink{after-bottom}{Continue reading the main story}

\hypertarget{site-index}{%
\subsection{Site Index}\label{site-index}}

\hypertarget{site-information-navigation}{%
\subsection{Site Information
Navigation}\label{site-information-navigation}}

\begin{itemize}
\tightlist
\item
  \href{https://help.nytimes.com/hc/en-us/articles/115014792127-Copyright-notice}{©~2020~The
  New York Times Company}
\end{itemize}

\begin{itemize}
\tightlist
\item
  \href{https://www.nytco.com/}{NYTCo}
\item
  \href{https://help.nytimes.com/hc/en-us/articles/115015385887-Contact-Us}{Contact
  Us}
\item
  \href{https://www.nytco.com/careers/}{Work with us}
\item
  \href{https://nytmediakit.com/}{Advertise}
\item
  \href{http://www.tbrandstudio.com/}{T Brand Studio}
\item
  \href{https://www.nytimes.com/privacy/cookie-policy\#how-do-i-manage-trackers}{Your
  Ad Choices}
\item
  \href{https://www.nytimes.com/privacy}{Privacy}
\item
  \href{https://help.nytimes.com/hc/en-us/articles/115014893428-Terms-of-service}{Terms
  of Service}
\item
  \href{https://help.nytimes.com/hc/en-us/articles/115014893968-Terms-of-sale}{Terms
  of Sale}
\item
  \href{https://spiderbites.nytimes.com}{Site Map}
\item
  \href{https://help.nytimes.com/hc/en-us}{Help}
\item
  \href{https://www.nytimes.com/subscription?campaignId=37WXW}{Subscriptions}
\end{itemize}
