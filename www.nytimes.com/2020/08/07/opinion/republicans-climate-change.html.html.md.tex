Sections

SEARCH

\protect\hyperlink{site-content}{Skip to
content}\protect\hyperlink{site-index}{Skip to site index}

\href{https://myaccount.nytimes.com/auth/login?response_type=cookie\&client_id=vi}{}

\href{https://www.nytimes.com/section/todayspaper}{Today's Paper}

\href{/section/opinion}{Opinion}\textbar{}I'm a Conservative Christian
Environmentalist. No, That's Not an Oxymoron.

\href{https://nyti.ms/3a2AcwO}{https://nyti.ms/3a2AcwO}

\begin{itemize}
\item
\item
\item
\item
\item
\end{itemize}

Advertisement

\protect\hyperlink{after-top}{Continue reading the main story}

\href{/section/opinion}{Opinion}

Supported by

\protect\hyperlink{after-sponsor}{Continue reading the main story}

\hypertarget{im-a-conservative-christian-environmentalist-no-thats-not-an-oxymoron}{%
\section{I'm a Conservative Christian Environmentalist. No, That's Not
an
Oxymoron.}\label{im-a-conservative-christian-environmentalist-no-thats-not-an-oxymoron}}

The G.O.P. may have had a politically expedient change of heart. Better
late than never.

By Ericka Andersen

Ms. Andersen worked in communications for National Review, the Heritage
Foundation and, from 2009-11, the Republican House Conference.

\begin{itemize}
\item
  Aug. 7, 2020
\item
  \begin{itemize}
  \item
  \item
  \item
  \item
  \item
  \end{itemize}
\end{itemize}

\includegraphics{https://static01.nyt.com/images/2020/08/08/opinion/07Anderson/07Anderson-articleLarge.jpg?quality=75\&auto=webp\&disable=upscale}

It's been a long time coming, but some Republicans seem to have finally
gotten serious about climate change. In June, a handful of
\href{https://www.politico.com/newsletters/the-long-game/2020/08/04/betting-the-farm-489964}{senators
rolled out} a bipartisan climate change bill. It is co-sponsored by
Debbie Stabenow, Democrat of Michigan, and Lindsey Graham, Republican of
South Carolina.

The bill, the Growing Climate Solutions Act, makes it easier to pay
farmers to capture carbon. It is the latest in a series of actions
Republicans have taken in the past year to combat climate change.

In March, Kevin McCarthy, the House minority leader,
\href{https://www.politico.com/news/2020/02/13/gop-climate-change-kevin-mccarthy-115025}{unveiled
the first} in a series of three original proposals to help slow the
earth's warming. The bills aim to help cut emissions by expanding a tax
credit for carbon-capture technology and draw on federal funds for
research and development.

With a growing majority of Americans concerned about the effects of
climate change ---
\href{https://www.pewresearch.org/fact-tank/2020/04/21/how-americans-see-climate-change-and-the-environment-in-7-charts/}{67
percent} say the government isn't doing enough to combat it ---
Republicans may have had a politically expedient change of heart. Better
late than never. The latest legislation offers the parties a common
ground where meaningful change can flourish.

As a conservative Christian environmentalist, I've witnessed how the
Republican base of Christian voters has helped push its leaders in this
direction. The faith-based world is an overlooked source of activism for
climate policy. When it comes to theology, a growing number of them are
taking the Bible's Genesis call to care for Creation very seriously, and
younger Christians increasingly seek policies that speak to this.
Republicans have cultivated options that don't negate the conservative
values they hold dear.

Mr. McCarthy's approach bypasses government mandates and regulations.
Instead, it focuses on clean energy, carbon capture and conservation.
Conservatives have historically opposed expensive, large-scale federal
policy, but these innovative solutions provide tangible steps without
sacrificing conservative principles. This is the Republican Party's
bread and butter: creative concepts that don't require significant
mandates or regulations to meet societal needs.

Other Republicans have followed in his footsteps. This summer,
\href{https://www.braun.senate.gov/energy-202-two-gop-senators-join-democrats-back-bill-help-cut-emissions-farms}{Senator
Mike Braun} of Indiana joined Mr. Graham on the Growing Climate
Solutions Act, and
\href{https://www.murkowski.senate.gov/press/release/senators-introduce-bill-to-combat-impacts-of-climate-change-}{Senator
Lisa Murkoswki} of Alaska offered a bipartisan bill (with Sheldon
Whitehouse, Democrat of Rhode Island) addressing the impact of oceans in
capturing carbon. These proposals come
\href{https://www.washingtonpost.com/politics/2020/07/28/energy-202-hunting-fishing-groups-urge-congress-work-together-climate-change/}{as
hunting and fishing groups rally Congress} to pursue bipartisan efforts
that can pass the House and the Senate.

There is also an opportunity for new partnerships, both with younger
Republicans and Christian communities engaged in the climate fight.
Because about 80 percent of Republicans
\href{https://www.pewforum.org/religious-landscape-study/compare/christians/by/party-affiliation/}{identify}
as Christian, political focus on climate policy will draw new interest
from this characteristically passionate, activated group.

``By focusing on mobilizing Christians on this issue, other Christians
will begin to see people like them engaging, and begin to recognize
themselves in that movement,'' said Kyle Meyaard-Schaap, a
representative for Young Evangelicals for Climate Action, in a phone
interview.

Most churchgoing Christians view scripture as holy. Therefore, earthcare
becomes a sacred act of worship. For the younger generation,
environmental responsibility and combating climate change is both
personal and spiritual.

``We, as Christians, have a responsibility to steward the earth we've
been given, and we can't do that without practical solutions,'' Bethany
Bowra, a conservative Christian in her 20s, wrote in an email. ``God
gave us a beautiful world that reflects Him at every turn, and my faith
plays a role in the way I view our responsibility to engage on
environmental issues.''

Young Evangelicals for Climate Action is just one of a growing number of
faith-based organizations focused on the environment.
\href{https://www.interfaithpowerandlight.org/}{Interfaith Power and
Light} exists to
\href{https://www.interfaithpower.org/about-us/national-interfaith-power-light-campaign/}{mobilize}
``a religious response to global warming,'' and the
\href{https://creationcare.org/}{Evangelical Environmental Network} aims
to ``to be
\href{https://creationcare.org/who-we-are/mission.html}{faithful
stewards} of God's provision.''

Later this year, a Creation Care Prayer Breakfast, hosted by a group of
evangelical environmental organizations, is scheduled to take place in
Washington and feature a keynote address from Walter Kim, the president
of the National Association of Evangelicals.

The Yale Program on Climate Change Communication
\href{https://climatecommunication.yale.edu/publications/engaging-christians-in-the-issue-of-climate-change/}{concluded
that Christians cite} ``protecting God's creation'' as their primary
concern. They also found that presenting environmental concerns in a
biblical, ``stewardship frame'' revealed ``significant increases in
pro-environmental and climate change beliefs.''

Not everyone is welcoming of the conservative plans for the environment.
A writer at The New Republic
\href{https://newrepublic.com/article/156269/republicans-climate-change-plan-big-oils-climate-change-plan}{called}
Mr. McCarthy's approach ``a package only a fossil fuel executive could
love.'' The Sierra Club balked at a proposal from the Trump
administration related to logging, denouncing it as
\href{https://www.sierraclub.org/sierra/donald-trump-s-greenwashing-climate-crisis}{``cynical
exploitation''} and ``greenwashing.''

A purity standard on climate action may lead only to more gridlock.
Progressive climate activists might consider the upside of these new
Republican policies: They give environmentalists an ``in'' with
churchgoers, who are a very powerfully activated demographic. And it's
something Joe Biden and his Democratic colleagues could work with if
they take the presidency in November.

It might be difficult for progressives to believe in the environmental
transformation of Republicans or the religious. Indeed, conservatives
have generally shunned taking action on climate change. But that is
changing. In 2019,
\href{https://www.mcclatchydc.com/news/politics-government/congress/article232467697.html}{Senator
Graham said Republicans needed to ``up our game}'' on climate change,
and the party didn't wait long to move on that.

If polling on what Americans care about is any measure, they won't be
letting up the fight for conservative climate change policy anytime
soon. It demonstrates that a vocal group of concerned citizens really
can lead their political leaders on the issues they care about.

Democrats have led the way on environmental policy issues before this,
but it's time for a longer table. Friends from the other side of the
aisle are asking to join.

Ericka Andersen
(\href{https://twitter.com/ErickaAndersen?ref_src=twsrc\%5Egoogle\%7Ctwcamp\%5Eserp\%7Ctwgr\%5Eauthor}{@ErickaAndersen}),
a freelance writer in Indianapolis, worked in communications for
National Review, the Heritage Foundation and, from 2009-11, the
Republican House Conference.

\emph{The Times is committed to publishing}
\href{https://www.nytimes.com/2019/01/31/opinion/letters/letters-to-editor-new-york-times-women.html}{\emph{a
diversity of letters}} \emph{to the editor. We'd like to hear what you
think about this or any of our articles. Here are some}
\href{https://help.nytimes.com/hc/en-us/articles/115014925288-How-to-submit-a-letter-to-the-editor}{\emph{tips}}\emph{.
And here's our email:}
\href{mailto:letters@nytimes.com}{\emph{letters@nytimes.com}}\emph{.}

\emph{Follow The New York Times Opinion section on}
\href{https://www.facebook.com/nytopinion}{\emph{Facebook}}\emph{,}
\href{http://twitter.com/NYTOpinion}{\emph{Twitter (@NYTopinion)}}
\emph{and}
\href{https://www.instagram.com/nytopinion/}{\emph{Instagram}}\emph{.}

Advertisement

\protect\hyperlink{after-bottom}{Continue reading the main story}

\hypertarget{site-index}{%
\subsection{Site Index}\label{site-index}}

\hypertarget{site-information-navigation}{%
\subsection{Site Information
Navigation}\label{site-information-navigation}}

\begin{itemize}
\tightlist
\item
  \href{https://help.nytimes.com/hc/en-us/articles/115014792127-Copyright-notice}{©~2020~The
  New York Times Company}
\end{itemize}

\begin{itemize}
\tightlist
\item
  \href{https://www.nytco.com/}{NYTCo}
\item
  \href{https://help.nytimes.com/hc/en-us/articles/115015385887-Contact-Us}{Contact
  Us}
\item
  \href{https://www.nytco.com/careers/}{Work with us}
\item
  \href{https://nytmediakit.com/}{Advertise}
\item
  \href{http://www.tbrandstudio.com/}{T Brand Studio}
\item
  \href{https://www.nytimes.com/privacy/cookie-policy\#how-do-i-manage-trackers}{Your
  Ad Choices}
\item
  \href{https://www.nytimes.com/privacy}{Privacy}
\item
  \href{https://help.nytimes.com/hc/en-us/articles/115014893428-Terms-of-service}{Terms
  of Service}
\item
  \href{https://help.nytimes.com/hc/en-us/articles/115014893968-Terms-of-sale}{Terms
  of Sale}
\item
  \href{https://spiderbites.nytimes.com}{Site Map}
\item
  \href{https://help.nytimes.com/hc/en-us}{Help}
\item
  \href{https://www.nytimes.com/subscription?campaignId=37WXW}{Subscriptions}
\end{itemize}
