\href{/section/technology}{Technology}\textbar{}At Talkspace, Start-Up
Culture Collides With Mental Health Concerns

\href{https://nyti.ms/31voXsN}{https://nyti.ms/31voXsN}

\begin{itemize}
\item
\item
\item
\item
\item
\end{itemize}

\includegraphics{https://static01.nyt.com/images/2020/08/09/business/07Talkspace-illo/07Talkspace-illo-articleLarge.jpg?quality=75\&auto=webp\&disable=upscale}

Sections

\protect\hyperlink{site-content}{Skip to
content}\protect\hyperlink{site-index}{Skip to site index}

\hypertarget{at-talkspace-start-up-culture-collides-with-mental-health-concerns}{%
\section{At Talkspace, Start-Up Culture Collides With Mental Health
Concerns}\label{at-talkspace-start-up-culture-collides-with-mental-health-concerns}}

The therapy-by-text company made burner phones available for fake
reviews and doesn't adequately respect client privacy, former employees
say.

Credit...Scott Anderson

Supported by

\protect\hyperlink{after-sponsor}{Continue reading the main story}

By \href{https://www.nytimes.com/by/kashmir-hill}{Kashmir Hill} and
Aaron Krolik

\begin{itemize}
\item
  Aug. 7, 2020
\item
  \begin{itemize}
  \item
  \item
  \item
  \item
  \item
  \end{itemize}
\end{itemize}

In 2016, Ricardo Lori was an avid user of Talkspace --- an app that lets
people text and chat with a licensed therapist throughout the day. A
part-time actor in New York City, Mr. Lori struggled with depression and
anxiety, and he credited the app with helping him get out of an abusive
relationship. He was a believer in Talkspace's stated mission to make
``\href{https://help.talkspace.com/hc/en-us?_ga=2.27016189.171121636.1593548307-1918767659.1593034870}{therapy
available and affordable for all},'' and when the start-up offered him a
job in its customer support department, Mr. Lori was ecstatic.

Talkspace, which has raised more than \$100 million from investors, had
an office in the old
\href{https://www.nytimes.com/2018/10/04/movies/studio-54-review.html}{Studio
54} building in Midtown Manhattan, with all the usual perks --- a
Ping-Pong table in the conference room and beer and wine in the company
fridge, plus all the therapy employees wanted. ``I felt like I was at
the best place in the world,'' Mr. Lori said.

After he
\href{https://www.talkspace.com/blog/ricardo-used-talkspace-overcome-depression-anxiety-abuse/}{wrote}
a general account of his therapy sessions on the company blog, an
executive named Linda Sacco came to Mr. Lori with an intimate request.
She wanted to give employees a sense of a typical user's experience.
Could she and one of the company's co-founders, Roni Frank, read through
two weeks of his therapy chat logs and then share excerpts with the
staff?

Mr. Lori thought about his sessions, which included deeply personal
information about his sex life and insecurities. Ms. Sacco assured Mr.
Lori that they would keep him anonymous. ``If I wasn't such a true
believer, I probably would have said, `Are you nuts?' But I was so
enamored of the place,'' said Mr. Lori. He agreed.

\includegraphics{https://static01.nyt.com/images/2020/08/09/business/00TalkSpace-lori/00TalkSpace-lori-articleLarge.jpg?quality=75\&auto=webp\&disable=upscale}

At an all-hands meeting on a Friday afternoon in December 2016,
employees gathered in a 13th-floor conference room. The Ping-Pong table
was folded up so that Ms. Sacco and Ms. Frank could sit on the floor,
cross-legged and back-to-back, for a dramatic reading. Ms. Sacco played
the role of the therapist; Ms. Frank played a female version of Mr.
Lori.

As Mr. Lori drank a tall glass of red wine and watched, he noticed that
a few employees kept glancing his way. Afterward, a member of the
marketing department approached and asked if he was OK. Later, Oren
Frank, Ms. Frank's husband and the chief executive, thanked him in the
elevator. Somehow, word had gotten around that Mr. Lori was the client
in the re-enactment.

Mr. Lori began to reconsider whether Talkspace was the dream employer
he'd imagined --- and whether it could be trusted to protect the privacy
of its users.

``Everything was done with employee-informed consent,'' said Ms. Sacco,
who no longer works at Talkspace. John Reilly, a lawyer for Talkspace,
said, ``At the time, the employee expressed great pride over their
Talkspace treatment with their therapist, and willingly told multiple
co-workers that the transcript was theirs.'' Mr. Lori said he did so
only after it became clear that his identity was widely known.

Image

Roni Frank, left, a co-founder of Talkspace, and another executive read
anonymized excerpts from Mr. Lori's therapy sessions during a company
meeting. Somehow, word got around that Mr. Lori was the client in the
re-enactment.Credit...Ricardo Lori

Despite the embarrassing episode, Mr. Lori stayed with the company for
two more years, until he was let go in 2018. He sued Talkspace for
discrimination and wrongful termination, claiming he was told that his
anxiety and depression were interfering with his work. The lawsuit
settled at the beginning of 2020. Mr. Lori asked the company to take
down his blog post; the company didn't, which is part of why Mr. Lori
decided to share his story with a reporter.

Mr. Lori and other former Talkspace employees, who asked not to be named
for fear of being sued, describe a company with an admirable ambition to
destigmatize therapy --- but that they say has questionable marketing
practices and regards treatment transcripts as another data resource to
be mined. Their accounts suggest that the needs of a venture
capital-backed start-up to grow quickly can sometimes be in conflict
with the core values of professional therapy, including strict
confidentiality and patient welfare.

This year, with a pandemic, a recession and an election shredding
Americans' nerves, those concerns are relevant to more people than ever
before: In May, Talkspace told The Washington Post that its client base
had
\href{https://www.washingtonpost.com/health/2020/05/04/mental-health-coronavirus/}{jumped
65 percent} since mid-February.

``The app-ification of mental health care has real problems,'' said
Hannah Zeavin, a lecturer in the English department at the University of
California, Berkeley whose book about teletherapy is scheduled to be
published next year by MIT Press. ``These are corporate platforms first.
And they offer therapy second.''

``Talkspace has democratized access to therapy and psychiatry by meeting
patients where they are in their lives and making treatment more
affordable,'' said Neil Leibowitz, Talkspace's chief medical officer.
``The need is profound, especially now in this time of unease, and we
are so proud of what our team of therapists is achieving.''

\hypertarget{burner-phones}{%
\subsection{Burner phones}\label{burner-phones}}

Image

Talkspace debuted in 2014 to positive press but lukewarm customer
reviews.Credit...Ricardo Lori

Signing up with Talkspace is quick. Users create an account, fill out a
questionnaire, and get a choice of therapists, who work for the platform
as
\href{https://www.theverge.com/2016/12/19/14004442/talkspace-therapy-app-reviews-patient-safety-privacy-liability-online}{independent
contractors}. Those who sign up for the ``Unlimited Messaging Therapy
Plus'' plan, at \$260 a month, can send a therapist messages at any time
and are promised daily responses. Higher-priced subscription tiers offer
``live sessions'' of 30 minutes. While users can send messages by text,
audio and video, Talkspace is known popularly as a platform for texting.

The company was founded in 2011 by Oren and Roni Frank, an Israeli
couple who felt inspired after their relationship was
``\href{https://blogs.wsj.com/venturecapital/2014/05/12/talkspace-raises-2-5-million-to-deliver-affordable-therapy-online/}{saved}''
by marriage counseling. Mr. Frank had a background in marketing, and Ms.
Frank was a software developer.

Ms. Frank is the company's head of clinical services; as of Aug. 6, her
LinkedIn page said she had a master's degree in psychoanalysis and
psychotherapy from the New York Graduate School of Psychoanalysis, but
she never completed the program. The degree claim was deleted after an
inquiry from The Times. Mr. Reilly said Ms. Frank ``studied for an M.A.
but left her program before completion to launch Talkspace. Her LinkedIn
profile was created while she was studying, the inadvertent error was
corrected as soon as the NYT brought this to our attention.''

The app launched in 2014 to
\href{https://www.theverge.com/2014/6/5/5765732/talkspace-smartphone-therapy-apps}{positive
press} but lukewarm customer reviews, with ratings of about three stars
out of five on both the Google and Apple app stores, according to a
Times analysis. Users complained about glitchy software and unresponsive
therapists.

In 2015 and 2016, according to four former employees, the company sought
to improve its ratings: It asked workers to write positive reviews. One
employee said that Talkspace's head of marketing at the time asked him
to compile 100 fake reviews in a Google spreadsheet, so that employees
could submit them to app stores.

Mr. Lori said that Talkspace gave employees ``burner'' phones to help
evade the app stores' techniques for detecting false reviews. ``They
said, `Don't do it here. Do it at home. Give us five-star ratings
because we have too many bad reviews,''' Mr. Lori said.

Mr. Reilly, the Talkspace lawyer, disputed this account, saying that
employees were free to write reviews any way they liked. ``We alerted
employees if they were to leave a review, to do it from their personal
phones --- not from the Talkspace office network, as that would cause
issues with the app store,'' Mr. Reilly said in an emailed statement.
``To be clear: We have never used fake identities or encouraged anybody
to do so; there is no event involving `burner' phones, and the idea in
and of itself is nonsensical relative to the large number of reviews
outstanding.''

Image

Mr. Lori says Talkspace provided a ``burner'' iPhone with an Apple App
Store login for the purpose of leaving a positive review of the
app.Credit...Ricardo Lori~

Mr. Lori still has the iPhone 4 that Talkspace gave him. On the back,
there is a white sticker on which someone has written ``\#7 App Store
login,'' along with a Yahoo email address and password. Two other former
employees said burner phones were made available to workers.

``Fake reviews are deceptive to consumers,'' said Eric Goldman,
co-director of the High Tech Law Institute at Santa Clara University. If
the Talkspace employees didn't disclose their role when leaving reviews,
``then the company-encouraged reviews are problematic on multiple legal
fronts,'' Mr. Goldman said.

Posting fake online reviews is considered a deceptive business practice
and can violate laws against false advertising. The
\href{https://ag.ny.gov/press-release/2013/ag-schneiderman-announces-agreement-19-companies-stop-writing-fake-online-reviews}{New
York attorney general} and the
\href{https://www.ftc.gov/news-events/press-releases/2019/02/ftc-brings-first-case-challenging-fake-paid-reviews-independent}{Federal
Trade Commission} have fined companies for posting such reviews, though
consequences can also be less severe. After the F.T.C. accused the
cosmetics brand Sunday Riley of posting fake reviews, it simply
\href{https://www.nytimes.com/2019/10/22/us/sunday-riley-fake-reviews.html}{made
the company agree} not to do so again.

\href{https://support.google.com/googleplay/android-developer/answer/9898684}{Google}
and Apple forbid developers from soliciting fraudulent reviews. Apple
\href{https://developer.apple.com/app-store/review/guidelines/}{says}
violators may have their apps removed from the App Store.

\hypertarget{irreverence-unusual-to-health-care}{%
\subsection{Irreverence unusual to health
care}\label{irreverence-unusual-to-health-care}}

Talkspace has also seized on moments of national anxiety as
opportunities for promotion. On Nov. 9, 2016, the morning after the
election of Donald Trump, Mr. Frank
\href{https://twitter.com/orenfrank/status/796362326073081856}{wrote on
Twitter}: ``Long night in NYC. Woke up this morning to record sales.''
The company told reporters that users were flocking to the app to help
process the news. CNBC and
\href{https://www.washingtonpost.com/news/on-small-business/wp/2016/11/17/trumps-victory-is-proving-to-be-great-newsif-youre-in-the-therapy-business/}{The
Washington Post} published stories about Talkspace's
``\href{https://www.cnbc.com/2016/11/14/online-therapy-startup-sees-a-7-fold-spike-in-traffic-after-trump-victory.html}{7-fold
spike in traffic},'' and Mr. Frank
\href{https://twitter.com/orenfrank/status/796446638923513857}{shared} a
Fast Company link claiming a ``7x spike in sales.''

Image

Talkspace told reporters that usage of its app spiked after Donald Trump
was declared the winner of the presidential election in
2016.Credit...Eric Thayer for The New York Times

According to data from two app analytics firms, App Annie and Sensor
Tower, the number of Talkspace downloads declined in the months after
the election. The Times analyzed more than 3,600 reviews of the
Talkspace app. There was no significant increase in the number of
reviews, positive or negative, following the 2016 election.

Dr. Leibowitz, Talkspace's chief medical officer, who joined the company
in 2018, said in an email: ``We saw an uptick in use after the election,
including, as the piece mentions, an uptick in traffic from existing
clients concerned about election results. App analytics fail to capture
a few elements: Much of our traffic is on the web.''

The Trump election tweets are examples of the sometimes unfiltered
social media presence of Mr. Frank and Talkspace ---~an irreverence
familiar among start-ups but unusual among organizations devoted to
mental health care.

In 2016, a man named Ross complained on Twitter that the company's
subway ads ``were designed to trigger you into needing their services.''
Talkspace's official Twitter account responded, ``Ads for food make
people hungry, right?'' and added, ``I get what you're saying, Ross, but
medical professionals need people to buy things.'' The company later
deleted the messages and blocked the man. (Ross wrote about the exchange
in a Medium post; when The Times asked for comment recently, he deleted
it and asked that his full name be withheld, citing personal reasons.)

From his own Twitter account, Mr. Frank called the man a ``sweet bored
troll'' and mocked him for spending \$20,000 a year on therapy, saying
Talkspace could offer ``a more affordable alternative.'' The company
declined to comment about the episode.

\hypertarget{we-need-data-all-of-our-data}{%
\subsection{`We need data. All of our
data.'}\label{we-need-data-all-of-our-data}}

Therapy sessions are incredibly sensitive by their nature --- they are
intended to be a sacrosanct space for people to confess their secrets
and share their deepest vulnerabilities.

Talkspace's website promises users that their conversations will be
``safe and confidential,'' but people may not have as much control as
they might think over what happens to their data. Users
\href{https://help.talkspace.com/hc/en-us/articles/360000286663-Am-I-able-to-delete-my-chat-transcript-}{can't
delete} their transcripts, for example, because they are considered
medical records.

Talkspace's
\href{https://www.talkspace.com/public/privacy-policy}{privacy policy}
states that ``non-identifying and aggregate information'' may be used
``to better design our website'' and ``in research and trend analysis.''
The impression left is a detached and impersonal process. But former
employees and therapists told The Times that individual users'
anonymized conversations were routinely reviewed and mined for insights.

Karissa Brennan, a New York-based therapist, provided services via
Talkspace from 2015 to 2017, including to Mr. Lori. She said that after
she provided a client with links to therapy resources outside of
Talkspace, a company representative contacted her, saying she should
seek to keep her clients inside the app.

``I was like, `How do you know I did that?''' Ms. Brennan said. ``They
said it was private, but it wasn't.''

The company says this would only happen if an algorithmic review flagged
the interaction for some reason --- for example, if the therapist
recommended medical marijuana to a client. Ms. Brennan says that to the
best of her recollection, she had sent a link to an anxiety worksheet.

Talkspace also has been analyzing transcripts in order to develop
\href{https://news.bloomberglaw.com/privacy-and-data-security/talkspace-wants-to-build-a-better-therapist-with-ai-listening-in}{bots}
that monitor and augment therapists' work. During a presentation in
2019, a Talkspace engineer specializing in machine learning
\href{https://www.youtube.com/watch?v=aq0AhbvxBkc\&feature=emb_logo}{said
the research} was important because certain cues that a client is in
distress that could be caught during in-person sessions might be missed
when a therapist is only communicating by text. Software might better
catch those cues.

Last year, Mr. Frank wrote
\href{https://www.nytimes.com/2019/10/02/opinion/health-care-data-privacy.html}{an
opinion article for The Times} encouraging people to make their health
data available to researchers. ``We need data. All of our data. Mine and
yours,'' he wrote, arguing that analysis of anonymous data sets could
improve treatment.

The anonymous data Talkspace collects is not used just for medical
advancements; it's used to better sell Talkspace's product. Two former
employees said the company's data scientists shared common phrases from
clients' transcripts with the marketing team so that it could better
target potential customers.

The company disputes this. ``We are a data-focused company, and data
science and clinical leadership will from time to time share insights
with their colleagues,'' Mr. Reilly said. ``This can include evaluating
critical information that can help us improve best practices.''

He added: ``It never has and never will be used for marketing
purposes.''

\hypertarget{engagement-based-therapy}{%
\subsection{`Engagement'-based therapy}\label{engagement-based-therapy}}

Many licensed therapists sign up with Talkspace for reasons similar to
why drivers work for Uber. The company provides a steady stream of
clients, takes care of administrative tasks and deals with some
insurance issues.

``The beauty of text-based therapy is we are meeting clients where they
are, and giving them access to something different,'' said Reshawna
Chapple, a Talkspace therapist whom the company made available for an
interview. ``It's about convenience for me.''

``The thing that Talkspace allows me to do is to put my hands in a lot
of different pots,'' said Dr. Chapple, who communicates with 30 clients
via Talkspace, treats 15 in person, and works as a full-time professor
at the University of Central Florida. She also has a contract with
Talkspace to advise other therapists.

The approximately 3,000 therapists who work on the platform are paid by
``engagement,'' according to the company, based on the number of words
they write to users or how often they talk by video or audio, with
bonuses for client retention.

According to multiple therapists, Talkspace paid special attention to
their interactions with clients who worked at places like Google, Kroger
and JetBlue --- ``enterprise partners'' that provide Talkspace to
employees as a perk. (The New York Times offers Talkspace to its workers
as a benefit.)

A college professor who provided therapy via Talkspace for two years
said the company reached out to her when it thought two clients from
Google had been waiting too long for a response.

``Like all businesses, we focus on clients based on size and scope,''
said Dr. Leibowitz, the chief medical officer.

Last year, Talkspace introduced a new feature: a
\href{https://help.talkspace.com/hc/en-us/articles/360029308131-How-does-Guaranteed-Response-Time-work-}{button}
that users could press after sending a message that required the
therapist to respond within a certain time frame. If the therapists
don't respond in time, their pay can be docked.

Some therapists on the platform were alarmed, in part because the
function required them to work on demand, rather than on their own
schedule. More significantly, they asked: Is it harmful to give clients
with anxiety and boundary issues a button to press for immediate
gratification?

``That's a corporate model: You need to respond to the customer no
matter what,'' said Shara Sand, a psychologist with her own practice in
New York. ``Limit-setting and boundary-setting is part of the therapy.
If you can't manage not to talk to your therapist for four hours, you
are very ill and need a higher level of care than a texting app.''

\hypertarget{pushback-on-clinical-benefits}{%
\subsection{Pushback on clinical
benefits}\label{pushback-on-clinical-benefits}}

Talkspace is advertised to users as unlimited, ``24/7'' messaging
therapy. ``Your therapist will see your messages and respond to you
throughout the day,'' the company
\href{https://www.talkspace.com/online-therapy/unlimited-messaging-therapy/}{says}.
Therapists \href{https://www.talkspace.com/join-as-a-therapist}{get a
different pitch}: ``Set your business hours, and check in on your
clients daily, five days per week.''

The company says the two messages are not in conflict. ``I don't think
it's a discrepancy in expectations,'' said Rachel O'Neill, a licensed
therapist at Talkspace whose title is director of clinical
effectiveness. ``It's not 24/7 therapy, it's 24/7 ability to
communicate.''

Some traditional mental health professionals question the free-flowing
format, saying that the benefits of therapy stem from regular, scheduled
check-ins --- sessions with clear beginnings and endings that help mark
progress.

``It's called the `frame' in psychoanalysis. It's the room. It's how
long the session will last. How much it will cost,'' said Berkeley's Ms.
Zeavin. ``Boundaries are really important to the history of therapy. If
texting is equated with no boundaries, that's a real problem.''

There has been
\href{https://www.nytimes.com/wirecutter/blog/text-therapy/}{limited
study} into how effective teletherapy is. Much of it either has been
conducted by
\href{https://www.researchgate.net/publication/326003323_The_Effect_of_Messaging_Therapy_for_Depression_and_Anxiety_on_Employee_Productivity}{Talkspace
itself} or has involved therapy via
\href{https://pubmed.ncbi.nlm.nih.gov/26864655/}{video sessions}, not
just text.

``Talkspace's No. 1 priority is quality of care for patients and driving
the clinical outcomes desired by patients,'' Dr. Leibowitz said.
``Talkspace has conducted research in partnership with many of the top
academic universities,'' he said, adding that the work has yielded ``10
vetted papers in peer-reviewed journals.''

Lynn Bufka, the senior director for practice transformation and quality
at the American Psychological Association, or A.P.A., said the research
on text-based therapy has been based on surveys of whether people find
it satisfactory.

``There's been much less research into whether there's a clinical
benefit,'' Dr. Bufka said. ``We would offer cautions around relying on
text therapy, particularly when there is greater severity in terms of
symptoms. We would urge people to seek direct care, which at this time
would be by phone or video.''

In 2018, a therapists advocacy group called the Psychotherapy Action
Network wrote a letter to the A.P.A. and to the Olympian Michael Phelps,
who has appeared in ads for Talkspace, calling the company a
``problematic treatment provider who aggressively sells an untested,
risky treatment.'' After receiving the letter, the A.P.A. changed its
\href{https://irp-cdn.multiscreensite.com/74e2b053/files/uploaded/APA_Tech\%20Advertising\%20Policy_2020_rev.pdf}{policy}
on therapy-tech ads and stopped letting Talkspace exhibit at
conferences.

Image

The Olympian Michael Phelps has been a prominent spokesman for
Talkspace, appearing in advertisements for the
service.Credit...Talkspace

In 2019, after Talkspace signed a deal with Optum, a unit of the health
care giant UnitedHealth, to provide teletherapy to its two million
customers, the advocacy group wrote another
\href{https://mppsyd.com/post/185548172126/mental-health-app-raises-50-million-for}{letter
of ``alarm''} to the A.P.A. Talkspace
\href{https://casetext.com/case/groop-internet-platform-inc-v-psychotherapy-action-network}{sued
the group for defamation}, claiming damages of \$40 million. The lawsuit
was dismissed for jurisdictional reasons.

``Maybe their products and services are helpful to certain people,''
said Linda Michaels, a founder of the Psychotherapy Action Network.
``But it's just not therapy.''

Until 2018, the
\href{https://web.archive.org/web/20180817042804/https://www.talkspace.com/public/user-agreement}{Talkspace
user agreement} said the same thing: ``This Site Does Not Provide
Therapy. It provides Therapeutic conversation with a licensed
therapist.'' The company has since removed the clause.

``That is very old,'' Dr. Leibowitz said. ``The company has evolved
quite a bit.''

Mr. Lori no longer uses the Talkspace app. But he is still seeing the
therapist, Ms. Brennan, whom he originally met via the platform.

``Even through this toxic company, wonderful things can happen,'' he
said. ``It's such a sad story in totality, of what the company could
have been versus what it is.''

Susan Beachy contributed research.

Advertisement

\protect\hyperlink{after-bottom}{Continue reading the main story}

\hypertarget{site-index}{%
\subsection{Site Index}\label{site-index}}

\hypertarget{site-information-navigation}{%
\subsection{Site Information
Navigation}\label{site-information-navigation}}

\begin{itemize}
\tightlist
\item
  \href{https://help.nytimes.com/hc/en-us/articles/115014792127-Copyright-notice}{©~2020~The
  New York Times Company}
\end{itemize}

\begin{itemize}
\tightlist
\item
  \href{https://www.nytco.com/}{NYTCo}
\item
  \href{https://help.nytimes.com/hc/en-us/articles/115015385887-Contact-Us}{Contact
  Us}
\item
  \href{https://www.nytco.com/careers/}{Work with us}
\item
  \href{https://nytmediakit.com/}{Advertise}
\item
  \href{http://www.tbrandstudio.com/}{T Brand Studio}
\item
  \href{https://www.nytimes.com/privacy/cookie-policy\#how-do-i-manage-trackers}{Your
  Ad Choices}
\item
  \href{https://www.nytimes.com/privacy}{Privacy}
\item
  \href{https://help.nytimes.com/hc/en-us/articles/115014893428-Terms-of-service}{Terms
  of Service}
\item
  \href{https://help.nytimes.com/hc/en-us/articles/115014893968-Terms-of-sale}{Terms
  of Sale}
\item
  \href{https://spiderbites.nytimes.com}{Site Map}
\item
  \href{https://help.nytimes.com/hc/en-us}{Help}
\item
  \href{https://www.nytimes.com/subscription?campaignId=37WXW}{Subscriptions}
\end{itemize}
