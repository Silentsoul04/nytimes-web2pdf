Sections

SEARCH

\protect\hyperlink{site-content}{Skip to
content}\protect\hyperlink{site-index}{Skip to site index}

\href{https://www.nytimes.com/section/politics}{Politics}

\href{https://myaccount.nytimes.com/auth/login?response_type=cookie\&client_id=vi}{}

\href{https://www.nytimes.com/section/todayspaper}{Today's Paper}

\href{/section/politics}{Politics}\textbar{}Brent Scowcroft, a Force on
Foreign Policy for 40 Years, Dies at 95

\href{https://nyti.ms/33CKZMU}{https://nyti.ms/33CKZMU}

\begin{itemize}
\item
\item
\item
\item
\item
\item
\end{itemize}

Advertisement

\protect\hyperlink{after-top}{Continue reading the main story}

Supported by

\protect\hyperlink{after-sponsor}{Continue reading the main story}

\hypertarget{brent-scowcroft-a-force-on-foreign-policy-for-40-years-dies-at-95}{%
\section{Brent Scowcroft, a Force on Foreign Policy for 40 Years, Dies
at
95}\label{brent-scowcroft-a-force-on-foreign-policy-for-40-years-dies-at-95}}

He was a national security adviser to President Ford and the first
President Bush and an influential voice in Washington for decades.

\includegraphics{https://static01.nyt.com/images/2020/08/08/obituaries/00Scowcroft1/00Scrowcroft1-articleLarge.jpg?quality=75\&auto=webp\&disable=upscale}

By \href{https://www.nytimes.com/by/robert-d-mcfadden}{Robert D.
McFadden}

\begin{itemize}
\item
  Aug. 7, 2020
\item
  \begin{itemize}
  \item
  \item
  \item
  \item
  \item
  \item
  \end{itemize}
\end{itemize}

Brent Scowcroft, a pre-eminent foreign policy expert who helped shape
America's international and strategic decisions for decades as the
national security adviser to Presidents Gerald R. Ford and George Bush
and as a counselor to seven administrations, died on Thursday at his
home in Falls Church, Va. He was 95.

His death was announced by a family spokesman, Jim McGrath, who did not
cite a specific cause.

Mr. Scowcroft wanted to be a fighter pilot after World War II, but a
plane crash changed the young man's life and, as it turned out, gave the
nation one of its most authoritative military intellectuals --- a
diplomat, linguist, tactician on nuclear arms and missile systems and a
scholar of global politics who became an influential voice in Washington
for more than 40 years.

He accompanied President Richard M. Nixon to China in 1972, oversaw the
Ford administration's evacuation of Americans from Saigon in 1975, laid
groundwork for President Jimmy Carter's Strategic Arms Limitation Treaty
with the Soviet Union in 1979, evaluated the MX missile systems for
President Ronald Reagan in the 1980s and directed President Bush's
strategy in the Persian Gulf war in 1991.

Mr. Scowcroft was a principal architect of American policy toward
post-communist Russia, a leading Republican voice
\href{https://www.wsj.com/articles/SB1029371773228069195}{opposing the
American-led invasion of Iraq} after the Sept. 11 terrorist attacks and
a voice in President Barack Obama's selection of a national security
team after the 2008 elections.

He also wrote books, taught at universities and counted among his many
protégés Condoleezza Rice and Robert M. Gates, both national security
experts who became secretaries of state and defense for President George
W. Bush.

\includegraphics{https://static01.nyt.com/images/2020/08/08/obituaries/00scowcroft2/0000scowcroft2-articleLarge.jpg?quality=75\&auto=webp\&disable=upscale}

Most closely associated with moderate Republicans like Ford,
\href{https://www.nytimes.com/2014/06/27/us/politics/howard-h-baker-jr-great-conciliator-of-senate-dies-at-88.html}{Howard
H. Baker Jr.} and Colin L. Powell, Mr. Scowcroft (pronounced SKO-croft)
was a self-effacing former Air Force general who did not smoke or drink.
He preferred working quietly in small groups.

In making foreign policy, a national security adviser coordinates the
work of the National Security Council --- the president, vice president,
secretaries of state and defense and others, supported by a staff that
writes papers and proposals --- and makes sure that the president hears
all sides of the debate before making decisions.

Mr. Scowcroft called himself a traditionalist, who believed that the
nation should work with allies and international organizations, as
opposed to a ``transformationalist,'' like the second President Bush,
who argued that America should fight terrorism by spreading democracy in
the world --- by force if necessary --- and be free to act swiftly
without relying on overly cautious allies or a cumbersome United
Nations.

After leaving government in 1993, Mr. Scowcroft headed the
Washington-based Scowcroft Group, a consulting firm for international
businesses, and was chairman of an advisory board that made policy
recommendations to President George W. Bush.

Nevertheless, he was among the few prominent Republicans who challenged
President Bush in 2002 as the administration made its case to go to war
in Iraq.

In an op-ed article in The Wall Street Journal titled ``Don't Attack
Saddam,'' Mr. Scowcroft said there was ``scant evidence'' of ties
between Iraq and Al Qaeda or the Sept. 11 attacks, as Mr. Bush claimed.
And he argued that an invasion to oust the Iraqi leader, Saddam Hussein,
would ``seriously jeopardize, if not destroy, the global
counterterrorist campaign we have undertaken.''

The Iraq war, he told the Op-Ed columnist Roger Cohen of The New York
Times in 2007, had also
\href{https://www.nytimes.com/2007/09/03/opinion/03cohen.html}{undermined
faith in America}.

``Historically, the world has always given us the benefit of the doubt
because it believed we meant well,'' Mr. Scowcroft said. ``It no longer
does. It is easy to lose trust, but it takes a lot of work to gain it.
Can the sense of confidence in us be restored? Sure. But not easily.''

Image

From left, John H. Sununu, the White House chief of staff, Mr. Scowcroft
and Gen. H. Norman Schwarzkopf at the Pentagon in 1990.Credit...Kevin
Larkin/Agence France-Press --- Getty Images

President Obama liked Mr. Scowcroft and his restrained foreign policy,
Jeffrey Goldberg
\href{https://www.theatlantic.com/magazine/archive/2016/04/the-obama-doctrine/471525/}{noted
in The Atlantic in 2016}. ``Obama, unlike liberal interventionists, is
an admirer of the foreign-policy realism of President George H.W. Bush
and, in particular, of Bush's national security adviser, Brent
Scowcroft,'' he wrote. ``As Obama was writing his campaign manifesto,
`The Audacity of Hope,' in 2006, Susan Rice, then an informal adviser,
felt it necessary to remind him to include at least one line of praise
for the foreign policy of President Bill Clinton, to partially balance
the praise he showered on Bush and Scowcroft.''

Long after his retirement, Mr. Scowcroft remained a pillar of the
Republican national security establishment. In the run-up to the 2016
presidential election, he joined more than 120 other Republican foreign
policy veterans who crossed party lines and endorsed Hillary Clinton.
Mr. Scowcroft
\href{https://time.com/4378850/brent-scowcroft-hillary-clinton-donald-trump/}{said
she possessed} ``truly unique experience and perspective'' to ``lead our
country at this critical time.'' He did not mention Donald J. Trump in
his endorsement.

But days after Mr. Trump's election, the frail and ailing Mr. Scowcroft
made an emotional appeal at an off-the-record Washington luncheon in his
honor, calling on fellow Republicans, and Democrats, to put country
above political party and accept posts in the incoming Trump
administration if asked to do so --- even though, by some accounts, he
remained concerned that Mr. Trump was ill-prepared and unsuited for the
presidency.

``He needs you, your country needs you,'' one attendee said,
characterizing Mr. Scowcroft's message.

His appeal for public service was a classic reminder of a less partisan
age, when presidents often reached out to experienced talent, regardless
of party loyalties.

Brent Scowcroft was born on March 19, 1925, in Ogden, Utah, the son of
James and Lucile (Ballantyne) Scowcroft. He graduated from the United
States Military Academy at West Point in 1947, joined the Air Force and
envisioned life as a fighter pilot.

But on Jan. 6, 1949, his P-51 Mustang developed engine trouble after
taking off from Grenier Army Air Field in New Hampshire (now
Manchester-Boston Regional Airport), and crash-landed. His injuries were
not critical, but he assumed he would never fly again and considered
other military career options.

In 1951, he married Marian Horner. She died in 1995. He is survived by
their daughter, Karen Scowcroft, and a granddaughter.

Image

Mr. Kissinger and Mr. Scowcroft monitored developments~in the seizure
of~the S.S. Mayagüez in 1975.Credit...David Hume Kennerly/Gerald R. Ford
Library

Mr. Scowcroft earned a master's degree in international relations from
Columbia University in 1953. He taught Russian history for four years at
West Point, studied Slavic languages at Georgetown University in 1958
and, from 1959 to 1961, used his Serbo-Croatian skills as an assistant
air attaché at the American Embassy in Belgrade, Yugoslavia. He taught
political science at the Air Force Academy in Colorado in 1962-63.

He then joined the Air Force planning division in Washington, and in
1967 earned a doctorate in international relations at Columbia. Starting
in 1968, he held various Pentagon posts, becoming a special assistant to
Gen. John W. Vogt, director of the Joint Chiefs of Staff. In 1972, by
then a general, he became a military aide to President Nixon.

Mr. Scowcroft accompanied Nixon on his historic trip to China to
establish diplomatic relations after decades of estrangement. Fluent in
Russian, he next went to Moscow to prepare for Nixon's spring visit
there, a delicate task because America was bombing North Vietnam, a
Soviet ally. Impressed, Henry A. Kissinger, then head of national
security, chose him as his deputy in 1973. That fall, Mr. Kissinger
became secretary of state, and Mr. Scowcroft ran Security Council
meetings in his absence.

At the White House, Mr. Scowcroft put in 18-hour days, working from a
cramped, cable-strewn cubby near the Oval Office. In 1975, after Nixon
had resigned in the Watergate scandal, Mr. Scowcroft briefed the new
president, Ford, on national security. Ford chose him to succeed Mr.
Kissinger as the national security adviser. To accept, he resigned his
commission as a lieutenant general.

Image

Mr. Scowcroft speaking before the Senate Armed Services Committee in
1983.Credit...George Tames/The New York Times

As American involvement in the Vietnam War ended in 1975, Mr. Scowcroft
planned the evacuation of American personnel from Saigon, and later a
military response to the Cambodian seizure of the American merchant ship
S.S. Mayagüez. The ship and 39 crewmen were saved, but 41 American
servicemen died.

Mr. Scowcroft left the White House in 1977, when Jimmy Carter became
president. He later went to work for Mr. Kissinger's international
consulting firm, Kissinger Associates. But he remained a member of the
President Carter's advisory committee on arms control and helped
formulate the Strategic Arms Limitation Treaty II signed by Mr. Carter
and the Soviet leader Leonid Brezhnev in 1979. The treaty, though not
ratified by Congress, was honored until 1986, when the Reagan
administration withdrew from it.

President Reagan named Mr. Scowcroft to head a commission that evaluated
options for deployment of the MX missiles. Later, Reagan appointed him
to a commission led by former Senator John G. Tower that investigated
the Iran-contra scandal, in which money from arms sales to Iran was
diverted without authorization to anti-communist rebels in Nicaragua.
The commission found no evidence that Mr. Reagan had known of
skulduggery, but it criticized him for failing to monitor subordinates.

While it is rare for officials to return to White House jobs, Mr.
Scowcroft was an early and easy choice of President Bush for national
security adviser in 1989. Mr. Scowcroft chose Mr. Gates as his deputy
and Ms. Rice, a Soviet expert, as a council member.

Mr. Scowcroft was instrumental in developing policies toward
post-communist Russia that leaned away from trying to reshape its
political and economic systems --- a response to the Soviet collapse
that critics said was too passive, one that missed chances to promote
democratic institutions. Mr. Scowcroft, however, defended the
administration's approach as prudently cautious in a world turned upside
down after the Cold War.

Image

Mr. Scowcroft in 2002.~Long after his retirement, he remained a pillar
of the Republican national security establishment.Credit...Paul
Hosefros/The New York Times

Mr. Scowcroft made two secret trips to China in 1989: one to underscore
America's shock over China's crackdown on pro-democracy demonstrators in
Tiananmen Square, which had left hundreds dead, and a second to mend
relations with Beijing after Mr. Bush had canceled contacts between
American and Chinese leaders.

Disclosure of the trips gave the public a rare glimpse of the diplomatic
dance that often plays out behind the scenes --- a two-step of rejection
and reconciliation more typical of quarrelsome couples.

In 1991 Mr. Scowcroft was the guiding hand behind what Mr. Bush regarded
as the triumph of his political life, Operation Desert Storm, in which
the president mobilized an international coalition to oust an invading
Saddam Hussein from neighboring Kuwait. Mr. Scowcroft generally drew
narrow military goals, and the Persian Gulf war was no exception. He
urged Mr. Bush to limit operations to evicting Iraqi troops from Kuwait
and not to depose Saddam Hussein, unless the dictator resorted to
chemical or biological warfare against coalition forces or Israel.

Mr. Bush awarded Mr. Scowcroft the Presidential Medal of Freedom, the
nation's highest civilian honor.

Mr. Scowcroft wrote ``A World Transformed,'' (1998) with George Bush, on
policy issues after the Cold War. His talks with
\href{https://www.nytimes.com/2017/05/26/us/zbigniew-brzezinski-dead-national-security-adviser-to-carter.html}{Zbigniew
Brzezinski}, Mr. Carter's national security adviser, were published as
``America and the World: Conversations on the Future of American Foreign
Policy'' (2008).

A biography by Bartholomew Sparrow, ``The Strategist: Brent Scowcroft
and the Call of National Security,'' was published in 2015.

Advertisement

\protect\hyperlink{after-bottom}{Continue reading the main story}

\hypertarget{site-index}{%
\subsection{Site Index}\label{site-index}}

\hypertarget{site-information-navigation}{%
\subsection{Site Information
Navigation}\label{site-information-navigation}}

\begin{itemize}
\tightlist
\item
  \href{https://help.nytimes.com/hc/en-us/articles/115014792127-Copyright-notice}{©~2020~The
  New York Times Company}
\end{itemize}

\begin{itemize}
\tightlist
\item
  \href{https://www.nytco.com/}{NYTCo}
\item
  \href{https://help.nytimes.com/hc/en-us/articles/115015385887-Contact-Us}{Contact
  Us}
\item
  \href{https://www.nytco.com/careers/}{Work with us}
\item
  \href{https://nytmediakit.com/}{Advertise}
\item
  \href{http://www.tbrandstudio.com/}{T Brand Studio}
\item
  \href{https://www.nytimes.com/privacy/cookie-policy\#how-do-i-manage-trackers}{Your
  Ad Choices}
\item
  \href{https://www.nytimes.com/privacy}{Privacy}
\item
  \href{https://help.nytimes.com/hc/en-us/articles/115014893428-Terms-of-service}{Terms
  of Service}
\item
  \href{https://help.nytimes.com/hc/en-us/articles/115014893968-Terms-of-sale}{Terms
  of Sale}
\item
  \href{https://spiderbites.nytimes.com}{Site Map}
\item
  \href{https://help.nytimes.com/hc/en-us}{Help}
\item
  \href{https://www.nytimes.com/subscription?campaignId=37WXW}{Subscriptions}
\end{itemize}
