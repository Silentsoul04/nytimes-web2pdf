Sections

SEARCH

\protect\hyperlink{site-content}{Skip to
content}\protect\hyperlink{site-index}{Skip to site index}

\href{https://www.nytimes.com/section/politics}{Politics}

\href{https://myaccount.nytimes.com/auth/login?response_type=cookie\&client_id=vi}{}

\href{https://www.nytimes.com/section/todayspaper}{Today's Paper}

\href{/section/politics}{Politics}\textbar{}Trump Threatens to Bypass
Congress as Stimulus Talks Fail Again

\href{https://nyti.ms/2Pwiicv}{https://nyti.ms/2Pwiicv}

\begin{itemize}
\item
\item
\item
\item
\item
\end{itemize}

Advertisement

\protect\hyperlink{after-top}{Continue reading the main story}

Supported by

\protect\hyperlink{after-sponsor}{Continue reading the main story}

\hypertarget{trump-threatens-to-bypass-congress-as-stimulus-talks-fail-again}{%
\section{Trump Threatens to Bypass Congress as Stimulus Talks Fail
Again}\label{trump-threatens-to-bypass-congress-as-stimulus-talks-fail-again}}

Democrats said the talks had been ``disappointing,'' and President Trump
promised to use executive orders to provide relief if no agreement could
be reached.

\includegraphics{https://static01.nyt.com/images/2020/08/07/us/politics/07dc-virus-cong-01/merlin_175432500_d6a89223-8234-4d82-a674-6c57f7b3cec7-articleLarge.jpg?quality=75\&auto=webp\&disable=upscale}

\href{https://www.nytimes.com/by/emily-cochrane}{\includegraphics{https://static01.nyt.com/images/2018/11/28/multimedia/author-emily-cochrane/author-emily-cochrane-thumbLarge-v3.png}}\href{https://www.nytimes.com/by/jim-tankersley}{\includegraphics{https://static01.nyt.com/images/2018/10/19/multimedia/author-jim-tankersley/author-jim-tankersley-thumbLarge.png}}

By \href{https://www.nytimes.com/by/emily-cochrane}{Emily Cochrane} and
\href{https://www.nytimes.com/by/jim-tankersley}{Jim Tankersley}

\begin{itemize}
\item
  Aug. 7, 2020
\item
  \begin{itemize}
  \item
  \item
  \item
  \item
  \item
  \end{itemize}
\end{itemize}

WASHINGTON --- Crisis negotiations between the White House and top
Democrats teetered on the brink of collapse on Friday, as both sides
said they remained deeply divided on an economic recovery package and
President Trump threatened to bypass Congress and act on his own to
provide relief if no deal could be reached.

In a hastily called evening news conference after talks on Capitol Hill
broke up without a compromise, Mr. Trump said he could sign executive
orders within a week to delay payroll tax collections, extend an
eviction moratorium, give flexibility to Americans who owe student loans
and supplement unemployment benefits through the end of the year.

``If Democrats continue to hold this critical relief hostage, I will act
under my authority as president to get Americans the relief they need,''
Mr. Trump said in a ballroom at his golf resort in Bedminster, N.J.,
where dozens of club members, some sipping wine, gathered to watch as
the president claimed that the economy was quickly bouncing back.

It was not clear what power Mr. Trump might have to move unilaterally to
extend jobless aid or otherwise redirect federal relief money as he sees
fit because Congress controls spending. And Mr. Trump conceded that such
a move was likely to be met with a legal challenge that would block any
help from reaching the tens of millions of Americans who have depended
for months on \$600 weekly federal jobless payments that vanished last
week in the absence of a deal to extend them.

``Probably, we'll get sued,'' Mr. Trump said.

But his threat to do so reflected the failure of days of marathon talks
on Capitol Hill to reach a bipartisan compromise to pump more aid into
the slowing economic recovery --- and the president's determination to
be seen as acting in the face of the gridlock.

It came after another in a string of unproductive meetings between Mark
Meadows, the White House chief of staff, and Steven Mnuchin, the
Treasury secretary, and Democratic leaders, which ended with no
agreement and no additional talks scheduled.

Lawmakers and lobbyists said Mr. Trump was looking at possibly invoking
a law that governs federal response to natural disasters like hurricanes
to unilaterally restore enhanced jobless benefits that expired last
week. Administration officials have suggested Mr. Trump might seek to
repurpose funds that Congress allocated for coronavirus relief this year
that have not yet been spent by states or federal agencies.

Critical issues continue to divide the negotiators, including the
overall price tag of the bill and whether to help states and localities
across the country bridge budget shortfalls that are a direct result of
the
\href{https://www.nytimes.com/2020/06/08/business/economy/us-economy-recession-2020.html}{pandemic
recession}. The central tension point has been how much additional
assistance to send to unemployed Americans. Democrats are pressing to
extend the \$600 weekly benefit, while Republicans have sought to cut
it.

Mr. Trump declined to say how much jobless aid he would seek to provide
by executive order.

Republicans have shifted position on the additional unemployment
benefits, first proposing to extend them at a much lower rate, then
raising their bid in negotiations with Democrats. But Mr. Trump has
remained steadfast in his public opposition to any money for states and
local governments, which he has falsely said would go only to states run
by Democrats and does not have any relationship to the current crisis.

It was not clear whether any unilateral move by the president would
effectively scuttle further talks, or serve to accelerate them.
Democrats emerged from what they called a ``disappointing'' afternoon
negotiating session accusing the administration of refusing to
compromise.

``This isn't about negotiating or leverage or anything,'' Speaker Nancy
Pelosi said. ``It's about meeting the needs of the American people.''

Even as they said they would advise Mr. Trump to issue executive orders
to provide economic relief, administration officials held out the
possibility that further negotiations could yet yield an agreement.

``I don't want to speculate as to whether there is an agreement or
not,'' Mr. Mnuchin said. ``We will continue to try to get an agreement
that's in the best interest of the people, and that's why we're here.''

Mr. Meadows conceded that executive orders were not ``a perfect answer
--- we'll be the first ones to say that.''

``But it is all that we can do and all the president can do within the
confines of his executive power,'' he added, ``and we're going to
encourage him to do it.''

Democrats, who had earlier said they would be willing to lower their
spending demands to \$2.4 trillion from \$3.4 trillion, said the White
House needed to return with a higher overall price tag, after Mr.
Trump's negotiators declined to accept that offer. Republicans have
proposed a \$1 trillion plan.

``The House is Democratic, they need a majority of Democratic votes in
the Senate,'' said Senator Chuck Schumer of New York, the Democratic
leader. ``Meet us in the middle --- for God's sake, please --- for the
sake of America, meet us in the middle.''

Mr. Mnuchin and Mr. Meadows demanded that Democrats agree to lower the
amount of aid for state and local governments, and provide more
specifics about how they were proposing to revive lapsed unemployment
benefits.

A
\href{https://www.nytimes.com/2020/08/07/business/economy/july-jobs-report.html}{jobs
report on Friday} that found that the United States economy added 1.8
million jobs in July further muddled the talks, giving each side support
for its talking points. Democrats seized on it as a call to action,
quickly issuing a statement calling for a face-to-face meeting to
continue negotiations on the stimulus package.

But the White House and Republicans, who are pressing for a narrower
recovery measure, cheered the report, which beat economists'
expectations, and were likely to see it as confirmation of their
argument that it is time to scale back federal help.

The overall scope and cost of any agreement remains one of the most
significant sticking points, along with how much aid should be provided
to state and local governments. Republicans have offered \$150 billion
in new relief, arguing that a significant portion of aid allocated in
the \$2.2 trillion stimulus law has not yet been spent, while Democrats
included nearly \$900 billion in their opening proposal.

\includegraphics{https://static01.nyt.com/images/2020/08/07/us/politics/07dc-virus-cong/merlin_175437309_e765af09-6663-4ca5-9fa1-7d9f8ec8a856-articleLarge.jpg?quality=75\&auto=webp\&disable=upscale}

In a private call with Republican senators on Friday morning, Mr.
Mnuchin and Mr. Meadows indicated that the funding for state and local
governments remained one of the most stark divisions between the two
parties, according to two people familiar with the discussion, who spoke
on the condition of anonymity to disclose the details of a private phone
call. One person said that the two administration officials singled out
education funding as another sticking point, given that Democrats were
pressing for far more relief to send to schools, as well as a Democratic
demand for additional relief for the Postal Service and election
protections ahead of the general election in November.

Mr. Trump has threatened all week to act on his own if no deal can be
reached, telling reporters that he could move as soon as Friday or
Saturday. At his news conference on Friday night, he said the action
could come next week.

Any move to reprogram unspent federal dollars for unemployment could
prompt legal challenges that could stall its delivery. Even without
those challenges, overburdened state unemployment offices could need
weeks or as much as a month to begin supplementing traditional benefit
checks again with federal dollars. The orders would do nothing to help
small businesses that have already exhausted federal assistance or state
and local governments bracing for layoffs amid declining tax revenues.

Suspending the payroll tax will do nothing to help unemployed Americans,
and analysts warn it could likely not help workers, either. That is
because workers would still be on the hook to pay the deferred taxes
next year, and many employers will choose to continue to withhold the
taxes to guard against workers owing penalties on unpaid taxes.

Economic forecasters expect the job market to cool even further this
month and in September, particularly if consumer spending declines
because of the expiration of unemployment benefits.

Nicholas Fandos contributed reporting.

Advertisement

\protect\hyperlink{after-bottom}{Continue reading the main story}

\hypertarget{site-index}{%
\subsection{Site Index}\label{site-index}}

\hypertarget{site-information-navigation}{%
\subsection{Site Information
Navigation}\label{site-information-navigation}}

\begin{itemize}
\tightlist
\item
  \href{https://help.nytimes.com/hc/en-us/articles/115014792127-Copyright-notice}{©~2020~The
  New York Times Company}
\end{itemize}

\begin{itemize}
\tightlist
\item
  \href{https://www.nytco.com/}{NYTCo}
\item
  \href{https://help.nytimes.com/hc/en-us/articles/115015385887-Contact-Us}{Contact
  Us}
\item
  \href{https://www.nytco.com/careers/}{Work with us}
\item
  \href{https://nytmediakit.com/}{Advertise}
\item
  \href{http://www.tbrandstudio.com/}{T Brand Studio}
\item
  \href{https://www.nytimes.com/privacy/cookie-policy\#how-do-i-manage-trackers}{Your
  Ad Choices}
\item
  \href{https://www.nytimes.com/privacy}{Privacy}
\item
  \href{https://help.nytimes.com/hc/en-us/articles/115014893428-Terms-of-service}{Terms
  of Service}
\item
  \href{https://help.nytimes.com/hc/en-us/articles/115014893968-Terms-of-sale}{Terms
  of Sale}
\item
  \href{https://spiderbites.nytimes.com}{Site Map}
\item
  \href{https://help.nytimes.com/hc/en-us}{Help}
\item
  \href{https://www.nytimes.com/subscription?campaignId=37WXW}{Subscriptions}
\end{itemize}
