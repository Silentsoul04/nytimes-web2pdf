Sections

SEARCH

\protect\hyperlink{site-content}{Skip to
content}\protect\hyperlink{site-index}{Skip to site index}

\href{https://www.nytimes.com/section/world/americas}{Americas}

\href{https://myaccount.nytimes.com/auth/login?response_type=cookie\&client_id=vi}{}

\href{https://www.nytimes.com/section/todayspaper}{Today's Paper}

\href{/section/world/americas}{Americas}\textbar{}Álvaro Uribe's
Detention Deepens Colombia's Divisions

\href{https://nyti.ms/2Pzag2A}{https://nyti.ms/2Pzag2A}

\begin{itemize}
\item
\item
\item
\item
\item
\item
\end{itemize}

Advertisement

\protect\hyperlink{after-top}{Continue reading the main story}

Supported by

\protect\hyperlink{after-sponsor}{Continue reading the main story}

\hypertarget{uxe1lvaro-uribes-detention-deepens-colombias-divisions}{%
\section{Álvaro Uribe's Detention Deepens Colombia's
Divisions}\label{uxe1lvaro-uribes-detention-deepens-colombias-divisions}}

Colombia's most powerful politician is now under house arrest, drawing
the country back into the pitched political battle it had been trying to
overcome for years.

\includegraphics{https://static01.nyt.com/images/2020/08/05/world/05uribe/05uribe-articleLarge-v2.jpg?quality=75\&auto=webp\&disable=upscale}

By \href{https://www.nytimes.com/by/julie-turkewitz}{Julie Turkewitz}

\begin{itemize}
\item
  Aug. 7, 2020
\item
  \begin{itemize}
  \item
  \item
  \item
  \item
  \item
  \item
  \end{itemize}
\end{itemize}

\href{https://www.nytimes.com/es/2020/08/07/espanol/america-latina/alvaro-uribe-colombia.html}{Leer
en español}

BOGOTÁ, Colombia --- Former president Álvaro Uribe has dominated
Colombia's political landscape for decades as the country's most beloved
--- and most hated --- politician.

To some Colombians, he is a savior, the only leader who was willing to
take the tough measures necessary to restore security in a nation
battered by a long, cruel civil war.

To others, he is a criminal whose no-holds-barred fight against
insurgents showed little regard for human rights and left thousands
dead, many of them civilians.

His house arrest,
\href{https://www.nytimes.com/2020/08/04/world/americas/colombia-president-uribe-charged.html}{ordered}
by the Supreme Court this week in connection with a case that harkens to
some of the grimmest aspects of the war, has intensified the country's
deep left-right rift, drawing Colombians back into the pitched political
battle the country has been trying to overcome for years.

``The country has so many wounds,'' said Paloma Valencia, a senator and
supporter of Mr. Uribe who began following him as a college student,
``this makes any kind of reconciliation many times more difficult.''

Just hours after the announcement of Mr. Uribe's detention, his
supporters on the right and his detractors on the left poured into the
streets around the country, honking their horns or banging pots in
outrage or celebration. Political commentators said the move threatened
the country's fragile reconciliation following a
\href{https://www.nytimes.com/2016/08/25/world/americas/colombia-farc-peace-deal.html}{2016
peace deal} that ended the conflict, which had been the longest running
war in the Americas.

\includegraphics{https://static01.nyt.com/images/2020/08/07/world/07uribe2/merlin_175319346_da80a927-8b27-4a75-a484-4cae62764fff-articleLarge.jpg?quality=75\&auto=webp\&disable=upscale}

By the next morning, Mr. Uribe's party had revived a call to overhaul
the justice system --- an apparent move to stop future detentions they
viewed as unfair --- and the current president, Iván Duque, a staunch
Uribe ally, assailed the court's decision to detain his mentor.

Soon, the office of the inspector general, which oversees the conduct of
public employees, was issuing an
\href{https://twitter.com/PGN_COL/status/1291128939382022145}{urgent
call} for public servants to ``respect and not attack the justice
system.''

Colombians, the office said, must ``stop the aggression and the extreme
polarization that could bring new scenes of violence. To the crisis
created by the pandemic of Covid-19, we cannot add a pandemic of hate
that clouds the future, threatens democracy and submerges us in a new
night of pain.''

At a crowded pro-Uribe gathering in Medellín following the decision, a
throng of cars cloaked in Colombian flags lined a major downtown avenue.
And protesters said they were outraged that their hero had been detained
while, under the terms of the 2016 peace deal, thousands of former
guerrilla fighters have gone free.

Santiago Vásquez, 23, called Mr. Uribe ``the best president Colombia has
ever had,'' describing him as the man who crippled the country's largest
rebel group, known as the FARC. He feared the former president's
detention would strengthen the left, ushering in the old days of
violence.

``Uribe! Amigo! Colombia is with you!'' Mr. Uribe's allies shouted.

Hundreds of miles away, in the capital of Bogotá, Colombians leaned out
of homes across the city, banging pots in frenzied celebration. Families
of those who had died in the war had thought Mr. Uribe would never be
called before a court to answer for his role and found themselves barely
able to believe the news.

``I pray that he pays for all the pain,'' said Lucero Carmona Martínez,
61, who said her son Omar, 26, was killed by security forces at a time
when Mr. Uribe was president and the military, under pressure to
increase the body count in combat, was killing civilians along with
rebel fighters.

Mr. Uribe, over the last 40 years, rose from being a relatively
small-time bureaucrat to the most powerful politician in the country,
wielding his charisma to create an entire political movement ---
Uribismo --- in his name.

He has long said that his father was killed by the FARC, something the
group has denied.

When he became president in 2002, a decades-long insurgency that had
begun as a fight over inequality had grown devastatingly violent.
Highway blockades, kidnappings and city bombings were regular
occurrences, and much of the nation was desperate for someone to restore
order and to defeat the FARC.

Mr. Uribe made combating the insurgents his government's top priority.
Many people credit him for significantly weakening the FARC and putting
an end to much of that terror.

``Without President Uribe, Colombia would not be a democracy,'' said Ms.
Valencia, the senator. ``It would be a failed state like Venezuela.''

Image

A group of FARC rebels in the jungle, months before the peace deal in
2016.Credit...Federico Rios for The New York Times

But as Mr. Uribe fought leftist guerrillas, his critics accused him of
overseeing a period of horrific abuses committed not just by the army,
but also by paramilitary groups allegedly doing the dirty work of the
government.

``He believed the ends justified the means,'' said Iván Cepeda, a
political opponent.

While Mr. Uribe was president, Colombian soldiers killed
\href{https://www2.ohchr.org/english/bodies/hrcouncil/docs/14session/A.HRC.14.24.Add.2_en.pdf}{thousands}
of innocent people, many of them peasants, according to years of
\href{https://www.hrw.org/report/2015/06/24/their-watch/evidence-senior-army-officers-responsibility-false-positive-killings}{investigation}
by prosecutors and human rights groups. Soldiers often tried to pass the
dead off as guerrilla fighters to show they were winning the war.

José Miguel Vivanco, who leads the Americas division for Human Rights
Watch, said he raised the problem many times with Mr. Uribe over the
years, but found the former president dismissive, quick to anger and
unwilling to tackle the issue.

``His human rights record is deplorable,'' said Mr. Vivanco.

Mr. Uribe has long denied a connection to paramilitary groups, instead
saying he fought against them.

In an unexpected twist, the investigation that has led to Mr. Uribe's
house arrest examines relatively small-time crimes --- at least when
compared to the crimes at the core of other investigations involving
him.

In the current case, the Supreme Court is examining whether Mr. Uribe
participated in bribery, fraud and witness tampering in an effort to
influence the testimony of an alleged paramilitary member, Juan
Guillermo Monsalve. He is suspected of pushing Mr. Monsalve to retract a
statement in which he linked Mr. Uribe to the creation of paramilitary
groups.

Among the other inquiries into Mr. Uribe's conduct are several that
examine possible connection to paramilitary massacres. His brother
Santiago
\href{https://www.nytimes.com/2018/07/08/world/americas/colombia-uribe-death-squad.html}{has
been charged} for alleged involvement with a paramilitary group.

The former president, who is now a senator, but is likely to be
suspended from that post, has not been formally charged in the case in
question. But the Colombia justice system allows for him to be held as
the investigation continues if judges believe witness tampering could
take place.

If found guilty, the former president could spend approximately six to
eight years in prison, according to the law professor Francisco Bernate.

Mr. Uribe's lawyer, Jaime Granados, denied the charges Wednesday, saying
that ``President Uribe did not ask anyone to bribe any witnesses.''

His supporters, including Mr. Duque, have denounced the detention as
unjust.

``It hurts, as a Colombian,'' Mr. Duque said, that ``an exemplary public
servant, who has occupied the highest post in the state, is not allowed
to defend himself in liberty, with the presumption of innocence.''

Mr. Uribe is now ensconced in a countryside home called El Ubérrimo in
Colombia's north. On Wednesday, people close to him announced that he
had tested positive for Covid-19, adding that he was not in serious
condition.

The home, set on expansive terrain, has a horse track, a swimming pool
and a stable. At the moment his house arrest does not require guards or
police, said the court, but simply requires that he sign a contract and
pay a bond.

Image

Mr. Uribe on the grounds of his home in 2016.Credit...Raul
Arboleda/Agence France-Presse --- Getty Images

Mr. Uribe served as president until 2010, leaving after a court decision
prevented him from running for a third term. But he retains significant
power. Mr. Uribe's support was essential to the victory of Mr. Duque,
who swore to uphold his mentor's legacy.

When the government
\href{https://www.nytimes.com/2016/08/25/world/americas/colombia-farc-peace-deal.html}{reached
an agreement} with the FARC, ending more than five decades of bloody
conflict, many hoped the historic treaty would help heal deep wounds.
But the country's divisions remained strong in the years that followed.

The deal's opponents argued it was too lenient on rebel fighters --- and
were angered that it was passed despite
\href{https://www.nytimes.com/2016/10/03/world/colombia-peace-deal-defeat.html}{a
national vote against it}. And its supporters accuse Mr. Duque of
lacking the will to fully implement it. Hundreds of former fighters and
community leaders have been killed since it was passed, leading critics
to accuse Mr. Duque of failing to protect them. And many rural
communities
\href{https://www.nytimes.com/2019/05/17/world/americas/colombia-farc-peace-deal.html}{are
still awaiting} the roads, schools and electricity that had been
promised.

Among the chief opponents to the terms of the deal was Mr. Uribe, who
thought the accord was too easy on rebel fighters.

His detention, many said this week, reinforced those rifts, fostering
resentment on the right and strengthening the idea on the left that the
former president is a criminal.

``This is an important advance in terms of justice,'' said Francisco
Gutiérrez Sanín, a Colombian political scientist, highlighting the fact
that many of the country's powerful figures have not had to answer to
the justice system. ``But on the other hand it radicalizes and makes
Uribismo more extreme.''

In Medellín this week, Nora Villa, 58, an Uribe loyalist at the support
march, vowed to fight the left. ``We are going to see more division,''
she said.

While in Bogotá, Luz Marina Bernal, 60, an activist whose son, Fair, 26,
was killed by security forces during Mr. Uribe's mandate, said something
about Mr. Uribe that she could not have imagined saying just a few days
ago: ``I think there is a possibility that he will be convicted of all
he has done.''

Reporting was contributed by Jenny Carolina González and Sofía Villamil
in Bogotá and Megan Janetsky in Medellín.

Advertisement

\protect\hyperlink{after-bottom}{Continue reading the main story}

\hypertarget{site-index}{%
\subsection{Site Index}\label{site-index}}

\hypertarget{site-information-navigation}{%
\subsection{Site Information
Navigation}\label{site-information-navigation}}

\begin{itemize}
\tightlist
\item
  \href{https://help.nytimes.com/hc/en-us/articles/115014792127-Copyright-notice}{©~2020~The
  New York Times Company}
\end{itemize}

\begin{itemize}
\tightlist
\item
  \href{https://www.nytco.com/}{NYTCo}
\item
  \href{https://help.nytimes.com/hc/en-us/articles/115015385887-Contact-Us}{Contact
  Us}
\item
  \href{https://www.nytco.com/careers/}{Work with us}
\item
  \href{https://nytmediakit.com/}{Advertise}
\item
  \href{http://www.tbrandstudio.com/}{T Brand Studio}
\item
  \href{https://www.nytimes.com/privacy/cookie-policy\#how-do-i-manage-trackers}{Your
  Ad Choices}
\item
  \href{https://www.nytimes.com/privacy}{Privacy}
\item
  \href{https://help.nytimes.com/hc/en-us/articles/115014893428-Terms-of-service}{Terms
  of Service}
\item
  \href{https://help.nytimes.com/hc/en-us/articles/115014893968-Terms-of-sale}{Terms
  of Sale}
\item
  \href{https://spiderbites.nytimes.com}{Site Map}
\item
  \href{https://help.nytimes.com/hc/en-us}{Help}
\item
  \href{https://www.nytimes.com/subscription?campaignId=37WXW}{Subscriptions}
\end{itemize}
