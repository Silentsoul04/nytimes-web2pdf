Sections

SEARCH

\protect\hyperlink{site-content}{Skip to
content}\protect\hyperlink{site-index}{Skip to site index}

\href{https://www.nytimes.com/section/world/asia}{Asia Pacific}

\href{https://myaccount.nytimes.com/auth/login?response_type=cookie\&client_id=vi}{}

\href{https://www.nytimes.com/section/todayspaper}{Today's Paper}

\href{/section/world/asia}{Asia Pacific}\textbar{}For P.O.W., Landmark
Verdict Against North Korea Is Long-Overdue Justice

\href{https://nyti.ms/3gGATyo}{https://nyti.ms/3gGATyo}

\begin{itemize}
\item
\item
\item
\item
\item
\end{itemize}

Advertisement

\protect\hyperlink{after-top}{Continue reading the main story}

Supported by

\protect\hyperlink{after-sponsor}{Continue reading the main story}

The Saturday Profile

\hypertarget{for-pow-landmark-verdict-against-north-korea-is-long-overdue-justice}{%
\section{For P.O.W., Landmark Verdict Against North Korea Is
Long-Overdue
Justice}\label{for-pow-landmark-verdict-against-north-korea-is-long-overdue-justice}}

A former South Korean soldier may never see a penny of the \$17,600
awarded by a court last month. But the verdict set a precedent for
thousands of others seeking to hold the North and Kim Jong-un
accountable for human rights violations.

\includegraphics{https://static01.nyt.com/images/2020/08/06/world/00korea-pow-1/merlin_175364766_f1dadd0f-7e2b-46f7-a396-89be19128124-articleLarge.jpg?quality=75\&auto=webp\&disable=upscale}

\href{https://www.nytimes.com/by/choe-sang-hun}{\includegraphics{https://static01.nyt.com/images/2018/07/18/multimedia/author-choe-sang-hun/author-choe-sang-hun-thumbLarge.png}}

By \href{https://www.nytimes.com/by/choe-sang-hun}{Choe Sang-Hun}

\begin{itemize}
\item
  Aug. 7, 2020
\item
  \begin{itemize}
  \item
  \item
  \item
  \item
  \item
  \end{itemize}
\end{itemize}

PYEONGTAEK, South Korea --- He was only 17 when Chinese troops backing
North Korea overran a hill being defended by his South Korean Army squad
and took him prisoner in the early hours of Dec. 28, 1951.

He spent the next 40 years toiling in ​North Korean ​coal mines as a
prisoner of the war between the Koreas. ``We P.O.W.s lived inside a
fenced-off camp guarded by armed sentries at four corners and were
escorted to work by officers carrying pistols,'' he said.

``We were nothing but slaves.''

Decades later, the former P.O.W., now 86, scored a landmark legal
victory when the Seoul Central District Court ordered North Korea and
its leader, Kim Jong-un, to pay him the equivalent of \$17,600 in
damages for holding him against his will ​and ​forcing him to work in
the mines. ​ The verdict marked the first time that a court in the South
recognized P.O.W.s who were illegally held in the North --- an
acknowledgment of their suffering there.

In its ruling, the court blocked part of the man's name from the public,
and fearing that North Korea might retaliate against ​his children still
in that country, the former P.O.W. spoke only on the condition that he
be identified by his last name, Han, and that his face be partly
obscured.

There is little chance that North Korea or Mr. Kim will pay what's owed
to Mr. Han. And it could take years for his lawyers to find and
\href{https://www.nbcnews.com/news/us-news/seized-north-korean-cargo-ship-awarded-united-states-n1069761}{confiscate
any North Korean assets}.

Still, for Mr. Han, the verdict was justice served, and justice long
overdue.

``I could not understand the judge's words in the courtroom,'' Mr. Han
said, referring to the unfamiliar legal terms​. ​``But when my lawyer
held my hand and explained that we had won, tears came to my eyes,'' he
recalled in an interview at his two-room apartment in Pyeongtaek, a city
south of Seoul, where he lives with his wife, who has Alzheimer's
disease.

The families of two victims of North Korean torture --- the American
college student
\href{https://www.nytimes.com/2018/12/24/us/otto-warmbier-north-korea.html?searchResultPosition=1}{Otto
Warmbier} and a South Korean minister, the Rev.
\href{https://www.nytimes.com/2015/04/16/world/asia/after-15-years-victory-of-sorts-for-family-of-pastor-believed-abducted-by-north-korea.html}{Kim
Dong-shik} --- previously won court judgments against the North in the
United States​. But Mr. Han's case showed that a similar lawsuit could
be successfully waged in South Korea, where efforts to bring P.O.W.s
home or to hold the North accountable for holding them against their
will have long been considered a lost cause, given the political
tensions between the two Koreas.

The case set a precedent for thousands of South Koreans whose relatives
were kidnapped by ​the ​North or who lost
\href{https://www.nytimes.com/2010/05/20/world/asia/20korea.html}{family
members} or
\href{https://www.nytimes.com/2020/07/17/world/asia/korea-kim-yo-jong-lawsuit.html}{propert}y
to ​the North's ​military.

Mr. Han was one of six children from a farming family in Jeongeup​,
South Korea, when he was taken prisoner. He and two friends from his
village had volunteered for the South Korean Army in the spring of 1951,
less than a year after North Korea invaded the South --- setting off a
war that has not officially ended.

He would spend the next half-century in the North, most of that time
doing backbreaking work in its coal mines.

Over the years, North Korea officials allowed the P.O.W.s to form some
semblance of a life, giving the miners citizenship in 1956 and allowing
them to marry. Mr. Han wed a North Korean woman that year, and together
they had five children.

But the former P.O.W.s from the South and their children were ​assigned
to the bottom of the North's songbun, or class system​, and often given
the most dangerous jobs in the mines. Six days a week, Mr. Han said, he
rode up to half a mile underground into the dark tunnels, where he
toiled 12 hours a day in sweltering heat, with methane gas a constant
hazard. Prisoners who tried to ​escape were hunted down and never heard
from again.

\includegraphics{https://static01.nyt.com/images/2020/08/06/world/00korea-pow-2/merlin_175364772_20c34d3a-6c04-4c27-b84a-7911dc6b23f9-articleLarge.jpg?quality=75\&auto=webp\&disable=upscale}

``When there was a methane gas explosion, we could hardly recognize the
bodies because they were literally cooked in flames,'' Mr. Han said,
recalling the names of P.O.W. friends he had lost. ``​We smelled like
nothing else, working soaking wet with sweat but ​having no time to wash
our clothes.''

When ​an armistice was signed ​ in 1953​ to halt the fighting, 82,000
South Korean soldiers remained missing or were believed to have been
taken prisoner. In 2014, the United Nations'
\href{https://www.ohchr.org/EN/HRBodies/HRC/CoIDPRK/Pages/CommissionInquiryonHRinDPRK.aspx}{Commission
of Inquiry} estimated that at least 50,000 South Korean P.O.W.s were not
repatriated. North Korea returned only 8,300​​, keeping many more for
forced labor in postwar coal mines. Men were in such short supply that
women, including Mr. Han's North Korean mother-in-law, also worked in
the mines.

``We thought that relations between South and North Korea would
improve,'' Yoo Young-bok, a former P.O.W. who escaped the North in 2000,
told the U.N. commission. ``But five decades have passed and nobody came
looking for us and tried to save us.''

North Korea ​has denied holding ​any South Koreans against their will.
The missing soldiers were ​eventually ​counted among the war dead and
largely forgotten in the South. Mr. Han's mother died in 1961 believing
that her son ​had been killed in battle. (His father died before the
war.)

Then, in 1994, an emaciated refugee from North Korea named Cho Chang-ho
was found adrift on a ramshackle wooden boat off South Korea. He turned
out to be a South Korean lieutenant who had survived prison camps and
coal mines in the North. In the South, he was found to have black lung
disease.

​More aging P.O.W.s fled to the South in the following years, as a
famine forced the North to ease control on its people. They all
testified in government debriefings, memoirs, news conferences and
interviews to forced labor, starvation and deaths in North Korean mines
and ​identified hundreds of fellow war prisoners still alive in the
North. ​ Shocked that their long-lost sons and siblings were ​still
​alive, South Korean families wanted to help them flee the North​. Soon,
a cottage industry developed for human traffickers to smuggle refugees
out.

So far, 80 P.O.W.s have made it to South Korea, some later testifying in
court in support of Mr. Han's case --- the seeds of which were planted
by South Korean activists, who suggested the lawsuit in 2016.

Mr. Han, who retired from the Hamyon coal mines at age 60, was living in
Kyongwon, in northeast North Korea, when a man showed up in August 2001,
asking whether he wanted to meet his South Korean relatives. Mr. Han
said he followed the man across the river border to China, his youngest
son tagging along.

Around that time, Han Jae-eun, Mr. Han's youngest brother in South
Korea, got a call from a human trafficker.

``I ​first ​could not tell whether the man was telling the truth or ​it
was a scam,'' said his brother, a taxi driver in Incheon, west of Seoul.
``The brother we all thought was dead more than a half century ago
turned up alive.''

The brothers had a tearful reunion in Hunchun, China, across the border
​from Kyongwon. The younger brother gave Mr. Han what money he had
brought with him​, \$8,000, ​and asked him to decide whether to travel
​on to South Korea or return to ​the North with the money​.

Image

Messages wishing for the reunification of the two Koreas in Paju, near
the border with North Korea, last month. The two Koreas are still
technically at war because the Korean War never formally
ended.Credit...Ahn Young-Joon/Associated Press

Mr. Han said he thought: ``How ​could I live comfortably in the South
while my children and grandchildren​ in the North​ did not even have
enough corn to eat? With the money, I could buy tons of corn in the
North​, but I would never see my brothers again.''

He used the money in November 2001 to smuggle himself, his youngest son,
the son's wife and their two children to the South. There, Mr. Han's old
7th Army Division promoted him to sergeant and formally discharged him.
He ​received his unpaid salary, military pension and other subsidies
that South Korea provides for returning P.O.W.s. He then smuggled his
wife, another son and a daughter ​out of the North.

But his family is still divided. Two sons, a daughter and four
grandchildren live in the South, and a son, a daughter and four
grandchildren in the North. The two Koreas do not allow their citizens
to meet or communicate with one another​, except during
\href{https://www.nytimes.com/2015/10/21/world/asia/south-and-north-koreans-separated-almost-a-lifetime-reunite-briefly.html}{occasional
official family reunions.}

After the court ruling last month, Yoh Sang-key, a spokesman for the
South's Unification Ministry, said the government would ``cooperate with
North Korea and the international community to make concrete progress in
resolving the problem of P.O.W.s.''

Mr. Han's life story epitomized that of thousands of South Korean
P.O.W.s who were abused by North Korea for decades but ignored in their
home country.

The thought haunts him still.

``Think of all those 50,000 men in the North,'' ​Mr. Han said. ``That
was a few army divisions worth of soldiers trapped in the enemy
territory, still in active service because they have never been
discharged. And what have you done for them?''

Advertisement

\protect\hyperlink{after-bottom}{Continue reading the main story}

\hypertarget{site-index}{%
\subsection{Site Index}\label{site-index}}

\hypertarget{site-information-navigation}{%
\subsection{Site Information
Navigation}\label{site-information-navigation}}

\begin{itemize}
\tightlist
\item
  \href{https://help.nytimes.com/hc/en-us/articles/115014792127-Copyright-notice}{©~2020~The
  New York Times Company}
\end{itemize}

\begin{itemize}
\tightlist
\item
  \href{https://www.nytco.com/}{NYTCo}
\item
  \href{https://help.nytimes.com/hc/en-us/articles/115015385887-Contact-Us}{Contact
  Us}
\item
  \href{https://www.nytco.com/careers/}{Work with us}
\item
  \href{https://nytmediakit.com/}{Advertise}
\item
  \href{http://www.tbrandstudio.com/}{T Brand Studio}
\item
  \href{https://www.nytimes.com/privacy/cookie-policy\#how-do-i-manage-trackers}{Your
  Ad Choices}
\item
  \href{https://www.nytimes.com/privacy}{Privacy}
\item
  \href{https://help.nytimes.com/hc/en-us/articles/115014893428-Terms-of-service}{Terms
  of Service}
\item
  \href{https://help.nytimes.com/hc/en-us/articles/115014893968-Terms-of-sale}{Terms
  of Sale}
\item
  \href{https://spiderbites.nytimes.com}{Site Map}
\item
  \href{https://help.nytimes.com/hc/en-us}{Help}
\item
  \href{https://www.nytimes.com/subscription?campaignId=37WXW}{Subscriptions}
\end{itemize}
