Sections

SEARCH

\protect\hyperlink{site-content}{Skip to
content}\protect\hyperlink{site-index}{Skip to site index}

\href{https://www.nytimes.com/section/world/europe}{Europe}

\href{https://myaccount.nytimes.com/auth/login?response_type=cookie\&client_id=vi}{}

\href{https://www.nytimes.com/section/todayspaper}{Today's Paper}

\href{/section/world/europe}{Europe}\textbar{}Europe's `Last Dictator,'
Facing Re-Election, Is Increasingly in Peril

\href{https://nyti.ms/2XCKXB2}{https://nyti.ms/2XCKXB2}

\begin{itemize}
\item
\item
\item
\item
\item
\end{itemize}

Advertisement

\protect\hyperlink{after-top}{Continue reading the main story}

Supported by

\protect\hyperlink{after-sponsor}{Continue reading the main story}

\hypertarget{europes-last-dictator-facing-re-election-is-increasingly-in-peril}{%
\section{Europe's `Last Dictator,' Facing Re-Election, Is Increasingly
in
Peril}\label{europes-last-dictator-facing-re-election-is-increasingly-in-peril}}

For 26 years, Aleksandr G. Lukashenko has ruled Belarus as if it were
his personal fief. In his sixth presidential election, the result is not
in doubt, but he is being challenged like never before.

\includegraphics{https://static01.nyt.com/images/2020/08/07/world/07belarus01/07belarus01-articleLarge-v2.jpg?quality=75\&auto=webp\&disable=upscale}

\href{https://www.nytimes.com/by/ivan-nechepurenko}{\includegraphics{https://static01.nyt.com/images/2018/10/18/multimedia/author-ivan-nechepurenko/author-ivan-nechepurenko-thumbLarge.png}}

By \href{https://www.nytimes.com/by/ivan-nechepurenko}{Ivan
Nechepurenko}

\begin{itemize}
\item
  Published Aug. 7, 2020Updated Aug. 8, 2020, 9:24 a.m. ET
\item
  \begin{itemize}
  \item
  \item
  \item
  \item
  \item
  \end{itemize}
\end{itemize}

MINSK, Belarus --- The man often described as ``Europe's last dictator''
has never looked so shaky.

During his 26 years in power,
\href{https://www.nytimes.com/2020/06/22/world/europe/belarus-lukashenko-russia.html}{Aleksandr
G. Lukashenko} --- the iron-fisted president of Belarus and the
longest-serving leader in the former Soviet Union --- has
\href{https://www.nytimes.com/2017/08/13/world/europe/belarus-russia-aleksandr-lukashenko.html}{danced
between Russia and the West}, alternating praise and blame as he
targeted one side or the other as the reason for his country's and his
own misfortunes.

But as he faces his most difficult challenge yet ahead of a presidential
election on Sunday, Mr. Lukashenko has lost his political balance,
attacking all sides at once as he struggles to explain an upsurge of
popular discontent.

After lashing out at Moscow last week over what he described as a squad
of Russian mercenaries sent to disrupt the election, Mr. Lukashenko on
Thursday claimed that Belarus was under attack from a new team of
saboteurs who could be Americans, might be Ukrainians or perhaps from
Russia.

``A hybrid war is going on against Belarus, and we should expect dirty
tricks from any side,'' he told security officials in Minsk, the
Belarusian capital. ``We don't even know who they are: Americans with
NATO, or someone from Ukraine, or our eastern brothers showing their
affection toward us this way.''

The outcome of Sunday's election is in little doubt: Mr. Lukashenko, 65,
will be declared the winner for a sixth time. But what is usually a
tightly choreographed rite of affirmation has been upset by the largest
protests in Belarus since the collapse of the Soviet Union nearly 30
years ago.

\includegraphics{https://static01.nyt.com/images/2020/08/07/world/07belarus02/merlin_175287702_c07ff097-6312-4be5-9933-fc7bb737bf24-articleLarge.jpg?quality=75\&auto=webp\&disable=upscale}

In the past, Mr. Lukashenko, who commands a large and often brutal
security apparatus, has never been shy about demonstrating that he can
crush any dissent. But this time he seems cornered, with opposition
rallies in Minsk and smaller cities attracting up to tens of thousands
of people.

On Thursday, thousands came out to the Kyiv public garden in Minsk to
support Svetlana G. Tikhanovskaya, a candidate whose platform has
consisted of one point: Get rid of Mr. Lukashenko. People waved, clapped
and shouted of the president, ``Go away!''

``People just lost patience,'' said Nikita, 27, who declined to give his
last name, citing fear of repercussions at his work, a state-run
operation.

Ms. Tikhanovskaya's emergence as a candidate was the result of efforts
by Mr. Lukashenko to clear the ballot of all strong competitors. She was
declared the united opposition candidate last month after the arrest of
her husband, Sergei, who had been a leading opposition contender after
attracting a sizable following largely through a
\href{https://www.youtube.com/channel/UCFPC7r3tWWXWzUIROLx46mg}{YouTube
show} in which he interviewed people in Belarus's provinces.

Another would-be rival, Viktor D. Babariko, the former head of a
Russian-owned bank in Belarus, was also jailed on suspicions of
financial wrongdoing. And the third most popular candidate, Valery V.
Tsepkalo, fled the country last month, saying that he was about to be
detained.

Image

Svetlana Tikhanovskaya, center, on the campaign trail this week in
central Belarus.~Credit...Misha Friedman/Getty Images

The president's mounting troubles, said Aleksandr I. Feduta, his
disenchanted former campaign manager, have left Mr. Lukashenko in a
situation he has never before experienced: almost entirely bereft of
allies outside the security system.

``It is a catastrophe for him,'' Mr. Feduta said. ``He can extend his
rule, but he cannot restore his power.''

The Belarusian economy is faltering in part because of a collapse in oil
prices. Members of the economic and government elite have turned against
Mr. Lukashenko. Tightly controlled media outlets like state television
have lost their grip in the face of vibrant online ones that often
support his opponents. And his response to the coronavirus pandemic has
also left him exposed.

For months, he
\href{https://www.nytimes.com/2020/04/25/world/europe/belarus-lukashenko-coronavirus.html}{denied
that the virus was a serious threat} and ridiculed that idea that it
could be fatal, suggesting that people drink vodka, ride tractors and
frequent a sauna to prevent infection.

At the end of last month, he claimed that he had himself been infected
but suffered no damage to his health. In an
\href{https://www.youtube.com/watch?v=R5UmsPFMUaw\&t=3067s}{interview
with a Ukrainian blogger on Wednesday}, he hinted that he had been
deliberately infected, but didn't specify who would plot against him.

Regarding the opposition rallies, he said in the interview that while
about 20 percent of Belarussians might be against him, most of the
country continued to support his policies.

``I am a realist --- I understand that the pandemic and everything else
have come together,'' he said. ``I don't have any jitters about the
election,'' he added. ``I just won't be comfortable if there will be
brawls on the streets that will need to be dispersed.''

Image

A paramedic checking on a patient in June amid the coronavirus pandemic.
Mr. Lukashenko long denied that the virus was a serious
threat.Credit...Sergei Gapon/Sputnik, via Agence France-Presse --- Getty
Images

One of his supporters, Lyudmila S. Krokhaleva, who is in her early 70s,
said it was also a matter of perspective.

``These people are young and inexperienced. They have nothing to compare
the current situation with,'' she said on Friday before going to a
polling station to give her support to Mr. Lukashenko in early voting.

``There is something you can compare Belarus with --- in the 1990s,
Belarus was in ruins,'' she added. ``Thanks to Mr. Lukashenko, to his
ability to organize and inspire people, we have not lost anything. On
the contrary, we are moving forward.''

Yet many now regard Mr. Lukashenko as weak, Mr. Feduta said, including
the president's own officials and members of law enforcement. ``The main
thing, though, is that Russia sees his weakness, too --- the country
that sponsored his regime.''

Although the two countries are longtime allies supposedly committed to
forming a ``union state,'' Russia and Belarus have been engaged in a
simmering feud for years as the Kremlin has shown
\href{https://www.nytimes.com/2020/02/07/world/europe/belarus-lukashenko-russia-putin.html}{increasing
reluctance to bankroll its smaller neighbor through reduced-price
energy}.

Image

Belarus lost \$400 million last year because of a tax system in Russia
that prevented Minsk from buying oil at reduced rates.Credit...Vasily
Fedosenko/Reuters

Mr. Lukashenko, in turn, has rejected pressure from President Vladimir
V. Putin of Russia to surrender some of his country's sovereignty in
exchange for financial help. Last year alone, Belarus
\href{https://blr.belta.by/economics/view/minfin-belarusi-atsenvae-straty-z-za-padatkovaga-maneuru-u-2020-godze-u-pamery-400-mln-82002-2019/}{lost
\$400 million} because of a Russian oil tax system that prevented Minsk
from buying oil at lower rates and then selling it on to Europe at
market prices.

The souring of the friendship hit a low point last week, when Mr.
Lukashenko
\href{https://www.nytimes.com/2020/07/29/world/europe/belarus-russian-mercenaries-lukashenko.html}{accused
Russia} of sending mercenaries to disrupt his re-election. Thirty-three
Russians were arrested, though the exact purpose of their trip was
unclear. Moscow demanded the release of its citizens, who it said had
simply been passing through Belarus on their way to other countries.

Mr. Lukashenko has also come under pressure from inside his own
government, where some have begun to turn against him.

Mr. Tsepkalo, his former ambassador to Washington and one of the most
popular opposition candidates, who fled to Russia last month, was one of
the highest-ranking members of Mr. Lukashenko's elite to openly abandon
him.

``No one in his immediate surrounding feels that he is part of a team,
and no one sees Mr. Lukashenko as an ally,'' Mr. Tsepkalo, 55, said in
an interview in Moscow. ``He finds himself alone now.''

Image

Belarusians waited outside the country's election commission last month
to file complaints after the commission refused to register two
opposition candidates.Credit...Vasily Fedosenko/Reuters

Yet despite the rising pressure, Mr. Lukashenko controls the electoral
system, which can produce any result he needs.

Before the vote, he also visited a number of military bases and
anti-riot troops. The fence around his residence in Minsk
\href{https://twitter.com/TadeuszGiczan/status/1267553203366899718}{has
been reinforced} with metal shields. And army reservists have been asked
to return to military service.

``In this situation, he will turn more to people in uniforms,'' said
Artyom Shraibman, the founder of Sense-Analytics, a Minsk consulting
firm and research group, said in an interview.

That could be enough for the president to claim yet another election
victory on Sunday. But Mr. Shraibman said Mr. Lukashenko's era was
ending.

``This is clearly the fall season for him,'' Mr. Shraibman said. ``The
question is what month it is ---~October or November?''

Advertisement

\protect\hyperlink{after-bottom}{Continue reading the main story}

\hypertarget{site-index}{%
\subsection{Site Index}\label{site-index}}

\hypertarget{site-information-navigation}{%
\subsection{Site Information
Navigation}\label{site-information-navigation}}

\begin{itemize}
\tightlist
\item
  \href{https://help.nytimes.com/hc/en-us/articles/115014792127-Copyright-notice}{©~2020~The
  New York Times Company}
\end{itemize}

\begin{itemize}
\tightlist
\item
  \href{https://www.nytco.com/}{NYTCo}
\item
  \href{https://help.nytimes.com/hc/en-us/articles/115015385887-Contact-Us}{Contact
  Us}
\item
  \href{https://www.nytco.com/careers/}{Work with us}
\item
  \href{https://nytmediakit.com/}{Advertise}
\item
  \href{http://www.tbrandstudio.com/}{T Brand Studio}
\item
  \href{https://www.nytimes.com/privacy/cookie-policy\#how-do-i-manage-trackers}{Your
  Ad Choices}
\item
  \href{https://www.nytimes.com/privacy}{Privacy}
\item
  \href{https://help.nytimes.com/hc/en-us/articles/115014893428-Terms-of-service}{Terms
  of Service}
\item
  \href{https://help.nytimes.com/hc/en-us/articles/115014893968-Terms-of-sale}{Terms
  of Sale}
\item
  \href{https://spiderbites.nytimes.com}{Site Map}
\item
  \href{https://help.nytimes.com/hc/en-us}{Help}
\item
  \href{https://www.nytimes.com/subscription?campaignId=37WXW}{Subscriptions}
\end{itemize}
