Sections

SEARCH

\protect\hyperlink{site-content}{Skip to
content}\protect\hyperlink{site-index}{Skip to site index}

\href{https://www.nytimes.com/section/opinion/sunday}{Sunday Review}

\href{https://myaccount.nytimes.com/auth/login?response_type=cookie\&client_id=vi}{}

\href{https://www.nytimes.com/section/todayspaper}{Today's Paper}

\href{/section/opinion/sunday}{Sunday Review}\textbar{}No Wrist
Corsages, Please

\href{https://nyti.ms/2PzTG2t}{https://nyti.ms/2PzTG2t}

\begin{itemize}
\item
\item
\item
\item
\item
\item
\end{itemize}

Advertisement

\protect\hyperlink{after-top}{Continue reading the main story}

\href{/section/opinion}{Opinion}

Supported by

\protect\hyperlink{after-sponsor}{Continue reading the main story}

\hypertarget{no-wrist-corsages-please}{%
\section{No Wrist Corsages, Please}\label{no-wrist-corsages-please}}

Has America grown since 1984, or will the knives still be out for
Biden's running mate?

\href{https://www.nytimes.com/by/maureen-dowd}{\includegraphics{https://static01.nyt.com/images/2018/04/02/opinion/maureen-dowd/maureen-dowd-thumbLarge.png}}

By \href{https://www.nytimes.com/by/maureen-dowd}{Maureen Dowd}

Opinion Columnist

\begin{itemize}
\item
  Aug. 8, 2020
\item
  \begin{itemize}
  \item
  \item
  \item
  \item
  \item
  \item
  \end{itemize}
\end{itemize}

\includegraphics{https://static01.nyt.com/images/2020/08/09/opinion/sunday/09Dowd1/09Dowd1-articleLarge.jpg?quality=75\&auto=webp\&disable=upscale}

WASHINGTON --- On the cusp of Joe Biden teaming up with a woman, I am
casting back to my time covering the first woman who was a serious
contender for veep.

The feminist fairy tale --- which began with women crying and popping
champagne on the convention floor in San Francisco in 1984 --- had a sad
ending. Cinderella with ashes in her mouth.

It's hard to fathom, but it took another 36 years for a man to choose to
put a woman on the Democratic ticket with him. To use Geraldine
Ferraro's favorite expression, ``Gimme a break!''

After Walter Mondale picked Ferraro, a Queens congresswoman, the first
man and woman to share a ticket had to consider all sorts of things:
Could he kiss her on the cheek? (No.) Could he call her ``dear'' or
``honey''? (No.) Could they hug? (No.) Could they tell jokes, as Johnny
Carson did, about how angry Joan Mondale would be when her husband kept
coming home late and saying he had been in private sessions with the
vice president? (No.)

They wanted to be seen as peers, more TV anchor team than suburban
couple. Mondale could not seem paternal or patronizing or use phrases
like ``a ticket with broad appeal.'' Ferraro, who walked faster, had to
stop bounding ahead of her running mate.

They knew that the way they conducted themselves would forever recast
the perception of men and women in politics. So they were wary in the
beginning.

As one Democratic consultant put it at the time, ``He looked like a
teenager on the first date with that `How in the world do you pin the
corsage on her?' problem.''

Before a fund-raiser in New York once, a Democratic official presented
Ferraro with
\href{https://www.nytimes.com/1984/10/10/us/ferraro-campaign-perspectives-that-startle.html}{a
wrist corsage}. She refused to put it on. ``That I will not do,'' she
told the man politely.

Sometimes, the introductory music for the petite blonde was the 1925
ditty, ``Five Foot Two, Eyes of Blue.'' One magazine hailed her as
``America's Bride.''

When the ticket headed South, Jim Buck Ross, Mississippi's 70-year-old
commissioner of agriculture, called the 48-year-old Ferraro ``young
lady'' and asked if she could bake blueberry muffins.

Ferraro's historic campaign was full of images never before seen on the
presidential trail. As she went onstage, Gerry, as she was universally
known, would hand off her pocketbook to an aide. Her charming press
spokesman, Francis O'Brien, sometimes ironed her dresses --- as her main
foreign affairs adviser, Madeleine Albright, looked on.

\includegraphics{https://static01.nyt.com/images/2020/08/09/opinion/sunday/09Dowd2/09Dowd2-articleLarge.jpg?quality=75\&auto=webp\&disable=upscale}

It was fascinating to see age-old customs through the eyes of a woman
candidate.

``People hand me their babies,'' Ferraro marveled. ``As a mother, my
instinctive reaction is how do you give your baby to someone who's a
total stranger to kiss, especially with so many colds going around? And
especially when the woman is wearing lipstick?''

It was the first time a candidate running for the White House had talked
about abortion using the phrase, ``If I were pregnant,'' and about
foreign policy with the phrase, ``As the mother of a draft-age son.''
The ``smartass white boys'' around Mondale, as many feminists called
them privately, got nervous when she talked about being a mother. How
could she be tough and a mother, they wondered, not seeing the obvious:
Mothers are tougher than anyone. Fearing white male backlash, they tried
to control her bouncy Queens persona.

Ferraro walked the same tightrope that tripped up Hillary Clinton when
she wondered if she should wheel around in that debate and tell the
creeping Donald Trump to scram.

If she got angry, would she seem shrill, that dread word, and turn off
voters? The Mondale inner circle wanted Ferraro to play the traditional
running-mate role of hatchet man. But Gloria Steinem warned, ``Nothing
makes men more anxious than for a woman to be masculine.''

George H.W. Bush excitedly proclaimed after his debate with Ferraro that
he had tried to
\href{https://www.nytimes.com/1984/10/14/us/aide-to-ferraro-demands-bush-make-apology.html}{``kick
a little ass'';} his press aide called Ferraro ``bitchy''; and Barbara
Bush said Ferraro was a word that ``rhymes with rich.''

What started as a goose bump blind date with history curdled, as Ferraro
got dragged into a financial mess involving her husband's real estate
business.

Right after the Reagan landslide, Democrats began muttering about
returning to white Anglo-Saxon men on the ticket and not having any more
``feminized'' tickets that didn't appeal to them.

I called women across the country for a
\href{https://www.nytimes.com/1984/12/30/magazine/reassessing-women-s-political-role-the-lasting-impact-of-geraldine-ferraro.html}{magazine
autopsy} I was writing and was shocked to hear how ambivalent women
still were about a woman running the country.

A 36-year-old mother of three from Bristol, Tenn., told me: ``I put
myself in her shoes. Could I sit down and logically make decisions for
everybody without cracking up? I think women in general are weak. I know
that sounds awful. But we women know we have our faults.''

The next year, Ferraro put out a memoir talking about how depressed and
paranoid she got, and how much she cried, admitting that she was not
``prepared for the depth of the fury, the bigotry, and the sexism my
candidacy would unleash.''

She said that Mondale's male aides were so condescending that she
instructed them to ``pretend every time they talk to me or even look at
me that I'm a gray-haired Southern gentleman, a senator from Texas.''
(In \href{https://www.nytimes.com/2009/11/15/books/15book.html}{her
memoir}, Sarah Palin aimed her sharpest barbs at John McCain's aides.)

We don't know whom Biden will choose but we do know the sort of hell she
will endure at the hands of Team Trump. Even after the \#MeToo
revolution, even with women deciding this election, have the
undercurrents of sexism in America changed so much? Hollywood, after
all, only just began forking over major budgets to women directors,
after years of absurdly stereotyping them.

Kimberly Guilfoyle, Kellyanne Conway, Kayleigh McEnany, Lara Trump and
Jeanine Pirro --- the Fox Force Five of retrograde Trumpworld --- will
have the knives out. Conservatives will undermine the veep candidate
with stereotypes. She's bitchy. She's a nag. She's aggressive. She's
ambitious. Who's wearing the pants here, anyhow?

I asked Francis O'Brien if he thought, three and a half decades after he
watched the sandstorm of sexism around Ferraro, whether her successor
would have an easier time.

``I think it's the same, in many ways,'' he said. ``This is a white
Anglo-Saxon country founded by white Anglo-Saxon men for white
Anglo-Saxon men. Sexism is like race. It'll pop out. It's in our DNA.
We're one of the few Western countries where women have never made it to
the top.''

But on the bright side, when Chuck Schumer wanted to call Nancy Pelosi a
lioness on Friday, referring to her negotiations with Republicans on the
relief bill, he checked with her first to see if she would prefer lion.

The Speaker chose lioness.

\emph{The Times is committed to publishing}
\href{https://www.nytimes.com/2019/01/31/opinion/letters/letters-to-editor-new-york-times-women.html}{\emph{a
diversity of letters}} \emph{to the editor. We'd like to hear what you
think about this or any of our articles. Here are some}
\href{https://help.nytimes.com/hc/en-us/articles/115014925288-How-to-submit-a-letter-to-the-editor}{\emph{tips}}\emph{.
And here's our email:}
\href{mailto:letters@nytimes.com}{\emph{letters@nytimes.com}}\emph{.}

\emph{Follow The New York Times Opinion section on}
\href{https://www.facebook.com/nytopinion}{\emph{Facebook}}\emph{,}
\href{http://twitter.com/NYTOpinion}{\emph{Twitter (@NYTopinion)}}
\emph{and}
\href{https://www.instagram.com/nytopinion/}{\emph{Instagram}}\emph{.}

Advertisement

\protect\hyperlink{after-bottom}{Continue reading the main story}

\hypertarget{site-index}{%
\subsection{Site Index}\label{site-index}}

\hypertarget{site-information-navigation}{%
\subsection{Site Information
Navigation}\label{site-information-navigation}}

\begin{itemize}
\tightlist
\item
  \href{https://help.nytimes.com/hc/en-us/articles/115014792127-Copyright-notice}{©~2020~The
  New York Times Company}
\end{itemize}

\begin{itemize}
\tightlist
\item
  \href{https://www.nytco.com/}{NYTCo}
\item
  \href{https://help.nytimes.com/hc/en-us/articles/115015385887-Contact-Us}{Contact
  Us}
\item
  \href{https://www.nytco.com/careers/}{Work with us}
\item
  \href{https://nytmediakit.com/}{Advertise}
\item
  \href{http://www.tbrandstudio.com/}{T Brand Studio}
\item
  \href{https://www.nytimes.com/privacy/cookie-policy\#how-do-i-manage-trackers}{Your
  Ad Choices}
\item
  \href{https://www.nytimes.com/privacy}{Privacy}
\item
  \href{https://help.nytimes.com/hc/en-us/articles/115014893428-Terms-of-service}{Terms
  of Service}
\item
  \href{https://help.nytimes.com/hc/en-us/articles/115014893968-Terms-of-sale}{Terms
  of Sale}
\item
  \href{https://spiderbites.nytimes.com}{Site Map}
\item
  \href{https://help.nytimes.com/hc/en-us}{Help}
\item
  \href{https://www.nytimes.com/subscription?campaignId=37WXW}{Subscriptions}
\end{itemize}
