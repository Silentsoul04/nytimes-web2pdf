Sections

SEARCH

\protect\hyperlink{site-content}{Skip to
content}\protect\hyperlink{site-index}{Skip to site index}

\href{https://www.nytimes.com/section/opinion/sunday}{Sunday Review}

\href{https://myaccount.nytimes.com/auth/login?response_type=cookie\&client_id=vi}{}

\href{https://www.nytimes.com/section/todayspaper}{Today's Paper}

\href{/section/opinion/sunday}{Sunday Review}\textbar{}America Could
Control the Pandemic by October. Let's Get to It.

\href{https://nyti.ms/2F6hokX}{https://nyti.ms/2F6hokX}

\begin{itemize}
\item
\item
\item
\item
\item
\item
\end{itemize}

Advertisement

\protect\hyperlink{after-top}{Continue reading the main story}

\href{/section/opinion}{Opinion}

Supported by

\protect\hyperlink{after-sponsor}{Continue reading the main story}

\hypertarget{america-could-control-the-pandemic-by-october-lets-get-to-it}{%
\section{America Could Control the Pandemic by October. Let's Get to
It.}\label{america-could-control-the-pandemic-by-october-lets-get-to-it}}

The solutions to combating the coronavirus are no mystery. It's time to
do this right.

By
\href{https://www.nytimes.com/interactive/opinion/editorialboard.html}{The
Editorial Board}

The editorial board is a group of opinion journalists whose views are
informed by expertise, research, debate and certain longstanding ****
\href{https://www.nytimes.com/interactive/2018/opinion/editorialboard.html}{values}.
It is separate from the newsroom.

\begin{itemize}
\item
  Aug. 8, 2020, 2:28 p.m. ET
\item
  \begin{itemize}
  \item
  \item
  \item
  \item
  \item
  \item
  \end{itemize}
\end{itemize}

\includegraphics{https://static01.nyt.com/images/2020/08/09/opinion/09covid-editorial-08/09covid-editorial-08-articleLarge.jpg?quality=75\&auto=webp\&disable=upscale}

Six to eight weeks. That's how long some of the nation's leading public
health experts say it would take to finally get the United States'
coronavirus epidemic under control. If the country were to take the
right steps, many thousands of people could be spared from the ravages
of Covid-19. The economy could finally begin to repair itself, and
Americans could start to enjoy something more like normal life.

\href{https://www.nejm.org/doi/full/10.1056/NEJMms2024920}{Six to eight
weeks}. For proof, look at
\href{https://www.nytimes.com/2020/08/05/world/europe/germany-coronavirus-test-travelers.html}{Germany}.
Or
\href{https://www.nytimes.com/2020/07/16/world/asia/coronavirus-thailand-photos.html}{Thailand}.
Or
\href{https://www.nytimes.com/2020/06/05/world/europe/coronavirus-france-macron-reopening.html}{France}.
Or nearly any other
\href{https://www.nytimes.com/2020/08/06/us/coronavirus-us.html}{country}
in the world.

In the United States, after a brief period of multistate
curve-flattening, case counts and death tolls are rising in so many
places that Dr. Deborah Birx, the Trump administration's coronavirus
response coordinator,
\href{https://www.nytimes.com/2020/08/02/health/dr-birx-coronavirus-phase.html}{described}
the collective uptick as a sprawling ``new phase'' of the pandemic.
Rural communities are as troubled as urban ones, and even clear
victories over the virus, in places like New York and Massachusetts,
feel imperiled.

At the same time, Americans are fatigued from spending months under
semi-lockdown. Bars and restaurants are reopening in some places, for
indoor service --- and
\href{https://www.nytimes.com/2020/07/23/sunday-review/reopening-schools-coronavirus.html}{debates
are underway} over if and when and how to do the same for schools ---
even as the virus continues to spread unchecked. Long
\href{https://www.nytimes.com/2020/07/23/nyregion/coronavirus-testing-nyc.html}{delays
in testing} have become an accepted norm: It can still take up to two
weeks to get results in some places. As the
\href{https://www.nytimes.com/interactive/2020/us/coronavirus-us-cases.html}{national
death toll} climbs above 160,000, mask wearing is still not universal.

It's no mystery how America got here. The Trump administration's
response has been disjointed and often contradictory, indifferent to
science, suffused with politics and eager to hand off responsibility to
state leaders. Among the states, the response has also been wildly
uneven.

It's also no surprise where the country is headed. Unless something
changes quickly, millions more people will be sickened by the virus, and
well over a million may ultimately die from it. The economy will
\href{https://www.nytimes.com/2020/07/30/business/economy/q2-gdp-coronavirus-economy.html}{contract
further} as new surges of viral spread overwhelm hospitals and force
further shutdowns and compound suffering, especially in low-income
communities and communities of color.

The path to avoiding those outcomes is as clear as the failures of the
past several months.

Scientists have learned a lot about this coronavirus since the first
cases were reported in the United States earlier this year. For
instance, they know now that airborne transmission is a far greater risk
than contaminated surfaces, that the virus spreads through singing and
shouting as much as through coughing, and that while any infected person
is a potential vector,
\href{https://www.nytimes.com/2020/06/30/science/how-coronavirus-spreads.html}{superspreading}
events --- as in
\href{https://www.nytimes.com/interactive/2020/us/coronavirus-nursing-homes.html}{nursing
homes}, meatpacking plants, churches and bars --- are major drivers of
the pandemic. By most estimates, just 10 to 20 percent of coronavirus
infections
\href{https://www.scientificamerican.com/article/how-superspreading-events-drive-most-covid-19-spread1/\#:~:text=In\%20fact\%2C\%20research\%20on\%20actual,percent\%20of\%20the\%20coronavirus's\%20spread.}{account}
for 80 percent of transmissions.

\includegraphics{https://static01.nyt.com/images/2020/08/08/autossell/CovidUpdate_E3V2-thumb/CovidUpdate_E3V2-thumb-videoSixteenByNineJumbo1600.jpg}

Experts have also learned a lot about what it takes to get a coronavirus
outbreak under control. Most of the necessary steps are the same ones
public health experts have been urging for months.

Just because America has largely bungled these steps so far doesn't mean
it can't turn things around. The nation can do better. It must.

\hypertarget{clear-consistent-messaging}{%
\subsubsection{Clear, Consistent
Messaging}\label{clear-consistent-messaging}}

President Trump and his closest advisers have repeatedly contradicted
the scientific evidence, and even themselves, on the severity of the
pandemic and the best ways to respond to it. They've sown confusion on
the importance of
\href{https://www.nytimes.com/2020/07/21/us/politics/trump-coronavirus-masks.html}{mask
wearing}, the dangers of large gatherings, the potential of untested
\href{https://www.nytimes.com/2020/04/26/us/politics/trump-disinfectant-coronavirus.html}{treatments},
the availability of
\href{https://www.nytimes.com/2020/07/31/us/politics/trump-coronavirus-testing.html}{testing}
and the basic matter of who is in charge of what in the pandemic
response.

That confusion seems to have bred a national apathy --- and a dangerous
partisanship over public health measures --- that will be difficult to
undo. But leaders at every level can improve the situation by
coordinating their messaging: Masks are essential and will be required
in all public places. Social distancing is a civic responsibility. The
virus is not going away anytime soon, but we can get it under control
quickly if we work together.

Such messaging works best when it comes from the very top, but state and
local leaders don't have to wait for federal leaders to step up.

\hypertarget{better-use-of-data}{%
\subsubsection{Better Use of Data}\label{better-use-of-data}}

As Dr. Tom Frieden, the former director of the Centers for Disease
Control and Prevention,
\href{https://www.statnews.com/2020/07/21/group-calls-for-standardized-data-collection-to-track-covid19/}{has
noted}: The United States has a glut of data and a dearth of
information.

Data on who is getting sick and where is not being used to guide
interventions, and crucial figures like test result times and the
portion of new cases that were found through contact tracing are
\href{https://preventepidemics.org/covid19/resources/indicators/}{not
consistently} or routinely reported. If scientists had better access to
such figures, they could use it to forecast Covid-19 conditions the same
way they forecast the weather: warning when a given outbreak is
spreading and advising people to adjust their plans accordingly. State
and local leaders can make all their data public, and the C.D.C. ought
to help them get that data into a usable form.

\hypertarget{smarter-shutdowns}{%
\subsubsection{Smarter Shutdowns}\label{smarter-shutdowns}}

In places like
\href{https://www.nytimes.com/2020/08/04/world/australia/coronavirus-melbourne-lockdown.html}{Melbourne},
Australia, and Harris County, Texas, health officials have created
numerical and color-coded threat assessments that tell officials and
citizens exactly what to do, based on how extensively the coronavirus is
spreading in their communities. The highest alert levels call for
full-on shelter in place, while the lowest call for careful monitoring
of high-risk establishments.

It would behoove the C.D.C. to create a similar, evidence-based scale
and work with state and local leaders to employ it in individual
communities. In places where the virus is still rampant, that would mean
much more aggressive shutdowns than have been carried out in the past.
(The United States has not had a true national lockdown, shuttering only
about half the country, compared with 90 percent in other countries with
more successful outbreak control.)

Smarter shutdowns may also mean closing bars and indoor dining in many
places so schools there can reopen more safely; closing meat processing
plants until better protections are in place; and tightening state
borders in a sensible, as-needed fashion.

\hypertarget{testing-tracing-isolation-and-quarantine}{%
\subsubsection{Testing, Tracing, Isolation and
Quarantine}\label{testing-tracing-isolation-and-quarantine}}

The most consistent mantra of experts trying to get the coronavirus
pandemic under control has been that the nation needs much better
testing, tracing, isolation and quarantine protocols. Despite examples
across the globe for how to achieve all four, the United States has
largely failed on these fronts. Testing delays make contact tracing ---
not to mention isolation and quarantine --- impossible to execute.

To resolve the crisis, federal officials need to commandeer the
intellectual property of companies that have developed effective rapid
diagnostics and utilize the
\href{https://www.nytimes.com/2020/07/22/us/politics/coronavirus-defense-production-act.html}{Defense
Production Act} to make and distribute as many of those tests as
possible. As testing is brought up to speed, officials also need to
expand contact tracing and quarantine programs so that once outbreaks
are brought under control, states are prepared to keep them in check.

The causes of America's great pandemic failure run deep, exacerbated by
innumerable longstanding problems, from a weak public health
infrastructure to institutional racism to systemic inequality in health
care, housing and employment. If the pandemic forces the nation to
meaningfully grapple with any of those issues, then perhaps all this
suffering will not have been in vain. But that work can't really begin
until Americans solve the problem that's right in front of them, with
the tools that are already at their disposal.

\emph{The Times is committed to publishing}
\href{https://www.nytimes.com/2019/01/31/opinion/letters/letters-to-editor-new-york-times-women.html}{\emph{a
diversity of letters}} \emph{to the editor. We'd like to hear what you
think about this or any of our articles. Here are some}
\href{https://help.nytimes.com/hc/en-us/articles/115014925288-How-to-submit-a-letter-to-the-editor}{\emph{tips}}\emph{.
And here's our email:}
\href{mailto:letters@nytimes.com}{\emph{letters@nytimes.com}}\emph{.}

\emph{Follow The New York Times Opinion section on}
\href{https://www.facebook.com/nytopinion}{\emph{Facebook}}\emph{,}
\href{http://twitter.com/NYTOpinion}{\emph{Twitter (@NYTopinion)}}
\emph{and}
\href{https://www.instagram.com/nytopinion/}{\emph{Instagram}}\emph{.}

Advertisement

\protect\hyperlink{after-bottom}{Continue reading the main story}

\hypertarget{site-index}{%
\subsection{Site Index}\label{site-index}}

\hypertarget{site-information-navigation}{%
\subsection{Site Information
Navigation}\label{site-information-navigation}}

\begin{itemize}
\tightlist
\item
  \href{https://help.nytimes.com/hc/en-us/articles/115014792127-Copyright-notice}{©~2020~The
  New York Times Company}
\end{itemize}

\begin{itemize}
\tightlist
\item
  \href{https://www.nytco.com/}{NYTCo}
\item
  \href{https://help.nytimes.com/hc/en-us/articles/115015385887-Contact-Us}{Contact
  Us}
\item
  \href{https://www.nytco.com/careers/}{Work with us}
\item
  \href{https://nytmediakit.com/}{Advertise}
\item
  \href{http://www.tbrandstudio.com/}{T Brand Studio}
\item
  \href{https://www.nytimes.com/privacy/cookie-policy\#how-do-i-manage-trackers}{Your
  Ad Choices}
\item
  \href{https://www.nytimes.com/privacy}{Privacy}
\item
  \href{https://help.nytimes.com/hc/en-us/articles/115014893428-Terms-of-service}{Terms
  of Service}
\item
  \href{https://help.nytimes.com/hc/en-us/articles/115014893968-Terms-of-sale}{Terms
  of Sale}
\item
  \href{https://spiderbites.nytimes.com}{Site Map}
\item
  \href{https://help.nytimes.com/hc/en-us}{Help}
\item
  \href{https://www.nytimes.com/subscription?campaignId=37WXW}{Subscriptions}
\end{itemize}
