Sections

SEARCH

\protect\hyperlink{site-content}{Skip to
content}\protect\hyperlink{site-index}{Skip to site index}

\href{https://www.nytimes.com/section/opinion/sunday}{Sunday Review}

\href{https://myaccount.nytimes.com/auth/login?response_type=cookie\&client_id=vi}{}

\href{https://www.nytimes.com/section/todayspaper}{Today's Paper}

\href{/section/opinion/sunday}{Sunday Review}\textbar{}What to Do When
Covid Doesn't Go Away

\href{https://nyti.ms/3gBqbJs}{https://nyti.ms/3gBqbJs}

\begin{itemize}
\item
\item
\item
\item
\item
\item
\end{itemize}

Advertisement

\protect\hyperlink{after-top}{Continue reading the main story}

\href{/section/opinion}{Opinion}

Supported by

\protect\hyperlink{after-sponsor}{Continue reading the main story}

\hypertarget{what-to-do-when-covid-doesnt-go-away}{%
\section{What to Do When Covid Doesn't Go
Away}\label{what-to-do-when-covid-doesnt-go-away}}

Lessons for coronavirus long-haulers from my own experience with chronic
illness.

\href{https://www.nytimes.com/by/ross-douthat}{\includegraphics{https://static01.nyt.com/images/2018/04/03/opinion/ross-douthat/ross-douthat-thumbLarge.png}}

By \href{https://www.nytimes.com/by/ross-douthat}{Ross Douthat}

Opinion Columnist

\begin{itemize}
\item
  Aug. 8, 2020, 3:27 p.m. ET
\item
  \begin{itemize}
  \item
  \item
  \item
  \item
  \item
  \item
  \end{itemize}
\end{itemize}

\includegraphics{https://static01.nyt.com/images/2020/08/09/opinion/sunday/09Douthat/09Douthat-articleLarge.jpg?quality=75\&auto=webp\&disable=upscale}

Among the many things that nobody knows about the disease that has
overturned our lives is how long its effects last. I don't just mean the
possibility of coronavirus damage lurking invisibly in the heart or
lungs or brain. I mean the simpler question of what it takes, and how
long, for some uncertain percentage of the sick to actually feel better.

Two months ago Ed Yong of The Atlantic
\href{https://www.theatlantic.com/health/archive/2020/06/covid-19-coronavirus-longterm-symptoms-months/612679/}{reported}
on Covid's ``long-haulers'' --- people who are sick for months rather
than the two or three weeks that's supposed to be the norm. They don't
just have persistent coughs: Instead their disease is a systemic
experience, with brain fog, internal organ pain, bowel problems,
tremors, relapsing fevers, more.

One of Yong's subjects, a New Yorker named Hannah Davis, was on Day 71
when his story appeared. When she passed the four-month mark, in late
July, she
\href{https://twitter.com/ahandvanish/status/1287525539859910657}{tweeted}
a list of symptoms that included everything from ``phantom smells (like
someone BBQing bad meat)'' to ``sensitivity to noise and light'' to
``extreme back/kidney/rib pain'' to ``a feeling like my body has
forgotten to breathe.''

That same week, the Centers for Disease Control and Prevention released
a
\href{https://www.nbcnews.com/health/health-news/monumental-acknowledgment-cdc-reports-long-term-covid-19-patients-n1234814}{survey}
of Covid patients who were never sick enough to be hospitalized. One in
three reported still feeling sick three weeks into the disease.

I was probably a long-hauler, under this definition. My whole family was
sick in March
\href{https://www.nytimes.com/2020/03/24/opinion/coronavirus-testing.html}{with
Covid-like symptoms}, and though the one test we obtained was negative,
I'm pretty sure we had the thing itself --- and my own symptoms took
months rather than weeks to disappear.

But unlike many of the afflicted, I didn't find the experience
particularly shocking, because I have a prior long-haul experience of my
own. In the spring of 2015, I was bitten by a deer tick, and the effects
of the subsequent illness --- a combination of Lyme disease and a more
obscure tick-borne infection, Bartonella --- have been with me ever
since.

Lyme disease in its chronic form --- or, per official medical parlance,
``post-treatment Lyme disease syndrome'' --- is a fiendishly complicated
and
\href{https://www.newyorker.com/magazine/2013/07/01/the-lyme-wars}{controversial
subject}, and what I learned from the experience would (and will, at
some point) fill a book.

But there are a few lessons that are worth passing along to anyone whose
encounter with the pandemic of 2020 has left them feeling permanently
transformed for the worse.

\hypertarget{impatience-is-your-friend}{%
\subsubsection{\texorpdfstring{\textbf{Impatience is your
friend.}}{Impatience is your friend.}}\label{impatience-is-your-friend}}

With most illnesses, \emph{get some rest and drink fluids and you'll
probably feel better} is excellent advice, which is why doctors offer it
so consistently. But if you don't feel better after a reasonable
duration, then you shouldn't just try to endure stoically while hoping
that maybe you're making microscopic progress. (I lost months to my own
illness taking that approach.) If you feel like you need something else
to get better, some outside intervention, something more than just your
own beleaguered body's resources, be impatient --- and find a way to go
in search of it.

\hypertarget{if-your-doctor-struggles-to-help-you-youll-need-to-help-yourself}{%
\subsubsection{\texorpdfstring{\textbf{If your doctor struggles to help
you, you'll need to help
yourself.}}{If your doctor struggles to help you, you'll need to help yourself.}}\label{if-your-doctor-struggles-to-help-you-youll-need-to-help-yourself}}

Modern medicine works marvels, but it's built to treat acute conditions
and well-known diseases. A completely novel virus that seems to hang
around for months is neither. Add in all the other burdens on the
medical system at the moment, and the understandable focus on the most
life-threatening Covid cases, and it may be extremely difficult to find
a doctor who can guide and support a labyrinthine recovery process. So
to some uncertain extent, you may need to become your own doctor --- or
if you're too sick for that, to find someone who can help you on your
journey, notwithstanding the absence of an M.D. beside their name.

\hypertarget{trust-your-own-experience-of-your-body}{%
\subsubsection{\texorpdfstring{\textbf{Trust your own experience of your
body.}}{Trust your own experience of your body.}}\label{trust-your-own-experience-of-your-body}}

Yong's Atlantic piece notes that many Covid long-haulers ``have been
frustrated by their friends' and families' inability to process a
prolonged illness'' and have dealt with skepticism from doctors as well.
In such circumstances, it's natural to doubt yourself as well, and to
think \emph{maybe it really is all in my head.}

In some cases, presumably, it is: Hypochondria certainly exists, and the
combination of high anxiety and pandemic headlines no doubt inspires
some phantom illnesses. But for a field officially grounded in hard
materialism, contemporary medicine is far too quick to retreat to a kind
of mysterianism, a hand-waving about mind-body connections, when it
comes to chronic illnesses that we can't yet treat. If you don't have a
history of imagined illness, if you were generally healthy up until a
few months ago, if your body felt normal and now it feels invaded, you
should have a reasonable level of trust that it isn't just ``in your
head'' --- that you're dealing with a real infection or immune response,
not some miasma in your subconscious.

\hypertarget{experiment-experiment-experiment}{%
\subsubsection{\texorpdfstring{\textbf{Experiment, experiment,
experiment.}}{Experiment, experiment, experiment.}}\label{experiment-experiment-experiment}}

There is no treatment yet for ``long haul'' Covid that meets the
standard of a randomized, double-blind, placebo-controlled trial, which
means that the F.D.A.-stamped medical consensus can't be your only guide
if you're trying to break a systemic, debilitating curse. The realm
beyond that consensus has, yes, plenty of quacks, perils and overpriced
placebos. But it also includes treatments that may help you --- starting
with the most basic herbs and vitamins, and expanding into things that,
well, let's just say I wouldn't have \emph{ever} imagined myself trying
before I become ill myself.

So please don't drink bleach, or believe everything you read on
Goop.com. But if you find yourself decanting Chinese tinctures, or lying
on a chiropractor's table with magnets placed strategically around your
body, or listening to an
``\href{https://open.spotify.com/album/6ypeVM7NafQtp8IX3ZpctL}{Anti-Coronavirus
Frequency}'' on Spotify, and you think, \emph{how did I end up here?},
know that you aren't alone, and you aren't being irrational. The
irrational thing is to be sick, to have no official treatment available,
and to fear the outré or strange more than you fear the permanence of
your disease.

\hypertarget{the-internet-is-your-friend}{%
\subsubsection{\texorpdfstring{\textbf{The internet is your
friend.}}{The internet is your friend.}}\label{the-internet-is-your-friend}}

For experimental purposes, that is. My profession is obsessed,
understandably, with the dangers of online Covid misinformation. But the
internet also creates communities of shared medical experience, where
you can sift testimonies from fellow sufferers who have tried different
approaches, different doctors, different regimens. For now, that kind of
collective offers a crowdsourced empiricism, an imperfect but still
evidence-based guide to treatment possibilities. Use it carefully, but
use it.

\hypertarget{ask-god-to-help-you-and-keep-asking-when-he-doesnt-seem-to-answer}{%
\subsubsection{\texorpdfstring{\textbf{Ask God to help you. And keep
asking when He doesn't seem to
answer.}}{Ask God to help you. And keep asking when He doesn't seem to answer.}}\label{ask-god-to-help-you-and-keep-asking-when-he-doesnt-seem-to-answer}}

I mean this very seriously.

\hypertarget{you-can-get-better}{%
\subsubsection{\texorpdfstring{\textbf{You can get
better.}}{You can get better.}}\label{you-can-get-better}}

I said earlier that my own illness is still with me five years later.
But not in anything like the same way. I was wrecked, destroyed,
despairing. Now I'm better, substantially better --- and I believe that
with enough time and experimentation, I will actually be well.

That belief is essential. Hold on to it. In the long haul, it may see
you through.

\emph{The Times is committed to publishing}
\href{https://www.nytimes.com/2019/01/31/opinion/letters/letters-to-editor-new-york-times-women.html}{\emph{a
diversity of letters}} \emph{to the editor. We'd like to hear what you
think about this or any of our articles. Here are some}
\href{https://help.nytimes.com/hc/en-us/articles/115014925288-How-to-submit-a-letter-to-the-editor}{\emph{tips}}\emph{.
And here's our email:}
\href{mailto:letters@nytimes.com}{\emph{letters@nytimes.com}}\emph{.}

\emph{Follow The New York Times Opinion section on}
\href{https://www.facebook.com/nytopinion}{\emph{Facebook}}\emph{,}
\href{http://twitter.com/NYTOpinion}{\emph{Twitter (@NYTOpinion)}}
\emph{and}
\href{https://www.instagram.com/nytopinion/}{\emph{Instagram}}\emph{,
join the Facebook political discussion group,}
\href{https://www.facebook.com/groups/votingwhilefemale/}{\emph{Voting
While Female}}\emph{.}

Advertisement

\protect\hyperlink{after-bottom}{Continue reading the main story}

\hypertarget{site-index}{%
\subsection{Site Index}\label{site-index}}

\hypertarget{site-information-navigation}{%
\subsection{Site Information
Navigation}\label{site-information-navigation}}

\begin{itemize}
\tightlist
\item
  \href{https://help.nytimes.com/hc/en-us/articles/115014792127-Copyright-notice}{©~2020~The
  New York Times Company}
\end{itemize}

\begin{itemize}
\tightlist
\item
  \href{https://www.nytco.com/}{NYTCo}
\item
  \href{https://help.nytimes.com/hc/en-us/articles/115015385887-Contact-Us}{Contact
  Us}
\item
  \href{https://www.nytco.com/careers/}{Work with us}
\item
  \href{https://nytmediakit.com/}{Advertise}
\item
  \href{http://www.tbrandstudio.com/}{T Brand Studio}
\item
  \href{https://www.nytimes.com/privacy/cookie-policy\#how-do-i-manage-trackers}{Your
  Ad Choices}
\item
  \href{https://www.nytimes.com/privacy}{Privacy}
\item
  \href{https://help.nytimes.com/hc/en-us/articles/115014893428-Terms-of-service}{Terms
  of Service}
\item
  \href{https://help.nytimes.com/hc/en-us/articles/115014893968-Terms-of-sale}{Terms
  of Sale}
\item
  \href{https://spiderbites.nytimes.com}{Site Map}
\item
  \href{https://help.nytimes.com/hc/en-us}{Help}
\item
  \href{https://www.nytimes.com/subscription?campaignId=37WXW}{Subscriptions}
\end{itemize}
