Sections

SEARCH

\protect\hyperlink{site-content}{Skip to
content}\protect\hyperlink{site-index}{Skip to site index}

\href{https://www.nytimes.com/section/health}{Health}

\href{https://myaccount.nytimes.com/auth/login?response_type=cookie\&client_id=vi}{}

\href{https://www.nytimes.com/section/todayspaper}{Today's Paper}

\href{/section/health}{Health}\textbar{}Surfaces? Sneezes? Sex? How the
Coronavirus Can and Cannot Spread

\url{https://nyti.ms/2y13VqV}

\begin{itemize}
\item
\item
\item
\item
\item
\end{itemize}

\href{https://www.nytimes.com/news-event/coronavirus?action=click\&pgtype=Article\&state=default\&region=TOP_BANNER\&context=storylines_menu}{The
Coronavirus Outbreak}

\begin{itemize}
\tightlist
\item
  live\href{https://www.nytimes.com/2020/08/04/world/coronavirus-cases.html?action=click\&pgtype=Article\&state=default\&region=TOP_BANNER\&context=storylines_menu}{Latest
  Updates}
\item
  \href{https://www.nytimes.com/interactive/2020/us/coronavirus-us-cases.html?action=click\&pgtype=Article\&state=default\&region=TOP_BANNER\&context=storylines_menu}{Maps
  and Cases}
\item
  \href{https://www.nytimes.com/interactive/2020/science/coronavirus-vaccine-tracker.html?action=click\&pgtype=Article\&state=default\&region=TOP_BANNER\&context=storylines_menu}{Vaccine
  Tracker}
\item
  \href{https://www.nytimes.com/2020/08/02/us/covid-college-reopening.html?action=click\&pgtype=Article\&state=default\&region=TOP_BANNER\&context=storylines_menu}{College
  Reopening}
\item
  \href{https://www.nytimes.com/live/2020/08/04/business/stock-market-today-coronavirus?action=click\&pgtype=Article\&state=default\&region=TOP_BANNER\&context=storylines_menu}{Economy}
\end{itemize}

Advertisement

\protect\hyperlink{after-top}{Continue reading the main story}

Supported by

\protect\hyperlink{after-sponsor}{Continue reading the main story}

\hypertarget{surfaces-sneezes-sex-how-the-coronavirus-can-and-cannot-spread}{%
\section{Surfaces? Sneezes? Sex? How the Coronavirus Can and Cannot
Spread}\label{surfaces-sneezes-sex-how-the-coronavirus-can-and-cannot-spread}}

What you need to know about how the virus is transmitted.

\includegraphics{https://static01.nyt.com/images/2020/02/27/world/27xp-spread-pix3/27xp-spread-pix3-articleLarge.jpg?quality=75\&auto=webp\&disable=upscale}

By \href{https://www.nytimes.com/by/heather-murphy}{Heather Murphy}

\begin{itemize}
\item
  May 26, 2020
\item
  \begin{itemize}
  \item
  \item
  \item
  \item
  \item
  \end{itemize}
\end{itemize}

\href{https://cn.nytimes.com/health/20200303/coronavirus-how-it-spreads/}{阅读简体中文版}\href{https://cn.nytimes.com/health/20200303/coronavirus-how-it-spreads/zh-hant/}{閱讀繁體中文版}\href{https://www.nytimes.com/es/2020/03/03/espanol/ciencia-y-tecnologia/coronavirus-como-se-transmite.html}{Leer
en español}

\emph{{[}This article is part of the developing coronavirus coverage,
and may be outdated.}
\href{https://www.nytimes.com/news-event/coronavirus}{\emph{Go here for
the latest on the coronavirus}}\emph{.{]}}

A delicate but highly contagious virus, roughly one-900th the width of a
human hair, is spreading from person to person around the world. The
\href{https://www.nytimes.com/2020/03/03/world/coronavirus-news.html}{coronavirus},
as it's known, has already
\href{https://www.nytimes.com/interactive/2020/world/coronavirus-maps.html}{infected
more than 200,000 people in 140 countries}.

Because this virus is so new, experts' understanding of how it spreads
is limited. They can, however, offer some guidance about how it does ---
and does not --- seem to be transmitted.

\hypertarget{if-i-cross-paths-with-a-sick-person-will-i-get-sick-too}{%
\subsection{If I cross paths with a sick person, will I get sick,
too?}\label{if-i-cross-paths-with-a-sick-person-will-i-get-sick-too}}

You walk into a crowded grocery store. A shopper has the coronavirus.
What puts you most at risk of getting infected by that person?

Experts agree they have a great deal to learn, but four factors are
likely to play some role: how close you get; how long you are near the
person; whether that person projects viral droplets on you; and how much
you touch your face. (Of course,
\href{https://www.nytimes.com/2020/02/20/health/coronavirus-men-women.html}{your
age and health} are also major factors.)

Also, the larger the number of people in the store --- or in any other
situation --- the greater the chance that you'll cross paths with an
infected person, which is why so many health officials are now urging
people to avoid crowds and to
\href{https://www.cdc.gov/coronavirus/2019-ncov/community/large-events/mass-gatherings-ready-for-covid-19.html}{cancel
gatherings} large and small.

\hypertarget{whats-a-viral-droplet}{%
\subsection{What's a viral droplet?}\label{whats-a-viral-droplet}}

It is a droplet containing viral particles. A virus is a tiny
codependent microbe that attaches to a cell, takes over, makes more of
itself and moves on to its next host. This is its ``lifestyle,'' said
Gary Whittaker, a professor of virology at the Cornell University
College of Veterinary Medicine.

A ``naked'' virus can't go anywhere unless it's hitching a ride with a
droplet of mucus or saliva, said Kin-on Kwok, a professor at the Jockey
Club School of Public Health and Primary Care at the Chinese University
of Hong Kong.

These mucus and saliva droplets are ejected from the mouth or the nose
as we cough, sneeze, laugh, sing, breathe and talk. If they don't hit
something along the way, they typically land on the floor or the ground.
When the virus becomes suspended in droplets smaller than five
micrometers --- known as aerosols --- it can stay suspended for about a
half-hour,
\href{https://www.nytimes.com/2020/03/17/health/coronavirus-surfaces-aerosols.html}{research
suggests}.

\hypertarget{latest-updates-global-coronavirus-outbreak}{%
\section{\texorpdfstring{\href{https://www.nytimes.com/2020/08/04/world/coronavirus-cases.html?action=click\&pgtype=Article\&state=default\&region=MAIN_CONTENT_1\&context=storylines_live_updates}{Latest
Updates: Global Coronavirus
Outbreak}}{Latest Updates: Global Coronavirus Outbreak}}\label{latest-updates-global-coronavirus-outbreak}}

Updated 2020-08-05T06:48:23.151Z

\begin{itemize}
\tightlist
\item
  \href{https://www.nytimes.com/2020/08/04/world/coronavirus-cases.html?action=click\&pgtype=Article\&state=default\&region=MAIN_CONTENT_1\&context=storylines_live_updates\#link-762df92}{As
  talks drag on, McConnell signals openness to jobless aid extension,
  and negotiators agree on a deadline.}
\item
  \href{https://www.nytimes.com/2020/08/04/world/coronavirus-cases.html?action=click\&pgtype=Article\&state=default\&region=MAIN_CONTENT_1\&context=storylines_live_updates\#link-1228a480}{Novavax
  sees encouraging results from two studies of its experimental
  vaccine.}
\item
  \href{https://www.nytimes.com/2020/08/04/world/coronavirus-cases.html?action=click\&pgtype=Article\&state=default\&region=MAIN_CONTENT_1\&context=storylines_live_updates\#link-794484ed}{Mississippians
  must now wear masks in public, governor says.}
\end{itemize}

\href{https://www.nytimes.com/2020/08/04/world/coronavirus-cases.html?action=click\&pgtype=Article\&state=default\&region=MAIN_CONTENT_1\&context=storylines_live_updates}{See
more updates}

More live coverage:
\href{https://www.nytimes.com/live/2020/08/04/business/stock-market-today-coronavirus?action=click\&pgtype=Article\&state=default\&region=MAIN_CONTENT_1\&context=storylines_live_updates}{Markets}

To gain access to your cells, the viral droplets must enter through the
eyes, the nose or the mouth. Some experts believe that sneezing and
coughing are most likely the primary forms of transmission. Professor
Kwok said talking face-to-face or sharing a meal with someone could pose
a risk.

Julian Tang, a virologist and a professor at the University of Leicester
in England who is researching the coronavirus with Professor Kwok,
agreed.

``If you can smell what someone had for lunch --- garlic, curry, etc.
--- you are inhaling what they are breathing out, including any virus in
their breath,'' he said.

The virus does not linger in the air at high enough levels to be a risk
to most people. But the techniques health care workers use to care for
sick people can generate high levels of aerosols. This is part of why
it's so important that they have
\href{https://www.nytimes.com/interactive/2020/03/11/us/virus-health-workers.html}{proper
protective equipment.}

\hypertarget{how-close-is-too-close}{%
\subsection{How close is too close?}\label{how-close-is-too-close}}

The
\href{https://www.cdc.gov/coronavirus/2019-ncov/hcp/clinical-criteria.html}{Centers
for Disease Control and Prevention} recommends keeping a distance of six
feet from other people to minimize the possibility of infection. (A
useful way to think about six feet is that it's roughly twice the length
of the average person's extended arm.)

Three feet is the distance the W.H.O. emphasizes as particularly risky
when standing near a person who is coughing or sneezing.

Still, other public health experts say that at this crucial moment, when
the world still has an opportunity to
\href{https://www.nytimes.com/interactive/2020/03/19/world/coronavirus-flatten-the-curve-countries.html}{slow
the transmission} of the coronavirus, any number of feet is too close.
By cutting out all but essential in-person interactions, we can help
\href{https://www.nytimes.com/2020/03/11/science/coronavirus-curve-mitigation-infection.html}{flatten
the curve}, they say, keeping the number of sick people to levels that
medical providers can manage.

\hypertarget{how-long-is-too-long-to-be-near-an-infected-person}{%
\subsection{How long is too long to be near an infected
person?}\label{how-long-is-too-long-to-be-near-an-infected-person}}

It's not yet clear, but most experts agree that more time equals more
risk.

\hypertarget{will-you-know-a-person-is-sick}{%
\subsection{Will you know a person is
sick?}\label{will-you-know-a-person-is-sick}}

Not necessarily.

\href{https://www.nytimes.com/article/coronavirus-body-symptoms.html}{Fever,
coughing, chest pain and shortness of breath} may signal that someone
has been infected with the coronavirus. (Covid-19 is the name for the
disease caused by the virus.)

But it has become increasingly clear that
\href{https://www.nytimes.com/2020/02/26/health/coronavirus-asymptomatic.html}{people
without symptoms can also infect others}. In some cases, these people
may later feel terrible enough to
\href{https://www.nytimes.com/2020/03/18/nyregion/coronavirus-testing-positive.html}{try
to get tested}, isolate themselves,
\href{https://www.nytimes.com/2020/03/17/science/coronavirus-treatment.html}{seek
treatment} and notify friends and colleagues about potential risk. In
still other cases, people with the virus may never experience the
physical discomfort that would tip them off to the fact that they have
been a danger to others.

\includegraphics{https://static01.nyt.com/images/2020/03/02/multimedia/02xp-spread3/merlin_169500543_c7f77cb5-99dd-4495-8aed-dbb65fc11be4-articleLarge.jpg?quality=75\&auto=webp\&disable=upscale}

\hypertarget{can-the-virus-last-on-a-bus-pole-a-touch-screen-or-other-surface}{%
\subsection{Can the virus last on a bus pole, a touch screen or other
surface?}\label{can-the-virus-last-on-a-bus-pole-a-touch-screen-or-other-surface}}

Yes. After numerous people who attended a Buddhist temple in Hong Kong
fell ill, the city's Center for Health Protection collected samples from
the site. Restroom faucets and the cloth covers over Buddhist texts
tested positive for the coronavirus, the agency said.

This coronavirus is just the latest of many similarly shaped viruses.
(Coronaviruses are
\href{https://www.nytimes.com/article/what-is-coronavirus.html}{named
for the spikes} that protrude from their surfaces, which resemble a
crown or the sun's corona.)

A
\href{https://www.nytimes.com/2020/03/17/health/coronavirus-surfaces-aerosols.html}{recent
study} of the novel coronavirus found that it could live for three days
on plastic and steel. If you are ordering lots of supplies online, you
may be relieved to know that the virus did poorly on cardboard --- it
disintegrated over the course of a day. It survived for about four hours
on copper.

Whether a surface looks dirty or clean is irrelevant. If an infected
person sneezed and a droplet landed on a surface, a person who then
touched that surface could pick it up. How much is required to infect a
person is unclear.

But as long as you
\href{https://www.nytimes.com/2016/04/21/health/washing-hands.html}{wash
your hands} before
\href{https://www.nytimes.com/2020/03/02/well/live/coronavirus-spread-transmission-face-touching-hands.html}{touching
your face,} you should be OK, because viral droplets don't pass through
skin.

\href{https://www.nytimes.com/news-event/coronavirus?action=click\&pgtype=Article\&state=default\&region=MAIN_CONTENT_3\&context=storylines_faq}{}

\hypertarget{the-coronavirus-outbreak-}{%
\subsubsection{The Coronavirus Outbreak
›}\label{the-coronavirus-outbreak-}}

\hypertarget{frequently-asked-questions}{%
\paragraph{Frequently Asked
Questions}\label{frequently-asked-questions}}

Updated August 4, 2020

\begin{itemize}
\item ~
  \hypertarget{i-have-antibodies-am-i-now-immune}{%
  \paragraph{I have antibodies. Am I now
  immune?}\label{i-have-antibodies-am-i-now-immune}}

  \begin{itemize}
  \tightlist
  \item
    As of right
    now,\href{https://www.nytimes.com/2020/07/22/health/covid-antibodies-herd-immunity.html?action=click\&pgtype=Article\&state=default\&region=MAIN_CONTENT_3\&context=storylines_faq}{that
    seems likely, for at least several months.} There have been
    frightening accounts of people suffering what seems to be a second
    bout of Covid-19. But experts say these patients may have a
    drawn-out course of infection, with the virus taking a slow toll
    weeks to months after initial exposure. People infected with the
    coronavirus typically
    \href{https://www.nature.com/articles/s41586-020-2456-9}{produce}
    immune molecules called antibodies, which are
    \href{https://www.nytimes.com/2020/05/07/health/coronavirus-antibody-prevalence.html?action=click\&pgtype=Article\&state=default\&region=MAIN_CONTENT_3\&context=storylines_faq}{protective
    proteins made in response to an
    infection}\href{https://www.nytimes.com/2020/05/07/health/coronavirus-antibody-prevalence.html?action=click\&pgtype=Article\&state=default\&region=MAIN_CONTENT_3\&context=storylines_faq}{.
    These antibodies may} last in the body
    \href{https://www.nature.com/articles/s41591-020-0965-6}{only two to
    three months}, which may seem worrisome, but that's perfectly normal
    after an acute infection subsides, said Dr. Michael Mina, an
    immunologist at Harvard University. It may be possible to get the
    coronavirus again, but it's highly unlikely that it would be
    possible in a short window of time from initial infection or make
    people sicker the second time.
  \end{itemize}
\item ~
  \hypertarget{im-a-small-business-owner-can-i-get-relief}{%
  \paragraph{I'm a small-business owner. Can I get
  relief?}\label{im-a-small-business-owner-can-i-get-relief}}

  \begin{itemize}
  \tightlist
  \item
    The
    \href{https://www.nytimes.com/article/small-business-loans-stimulus-grants-freelancers-coronavirus.html?action=click\&pgtype=Article\&state=default\&region=MAIN_CONTENT_3\&context=storylines_faq}{stimulus
    bills enacted in March} offer help for the millions of American
    small businesses. Those eligible for aid are businesses and
    nonprofit organizations with fewer than 500 workers, including sole
    proprietorships, independent contractors and freelancers. Some
    larger companies in some industries are also eligible. The help
    being offered, which is being managed by the Small Business
    Administration, includes the Paycheck Protection Program and the
    Economic Injury Disaster Loan program. But lots of folks have
    \href{https://www.nytimes.com/interactive/2020/05/07/business/small-business-loans-coronavirus.html?action=click\&pgtype=Article\&state=default\&region=MAIN_CONTENT_3\&context=storylines_faq}{not
    yet seen payouts.} Even those who have received help are confused:
    The rules are draconian, and some are stuck sitting on
    \href{https://www.nytimes.com/2020/05/02/business/economy/loans-coronavirus-small-business.html?action=click\&pgtype=Article\&state=default\&region=MAIN_CONTENT_3\&context=storylines_faq}{money
    they don't know how to use.} Many small-business owners are getting
    less than they expected or
    \href{https://www.nytimes.com/2020/06/10/business/Small-business-loans-ppp.html?action=click\&pgtype=Article\&state=default\&region=MAIN_CONTENT_3\&context=storylines_faq}{not
    hearing anything at all.}
  \end{itemize}
\item ~
  \hypertarget{what-are-my-rights-if-i-am-worried-about-going-back-to-work}{%
  \paragraph{What are my rights if I am worried about going back to
  work?}\label{what-are-my-rights-if-i-am-worried-about-going-back-to-work}}

  \begin{itemize}
  \tightlist
  \item
    Employers have to provide
    \href{https://www.osha.gov/SLTC/covid-19/standards.html}{a safe
    workplace} with policies that protect everyone equally.
    \href{https://www.nytimes.com/article/coronavirus-money-unemployment.html?action=click\&pgtype=Article\&state=default\&region=MAIN_CONTENT_3\&context=storylines_faq}{And
    if one of your co-workers tests positive for the coronavirus, the
    C.D.C.} has said that
    \href{https://www.cdc.gov/coronavirus/2019-ncov/community/guidance-business-response.html}{employers
    should tell their employees} -\/- without giving you the sick
    employee's name -\/- that they may have been exposed to the virus.
  \end{itemize}
\item ~
  \hypertarget{should-i-refinance-my-mortgage}{%
  \paragraph{Should I refinance my
  mortgage?}\label{should-i-refinance-my-mortgage}}

  \begin{itemize}
  \tightlist
  \item
    \href{https://www.nytimes.com/article/coronavirus-money-unemployment.html?action=click\&pgtype=Article\&state=default\&region=MAIN_CONTENT_3\&context=storylines_faq}{It
    could be a good idea,} because mortgage rates have
    \href{https://www.nytimes.com/2020/07/16/business/mortgage-rates-below-3-percent.html?action=click\&pgtype=Article\&state=default\&region=MAIN_CONTENT_3\&context=storylines_faq}{never
    been lower.} Refinancing requests have pushed mortgage applications
    to some of the highest levels since 2008, so be prepared to get in
    line. But defaults are also up, so if you're thinking about buying a
    home, be aware that some lenders have tightened their standards.
  \end{itemize}
\item ~
  \hypertarget{what-is-school-going-to-look-like-in-september}{%
  \paragraph{What is school going to look like in
  September?}\label{what-is-school-going-to-look-like-in-september}}

  \begin{itemize}
  \tightlist
  \item
    It is unlikely that many schools will return to a normal schedule
    this fall, requiring the grind of
    \href{https://www.nytimes.com/2020/06/05/us/coronavirus-education-lost-learning.html?action=click\&pgtype=Article\&state=default\&region=MAIN_CONTENT_3\&context=storylines_faq}{online
    learning},
    \href{https://www.nytimes.com/2020/05/29/us/coronavirus-child-care-centers.html?action=click\&pgtype=Article\&state=default\&region=MAIN_CONTENT_3\&context=storylines_faq}{makeshift
    child care} and
    \href{https://www.nytimes.com/2020/06/03/business/economy/coronavirus-working-women.html?action=click\&pgtype=Article\&state=default\&region=MAIN_CONTENT_3\&context=storylines_faq}{stunted
    workdays} to continue. California's two largest public school
    districts --- Los Angeles and San Diego --- said on July 13, that
    \href{https://www.nytimes.com/2020/07/13/us/lausd-san-diego-school-reopening.html?action=click\&pgtype=Article\&state=default\&region=MAIN_CONTENT_3\&context=storylines_faq}{instruction
    will be remote-only in the fall}, citing concerns that surging
    coronavirus infections in their areas pose too dire a risk for
    students and teachers. Together, the two districts enroll some
    825,000 students. They are the largest in the country so far to
    abandon plans for even a partial physical return to classrooms when
    they reopen in August. For other districts, the solution won't be an
    all-or-nothing approach.
    \href{https://bioethics.jhu.edu/research-and-outreach/projects/eschool-initiative/school-policy-tracker/}{Many
    systems}, including the nation's largest, New York City, are
    devising
    \href{https://www.nytimes.com/2020/06/26/us/coronavirus-schools-reopen-fall.html?action=click\&pgtype=Article\&state=default\&region=MAIN_CONTENT_3\&context=storylines_faq}{hybrid
    plans} that involve spending some days in classrooms and other days
    online. There's no national policy on this yet, so check with your
    municipal school system regularly to see what is happening in your
    community.
  \end{itemize}
\end{itemize}

Also, coronaviruses are relatively easy to destroy. Using a simple
disinfectant on a surface is nearly guaranteed to break the delicate
envelope that surrounds the tiny microbe, rendering it harmless,
Professor Whittaker said.

Image

Passengers on a bus in Seoul, South Korea. If an infected person sneezed
and a droplet landed on a surface, a person who then touched that
surface could pick it up, experts said.Credit...Kim Hong-Ji/Reuters

\hypertarget{does-the-brand-or-type-of-soap-you-use-matter}{%
\subsection{Does the brand or type of soap you use
matter?}\label{does-the-brand-or-type-of-soap-you-use-matter}}

No, several experts said.

\hypertarget{my-neighbor-is-coughing-should-i-be-worried}{%
\subsection{My neighbor is coughing. Should I be
worried?}\label{my-neighbor-is-coughing-should-i-be-worried}}

There is no evidence that viral particles can go through walls or glass,
said Dr. Ashish K. Jha, director of the Harvard Global Health Institute.

He said he was more concerned about the dangers posed by common spaces
than those posed by vents, provided there is good air circulation in a
room.

An infected neighbor might sneeze on a railing and if you touched it,
``that would be a more natural way to get it from your neighbor,'' he
said.

\hypertarget{can-i-get-it-from-making-out-with-someone}{%
\subsection{Can I get it from making out with
someone?}\label{can-i-get-it-from-making-out-with-someone}}

Kissing could definitely spread it, several experts said.

Though coronaviruses are not typically sexually transmitted, it's too
soon to know, the W.H.O. said.

\hypertarget{is-it-safe-to-eat-where-people-are-sick-with-the-coronavirus}{%
\subsection{Is it safe to eat where people are sick with the
coronavirus?}\label{is-it-safe-to-eat-where-people-are-sick-with-the-coronavirus}}

If a sick person handles the food or it's a high-traffic buffet, then
risks cannot be ruled out --- but heating or reheating food should kill
the virus, Professor Whittaker said.

Dr. Jha concurred.

``As a general rule, we haven't seen that food is a mechanism for
spreading,'' he said.

Image

Waiters in February at a restaurant in St. Mark's Square in Venice,
which would usually be full of tourists.Credit...Manuel
Silvestri/Reuters

\hypertarget{can-my-dog-or-cat-safely-join-me-in-quarantine}{%
\subsection{Can my dog or cat safely join me in
quarantine?}\label{can-my-dog-or-cat-safely-join-me-in-quarantine}}

Thousands of people have already begun various types of quarantines.
Some have been mandated by health officials, while others are voluntary
and primarily involve staying home.

Can a cat or dog join someone to make quarantine less lonely?

Professor Whittaker, who has studied the spread of coronaviruses in
animals and humans, said that he had seen no evidence that people who
have the virus could be a danger to their pets.

Apoorva Mandavilli contributed reporting.

Advertisement

\protect\hyperlink{after-bottom}{Continue reading the main story}

\hypertarget{site-index}{%
\subsection{Site Index}\label{site-index}}

\hypertarget{site-information-navigation}{%
\subsection{Site Information
Navigation}\label{site-information-navigation}}

\begin{itemize}
\tightlist
\item
  \href{https://help.nytimes.com/hc/en-us/articles/115014792127-Copyright-notice}{©~2020~The
  New York Times Company}
\end{itemize}

\begin{itemize}
\tightlist
\item
  \href{https://www.nytco.com/}{NYTCo}
\item
  \href{https://help.nytimes.com/hc/en-us/articles/115015385887-Contact-Us}{Contact
  Us}
\item
  \href{https://www.nytco.com/careers/}{Work with us}
\item
  \href{https://nytmediakit.com/}{Advertise}
\item
  \href{http://www.tbrandstudio.com/}{T Brand Studio}
\item
  \href{https://www.nytimes.com/privacy/cookie-policy\#how-do-i-manage-trackers}{Your
  Ad Choices}
\item
  \href{https://www.nytimes.com/privacy}{Privacy}
\item
  \href{https://help.nytimes.com/hc/en-us/articles/115014893428-Terms-of-service}{Terms
  of Service}
\item
  \href{https://help.nytimes.com/hc/en-us/articles/115014893968-Terms-of-sale}{Terms
  of Sale}
\item
  \href{https://spiderbites.nytimes.com}{Site Map}
\item
  \href{https://help.nytimes.com/hc/en-us}{Help}
\item
  \href{https://www.nytimes.com/subscription?campaignId=37WXW}{Subscriptions}
\end{itemize}
