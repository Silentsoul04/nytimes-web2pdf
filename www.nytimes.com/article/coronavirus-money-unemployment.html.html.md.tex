Sections

SEARCH

\protect\hyperlink{site-content}{Skip to
content}\protect\hyperlink{site-index}{Skip to site index}

\href{https://www.nytimes.com/section/your-money}{Your Money}

\href{https://myaccount.nytimes.com/auth/login?response_type=cookie\&client_id=vi}{}

\href{https://www.nytimes.com/section/todayspaper}{Today's Paper}

\href{/section/your-money}{Your Money}\textbar{}Your Money: A Hub for
Help During the Coronavirus Crisis

\url{https://nyti.ms/33vnSBV}

\begin{itemize}
\item
\item
\item
\item
\item
\end{itemize}

\href{https://www.nytimes.com/news-event/coronavirus?action=click\&pgtype=Article\&state=default\&region=TOP_BANNER\&context=storylines_menu}{The
Coronavirus Outbreak}

\begin{itemize}
\tightlist
\item
  live\href{https://www.nytimes.com/2020/08/04/world/coronavirus-cases.html?action=click\&pgtype=Article\&state=default\&region=TOP_BANNER\&context=storylines_menu}{Latest
  Updates}
\item
  \href{https://www.nytimes.com/interactive/2020/us/coronavirus-us-cases.html?action=click\&pgtype=Article\&state=default\&region=TOP_BANNER\&context=storylines_menu}{Maps
  and Cases}
\item
  \href{https://www.nytimes.com/interactive/2020/science/coronavirus-vaccine-tracker.html?action=click\&pgtype=Article\&state=default\&region=TOP_BANNER\&context=storylines_menu}{Vaccine
  Tracker}
\item
  \href{https://www.nytimes.com/2020/08/02/us/covid-college-reopening.html?action=click\&pgtype=Article\&state=default\&region=TOP_BANNER\&context=storylines_menu}{College
  Reopening}
\item
  \href{https://www.nytimes.com/live/2020/08/04/business/stock-market-today-coronavirus?action=click\&pgtype=Article\&state=default\&region=TOP_BANNER\&context=storylines_menu}{Economy}
\end{itemize}

Advertisement

\protect\hyperlink{after-top}{Continue reading the main story}

Supported by

\protect\hyperlink{after-sponsor}{Continue reading the main story}

Updated~

Aug. 4, 2020, 9:06 a.m. ET

Aug. 4, 2020, 9:06 a.m. ET

\hypertarget{your-money-a-hub-for-help-during-the-coronavirus-crisis}{%
\section{Your Money: A Hub for Help During the Coronavirus
Crisis}\label{your-money-a-hub-for-help-during-the-coronavirus-crisis}}

\href{https://www.nytimes.com/by/ron-lieber}{\includegraphics{https://static01.nyt.com/images/2018/10/22/multimedia/author-ron-lieber/author-ron-lieber-thumbLarge.png}}\href{https://www.nytimes.com/by/tara-siegel-bernard}{\includegraphics{https://static01.nyt.com/images/2019/01/18/multimedia/author-tara-siegel-bernard/author-tara-siegel-bernard-thumbLarge.png}}

By \href{https://www.nytimes.com/by/ron-lieber}{Ron Lieber} and
\href{https://www.nytimes.com/by/tara-siegel-bernard}{Tara Siegel
Bernard}

\includegraphics{https://static01.nyt.com/images/2020/03/22/business/18virus-hub/18virus-hub-articleLarge.jpg?quality=75\&auto=webp\&disable=upscale}

We're here to help.

The pandemic has put millions of people out of work, forced painful
sacrifices and put many in the position of needing help they never
imagined would be necessary. We assembled this guide to connect you with
information about government benefits, free services and financial
strategies to get you through this crisis.

If you have a question that we have not answered about different kinds
of relief, please write to
\href{mailto:hubforhelp@nytimes.com}{\nolinkurl{hubforhelp@nytimes.com}}.
Ron and Tara will read every message.

\hypertarget{jump-to-information-about}{%
\subsubsection{Jump to information
about:}\label{jump-to-information-about}}

\begin{itemize}
\tightlist
\item
  \protect\hyperlink{link-2f8736c0}{Unemployment Insurance}
\item
  \protect\hyperlink{link-4ccc30a7}{Paid Sick Leave and Family Leave}
\item
  \protect\hyperlink{link-3c2140bc}{Food Assistance}
\item
  \protect\hyperlink{link-39bfe27a}{Mortgage Relief}
\item
  \protect\hyperlink{link-1da1a7c4}{Rent Relief}
\item
  \protect\hyperlink{link-74e96196}{Health Insurance}
\item
  \protect\hyperlink{link-75bcf9dc}{Credit Cards and Auto Loans}
\item
  \protect\hyperlink{link-49737032}{Mental Health}
\item
  \protect\hyperlink{link-16f012f2}{More Helpful Advice}
\end{itemize}

\hypertarget{unemployment-insurance}{%
\subsection{Unemployment Insurance}\label{unemployment-insurance}}

A giant pandemic
\href{https://www.nytimes.com/2020/03/27/world/coronavirus-live-news-updates.html\#link-1900f91a}{relief
package} made significant --- but temporary --- changes to the way the
unemployment insurance system works. These changes expand the kinds of
workers who are eligible for unemployment, extend the amount of time
people can receive benefits and increase the amount people can receive.

\textbf{Who is eligible?} The state programs that make up the
unemployment system now cover far more people than usual, including
self-employed people and part-time workers. Those who are unemployed,
are partly unemployed or cannot work for a wide variety of
coronavirus-related reasons will be more likely to receive benefits ---
and you don't necessarily need to lose your job to qualify. For example,
if you're quarantined or have been furloughed --- that is, you're not
being paid but expect to return to your job eventually --- you may be
eligible.

\textbf{How much will someone get?} States set many of their own rules,
including for benefit amounts, which are generally calculated as a
percentage of your income over the past year, up to a certain maximum.
Some states are more generous than others, but unemployment typically
replaces roughly 45 percent of your lost income.

Whatever your benefit amount, the CARES Act also provides a temporary
increase of
\href{https://www.nytimes.com/interactive/2020/04/23/business/economy/unemployment-benefits-stimulus-coronavirus.html}{\$600
weekly}, but only through July 31.

\textbf{How long will it last?} Benefits could last nine months or more,
through a combination of state and federal programs. But the details
depend on your state.

Most states pay benefits for 26 weeks, though some offer less. After
that, federal legislation extends benefits by another 13 weeks. (Here's
a \href{https://labor.ny.gov/ui/cares-act.shtm}{helpful illustration}
that breaks down how the program works in New York State.)

In periods of high unemployment, your state may also offer its own
extended benefit program. Extended benefits usually last for half the
length of the state's standard benefit period.

\textbf{What else should I know?} Being eligible for benefits doesn't
mean the process is easy.

Many states administering these benefits are relying on
\href{https://www.nytimes.com/2020/04/17/nyregion/coronavirus-pandemic-unemployment-assistance-ny-delays.html}{archaic
systems}, which have been overwhelmed by the influx of claims. That has
left
\href{https://www.nytimes.com/2020/04/23/us/florida-coronavirus-unemployment.html}{many
people}
\href{https://www.nytimes.com/2020/05/08/nyregion/unemployment-benefits-ny-coronavirus.html}{beyond
frustrated} because they were locked out, unable to submit applications
or wondering if and when a check would ever arrive. If you're still
encountering difficulty, try contacting your elected state and federal
representatives for help.
\href{https://www.americanbar.org/groups/legal_services/flh-home/flh-free-legal-help/}{Legal
Aid} is another good resource for lower-income households.

\hypertarget{paid-sick-leave-and-family-leave}{%
\subsection{Paid Sick Leave and Family
Leave}\label{paid-sick-leave-and-family-leave}}

The CARES Act gives many American workers paid leave if they need to
take time off because of the outbreak, but
\href{https://www.nytimes.com/2020/03/14/opinion/coronavirus-pelosi-sick-leave.html}{there
are a lot of exceptions,} and how the benefits work depends on your
\href{https://www.nytimes.com/2020/05/08/upshot/virus-paid-leave-pandemic.html}{circumstances}.

\textbf{Who is eligible?} Most workers at small and midsize companies,
as well as government employees. And that includes part-time workers.

\textbf{How much will they receive?} Eligible employees get two weeks of
paid sick leave if they are ill or seeking care, as long as they've been
employed at least 30 days. They can receive their full pay, up to \$511
a day.

Some workers can also get 12 weeks of paid leave to care for children
whose schools are closed, or whose child care provider is unavailable
because of the outbreak, but fewer workers qualify for this type of
leave. They can receive two-thirds of their usual pay, up to \$200 a
day.

Part-time workers will be paid the amount they typically earn in a
two-week period, up to the daily limits. People who are self-employed
--- including gig workers like Uber drivers and Instacart shoppers ---
can also receive paid leave, but they must calculate their average daily
income and claim it as a tax credit.

\textbf{Who is left out?} Employers with fewer than 50 workers can
\href{https://www.nytimes.com/2020/04/02/us/politics/coronavirus-paid-leave.html}{deny
workers the child-care leave} (but not the sick leave) if it would be
hard on their businesses, and companies with more than 500 employees are
excluded from the rules entirely.
\href{https://www.bls.gov/ncs/ebs/benefits/2019/ownership/private/table31a.pdf}{Many
workers at big businesses} already have paid sick leave, but their
employers' low-wage workers are the least likely to be covered. The New
America Foundation has published
\href{https://www.newamerica.org/better-life-lab/reports/which-companies-still-arent-offering-paid-sick-days/}{a
detailed list} of large employers (mostly consumer-facing companies like
retailers, restaurant chains and hotels) and their policies.

\textbf{How long do benefits last?} These changes aren't permanent ---
the leave law expires Dec. 31. You can find out more from the Department
of Labor, which has posted a
\href{https://www.dol.gov/agencies/whd/pandemic/ffcra-employee-paid-leave}{fact
sheet for workers} and a
\href{https://www.dol.gov/agencies/whd/pandemic/ffcra-questions}{Q\&A}.

\hypertarget{food-assistance}{%
\subsection{Food Assistance}\label{food-assistance}}

If you are experiencing food insecurity for the first time, you're not
alone. If you've never used a food pantry, it might help to read
\href{https://medium.com/wake-up-call/i-went-to-the-food-bank-for-the-first-time-37450c89b959}{a
few}
\href{https://workingclassstudies.wordpress.com/2011/12/05/a-visit-to-the-food-pantry/}{dispatches}
\href{https://www.npr.org/sections/thesalt/2019/06/30/735881297/opinion-being-hungry-in-america-is-hard-work-food-banks-need-your-help}{from
others} who started getting groceries at local food banks. Here's a list
of
\href{https://solvehungertoday.org/blog/visiting-food-pantry-myths-facts/}{myths}
about food pantries and
\href{https://extension.sdstate.edu/tips-visiting-food-pantry}{tips} for
visiting them. To find your nearest food pantry, start with the map
\href{https://ampleharvest.org/find-pantry/}{here}.

Many people who experience even a temporary loss of income can become
eligible for food stamps but do not realize it. The formal name for the
program is Supplemental Nutrition Assistance Program, or SNAP, and
eligibility may vary by state. Here's
\href{https://www.nytimes.com/2020/07/17/your-money/food-stamps-coronavirus.html}{Ron's
column} explaining how the system works.

The federal F.A.Q. about SNAP eligibility is
\href{https://www.fns.usda.gov/snap/recipient/eligibility}{here}, and
you can learn more about your state's rules via
\href{https://www.fns.usda.gov/snap/state-directory}{this map}. As Tara
noted in a recent
\href{https://www.nytimes.com/2020/05/01/your-money/food-stamps-snap-coronavirus.html}{article},
it isn't always possible to use SNAP benefits when buying groceries
online.

\hypertarget{mortgage-relief}{%
\subsection{Mortgage Relief}\label{mortgage-relief}}

\href{https://www.nytimes.com/2020/05/15/business/coronavirus-mortgage-relief.html}{Millions
of homeowners} have pressed the pause button on their mortgage payments,
a form of relief extended by the CARES Act. Not all homeowners are
covered under the new law, however, and many borrowers seeking relief
have been given inaccurate information. Here's what you need to know.

\textbf{Who is covered by the law?} Homeowners with mortgages backed by
the federal government are permitted to
\href{https://www.fhfa.gov/Homeownersbuyer/MortgageAssistance/Pages/Coronavirus-Assistance-Information.aspx}{temporarily
suspend their payments}, a process called forbearance, for up to a year.
This covers about 70 percent of mortgage holders and includes loans
backed by Fannie Mae or Freddie Mac, loans insured by the Federal
Housing Administration (known as F.H.A. loans) and those guaranteed by
the Department of Veterans Affairs and the Department of Agriculture.

About 30 percent of mortgage holders have loans owned by banks or other
private investors. They are not covered by the new law, but many of
these homeowners have received similar relief, often granted in
three-month increments.

\textbf{Who controls my mortgage?} You can search your address on
\href{https://www.consumerfinance.gov/ask-cfpb/how-can-i-tell-who-owns-my-mortgage-en-214/}{various}
\href{https://www.makinghomeaffordable.gov/get-answers/Pages/get-answers-find-out-mortgage.aspx}{government
websites}.

\textbf{What happens after forbearance?} You must make up for the
payments you skip. The process will vary depending on your personal
circumstances --- and who owns your loan.

If you have a federally backed loan, your servicer should present you
with several ways to become current on your mortgage --- and none of
them require you to immediately pay the money back in a lump sum
(although you can if you want to). If you can afford to resume your
regular payments, you may pay the money back over several months, for
example, or settle up when the home is sold, refinanced or when the
mortgage term is up.

People who still cannot afford to make their mortgage payments after the
forbearance period expires will probably have to lower their monthly
payment by modifying their loan, a more formal process that will require
an application.

For mortgage holders with loans owned by banks or private investors,
\href{https://www.nytimes.com/2020/05/15/business/coronavirus-mortgage-relief.html}{the
options} aren't always as clear or as accommodating.

\textbf{What about foreclosure?} Federal housing officials
\href{https://www.fhfa.gov/Media/PublicAffairs/Pages/FHFA-Extends-Foreclosure-and-Eviction-Moratorium-6172020.aspx}{recently
extended} a nationwide eviction and foreclosure moratorium for borrowers
with loans backed by Fannie Mae, Freddie Mac and the F.H.A. This
includes foreclosures that are already in progress.

\textbf{Where can I get assistance?} If you don't feel like you are
being treated fairly --- or are simply overwhelmed by the process --- it
might help to
\href{https://apps.hud.gov/offices/hsg/sfh/hcc/hcs.cfm}{find a housing
counselor}. For more information, check out our short
\href{https://www.nytimes.com/2020/05/15/business/covid-mortgage-forbearance.html}{resource
guide here}.

\hypertarget{rent-relief}{%
\subsection{Rent Relief}\label{rent-relief}}

While rent payments nationwide
\href{https://www.nytimes.com/2020/05/31/business/economy/coronavirus-rent-landlords-tenants.html}{have
not yet} fallen precipitously, every person who can't pay is in crisis,
and fear remains high that more
\href{https://www.nytimes.com/2020/05/27/us/coronavirus-evictions-renters.html?action=click\&module=RelatedLinks\&pgtype=Article}{trouble}
is coming.

\textbf{Where can I get help?} ** Here's Ron's
\href{https://www.nytimes.com/2020/07/11/your-money/coronavirus-eviction-prevention-renters-landlord.html}{column}
on how to prevent your own eviction, and it includes a number of links
with plenty of detail.

If your landlord won't give you a break and you want to see what legal
options you might have, you can search for a low or no-cost legal
assistance office near you via the Legal Services Corporation's
\href{https://www.lsc.gov/what-legal-aid/find-legal-aid}{map}. Just
Shelter, a tenant advocacy group formed by Matthew Desmond and Tessa
Lowinske Desmond, also offers
\href{https://justshelter.org/community-resources/}{information} on
local organizations that can provide advice to renters in distress.

\textbf{What are governments doing?} ** State and local governments have
offered some eviction protection. Mr. Desmond, the author of the book
``\href{https://www.nytimes.com/2016/02/28/books/review/matthew-desmonds-evicted-poverty-and-profit-in-the-american-city.html}{Evicted},''
is also the founder of Eviction Lab, which maintains
\href{https://evictionlab.org/covid-eviction-policies/}{a list} of local
and regional actions to pause evictions of renters. It has also
published a
\href{https://evictionlab.org/covid-policy-scorecard/}{scorecard} that
examines state policies and how they've changed since the pandemic took
hold.

The CARES Act put a temporary, nationwide
\href{https://crsreports.congress.gov/product/pdf/IN/IN11320}{eviction
moratorium} in place for any renters whose landlords have mortgages
backed or owned by Fannie Mae, Freddie Mac or the Federal Housing
Administration. This will last through the end of July, and landlords
can't charge any fees or penalties for nonpayment of rent either. The
moratorium applies only to eviction for nonpayment; tenants can still be
evicted for other reasons.

\textbf{What about my landlord?} Regulators have also
\href{https://www.fhfa.gov/Media/PublicAffairs/Pages/FHFA-Moves-to-Provide-Eviction-Suspension-Relief-for-Renters-in-Multifamily-Properties.aspx}{told
landlords} whose own mortgages are owned by Fannie or Freddie that they
can use forbearance on their own mortgages, just as long as they do not
evict tenants after they pause their mortgage payments. The challenge
for renters is figuring out whether their landlord has such a mortgage.
This information sometimes appears if you look up the address in the
\href{https://preservationdatabase.org/about-the-database/}{National
Housing Preservation Database} or in
\href{https://nlihc.org/federal-moratoriums}{another one} that the
National Low Income Housing Coalition created.

If the landlord's mortgage is not in forbearance, renters who skip
payments could be risking eviction if there has not been a local
prohibition.

\hypertarget{health-insurance}{%
\subsection{Health Insurance}\label{health-insurance}}

Millions of Americans
\href{https://www.kff.org/coronavirus-covid-19/issue-brief/eligibility-for-aca-health-coverage-following-job-loss/}{most
likely lost} their health coverage along with their jobs. And many
others can no longer afford the policy they were paying for on their
own.

If your situation has recently
changed,\href{https://www.nytimes.com/2020/03/25/upshot/coronavirus-health-insurance-faq.html}{you
have more options} now than in previous crises. But navigating the
\href{https://www.kff.org/health-reform/issue-brief/changes-in-income-and-health-coverage-eligibility-after-job-loss-due-to-covid-19/}{complex
web} of alternatives and figuring out how to qualify can be a challenge.

\textbf{If your income has dwindled to almost nothing.} People earning
very little will most likely be eligible for the federal-state health
insurance program known as Medicaid in 36 states and the District of
Columbia. Because of the Affordable Care Act, most states now allow all
residents to qualify for Medicaid if their household's monthly income is
below a certain threshold --- around \$1,400 a month for a single person
or \$2,950 for a family of four. That calculation should include any
normal unemployment benefits you are receiving, but not the additional
\$600 a week being paid temporarily and not the direct
\href{https://www.nytimes.com/interactive/2020/07/24/business/economy/600-unemployment-benefits.html}{stimulus}
payment authorized under recent
\href{https://www.nytimes.com/article/coronavirus-stimulus-package-questions-answers.html}{relief
legislation}.

\textbf{If your income is too high for Medicaid.} Those earning more can
probably buy coverage through the marketplaces established under the
Affordable Care Act --- and you may qualify for substantial subsidies.
If you lose your job for any reason, you are permitted to sign up during
\href{https://www.healthcare.gov/have-job-based-coverage/if-you-lose-job-based-coverage/}{a
special enrollment period.}

People who want to buy coverage even in the absence of a job loss might
be able to do so if they live in states that run their own marketplaces;
some of those states have established special enrollment periods. But
the
\href{https://www.nytimes.com/2020/04/01/upshot/obamacare-markets-coronavirus-trump.html}{Trump
administration decided in April that it would not} reopen the federal
Healthcare.gov marketplaces to new customers. Those marketplaces are
used in 38 states.

If you already have a marketplace plan but have experienced a drop in
income, you can go back into the system --- even outside of an open
enrollment period --- and adjust your income, which may result in
greater subsidies.

It's also possible to keep your insurance if you lost your job, but that
tends to be more expensive than buying coverage in the Obamacare
marketplaces.

\textbf{I have a job and a workplace plan. What about me?} You may have
a chance to
\href{https://www.nytimes.com/2020/05/12/business/employer-health-plans-coronavirus.html}{change
your coverage} or add family members outside of an open enrollment
period, which usually isn't possible. The Internal Revenue Service
recently made an exception, but your employer doesn't have to offer this
option.

For more details on the various coverage options, check out
\href{https://www.nytimes.com/2020/03/25/upshot/coronavirus-health-insurance-faq.html}{this
piece} by Margot Sanger-Katz and Reed Abelson. (At least one health
insurance company, UnitedHealth,
is\href{https://www.nytimes.com/2020/05/07/health/unitedhealth-coronavirus.html}{offering
modest relief} by providing enrollees with a break on premiums.)

\hypertarget{credit-cards-and-auto-loans}{%
\subsection{Credit Cards and Auto
Loans}\label{credit-cards-and-auto-loans}}

If you need temporary relief on your credit card or auto loan payments,
many lenders are offering at least some help.

Start with the website for your lenders and read what they have posted.
Some have made their policies more stingy since Ron
\href{https://www.nytimes.com/2020/03/17/your-money/loan-waivers-coronavirus.html}{first
reported} on changes in March.

If you call for help via phone, record the conversation if you can or at
least get written documentation of any changes the lender agrees to.
\href{https://www.nytimes.com/2020/05/16/business/coronavirus-financial-help.html}{This
column from Ron} explains how and why.

Among the options you can ask for are permission to skip payments (with
waived interest charges during the months you skip), the elimination of
late or other fees and a permanently lower interest rate. Ask how any
change might affect your credit score and whether you'll have to make up
missed payments in one lump sum right after the zero-payment months.

\hypertarget{mental-health}{%
\subsection{Mental Health}\label{mental-health}}

Financial losses often come with emotional strain, at the very point
when people may be least likely to spend money on care for themselves.

If you are in severe distress, the number for the free, confidential
\href{https://suicidepreventionlifeline.org/our-crisis-centers/}{National
Suicide Prevention Hotline} is 1-800-273-8255. It's open at all hours.
Or text HELLO to \href{https://www.crisistextline.org/text-us/}{741741}.

Many mental health practitioners do pro bono work or charge fees on a
sliding scale. There does not appear to be a national directory of such
providers, but there is no reason not to contact local ones to ask about
low or no-cost services.

The National Alliance on Mental Illness maintains
\href{https://nami.org/Support-Education/NAMI-HelpLine/NAMI-HelpLine-FAQs}{a
help line} that can provide referrals to local resources as well. Its
number is 1-800-950-6264.

\hypertarget{more-helpful-advice}{%
\subsection{More Helpful Advice}\label{more-helpful-advice}}

\begin{itemize}
\item
  \textbf{Help for the Self-Employed.} The self-employed often have
  fewer protections than employees working for companies and other
  organizations, but two legislative packages extended several new
  benefits to help them cope during the pandemic. Paid sick and family
  leave is now available in the form of a tax credit. Unemployment
  insurance is also newly available to gig workers, independent
  contractors and freelancers who are usually ineligible. And
  self-employed people who can no longer afford their health insurance
  or want to buy new polices may have more options.
  \href{https://www.nytimes.com/article/self-employed-workers-unemployment-coronavirus-stimulus-package.html}{Tara's
  story} has more details.
\item
  \textbf{You have some flexibility with your federal student loans.} In
  fact, you should have automatically received relief without lifting a
  finger: Borrowers have been placed in so-called administrative
  forbearance, which allows you to temporarily stop making payments
  until Sept. 30.

  No interest will accrue during this period, and borrowers who want to
  continue making loan payments can do so.

  The Education Department says that these skipped payments will still
  count toward loan forgiveness for borrowers in income-driven repayment
  and public service loan forgiveness programs, as long as the other
  usual requirements are fulfilled.

  If you have more questions, check out the Education Department's Q\&A
  \href{https://studentaid.gov/announcements-events/coronavirus\#borrower-questions}{here}.
  Some private lenders are offering relief programs, too.
\item
  \textbf{Staying in touch.} A number of large companies have agreed not
  to terminate the service of residential or small business customers
  who can't pay their bills until at least June 30, including AT\&T,
  Comcast, Cox, RCN, Sprint, T-Mobile and Verizon. A
  \href{https://www.fcc.gov/keep-americans-connected\#pledges}{full list
  of companies} is available on the Federal Communications Commission
  site.
\item
  \textbf{How to help.} There is no shortage of need right now --- and
  no shortage of guides to helping. The New York Times has a
  \href{https://www.nytimes.com/article/coronavirus-how-to-help-donations-charities.html}{basic
  guide} to coronavirus giving,
  \href{https://www.nytimes.com/2020/03/27/smarter-living/coronavirus-charity-donations.html}{suggestions}
  on where to donate money, some practical tips on
  \href{https://www.nytimes.com/2020/04/10/nyregion/coronavirus-help-healthcare-workers.html}{what
  not to do} and an
  \href{https://www.nytimes.com/2020/04/13/style/self-care/donate-clothes-coronavirus.html}{explainer}
  on donating clothes.

  Ron wrote
  \href{https://www.nytimes.com/2020/05/30/your-money/philanthropy-charity-giving-coronavirus.html}{a
  column} about the kind of direct giving that allows you to channel
  money to individuals with immediate cash needs. The New York Times has
  also started
  \href{https://www.nytimes.com/2020/04/01/reader-center/neediest-cases-covid-19-relief-campaign.html}{its
  own campaign} as part of its Neediest Cases fund.
\item
  \textbf{Don't forget about property taxes.} Despite the economic
  strain caused by the virus, in many places, homeowners are still
  expected to make property tax payments by the usual deadlines. If they
  were postponed, they could wreak havoc on local budgets. Our colleague
  Ann
  Carrns\href{https://www.nytimes.com/2020/04/10/your-money/coronavirus-property-taxes.html?campaign_id=12\&emc=edit_my_20200413\&instance_id=17606\&nl=your-money\&regi_id=8921505\&segment_id=25000\&te=1\&user_id=1f51a0e7a2edf91cad2fd25cabf8cd78}{has
  a story} with more information. You can find details on jurisdictions
  that may offer some leeway in
  \href{https://www.inmyarea.com/research/covid-property-tax-breaks-by-state\#special-relief-program-deadline-extensions}{this
  chart}.

  There are situations where you may get a break. If you have paused
  payments on your federally backed loan and you pay taxes and insurance
  from an escrow account, your mortgage servicer should continue to
  advance those payments as well, according to the Federal Housing
  Finance Agency. But if you don't use an escrow account for taxes and
  insurance, you will need to continue making those payments on your own
  unless your locality provides some flexibility or relief.
\item
  \textbf{Mistaken stimulus payments.} Think you received a payment by
  mistake, say, for a relative who is among the over one million
  deceased individuals to whom the government made payments? Don't spend
  the money. The Internal Revenue Service may well realize its mistake
  and ask for it back come tax time in 2021.
\item
  \textbf{Financial planners are offering free advice.}
  \href{https://www.xyplanningnetwork.com/?_advisor_search=\%22coronavirus\%20relief\%22}{Dozens
  of members} of the XY Planning Network have offered to help people
  through phone consultations. The Financial Planning Association has
  \href{https://www.onefpa.org/Pages/ProBonoPlanning.aspx}{its own list}
  of volunteer certified financial planners,
  \href{https://www.napfa.org/find-an-advisor}{as does} the National
  Association of Personal Financial Advisors.
\item
  \textbf{What to know about Social Security.} Older workers who have
  lost their jobs and are short on savings may be contemplating whether
  they should file for Social Security earlier than they had
  anticipated. Filing before your
  \href{https://www.ssa.gov/planners/retire/retirechart.html}{full
  retirement age} has serious implications, which may reduce your
  monthly check forevermore. Before you decide, consider
  \href{https://www.nytimes.com/2020/04/17/business/retiring-social-security-jobs-coronavirus-pandemic.html}{the
  following strategies}. (And if you're eligible for unemployment, you
  might apply for that first.)

  The Social Security Administration has mostly closed its 1,200 offices
  for routine requests like help with benefit claims. Those requests
  should go through the agency's toll-free phone line, 1-800-772-1213,
  and its \href{https://www.ssa.gov/onlineservices/}{website}. In-person
  assistance is still available for crucial services, like reinstatement
  of benefits and assistance for those with severe disabilities. Those
  seeking in-person help must call in advance. Mark Miller has details
  \href{https://www.nytimes.com/2020/03/17/business/retirement/coronavirus-social-security.html}{here}.
\item
  \textbf{You can use a retirement account in new ways.} Many people who
  are out of work may be turning to their retirement accounts for
  emergency cash. Under normal circumstances, that would trigger taxes
  and penalties. But the CARES Act provides more flexible hardship
  options for 401(k) and individual retirement accounts. But even newly
  jobless people who don't need to tap their savings have a decision to
  make: Leave the money in a former employer's plan or roll it over to
  an individual retirement account? All of these situations require some
  analysis.
  \href{https://www.nytimes.com/2020/05/10/business/401k-rollover-faq.html}{This
  story} can help.
\item
  \textbf{Get your free credit report.} Consumers can now check their
  credit reports from each of the Big Three credit firms each week, free
  of charge, instead of just once a year. Routine checks have always
  been wise, but now they are essential --- particularly if you're
  skipping payments with the permission of your lender. Even if your
  lender says this relief won't hurt your credit profile, mistakes are
  bound to happen. To find out more about how to check your report and
  what to look for, read
  \href{https://www.nytimes.com/2020/05/15/your-money/coronavirus-credit-reports.html}{Ann's
  story here.}
\item
  \textbf{Watch out for fraud.} Whether it's a shady sales pitch for a
  gravity-defying investment or a website offering masks that never
  arrive, coronavirus-related fraud is on the rise. These solicitations
  may arrive by telephone, text messages, email, social media, even in
  \href{https://www.nytimes.com/2020/04/05/us/politics/coronavirus-scams-fraud-price-gouging.html}{store
  parking lots,} which is why consumers must remain hypervigilant.
  This\href{https://www.nytimes.com/2020/03/13/business/coronavirus-scams.html}{story
  from Tara} looks at overhyped pitches for complex investments, while
  \href{https://www.nytimes.com/2020/04/17/your-money/coronavirus-fraud.html}{this
  piece from Ann} surveys the landscape of bogus practices.
  The\href{https://www.consumerfinance.gov/about-us/blog/beware-coronavirus-related-scams/}{Consumer
  Financial Protection Bureau} and
  the\href{https://www.consumer.ftc.gov/features/coronavirus-scams-what-ftc-doing}{Federal
  Trade Commission} have also posted warnings about coronavirus fraud,
  and a Bentley University professor named Steve Weisman also keeps a
  \href{https://scamicide.com/tag/coronavirus-scams/}{running list} of
  virus-related scams.
\end{itemize}

Advertisement

\protect\hyperlink{after-bottom}{Continue reading the main story}

\hypertarget{site-index}{%
\subsection{Site Index}\label{site-index}}

\hypertarget{site-information-navigation}{%
\subsection{Site Information
Navigation}\label{site-information-navigation}}

\begin{itemize}
\tightlist
\item
  \href{https://help.nytimes.com/hc/en-us/articles/115014792127-Copyright-notice}{©~2020~The
  New York Times Company}
\end{itemize}

\begin{itemize}
\tightlist
\item
  \href{https://www.nytco.com/}{NYTCo}
\item
  \href{https://help.nytimes.com/hc/en-us/articles/115015385887-Contact-Us}{Contact
  Us}
\item
  \href{https://www.nytco.com/careers/}{Work with us}
\item
  \href{https://nytmediakit.com/}{Advertise}
\item
  \href{http://www.tbrandstudio.com/}{T Brand Studio}
\item
  \href{https://www.nytimes.com/privacy/cookie-policy\#how-do-i-manage-trackers}{Your
  Ad Choices}
\item
  \href{https://www.nytimes.com/privacy}{Privacy}
\item
  \href{https://help.nytimes.com/hc/en-us/articles/115014893428-Terms-of-service}{Terms
  of Service}
\item
  \href{https://help.nytimes.com/hc/en-us/articles/115014893968-Terms-of-sale}{Terms
  of Sale}
\item
  \href{https://spiderbites.nytimes.com}{Site Map}
\item
  \href{https://help.nytimes.com/hc/en-us}{Help}
\item
  \href{https://www.nytimes.com/subscription?campaignId=37WXW}{Subscriptions}
\end{itemize}
