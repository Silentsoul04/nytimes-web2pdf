Sections

SEARCH

\protect\hyperlink{site-content}{Skip to
content}\protect\hyperlink{site-index}{Skip to site index}

\href{https://www.nytimes.com/section/business}{Business}

\href{https://myaccount.nytimes.com/auth/login?response_type=cookie\&client_id=vi}{}

\href{https://www.nytimes.com/section/todayspaper}{Today's Paper}

\href{/section/business}{Business}\textbar{}F.A.Q. on Stimulus Checks,
Unemployment and the Coronavirus Plan

\url{https://nyti.ms/2vQbnVj}

\begin{itemize}
\item
\item
\item
\item
\item
\item
\end{itemize}

\href{https://www.nytimes.com/news-event/coronavirus?action=click\&pgtype=Article\&state=default\&region=TOP_BANNER\&context=storylines_menu}{The
Coronavirus Outbreak}

\begin{itemize}
\tightlist
\item
  live\href{https://www.nytimes.com/2020/08/01/world/coronavirus-covid-19.html?action=click\&pgtype=Article\&state=default\&region=TOP_BANNER\&context=storylines_menu}{Latest
  Updates}
\item
  \href{https://www.nytimes.com/interactive/2020/us/coronavirus-us-cases.html?action=click\&pgtype=Article\&state=default\&region=TOP_BANNER\&context=storylines_menu}{Maps
  and Cases}
\item
  \href{https://www.nytimes.com/interactive/2020/science/coronavirus-vaccine-tracker.html?action=click\&pgtype=Article\&state=default\&region=TOP_BANNER\&context=storylines_menu}{Vaccine
  Tracker}
\item
  \href{https://www.nytimes.com/interactive/2020/07/29/us/schools-reopening-coronavirus.html?action=click\&pgtype=Article\&state=default\&region=TOP_BANNER\&context=storylines_menu}{What
  School May Look Like}
\item
  \href{https://www.nytimes.com/live/2020/07/31/business/stock-market-today-coronavirus?action=click\&pgtype=Article\&state=default\&region=TOP_BANNER\&context=storylines_menu}{Economy}
\end{itemize}

Advertisement

\protect\hyperlink{after-top}{Continue reading the main story}

Supported by

\protect\hyperlink{after-sponsor}{Continue reading the main story}

\hypertarget{faq-on-stimulus-checks-unemployment-and-the-coronavirus-plan}{%
\section{F.A.Q. on Stimulus Checks, Unemployment and the Coronavirus
Plan}\label{faq-on-stimulus-checks-unemployment-and-the-coronavirus-plan}}

The \$2 trillion relief package is sending money directly to Americans,
greatly expanding unemployment coverage and making a number of other
changes.

\includegraphics{https://static01.nyt.com/images/2020/03/27/business/26virus-qa/26virus-qa-articleLarge.jpg?quality=75\&auto=webp\&disable=upscale}

\href{https://www.nytimes.com/by/tara-siegel-bernard}{\includegraphics{https://static01.nyt.com/images/2019/01/18/multimedia/author-tara-siegel-bernard/author-tara-siegel-bernard-thumbLarge.png}}\href{https://www.nytimes.com/by/ron-lieber}{\includegraphics{https://static01.nyt.com/images/2018/10/22/multimedia/author-ron-lieber/author-ron-lieber-thumbLarge.png}}

By \href{https://www.nytimes.com/by/tara-siegel-bernard}{Tara Siegel
Bernard} and \href{https://www.nytimes.com/by/ron-lieber}{Ron Lieber}

\begin{itemize}
\item
  July 24, 2020
\item
  \begin{itemize}
  \item
  \item
  \item
  \item
  \item
  \item
  \end{itemize}
\end{itemize}

\emph{{[}}\href{https://www.nytimes.com/article/small-business-loans-stimulus-grants-freelancers-coronavirus.html}{\emph{Read
our Coronavirus Relief Small Business F.A.Q.}}\emph{{]}}

President Trump has signed
\href{https://www.nytimes.com/2020/03/26/us/coronavirus-senate-stimulus-package.html}{a
bipartisan \$2 trillion economic relief plan} to offer assistance to
tens of millions of American households affected by
the\href{https://www.nytimes.com/interactive/2020/07/24/business/economy/600-unemployment-benefits.html}{coronavirus}
pandemic. Its components include
\href{https://www.nytimes.com/interactive/2020/07/24/business/economy/600-unemployment-benefits.html}{stimulus}
payments to individuals, expanded unemployment coverage, student loan
changes, different retirement account rules and more.

Here are the answers to common questions about the
\href{https://www.nytimes.com/2020/06/02/business/economy/major-employers-coronavirus-relief.html}{relief
package}. We'll update this article as we know more.

More information on getting assistance can be found at our
\href{https://www.nytimes.com/article/coronavirus-money-unemployment.html}{Hub
for Help}.

\hypertarget{stimulus-payments}{%
\subsection{Stimulus Payments}\label{stimulus-payments}}

\textbf{How large will the payments be?}

Most adults will get \$1,200, although some will get less. For every
qualifying child age 16 or under, the payment will be an additional
\$500.

\textbf{How many payments will there be?}

Just one. Future bills could order up additional payments, though.

\textbf{Is there a place I can check to see where my stimulus payment is
and when it is arriving?}

Yes. Go to \href{https://www.irs.gov/coronavirus/get-my-payment}{this
page} on the I.R.S. website.

\textbf{How do I know if I will get the full amount?}

It depends on your income. Single adults with Social Security numbers
who have an adjusted gross income of \$75,000 or less will get the full
amount. Married couples with no children earning \$150,000 or less will
receive a total of \$2,400. And taxpayers filing as head of household
will get the full payment if they earned \$112,500 or less.

Above those income figures, the payment decreases until it stops
altogether for single people earning \$99,000 or married people who have
no children and earn \$198,000. According to the Senate Finance
Committee, a family with two children will no longer be eligible for any
payments if its income surpassed \$218,000.

You can't get a payment if someone claims you as a dependent, even if
you're an adult. In any given family and in most instances, everyone
must have a valid Social Security number in order to be eligible. There
is an exception for members of the military.

You can find your adjusted gross income on Line 8b of the 2019 1040
federal tax return.

\textbf{Do college students get anything?}

Not if anyone claims them as a dependent on a tax return. Usually,
students under the age of 24 are dependents in the eyes of the taxing
authorities if a parent pays for at least half of their expenses.

\textbf{What year's income should I be looking at?}

2019. If you haven't prepared a tax return yet, you can use your 2018
return. If you haven't filed that yet, you can use a 2019 Social
Security statement showing your income to see what an employer reported
to the I.R.S.

\textbf{What if my recent income made me ineligible, but I anticipate
being eligible because of a loss of income in 2020? Do I get a payment?}

The plan does not help people in that circumstance now, but you may
benefit once you file your 2020 taxes. That's because the payment is
technically an advance on a tax credit that is available for the entire
year. So it will depend on how much you earn.

And there are many other provisions in the legislation. You may be able
to file for unemployment or for one of the new loans for small business
owners or sole proprietors.

\textbf{Will I have to apply to receive a payment?}

No. If the Internal Revenue Service already has your bank account
information from your 2019 or 2018 return, it will transfer the money to
you via direct deposit based on the recent income-tax figures it already
has.

The payments will also be automatic for people who receive Social
Security retirement, survivor or disability benefits and Supplemental
Security Income. Recipients of federally managed railroad retirement
benefits will receive money automatically, too. But if you receive any
of these benefits and do not file a tax return, you'll need to take an
extra step if you have qualifying children who will make you eligible
for an additional \$500 payment. To collect that, you'll need to enter
additional information through a
\href{http://www.irs.gov/coronavirus/economic-impact-payments}{tool} on
IRS.gov. (Go to the non-filers section.)

\textbf{What if my direct deposit information has changed or I want to
add it for the first time?}

The I.R.S. has set up
\href{https://www.irs.gov/coronavirus/get-my-payment}{a page on its
website} to collect this information. If it doesn't load because of high
demand, keep trying --- or attempt to access it at a time with lower
demand, like late at night or early in the morning. The system is
sensitive; if it doesn't recognize your information, consider all of the
different ways you could render your address and other data.

\textbf{When will the payment arrive?}

Payments have started showing up in bank accounts. Treasury Secretary
Steven Mnuchin has said he expected most people to get their payments by
April 17. Presumably those people using the new portals would not get
money until a few weeks after they are first able to provide their
information.

Many people receiving paper checks will have to wait longer because the
federal government is producing and distributing them in batches.

\textbf{What if I haven't filed tax returns recently? Will that affect
my ability to receive a payment?}

It could.

On April 10, the I.R.S.
\href{https://www.irs.gov/coronavirus/non-filers-enter-payment-info-here}{posted
a notice} and instructions for people whose gross income did not exceed
\$12,200 (\$24,400 for married couples) in 2019 and others who were not
required to file federal income tax returns. In order to receive a
payment, they must fill out a form that is linked from the notice. The
form asks for checking account information; people who have no ability
to receive a direct deposit will get a paper check in the mail instead.

As part of the announcement, the I.R.S.
\href{https://www.irs.gov/newsroom/treasury-irs-launch-new-tool-to-help-non-filers-register-for-economic-impact-payments}{also
said} it was still working on ways to push payments automatically to
certain people who do not normally file a return, including those who
receive veterans' pension payments. People in those categories can use
the new form now to get payments faster if they wish.

The I.R.S. has been quick to reassure those people that they would not
somehow end up owing tax just because they are filing tax returns now,
in order to make it easier to receive this one-time payment.

If you're worried about money that you already owe that you cannot pay,
the I.R.S. recommends consulting a tax professional who can help you
request an alternative payment plan or some other resolution. You may
also be able to
\href{https://www.irs.gov/payments/payment-plans-installment-agreements\#eligibility}{apply
online} without the help of a pro.

\hypertarget{latest-updates-economy}{%
\section{\texorpdfstring{\href{https://www.nytimes.com/live/2020/07/31/business/stock-market-today-coronavirus?action=click\&pgtype=Article\&state=default\&region=MAIN_CONTENT_1\&context=storylines_live_updates}{Latest
Updates:
Economy}}{Latest Updates: Economy}}\label{latest-updates-economy}}

\href{https://www.nytimes.com/live/2020/07/31/business/stock-market-today-coronavirus?action=click\&pgtype=Article\&state=default\&region=MAIN_CONTENT_1\&context=storylines_live_updates\#kodaks-chief-executive-was-given-stock-options-then-the-share-price-spiked-1000-percent}{34h
ago}

\href{https://www.nytimes.com/live/2020/07/31/business/stock-market-today-coronavirus?action=click\&pgtype=Article\&state=default\&region=MAIN_CONTENT_1\&context=storylines_live_updates\#kodaks-chief-executive-was-given-stock-options-then-the-share-price-spiked-1000-percent}{Kodak's
chief executive was given stock options. Then the share price spiked
1,000 percent.}

\href{https://www.nytimes.com/live/2020/07/31/business/stock-market-today-coronavirus?action=click\&pgtype=Article\&state=default\&region=MAIN_CONTENT_1\&context=storylines_live_updates\#fitch-ratings-downgrades-its-outlook-on-us-debt}{37h
ago}

\href{https://www.nytimes.com/live/2020/07/31/business/stock-market-today-coronavirus?action=click\&pgtype=Article\&state=default\&region=MAIN_CONTENT_1\&context=storylines_live_updates\#fitch-ratings-downgrades-its-outlook-on-us-debt}{Fitch
Ratings downgrades its outlook on U.S. debt.}

\href{https://www.nytimes.com/live/2020/07/31/business/stock-market-today-coronavirus?action=click\&pgtype=Article\&state=default\&region=MAIN_CONTENT_1\&context=storylines_live_updates\#us-sanctions-more-chinese-officials-over-human-rights-violations-as-tensions-flare}{44h
ago}

\href{https://www.nytimes.com/live/2020/07/31/business/stock-market-today-coronavirus?action=click\&pgtype=Article\&state=default\&region=MAIN_CONTENT_1\&context=storylines_live_updates\#us-sanctions-more-chinese-officials-over-human-rights-violations-as-tensions-flare}{U.S.
sanctions more Chinese officials over human rights violations as
tensions flare}

\href{https://www.nytimes.com/live/2020/07/31/business/stock-market-today-coronavirus?action=click\&pgtype=Article\&state=default\&region=MAIN_CONTENT_1\&context=storylines_live_updates}{See
more updates}

More live coverage:
\href{https://www.nytimes.com/2020/08/01/world/coronavirus-covid-19.html?action=click\&pgtype=Article\&state=default\&region=MAIN_CONTENT_1\&context=storylines_live_updates}{Global}

\textbf{Will most people who are receiving Social Security retirement
and disability payments each month also get a stimulus payment?}

Yes.

\textbf{Will eligible unemployed people get these stimulus payments?}

Yes.

\textbf{Will U.S. citizens living abroad get a payment?}

Yes, as long as they meet the income requirements and have a Social
Security number.

\textbf{If my payment doesn't come soon, how can I be sure that it
wasn't misdirected?}

During the week of April 13, the I.R.S. intends to release an online
tool that will allow you to track the status of your payment.

According to the relief law, you will get a paper notice in the mail no
later than a few weeks after your payment has been disbursed. That
notice will contain information about where the payment ended up and in
what form it was made. If you couldn't locate the payment at that point,
it would be time to contact the I.R.S. using the information on the
notice.

\textbf{Do I have to pay income taxes on the amount of my payment?}

No.

\textbf{If my income tax refunds are currently being garnished because
of a student loan default, will this payment be garnished as well?}

No. In fact, the bill temporarily suspends nearly all efforts to garnish
tax refunds to repay debts, including those to the I.R.S. itself. But
this waiver may not apply to people who are behind on child support.

\textbf{What about other garnishment orders that I may be subject to?}

The National Consumer Law Center has published
\href{https://library.nclc.org/protecting-against-creditor-seizure-stimulus-checks}{a
guide} for people who may be in this spot. The short version:
``Consumers should consider withdrawing the funds in cash or
transferring the funds electronically or through a debit card payment to
pay for necessary goods or services'' immediately after the payment
arrives.

\hypertarget{unemployment-benefits}{%
\subsection{Unemployment Benefits}\label{unemployment-benefits}}

\textbf{Who will be covered by the expanded program?}

The plan wraps in far more workers than are usually eligible for
unemployment benefits, including self-employed people and part-time
workers.

The bottom line: Those who are unemployed, are partly unemployed or
cannot work for a wide variety of coronavirus-related reasons will be
more likely to receive benefits.

\textbf{How much will I receive?}

It depends on your state.

Benefits will be expanded in an attempt to replace the average worker's
paycheck, explained Andrew Stettner, a senior fellow at the Century
Foundation, a public policy research group. The average worker earns
about \$1,000 a week, and unemployment benefits often replace roughly 40
to 45 percent of that. The expansion will pay an extra amount to fill
the gap.

Eligible workers will get an extra \$600 per week on top of their state
benefit, until July 31. But some states are more generous than others.
According to the Century Foundation, the maximum weekly benefit in
Alabama is \$275, but it's \$450 in California and \$713 in New Jersey.

So let's say a worker was making \$1,100 per week in New York; she'd be
eligible for the maximum state unemployment benefit of \$504 per week.
Under the new expansion, she gets an additional \$600 of federal
pandemic unemployment compensation, for a total of \$1,104, essentially
replacing her original paycheck.

States have the option of providing the entire amount in one payment, or
sending the extra portion separately. But it must all be done on the
same weekly basis.

\textbf{Do I have to apply for the extra \$600 separately?}

No. Eligibility depends on whether you qualify for state or other
federal unemployment benefits.

\textbf{Will I get the full \$600?}

If you're eligible for at least \$1 of state-level or federal
unemployment compensation, you get the full \$600, according to the
Labor Department.

\textbf{Are gig workers, freelancers and independent contractors
covered?}

Yes,
\href{https://www.nytimes.com/article/self-employed-workers-unemployment-coronavirus-stimulus-package.html}{self-employed
people are newly eligible} for unemployment benefits for up to 39 weeks
through the so-called pandemic unemployment assistance program, which
will be administered through the states.

Benefit amounts will be calculated based on previous income, using a
formula from the
\href{https://www.law.cornell.edu/cfr/text/20/625.6}{Disaster
Unemployment Assistance}
\href{https://www.benefits.gov/benefit/597}{program}, according to a
congressional aide. There will be a minimum benefit equal to one-half
the state's average weekly unemployment insurance amount. The national
average is about \$190 per week, the National Employment Law Project
said.

Self-employed workers will also be eligible for the additional \$600
weekly benefit provided by the federal government.

\textbf{What if I'm a part-time worker who lost my job because of a
coronavirus reason, but my state doesn't cover part-time workers? Am I
still eligible?}

Yes. Part-time workers are eligible for benefits, but the benefit amount
and how long benefits will last depend on your state. They are also
eligible for the additional \$600 weekly benefit.

\textbf{What if I have Covid-19 or need to care for a family member who
has it?}

If you've received a diagnosis, are experiencing symptoms or are seeking
a diagnosis --- and you're unemployed, partly unemployed or cannot work
as a result --- you will be covered. The same goes if you must care for
a member of your family or household who has received a diagnosis.

\textbf{What if my child's school or day care shut down?}

If you rely on a school, a day care or another facility to care for a
child, elderly parent or another household member so that you can work
--- and that facility has been shut down because of coronavirus --- you
are eligible.

\textbf{What if I've been advised by a health care provider to
quarantine myself because of exposure to coronavirus}? \textbf{And what
about broader orders to stay home?}

People who must self-quarantine are covered. The legislation also says
that individuals who are unable to get to work because of a quarantine
imposed as a result of the outbreak are eligible.

\textbf{I was about to start a new job and now can't get there because
of an outbreak.}

You're eligible for benefits. You will also be covered if you were
immediately laid off from a new job and did not have a sufficient work
history to qualify for benefits under normal circumstances.

\textbf{I had to quit my job as a direct result of coronavirus. Would I
be eligible to apply for benefits?}

It depends. Let's say your employer didn't lay you off but you had to
quit because of a quarantine recommended by a health care provider, or
because your child's day care closed and you're the primary caregiver.
Situations like that are covered.

But this provision wasn't intended to cover people who quit (or want to
quit) because they fear that continuing to work puts them at risk of
contracting coronavirus, according to congressional aides.

\textbf{My employer shut down my workplace because of coronavirus. Am I
eligible?}

Yes. If you are unemployed, partly unemployed or unable to work because
your employer closed down, you're covered under the bill.

\textbf{The breadwinner of my household has died as a result of
coronavirus. I relied on that person for income, and I'm not working. Is
that covered?}

Yes.

\textbf{Whom does the bill leave out?}

Workers who are able to work from home, and those receiving paid sick
leave or paid family leave, are not covered. New entrants to the work
force who cannot find jobs and undocumented workers are also ineligible.

\textbf{How long will the payments last?}

That depends on your state, but many people will get at least 39 weeks
through a variety of programs that can kick in at different times. Some
may get a year or more if their state's programs are particularly
generous.

To start, you'll receive your state benefit. (Many states provide 26
weeks of payments, but some offer less. Others use a sliding scale tied
to unemployment levels.) After that, a new 13-week federal benefit will
kick in.

What happens next depends on several factors.

States can offer so-called extended benefits in times of high
unemployment, and each state has its own formula. The number of weeks
offered by each state varies, but it's usually half the length of the
standard benefit. Some offer more.

People who remain unemployed because of a coronavirus-related reason may
be able to tap into an additional pandemic unemployment assistance
program from the federal government that could augment your state-level
benefits, up to a total of 39 weeks. (The 13-week federal benefit
doesn't count toward this total.)

The extra \$600 payment will last for up to four months, covering weeks
of unemployment ending July 31.

\textbf{How long would the broader program last?}

Expanded coverage would be available to workers who were newly eligible
for unemployment benefits for weeks starting on Jan. 27, 2020, and
through Dec. 31, 2020.

\textbf{I'm already receiving unemployment benefits. Will I receive any
help?}

Yes. Even if you're already receiving unemployment benefits for reasons
unrelated to the coronavirus, your state-level benefits will still be
extended by 13 weeks. You will also receive the extra \$600 weekly
benefit from the federal government.

\textbf{My unemployment recently ran out --- could I sign up again?}

Yes. If you've exhausted your benefits, eligible workers can generally
reapply. But how much you get and for how long depends on the state
where you worked. Everyone gets at least another 13 weeks, along with
the extra \$600 payment through July 31.

\textbf{Are any unemployment benefits retroactive?}

Maybe. If you are newly eligible for benefits, you may be able to claim
state-level benefits retroactively, back to Jan. 27. But it will
ultimately be determined by your state, which will consider the date
that you became unemployed and any extenuating circumstances that
prevented you from filing earlier, according to a representative for the
Department of Labor.

People who are already receiving unemployment will not get any
retroactive benefits. If your benefits run out, you'll be eligible for
the added 13 weeks of state-level benefits (as long as you continue to
meet the eligibility criteria).

The extra \$600 payment being paid by the federal government is also not
retroactive.

\textbf{Will this income disqualify me from any other programs?}

Maybe. The additional \$600 benefit counts as income when determining
eligibility for means-tested programs, except for Medicaid and the
Children's Health Insurance Program, known as CHIP.

\textbf{How long will I need to wait for benefits?}

States have been incentivized to waive the one-week waiting period, but
it's unclear how long it will take to process claims --- especially with
state offices so strained by
\href{https://www.nytimes.com/2020/03/19/business/coronavirus-unemployment-states.html}{a
flood of applicants}.

The arrival of the extra \$600 depends on when your state signed an
agreement with the Department of Labor. The week ending April 4 or 5
(depending how your state lays out its calendar) is the first week for
which unemployed workers can claim the new federal benefit.

But that doesn't necessarily mean benefits will flow right away. States
that are unable to immediately pay the federal pandemic benefit after
they sign agreements will pay them retroactively for the weeks you're
entitled to receive them.

\textbf{Are benefits taxable?}

Yes. Benefits are subject to federal income taxes and most state income
taxes, according to the Department of Labor. The same goes for the
\$600. You should be able to elect to have taxes withheld.

Child support obligations can also be deducted from your benefits.

\textbf{What happens if I worked in more than one state? Where do I
file?}

People should apply in the state they worked in last --- and be prepared
to submit documentation for all income earned in every other state as
well, said Michele Evermore, a senior policy analyst for social
insurance at the National Employment Law Project. But if you worked in
one or more places simultaneously, then she suggested starting with the
state you live in.

\hypertarget{student-loans}{%
\subsection{Student Loans}\label{student-loans}}

\textbf{The federal government}
\textbf{\href{https://www.nytimes.com/article/coronavirus-money-unemployment.html\#link-3c2b8d5f}{has
already}} \textbf{waived two months of payments and interest for many
federal student loan borrowers. Is there a bigger break now with the new
bill?}

Yes. Until Sept. 30, there will be automatic payment suspensions for any
student loan held by the federal government, and it's retroactive to
March 13. It is hard to contact many of the loan servicers right now, so
check your account online in the coming weeks. Once you are logged in,
look for the current amount due. There, you should be able to see if the
servicer has reset its billing systems so that you are showing no
payment due.

\textbf{How do I know if my loan is eligible?}

If you've borrowed money from the federal government --- a so-called
direct loan --- in the past 10 years, you're definitely eligible.
According to the
\href{https://ticas.org/our-work/student-debt/}{Institute for College
Access \& Success}, 90 percent of loans (in dollar terms) will be
eligible.

Older Federal Family Educational Loans (F.F.E.L.) that the U.S.
Department of Education does not own are not eligible, nor are Perkins
loans that your school owns (ask your financial aid office if you're not
sure), loans from state agencies, or loans from private lenders like
Discover, Sallie Mae and Wells Fargo. The holders of all those kinds of
loans may be offering their own assistance programs.

Within a few weeks, you are supposed to receive notice indicating what
has happened with your federal loans. You can choose to keep paying down
your principal if you want, and you should contact your loan servicer if
that is the case. Then, after Aug. 1, you should get multiple notices
letting you know about the cessation of the suspension period and that
you may be eligible to enroll in an
\href{https://studentaid.gov/manage-loans/repayment/plans/income-driven}{income-driven
repayment plan}.

\textbf{I'm signed up for automatic payments. Will my servicer turn them
off by itself during this period?}

Yes, that is how it's supposed to happen, according to information that
the Education Department posted.

\textbf{What happens if I've already made a payment since March 13?}

You can ask your loan servicer to refund it to you. But **** keep in
mind that it is taking time for servicers to interpret Education
Department guidance so they can change their websites and update their
customer service representatives.

\textbf{Will my loan servicer charge me interest during the six-month
period?}

The bill says that interest ``shall not accrue'' on the loan during the
suspension period.

After repeated questions, the Education Department said any unpaid
interest from before the period began will not be added to your loan's
principal --- a process known as capitalization --- because of the
six-month suspension.

In short: No one is supposed to have a larger balance after the
suspension than before because of the bill and the relief it offers.

At the end of the suspension, keep a close eye on what your loan
servicer does (or does not do) to put you back into your previous
repayment mode. Servicer errors are common.

\textbf{Will the six-month suspension cost me money, since I'm trying to
qualify for the public service loan forgiveness program by making 120
monthly payments?}

No. The legislation says that your payment count will still go up by one
payment each month during the six-month suspension, even though you will
not actually be making any payments. This is true for all forgiveness or
loan-rehabilitation programs.

\textbf{Is wage garnishment that resulted from being behind on my loan
payments suspended during this six-month period?}

Yes. So is the seizure of tax refunds, the reduction of any other
federal benefit payments and other involuntary collection efforts.

\textbf{Are there changes to the rules if my employer repays some of my
student loans?}

Yes. Some employers do this as an employee benefit. Between the date the
bill is signed and the end of 2020, they can offer up to \$5,250 of
assistance without that money counting as part of the employee's income.
If the employer pays tuition for classes an employee is taking, that
money will also count toward the \$5,250.

\hypertarget{retirement-accounts}{%
\subsection{Retirement Accounts}\label{retirement-accounts}}

\textbf{Which retirement account rules are suspended?}

For the calendar year 2020, no one will have to take a
\href{https://www.irs.gov/retirement-plans/plan-participant-employee/retirement-topics-required-minimum-distributions-rmds}{required
minimum distribution} from any individual retirement accounts or
workplace retirement savings plans, like a 401(k). That way, you aren't
forced to sell investments that may have fallen in value, which would
lock in losses. If you don't need the money now, you can let the
investments sit and hope that they recover.

This change would not affect old-fashioned pensions.

\textbf{What if I have to take money out of my I.R.A. or workplace
retirement plan early?}

You can withdraw up to \$100,000 this year without the usual 10 percent
penalty, as long as it's because of the outbreak.

You will also be able to spread out any income taxes that you owe over
three years from the date you took the distribution. And if you want,
you could put the money back into the account before those three years
are up, even though the rules may normally keep you from making a
contribution that large.

This exception applies only to coronavirus-related withdrawals. You
qualify if you tested positive, a spouse or dependent did or you
experienced a variety of other negative economic consequences related to
the pandemic. Employers can allow workers to self-certify that they are
qualified to pull money from a workplace retirement account.

\textbf{Can I still borrow from my 401(k) or other workplace retirement
plan?}

Yes, and you can take out twice the usual amount. For 180 days after the
bill passes, with certification that you've been affected by the
pandemic, you'll be able to take out a loan of up to \$100,000. Usually
you can't take out more than half your balance, but that rule is
suspended.

If you already have a loan and were supposed to finish repaying it
before Dec. 31, you get an extra year.

\hypertarget{charitable-contributions}{%
\subsection{Charitable Contributions}\label{charitable-contributions}}

\textbf{I want to help people who are suffering from the pandemic. Does
the bill do anything about charitable donations?}

Yes. The bill makes a new deduction available --- and not just for 2020
--- for up to \$300 in annual charitable contributions. It's available
only to people who don't itemize their deductions, and you calculate
this new one by subtracting the amount you give from your gross income.

To qualify, you have to give cash to a qualified charity and not to a
donor-advised fund, which is a charitable account that affluent people
often use to bunch contributions in a particular year in order to
maximize deductions. If you've already given money since Jan. 1, that
contribution counts toward the \$300 cap.

\textbf{I am lucky to have substantial wealth, and I want to give more
to charity than I usually do. Have the limits on charitable deductions
changed?}

Yes, they have. As part of the bill, donors can deduct 100 percent of
their gift against their 2020 adjusted gross income. If you have \$1
million of income, you can give \$1 million to a public charity and
deduct the full amount in 2020.

The new deduction is only for cash gifts that go to a public charity. If
you give cash to, say, your private foundation, the old deduction rules
apply. And while the organizations that manage donor-advised funds are
public charities, you do not get the higher deduction for donating cash
to your donor-advised fund.

If your assets are substantial enough that you can give more than your
income this year, you won't lose the deduction for the excess amount.
You can use it next year, as has always been the case.

\hypertarget{other-features-of-the-bill}{%
\subsection{Other Features of the
Bill}\label{other-features-of-the-bill}}

\textbf{How does the aid for small businesses and nonprofits work?}

Good news here, as you may be eligible for forgivable loans. Our
colleague Emily Flitter covered the details in
\href{https://www.nytimes.com/2020/03/26/business/coronavirus-stimulus-small-business.html}{a
separate article}. Aides to Senator Marco Rubio, Republican of Florida,
also wrote a
\href{https://www.rubio.senate.gov/public/_cache/files/28e8263e-e7d4-4da7-a67b-077c54ba4220/9F7B494B2E355791B24536DC2162CF16.final-one-pager-keeping-american-workers-paid-and-employed-act-.pdf}{one-page
summary} of those provisions.

\textbf{Will there be damage to my credit report if I take advantage of
any virus-related payment relief, including the student loan
suspension?}

No. There is not supposed to be, at least.

The bill states that during the period beginning on Jan. 31 and
continuing 120 days after the end of the national emergency declaration,
lenders and others should mark your credit file as current, even if you
take advantage of payment modifications.

If you had black marks in your file before the virus hit, those will
remain unless you fix the issues during the emergency period.

Credit reporting agencies can make errors. Be sure to
\href{https://www.annualcreditreport.com/index.action}{check your credit
report} a few times each year, especially if you accept any help from
any financial institution or biller this year.

\textbf{What if I find black marks anyway?}

File a dispute with the credit bureau, but it may take a while to fix
them. The Consumer Financial Protection Bureau
\href{https://files.consumerfinance.gov/f/documents/cfpb_credit-reporting-policy-statement_cares-act_2020-04.pdf}{has
told} credit bureaus and others that during the pandemic they can take
longer than the usual 30 to 45 days to meet the dispute-response
deadline, as long as they are making ``good faith'' efforts.

\textbf{Is there any relief for renters in the bill?}

Yes. The bill puts a temporary, nationwide eviction moratorium in place
for any renters whose landlords have mortgages backed or owned by Fannie
Mae, Freddie Mac and other federal entities. About 70 percent of all
mortgages fall into this category. If you want to figure out whether
your landlord has such a mortgage, try plugging the address in to the
\href{https://preservationdatabase.org/about-the-database/}{National
Housing Preservation Database}.

In addition, the bill stipulates that landlords cannot charge any fees
or penalties for nonpayment of rent. The eviction suspension applies
only to nonpayment; damaging your place is still grounds for action.
This moratorium will last for 120 days after the bill passes.

\textbf{Does this bill change any rules for health savings accounts and
health care flexible spending accounts?}

Yes. After \href{https://www.wsj.com/articles/SB106556104757114900}{at
least 15 years} of
\href{https://www.periodequity.org/mission-and-history}{lobbying} and
debate, menstrual products are now eligible for reimbursement.

\textbf{Did the legislation make it illegal for any internet provider to
cut off service to an individual or small business that can't pay its
bills?}

No.

\textbf{Did the legislation make it illegal for utility providers to cut
off service?}

No.

Paul Sullivan contributed reporting.

Advertisement

\protect\hyperlink{after-bottom}{Continue reading the main story}

\hypertarget{site-index}{%
\subsection{Site Index}\label{site-index}}

\hypertarget{site-information-navigation}{%
\subsection{Site Information
Navigation}\label{site-information-navigation}}

\begin{itemize}
\tightlist
\item
  \href{https://help.nytimes.com/hc/en-us/articles/115014792127-Copyright-notice}{©~2020~The
  New York Times Company}
\end{itemize}

\begin{itemize}
\tightlist
\item
  \href{https://www.nytco.com/}{NYTCo}
\item
  \href{https://help.nytimes.com/hc/en-us/articles/115015385887-Contact-Us}{Contact
  Us}
\item
  \href{https://www.nytco.com/careers/}{Work with us}
\item
  \href{https://nytmediakit.com/}{Advertise}
\item
  \href{http://www.tbrandstudio.com/}{T Brand Studio}
\item
  \href{https://www.nytimes.com/privacy/cookie-policy\#how-do-i-manage-trackers}{Your
  Ad Choices}
\item
  \href{https://www.nytimes.com/privacy}{Privacy}
\item
  \href{https://help.nytimes.com/hc/en-us/articles/115014893428-Terms-of-service}{Terms
  of Service}
\item
  \href{https://help.nytimes.com/hc/en-us/articles/115014893968-Terms-of-sale}{Terms
  of Sale}
\item
  \href{https://spiderbites.nytimes.com}{Site Map}
\item
  \href{https://help.nytimes.com/hc/en-us}{Help}
\item
  \href{https://www.nytimes.com/subscription?campaignId=37WXW}{Subscriptions}
\end{itemize}
