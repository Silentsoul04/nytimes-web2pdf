Sections

SEARCH

\protect\hyperlink{site-content}{Skip to
content}\protect\hyperlink{site-index}{Skip to site index}

\href{https://www.nytimes.com/section/health}{Health}

\href{https://myaccount.nytimes.com/auth/login?response_type=cookie\&client_id=vi}{}

\href{https://www.nytimes.com/section/todayspaper}{Today's Paper}

\href{/section/health}{Health}\textbar{}Trump Says He `Aced' a Cognitive
Test. What Does That Really Mean?

\url{https://nyti.ms/30xbWP0}

\begin{itemize}
\item
\item
\item
\item
\item
\item
\end{itemize}

\begin{itemize}
\item
  \href{https://www.nytimes.com/2020/07/31/us/elections/biden-vs-trump.html?action=click\&pgtype=Article\&state=default\&region=TOP_BANNER\&context=storylines_menu}{Election
  Updates}
\item
  \href{https://www.nytimes.com/article/biden-vice-president-2020.html?action=click\&pgtype=Article\&state=default\&region=TOP_BANNER\&context=storylines_menu}{Biden's
  V.P. Search}
\item
  \href{https://www.nytimes.com/interactive/2020/07/24/us/politics/trump-biden-campaign-donors.html?action=click\&pgtype=Article\&state=default\&region=TOP_BANNER\&context=storylines_menu}{Map
  of Donations}
\item
  \href{https://www.nytimes.com/interactive/2020/us/elections/delegate-count-primary-results.html?action=click\&pgtype=Article\&state=default\&region=TOP_BANNER\&context=storylines_menu}{Delegate
  Count}
\item
  \href{https://www.nytimes.com/interactive/2019/us/politics/2020-presidential-candidates.html?action=click\&pgtype=Article\&state=default\&region=TOP_BANNER\&context=storylines_menu}{The
  Candidates}
\item
  \href{https://www.nytimes.com/newsletters/politics?action=click\&pgtype=Article\&state=default\&region=TOP_BANNER\&context=storylines_menu}{Politics
  Newsletter}
\end{itemize}

Advertisement

\protect\hyperlink{after-top}{Continue reading the main story}

Supported by

\protect\hyperlink{after-sponsor}{Continue reading the main story}

\hypertarget{trump-says-he-aced-a-cognitive-test-what-does-that-really-mean}{%
\section{Trump Says He `Aced' a Cognitive Test. What Does That Really
Mean?}\label{trump-says-he-aced-a-cognitive-test-what-does-that-really-mean}}

Experts say one popular test that the president might have taken is by
no means definitive, or even diagnostic.

\includegraphics{https://static01.nyt.com/images/2020/07/20/science/20SCI-TRUMPTESTS/merlin_174434709_c97e4c5d-51a8-424d-a4ed-1f23d185eb3a-articleLarge.jpg?quality=75\&auto=webp\&disable=upscale}

\href{https://www.nytimes.com/by/gina-kolata}{\includegraphics{https://static01.nyt.com/images/2018/02/16/multimedia/author-gina-kolata/author-gina-kolata-thumbLarge.jpg}}

By \href{https://www.nytimes.com/by/gina-kolata}{Gina Kolata}

\begin{itemize}
\item
  July 20, 2020
\item
  \begin{itemize}
  \item
  \item
  \item
  \item
  \item
  \item
  \end{itemize}
\end{itemize}

In
\href{https://www.nytimes.com/2020/07/10/us/politics/trump-cognitive-test-health.html}{recent
interviews}, President Trump has been boasting about a
\href{https://www.nytimes.com/2020/07/22/us/politics/trump-cognitive-test-results.html}{cognitive
test} that he took recently and ``aced.'' It's not the first time --- he
made
\href{https://www.nytimes.com/2018/01/19/health/trump-cognitive-screening-dementia.html}{similar
remarks in January 2018}. But he once again has not said what test he
took, or what his score was, making it difficult to know what to make of
his claims.

He has, however, provided a few clues.

In
\href{https://www.nytimes.com/2020/07/19/us/politics/trump-fox-interview-coronavirus-race.html}{an
interview with Chris Wallace on ``Fox News Sunday},'' in which Mr.
Wallace said one question was to identify an elephant, Mr. Trump did not
contradict him. One popular test, the Montreal
\href{https://www.nytimes.com/2020/07/10/us/politics/trump-cognitive-test-health.html}{Cognitive}
Assessment, or MOCA, has a drawing of three animals that patients are
asked to identify. It is
\href{https://www.mocatest.org/wp-content/uploads/2017/01/MoCA-New-Test-8.1-2017-04.pdf}{a
10-minute screening exam} meant to highlight possible problems with
thinking and memory.

But it is by no means definitive, or even diagnostic, experts pointed
out. And as the president tries to draw a contrast between his mental
faculties and those of Vice President Joseph R. Biden Jr., a fellow
septuagenarian he expects to face off against in November's election,
some experts criticize his politicization of such cognitive screening
tests.

``The way our president is having a conversation about mental health is
not helpful,'' said Dr. Jason Karlawish, a dementia researcher at the
University of Pennsylvania's Perelman School of Medicine

``You would think he would understand clearly what the test result was
and why the test was done,'' Dr. Karlawish said, ``and not turn it into
a competition about mental health.''

From January 2018, when we
\href{https://www.nytimes.com/2018/01/19/health/trump-cognitive-screening-dementia.html}{originally
examined this topic}, here are some answers to questions about cognitive
exams, what they measure, and how specialists decide whether a patient
really is impaired.

\hypertarget{heres-what-you-need-to-know}{%
\subsubsection{Here's what you need to
know:}\label{heres-what-you-need-to-know}}

\begin{itemize}
\tightlist
\item
  \protect\hyperlink{link-65e034cc}{What is the MOCA?}
\item
  \protect\hyperlink{link-1a1aa9aa}{What does the test ask?}
\item
  \protect\hyperlink{link-1777dbe1}{Do medical societies recommend
  cognitive screenings?}
\item
  \protect\hyperlink{link-2fe34d74}{Shouldn't a perfect score be
  reassuring?}
\item
  \protect\hyperlink{link-31dbf01d}{So how do doctors detect cognitive
  problems?}
\item
  \protect\hyperlink{link-5943b1a8}{What about Alzheimer's?}
\end{itemize}

\hypertarget{what-is-the-moca}{%
\subsection{What is the MOCA?}\label{what-is-the-moca}}

This screening test was designed about 20 years ago as a possible
replacement for another test, the
\href{https://www.uml.edu/docs/Mini\%20Mental\%20State\%20Exam_tcm18-169319.pdf}{Mini-Mental
State Examination}, which had been widely used since the 1970s to look
for outright dementia. The MOCA is used in all 31 of the National
Institute on Aging's Alzheimer Disease Centers.

While there are many such screening tests, the MOCA is gaining
acceptance because it is a bit harder than the Mini-Mental and can pick
up problems that occur in the earliest stage of dementia, mild cognitive
impairment --- a sort of everyday forgetfulness.

About one in five people over age 65 have M.C.I., and roughly a third
will develop Alzheimer's disease within five years.

\hypertarget{latest-updates-2020-election}{%
\section{\texorpdfstring{\href{https://www.nytimes.com/2020/07/31/us/elections/biden-vs-trump.html?action=click\&pgtype=Article\&state=default\&region=MAIN_CONTENT_1\&context=storylines_live_updates}{Latest
Updates: 2020
Election}}{Latest Updates: 2020 Election}}\label{latest-updates-2020-election}}

Updated 2020-08-01T01:26:45.732Z

\begin{itemize}
\tightlist
\item
  \href{https://www.nytimes.com/2020/07/31/us/elections/biden-vs-trump.html?action=click\&pgtype=Article\&state=default\&region=MAIN_CONTENT_1\&context=storylines_live_updates\#link-29fdff45}{Kamala
  Harris, a top vice-presidential contender, confronts double
  standards.}
\item
  \href{https://www.nytimes.com/2020/07/31/us/elections/biden-vs-trump.html?action=click\&pgtype=Article\&state=default\&region=MAIN_CONTENT_1\&context=storylines_live_updates\#link-13ec3d9c}{Karen
  Bass and Susan Rice are rising on Biden's vice-presidential
  shortlist.}
\item
  \href{https://www.nytimes.com/2020/07/31/us/elections/biden-vs-trump.html?action=click\&pgtype=Article\&state=default\&region=MAIN_CONTENT_1\&context=storylines_live_updates\#link-49e9a016}{Trump
  says Russian bounties to kill U.S. troops `never took place.'}
\end{itemize}

\href{https://www.nytimes.com/2020/07/31/us/elections/biden-vs-trump.html?action=click\&pgtype=Article\&state=default\&region=MAIN_CONTENT_1\&context=storylines_live_updates}{See
more updates}

\hypertarget{what-does-the-test-ask}{%
\subsection{What does the test ask?}\label{what-does-the-test-ask}}

MOCA has approximately 30 questions meant to briefly assess memory,
attention and concentration, control and self-regulation, and other
mental skills.

To test memory, for example, the examiner reads five words at a rate of
one per second and asks the subject to repeat them immediately and then
again after some time has passed.

To assess attention and concentration, subjects are read a list of five
digits and asked to repeat them in the order they were provided and then
in reverse order. The subjects also are asked to count backward from 100
in increments of 7.

Other exercises include drawing a clock with the hands pointing to
11:10, and identifying a lion, rhino or camel. A perfect score is 30. A
score from 26 to 30 is considered normal.

\includegraphics{https://static01.nyt.com/images/2018/01/20/science/20SCI-TRUMPTESTS2/merlin_132534770_353a0abb-2b1d-457e-ab28-3562cac188ab-articleLarge.jpg?quality=75\&auto=webp\&disable=upscale}

\hypertarget{do-medical-societies-recommend-cognitive-screenings}{%
\subsection{Do medical societies recommend cognitive
screenings?}\label{do-medical-societies-recommend-cognitive-screenings}}

No. Such exams are not like mammograms for breast cancer and
colonoscopies for colon cancer. With those tests, doctors can get a
diagnosis and begin treatment.

But they are only part of an assessment of the mental functioning of an
older adult. It can be more valuable to ask family members or others who
know the patient well whether the person has been inefficient at tasks
they once did well, like negotiating a new route when driving or
following a recipe.

Screening tests like the MOCA cannot rule out declines in reasoning or
memory, or difficulties with planning or judgment. The test is just too
blunt an instrument, and for many high-functioning people, too easy.

For that reason, dementia specialists say they would not make a
diagnosis based on a screening exam like the MOCA.

Nonetheless, Medicare recipients are often given cognitive screenings,
Dr. Karlawish said. That is because Congress instituted a requirement
that Medicare cover a brief cognitive screening test as part of the
annual wellness exam.

\hypertarget{shouldnt-a-perfect-score-be-reassuring}{%
\subsection{Shouldn't a perfect score be
reassuring?}\label{shouldnt-a-perfect-score-be-reassuring}}

Maybe. But the test is not that difficult, and the problem with a single
test is that the doctor doesn't know what the subject's starting point
was. Usually it's the trend over time that suggests a problem.

And even then, the tests can be too blunt an instrument to detect
declines in many highly educated people for whom, say, counting backward
from 100 by 7's is not a challenge until dementia is well established.

\hypertarget{so-how-do-doctors-detect-cognitive-problems}{%
\subsection{So how do doctors detect cognitive
problems?}\label{so-how-do-doctors-detect-cognitive-problems}}

It's not easy. What physicians look for is a slow decline. They start by
simply talking to the patient: Has she noticed memory problems, or
issues with judgment or reasoning?

It's also important for the physician to talk to someone who knows the
patient well, because people who are slipping cognitively do not always
recognize it. ``Lack of awareness or insight can be part of the
package'' of dementia, said Dr. Ronald Petersen, director of the
Alzheimer's Disease Research Center at the Mayo Clinic in Rochester,
Minn. (He emphasized that he was speaking in general terms, not
specifically about President Trump's case.)

If the doctor is concerned, and if a family member also says the subject
is forgetful or repeating himself, and if this behavior is becoming a
pattern --- all those factors will influence the decision to ``pursue
this to the next level,'' Dr. Petersen said.

Some patients simply prefer not to know if they are developing dementia.
But those who do are given a neuropsychological test much more difficult
and intense than the MOCA. And doctors will repeat it over time.

In such a test, for example, the examiner reads a short story and asks
the subject to repeat it. Thirty minutes later, the subject is asked to
repeat the story again.

The subject is also asked to draw geometric shapes and to remember them
a half-hour later. The examiner may ask the subject to recall a list of
15 words as many as five times, and then recall them 30 minutes later.

\hypertarget{what-about-alzheimers}{%
\subsection{What about Alzheimer's?}\label{what-about-alzheimers}}

The results of neuropsychological tests can tell doctors how a subject
is performing relative to others of the same age, sex and education
level. If the doctor thinks something is amiss, a clinical exam might
follow to figure out what might be causing the problem.

Most cases of dementia result from Alzheimer's disease. An M.R.I. scan
can help with diagnosis. It can detect a stroke and other conditions. It
also can determine if the hippocampus, the memory center of the brain,
is shrinking, as happens in Alzheimer's.

A PET scan that uses glucose measures the activity of brain cells. Cells
starting to falter and die, especially in certain areas of the brain,
may be a sign of Alzheimer's disease.

Neither scan is itself diagnostic, Dr. Petersen said. Instead, the
results add to the weight of evidence suggestive of Alzheimer's disease.

Another test, which costs \$5,000 to \$7,000 and generally is not
covered by insurance, is a scan to look for amyloid protein in the
brain. Occasionally people have these accumulations but not dementia.

But because amyloid is a part of the Alzheimer's pathology, a lack of it
means the subject does not have Alzheimer's disease.

\textbf{\emph{{[}}\href{http://on.fb.me/1paTQ1h}{\emph{Like the Science
Times page on Facebook.}}} ****** \emph{\textbar{} Sign up for the}
\textbf{\href{http://nyti.ms/1MbHaRU}{\emph{Science Times
newsletter.}}\emph{{]}}}

\hypertarget{our-2020-election-guide}{%
\section{Our 2020 Election Guide}\label{our-2020-election-guide}}

Updated July 31, 2020

\begin{itemize}
\item
  \begin{center}\rule{0.5\linewidth}{\linethickness}\end{center}

  \hypertarget{the-latest}{%
  \subsection{The Latest}\label{the-latest}}

  \begin{itemize}
  \tightlist
  \item
    President Trump's assault on the Postal Service is intersecting with
    his attacks on mail-in voting.
    \href{https://www.nytimes.com/2020/07/31/us/politics/trump-usps-mail-delays.html?action=click\&pgtype=Article\&state=default\&region=BELOW_MAIN_CONTENT\&context=storylines_guide}{Voting
    rights groups say it is a recipe for disaster.}
  \end{itemize}
\item
  \begin{center}\rule{0.5\linewidth}{\linethickness}\end{center}

  \hypertarget{bidens-vp-search}{%
  \subsection{Biden's V.P. Search}\label{bidens-vp-search}}

  \begin{itemize}
  \tightlist
  \item
    \href{https://www.nytimes.com/article/biden-vice-president-2020.html?action=click\&pgtype=Article\&state=default\&region=BELOW_MAIN_CONTENT\&context=storylines_guide}{Here
    are 13 women} who have been under consideration to be Joe Biden's
    running mate, and why each might be chosen --- and might not be.
  \end{itemize}
\item
  \begin{center}\rule{0.5\linewidth}{\linethickness}\end{center}

  \hypertarget{keep-up-with-our-coverage}{%
  \subsection{Keep Up With Our
  Coverage}\label{keep-up-with-our-coverage}}

  \begin{itemize}
  \tightlist
  \item
    Get an
    \href{https://www.nytimes.com/newsletters/politics?action=click\&pgtype=Article\&state=default\&region=BELOW_MAIN_CONTENT\&context=storylines_guide}{email}
    recapping the day's news
  \end{itemize}

  \begin{itemize}
  \tightlist
  \item
    Download our mobile app on
    \href{https://apps.apple.com/us/app/nytimes/id284862083?ls=1\&mat_click_id=5c79ae7455014fd1bd66b5610c05b8f2-20191112-16948\&referrer=mat_click_id\%3D5c79ae7455014fd1bd66b5610c05b8f2-20191112-16948\%26link_click_id\%3D722930677036718082}{iOS}
    and
    \href{http://a.localytics.com/android?id=com.nytimes.android\&referrer=utm_source\%3Dother_nyt_mobile_web\%26utm_medium\%3DWeb\%2520page\%26utm_term\%3DGeneral\%2520Mobile\%2520Page\%26utm_campaign\%3DNYT\%2520Mobile\%2520General\%2520Page}{Android}
    and turn on Breaking News and Politics alerts
  \end{itemize}
\end{itemize}

Advertisement

\protect\hyperlink{after-bottom}{Continue reading the main story}

\hypertarget{site-index}{%
\subsection{Site Index}\label{site-index}}

\hypertarget{site-information-navigation}{%
\subsection{Site Information
Navigation}\label{site-information-navigation}}

\begin{itemize}
\tightlist
\item
  \href{https://help.nytimes.com/hc/en-us/articles/115014792127-Copyright-notice}{©~2020~The
  New York Times Company}
\end{itemize}

\begin{itemize}
\tightlist
\item
  \href{https://www.nytco.com/}{NYTCo}
\item
  \href{https://help.nytimes.com/hc/en-us/articles/115015385887-Contact-Us}{Contact
  Us}
\item
  \href{https://www.nytco.com/careers/}{Work with us}
\item
  \href{https://nytmediakit.com/}{Advertise}
\item
  \href{http://www.tbrandstudio.com/}{T Brand Studio}
\item
  \href{https://www.nytimes.com/privacy/cookie-policy\#how-do-i-manage-trackers}{Your
  Ad Choices}
\item
  \href{https://www.nytimes.com/privacy}{Privacy}
\item
  \href{https://help.nytimes.com/hc/en-us/articles/115014893428-Terms-of-service}{Terms
  of Service}
\item
  \href{https://help.nytimes.com/hc/en-us/articles/115014893968-Terms-of-sale}{Terms
  of Sale}
\item
  \href{https://spiderbites.nytimes.com}{Site Map}
\item
  \href{https://help.nytimes.com/hc/en-us}{Help}
\item
  \href{https://www.nytimes.com/subscription?campaignId=37WXW}{Subscriptions}
\end{itemize}
