Sections

SEARCH

\protect\hyperlink{site-content}{Skip to
content}\protect\hyperlink{site-index}{Skip to site index}

\href{https://www.nytimes.com/column/out-there}{Out There}

\href{https://myaccount.nytimes.com/auth/login?response_type=cookie\&client_id=vi}{}

\href{https://www.nytimes.com/section/todayspaper}{Today's Paper}

Advertisement

\protect\hyperlink{after-top}{Continue reading the main story}

Supported by

\protect\hyperlink{after-sponsor}{Continue reading the main story}

\hypertarget{science}{%
\subsubsection{\texorpdfstring{\href{/section/science}{Science}}{Science}}\label{science}}

\includegraphics{https://static01.nyt.com/images/2018/07/30/multimedia/author-dennis-overbye/author-dennis-overbye-thumbLarge.png}

\hypertarget{out-there}{%
\section{Out There}\label{out-there}}

\hypertarget{a-collection-of-out-there-columns-published-in-the-new-york-times-more}{%
\subsection{A collection of ``Out There'' columns published in The New
York Times.
More**}\label{a-collection-of-out-there-columns-published-in-the-new-york-times-more}}

\hypertarget{a-collection-of-out-there-columns-published-in-the-new-york-times-more-1}{%
\subsection{A collection of ``Out There'' columns published in The New
York Times.
More**}\label{a-collection-of-out-there-columns-published-in-the-new-york-times-more-1}}

Mr. Overbye's reporting can range from zero-gravity fashion shows and
science in the movies to the status of Pluto, the death of the Earth and
the fate of the universe.

He joined The Times in 1998 as deputy science editor, resuming a
newspaper career that had been disrupted in the ninth grade when he lost
his job as editor of the junior high paper after being in a classroom
after hours where erasers were thrown. In the meantime, he graduated
from M.I.T. with a physics degree, failed to finish a novel and worked
as a writer and editor at Sky and Telescope and Discover magazines.

He has written two books: "Lonely Hearts of the Cosmos, The Scientific
Search for the Secret of the Universe" (HarperCollins 1991, and Little,
Brown, 1999), and "Einstein in Love, A Scientific Romance" (Viking,
2000). As a result of the latter, there are few occasions for which he
cannot rustle up a quotation - appropriate or not - from Albert
Einstein.

In 2001, realizing that the reporters were having more fun and got to
take cooler trips than editors, he switched to being a reporter. He has
been covering the universe for more than 30 years, but lately he
professes to be amazed that a huge chunk of his work is devoted to two
topics that did not exist only a decade or so ago: the proliferation of
planets beyond our own solar system; and the mysterious dark energy that
seems to be souping up the expansion of the universe and spurring
metaphysical-sounding debates among astronomers and physicists.

He lives with his wife, Nancy, and daughter, Mira, in Morningside
Heights. In their house, he reports, Pluto is still a planet.

\begin{itemize}
\tightlist
\item
  \protect\hyperlink{stream-panel}{Latest}
\item
  Search
\end{itemize}

\begin{enumerate}
\def\labelenumi{\arabic{enumi}.}
\item
  \href{/2020/07/27/science/mars-sarah-stewart-johnson.html}{}

  \includegraphics{https://static01.nyt.com/images/2020/07/27/science/27SCI-OUTTHERE-MARS-promo/27SC-OUTTHERE-MARS-promo-thumbWide.jpg?quality=75\&auto=webp\&disable=upscale}

  \hypertarget{coming-of-age-on-mars}{%
  \subsection{Coming of Age on Mars}\label{coming-of-age-on-mars}}

  In a new book, planetary scientist Sarah Stewart Johnson recalls how
  the Red Planet drew her to become a scientist.

  By Dennis Overbye
\item
  \href{/2020/07/10/science/astronomy-galaxies-attractor-universe.html}{}

  \includegraphics{https://static01.nyt.com/images/2020/07/21/science/10cosmicwall-mw/10cosmicwall-mw-thumbWide-v3.jpg?quality=75\&auto=webp\&disable=upscale}

  \hypertarget{beyond-the-milky-way-a-galactic-wall}{%
  \subsection{Beyond the Milky Way, a Galactic
  Wall}\label{beyond-the-milky-way-a-galactic-wall}}

  Astronomers have discovered a vast assemblage of galaxies hidden
  behind our own, in the ``zone of avoidance.''

  By Dennis Overbye
\item
  \href{/2020/06/25/science/black-hole-collision-ligo.html}{}

  \includegraphics{https://static01.nyt.com/images/2020/06/30/science/25SCI-BLACKHOLE/25SCI-BLACKHOLE-thumbWide.jpg?quality=75\&auto=webp\&disable=upscale}

  \hypertarget{two-black-holes-colliding-not-enough-make-it-three}{%
  \subsection{Two Black Holes Colliding Not Enough? Make It
  Three}\label{two-black-holes-colliding-not-enough-make-it-three}}

  Astronomers claim to have seen a flash from the merger of two black
  holes within the maelstrom of a third, far bigger one.

  By Dennis Overbye
\item
  \href{/2020/06/24/science/black-hole-ligo-gravitational.html}{}

  \includegraphics{https://static01.nyt.com/images/2020/07/07/science/23SCI-OUTTHERE-LIGO/23SCI-OUTTHERE-LIGO-thumbWide.jpg?quality=75\&auto=webp\&disable=upscale}

  \hypertarget{a-black-holes-lunch-provides-a-treat-for-astronomers}{%
  \subsection{A Black Hole's Lunch Provides a Treat for
  Astronomers}\label{a-black-holes-lunch-provides-a-treat-for-astronomers}}

  Scientists have discovered the heaviest known neutron star, or maybe
  the lightest known black hole: ``Either way it breaks a record.''

  By Dennis Overbye
\item
  \href{/2020/06/17/science/xenon-axions-neutrinos-tritium.html}{}

  \includegraphics{https://static01.nyt.com/images/2020/06/23/science/17SCI-OUTTHERE-XENON1/17SCI-OUTTHERE-XENON1-thumbWide.jpg?quality=75\&auto=webp\&disable=upscale}

  \hypertarget{seeking-dark-matter-they-detected-another-mystery}{%
  \subsection{Seeking Dark Matter, They Detected Another
  Mystery}\label{seeking-dark-matter-they-detected-another-mystery}}

  Do signals from beneath an Italian mountain herald a revolution in
  physics?

  By Dennis Overbye
\item
  \href{/2020/06/15/science/oumuamua-astronomy-comets.html}{}

  \includegraphics{https://static01.nyt.com/images/2020/06/16/science/16SCI-OUTTHERE-ICEBERG/16SCI-OUTTHERE-ICEBERG-thumbWide.jpg?quality=75\&auto=webp\&disable=upscale}

  \hypertarget{oumuamua-neither-comet-nor-asteroid-but-a-cosmic-iceberg}{%
  \subsection{Oumuamua: Neither Comet nor Asteroid, but a Cosmic
  Iceberg}\label{oumuamua-neither-comet-nor-asteroid-but-a-cosmic-iceberg}}

  A new study suggests the interloper may have arisen in an interstellar
  cloud, where stars are sometimes born.

  By Dennis Overbye
\item
  \href{/2020/06/05/science/black-hole-astronomy.html}{}

  \includegraphics{https://static01.nyt.com/images/2020/06/09/science/05SCI-OUTTHERE-BLACKHOLE1/05SCI-OUTTHERE-BLACKHOLE1-thumbWide.jpg?quality=75\&auto=webp\&disable=upscale}

  \hypertarget{watch-this-black-hole-blow-bubbles}{%
  \subsection{Watch This Black Hole Blow
  Bubbles}\label{watch-this-black-hole-blow-bubbles}}

  A black hole was seen shooting electrified gas and energy into space.
  Each blob contained about 400 million billion pounds of matter.

  By Dennis Overbye
\item
  \href{/2020/06/02/science/coronavirus-space-travel-colonization.html}{}

  \includegraphics{https://static01.nyt.com/images/2020/06/02/science/00SCI-OUTTHERE-GREATFILTER1/00SCI-OUTTHERE-GREATFILTER1-thumbWide.jpg?quality=75\&auto=webp\&disable=upscale}

  \hypertarget{going-viral-or-not-in-the-milky-way}{%
  \subsection{Going Viral, or Not, in the Milky
  Way}\label{going-viral-or-not-in-the-milky-way}}

  Is the pandemic a rehearsal for our own cosmic mortality?

  By Dennis Overbye
\item
  \href{/2020/05/20/science/nancy-grace-roman-telescope.html}{}

  \includegraphics{https://static01.nyt.com/images/2020/05/20/science/20OUTTHERE-TELESCOPE2/20OUTTHERE-TELESCOPE2-thumbWide.jpg?quality=75\&auto=webp\&disable=upscale}

  \hypertarget{nasa-names-dark-energy-telescope-for-nancy-grace-roman}{%
  \subsection{NASA Names Dark Energy Telescope for Nancy Grace
  Roman}\label{nasa-names-dark-energy-telescope-for-nancy-grace-roman}}

  Dr. Roman was a pioneer at NASA, joining the agency in its early days
  and becoming its first chief astronomer.

  By Dennis Overbye
\item
  \href{/2020/05/20/science/galaxy-early-universe-astronomy.html}{}

  \includegraphics{https://static01.nyt.com/images/2020/05/26/science/20SCI-GALAXY1/20SCI-GALAXY1-thumbWide.jpg?quality=75\&auto=webp\&disable=upscale}

  \hypertarget{the-galaxy-that-grew-up-too-fast}{%
  \subsection{The Galaxy That Grew Up Too
  Fast}\label{the-galaxy-that-grew-up-too-fast}}

  A vast wheel of gas in the primordial cosmos is forcing astronomers to
  rethink how some of the universe's largest structures may have formed.

  By Dennis Overbye
\end{enumerate}

Show More

Advertisement

\protect\hyperlink{after-mid1}{Continue reading the main story}

Advertisement

\protect\hyperlink{after-mktg}{Continue reading the main story}

\hypertarget{site-index}{%
\subsection{Site Index}\label{site-index}}

\hypertarget{site-information-navigation}{%
\subsection{Site Information
Navigation}\label{site-information-navigation}}

\begin{itemize}
\tightlist
\item
  \href{https://help.nytimes.com/hc/en-us/articles/115014792127-Copyright-notice}{©~2020~The
  New York Times Company}
\end{itemize}

\begin{itemize}
\tightlist
\item
  \href{https://www.nytco.com/}{NYTCo}
\item
  \href{https://help.nytimes.com/hc/en-us/articles/115015385887-Contact-Us}{Contact
  Us}
\item
  \href{https://www.nytco.com/careers/}{Work with us}
\item
  \href{https://nytmediakit.com/}{Advertise}
\item
  \href{http://www.tbrandstudio.com/}{T Brand Studio}
\item
  \href{https://www.nytimes.com/privacy/cookie-policy\#how-do-i-manage-trackers}{Your
  Ad Choices}
\item
  \href{https://www.nytimes.com/privacy}{Privacy}
\item
  \href{https://help.nytimes.com/hc/en-us/articles/115014893428-Terms-of-service}{Terms
  of Service}
\item
  \href{https://help.nytimes.com/hc/en-us/articles/115014893968-Terms-of-sale}{Terms
  of Sale}
\item
  \href{https://spiderbites.nytimes.com}{Site Map}
\item
  \href{https://help.nytimes.com/hc/en-us}{Help}
\item
  \href{https://www.nytimes.com/subscription?campaignId=37WXW}{Subscriptions}
\end{itemize}
