Sections

SEARCH

\protect\hyperlink{site-content}{Skip to
content}\protect\hyperlink{site-index}{Skip to site index}

\href{https://www.nytimes.com/es/section/mundo}{Mundo}

\href{https://myaccount.nytimes.com/auth/login?response_type=cookie\&client_id=vi}{}

\href{https://www.nytimes.com/section/todayspaper}{Today's Paper}

\href{/es/section/mundo}{Mundo}\textbar{}Coronavirus: ¿qué tan grave
será la crisis?

\url{https://nyti.ms/3b7sVvQ}

\begin{itemize}
\item
\item
\item
\item
\item
\end{itemize}

\href{https://www.nytimes.com/es/spotlight/coronavirus?action=click\&pgtype=Article\&state=default\&region=TOP_BANNER\&context=storylines_menu}{El
brote de coronavirus}

\begin{itemize}
\tightlist
\item
  \href{https://www.nytimes.com/es/interactive/2020/espanol/mundo/coronavirus-en-estados-unidos.html?action=click\&pgtype=Article\&state=default\&region=TOP_BANNER\&context=storylines_menu}{Mapa
  y casos en EE. UU.}
\item
  \href{https://www.nytimes.com/es/2020/07/23/espanol/america-latina/bolivia-cloro-coronavirus-ivermectina.html?action=click\&pgtype=Article\&state=default\&region=TOP_BANNER\&context=storylines_menu}{Dióxido
  de cloro, ivermectina y más: ¿funcionan?}
\item
  \href{https://www.nytimes.com/es/interactive/2020/science/coronavirus-tratamientos-curas.html?action=click\&pgtype=Article\&state=default\&region=TOP_BANNER\&context=storylines_menu}{Fármacos
  y tratamientos}
\item
  \href{https://www.nytimes.com/es/2020/07/28/espanol/ciencia-y-tecnologia/anticuerpos-coronavirus-inmunidad.html?action=click\&pgtype=Article\&state=default\&region=TOP_BANNER\&context=storylines_menu}{Anticuerpos
  e inmunidad}
\item
  \href{https://www.nytimes.com/es/2020/04/29/espanol/estilos-de-vida/oximetro-para-que-sirve.html?action=click\&pgtype=Article\&state=default\&region=TOP_BANNER\&context=storylines_menu}{Oxímetros}
\end{itemize}

Advertisement

\protect\hyperlink{after-top}{Continue reading the main story}

Supported by

\protect\hyperlink{after-sponsor}{Continue reading the main story}

\hypertarget{coronavirus-quuxe9-tan-grave-seruxe1-la-crisis}{%
\section{Coronavirus: ¿qué tan grave será la
crisis?}\label{coronavirus-quuxe9-tan-grave-seruxe1-la-crisis}}

Conforme el brote de coronavirus se propaga en China, un torrente de
primeras investigaciones ofrece un panorama más claro de cómo se
comporta el patógeno y los factores clave que determinarán si puede
contenerse.

\includegraphics{https://static01.nyt.com/images/2020/02/03/world/03coronavirus-es/merlin_168335130_a2d47cf5-d39d-4fa0-86b6-c869c0b68ed4-articleLarge.jpg?quality=75\&auto=webp\&disable=upscale}

Por \href{https://www.nytimes.com/by/knvul-sheikh}{Knvul Sheikh},
\href{https://www.nytimes.com/by/derek-watkins}{Derek Watkins},
\href{https://www.nytimes.com/by/jin-wu}{Jin Wu} y
\href{https://www.nytimes.com/by/mika-grondahl}{Mika Gröndahl}

\begin{itemize}
\item
  4 de febrero de 2020
\item
  \begin{itemize}
  \item
  \item
  \item
  \item
  \item
  \end{itemize}
\end{itemize}

\href{https://www.nytimes.com/interactive/2020/world/asia/china-coronavirus-contain.html}{Read
in English}

Aunque el virus es una preocupación seria de salud pública, el riesgo
para la mayoría de las personas fuera de China sigue siendo muy bajo, de
hecho, la gripe estacional es una amenaza más inmediata. Para evitar
cualquier enfermedad viral, los expertos aconsejan que
\href{https://www.nytimes.com/2020/01/28/opinion/coronavirus-prevention-tips.html}{nos
lavemos las manos} con frecuencia y no vayamos a la escuela ni a la
oficina cuando estemos enfermos. La mayoría de las personas sanas
\href{https://www.nytimes.com/2020/01/29/health/coronavirus-masks-hoarding.html}{no
necesitan tapabocas}, y acumularlos quizá contribuya a periodos de
escasez de dicho producto para los trabajadores del sector salud que sí
los necesitan, señalaron expertos.

\hypertarget{quuxe9-tan-contagioso-es-el-virus}{%
\subsection{¿Qué tan contagioso es el
virus?}\label{quuxe9-tan-contagioso-es-el-virus}}

Parece ser moderadamente infeccioso, similar al SARS.

\includegraphics{https://static01.nyt.com/images/2020/02/03/world/03coronavirus-es3/merlin_168335982_c3270db4-0217-4837-9f28-8f5a741291b1-articleLarge.jpg?quality=75\&auto=webp\&disable=upscale}

La escala de un brote depende de la velocidad y la facilidad con que se
transmita un virus de persona a persona. Aunque las investigaciones
apenas comenzaron, los
\href{https://www.nejm.org/doi/full/10.1056/NEJMoa2001316}{científicos}
\href{https://www.imperial.ac.uk/mrc-global-infectious-disease-analysis/news--wuhan-coronavirus/}{han}
\href{https://papers.ssrn.com/sol3/papers.cfm?abstract_id=3524675}{calculado}
que cada persona con el nuevo coronavirus podría infectar de 1,5 a 3,5
personas si no se toman medidas de contención eficaces.

Eso haría que el virus sea aproximadamente tan contagioso como el SARS,
otro coronavirus que circuló en China en 2003 y fue contenido después de
que enfermó a 8098 personas y provocó la muerte de 774. Los virus
respiratorios como este pueden transmitirse por el aire, envueltos en
pequeñas gotas que se producen cuando una persona enferma respira,
habla, tose o estornuda.

Esas gotas caen al suelo a varios centímetros. Eso hace que el virus sea
más difícil de contraer que otros patógenos como el sarampión, la
varicela y la tuberculosis, que pueden viajar a decenas de metros a
través del aire. Sin embargo, es más fácil de contraer que el VIH o la
hepatitis, que solo se transmiten a través del contacto directo con los
fluidos corporales de una persona infectada.

Si cada persona infectada con el nuevo coronavirus contagia a dos o tres
más, eso quizá sea suficiente para sostener y acelerar un brote en caso
de que no se haga nada por reducirlo.

Los coronavirus como el virus de Wuhan pueden viajar solo unos dos
metros a partir de la persona infectada. Se desconoce cuánto pueden
sobrevivir en las superficies. Otros virus, como el sarampión, pueden
viajar hasta 30 metros y sobrevivir durante horas sobre las superficies.

Comparemos eso con un virus menos contagioso, como la gripe estacional.
La gente que tiene gripe suele infectar, en promedio, a 1,3 personas.

Sin embargo, las cifras de transmisión de cualquier enfermedad no son
fijas. Pueden reducirse mediante medidas sanitarias públicas y eficaces,
como aislar a las personas enfermas y dar seguimiento a los individuos
con los que han tenido contacto. Cuando las autoridades mundiales del
sector salud dieron seguimiento metódicamente y aislaron a personas
infectadas con el SARS en 2003, pudieron reducir el número promedio de
individuos contagiados por cada persona enferma a 0,4; lo suficiente
para detener el brote.

Las autoridades en materia de salubridad en todo el mundo
\href{https://www.nytimes.com/2020/01/29/health/china-coronavirus-outbreak.html}{están
llevando a cabo un gran esfuerzo} para tratar de repetir esa estrategia.

Hasta ahora, el número de casos afuera de China ha sido pequeño. No
obstante, en días recientes, han surgido casos en varios países,
incluyendo Estados Unidos, con personas que no han visitado China. Y el
número de casos dentro de China se ha acelerado, superando por mucho la
tasa de casos nuevos del SARS en 2003.

\hypertarget{cuuxe1nto-tiempo-se-necesita-para-presentar-suxedntomas}{%
\subsection{¿Cuánto tiempo se necesita para presentar
síntomas?}\label{cuuxe1nto-tiempo-se-necesita-para-presentar-suxedntomas}}

Posiblemente entre 2 y 14 días, lo cual hace que la enfermedad pase
desapercibida.

Image

Personal médico revisa la temperatura de un hombre a su llegada a un
centro de salud comunitario en Guangzhou.Credit...Alex Plavevski/EPA vía
Shutterstock

El tiempo que tardan en aparecer los síntomas después de que se infecta
una persona puede ser vital para la prevención y el control. Ese tiempo,
conocido como periodo de incubación, puede permitir que los funcionarios
de salud pongan en cuarentena u observen a las personas que hayan estado
expuestas al virus. Sin embargo, si el periodo de incubación es
demasiado largo o demasiado corto, puede ser difícil implementar estas
medidas.

Algunas enfermedades, como la influenza, tienen un periodo de incubación
breve de dos o tres días. No obstante, el SARS tuvo un periodo de
incubación de casi cinco días. Además, se necesitaban de cuatro a cinco
días después de comenzados los síntomas para que la gente pudiera
transmitir el virus. Eso les dio a los funcionarios tiempo para detener
el virus y contener el brote de manera efectiva, dijo Allison McGeer,
especialista en enfermedades infecciosas en el Hospital Mount Sinai en
Toronto, que estuvo a la vanguardia en Canadá de la respuesta al SARS.

Los funcionarios de los Centros para el Control y la Prevención de
Enfermedades calculan que el nuevo coronavirus tiene un periodo de
incubación de dos a catorce días. No obstante, aún no está claro si una
persona puede transmitir el virus antes de desarrollar síntomas o si la
gravedad de la enfermedad afecta la facilidad con que un paciente puede
propagar el virus.

``Eso me preocupa porque significa que la infección podría eludir la
detección'', dijo Mark Denison, experto en enfermedades infecciosas de
la Universidad de Vanderbilt en Nashville, Tennessee.

\hypertarget{cuuxe1n-mortuxedfero-es-el-virus}{%
\subsection{¿Cuán mortífero es el
virus?}\label{cuuxe1n-mortuxedfero-es-el-virus}}

Aún es difícil saberlo. Pero la tasa de fatalidad es probablemente menor
al 3 por ciento, mucho menor que el SARS.

Image

En Indonesia, funcionarios de salud llevan a cabo un simulacro de
transporte de paciente en aislamiento.Credit...Ivan Damanik/Agence
France-Presse --- Getty Images

Este es uno de los factores más importantes para saber cuán nocivo será
el brote, así como uno de los menos comprendidos.

Es difícil evaluar la letalidad de un nuevo virus. Los peores casos
generalmente se detectan primero, lo cual puede afectar nuestro
entendimiento de cuán probable es que mueran los pacientes. Alrededor de
\href{https://www.thelancet.com/journals/lancet/article/PIIS0140-6736(20)30183-5/fulltext}{un
tercio de los primeros 41 pacientes reportados en Wuhan} tuvieron que
ser atendidos en una unidad de cuidados intensivos, muchos con síntomas
de fiebre, tos crónica, dificultad para respirar y neumonía. No
obstante, las personas con casos leves quizá nunca visiten a un médico.
Por eso, quizá hay más casos de los que conocemos, y el índice de
muertes podría ser más bajo de lo que se pensaba en un principio.

Al mismo tiempo, quizá se reporten menos muertes por el virus de las que
hay. Las ciudades chinas que se encuentran en el epicentro del brote
enfrentan una escasez de kits de pruebas y camas de hospital, y muchas
personas enfermas no han podido ver a un médico.

``Aún hay mucha incertidumbre sobre cómo es el virus y cómo está
actuando'', comentó McGeer, del Hospital Mount Sinai en Toronto.

Los primeros indicios sugieren que el índice de letalidad del virus es
considerablemente más bajo que otro coronavirus, el MERS-CoV, que mata a
casi una de cada tres personas que se infectan, y el SARS, que mata a
una de cada diez. Todas las enfermedades parecen aferrarse a las
proteínas que se encuentran en la superficie de las células pulmonares,
pero el MERS-CoV y el SARS parecen ser más destructivos para el tejido
pulmonar. Hasta el 31 de enero, menos de una de cada 40 personas con
infecciones confirmadas había muerto. Muchos de los que murieron eran
hombres mayores con problemas de salud subyacentes.

Los patógenos de todos modos pueden ser muy peligrosos, aunque su índice
de letalidad sea bajo, explicó McGeer. Si bien la influenza tiene un
índice de letalidad de menos de uno por cada mil personas, cada año en
Estados Unidos aproximadamente 200.000 personas terminan hospitalizadas
debido a este virus y alrededor de 35.000 mueren.

\hypertarget{quuxe9-tan-eficaz-seruxe1-la-respuesta}{%
\subsection{¿Qué tan eficaz será la
respuesta?}\label{quuxe9-tan-eficaz-seruxe1-la-respuesta}}

La Organización Mundial de la Salud (OMS) ha aplaudido los esfuerzos de
China, pero los críticos temen que las medidas de aislamiento no sean
suficientes.

Image

El 3 de enero se desinfectaba un centro comercial en
Pekín.~Credit...Carlos García Rawlins/Reuters

Además de suspender los medios de transporte, los funcionarios
\href{https://www.nytimes.com/es/2020/01/27/espanol/coronavirus-murcielago.html}{cerraron
un mercado en Wuhan} en el que se vendían pollos, mariscos y animales
silvestres vivos, donde se piensa que se originó el coronavirus, y más
tarde suspendió el comercio de animales silvestres en todo el país. Se
han cerrado las escuelas, está restringido el acceso a la Gran Muralla
de Pekín y han detenido a los grupos de turistas de China. Los
funcionarios de la OMS han elogiado la respuesta agresiva de China ante
el virus.

Sin embargo, las medidas también han tenido efectos imprevistos. Los
residentes de Wuhan que están enfermos deben caminar o transportarse en
bicicleta a lo largo de kilómetros para llegar a los hospitales. Ahí,
muchos se quejan de que los envían de regreso debido a la escasez de
camas de hospital, empleados y suministros que ha empeorado debido a la
cuarentena.

Además, durante los primeros días críticos del brote, las autoridades
chinas prefirieron el hermetismo y el orden antes que enfrentarse
abiertamente a la crisis, por lo que
s\href{https://www.nytimes.com/es/2020/01/24/espanol/mundo/que-es-coronavirus-sintomas.html}{ilenciaron
a los profesionales médicos que alertaron sobre el problema}. El hecho
de que no quisieran hacer público el problema retrasó una posible
respuesta concertada de salud pública.

El 30 de enero, la OMS declaró que el brote es
\href{https://www.nytimes.com/2020/01/30/health/coronavirus-world-health-organization.html}{una
emergencia global de salud} y reconoció que la enfermedad representa un
riesgo más allá de las fronteras chinas.

Estados Unidos y Australia están negando temporalmente la entrada a
quienes no son ciudadanos de esos países que viajaron hace poco a China,
y varias grandes aerolíneas señalaron que esperan suspender los
servicios directos a China continental durante meses. Otros países
---incluyendo Kazakstán, Rusia y Vietnam--- han restringido
temporalmente los viajes y las visas. No obstante, los detractores temen
que estas medidas no sean suficientes.

\hypertarget{cuuxe1nto-han-viajado-las-personas-infectadas}{%
\subsection{¿Cuánto han viajado las personas
infectadas?}\label{cuuxe1nto-han-viajado-las-personas-infectadas}}

El virus se extendió rápidamente porque se originó en un centro de
transporte.

Image

Ciudadanos portugueses y brasileños aterrizaron en Lisboa desde
WuhanCredit...Mario Cruz

Wuhan es un lugar difícil para contener un brote. Tiene once millones de
habitantes, más que la ciudad de Nueva York. En un día promedio, 3500
pasajeros toman vuelos directos de Wuhan a ciudades en otros países.
Estas ciudades fueron las primeras en reportar casos del virus fuera de
China.

Wuhan también es un gran centro de transporte dentro de China, conectado
con Pekín, Shanghái y otras ciudades principales mediante líneas de tren
de alta velocidad y aerolíneas nacionales. En octubre y noviembre del
año pasado, cerca de dos millones de personas volaron de Wuhan a otros
lugares dentro de China.

China no estaba tan bien conectada en 2003 durante el brote del SARS.
Grandes números de trabajadores migrantes ahora viajan nacional e
internacionalmente a África, otras partes de Asia y Latinoamérica, donde
China está llevando a cabo un gran impulso de infraestructura con su
iniciativa del Cinturón y la Ruta de la Seda.

En general, China tiene alrededor de cuatro veces el número de trenes y
pasajeros aéreos del que tenía durante el brote del SARS.

China ha tomado la medida sin precedentes de
\href{https://www.nytimes.com/2020/01/26/world/asia/coronavirus-wuhan-china-hubei.html}{imponer
restricciones de viaje} a decenas de millones de personas que viven en
Wuhan y en ciudades vecinas. Sin embargo, los expertos advirtieron que
la cuarentena quizá se impuso demasiado tarde. El alcalde de Wuhan
reconoció que cinco millones de personas habían salido de la ciudad
antes de que comenzaran las restricciones.

``No se puede contener un germen. Una infección nueva se va a
propagar'', dijo Lawrence Gostin, profesor de Derecho en la Universidad
de Georgetown y director del Centro de Colaboración de la OMS para el
Derecho de Salud Nacional y Global. ``Se escapará; siempre es así''.

\hypertarget{cuuxe1nto-tiempo-tomaruxe1-desarrollar-una-vacuna}{%
\subsection{¿Cuánto tiempo tomará desarrollar una
vacuna?}\label{cuuxe1nto-tiempo-tomaruxe1-desarrollar-una-vacuna}}

Al menos falta un año para contar con una vacuna.

Image

Así operaba un equipo de prevención y control en el laboratorio del
Centro de Control y Prevención de Enfermedades de
NingxiaCredit...Reuters

Una vacuna de coronavirus podría evitar infecciones y detener la
propagación de la enfermedad. Sin embargo, las vacunas toman tiempo.

Después del brote del SARS en 2003, a los investigadores les tomó cerca
de veinte meses tener una vacuna lista para las pruebas con humanos. (La
vacuna jamás se necesitó porque terminaron por contener la enfermedad).
Con el brote de zika en 2015, los investigadores lograron reducir el
periodo de desarrollo de la vacuna a seis meses.

Ahora esperan que el trabajo de los brotes previos ayude a reducir ese
periodo aún más. Los investigadores
\href{https://ncbiinsights.ncbi.nlm.nih.gov/2020/01/13/novel-coronavirus/}{ya
han estudiado el genoma del nuevo coronavirus} y hallaron las proteínas
que son cruciales para la infección. Los científicos de los Institutos
Nacionales de Salud, de Australia y por lo menos de tres compañías están
trabajando en posibles vacunas.

``Si no nos topamos con obstáculos imprevistos, podremos llevar a cabo
una primera fase de pruebas en los próximos tres meses'', dijo Anthony
Fauci, director del Instituto Nacional de Alergias y Enfermedades
Infecciosas.

Fauci advirtió que aún podrían tardarse meses, e incluso años, después
de las pruebas iniciales para llevar a cabo pruebas exhaustivas que
demuestren que la vacuna es segura y eficaz. En el mejor de los casos,
quizá haya una vacuna disponible para el público dentro de un año.

\begin{center}\rule{0.5\linewidth}{\linethickness}\end{center}

\textbf{Fuentes}: Datos sobre casos reportados diariamente de la
Comisión de Salud de la Provincia de Hubei, la Comisión Nacional de
Salud de la República Popular China y la Organización Mundial de la
Salud.

Datos sobre las tasas de mortalidad y el número de transmisiones por
persona enferma de la Organización Mundial de la Salud, Centros para el
Control y la Prevención de Enfermedades de Estados Unidos, Universidad
Johns Hopkins, Intercambio Global de Datos de Salud, Organización de
Alimentos y Agricultura de Estados Unidos, Mapeo Global del Riesgo de
Enfermedades Infecciosas, Institutos Nacionales de Salud, Centro Europeo
para la Prevención y Control de Enfermedades, Universidad de Oxford,
Instituto Coreano de Medicina Oriental, Inserm, Imperial College,
Universidad de Harvard, Universidad de Hong Kong, Universidad de
Lancaster y Universidad de Berna.

Datos de volumen de viaje de la Asociación Internacional de Transporte
Aéreo DDS y, a través de CEIC, la Corporación Ferroviaria de China, la
Administración Nacional de Ferrocarriles y la Administración de Aviación
Civil de China.

Advertisement

\protect\hyperlink{after-bottom}{Continue reading the main story}

\hypertarget{site-index}{%
\subsection{Site Index}\label{site-index}}

\hypertarget{site-information-navigation}{%
\subsection{Site Information
Navigation}\label{site-information-navigation}}

\begin{itemize}
\tightlist
\item
  \href{https://help.nytimes.com/hc/en-us/articles/115014792127-Copyright-notice}{©~2020~The
  New York Times Company}
\end{itemize}

\begin{itemize}
\tightlist
\item
  \href{https://www.nytco.com/}{NYTCo}
\item
  \href{https://help.nytimes.com/hc/en-us/articles/115015385887-Contact-Us}{Contact
  Us}
\item
  \href{https://www.nytco.com/careers/}{Work with us}
\item
  \href{https://nytmediakit.com/}{Advertise}
\item
  \href{http://www.tbrandstudio.com/}{T Brand Studio}
\item
  \href{https://www.nytimes.com/privacy/cookie-policy\#how-do-i-manage-trackers}{Your
  Ad Choices}
\item
  \href{https://www.nytimes.com/privacy}{Privacy}
\item
  \href{https://help.nytimes.com/hc/en-us/articles/115014893428-Terms-of-service}{Terms
  of Service}
\item
  \href{https://help.nytimes.com/hc/en-us/articles/115014893968-Terms-of-sale}{Terms
  of Sale}
\item
  \href{https://spiderbites.nytimes.com}{Site Map}
\item
  \href{https://help.nytimes.com/hc/en-us}{Help}
\item
  \href{https://www.nytimes.com/subscription?campaignId=37WXW}{Subscriptions}
\end{itemize}
