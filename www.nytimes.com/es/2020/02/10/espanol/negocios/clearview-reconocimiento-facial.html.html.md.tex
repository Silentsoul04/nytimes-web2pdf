Sections

SEARCH

\protect\hyperlink{site-content}{Skip to
content}\protect\hyperlink{site-index}{Skip to site index}

\href{https://www.nytimes.com/es/section/negocios}{Negocios}

\href{https://myaccount.nytimes.com/auth/login?response_type=cookie\&client_id=vi}{}

\href{https://www.nytimes.com/section/todayspaper}{Today's Paper}

\href{/es/section/negocios}{Negocios}\textbar{}Una aplicación de
reconocimiento facial ha identificado a víctimas de abuso infantil

\url{https://nyti.ms/37blxw9}

\begin{itemize}
\item
\item
\item
\item
\item
\end{itemize}

Advertisement

\protect\hyperlink{after-top}{Continue reading the main story}

Supported by

\protect\hyperlink{after-sponsor}{Continue reading the main story}

\hypertarget{una-aplicaciuxf3n-de-reconocimiento-facial-ha-identificado-a-vuxedctimas-de-abuso-infantil}{%
\section{Una aplicación de reconocimiento facial ha identificado a
víctimas de abuso
infantil}\label{una-aplicaciuxf3n-de-reconocimiento-facial-ha-identificado-a-vuxedctimas-de-abuso-infantil}}

Aunque la herramienta podría ayudar a resolver casos, esa tecnología
podría permitir que la hermética empresa Clearview recopile datos e
imágenes extraordinariamente sensibles.

\includegraphics{https://static01.nyt.com/images/2020/02/07/business/10ClearviewES-1/merlin_167287035_0c3ff0e2-b4b7-4c2b-a1a7-e5054b500409-articleLarge.jpg?quality=75\&auto=webp\&disable=upscale}

Por \href{https://www.nytimes.com/by/kashmir-hill}{Kashmir Hill} y
\href{https://www.nytimes.com/by/gabriel-dance}{Gabriel J.X. Dance}

\begin{itemize}
\item
  10 de febrero de 2020
\item
  \begin{itemize}
  \item
  \item
  \item
  \item
  \item
  \end{itemize}
\end{itemize}

\href{https://www.nytimes.com/2020/02/07/business/clearview-facial-recognition-child-sexual-abuse.html}{Read
in English}

Agencias de seguridad de Estados Unidos y Canadá están usando Clearview
AI ---una hermética empresa que usa tecnología de reconocimiento facial
a partir de una base de datos de 3000 millones de imágenes--- para
identificar a niños que son víctimas de abuso sexual. Es un poderoso
incentivo para usar la tecnología, pero genera nuevos cuestionamientos
sobre la precisión de esa herramienta y el manejo de los datos que hace
la empresa.

Según los investigadores, las herramientas de Clearview les permiten
saber los nombres o el paradero de menores que aparecen en videos o
fotos de explotación sexual y que, de otro modo, tal vez no habrían sido
identificados. En un caso en Indiana, los detectives pasaron imágenes de
21 víctimas del mismo delincuente por la aplicación de Clearview y
recibieron catorce identidades, de acuerdo con Charles Cohen, un jefe
retirado de la policía estatal. La víctima más joven tenía 13 años.

``Eran chicos o mujeres jóvenes. Queríamos encontrarlos para decirles
que habíamos arrestado a este tipo y ver si querían declarar en calidad
de víctimas'', señaló Cohen.

Otro funcionario, uno de los responsables de la identificación de
víctimas en Canadá, quien no estaba autorizado para hablar en público
sobre las investigaciones, describió la tecnología de Clearview como
``el avance más importante en la última década'' en el terreno de los
delitos de abuso sexual infantil.

Sin embargo, según defensores de la privacidad, no se ha probado ni
regulado la base de datos de la empresa, la cual podría causar nuevos
tipos de daños. En sus servidores, Clearview almacena fotos que suben
los investigadores ---conocidas como ``imágenes de sondas''---, es decir
que podría acumular un conjunto de datos de una delicadeza
extraordinaria sobre víctimas infantiles de abuso sexual y explotación.

``Comprendemos la sensibilidad extrema involucrada con la identificación
de niños'', escribió en un correo electrónico Hoan Ton-That, fundador de
Clearview. ``Nuestra misión es proteger a los menores''.

De acuerdo con un
\href{https://int.nyt.com/data/documenthelper/6690-clearview-faq/c8b081a0bcca12e7903a/optimized/full.pdf\#page=1}{documento}
que la empresa comparte con sus clientes, ``las búsquedas se guardan
para siempre'' por omisión, pero los administradores pueden cambiar sus
configuraciones para que las imágenes de esas búsquedas sean purgadas
después de 30 días.

En su mayor parte, Clearview había operado en las sombras hasta el mes
pasado, cuando la publicación de un
\href{https://www.nytimes.com/es/2020/01/20/espanol/negocios/clearview-reconocimiento-facial.html}{reportaje
de The New York Times} reveló el uso que le daban agencias de seguridad
a nivel local y federal de todo Estados Unidos. La empresa ha
recolectado miles de millones de fotos de individuos de todo el internet
público, en sitios como Facebook, Twitter, Venmo y YouTube. Cuando un
usuario sube la foto de una persona a Clearview, la aplicación muestra
otras imágenes de la persona y las direcciones web donde aparece.

En numerosos documentos publicitarios, Clearview promueve el uso que le
dan las agencias de seguridad a su tecnología para resolver casos de
abuso sexual infantil. No obstante, hasta hace poco, la empresa se
centraba en su participación para identificar a los perpetradores, no a
las víctimas.

Quienes han alertado sobre los riesgos de Clearview han argumentado que
los beneficios de ese tipo de base de datos no son mayores que los daños
que podían causar.

``Es difícil. Todo el mundo quiere seguridad y salvar a los niños'',
dijo Liz O'Sullivan, directora de tecnología en Surveillance Technology
Oversight Project. ``Siempre hay alguna manera de normalizar la
vigilancia, pero sería peligroso que nos enfocáramos en las ventajas
potenciales. El reconocimiento facial comete muchos errores''.

O'Sullivan mencionó que le preocupaba que una agencia independiente no
hubiera probado la precisión del software de Clearview. Los algoritmos
de reconocimiento facial pueden ser deficientes en la gente joven, en
parte porque sus rostros cambian con la edad y también porque los niños
a menudo no están incluidos en los archivos de datos que se usan para
entrenar a los algoritmos.

O'Sullivan advirtió que, si la herramienta se equivoca en una
coincidencia, podría provocar efectos devastadores para los niños que
hayan sido identificados de forma errónea y para sus familias.
``Intercambiar la libertad y la privacidad por alguna evidencia
anecdótica que pudiera servirles a ciertas personas nunca es suficiente
para ceder nuestras libertades civiles'', opinó O'Sullivan.

Las agencias de seguridad deben verificar cada identidad cuando usan la
aplicación de Clearview, comentó Ton-That en su correo electrónico. Sin
embargo, no supo decir cuántos niños había en su base de datos.

``No monitoreamos el desglose de edad, género o raza de nuestra base de
datos de imágenes'', comentó. ``Somos un motor de búsqueda de imágenes
públicas, no un sistema de vigilancia''.

Las fuerzas especiales de Florida, Indiana y Dakota del Sur dedicadas a
investigar casos de abuso infantil, así como el Departamento de
Seguridad Nacional estadounidense y las agencias de seguridad de Canadá,
utilizan la aplicación.

Al igual que varios funcionarios que declararon para el Times, el
investigador canadiense fue reacio a hablar sobre Clearview por temor a
que los delincuentes cambiaran sus tácticas. ``Nos preocupa que, cuando
los criminales sepan que está disponible, cubran más el rostro de sus
víctimas'', comentó el oficial. ``No queremos que ellos sepan que se
puede hacer esto''.

Hay riesgos legales asociados con el manejo de este tipo de imágenes. La
empresa violaría la ley si recibe imágenes de abuso y no informa de
inmediato a las autoridades y borra el material de sus servidores.
Ton-That mencionó que la aplicación de Clearview solo transmite rostros,
no imágenes completas.

El Times verificó esto al analizar una versión para Android de la
aplicación de Clearview, pero no pudo examinar la oferta de la empresa
para iOS ni una versión para web.

Ninguna de las agencias de seguridad con las que habló el Times mencionó
si había realizado una auditoría técnica a Clearview antes de usar el
software. Tampoco respondieron a las preguntas relacionadas con el uso
específico de la aplicación, al argumentar que no comentaban sobre sus
técnicas de investigación.

Britney Walker, una vocera de la Unidad de Investigación de Explotación
Infantil del Departamento de Seguridad Nacional de Estados Unidos,
mencionó que la división colabora con otras agencias para que le ayuden
con las investigaciones, pero que la estrategia ``centrada en las
víctimas'' que utiliza la unidad prohíbe compartir imágenes ilegales.

``La agencia, en ninguna circunstancia, compartiría material de abuso
sexual infantil a empresas privadas'', comentó Walker.

Otras compañías ya trabajan de cerca con funcionarios de agencias de
seguridad para investigar casos de abuso sexual infantil. Johann
Hofmann, director ejecutivo de Griffeye, señaló que el software de la
empresa que analiza las imágenes fue instalado dentro de las redes de
las agencias de seguridad y estaba diseñado para evitar el envío de
imágenes a terceros, incluida la misma Griffeye.

Otra empresa que ofrece herramientas analíticas a los investigadores de
abuso sexual infantil, CameraForensics, también comentó que sus sistemas
estaban diseñados para nunca recibir ningún tipo de imágenes de agencias
de seguridad, incluidos rostros. El fundador de la empresa, Matt Burns,
mencionó que su firma había considerado incorporar la tecnología de
reconocimiento facial en su software, pero había decidido no hacerlo por
``razones éticas''.

``Consideramos que era una herramienta demasiado controversial porque es
muy fácil caer en casos de abuso con esa funcionalidad'', dijo. ``Y
también es una pesadilla legal''.

Pero Burns entiende por qué los investigadores querrían usar un software
de reconocimiento facial. ``Se enfrentan a una tarea muy sombría y, si
hay una herramienta que les da la posibilidad de proteger a las
víctimas, no los culpo por usarla'', dijo.

Desde que salieron a la luz las prácticas de Clearview, compañías como
Facebook, LinkedIn, Twitter, Venmo y YouTube han
\href{https://www.nytimes.com/2020/01/22/technology/clearview-ai-twitter-letter.html}{enviado}
cartas en las que le solicitan a la empresa dejar de tomar fotos de sus
sitios y borrar las imágenes existentes de sus bases de datos. El fiscal
general de Nueva Jersey les
\href{https://www.nytimes.com/2020/01/24/technology/clearview-ai-new-jersey.html}{prohibió}
el uso de Clearview a los funcionarios del estado y solicitó una
investigación sobre el uso que les daban las agencias de seguridad a la
empresa y a las tecnologías similares. En Illinois, donde una dura ley
de privacidad de los datos biométricos prohíbe el uso de los rostros de
los habitantes sin su consentimiento, se presentó una demanda colectiva
en busca de certificación. El 3 de febrero, se presentó otra en
Virginia.

Recientemente, en Nueva York y Washington, se han presentado proyectos
de ley que le prohíben el uso del reconocimiento facial a la policía.
Además, Clearview recibió una carta de Edward Markey, senador demócrata
de Massachusetts, en la que le pidió una lista de las agencias de
seguridad que han usado la aplicación y si se ha recabado la información
biométrica de niños menores de 13 años.

``Aunque este tipo de tecnología existe desde hace bastante tiempo,
creemos que hemos creado algo que permite a las fuerzas del orden
público resolver crímenes que antes no se podían resolver y, lo más
importante, proteger a menores vulnerables'', dijo Ton-That en su correo
electrónico. ``Al mismo tiempo, estamos respondiendo a las solicitudes
de información del gobierno estadounidense y otras partes interesadas,
según corresponda, y esperamos entablar conversaciones constructivas con
ellos mientras trabajamos para hacer que nuestras comunidades sean más
seguras''.

En octubre, grupos de agencias policiales enviaron
\href{https://www.ascia.org/pdf/news/le_group_letter_to_congress__facial_recogniton_technology__october_2019.pdf}{una
carta} a los miembros del congreso de Estados Unidos en la que hacen un
llamado a no prohibir el uso del reconocimiento facial para sus
investigaciones. ``Entendemos la preocupación del público sobre los
derechos civiles y la protección de su privacidad'', escribieron. ``Con
políticas claras y disponibles al público, creemos que esas inquietudes
pueden abordarse''.

Muchas agencias ya habían estado usando Clearview durante meses, pero la
carta no mencionaba eso.

Michael H. Keller y Aaron Krolik colaboraron en este reportaje.

Kashmir Hill es una reportera de tecnología radicada en Nueva York.
Escribe sobre las formas inesperadas en las que la tecnología está
cambiando nuestras vidas, particularmente cuando se trata de nuestra
privacidad. \href{https://twitter.com/kashhill}{@kashhill}

Gabriel Dance es el editor adjunto de investigaciones. Antes fue editor
interactivo del diario The Guardian y formó parte del equipo que recibió
el Pulitzer en 2014 por la cobertura de la vigilancia secreta por parte
de la Agencia de Seguridad Nacional de Estados Unidos.
\href{https://twitter.com/gabrieldance}{@gabrieldance}

\begin{center}\rule{0.5\linewidth}{\linethickness}\end{center}

Advertisement

\protect\hyperlink{after-bottom}{Continue reading the main story}

\hypertarget{site-index}{%
\subsection{Site Index}\label{site-index}}

\hypertarget{site-information-navigation}{%
\subsection{Site Information
Navigation}\label{site-information-navigation}}

\begin{itemize}
\tightlist
\item
  \href{https://help.nytimes.com/hc/en-us/articles/115014792127-Copyright-notice}{©~2020~The
  New York Times Company}
\end{itemize}

\begin{itemize}
\tightlist
\item
  \href{https://www.nytco.com/}{NYTCo}
\item
  \href{https://help.nytimes.com/hc/en-us/articles/115015385887-Contact-Us}{Contact
  Us}
\item
  \href{https://www.nytco.com/careers/}{Work with us}
\item
  \href{https://nytmediakit.com/}{Advertise}
\item
  \href{http://www.tbrandstudio.com/}{T Brand Studio}
\item
  \href{https://www.nytimes.com/privacy/cookie-policy\#how-do-i-manage-trackers}{Your
  Ad Choices}
\item
  \href{https://www.nytimes.com/privacy}{Privacy}
\item
  \href{https://help.nytimes.com/hc/en-us/articles/115014893428-Terms-of-service}{Terms
  of Service}
\item
  \href{https://help.nytimes.com/hc/en-us/articles/115014893968-Terms-of-sale}{Terms
  of Sale}
\item
  \href{https://spiderbites.nytimes.com}{Site Map}
\item
  \href{https://help.nytimes.com/hc/en-us}{Help}
\item
  \href{https://www.nytimes.com/subscription?campaignId=37WXW}{Subscriptions}
\end{itemize}
