Sections

SEARCH

\protect\hyperlink{site-content}{Skip to
content}\protect\hyperlink{site-index}{Skip to site index}

\href{https://www.nytimes.com/es/section/mundo}{Mundo}

\href{https://myaccount.nytimes.com/auth/login?response_type=cookie\&client_id=vi}{}

\href{https://www.nytimes.com/section/todayspaper}{Today's Paper}

\href{/es/section/mundo}{Mundo}\textbar{}La medicina moderna pelea con
los fantasmas del Tercer Reich en Israel

\url{https://nyti.ms/2ApM524}

\begin{itemize}
\item
\item
\item
\item
\item
\end{itemize}

Advertisement

\protect\hyperlink{after-top}{Continue reading the main story}

Supported by

\protect\hyperlink{after-sponsor}{Continue reading the main story}

Medio Oriente

\hypertarget{la-medicina-moderna-pelea-con-los-fantasmas-del-tercer-reich-en-israel}{%
\section{La medicina moderna pelea con los fantasmas del Tercer Reich en
Israel}\label{la-medicina-moderna-pelea-con-los-fantasmas-del-tercer-reich-en-israel}}

Un cirujano palestino, un paciente judío, un texto médico nazi y el
inesperado vínculo que los une.

\includegraphics{https://static01.nyt.com/images/2020/04/21/world/15israel-nazi-ES-01/merlin_171330726_de593ea3-7d90-4eb9-b14f-c01b37e1b762-articleLarge.jpg?quality=75\&auto=webp\&disable=upscale}

\href{https://www.nytimes.com/by/isabel-kershner}{\includegraphics{https://static01.nyt.com/images/2018/10/12/multimedia/author-isabel-kershner/author-isabel-kershner-thumbLarge.png}}

Por \href{https://www.nytimes.com/by/isabel-kershner}{Isabel Kershner}

\begin{itemize}
\item
  15 de mayo de 2020
\item
  \begin{itemize}
  \item
  \item
  \item
  \item
  \item
  \end{itemize}
\end{itemize}

\href{https://www.nytimes.com/2020/05/12/world/middleeast/nazi-medical-text-israel.html}{Read
in
English}\href{https://www.nytimes.com/2020/05/12/world/middleeast/nazi-medical-text-israel.html}{Read
in English}

\href{https://www.nytimes.com/newsletters/el-times}{Regístrate para
recibir nuestro boletín} con lo mejor de The New York Times.

\begin{center}\rule{0.5\linewidth}{\linethickness}\end{center}

JERUSALÉN --- La explosión lo lanzó al cielo, con las piernas hacia
arriba, antes de estrellarlo contra el suelo.

Era junio de 2002, en el apogeo de la segunda intifada palestina. Dvir
Musai, entonces un estudiante israelita de 13 años, de un asentamiento
religioso judío, estaba en un viaje de recolección de cerezas en el sur
de Cisjordania. En su camino de regreso al bus pisó una mina colocada
por militantes palestinos y quedó gravemente herido junto con otros dos
niños.

``Había mucho humo, caían grumos de tierra, un olor a quemado y
pólvora'', recuerda Musai, ahora de 31 años.

Siguieron décadas de agonía. El pie derecho de Musai se sentía como si
estuviera permanentemente en llamas. Y entonces, el año pasado, un
cirujano le ofreció esperanza, y una revelación inquietante.

En el preoperatorio del Centro Médico Hadassah, en Jerusalén, el doctor
Madi el-Haj le dijo a su paciente que el atlas anatómico que usaría para
guiarse a través de las intrincadas vías nerviosas había sido producido
por los nazis. Se cree que sus ilustraciones se basan en las víctimas
diseccionadas del sistema judicial nazi bajo el Tercer Reich de Hitler.

Si tenían objeciones, dijo el-Haj a la familia Musai, podría operar sin
él, pero sería más difícil. Señaló que había aprobación rabínica para el
uso del libro.

La madre de Musai, Chana, había perdido familiares en el Holocausto.

``Ella dijo: `Si ahora puede ayudar, lo usaremos''', recordó Musai.

Esa decisión desgarradora estuvo al centro de un debate de larga data
sobre la ética de aprovechar el conocimiento derivado de la amplia
experimentación médica y científica de los nazis, y en este caso, la
ética de usar el libro de texto
\href{https://www.nytimes.com/1996/11/26/science/doctors-question-use-of-nazi-s-medical-atlas.html}{\emph{Atlas
de Topografía y Anatomía Humana Aplicada}}.

El libro, de Eduard Pernkopf, destaca por su precisión y detalle, e
incluso en una era de imágenes de vanguardia, algunos cirujanos ---entre
ellos aquellos que realizan procedimientos de nervios periféricos--- aún
encuentran que esos dibujos resultan invaluables.

En un giro perverso, cuanto más avanzado se vuelve el campo
relativamente nuevo de la cirugía de nervios periféricos, algunos de los
médicos que la practican dicen que se vuelven más dependiente del atlas.
Esto se debe a que incluso las imágenes de alta tecnología tienen uso
limitado en esta compleja disciplina, en la que los doctores tratan
problemas como el dolor crónico causado por nervios que están dañados o
atrapados.

\includegraphics{https://static01.nyt.com/images/2020/04/21/world/15israel-nazi-ES-02/merlin_171330960_5ce1814f-a53c-4c98-bf2c-0b19454cc66a-articleLarge.jpg?quality=75\&auto=webp\&disable=upscale}

Pernkopf comenzó a trabajar con el atlas en la Universidad de Viena,
cuando se convirtió en director de anatomía en 1933, el año en que se
unió al partido Nazi. Con la anexión de Austria por Hitler en 1938, se
convirtió en decano de la facultad de medicina y, después en presidente
de la universidad.

Los ilustradores a quienes Pernkopf recurrió para producir el atlas
también eran entusiastas nazis. Tres de los cuatro ilustradores
incorporaron esvásticas, relámpagos de las SS y otras insignias nazis en
sus firmas, distintivos del mal que fueron borrados en ediciones
posteriores.

Es menos claro quiénes eran las personas cuyos cuerpos fueron disecados
para que los ilustradores pudieran producir su trabajo. A lo largo de
los años, se ha cuestionado si algunos fueron asesinados en los campos
de exterminio de Hitler. Esas preguntas siguen sin resolverse, pero
muchos expertos creen que la mayor parte de los prisioneros eran
austríacos condenados en los tribunales.

Después de la guerra, Pernkopf pasó tres años en un campo de prisioneros
de los aliados, pero no fue acusado de crímenes de guerra. Continuó
trabajando en el atlas hasta su muerte, en 1955.

Una edición de dos volúmenes se publicó en cinco idiomas, y la primera
edición estadounidense salió en 1963. Elsevier, una editorial científica
europea que actualmente tiene los derechos de autor, dejó de imprimirlo
por razones éticas, pero los volúmenes se pueden encontrar en
colecciones privadas y se pueden comprar en eBay y Amazon.

Los académicos plantearon por primera vez preguntas sobre los orígenes
del atlas en los años ochenta, cuando el ``Gran Silencio'' de la Guerra
Fría sobre el legado médico de los nazis comenzó a resquebrajarse.

En la década de los noventa, la controversia atraía mayor atención
pública. Howard Israel, un cirujano oral de la Universidad de Columbia
que usó habitualmente el atlas, expuso los símbolos nazis en las firmas
de los artistas incluidos en las primeras ediciones del libro.

Entonces el doctor Israel y el doctor William Seidelman, un médico de
Toronto, solicitaron ayuda al memorial oficial del Holocausto de Israel,
Yad Vashem, pidiéndole que presione a la Universidad de Viena para
investigar los antecedentes del atlas y de los cadáveres disecados que
usaron los autores. Después de cierta reticencia inicial, la universidad
estuvo de acuerdo.

``Las cosas comenzaron a aclararse'', contó Seidelman, quien ahora vive
en Jerusalén.

De 1938 a 1945, el instituto anatómico de la universidad recibió más de
1370 cuerpos de prisioneros ejecutados por el sistema judicial de Viena,
según los hallazgos de un comité de investigación. Más de la mitad
habían sido prisioneros políticos, personas atacadas por el régimen
nazi. En ese momento en Austria, bromear sobre Hitler era suficiente
para garantizar la ejecución, a menudo por decapitación.

El cirujano de Hadassah, el doctor el-Haj, dijo que supo por primera vez
del atlas cuando estudiaba con Susan Mackinnon, pionera en cirugía de
nervios periféricos, en la Universidad de Washington en Saint Louis.

``Ella sabía que yo venía de Israel, pensó que yo era judío'', recordó.

Que él era, en realidad, un árabe musulmán de Galilea no cambió las
cosas.

``Estaba horrorizado'', dijo. ``Es una cuestión de humanidad''.

Image

Dvir Musai tenía 13 años cuando pisó una mina. El doctor el-Haj lo operó
años después.Credit...Dan Balilty para The New York Times

La doctora Mackinnon compró su primera copia a inicios de los años
ochenta, cuando era una joven cirujana plástica en Baltimore, y lo usó
para guiar muchos de sus procedimientos quirúrgicos.

Pero preocupada por la procedencia de las ilustraciones, algunos años
después Mackinnon fotocopió los primeros artículos académicos sobre el
pasado de Pernkopf y los guardó en el libro, como un recordatorio
constante.

En 2015, Mackinnon y su socio de años, Andrew Yee, querían compartir
dibujos del atlas en una plataforma de enseñanza en línea, y buscaron la
opinión de Sabine Hildebrandt, una médica de Boston que ha estudiado el
Tercer Reich.

Ya había en marcha un esfuerzo internacional para determinar cómo
manejar los restos humanos desenterrados y las muestras médicas de la
era del Holocausto.

Hildebrandt aceptó la consulta de Mackinnon y recurrió a otros expertos,
dando lugar a un conjunto especial de recomendaciones sobre el atlas de
Pernkopf en un documento conocido como el
\href{https://www.bu.edu/jewishstudies/files/2018/08/HOW-TO-DEAL-WITH-HOLOCAUST-ERA-REMAINS.FINAL_.pdf}{\emph{Protocolo
de Viena}}. Fue escrito por un prominente rabí y eticista
estadounidense, Joseph A. Polak, y fue adoptado formalmente en 2017, en
un simposio de expertos en Yad Vashem. Según el protocolo, el atlas
puede ser usado si se revelan por completo sus orígenes.

En una encuesta reciente de un grupo internacional de cirujanos de
nervios, Mackinnon y Yee encontraron que el 59 por ciento de los 182
encuestados conocían el atlas de Pernkopf, el 41 por ciento lo había
usado en algún momento y el 13 por ciento lo estaba usando en la
actualidad.

Pero el debate no está resuelto del todo.

Justin M. Sacks, jefe de la división de cirugía plástica y
reconstructiva de la Universidad de Washington, dijo que nunca había
encontrado el atlas hasta que llegó a su departamento este año. Él
argumentó que era moral y éticamente incorrecto usarlo y que había
sustitutos perfectamente adecuados, disponibles en forma impresa o en
línea.

``No busco provocar una controversia'', dijo en una entrevista, ``busco
ponerlo donde pertenece: en un museo''.

El doctor el-Haj dijo que, si bien las alternativas podrían ser lo
suficientemente buenas en otros campos médicos, cuando se trata de
cirugía de nervios periféricos, no eran rivales de Pernkopf.

Uno entre ocho hermanos, el-Haj creció en una aldea agrícola y aspiraba
a convertirse en un cirujano de nervios, dijo, con la esperanza de
ayudar a su padre, quien cuando era joven quedó con un brazo y una
pierna paralizados debido a un accidente laboral. Después de estudiar en
Estados Unidos, el-Haj regresó a Jerusalén en agosto de 2018 con sus
propios volúmenes de Pernkopf.

Casi al mismo tiempo, Musai, quien había sufrido docenas de operaciones
desde su lesión, regresó a sus médicos. Ahora era un hombre casado y
padre de dos hijos que apenas podía caminar. Su pie no toleraba ni el
peso de una sábana por la noche.

Fue derivado al doctor el-Haj.

Desde sus días como estudiante de medicina en Hadassah, el-Haj, de 40
años, recordaba a Musai como un adolescente enojado con un dolor
terrible, alguien que albergaba odio hacia los árabes.

Musai reconoce que así era.

Image

El origen de los cadáveres usados para el libro han causado dilemas a
los cirujanos.Credit...Dan Balilty para The New York Times

``La verdad es que si me hubieran enviado a Madi al inicio de mi lesión,
habría dicho que no,'' dijo Musai. ``No por el atlas, sino porque tuve
un gran problema con la población árabe. Veía en todos al terrorista que
me hirió''.

Pero ahora, años después, el-Haj realizó algunas pruebas y programó la
cirugía. Guiado por el atlas de Pernkopf, que llevó a la sala de
operaciones, encontró un collar de metralla alrededor del nervio,
localizó las ramas principales que causaban el dolor y las quitó,
aliviando su sufrimiento.

``Suena como un buen chiste,'' dijo Musai. ``El cirujano musulmán con el
atlas nazi operando a un judío''.

Las vidas de el-Haj y Musai se han entrelazado desde entonces.

Musai ha visitado a la familia del médico en su pueblo. Y cuando la
madre de el-Haj fue hospitalizada en Hadassah, Musai, quien ahora
trabaja allí como guía, la visitó. El doctor el-Haj también ha llevado a
sus hijos a visitar a los Musai a su asentamiento en Cisjordania.

El doctor el-Haj dijo que ha usado el atlas en aproximadamente el 90 por
ciento de sus operaciones, siempre explicando los antecedentes a sus
pacientes.

``Ningún paciente se ha negado'', dijo. ``Jamás. Porque estas personas
son capaces de hacer un pacto con el diablo para salir de su dolor''.

Isabel Kershner, corresponsal en Jerusalén, ha reportado sobre la
política israelí y palestina desde 1990. Es autora de \emph{Barrier: The
Seam of the Israeli-Palestinian Conflict}.
\href{https://twitter.com/IKershner}{@IKershner} •
\href{https://www.facebook.com/100013443257747}{Facebook}

\begin{center}\rule{0.5\linewidth}{\linethickness}\end{center}

Advertisement

\protect\hyperlink{after-bottom}{Continue reading the main story}

\hypertarget{site-index}{%
\subsection{Site Index}\label{site-index}}

\hypertarget{site-information-navigation}{%
\subsection{Site Information
Navigation}\label{site-information-navigation}}

\begin{itemize}
\tightlist
\item
  \href{https://help.nytimes.com/hc/en-us/articles/115014792127-Copyright-notice}{©~2020~The
  New York Times Company}
\end{itemize}

\begin{itemize}
\tightlist
\item
  \href{https://www.nytco.com/}{NYTCo}
\item
  \href{https://help.nytimes.com/hc/en-us/articles/115015385887-Contact-Us}{Contact
  Us}
\item
  \href{https://www.nytco.com/careers/}{Work with us}
\item
  \href{https://nytmediakit.com/}{Advertise}
\item
  \href{http://www.tbrandstudio.com/}{T Brand Studio}
\item
  \href{https://www.nytimes.com/privacy/cookie-policy\#how-do-i-manage-trackers}{Your
  Ad Choices}
\item
  \href{https://www.nytimes.com/privacy}{Privacy}
\item
  \href{https://help.nytimes.com/hc/en-us/articles/115014893428-Terms-of-service}{Terms
  of Service}
\item
  \href{https://help.nytimes.com/hc/en-us/articles/115014893968-Terms-of-sale}{Terms
  of Sale}
\item
  \href{https://spiderbites.nytimes.com}{Site Map}
\item
  \href{https://help.nytimes.com/hc/en-us}{Help}
\item
  \href{https://www.nytimes.com/subscription?campaignId=37WXW}{Subscriptions}
\end{itemize}
