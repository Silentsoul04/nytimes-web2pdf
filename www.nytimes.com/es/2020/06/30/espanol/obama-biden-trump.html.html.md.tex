Sections

SEARCH

\protect\hyperlink{site-content}{Skip to
content}\protect\hyperlink{site-index}{Skip to site index}

\href{https://www.nytimes.com/es/}{en Español}

\href{https://myaccount.nytimes.com/auth/login?response_type=cookie\&client_id=vi}{}

\href{https://www.nytimes.com/section/todayspaper}{Today's Paper}

\href{/es/}{en Español}\textbar{}Barack Obama abandona el retiro,
obligado por la campaña de Trump

\url{https://nyti.ms/3iiXNNq}

\begin{itemize}
\item
\item
\item
\item
\item
\item
\end{itemize}

\begin{itemize}
\item
  \href{https://www.nytimes.com/2020/07/31/us/elections/biden-vs-trump.html?action=click\&pgtype=Article\&state=default\&region=TOP_BANNER\&context=storylines_menu}{Election
  Updates}
\item
  \href{https://www.nytimes.com/article/biden-vice-president-2020.html?action=click\&pgtype=Article\&state=default\&region=TOP_BANNER\&context=storylines_menu}{Biden's
  V.P. Search}
\item
  \href{https://www.nytimes.com/interactive/2020/07/24/us/politics/trump-biden-campaign-donors.html?action=click\&pgtype=Article\&state=default\&region=TOP_BANNER\&context=storylines_menu}{Map
  of Donations}
\item
  \href{https://www.nytimes.com/interactive/2020/us/elections/delegate-count-primary-results.html?action=click\&pgtype=Article\&state=default\&region=TOP_BANNER\&context=storylines_menu}{Delegate
  Count}
\item
  \href{https://www.nytimes.com/interactive/2019/us/politics/2020-presidential-candidates.html?action=click\&pgtype=Article\&state=default\&region=TOP_BANNER\&context=storylines_menu}{The
  Candidates}
\item
  \href{https://www.nytimes.com/newsletters/politics?action=click\&pgtype=Article\&state=default\&region=TOP_BANNER\&context=storylines_menu}{Politics
  Newsletter}
\end{itemize}

Advertisement

\protect\hyperlink{after-top}{Continue reading the main story}

Supported by

\protect\hyperlink{after-sponsor}{Continue reading the main story}

Estados Unidos

\hypertarget{barack-obama-abandona-el-retiro-obligado-por-la-campauxf1a-de-trump}{%
\section{Barack Obama abandona el retiro, obligado por la campaña de
Trump}\label{barack-obama-abandona-el-retiro-obligado-por-la-campauxf1a-de-trump}}

El presidente número 44 de Estados Unidos anhelaba alejarse de la
política. Tres años después, está de regreso.

\includegraphics{https://static01.nyt.com/images/2020/06/30/us/politics/30Obama-ES-01/merlin_114147145_5babe815-5725-413c-9128-61bf3f9f1a39-articleLarge.jpg?quality=75\&auto=webp\&disable=upscale}

Por \href{https://www.nytimes.com/by/glenn-thrush}{Glenn Thrush} y
\href{https://www.nytimes.com/by/elaina-plott}{Elaina Plott}

\begin{itemize}
\item
  30 de junio de 2020
\item
  \begin{itemize}
  \item
  \item
  \item
  \item
  \item
  \item
  \end{itemize}
\end{itemize}

\href{https://www.nytimes.com/2020/06/28/us/politics/obama-biden-trump.html}{Read
in English}

\href{https://www.nytimes.com/newsletters/el-times}{Regístrate para
recibir nuestro boletín} con lo mejor de The New York Times.

\begin{center}\rule{0.5\linewidth}{\linethickness}\end{center}

Justo después de saber que
\href{https://www.nytimes.com/es/2020/06/22/espanol/donald-trump-2020-tulsa.html}{Donald
Trump}había sido electo presidente, Barack Obama se desplomó en su silla
del Despacho Oval y se dirigió a un asistente que estaba de pie cerca de
un frutero con manzanas colocado en un lugar prominente, un emblema de
su política de refrigerios saludables que, como tantas otras, estaba a
punto de desaparecer.

``Ya estoy harto'', dijo Obama acerca de su trabajo, según varias
personas familiarizadas con el intercambio

Pero él sabía, aún entonces, que un retiro convencional de la Casa
Blanca no era opción. Obama, quien en ese momento tenía 55 años, se
había quedado varado con la estafeta que planeaba pasarle a Hillary
Clinton todavía en la mano. Encima debía lidiar con un sucesor que,
creía, tenía una fijación en su contra basada en una extraña antipatía
personal y una política de reacción racial violenta ejemplificada en la
mentira sobre el lugar de nacimiento de Obama.

``No hay ningún modelo capaz de predecir el tipo de vida que tendré
después de la presidencia'', le dijo Obama al asistente. ``Es evidente
que no puede dejar de pensar en mí''.

Lo que no quiere decir que Obama estuviera dispuesto a olvidar cómo
había vislumbrado su retiro antes del triunfo de Trump: una vida plácida
dedicada a escribir, disfrutar juegos de golf en días soleados, impulsar
políticas a través de su fundación, producir documentales con Netflix y
gozar mucho tiempo en familia en su nueva finca de 11,7 millones de
dólares en Martha's Vineyard.

De cualquier forma, más de tres años después de su salida, el 44.°
presidente de Estados Unidos está otra vez en el campo de batalla
político que tanto deseó abandonar. Lo obligan a participar en el
enfrentamiento un enemigo empecinado en borrarlo de la historia
---Trump---, y un amigo que ha demostrado la misma determinación por
aprovechar su presencia, Joe Biden.

Era bien sabido que volver al campo de batalla sería muy arriesgado.
Obama ha demostrado un gran interés en proteger su legado, en especial
de los múltiples ataques de Trump. Pero después de realizar entrevistas
con más de 50 personas que rodean al expresidente, el retrato que
percibimos es el de un combatiente atribulado que intenta equilibrar el
profundo enojo causado por su sucesor con el instinto de evitar el
enfrentamiento por temor a que pudiera dañar su popularidad y afectar su
lugar en la historia.

Sin embargo, es posible que el cálculo de ese equilibrio haya comenzado
a cambiar tras el asesinato de George Floyd a manos de la policía en
Mineápolis. Como el primer presidente negro de Estados Unidos, ahora el
primer ex-presidente negro, Obama ve la concientización social y racial
actual como una oportunidad para darle un valor más significativo a las
elecciones de 2020, que habían estado marcadas por el estilo de lucha
sucia de Trump, y canalizar un nuevo movimiento juvenil hacia un
objetivo político, como sucedió en 2008.

Actúa con cautela, con su intención característica de mantener la calma,
ser fiel a su reputación, conservar su capital político y mantener
intactas sus aspiraciones de un retiro tranquilo.

``No creo que tenga dudas. Más bien, creo que ha adoptado una actitud
estratégica'', señaló Dan Pfeiffer, uno de sus principales asesores
durante más de una década. ``Siempre ha usado su voz de manera
estratégica; es su posesión más valiosa''.

Obama también está atento a un ejemplo aleccionador: en 2008, los
ataques de Bill Clinton en su contra fracasaron de tal manera que el
personal de campaña de su esposa tuvo que reducir sus apariciones.

Muchos seguidores ejercen cada vez más presión para que sea más
agresivo.

``Para variar, sería bueno que Barack Obama saliera de su cueva y
ofreciera (o más bien EXIGIERA) una ruta para seguir adelante'',
escribió el columnista Drew Magary en una publicación de Medium que se
ha compartido muchísimo desde su aparición en abril con el título
\href{https://gen.medium.com/where-the-hell-is-barack-obama-397ce8d7bbe2}{\emph{¿Dónde
diablos está Barack Obama?}}.

El argumento para rebatir esta postura es que Obama cumplió su trabajo y
merece que lo dejen en paz.

\includegraphics{https://static01.nyt.com/images/2020/06/26/us/politics/30Obama-ES-02/merlin_115707257_44ee1a9d-c059-4d5a-8d1d-1afcff6e5510-articleLarge.jpg?quality=75\&auto=webp\&disable=upscale}

``Obama ha estado fuera del cargo durante tres años y medio, y todavía
se enfrenta a este tipo de escrutinio: nadie está presionando de la
misma manera a ex presidentes blancos como George W. Bush y Jimmy
Carter'', dijo Monique Judge, editora de noticias de la revista en línea
The Root y autora de un artículo de 2018 que argumentaba que Obama
\href{https://www.theroot.com/obama-doesn-t-owe-this-country-shit-1826309455}{ya
no le debía nada al país}.

Obama mismo parece posicionarse en algún lugar intermedio. No planea
descartar sus vacaciones de verano en Vineyard y todavía le preocupa la
fecha de publicación de su esperada autobiografía. No obstante, la
semana pasada redobló sus críticas ``indirectas'' al gobierno de Trump
cuando condenó el
``\href{https://www.nytimes.com/2020/06/23/us/politics/obama-biden-fundraiser.html}{enfoque
de gobierno caótico, desorganizado y malintencionado}'' durante un
evento en línea para recaudar fondos para Biden. Además, expresó una
especie de compromiso cuando les dijo a los seguidores de Biden: ``Lo
que han hecho hasta ahora no ha sido suficiente. Y lo mismo va para mí,
para Michelle y para nuestras hijas''.

El 25 de junio, durante un evento de recaudación por Zoom accesible solo
con invitación, Obama expresó su indignación porque el presidente
utilizó las frases ``\emph{kung flu}'' y ``China virus'' para describir
al coronavirus. ``No quiero un país en el que el presidente de Estados
Unidos promueva de manera activa la discriminación contra los asiáticos
y encima le parezca gracioso. No quiero eso. Todavía me da escalofríos y
me enfurece'', dijo Obama, según una transcripción de sus comentarios
proporcionada por alguien que participó en el evento.

\hypertarget{latest-updates-2020-election}{%
\section{\texorpdfstring{\href{https://www.nytimes.com/2020/07/31/us/elections/biden-vs-trump.html?action=click\&pgtype=Article\&state=default\&region=MAIN_CONTENT_1\&context=storylines_live_updates}{Latest
Updates: 2020
Election}}{Latest Updates: 2020 Election}}\label{latest-updates-2020-election}}

Updated 2020-08-01T01:26:45.732Z

\begin{itemize}
\tightlist
\item
  \href{https://www.nytimes.com/2020/07/31/us/elections/biden-vs-trump.html?action=click\&pgtype=Article\&state=default\&region=MAIN_CONTENT_1\&context=storylines_live_updates\#link-29fdff45}{Kamala
  Harris, a top vice-presidential contender, confronts double
  standards.}
\item
  \href{https://www.nytimes.com/2020/07/31/us/elections/biden-vs-trump.html?action=click\&pgtype=Article\&state=default\&region=MAIN_CONTENT_1\&context=storylines_live_updates\#link-13ec3d9c}{Karen
  Bass and Susan Rice are rising on Biden's vice-presidential
  shortlist.}
\item
  \href{https://www.nytimes.com/2020/07/31/us/elections/biden-vs-trump.html?action=click\&pgtype=Article\&state=default\&region=MAIN_CONTENT_1\&context=storylines_live_updates\#link-49e9a016}{Trump
  says Russian bounties to kill U.S. troops `never took place.'}
\end{itemize}

\href{https://www.nytimes.com/2020/07/31/us/elections/biden-vs-trump.html?action=click\&pgtype=Article\&state=default\&region=MAIN_CONTENT_1\&context=storylines_live_updates}{See
more updates}

Obama habla frecuentemente con el exvicepresidente y los principales
asesores de la campaña para darles sugerencias sobre el personal y los
mensajes. En mayo le aconsejó sin rodeos a Biden mantener sus discursos
cortos, hacer entrevistas entusiastas y recortar la extensión de sus
tuits, pues lo mejor es hacer que la campaña funcione como un referendo
sobre Trump y la economía, según algunos funcionarios demócratas.

Los funcionarios mencionaron que un aspecto de particular interés para
el expresidente Obama es la operación digital de Biden, que está en
preparación y para la cual ha buscado que aliados poderosos como el
fundador de LinkedIn, Reid Hoffman, y el exdirector ejecutivo de Google,
Eric Schmidt, compartan sus conocimientos.

Con todo, todavía se toma su tiempo para responder a algunas
solicitudes, en especial las que se refieren a encabezar más actividades
de recaudación de fondos. Algunos colaboradores de Obama dieron a
entender que no quiere eclipsar al candidato, pero los partidarios de
Biden no están convencidos de que sea así.

``Por favor, que venga y nos eclipse'', bromeó uno de ellos.

Image

Una reunión en la Casa Blanca poco después de que Donald Trump ganó las
elecciones presidenciales.Credit...Stephen Crowley/The New York Times

\hypertarget{obama-no-podruxe1-descansar}{%
\subsection{`Obama no podrá
descansar'}\label{obama-no-podruxe1-descansar}}

Desde el momento en que se anunció el triunfo de Trump, Obama adoptó un
enfoque minimalista: criticaba sus decisiones de políticas públicas,
pero no al hombre que las tomaba, un comportamiento conforme a la norma
de civilidad observada por sus predecesores, en especial George W. Bush.

El problema es que para Trump las normas no significan nada. Desde un
principio dejó muy claro que quería erradicar cualquier rastro de la
presencia de Obama en el Ala Oeste. ``Tenía el peor gusto'', le dijo
Trump a un visitante a principios de 2017 mientras presumía sus nuevas
cortinas (que no eran muy distintas de las de Obama, en opinión de otras
personas que entraron al despacho durante ese periodo caótico).

Esos esfuerzos por hacerlo desaparecer fueron más enfáticos en lo
referente a las políticas. Un exfuncionario de la Casa Blanca comentó
que Trump interrumpió una presentación para verificar que una propuesta
del personal no fuera ``una cosa de Obama''.

Durante la transición, en lo que en retrospectiva parece un anticipo de
la presidencia, a un colaborador de Trump se le ocurrió imprimir una
lista detallada de las promesas de campaña de Obama del sitio web
oficial de la Casa Blanca y utilizarla como una especie de lista de
objetivos a abatir, según dos personas con conocimiento de la medida.

``Es algo personal para Trump; todo se trata del presidente Obama y de
acabar con su legado. Es su obsesión'', explicó Omarosa Manigault
Newman, veterana del programa \emph{Apprentice} y, hasta su abrupta
salida, una de las contadas funcionarias negras en el Ala Oeste de
Trump. ``El presidente Obama no podrá descansar mientras Trump
respire''.

Cuando los dos hombres se encontraron en noviembre de 2016 para una
forzada reunión posterior a la elección, el presidente electo fue
cortés, por lo que Obama aprovechó la oportunidad para aconsejarle no
desmantelar Obamacare. ``Mira, puedes quitarle mi nombre; no me
importa'', le dijo, según los asesores.

Trump asintió sin comprometerse.

Cuando la transición comenzó a hacerse eterna, Obama experimentó una
creciente inquietud ante una actitud que le parecía la alegre
indiferencia del nuevo presidente y su equipo de novatos. Muchos de
ellos ignoraron por completo los documentos de información que el
personal de Obama había preparado con tanto empeño, recuerdan sus
antiguos colaboradores, y en lugar de centrarse en la política o en el
funcionamiento del Ala Oeste, preguntaron por la calidad de los tacos en
el comedor del sótano o dónde encontrar un buen apartamento.

En cuanto a Trump, no tiene ``ni la menor idea de qué está haciendo'',
Obama le dijo a un asistente después de su encuentro en el Despacho
Oval.

Jared Kushner, yerno y asesor cercano de Trump, causó una impresión
igualmente imborrable. Durante un recorrido por el edificio preguntó
abruptamente: ``Entonces, ¿cuántas de estas personas se quedan?''.

La respuesta fue ninguna, respondió su escolta. (Los funcionarios del
Ala Oeste sirven a gusto del presidente, como Trump dejaría claro
ampliamente en los siguientes meses).

Cuando la historia de Kushner fue transmitida a Obama, recuerdan los
asesores, se rió y la repitió a sus amigos, e incluso a algunos
periodistas, para ilustrar a qué se enfrentaba el país.

Un portavoz de la Casa Blanca no negó el relato, pero sugirió que
Kushner podría haber hablado del personal de seguridad y mantenimiento
en lugar de los nombramientos políticos.

Durante otras conversaciones con editores que respetaba, incluidos David
Remnick de The New Yorker y Jeffrey Goldberg de The Atlantic, Obama se
mostró más reflexivo, según personas familiarizadas con las
interacciones. A veces, flotaba alguna versión de esta pregunta: ¿Podría
haber hecho algo para mitigar la reacción violenta de Trump?

Image

Obama dio su discurso de despedida en McCormick Place en Chicago, el 10
de enero de 2017.Credit...Doug Mills/The New York Times

Barack Obama finalmente llegó a la conclusión de que era una
inevitabilidad histórica, y le dijo a las personas a su alrededor que lo
mejor que podía hacer era ``establecer un contraejemplo''.

Otros pensaron que necesitaba hacer más. Durante la transición, Paulette
Aniskoff, asistente veterana en el Ala Oeste, comenzó a formar una
organización política con antiguos asesores para ayudar a Obama a
defender su legado, colaborar con otros demócratas y planear sus
actividades de respaldo en las elecciones intermedias de 2018.

Aunque se mostró abierto al planteamiento, lo que más le interesaba a
Obama eran las salidas. ``Haré lo que me pidan'', le dijo al equipo de
Aniskoff, pero les pidió identificar y descartar con cuidado las
apariciones que pudieran ser una pérdida de tiempo o un despilfarro de
su capital político.

Entonces como ahora, Obama estaba tan determinado a evitar mencionar el
nombre del nuevo presidente que un colaborador en broma sugirió que
hicieran referencia a él como ``el que no debe ser nombrado'', en
alusión al archienemigo de Harry Potter, Lord Voldemort.

Por su parte, Trump no tenía el menor problema en mencionar nombres. En
marzo de 2017, acusó en falso a Obama de haber ordenado que se vigilaran
las oficinas generales de su campaña, como dijo en un tuit: ``¡Qué bajo
ha caído el presidente Obama, que intervino mis teléfonos durante el
sagrado proceso de las elecciones! Es Nixon/Watergate. ¡Qué tipo malvado
(o enfermo)!''.

Fue algo así como un punto de inflexión. Obama les dijo a Aniskoff y su
equipo que hablaría de su sucesor en las elecciones intermedias de 2018.
Pero no mucho.

Fue muy reveladora la forma en que Obama habló de Trump ese otoño: no
tanto como una persona sino como una especie de padecimiento
epidemiológico que sufría el cuerpo político, diseminado por sus
secuaces republicanos.

``No empezó con Donald Trump; él es más bien un síntoma, no la causa'',
afirmó durante su discurso inicial en la Universidad de Illinois en
septiembre de 2018. Añadió que el sistema político estadounidense no
gozaba de ``salud'' suficiente para formar los ``anticuerpos''
necesarios y combatir el contagio del ``nacionalismo racial''.

La pandemia, si acaso, lo ha hecho más partidario de la comparación.

El virus, dijo durante su aparición con Biden la semana pasada, ``es una
metáfora'' para mucho más.

Image

Una salida de golf cerca de Dundee, Escocia, después de que Obama dejó
el cargoCredit...Andrew Milligan/Press Association Images, vía Getty
Images

\hypertarget{al-golf-le-va-mejor-que-a-mi-libro}{%
\subsection{Al golf le va `mejor que a mi
libro'}\label{al-golf-le-va-mejor-que-a-mi-libro}}

Obama consideró que una de las mejores maneras de salvaguardar su legado
era escribir su libro, que imaginó tanto como una crónica detallada de
su presidencia como un seguimiento literario a sus muy elogiadas
memorias de 1995, \emph{Los sueños de mi padr}e.

A fines de 2016, el agente de Obama, Bob Barnett, comenzó a negociar un
acuerdo global para las memorias de Obama y la autografía de Michelle
Obama. Random House finalmente ganó la guerra de ofertas con una
propuesta récord de 65 millones de dólares.

El proceso ha sido un castigo dorado. A un ex funcionario de la Casa
Blanca que se comunicó con él a mediados de 2018 Obama le dijo que el
proyecto ``era como hacer la tarea''.

Otro asociado, que se encontró con el ex presidente en un evento el año
pasado, comentó cuán en forma se veía. Obama respondió: ``Digamos que a
mi juego de golf le va mucho mejor que a mi libro''.

No fue especialmente fácil para el ex presidente ver cómo el libro de su
esposa, \emph{Mi historia}, se publicó en 2018 y rápidamente se
convirtió en un éxito de ventas internacional.

``Ella tuvo un escritor fantasma'', le dijo Obama a un amigo que le
preguntó sobre el trabajo veloz de su esposa. ``Estoy escribiendo cada
palabra yo mismo, y es por eso que me está tomando más tiempo''.

La fecha de lanzamiento libro sigue siendo uno de los temas más
delicados. Obama, un escritor deliberado propenso a la procrastinación
---y a largas digresiones--- insistió en que no haya un plazo
establecido, según varias personas al tanto del proceso.

En una entrevista poco después de que Obama dejó el cargo, uno de sus
asesores más cercanos había predicho que el libro saldría a mediados de
2019, antes de que las elecciones primarias comenzaran en serio, una
opción preferida por muchos que trabajan en el proyecto.

Pero Obama no terminó y no circuló un borrador de entre 600 y 800
páginas sino hasta cerca de Año Nuevo, demasiado tarde para publicar
antes de las elecciones, de acuerdo con conocedores de la situación

Ahora considera seriamente dividir el proyecto en dos volúmenes, con la
esperanza de publicarlo rápidamente, antes de las elecciones, quizás a
tiempo para la temporada navideña, según personas cercanas al proceso.

La otra gran empresa creativa de Obama, un acuerdo multimillonario de
2018 con Netflix para producir documentales y películas con su esposa,
ha sido un estimulante, y, en comparación, un trabajo rápido.

Obama disfrutó la revisión de decenas de proyectos potenciales y ofreció
sugerencias específicas ---garabateadas en la libreta amarilla que usa
para escribir su libro--- a directores y escritores.
\href{https://www.nytimes.com/2019/04/30/business/media/obama-netflix-shows.html}{Su
compañía de producción}, Higher Ground Productions, tiene un pequeño
bungalow en un estudio de Hollywood que una vez fue hogar de la compañía
de Charlie Chaplin. El expresidente pasó un día entrometiéndose con el
trabajo del reducido personal durante una visita en noviembre.

Uno de los primeros proyectos fue \emph{Crip Camp}, un galardonado
documental sobre un campamento de verano en el estado de Nueva York,
fundado a inicios de los años 70, que se convirtió en un punto focal del
movimiento por los derechos de las personas con discapacidad.

Obama vio el proyecto como un vehículo para su visión del cambio
político de base, y proporcionó retroalimentación durante los 18 meses
que duró la producción de la película.

``Vimos imágenes que los cineastas habían comenzado a editar y se las
enviamos al presidente para que las viera'', dijo Priya Swaminathan,
codirectora de \emph{Higher Ground.} ``Quería saber cómo podríamos
ayudar a los realizadores a hacer de este el mejor relato de la historia
y ellos se involucraron en la colaboración. Vimos muchos, muchos cortes
juntos''.

Image

Un discurso pronunciado en Cleveland en 2018 a favor de Richard Cordray,
un demócrata que postulaba para gobernador.Credit...Maddie McGarvey para
The New York Times

\hypertarget{un-momento-a-la-medida}{%
\subsection{Un momento `a la medida'}\label{un-momento-a-la-medida}}

Parte de lo que Obama encuentra tan atractivo sobre el cine es que le
permite controlar la narrativa. En ese sentido, la campaña de 2020 ha
sido una experiencia desorientadora: se supone que su carrera política
ha terminado, pero tiene un papel semi-protagonista en una producción
que no ha escrito ni dirigido.

Esa leve frustración ha sido más evidente en su complicada relación con
Biden, quien al mismo tiempo codicia su apoyo y está firmemente decidido
a triunfar en la elección por cuenta propia.

Obama apoyó a Biden, personalmente, desde el inicio de la campaña, pero
le prometió al senador Bernie Sanders, en una de sus primeras
conversaciones, que su profesión pública de neutralidad era genuina y
que no estaba trabajando en secreto para elegir a su amigo, según un
funcionario del partido al tanto del intercambio.

Además, Obama siempre ha tenido claras las vulnerabilidades de su amigo,
instando a los ayudantes de Biden a asegurarse de que no ``pase por una
situación embarazosa'' o ``dañe su legado'', gane o pierda.

Cuando un donante demócrata planteó la cuestión de la edad de Biden a
fines del año pasado ---tiene 77 años--- Obama reconoció esas
preocupaciones y dijo: ``Ni siquiera tenía 50 años cuando fui elegido, y
ese trabajo consumió cada gramo de energía que tenía'', según la
persona.

Aún así, es un partidario entusiasta y jugó un papel central al
presionar a Sanders para que
``\href{https://www.nytimes.com/2020/04/14/us/politics/obama-biden-democratic-primary.html}{acelere
el final del juego}'' que llevó a la victoria antes de lo esperado de
Biden, en abril. Las siguientes semanas las dedicó a arreglar algunos
cabos sueltos políticos, trabajar para mejorar su relación con la
senadora Elizabeth Warren, quien lo molestó al criticar sus conferencias
pagadas en Wall Street como emblemáticas del flagelo del dinero en la
política, al describirlo como una
\href{https://www.masslive.com/opinion/2017/04/warren_is_right_about_speaking.html}{``serpiente
que se desliza por Washington''}.

Sus asistentes insisten que nunca ha visto la campaña de Biden como una
guerra indirecta entre él y Trump. Sin embargo, le entusiasman las
métricas desequilibradas de la competencia en los últimos tiempos.

Obama supervisa de cerca sus respectivos números electorales ---obtiene
datos de circulación privada del Comité Nacional Demócrata---y se
enorgullece con el hecho de que tiene muchos más millones de seguidores
en Twitter que el presidente que confía en la plataforma mucho más que
él, dijeron personas cercanas.

El expresidente devora las noticias en línea, y recurre constantemente a
las páginas web de The New York Times, The Washington Post y The
Atlantic desde su iPad, y conserva el horario noctámbulo de sus días en
la Casa Blanca: envía mensajes de textos y enlaces a historias a sus
amigos entre la medianoche y las dos de la mañana. Incluso durante la
pandemia se levanta temprano, al menos entre semana, y con frecuencia
está en su bicicleta Peloton a las ocho de la mañana, cuando envía otra
nueva ronda de mensajes de texto, a menudo sobre el último escándalo de
Trump.

Obama ya estaba intensificando su crítica a Trump antes del asesinato de
Floyd en mayo. Aniskoff organizó
\href{https://www.nytimes.com/2020/05/09/us/politics/obama-flynn-coronavirus-trump.html}{una
reunión en línea con 3000 exfuncionarios del gobierno} cuyo propósito,
en parte, era suavizar su línea más dura. (Demócratas cercanos a Obama
filtraron amablemente la grabación de sus observaciones).

El creciente clamor de justicia racial le ha dado a la campaña de 2020
la coherencia que necesitaba Obama, un político que se siente más cómodo
si puede disfrazar sus críticas contra un oponente ---ya sea Hillary
Clinton o Donald Trump--- con lenguaje a favor de un movimiento.

Image

Prendedores a la venta afuera de la escuela secundaria East en
Cleveland, Ohio, cuando el expresidente Barack Obama dio un discurso en
un mitin demócrata en 2018.Credit...Maddie McGarvey para The New York
Times

La primera reacción de Obama a las manifestaciones, según sus
colaboradores cercanos, fue de ansiedad, pues temía que los brotes de
vandalismo se salieran de control y respaldaran la narrativa de Trump de
una izquierda anárquica.

Por fortuna, los manifestantes pacíficos asumieron el control y
despertaron un movimiento nacional que representó un reto para Trump sin
convertir al presidente en su punto focal.

Poco después, durante una llamada estratégica con colaboradores
políticos y expertos en políticas de su fundación, Obama dijo emocionado
que había llegado ``un momento hecho a la medida''.

Obama ha estado últimamente en contacto cercano con su primer fiscal
general, Eric H. Holder Jr., compartiendo su indignación por la manera
en que el actual fiscal general, William P. Barr, inspeccionó
personalmente la falange de los agentes federales del orden público que
lanzaron gases lacrimógenos a manifestantes para despejar el paso a la
caminata de Trump para hacerse una foto en una iglesia histórica cerca
de la Casa Blanca.

Holder tiene pocos reparos en llamar a Trump de racista en frente del
expresidente. Obama nunca lo contradijo, pero evita el término, incluso
en privado, y prefiere una acusación más indirecta de ``demagogia
racial'', según varias personas cercanas a ambos.

Su respuesta al asesinato de Floyd no consistió en atacar a Trump, sino
en alentar a votar a los jóvenes, que no han mostrado gran entusiasmo
por apoyar a Biden. Cuando decidió hablar en público
\href{https://www.nytimes.com/2020/06/23/us/politics/obama-biden-fundraiser.html}{fue
para encabezar un foro en línea para destacar una lista de reformas a la
policía que no prosperaron durante su segundo mandato}.

En ese sentido, el papel en el que se desempeña con más comodidad es el
cargo del que llegó a estar harto en cierto momento.

El 4 de junio, más o menos una hora antes de las
\href{https://www.nytimes.com/2020/06/04/us/floyd-memorial-funeral.html}{honras
fúnebres de Floyd} en Mineápolis, el expresidente llamó al hermano de
este, Philonise Floyd, del mismo modo que hizo con las familias en duelo
durante sus ocho años en el cargo.

``Quiero que tengas esperanza. Quiero que sepas que no estás solo.
Quiero que sepas que Michelle y yo haremos todo lo que quieras que
haga'', dijo Obama durante la emotiva conversación de 25 minutos, según
el reverendo Al Sharpton, que estaba presente. Otras dos personas con
conocimiento de la llamada confirmaron su contenido.

``Esa fue la primera vez, creo, que la familia Floyd realmente
experimentó consuelo desde que él murió'', dijo Sharpton en una
entrevista.

\hypertarget{our-2020-election-guide}{%
\section{Our 2020 Election Guide}\label{our-2020-election-guide}}

Updated July 31, 2020

\begin{itemize}
\item
  \begin{center}\rule{0.5\linewidth}{\linethickness}\end{center}

  \hypertarget{the-latest}{%
  \subsection{The Latest}\label{the-latest}}

  \begin{itemize}
  \tightlist
  \item
    President Trump's assault on the Postal Service is intersecting with
    his attacks on mail-in voting.
    \href{https://www.nytimes.com/2020/07/31/us/politics/trump-usps-mail-delays.html?action=click\&pgtype=Article\&state=default\&region=BELOW_MAIN_CONTENT\&context=storylines_guide}{Voting
    rights groups say it is a recipe for disaster.}
  \end{itemize}
\item
  \begin{center}\rule{0.5\linewidth}{\linethickness}\end{center}

  \hypertarget{bidens-vp-search}{%
  \subsection{Biden's V.P. Search}\label{bidens-vp-search}}

  \begin{itemize}
  \tightlist
  \item
    \href{https://www.nytimes.com/article/biden-vice-president-2020.html?action=click\&pgtype=Article\&state=default\&region=BELOW_MAIN_CONTENT\&context=storylines_guide}{Here
    are 13 women} who have been under consideration to be Joe Biden's
    running mate, and why each might be chosen --- and might not be.
  \end{itemize}
\item
  \begin{center}\rule{0.5\linewidth}{\linethickness}\end{center}

  \hypertarget{keep-up-with-our-coverage}{%
  \subsection{Keep Up With Our
  Coverage}\label{keep-up-with-our-coverage}}

  \begin{itemize}
  \tightlist
  \item
    Get an
    \href{https://www.nytimes.com/newsletters/politics?action=click\&pgtype=Article\&state=default\&region=BELOW_MAIN_CONTENT\&context=storylines_guide}{email}
    recapping the day's news
  \end{itemize}

  \begin{itemize}
  \tightlist
  \item
    Download our mobile app on
    \href{https://apps.apple.com/us/app/nytimes/id284862083?ls=1\&mat_click_id=5c79ae7455014fd1bd66b5610c05b8f2-20191112-16948\&referrer=mat_click_id\%3D5c79ae7455014fd1bd66b5610c05b8f2-20191112-16948\%26link_click_id\%3D722930677036718082}{iOS}
    and
    \href{http://a.localytics.com/android?id=com.nytimes.android\&referrer=utm_source\%3Dother_nyt_mobile_web\%26utm_medium\%3DWeb\%2520page\%26utm_term\%3DGeneral\%2520Mobile\%2520Page\%26utm_campaign\%3DNYT\%2520Mobile\%2520General\%2520Page}{Android}
    and turn on Breaking News and Politics alerts
  \end{itemize}
\end{itemize}

Advertisement

\protect\hyperlink{after-bottom}{Continue reading the main story}

\hypertarget{site-index}{%
\subsection{Site Index}\label{site-index}}

\hypertarget{site-information-navigation}{%
\subsection{Site Information
Navigation}\label{site-information-navigation}}

\begin{itemize}
\tightlist
\item
  \href{https://help.nytimes.com/hc/en-us/articles/115014792127-Copyright-notice}{©~2020~The
  New York Times Company}
\end{itemize}

\begin{itemize}
\tightlist
\item
  \href{https://www.nytco.com/}{NYTCo}
\item
  \href{https://help.nytimes.com/hc/en-us/articles/115015385887-Contact-Us}{Contact
  Us}
\item
  \href{https://www.nytco.com/careers/}{Work with us}
\item
  \href{https://nytmediakit.com/}{Advertise}
\item
  \href{http://www.tbrandstudio.com/}{T Brand Studio}
\item
  \href{https://www.nytimes.com/privacy/cookie-policy\#how-do-i-manage-trackers}{Your
  Ad Choices}
\item
  \href{https://www.nytimes.com/privacy}{Privacy}
\item
  \href{https://help.nytimes.com/hc/en-us/articles/115014893428-Terms-of-service}{Terms
  of Service}
\item
  \href{https://help.nytimes.com/hc/en-us/articles/115014893968-Terms-of-sale}{Terms
  of Sale}
\item
  \href{https://spiderbites.nytimes.com}{Site Map}
\item
  \href{https://help.nytimes.com/hc/en-us}{Help}
\item
  \href{https://www.nytimes.com/subscription?campaignId=37WXW}{Subscriptions}
\end{itemize}
