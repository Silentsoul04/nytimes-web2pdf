Sections

SEARCH

\protect\hyperlink{site-content}{Skip to
content}\protect\hyperlink{site-index}{Skip to site index}

\href{https://www.nytimes.com/es/section/ciencia-y-tecnologia}{Ciencia y
Tecnología}

\href{https://myaccount.nytimes.com/auth/login?response_type=cookie\&client_id=vi}{}

\href{https://www.nytimes.com/section/todayspaper}{Today's Paper}

\href{/es/section/ciencia-y-tecnologia}{Ciencia y
Tecnología}\textbar{}El coronavirus se propaga por el aire: qué debes
hacer ahora

\url{https://nyti.ms/2ZPc5Ny}

\begin{itemize}
\item
\item
\item
\item
\item
\item
\end{itemize}

\href{https://www.nytimes.com/es/spotlight/coronavirus?action=click\&pgtype=Article\&state=default\&region=TOP_BANNER\&context=storylines_menu}{El
brote de coronavirus}

\begin{itemize}
\tightlist
\item
  \href{https://www.nytimes.com/es/interactive/2020/espanol/america-latina/coronavirus-en-mexico.html?action=click\&pgtype=Article\&state=default\&region=TOP_BANNER\&context=storylines_menu}{Mapa
  y casos en México}
\item
  \href{https://www.nytimes.com/es/2020/07/31/espanol/ciencia-y-tecnologia/ninos-contagio-coronavirus.html?action=click\&pgtype=Article\&state=default\&region=TOP_BANNER\&context=storylines_menu}{Los
  niños y el virus}
\item
  \href{https://www.nytimes.com/es/interactive/2020/science/coronavirus-tratamientos-curas.html?action=click\&pgtype=Article\&state=default\&region=TOP_BANNER\&context=storylines_menu}{Fármacos
  y tratamientos}
\item
  \href{https://www.nytimes.com/es/2020/07/06/espanol/ciencia-y-tecnologia/coronavirus-transmision-aire.html?action=click\&pgtype=Article\&state=default\&region=TOP_BANNER\&context=storylines_menu}{Cómo
  se transmite el coronavirus}
\item
  \href{https://www.nytimes.com/es/2020/07/14/espanol/estilos-de-vida/botiquin-medicina-coronavirus.html?action=click\&pgtype=Article\&state=default\&region=TOP_BANNER\&context=storylines_menu}{Prepara
  tu botiquín}
\end{itemize}

Advertisement

\protect\hyperlink{after-top}{Continue reading the main story}

Supported by

\protect\hyperlink{after-sponsor}{Continue reading the main story}

Salud

\hypertarget{el-coronavirus-se-propaga-por-el-aire-quuxe9-debes-hacer-ahora}{%
\section{El coronavirus se propaga por el aire: qué debes hacer
ahora}\label{el-coronavirus-se-propaga-por-el-aire-quuxe9-debes-hacer-ahora}}

¿Cómo protegerse de un virus que puede estar flotando en el interior?
Mejor ventilación, para empezar. Y no te quites el cubrebocas.

\includegraphics{https://static01.nyt.com/images/2020/07/06/science/07aerosol-explainer-ES-01/06virus-aerosol-explainer01-sub-articleLarge-v2.jpg?quality=75\&auto=webp\&disable=upscale}

Por \href{https://www.nytimes.com/by/apoorva-mandavilli}{Apoorva
Mandavilli}

\begin{itemize}
\item
  8 de julio de 2020
\item
  \begin{itemize}
  \item
  \item
  \item
  \item
  \item
  \item
  \end{itemize}
\end{itemize}

\href{https://www.nytimes.com/2020/07/06/health/coronavirus-airborne-aerosols.html}{Read
in English}

\href{https://www.nytimes.com/newsletters/el-times}{Regístrate para
recibir nuestro boletín} con lo mejor de The New York Times.

\begin{center}\rule{0.5\linewidth}{\linethickness}\end{center}

El coronavirus puede quedarse suspendido en el aire durante varias horas
si no hay corrientes, y la gente podría contagiarse al inhalar,
según\href{https://www.nytimes.com/es/2020/07/06/espanol/ciencia-y-tecnologia/coronavirus-transmision-aire.html}{sugiere
un número cada vez mayor de evidencia científica.}

El riesgo es mayor en espacios interiores donde la ventilación es
deficiente, lo cual podría ayudar a explicar eventos de superpropagación
que se han reportado en plantas de procesamiento de carne, iglesias y
restaurantes.

No está muy claro con qué frecuencia se propaga el virus a través de
estas diminutas gotículas, o aerosoles, en comparación con las gotas más
grandes que se expelen cuando alguien infectado tose o estornuda, o se
transmiten a través del contacto con superficies contaminadas, afirmó
Linsey Marr, una experta en aerosoles del Instituto Politécnico y
Universidad Estatal de Virginia.

Los aerosoles se liberan incluso cuando una persona asintomática exhala,
habla o canta, de acuerdo con Marr y más de otros 200 expertos, quienes
han detallado la evidencia en una carta abierta a la Organización
Mundial de la Salud.

Lo que sí está claro, dicen en la misiva, es que la gente debería
intentar minimizar el tiempo que pasa en interiores con personas que no
pertenezcan a su unidad familiar. Las escuelas, los asilos y las
empresas deberían considerar añadir nuevos filtros de aire potentes y
luces ultravioletas que pueden matar a los virus de tranmisión aérea.

A continuación, presentamos algunas respuestas a las preguntas que se
suscitan a raíz de la nueva información.

\hypertarget{quuxe9-significa-que-un-virus-se-propague-por-vuxeda-auxe9rea}{%
\subsection{¿Qué significa que un virus se propague por vía
aérea?}\label{quuxe9-significa-que-un-virus-se-propague-por-vuxeda-auxe9rea}}

Para que se considere que un virus se propaga por vía aérea este tendría
que mantener viabilidad viral al transportarse por el aire. Para la
mayoría de los patógenos es claro si se transmiten de esta manera o no.
El virus del VIH, que es demasiado delicado como para sobrevivir fuera
del cuerpo, no se queda suspendido en el aire. El sarampión sí se
propaga por vía aérea, lo cual lo hace peligroso ya que puede sobrevivir
en el aire hasta por dos horas.

En el caso del coronavirus, la definición ha sido más complicada. Los
expertos concuerdan en que el virus no recorre grandes distancias ni
permanece viable en exteriores. Sin embargo, la evidencia sugiere que
puede llegar de un extremo a otro en una habitación y, en una serie de
situaciones
experimentales,\href{https://www.nytimes.com/es/2020/03/25/espanol/coronavirus-aire-como-se-transmite.html}{permaneció
viable hasta por tres hora}s.

\includegraphics{https://static01.nyt.com/images/2017/01/29/podcasts/the-daily-album-art/the-daily-album-art-articleInline-v2.jpg?quality=75\&auto=webp\&disable=upscale}

\hypertarget{listen-to-the-daily-four-new-insights-about-the-coronavirus}{%
\subsubsection{Listen to `The Daily': Four New Insights About the
Coronavirus}\label{listen-to-the-daily-four-new-insights-about-the-coronavirus}}

A look at what scientists have learned about how the virus takes hold in
the body and where the risk of infection is highest.

transcript

Back to The Daily

bars

0:00/29:28

-29:28

transcript

\hypertarget{listen-to-the-daily-four-new-insights-about-the-coronavirus-1}{%
\subsection{Listen to `The Daily': Four New Insights About the
Coronavirus}\label{listen-to-the-daily-four-new-insights-about-the-coronavirus-1}}

\hypertarget{hosted-by-michael-barbaro-produced-by-by-alexandra-leigh-young-and-austin-mitchell-and-edited-by-larissa-anderson}{%
\subsubsection{Hosted by Michael Barbaro, produced by by Alexandra Leigh
Young and Austin Mitchell, and edited by Larissa
Anderson}\label{hosted-by-michael-barbaro-produced-by-by-alexandra-leigh-young-and-austin-mitchell-and-edited-by-larissa-anderson}}

\hypertarget{a-look-at-what-scientists-have-learned-about-how-the-virus-takes-hold-in-the-body-and-where-the-risk-of-infection-is-highest}{%
\paragraph{A look at what scientists have learned about how the virus
takes hold in the body and where the risk of infection is
highest.}\label{a-look-at-what-scientists-have-learned-about-how-the-virus-takes-hold-in-the-body-and-where-the-risk-of-infection-is-highest}}

\begin{itemize}
\item
  michael barbaro\\
  From The New York Times, I'm Michael Barbaro. This is ``The Daily.''
\item
  {[}music{]}\\
  Today: As infection rates break new records across the U.S. this July
  4 weekend, four new insights into the virus from my colleague, science
  reporter Donald G. McNeil Jr. It's Monday, July 6.

  Let me start by, Donald, saying happy 4th of July.
\item
  donald g. mcneil jr.\\
  Thank you.
\item
  michael barbaro\\
  How did you spend it?
\item
  donald g. mcneil jr.\\
  Saturday, I played softball in the morning --- socially distant
  softball.
\item
  michael barbaro\\
  Softball?
\item
  donald g. mcneil jr.\\
  Yeah.
\item
  michael barbaro\\
  In New York City?
\item
  donald g. mcneil jr.\\
  In New York City. Yeah, Riverside Park. And then we went to dinner
  last night in a friend's backyard on Long Island, where we all sat as
  couples together, but six feet apart from each other, and sort of took
  turns going up to the table to get to the food, and sat, and had a
  really nice time.
\item
  michael barbaro\\
  I have to imagine that even a socially distanced meal with you is
  challenging, in that I think I would feel quite seen and judged, given
  your role.
\item
  donald g. mcneil jr.\\
  Why?
\item
  {[}laughter{]}
\item
  donald g. mcneil jr.\\
  I mean, I do tend to say six feet, six feet, six feet, a lot at
  people.
\item
  michael barbaro\\
  See?
\item
  donald g. mcneil jr.\\
  Because masks give people a false sense of security. I mean, this is a
  big thing on the softball field, is that people would both put on
  masks and they'd sit next to each other in the dugout, making up the
  batting order and stuff like that. And I'd go, no! Air leaks out of
  the side of your masks. And you're not always wearing masks. Sometimes
  you're wearing it as a sort of a Captain Ahab blue beard under your
  chin. So it's better to stay six feet apart. That way if your mask ---
\item
  michael barbaro\\
  Do your remember a couple seconds ago how you asked me why it might be
  challenging to have a ---
\item
  donald g. mcneil jr.\\
  {[}LAUGHS{]}
\item
  michael barbaro\\
  --- a meal with you?
\item
  donald g. mcneil jr.\\
  All right. {[}LAUGHTER{]}
\item
  michael barbaro\\
  Ask and ---
\item
  donald g. mcneil jr.\\
  All right, touche, touche.
\item
  michael barbaro\\
  So everything you just described, of course, is occurring in New York,
  where the infection rate has been generally declining or at least
  stabilizing. So let's talk about the state of the pandemic in the rest
  of the country. I wonder if you can give us a quick status update on
  this end of July 4 weekend.
\item
  donald g. mcneil jr.\\
  OK. I think it's 39 states now have cases going up. And it's hitting
  hardest in the South and in the West. Texas, Florida, Arizona,
  California, a number of other states. And it's exactly what was
  predicted back in May when states were debating opening by Memorial
  Day. All public health experts said, if you open up when your case
  counts are rising, they're going to continue rising and start rising
  even faster. And now we're seeing that.

  For example, in Houston, doctors who knew the situation in New York
  are saying that what's happening there looks like what happened in New
  York in early April. Finding new beds, finding ventilators, lots of
  patients who were sick, patients who were on oxygen. Not as many dying
  yet, but with people on oxygen and on ventilators they may find
  themselves in the situation where they have to park refrigerated
  trucks behind hospitals to hold the bodies, as they did in New York.
\item
  michael barbaro\\
  So Donald, correct me if I'm wrong. I think the U.S. is at about
  50,000 infections a day right now. Dr. Anthony Fauci said we could get
  up to 100,000. And if that's the case, what do we expect the attendant
  death count to start to look like in the U.S.? I assume that's going
  to catch up with that.
\item
  donald g. mcneil jr.\\
  Yeah, it'll catch up with it. I mean, but there's no reason to believe
  that 100,000 is the upper limit. It all depends on how much social
  distancing we practice. I mean, this is the dance. And do you close
  bars and restaurants? Do you open or close schools? Do you wear masks?
  How much attention do people pay to the directions you give them? How
  much do they practice good social distancing. That very much affects
  the rate of spread of the virus.
\item
  michael barbaro\\
  Well, with that in mind, those big questions of kind of how we proceed
  through the rest of this pandemic, you have been doing a lot of
  reporting about the latest learnings and insights into the virus that
  will very much guide how we answer those questions. And we want to
  talk through those with you. So where do you think we should start?
\item
  {[}music{]}
\item
  donald g. mcneil jr.\\
  Some of these insights are really more theories with some evidence to
  them. And some there's quite a bit of confidence in. So we probably
  ought to start with something that there's a pretty high degree of
  confidence in among doctors.
\item
  michael barbaro\\
  And what is that?
\item
  donald g. mcneil jr.\\
  When this all started, we thought of it as a lung disease, a
  respiratory disease, because the first cases we heard about, people
  got pneumonia. And that of course reminded us that the model for this
  disease was the 1918 influenza epidemic. But we're learning that this
  coronavirus is very different from an influenza virus.

  The influenza viruses attach to receptors in the lungs and the airway.
  This gets into the body through the airway, through the lungs. But it
  really attaches to the insides of the blood vessels. And so that makes
  it a vascular disease, a blood vessel disease.
\item
  michael barbaro\\
  And what are the implications of a virus like this being a vascular
  disease, a blood vessel disease, and not just a respiratory disease?
\item
  donald g. mcneil jr.\\
  It means it affects every organ in the body that has lots of fine
  blood vessels in it, and not even just organs.

  I mean, so it affects the lungs, which are the filter where the air
  gets into the blood, and you have lots of little fine blood vessels
  surrounding the little sacs at the ends of your breathing tubes. It
  attacks the kidneys, because that's the filter where the urine comes
  out of the blood. So you have very fine networks of blood vessels
  there. It attacks the gut, because you have a network of blood vessels
  in your gut where food gets into your body. It attacks the brain,
  because you have lots of fine blood vessels in the brain. It doesn't
  attack the nerve cells in the brain, which most of the brain is made
  of. It doesn't attack the muscle cells in the heart. But it attacks
  the blood vessels that go through all those other parts.

  And so when they do autopsies they find thousands of tiny little blood
  clots all over the body. We have lots of people who have strokes. And
  as those blood clots clot up blood vessels to small areas of the
  brain, you may get dementia or disorientation. And then in kids, when
  you have `Covid toes' in teenagers and young adults, this is the
  little capillaries in the hands and feet getting blocked, and getting
  this inflamed, painful, red or purple toe and finger syndrome. So it's
  more complicated to deal with a disease that can travel to any organ
  in the body.
\item
  michael barbaro\\
  So how does this new insight about the coronavirus, how does it change
  the way we are going to approach the pandemic?
\item
  donald g. mcneil jr.\\
  Before, the main thing you're looking for when you're looking to see
  if somebody is having a problem is their blood oxygen level. Because
  you're assuming that their problem is going to be pneumonia. But if
  you realize that the problem could be kidney damage, heart damage, you
  do a whole different battery of blood tests.
\item
  michael barbaro\\
  So what you're saying is that doctors who previously were diagnosing
  Covid-19 through a set of well-established symptoms now need to expand
  that set of symptoms pretty broadly. Because it turns out this is
  looking to be vascular, not respiratory.
\item
  donald g. mcneil jr.\\
  Yeah, that's right. It means that virtually anybody who comes into a
  doctor's office feeling sick might have the coronavirus. If they come
  in with symptoms of a stroke, it might be Covid. If they come in with
  symptoms of a heart attack, it might be Covid. If they come in with
  what seems like arthritis in their feet, it might be Covid toe.

  And because we often don't have enough tests, or it takes a long time
  to get test results, the patient's at a real disadvantage. Because if
  you don't know your patient has coronavirus, whatever symptom they've
  got now might become greater, might spread to other organs. So the
  problem in your toes might literally spread to your kidneys or your
  brain. And you want to know that that patient has a disease that can
  spread throughout the body.
\item
  michael barbaro\\
  In other words, more testing, fast testing becomes more imperative
  once we have learned that so many symptoms may actually be a sign of
  Covid-19.
\item
  donald g. mcneil jr.\\
  Absolutely.
\item
  michael barbaro\\
  OK. So what is the next big new insight we have into the coronavirus?
\item
  donald g. mcneil jr.\\
  Well, people are always asking, is the virus mutating? Is it becoming
  different? And the answer is yes. This virus always mutates. It makes
  one mutation about every two weeks.
\item
  michael barbaro\\
  Wow.
\item
  donald g. mcneil jr.\\
  The question is are any of those mutations important. And most of them
  aren't. Most of them don't change the function of the virus at all.
  But there has been one mutation that has become the object of a great
  deal of interest. We know for sure that there are sort of two general
  clades of the virus, the Wuhan strain and the other one called the
  Italian strain or sometimes a European strain. Now, the Wuhan strain
  is obviously the original one. That's where the virus started. But it
  went around Asia. Then it went to Iran. Then it went to Italy. And in
  Italy sometime in February, presumably, this mutation took place. Now,
  it has definitely not made the virus more dangerous, more lethal, more
  likely to kill you. But it appears to have made it more transmissible.
\item
  michael barbaro\\
  How so?
\item
  donald g. mcneil jr.\\
  Well, it appears that it transmits between people five to 10 times
  more easily. Now, this is in dispute. But there's been work done in
  cells in the laboratory where they infected them with the two
  different strains. And the mutation in the Italian strain seems to
  make the spikes on the outside of the virus --- the spikes of the
  corona --- more stable. Better able to infect. And so that they appear
  to be five to 10 times more capable at infecting cells as the old
  Wuhan version.
\item
  michael barbaro\\
  So the strain of this virus that has a better spike --- the Italian
  strain --- and is therefore more transmissible, is crowding out the
  previous strain, because it's just doing a better, more effective job
  of infecting people.
\item
  donald g. mcneil jr.\\
  Yeah, that's right. It's the natural progression for a virus. It's the
  way they tend to go.
\item
  michael barbaro\\
  What do you mean?
\item
  donald g. mcneil jr.\\
  Well, viruses, over the course of infecting lots of hosts, tend to
  become less lethal to those hosts and more transmissible. Like, for
  example, if I have the virus and it mutates inside me, and it turns
  into a more deadly strain, I've now got two strains. And I pass on
  that virus to two people, the person who gets the more deadly strain
  is more likely to go home, go to bed and die. Whereas the person who
  gets the less lethal, more transmissible strain is going to go out to
  a disco and infect 40 people.

  And if you do that enough times in the course of the virus, the virus
  always sort of naturally moves in the direction of the more
  transmissible, less lethal one, because that's the one that spreads
  whenever it's given that kind of fork in the road.

  And so this is what happened in 1918. The virus started off extremely
  deadly. It blew through an enormous chunk of the population, probably
  60 to 70 percent of all the people in the world. And then it
  disappeared for a while. Then it turned up in pigs, and it was a pig
  virus for a while. And then when enough humans who'd never had the
  virus were born, it reappeared in people. But it reappeared as the
  H1N1 seasonal flu, the one that we know about as one of the seasonal
  flus every year. But that became less lethal and more transmissible.
  And basically all viruses do that. And we might be beginning to see
  the very first hints of that happening with this virus.
\item
  michael barbaro\\
  So if I'm in Texas or Arizona right now and I'm testing positive for
  Covid-19, it sounds quite likely that I've gotten the Italian mutation
  of this virus, right? And that means I'm quite likely to spread it to
  somebody else and not have the most horrible symptoms. So does that
  partly help explain why infection rates are rising so rapidly in the
  U.S.?
\item
  donald g. mcneil jr.\\
  Well, infection rates are rising rapidly in the U.S. more because of
  human behavior than because of any changes in the virus. I think it's
  wishful thinking to think that this virus is not dangerous. It's
  really dangerous, and it's highly transmissible.
\item
  michael barbaro\\
  But because the Italian version of the virus spreads more effectively,
  that does suggest that the virus is becoming better at doing the thing
  it was designed to do, which is to infect lots and lots of people.
\item
  donald g. mcneil jr.\\
  Yes. But I mean, the Italian version versus the Wuhan version isn't
  the dead end. There are going to be many more mutations. It mutates
  every two weeks. There may be other mutations turning up in the virus
  that turn out to be important. And we may call those the Texas strain
  or the California strain, or whatever.

  But we don't know them yet. There's a lot of disagreement about this
  among scientists as to whether or not it really is more transmissible.
  And there's zero agreement that it --- not even really any thought
  that it's less dangerous. That completely remains to be seen.
\item
  {[}music{]}
\item
  michael barbaro\\
  We'll be right back.

  So Donald, what is the next big new understanding we have into the
  virus at this point?
\item
  donald g. mcneil jr.\\
  Well, there's more and more confirmation that you are much safer
  outdoors than you are indoors. There's a study in China that looked at
  318 clusters of transmission. And only one case involved outdoor
  transmission. And that was between two neighbors who had a long
  conversation with each other. And there's recently been another study
  from Japan that suggests that your chances of getting the virus
  indoors are 20 times as high as it would be outdoors.
\item
  michael barbaro\\
  And what are these studies finding about why exactly that is? I think
  we all have some understanding that when you're outside the virus is
  just going to disperse and become more diffuse. Is that as complicated
  as it is?
\item
  donald g. mcneil jr.\\
  Well, there's always a little bit of wind outside. Humidity also makes
  droplets fall out of the air. But mostly it's the wind. And when
  people talk within a few feet of each other, especially when they talk
  loud, or when they laugh, or when they sing or shout or do anything
  like that, you put out this kind of invisible mist of little tiny
  droplets that spews out of your mouth and sort of hangs around your
  head. But it also drifts towards the other person. And so you're
  sitting inside each other's droplet cloud. And those little tiny
  droplets, even if you're not feeling the other person in effect
  spitting on your face, that droplet cloud can hold enough virus to
  transmit the disease from one person to the other.

  And indoors when there's no windows open, it can sort of drift through
  the room, more or less at head level, and go past one person after
  another at a cocktail party or inside a bar like that. And each person
  inhaling a little bit of that droplet cloud, until the disease has
  spread to 20, 30, 40 people. Whereas outdoors, the breeze just blows
  that away. So standing six feet away outdoors, even without masks, is
  considered safe.
\item
  michael barbaro\\
  This is the idea that the virus becomes aerosolized. And you're saying
  that indoors, that poses a very significant danger. Outdoors, because
  of wind, nowhere near as much.
\item
  donald g. mcneil jr.\\
  Yeah, that's right.
\item
  michael barbaro\\
  So if being outdoors is less risky, and it's now been clinically
  shown, I wonder if that explains something you mentioned the last time
  that we spoke, which is that you did not have a tremendous amount of
  fear that these protests that have occurred all over the United States
  over race and policing, that they would be a major source of
  infection. And is that because they occurred outdoors? And is it so
  far the case that they haven't led to a meaningful spike in
  infections?
\item
  donald g. mcneil jr.\\
  We have not seen any big spike in infections in the cities where most
  of the protests took place. So it looks like they didn't lead to a lot
  of transmission. That doesn't imply that everything is safe just
  because it's outdoors. The important thing is how far apart people are
  when they're outdoors. So sitting right next to somebody else in front
  of a stage at Mount Rushmore, for example, where the chairs are zip
  tied together, is not safe. Masks or no masks, you still really want
  to try to keep six feet distance.
\item
  michael barbaro\\
  Donald, a couple of moments ago you mentioned the danger of being
  indoors because of this aerosolized virus mist that is not as great a
  danger outdoors. But I want to linger on this question of the indoors
  for a moment. Because the more we think about it, that aerosolized
  mist would seem to make any indoor activity inherently dangerous. I
  wonder if that's an accurate assessment?
\item
  donald g. mcneil jr.\\
  Yes. I mean, we've seen transmission of virus to large numbers of
  people in funerals, in choir practices, at birthday parties, inside
  bars, in business meetings. Virtually any kind of indoor environment
  you can imagine, there have been super spreader events. There may be
  ways to eventually make indoor spaces safer. There's going to be no
  way to make them completely safe.

  And all this talk about what's safe to do indoors brings us to really
  the most important question, which is the most important indoor space
  we want to get functioning again, which is schools. Can kids go to
  school safely? And again, the science isn't firm yet. But there are
  more and more hints that it may be safe, or pretty safe, to open the
  schools in the fall, especially for very young kids.

  There's growing evidence that kids are not big transmitters of the
  virus to adults. Denmark opened its schools in April. Did not see a
  big spike up in cases. Finland opened its schools in May. Did not see
  a big spike up in cases. Even from the beginning in China, the Chinese
  said, every time they looked at clusters in families, almost never did
  they see a case where the child, particularly the youngest child, was
  the one who introduced the virus into the family. Usually it was
  parent infecting the kids, not the other way around.

  We know that kids are big transmitters of flu viruses. And they do it
  because they cough and sneeze like crazy. But if the biggest symptom
  that they're getting is inflammation, rather than coughing and
  sneezing, --- and that's the case; kids tend to get more sort of
  cranky, inflammatory, unpleasant manifestations of the disease, rather
  than something that looks like a cold. Then it would make sense that
  that might be a reason why they're not big transmitters.
\item
  michael barbaro\\
  And what is this new insight about kids being less likely to transmit
  mean for the teachers who are going to stand or sit in front of them
  all day? Does it mean that an adult teacher in a school is pretty safe
  teaching? Or does it not mean that at all?
\item
  donald g. mcneil jr.\\
  I don't think we know the answer to that yet. I mean, schools ---
  you're bringing together a lot of kids. But schools also bring
  together a lot of adults. Teachers, staff, parents picking up the
  kids, things like that. So schools are not going to be completely safe
  under any circumstances.

  But opening schools is so important to society, much more important
  than opening restaurants, much more important than opening movie
  theaters. It probably needs to be done really carefully. Not just all
  back into the classroom, 30 kids to a classroom, at all. But it looks
  like it could be done. And that's really important. Because it's
  important for the kids, for their development, for their feeding, for
  their socialization. And it's also important for the parents. Parents
  can't go back to work if they're stuck at home with their kids. So
  it's a crucial part of getting both the economy going and just the
  health of kids and health of parents.
\item
  michael barbaro\\
  So of all the insights that you have shared today, this one seems like
  the silver lining. That reopening schools may be a somewhat safe
  undertaking.
\item
  donald g. mcneil jr.\\
  Yeah. And that would be very good news for us.
\item
  michael barbaro\\
  Because if I'm being candid, everything else you have said sounds
  pretty bad, right? I mean, it seems to be vascular, not respiratory.
  So it's going to be easy to miss symptoms. It seems it's becoming more
  transmissible through mutations. And the indoors presents very
  significant threats for non-kids because of this aerosolized mist. And
  once the temperature drops, which it will do in a few months, and tens
  of millions of us are suddenly stuck indoors, then we're in for a lot
  of trouble.
\item
  donald g. mcneil jr.\\
  Yeah, and the number of cases per day could rise well over 100,000 if
  we're not careful. So yeah, I guess, it's mostly bad news.
\item
  {[}laughter{]}\\
  Sorry. I'm hoping that the fact that the virus is becoming more
  transmissible also means that it will become less lethal, which would
  be good news. But it hasn't done that yet. So more transmission of a
  virus that's already bad is not a good thing. No question about it.
\item
  michael barbaro\\
  And all these things that we have just talked about would also seem to
  reinforce the need, not just for social distancing, but for these
  government-mandated lockdowns. I mean, specific requirements that say,
  don't go to a bar. Don't go to a restaurant. And those will become
  even more urgent as the warm weather yields to cold weather.
\item
  donald g. mcneil jr.\\
  Yeah. We have to realize we are just in the opening phases of this
  pandemic. I mean, this is the second inning. And there's still ---
  there's more than 120,000 people dead. So we are doing the dance in,
  dance out of various forms of lockdown. But we need to get to the
  point where we're all basically dancing to the same music. Where all
  governors accept the notion that when they have a problem that's
  getting out of control in their state, they react quickly.

  And if they do that, they will save lives of their own citizens. And I
  think we're beginning to see that.

  In places like Texas, places like Arizona, places like Florida the
  governors have made major about-faces in the last couple of weeks. And
  they're getting the science that the thing you do today doesn't
  produce good effects until a month from today, because the people who
  got infected yesterday are the ones who are going to be in your
  hospital three weeks from now. So they're beginning to catch on.

  But we need to arrive at sort of a common understanding that we don't
  all have to move in lockstep as a nation, but at the crucial moments
  we need to take similar steps to save lives.
\item
  {[}music{]}
\item
  michael barbaro\\
  Thank you, Donald. We appreciate it.
\item
  donald g. mcneil jr.\\
  Thank you. I was glad to be here.
\item
  michael barbaro\\
  We'll be right back.

  Here's what else you need to know today.
\item
  archived recording (donald trump)\\
  In our schools, our newsrooms, even our corporate boardrooms, there is
  a new far-left fascism that demands absolute allegiance.
\end{itemize}

michael barbaro

In a pair of back-to-back speeches over the weekend, President Trump
delivered harsh attacks against what he called the radical far-left
forces who are protesting police brutality and tearing down monuments to
America's racist past, describing them as a threat to American values
and heritage.

\begin{itemize}
\tightlist
\item
  archived recording (donald trump)\\
  If you do not speak its language, perform its rituals, recite its
  mantras and follow its commandments, then you will be censored,
  banished, blacklisted, persecuted and punished. It's not going to
  happen to us.
\end{itemize}

michael barbaro

The Times reports that the speeches, delivered in front of Mount
Rushmore and the White House, signaled that Trump would seek, once
again, to exploit racial and cultural divisions in an effort to win
re-election.

\begin{itemize}
\tightlist
\item
  archived recording (donald trump)\\
  I am here as your president to proclaim before the country and before
  the world this monument will never be desecrated. These heroes will
  never be defaced. Their legacy will never, ever be destroyed. Their
  achievements will never be forgotten. And Mount Rushmore will stand
  forever as an eternal tribute to our forefathers and to our freedom.
\end{itemize}

michael barbaro

Neither event enforced social distancing rules. And both were held
despite pleas from public health officials that they be canceled to
avoid spreading the coronavirus.

That's it for ``The Daily.'' I'm Michael Barbaro. See you tomorrow.

\hypertarget{en-quuxe9-se-diferencian-los-aerosoles-de-las-gotuxedculas}{%
\subsection{¿En qué se diferencian los aerosoles de las
gotículas?}\label{en-quuxe9-se-diferencian-los-aerosoles-de-las-gotuxedculas}}

Los aerosoles son gotículas y las gotículas son aerosoles, la única
diferencia es el tamaño. A veces los científicos llaman aerosoles a las
gotículas con un diámetro de menos de 5 micras. (En comparación, una
célula roja mide aproximadamente 5 micras de diámetro; un cabello humano
mide 50 micras de ancho).

Desde el inicio de la pandemia, la OMS y otras organizaciones de salud
pública se han enfocado en la capacidad del virus para propagarse a
través de grandes gotículas que son exhaladas cuando una persona con
síntomas tose o estornuda.

Estas gotículas son relativamente pesadas y caen rápidamente al suelo a
una superficie que otros podrían tocar. Por este motivo, las autoridades
de salud pública han recomendado mantener al menos una distancia de dos
metros respecto a otras personas y lavarse las manos con frecuencia.

Pero desde hace meses algunos expertos han dicho que las personas
infectadas también emiten aerosoles cuando tosen y estornudan. Lo que es
más importante, emiten aerosoles incluso cuando respiran, hablan o
cantan, especialmente cuando lo hacen con esfuerzo.

Los científicos ya saben que la gente puede transmitir el virus incluso
aunque no presente síntomas, es decir, sin toser o estornudar, y los
aerosoles podrían explicar por qué se da ese fenómeno.

Debido a que los aerosoles son más pequeños, contienen mucho menos virus
que las gotículas. Pero, puesto que son más ligeros, pueden sobrevivir
en el aire mucho más tiempo, sobre todo si no hay aire fresco. En un
espacio interior con mucha gente,
\href{https://www.nytimes.com/es/2020/07/03/espanol/el-misterio-de-los-superpropagadores-de-coronavirus.html}{una
sola persona infectada puede liberar suficiente virus aerosolizado para
infectar a muchas personas,} quizá provocando un evento de
superpropagación.

Para que las gotículas sean responsables de una propagación a ese nivel,
una sola persona tendría que estar a menos de un metro de todas las
demás o haber
contaminado\href{https://www.nytimes.com/2020/05/28/well/live/whats-the-risk-of-catching-coronavirus-from-a-surface.html}{un
objeto que todos tocaron.} Todo eso les parece poco probable a muchos
expertos: ``Tendría que hacer mucha gimnasia mental para explicar esas
otras rutas de transmisión en comparación con la transmisión de
aerosoles, lo cual es mucho más simple'', afirmó Marr.

\includegraphics{https://static01.nyt.com/images/2020/07/06/science/07aerosol-explainer-ES-02/merlin_174294708_78421d85-ec71-4ddc-bc95-f5e4adedd5d8-articleLarge.jpg?quality=75\&auto=webp\&disable=upscale}

\hypertarget{ya-puedo-dejar-de-preocuparme-por-la-sana-distancia-y-lavarme-las-manos}{%
\subsection{¿Ya puedo dejar de preocuparme por la sana distancia y
lavarme las
manos?}\label{ya-puedo-dejar-de-preocuparme-por-la-sana-distancia-y-lavarme-las-manos}}

El distanciamiento social sigue siendo muy importante. Entre más cerca
estés de una persona infectada, más estarás expuesto a sus aerosoles y
gotículas. Lavarte las manos con frecuencia sigue siendo una buena idea.

Lo nuevo es que quizá esas dos cosas no sean suficiente. ``Deberíamos
hacer tanto énfasis en los protectores faciales y la ventilación como lo
hacemos con el lavado de las manos'', explicó Marr. ``Hasta donde
sabemos, esto es igual de importante, si no es que más''.

\hypertarget{deberuxeda-empezar-a-usar-mascarillas-quiruxfargicas-en-interiores-cuuxe1nto-tiempo-podemos-estar-en-interiores}{%
\subsection{¿Debería empezar a usar mascarillas quirúrgicas en
interiores? ¿Cuánto tiempo podemos estar en
interiores?}\label{deberuxeda-empezar-a-usar-mascarillas-quiruxfargicas-en-interiores-cuuxe1nto-tiempo-podemos-estar-en-interiores}}

Quizá todos los médicos y trabajadores sanitarios deberían usar máscaras
N95, que filtran la mayoría de los aerosoles. Por ahora, solo se les
requiere que lo hagan cuando realizan ciertos procedimientos médicos que
se piensan que producen aerosoles.

Para el resto de nosotros, las mascarillas de tela aún reducen mucho el
riesgo, siempre y cuando todos las usen. En el hogar, cuando estás con
tu familia o con quienes compartes la casa y estás seguro que tienen
cuidado, las mascarillas todavía no son necesarias. Pero los expertos
sostienen que es prudente usarlas en otros espacios interiores.

Y sobre cuánto tiempo es seguro estar adentro con otras personas, eso es
difícil de decir, desafortunadamente. Mucho depende de si el lugar está
muy concurrido como para que no se pueda guardar una sana distancia y si
hay aire fresco que ventile la habitación.

\hypertarget{quuxe9-significa-la-transmisiuxf3n-auxe9rea-para-la-reapertura-de-escuelas-y-universidades}{%
\subsection{¿Qué significa la transmisión aérea para la reapertura de
escuelas y
universidades?}\label{quuxe9-significa-la-transmisiuxf3n-auxe9rea-para-la-reapertura-de-escuelas-y-universidades}}

Este es un tema de intenso debate. Muchas escuelas tienen poca
ventilación y no cuentan con financiamiento suficiente para invertir en
nuevos sistemas de filtración. ``Existe una gran vulnerabilidad a la
transmisión de infecciones a través de aerosoles en las escuelas'', dijo
Don Milton, un experto en aerosoles de la Universidad de Maryland.

La mayoría de los niños menores de 12 años parecen tener síntomas leves,
si es que los presentan, por lo que las escuelas primarias pueden
sobrevivir. ``Hasta el momento, no tenemos evidencia de que las escuelas
primarias sean un problema, pero creo que los grados superiores serían
más propensos a ser un problema'', dijo Milton.

Los dormitorios universitarios y las aulas también son motivo de
preocupación.

El doctor Milton dijo que el gobierno debería pensar en soluciones a
largo plazo para estos problemas. Tener las escuelas públicas cerradas
``obstruye toda la economía, y es una gran vulnerabilidad'', dijo.

``Hasta que comprendamos que esto es parte de nuestra defensa nacional y
lo financiemos adecuadamente, seguiremos siendo extremadamente
vulnerables a este tipo de amenazas biológicas''.

\hypertarget{quuxe9-puedo-hacer-para-minimizar-el-riesgo}{%
\subsection{¿Qué puedo hacer para minimizar el
riesgo?}\label{quuxe9-puedo-hacer-para-minimizar-el-riesgo}}

Haz todas las actividades que puedas al aire libre. A pesar de que estés
viendo muchas fotos con gente en la playa, aunque estén concurridas, y
sobre todo si hay algo de viento, seguramente será un lugar mucho más
seguro que dentro de un bar o restaurante con aire reciclado.

Pero aunque estés afuera, usa una mascarilla si estarás cerca de otras
personas durante mucho tiempo.

En espacios cerrados, algo sencillo que se puede hacer es ``abrir las
ventanas y puertas siempre que sea posible'', sostuvo Marr. También
puedes cambiar los filtros en tu sistema de aire acondicionado o ajustar
la configuración para que se use más aire del exterior y menos aire que
ya ha circulado dentro.

Tal vez sea buena idea que los edificios gubernamentales y las empresas
inviertan en purificadores de aire y
\href{https://www.nytimes.com/2020/05/07/science/ultraviolet-light-coronavirus.html}{luces
ultravioleta que puedan matar virus}. A pesar de su mala reputación, es
probable que los ascensores no presenten un riesgo tan grande, dijo Don
Milton, un experto en aerosoles en la Universidad de Maryland,
\href{https://www.nytimes.com/es/2020/06/19/espanol/coronavirus-infeccion-inodoro.html}{si
se comparan con baños público}s u oficinas con aire estancado donde la
gente pasa mucho tiempo.

Si nada de eso es posible, intenta reducir el tiempo que pasas en un
lugar cerrado, sobre todo si no usarás una mascarilla. Entre más tiempo
estés adentro,
\href{https://www.nytimes.com/2020/05/29/health/coronavirus-transmission-dose.html}{podrías
estar inhalando más dosis de virus.}

Advertisement

\protect\hyperlink{after-bottom}{Continue reading the main story}

\hypertarget{site-index}{%
\subsection{Site Index}\label{site-index}}

\hypertarget{site-information-navigation}{%
\subsection{Site Information
Navigation}\label{site-information-navigation}}

\begin{itemize}
\tightlist
\item
  \href{https://help.nytimes.com/hc/en-us/articles/115014792127-Copyright-notice}{©~2020~The
  New York Times Company}
\end{itemize}

\begin{itemize}
\tightlist
\item
  \href{https://www.nytco.com/}{NYTCo}
\item
  \href{https://help.nytimes.com/hc/en-us/articles/115015385887-Contact-Us}{Contact
  Us}
\item
  \href{https://www.nytco.com/careers/}{Work with us}
\item
  \href{https://nytmediakit.com/}{Advertise}
\item
  \href{http://www.tbrandstudio.com/}{T Brand Studio}
\item
  \href{https://www.nytimes.com/privacy/cookie-policy\#how-do-i-manage-trackers}{Your
  Ad Choices}
\item
  \href{https://www.nytimes.com/privacy}{Privacy}
\item
  \href{https://help.nytimes.com/hc/en-us/articles/115014893428-Terms-of-service}{Terms
  of Service}
\item
  \href{https://help.nytimes.com/hc/en-us/articles/115014893968-Terms-of-sale}{Terms
  of Sale}
\item
  \href{https://spiderbites.nytimes.com}{Site Map}
\item
  \href{https://help.nytimes.com/hc/en-us}{Help}
\item
  \href{https://www.nytimes.com/subscription?campaignId=37WXW}{Subscriptions}
\end{itemize}
