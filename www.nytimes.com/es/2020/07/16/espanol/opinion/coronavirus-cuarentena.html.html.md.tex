Sections

SEARCH

\protect\hyperlink{site-content}{Skip to
content}\protect\hyperlink{site-index}{Skip to site index}

\href{https://www.nytimes.com/es/section/opinion}{Opinión}

\href{https://myaccount.nytimes.com/auth/login?response_type=cookie\&client_id=vi}{}

\href{https://www.nytimes.com/section/todayspaper}{Today's Paper}

\href{/es/section/opinion}{Opinión}\textbar{}La pandemia podría empeorar
y por eso debemos actuar ahora

\href{https://nyti.ms/3evhPkM}{https://nyti.ms/3evhPkM}

\begin{itemize}
\item
\item
\item
\item
\item
\end{itemize}

\href{https://www.nytimes.com/es/spotlight/coronavirus?action=click\&pgtype=Article\&state=default\&region=TOP_BANNER\&context=storylines_menu}{El
brote de coronavirus}

\begin{itemize}
\tightlist
\item
  \href{https://www.nytimes.com/es/interactive/2020/espanol/america-latina/coronavirus-en-mexico.html?action=click\&pgtype=Article\&state=default\&region=TOP_BANNER\&context=storylines_menu}{Mapa
  y casos en México}
\item
  \href{https://www.nytimes.com/es/2020/07/31/espanol/ciencia-y-tecnologia/ninos-contagio-coronavirus.html?action=click\&pgtype=Article\&state=default\&region=TOP_BANNER\&context=storylines_menu}{Los
  niños y el virus}
\item
  \href{https://www.nytimes.com/es/interactive/2020/science/coronavirus-tratamientos-curas.html?action=click\&pgtype=Article\&state=default\&region=TOP_BANNER\&context=storylines_menu}{Fármacos
  y tratamientos}
\item
  \href{https://www.nytimes.com/es/2020/07/06/espanol/ciencia-y-tecnologia/coronavirus-transmision-aire.html?action=click\&pgtype=Article\&state=default\&region=TOP_BANNER\&context=storylines_menu}{Cómo
  se transmite el coronavirus}
\item
  \href{https://www.nytimes.com/es/2020/07/14/espanol/estilos-de-vida/botiquin-medicina-coronavirus.html?action=click\&pgtype=Article\&state=default\&region=TOP_BANNER\&context=storylines_menu}{Prepara
  tu botiquín}
\end{itemize}

Advertisement

\protect\hyperlink{after-top}{Continue reading the main story}

\href{/es/section/opinion}{Opinión}

Supported by

\protect\hyperlink{after-sponsor}{Continue reading the main story}

Comentario

\hypertarget{la-pandemia-podruxeda-empeorar-y-por-eso-debemos-actuar-ahora}{%
\section{La pandemia podría empeorar y por eso debemos actuar
ahora}\label{la-pandemia-podruxeda-empeorar-y-por-eso-debemos-actuar-ahora}}

Cuando se mezcla ciencia y política, la segunda suele ganar. Estados
Unidos ha probado que esa fórmula no ha funcionado para enfrentar al
coronavirus. Para planear una reapertura completa y segura tenemos que
volver a la ciencia.

\includegraphics{https://static01.nyt.com/images/2020/07/14/opinion/sunday/14barry/14barry-articleLarge.jpg?quality=75\&auto=webp\&disable=upscale}

Por John M. Barry

Es el autor de \emph{The Great Influenza: The Story of the Deadliest
Pandemic in History}.

\begin{itemize}
\item
  16 de julio de 2020
\item
  \begin{itemize}
  \item
  \item
  \item
  \item
  \item
  \end{itemize}
\end{itemize}

\href{https://www.nytimes.com/2020/07/14/opinion/coronavirus-shutdown.html}{Read
in English}

\href{https://www.nytimes.com/newsletters/el-times}{Regístrate para
recibir nuestro boletín} con lo mejor de The New York Times.

\begin{center}\rule{0.5\linewidth}{\linethickness}\end{center}

Cuando mezclas ciencia y política, obtienes política. Con el
coronavirus, Estados Unidos ha probado que la política no ha funcionado.
Si queremos planear la reapertura completa tanto de la economía como de
las escuelas de manera segura ---lo cual se puede lograr--- tenemos que
volver a la ciencia.

Para entender cuán mal está la situación en Estados Unidos y, lo más
importante, qué se puede hacer al respecto, es necesario hacer
comparaciones. En el momento de escribir este artículo,
\href{https://www.nytimes.com/interactive/2020/world/coronavirus-maps.html}{Italia},
que hace unos meses era la imagen de la devastación por el coronavirus y
cuya población es del doble de la de Texas, recientemente ha promediado
alrededor de 200 nuevos casos al día mientras que
\href{https://www.nytimes.com/interactive/2020/us/texas-coronavirus-cases.html}{Texas}
ha tenido más de 9000. Alemania, con una población cuatro veces la de
Florida, ha tenido menos de 400 nuevos casos al día. El 12 de julio,
Florida
\href{https://www.nytimes.com/2020/07/12/us/florida-coronavirus-covid-cases.html}{reportó}
más de 15.300, el total más alto para un solo día de cualquier estado de
Estados Unidos.

La Casa Blanca dice que el país tiene que aprender a vivir con el virus.
Una cosa sería si los nuevos casos ocurrieran al ritmo que suceden en
Italia o Alemania, sin mencionar a Corea del Sur, Australia o Vietnam
(que hasta el momento tiene cero muertes). Pero Estados Unidos tiene la
\href{https://coronavirus.jhu.edu/data/new-cases}{tasa de crecimiento
más alta} de nuevos casos en el mundo, incluso por encima de Brasil.

Italia, Alemania y decenas de otros países han reabierto casi por
completo, y tenían toda la razón en hacerlo. Todos tomaron el virus en
serio y actuaron de manera decisiva, y continúan haciéndolo: Australia
acaba de emitir multas por un total de 18.000 dólares porque demasiadas
personas asistieron a una fiesta de cumpleaños en una casa.

En Estados Unidos, los expertos en salud pública estuvieron de acuerdo
prácticamente de manera unánime en que replicar el éxito europeo
requería, primero, mantener el confinamiento hasta que alcanzáramos una
tendencia descendente pronunciada en el número de casos; segundo, lograr
un cumplimiento generalizado de las recomendaciones de salud pública, y,
tercero, crear una fuerza de trabajo de por lo menos 100.000 personas
---algunos expertos consideran que se necesitarían 300.000--- para
evaluar, rastrear y aislar casos. En el ámbito nacional, no estamos ni
cerca de cualquiera de esas metas, aunque algunos estados lo
consiguieron y ahora están reabriendo de manera cuidadosa y segura.
Otros estados distaron mucho de lograrlo, pero reabrieron de todos
modos. Ahora vemos los resultados.

Aunque la ciudad de Nueva York acaba de registrar su
\href{https://www.npr.org/sections/coronavirus-live-updates/2020/07/13/890427225/nyc-has-its-first-day-in-months-with-no-covid-19-deaths}{primer
día en meses} sin una muerte por la COVID-19, la
\href{https://www.nytimes.com/es/interactive/2020/espanol/mundo/coronavirus-en-estados-unidos.html}{pandemia
crece} en 39 estados. En el condado de Miami-Dade en Florida, seis
hospitales han llegado al límite de su capacidad. En Houston, donde se
desató uno de los peores brotes en el país, los funcionarios han
exhortado al gobernador a emitir una orden de permanecer en casa.

Como si el crecimiento explosivo en muchos estados no fuera lo
suficientemente malo, también sufrimos las mismas carencias que
afectaron a los hospitales en marzo y abril. En Nueva Orleans, los
suministros para pruebas están tan limitados que un lugar comenzó a
realizar pruebas a las 08:00 de la mañana, pero solo contaba con
suficientes para atender a las personas que ya estaban formadas a las
7:33 de la mañana.

Las pruebas por sí mismas tienen poco efecto sin una infraestructura no
solo para rastrear y contactar a personas posiblemente infectadas, sino
también para atender y apoyar a quienes resulten positivos y deban
aislarse, así como aquellos que deban someterse a una cuarentena
inmediata. Con demasiada frecuencia esto no ha sucedido; en Miami,
\href{https://miami.cbslocal.com/2020/07/09/mayors-coalition-wants-more-contact-tracers-miami-dade-county/}{solo
el 17 por ciento} de quienes dieron positivo por el coronavirus
completaron cuestionarios para colaborar con el rastreo de contactos,
una acción crucial para disminuir la propagación. Muchos estados ahora
tienen tantos casos que el rastreo de contactos se ha vuelto imposible.

¿Cuál es la solución?

El distanciamiento social, el
\href{https://www.nytimes.com/es/2020/03/19/espanol/ciencia-y-tecnologia/como-lavarse-las-manos-coronavirus.html}{lavado
de manos} y el confinamiento voluntario siguen siendo cruciales. Se ha
hecho muy poco énfasis en la ventilación, que también importa. En áreas
públicas, se pueden instalar luces ultravioletas. Estas cosas reducirán
la propagación, y el presidente Donald Trump finalmente usó un
cubrebocas en público, lo que
\href{https://www.nytimes.com/es/2020/06/11/espanol/opinion/coronavirus-economia-krugman.html}{podría
de alguna manera despolitizar} el asunto. Sin embargo, en este punto,
todas estas acciones juntas, incluso con un cumplimiento generalizado,
solo pueden ayudar a reducir tendencias peligrosas donde están
ocurriendo. El virus está demasiado extendido como para que estas
acciones aplanen la curva de manera rápida.

Para reabrir las escuelas de la manera más segura, lo que tal vez sea
imposible en algunas instancias, y reiniciar la economía, debemos
reducir el conteo de casos a niveles manejables, hasta alcanzar los
niveles de los países europeos. La amenaza del gobierno de Trump de
retener los fondos federales de las escuelas que no reabran no logrará
ese objetivo. Para hacer eso, solo las medidas decisivas funcionarán en
lugares que experimentan un crecimiento explosivo: como mínimo, fijar
límites incluso en reuniones privadas e imponer cierres selectivos que
deben incluir no solo los lugares obvios como los bares, sino también
las iglesias, que son una fuente bien documentada de propagación a gran
escala.

Dependiendo de las circunstancias locales, eso podría ser insuficiente;
quizá sea necesaria una cuarentena completa como la de abril. Esta
podría ser con base en cada condado, pero las medidas a medias lograrán
poco, excepto evitar que los hospitales se saturen. Las medidas a medias
dejarán la transmisión a un nivel que excederá por mucho aquellos de
diversos países que han logrado contener el virus. Las medidas a medias
causarán que muchos estadounidenses no vivan con el virus, sino que
fallezcan debido a él.

Durante la
\href{https://www.nytimes.com/2020/03/17/opinion/coronavirus-1918-spanish-flu.html}{pandemia
de la influenza de 1918}, casi todas las ciudades suspendieron gran
parte de sus actividades. El temor y el cuidado de los familiares
enfermos logró el resto; el ausentismo incluso en la industria bélica
excedió el 50 por ciento y destruyó la economía. Muchas ciudades
reabrieron demasiado pronto y tuvieron que cerrar por segunda vez ---en
algunos casos, una tercera vez--- y enfrentaron una intensa resistencia.
Sin embargo, se salvaron vidas.

Si lo hubiéramos hecho de la manera correcta desde la primera vez, ya
operaríamos a casi el 100 por ciento, las escuelas se alistarían para un
año escolar casi normal, los equipos de fútbol se prepararían para las
prácticas, y decenas de miles de estadounidenses no habrían muerto.

Esta es nuestra segunda oportunidad. No tendremos una tercera. Si no
contenemos el crecimiento de esta pandemia ahora, dentro de algunos
meses, cuando el clima se vuelva frío y obligue a las personas a pasar
más tiempo en interiores, podríamos enfrentar un desastre que haga lucir
minúscula a la situación actual.

\href{http://www.johnmbarry.com/index.htm}{John M. Barry} es profesor en
la Escuela de Salud Pública y Medicina Tropical en la Universidad de
Tulane y el autor de \emph{The Great Influenza: The Story of the
Deadliest Pandemic in History}.

Advertisement

\protect\hyperlink{after-bottom}{Continue reading the main story}

\hypertarget{site-index}{%
\subsection{Site Index}\label{site-index}}

\hypertarget{site-information-navigation}{%
\subsection{Site Information
Navigation}\label{site-information-navigation}}

\begin{itemize}
\tightlist
\item
  \href{https://help.nytimes.com/hc/en-us/articles/115014792127-Copyright-notice}{©~2020~The
  New York Times Company}
\end{itemize}

\begin{itemize}
\tightlist
\item
  \href{https://www.nytco.com/}{NYTCo}
\item
  \href{https://help.nytimes.com/hc/en-us/articles/115015385887-Contact-Us}{Contact
  Us}
\item
  \href{https://www.nytco.com/careers/}{Work with us}
\item
  \href{https://nytmediakit.com/}{Advertise}
\item
  \href{http://www.tbrandstudio.com/}{T Brand Studio}
\item
  \href{https://www.nytimes.com/privacy/cookie-policy\#how-do-i-manage-trackers}{Your
  Ad Choices}
\item
  \href{https://www.nytimes.com/privacy}{Privacy}
\item
  \href{https://help.nytimes.com/hc/en-us/articles/115014893428-Terms-of-service}{Terms
  of Service}
\item
  \href{https://help.nytimes.com/hc/en-us/articles/115014893968-Terms-of-sale}{Terms
  of Sale}
\item
  \href{https://spiderbites.nytimes.com}{Site Map}
\item
  \href{https://help.nytimes.com/hc/en-us}{Help}
\item
  \href{https://www.nytimes.com/subscription?campaignId=37WXW}{Subscriptions}
\end{itemize}
