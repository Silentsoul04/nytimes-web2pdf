Sections

SEARCH

\protect\hyperlink{site-content}{Skip to
content}\protect\hyperlink{site-index}{Skip to site index}

\href{https://www.nytimes.com/es/section/america-latina}{América Latina}

\href{https://myaccount.nytimes.com/auth/login?response_type=cookie\&client_id=vi}{}

\href{https://www.nytimes.com/section/todayspaper}{Today's Paper}

\href{/es/section/america-latina}{América Latina}\textbar{}Coronavirus
en América Latina: algunas autoridades respaldan tratamientos
cuestionables

\url{https://nyti.ms/3fUL6Hh}

\begin{itemize}
\item
\item
\item
\item
\item
\item
\end{itemize}

\href{https://www.nytimes.com/es/spotlight/coronavirus?action=click\&pgtype=Article\&state=default\&region=TOP_BANNER\&context=storylines_menu}{El
brote de coronavirus}

\begin{itemize}
\tightlist
\item
  \href{https://www.nytimes.com/es/interactive/2020/espanol/mundo/coronavirus-en-estados-unidos.html?action=click\&pgtype=Article\&state=default\&region=TOP_BANNER\&context=storylines_menu}{Mapa
  y casos en EE. UU.}
\item
  \href{https://www.nytimes.com/es/2020/07/23/espanol/america-latina/bolivia-cloro-coronavirus-ivermectina.html?action=click\&pgtype=Article\&state=default\&region=TOP_BANNER\&context=storylines_menu}{Dióxido
  de cloro, ivermectina y más: ¿funcionan?}
\item
  \href{https://www.nytimes.com/es/interactive/2020/science/coronavirus-tratamientos-curas.html?action=click\&pgtype=Article\&state=default\&region=TOP_BANNER\&context=storylines_menu}{Fármacos
  y tratamientos}
\item
  \href{https://www.nytimes.com/es/2020/07/28/espanol/ciencia-y-tecnologia/anticuerpos-coronavirus-inmunidad.html?action=click\&pgtype=Article\&state=default\&region=TOP_BANNER\&context=storylines_menu}{Anticuerpos
  e inmunidad}
\item
  \href{https://www.nytimes.com/es/2020/04/29/espanol/estilos-de-vida/oximetro-para-que-sirve.html?action=click\&pgtype=Article\&state=default\&region=TOP_BANNER\&context=storylines_menu}{Oxímetros}
\end{itemize}

Advertisement

\protect\hyperlink{after-top}{Continue reading the main story}

Supported by

\protect\hyperlink{after-sponsor}{Continue reading the main story}

Sudamérica

\hypertarget{coronavirus-en-amuxe9rica-latina-algunas-autoridades-respaldan-tratamientos-cuestionables}{%
\section{Coronavirus en América Latina: algunas autoridades respaldan
tratamientos
cuestionables}\label{coronavirus-en-amuxe9rica-latina-algunas-autoridades-respaldan-tratamientos-cuestionables}}

Una solución de cloro, muy popular en Bolivia, es solo uno de varios
remedios de eficacia no comprobada que ganan terreno en una región
necesitada de esperanza.

\includegraphics{https://static01.nyt.com/images/2020/07/22/world/22virus-bolivia-chlorine-ES/22virus-bolivia-chlorine-articleLarge.jpg?quality=75\&auto=webp\&disable=upscale}

Por María Silvia Trigo, Anatoly Kurmanaev y José María León Cabrera

\begin{itemize}
\item
  Publicado 23 de julio de 2020Actualizado 24 de julio de 2020
\item
  \begin{itemize}
  \item
  \item
  \item
  \item
  \item
  \item
  \end{itemize}
\end{itemize}

\href{https://www.nytimes.com/2020/07/23/world/americas/chlorine-coronavirus-bolivia-latin-america.html}{Read
in English}

\href{https://www.nytimes.com/newsletters/el-times}{Regístrate para
recibir nuestro boletín} con lo mejor de The New York Times.

\begin{center}\rule{0.5\linewidth}{\linethickness}\end{center}

TARIJA, Bolivia --- En Cochabamba, en lo alto de los Andes bolivianos,
la gente hace fila todos los días fuera de las farmacias de la plaza
central, ansiosa por comprar el escaso elixir que, esperan, alejará a la
COVID-19: dióxido de cloro, un tipo de blanqueador que se usa para
desinfectar piscinas y pisos.

Los expertos dicen que, en el mejor de los casos, beberlo no tiene
sentido y, en el peor de ellos, es peligroso. Pero en Bolivia, donde
varias personas han sido hospitalizadas después de ingerir dióxido de
cloro, las autoridades regionales lo están probando en presos, el Senado
nacional aprobó la semana pasada su uso y un importante legislador
amenazó con expulsar a la Organización Mundial de la Salud por oponerse
a su uso médico.

Julio César Baldivieso, un héroe local de fútbol y excapitán de la
selección nacional,
\href{https://www.facebook.com/watch/live/?v=214675996308544\&ref=watch_permalink}{dijo
a un canal de televisión} que debido a que los hospitales de Cochabamba
``no tienen reactivos, no tienen insumos, no tienen equipos de
bioseguridad'', él y su familia habían recurrido al dióxido de cloro
para tratar sus síntomas de coronavirus.

Los bolivianos tienen mucha compañía
---\href{https://www.nytimes.com/2020/06/03/health/hydroxychloroquine-coronavirus-trump.html}{incluido
el presidente de Estados Unidos, Donald Trump}--- al recurrir a
tratamientos no aprobados e incluso peligrosos para prevenir o tratar
infecciones. En cada parte del mundo, la ciencia dura ha tenido que
competir por la atención con teorías populares, rumores y creencias
tradicionales durante esta pandemia, como en el pasado.

Pero el interés en medicamentos cuestionables ha sido especialmente
elevado recientemente en América Latina, donde el virus causa estragos
sin control y muchos líderes políticos de derecha e izquierda los
promueven, ya sea por fe genuina o por el deseo de ofrecer esperanza y
desviar la culpa.

En una región donde pocas personas pueden pagar una atención médica de
calidad, los tratamientos alternativos son ampliamente promocionados en
las redes sociales y explotados por los especuladores.

``Hay una desesperación de la gente frente a la COVID-19'', dijo
Santiago Ron, un profesor de biología ecuatoriano, quien se ha
enfrentado a los defensores de los supuestos tratamientos peligrosos,
\href{https://www.facebook.com/jambato/posts/10223907884827653}{incluidos
legisladores}. ``La gente está muy vulnerable a los discursos
pseudocientíficos''.

\includegraphics{https://static01.nyt.com/images/2020/07/22/world/22virus-bolivia-chlorine3-ES/merlin_174813969_70c99bfb-2d19-43d2-9240-6a5628c9b216-articleLarge.jpg?quality=75\&auto=webp\&disable=upscale}

El coronavirus ha infectado a más de tres millones de personas y ha
matado a unas 160.000 en América Latina, según cifras oficiales, lo que
convierte a la región en una de las más afectadas por la pandemia. Los
\href{https://www.nytimes.com/interactive/2020/04/21/world/coronavirus-missing-deaths.html}{expertos
y los análisis estadísticos} indican que las cifras de víctimas mortales
son mucho mayores a las oficiales y están distorsionadas debido a la
\href{https://www.nytimes.com/es/2020/04/23/espanol/america-latina/virus-ecuador-muertes.html}{capacidad
limitada de pruebas} y recursos médicos y a
\href{https://www.nytimes.com/es/2020/05/08/espanol/america-latina/mexico-coronavirus.html}{la
resistencia de algunos gobiernos} a reconocer públicamente
\href{https://www.nytimes.com/es/2020/06/08/espanol/america-latina/brasil-cifras-coronavirus.html}{el
alcance de la crisis}.

La COVID-19 ha maltratado los ya frágiles sistemas de atención médica, y
las medidas de confinamiento han devastado a las economías sin lograr
controlar el virus.

Los científicos están probando un amplio espectro de tratamientos no
comprobados, pero las probabilidades de que algunos de ellos sean útiles
se consideran bajas, y se sabe que algunos son potencialmente dañinos.
En muchos casos, no hay evidencia sólida de que funcionan contra el
coronavirus.

Uno de los fármacos que despierta ese tipo de interés es la ivermectina,
que se usa para tratar parásitos intestinales. Dos ministros brasileños
anunciaron el lunes que habían dado positivo por el coronavirus, y uno
de ellos dijo que se estaba tratando con ivermectina, entre otros
medicamentos.

El gobierno de Perú compró ivermectina para combatir la pandemia, y ha
continuado promocionándola, incluso después de que
\href{https://www.paho.org/es/documentos/recomendacion-sobre-uso-ivermectina-tratamiento-covid-19}{la
OMS dijo} que no debía usarse para tratar al coronavirus. Esto provocó
la explosión de un mercado ilegal de la versión veterinaria de la
ivermectina, lo que obligó al gobierno peruano ---y a la
\href{https://www.fda.gov/animal-veterinary/product-safety-information/fda-letter-stakeholders-do-not-use-ivermectin-intended-animals-treatment-covid-19-humans}{Administración
de Drogas y Alimentos} de Estados Unidos (FDA, por su sigla en
inglés)--- a advertir a los ciudadanos contra el uso de medicinas para
animales de granja.

Aún así, en el pequeño pueblo de Nauta, en la Amazonía peruana, el
gobierno local y los grupos religiosos llegaron a dar ivermectina
veterinaria a adultos y niños de hasta cuatro años,
\href{https://www.youtube.com/watch?v=b7d6ICDRbzo\&feature=youtu.be\&fbclid=IwAR2AxONvP6C8FWbkRAtQRgA-iC4wQm4DNutT1RbaRnk0kf9hivTQx8Fgmm0}{según
los medios locales} y un grupo de derechos humanos.

El presidente Trump ha comentado ideas infundadas como tratar el virus
con
\href{https://www.nytimes.com/2020/04/24/health/sunlight-coronavirus-trump.html}{luces
potentes o inyecciones de desinfectante}. En repetidas ocasiones ha
promocionado la hidroxicloroquina, un medicamento contra la malaria,
llamándola un
``\href{https://www.nytimes.com/2020/04/17/health/trump-hydroxychloroquine-coronavirus.html}{punto
de inflexión}'' en la pandemia, a pesar de las investigaciones
científicas en contra, y ha dicho que la tomó durante dos semanas.

Pero en Estados Unidos, la hidroxicloroquina no lleva el sello casi
oficial que sí tiene en algunas partes de América Latina.

En Brasil, con el segundo lugar en número de casos y de fallecimientos
de coronavirus en el mundo después de Estados Unidos, el presidente Jair
Bolsonaro ha promovido sin tregua el medicamento,
\href{https://www.nytimes.com/2020/07/08/world/americas/brazil-bolsonaro-covid-coronavirus.html}{incluso
después de que él mismo desarrolló la COVID-19}, después de haberlo
estado tomando durante meses. Él ha ordenado a los militares que la
produzcan en masa, y después de su diagnóstico,
\href{https://twitter.com/SamPancher/status/1284974259698380801?s=19}{agitó
un paquete} frente a un grupo de entusiastas seguidores.

Image

El presidente de Brasil, Jair Bolsonaro, muestra una caja de
hidroxicloroquina a sus partidarios fuera del Palacio de la Alvorada en
Brasilia, Brasil, en una captura del video publicado en su página
oficial de Facebook el 19 de julio.

Los gobiernos en El Salvador, Perú y Paraguay compraron
hidroxicloroquina para tratar el coronavirus.

Los estudios han encontrado que el medicamento
\href{https://www.nytimes.com/2020/06/03/health/hydroxychloroquine-coronavirus-trump.html}{no
disminuyó la posibilidad de infección},
\href{https://www.acpjournals.org/doi/10.7326/M20-4207}{redujo la
gravedad} de la COVID-19 o
\href{https://www.recoverytrial.net/news/statement-from-the-chief-investigators-of-the-randomised-evaluation-of-covid-19-therapy-recovery-trial-on-hydroxychloroquine-5-june-2020-no-clinical-benefit-from-use-of-hydroxychloroquine-in-hospitalised-patients-with-covid-19}{aceleró
la recuperación}. Pero es potencialmente peligroso, particularmente para
personas con ritmos cardíacos anormales.

En Venezuela, el gobierno del presidente Nicolás Maduro, que tiene
problemas incluso para dotar de agua potable y jabón a sus hospitales en
ruinas, se ha jactado de haber obtenido de su aliada Cuba decenas de
miles de dosis de un medicamento, interferón alfa-2b, utilizado contra
algunos virus y tipos cáncer, para combatir la pandemia. Las clínicas
del Estado ahora requieren que los pacientes con síntomas del
coronavirus tomen el fármaco.

Pero no hay evidencia concluyente de que este medicamento en particular,
uno de los muchos que constituyen esta clase de interferón, funciona
contra el coronavirus, y en Estados Unidos los Institutos Nacionales de
Salud
\href{https://www.covid19treatmentguidelines.nih.gov/immune-based-therapy/immunomodulators/interferons/}{no
recomiendan actualmente su uso en pacientes con la COVID-19}.

Siguiendo el ejemplo de Bolivia, la Asamblea de Ecuador recientemente
debatió si debía permitir el dióxido de cloro como tratamiento contra el
coronavirus y 10 obispos católicos han
\href{https://www.ecuavisa.com/articulo/noticias/nacional/621671-obispos-solicitan-renuncia-del-ministro-salud}{hecho
un llamado para que se utilice}.

El químico se ha publicitado desde hace mucho tiempo sin aprobación
oficial, también en Estados Unidos, como cura para padecimientos como el
sida y el autismo. La FDA repetidamente ha dicho que no tiene valor
médico y que puede tener
\href{https://web.archive.org/web/20190814102219/https:/www.fda.gov/news-events/press-announcements/fda-warns-consumers-about-dangerous-and-potentially-life-threating-side-effects-miracle-mineral}{efectos
potencialmente mortales}, entre ellos ``vómitos severos, diarrea severa,
presión arterial baja potencialmente mortal causada por deshidratación e
insuficiencia hepática aguda''.

Al menos 10 bolivianos han sido hospitalizados con envenenamiento por
dióxido de cloro en semanas recientes, de acuerdo con el Ministerio de
Salud.

Pero el miércoles, Efraín Chambi, el líder de la mayoría en el Senado de
Bolivia, dijo que su partido pediría que la OMS abandone el país si
siguen recomendando a la gente que no tome dióxido de cloro.

``No hacen ningún favor al pueblo boliviano'', dijo. ``Creemos que están
del lado de grandes transnacionales''.

Image

Trabajadores sanitarios atendían a un paciente de coronavirus en un
hospital de campaña en Santa Cruz este mesCredit...Enrique Canedo/Agence
France-Presse --- Getty Images

Después de contener con éxito la enfermedad durante meses, Bolivia, uno
de los países más pobres de América Latina, sucumbió a
\href{https://www.nytimes.com/2020/07/09/world/americas/bolivia-president-jeanine-anez-coronavirus.html?searchResultPosition=1}{un
agresivo brote}este mes que ha abrumado a los hospitales. Esta semana,
la policía recolectó cientos de cuerpos de presuntas víctimas de la
COVID-19 de las calles y hogares en las ciudades de Santa Cruz y La Paz,
y, el jueves, el gobierno pospuso las elecciones nacionales de
septiembre a octubre, aludiendo preocupaciones de seguridad.

El virus se extendió rápidamente hasta los niveles más altos del poder,
infectó a la
\href{https://www.nytimes.com/2020/07/09/world/americas/bolivia-president-jeanine-anez-coronavirus.html}{presidenta
interina, Jeanine Añez}, y a la mitad de su gabinete, lo que alimentó
una sensación de desamparo. Los políticos y las figuras públicas
populares comenzaron a promover el dióxido de cloro como un tratamiento
alternativo.

El Senado, controlado por la oposición, aprobó la semana pasada un
proyecto de ley que permitiría a los gobiernos locales suministrar la
solución de forma gratuita para uso médico, a pesar de las protestas del
Ministerio de Salud. Añez ha guardado silencio sobre la controversia,
mientras su candidatura electoral pierde apoyo.

En Cochabamba, en el centro del país, donde una botella de 3,78 litros
de dióxido de cloro se vende por ocho dólares ---cuando se puede
encontrar--- los residentes bloquearon el camino a la planta municipal
de tratamiento de residuos hasta que las autoridades locales prometieron
proporcionarlo de forma gratuita.

Baldivieso, de 48 años, el futbolista, dijo que él y toda su familia
comenzaron a beber el químico después de experimentar por primera vez
los síntomas del coronavirus. Dijo que había tenido que esperar 15 días
para obtener un resultado de la prueba, que resultó positivo.

``¿Qué podía haber pasado si nosotros no tomábamos ningún tipo de
precauciones?'', dijo.

Image

Una brigada de trabajadores de la salud llevó a cabo pruebas a domicilio
en Villa El Rosal, cerca de La Paz,~este mesCredit...Juan
Karita/Associated Press

En la capital boliviana, Sucre, funcionarios de salud local empezaron la
semana pasada a
\href{https://www.facebook.com/cadenaarednacional/videos/648916312499999}{probar
el dióxido de cloro} en 200 guardias y presos, algunos de los cuales
presentan síntomas de coronavirus. El funcionario al mando de la
prisión, Ludwin Miranda, dijo que todos los participantes habían firmado
formularios de consentimiento.

En San José de Chiquitos, un pueblo al este de Bolivia de 30.000
habitantes, el alcalde distribuyó dióxido de cloro a los centros médicos
locales para que trataran el virus.

``Ha resultado perfectamente la aplicación de dióxido de cloro en la
recuperación de pacientes críticos'', dijo el alcalde Germaín Caballero,
a una estación local de televisión la semana pasada. ``Hemos logrado un
nivel de control y de freno al avance de la pandemia''.

Los expertos médicos dicen que el dióxido de cloro es, en el mejor de
los casos, un placebo, y, como con cualquier placebo, las personas
podrían atribuirle su recuperación.

Quienes defienden el uso del dióxido de cloro ``crean una falsa
seguridad'', dijo en una entrevista Virgilio Prieto, director de
epidemiología del Ministerio de Salud de Bolivia. ``Al promover su uso
indiscriminado e irresponsable están poniendo en riesgo a la
población''.

María Silvia Trigo reportó desde Tarija, Bolivia; Anatoly Kurmanaev,
desde Caracas, Venezuela, y José María León Cabrera, desde Quito,
Ecuador. Mitra Taj colaboró con reportería en Lima, Perú; Isayen Herrera
en Caracas, Venezuela; Manuela Andreoni en Nova Friburgo, Brasil; Norman
Chinchilla en Cochabamba, Bolivia, y Jenny Carolina González en Bogotá,
Colombia.

\begin{center}\rule{0.5\linewidth}{\linethickness}\end{center}

Advertisement

\protect\hyperlink{after-bottom}{Continue reading the main story}

\hypertarget{site-index}{%
\subsection{Site Index}\label{site-index}}

\hypertarget{site-information-navigation}{%
\subsection{Site Information
Navigation}\label{site-information-navigation}}

\begin{itemize}
\tightlist
\item
  \href{https://help.nytimes.com/hc/en-us/articles/115014792127-Copyright-notice}{©~2020~The
  New York Times Company}
\end{itemize}

\begin{itemize}
\tightlist
\item
  \href{https://www.nytco.com/}{NYTCo}
\item
  \href{https://help.nytimes.com/hc/en-us/articles/115015385887-Contact-Us}{Contact
  Us}
\item
  \href{https://www.nytco.com/careers/}{Work with us}
\item
  \href{https://nytmediakit.com/}{Advertise}
\item
  \href{http://www.tbrandstudio.com/}{T Brand Studio}
\item
  \href{https://www.nytimes.com/privacy/cookie-policy\#how-do-i-manage-trackers}{Your
  Ad Choices}
\item
  \href{https://www.nytimes.com/privacy}{Privacy}
\item
  \href{https://help.nytimes.com/hc/en-us/articles/115014893428-Terms-of-service}{Terms
  of Service}
\item
  \href{https://help.nytimes.com/hc/en-us/articles/115014893968-Terms-of-sale}{Terms
  of Sale}
\item
  \href{https://spiderbites.nytimes.com}{Site Map}
\item
  \href{https://help.nytimes.com/hc/en-us}{Help}
\item
  \href{https://www.nytimes.com/subscription?campaignId=37WXW}{Subscriptions}
\end{itemize}
