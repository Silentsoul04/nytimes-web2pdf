Sections

SEARCH

\protect\hyperlink{site-content}{Skip to
content}\protect\hyperlink{site-index}{Skip to site index}

\href{https://www.nytimes.com/es/section/ciencia-y-tecnologia}{Ciencia y
Tecnología}

\href{https://myaccount.nytimes.com/auth/login?response_type=cookie\&client_id=vi}{}

\href{https://www.nytimes.com/section/todayspaper}{Today's Paper}

\href{/es/section/ciencia-y-tecnologia}{Ciencia y
Tecnología}\textbar{}Cómo reabrir las escuelas: lo que la ciencia y la
experiencia de varios países nos enseñan

\url{https://nyti.ms/3f7Dm3j}

\begin{itemize}
\item
\item
\item
\item
\item
\item
\end{itemize}

\href{https://www.nytimes.com/es/spotlight/coronavirus?action=click\&pgtype=Article\&state=default\&region=TOP_BANNER\&context=storylines_menu}{El
brote de coronavirus}

\begin{itemize}
\tightlist
\item
  \href{https://www.nytimes.com/es/interactive/2020/espanol/mundo/coronavirus-en-estados-unidos.html?action=click\&pgtype=Article\&state=default\&region=TOP_BANNER\&context=storylines_menu}{Mapa
  y casos en EE. UU.}
\item
  \href{https://www.nytimes.com/es/2020/07/23/espanol/america-latina/bolivia-cloro-coronavirus-ivermectina.html?action=click\&pgtype=Article\&state=default\&region=TOP_BANNER\&context=storylines_menu}{Dióxido
  de cloro, ivermectina y más: ¿funcionan?}
\item
  \href{https://www.nytimes.com/es/interactive/2020/science/coronavirus-tratamientos-curas.html?action=click\&pgtype=Article\&state=default\&region=TOP_BANNER\&context=storylines_menu}{Fármacos
  y tratamientos}
\item
  \href{https://www.nytimes.com/es/2020/07/28/espanol/ciencia-y-tecnologia/anticuerpos-coronavirus-inmunidad.html?action=click\&pgtype=Article\&state=default\&region=TOP_BANNER\&context=storylines_menu}{Anticuerpos
  e inmunidad}
\item
  \href{https://www.nytimes.com/es/2020/04/29/espanol/estilos-de-vida/oximetro-para-que-sirve.html?action=click\&pgtype=Article\&state=default\&region=TOP_BANNER\&context=storylines_menu}{Oxímetros}
\end{itemize}

Advertisement

\protect\hyperlink{after-top}{Continue reading the main story}

Supported by

\protect\hyperlink{after-sponsor}{Continue reading the main story}

\hypertarget{cuxf3mo-reabrir-las-escuelas-lo-que-la-ciencia-y-la-experiencia-de-varios-pauxedses-nos-enseuxf1an}{%
\section{Cómo reabrir las escuelas: lo que la ciencia y la experiencia
de varios países nos
enseñan}\label{cuxf3mo-reabrir-las-escuelas-lo-que-la-ciencia-y-la-experiencia-de-varios-pauxedses-nos-enseuxf1an}}

La presión para que los estudiantes estadounidenses vuelvan a las aulas
es intensa, pero evaluar el riesgo es complicado cuando las infecciones
aún están fuera de control en muchas comunidades.

\includegraphics{https://static01.nyt.com/images/2020/07/12/science/27reopenschools-ES-00/merlin_170865825_2993c63a-7bb5-4ae4-853c-2f355b29af24-articleLarge.jpg?quality=75\&auto=webp\&disable=upscale}

Por \href{https://www.nytimes.com/by/pam-belluck}{Pam Belluck},
\href{https://www.nytimes.com/by/apoorva-mandavilli}{Apoorva Mandavilli}
y \href{https://www.nytimes.com/by/benedict-carey}{Benedict Carey}

\begin{itemize}
\item
  27 de julio de 2020
\item
  \begin{itemize}
  \item
  \item
  \item
  \item
  \item
  \item
  \end{itemize}
\end{itemize}

\href{https://www.nytimes.com/2020/07/11/health/coronavirus-schools-reopen.html}{Read
in English}

\href{https://www.nytimes.com/newsletters/el-times}{Regístrate para
recibir nuestro boletín} con lo mejor de The New York Times.

\begin{center}\rule{0.5\linewidth}{\linethickness}\end{center}

En tanto los distritos escolares estadounidenses consideran si van a
reiniciar las clases presenciales y cómo hacerlo, su desafío se complica
por un par de incertidumbres fundamentales: ningún país ha tratado de
enviar a los niños a la escuela con el virus en niveles como los de
Estados Unidos y la investigación científica sobre la transmisión en las
aulas es limitada.

La Organización Mundial de la Salud ha concluido que
\href{https://slack-redir.net/link?url=https\%3A\%2F\%2Fwww.nytimes.com\%2F2020\%2F07\%2F09\%2Fhealth\%2Fvirus-aerosols-who.html}{el
virus se transmite por el aire} en espacios
\href{https://www.nytimes.com/es/2020/07/08/espanol/ciencia-y-tecnologia/coronavirus-aire-aerosoles.html}{interiores
abarrotados y con poca ventilación}, una descripción que concuerda con
la realidad ed muchas escuelas estadounidenses. Pero hay una enorme
presión para traer de vuelta a los estudiantes: de padres, pediatras y
especialistas en desarrollo infantil, y del presidente de Estados Unidos
Donald Trump.

``Voy a decirlo: parece que estamos jugando a la ruleta rusa con
nuestros niños y nuestro personal'', dijo Robin Cogan, enfermera en la
escuela Yorkship en Camden, Nueva Jersey, que forma parte del comité
estatal para reabrir las escuelas.

\href{https://www.cdc.gov/coronavirus/2019-ncov/hcp/pediatric-hcp.html\#burden-disease-risk-factors}{Los
datos de todo el mundo} muestran claramente que los niños tienen muchas
menos probabilidades de enfermarse gravemente por el coronavirus que los
adultos. Pero hay grandes preguntas sin respuesta, que incluyen con qué
frecuencia los niños se infectan y qué papel juegan en la transmisión
del virus. Algunas investigaciones sugieren que los niños más pequeños
tienen menos probabilidades de infectar a otras personas que los
adolescentes, lo que haría que abrir las escuelas primarias sea menos
riesgoso que hacerlo con las escuelas secundarias, pero la evidencia no
es concluyente.

La experiencia en el extranjero ha demostrado que medidas como el
distanciamiento físico y el uso de cubrebocas en las escuelas pueden
marcar la diferencia. Otra variable importante es qué tan extendido está
el virus en la comunidad en general, porque eso afectará al número de
personas que podrían llevarlo a la escuela.

Para la mayoría de los distritos, la solución no será un claro todo o
nada.
\href{https://bioethics.jhu.edu/research-and-outreach/projects/eschool-initiative/school-policy-tracker/}{Muchos
sistemas escolares}, incluido el más grande del país, en la ciudad de
Nueva York, idean híbridos que implicarán pasar algunos días en las
aulas y otros días en línea.

``Hay que hacer mucho más que agitar las manos y decir hazlo así'', dijo
Joshua Sharfstein, profesor en la Escuela Bloomberg de Salud Pública de
Johns Hopkins. ``Primero tienes que controlar la propagación de la
comunidad y luego debes abrir las escuelas cuidadosamente''.

\hypertarget{el-acertijo-de-la-transmisiuxf3n}{%
\subsection{El acertijo de la
transmisión}\label{el-acertijo-de-la-transmisiuxf3n}}

Aunque los niños tienen un riesgo mucho menor de enfermarse gravemente
por el coronavirus que los adultos, el riesgo existe. Un pequeño número
de niños murió y otros necesitaron cuidados intensivos porque
\href{https://www.nytimes.com/2020/04/06/health/coronavirus-children-us.html}{sufrieron
insuficiencia respiratoria} o un
\href{https://www.nytimes.com/es/2020/05/18/espanol/sindrome-coronavirus-ninos.html}{síndrome
inflamatorio} que causó problemas cardíacos o circulatorios.

La mayor preocupación con la reapertura de las escuelas es la
posibilidad de que los niños se infecten, muchos sin síntomas, y luego
transmitan el virus a otros, incluidos los miembros de la familia, sus
maestros y otros empleados de la escuela. La mayoría de la evidencia
hasta la fecha sugiere que, incluso si los niños menores de 12 años
están infectados en las mismas tasas que los adultos que los rodean, es
menos probable que lo propaguen. La Academia Estadounidense de Pediatría
ha citado algunos de estos datos para
\href{https://services.aap.org/en/pages/2019-novel-coronavirus-covid-19-infections/clinical-guidance/covid-19-planning-considerations-return-to-in-person-education-in-schools/}{recomendar
que las escuelas vuelvan a abrir} con las debidas precauciones de
seguridad.

Pero la mayor parte de la evidencia se recopiló en países que ya estaban
confinados o que habían comenzado a implementar otras medidas
preventivas. Y pocos países han examinado sistemáticamente a los niños
para detectar el virus o los anticuerpos que indicarían si habían estado
expuestos al virus.

Los especialistas en enfermedades infecciosas han modelado desde febrero
el impacto de las escuelas en la propagación comunitaria.

\includegraphics{https://static01.nyt.com/images/2020/07/10/science/27reopenschools-ES-01/merlin_170483466_ca28d6d9-7b78-4509-9b49-9fcddef888b3-articleLarge.jpg?quality=75\&auto=webp\&disable=upscale}

En marzo, la mayoría de los expertos en modelos acordaron que cerrar las
escuelas
\href{https://www.nytimes.com/2020/05/05/health/coronavirus-children-transmission-school.html}{retrasaría
la progresión de las infecciones}. Pero medidas más amplias, como el
distanciamiento social, demostraron tener un efecto de contención mucho
mayor, lo que eclipsó los resultados del cierre de escuelas,
\href{https://www.medrxiv.org/content/10.1101/2020.04.16.20068403v1}{según
análisis recientes}.

El riesgo de reapertura ``dependerá de qué tan bien las escuelas
contengan la transmisión, con cubrebocas, por ejemplo, o al limitar el
aforo'', dijo Lauren Ancel Meyers, profesora de biología y estadística
en la Universidad de Texas, en Austin, quien ha estado asesorando a la
ciudad y los distritos escolares. ``La tasa de transmisión comunitaria
en agosto también será un factor''.

En Austin, Texas, por ejemplo, que al igual que las ciudades de Florida
y Arizona ha visto una aceleración reciente en nuevos casos, la tasa de
infección estimada a principios de julio era de aproximadamente siete
por cada 1000 residentes. Eso significa que una escuela con 500
estudiantes tendría unos cuatro con coronavirus. ``La escuela podría
contenerlos, dependiendo de las medidas que tome'', dijo Meyers.

De lo contrario, las escuelas podrían ayudar a incubar brotes, dado que
son instalaciones cerradas donde es probable que los estudiantes,
especialmente los más jóvenes, tengan grandes dificultades para el
distanciamiento social, por no hablar del uso de cubrebocas. Incluso si
resulta que los niños no transmiten el virus de manera eficiente, todo
lo que se necesitaría es uno o dos para sembrar nuevas cadenas.

\hypertarget{la-evidencia-del-exterior}{%
\subsection{La evidencia del exterior}\label{la-evidencia-del-exterior}}

Hasta ahora, los países que reabrieron las escuelas después de reducir
los niveles de infección ---e imponer requisitos como distanciamiento
físico y limitar el tamaño de los grupos---
\href{https://globalhealth.washington.edu/sites/default/files/COVID-19\%20Schools\%20Summary\%20\%282\%29.pdf?mkt_tok=eyJpIjoiTkRreE5XWXlORFF3TXpNeCIsInQiOiJIbVNQTTVySEo0Vzk1cHVBZVVqWnFGVmR1UEJxRGdpd01mTXg4OGw3Mk5nTnpmaUoyMGt2UXIwWVZBOE5GVjIybHA5aStrbzJ3MUxsanoxamZibmlocmpSbXZyVFVoV0VHYU1aTGx0RnpsMXlmOEtXSVJqaDJsZ0RJU1BQcVZjZSJ9}{no
han visto un aumento} en los casos de coronavirus.

Noruega y Dinamarca son buenos ejemplos. Ambos países abrieron sus
escuelas en abril, aproximadamente un mes después de cerrar, pero
inicialmente solo para los niños más pequeños, y dejaron cerradas las
escuelas secundarias hasta más tarde. Fortalecieron los procedimientos
de desinfección y establecieron clases de tamaño reducido, grupos
pequeños de niños en el recreo y mayor espacio entre los escritorios.
Ninguno de esos países ha visto un aumento significativo en los casos.

Todavía no se han realizado estudios científicos rigurosos sobre el
potencial de propagación de las escuelas, pero un puñado de informes de
casos, la mayoría de ellos aún sin revisión por pares, refuerzan la idea
de que un alto riesgo no es inevitable.

Image

Los estudiantes de una escuela primaria en Bangkok regresaron el 1 de
julio, un comienzo retrasado de su año académico.Credit...Adam Dean para
The New York Times

\href{https://www.eurosurveillance.org/content/10.2807/1560-7917.ES.2020.25.21.2000903\#html_fulltext}{Un
estudio en Irlanda ofrece una instantánea del panorama} con seis
personas infectadas (dos estudiantes de secundaria, un estudiante de
primaria y tres adultos) que acudieron un tiempo a las escuelas antes de
su cierre en marzo. Los investigadores analizaron a 1155 contactos de
esos seis pacientes para ver si alguno confirmaba una infección por
coronavirus. Los contactos incluyeron a quienes habían participado en
actividades escolares que podrían ser un terreno fértil para la
transmisión, como lecciones de música con instrumentos de viento de
madera, ensayos del coro y deportes. Ninguno de los estudiantes parecía
haber infectado a otras personas, informaron los autores, y agregaron
que la única transmisión documentada del virus fue a dos adultos que no
pertenecían a la escuela y estaban en contacto con uno de los adultos
infectados.

Pero ha habido brotes en las escuelas en países con niveles más altos de
infección comunitaria y países que aparentemente flexibilizaron las
pautas de seguridad demasiado pronto. En Israel, el virus infectó a más
de 200 estudiantes y personal escolar después de que a inicios de mayo
reabrieron las escuelas y, pocas semanas después, se eliminaron los
límites en el tamaño de las clases, según un
\href{https://globalhealth.washington.edu/sites/default/files/COVID-19\%20Schools\%20Summary\%20\%282\%29.pdf?mkt_tok=eyJpIjoiTkRreE5XWXlORFF3TXpNeCIsInQiOiJIbVNQTTVySEo0Vzk1cHVBZVVqWnFGVmR1UEJxRGdpd01mTXg4OGw3Mk5nTnpmaUoyMGt2UXIwWVZBOE5GVjIybHA5aStrbzJ3MUxsanoxamZibmlocmpSbXZyVFVoV0VHYU1aTGx0RnpsMXlmOEtXSVJqaDJsZ0RJU1BQcVZjZSJ9}{informe
de investigadores de la Universidad de Washington}.

Los estudios de caso en algunos países sugieren que hay diferencias en
la transmisión del virus en niños más pequeños en comparación con los
niños mayores.

En una comunidad en el norte de Francia, Crépy-en-Valois, dos maestros
de secundaria se enfermaron con la COVID-19 a inicios de febrero, antes
del cierre de las escuelas. Científicos del Instituto Pasteur evaluaron
luego a los estudiantes y el personal de la escuela en busca de
anticuerpos contra el coronavirus. Encontraron anticuerpos en el 38 por
ciento de los estudiantes, 43 por ciento de los profesores y 59 por
ciento del resto del personal escolar, dijo Arnaud Fontanet,
epidemiólogo que dirigió
\href{https://www.medrxiv.org/content/10.1101/2020.04.18.20071134v1}{el
estudio} y es miembro de un comité que asesora al gobierno francés.

``Claramente sabes que el virus circulaba en la escuela secundaria'',
dijo Fontanet.

Más tarde, el equipo evaluó a estudiantes y personal de seis
\href{https://www.medrxiv.org/content/10.1101/2020.06.25.20140178v2}{escuelas
primarias} en la comunidad. El cierre de las escuelas a mediados de
febrero brindó la oportunidad de ver si los niños más pequeños se habían
infectado cuando las escuelas estaban abiertas, el momento en el que el
virus alcanzó a los estudiantes de secundaria.

Los investigadores encontraron anticuerpos en solo el nueve por ciento
de los estudiantes de primaria, el siete por ciento de los profesores y
el cuatro por ciento del resto del personal. Identificaron a tres
estudiantes en tres escuelas primarias diferentes que habían asistido a
clases con síntomas agudos de coronavirus antes de que cerraran las
escuelas. Ninguno parecía haber infectado a otros niños, maestros o
personal, dijo Fontanet. Dos de esos estudiantes sintomáticos tenían
hermanos en la escuela secundaria y el tercero tenía una hermana que
trabajaba en la escuela secundaria, dijo.

La investigación también indicó que cuando un estudiante de primaria dio
positivo por anticuerpos contra el coronavirus, había una probabilidad
muy alta de que los padres del estudiante también hubieran sido
infectados, dijo Fontanet. La probabilidad no era tan alta para los
padres de estudiantes de secundaria. ``Al mirar el calendario, creemos
que comenzó en la escuela secundaria, se trasladó a las familias y luego
a los estudiantes más jóvenes'', dijo.

Fontanet dijo que los hallazgos sugieren que los niños mayores pueden
transmitir el virus más fácilmente que los niños más pequeños.

Ese patrón también puede reflejarse en la experiencia de Israel, donde
uno de los brotes escolares más grandes, que involucró a unos 175
estudiantes y personal, ocurrió en Gymnasia Rehavia, una escuela
primaria y secundaria en Jerusalén.

Existen diferentes teorías sobre por qué los niños mayores tendrían más
probabilidades de transmitir el virus que los niños más pequeños.
Algunos científicos dicen que los niños más pequeños tienen menos
probabilidades de tener síntomas de la COVID-19 como tos y es menos
probable que tengan voces fuertes; ambas cosas pueden transmitir el
virus por gotículas. Otros investigadores examinan si las proteínas que
permiten al virus entrar y replicarse en las células pulmonares son
menos abundantes en los niños, lo que limitaría la gravedad de su
infección y, potencialmente, su capacidad de transmitir el virus.

\hypertarget{quuxe9-pueden-hacer-las-escuelas}{%
\subsection{Qué pueden hacer las
escuelas}\label{quuxe9-pueden-hacer-las-escuelas}}

Las pruebas para detectar infecciones son esenciales, dijeron expertos
en salud pública. En Estados Unidos los Centros para el Control y la
Prevención de Enfermedades (CDC, por su sigla en inglés) recomiendan
evaluar a los estudiantes o maestros solo en función de los síntomas o
antecedentes de exposición. Pero eso no detectará a todos los
infectados.

``Sabemos que la propagación asintomática o presintomática es real, y
sabemos que es menos probable ---en comparación con los adultos--- que
los niños muestren síntomas si están infectados'', dijo Megan Ranney,
emergencióloga y experta en salud adolescente en la Universidad de
Brown. Las escuelas deberían evaluar al azar a estudiantes y maestros,
dijo, pero eso puede ser imposible dada la falta de fondos y pruebas
limitadas, incluso en hospitales.

Los países que han reabierto las escuelas han implementado una serie de
pautas de seguridad.

Image

~Los estudiantes regresaron a la escuela en Thun, Suiza, el 11 de mayo,
después de que un confinamiento por coronavirus los mantuvo en
casa.Credit...Peter Schneider/EPA, vía Shutterstock

Algunos países inicialmente solo permitieron volver a las aulas a una
parte de sus estudiantes: a los niños más pequeños en Dinamarca,
Noruega, Bélgica, Suiza y Grecia; a los chicos mayores en Alemania,
según el
\href{https://globalhealth.washington.edu/sites/default/files/COVID-19\%20Schools\%20Summary\%20\%282\%29.pdf?mkt_tok=eyJpIjoiTkRreE5XWXlORFF3TXpNeCIsInQiOiJIbVNQTTVySEo0Vzk1cHVBZVVqWnFGVmR1UEJxRGdpd01mTXg4OGw3Mk5nTnpmaUoyMGt2UXIwWVZBOE5GVjIybHA5aStrbzJ3MUxsanoxamZibmlocmpSbXZyVFVoV0VHYU1aTGx0RnpsMXlmOEtXSVJqaDJsZ0RJU1BQcVZjZSJ9}{informe
de los investigadores de la Universidad de Washington}. Bélgica llevó a
los estudiantes por turnos, en días alternos.

Varios países limitaron el aforo de las clases, lo que a menudo permite
un máximo de 10 a 15 estudiantes en un aula. Muchos colocaron
escritorios a varios metros de distancia. Varios países agrupan a los
niños en grupos o cápsulas con interacción social restringida en gran
medida a esos mismos grupos, especialmente durante el recreo y la hora
de almuerzo.

Las políticas de uso de cubrebocas varía. En Asia, donde la práctica de
usar cubrebocas durante la temporada de gripe es común, muchos países
requieren cubrebocas en las escuelas. En otros lugares, algunos pedían
cubrebocas solo para algunos estudiantes o personal, como los profesores
en Bélgica y los estudiantes de secundaria en Francia, según el reporte
de la Universidad de Washington.

En Alemania, los estudiantes que den negativo para el virus no tienen
que usar cubrebocas, de acuerdo con el informe, que dice que desde la
apertura de las escuelas, Alemania ha visto una mayor transmisión del
virus entre los estudiantes, pero no entre el personal de la escuela.

Los CDC han esbozado los pasos que las escuelas pueden tomar para
minimizar el riesgo para los estudiantes, entr los que se encuentran
mantener una distancia de dos metros, lavarse las manos y usar
cubrebocas.

``Las pautas ya son excepcionalmente débiles'', dijo Carl Bergstrom,
experto en enfermedades infecciosas en la Universidad de Washington en
Seattle. Él y otros dijeron que temían que las recomendaciones se
diluyeran aún más en respuesta a la presión política.

Los CDC han trabajado en nuevas recomendaciones para reabrir las
escuelas durante varias semanas, en consulta con organizaciones como la
Asociación Nacional de Enfermeras Escolares, según una portavoz de los
CDC. Los cinco documentos que se planifican incluyen orientación sobre
la detección de síntomas y los cubrebocas, y una lista para los padres o
tutores que intentan decidir si enviarán o no a sus niños a la escuela.
Pero no incluyen ninguna información sobre cómo mejorar la ventilación o
reducir la propagación del virus en el aire.

Las escuelas deberán asegurarse de que el aire fresco circule, ya sea
filtrando el aire, bombeando desde el exterior o simplemente abriendo
las ventanas, dijo Saskia Popescu, epidemióloga de la Universidad de
Arizona. Los enfermeros escolares, como Cogan, también necesitarán
equipo de protección, como guantes, trajes y mascarillas N95.

Hay diferencias en cómo otros países responden cuando se identifican
casos de coronavirus en las escuelas, con algunos, como Israel, que
cierran escuelas enteras por un simple caso y otros que adoptan el
enfoque más específico de enviar a los estudiantes y maestros de una
clase afectada a hacer cuarentena en su casa durante dos semanas.

Kathryn Edwards, especialista en enfermedades infecciosas y profesora de
pediatría en la Escuela de Medicina de la Universidad de Vanderbilt,
asesora a las escuelas de Nashville sobre enfoques para reabrir. Dijo
que el distrito aún evalúa qué tan separados deberían estar los
escritorios. ``Algunas personas dicen que solo necesitas medio metro y
otras dicen que necesitas dos metros, y otros se preguntan si con el
problema del aerosol, ¿necesitamos más distancia?''.

Edwards dijo que estaba decepcionada con la decisión de Nashville,
anunciada el jueves 23, de impartir clases
\href{https://www.tennessean.com/story/news/education/2020/07/09/metro-schools-academic-year-start-online-nashville-students/5383315002/}{en
línea durante el primer mes de clases}, al menos hasta el primer lunes
de septiembre.

Mantener las escuelas cerradas por un período prolongado tiene
implicaciones preocupantes para el desarrollo social y académico, dicen
los expertos en desarrollo infantil. También se hizo evidente esta
primavera que negarles a los niños un día escolar real profundizó las
desigualdades raciales y económicas.

``Los chicos realmente se ven afectados si no van a la escuela'', dijo
Edwards. ``Creo que tenemos que pensar en los niños y llevarlos de
regreso a la escuela de forma segura''.

Pam Belluck es una reportera de ciencia y salud cuyos galardones
incluyen compartir un Premio Pulitzer 2015 y ganar el premio Nellie Bly
a la mejor historia de primera plana. Es autora de \emph{Island
Practice}, un libro sobre un doctor peculiar.
\href{https://twitter.com/PamBelluck}{@PamBelluck}

Apoorva Mandavilli es reportera del Times y se enfoca en ciencia y salud
global. En 2019 ganó el Premio Victor Cohn a la excelencia en reportaje
sobre ciencias médicas.
\href{https://twitter.com/apoorva_nyc}{@apoorva\_nyc}

Benedict Carey ha sido reportero científico en el Times desde 2004.
También ha escrito tres libros: \emph{Aprender a aprender} sobre la
ciencia cognitiva del aprendizaje y en inglés de \emph{Poison Most Vial}
e \emph{Island of the Unknowns}, sobre misterios científicos para
estudiantes de secundaria.

\begin{center}\rule{0.5\linewidth}{\linethickness}\end{center}

Advertisement

\protect\hyperlink{after-bottom}{Continue reading the main story}

\hypertarget{site-index}{%
\subsection{Site Index}\label{site-index}}

\hypertarget{site-information-navigation}{%
\subsection{Site Information
Navigation}\label{site-information-navigation}}

\begin{itemize}
\tightlist
\item
  \href{https://help.nytimes.com/hc/en-us/articles/115014792127-Copyright-notice}{©~2020~The
  New York Times Company}
\end{itemize}

\begin{itemize}
\tightlist
\item
  \href{https://www.nytco.com/}{NYTCo}
\item
  \href{https://help.nytimes.com/hc/en-us/articles/115015385887-Contact-Us}{Contact
  Us}
\item
  \href{https://www.nytco.com/careers/}{Work with us}
\item
  \href{https://nytmediakit.com/}{Advertise}
\item
  \href{http://www.tbrandstudio.com/}{T Brand Studio}
\item
  \href{https://www.nytimes.com/privacy/cookie-policy\#how-do-i-manage-trackers}{Your
  Ad Choices}
\item
  \href{https://www.nytimes.com/privacy}{Privacy}
\item
  \href{https://help.nytimes.com/hc/en-us/articles/115014893428-Terms-of-service}{Terms
  of Service}
\item
  \href{https://help.nytimes.com/hc/en-us/articles/115014893968-Terms-of-sale}{Terms
  of Sale}
\item
  \href{https://spiderbites.nytimes.com}{Site Map}
\item
  \href{https://help.nytimes.com/hc/en-us}{Help}
\item
  \href{https://www.nytimes.com/subscription?campaignId=37WXW}{Subscriptions}
\end{itemize}
