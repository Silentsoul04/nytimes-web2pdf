Sections

SEARCH

\protect\hyperlink{site-content}{Skip to
content}\protect\hyperlink{site-index}{Skip to site index}

\href{https://www.nytimes.com/es/section/opinion}{Opinión}

\href{https://myaccount.nytimes.com/auth/login?response_type=cookie\&client_id=vi}{}

\href{https://www.nytimes.com/section/todayspaper}{Today's Paper}

\href{/es/section/opinion}{Opinión}\textbar{}Las enseñanzas de educar
durante la pandemia

\url{https://nyti.ms/300hXVt}

\begin{itemize}
\item
\item
\item
\item
\item
\item
\end{itemize}

Advertisement

\protect\hyperlink{after-top}{Continue reading the main story}

\href{/es/section/opinion}{Opinión}

Supported by

\protect\hyperlink{after-sponsor}{Continue reading the main story}

Comentario

\hypertarget{las-enseuxf1anzas-de-educar-durante-la-pandemia}{%
\section{Las enseñanzas de educar durante la
pandemia}\label{las-enseuxf1anzas-de-educar-durante-la-pandemia}}

He sido profesor universitario durante más de dos décadas, y estos meses
de enseñar durante la pandemia me han hecho entender que la educación ha
cambiado para siempre.

\includegraphics{https://static01.nyt.com/images/2020/07/24/opinion/00herrcher/00herrcher-articleLarge.jpg?quality=75\&auto=webp\&disable=upscale}

Por Roberto Herrscher

Es periodista y profesor.

\begin{itemize}
\item
  27 de julio de 2020
\item
  \begin{itemize}
  \item
  \item
  \item
  \item
  \item
  \item
  \end{itemize}
\end{itemize}

SANTIAGO --- Durante más de dos décadas he sido profesor universitario.
Empecé con pizarrón y tiza, pasé a las transparencias y a una tele
abombada para pasar las películas en VHS, y de ahí al Power Point y a
los videos de YouTube en las aulas de la Universidad Alberto Hurtado, en
el centro de la capital chilena, donde hoy trabajo. Pero en este
semestre desaparecieron las aulas y debimos reinventarnos.

Como casi todos los oficios, el de la educación ha cambiado con la
llegada del coronavirus. Ahora, los profesores y los estudiantes
llevamos cuatro meses de dar y tomar clases y exámenes a distancia. ¿Qué
pasará cuando esto termine?

Hay gerentes de universidades, sobre todo privadas, que preferirían que,
tras la pandemia, sigamos así, sin gastar en aulas, lugares comunes,
limpieza, seguridad, luz y calefacción. En el otro costado, hay
profesores de la vieja guardia que anhelan volver a la clase magistral y
el pizarrón de antes, como si nada hubiera pasado.

Yo pienso que ambas opciones son malas. En esta total transformación de
la educación superior perdimos y ganamos, y es hora de aprender de esto
para educar y aprender mejor.

\hypertarget{quuxe9-perdimos}{%
\subsection{¿Qué perdimos?}\label{quuxe9-perdimos}}

Algo vital: el lugar donde estábamos juntos, donde los alumnos,
cualquiera que sea su origen de clase o su barrio, compartían un mismo
espacio.

Como
\href{https://www.nytimes.com/es/2020/05/08/espanol/opinion/zoom-escuela-clases.html}{escribe}
la profesora norteamericana Karen Strassler, los espacios universitarios
no garantizaron nunca la igualdad entre estudiantes de distintas
procedencias y clases sociales, pero sí crearon un ambiente propicio, un
``paraíso de aprendizaje'' en el que verse y medirse compartiendo las
mismas herramientas y espacios ---escapando de las propias
circunstancias--- les permite a estudiantes de disímil procedencia
imaginarse en otro sitio y pensar en la transformación del sitio del
cual vienen.

Según un
\href{https://www.iesalc.unesco.org/en/2019/09/25/unesco-iesalc-reveals-that-only-38-of-mobility-from-latin-america-and-the-caribbean-is-to-the-same-region/}{informe
de 2019 de la UNESCO}, en los últimos cinco años en Latinoamérica, el
número de alumnos en educación superior creció un 16 por ciento, hasta
llegar a los 27,4 millones de ese año. Esto hizo aumentar mucho el
porcentaje de alumnos que son primera generación en la educación
superior.

En la universidad en la que enseño, el
\href{https://www.uahurtado.cl/wp-images/uploads/2019/08/Memoria-UAH-2018_-por-pagina.pdf}{72
por ciento} de los alumnos son los primeros universitarios de sus
familias. Las salas de estudio, las bibliotecas, los equipos y softwares
y los laboratorios hacen que todos tengan similares posibilidades. Antes
de la pandemia, en las amplias mesas de madera de la biblioteca reinaba
un silencio de concentración propicio para el estudio.

Y esto hace que el confinamiento en la casa sea ahora especialmente duro
y perjudicial. Según expertos
\href{https://blogs.worldbank.org/es/education/educational-challenges-and-opportunities-covid-19-pandemic}{en
educación} y
\href{https://www.uchile.cl/noticias/162982/academicos-uch-exponen-los-problemas-de-la-educacion-remota-en-el-pais}{psicología},
las enormes dificultades de los sectores más vulnerables para estudiar
fuera del aula producen angustia, problemas psicológicos, retrasos y
deterioro en el rendimiento en todos los niveles educativos.

Al estar ahora obligados a quedarse en casa, unos tienen silencio y
otros, ruido constante; unas comparten habitación y escritorio con dos
hermanas pequeñas mientras otras tienen el lujo de un cuarto propio;
algunas tienen mejor conexión a internet que otras, hay muchos que deben
compartir la computadora de la casa con sus padres que hacen
teletrabajo. O hay incluso quienes han tenido que dejar la universidad
para trabajar y ayudar a sus familias.

Por eso tenemos que volver a las aulas como sitios donde, aunque no se
consigan borrar las diferencias sociales, el terreno está algo más
nivelado y las condiciones son mejores para los que menos tienen.

\hypertarget{quuxe9-ganamos}{%
\subsection{¿Qué ganamos?}\label{quuxe9-ganamos}}

A mediados de mayo, en una clase por Zoom con 38 alumnos de primer año
de Periodismo sobre qué debe tener un buen reportero, mis alumnos
empezaron a escribir en el chat los nombres de los periodistas que
admiran. Periodistas como Bob Woodward y Carl Bernstein o
\href{https://www.uc.cl/universidad/premios-nacionales/raquel-correa-prats/}{Raquel
Correa}, una profesional valiente durante la dictadura en Chile.

Y de pronto en el chat alguien escribió el nombre de Juan Carlos
Bodoque, un conejo del programa infantil chileno \emph{31 minutos}. Le
pedí que abriera su micrófono y se explicara: de niño quiso ser
periodista por este reportero de largas orejas rojas que se preguntaba,
entre otras cosas, por qué los pueblos se quedan sin agua. Mi alumno
tenía razón: este personaje humorístico cubría temas que muchos de los
medios ``serios'' latinoamericanos de entonces no abordaban, como el
medioambiente.

Pero hay algo más. Sé que difícilmente un estudiante presentaría a este
muñeco de trapo como su modelo en una clase presencial: debería superar
el temor al desdén del profesor o a las burlas de los compañeros, o a
ambos.

Terminé la clase pensando en que una de las cosas buenas que nos dejará
esta pandemia es que la lejanía puede acercarnos a nuestros alumnos. En
mi experiencia de estos meses, los alumnos nativos digitales se sienten
con ánimo y coraje de hablar más libremente.

Muchos colegas vivieron experiencias similares: Federico Navarro, quien
enseña en la Universidad de O'Higgins de Chile, me contó que sus alumnos
innovaron y profundizaron haciendo etnografías familiares, aprovechando
el encierro para mirar hacia adentro y pensar en su entorno.

Y también, este momento tan duro, ha llevado a los profesores a entender
mejor las condiciones en que sus alumnos menos favorecidos deben
estudiar.

En el tiempo en que escribo este texto he recibido tres correos de mis
estudiantes explicando las dificultades para hacer el examen que debían
entregar la semana pasada. Muchos de sus relatos me duelen, pero también
me ayudan a conocerlos mejor.

\hypertarget{cuxf3mo-debemos-volver-a-las-aulas-a-partir-de-lo-que-ganamos-y-perdimos}{%
\subsection{¿Cómo debemos volver a las aulas a partir de lo que ganamos
y
perdimos?}\label{cuxf3mo-debemos-volver-a-las-aulas-a-partir-de-lo-que-ganamos-y-perdimos}}

No todo es malo: debemos conservar lo que hemos ganado trabajando en el
entorno digital, en la que esta generación de veinteañeros se siente
como peces en el agua, aunque no todos sienten que su comodidad en redes
se traspase a sus labores educativas. Hay que seguir compartiendo
contenidos en internet, las herramientas que aprendimos a usar en
cuarentena.

Eso postula Diego Mardones, profesor de la Universidad de Chile. Él y
otros académicos latinoamericanos están creando formas de transmisión de
conocimiento y evaluaciones de aprendizaje a distancia, que en su
criterio cambiarán para siempre la forma de dar y recibir clases.

En un reciente encuentro por Zoom, Mardones me mostró una de sus clases
de Introducción a la física, donde los alumnos se adentran en los
secretos de la ciencia como en un simulador de vuelo. Siguiendo su clase
como si fuera su alumno, siento que ahora sí puedo entender la física
del siglo XXI. A mi ritmo, deteniéndome en lo que no comprendo.

Y junto con la ganancia digital, tenemos que revalorar lo presencial.
Usar mejor las clases, los ejercicios grupales, los aspectos
intransferibles de encontrarnos en un mismo lugar para conocernos mejor
y aprender juntos mirándonos a las caras.

Roberto Herrscher es periodista argentino y director de la carrera de
Periodismo de la Universidad Alberto Hurtado de Chile. Su libro más
reciente es \emph{Periodismo narrativo}.

Advertisement

\protect\hyperlink{after-bottom}{Continue reading the main story}

\hypertarget{site-index}{%
\subsection{Site Index}\label{site-index}}

\hypertarget{site-information-navigation}{%
\subsection{Site Information
Navigation}\label{site-information-navigation}}

\begin{itemize}
\tightlist
\item
  \href{https://help.nytimes.com/hc/en-us/articles/115014792127-Copyright-notice}{©~2020~The
  New York Times Company}
\end{itemize}

\begin{itemize}
\tightlist
\item
  \href{https://www.nytco.com/}{NYTCo}
\item
  \href{https://help.nytimes.com/hc/en-us/articles/115015385887-Contact-Us}{Contact
  Us}
\item
  \href{https://www.nytco.com/careers/}{Work with us}
\item
  \href{https://nytmediakit.com/}{Advertise}
\item
  \href{http://www.tbrandstudio.com/}{T Brand Studio}
\item
  \href{https://www.nytimes.com/privacy/cookie-policy\#how-do-i-manage-trackers}{Your
  Ad Choices}
\item
  \href{https://www.nytimes.com/privacy}{Privacy}
\item
  \href{https://help.nytimes.com/hc/en-us/articles/115014893428-Terms-of-service}{Terms
  of Service}
\item
  \href{https://help.nytimes.com/hc/en-us/articles/115014893968-Terms-of-sale}{Terms
  of Sale}
\item
  \href{https://spiderbites.nytimes.com}{Site Map}
\item
  \href{https://help.nytimes.com/hc/en-us}{Help}
\item
  \href{https://www.nytimes.com/subscription?campaignId=37WXW}{Subscriptions}
\end{itemize}
