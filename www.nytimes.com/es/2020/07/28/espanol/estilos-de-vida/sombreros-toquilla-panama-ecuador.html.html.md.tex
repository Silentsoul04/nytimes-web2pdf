Sections

SEARCH

\protect\hyperlink{site-content}{Skip to
content}\protect\hyperlink{site-index}{Skip to site index}

\href{/es/section/estilos-de-vida}{Estilos de Vida}\textbar{}Una visita
a los talleres de los mejores fabricantes de sombreros en Ecuador

\url{https://nyti.ms/30XjpHz}

\begin{itemize}
\item
\item
\item
\item
\item
\item
\end{itemize}

\includegraphics{https://static01.nyt.com/images/2020/07/21/travel/28sombreros-toquilla-ES-00/20travel-panama-promo-articleLarge.jpg?quality=75\&auto=webp\&disable=upscale}

EL MUNDO A TRAVÉS DE UNA LENTE

\hypertarget{una-visita-a-los-talleres-de-los-mejores-fabricantes-de-sombreros-en-ecuador}{%
\section{Una visita a los talleres de los mejores fabricantes de
sombreros en
Ecuador}\label{una-visita-a-los-talleres-de-los-mejores-fabricantes-de-sombreros-en-ecuador}}

Cremoso como la seda y más costoso que su peso en oro, un sombrero de
Panamá Montecristi superfino no solo es un accesorio de moda, también es
una obra de arte.

Gabriel Lucas plancha un sombrero en su taller en Montecristi,
Ecuador.Credit...Roff Smith

Supported by

\protect\hyperlink{after-sponsor}{Continue reading the main story}

Por Roff Smith

\begin{itemize}
\item
  28 de julio de 2020
\item
  \begin{itemize}
  \item
  \item
  \item
  \item
  \item
  \item
  \end{itemize}
\end{itemize}

\href{https://www.nytimes.com/2020/07/20/travel/panama-hats-ecuador.html}{Read
in English}

\href{https://www.nytimes.com/newsletters/el-times}{Regístrate para
recibir nuestro boletín} con lo mejor de The New York Times.

\begin{center}\rule{0.5\linewidth}{\linethickness}\end{center}

\emph{Mientras duran las restricciones de viaje, hemos lanzado una nueva
serie,}
\href{https://www.nytimes.com/column/the-world-through-a-lens}{\emph{El
mundo a través de una lente}}\emph{, en la cual fotoperiodistas te
transportan, virtualmente, a algunos de los lugares más hermosos e
intrigantes del planeta. En esta entrega escribe Roff Smith, quien
comparte una colección de fotografías de los talleres de artesanos del
sombrero en Ecuador.}

\begin{center}\rule{0.5\linewidth}{\linethickness}\end{center}

Cremoso como la seda, más costoso en peso que el oro, del color del fino
marfil viejo, un sombrero panamá Montecristi superfino es tanto una obra
de arte como un accesorio de moda. Los mejores ejemplares tienen más de
4000 fibras en seis centímetros cuadrados, un tejido tan fino que se
necesita una lupa de joyero para contar las filas. Y cada uno de estos
tejidos se hace a mano. No se utiliza telar: solo dedos diestros, ojos
afilados y concentración zen.

``No puedes permitir que tu mente divague ni siquiera por un segundo'',
dice Simón Espinal, un hombre modesto y de voz suave que es considerado
por sus pares como el mejor tejedor vivo de sombreros panamá,
posiblemente el más grande de la historia. ``Cuando estás tejiendo, solo
eres tú y la paja''.

Image

Simón Espinal examina el tejido en uno de sus sombreros, su obra
maestra.

Image

Espinal sostiene una de las pajitas más delgadas con las que tejerá un
sombrero que vale más que su peso en oro.

\includegraphics{https://static01.nyt.com/images/2020/07/20/travel/28sombreros-toquilla-ES-03/merlin_174060129_b7033e25-7181-4ce9-9219-8377acfa3bbc-articleLarge.jpg?quality=75\&auto=webp\&disable=upscale}

Los sombreros de Espinal tienen un promedio de alrededor de 465 fibras
por centímetro cuadrado, una finura a la que pocos tejedores se han
acercado. Su mejor tejido tiene 651 por centímetro cuadrado y le tomó
cinco meses de elaboración.

Image

Gabriel Lucas reemplaza una pajita en un sombrero panamá en su taller de
Montecristi.

El ecuatoriano de 52 años es uno de los pocos tejedores de los sombreros
panamá de élite que aún quedan; casi todos ellos viven en Pile, una
oscura aldea escondida en las estribaciones detrás de Montecristi, una
ciudad a poca altura a más de 160 kilómetros arriba de la costa desde
Guayaquil.

Image

Un sombrero superfino en proceso de ser tejido.

Image

~El taller de Gabriel Lucas, uno de los grandes artesanos de acabados en
Montecristi.

Me interesé por los sombreros hace unos 15 años, por casualidad, cuando
leí sobre unos sombreros de paja que podían costar miles de dólares.
Intrigado, comencé a investigar los sombreros, hice un viaje a Ecuador
---donde se tejen todos los sombreros panamá legítimos--- y descubrí
este mundo curioso y ligeramente anacrónico de los tejedores de
sombreros de Montecristi.

Image

Patricia López muestra los comienzos de un sombrero panamá.

Aunque el tejedor es la estrella del espectáculo, la fabricación de un
Montecristi es un arte colaborativo. Después de que el tejedor o la
tejedora ha terminado su parte, el cuerpo del sombrero crudo pasa a
través de las manos de un equipo de artesanos especializados cuyos
títulos ---el rematador, el cortador, el apaleador y el planchador ---
le prestan algo de la apasionada formalidad de una plaza de toros a la
fabricación de un panamá Montecristi. (El término rematador se deriva
directamente de las corridas de toros: allí, es el finalizador, uno
``que realiza algún acto que proporcionará un clímax emocional o
artístico'', como lo describe Hemingway en \emph{Muerte en la tarde}).

Image

~La paja colgada para el secado. Para prepararla para el tejido, la paja
se hierve ligeramente durante aproximadamente un minuto y luego se deja
secar durante la noche al aire libre.

En Montecristi, el rematador es el tejedor especializado que realiza el
complicado entretejido para sellar el borde, lo que lleva a un cierre
artístico en la fase de tejido de la creación del sombrero. Después de
eso, el exceso de paja es recortado por el cortador, quien entonces le
pasa al sombrero una cuchilla de afeitar muy al ras para eliminar
cualquier rebaba en la paja.

``A veces, cuando estoy cortando, me encuentro con una pajita que se ha
decolorado o no se ha tejido correctamente'', dice Gabriel Lucas, uno de
los mejores artesanos de acabados de Montecristi, mientras realiza una
delicada operación con un sombrero fino que valdrá miles cuando esté
terminado. ``Los llamamos hijos perdidos, las pajitas que faltan. Tengo
que cortarlas con cuidado y tejer con una nueva pajita para
reemplazarla''.

Image

Uno de los trabajos de los artesanos de acabado es inspeccionar el
sombrero en busca de pajitas mal tejidas o descoloridas. Si se
encuentran, se cortan y se reemplazan.

Image

El cortador recorta el exceso de paja del cuerpo de un sombrero recién
tejido, luego le da el mejor afeitado con una cuchilla para recortar
cualquier parte espinosa. Aquí, el artesano Gabriel Lucas, de 34 años,
realiza la tarea en su taller en Montecristi.

Una vez que ha sido debidamente barberado, el sombrero es golpeado con
un mazo de madera dura por el apaleador para ayudar a acomodar las
fibras, luego el planchador lo desarruga rápidamente para darle la
cantidad adecuada de rigidez en preparación para la etapa final del
bloqueo, o el esculpido a mano del sombrero sin forma hasta que adquiere
alguno de sus estilos reconocibles: fedora, óptimo, plantación.

Image

Gabriel Lucas plancha firmemente un sombrero para ayudar a la paja a
mantener su estructura.

Los sombreros panamá son exclusivamente ecuatorianos, a pesar de su
curiosa denominación errónea. El término ``sombrero de Panamá'' ha
estado en uso desde al menos la década de 1830, y surgió porque los
sombreros a menudo se vendían en puestos comerciales del istmo de
Panamá, que ya era un centro de transporte mucho antes de que se
construyera el canal. El nombre fue popularizado durante la fiebre del
oro de California, cuando decenas de miles de buscadores pasaron por
Panamá en su camino hacia la aventura, y muchos de ellos adquirían un
sombrero en el camino.

Image

Los sombreros panamá se tejen a partir de las fibras de la paja
toquilla~ Carludovica palmata.

Image

Los brotes inmaduros de la palma son descortezados, y las fibras planas
parecidas a un fettuccini se dividen una y otra vez para hacer fina la
paja requerida para un hermoso sombrero.

Este tipo de sombrero se afirmó mucho más en la imaginación popular
después de la Exposición de París en 1855, cuando un francés que había
estado viviendo en Panamá le regaló a Napoleón III un sombrero finamente
tejido. Su Alteza amó el sombrero y lo usaba en todas partes.

Entonces, como ahora, las celebridades marcaron la pauta en asuntos de
moda, y nadie era más célebre que el emperador de Francia. Los sombreros
panamá finos y sedosos para la primavera y el verano se volvieron de
rigor entre los ricos y famosos. Se dice que el rey Eduardo VII ordenó a
su sombrerero que no escatimara gastos para conseguirle el mejor
ejemplar disponible. Él y otros pagaron sumas fabulosas por los mejores
sombreros. Un
\href{https://www.newyorker.com/magazine/1930/07/05/thousand-dollar-hats}{artículo
de The Talk of The Town} en The New Yorker de julio de 1930 describe un
panamá de mil dólares ---unos 16.000 dólares de hoy--- en exhibición en
la tienda de sombreros Dobbs en la ciudad. Se mencionaba a
\href{https://www.pbs.org/wnet/broadway/stars/florenz-ziegfeld/}{Florenz
Ziegfeld} como un posible comprador.

Image

La parte superior de un panamá se llama la plantilla, esta fue tejida
por Espinal

En estos días, la abrumadora mayoría de los llamados sombreros de Panamá
se tejen en Cuenca, una atractiva ciudad de los Andes cuyos residentes,
motivados por el gobierno local, comenzaron a tejer sombreros a mediados
de 1800, una vez que los panamá se hicieron populares. Estos son los que
se encuentran en grandes almacenes y en la mayoría de las sombrererías.
Son bonitos, se confeccionan con un tejido ligero y sencillo llamado
``brisa'', que se puede producir rápidamente y en cantidades
comerciales.

Montecristi, por otro lado, es la sede del arte. Los lugareños han
tejido sombreros finos con las fibras de la
\href{https://timesmachine.nytimes.com/timesmachine/1900/09/02/101066082.html?pageNumber=24}{paja
toquilla} durante siglos. Aquí, la fabricación de sombreros se ha
mantenido como una industria artesanal, los tejedores se reúnen y
preparan sus propias pajas como lo han hecho durante generaciones,
tejiendo sus sombreros en su patrón artístico y liso, un estilo muy
bonito en espiga.

Image

Espinal se enfoca en mantener rectas las innumerables hebras mientras
teje otra de sus obras maestras.

Su producción es obligadamente pequeña, y la de los tejedores de élite
de Pile es aún más pequeña. En un buen año, Simón Espinal podría hacer
tres sombreros.

Últimamente, el gobierno ha instado a los tejedores de Pile a ser más
comerciales, a abandonar las viejas costumbres, a no tejer sombreros tan
finos, pero ellos se han negado. ``Esto'', dice Simón Espinal, ``es un
regalo de Dios''.

\href{http://www.roffsmithphotography.com/}{\emph{Roff Smith}} \emph{es
un escritor y fotógrafo con sede en Inglaterra. Puedes seguir sus
aventuras en Instagram:}
\href{https://www.instagram.com/roffsmith/}{\emph{@roffsmith}}\emph{.}

Advertisement

\protect\hyperlink{after-bottom}{Continue reading the main story}

\hypertarget{site-index}{%
\subsection{Site Index}\label{site-index}}

\hypertarget{site-information-navigation}{%
\subsection{Site Information
Navigation}\label{site-information-navigation}}

\begin{itemize}
\tightlist
\item
  \href{https://help.nytimes.com/hc/en-us/articles/115014792127-Copyright-notice}{©~2020~The
  New York Times Company}
\end{itemize}

\begin{itemize}
\tightlist
\item
  \href{https://www.nytco.com/}{NYTCo}
\item
  \href{https://help.nytimes.com/hc/en-us/articles/115015385887-Contact-Us}{Contact
  Us}
\item
  \href{https://www.nytco.com/careers/}{Work with us}
\item
  \href{https://nytmediakit.com/}{Advertise}
\item
  \href{http://www.tbrandstudio.com/}{T Brand Studio}
\item
  \href{https://www.nytimes.com/privacy/cookie-policy\#how-do-i-manage-trackers}{Your
  Ad Choices}
\item
  \href{https://www.nytimes.com/privacy}{Privacy}
\item
  \href{https://help.nytimes.com/hc/en-us/articles/115014893428-Terms-of-service}{Terms
  of Service}
\item
  \href{https://help.nytimes.com/hc/en-us/articles/115014893968-Terms-of-sale}{Terms
  of Sale}
\item
  \href{https://spiderbites.nytimes.com}{Site Map}
\item
  \href{https://help.nytimes.com/hc/en-us}{Help}
\item
  \href{https://www.nytimes.com/subscription?campaignId=37WXW}{Subscriptions}
\end{itemize}
