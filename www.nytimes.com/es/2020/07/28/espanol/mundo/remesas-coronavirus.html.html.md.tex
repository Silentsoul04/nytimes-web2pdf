Sections

SEARCH

\protect\hyperlink{site-content}{Skip to
content}\protect\hyperlink{site-index}{Skip to site index}

\href{https://www.nytimes.com/es/section/mundo}{Mundo}

\href{https://myaccount.nytimes.com/auth/login?response_type=cookie\&client_id=vi}{}

\href{https://www.nytimes.com/section/todayspaper}{Today's Paper}

\href{/es/section/mundo}{Mundo}\textbar{}Migraron para sacar a sus
familias de la pobreza. Ahora necesitan ayuda

\url{https://nyti.ms/3g8m7Qx}

\begin{itemize}
\item
\item
\item
\item
\item
\item
\end{itemize}

\href{https://www.nytimes.com/es/spotlight/coronavirus?action=click\&pgtype=Article\&state=default\&region=TOP_BANNER\&context=storylines_menu}{El
brote de coronavirus}

\begin{itemize}
\tightlist
\item
  \href{https://www.nytimes.com/es/interactive/2020/espanol/mundo/coronavirus-en-estados-unidos.html?action=click\&pgtype=Article\&state=default\&region=TOP_BANNER\&context=storylines_menu}{Mapa
  y casos en EE. UU.}
\item
  \href{https://www.nytimes.com/es/2020/07/23/espanol/america-latina/bolivia-cloro-coronavirus-ivermectina.html?action=click\&pgtype=Article\&state=default\&region=TOP_BANNER\&context=storylines_menu}{Dióxido
  de cloro, ivermectina y más: ¿funcionan?}
\item
  \href{https://www.nytimes.com/es/interactive/2020/science/coronavirus-tratamientos-curas.html?action=click\&pgtype=Article\&state=default\&region=TOP_BANNER\&context=storylines_menu}{Fármacos
  y tratamientos}
\item
  \href{https://www.nytimes.com/es/2020/07/28/espanol/ciencia-y-tecnologia/anticuerpos-coronavirus-inmunidad.html?action=click\&pgtype=Article\&state=default\&region=TOP_BANNER\&context=storylines_menu}{Anticuerpos
  e inmunidad}
\item
  \href{https://www.nytimes.com/es/2020/04/29/espanol/estilos-de-vida/oximetro-para-que-sirve.html?action=click\&pgtype=Article\&state=default\&region=TOP_BANNER\&context=storylines_menu}{Oxímetros}
\end{itemize}

Advertisement

\protect\hyperlink{after-top}{Continue reading the main story}

Supported by

\protect\hyperlink{after-sponsor}{Continue reading the main story}

\hypertarget{migraron-para-sacar-a-sus-familias-de-la-pobreza-ahora-necesitan-ayuda}{%
\section{Migraron para sacar a sus familias de la pobreza. Ahora
necesitan
ayuda}\label{migraron-para-sacar-a-sus-familias-de-la-pobreza-ahora-necesitan-ayuda}}

La pandemia ha deteriorado los salarios de los trabajadores inmigrantes,
quienes envían menos remesas a sus hogares. Esto podría generar un
aumento de la pobreza en varias partes del mundo, de América Latina al
sur de Asia.

\includegraphics{https://static01.nyt.com/images/2020/07/24/business/28Remesas-ES-1/00Remittances-1-articleLarge.jpg?quality=75\&auto=webp\&disable=upscale}

\href{https://www.nytimes.com/by/peter-s-goodman}{\includegraphics{https://static01.nyt.com/images/2018/02/16/multimedia/author-peter-s-goodman/author-peter-s-goodman-thumbLarge-v2.png}}

Por \href{https://www.nytimes.com/by/peter-s-goodman}{Peter S. Goodman}

\begin{itemize}
\item
  28 de julio de 2020
\item
  \begin{itemize}
  \item
  \item
  \item
  \item
  \item
  \item
  \end{itemize}
\end{itemize}

\href{https://www.nytimes.com/2020/07/27/business/global-remittances-coronavirus.html}{Read
in English}

\href{https://www.nytimes.com/newsletters/el-times}{Regístrate para
recibir nuestro boletín} con lo mejor de The New York Times.

\begin{center}\rule{0.5\linewidth}{\linethickness}\end{center}

LONDRES --- Durante más de una década, Flavius Tudor ha compartido con
su madre, quien vive en Rumania, el dinero que ha ganado en el Reino
Unido y le ha enviado de manera periódica la suma que le permitía
comprar sus medicamentos.

El mes pasado, el flujo se revirtió. Su madre de 82 años le envió dinero
para que pudiera pagar sus cuentas.

Cuando le subió la fiebre y le dio una tos persistente en medio de la
pandemia de coronavirus, Tudor, de 52 años, ya no pudo ingresar al asilo
de ancianos donde trabajaba como cuidador. Así que su madre usó su
pensión, obtenida gracias a toda una vida como bibliotecaria en uno de
los países más pobres de Europa, y le envió dinero a su hijo que radica
en uno de los países más ricos del mundo.

``Son épocas muy difíciles'', comentó. ``Estoy perdido''.

En todo el mundo, la pandemia ha puesto en riesgo una arteria vital de
financiamiento que sustenta a cientos de millones de familias: las
remesas que los trabajadores migrantes que laboran en países ricos
envían a sus países de origen. Como el coronavirus paralizó las
economías y ha originado desempleo, las personas que estaban
acostumbradas a cuidar a sus familiares que permanecen en sus países han
perdido sus ingresos, lo que las ha obligado a depender de quienes han
dependido de ellos.

Según el
\href{https://www.bancomundial.org/es/news/press-release/2020/04/22/world-bank-predicts-sharpest-decline-of-remittances-in-recent-history}{Banco
Mundial}, el año pasado, los trabajadores migrantes enviaron a sus
países una suma histórica de 554.000 millones de dólares, más del triple
que la cantidad de ayuda para el desarrollo otorgada por los países
ricos. Pero es probable que esas remesas se reduzcan a una quinta parte
este año, lo que representa la contracción más importante de la
historia.

La caída augura una catástrofe y aumenta la posibilidad de que la
pandemia genere el primer incremento de la pobreza a nivel global desde
la crisis financiera de Asia en 1998. Se cree que, aproximadamente, de
40 a 60 millones de personas caerán en
\href{https://www.bancomundial.org/es/topic/poverty/overview}{pobreza
extrema} este año, lo que el Banco Mundial define como vivir con 1,90
dólares o menos al día.

La disminución de las remesas es tanto un resultado de la crisis que
abruma al mundo como un presagio de otros problemas que se avecinan. Con
base en el poder adquisitivo, los países en desarrollo representan el 60
por ciento de la economía mundial, según el Fondo Monetario
Internacional. Un menor gasto en los países más pobres se traduce en
menos crecimiento económico para el mundo.

Al igual que la pandemia que la ha provocado, la disminución de las
remesas es global. Se espera que Europa y Asia Central sufran una caída
de casi el 28 por ciento en las retribuciones que se envían desde otros
países, mientras que en África subsahariana se contempla una caída del
23 por ciento. Parece que Asia del sur se prepara para un descenso del
22 por ciento, mientras que Medio Oriente, el norte de África,
Latinoamérica y el Caribe podrían asumir una reducción de más del 19 por
ciento.

En general, de acuerdo con un cálculo de la
\href{https://migrationnetwork.un.org/sites/default/files/policy_brief-_remittances_in_the_time_of_covid-19.pdf}{Red
de Naciones Unidas sobre la Migración}, la pandemia ha deteriorado el
potencial de ingreso de 164 millones de trabajadores migrantes que
apoyan al menos a 800 millones de sus familiares que viven en países
menos ricos.

``Estamos hablando de un número extraordinario de personas que se
benefician de las remesas'', señaló Dilip Ratha, economista principal
sobre migración y remesas del Banco Mundial en Washington.

Aventurarse a trabajar en el extranjero está vinculado con el peligro,
ya que los trabajadores migrantes están expuestos a agentes de
reclutamiento deshonestos, a empleadores que los explotan y a los
peligros físicos del trabajo físico. También es una manera
particularmente eficaz de aspirar a un ascenso social.

\includegraphics{https://static01.nyt.com/images/2020/07/24/business/28Remesas-ES-2/merlin_152944797_91edcdcc-e7ca-429d-ae78-d7c9b8f20b5b-articleLarge.jpg?quality=75\&auto=webp\&disable=upscale}

Las familias que reciben remesas se alimentan mejor y tienen más
probabilidades de que sus hijos estudien en vez de verse presionados a
sumarse a la fuerza laboral. Los bebés que nacen dentro de familias que
reciben remesas tienden a tener un mayor peso al nacer.

En algunos países, los trabajadores migrantes pueden aprovechar el
seguro de desempleo y otros programas gubernamentales, especialmente los
europeos orientales de las naciones de la Unión Europea que han
trabajado en otros Estados miembro. Pero en muchos países los migrantes
operan en áreas grises, sin protección del gobierno y especialmente
vulnerables en tiempos difíciles.

``Algunas personas, ingenuamente o con buenas intenciones, dicen que la
COVID-19 nos democratiza, y que todos estamos expuestos por igual'',
dijo Mahmoud Mohieldin, un economista egipcio que se desempeña como
enviado especial de las Naciones Unidas para el financiamiento del
desarrollo sostenible. ``Esto no es verdad. Los impactos son muy
desproporcionados''.

Para las familias en los países pobres, enviar a un pariente al
extranjero para ganar dinero tiende a ser una empresa colectiva. Las
personas juntan su efectivo para financiar viajes en lo que equivale a
la mayor inversión que harán en sus vidas.

La pandemia ha convertido tales emprendimientos en desastres.

Mahammed Heron salió hace tres años de su pueblo en las afueras de Daca,
Bangladés, para trabajar en Catar, el país rico en recursos energéticos,
siguiendo la ruta emprendida por decenas de millones de migrantes de
Asia del sur.

Les pidió prestados 400.000 takas bangladesíes (cerca de 4700 dólares) a
sus familiares y se puso en contacto con un agente de reclutamiento
local que le compró un boleto de avión, le garantizó una visa de trabajo
y le prometió un empleo. Era una cantidad descomunal de dinero en
Bangladés, más del doble del ingreso nacional per cápita
(aproximadamente 1855 dólares). Su esposa, Monowara Begum, estaba
aterrada. Su primer esposo ---el hermano mayor de Heron--- había muerto
a manos de un chofer ebrio hacía más de una década en Arabia Saudita,
donde estuvo trabajando como conserje de un hospital.

Image

Monowara Begum vive en una cabaña en una aldea a las afueras de Daca,
Bangladés, que está construida con aluminio y es vulnerable a las
lluvias torrenciales del monzón. No hay agua potable.Credit...Salahuddin
Ahmed para The New York Times

Pero si era atemorizante la expectativa de que su esposo se arriesgara
en el golfo Pérsico, parecía todavía más arriesgado que se quedara.

Su familia vivía en una choza hecha de aluminio corrugado vulnerable a
las lluvias torrenciales del monzón. No tenían agua potable. Heron
ganaba unos 300 takas (cerca de 3,50 dólares) al día por su trabajo en
los arrozales de los alrededores. Casi nunca podían darse el lujo de
comer carne o pescado y subsistían con arroz y papas. Su hijo mayor
tenía una afección cardiaca, por lo que tenía que tomar medicamentos.

La única manera de salir de la pobreza era invertir en la educación de
sus hijos, pero las colegiaturas llegaban a 6000 tacas (más de 70
dólares) por un año.

``Nuestra situación económica nunca fue buena'', explicó Begum en una
entrevista por video, mientras los pájaros trinaban ruidosamente en la
aldea. A su pesar, aceptó ese plan.

Image

Begum dijo que la posibilidad de que su esposo se aventurara al golfo
Pérsico era aterradora, pero quedarse ahí parecía aún más riesgoso.
``Nuestra situación económica nunca fue buena''.Credit...Salahuddin
Ahmed para The New York Times

En septiembre de 2018, cuando Heron aterrizó en Doha, no solo recibió el
impacto del calor abrasador, sino la noticia de que la agencia de
reclutamiento no le había conseguido empleo. ``Me engañaron'', contó en
una entrevista por video.

Buscó trabajo con desesperación y finalmente consiguió un puesto en una
agencia de empleos que le asignó una diversidad de tareas: limpiar
oficinas, hacer jardinería y cavar en el suelo arenoso para instalar
cables de fibra óptica.

A Heron le pagaban un salario mensual de 900 riales cataríes (unos 250
dólares) y le asignaron una litera dentro de una habitación en un
dormitorio que compartía con otros 15 hombres, todos ellos de Bangladés.

Cada dos o tres meses, enviaba a casa unos 30.000 takas (cerca de 350
dólares), pero todo eso era para pagar su deuda\ldots{} y solo había
pagado una cuarta parte.

Posteriormente, en mayo, cuando se paralizó gran parte de la vida en
Doha por el coronavirus, la agencia dejó de pagarles a los trabajadores,
dijo Heron. Tuvo un ataque de asma que requirió hospitalización, lo que
le consumió todos sus recursos y dejó de mandar dinero a casa.

Según \href{https://www.bb.org.bd/econdata/wageremitance.php}{el banco
central del país}, para Bangladés, en general, las remesas que se
recibieron de otros países cayeron un 23 por ciento en abril, en
comparación con el año anterior, y en mayo disminuyeron en 13 por
ciento, aunque en junio hubo un incremento.

Las escuelas siguen cerradas en Bangladés, pero Begum no ve
posibilidades de enviar a su hijo de 16 años, Hasan, cuando vuelvan a
abrir.

Image

Begum posa con su hija e hijos. Las escuelas permanecen cerradas en
Bangladés, pero cada vez que abren, Begum no ve la manera de pagar para
enviar a Hasan, su hijo mayor.Credit...Salahuddin Ahmed para The New
York Times

Begum ha estado presionando a Hasan para que busque trabajo, tal vez en
la construcción, o quizás en un taller mecánico. Él se ha rehusado y
prefiere quedarse en casa a leer libros de texto.

``Quiero seguir con mis estudios'', afirmó. Imagina su vida como
ingeniero de software. El rostro se le ilumina ---es un adolescente
delgado, parado sin camisa frente a su choza mientras los gallos
cacarean--- cuando cuenta que se imagina en una oficina reluciente
inclinado frente a una computadora.

Cada tantos días, Hasan y su madre usan una aplicación de celular y una
tarjeta de prepago de internet para hablar con Heron, quien se encuentra
varado en el dormitorio de Catar. Está demasiado enfermo para trabajar,
dijo, pero no tiene dinero para tomar un avión de regreso a casa.
Después de otro año, la agencia de empleos está obligada
contractualmente a pagarle el vuelo de regreso a casa. Desea una nueva
oportunidad, con la esperanza de recuperar su salud, con la esperanza de
que vuelvan a pagarle y lograr que sus propios hijos se libren de su
destino.

``Sueño con que mis hijos hagan algo con sus vidas'', dijo.

Image

Algunos trabajadores inmigrantes juegan críquet en un terreno baldío en
Doha el año pasado. Mahammed Heron, quien llegó a ese país en 2018,
ahora está muy enfermo como para trabajar, dijo, pero le falta dinero
para regresar a Bangladés.Credit...Petr David Josek/Associated Press

En el pueblo de Patzún, Guatemala, la familia de Édgar Tzirin usaba el
dinero que él ganaba trabajando como cocinero en un restaurante de sopa
y sándwiches de Nueva York para construir una casa nueva. Tzirin ganaba
unos 2000 dólares al mes. Cada dos semanas, enviaba sin falta a su
familia de 500 a 700 dólares.

Este dinero resultó ser indispensable en el momento en que la pandemia
dejó sin empleo a sus tres hermanas. Cuando tuvieron que hospitalizar a
su madre ---tal vez con la COVID--- él se hizo cargo de los gastos.

Pero en abril, con la suspensión de actividades en Nueva York, Tzirin
perdió su empleo. Cuando murió su abuelo al mes siguiente, no pudo
enviarles dinero para su funeral y eso le causó un profundo dolor. Solía
hablar con su familia cada dos o tres días, pero ya no puede soportarlo
y se ha retraído en su aislamiento y soledad. No les ha contado que
perdió el empleo.

``Mi familia me necesita'', afirmó.

Tzirin se levanta a las 5:30 todos los días y sale a buscar trabajo en
la construcción o empleos temporales como jornalero, pero casi siempre
regresa a casa con las manos vacías. ``No encuentro nada'', señaló.

Ya debe tres meses de renta. Tiene pensado regresar a Guatemala por
primera vez en una década, pero ¿qué puede hacer ahí?

``Es una experiencia muy dura'', comentó Tzirin. ``La gente se
desespera''.

Muchos trabajadores migrantes se enfrentan a dos emergencias al mismo
tiempo: la pérdida de sus ingresos y la amenaza del virus.

Tudor, el inmigrante rumano que vive en el Reino Unido, salió de su
región de origen en Transilvania cuando tenía veintitantos años. Después
de dejar una vida peligrosa trabajando en una mina de carbón, primero
llegó a España, donde trabajó en el área de seguridad. Cuando la crisis
financiera global hundió al país en una verdadera depresión en 2009, se
fue al Reino Unido y se estableció en Weston-super-Mare, un pueblo
costero de 76.000 habitantes a unos 250 kilómetros al oeste de Londres.

Las agencias de empleos lo asignaban a centros de cuidado de gente de
edad avanzada por temporadas. Su empleo más reciente fue en un asilo de
ancianos con fines lucrativos llamado The Heathers. Ganaba 848 libras
(aproximadamente 1070 dólares) a la semana. Su esposa trabajaba en el
aseo de las habitaciones de un hotel y ganaba 1200 libras (unos 1536
dólares) al mes.

Cuando llegó el coronavirus, le redujeron las horas a su esposa. Los
hospitales empezaron a trasladar a los asilos a los pacientes mayores
que tenían el virus.

Image

La familia de Edgar Tzirin usó el dinero que ganó en su trabajo como
cocinero en Nueva York para construir una casa en Guatemala. Pero en
abril, perdió su trabajo y ahora tiene tres meses de retraso en el
alquiler.Credit...Justin French para The New York Times

Según Tudor, a principios de marzo, 23 de las 30 habitaciones en The
Heathers estaban llenas de pacientes con coronavirus. En una semana,
nueve estaban muertos, dijo. Él y sus colegas solo recibieron cubrebocas
quirúrgicos desechables. Un colega exigió más equipo de protección y fue
despedido. Dijo que durante su última semana en las instalaciones, el
administrador colocó a una mujer con demencia que no tenía el virus en
la misma habitación de alguien que sí lo tenía.

``Fue horrible'', dijo Tudor. ``Solo se trata de negocios. Se trata de
dinero''.

Contactado por teléfono, un copropietario de The Heathers, Bipin Patel,
se negó a responder preguntas. ``No estamos haciendo ningún
comentario'', dijo.

Tudor pronto empezó con tos y fiebre, lo que lo obligó a dejar de
asistir a trabajar. Dio negativo para el coronavirus dos veces, pero no
ha podido encontrar otro empleo.

En años recientes, el Reino Unido
\href{https://www.nytimes.com/2018/05/28/world/europe/uk-austerity-poverty.html}{ha
reducido de manera considerable los programas gubernamentales de apoyo}
para los desempleados y para quienes tienen dificultades al momento de
pagar sus cuentas y los ha incorporado a un esquema de pagos conocido
como crédito universal.

Tudor ha cambiado su sueldo por un pago de crédito universal de 1000
libras (1280 dólares) al mes, lo que ha reducido su ingreso
prácticamente a la mitad. Se le han roto sus anteojos, pero no puede
costear la compra de otros. Cuando se venció la renta del mes pasado,
pudo pagarla gracias a la ayuda de su madre que vive en Rumania.

``El mundo no sabe hacia dónde va'', comentó. ``Ninguna sociedad puede
manejar esta situación''.

Nic Wirtz colaboró con informes desde Antigua, Guatemala; Hari Kumar,
desde Nueva Delhi, y Ginebra Abdul colaboró con investigaciones desde
Londres.

Peter S. Goodman es corresponsal de economía europea con sede en
Londres. Antes fue corresponsal económico de Estados Unidos en Nueva
York. También trabajó en The Washington Post como corresponsal en China
y fue editor global jefe del International Business Times.
\href{https://twitter.com/petersgoodman}{@petersgoodman}

Advertisement

\protect\hyperlink{after-bottom}{Continue reading the main story}

\hypertarget{site-index}{%
\subsection{Site Index}\label{site-index}}

\hypertarget{site-information-navigation}{%
\subsection{Site Information
Navigation}\label{site-information-navigation}}

\begin{itemize}
\tightlist
\item
  \href{https://help.nytimes.com/hc/en-us/articles/115014792127-Copyright-notice}{©~2020~The
  New York Times Company}
\end{itemize}

\begin{itemize}
\tightlist
\item
  \href{https://www.nytco.com/}{NYTCo}
\item
  \href{https://help.nytimes.com/hc/en-us/articles/115015385887-Contact-Us}{Contact
  Us}
\item
  \href{https://www.nytco.com/careers/}{Work with us}
\item
  \href{https://nytmediakit.com/}{Advertise}
\item
  \href{http://www.tbrandstudio.com/}{T Brand Studio}
\item
  \href{https://www.nytimes.com/privacy/cookie-policy\#how-do-i-manage-trackers}{Your
  Ad Choices}
\item
  \href{https://www.nytimes.com/privacy}{Privacy}
\item
  \href{https://help.nytimes.com/hc/en-us/articles/115014893428-Terms-of-service}{Terms
  of Service}
\item
  \href{https://help.nytimes.com/hc/en-us/articles/115014893968-Terms-of-sale}{Terms
  of Sale}
\item
  \href{https://spiderbites.nytimes.com}{Site Map}
\item
  \href{https://help.nytimes.com/hc/en-us}{Help}
\item
  \href{https://www.nytimes.com/subscription?campaignId=37WXW}{Subscriptions}
\end{itemize}
