Sections

SEARCH

\protect\hyperlink{site-content}{Skip to
content}\protect\hyperlink{site-index}{Skip to site index}

\href{https://www.nytimes.com/es/section/ciencia-y-tecnologia}{Ciencia y
Tecnología}

\href{https://myaccount.nytimes.com/auth/login?response_type=cookie\&client_id=vi}{}

\href{https://www.nytimes.com/section/todayspaper}{Today's Paper}

\href{/es/section/ciencia-y-tecnologia}{Ciencia y
Tecnología}\textbar{}Traer rocas de Marte a la Tierra: nuestra maniobra
interplanetaria más asombrosa

\url{https://nyti.ms/3jNIs8l}

\begin{itemize}
\item
\item
\item
\item
\item
\end{itemize}

Advertisement

\protect\hyperlink{after-top}{Continue reading the main story}

Supported by

\protect\hyperlink{after-sponsor}{Continue reading the main story}

\hypertarget{traer-rocas-de-marte-a-la-tierra-nuestra-maniobra-interplanetaria-muxe1s-asombrosa}{%
\section{Traer rocas de Marte a la Tierra: nuestra maniobra
interplanetaria más
asombrosa}\label{traer-rocas-de-marte-a-la-tierra-nuestra-maniobra-interplanetaria-muxe1s-asombrosa}}

La NASA y la Agencia Espacial Europea planean lanzar rocas de una nave
espacial a otra antes de que las muestras finalmente lleguen a nuestro
planeta en 2031.

\includegraphics{https://static01.nyt.com/images/2020/07/28/science/28MARSSAMPLE/28MARSSAMPLE-articleLarge.jpg?quality=75\&auto=webp\&disable=upscale}

\href{https://www.nytimes.com/by/kenneth-chang}{\includegraphics{https://static01.nyt.com/images/2018/02/16/multimedia/author-kenneth-chang/author-kenneth-chang-thumbLarge.jpg}}

Por \href{https://www.nytimes.com/by/kenneth-chang}{Kenneth Chang}

\begin{itemize}
\item
  29 de julio de 2020
\item
  \begin{itemize}
  \item
  \item
  \item
  \item
  \item
  \end{itemize}
\end{itemize}

\href{https://www.nytimes.com/2020/07/28/science/mars-sample-return-mission.html}{Read
in English}

\href{https://www.nytimes.com/newsletters/el-times}{Regístrate para
recibir nuestro boletín} con lo mejor de The New York Times.

\begin{center}\rule{0.5\linewidth}{\linethickness}\end{center}

Enviar una nave espacial robótica a Marte, hacer que tome algunas rocas
y polvo y que los traiga a la Tierra.

¿Qué tan difícil podría ser?

Se parece más a un acto circense interplanetario de lo que podrías
imaginar, pero la NASA y la Agencia Espacial Europea creen que ahora es
el momento de por fin llevar a cabo la coreografía compleja de echar las
rocas de una nave espacial hacia otra antes de que las muestras
aterricen de manera definitiva en la Tierra, en 2031.

``Es evidente que la comunidad científica ha deseado hacer esto desde
hace bastante tiempo'', dijo James Watzin, director del programa de
exploración de Marte de la NASA.

En las últimas décadas, los exploradores robóticos han revelado una
imagen cada vez más compleja de Marte, pero los científicos planetarios
están limitados por la cantidad de ciencia que se puede empaquetar en
una nave espacial.

``No puedes llevar tanta instrumentación al campo, robóticamente'', dijo
Watzin. ``Para entrar realmente en algunas de las preguntas
verdaderamente intrigantes a nivel de detalle significa que necesitamos
analizar la evidencia a nivel molecular e intentar extraer la
información de un material muy muy antiguo. Y eso requiere un conjunto
completo de instrumentación que era claramente demasiado grande para
encogerse y enviarse a otro planeta''.

Con rocas frescas de Marte en la Tierra, más científicos podrán
examinarlas mientras emplean una amplia gama de los equipos más
sofisticados en laboratorios de todo el mundo.

El primer paso de este proyecto épico, conocido como Regreso de Muestras
de Marte (MSR, por su sigla en inglés), comienza pronto con el
Perseverance, el próximo vehículo explorador de la NASA. El Perseverance
tiene programado despegar el 30 de julio con dirección a Jezero, un
cráter que fue un lago hace unos 3500 millones de años y es un lugar
prometedor donde se podrían conservar signos de vida pasada en Marte.

Una de las tareas clave del Perseverance es perforar hasta 39 núcleos
rocosos, cada uno de 1,3 centímetros de ancho y 6 centímetros de largo,
que luzcan lo suficientemente interesantes como para ameritar un
escrutinio adicional en la Tierra. Cada muestra de roca y polvo, con un
peso de alrededor de 14 gramos, estará sellada en un tubo de metal
ultralimpio del tamaño de un puro.

Sin embargo, en un inicio, la NASA no tenía planeado traer esos tubos a
la Tierra. El Perseverance no cuenta con ningún mecanismo para lanzar
las rocas fuera de Marte.

\hypertarget{perseverance}{%
\subsection{Perseverance}\label{perseverance}}

La misión de la NASA incluye al Perseverance, un vehículo explorador de
997 kilos, e Ingenuity, un helicóptero experimental para Marte.

Helicóptero Ingenuity

Esta aeronave de unos dos kilos se comunicará de forma inalámbrica con
el vehículo explorador Perseverance.

Panel solar

Aspas

Cuatro aspas de fibra de carbono girarán a aproximadamente 2400
revoluciones por minuto.

Energía

La fuente de alimentación a base de plutonio cargará las baterías del
vehículo explorador.

Mástil

Los instrumentos tomarán videos, panoramas y fotografías. Un láser
estudiará la química de las rocas marcianas.

PiXl

Identificará elementos químicos para buscar signos de vidas pasadas en
Marte.

Antena

Transmitirá los datos directamente a la Tierra.

Brazo robótico

Una torreta con muchos instrumentos está unida a un brazo robótico de
dos metros. Un taladro extraerá muestras de rocas marcianas. El
dispositivo Sherloc identificará moléculas y minerales para detectar
posibles biofirmas, con la ayuda de la cámara Watson.

Vehículo explorador Perseverance

El vehículo explorador de 997 kilos explorará el cráter Jezero. Tiene
ruedas de aluminio y un sistema de suspensión para conducir sobre
obstáculos.

Ingenuity Helicopter

The aircraft will communicate wirelessly with the rover.

Solar Panel

Blades

Power

The plutonium-based power supply will charge the rover's batteries.

MAST

Instruments will take videos, panoramas and photographs. A laser will
study the chemistry of Martian rocks.

PiXl

Will identify chemical elements to seek signs of past life on Mars.

Antenna

Robotic arm

A turret with many instruments is attached to a 7-foot robotic arm. A
drill will extract samples from Martian rocks. The Sherloc device will
identify molecules and minerals to detect potential biosignatures, with
help from the Watson camera.

Perseverance Rover

The 2,200 pound rover will explore Jezero Crater. It has aluminum wheels
and a suspension system to drive over obstacles.

Solar panel

Ingenuity Helicopter

Blades

Power

Mast

PIXL

Antenna

Suspension

Perseverance rover

Robotic arm

A turret with many instruments is attached to a 7-foot robotic arm. A
drill will extract samples from Martian rocks. The Sherloc device will
identify molecules and minerals to detect potential biosignatures, with
help from the Watson camera. PiXl will identify chemical elements to
seek signs of past life on Mars.

Por Eleanor Lutz \textbar{} Fuente: NASA

Hace tres años, un equipo de ingenieros del Laboratorio de Propulsión a
Chorro de la NASA en California, comenzó a analizar más de cerca cuándo
podría empezar la parte del ``regreso'' del programa de Regreso de
Muestras de Marte. Los ingenieros consideraron la posibilidad de lanzar
la nave espacial de recuperación en 2026 y que las muestras regresasen
tres años más tarde.

Se dieron cuenta de que su cálculo del tiempo era demasiado ambicioso.

No obstante, si el aterrizaje en la Tierra se postergaba hasta 2031, el
calendario parecía factible. ``En verdad pensamos que podríamos
hacerlo'', dijo Watzin.

La solicitud de presupuesto del gobierno de Trump para la NASA para el
año fiscal 2021 incluyó 233 millones de dólares para continuar el
desarrollo, dos años después de que la agencia recibiera 50 millones
para los estudios iniciales. El mes pasado, los 22 países miembro de la
Agencia Espacial Europea dieron el visto bueno a la colaboración con la
NASA.

El equipo científico del Perseverance ya ha comenzado un análisis
geológico preliminar sobre lo que debería traerse a la Tierra.

``Nos centramos cada vez más en cómo hacerlo bien'', dijo Kenneth
Farley, el científico del proyecto para el Perseverance. ``Hemos hecho
la transición de un `sí, algún día estas muestras serán recogidas' a
`sí, podrían ser recogidas muy pronto'. Ha sido una evolución
importante''.

Los funcionarios de la agencia espacial aún no han anunciado el costo
total, pero se espera que sean varios miles de millones de dólares.

``Intentamos mantener la operación dentro de cierto presupuesto'', dijo
Brian K. Muirhead, quien dirige el diseño del regreso de las muestras en
el Laboratorio de Propulsión a Chorro. ``Realmente estamos planteando
los estimados y decimos: `Creemos que esto costará'. Y, hasta el
momento, la NASA ha dicho: `Bien, sigan adelante'''.

Si todo resulta según lo planeado, dos naves espaciales despegarán hacia
Marte en 2026. Una de ellas será un aterrizador fabricado por la NASA,
el vehículo más pesado que se haya llevado a la superficie de Marte.
Esta nave transportará un vehículo explorador, construido por los
europeos, para buscar las muestras rocosas, y un pequeño cohete que
lanzará las rocas en órbita alrededor de Marte.

El aterrizador realizará una trayectoria indirecta a Marte, para llegar
en agosto de 2028, al comienzo de la primavera marciana. Luego, el
vehículo explorador impulsado por energía solar saldrá del aterrizador,
hará una carrera para recolectar al menos algunas muestras rocosas y las
llevará de regreso al aterrizador. A su vez, mecanismos robotizados
moverán las muestras hasta la parte más alta del vehículo de ascenso a
Marte, el cohete que lanzará las rocas fuera del planeta rojo.

La segunda nave especial, el Orbitador de Retorno a la Tierra, será un
producto de la Agencia Espacial Europea. Tomará una ruta más veloz a
Marte, para ponerse en órbita antes de la llegada del aterrizador. Esto
permitirá que el orbitador sirva de retransmisor para las comunicaciones
del aterrizador conforme se acerque a la superficie.

El lanzamiento del vehículo de ascenso colocará un contenedor, más o
menos del tamaño de un balón de fútbol, de modo que las muestras rocosas
estén dando vueltas alrededor de Marte a unos 321 kilómetros de la
superficie. Luego el orbitador debe encontrar este contenedor, como un
jardinero de béisbol que busca un batazo. El orbitador no posee ningún
propulsor ni radiobaliza. Sin embargo, es blanco, y esto debería
facilitar su detección en contraste con la oscuridad del espacio.

``Evidentemente, este es uno de los problemas clave: ¿cómo
encontrarlo?'', dijo Muirhead. ``En cuanto se conoce la órbita donde se
encuentra, es muy fácil llegar a ella''.

Se abrirá una puerta en el orbitador para capturar el contenedor. Luego,
un aparato de 450 kilogramos dentro del orbitador rota y desliza el
contenedor hacia la configuración adecuada al interior de la nave
espacial, con cuidado de aislar la posibilidad de que algo de Marte
pueda contaminar cualquier parte del exterior del contenedor que tiene
las muestras.

Después, el orbitador se iría de Marte. A medida que se aproxime a la
Tierra, expulsará las muestras, que en ese momento estarán dentro del
llamado vehículo de entrada a la Tierra, en una trayectoria de colisión
hacia el desierto de Utah.

Los paracaídas fueron otra complicación innecesaria para los ingenieros,
así que el vehículo de entrada, el cual parece un sombrero grande,
tocará el suelo a una velocidad que podría compararse con la de un
accidente de auto en carretera: 144 kilómetros por hora.

El cargamento científico ---las rocas y el polvo, que no son frágiles---
sobrevivirán el impacto sin problemas.

Aún hay muchos detalles sin determinar, como el lugar donde llegará el
aterrizador. Si el Perseverance sigue en buenas condiciones, podría
enviarse a un segundo sitio fuera de Jezero, donde quizá alguna vez hubo
manantiales geotérmicos, otro entorno donde se pudo haber desarrollado
vida.

Sin embargo, estas decisiones no se tienen que tomar sino hasta dentro
de años, y las mejores respuestas podrían no revelarse sino hasta que el
Perseverance le eche un buen vistazo a Jezero.

Si se rompe una pieza, la misión para regresar las muestras no
necesariamente fallará. El Perseverance probablemente dejará caer
algunos de los tubos de muestra en el piso en caso de sufrir un mal
funcionamiento más adelante en la misión. Si el vehículo explorador se
rompe, entonces el Perseverance podría llevar muestras al módulo de
aterrizaje.

Incluso si el orbitador falla, su contenedor del tamaño de una pelota de
fútbol con muestras podría permanecer dando vueltas alrededor de Marte
durante años, hasta que se pueda enviar otra nave espacial para
atraparlo.

``Ese ha sido mi trabajo como arquitecto'', dijo Muirhead. ``Pensar en
el proceso desde el concepto de operaciones, desarrollar los conceptos
que pueden lograr los objetivos de las diferentes fases y asegurarse de
que haya buenos márgenes incorporados en todas partes. Para que el
diseño no sea frágil''.

Kenneth Chang ha estado en el Times desde 2000 y ha escrito sobre
física, geología, química y los planetas. Antes de escribir sobre
ciencias, fue un estudiante de posgrado cuya investigación involucraba
el control del caos. \href{https://twitter.com/kchangnyt}{@kchangnyt}

Advertisement

\protect\hyperlink{after-bottom}{Continue reading the main story}

\hypertarget{site-index}{%
\subsection{Site Index}\label{site-index}}

\hypertarget{site-information-navigation}{%
\subsection{Site Information
Navigation}\label{site-information-navigation}}

\begin{itemize}
\tightlist
\item
  \href{https://help.nytimes.com/hc/en-us/articles/115014792127-Copyright-notice}{©~2020~The
  New York Times Company}
\end{itemize}

\begin{itemize}
\tightlist
\item
  \href{https://www.nytco.com/}{NYTCo}
\item
  \href{https://help.nytimes.com/hc/en-us/articles/115015385887-Contact-Us}{Contact
  Us}
\item
  \href{https://www.nytco.com/careers/}{Work with us}
\item
  \href{https://nytmediakit.com/}{Advertise}
\item
  \href{http://www.tbrandstudio.com/}{T Brand Studio}
\item
  \href{https://www.nytimes.com/privacy/cookie-policy\#how-do-i-manage-trackers}{Your
  Ad Choices}
\item
  \href{https://www.nytimes.com/privacy}{Privacy}
\item
  \href{https://help.nytimes.com/hc/en-us/articles/115014893428-Terms-of-service}{Terms
  of Service}
\item
  \href{https://help.nytimes.com/hc/en-us/articles/115014893968-Terms-of-sale}{Terms
  of Sale}
\item
  \href{https://spiderbites.nytimes.com}{Site Map}
\item
  \href{https://help.nytimes.com/hc/en-us}{Help}
\item
  \href{https://www.nytimes.com/subscription?campaignId=37WXW}{Subscriptions}
\end{itemize}
