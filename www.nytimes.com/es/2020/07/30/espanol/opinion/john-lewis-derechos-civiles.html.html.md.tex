Sections

SEARCH

\protect\hyperlink{site-content}{Skip to
content}\protect\hyperlink{site-index}{Skip to site index}

\href{/es/section/opinion}{Opinión}\textbar{}John Lewis: Juntos, ustedes
pueden recuperar el alma de Estados Unidos

\url{https://nyti.ms/3ffIN00}

\begin{itemize}
\item
\item
\item
\item
\item
\item
\end{itemize}

\includegraphics{https://static01.nyt.com/newsgraphics/2020/07/14/op-header/c998714d96ba195218174d25716062b22597e147/h_14047247.jpg}
\includegraphics{https://static01.nyt.com/newsgraphics/2020/07/14/op-header/c998714d96ba195218174d25716062b22597e147/h_14047247.jpg}

 Opinión John Lewis

\hypertarget{juntos-ustedes}{%
\section{Juntos, ustedes}\label{juntos-ustedes}}

pueden recuperar el alma\\
de Estados Unidos

\hypertarget{juntos-ustedes-1}{%
\section{Juntos, ustedes}\label{juntos-ustedes-1}}

pueden recuperar el alma\\
de Estados Unidos

\hypertarget{aunque-me-haya-ido-los-animo-a}{%
\subsection{Aunque me haya ido, los animo
a}\label{aunque-me-haya-ido-los-animo-a}}

responder al llamado más elevado de su corazón y a defender lo que
realmente creen.

\hypertarget{aunque-me-haya-ido-los-animo-a-1}{%
\subsection{Aunque me haya ido, los animo
a}\label{aunque-me-haya-ido-los-animo-a-1}}

responder al llamado más elevado\\
de su corazón y a defender\\
lo que realmente creen.

Supported by

\protect\hyperlink{after-sponsor}{Continue reading the main story}

\hypertarget{john-lewis-juntos-ustedes-pueden-recuperar-el-alma-de-estados-unidos}{%
\section{John Lewis: Juntos, ustedes pueden recuperar el alma de Estados
Unidos}\label{john-lewis-juntos-ustedes-pueden-recuperar-el-alma-de-estados-unidos}}

Aunque me haya ido, los animo a responder al llamado más elevado de su
corazón y a defender lo que realmente creen.

Por John Lewis

El líder de los derechos civiles murió el 17 de julio y escribió este
ensayo poco antes de su muerte para ser publicado el día de su funeral.
La editora del Comité Editorial, Kathleen Kingsbury, escribió sobre este
artículo y el legado de Lewis en
\href{https://www.nytimes.com/2020/07/30/opinion/john-lewis-op-ed.html}{la
edición del jueves de nuestro boletín Opinion Today} (en inglés).

\begin{itemize}
\item
  30 de julio de 2020
\item
  \begin{itemize}
  \item
  \item
  \item
  \item
  \item
  \item
  \end{itemize}
\end{itemize}

\href{https://www.nytimes.com/2020/07/30/opinion/john-lewis-civil-rights-america.html}{Read
in English}

\href{https://www.nytimes.com/newsletters/el-times}{Regístrate para
recibir nuestro boletín} con lo mejor de The New York Times.

\begin{center}\rule{0.5\linewidth}{\linethickness}\end{center}

\textbf{S}i bien mi tiempo aquí ha llegado a su fin, quiero que sepan
que en los últimos días y horas de mi vida ustedes me han inspirado.
Ustedes me han llenado de esperanza sobre el próximo capítulo de la gran
historia estadounidense al usar su poder para marcar una diferencia en
nuestra sociedad. Millones de personas motivadas simplemente por la
compasión humana depusieron el lastre de la división. Alrededor del país
y del mundo, ustedes han puesto de lado la raza, la clase, la edad, el
idioma y la nacionalidad para exigir respeto a la vida humana.

Es por eso que tuve que visitar la plaza Black Lives Matter en
Washington, aunque me internaron en el hospital al día siguiente. Tenía
que ver y sentir por mí mismo que, después de muchos años de ser testigo
mudo, la verdad sigue su marcha.

\begin{center}\rule{0.5\linewidth}{\linethickness}\end{center}

Puedes escuchar este artículo de opinión en inglés

\hypertarget{listen-to-this-op-ed}{%
\subsubsection{Listen to This Op-Ed}\label{listen-to-this-op-ed}}

Audio Recording by Audm

Emmett Till fue mi George Floyd. Él fue mi Rayshard Brooks, Sandra Bland
y Breonna Taylor. Él tenía 14 años cuando fue asesinado, y en ese
entonces yo solo tenía 15 años. Nunca olvidaré el momento en que quedó
tan claro que él fácilmente podría haber sido yo. En aquellos días, el
miedo nos restringía como una prisión imaginaria y los pensamientos
inquietantes de brutalidad potencial cometida sin ninguna razón
comprensible eran las rejas.

Aunque estaba rodeado de dos padres amorosos, muchos hermanos, hermanas
y primos, su amor no podía protegerme de la opresión impía que esperaba
a las afueras de ese círculo familiar. La violencia desenfrenada y sin
control y el terror aprobado por el gobierno tenían el poder de
convertir en una pesadilla un simple paseo a la tienda por algunos
caramelos Skittles o un inocente trote por la mañana en una carretera
rural solitaria. Si queremos sobrevivir como nación unificada, debemos
descubrir lo que se arraiga tan fácilmente en nuestros corazones y que
podría despojar a la Iglesia Madre Emanuel en Carolina del Sur de sus
mejores y más brillantes feligreses, matar a desprevenidos asistentes de
un concierto en Las Vegas y estrangular hasta la muerte las esperanzas y
sueños de un violinista talentoso como Elijah McClain.

Al igual que muchos jóvenes de hoy, buscaba una salida ---algunos
podrían decir una entrada--- y entonces escuché la voz del doctor Martin
Luther King Jr. en una vieja radio. Hablaba sobre la filosofía y la
disciplina de la no violencia. Dijo que todos somos cómplices cuando
toleramos la injusticia. Dijo que no es suficiente decir que todo
mejorará poco a poco. Dijo que cada uno de nosotros tiene la obligación
moral de ponerse de pie, alzar la voz y manifestarse. Cuando vean algo
que no está bien, deben decir algo. Deben hacer algo. La democracia no
es un estado. Es un acto, y cada generación tiene que hacer su parte
para ayudar a construir lo que llamamos la Comunidad Amada, una nación y
una sociedad mundial en paz consigo misma.

Con una visión extraordinaria las personas comunes pueden rescatar el
alma de Estados Unidos al meterse en lo que yo llamo buenos problemas,
problemas necesarios. Votar y participar en el proceso democrático es
clave. El voto es el agente de cambio no violento más poderoso que se
tiene en una sociedad democrática. Deben usarlo porque no está
garantizado. Pueden perderlo.

También deben estudiar y leer las lecciones de la historia, porque la
humanidad ha estado en esta lucha existencial y desgarradora durante
mucho tiempo. La gente de todos los continentes ha estado en su lugar,
décadas y siglos antes que ustedes. La verdad no cambia, y es por eso
que las respuestas elaboradas hace mucho tiempo pueden ayudarlos a
encontrar soluciones a los desafíos de nuestro tiempo. Continúen
construyendo la unión entre movimientos que se extienden por todo el
mundo, porque debemos dejar de lado nuestra voluntad de sacar provecho
de la explotación de los demás.

Aunque puede que ya no esté con ustedes, los animo a responder el
llamado más elevado de su corazón y a defender lo que realmente creen.
En mi vida he hecho todo lo posible para demostrar que el camino de la
paz, el camino del amor y la no violencia es el camino más excelente.
Ahora es el turno de ustedes de hacer repicar la libertad.

Cuando los historiadores tomen sus bolígrafos para escribir la historia
del siglo XXI, que digan que fue su generación la que finalmente
derrumbó las pesadas cargas de odio y que la paz finalmente triunfó
sobre la violencia, la agresión y la guerra. Así que les digo, caminen
con el viento, hermanos y hermanas, y dejen que el espíritu de la paz y
el poder del amor eterno sean su guía.

John Lewis, líder de los derechos civiles y congresista que murió el 17
de julio, escribió este ensayo poco antes de su muerte.

\emph{Fotografía de John Lewis por David Deal/Redux}

Advertisement

\protect\hyperlink{after-bottom}{Continue reading the main story}

\hypertarget{site-index}{%
\subsection{Site Index}\label{site-index}}

\hypertarget{site-information-navigation}{%
\subsection{Site Information
Navigation}\label{site-information-navigation}}

\begin{itemize}
\tightlist
\item
  \href{https://help.nytimes.com/hc/en-us/articles/115014792127-Copyright-notice}{©~2020~The
  New York Times Company}
\end{itemize}

\begin{itemize}
\tightlist
\item
  \href{https://www.nytco.com/}{NYTCo}
\item
  \href{https://help.nytimes.com/hc/en-us/articles/115015385887-Contact-Us}{Contact
  Us}
\item
  \href{https://www.nytco.com/careers/}{Work with us}
\item
  \href{https://nytmediakit.com/}{Advertise}
\item
  \href{http://www.tbrandstudio.com/}{T Brand Studio}
\item
  \href{https://www.nytimes.com/privacy/cookie-policy\#how-do-i-manage-trackers}{Your
  Ad Choices}
\item
  \href{https://www.nytimes.com/privacy}{Privacy}
\item
  \href{https://help.nytimes.com/hc/en-us/articles/115014893428-Terms-of-service}{Terms
  of Service}
\item
  \href{https://help.nytimes.com/hc/en-us/articles/115014893968-Terms-of-sale}{Terms
  of Sale}
\item
  \href{https://spiderbites.nytimes.com}{Site Map}
\item
  \href{https://help.nytimes.com/hc/en-us}{Help}
\item
  \href{https://www.nytimes.com/subscription?campaignId=37WXW}{Subscriptions}
\end{itemize}
