Sections

SEARCH

\protect\hyperlink{site-content}{Skip to
content}\protect\hyperlink{site-index}{Skip to site index}

\href{https://www.nytimes.com/es/section/estados-unidos}{Estados Unidos}

\href{https://myaccount.nytimes.com/auth/login?response_type=cookie\&client_id=vi}{}

\href{https://www.nytimes.com/section/todayspaper}{Today's Paper}

\href{/es/section/estados-unidos}{Estados Unidos}\textbar{}Hijo de su
padre: el presidente Donald Trump aprendió en casa a no mostrar
aflicción

\url{https://nyti.ms/2PauO13}

\begin{itemize}
\item
\item
\item
\item
\item
\item
\end{itemize}

\href{https://www.nytimes.com/es/spotlight/coronavirus?action=click\&pgtype=Article\&state=default\&region=TOP_BANNER\&context=storylines_menu}{El
brote de coronavirus}

\begin{itemize}
\tightlist
\item
  \href{https://www.nytimes.com/es/interactive/2020/espanol/america-latina/coronavirus-en-mexico.html?action=click\&pgtype=Article\&state=default\&region=TOP_BANNER\&context=storylines_menu}{Mapa
  y casos en México}
\item
  \href{https://www.nytimes.com/es/2020/07/31/espanol/ciencia-y-tecnologia/ninos-contagio-coronavirus.html?action=click\&pgtype=Article\&state=default\&region=TOP_BANNER\&context=storylines_menu}{Los
  niños y el virus}
\item
  \href{https://www.nytimes.com/es/interactive/2020/science/coronavirus-tratamientos-curas.html?action=click\&pgtype=Article\&state=default\&region=TOP_BANNER\&context=storylines_menu}{Fármacos
  y tratamientos}
\item
  \href{https://www.nytimes.com/es/2020/07/06/espanol/ciencia-y-tecnologia/coronavirus-transmision-aire.html?action=click\&pgtype=Article\&state=default\&region=TOP_BANNER\&context=storylines_menu}{Cómo
  se transmite el coronavirus}
\item
  \href{https://www.nytimes.com/es/2020/07/14/espanol/estilos-de-vida/botiquin-medicina-coronavirus.html?action=click\&pgtype=Article\&state=default\&region=TOP_BANNER\&context=storylines_menu}{Prepara
  tu botiquín}
\end{itemize}

Advertisement

\protect\hyperlink{after-top}{Continue reading the main story}

Supported by

\protect\hyperlink{after-sponsor}{Continue reading the main story}

\hypertarget{hijo-de-su-padre-el-presidente-donald-trump-aprendiuxf3-en-casa-a-no-mostrar-aflicciuxf3n}{%
\section{Hijo de su padre: el presidente Donald Trump aprendió en casa a
no mostrar
aflicción}\label{hijo-de-su-padre-el-presidente-donald-trump-aprendiuxf3-en-casa-a-no-mostrar-aflicciuxf3n}}

Ya sea que enfrente la pérdida de un familiar o la muerte de casi
150.000 estadounidenses en una pandemia creciente, el mandatario
estadounidense casi nunca exhibe empatía. Se lo inculcó su padre.

\includegraphics{https://static01.nyt.com/images/2020/07/28/us/politics/30dc-fred-Trump-ES-00/merlin_97610656_78a7876d-c54b-430a-a599-163bcf354205-articleLarge.jpg?quality=75\&auto=webp\&disable=upscale}

\href{https://www.nytimes.com/by/annie-karni}{\includegraphics{https://static01.nyt.com/images/2019/02/05/multimedia/author-annie-karni/author-annie-karni-thumbLarge.png}}\href{https://www.nytimes.com/by/katie-rogers}{\includegraphics{https://static01.nyt.com/images/2018/06/12/multimedia/author-katie-rogers/author-katie-rogers-thumbLarge-v2.png}}

Por \href{https://www.nytimes.com/by/annie-karni}{Annie Karni} y
\href{https://www.nytimes.com/by/katie-rogers}{Katie Rogers}

\begin{itemize}
\item
  31 de julio de 2020
\item
  \begin{itemize}
  \item
  \item
  \item
  \item
  \item
  \item
  \end{itemize}
\end{itemize}

\href{https://www.nytimes.com/2020/07/28/us/politics/donald-fred-trump.html}{Read
in
English}\href{https://www.nytimes.com/2020/07/28/us/politics/donald-fred-trump.html}{Read
in English}

\href{https://www.nytimes.com/newsletters/el-times}{Regístrate para
recibir nuestro boletín} con lo mejor de The New York Times.

\begin{center}\rule{0.5\linewidth}{\linethickness}\end{center}

WASHINGTON --- La iglesia Marble Collegiate en la Quinta Avenida de
Manhattan estaba repleta de desarrolladores inmobiliarios, políticos y
celebridades neoyorquinas, más de 600 personas en total, para el funeral
de Fred C. Trump, el constructor cuyas torres de alquiler con diseño de
ladrillos sin adornos transformaron Brooklyn y Queens.

Tres de los cuatro hijos que le sobrevivieron, quienes crecieron
escuchando los sermones del ministro más famoso de la iglesia, Norman
Vincent Peale, ofrecieron cariñosos panegíricos a su padre. Luego tocó
el turno de Donald Trump.

Empezó hablando de sí mismo.

Se había enterado de la muerte de su padre, le dijo a la multitud ese
día de junio de 1999, hacía apenas unos momentos, después de leer un
artículo de primera plana en The New York Times sobre su mayor
desarrollo hasta la fecha, Trump Place.

``Donald comenzó su panegírico diciendo: `Estaba teniendo el mejor año
de mi carrera empresarial y me había sentado a desayunar y pensaba en lo
bien que me estaba yendo''', al enterarse de la muerte de su padre, dijo
Alan Marcus, exconsultor de relaciones públicas de la Organización
Trump. ``El panegírico de Donald solo hablaba de Donald y todos en la
iglesia de Vincent Peale lo sabían'', recordó.

Gwenda Blair, biógrafa de la familia Trump, también asistió al funeral.
Ella tampoco pudo evitar fijarse en el discurso, que describió en su
libro \emph{The Trumps}.

``¿Nos sorprendió?'', dijo Blair en una entrevista. ``No. ¿Fue
impactante? Sí''.

\includegraphics{https://static01.nyt.com/images/2020/07/29/us/politics/30dc-fred-Trump-ES-01/28dc-fredtrump-articleLarge.jpg?quality=75\&auto=webp\&disable=upscale}

Ya sea que enfrente la pérdida de un familiar, la muerte de casi 150.000
estadounidenses en una pandemia creciente, más de 30 millones de
personas desempleadas o la agitación racial provocada por los asesinatos
de afroestadounidenses a manos de agentes de policía blancos, el
presidente Trump casi nunca muestra empatía en público.
\href{https://www.nytimes.com/es/2020/07/09/espanol/mundo/libro-mary-trump-sobrina.html}{Un
libro publicado este verano} por su sobrina, Mary L. Trump, renovó la
atención en este rasgo.

Trump no ha proclamado ningún día de luto nacional en honor de las
víctimas del virus. En los eventos del Jardín de las Rosas se rodea de
ejecutivos empresariales que presionan para reabrir la economía en lugar
de familias que han perdido sus trabajos o a sus seres queridos. En
sombríos discursos durante el fin de semana del 4 de julio, denunció con
rabia lo que calificó de ``nuevo fascismo de extrema izquierda''. No
mencionó ni una sola vez a George Floyd, el hombre negro cuya muerte
bajo custodia policial ha desencadenado protestas mundiales por la
injusticia racial.

Existen muchas razones, entre ellas la negación y la desorganización,
por las que el manejo del virus por parte de Trump ha provocado crisis
catastróficas en Estados Unidos. Sin embargo, hasta los republicanos
dicen que una de las principales causas es que el presidente no es capaz
de ponerse en el lugar de los demás y encauzar su dolor. Su falta de
voluntad, o su incapacidad, para consolar a una nación angustiada ha
consternado a los críticos, sorprendido a los aliados y ha exasperado al
personal de la Casa Blanca, cuyos miembros siguen perplejos, sin poder
contestar por qué a Trump se le escapa esta parte tan básica del
liderazgo presidencial.

``Su estilo como líder es tener que ser un tipo duro'', dijo en una
entrevista el congresista por Nueva York Peter T. King, uno de los
aliados del presidente. ``No puedes mostrar ningún tipo de debilidad. No
quiere demostrar que esto lo supera'', afirmó King.

Trump se ha comportado así toda su vida, según sus amigos y familiares.
Lo aprendió, dicen, en casa, en especial de su padre, que lo crió con
estilo disciplinario y gastó cientos de millones de dólares en financiar
la carrera del hijo y le enseñó a dominar o someterse. En el mundo de
Fred Trump, mostrar tristeza o dolor era un signo de debilidad.

``Lo único que le importaba a Trump era eso de: `Tengo que ganar.
Enséñame a ganar''', dijo en una entrevista George White, un antiguo
compañero de escuela del presidente Trump en la Academia Militar de
Nueva York, quien convivió con padre e hijo durante años.

Al recordar la dura influencia de Fred, White dijo que el antiguo mentor
escolar de Trump, un veterano de combate de la Segunda Guerra Mundial
llamado Theodore Dobias, una vez le dijo que ``nunca había visto a un
cadete cuyo padre fuera más duro que el padre de Donald Trump''. Fred
Trump visitaba la academia casi todos los fines de semana para vigilar a
su hijo, dijo White.

El padre de Trump aún es parte de su vida, dijo Andrew Stein, un
expresidente del condado de Manhattan que ha conocido al mandatario
desde hace décadas y se ha reunido regularmente con él en la Casa
Blanca. Cuando han estado a solas en la Oficina Oval, dijo, Trump a
menudo ha señalado hacia el techo y se ha referido a su padre. ``Levanta
la vista al cielo, y dice: `Fred, ¿puedes creerlo?''', dijo Stein.

Este artículo se basa en entrevistas con más de 20 amigos, aliados
políticos, miembros del gobierno, familiares y empleados y exempleados
de Trump.

La relación dominante de Fred Trump con sus hijos, y la forma en que
moldeó al segundo de ellos, es ahora la fuerza central detrás del éxito
editorial \emph{Too Much and Never Enough: How My Family Created the
World's Most Dangerous Man}, de Mary Trump, psicóloga clínica y única
sobrina del presidente Trump.

``Rendir homenaje a las víctimas de la COVID-19 sería asociarse con su
debilidad, un rasgo que su padre le enseñó a despreciar'', escribió Mary
Trump.

Robert Trump, el hermano menor del presidente, que junto con Trump trató
de detener la publicación del libro, se mostró en desacuerdo con esa
caracterización. En una declaración para este artículo, dijo que sabía
``lo abnegado que era mi padre y que es Donald, mucho más de lo que
nadie creería''.

\hypertarget{dominar-o-someterse}{%
\subsection{Dominar o someterse}\label{dominar-o-someterse}}

Donald Trump, quien nació en 1946 en medio del optimismo y la vitalidad
de la posguerra en Estados Unidos, creció en una mansión de millonario
de ladrillos rojos y columnas blancas construida por su padre en lo que
entonces era una comunidad cerrada, casi toda de vecinos blancos, en
Queens. Como Trump mismo admitió en su autobiografía, \emph{El arte de
la negociación}, era un niño difícil y voluble. Una de sus actividades
favoritas era poner a prueba a los demás, desde los niños de su barrio
hasta las figuras de autoridad. En una ocasión, los vecinos
\href{https://www.washingtonpost.com/lifestyle/style/young-donald-trump-military-school/2016/06/22/f0b3b164-317c-11e6-8758-d58e76e11b12_story.html}{lo
sorprendieron lanzando piedras} por encima de una cerca a un niño
pequeño que estaba en un corralito.

``Incluso en la escuela primaria, yo era un niño muy asertivo y
agresivo'', escribió Trump.

El hogar era estricto. Fred Trump era ``rígido y formal'', dijo una
vecina, Annamaria Forcier, y se enfocaba en el trabajo y el dinero. (Su
padre, Frederick Trump, había hecho una fortuna en la fiebre del oro
antes de morir de gripe española en 1918).

La madre del presidente, Mary Anne MacLeod Trump, era la hija de un
pescador de un pueblo escocés de las islas Hébridas Exteriores que llegó
a Nueva York en 1930 a la edad de 18 años. Mary Anne encontró un trabajo
como empleada doméstica en la casa de la viuda de Andrew Carnegie, según
los registros del censo que la periodista Nina Burleigh desenterró para
su libro \emph{Golden Handcuffs: The Secret History of Trump's Women}.
Actualmente, la casa es el Museo Cooper Hewitt en Manhattan.

El roce de Mary Trump con la alta sociedad engendró en ella un amor de
nuevo rico por la ceremonia y la pompa que compartía con su hijo. En su
libro, Burleigh escribió que ``los aires de Mary eran la antítesis'' de
las tendencias germánicas de su esposo. El sentido del humor de ella a
menudo podía dirigirse hacia Donald Trump, dijo uno de los hijos del
presidente.

Image

Una foto del anuario de Trump de su tiempo en la Academia Militar de
Nueva York, donde asistió a la escuela secundaria.Credit...Fred R.
Conrad para The New York Times

Pero Fred Trump era la máxima autoridad y los niños aprendieron a ser
estoicos ante la pérdida, incluso cuando su madre se enfermó gravemente
de peritonitis, una inflamación del revestimiento de la cavidad
abdominal, además de pasar una larga temporada en el hospital y padecer
una enfermedad persistente después del nacimiento de su quinto y último
hijo.

``Mi padre llegó a casa y me dijo que no se esperaba que ella viviera,
pero que debía ir a la escuela y que me llamaría si pasaba algo``, dijo
Maryanne Trump Barry, una de sus hijas, en una entrevista con Blair.
``Así es, ve a la escuela como siempre'', dijo.

En opinión de Mary Trump, Donald Trump, que tenía dos años y medio en
ese momento, sufrió el año que su madre estuvo enferma. ''Las
necesidades de Donald, que habían sido satisfechas de manera desigual
antes de la enfermedad de su madre, a duras penas fueron atendidas por
su padre'', escribió Mary Trump. ``El hecho de que Fred se convirtiera,
por defecto, en la principal fuente de consuelo de Donald cuando era
mucho más probable que fuera una fuente de miedo o rechazo puso a Donald
en una posición intolerable: la dependencia total de un cuidador que
también era probable que fuera el causante de su pánico''.

Como resultado, escribió la sobrina de Trump, ``sufrió privaciones que
lo marcarían de por vida''.

El padre de la familia presionó fuertemente a Fred Trump Jr., el segundo
en nacer y el primer hombre, para que fuera el presunto heredero del
negocio familiar. Sin embargo, Fred Jr. nunca se dedicó al negocio
inmobiliario y murió solo en el hospital en 1981, después de una larga
lucha contra el alcoholismo. Tenía 42 años. Según Mary Trump, su hija,
Donald Trump fue al cine esa noche, y Fred Trump padre no lo visitó.

La familia rara vez hablaba sobre la muerte de Fred Jr., pero en una
entrevista de 1990 en Playboy, Donald Trump dedicó unos momentos a
reflexionar sobre ello. ``Vi que la gente realmente se aprovechaba de
Fred y la lección que aprendí fue siempre mantenerme en guardia al 100
por ciento, a diferencia de él'', dijo Trump. ``Él no sentía que hubiera
una razón real para eso, lo que es un error fatal en la vida. La gente
es demasiado confiada. Soy un tipo muy desconfiado''.

\hypertarget{no-tiene-tiempo-para-la-empatuxeda}{%
\subsection{`No tiene tiempo para la
empatía'}\label{no-tiene-tiempo-para-la-empatuxeda}}

Dan P. McAdams, profesor de Psicología y Desarrollo Humano en la
Universidad de Northwestern que ha escrito sobre Donald Trump, dijo en
una entrevista que desde la infancia el futuro presidente (con la ayuda
de su padre) se condicionó a sí mismo a enfrentar la vida como una serie
de batallas por ganar.

``No tiene tiempo para la empatía porque el mundo está contra él'', dijo
McAdams.

Después de la muerte de su hermano, Trump se convirtió en el heredero, y
durante las siguientes décadas él y su padre fueron socios cercanos en
las estratagemas y evasiones de impuestos que formaban parte del negocio
familiar. Hablaban casi todos los días y pasaban tiempo juntos los fines
de semana.

``Nunca me sentí intimidado por mi padre, como le sucedía a la mayoría
de la gente'', escribió Trump en su autobiografía. ``Me le enfrentaba y
él respetaba eso. Teníamos una relación que era casi de negocios'',
agregó.

Al igual que su padre, Trump seguía adelante ante las pérdidas. En la
Organización Trump no era un jefe que se acercara a expresar
condolencias. ``Uno de sus banqueros murió y alguien en este pequeño
círculo le dijo: `Donald, ¿no crees que deberías llamar a la
familia?''', recordó Marcus, el exconsultor de relaciones públicas de la
Organización Trump. ``Él respondió: `¿Por qué? Está muerto'''.

La frialdad de Trump frente a la enfermedad conmocionó incluso a algunos
de sus asociados más cercanos. Después de que Roy Cohn, abogado personal
de Trump durante mucho tiempo, se enteró de que tenía sida en la década
de 1980, Trump interrumpió abruptamente el contacto con él, un cambio
dramático de la estrecha relación que habían tenido durante años, que
los asociados recordaron que involucraba hablar por teléfono al menos
cinco veces al día.

``Descarta a las personas que ya no son útiles, y no importa qué es lo
que haya hecho que la persona ya no sea útil'', dijo Michael D'Antonio,
biógrafo de Trump. ``Si caes en desgracia, o estás muriendo, o muerto,
ya no existes para él''. D'Antonio recordó que Trump le dijo que le
había dado a Cohn un lugar para quedarse al final de su vida. ``Donald
pensó que proporcionarle algo de valor material era adecuado'', dijo.

Image

El casino Trump Taj Mahal en Atlantic City en 1990. Trump pareció
priorizar sus negocios sobre la empatía hacia los demás.Credit...Tony
Ward/Mirrorpix, vía Getty Images

En 1989, un helicóptero que volaba de Nueva York a Atlantic City, Nueva
Jersey, se estrelló y
\href{https://www.nytimes.com/1989/10/11/nyregion/copter-crash-kills-3-aides-of-trump.html}{mató
a tres altos ejecutivos de los casinos de Trump en Atlantic City}.
Trump, de manera infame, utilizó la tragedia para su propio beneficio,
sembrando noticias en los periódicos locales según las cuales estaba
programado que él abordara el avión hasta el último momento y que por
poco había escapado de la muerte. En un libro posterior admitió que
nunca había estado programado que él volase en ese helicóptero.

Jack O'Donnell, quien era presidente del Trump Plaza Hotel y Casino en
ese momento y escribió un libro mordaz sobre Trump, dijo que el actual
mandatario procesó las muertes mayormente como un golpe meteórico contra
su negocio.

Sin embargo, O'Donnell recordó que la noche del accidente, Trump hizo
algo inusual para él.

``No creí que fuera capaz de hacerlo'', dijo O'Donnell. ``Pero voló a
Atlantic City, y fue personalmente a las casas de las viudas y pasó
tiempo con ellas''. Sin embargo, meses después, ``culpó a esos mismos
tipos por los problemas que él generó. Fue por eso que acabé por
dejarlo, luego de una gran discusión'', recordó O'Donnell.

Poco más de una década después, cuando la madre de Trump estaba enferma
de gravedad, sus hermanos tuvieron que recordarle que se alejara del
trabajo y la visitara en el hospital, recordó Marcus. Murió a la edad de
88 años en el año 2000, apenas un año después que su marido.

\hypertarget{un-gran-duxeda-para-todos}{%
\subsection{`Un gran día para todos'}\label{un-gran-duxeda-para-todos}}

Image

Trump ha utilizado los eventos de la Casa Blanca para reunirse con
líderes empresariales, en lugar de lamentar la muerte de las víctimas de
la pandemia del coronavirus.Credit...Doug Mills/The New York Times

En respuesta a este artículo, Hogan Gidley, un exvocero de la Casa
Blanca que desde entonces ha hecho la transición a la campaña, dijo que
el presidente sí ha mostrado empatía. Como evidencia, envió tres
recortes de prensa de cobertura positiva para Trump.

\href{https://www.jta.org/1988/07/20/archive/orthodox-child-with-rare-ailment-is-rescued-aboard-tycoons-jet}{Uno
de 1988} relataba cómo Trump donó el uso de su jet privado para llevar a
un niño enfermo a Nueva York para que recibiera tratamiento por un
problema médico poco común. Otro detallaba un esfuerzo de Trump en 1986
para ayudar a una viuda a recaudar dinero que
\href{https://apnews.com/24c831825e0dab47d51d8d25bffe45f5}{cubriera los
pagos de su hipoteca}. El tercero daba cuenta de la donación de
\href{https://www.aol.com/2013/11/08/trump-gift-barton-buffalo/}{10.000
dólares} que hizo Trump en 2013 a un conductor de autobús que evitó que
una mujer saltara de un puente.

El mes pasado en el Jardín de las Rosas cuando Trump destacó una
disminución en la tasa de desempleo,
\href{https://www.nytimes.com/2020/06/05/us/politics/trump-jobs-report-george-floyd.html}{mencionó
a George Floyd}.

``Esperemos que George esté mirando hacia abajo ahora mismo y diciendo
que esto es algo grandioso para nuestro país``, dijo. ``Es un gran día
para él, un gran día para todos''.

Para Marcus, el expublicista que había asistido al funeral de Fred Trump
padre 21 años antes, las palabras del presidente provocaron una
sensación de `déjà vu'. ``Se asemejó un poco al panegírico que pronunció
por su padre'', dijo Marcus. Una vez más, dijo, ``se trataba solo de
él''.

Annie Karni es corresponsal de la Casa Blanca. Anteriormente cubrió la
Casa Blanca y la campaña presidencial de 2016 de Hillary Clinton para
Politico, y ha cubierto noticias locales y política en Nueva York para
el New York Post y el New York Daily News.
\href{https://twitter.com/AnnieKarni}{@AnnieKarni}

Katie Rogers es una corresponsal de la Casa Blanca en el buró de
Washington, que cubre el impacto cultural del gobierno de Trump en la
capital de Estados Unidos y otros lugares.
\href{https://twitter.com/katierogers}{@katierogers}

\begin{center}\rule{0.5\linewidth}{\linethickness}\end{center}

Advertisement

\protect\hyperlink{after-bottom}{Continue reading the main story}

\hypertarget{site-index}{%
\subsection{Site Index}\label{site-index}}

\hypertarget{site-information-navigation}{%
\subsection{Site Information
Navigation}\label{site-information-navigation}}

\begin{itemize}
\tightlist
\item
  \href{https://help.nytimes.com/hc/en-us/articles/115014792127-Copyright-notice}{©~2020~The
  New York Times Company}
\end{itemize}

\begin{itemize}
\tightlist
\item
  \href{https://www.nytco.com/}{NYTCo}
\item
  \href{https://help.nytimes.com/hc/en-us/articles/115015385887-Contact-Us}{Contact
  Us}
\item
  \href{https://www.nytco.com/careers/}{Work with us}
\item
  \href{https://nytmediakit.com/}{Advertise}
\item
  \href{http://www.tbrandstudio.com/}{T Brand Studio}
\item
  \href{https://www.nytimes.com/privacy/cookie-policy\#how-do-i-manage-trackers}{Your
  Ad Choices}
\item
  \href{https://www.nytimes.com/privacy}{Privacy}
\item
  \href{https://help.nytimes.com/hc/en-us/articles/115014893428-Terms-of-service}{Terms
  of Service}
\item
  \href{https://help.nytimes.com/hc/en-us/articles/115014893968-Terms-of-sale}{Terms
  of Sale}
\item
  \href{https://spiderbites.nytimes.com}{Site Map}
\item
  \href{https://help.nytimes.com/hc/en-us}{Help}
\item
  \href{https://www.nytimes.com/subscription?campaignId=37WXW}{Subscriptions}
\end{itemize}
