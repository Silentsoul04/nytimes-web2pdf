Sections

SEARCH

\protect\hyperlink{site-content}{Skip to
content}\protect\hyperlink{site-index}{Skip to site index}

\href{https://www.nytimes.com/es/section/america-latina}{América Latina}

\href{https://myaccount.nytimes.com/auth/login?response_type=cookie\&client_id=vi}{}

\href{https://www.nytimes.com/section/todayspaper}{Today's Paper}

\href{/es/section/america-latina}{América Latina}\textbar{}Menos sexo,
más audiencia: la pandemia reanima a las telenovelas mexicanas

\url{https://nyti.ms/2Xhtz4s}

\begin{itemize}
\item
\item
\item
\item
\item
\item
\end{itemize}

\href{https://www.nytimes.com/es/spotlight/coronavirus?action=click\&pgtype=Article\&state=default\&region=TOP_BANNER\&context=storylines_menu}{El
brote de coronavirus}

\begin{itemize}
\tightlist
\item
  \href{https://www.nytimes.com/es/interactive/2020/espanol/america-latina/coronavirus-en-mexico.html?action=click\&pgtype=Article\&state=default\&region=TOP_BANNER\&context=storylines_menu}{Mapa
  y casos en México}
\item
  \href{https://www.nytimes.com/es/2020/07/31/espanol/ciencia-y-tecnologia/ninos-contagio-coronavirus.html?action=click\&pgtype=Article\&state=default\&region=TOP_BANNER\&context=storylines_menu}{Los
  niños y el virus}
\item
  \href{https://www.nytimes.com/es/interactive/2020/science/coronavirus-tratamientos-curas.html?action=click\&pgtype=Article\&state=default\&region=TOP_BANNER\&context=storylines_menu}{Fármacos
  y tratamientos}
\item
  \href{https://www.nytimes.com/es/2020/07/06/espanol/ciencia-y-tecnologia/coronavirus-transmision-aire.html?action=click\&pgtype=Article\&state=default\&region=TOP_BANNER\&context=storylines_menu}{Cómo
  se transmite el coronavirus}
\item
  \href{https://www.nytimes.com/es/2020/07/14/espanol/estilos-de-vida/botiquin-medicina-coronavirus.html?action=click\&pgtype=Article\&state=default\&region=TOP_BANNER\&context=storylines_menu}{Prepara
  tu botiquín}
\end{itemize}

Advertisement

\protect\hyperlink{after-top}{Continue reading the main story}

Supported by

\protect\hyperlink{after-sponsor}{Continue reading the main story}

\hypertarget{menos-sexo-muxe1s-audiencia-la-pandemia-reanima-a-las-telenovelas-mexicanas}{%
\section{Menos sexo, más audiencia: la pandemia reanima a las
telenovelas
mexicanas}\label{menos-sexo-muxe1s-audiencia-la-pandemia-reanima-a-las-telenovelas-mexicanas}}

Desdeñados por ser muy anticuados para competir con las series
transmitidas por internet, los melodramas televisivos recuperan a un
público ansioso que busca entretenimiento familiar y reconfortante en
tiempos de incertidumbre.

\includegraphics{https://static01.nyt.com/images/2020/07/23/world/00mexico-melodrama-ES-00/merlin_173965713_f45fb63a-5152-433a-b9e9-4cfa786e4780-articleLarge.jpg?quality=75\&auto=webp\&disable=upscale}

\href{https://www.nytimes.com/by/natalie-kitroeff}{\includegraphics{https://static01.nyt.com/images/2019/03/01/multimedia/author-natalie-kitroeff/author-natalie-kitroeff-thumbLarge.png}}

Por \href{https://www.nytimes.com/by/natalie-kitroeff}{Natalie Kitroeff}

\begin{itemize}
\item
  2 de agosto de 2020
\item
  \begin{itemize}
  \item
  \item
  \item
  \item
  \item
  \item
  \end{itemize}
\end{itemize}

\href{https://www.nytimes.com/2020/08/02/world/americas/mexico-tv-virus-telenovela.html}{Read
in English}

\href{https://www.nytimes.com/newsletters/el-times}{Regístrate para
recibir nuestro boletín} con lo mejor de The New York Times.

\begin{center}\rule{0.5\linewidth}{\linethickness}\end{center}

CIUDAD DE MÉXICO --- El romance de México con el melodrama había
terminado.

Después de décadas de reinado supremo en los horarios de máxima
audiencia, las icónicas telenovelas del país perdían espectadores. Los
ejecutivos de la industria las declararon obsoletas, demasiado cursis y
simplistas para competir con programas más complejos y de mayor
presupuesto.

Ahora, gracias a la pandemia, la telenovela arrasa de nuevo.

Confinados en sus hogares, millones de mexicanos han dedicado sus noches
a los melodramas tradicionales y otros clásicos kitsch, y han encontrado
en los rostros familiares y en la garantía de los finales felices un
bálsamo para las ansiedades surgidas de una crisis de salud
\href{https://www.nytimes.com/es/interactive/2020/espanol/america-latina/coronavirus-en-mexico.html}{que
ha dejado al menos 43.000 muertos} y millones de
\href{https://www.nytimes.com/es/2020/06/08/espanol/america-latina/mexico-amlo-deuda-coronavirus.html}{desempleados}.

``No hay miedo ni horror ni miseria'', dijo Enrique Millán, de 75 años,
sobre las telenovelas que se adueñaron de toda su atención después de
que la pandemia colocó al fútbol en pausa. ``Puedo imaginar lo que va a
pasar al final de cada episodio. No hay estrés''.

\includegraphics{https://static01.nyt.com/images/2020/07/28/world/00mexico-melodrama-ES-01/merlin_174535812_5c6ee301-bd88-4ec0-b4dc-48a5b3416a9a-articleLarge.jpg?quality=75\&auto=webp\&disable=upscale}

Los índices de audiencia para estos programas se han disparado en los
últimos meses, reviviendo un género que moldeó a generaciones y se
convirtió en una de las exportaciones culturales más importantes del
país.

El inicio de una recesión económica mundial ha hecho que este tipo de
programas sean más atractivos por definición. Las telenovelas se emiten
por señal abierta, lo que las hace más accesibles para la familia
mexicana promedio que Netflix o los canales de cable premium.

Pero su atractivo también emana de un estilo particular de narración sin
complicaciones que alivia el aburrimiento de la vida en cuarentena al
tiempo que calma los temores y brinda la intimidad emocional que las
interacciones diarias han perdido con el virus.

``Prendo la televisión, el tiempo pasa y no sientes que no estás
haciendo nada'', dijo Minerva Becerril, quien ve telenovelas y otros
melodramas todas las noches con su madre de 90 años en su casa en las
afueras de Ciudad de México. ``Es un rato de calma y ves escenas de
amor, algo que me gusta porque soy romántica''.

Durante la pandemia, Becerril comenzaba sus tardes con \emph{Te doy la
vida}, una telenovela que presenta un triángulo amoroso, y luego pasa a
\emph{La rosa de Guadalupe}, un drama con matices religiosos. A veces ve
\emph{Destilando amor}, pero no le gusta \emph{Rubí}, una nueva versión
de la telenovela de 2004 basada en una historieta de finales de los años
60. ``La versión de la revista era mejor'', dijo.

El resurgimiento de los melodramas en México ha sido una bendición para
Televisa, el otrora monopolio mediático que en años recientes se ha
visto afectado por los servicios de transmisión en continuo y otros
competidores.

Image

Imágenes de las estrellas de Televisa en la pared de~ las oficinas de la
producción. Una línea amarilla ayuda a recordar el distanciamiento
social.~Credit...Meghan Dhaliwal para The New York Times

Durante el segundo trimestre de este año, 6,6 millones de personas
vieron cada noche el canal insignia de Televisa durante el horario
estelar, cuando se transmiten las telenovelas y otros melodramas.
Durante el mismo período en 2019, según la cadena, fueron cinco
millones. Los índices de audiencia para el canal aumentaron dos veces
más que la audiencia general de televisión en México de mayo a junio.

Según los índices de audiencia de Nielsen, Televisa estima que más de
diez millones de personas vieron el final de \emph{Te doy la vida}, que
se emitió a principios de este mes, con lo que se convirtió en el
episodio de telenovela más visto de la cadena desde 2016.

``De repente, los índices de audiencia subieron'', dijo Isaac Lee,
exejecutivo de Televisa y Univisión. ``Nadie sabe si es un momento, un
coletazo, una tendencia o si la telenovela ha vuelto''.

Cuando Lee se convirtió en director de contenidos de Televisa en 2017,
la cadena estaba en crisis. Desde hacía décadas, los ingresos de los
mexicanos habían ido en aumento y el acceso a internet se extendía por
todo el país, lo que alejó a la gente de los clásicos melodramas que
fueron el pan de cada día de Televisa durante medio siglo.

Los ejecutivos de la industria querían más acción, más violencia y
mayores presupuestos, ingredientes que parecían explicar el éxito de los
dramas sobre narcotraficantes en Telemundo y series como \emph{Narcos}
en Netflix.

Lee comenzó a mirar sin parar toda esa programación y pronto se dio
cuenta de lo que debería haber sido obvio: él no era el público
objetivo. Y tampoco lo eran los otros ejecutivos de la compañía que
habían tomado decisiones sobre los programas.

``Decidí no mirar el contenido'', dijo, ``porque sabía que lo
arruinaría''.

Image

Minerva Becerril, izquierda, ve telenovelas con su madre Gorgonia
Becerril Rocha, porque ofrecen ``un rato de calma''Credit...Meghan
Dhaliwal para The New York Times

Después de muchas conversaciones con los espectadores, quedó claro que
el melodrama solo necesitaba un cambio de imagen, dijo. Televisa comenzó
a modernizar sus telenovelas, atenuando las bofetadas y los barítonos
operísticos en favor de personajes que hablaban en voz normal sobre
problemas reales.

Su apuesta fue \emph{La rosa de Guadalupe}, un drama de Televisa que
tenía una década y había sido subestimado durante mucho tiempo por los
propios ejecutivos de la cadena.

\emph{La rosa de Guadalupe} no es una telenovela con personajes y
conflictos establecidos, pero es el pináculo del melodrama. Cada
episodio de una hora cuenta una historia independiente que siempre sigue
el mismo arco: alguien enfrenta un problema y reza para pedir la ayuda
de la Virgen de Guadalupe. Aparece una rosa blanca, un viento santo
sopla sobre sus rostros y pronto sus problemas han terminado.

Lo que el programa tenía que le faltaba a las telenovelas de la cadena
era actualidad cultural. Los temas que aborda \emph{La rosa de
Guadalupe} a menudo se inspiran en los titulares, como el episodio
dedicado a una familia separada por la deportación de Estados Unidos, o
aquel sobre los adolescentes que consumían alcohol
\href{https://www.ncbi.nlm.nih.gov/pmc/articles/PMC4009175/}{vertiéndolo
sobre sus globos oculares}, una broma peligrosa que circulaba en las
redes sociales.

El drama también ha atraído a una sorprendente cantidad de seguidores
entre los jóvenes mexicanos, aunque muchos juran que, a diferencia de
sus abuelas, lo sintonizan de forma irónica, para burlarse de sus tramas
exageradas. Tik Tok, Twitter y YouTube están llenos de memes y videos
que ridiculizan el programa.

Image

Una escena de ``La rosa de Guadalupe'' en la que un personaje lucha por
reunirse con su familia después de que lo deportan de Estados
Unidos.~Credit...Televisa

``Lo consideramos como absurdo'', dijo Héctor Ortega, de 22 años,
creador de la cuenta de Twitter `Out of Context Rosa', donde publica
videos breves de los momentos más exagerados de la teleserie. ``Si ni
siquiera veo el programa. Vi como todos los memes y el impacto que tiene
sobre generaciones como la mía, que no son precisamente su \emph{target
market}''.

Por supuesto, muchos de los que lo critican resultan ser espectadores
leales del programa. \emph{La rosa de Guadalupe} ha visto un gran
crecimiento en su audiencia más joven en los últimos meses,
especialmente entre los televidentes masculinos de 13 a 31 años, cuyas
cifras han aumentado en aproximadamente un 40 por ciento en comparación
con el año pasado.

No está claro, ni siquiera para los ejecutivos de Televisa, si el éxito
puede seguir después de una pandemia que ha eliminado las muestras
físicas de afecto de ese deporte de contacto que es una telenovela.

``No hay besos, no hay abrazos, no hay apapachos y escenas de cama'',
dijo Miguel Ángel Herros, productor ejecutivo de \emph{La rosa de
Guadalupe.}

Cualquier contacto es ``manos solamente y las charlas normalmente máximo
esta distancia'', dijo, y señaló los casi tres metros entre su
escritorio y su asistente.

Herros, de 80 años, ahora filma en períodos más cortos, en locaciones
que permiten mantener un amplio espacio para su equipo. A los actores se
les toma la temperatura cuando llegan al plató y ensayan con cubrebocas
y protectores faciales. Y la cadena ya tuvo que enviar a cuarentena a
una actriz, de la telenovela \emph{Te doy la vida}, después de que dio
positivo por coronavirus.

Image

Miguel Ángel Herros, productor ejecutivo de ``La rosa de
Guadalupe''.Credit...Meghan Dhaliwal para The New York Times

Pero Herros no ve a la epidemia como una amenaza. \emph{La rosa de
Guadalupe} dejó de filmar solo brevemente durante la pandemia, por orden
del gobierno de la ciudad, pero rápidamente reanudó actividades.

``Yo vengo todos los días a la oficina'', dijo Herros, sentado en una
oficina adornada con iconografía religiosa en medio de la amplia sede de
Televisa en San Ángel, justo al sur del centro de Ciudad de México. ``No
hemos parado desde el mes de marzo''.

Por el momento, al menos,Televisa tiene algunas ventajas sobre los
servicios de transmisión en continuo en México. La cadena tiene
contratos a largo plazo con actores y equipos que pueden mantenerse en
entornos estrictamente controlados para contener la propagación del
virus.

Y cuando se trata de ofrecer alimentos reconfortantes a una audiencia
angustiada, el melodrama anticuado de toda la vida no tiene rival.

``A diferencia de Netflix, le damos certeza a la gente'', dijo Carlos
Mercado, creador del programa y su guionista principal. ``Con \emph{La
rosa de Guadalupe} tú tienes la certeza de lo que vas a ver, incluso
para burlarse''.

\begin{center}\rule{0.5\linewidth}{\linethickness}\end{center}

Advertisement

\protect\hyperlink{after-bottom}{Continue reading the main story}

\hypertarget{site-index}{%
\subsection{Site Index}\label{site-index}}

\hypertarget{site-information-navigation}{%
\subsection{Site Information
Navigation}\label{site-information-navigation}}

\begin{itemize}
\tightlist
\item
  \href{https://help.nytimes.com/hc/en-us/articles/115014792127-Copyright-notice}{©~2020~The
  New York Times Company}
\end{itemize}

\begin{itemize}
\tightlist
\item
  \href{https://www.nytco.com/}{NYTCo}
\item
  \href{https://help.nytimes.com/hc/en-us/articles/115015385887-Contact-Us}{Contact
  Us}
\item
  \href{https://www.nytco.com/careers/}{Work with us}
\item
  \href{https://nytmediakit.com/}{Advertise}
\item
  \href{http://www.tbrandstudio.com/}{T Brand Studio}
\item
  \href{https://www.nytimes.com/privacy/cookie-policy\#how-do-i-manage-trackers}{Your
  Ad Choices}
\item
  \href{https://www.nytimes.com/privacy}{Privacy}
\item
  \href{https://help.nytimes.com/hc/en-us/articles/115014893428-Terms-of-service}{Terms
  of Service}
\item
  \href{https://help.nytimes.com/hc/en-us/articles/115014893968-Terms-of-sale}{Terms
  of Sale}
\item
  \href{https://spiderbites.nytimes.com}{Site Map}
\item
  \href{https://help.nytimes.com/hc/en-us}{Help}
\item
  \href{https://www.nytimes.com/subscription?campaignId=37WXW}{Subscriptions}
\end{itemize}
