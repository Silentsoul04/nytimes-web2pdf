Sections

SEARCH

\protect\hyperlink{site-content}{Skip to
content}\protect\hyperlink{site-index}{Skip to site index}

\href{https://www.nytimes.com/es/section/estilos-de-vida}{Estilos de
Vida}

\href{https://myaccount.nytimes.com/auth/login?response_type=cookie\&client_id=vi}{}

\href{https://www.nytimes.com/section/todayspaper}{Today's Paper}

\href{/es/section/estilos-de-vida}{Estilos de Vida}\textbar{}¿Por qué me
dejó ahí?

\url{https://nyti.ms/2Deu8Fo}

\begin{itemize}
\item
\item
\item
\item
\item
\end{itemize}

Advertisement

\protect\hyperlink{after-top}{Continue reading the main story}

Supported by

\protect\hyperlink{after-sponsor}{Continue reading the main story}

Modern Love

\hypertarget{por-quuxe9-me-dejuxf3-ahuxed}{%
\section{¿Por qué me dejó ahí?}\label{por-quuxe9-me-dejuxf3-ahuxed}}

Un joven regresa al orfanato en Vietnam que había tratado de olvidar
durante 25 años.

\includegraphics{https://static01.nyt.com/images/2020/07/26/fashion/26MODERN-ORPHANAGE/26MODERN-ORPHANAGE-articleLarge.jpg?quality=75\&auto=webp\&disable=upscale}

Por Kacey Vu Shap

\begin{itemize}
\item
  2 de agosto de 2020
\item
  \begin{itemize}
  \item
  \item
  \item
  \item
  \item
  \end{itemize}
\end{itemize}

\href{https://www.nytimes.com/2020/07/24/style/modern-love-adoption-vietnam-why-did-she-leave-me-there.html}{Read
in English}

\href{https://www.nytimes.com/newsletters/el-times}{Regístrate para
recibir nuestro boletín} con lo mejor de The New York Times.

\begin{center}\rule{0.5\linewidth}{\linethickness}\end{center}

El portón del orfanato era más pequeño de lo que recordaba. Habían
pasado casi 25 años desde que había vivido ahí. Me preguntaba si
regresar era buena idea.

Mis mejores amigos, Phu, Francis y Will habían planeado este viaje a
Vietnam y me invitaron. Los conocí 15 años atrás cuando estaba en la
secundaria. Ellos estaban en la universidad y habían comenzado un grupo
de apoyo para jóvenes asiáticos homosexuales.

En ese entonces, me había teñido luces rubias en el cabello, usaba
lentes de contacto azules y camisetas de Abercrombie \& Fitch con
pantalones de mezclilla holgados y desgarrados. Phu y Francis pensaban
que me esforzaba demasiado en proyectar cierta imagen, pero me dejaron
juntarme con ellos. Se convirtieron en hermanos sobreprotectores, y me
regañaban por todo.

Mis amigos sabían que había vivido en un orfanato cerca de la ciudad de
Ho Chi Minh y sugirieron que fuéramos de visita como parte de nuestro
viaje. Por si fuera poco, querían hacer una campaña de Kickstarter para
recaudar dinero para los huérfanos que ahora vivían ahí.

Pensé que eso era una locura. Faltaba una semana para irnos. Y no tenía
ganas de regresar a un lugar que había tratado de olvidar durante casi
toda mi vida.

``Esta es una oportunidad que se vive una vez en la vida'', dijo
Francis. ``No estarás solo. Te acompañaremos''.

Lograron convencerme y, en cuestión de días, habíamos recaudado más de
5000 dólares para comprar ropa, juguetes y otros productos básicos.

La semana siguiente, tras descargar los donativos de una camioneta
rentada, entramos al orfanato y nos recibieron pequeños rostros y manos
huesudas extendidas. Qué extraño es pensar que alguna vez fui una de
esas caras aterradas y emocionadas que querían la atención de un extraño
como yo.

En ese entonces, el enorme edificio estaba en ruinas, con la pintura
blanca cubierta de mugre y los muros maltratados. Ya lo habían arreglado
y ampliado, pero una cosa seguía presente: el peculiar olor a talco de
bebé, sudor, orina, putrefacción, desesperanza y desolación.

Aunque tenía cinco años cuando llegué, era demasiado pequeño para estar
con los niños mayores, pero demasiado grande para estar con los bebés,
así que me pusieron con los que tenían deformidades, extremidades
faltantes o enfermedades mentales. Los recuerdos regresaron de pronto
mientras mis amigos y yo entrábamos. Mis ojos comenzaron a llenarse de
lágrimas, el corazón me latía fuerte y empecé a sentir ansiedad. No
tardé en salir corriendo hacia la entrada mientras mis amigos me
llamaban.

De niño, cuando le contaba a la gente que era adoptado, solía decir que
había llegado prefabricado, que simplemente aparecí una noche de verano
en el aeropuerto de Baltimore, donde me recibieron mi mamá, mi papá y mi
hermana, con dulces, flores y besos. Era más fácil embellecer mi
historia y hablar únicamente de mi vida como Kacey, quien era amado y
deseado, que narrar mi vida como Vu, a quien habían abandonado y no
habían querido.

Jamás conocí a mi madre biológica, que murió cuando yo tenía dos años en
la sala de parto junto con mi hermano. Conocí poco a mi padre biológico,
un trabajador migrante que jamás estuvo cerca. Cuando tenía cinco años,
mi hermana mayor se ahogó en un río cerca de la casa de mi abuela. Vi a
tres metros de distancia cómo pataleaba y desaparecía en el agua turbia.

Le había rogado que fuéramos al río a jugar con los otros niños, a pesar
de que mi abuela nos había prohibido ir cuando ella no estaba. Deseé
haber sido yo quien se ahogó ese día.

Después éramos solo mi abuela y yo quienes vivíamos juntos en una
empobrecida aldea agrícola en el sur de Vietnam. Si mi abuela hubiera
sido un gato, yo habría sido su cola, porque, adonde iba, yo la seguía.
Me encantaba estar cerca de ella en la cocina. Las especias exóticas
mezcladas con carne sazonada y hierbas frescas nos cobijaban en su
delicioso abrazo mientras yo atosigaba a mi abuela con preguntas acerca
de nuestro tema favorito: mi madre.

``Abuela, tú tienes mis ojos, mi nariz y mis mejillas'', le decía.
``¿Crees que mi madre también se veía como yo?''.

``Claro que sí, tontito. ¿Quién crees que le dio a tu madre esos rasgos
tan hermosos?''. Me enseñaba su sonrisa chimuela. Después dejaba de
picar vegetales y decía: ``¿Te cuento algo muy secreto? Tu madre era la
favorita de mis hijos. Siempre trataba de hacer reír a todos. Quiero que
seas bueno, como tu madre. ¿Sí?''.

``Está bien'', le respondía.

Después de la muerte de mi hermana, me enteré de que mi padre también
había muerto, y no pasó mucho tiempo antes de que mi abuela me pidiera
que empacara mis cosas para ir de viaje. Estaba extasiado, pues jamás
había salido de viaje.

Llegamos a un enorme edificio blanco lleno de niños. Después de dar un
recorrido por el lugar, parecía que mi abuela no quería irse. Al final,
se agachó y me dijo: ``Me iré a casa, pero tú te quedarás aquí''.

Me quedé ahí paralizado.

Mi abuela cubrió mis mejillas con sus manos curtidas y levantó mi rostro
hacia el suyo. Su mirada, que siempre era intensa, estaba llena de
tristeza. Se quitó un pañuelo con estampado floral del cuello y me lo
puso. Era su favorito, empapado de su perfume de siempre. Después se
levantó y se alejó sin mirar atrás.

Traté de seguirla, pero unas manos fuertes me tomaron. Le grité a mi
abuela y le rogué que me llevara a casa. Tras su partida, esperé durante
días en el portón de la entrada, con la esperanza de su regreso.

Algunos meses después, una pareja judía del norte de Virginia estaba en
las últimas etapas de una adopción que no se concretó. Devastados, a
punto de darse por vencidos, recibieron mi foto, que les había enviado
la agencia de adopción, y decidieron que querían que fuera su hijo, un
proceso difícil que tomó dos años. Yo no sabía nada de mi adopción hasta
el día en que me llevaron al aeropuerto. Después me enteré de que, entre
los cientos de niños del orfanato, solo un puñado llegaba a Estados
Unidos. La mayoría eran bebés. Yo tenía 7 años.

Ha pasado un cuarto de siglo desde que mi abuela me abandonó ese día.
Aún llevo el pañuelo conmigo adonde voy, pero ya no tiene su perfume.
Hay muchas cosas que he querido contarle acerca de mi vida en Estados
Unidos: mis padres amorosos, mis amigos, mi perro, el departamento de
Los Ángeles y el título de doctor en Psicología Social que acabo de
obtener. También hay muchas preguntas que he querido hacerle.

Cuando le decía a la gente que era adoptado, no les contaba sobre el día
en que me abandonaron, por miedo a que mis amigos y mi familia
descubrieran que no valía nada y que por eso me lo había merecido.

Ahora, mis amigos lo habían visto. Lo sabían. Cuando salieron y me
encontraron cerca del portón, me preguntaron por qué me había ido de
pronto.

``Sabía que, en cuanto vieran mi orfanato, me verían como alguien
inferior y ya no querrían ser mis amigos'', les dije con demasiada
prisa.

``¿Es en serio?'', dijo Phu. ``Viajamos al otro lado del mundo, llenos
de picaduras de mosquitos, empapados de sudor, ¿y a ti te preocupa que
creamos que eres inferior? Nos has sometido a cosas peores. Están el
Kacey que siempre llega tarde, el Kacey testarudo y el Kacey que quiere
estar con hombres que no están disponibles emocionalmente. Si todo eso
no nos alejó, nada lo hará''.

Mis amigos me rodearon y me envolvieron en sus brazos.

``Eres nuestra familia'', dijo Francis. ``Te amamos. Además, tu amistad
es como el herpes. Es muy contagiosa, fácilmente tratable, pero
imposible de eliminar. Y ya llevamos así 15 años''.

Después, Will dijo: ``Y quizá tu abuela sí te amaba. Tal vez dejarte ahí
fue su último acto de amor para que tuvieras la oportunidad de tener una
mejor vida''.

Es algo que me he preguntado durante mucho tiempo. ¿Me dejó porque era
una carga o para que no sufriera una vida brutal de pobreza?

Después mis amigos me dijeron que, mientras yo estaba afuera, pudieron
encontrar la última dirección registrada de mi abuela en los expedientes
del orfanato. Era probable que todavía viviera ahí, tan solo a media
hora de distancia.

Si mi abuela aún vivía ahí, podría tener mi respuesta. Lo pensé, y
también pensé en el amor y el apoyo de mis amigos, mi familia y otras
personas que habían vuelto esto posible.

``No'', les dije. ``No necesito saber su dirección. Podemos irnos
ahora''. Por primera vez, pude elegir no ser definido por el abandono.

Luego de eso, nos fuimos del orfanato y fuimos a explorar la ciudad de
Ho Chi Minh, donde el olor dulce del cerdo asado se mezclaba con la risa
de los niños que se correteaban, como si las calles fueran un patio
gigantesco de juegos.

Kacey Vu Shap es investigador, escritor y emprendedor social en Los
Ángeles.

Advertisement

\protect\hyperlink{after-bottom}{Continue reading the main story}

\hypertarget{site-index}{%
\subsection{Site Index}\label{site-index}}

\hypertarget{site-information-navigation}{%
\subsection{Site Information
Navigation}\label{site-information-navigation}}

\begin{itemize}
\tightlist
\item
  \href{https://help.nytimes.com/hc/en-us/articles/115014792127-Copyright-notice}{©~2020~The
  New York Times Company}
\end{itemize}

\begin{itemize}
\tightlist
\item
  \href{https://www.nytco.com/}{NYTCo}
\item
  \href{https://help.nytimes.com/hc/en-us/articles/115015385887-Contact-Us}{Contact
  Us}
\item
  \href{https://www.nytco.com/careers/}{Work with us}
\item
  \href{https://nytmediakit.com/}{Advertise}
\item
  \href{http://www.tbrandstudio.com/}{T Brand Studio}
\item
  \href{https://www.nytimes.com/privacy/cookie-policy\#how-do-i-manage-trackers}{Your
  Ad Choices}
\item
  \href{https://www.nytimes.com/privacy}{Privacy}
\item
  \href{https://help.nytimes.com/hc/en-us/articles/115014893428-Terms-of-service}{Terms
  of Service}
\item
  \href{https://help.nytimes.com/hc/en-us/articles/115014893968-Terms-of-sale}{Terms
  of Sale}
\item
  \href{https://spiderbites.nytimes.com}{Site Map}
\item
  \href{https://help.nytimes.com/hc/en-us}{Help}
\item
  \href{https://www.nytimes.com/subscription?campaignId=37WXW}{Subscriptions}
\end{itemize}
