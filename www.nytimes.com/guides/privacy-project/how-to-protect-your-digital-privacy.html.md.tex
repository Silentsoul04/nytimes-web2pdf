\href{https://www.nytimes.com/interactive/2019/opinion/internet-privacy-project.html}{The
Privacy Project}

How to Protect Your Digital Privacy

Intro

\begin{itemize}
\tightlist
\item
  Use These Tools
\item
  Good Practices
\item
  In Case of Emergency
\end{itemize}

Open share options

Bookmark the page

Save for Later

Close your current view

Cancel

Share

\begin{itemize}
\tightlist
\item
  Facebook Icon
\item
  Twitter Icon
\item
  Pinterest Icon
\item
  Email Icon
\end{itemize}

\begin{itemize}
\tightlist
\item
  Use These Tools
\item
  Good Practices
\item
  In Case of Emergency
\end{itemize}

\hypertarget{how-to-protect-your-digital-privacy}{%
\section{How to Protect Your Digital
Privacy}\label{how-to-protect-your-digital-privacy}}

By Thorin Klosowski

Illustrations by Jon Han

Share on Facebook

Share on Twitter

Share in an email

Bookmark the page

By making a few simple changes to your devices and accounts, you can
maintain security against outside parties' unwanted attempts to access
your data as well as protect your privacy from those you don't consent
to sharing your information with. Getting started is easy. Here's a
guide to the few simple changes you can make to protect yourself and
your information online.~

\hypertarget{use-these-tools}{%
\subsection{Use These Tools}\label{use-these-tools}}

Make smart use of the tools available to keep your data safe.~

\hypertarget{secure-your-accounts}{%
\subsubsection{Secure your accounts}\label{secure-your-accounts}}

\textbf{Why:} In the past decade, data breaches and password leaks have
struck companies such as
\href{https://www.nytimes.com/interactive/2017/your-money/equifax-data-breach-credit.html\#first}{Equifax},
\href{https://www.nytimes.com/2019/03/21/technology/personaltech/facebook-passwords.html}{Facebook},
\href{https://bits.blogs.nytimes.com/2014/09/08/home-depot-confirms-that-it-was-hacked/}{Home
Depot},
\href{https://www.nytimes.com/2018/11/30/business/marriott-data-breach.html}{Marriott},
\href{https://www.nytimes.com/2014/01/18/business/a-sneaky-path-into-target-customers-wallets.html}{Target},
\href{https://www.nytimes.com/2017/10/03/technology/yahoo-hack-3-billion-users.html}{Yahoo},
and countless others. If you have online accounts, hackers have likely
leaked data from at least one of them. Want to know which of your
accounts have been compromised? Search for your email address on
\href{https://haveibeenpwned.com/}{Have I Been Pwned?} to
cross-reference your email address with hundreds of data breaches.~

\textbf{How:} Everyone should use a password manager to generate and
remember different, complex passwords for every account --- this is the
most important thing people can do to protect their privacy and security
today.
\href{https://thewirecutter.com/reviews/best-password-managers/}{Wirecutter's
favorite password managers} are
\href{https://www.lastpass.com/}{LastPass} and
\href{https://1password.com/}{1Password}. Both can generate passwords,
monitor accounts for security breaches, suggest changing weak passwords,
and sync your passwords between your computer and phone. Password
managers seem intimidating to set up, but once you've installed one you
just need to browse the Internet as usual. As you log in to accounts,
the password manager saves your passwords and suggests changing weak or
duplicate passwords. Over the course of a couple of weeks, you end up
with new passwords for most of your accounts. Take this time to also
change the default passwords for any devices in your house --- if your
home router, smart light bulbs, or security cameras are still using
``password'' or ``1234'' as the password, change them.

Everyone should also use
\href{https://www.nytimes.com/2019/03/27/technology/personaltech/two-step-authentication.html}{two-step
authentication} whenever possible for their online accounts. Most banks
and major social networks provide this option. As the name suggests,
two-step authentication requires two steps: entering your password and
entering a number only you can access. For example, step one is logging
in to Facebook with your username and password. In step two, Facebook
sends a temporary code to you in a text message or, even better, through
an app like Google Authenticator, and you enter that code to log in.~

\hypertarget{protect-your-web-browsing}{%
\subsubsection{Protect your Web
browsing}\label{protect-your-web-browsing}}

\textbf{Why:} Companies and websites track everything you do online.
Every ad, social network button, and website collects information about
your location, browsing habits, and more. The data collected
\href{https://www.nytimes.com/2018/08/15/technology/personaltech/stop-targeted-stalker-ads.html}{reveals
more about you} than you might expect. You might think yourself clever
for never tweeting your medical problems or sharing all your religious
beliefs on Facebook, for instance, but chances are good that the
websites you visit regularly provide all the data advertisers need to
pinpoint the type of person you are. This is part of how targeted ads
remain one of the Internet's most unsettling innovations.

\textbf{How:} A browser extension like
\href{https://github.com/gorhill/uBlock}{uBlock Origin} blocks ads and
the data they collect. The uBlock Origin extension also prevents malware
from running in your browser and gives you an easy way to
\href{https://github.com/gorhill/uBlock/wiki/How-to-whitelist-a-web-site}{turn
the ad blocking off} when you want to support sites you know are secure.
Combine uBlock with \href{https://www.eff.org/privacybadger}{Privacy
Badger}, which blocks trackers, and ads won't follow you around as much.
To slow down stalker ads even more, disable interest-based ads from
\href{https://support.apple.com/en-us/HT202074}{Apple},
\href{https://www.facebook.com/help/568137493302217}{Facebook},
\href{https://support.google.com/ads/answer/2662922?hl=en}{Google}, and
\href{https://business.twitter.com/en/help/ads-policies/other-policy-requirements/interest-based-opt-out-policy.html}{Twitter}.
A lot of websites offer means to opt out of data collection, but you
need to do so manually. \href{http://simpleoptout.com/}{Simple Opt Out}
has direct links to opt-out instructions for major sites like Netflix,
Reddit, and more. Doing this won't eliminate the problem completely, but
it will significantly cut down the amount of data collected.

You should also install the
\href{https://www.eff.org/https-everywhere/faq}{HTTPS Everywhere}
extension. HTTPS Everywhere automatically directs you to the secure
version of a site when the site supports that, making it difficult for
an attacker --- especially if you're on public Wi-Fi at a coffee shop,
airport, or hotel --- to digitally eavesdrop on what you're doing.

Some people may want to use a virtual private network (VPN), but it's
not necessary for everyone. If you frequently connect to public Wi-Fi, a
VPN is useful because it adds a layer of security to your browsing when
HTTPS isn't available. It can also provide some privacy from your
Internet service provider and help minimize tracking based on your IP
address. But all your Internet activity still flows through the VPN
provider's servers, so in using a VPN you're choosing to trust that
company over your ISP not to store or sell your data. Make sure you
\href{https://thewirecutter.com/reviews/what-is-a-vpn/}{understand the
pros and cons} first, but if you want a VPN,
\href{https://thewirecutter.com/reviews/best-vpn-service/}{Wirecutter
recommends IVPN}.

\hypertarget{use-antivirus-software-on-your-computer}{%
\subsubsection{Use antivirus software on your
computer}\label{use-antivirus-software-on-your-computer}}

\textbf{Why:} Viruses might not seem as common as they were a decade
ago, but they still exist. Malicious software on your computer can wreak
all kinds of havoc, from annoying pop-ups to
\href{https://www.zdnet.com/article/cryptocurrency-mining-malware-why-it-is-such-a-menace-and-where-its-going-next/}{covert
bitcoin mining} to scanning for personal information. If you're at risk
for clicking perilous links, or if you share a computer with multiple
people in a household, it's worthwhile to set up antivirus software,
especially on Windows computers.~

\textbf{How:} If your computer runs Windows 10, you should use
Microsoft's built-in software,
\href{https://support.microsoft.com/en-us/help/17464/windows-10-help-protect-my-device-with-windows-security}{Windows
Defender}. Windows Defender offers plenty of security for most people,
and it's the main antivirus option that
\href{https://thewirecutter.com/blog/best-antivirus/}{Wirecutter
recommends}; we reached that conclusion after speaking with several
experts. If you run an older version of Windows (even though we
recommend updating to Windows 10) or you use a shared computer, a second
layer of protection might be necessary. For this purpose,
\href{https://www.malwarebytes.com/}{Malwarebytes Premium} is your best
bet. Malwarebytes is unintrusive, it works well with Windows Defender,
and it doesn't push out dozens of annoying notifications like most
antivirus utilities tend to do.

Mac users are typically okay with the protections included in macOS,
especially if you download software only from Apple's App Store and
stick to well-known browser extensions. If you do want a second layer of
security, Malwarebytes Premium is also available for Mac. You should
avoid antivirus applications on your phone altogether and stick to
downloading trusted apps from official stores.~

\hypertarget{read-more-from-the-privacy-project}{%
\subsubsection{Read More from The Privacy
Project}\label{read-more-from-the-privacy-project}}

\href{https://www.nytimes.com/2019/06/20/opinion/queer-dating-apps.html}{}

\hypertarget{queer-dating-apps-are-unsafe-by-design}{%
\paragraph{Queer Dating Apps Are Unsafe by
Design}\label{queer-dating-apps-are-unsafe-by-design}}

June 20, 2019

\href{https://www.nytimes.com/2019/06/19/opinion/facebook-currency-libra.html}{}

\hypertarget{launching-a-global-currency-is-a-bold-bad-move-for-facebook}{%
\paragraph{Launching a Global Currency Is a Bold, Bad Move for
Facebook}\label{launching-a-global-currency-is-a-bold-bad-move-for-facebook}}

June 19, 2019

\href{https://www.nytimes.com/2019/06/19/opinion/facebook-google-privacy.html}{}

\hypertarget{what-if-we-all-just-sold-non-creepy-advertising}{%
\paragraph{What if We All Just Sold Non-Creepy
Advertising?}\label{what-if-we-all-just-sold-non-creepy-advertising}}

June 19, 2019

\href{https://www.nytimes.com/2019/06/18/opinion/facebook-court-privacy.html}{}

\hypertarget{facebook-under-oath-you-have-no-expectation-of-privacy}{%
\paragraph{Facebook Under Oath: You Have No Expectation of
Privacy}\label{facebook-under-oath-you-have-no-expectation-of-privacy}}

June 18, 2019

\hypertarget{good-practices}{%
\subsection{Good Practices}\label{good-practices}}

Adopt healthy internet habits to ensure that you don't leave yourself
prone.~

\hypertarget{update-your-software-and-devices}{%
\subsubsection{Update your software and
devices}\label{update-your-software-and-devices}}

\textbf{Why:} Phone and computer operating systems, Web browsers,
popular apps, and even smart-home devices receive frequent updates with
new features and security improvements. These security updates are
typically far better at thwarting hackers than antivirus software.

\textbf{How:} All three major operating systems can update
automatically, but you should take a moment to double-check that you
have automatic updates enabled for your OS of choice:
\href{https://support.microsoft.com/en-us/help/12373/windows-update-faq}{Windows},
\href{https://support.apple.com/guide/mac-help/get-macos-updates-mchlpx1065/mac}{macOS},
or
\href{https://support.google.com/chrome/a/answer/3168106?hl=en}{Chrome
OS}. Although it's frustrating to turn your computer on and have to wait
out an update that
\href{https://www.nytimes.com/2017/02/24/technology/personaltech/automatic-update-headaches.html}{might
break the software you use}, the
\href{https://www.nytimes.com/2019/03/27/opinion/asus-malware-hack.html}{security
benefits are worth the trouble}. These updates include new versions of
Microsoft's Edge browser and Apple's Safari. Most third-party Web
browsers, including Google's Chrome and Mozilla Firefox, also update
automatically. If you tend to leave your browser open all the time,
remember to reboot it now and again to get those updates. Your phone
also has automatic-update options. On Apple's iPhone, enable automatic
updates under \emph{Settings \textgreater{} General \textgreater{}
Software Update}. On Google's Android operating system, security updates
should happen automatically, but you can double-check by opening up
\emph{Settings \textgreater{} System \textgreater{} Advanced
\textgreater{} System Update}.

For third-party software and apps, you may need to find and enable a
\emph{Check for updates} option in the software's settings. Smart-home
devices such as cameras, thermostats, and light bulbs can receive
updates to the app as well as to the hardware itself. Check the settings
using the device's app to make sure these updates happen automatically;
if you don't find an automatic-update option, you may have to manually
reboot the device on occasion (a monthly calendar reminder might help).

\hypertarget{dont-install-sketchy-software}{%
\subsubsection{Don't install sketchy
software}\label{dont-install-sketchy-software}}

\textbf{Why:} Every weird app you install on your phone and every
browser extension or piece of software you download from a sketchy
website represents another potential privacy and security hole.
Countless mobile apps track your
\href{https://www.nytimes.com/interactive/2018/12/10/business/location-data-privacy-apps.html}{location
everywhere you go} and
\href{https://www.nytimes.com/2018/05/19/technology/phone-apps-stalking.html}{harvest
your data} without asking consent,
\href{https://www.nytimes.com/interactive/2018/09/12/technology/kids-apps-data-privacy-google-twitter.html}{even
in children's apps}.~

\textbf{How:} Stop downloading garbage software, and stick to
downloading programs and browser extensions directly from their makers
and official app stores. You don't need half the apps on your phone, and
\href{https://www.nytimes.com/2019/04/18/smarter-living/wirecutter/declutter-speed-up-phone.html}{getting
rid of what you don't need} can make your phone feel faster. Once you
clear out the apps you don't use, audit the privacy permissions of
what's left. If you have an iPhone, open \emph{Settings} and tap the
\emph{Privacy} option. On Android, head to \emph{Settings \textgreater{}
Apps}, and then tap the gear icon and select \emph{App Permissions}.
Here, you can see which apps have
\href{https://www.nytimes.com/2018/12/10/technology/prevent-location-data-sharing.html}{access
to your location}, contacts, microphone, and other data. Disable
permissions where they don't make sense---for example, Google Maps needs
your location to function, but your notes app doesn't. In the future,
think about app permissions as you install new software; if an app is
free, it's possibly collecting and selling your data.

The same rules go for your computer. If you're not sure what to delete
from your Windows computer,
\href{https://www.shouldiremoveit.com/}{Should I Remove It?} can help
you choose. (Yes, it's more software, but you should delete it after
you're done using it.) Mac users don't have an equivalent, but all
software resides in the Applications folder, so it's easy to sift
through. If you find an app you don't remember installing, search for it
on Google, and then drag it to the trash to delete it if you don't need
it.

\hypertarget{the-privacy-project}{%
\subsection{The Privacy Project}\label{the-privacy-project}}

Technology has made our lives easier. But it also means that your data
is no longer your own. We'll examine who is hoarding your information
--- and give you a guide for what you can do about it.

Your email address

Sign up

\href{https://www.nytimes.com/newsletters/sample/real-estate}{See
sample} \textbar{}
\href{https://www.nytimes.com/content/help/rights/privacy/policy/privacy-policy.html}{Privacy
Policy} \textbar{} Opt out or
\href{https://www.nytimes.com/help/index.html}{contact us} anytime

\hypertarget{in-case-of-emergency}{%
\subsection{In Case of Emergency}\label{in-case-of-emergency}}

Think ahead just in case you lose your phone or computer.~

\hypertarget{lock-down-your-phone-in-case-you-lose-it}{%
\subsubsection{Lock down your phone in case you lose
it}\label{lock-down-your-phone-in-case-you-lose-it}}

\textbf{Why:} You need to ensure nobody can get into your phone if you
lose it or someone steals it. Smartphones are encrypted by default,
which is great, but you still need to take a few steps to ensure your
phone is properly locked down if it disappears.~

\textbf{How:} You have two main defenses here. The first is to use a
strong passcode alongside your biometric (fingerprint or face) login.
The second is to set up your phone's remote-tracking feature. If you
haven't taken the first step, set up a PIN number or pattern, and enable
the biometric login on your phone. You can find these options on an
iPhone under \emph{Settings \textgreater{} Face ID \& Passcode} or
\emph{Touch ID \& Passcode}, and on an Android phone under
\emph{Settings \textgreater{} Security and location}.

Next, set up your phone's remote-tracking feature. If you lose your
phone, you'll be able to see where it is, and you can remotely delete
everything on the phone if you can't recover it. On an iPhone, head to
\emph{Settings}, tap your name, and then go to \emph{iCloud
\textgreater{} Find My iPhone}. On an Android phone, tap \emph{Settings
\textgreater{} Security \& location} and enable \emph{Find My Device}.~

\hypertarget{enable-encryption-on-your-laptop-its-easier-than-it-sounds}{%
\subsubsection{Enable encryption on your laptop (it's easier than it
sounds)}\label{enable-encryption-on-your-laptop-its-easier-than-it-sounds}}

\textbf{Why:} If you lose your laptop or someone steals it, the thief
gets both a sweet new piece of hardware and access to your data. Even
without your password, a thief can usually still copy files off the
laptop if they know what they're doing. If a stranger poked around your
laptop, they might get a look at all your photos, say, or your tax
returns, or maybe an unfinished bit of \emph{Game of Thrones}
fanfiction.~

\textbf{How:} When you encrypt the storage drive on your laptop, your
password and a security key protect your data; without your password or
the key, the data becomes nonsense. Although encryption might sound like
something from a high-tech spy movie, it's simple and free to enable
with built-in software. Follow
\href{https://www.nytimes.com/2018/03/13/smarter-living/how-to-encrypt-your-computers-data.html}{these
directions on how to set up encryption} on both Windows and Mac.~

Speaking of computer theft, if you store a lot of data on your computer,
it's worth the effort to
\href{https://thewirecutter.com/reviews/how-to-back-up-your-computer/}{back
it up securely}. For this purpose,
\href{https://thewirecutter.com/reviews/best-online-backup-service/}{Wirecutter
likes} the online backup service
\href{https://www.backblaze.com/cloud-backup.html}{Backblaze}, which
\href{https://www.backblaze.com/backup-encryption.html}{encrypts all its
data} in a way that even the folks at Backblaze don't have access to
it.~

\hypertarget{the-importance-of-paranoia}{%
\subsubsection{The Importance of
Paranoia}\label{the-importance-of-paranoia}}

Ultimately, security and privacy are linked, so you need to get in the
habit of protecting both. It might seem like a time-consuming,
overwhelming headache, but once you follow these steps, all that's left
is to cultivate your judgment and establish good online behaviors.

Be suspicious of links in emails and on social media. Make your accounts
private and don't share anything you wouldn't mind getting out anyway.
Keep your main email address and phone number relatively private. Use a
burner email account you don't care about for shopping and other online
activities; that way, if an account is hacked, it's not linked to an
important personal account, like that of your bank. Likewise, avoid
using your real name and number when you have to sign up for a service
you don't care about, such as
\href{https://www.nytimes.com/2013/03/26/technology/facebook-expands-targeted-advertising-through-outside-data-sources.html}{discount
cards at a grocery store} (your area code plus
\href{https://en.wikipedia.org/wiki/867-5309/Jenny}{Jenny's number}
usually gets you whatever club-card discount a retailer offers). Don't
link together services, like Facebook and Spotify, or Twitter and
Instagram, unless you gain a useful feature from doing so. Don't buy
Internet of Things devices unless you're willing to give up a little
privacy for whatever convenience they provide.~

Once you settle into a low-key, distrustful paranoia about new apps and
services, you're well on your way to avoiding many privacy-invading
practices.

\hypertarget{keep-reading-about-digital-datas-impact}{%
\subsubsection{Keep Reading about Digital Data's
Impact}\label{keep-reading-about-digital-datas-impact}}

\href{https://www.nytimes.com/2019/05/20/opinion/car-repair-data-privacy.html}{}

\hypertarget{your-car-knows-when-you-gain-weight}{%
\paragraph{Your Car Knows When You Gain
Weight}\label{your-car-knows-when-you-gain-weight}}

May 20, 2019

\href{https://www.nytimes.com/2019/06/13/opinion/privacy-law-enforcment-congress.html}{}

\hypertarget{im-a-judge-heres-how-surveillance-is-challenging-our-legal-system}{%
\paragraph{I'm a Judge. Here's How Surveillance Is Challenging Our Legal
System.}\label{im-a-judge-heres-how-surveillance-is-challenging-our-legal-system}}

June 13, 2019

\href{https://www.nytimes.com/2019/04/30/opinion/police-phone-privacy.html}{}

\hypertarget{would-you-let-the-police-search-your-phone}{%
\paragraph{Would You Let the Police Search Your
Phone?}\label{would-you-let-the-police-search-your-phone}}

May 1, 2019

\href{https://www.nytimes.com/2019/02/27/technology/personaltech/digital-footprint-surveillance.html}{}

\hypertarget{limiting-your-digital-footprints-in-a-surveillance-state}{%
\paragraph{Limiting Your Digital Footprints in a Surveillance
State}\label{limiting-your-digital-footprints-in-a-surveillance-state}}

February 27, 2019

\hypertarget{about-the-author}{%
\subsection{About the Author}\label{about-the-author}}

Thorin Klosowski is a staff writer for Wirecutter who focuses on DIY
electronics, technology, and security.~

\begin{itemize}
\tightlist
\item
  \href{https://www.nytimes.com}{NYTimes.com}
\item
  \href{//www.nytimes.com/spotlight/guides}{Guides}
\item
  \href{https://nyt.qualtrics.com/jfe/form/SV_7VuAQJbpWqaxzUh?AGENT_ID=}{Send
  us Feedback}
\end{itemize}

\begin{itemize}
\tightlist
\item
  \href{http://www.nytimes.com/privacy}{Privacy Policy}
\item
  \href{http://www.nytimes.com/ref/membercenter/help/agree.html}{Terms
  of Service}
\item
  \href{http://www.nytimes.com/content/help/rights/copyright/copyright-notice.html}{©
  The New York Times Company}
\end{itemize}
