 **NYTimes.com no longer supports Internet Explorer 9 or earlier. Please
upgrade your browser.
\href{http://www.nytimes.com/content/help/site/ie9-support.html}{LEARN
MORE »}

**Sections

**Home

**Search

\hypertarget{the-new-york-times}{%
\subsection{\texorpdfstring{\href{http://www.nytimes.com/}{The New York
Times}}{The New York Times}}\label{the-new-york-times}}

 \href{https://www.nytimes.com/section/technology}{Technology}
\textbar{}How Game Apps That Captivate Kids Have Been Collecting Their
Data

**Close search

\hypertarget{site-search-navigation}{%
\subsection{Site Search Navigation}\label{site-search-navigation}}

Search NYTimes.com

**Clear this text input

Go

\url{https://nyti.ms/2MpFuoG}

\hypertarget{site-navigation}{%
\subsection{Site Navigation}\label{site-navigation}}

\hypertarget{site-mobile-navigation}{%
\subsection{Site Mobile Navigation}\label{site-mobile-navigation}}

\hypertarget{how-game-apps-that-captivate-kids-have-been-collecting-their-data}{%
\section{How Game Apps That Captivate Kids Have Been Collecting Their
Data}\label{how-game-apps-that-captivate-kids-have-been-collecting-their-data}}

A lawsuit by New Mexico's attorney general accuses a popular app maker,
as well as online ad businesses run by Google and Twitter, of violating
children's privacy law.

\hypertarget{how-game-apps-that-captivate-kids-have-been-collecting-their-data-1}{%
\section{How Game Apps That Captivate Kids Have Been Collecting Their
Data}\label{how-game-apps-that-captivate-kids-have-been-collecting-their-data-1}}

A lawsuit by New Mexico's attorney general accuses a popular app maker,
as well as online ad businesses run by Google and Twitter, of violating
children's privacy law.

By JENNIFER VALENTINO-DeVRIES,
\href{http://www.nytimes.com/by/natasha-singer}{NATASHA SINGER}, AARON
KROLIK and MICHAEL H. KELLER SEPT. 12, 2018

Before Kim Slingerland downloaded the Fun Kid Racing app for her
then-5-year-old son, Shane, she checked to make sure it was in the
family section of the Google Play store and rated as age-appropriate.
The game, which lets children race cartoon cars with animal drivers, has
been downloaded millions of times.

Until last month, the app also shared users' data, sometimes including
the precise location of devices, with more than a half-dozen advertising
and online tracking companies. On Tuesday evening, New Mexico's attorney
general filed a lawsuit claiming that the maker of Fun Kid Racing had
violated a federal children's privacy law through dozens of Android apps
that shared children's data.

``I don't think it's right,'' said Ms. Slingerland, a mother of three in
Alberta, Canada. ``I don't think that's any of their business, location
or anything like that.''

The suit accuses the app maker, Tiny Lab Productions, along with online
ad businesses run by Google, Twitter and three other companies, of
flouting a law intended to prevent the personal data of children under
13 from falling into the hands of predators, hackers and manipulative
marketers. The suit also contends that Google misled consumers by
including the apps in the family section of its store.

\emph{{[}Read the
\href{https://int.nyt.com/data/documenthelper/295-new-mexico-kid-apps-complaint/206d4ea39896e264fe3a/optimized/full.pdf\#page=1?action=click\&module=Intentional\&pgtype=Article}{full
complaint}{]}}

An analysis by The New York Times found that children's apps by other
developers were also collecting data. The review of 20 children's apps
--- 10 each on Google Android and Apple iOS --- found examples on both
platforms that sent data to tracking companies, potentially violating
children's privacy law; the iOS apps sent less data over all.

These findings are consistent with those
\href{https://blues.cs.berkeley.edu/wp-content/uploads/2018/04/popets-2018-0021.pdf}{published
this spring} by academic researchers who analyzed nearly 6,000 free
children's Android apps. They reported that more than half of the apps,
including those by Tiny Lab, shared details with outside companies in
ways that may have violated the law.

Although federal law doesn't provide many digital privacy protections
for adults, there are safeguards for children under 13. The
\href{https://www.ftc.gov/tips-advice/business-center/privacy-and-security/children\%27s-privacy}{Children's
Online Privacy Protection Act} protects them from being improperly
tracked, including for advertising purposes. Without explicit,
verifiable permission from parents, children's sites and apps are
prohibited from collecting personal details including names, email
addresses, geolocation data and tracking codes like ``cookies'' if
they're used for targeted ads.

But the New Mexico lawsuit and the analyses of children's apps suggest
that some app developers, ad tech companies and app stores are falling
short in protecting children's privacy.

``These sophisticated tech companies are not policing themselves,'' the
New Mexico attorney general, Hector Balderas, said. ``The children of
this country ultimately pay the price.''

Jessica Rich, a former consumer protection director at the Federal Trade
Commission, called the findings ``significant and disturbing.'' They
suggest, she said, ``that the `safe spaces' for kids in the apps stores
aren't safe at all.''

A Google spokesman, Aaron Stein, said that developers are responsible
for declaring whether their apps are primarily for children, and that
apps in the store's family section ``must comply with more stringent
policies.''

A Twitter spokesman said that the company's ad platform, MoPub, does not
allow its services to be used to collect information from children's
apps for targeted advertising and that it suspended the maker of Fun Kid
Racing in September of 2017 for violating its policies.

Jonas Abromaitis, founder of the Lithuania-based Tiny Lab, said he
believed he had followed the law and Google's requirements, because the
app asked for users' ages and tracked those who identified as over 13.
``We thought we were doing everything the right way,'' he said.

\hypertarget{a-market-for-tracking}{%
\paragraph{A Market for Tracking}\label{a-market-for-tracking}}

Dozens of companies now track consumers on their phones to build
behavioral profiles that help tailor the ads they see. Two of the
largest are AdMob and MoPub.

To make money, app developers generally have two options: publish free
apps supported by ads, or charge users. But children don't have the
money to make purchases, and under federal law they can't be tracked for
ad targeting.

\hypertarget{how-companies-track-childrens-personal-data-to-target-ads}{%
\subsection{How companies track children's personal data to target
ads}\label{how-companies-track-childrens-personal-data-to-target-ads}}

By Anjali Singhvi and Rich Harris \textbar{} Source:

The app industry has had trouble adapting to children, said Dylan
Collins, the chief executive of
\href{https://www.superawesome.tv/about-us}{SuperAwesome, a technology
firm} that helps companies build apps for children without tracking
them.

Mr. Collins said some top children's app makers had started charging
parents for subscriptions or showing ads that didn't use tracking. But,
he noted, small developers typically sell fewer subscriptions and don't
always sell enough ads using only child-friendly ad networks. ``As a
result, there's still a huge amount of data being collected on kids,''
he said.

In 2013 Apple introduced a children's section in its App Store. It
\href{https://asciiwwdc.com/2014/sessions/717}{told developers} that, to
be listed there, they could ``do no tracking across sites or across
apps.'' Apple \href{https://www.apple.com/families/}{tells parents} that
it reviews each app in the section ``to make sure it does what it says
it does.''

Google introduced a similar program, Designed for Families, in 2015. The
company
\href{https://web.archive.org/web/20180526145142/https://developer.android.com/distribute/google-play/families}{informed
Android developers} that apps that were ``primarily child-directed must
participate'' in the program and that developers must confirm that their
apps complied with the children's privacy law. Google has said it
developed its family section to help parents find
\href{https://web.archive.org/web/20180507231711/https://developer.android.com/distribute/google-play/families}{``suitable,
trusted, high-quality apps''} for their children.

\hypertarget{for-children-vs-for-families}{%
\paragraph{`For Children' vs. `for
Families'}\label{for-children-vs-for-families}}

Mr. Abromaitis, the Tiny Lab founder, created Fun Kid Racing in 2013,
after searching unsuccessfully for a racing game to play with his
3-year-old nephew. Other Tiny Lab apps include simple games with titles
such as Run Cute Little Pony.

Still, Mr. Abromaitis said in an interview, the company's apps were
directed at ``mixed audiences,'' with children under 13 forming only
part of the market.

The distinction is important: Under privacy law, apps aimed at younger
children are prohibited from tracking any users for ads without parental
consent, but those intended for a general audience can ask players their
age and track older users.

When Tiny Lab submitted apps to Google's store, it indicated they were
for families, not just children, and Google accepted the apps.

In The Times's tests of Fun Kid Racing in July, the app asked that
players select their birth year from a list. But with the default set
between 2000 and 2001, a young child eager to get to the next screen
could simply tap through quickly and be counted as a teenager. In the
tests, the app didn't collect location data if the player identified as
under 13.

In early June, emails show, the
\href{https://www.appcensus.mobi/}{academic researchers} who had done
the earlier study informed Google that app developers ``seem to have an
incentive to mischaracterize'' their children's apps as ``not primarily
directed to children,'' freeing them to track users for targeted ads.
They cited 84 apps from Tiny Lab as examples and said they had
identified nearly 3,000 apps in all that appeared to be similarly
mislabeled.

In July, a Google manager responded that the company had investigated
the Tiny Lab apps and had found they had not violated the privacy law.
Google, he said, did not consider ``these apps to be designed primarily
for children, but for families in general.''

A month later, Google appeared to reverse course: The company told Mr.
Abromaitis it had identified a Tiny Lab app that should be designated
for children. Google gave Tiny Lab a week to change that app and any
others like it. Tiny Lab labeled 10 of its apps for children and used ad
networks in them designed for children's apps. Google approved the
updates but flagged more apps at the end of August, Mr. Abromaitis said,
so he made another round of changes.

Then, this week, after inquiries from The Times, Google terminated Tiny
Lab's account and removed all of its apps from the Play store, citing
multiple policy violations.

Asked about the earlier emails, Google said the statements were made in
error and that it doesn't certify whether apps in the Play store comply
with the children's privacy law.

Mr. Abromaitis said he hoped to work with Google to get back into the
store.

\hypertarget{widespread-tracking-of-children}{%
\paragraph{Widespread Tracking of
Children}\label{widespread-tracking-of-children}}

The study this spring showed not only that more than half of children's
apps on Android were sharing tracking ID numbers but also that 5 percent
collected children's location or contact information without their
parents' permission.

To evaluate tracking on iOS as well as Android, The Times conducted a
small study, looking at 10 apps on each platform. The Times chose a mix
of the most popular children's apps and smaller apps that had been
flagged in the academics' research for sharing data, to test whether the
apps had problems on iOS and whether they had been fixed on Android.

Although it is difficult to know whether companies are actually
violating the federal rules, six of the Android apps shared data such as
precise location, IP addresses and tracking IDs in ways that could be
problematic. On iOS, five apps sent IDs to tracking companies in
questionable ways.

In addition to Fun Kid Racing, the tests showed one other Android app
sending precise location data to other companies: Masha and the Bear:
Free Animal Games for Kids, an animated game app with millions of
downloads. The iOS version sent advertising ID codes to a company that
generally prohibits children's apps from using its network.

In an email, Indigo Kids, the Cyprus-based maker of the Masha app, said
it was not responsible for harvesting children's information because
third-party companies collected the data. ``We, as a company, do not
collect or store any data of our users,'' the company said.

Other apps with data practices that could violate the children's privacy
rules sent data to multiple tracking companies that don't allow
children's apps, or sent the data with notes in the computer code
incorrectly indicating that it hadn't come from children.

Several apps reviewed by The Times also sent the advertising ID to other
companies but said this was for specific purposes allowed under the law,
such as preventing an ad from being shown too many times.

Tom Neumayr, an Apple spokesman, said that children's privacy in apps
``is something we take very seriously'' and that developers must follow
strict guidelines about tracking in children's apps.

\hypertarget{enforcing-the-law}{%
\paragraph{Enforcing the Law}\label{enforcing-the-law}}

Since the federal children's online privacy law was enacted in 1998, the
Federal Trade Commission has brought nearly 30 cases alleging violations
by companies including
\href{https://www.ftc.gov/news-events/press-releases/2008/12/sony-bmg-music-settles-charges-its-music-fan-websites-violated}{Sony
BMG Music Entertainment} and
\href{https://www.ftc.gov/news-events/press-releases/2014/09/yelp-tinyco-settle-ftc-charges-their-apps-improperly-collected}{Yelp}.
All of those firms ultimately settled with the agency.

``The F.T.C. has made enforcement of the Children's Online Privacy
Protection Act a high priority,'' said Juliana Gruenwald, an agency
spokeswoman.

But the New Mexico lawsuit is different. The state is not just going
after a single app maker or ad company; it's also implicating the ad
platforms of Google and Twitter, and the vetting process of Google's app
store.

``Google knows that Tiny Lab's apps track children unlawfully,'' the
complaint said. ``Google's bad acts are compounded because it represents
to parents'' that Tiny Lab's apps comply with the children's privacy law
and are ``safe for children.''

The case is particularly fraught
\href{https://www.ftc.gov/news-events/press-releases/2012/08/google-will-pay-225-million-settle-ftc-charges-it-misrepresented}{for
Google}
\href{https://www.ftc.gov/news-events/press-releases/2011/03/ftc-accepts-final-settlement-twitter-failure-safeguard-personal-0}{and}\href{https://www.ftc.gov/news-events/press-releases/2011/03/ftc-accepts-final-settlement-twitter-failure-safeguard-personal-0}{Twitter},
which are each already subject to federal settlements over consumer
privacy or security violations. Those settlements prohibit the companies
from misrepresenting their consumer data protections, and violations
could trigger hefty fines.

``I don't see any way that anything would change unless there are
enforcement actions,'' said Serge Egelman, a researcher at the
University of California, Berkeley, who helped lead the study this
spring.

New Mexico's attorney general said he hoped the F.T.C. and others in
Washington would follow his lead. ``This is as much a black eye on the
federal government as the tech space,'' Mr. Balderas said. ``I'm trying
to get lawmakers at the federal level to wake up.''

\hypertarget{related-coverage}{%
\subsection{Related Coverage}\label{related-coverage}}

\begin{itemize}
\tightlist
\item
  \href{https://www.nytimes.com/2018/08/26/technology/tech-industry-federal-privacy-law.html}{}
\item
  \href{https://www.nytimes.com/2018/06/28/technology/california-online-privacy-law.html}{}
\item
  \href{https://www.nytimes.com/2018/05/19/technology/phone-apps-stalking.html}{}
\item
  \href{https://www.nytimes.com/2018/07/29/business/for-sale-survey-data-on-millions-of-high-school-students.html}{}
\end{itemize}

2018

\hypertarget{more-on-nytimescom}{%
\subsection{More on NYTimes.com}\label{more-on-nytimescom}}

Advertisement

\hypertarget{site-information-navigation}{%
\subsection{Site Information
Navigation}\label{site-information-navigation}}

\begin{itemize}
\tightlist
\item
  \href{https://help.nytimes.com/hc/en-us/articles/115014792127-Copyright-notice}{©
  2020 The New York Times Company}
\item
  \href{https://www.nytimes.com}{Home}
\item
  \href{https://www.nytimes.com/search/}{Search}
\item
  Accessibility concerns? Email us at
  \href{mailto:accessibility@nytimes.com}{\nolinkurl{accessibility@nytimes.com}}.
  We would love to hear from you.
\item
  \href{https://help.nytimes.com/hc/en-us/articles/115015385887-Contact-Us}{Contact
  Us}
\item
  \href{https://www.nytco.com/careers/}{Work with us}
\item
  \href{https://nytmediakit.com/}{Advertise}
\item
  \href{https://help.nytimes.com/hc/en-us/articles/115014892108-Privacy-policy\#pp}{Your
  Ad Choices}
\item
  \href{https://help.nytimes.com/hc/en-us/articles/115014892108-Privacy-policy}{Privacy}
\item
  \href{https://help.nytimes.com/hc/en-us/articles/115014893428-Terms-of-service}{Terms
  of Service}
\item
  \href{https://help.nytimes.com/hc/en-us/articles/115014893968-Terms-of-sale}{Terms
  of Sale}
\end{itemize}

\hypertarget{site-information-navigation-1}{%
\subsection{Site Information
Navigation}\label{site-information-navigation-1}}

\begin{itemize}
\tightlist
\item
  \href{https://spiderbites.nytimes.com}{Site Map}
\item
  \href{https://help.nytimes.com/hc/en-us}{Help}
\item
  \href{https://help.nytimes.com/hc/en-us/articles/115015385887-Contact-Us?redir=myacc}{Site
  Feedback}
\item
  \href{https://www.nytimes.com/subscription?campaignId=37WXW}{Subscriptions}
\end{itemize}
