 **NYTimes.com no longer supports Internet Explorer 9 or earlier. Please
upgrade your browser.
\href{http://www.nytimes.com/content/help/site/ie9-support.html}{LEARN
MORE »}

**Sections

**Home

**Search

\hypertarget{the-new-york-times}{%
\subsection{\texorpdfstring{\href{http://www.nytimes.com/}{The New York
Times}}{The New York Times}}\label{the-new-york-times}}

\hypertarget{-world-}{%
\subsubsection{\texorpdfstring{ \href{/section/world}{World}
}{ World }}\label{-world-}}

 \href{/section/world/middleeast}{Middle East} \textbar{}The Tragedy of
Saudi Arabia's War in Yemen

**Close search

\hypertarget{site-search-navigation}{%
\subsection{Site Search Navigation}\label{site-search-navigation}}

Search NYTimes.com

**Clear this text input

Go

\url{https://nyti.ms/2JgvGNK}

\hypertarget{site-navigation}{%
\subsection{Site Navigation}\label{site-navigation}}

\hypertarget{site-mobile-navigation}{%
\subsection{Site Mobile Navigation}\label{site-mobile-navigation}}

\hypertarget{the-tragedy-of-saudi-arabias-war-in-yemen}{%
\section{The Tragedy of Saudi Arabia's War in
Yemen}\label{the-tragedy-of-saudi-arabias-war-in-yemen}}

The Khashoggi killing has cast light on Saudi tactics in Yemen, where an
economic war has pushed millions to the brink of starvation.

The Tragedy of Saudi Arabia's War

\textbf{Amal Hussain, 7, is wasting away from hunger.} The Saudi-led war
in Yemen has pushed millions to the brink of starvation.

Written by \textbf{\href{https://www.nytimes.com/by/declan-walsh}{Declan
Walsh}}

Photographs by \textbf{Tyler Hicks}

Oct. 26, 2018

Written by \textbf{\href{https://www.nytimes.com/by/declan-walsh}{Declan
Walsh}}

Photographs by \textbf{Tyler Hicks}

Oct. 29, 2018

\emph{\textbf{Declan Walsh} and \textbf{Tyler Hicks} reported from
Hajjah and other areas of northern Yemen this month. Our editors explain
\href{https://www.nytimes.com/2018/10/26/reader-center/yemen-photos-starvation.html}{why
The Times has decided} to publish such unsettling images.}

Chest heaving and eyes fluttering, the 3-year-old boy lay silently on a
hospital bed in the highland town of Hajjah, a bag of bones fighting for
breath.

His father, Ali al-Hajaji, stood anxiously over him. Mr. Hajaji had
already lost one son three weeks earlier to the epidemic of hunger
sweeping across Yemen. Now he feared that a second was slipping away.

It wasn't for a lack of food in the area: The stores outside the
hospital gate were filled with goods and the markets were bustling. But
Mr. Hajaji couldn't afford any of it because prices were rising too
fast.

``I can barely buy a piece of stale bread,'' he said. ``That's why my
children are dying before my eyes.''

The
\href{https://www.nytimes.com/interactive/2018/10/20/world/middleeast/saudi-arabia-invisible-war-yemen.html}{devastating
war in Yemen} has gotten more attention recently as outrage over
\href{https://www.nytimes.com/2018/10/25/world/middleeast/saudi-arabia-jamal-khashoggi-turkey.html}{the}\href{https://www.nytimes.com/2018/10/25/world/middleeast/saudi-arabia-jamal-khashoggi-turkey.html}{killing
of a Saudi dissident} in Istanbul has turned a spotlight on Saudi
actions elsewhere. The harshest criticism of the Saudi-led war has
focused on the
\href{https://www.nytimes.com/2018/08/28/world/middleeast/un-yemen-war-crimes.html}{airstrikes
that have killed thousands} of civilians
\href{https://www.nytimes.com/2018/04/23/world/middleeast/yemen-wedding-bombing.html}{at
weddings},
\href{https://www.nytimes.com/2016/10/09/world/middleeast/yemen-saudi-arabia-houthis-rebels.html}{funerals}
and
\href{https://www.nytimes.com/2018/08/09/world/middleeast/yemen-airstrike-school-bus-children.html}{on
school buses}, aided by American-supplied bombs and intelligence.

But aid experts and United Nations officials say a more insidious form
of warfare is also being waged in Yemen, an economic war that is
exacting a far greater toll on civilians and now risks tipping the
country into a famine of catastrophic proportions.

\hypertarget{hunger-is-expected-to-spread-over-hundreds-of-miles-in-war-torn-yemen}{%
\subsection{Hunger is Expected to Spread Over Hundreds of Miles in
War-torn
Yemen}\label{hunger-is-expected-to-spread-over-hundreds-of-miles-in-war-torn-yemen}}

Oman

Saudi Arabia

Sparsely

populated

areas

HADRAMAWT

Region

Al Ghaydah

Aslam

Yemen

Hajjah

Sana

Juberia

Marib

Zabid

Rida

Al Mukalla

Densely

populated

areas

Taiz

Where the Hunger Crisis Is Worst

more severe

100 miles

Aden

Where the Hunger Crisis Is Worst

more severe

Saudi Arabia

Sparsely

populated

areas

Hajjah

Al Ghaydah

Yemen

Sana

Al Mukalla

Rida

Taiz

Aden

100 miles

Oman

Saudi Arabia

Sparsely

populated

areas

HADRAMAWT

Region

Al Ghaydah

Yemen

Hajjah

Marib

Sana

Al Mukalla

Zabid

Rida

Taiz

Where the Hunger Crisis Is Worst

Aden

more severe

100 miles

Oman

Sparsely

populated

areas

Saudi Arabia

HADRAMAWT

Region

Al Ghaydah

Aslam

Hajjah

Yemen

Sana

Juberia

Marib

Zabid

Rida

Al Mukalla

Densely

populated

areas

Taiz

Where the Hunger Crisis Is Worst

more severe

100 miles

Aden

By Jugal K. Patel and Troy Griggs

Note: Hunger severity data projected for next three months \textbar{}
Source: Famine Early Warning Systems Network

Under the leadership of Crown Prince Mohammed bin Salman, the Saudi-led
coalition and its Yemeni allies have imposed a raft of punitive economic
measures aimed at undercutting the Houthi rebels who control northern
Yemen. But these actions --- including periodic blockades, stringent
import restrictions and withholding the salaries of about a million
civil servants --- have landed on the backs of civilians, laying the
economy to waste and driving millions deeper into poverty.

Those measures have inflicted a slow-burn toll: infrastructure
destroyed, jobs lost, a weakening currency and soaring prices. But in
recent weeks the economic collapse has gathered pace at alarming speed,
causing top United Nations officials to revise their predictions of
famine.

``There is now a clear and present danger of an imminent and great, big
famine engulfing Yemen,'' Mark Lowcock, the under secretary for
humanitarian affairs, told the Security Council on Tuesday. Eight
million Yemenis already depend on emergency food aid to survive, he
said, a figure that could soon rise to 14 million, or half Yemen's
population.

``People think famine is just a lack of food,'' said Alex de Waal,
author of ``Mass Starvation'' which analyzes recent man-made famines.
``But in Yemen it's about a war on the economy.''

Ali al-Hajaji and his wife, Mohamediah Mohammed, lost one son to hunger.
Now they fear losing a second. Tyler Hicks/The New York Times

When Shaher became ill, Mr. Hajaji tried a folk remedy and burned him,
leaving scars on his chest.

 Tyler Hicks/The New York Times

Shaadi, 4 years old, lies in a grave marked by a single broken rock.
Tyler Hicks/The New York Times

The signs are everywhere, cutting across boundaries of class, tribe and
region. Unpaid university professors issue desperate appeals for help on
social media. Doctors and teachers are forced to sell their gold, land
or cars to feed their families. On the streets of the capital, Sana, an
elderly woman begs for alms with a loudspeaker.

``Help me,'' the woman, Zahra Bajali, calls out. ``I have a sick
husband. I have a house for rent. Help.''

And in the hushed hunger wards, ailing infants hover between life and
death. Of nearly two million malnourished children in Yemen, 400,000 are
considered critically ill --- a figure projected to rise by one quarter
in the coming months.

``We are being crushed,'' said Dr. Mekkia Mahdi at the health clinic in
Aslam, an impoverished northwestern town that has been swamped with
refugees fleeing the fighting in Hudaydah, an embattled port city 90
miles to the south.

Flitting between the beds at her spartan clinic, she cajoled mothers,
dispensed orders to medics and spoon-fed milk to sickly infants. For
some it was too late: the night before, an 11-month old boy had died. He
weighed five and a half pounds.

Looking around her, Dr. Mahdi could not fathom the Western obsession
with the Saudi killing of Jamal Khashoggi in Istanbul.

``We're surprised the Khashoggi case is getting so much attention while
millions of Yemeni children are suffering,'' she said. ``Nobody gives a
damn about them.''

She tugged on the flaccid skin of a drowsy 7-year-old girl with
stick-like arms. ``Look,'' she said. ``No meat. Only bones.''

The embassy of Saudi Arabia in Washington did not respond to questions
about the country's policies in Yemen. But Saudi officials have defended
their actions, citing rockets fired across their border by the Houthis,
an armed group professing Zaidi Islam, an offshoot of Shiism, that Saudi
Arabia, a Sunni monarchy, views as a proxy for its regional rival, Iran.

The Saudis point out that they, along with the United Arab Emirates, are
among the most generous donors to Yemen's humanitarian relief effort.
Last spring, the two allies pledged \$1 billion in aid to Yemen. In
January, Saudi Arabia deposited \$2 billion in Yemen's central bank to
prop up its currency.

But those efforts have been overshadowed by the coalition's attacks on
Yemen's economy, including the denial of salaries to civil servants, a
partial blockade that has driven up food prices, and the printing of
vast amounts of bank notes, which caused the currency to plunge.

And the offensive to capture Hudaydah, which started in June, has
endangered the main lifeline for imports to northern Yemen, displaced
570,000 people and edged many more closer to starvation.

A famine here, Mr. Lowcock warned, would be ``much bigger than anything
any professional in this field has seen during their working lives.''

When Ali Hajaji's son fell ill with diarrhea and vomiting, the desperate
father turned to extreme measures. Following the advice of village
elders, he pushed the red-hot tip of a burning stick into Shaher's
chest, a folk remedy to drain the ``black blood'' from his son.

``People said burn him in the body and it will be O.K.,'' Mr. Hajaji
said. ``When you have no money, and your son is sick, you'll believe
anything.''

The burns were a mark of the rudimentary nature of life in Juberia, a
cluster of mud-walled houses perched on a rocky ridge. To reach it, you
cross a landscape of sandy pastures, camels and beehives, strewn with
giant, rust-colored boulders, where women in black cloaks and yellow
straw boaters toil in the fields.

In the past, the men of the village worked as migrant laborers in Saudi
Arabia, whose border is 80 miles away. They were often treated with
disdain by their wealthy Saudi employers but they earned a wage. Mr.
Hajaji worked on a suburban construction site in Mecca, the holy city
visited by millions of Muslim pilgrims every year.

When the war broke out in 2015, the border closed.

The fighting never reached Juberia, but it still took a toll there.

Last year a young woman died of cholera, part of an epidemic that
infected 1.1 million Yemenis. In April, a coalition airstrike hit a
wedding party in the district, killing 33 people, including the bride. A
local boy who went to fight for the Houthis was killed in an airstrike.

Bassam Mohammed Hassan, who suffers from severe malnutrition and
cerebral palsy, at a hospital in Sana, Yemen. Tyler Hicks/The New York
Times

``The big countries say they are fighting each other in Yemen,'' Mr.
Hajaji said. ``But it feels to us like they are fighting the poor
people.'' Tyler Hicks/The New York Times

Ahmed Ibrahim al-Junid, 5 months old, in Aslam. Tyler Hicks/The New York
Times

But for Mr. Hajaji, who had five sons under age 7, the deadliest blow
was economic.

He watched in dismay as the riyal lost half its value in the past year,
causing prices to soar. Suddenly, groceries cost twice as much as they
had before the war. Other villagers sold their assets, such as camels or
land, to get money for food.

But Mr. Hajaji, whose family lived in a one-room, mud-walled hut, had
nothing to sell.

At first he relied on the generosity of neighbors. Then he pared back
the family diet, until it consisted only of bread, tea and halas, a vine
leaf that had always been a source of food but now occupied a central
place in every meal.

Soon his first son to fall ill, Shaadi, was vomiting and had diarrhea,
classic symptoms of malnutrition. Mr. Hajaji wanted to take the ailing
4-year-old to the hospital, but that was out of the question: fuel
prices had risen by 50 percent over the previous year.

One morning in late September, Mr. Hajaji walked into his house to find
Shaadi silent and immobile, with a yellow tinge to his skin. ``I knew he
was gone,'' he said. He kissed his son on the forehead, bundled him up
in his arms, and walked along a winding hill path to the village mosque.

That evening, after prayers, the village gathered to bury Shaadi. His
grave, marked by a single broken rock, stood under a grove of Sidr trees
that, in better times, were famous for their honey.

Shaadi was the first in the village to die from hunger.

A few weeks later, when Shaher took ill, Mr. Hajaji was determined to do
something. When burning didn't work, he carried his son down the stony
path to a health clinic, which was ill-equipped for the task. Half of
Yemen's health facilities are closed because of the war.

So his family borrowed \$16 for the journey to the hospital in Hajjah.

``All the big countries say they are fighting each other in Yemen,'' Mr.
Hajaji said. ``But it feels to us like they are fighting the poor
people.''

Yemen's economic crisis was not some unfortunate but unavoidable side
effect of the fighting.

In 2016, the Saudi-backed Yemeni government transferred the operations
of the central bank from the Houthi-controlled capital, Sana, to the
southern city of Aden. The bank, whose policies are dictated by Saudi
Arabia, a senior Western official said, started printing vast amounts of
new money --- at least 600 billion riyals, according to one bank
official. The new money caused an inflationary spiral that eroded the
value of any savings people had.

The bank also stopped paying salaries to civil servants in
Houthi-controlled areas, where 80 percent of Yemenis live. With the
government as the largest employer, hundreds of thousands of families in
the north suddenly had no income.

At the Sabeen hospital in Sana, Dr. Huda Rajumi treats the country's
most severely malnourished children. But her own family is suffering,
too, as she falls out of Yemen's vanishing middle class.

In the past year, she has received only a single month's salary. Her
husband, a retired soldier, is no longer getting his pension, and Dr.
Rajumi has started to skimp on everyday pleasures, like fruit, meat and
taxi rides, to make ends meet.

``We get by because people help each other out,'' she said. ``But it's
getting hard.''

Economic warfare takes other forms, too. In a recent paper, Martha
Mundy, a lecturer at the London School of Economics,
\href{https://sites.tufts.edu/wpf/strategies-of-the-coalition-in-the-yemen-war/}{analyzed
coalition airstrikes} in Yemen, finding that their attacks on bridges,
factories, fishing boats and even fields suggested that they aimed to
destroy food production and distribution in Houthi-controlled areas.

Airstrikes have destroyed bridges, like this one in Bani Hassan,
factories, fishing boats and fields, suggesting that disrupting the food
supply may have been a goal. Tyler Hicks/The New York Times

The fighting has displaced about a million Yemenis.

 Tyler Hicks/The New York Times

Airstrikes have destroyed homes in the Old City of Sana. Tyler Hicks/The
New York Times

Saudi Arabia's tight control over all air and sea movements into
northern Yemen has effectively made the area a prison for those who live
there. In September, the World Health Organization brokered the
establishment of a humanitarian air bridge to allow the sickest Yemenis
--- cancer patients and others with life-threatening conditions --- to
fly to Egypt.

Among those on the waiting list is Maimoona Naji, a 16-year-old girl
with a melon-size tumor on her left leg. At a hostel in Sana, her
father, Ali Naji, said they had obtained visas and money to travel to
India for emergency treatment. Their hopes soared in September when his
daughter was told she would be on the first plane out of Sana once the
airlift started.

But the agreement has stalled, blocked by the Yemeni government,
according to the senior Western official. Maimoona and dozens of other
patients have been left stranded, the clock ticking on their illnesses.

``First they told us `next week, next week,''' said Mr. Naji, shuffling
through reams of documents as tears welled up in his eyes. ``Then they
said no. Where is the humanity in that? What did we do to deserve
this?''

The Saudi coalition is not solely to blame for Yemen's food crisis.

In Houthi-held areas, aid workers say, commanders level illegal taxes at
checkpoints and frequently try to divert international relief aid to the
families of soldiers, or to line their own pockets.

At the United Nations on Tuesday, Mr. Lowcock, the humanitarian
official, said that aid workers in Yemen faced obstacles including
delayed visas, retracted work permits and interference in the work ---
problems, officials said privately, that were greatest in Houthi-held
areas.

Despite the harrowing scenes of suffering in the north, some Yemenis are
getting rich. Upmarket parts of Sana are enjoying a mini real estate
boom, partly fueled by Yemeni migrants returned from Saudi Arabia, but
also by newly enriched Houthi officials.

Local residents say they have seen Houthi officials from modest
backgrounds driving around the city in Lexus four-wheel drives, or
shopping in luxury stores, trailed by armed gunmen, to buy suits and
perfumes.

Tensions reached a climax this summer when the head of the United
Nations migration agency was forced to leave Sana after clashing with
the Houthi administration.

In an interview, the Houthi vice foreign minister, Hussain al-Ezzi,
denied reports of corruption, and insisted that tensions with the United
Nations had been resolved.

``We don't deny there have been some mistakes on our side,'' he said.
``We are working to improve them.''

``Nobody gives a damn about them.'' Tyler Hicks/The New York Times

Wadah Askri Mesheel, 11 months old, arrived at a clinic in Aslam, Yemen,
with severe malnutrition. He died eight hours later. Tyler Hicks/The New
York Times

In the hushed hunger wards, ailing infants hover between life and death.
Tyler Hicks/The New York Times

Only two famines have been officially declared by the United Nations in
the past 20 years, in Somalia and South Sudan. A United Nations-led
assessment due in mid-November will determine how close Yemen is to
becoming the third.

To stave it off, aid workers are not appealing for shipments of relief
aid but for urgent measures to rescue the battered economy.

``This is an income famine,'' said Lise Grande, the United Nations
humanitarian coordinator for Yemen. ``The key to stopping it is to
ensure that people have enough money to buy what they need to survive.''

The priority should be to stabilize the falling currency, she said, and
to ensure that traders and shipping companies can import the food that
Yemenis need.

Above all, she added, ``the fighting has to stop.''

One hope for Yemenis is that the international fallout from the death of
the Saudi dissident, Jamal Khashoggi, which has damaged Prince
Mohammed's international standing, might force him to relent in his
unyielding prosecution of the war.

Peter Salisbury, a Yemen specialist at Chatham House, said that was
unlikely.

``I think the Saudis have learned what they can get away with in Yemen
--- that western tolerance for pretty bad behavior is quite high,'' he
said. ``If the Khashoggi murder tells us anything, it's just how
reluctant people are to rein the Saudis in.''

Produced by Craig Allen, Herbert Buchsbaum, David Furst, Josh Keller,
Meghan Petersen and Andrew Rossback. Saeed Al-Batati contributed
reporting.

\href{https://www.nytimes.com/interactive/2018/10/20/world/middleeast/saudi-arabia-invisible-war-yemen.html}{Related
\includegraphics{https://static01.nyt.com/images/2018/10/21/world/middleeast/21-Yemen-slide-K4HL/XX-Yemen-slide-K4HL-jumbo.jpg}
This is the front line of Saudi Arabia's invisible war}

Produced by Craig Allen, Herbert Buchsbaum, David Furst, Josh Keller,
Meghan Petersen and Andrew Rossback. Saeed Al-Batati contributed
reporting.

\hypertarget{at-war}{%
\subsection{\texorpdfstring{\href{https://www.nytimes.com/spotlight/atwar}{At
War}}{At War}}\label{at-war}}

\begin{itemize}
\item
  \href{https://www.nytimes.com/interactive/2018/10/20/world/middleeast/saudi-arabia-invisible-war-yemen.html}{}

  \hypertarget{this-is-the-front-line-of-saudi-arabias-invisible-war}{%
  \subsection{This is the front line of Saudi Arabia's invisible
  war}\label{this-is-the-front-line-of-saudi-arabias-invisible-war}}

  Dec. 20, 2018

  The Khashoggi crisis has called attention to a largely overlooked
  Saudi-led war in Yemen. On a rare trip to the front line, we found
  Yemenis fighting and dying in a war that has gone nowhere.
\end{itemize}

Advertisement

\hypertarget{site-information-navigation}{%
\subsection{Site Information
Navigation}\label{site-information-navigation}}

\begin{itemize}
\tightlist
\item
  \href{https://help.nytimes.com/hc/en-us/articles/115014792127-Copyright-notice}{©
  2020 The New York Times Company}
\item
  \href{https://www.nytimes.com}{Home}
\item
  \href{https://www.nytimes.com/search/}{Search}
\item
  Accessibility concerns? Email us at
  \href{mailto:accessibility@nytimes.com}{\nolinkurl{accessibility@nytimes.com}}.
  We would love to hear from you.
\item
  \href{https://help.nytimes.com/hc/en-us/articles/115015385887-Contact-Us}{Contact
  Us}
\item
  \href{https://www.nytco.com/careers/}{Work with us}
\item
  \href{https://nytmediakit.com/}{Advertise}
\item
  \href{https://help.nytimes.com/hc/en-us/articles/115014892108-Privacy-policy\#pp}{Your
  Ad Choices}
\item
  \href{https://help.nytimes.com/hc/en-us/articles/115014892108-Privacy-policy}{Privacy}
\item
  \href{https://help.nytimes.com/hc/en-us/articles/115014893428-Terms-of-service}{Terms
  of Service}
\item
  \href{https://help.nytimes.com/hc/en-us/articles/115014893968-Terms-of-sale}{Terms
  of Sale}
\end{itemize}

\hypertarget{site-information-navigation-1}{%
\subsection{Site Information
Navigation}\label{site-information-navigation-1}}

\begin{itemize}
\tightlist
\item
  \href{https://spiderbites.nytimes.com}{Site Map}
\item
  \href{https://help.nytimes.com/hc/en-us}{Help}
\item
  \href{https://help.nytimes.com/hc/en-us/articles/115015385887-Contact-Us?redir=myacc}{Site
  Feedback}
\item
  \href{https://www.nytimes.com/subscription?campaignId=37WXW}{Subscriptions}
\end{itemize}
