 **NYTimes.com no longer supports Internet Explorer 9 or earlier. Please
upgrade your browser.
\href{http://www.nytimes.com/content/help/site/ie9-support.html}{LEARN
MORE »}

**Sections

**Home

**Search

\hypertarget{the-new-york-times}{%
\subsection{\texorpdfstring{\href{http://www.nytimes.com/}{The New York
Times}}{The New York Times}}\label{the-new-york-times}}

\hypertarget{-opinion-}{%
\subsubsection{\texorpdfstring{ \href{/section/opinion}{Opinion}
}{ Opinion }}\label{-opinion-}}

 \href{/section/opinion}{Opinion} \textbar{}`It's an Act of Murder': How
Europe Outsources Suffering as Migrants Drown

**Close search

\hypertarget{site-search-navigation}{%
\subsection{Site Search Navigation}\label{site-search-navigation}}

Search NYTimes.com

**Clear this text input

Go

\url{https://nyti.ms/2GGPtIN}

\hypertarget{site-navigation}{%
\subsection{Site Navigation}\label{site-navigation}}

\hypertarget{site-mobile-navigation}{%
\subsection{Site Mobile Navigation}\label{site-mobile-navigation}}

\hypertarget{its-an-act-of-murder-how-europe-outsources-suffering-as-migrants-drown}{%
\section{`It's an Act of Murder': How Europe Outsources Suffering as
Migrants
Drown}\label{its-an-act-of-murder-how-europe-outsources-suffering-as-migrants-drown}}

 Opinion `It's an Act of Murder': How Europe Outsources Suffering as
Migrants Drown

 Opinion `It's an Act of Murder': How Europe Outsources Suffering as
Migrants Drown This short film, produced by The Times's Opinion Video
team and the research groups Forensic Architecture and Forensic
Oceanography, reconstructs a tragedy at sea that left at least 20
migrants dead. Combining footage from more than 10 cameras, 3-D modeling
and interviews with rescuers and survivors, the documentary shows
Europe's role in the migrant crisis at sea.

By Charles Heller, Lorenzo Pezzani, Itamar Mann, Violeta Moreno-Lax and
Eyal Weizman

Producer Taylor Adams

Assistant Producer Leah Varjacques

Executive Producer Adam B. Ellick

Editor Kristin Bye

On Nov. 6, 2017, at least 20 people trying to reach Europe from Libya
drowned in the Mediterranean, foundering next to a sinking raft.

Not far from the raft was a ship belonging to
\href{https://sea-watch.org/en/}{Sea-Watch}, a German humanitarian
organization. That ship had enough space on it for everyone who had been
aboard the raft. It could have brought them all to the safety of Europe,
where they might have had a chance at being granted asylum.

Instead, 20 people drowned and 47 more were captured by the Libyan Coast
Guard, which brought the migrants back to Libya, where they suffered
abuse --- including rape and torture.

This confrontation at sea was not a simplistic case of Europe versus
Africa, with human rights and rescue on one side and chaos and danger on
the other. Rather it's a case of Europe versus Europe: of volunteers
struggling to save lives being undercut by European Union policies that
outsource border control responsibilities to the Libyan Coast Guard ---
with the aim of stemming arrivals on European shores.

While funding, equipping and directing the Libyan Coast Guard, European
governments have stymied the activities of nongovernmental organizations
like Sea-Watch, criminalizing them or impounding their ships, or turning
away from ports ships carrying survivors.

\href{https://missingmigrants.iom.int/region/mediterranean?migrant_route\%5B\%5D=1376}{More
than 14,000 people} have died or gone missing while trying to cross the
central Mediterranean since 2014. But unlike most of those deaths and
drownings, the incident on Nov. 6, 2017, was extensively documented.

Sea-Watch's ship and rescue rafts were outfitted with nine cameras,
documenting the entire scene in video and audio. The Libyans, too,
filmed parts of the incident on their mobile phones.

The research groups Forensic Architecture and Forensic Oceanography of
Goldsmiths, University of London, of which three of us --- Mr. Heller,
Mr. Pezzani and Mr. Weizman --- are a part, combined these video sources
with radio recordings, vessel tracking data, witness testimonies and
newly obtained official sources to
produce\href{https://www.forensic-architecture.org/case/sea-watch/}{}\href{https://www.forensic-architecture.org/case/sea-watch/}{a
minute-by-minute reconstruction} of the facts. Opinion Video at The New
York Times built on this work to create the above short documentary,
gathering further testimonials by some of the survivors and rescuers who
were there.

This investigation makes a few things clear: European governments are
avoiding their legal and moral responsibilities to protect the human
rights of people fleeing violence and economic desperation. More
worrying, the Libyan Coast Guard partners that Europe is collaborating
with are ready to blatantly violate those rights if it allows them to
prevent migrants from crossing the sea.

\includegraphics{https://static01.nyt.com/packages/flash/multimedia/ICONS/transparent.png}

Forensic Architecture and Forensic Oceanography's reconstruction allowed
the events of Nov. 6 to be oriented in space and time. The New York
Times/Forensic Oceanography

\hypertarget{stopping-migrants-whatever-the-cost}{%
\subsection{Stopping Migrants, Whatever the
Cost}\label{stopping-migrants-whatever-the-cost}}

To understand the cynicism of Europe's policies in the Mediterranean,
one must understand the legal context. According
to\href{https://www.amnesty.org/en/latest/news/2012/02/italy-historic-european-court-judgment-upholds-migrants-rights/}{}\href{https://www.amnesty.org/en/latest/news/2012/02/italy-historic-european-court-judgment-upholds-migrants-rights/}{a
2012 ruling} by the European Court of Human Rights, migrants rescued by
European civilian or military vessels must be taken to a safe port.
Because of the chaotic political situation in Libya
and\href{https://www.amnesty.org/en/latest/news/2018/05/libya-shameful-eu-policies-fuel-surge-in-detention-of-migrants-and-refugees/}{}\href{https://www.amnesty.org/en/latest/news/2018/05/libya-shameful-eu-policies-fuel-surge-in-detention-of-migrants-and-refugees/}{well-documented
human rights abuses} in detention camps there, that means a European
port, often in Italy or Malta.

But when the Libyan Coast Guard intercepts migrants, even outside Libyan
territorial waters, as it did on Nov. 6, the Libyans take them back to
detention camps in Libya, which is not subject to European Court of
Human Rights jurisdiction.

For Italy --- and Europe --- this is an ideal situation. Europe is able
to stop people from reaching its shores while washing its hands of any
responsibility for their safety.

This policy can be traced back to February 2017, when Italy and the
United Nations-supported Libyan Government of National Accord signed a
``\href{https://www.repubblica.it/esteri/2017/02/02/news/migranti_accordo_italia-libia_ecco_cosa_contiene_in_memorandum-157464439/?refresh_ce}{memorandum
of understanding}'' that provided a framework for collaboration on
development, to fight against ``illegal immigration,'' human trafficking
and the smuggling of contraband. This agreement defines clearly the aim,
``to stem the illegal migrants' flows,'' and committed Italy to provide
``technical and technological support to the Libyan institutions in
charge of the fight against illegal immigration.''

Libyan Coast Guard members have been trained by the European Union, and
the Italian government donated or repaired several patrol boats and
supported the establishment of a Libyan search-and-rescue zone. Libyan
authorities have since attempted --- in defiance of maritime law --- to
make that zone off-limits to nongovernmental organizations' rescue
vessels. Italian Navy ships, based in Tripoli, have coordinated Libyan
Coast Guard efforts.

Before these arrangements, Libyan actors were able to intercept and
return very few migrants leaving from Libyan shores. Now the Libyan
Coast Guard is an efficient partner, having intercepted some 20,000
people in 2017 alone.

The Libyan Coast Guard is efficient when it comes to stopping migrants
from reaching Europe. It's not as good, however, at saving their lives,
as the events of Nov. 6 show.

\includegraphics{https://static01.nyt.com/packages/flash/multimedia/ICONS/transparent.png}

Marco Minniti, center, the interior minister of Italy at the time, in
front of the ship involved in the Nov. 6 incident. Ismail
Zitouny/Reuters

\hypertarget{a-deadly-policy-in-action}{%
\subsection{A Deadly Policy in Action}\label{a-deadly-policy-in-action}}

That morning the migrant raft had encountered worsening conditions after
leaving Tripoli, Libya, over night. Someone onboard used a satellite
phone to call the Italian Coast Guard for help.

Because the Italians were required by law to alert nearby vessels of the
sinking raft, they alerted Sea-Watch to its approximate location. But
they also requested the intervention of their Libyan counterparts.

The Libyan Coast Guard vessel that was sent to intervene on that
morning, the Ras Jadir, was one of several that had been repaired by
Italy and handed back to the Libyans in May of 2017. Eight of the 13
crew members onboard had received training from the European Union
anti-smuggling naval program known as Operation Sophia.

Even so, the Libyans brought the Ras Jadir next to the migrants' raft,
rather than deploying a smaller rescue vessel, as professional rescuers
do. This offered no hope of rescuing those who had already fallen
overboard and only caused more chaos, during which at least five people
died.

These deaths were not merely a result of a lack of professionalism. Some
of the migrants who had been brought aboard the Ras Jadir were so afraid
of their fate at the hands of the Libyans that they jumped back into the
water to try to reach the European rescuers. As can be seen in the
footage, members of the Libyan Coast Guard beat the remaining migrants.

Sea-Watch's crew was also attacked by the Libyan Coast Guard, who
threatened them and threw hard objects at them to keep them away. This
eruption of violence was the result of a clash between the goals of
rescue and interception, with the migrants caught in the middle
desperately struggling for their lives.

Apart from those who died during this chaos, more than 15 people had
already drowned in the time spent waiting for any rescue vessel to
appear.

There was, however, no shortage of potential rescuers in the area: A
Portuguese surveillance plane had located the migrants' raft after its
distress call. An Italian Navy helicopter and a French frigate were
nearby and eventually offered some support during the rescue.

It's possible that this French ship, deployed as part of Operation
Sophia, could have reached the sinking vessel earlier, in time to save
more lives --- despite our requests, this information has not been
disclosed to us. But it remained at a distance throughout the incident
and while offering some support, notably refrained from taking migrants
onboard who would then have had to have been disembarked on European
soil. It's an example of a hands-off approach that seeks to make Libyan
intervention not only possible but also inevitable.

\includegraphics{https://static01.nyt.com/packages/flash/multimedia/ICONS/transparent.png}

The Libyan Coast Guard vessel's proximity to the migrants' sinking raft
created a dangerous situation. With the Libyans unable to reach those
who fell into the water, Sea-Watch rushed to fill the gap. Sea-Watch

\hypertarget{a-legal-challenge}{%
\subsection{A Legal Challenge}\label{a-legal-challenge}}

On the basis of the forensic reconstruction, the Global Legal Action
Network and the Association for Juridical Studies on Immigration, with
the support of Yale Law School students, have filed a case against Italy
at the European Court of Human Rights representing 17 survivors of this
incident.

Those working on the suit, who include two of us --- Mr. Mann and Ms.
Moreno-Lax --- argue that even though Italian or European personnel did
not physically intercept the migrants and bring them back to Libya,
Italy exercised effective control over the Libyan Coast Guard through
mutual agreements, support and on-the-ground coordination. Italy has
entrusted the Libyans with a task that Rome knows full well would be
illegal if undertaken directly: preventing migrants from seeking
protection in Europe by impeding their flight and sending them back to a
country
where\href{https://www.amnesty.org/download/Documents/EUR3089062018ENGLISH.pdf}{}\href{https://www.amnesty.org/download/Documents/EUR3089062018ENGLISH.pdf}{extreme
violence and exploitation await}.

We hope this legal complaint will lead the European court to rule that
countries cannot subcontract their legal and humanitarian obligations to
dubious partners, and that if they do, they retain responsibility for
the resulting violations. Such a precedent would force the entire
European Union to make sure its cooperation with partners like Libya
does not end up denying refugees the right to seek asylum.

This case is especially important right now. In
Italy's\href{https://www.nytimes.com/2018/03/04/world/europe/italy-election.html}{}\href{https://www.nytimes.com/2018/03/04/world/europe/italy-election.html}{elections
in March}, the far-right Lega party, which campaigned on radical
anti-immigrant rhetoric, took nearly 20 percent of the vote. The party
is now part of the governing coalition, of which its leader, Matteo
Salvini, is the interior minister.

His government has doubled down on animosity toward migrants. In June,
Italy took the drastic step
of\href{https://www.nytimes.com/2018/06/11/world/europe/italy-migrant-boat-aquarius.html}{}\href{https://www.nytimes.com/2018/06/11/world/europe/italy-migrant-boat-aquarius.html}{turning
away a humanitarian vessel} from the country's ports and has been
systematically blocking rescued migrants from being disembarked since
then, even when they had been assisted by the Italian Coast Guard.

\href{https://www.amnesty.org/download/Documents/EUR3089062018ENGLISH.pdf}{The
Italian crackdown} helps explain why seafarers off the Libyan coast have
\href{https://www.theguardian.com/world/2018/nov/09/us-navy-ship-ignored-sinking-migrants-cries-for-help-say-survivors}{refrained}
from assisting migrants in distress, leaving them to drift for days.
Under the new Italian government, a new batch of patrol boats has been
handed over to the Libyan Coast Guard, and the rate of migrants being
intercepted and brought back to Libya has increased. All this has made
the crossing even more dangerous than before.

Italy has been seeking to enact a practice that blatantly violates the
spirit of the Geneva Convention on refugees, which enshrines the right
to seek asylum and prohibits sending people back to countries in which
their lives are at risk. A judgment by the European Court sanctioning
Italy for this practice would help prevent the outsourcing of border
control and human rights violations that may prevent the world's most
disempowered populations from seeking protection and dignity.

The European Court of Human Rights cannot stand alone as a guardian of
fundamental rights. Yet an insistence on its part to uphold the law
would both reflect and bolster the movements seeking solidarity with
migrants across Europe.

\emph{Charles Heller and Lorenzo Pezzani are co-founders of Forensic
Oceanography, and Eyal Weizman is the director
of}\href{https://www.forensic-architecture.org/}{}\href{https://www.forensic-architecture.org/}{Forensic
Architecture}\emph{, at Goldsmiths, University of London. Itamar Mann is
a legal adviser to the Global Legal Action Network and an associate
professor of law at the University of Haifa. Violeta Moreno-Lax is a
legal adviser to the Global Legal Action Network and a senior lecturer
in law at Queen Mary University of London.}

\hypertarget{more-on-nytimescom}{%
\subsection{More on NYTimes.com}\label{more-on-nytimescom}}

Advertisement

\hypertarget{site-information-navigation}{%
\subsection{Site Information
Navigation}\label{site-information-navigation}}

\begin{itemize}
\tightlist
\item
  \href{https://help.nytimes.com/hc/en-us/articles/115014792127-Copyright-notice}{©
  2020 The New York Times Company}
\item
  \href{https://www.nytimes.com}{Home}
\item
  \href{https://www.nytimes.com/search/}{Search}
\item
  Accessibility concerns? Email us at
  \href{mailto:accessibility@nytimes.com}{\nolinkurl{accessibility@nytimes.com}}.
  We would love to hear from you.
\item
  \href{https://help.nytimes.com/hc/en-us/articles/115015385887-Contact-Us}{Contact
  Us}
\item
  \href{https://www.nytco.com/careers/}{Work with us}
\item
  \href{https://nytmediakit.com/}{Advertise}
\item
  \href{https://help.nytimes.com/hc/en-us/articles/115014892108-Privacy-policy\#pp}{Your
  Ad Choices}
\item
  \href{https://help.nytimes.com/hc/en-us/articles/115014892108-Privacy-policy}{Privacy}
\item
  \href{https://help.nytimes.com/hc/en-us/articles/115014893428-Terms-of-service}{Terms
  of Service}
\item
  \href{https://help.nytimes.com/hc/en-us/articles/115014893968-Terms-of-sale}{Terms
  of Sale}
\end{itemize}

\hypertarget{site-information-navigation-1}{%
\subsection{Site Information
Navigation}\label{site-information-navigation-1}}

\begin{itemize}
\tightlist
\item
  \href{https://spiderbites.nytimes.com}{Site Map}
\item
  \href{https://help.nytimes.com/hc/en-us}{Help}
\item
  \href{https://help.nytimes.com/hc/en-us/articles/115015385887-Contact-Us?redir=myacc}{Site
  Feedback}
\item
  \href{https://www.nytimes.com/subscription?campaignId=37WXW}{Subscriptions}
\end{itemize}
