Sections

SEARCH

\protect\hyperlink{site-content}{Skip to
content}\protect\hyperlink{site-index}{Skip to site index}

\href{/section/opinion}{Opinion}

\hypertarget{the-new-york-times-editorial-board}{%
\section{The New York Times Editorial
Board}\label{the-new-york-times-editorial-board}}

March 1, 2018

\begin{itemize}
\item
\item
\item
\item
\end{itemize}

The New York Times editorial board is made up of opinion journalists who
rely on research, debate and individual expertise to reach a shared view
of important issues. The board does not speak for the newsroom or The
Times as a whole. Rather, amid the contending individual voices of Times
Opinion, it aims to provide a consistent, independent view of the world
based on time-tested institutional values. The board argues for a world
that is both free and fair, believing that societies must struggle to
reconcile these values in order to succeed. It has long supported a
liberal order of nations in which freedom and progress advance through
democracy and capitalism. But it has also sought to guard against the
excesses of those systems by promoting honest governance, civil rights,
equality of opportunity, a healthy planet and a good life for society's
most vulnerable members. Since its founding in 1896, the board has,
above all, championed what Adolph Ochs called ``the free exercise of a
sound conscience,'' believing that the fearless exchange of information
and ideas is the surest means of resisting tyranny and realizing human
potential.

\hypertarget{the-new-york-times-editorial-board-1}{%
\section{The New York Times Editorial
Board}\label{the-new-york-times-editorial-board-1}}

The New York Times editorial board is made up of opinion journalists who
rely on research, debate and individual expertise to reach a shared view
of important issues. The board does not speak for the newsroom or The
Times as a whole. Rather, amid the contending individual voices of Times
Opinion, it aims to provide a consistent, independent view of the world
based on time-tested institutional values. The board argues for a world
that is both free and fair, believing that societies must struggle to
reconcile these values in order to succeed. It has long supported a
liberal order of nations in which freedom and progress advance through
democracy and capitalism. But it has also sought to guard against the
excesses of those systems by promoting honest governance, civil rights,
equality of opportunity, a healthy planet and a good life for society's
most vulnerable members. Since its founding in 1896, the board has,
above all, championed what Adolph Ochs called ``the free exercise of a
sound conscience,'' believing that the fearless exchange of information
and ideas is the surest means of resisting tyranny and realizing human
potential.

\begin{itemize}
\item
  \hypertarget{kathleen-kingsbury}{%
  \subsection{Kathleen Kingsbury}\label{kathleen-kingsbury}}

  \hypertarget{editorial-page-editor}{%
  \subsubsection{Editorial Page Editor}\label{editorial-page-editor}}

  \includegraphics{https://static01.nyt.com/newsgraphics/2020/01/12/editorial-board-page/5610fc92ee24704dfbe34ffd1d73d67323eff42c/katiekingsbury.jpg}

  Kathleen Kingsbury is editorial page editor of The New York Times. She
  joined The Times in 2017 from The Boston Globe, where she served as
  managing editor for digital.

  Ms. Kingsbury joined The Globe's editorial board in 2013 and later
  edited Ideas, the paper's Sunday section aimed at tackling the new
  thinking, intellectual trends and big ideas that shape our world. In
  this role, Ms. Kingsbury was also a deputy managing editor and the
  deputy editorial page editor.

  She was awarded the 2015 Pulitzer Prize for distinguished editorial
  writing for a series on low wages and the mistreatment of workers in
  the restaurant industry. The same eight-part series, ``Service Not
  Included,'' also received the Scripps Howard Foundation's 2014 Walker
  Stone Award for editorial writing and the Burl Osborne Award for
  editorial leadership from the American Society of News Editors. She
  also edited The Globe's 2016 Pulitzer Prize-winning commentary on race
  and education.

  Ms. Kingsbury previously worked as a New York-based staff writer and
  Hong Kong-based foreign correspondent for Time magazine.

  Follow on Twitter \href{http://twitter.com/katiekings}{@katiekings}
\item
  \hypertarget{binyamin-appelbaum}{%
  \subsection{Binyamin Appelbaum}\label{binyamin-appelbaum}}

  \hypertarget{economics-and-business}{%
  \subsubsection{Economics and Business}\label{economics-and-business}}

  \includegraphics{https://static01.nyt.com/images/2019/03/28/opinion/binya-appelbaum/binya-appelbaum-custom1-v3.jpg}

  Binyamin Appelbaum joined the Times editorial board in 2019. From 2010
  to 2019, he was a Washington correspondent for The Times, covering the
  Federal Reserve and other aspects of economic policy. He previously
  worked for The Washington Post, The Boston Globe, The Charlotte
  Observer and The Florida Times-Union. A native of Boston, he holds a
  B.A. in history from the University of Pennsylvania. He is based in
  Washington.

  Follow on Twitter \href{http://twitter.com/BCAppelbaum}{@BCAppelbaum}
\item
  \hypertarget{greg-bensinger}{%
  \subsection{Greg Bensinger}\label{greg-bensinger}}

  \hypertarget{technology}{%
  \subsubsection{Technology}\label{technology}}

  \includegraphics{https://static01.nyt.com/newsgraphics/2020/01/12/editorial-board-page/5610fc92ee24704dfbe34ffd1d73d67323eff42c/gregbesinger.jpg}

  Greg Bensinger joined The Times editorial board in 2020. He was a
  staff writer for the Washington Post and The Wall Street Journal where
  he covered the world's largest technology companies. Prior to that he
  worked for Bloomberg News writing about the telecommunications and
  auto industries. A native of Seattle, he holds a B.A. in English
  literature from the University of Virginia and an M.S. in Journalism
  from Columbia University. He now lives in San Francisco.

  Follow on Twitter
  \href{https://twitter.com/GregBensinger}{@greg.bensinger}
\item
  \hypertarget{michelle-cottle}{%
  \subsection{Michelle Cottle}\label{michelle-cottle}}

  \hypertarget{us-politics}{%
  \subsubsection{U.S. Politics}\label{us-politics}}

  \includegraphics{https://static01.nyt.com/images/2018/09/19/opinion/michelle_cottle/michelle_cottle-custom1.jpg}

  Michelle Cottle has covered Washington and national politics since the
  Clinton administration. She joined The Times in 2018 as the editorial
  board's national political writer after reporting on the nation's
  capital as a contributing editor for The Atlantic. Before that, Ms.
  Cottle was a senior writer at National Journal specializing in
  long-form profiles. From 2010 to 2014 she served as a Washington
  correspondent for Newsweek and the Daily Beast. Earlier, she was a
  longtime senior editor at The New Republic; some of her work there
  later appeared in "The Best American Political Writing of 2009." She
  also was an editor of The Washington Monthly magazine. Born and raised
  in the South, she has a B.A. in English from Vanderbilt University.

  Follow on Twitter \href{http://twitter.com/mcottle}{@mcottle}
\item
  \hypertarget{mara-gay}{%
  \subsection{Mara Gay}\label{mara-gay}}

  \hypertarget{new-york-state-and-local-affairs}{%
  \subsubsection{New York State and Local
  Affairs}\label{new-york-state-and-local-affairs}}

  \includegraphics{https://static01.nyt.com/images/2018/05/08/opinion/mara-gay-rectangle/mara-gay-rectangle-custom1-v2.jpg}

  Mara Gay joined The New York Times in 2018. Before coming to The
  Times, she was a City Hall reporter at The Wall Street Journal,
  covering Mayors Bill de Blasio and Michael Bloomberg, and dozens of
  other stories that have shaped the nation's largest, most dynamic
  city. Ms. Gay has also worked for the New York Daily News, The
  Atlantic and The Daily, an all-digital newspaper owned by News Corp.
  She has a degree in political science from the University of Michigan,
  Ann Arbor, is a New York City native and lives in Brooklyn.

  Follow on Twitter \href{http://twitter.com/MaraGay}{@MaraGay}
\item
  \hypertarget{carol-giacomo}{%
  \subsection{Carol Giacomo}\label{carol-giacomo}}

  \hypertarget{foreign-affairs}{%
  \subsubsection{Foreign Affairs}\label{foreign-affairs}}

  \includegraphics{https://static01.nyt.com/packages/images/opinion/EditorialBoard/rsz_1cgiacomo_1.jpg}

  Carol Giacomo, a former diplomatic correspondent for Reuters in
  Washington, covered foreign policy for the international wire service
  for more than two decades before joining The Times editorial board in
  August 2007. In her previous position, she traveled over 1 million
  miles to more than 100 countries with eight secretaries of state and
  various other senior U.S. officials. Her reporting for the editorial
  board involves regular independent overseas travel, including recent
  trips to North Korea, Iran and Myanmar. In 2018, she won the Arthur
  Ross Media award for foreign affairs commentary from The American
  Academy of Diplomacy, an organization of former U.S. diplomats. In
  2009, she won the Georgetown University Weintal Prize for diplomatic
  reporting. She is a member of the Council on Foreign Relations. In
  1999-2000, she was a senior fellow at the U.S. Institute of Peace,
  researching U.S. economic and foreign policy decision-making during
  the Asian financial crisis. She was a Ferris professor of journalism
  at Princeton University in 2013 and is a frequent public speaker at
  academic institutions, think tanks and on media shows. Raised in
  Connecticut, she holds a B.A. in English Literature from Regis
  College, Weston, Mass. She began her professional journalism career at
  the Lowell Sun in Lowell, Mass., and later worked for the Hartford
  Courant in the city hall, state capitol and Washington bureaus.

  Follow on Twitter \href{http://twitter.com/giacomonyt}{@giacomonyt}
\item
  \hypertarget{jeneen-interlandi}{%
  \subsection{Jeneen Interlandi}\label{jeneen-interlandi}}

  \hypertarget{health-and-science}{%
  \subsubsection{Health and Science}\label{health-and-science}}

  \includegraphics{https://static01.nyt.com/newsgraphics/2020/01/12/editorial-board-page/5610fc92ee24704dfbe34ffd1d73d67323eff42c/jeneen-interlandi-custom1-v3.jpg}

  Jeneen Interlandi joined the Times editorial board in 2018. She is a
  staff writer for The New York Times Magazine, where she has written
  about health, science and education since 2006. She has been a staff
  writer at Consumer Reports and Newsweek, and a freelance journalist
  for several national magazines. Ms. Interlandi was a 2013 Harvard
  University Nieman Fellow. She holds an M.A. in Environmental Science
  and M.S. in Journalism, both from Columbia University.

  Follow on Twitter \href{http://twitter.com/JInterlandi}{@JInterlandi}
\item
  \hypertarget{lauren-kelley}{%
  \subsection{Lauren Kelley}\label{lauren-kelley}}

  \hypertarget{women-and-reproductive-rights}{%
  \subsubsection{Women and Reproductive
  Rights}\label{women-and-reproductive-rights}}

  \includegraphics{https://static01.nyt.com/images/2018/02/28/opinion/lauren_kelly/lauren_kelly-custom1.jpg}

  Lauren Kelley joined the Times editorial board in 2018. Previously,
  she was the online politics editor at Rolling Stone, where she led
  coverage of the 2016 presidential election, the Trump administration
  and Congress. Before that, she was the managing editor at Rewire, an
  outlet focused on reproductive health and rights, as well as an editor
  at Alternet and a staff writer at Philanthropy News Digest. A native
  of Dallas, Ms. Kelley holds a B.A. in English literature from Texas
  Christian University. She now lives in Brooklyn, N.Y.

  Follow on Twitter
  \href{http://twitter.com/lauren_kelley}{@lauren\_kelley}
\item
  \hypertarget{alex-kingsbury}{%
  \subsection{Alex Kingsbury}\label{alex-kingsbury}}

  \hypertarget{technology-and-national-affairs}{%
  \subsubsection{Technology and National
  Affairs}\label{technology-and-national-affairs}}

  \includegraphics{https://static01.nyt.com/images/2019/04/01/opinion/alex-kigsbury-shot/alex-kigsbury-shot-custom2-v3.jpg}

  Alex Kingsbury has been with The Times and a member of its editorial
  board since 2018. Previously, he sat on the editorial board of The
  Boston Globe and was deputy editor of The Globe's Ideas section.
  Before that, Mr. Kingsbury was a senior associate producer at WBUR,
  Boston's NPR news station, for the programs "On Point With Tom
  Ashbrook" and "Radio Boston." From 2004 to 2011, he was an editor at
  U.S. News \& World Report. Born and raised in New England, he holds a
  B.A. in history from George Washington University and an M.S. from the
  Columbia University Graduate School of Journalism.
\item
  \hypertarget{serge-schmemann}{%
  \subsection{Serge Schmemann}\label{serge-schmemann}}

  \hypertarget{international-affairs}{%
  \subsubsection{International Affairs}\label{international-affairs}}

  \includegraphics{https://static01.nyt.com/images/2013/11/19/opinion/Schmemann_new/Schmemann_new-custom1.jpg}

  Serge Schmemann joined the Times in 1980. He served as the editorial
  page editor of the International Herald Tribune in Paris from 2003 to
  2013. He has been a Times correspondent and bureau chief in Moscow,
  Bonn, Jerusalem and the United Nations. He served as the deputy
  foreign editor in New York from 1999 to 2001. Mr. Schmemann received
  the Pulitzer Prize in 1991 for coverage of the reunification of
  Germany, and an Emmy in 2003 for his work on a television documentary
  about the Israeli-Palestinian conflict. He was previously a reporter
  with the Associated Press. Mr. Schmemann is a graduate of Harvard
  College and holds an M.A. from Columbia University, as well as an
  honorary doctorate from Middlebury College. He was born in Paris, is
  married and has three children.
\item
  \hypertarget{brent-staples}{%
  \subsection{Brent Staples}\label{brent-staples}}

  \hypertarget{education-criminal-justice-economics}{%
  \subsubsection{Education, Criminal Justice,
  Economics}\label{education-criminal-justice-economics}}

  \includegraphics{https://static01.nyt.com/packages/images/opinion/EditorialBoard/BrentStaples.jpg}

  Brent Staples joined The Times editorial board in 1990. His editorials
  and essays are included in dozens of college readers throughout the
  United States and abroad. Before joining the editorial page, he served
  as an editor of The New York Times Book Review and an assistant editor
  for metropolitan news. Mr. Staples holds a Ph.D. in psychology from
  the University of Chicago and is author of "Parallel Time," a memoir,
  which was a finalist for the Los Angeles Times Book Prize and winner
  of the Anisfield-Wolf Book Award.

  Follow on Twitter \href{http://twitter.com/BrentNYT}{@BrentNYT}
\item
  \hypertarget{jesse-wegman}{%
  \subsection{Jesse Wegman}\label{jesse-wegman}}

  \hypertarget{the-supreme-court-legal-affairs}{%
  \subsubsection{The Supreme Court, Legal
  Affairs}\label{the-supreme-court-legal-affairs}}

  \includegraphics{https://static01.nyt.com/images/2013/08/02/opinion/JesseWegman/JesseWegman-custom1.jpg}

  Jesse Wegman joined the editorial board in 2013. He was previously a
  senior editor at The Daily Beast and Newsweek, a legal news editor at
  Reuters, and the managing editor of The New York Observer. In 2010, he
  received a Soros Justice Fellowship to write a book about jailhouse
  lawyers. He graduated from New York University School of Law in 2005.
  Before that, he was a producer and reporter for several National
  Public Radio programs.

  Follow on Twitter \href{http://twitter.com/jessewegman}{@jessewegman}
\item
  \hypertarget{john-broder}{%
  \subsection{John Broder}\label{john-broder}}

  \hypertarget{associate-editor}{%
  \subsubsection{Associate Editor}\label{associate-editor}}

  \includegraphics{https://static01.nyt.com/images/2018/02/28/opinion/john-broder/john-broder-custom1-v2.jpg}

  Mr. Broder joined the editorial board at the start of 2018. He most
  recently served as deputy climate editor and director of news surveys,
  with responsibility for overseeing the New York Times Poll and polling
  analysis. Before that, he was digital editor for Europe, based in
  Paris. Mr. Broder was a correspondent for the Washington bureau from
  2006 to 2012, covering energy and environmental issues, as well as the
  2008 campaign. Previously, Mr. Broder had served as the Los Angeles
  bureau chief, Washington editor and White House correspondent during
  the Clinton administration. He also served as the newspaper's Business
  Day Washington reporter covering regulatory and legal issues,
  including tobacco negotiations, anti-trust enforcement, consumer and
  financial regulation. He joined The Times in 1996.

  Before joining The Times, Mr. Broder served as a reporter in the
  Washington bureau of the Los Angeles Times from 1987 to 1996, where he
  covered the White House, intelligence, international economics and
  defense. From 1985 to 1987, Mr. Broder was a business writer at the
  Los Angeles Times, covering banking and finance.

  From 1978 to 1984, he was roving state correspondent, state capitol
  correspondent and photo editor at The Detroit News. Mr. Broder began
  his journalism career as a photojournalist at The Cleveland Press, the
  Kettering-Oakwood (Ohio) Times and the Charlevoix (Michigan) Courier.

  Mr. Broder is married to Karolyn Wallace, an elementary school
  teacher. They have two adult children.
\item
  \hypertarget{nick-fox}{%
  \subsection{Nick Fox}\label{nick-fox}}

  \hypertarget{editor}{%
  \subsubsection{Editor}\label{editor}}

  \includegraphics{https://static01.nyt.com/images/2017/06/12/opinion/nick-fox/nick-fox-custom1.jpg}

  Nick Fox has been an editor at The Times since 1995, having previously
  worked as the assignment editor on the National desk, in the Dining
  section and with the online opinion forum Room for Debate. He
  previously worked for Newsday and The Bergen Record. He has a B.A.
  from Binghamton University and an M.A. from the Columbia University
  Graduate School of Journalism.
\end{itemize}

\begin{itemize}
\item
\item
\item
\item
\end{itemize}

Advertisement

\protect\hyperlink{after-bottom}{Continue reading the main story}

\hypertarget{site-index}{%
\subsection{Site Index}\label{site-index}}

\hypertarget{site-information-navigation}{%
\subsection{Site Information
Navigation}\label{site-information-navigation}}

\begin{itemize}
\tightlist
\item
  \href{https://help.nytimes.com/hc/en-us/articles/115014792127-Copyright-notice}{©~2020~The
  New York Times Company}
\end{itemize}

\begin{itemize}
\tightlist
\item
  \href{https://www.nytco.com/}{NYTCo}
\item
  \href{https://help.nytimes.com/hc/en-us/articles/115015385887-Contact-Us}{Contact
  Us}
\item
  \href{https://www.nytco.com/careers/}{Work with us}
\item
  \href{https://nytmediakit.com/}{Advertise}
\item
  \href{http://www.tbrandstudio.com/}{T Brand Studio}
\item
  \href{https://www.nytimes.com/privacy/cookie-policy\#how-do-i-manage-trackers}{Your
  Ad Choices}
\item
  \href{https://www.nytimes.com/privacy}{Privacy}
\item
  \href{https://help.nytimes.com/hc/en-us/articles/115014893428-Terms-of-service}{Terms
  of Service}
\item
  \href{https://help.nytimes.com/hc/en-us/articles/115014893968-Terms-of-sale}{Terms
  of Sale}
\item
  \href{https://spiderbites.nytimes.com}{Site Map}
\item
  \href{https://help.nytimes.com/hc/en-us}{Help}
\item
  \href{https://www.nytimes.com/subscription?campaignId=37WXW}{Subscriptions}
\end{itemize}
