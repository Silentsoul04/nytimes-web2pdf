 **NYTimes.com no longer supports Internet Explorer 9 or earlier. Please
upgrade your browser.
\href{http://www.nytimes.com/content/help/site/ie9-support.html}{LEARN
MORE »}

**Sections

**Home

**Search

\hypertarget{the-new-york-times}{%
\subsection{\texorpdfstring{\href{http://www.nytimes.com/}{The New York
Times}}{The New York Times}}\label{the-new-york-times}}

\hypertarget{-world-}{%
\subsubsection{\texorpdfstring{ \href{/section/world}{World}
}{ World }}\label{-world-}}

 \href{/section/world/americas}{Americas} \textbar{}Inside Gang
Territory in Honduras: `Either They Kill Us or We Kill Them.'

**Close search

\hypertarget{site-search-navigation}{%
\subsection{Site Search Navigation}\label{site-search-navigation}}

Search NYTimes.com

**Clear this text input

Go

\url{https://nyti.ms/2DRccxY}

\hypertarget{site-navigation}{%
\subsection{Site Navigation}\label{site-navigation}}

\hypertarget{site-mobile-navigation}{%
\subsection{Site Mobile Navigation}\label{site-mobile-navigation}}

\hypertarget{inside-gang-territory-in-honduras-either-they-kill-us-or-we-kill-them}{%
\section{Inside Gang Territory in Honduras: `Either They Kill Us or We
Kill
Them.'}\label{inside-gang-territory-in-honduras-either-they-kill-us-or-we-kill-them}}

By MAY 4, 2019

The Times spent weeks with a group of young men as they fought for their
lives in Honduras. All they had was a few blocks in one of the world's
deadliest cities. They would die to protect it.

Inside Gang Territory In Honduras

\hypertarget{either-they-kill-us-or-we-kill-them}{%
\section{`Either They Kill Us or We Kill
Them'}\label{either-they-kill-us-or-we-kill-them}}

In one of the deadliest cities in the world, an embattled group of young
men had little but their tiny patch of turf --- and they would die to
protect it. Journalists from The New York Times spent weeks recording
their struggle.

By Azam AhmedPhotographs by Tyler HicksMay 13, 2019

\href{https://www.nytimes.com/es/2019/05/04/honduras-mara-salvatrucha-violencia/}{Leer
en español}

SAN PEDRO SULA, Honduras --- Three sharp cracks rang out, followed by
three more in quick succession. The thoroughfare emptied. Two old men
ducked behind a corrugated fence. A taxi jerked onto a side street. A
mother shoved her barefooted toddler indoors.

The shooter, an MS-13 gunman in a tank top and black baseball cap, stood
calmly on the corner in broad daylight, the only person left on the
commercial strip. He stuck the gun in his waistband and watched the
neighborhood shake in terror.

Bryan, Reinaldo and Franklin scrambled into a neighbor's dirt yard,
scattering chickens. In panicked whispers, they traded notes on the
shooting, the third in less than a week. Only days earlier, a child had
been hit in a similar attack. Bryan, 19, wondered what response the few
young men still living in the neighborhood could muster, if any.

Mara Salvatrucha, the gang known as MS-13, was coming for them almost
every day now. It raided homes, deployed spies and taunted them with
whistles at dusk, a constant reminder that the enemy was right around
the corner, able to charge in at will.

There was no avoiding it. The neighborhood, a patch of unpaved roads no
bigger than a few soccer fields, was surrounded on all sides.

To the east, near the Chinese takeout where the three friends used to
splurge on fried rice, MS-13 was planning its takeover of the area. To
the south, past the house repurposed as an evangelical church, the 18th
Street gang was plotting to do the same. North and west were no better.
Gangs lined those borders, too.

In reality, not much differentiated the neighborhood where Bryan and his
friends had grown up from the ones already controlled by gangs. There
was a sameness to them --- the concrete homes worn by age; the handcarts
offering fried chicken and tortillas; the laborers trudging to work at
sunrise, waiting for buses on busy corners.

But for Franklin, whose family had been there for generations and who
had a child of his own on the way, the neighborhood was his entire
world. Reinaldo and Bryan felt the same way.

Only bad options remained for them: stay and fight, abandon their homes
and head elsewhere, maybe to the United States, or surrender and hope
one of the invading gangs showed them mercy.

All three had been members of the 18th Street gang, but were sickened by
the cadence of murder, extortion and robbery of their neighbors, the
people they had known all their lives. Seeking redemption, they kicked
the gang out of the neighborhood, vowing never to allow another back in.

Now, they were being hunted --- by their former comrades in 18th Street,
and by MS-13, which wanted their territory.

And so the young men doubled down for their own protection, transforming
back into the thing they hated most: a gang.

``The borders surround us like a noose,'' said Bryan, standing in the
yard with the others in their group, the Casa Blanca. ``We don't want
the gangs here, and for that we live in constant conflict.''

Reinaldo, 22, stood guard, watching the street for any signs of
movement.

``Lots of people ask me why we're fighting for this little plot of
land,'' he said. ``I tell them I'm not fighting for this territory. I'm
fighting for my life.''

Casa Blanca

territory

Los Olanchanos

territory

Much of Casa Blanca's

neighborhood is

surrounded by MS-13

Los Vatos Locos

territory

3 Calle

MS-13

territory

Los Tercereños

territory

Southeast

San Pedro Sula

RIVERA

HERNANDEZ

33 Calle

Bulevar del Este

Members of

Casa Blanca used to be

part of the 18th Street gang

18th Street gang

territory

Airport

Casa Blanca

territory

Los Olanchanos

territory

Los Vatos Locos

territory

3 Calle

MS-13

territory

MS-13

territory

Los Tercereños

territory

Southeast

San Pedro Sula

Much of Casa Blanca's

neighborhood is

surrounded by MS-13

RIVERA

HERNANDEZ

33 Calle

Bulevar del Este

Members of Casa Blanca

used to be part of the

18th Street gang

18th Street gang

territory

Airport

Casa Blanca

territory

Los Olanchanos

territory

Los Vatos Locos

territory

MS-13

territory

MS-13

territory

Los Tercereños

territory

Much of Casa Blanca's

neighborhood is

surrounded by MS-13

Southeast

San Pedro Sula

33 Calle

Members of Casa Blanca

used to be part of the

18th Street gang

Bulevar del Este

18th Street gang

territory

Airport

Casa Blanca

territory

Los Olanchanos

territory

MS-13

territory

Los Vatos Locos

territory

MS-13

territory

Los Tercereños

territory

Much of Casa Blanca's neighborhood is surrounded by MS-13

18th Street gang

territory

Bulevar del Este

Members of Casa Blanca used to be part of the 18th Street gang

San Pedro

Sula

By Derek Watkins \textbar{} Sources: Times reporting and the Asociación
para una Sociedad más Justa, a local nonprofit organization.

Some of the boundaries shown here are imprecise, and uncolored parts of
the map show areas where gang control is unclear.

From 2018 through early 2019, The New York Times followed the young men
of Casa Blanca in this tiny corner of San Pedro Sula, Honduras, one of
the deadliest cities in the world, and witnessed firsthand as they tried
to keep the gangs at bay.

Caribbean Sea

GUATEMALA

San Pedro Sula

HONDURAS

Tegucigalpa

EL SALVADOR

NICARAGUA

Pacific Ocean

Caribbean

Sea

GUATEMALA

San Pedro Sula

HONDURAS

Tegucigalpa

EL SALVADOR

NICARAGUA

Pacific

Ocean

Shootouts, armed raids and last-minute pleas to stop the bloodshed
formed the central threads of their stories. MS-13 wanted the
neighborhood to sell drugs. The other gangs wanted it to extort and
steal. But the members of Casa Blanca had promised never to let their
neighborhood fall prey to that again. And they would die for it, if they
had to.

Almost no one was trying to stop the coming war --- not the police, not
the government, not even the young men themselves. The only person
working to prevent it was a part-time pastor who had no church of his
own and bounced around the neighborhood in a beat-up yellow hatchback,
risking his life to calm the warring factions.

``I'm not in favor of any gang,'' said the pastor, Daniel Pacheco,
rushing to the Casa Blanca members after the shooting. ``I'm in favor of
life.''

The struggle to protect the neighborhood --- roughly four blocks of
single-story houses, overgrown lots and a few stores selling chips and
soda --- encapsulates the inescapable violence that entraps and expels
millions of people across Latin America.

Since the turn of this century, more than 2.5 million people have been
killed in the homicide crisis gripping Latin America and the Caribbean,
according to the Igarapé Institute, a research group that tracks
violence worldwide.

The region accounts for just 8 percent of the global population, yet 38
percent of the world's murders. It has 17 of the 20 deadliest nations on
earth.

And in just seven Latin American countries --- Brazil, Colombia,
Honduras, El Salvador, Guatemala, Mexico and Venezuela --- violence has
killed more people than the wars in Afghanistan, Iraq, Syria and Yemen
combined.

\hypertarget{most-of-the-worlds-most-dangerous-cities-are-in-latin-america}{%
\subsection{Most of the world's most dangerous cities are in Latin
America}\label{most-of-the-worlds-most-dangerous-cities-are-in-latin-america}}

Latin America

Africa

U.S.

Other

Safer cities

MORE DANGEROUS

Nova Iguaçu,

Brazil

Cancún,

Mexico

San Pedro Sula,

Honduras

Kingston,

Jamaica

San Salvador

London

Los Angeles

Paris

Tokyo

Istanbul

Los Cabos,

Mexico

Tijuana,

Mexico

Bogotá,

Colombia

St. Louis

Moscow

New Orleans

0

6.2 global avg.

20

40

60

80

100

120

Average homicide rate per 100k people

Latin America

Africa

U.S.

Other

Safer cities

MORE DANGEROUS

Cancún,

Mexico

Kingston,

Jamaica

San Pedro Sula,

Honduras

San Salvador

London

Los Angeles

Paris

Tokyo

Istanbul

Los Cabos,

Mexico

Tijuana,

Mexico

Bogotá,

Colombia

St. Louis

Moscow

New Orleans

6.2 global avg.

0

40

60

80

100

120

Average homicide rate per 100k people

Latin America

Africa

U.S.

Other

Tokyo

0

Paris

London

Istanbul

New York

Moscow

6.2

global average

MORE

DANGEROUS

cities

10

Los Angeles

Bogotá, Colombia

20

Average homicide

rate per 100k people

30

Kingston, Jamaica

40

New Orleans

50

Nova Iguaçu,

Brazil

Baltimore

60

St. Louis

Cape Town

Cancún, Mexico

70

San Pedro Sula,

Honduras

80

90

Los Cabos, Mexico

100

Tijuana, Mexico

110

120

San Salvador

By Allison McCann

Source: Igarapé Institute and the United Nations Office on Drugs and
Crime. Cities include the 50 highest homicide rates in the world and a
group of prominent others for comparison, all with populations of at
least 250,000. Average homicide rates are from 2016-2018 or the latest
data available.

The violence is all the more striking because the civil wars and
military dictatorships that once seized Latin America have almost all
ended --- decades ago, in many cases. Most of the region has trudged,
often very successfully, along the prescribed path to democracy. Yet the
killings continue at a staggering rate.

They come in many forms: state-sanctioned deaths by overzealous armed
forces; the murder of women in domestic disputes, a consequence of
pervasive gender inequality; the ceaseless exchange of drugs and guns
with the United States.

Underpinning nearly every killing is a climate of impunity that, in some
countries, leaves more than 95 percent of homicides unsolved. And the
state is a guarantor of the phenomenon --- governments hollowed out by
corruption are either incapable or unwilling to apply the rule of law,
enabling criminal networks to dictate the lives of millions.

For the masses fleeing violence and poverty in Central America, the
United States is both a cause and solution --- the author of countless
woes and a chance to escape them.

Frustrated with the stream of migrants treading north, President Trump
has vowed to cut aid to the most violent Central American nations,
threatening hundreds of millions of dollars meant to address the roots
of the exodus.

But the surviving members of Casa Blanca, who once numbered in the
dozens, do not want to flee, like tens of thousands of their countrymen
have. They say they have jobs to keep, children to feed, families,
neighbors and loved ones to protect.

``There is only one way for this to end,'' said Reinaldo. ``Either they
kill us or we kill them.''

\hypertarget{the-next-time-they-will-kill-me}{%
\paragraph{`The Next Time, They Will Kill
Me'}\label{the-next-time-they-will-kill-me}}

The men entered without a word, pushing through the thin curtain hanging
over Fanny's front door with the barrels of their AK-47s.

She let out a stifled yelp as they spread through the house, their
assault rifles shouldered. After the shooting the day before, the MS-13
gunman had watched Bryan, Reinaldo and Franklin race into Fanny's
backyard, one of the few places they felt safe.

Now it was night, and Fanny was alone. The men did a final sweep for
Casa Blanca members, then left as suddenly as they had entered. The
message was more terrifying for its silence: They could come and go as
they pleased.

A single mother of three, Fanny was a surrogate mother to the Casa
Blanca members. She had known them since childhood; they had defended
her son from bullies in grade school. As they grew up, her house became
a refuge, a place to escape broken homes.

And now, for her closeness with the young men, she had fallen into the
cross hairs of MS-13. Shaking with fear, she called her cousin, Pastor
Pacheco.

``The next time, they will kill me, I know it,'' she told him.

Fanny drew respect in the few blocks controlled by Casa Blanca, but she
had no sway beyond the neighborhood, which was where the pastor came in.
He knew the leaders of all the gangs.

He had a slight paunch and a wide face that permanently lingered on the
verge of a smile. An evangelical minister, he delivered Sunday sermons
outdoors in the stifling heat and worked construction to make ends meet.

Then in 2014, a 13-year-old girl in the neighborhood was kidnapped by
gang members. Her parents owned a small corner store and had failed to
pay their extortion demands. As retribution, they abducted the girl and
took her to a private home, where they raped and tortured her for three
days before killing her and burying her in the floor.

``People watched as they grabbed her from the street, yelling for help,
and no one did anything,'' recalled Mr. Pacheco, 40, known mostly as
Pastor Danny. ``They were all scared for their lives.''

Pastor Danny's daughter was the same age as the girl. Overwhelmed, he
visited the house after the police had cleared the scene. The shallow
grave was still open, a small hole in the living room, scraped out of
the clay floor. He filled it with his hands.

``I made a promise there,'' he said. ``I was going to do something.''

Four years on, he still kept the newspaper clippings of the murder, to
remind him of that promise.

Most days, he shuttled around the gouged-out streets in his hatchback, a
car recognized across gang lines. More than once, he had intervened when
the police were beating gang members, or placed himself between rival
gangs on the verge of killing one another.

He resented the government, the arbitrary brutality of the police and
the relentless corruption that had driven so many Hondurans to
\href{https://www.nytimes.com/2018/10/24/world/americas/migrant-caravan-trump.html}{leave
in caravans to the United States}. Though murders in his country were
dropping, he often said, the underlying problems weren't.

Now, with Fanny's life on the line, it was personal. The pastor knew
many of the Casa Blanca members and appreciated the quandary they faced.
He didn't want gangs to dominate the neighborhood, either.

But he was a realist --- there was no way to keep them out. MS-13 had
made its intentions clear. It was advancing across large swaths of San
Pedro Sula, using its numbers, tight organization and ruthlessness to
overwhelm smaller, less sophisticated groups.

The way he saw it, Casa Blanca was next. And the invasion was coming,
one way or another.

Casa Blanca now had fewer than a dozen members in all. Some had been
killed, others imprisoned. The remaining ones were the least experienced
in gang warfare. A few were barely old enough to shave.

Bryan worked 12-hour shifts in a factory and began his day at 5:30 a.m.
To avoid ambushes, he crept out of the neighborhood each morning, then
back in at night. He hardly slept. A combination of fear and candy kept
him awake on the job.

He arguably had the least to fight for, living alone in a one-bedroom
apartment, estranged from his mother. He only heard from her every other
week, when he got paid.

``She's not like other mothers,'' he tried to explain, embarrassed.

Franklin, 19, worked construction, when there was work. He had a steady
girlfriend and wanted out of the violence because of the child on the
way. But he had a brother who harbored no such fantasies. When the time
came, he said, his brother would die fighting.

Reinaldo was the quietest. When others boasted of their exploits, he
laughed softly but never joined in. He rarely raised his voice and was
tender at times, wrapping Fanny's youngest son in an embrace after she
scolded him for collecting gun shells from the street and shaking them
like dice.

Reinaldo wanted a way out, too, but refused to abandon his friends, or
the neighborhood. He could scarcely imagine himself anywhere else. His
expectations were as hemmed in as his movements.

If Casa Blanca had any leader left, it was Javi, in his early 20s,
frighteningly skinny and the most violent by nature. A braided scar ran
from his right cheek down to his throat, compliments of a gang that had
kidnapped him a year earlier. Everyone called him the Macheted.

In November, Javi had taken off for Guatemala in search of a fresh
start. Now, he was back.

``I can't leave this place,'' he explained. ``It's my home. I'm not
going to run away.''

Like the legions of young men swept up in the region's homicide
epidemic, they felt trapped in a cycle they were powerless to break.
Even in trying to escape the violence --- by defecting from gangs
altogether --- they had only managed to summon more of it.

Pastor Danny considered it a good sign that MS-13 had not harmed Fanny.
But the sudden escalation worried him. There would be more bullets, more
casualties. He was sure of it. Bryan, Franklin and the others could not
even spend a quiet afternoon in Fanny's backyard anymore. It was marked
now.

So the pastor made a plan, one that bordered on diplomatic lunacy.

He wanted to broker a meeting between Casa Blanca members and MS-13, the
gang threatening their lives.

\hypertarget{life-was-good}{%
\paragraph{`Life Was Good'}\label{life-was-good}}

Anner stood shirtless on his porch, watching his daughter play
tug-of-war with a small dog.

``This is going to be hard,'' he warned the pastor. ``These guys have
lost too much to just give up.''

Anner, 26, was a workingman. He stocked produce at a grocery store, and
felt proud of the small house and motorcycle it afforded him. He had
grown up with everyone in Casa Blanca. He was not a member, but two of
his brothers-in-law, including Franklin, were.

The pastor needed Anner to convince Casa Blanca that peace was the only
way. They grabbed Franklin and went inside, where the air-conditioner
ran full blast in a losing battle with the heat. Anner wanted the pastor
to understand what he was up against --- the feudal history of Casa
Blanca.

In the early 2000s, he explained, the territory belonged to 18th Street,
and the local members operated from a white house, or Casa Blanca.

In 2016, though, a police operation landed the leaders in jail, leaving
the neighborhood up for grabs. A new gang stepped in and the locals, who
still referred to themselves as Casa Blanca, joined.

But the new gang was brutal and petty. It killed residents for failing
to hand over extortion payments, and robbed them even when they
complied. The Casa Blanca members were ashamed --- people they had grown
up with were suffering at their hands.

They revolted, seeking the help of a faction of 18th Street. When they
prevailed months later, they joined 18th Street once again.

But the threats, robberies and violence continued. They had lost people,
and for what, Anner asked. Simply to swap out one gang's abuse for
another's?

So they mutinied again and won, kicking out 18th Street after months of
bloodshed.

``They converted into an anti-gang group,'' Anner said. ``Life was good.
No more robberies, no more extortion and no more violence against people
living in the neighborhood.''

``And then,'' he said, ``the police came.''

Through the summer of 2017, the police arrested half a dozen Casa Blanca
members. Others fled. The ranks were decimated, leaving the
lowest-profile members on the street.

``Now the young ones are left alone,'' said Anner.

He listed the survivors, and how they would respond to an MS-13
takeover. Franklin's older brother wouldn't take it well, he said. He
had shot at MS-13 in the past and refused even to sit down with the
pastor.

Franklin nodded in agreement. ``He says the only truce he needs is the
one he carries in his waistband,'' Franklin said, miming a gun with his
hand.

Others might be game for a truce, Anner said, but the older members,
when they got out of prison, could break whatever agreement was made.

Pastor Danny realized what he was wading into: Casa Blanca was
leaderless and unpredictable, governed by young men whose instincts for
self-preservation were in constant conflict with their bravado.

``If something doesn't change, there's going to be a massacre before the
end of the year,'' Pastor Danny snapped.

``End of the year?'' Anner snorted. ``I think more like end of the
week.''

At that moment, a loud crash erupted, the sound of a rock hitting
Anner's roof. The group raced outside. Franklin signaled for them to be
quiet.

``MS-13 is on the block,'' he whispered, pointing up the road.

The street was long and narrow, running for more than 30 meters, like a
firing range. The pastor, worried about MS-13 gunmen, called the police.

The block cleared out, except for a middle-aged woman walking slowly
down the street, alone. After she passed, Anner sighed in relief.

She was the sister of one of MS-13's leaders, and most likely a lookout,
Anner explained.

``She's a lookout?'' Pastor Danny asked sharply, pointing down the
block. ``That woman was a lookout?''

He was furious at the missed opportunity. Had he known, he told Anner,
he would have introduced himself, to ease the tension. As a religious
leader, he would pose no threat, he argued.

Instead, they waited on Anner's porch, praying the gangsters down the
way would hold their fire. After half an hour, the pastor made a break
for his car, flooring it on his way out.

As he cleared the neighborhood, the police arrived. The pastor rolled
down his window to brief them, surprised they had even shown up.

But before he could say anything, the officers ordered him out of the
car. The pastor thought it was a joke, until the officer's voice grew
stern.

``But I'm the one who called you guys,'' Pastor Danny protested.

The police made a few calls before waving the pastor on. He twisted his
hands over the steering wheel and muttered an expletive.

``And you wonder why we have to solve our own problems,'' he said.

\hypertarget{the-last-card-i-have-to-play}{%
\paragraph{`The Last Card I Have to
Play'}\label{the-last-card-i-have-to-play}}

The pastor slowed at the knot of unpaved streets separating MS-13 from
Casa Blanca. He switched on his hazards and eased past a desolate cinder
block structure, where the outlines of young men were visible in the
glow of cigarette tips.

A man with tattoos covering his arms and neck appeared at Pastor Danny's
window.

``What do you want,'' he asked, taking a long glance up and down the
street.

``I want to see Samuel,'' the pastor said. ``We know each other.''

Just hours after leaving Anner's house, the pastor had received an
alarming call. Armed men on motorcycles were kicking families out of
their homes in Casa Blanca's area, taking the neighborhood by force. He
couldn't wait any longer.

So Pastor Danny fell back on his usual tactic --- improvisation --- and
raced into MS-13 territory, hoping to place himself at the mercy of
Samuel, the MS-13 leader in the area, before someone died.

``This is the last card I have to play,'' he said.

The pastor scanned the vacant lots and darkened buildings, taking heavy
breaths to steady himself. He was used to taking risks, but this was
different --- Samuel was an important figure, not just a soldier with a
quick temper. Even asking for him could arouse suspicion. And scared
criminals were dangerous.

The tattooed man stepped back and surveyed the street a second time.
Satisfied, he pointed to a peach-colored home. ``Check there,'' he said.

The pastor drove by a well-lit corner, where two women were smoking with
a slender man in a collared shirt and jeans.

It was Samuel. The pastor slammed on the brakes and leapt out of the
car, leaving it in the middle of the street with the door still open.

Samuel excused himself from the conversation with the women and stubbed
out his cigarette. He looked to be in his 30s, with short hair and the
calm bearing of someone used to being in control.

He walked over and embraced the older man. ``Pastor Danny, how are
you?'' he asked.

``I'm not great, brother,'' the pastor said. He often took his time when
enlisting people's help, spooling them up slowly. He was, at heart,
something of a performer.

But now, nervous and somewhat stunned at finding Samuel, Pastor Danny
got straight to the point.

``I have to ask you a personal favor,'' he said.

Samuel raised his eyebrows and answered like a politician. ``If I can do
it, I will,'' he said.

``I know your guys are looking to move into the territory of Casa
Blanca,'' Pastor Danny continued. ``But I'm asking you, begging you,
please don't do it violently. Please don't kill anyone.''

Samuel listened impassively, saying nothing.

``I'm not in favor of any gang,'' the pastor went on, filling the
silence with his standard refrain. ``I just want to protect life. And I
have a cousin who lives there and I'm worried she and others could be
hurt.''

Samuel interrupted.

``We already own that territory,'' he said. ``It's already ours.''

The pastor didn't know whether he was speaking literally or
figuratively. MS-13, while advancing fast, had not yet taken over. This
much the pastor knew.

``But there are people there now, kicking a family out of their home,''
the pastor insisted. ``I have people in the community who are witnessing
it.''

Samuel leaned against the pastor's car, then, seeing it was covered in a
film of dirt, eased back off it.

``It can't be us. We don't have anyone there right now,'' he countered.
``What did they tell you?''

The pastor called Anner. ``What exactly is happening right now?'' he
said into the phone.

Anner told the pastor that men on motorcycles had come in masks and
kicked out a family half a block from Fanny's house.

``We don't have any motorcycles in that area,'' Samuel said, shaking his
head.

Anner corrected himself. The men had arrived on bicycles, he now said.
But he was sure they were kicking people out of their homes.

The back and forth continued, with Samuel asking the pastor and, by
extension, an unwitting Anner for more specifics. Anner grew suspicious.
The pastor tried his best to explain the location to Samuel, based on
the vague answers he could squeeze out of Anner.

``No that can't be right,'' Samuel said. ``Where you are talking about
is where the old woman sells firewood.''

Samuel sketched a map into the dirt covering the pastor's back
windshield. They took turns drawing streets and landmarks.

``I think where he is describing is here,'' Samuel said, tapping his
finger against the glass. ``And that's not in Casa Blanca territory.''

The pastor winced. Samuel was right. Whatever was happening, it was not
in Casa Blanca territory.

It didn't matter, though, Samuel said. Everyone knew that Casa Blanca
was weak. He had already ordered his lieutenant --- a man called Monster
--- to take over the neighborhood.

His men weren't forcing families out of homes tonight, he said, but they
would enter soon enough.

Samuel then asked the pastor to draw the exact location of Fanny's
house. ``Do not worry about your loved ones, we won't hurt them,'' he
promised.

And what about the Casa Blanca members, the pastor asked. Would they
also get a pass?

``Like I said, the territory is already ours,'' Samuel replied. ``If we
can avoid violence, we will. But that depends on them.''

Samuel relit his cigarette and walked into an abandoned building.

\hypertarget{we-make-our-money-selling-drugs}{%
\paragraph{`We Make Our Money Selling
Drugs'}\label{we-make-our-money-selling-drugs}}

Monster led the pastor into a backyard, where more than a dozen MS-13
soldiers stood in a circle, cloaked in a cloud of marijuana smoke. A boy
no older than 10 stood among them, his hat turned sideways, smoking a
cigarette.

Pastor Danny introduced himself. Two days had passed since his encounter
with Samuel. Now he was back in MS-13 territory, face to face with Casa
Blanca's enemies.

The gunman responsible for the shooting a few days earlier was there,
wearing the same black baseball cap. The men who raided Fanny's house
were there, too, standing next to a giant mound of dirt. The pastor kept
his gaze on Monster, the man ordered to take over the neighborhood.

When speaking to groups, Pastor Danny had a roundabout way of getting to
the point. He flattered, shared bits of intelligence, or preached
parables from the Bible, depending on his spot assessment of what would
get through to the crowd.

``You guys are a structure, a disciplined group with organization and
resources,'' he told them, drawing genuine smiles from the gang members.
``It will be hard for the members of Casa Blanca to fight back, and they
know that.''

At 26, Monster had become one of Samuel's top lieutenants. After
struggling to make a living in construction, the gang offered him
employment, and a community, he said.

It also taught him discipline, which was paramount: No lying to the
gang; no drug use (marijuana was the exception); and murder had to be
approved by the leadership, unless in self-defense.

``Killing someone isn't what helps you climb the ladder,'' Monster
explained. ``What matters is how you think, your intelligence,'' he
added, tapping his forefinger against his temple.

Monster spoke like a small-time official, spouting platitudes and
promises with ease. Top-notch security. Respect for residents. No forced
conscription. No extortion. It was a surprising speech for a member of a
gang that terrorizes people from Central America to the United States.

``We make our money selling drugs,'' Monster explained, ``so we don't
rob from the people who live in our areas.''

``We need them,'' he added.

It all sounded hopeful to the pastor, maybe too hopeful. There was no
way to know if Monster was telling the truth. They were killers, after
all, no matter what they said about peace.

Still, the pastor wanted to walk away with something concrete. The
conversation went on for more than an hour before he finally pushed his
plan.

``You know, it might help to meet one of them,'' the pastor said
casually, as if the idea had only just occurred to him. ``I mean, if
they're willing and you're willing.''

\hypertarget{paralyzed-by-fear}{%
\paragraph{`Paralyzed by Fear'}\label{paralyzed-by-fear}}

In the car, Fanny asked half-jokingly whether the pastor was taking her
to be killed. She had dressed up for the occasion, wearing bright red
lipstick.

``Don't be stupid, Fanny,'' he said. ``I'm trying to save your life.''

They were driving to his brother's house, outside of Casa Blanca
territory, so he could explain his meetings with MS-13.

``Fanny doesn't listen when she's at home,'' he explained. ``She's just
paralyzed by fear.''

Pastor Danny wore the same clothes for the third day in a row. Bags had
formed under his eyes. Between counseling Fanny and keeping Casa Blanca
from falling apart, there was little time for anything else, even his
own family.

His daughter had been hospitalized for a lung condition. When he wasn't
in the neighborhood, he was with his wife, checking on her. Bills were
piling up, and finances were not his thing. He preferred being in the
streets, his ministry of action.

And right now, Fanny's safety was his first concern.

``Fanny, you need to think about you and your family,'' the pastor said,
sensing her doubts. ``They told me they wouldn't touch you.''

Fanny began to cry. After the events of the last few days --- the
shooting, the invasion of her home --- the pastor thought she would be
happy with the news. But his promise that she would be safe merely
reminded her of all the others who wouldn't be.

``How would you feel if I told you that I could save your life, but
children you have known and loved since they were young might die?'' she
sobbed. ``How would you feel if I told you I could only save you?''

The pastor was confused, hurt even, after all the sacrifices he had
made, the chances he had taken. He often joked that there was no
gratitude for the work he did, and for the most part, he didn't expect
much. Still, he didn't want to be chastised for it.

He handed Fanny some tissue to wipe the mascara streaking down her face.

``If others in the neighborhood want to put up a fight and die, that's
their choice, I guess,'' Pastor Danny said, shrugging. ``I'm trying to
save the lives of those who want to be saved.''

Two days later, when the pastor decided to tell Casa Blanca about his
plan for a truce, Fanny didn't join. He gathered everyone at Anner's
house, including a few parents, hoping they might force the young men
into accepting it.

It was late evening. Bryan raced in after work, his hair still wet from
a shower. Franklin sat on a sofa, legs outstretched.

``They say they will pardon everyone as long as they can enter
peacefully,'' the pastor said, explaining MS-13's terms.

The pastor had a way of stretching the facts to their most optimistic
lengths. MS-13 had said it did not want to kill. But it never promised
to pardon everyone, not explicitly.

Bryan interjected, describing his most recent brush with MS-13 members.

``They didn't whistle, or look at me in any sort of aggressive way,'' he
marveled, crediting the pastor's efforts for the atypical behavior.

Whether the change was at all related, the meeting seemed to be going
well. And in the end, the pastor's true gospel was hope. If he could
make Casa Blanca believe that peace was possible, maybe it could be.

By the end of the discussion, Anner agreed to sit down with Monster.

``This is inevitable,'' Anner said. ``I mean, look at the odds --- it's
like 50,000 of them versus eight of us.''

\hypertarget{we-dont-want-any-problems}{%
\paragraph{`We Don't Want Any
Problems'}\label{we-dont-want-any-problems}}

Anner dressed in his work uniform, a polo shirt with the grocery store's
insignia stitched on the upper left pocket. His boss had given him a few
hours off, and Anner was anxious to get going.

In the back seat of Pastor Danny's car, Anner talked without pause, a
nervous habit that could make it hard to get a word in. The pastor hoped
he would settle down before they met Monster.

Then, suddenly, Anner grew quiet. He pressed his face to the tinted
window and stared.

``I haven't been on this street in seven years,'' he said as they passed
into MS-13 territory, struck by how such a small neighborhood could be
so rigidly divided --- and how isolated it left everyone.

They reached a building with a tin portico. Beneath it, Monster sat on a
low-slung chair, smoking weed. He smiled slightly as his visitors looked
for a seat. Anner found a splintered crate, the pastor an overturned
bucket.

After a brief introduction, Anner began to talk, in his nervous way, for
nearly the entire meeting --- about his kids, his job, his life in the
neighborhood. He even named a few MS-13 members he knew personally.

``I'm not involved in any of this, but I know all of these guys,'' he
explained.

Monster continued smoking. Inside the building, a pinball machine
clanged to life, playing ``Limbo Rock'' while gang members took turns.

``We don't want any problems with MS,'' Anner said, scooting his crate a
little too close to Monster.

``I don't want to see violence,'' he continued. ``I work and have a
family and I don't want to lose my house.''

Monster, now very high, shook his head and uttered a soft ``No.''

``What about the others?'' Anner asked. ``Some of these guys have shot
at MS before,'' he said. ``Sometimes out of fear.''

Monster started to speak, but Anner cut him off.

``I just want to ask as a favor that if they don't resist, if they don't
put up a fight, that you pardon them,'' he said.

Monster looked at the pastor, then at Anner.

``Our goal is not to kill anyone,'' he said. ``If they don't put up a
fight, if they go with the program, we won't need to.''

Anner slumped over slightly, his tension ebbing. ``Thank you, brother,
this is a big relief for me. We've all been so worried about what would
happen, every day. It's been like living in a war zone.''

Two cars rolled past and the drivers honked their horns to salute the
gathered MS-13 members. Children played nearby, kicking a small rubber
ball up and down the street.

``Look around,'' Monster boasted. ``People live more freely here than
anywhere else.''

``This could be how it is in Casa Blanca,'' he concluded.

\hypertarget{they-dont-care}{%
\paragraph{`They Don't Care'}\label{they-dont-care}}

The bodies appeared one January morning, mutilated, wrapped in black
trash bags and deposited on the border that divided Casa Blanca from the
18th Street gang.

The warning spoke for itself: 18th Street had learned of the burgeoning
truce with MS-13 --- and had no intention of accepting it.

A few weeks later, Reinaldo disappeared. He had been walking inside the
boundaries of Casa Blanca territory when someone snatched him.

Bryan and Franklin circulated his photo, in case anyone had seen him.
After a few days, the pastor learned that 18th Street had taken him.
They never got the body back.

The pastor's fragile peace began to crumble.

MS-13 never entered the neighborhood, as Samuel and Monster said it
would. Though it stopped attacking Casa Blanca, as promised, 18th Street
picked up where its rivals had left off.

The pastor tried to put the Casa Blanca members at ease, but he had
nothing new to offer. For all his efforts --- the one-man missions, the
clandestine meetings --- he had managed only to swap one enemy for
another.

Even that didn't last. Early this year, Samuel and Monster were
promoted. After they moved on, there was no one to guarantee the peace.
Monster's replacement in MS-13, Puyudo, resumed the attacks on Casa
Blanca --- why, exactly, the pastor did not know.

Casa Blanca was still outgunned, still outnumbered, still trapped. In
March, a young boy in its territory was wounded in a shootout. A few
days later, MS-13 took shots at Anner after work.

A week later, a member shot at Fanny while she was walking her son home
from school.

Pastor Danny's mission became much more daunting. He began saying that
his heart wasn't in it anymore. Trying to change the neighborhood, much
less all of San Pedro Sula, or the rest of Honduras, seemed futile.

In his mind, the fact that everything fell on him --- a solo
peacekeeping campaign, with no help from the government --- was a
reflection of how dire the situation was.

``All of the things that end here on the streets, it all starts with
government corruption,'' he said. ``I can't keep fighting against this
monster --- the government, the country. It doesn't matter to them. They
don't care.''

He told himself this would be his last intervention. However the Casa
Blanca standoff ended --- peacefully or not --- he vowed to find a life
where he wasn't fighting the monster, as he called the state, and could
take up a less demoralizing cause. Maybe he would even leave Honduras.

But that didn't last, either. His cynicism gave way to hope, as it
always did. A few weeks after MS-13 took shots at Fanny, the pastor
managed to meet with Puyudo, the new leader in the area. Pastor Danny's
disillusionment fell away.

He gave Puyudo an abridged version of the speech that, by now, he had
practiced a half-dozen times. He slipped right back into diplomacy mode.

``I think I can convince him to stop the shooting,'' the pastor said.
``We are supposed to meet again soon.''

\hypertarget{more-on-nytimescom}{%
\subsection{More on NYTimes.com}\label{more-on-nytimescom}}

Advertisement

\hypertarget{site-information-navigation}{%
\subsection{Site Information
Navigation}\label{site-information-navigation}}

\begin{itemize}
\tightlist
\item
  \href{https://help.nytimes.com/hc/en-us/articles/115014792127-Copyright-notice}{©
  2020 The New York Times Company}
\item
  \href{https://www.nytimes.com}{Home}
\item
  \href{https://www.nytimes.com/search/}{Search}
\item
  Accessibility concerns? Email us at
  \href{mailto:accessibility@nytimes.com}{\nolinkurl{accessibility@nytimes.com}}.
  We would love to hear from you.
\item
  \href{https://help.nytimes.com/hc/en-us/articles/115015385887-Contact-Us}{Contact
  Us}
\item
  \href{https://www.nytco.com/careers/}{Work with us}
\item
  \href{https://nytmediakit.com/}{Advertise}
\item
  \href{https://help.nytimes.com/hc/en-us/articles/115014892108-Privacy-policy\#pp}{Your
  Ad Choices}
\item
  \href{https://help.nytimes.com/hc/en-us/articles/115014892108-Privacy-policy}{Privacy}
\item
  \href{https://help.nytimes.com/hc/en-us/articles/115014893428-Terms-of-service}{Terms
  of Service}
\item
  \href{https://help.nytimes.com/hc/en-us/articles/115014893968-Terms-of-sale}{Terms
  of Sale}
\end{itemize}

\hypertarget{site-information-navigation-1}{%
\subsection{Site Information
Navigation}\label{site-information-navigation-1}}

\begin{itemize}
\tightlist
\item
  \href{https://spiderbites.nytimes.com}{Site Map}
\item
  \href{https://help.nytimes.com/hc/en-us}{Help}
\item
  \href{https://help.nytimes.com/hc/en-us/articles/115015385887-Contact-Us?redir=myacc}{Site
  Feedback}
\item
  \href{https://www.nytimes.com/subscription?campaignId=37WXW}{Subscriptions}
\end{itemize}
