Sections

SEARCH

\protect\hyperlink{site-content}{Skip to
content}\protect\hyperlink{site-index}{Skip to site index}

\hypertarget{comments}{%
\subsection{\texorpdfstring{\protect\hyperlink{commentsContainer}{Comments}}{Comments}}\label{comments}}

\href{}{48 Hours in the Strange and Beautiful World of
TikTok}\href{}{Skip to Comments}

The comments section is closed. To submit a letter to the editor for
publication, write to
\href{mailto:letters@nytimes.com}{\nolinkurl{letters@nytimes.com}}.

\hypertarget{48-hours-in-the-strange-and-beautiful-world-of-tiktok}{%
\section{48 Hours in the Strange and Beautiful World of
TikTok}\label{48-hours-in-the-strange-and-beautiful-world-of-tiktok}}

Oct. 10, 2019

\begin{itemize}
\item
\item
\item
\item
\item
  \emph{18}
\end{itemize}

The video app offers an endless scroll of creativity and goofing off,
told in 15-second snippets. What did five critics see when they went
down the rabbit hole? Art, artistry and a lot of dancing.

\hypertarget{48-hours-in-the-strange-and-beautiful-world-of-tiktok-1}{%
\section{48 Hours in the Strange and Beautiful World of
TikTok}\label{48-hours-in-the-strange-and-beautiful-world-of-tiktok-1}}

\hypertarget{the-video-app-offers-an-endless-scroll-of-creativity-and-goofing-off-told-in-15-second-snippets-what-did-five-critics-see-when-they-went-down-the-rabbit-hole-art-artistry-and-a-lot-of-dancing}{%
\paragraph{The video app offers an endless scroll of creativity and
goofing off, told in 15-second snippets. What did five critics see when
they went down the rabbit hole? Art, artistry and a lot of
dancing.}\label{the-video-app-offers-an-endless-scroll-of-creativity-and-goofing-off-told-in-15-second-snippets-what-did-five-critics-see-when-they-went-down-the-rabbit-hole-art-artistry-and-a-lot-of-dancing}}

\hypertarget{by-james-poniewozik}{%
\subsection{By James Poniewozik,}\label{by-james-poniewozik}}

\hypertarget{amanda-hess}{%
\subsection{Amanda Hess,}\label{amanda-hess}}

\hypertarget{jon-caramanica}{%
\subsection{Jon Caramanica,}\label{jon-caramanica}}

\hypertarget{gia-kourlas}{%
\subsection{Gia Kourlas}\label{gia-kourlas}}

\hypertarget{and-wesley-morris}{%
\subsection{and Wesley Morris}\label{and-wesley-morris}}

\hypertarget{in-2019}{%
\subsubsection{In 2019,}\label{in-2019}}

we take our entertainment in microdoses. A complete story may be
unspooled in a fleeting video snippet, a tweet, a GIF. The social media
app TikTok has built an entire world on that premise, amassing a vast
global collection of 15-second clips that are changing the way we sing,
dance, pose, joke, dress, collaborate and cook. It is home to comedy
sketches, dance challenges, makeshift runway shows and short-short
films. The most ambitious ones arrive as mini-epics, complete with
soundtracks, visual effects and narrative arcs.

\href{https://www.tiktok.com/@thisaintjay/video/6734489584509537541}{}

@thisaintjay

TikTok was started in 2017 by the Chinese company ByteDance, but only
recently has it gained real traction in the United States. (Though this
country is by no means the center of creative gravity, which is a part
of its appeal.) TikTok is a kind of Frankenstein's monster of
micro-video platforms --- reminiscent of the defunct app Vine, it is an
international clone of ByteDance's Chinese app Douyin, and last year
merged with the popular lip-syncing
app\href{http://musical.ly/}{}\href{http://musical.ly/}{Musical.ly} ---
but it is now a cultural force all its own. This year, it produced a
bona fide crossover celebrity: Lil Nas X emerged from the platform's
viral soup as a fully-formed pop star, shattering records with his song
``Old Town Road'' after it became a popular soundtrack for TikTok dance
routines.

And yet there is something about TikTok's presence in mainstream culture
--- as a testing ground for ``real'' stars, as an Emmys joke about what
the kids are into --- that underestimates the power of the thing itself.
It feels as if there are endless TikTok universes unfolding all at once.
And so last week, over 48 hours, five critics of The New York Times with
different specialties and varying familiarity with the app took a look
at what it has to offer.

\hypertarget{james-poniewozik}{%
\subparagraph{James Poniewozik}\label{james-poniewozik}}

\hypertarget{tv-critic}{%
\subparagraph{TV Critic}\label{tv-critic}}

\hypertarget{bite-size-low-tech-tv}{%
\subsubsection{Bite-Size, Low-Tech TV}\label{bite-size-low-tech-tv}}

The scene: someone's backyard. The star: a pine cone, to which someone
has attached googly eyes, Popsicle-stick arms and a string. An unseen
force tugs on the line, and the pine cone (his name is Willy) ascends
spiraling heavenward, to the gushing chorus of Josh Groban's ``You Raise
Me Up.''

The video is 11 seconds of perfect idiocy. I have laughed every time I
watched it; I am laughing as I type this out. It's absurd and low-tech
and parodic but also --- can an inanimate seed cluster be \emph{joyful}?
Well, this one is. The clip is like the climax to an inspirational movie
no one will ever make.

This, I discovered after recently downloading TikTok for the first time,
is the beauty of the platform. Like Vine before it, it's all climaxes.
It's all punch lines and dance outbursts and dramatic (or comedic)
reveals.

As a professional TV watcher in 2019, I'm immersed in maximalist video
--- Netflix binges, ``Game of Thrones.'' Yet what we often take away
from these giant entertainments are the moments: the ``Neverending
Story'' singalong in ``Stranger Things,'' Arya Stark leaping out of the
dark at the Night King.

TikTok gives you nothing but the singalongs and leaps. It is not a
21-course meal. It's a bottomless gumball machine, serving up ephemeral
treats. \emph{Flick ---} hamster eating a tiny pancake! \emph{Flick ---}
guy vacuum-sealing himself into a garbage bag!

\href{https://www.tiktok.com/@kaylacoste/video/6739219005397044485?u_code=d77ehf2j7m0d81\&preview_pb=0\&language=en\&timestamp=1570111059\&utm_campaign=client_share\&app=musically\&utm_medium=ios\&user_id=6713156519594181637\&tt_from=copy\&utm_source=copy\&enter_from=h5_m}{}

@itsyagirlheatha

This is not ``TV'' as recognizable to anyone who grew up when
televisions were furniture. It's what TV becomes when it's something you
pull out of your pocket, standing on a cashier line, sitting on a bus.

So fittingly, it belongs creatively to people who were holding a screen
since they could make a fist. Videos are set in teenage bedrooms,
classrooms, suburban living rooms. A girl dances to audio of her mom
arguing with her boyfriend (\#someonegetmymomagoodman), an unsettling
slice of dark irony. A group of friends creates an ``Avengers assemble''
tableau by jumping backward into a swimming pool, then reversing the
video. A dancer busts an intricate set of moves to a Drake song in front
of a kitchen island, while an older man --- her dad? --- wanders into
the frame in the background, unnoticed.

It's easy for An Old on TikTok --- certainly this middle-aged viewer,
maybe anyone north of 20 --- to feel like the wandering dad in the
hallway, an interloper in an impenetrable hangout of references,
\href{https://www.youtube.com/watch?v=nuYM4jKOick}{Billie Eilish memes}
and comedy bits about homework, driver's ed and \#dormlife.

But the more time I spent with the app, the more I realized that any
feeling of exclusion I had was my own baggage. The pervasive feeling I
got from TikTok was inclusion. It wasn't, like Instagram, trying to
persuade me of its users' happiness, or, like Twitter, of their
rightness. Instead the vibe is: Look at this cool thing I did. What can
you do?

I don't know if I've developed a new habit; my streaming and TiVo
backlogs have too great a hold on me. But the beauty of this new
relationship is how little commitment it demands. TikTok is there,
whenever I want to be raised up, for 11 seconds at a time.

\hypertarget{amanda-hess-1}{%
\subparagraph{Amanda Hess}\label{amanda-hess-1}}

\hypertarget{critic-at-large}{%
\subparagraph{Critic-at-Large}\label{critic-at-large}}

\hypertarget{a-selfish-social-network}{%
\subsubsection{A Selfish Social
Network}\label{a-selfish-social-network}}

TikTok is often referred to as a ``social'' network, but I'm not there
to make friends. I feel no need to follow anyone I know, or to follow
anyone at all, or to attract followers myself. When I am TikTok user
@amandahess88, I am free.

Platforms like Facebook and Instagram have trapped us in the slippage
between personal and public, consumption and production. We use them to
chase pleasure at the same time as we perform reputational maintenance,
and this seeds our delight with a sense of duty. We like being in these
spaces even as we feel that we are not quite allowed to leave. Our
images and observations are weighted with power relationships.
``Liking'' a thing may also fulfill an obligation to a friend, or help
to cultivate a work relationship.

But when I tap the heart icon on TikTok, I do so selfishly. I ``like''
to tell the algorithm that I like something, and to teach it how to
provide me more things that I like. Inverting the impulse of every other
major social app, TikTok defaults not to content from people you have
chosen to follow but to content the algorithm has chosen for you. This
has made it the only platform for which the term ``algorithm'' has, for
me, a positive connotation.

The ``algorithmic timelines'' on Instagram and Twitter breed suspicion
that looser, messier posts are being hidden from view, obscured by
carefully framed announcements of engagements and jobs, pregnancies and
euthanized pets. These platforms operate with a brittle predictability.
But TikTok deals in the illusion, at least, of revelation. Even if all
I'm doing is tapping my screen, ``discovering'' new videos has the feel
of an internet treasure hunt. (That may be deceptive: The Guardian
reports that
ByteDance\href{https://www.theguardian.com/technology/2019/sep/25/revealed-how-tiktok-censors-videos-that-do-not-please-beijing}{}\href{https://www.theguardian.com/technology/2019/sep/25/revealed-how-tiktok-censors-videos-that-do-not-please-beijing}{instructs
its moderators} to censor or downplay material that might offend the
Chinese government.) They have now become a central draw of \emph{other}
social platforms, as friends cross-post the gems from their feeds ---
the guy
who\href{http://vm.tiktok.com/fVUQAG/}{}\href{http://vm.tiktok.com/fVUQAG/}{cooks
in a stream} in the Chinese countryside, surfaced
by\href{https://twitter.com/jaycaspiankang}{}\href{https://twitter.com/jaycaspiankang}{Jay
Caspian Kang}, or
the\href{http://vm.tiktok.com/fVHnMK/}{}\href{http://vm.tiktok.com/fVHnMK/}{young
rural supermodel} found
by\href{https://twitter.com/dodaistewart}{}\href{https://twitter.com/dodaistewart}{Dodai
Stewart}.

\href{https://www.tiktok.com/@whiskeylover311/video/6695561872676048133?u_code=d74k67k9e5617b\&preview_pb=0\&language=en\&timestamp=1570040014\&utm_campaign=client_share\&app=musically\&utm_medium=ios\&user_id=6710526270067426310\&tt_from=more\&utm_source=more\&enter_from=h5_m}{}

@whiskeylover311

Though TikTok videos seem to arrive with the guilelessness of
under-the-covers childhood play, their creators are highly skilled
semiprofessionals who are entertaining me, and building TikTok into a
mammoth platform, for no direct monetary rewards. I realize that my
uncomplicated relationship with TikTok is a gift afforded by age; I am
so far outside of the demographic that aspires to be a TikTok creator
that I could be the demographic's mother. When I tap the heart on some
high school kid's weird video, I feel a flicker of pride, as if I am
supporting him in some way. But all I am really doing is demanding more.

\hypertarget{jon-caramanica-1}{%
\subparagraph{Jon Caramanica}\label{jon-caramanica-1}}

\hypertarget{pop-music-critic}{%
\subparagraph{Pop Music Critic}\label{pop-music-critic}}

\hypertarget{music-and-culture-in-a-warp-speed-blender}{%
\subsubsection{Music and Culture in a Warp-Speed
Blender}\label{music-and-culture-in-a-warp-speed-blender}}

A South Asian teen screaming at his haplessly silent grandmother over a
decontextualized hip-hop snippet. A goth girl doing a dance routine in a
Walmart. A young woman turning the profane bits of one of the raunchiest
rap songs of all time into guttural animal shrieks.

TikTok moves at something faster than warp speed, and crucial to its joy
is its reliance on the hard contrast: Why is this person doing this
thing, and why does the jolt feel so good?

Lost, or at least muddied, in that permanent triggering system is that
TikTok creates cultural blurring on an astounding scale. It is an
appropriation accelerant. Swimming in it for a while makes clear how
easily the app jumbles together inputs and references, with varying
levels of comfort and discomfort: white kids rapping along with hip-hop;
seemingly straight bros playing with queer aesthetics; young people of
all races embracing cowboy styles; and adolescents from many countries
using common soundtracks for skits about timeless teen problems. All you
need to participate is a phone and a desire to be seen.

\href{https://www.tiktok.com/@melon.dudee/video/6740756154743360774?u_code=cm1d04c8g06e8k\&preview_pb=0\&language=en\&timestamp=1570133566\&utm_campaign=client_share\&app=musically\&utm_medium=ios\&user_id=6532049790534942722\&tt_from=copy\&utm_source=copy\&enter_from=h5_m}{}

@melon.dudee

Music is key to TikTok --- essentially every clip has a soundtrack, and
you can press the spinning icon in the bottom right corner to see other
clips that use the same audio. Scrolling through that page is the
Infinite Remix.

Much of the music is hip-hop, though often stripped down to just a
couple of key sentences. With its reduction of people to human GIFs,
TikTok all but eradicates traditional norms about cultural ownership. It
can be difficult to identify source material --- often songs are listed
under completely different names than their actual titles, reminding me
of the wild file-sharing days of early Napster, when you were never
quite sure what you were downloading until you hit play.

After dipping in and out of TikTok for a few months now, I've come to
believe it perceives music as a bug, not a feature. The app's success
depends on its ability to vacuum up audio pulled from anywhere, at any
time, but rarely is it interested in the song as an end. Instead, it is
a means to a simple dance routine, or an easily replicated
micro-narrative.

Occasionally, a snippet that animates the app becomes a song that
escapes into the real world. Lil Nas X was able to reverse engineer
mainstream fame from the spare parts of viral engagement. But even now,
months later, it's unclear whether he's more meaningful as the
soundtrack to a million journeys, or as a destination of his own.

\hypertarget{gia-kourlas-1}{%
\subparagraph{Gia Kourlas}\label{gia-kourlas-1}}

\hypertarget{dance-critic}{%
\subparagraph{Dance Critic}\label{dance-critic}}

\hypertarget{dance-like-a-billion-people-are-watching}{%
\subsubsection{Dance Like a Billion People Are
Watching}\label{dance-like-a-billion-people-are-watching}}

You look at TikTok and think the world is a musical. This celebration
--- of movement, of bodies, of dance --- is addictive, and not just to
watch but to do.

Tutorials, in which steps are broken down, are ubiquitous ---
instructors demonstrate each action of a movement phrase and then
perform the dance full-out. (For some reason, these instructional videos
are less tedious than those found on YouTube.) Now the performances
happening on my train as it crosses the Williamsburg Bridge have
branched out, from seasoned street or subway dancers to amateurs
practicing footwork. These buoyant steps, repeatable and ranging in
complexity, should really be part of a metabolic workout, but you can do
them anywhere. I practice them when I'm stuck in traffic during a run.

What makes TikTok so magnetic is not just the frequency of dance, but
the unspoken truth of this performative genre. When it comes to dancing,
there are no rules: Everyone can move. Here, the amateur is just as
revered as the professional, a belief that has its roots with Judson
Dance Theater, the 1960s collective that ushered in postmodern dance.
Its founding members were highly trained, but its performers could come
from anywhere.

\href{https://www.tiktok.com/@live.collegiate.shag/video/6740735611616939270?u_code=cm1d04c8g06e8k\&preview_pb=0\&language=en\&timestamp=1570113651\&utm_campaign=client_share\&app=musically\&utm_medium=ios\&user_id=6532049790534942722\&tt_from=copy\&utm_source=copy\&enter_from=h5_m}{}

@live.collegiate.shag

Most of us in the dance world know that to be true --- that dancers come
in many forms --- but TikTok has opened up that philosophy to the
greater world. And it has the whole world moving: a basketball coach on
the court, a woman in front of a lake with a hoop as a partner, a couple
performing the collegiate shag in the rain in New Orleans.

While Instagram and YouTube love dance, TikTok has figured out how to
put on a show, one challenge at a time. There's the
\#DipandLeanChallenge, a rapid-fire showcase of different street dances,
like the Milly Rock, the Floss, the Dougie and the Roy Purdy, or
\#SwanDance, which features people extending their legs at different
levels to form a wing --- sometimes seriously, sometimes not. It's a
rabbit hole of interpretation and technique.

A dance --- even a short one --- erodes over time, as Twyla Tharp proved
in her 1970 classic ``The One Hundreds,'' which starts with two trained
dancers performing 100 11-second movement sequences; next, five dancers
perform 20 of them; and in the end, 100 ordinary people perform one of
the 11-second sequences. It's three different views of the same dance
performed under different circumstances, and it's usually a mess by the
end. TikTok is that, too, for dance: unruly, witty and beautiful, with
the continual possibility of invention and rejuvenation. There's always
a new challenge waiting to be born and, most important, willing bodies
to see it through.

\hypertarget{wesley-morris}{%
\subparagraph{Wesley Morris}\label{wesley-morris}}

\hypertarget{critic-at-large-1}{%
\subparagraph{Critic-at-Large}\label{critic-at-large-1}}

\hypertarget{when-15-seconds-is-not-enough}{%
\subsubsection{When 15 Seconds Is Not
Enough}\label{when-15-seconds-is-not-enough}}

Remember the shopping mall? It's dying. But now your phone's a mall.
Your phone is actually a mall full of other malls. One of those malls is
TikTok. And TikTok is an infinite mall. Excusing the ads, there's
nothing to buy. But there are people to watch, infinite amounts of
people. Lately, the first TikTok people I see are doing the ``I Used to
Be So Beautiful Now Look at Me'' dance. It's set to a snippet of a
catchy club-pop track, ``Absolutely Anything.'' But everybody calls it
the TikTok song.

In the opening halves of their clips, users mouth the lyrics, usually in
sweats. (This is the ``used to be so beautiful'' part.) Then, in time
with a stutter in the beat, they cross both arms above their heads or in
front of their middles, and whip them up and down two times, like salad
days Britney Spears. And suddenly they're hot zombies or flirty
princesses, walking in place but sexily. (The ``look at me'' part.)
There are scores and scores of ``Used to Be'' videos. And falling into a
pit of them is better than people-watching at the mall yet entirely in
the mall's bargain spirit, since you get two people in one.

Not everything is so literal-clever at the TikTok mall. The users you're
watching can be obnoxious. A guy heaves a watermelon into the ceiling of
a Target and runs when it falls explosively on the floor. (A million and
a half people liked that one.) They can be banal. Few of the
``nope/yup'' TikToks --- in which people lip-sync over the misty opening
chimes of E-40's ``Choices'' (``nope \ldots{} yup'') while questions
(``Am I straight?'') hover over them --- did it for me. The funny thing
about the app is how quickly it can turn you from onlooker to rummager
--- swiping for magic until you find some, until you settle for
something --- to judge.

After a few hours at the mall, I realized I was fishing hard for fun I
didn't need. I'm not a 10th grader unhappy in social studies. I don't
have time to be this bored. What I really needed was on Twitter. What I
needed was @SupremeDreams\_1, the handle of a person named Mark
Phillips. There's nothing I've seen on TikTok like the crisis-satire
video he posted last week: ``How They Expect You to React When You Get
an Amber Alert.'' In the clip, when an alert comes through, a dude
springs into action and joins an all-black-bro rescue caravan. This
thing moves with so much mocking emergency and shock craftsmanship that
brilliance rises out of the absurdity. It's like a ``Key \& Peele''
sketch under the director Tony Scott's command, 48 seconds long (that's
all!) and way more exciting than what currently seems possible with
TikTok's twee, one-dimensional starter kit.

\hypertarget{credits}{%
\paragraph{Credits}\label{credits}}

\hypertarget{background-tiktoks-in-order-of-appearance-lxurentaylor-billyandwilly7-ourfire-xowiejones-tmdad14-lesliecontreras_-eggrollblunt}{%
\subsubsection{\texorpdfstring{Background TikToks in order of
appearance:
\href{https://www.tiktok.com/@lxurentaylor/video/6740257829669702918?u_code=cm1d04c8g06e8k\&preview_pb=0\&language=en\&timestamp=1570113783\&utm_campaign=client_share\&app=musically\&utm_medium=ios\&user_id=6532049790534942722\&tt_from=copy\&utm_source=copy\&enter_from=h5_m}{@lxurentaylor},
\href{https://www.tiktok.com/@billyandwilly7/video/6734337294737689862?u_code=d77ehf2j7m0d81\&preview_pb=0\&language=en\&timestamp=1570110967\&utm_campaign=client_share\&app=musically\&utm_medium=ios\&user_id=6713156519594181637\&tt_from=copy\&utm_source=copy\&enter_from=h5_m}{@billyandwilly7},
\href{https://www.tiktok.com/@ourfire/video/6709507632148778246?u_code=dcg62\&preview_pb=0\&language=en\&timestamp=1570475625\&utm_campaign=client_share\&app=musically\&utm_medium=ios\&user_id=3792541\&tt_from=copy\&utm_source=copy\&enter_from=h5_m}{@ourfire},
\href{https://www.tiktok.com/@xowiejones/video/6733306306926497030?u_code=cm1d04c8g06e8k\&preview_pb=0\&language=en\&timestamp=1570133527\&utm_campaign=client_share\&app=musically\&utm_medium=ios\&user_id=6532049790534942722\&tt_from=copy\&utm_source=copy\&enter_from=h5_m}{@xowiejones},
\href{https://www.tiktok.com/@tmdad14/video/6741781657105304838?u_code=cm1d04c8g06e8k\&preview_pb=0\&language=en\&timestamp=1570133749\&utm_campaign=client_share\&app=musically\&utm_medium=ios\&user_id=6532049790534942722\&tt_from=copy\&utm_source=copy\&enter_from=h5_m}{@tmdad14},
\href{https://www.tiktok.com/@lesliecontreras_/video/6741114172869807366?u_code=d5hc8h6d3f7dm6\&preview_pb=0\&language=en\&timestamp=1570460952\&utm_campaign=client_share\&app=musically\&utm_medium=ios\&user_id=6678756474238338053\&tt_from=copy\&utm_source=copy\&enter_from=h5_m}{@lesliecontreras\_},
\href{https://www.tiktok.com/@eggrollblunt/video/6719592982695267589?u_code=d77ehf2j7m0d81\&preview_pb=0\&language=en\&timestamp=1570111008\&utm_campaign=client_share\&app=musically\&utm_medium=ios\&user_id=6713156519594181637\&tt_from=copy\&utm_source=copy\&enter_from=h5_m}{@eggrollblunt}}{Background TikToks in order of appearance: @lxurentaylor, @billyandwilly7, @ourfire, @xowiejones, @tmdad14, @lesliecontreras\_, @eggrollblunt}}\label{background-tiktoks-in-order-of-appearance-lxurentaylor-billyandwilly7-ourfire-xowiejones-tmdad14-lesliecontreras_-eggrollblunt}}

\hypertarget{produced-by-shannon-lin-and-alicia-desantis}{%
\subsubsection{Produced by Shannon Lin and Alicia
DeSantis.}\label{produced-by-shannon-lin-and-alicia-desantis}}

Read 18 Comments

\begin{itemize}
\item
\item
\item
\item
\end{itemize}

Advertisement

\protect\hyperlink{after-bottom}{Continue reading the main story}

\hypertarget{site-index}{%
\subsection{Site Index}\label{site-index}}

\hypertarget{site-information-navigation}{%
\subsection{Site Information
Navigation}\label{site-information-navigation}}

\begin{itemize}
\tightlist
\item
  \href{https://help.nytimes.com/hc/en-us/articles/115014792127-Copyright-notice}{©~2020~The
  New York Times Company}
\end{itemize}

\begin{itemize}
\tightlist
\item
  \href{https://www.nytco.com/}{NYTCo}
\item
  \href{https://help.nytimes.com/hc/en-us/articles/115015385887-Contact-Us}{Contact
  Us}
\item
  \href{https://www.nytco.com/careers/}{Work with us}
\item
  \href{https://nytmediakit.com/}{Advertise}
\item
  \href{http://www.tbrandstudio.com/}{T Brand Studio}
\item
  \href{https://www.nytimes.com/privacy/cookie-policy\#how-do-i-manage-trackers}{Your
  Ad Choices}
\item
  \href{https://www.nytimes.com/privacy}{Privacy}
\item
  \href{https://help.nytimes.com/hc/en-us/articles/115014893428-Terms-of-service}{Terms
  of Service}
\item
  \href{https://help.nytimes.com/hc/en-us/articles/115014893968-Terms-of-sale}{Terms
  of Sale}
\item
  \href{https://spiderbites.nytimes.com}{Site Map}
\item
  \href{https://help.nytimes.com/hc/en-us}{Help}
\item
  \href{https://www.nytimes.com/subscription?campaignId=37WXW}{Subscriptions}
\end{itemize}
