Sections

SEARCH

\protect\hyperlink{site-content}{Skip to
content}\protect\hyperlink{site-index}{Skip to site index}

\href{https://www.nytimes.com/section/politics}{Politics}

\href{https://myaccount.nytimes.com/auth/login?response_type=cookie\&client_id=vi}{}

\href{https://www.nytimes.com/section/todayspaper}{Today's Paper}

\href{/section/politics}{Politics}\textbar{}How Would Elizabeth Warren
Pay for Her Sweeping Policy Plans?

\url{https://nyti.ms/2CnfWWA}

\begin{itemize}
\item
\item
\item
\item
\item
\item
\end{itemize}

\begin{itemize}
\item
  \href{https://www.nytimes.com/2020/07/31/us/elections/biden-vs-trump.html?action=click\&pgtype=Article\&state=default\&region=TOP_BANNER\&context=storylines_menu}{Election
  Updates}
\item
  \href{https://www.nytimes.com/article/biden-vice-president-2020.html?action=click\&pgtype=Article\&state=default\&region=TOP_BANNER\&context=storylines_menu}{Biden's
  V.P. Search}
\item
  \href{https://www.nytimes.com/interactive/2020/07/24/us/politics/trump-biden-campaign-donors.html?action=click\&pgtype=Article\&state=default\&region=TOP_BANNER\&context=storylines_menu}{Map
  of Donations}
\item
  \href{https://www.nytimes.com/interactive/2020/us/elections/delegate-count-primary-results.html?action=click\&pgtype=Article\&state=default\&region=TOP_BANNER\&context=storylines_menu}{Delegate
  Count}
\item
  \href{https://www.nytimes.com/interactive/2019/us/politics/2020-presidential-candidates.html?action=click\&pgtype=Article\&state=default\&region=TOP_BANNER\&context=storylines_menu}{The
  Candidates}
\item
  \href{https://www.nytimes.com/newsletters/politics?action=click\&pgtype=Article\&state=default\&region=TOP_BANNER\&context=storylines_menu}{Politics
  Newsletter}
\end{itemize}

Advertisement

\protect\hyperlink{after-top}{Continue reading the main story}

\hypertarget{comments}{%
\subsection{\texorpdfstring{\protect\hyperlink{commentsContainer}{Comments}}{Comments}}\label{comments}}

\href{}{How Would Elizabeth Warren Pay for Her Sweeping Policy
Plans?}\href{}{Skip to Comments}

The comments section is closed. To submit a letter to the editor for
publication, write to
\href{mailto:letters@nytimes.com}{\nolinkurl{letters@nytimes.com}}.

\hypertarget{how-would-elizabeth-warren-pay-for-her-sweeping-policy-plans}{%
\section{How Would Elizabeth Warren Pay for Her Sweeping Policy
Plans?}\label{how-would-elizabeth-warren-pay-for-her-sweeping-policy-plans}}

By \href{https://www.nytimes.com/by/thomas-kaplan}{Thomas Kaplan},
\href{https://www.nytimes.com/by/aliza-aufrichtig}{Aliza Aufrichtig} and
\href{https://www.nytimes.com/by/derek-watkins}{Derek Watkins}Nov. 7,
2019

\begin{itemize}
\item
\item
\item
\item
\item
  \emph{831}
\end{itemize}

Senator Elizabeth Warren's policy plans would cost more than \$30
trillion over a decade, an agenda of monumental scale that would
significantly increase federal government spending. She is counting on
far-reaching changes to the tax system that would collect more revenue
from the richest Americans and from businesses.

Each

represents in planned spending.

Dots show a total of\\
 in spending.

Ms. Warren has proposed \textbf{universal child care, increased spending
on public schools, student debt cancellation and free college.} She
plans to pay for these proposals by creating a wealth tax on households
worth over \$50 million. How much revenue that tax would generate is a
matter of debate.

Her climate change agenda includes plans on \textbf{clean energy
technology and reducing carbon emissions.} She would cover much of the
cost by creating a new tax on corporate profits and reversing tax cuts
for wealthy people and big corporations in the 2017 Republican tax
overhaul.

Ms. Warren would increase \textbf{Social Security benefits} by \$200 a
month, among other changes. To pay for her plan, she would raise
investment and payroll taxes on high earners.

She wants to \textbf{build more affordable housing.} She would cover
most of the cost by expanding the estate tax.

She has called for new spending in a \textbf{variety of other areas,}
such as rural broadband and apprenticeship programs. Her plans to
address the opioid crisis and election security and help minority-owned
small businesses would be largely financed by eliminating a tax benefit
for inherited assets.

Last week, Ms. Warren laid out her most expensive plan yet: creating
\textbf{``Medicare for all,''} a government-run health insurance
program. Her price tag relies on aggressive assumptions and is lower
than several other estimates from experts.

Even still, that's twice the estimated cost of her other proposals.
Let's take a closer look at how she would pay for it.

First, Ms. Warren would impose a \textbf{new tax on employers} that is
similar to what they currently spend on their employees' health care.

Some money would come from \textbf{added revenue from existing taxes.}
Workers would no longer pay for health insurance premiums, so their
take-home pay would rise, and they would pay taxes on that extra money.
Ms. Warren also wants to strengthen tax enforcement to collect more
taxes.

A significant amount would come from \textbf{taxing the wealthy.} For
the top 1 percent of households, she would tax capital gains annually
instead of when investments are sold, and raise the tax rate on those
gains. She would also steepen her wealth tax on billionaires.

Ms. Warren would \textbf{make changes to corporate taxation,} such as
raising taxes on companies that earn money in foreign countries. She
would create a tax on financial transactions like stock trades and
impose a new fee on big banks.

She is counting on passing an \textbf{overhaul of immigration laws,}
which would be a momentous political achievement if it were to happen.
That would generate revenue from additional taxes paid by immigrants.

Finally, she would \textbf{cut military spending.}

The \textbf{total cost of Ms. Warren's proposals} over 10 years is
greater than what the federal government is currently projected to spend
on Social Security and Medicare combined over that time.

On paper, the revenue sources she has laid out would cover the
approximate cost of her plans. But her math relies on assumptions that
are far from assured, such as the extent to which the richest Americans
would pay her wealth tax rather than dodge it.

The federal government is projected to spend about \$58 trillion over
the next decade. Ms. Warren would need to persuade Congress to approve
her plans, a difficult feat for any one of her expansive proposals, let
alone her entire agenda. If she succeeded, she would increase federal
spending by \textbf{roughly 50 percent}.

\textbf{Notes}

This analysis includes proposals released by Ms. Warren's presidential
campaign that have clear price tags. Because the overall spending total
reflects only those proposals for which there are clear cost estimates,
it should be viewed as a partial estimate of how much her full agenda,
including ideas that do not have price tags, would cost. Projections for
what her plans would cost over a decade were used when those estimates
were available. The figures for Ms. Warren's most expensive plan,
Medicare for all, are for 2020 to 2029.
\href{https://www.cbo.gov/system/files/2019-08/55551-CBO-outlook-update_0.pdf\#page=11}{Projected
federal spending} under current law is also for 2020 to 2029. Ms.
Warren's plan for Medicare for all counts on \$6.1 trillion from states
and local governments in addition to \$20.5 trillion in new federal
spending.

Sources: Warren campaign; Moody's Analytics; Congressional Budget
Office. The
\href{https://int.nyt.com/data/documenthelper/6423-cost-estimate-for-warren-medicare-for-all/aea6a2d073e99921e256/optimized/full.pdf\#page=1}{cost
estimate} for Ms. Warren's
\href{https://medium.com/@teamwarren/ending-the-stranglehold-of-health-care-costs-on-american-families-bf8286b13086}{Medicare
for all plan} is from Dr. Donald M. Berwick, a former administrator of
the federal Centers for Medicare and Medicaid Services, and Simon
Johnson, a professor at the M.I.T. Sloan School of Management. The
\href{https://int.nyt.com/data/documenthelper/6424-revenue-estimate-for-warren-medicare-for-all/aea6a2d073e99921e256/optimized/full.pdf\#page=1}{estimate
of revenue sources} for Medicare for all is from Mr. Johnson; Betsey
Stevenson, a professor at the University of Michigan; and Mark Zandi,
the chief economist of Moody's Analytics.

Additional reporting by Margot Sanger-Katz.

Read 831 Comments

\begin{itemize}
\item
\item
\item
\item
\end{itemize}

\hypertarget{our-2020-election-guide}{%
\section{Our 2020 Election Guide}\label{our-2020-election-guide}}

Updated July 31, 2020

\begin{itemize}
\item
  \begin{center}\rule{0.5\linewidth}{\linethickness}\end{center}

  \hypertarget{the-latest}{%
  \subsection{The Latest}\label{the-latest}}

  \begin{itemize}
  \tightlist
  \item
    President Trump's assault on the Postal Service is intersecting with
    his attacks on mail-in voting.
    \href{https://www.nytimes.com/2020/07/31/us/politics/trump-usps-mail-delays.html?action=click\&pgtype=Article\&state=default\&region=BELOW_MAIN_CONTENT\&context=storylines_guide}{Voting
    rights groups say it is a recipe for disaster.}
  \end{itemize}
\item
  \begin{center}\rule{0.5\linewidth}{\linethickness}\end{center}

  \hypertarget{bidens-vp-search}{%
  \subsection{Biden's V.P. Search}\label{bidens-vp-search}}

  \begin{itemize}
  \tightlist
  \item
    \href{https://www.nytimes.com/article/biden-vice-president-2020.html?action=click\&pgtype=Article\&state=default\&region=BELOW_MAIN_CONTENT\&context=storylines_guide}{Here
    are 13 women} who have been under consideration to be Joe Biden's
    running mate, and why each might be chosen --- and might not be.
  \end{itemize}
\item
  \begin{center}\rule{0.5\linewidth}{\linethickness}\end{center}

  \hypertarget{keep-up-with-our-coverage}{%
  \subsection{Keep Up With Our
  Coverage}\label{keep-up-with-our-coverage}}

  \begin{itemize}
  \tightlist
  \item
    Get an
    \href{https://www.nytimes.com/newsletters/politics?action=click\&pgtype=Article\&state=default\&region=BELOW_MAIN_CONTENT\&context=storylines_guide}{email}
    recapping the day's news
  \end{itemize}

  \begin{itemize}
  \tightlist
  \item
    Download our mobile app on
    \href{https://apps.apple.com/us/app/nytimes/id284862083?ls=1\&mat_click_id=5c79ae7455014fd1bd66b5610c05b8f2-20191112-16948\&referrer=mat_click_id\%3D5c79ae7455014fd1bd66b5610c05b8f2-20191112-16948\%26link_click_id\%3D722930677036718082}{iOS}
    and
    \href{http://a.localytics.com/android?id=com.nytimes.android\&referrer=utm_source\%3Dother_nyt_mobile_web\%26utm_medium\%3DWeb\%2520page\%26utm_term\%3DGeneral\%2520Mobile\%2520Page\%26utm_campaign\%3DNYT\%2520Mobile\%2520General\%2520Page}{Android}
    and turn on Breaking News and Politics alerts
  \end{itemize}
\end{itemize}

Advertisement

\protect\hyperlink{after-bottom}{Continue reading the main story}

\hypertarget{site-index}{%
\subsection{Site Index}\label{site-index}}

\hypertarget{site-information-navigation}{%
\subsection{Site Information
Navigation}\label{site-information-navigation}}

\begin{itemize}
\tightlist
\item
  \href{https://help.nytimes.com/hc/en-us/articles/115014792127-Copyright-notice}{©~2020~The
  New York Times Company}
\end{itemize}

\begin{itemize}
\tightlist
\item
  \href{https://www.nytco.com/}{NYTCo}
\item
  \href{https://help.nytimes.com/hc/en-us/articles/115015385887-Contact-Us}{Contact
  Us}
\item
  \href{https://www.nytco.com/careers/}{Work with us}
\item
  \href{https://nytmediakit.com/}{Advertise}
\item
  \href{http://www.tbrandstudio.com/}{T Brand Studio}
\item
  \href{https://www.nytimes.com/privacy/cookie-policy\#how-do-i-manage-trackers}{Your
  Ad Choices}
\item
  \href{https://www.nytimes.com/privacy}{Privacy}
\item
  \href{https://help.nytimes.com/hc/en-us/articles/115014893428-Terms-of-service}{Terms
  of Service}
\item
  \href{https://help.nytimes.com/hc/en-us/articles/115014893968-Terms-of-sale}{Terms
  of Sale}
\item
  \href{https://spiderbites.nytimes.com}{Site Map}
\item
  \href{https://help.nytimes.com/hc/en-us}{Help}
\item
  \href{https://www.nytimes.com/subscription?campaignId=37WXW}{Subscriptions}
\end{itemize}
