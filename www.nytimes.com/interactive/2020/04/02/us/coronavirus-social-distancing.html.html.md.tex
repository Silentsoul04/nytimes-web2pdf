Sections

SEARCH

\protect\hyperlink{site-content}{Skip to
content}\protect\hyperlink{site-index}{Skip to site index}

\href{https://www.nytimes.com/section/us}{U.S.}

\href{https://myaccount.nytimes.com/auth/login?response_type=cookie\&client_id=vi}{}

\href{https://www.nytimes.com/section/todayspaper}{Today's Paper}

\href{/section/us}{U.S.}\textbar{}Where America Didn't Stay Home Even as
the Virus Spread

\url{https://nyti.ms/3aAql0E}

\begin{itemize}
\item
\item
\item
\item
\item
\item
\end{itemize}

\href{https://www.nytimes.com/news-event/coronavirus?action=click\&pgtype=Article\&state=default\&region=TOP_BANNER\&context=storylines_menu}{The
Coronavirus Outbreak}

\begin{itemize}
\tightlist
\item
  live\href{https://www.nytimes.com/2020/08/02/world/coronavirus-updates.html?action=click\&pgtype=Article\&state=default\&region=TOP_BANNER\&context=storylines_menu}{Latest
  Updates}
\item
  \href{https://www.nytimes.com/interactive/2020/us/coronavirus-us-cases.html?action=click\&pgtype=Article\&state=default\&region=TOP_BANNER\&context=storylines_menu}{Maps
  and Cases}
\item
  \href{https://www.nytimes.com/interactive/2020/science/coronavirus-vaccine-tracker.html?action=click\&pgtype=Article\&state=default\&region=TOP_BANNER\&context=storylines_menu}{Vaccine
  Tracker}
\item
  \href{https://www.nytimes.com/interactive/2020/07/29/us/schools-reopening-coronavirus.html?action=click\&pgtype=Article\&state=default\&region=TOP_BANNER\&context=storylines_menu}{What
  School May Look Like}
\item
  \href{https://www.nytimes.com/live/2020/07/31/business/stock-market-today-coronavirus?action=click\&pgtype=Article\&state=default\&region=TOP_BANNER\&context=storylines_menu}{Economy}
\end{itemize}

Advertisement

\protect\hyperlink{after-top}{Continue reading the main story}

\hypertarget{comments}{%
\subsection{\texorpdfstring{\protect\hyperlink{commentsContainer}{Comments}}{Comments}}\label{comments}}

\href{}{Where America Didn't Stay Home Even as the Virus
Spread}\href{}{Skip to Comments}

The comments section is closed. To submit a letter to the editor for
publication, write to
\href{mailto:letters@nytimes.com}{\nolinkurl{letters@nytimes.com}}.

\hypertarget{where-america-didnt-stay-home-even-as-the-virus-spread}{%
\section{Where America Didn't Stay Home Even as the Virus
Spread}\label{where-america-didnt-stay-home-even-as-the-virus-spread}}

By \href{https://www.nytimes.com/by/james-glanz}{James Glanz},
\href{https://www.nytimes.com/by/benedict-carey}{Benedict Carey},
\href{https://www.nytimes.com/by/josh-holder}{Josh Holder},
\href{https://www.nytimes.com/by/derek-watkins}{Derek Watkins},
\href{https://www.nytimes.com/by/jennifer-valentino-devries}{Jennifer
Valentino-DeVries}, \href{https://www.nytimes.com/by/rick-rojas}{Rick
Rojas} and \href{https://www.nytimes.com/by/lauren-leatherby}{Lauren
Leatherby}April 2, 2020

\begin{itemize}
\item
\item
\item
\item
\item
  \emph{3}
\end{itemize}

Outlined areas had known stay-at-home orders before March 27

Outlined areas had known stay-at-home orders before March 27

Known stay-at-home orders before March 27

\hypertarget{where-people-were-still-traveling}{%
\subsubsection{Where people were still
traveling}\label{where-people-were-still-traveling}}

\hypertarget{percent-change-in-average-travel-for-the-week-of-march-23-compared-with-travel-before-the-coronavirus-outbreak}{%
\paragraph{Percent change in average travel for the week of March 23,
compared with travel before the coronavirus
outbreak.}\label{percent-change-in-average-travel-for-the-week-of-march-23-compared-with-travel-before-the-coronavirus-outbreak}}

Closer to normal travel

No travel

Half of normal

Normal travel

Closer to normal travel

No travel

Half of

normal

Normal travel

Stay-at-home orders have nearly halted travel for most Americans, but
people in Florida, the Southeast and other places that waited to enact
such orders have continued to travel widely, potentially exposing more
people as the coronavirus outbreak accelerates, according to an analysis
of cellphone location data by The New York Times.

The divide in travel patterns, based on anonymous cellphone data from 15
million people, suggests that Americans in wide swaths of the West,
Northeast and Midwest have complied with orders from state and local
officials to stay home. Disease experts who reviewed the results say
those reductions in travel --- to less than a mile a day, on average,
from about five miles --- may be enough to sharply curb the spread of
the coronavirus in those regions, at least for now.

``That's huge,'' said Aaron A. King, a University of Michigan professor
who studies the ecology of infectious disease. ``By any measure this is
a massive change in behavior, and if we can make a similar reduction in
the number of contacts we make, every indication is that we can defeat
this epidemic.''

But not everybody has been staying home.

\hypertarget{which-counties-reduced-travel-the-most}{%
\subsubsection{Which counties reduced travel the
most}\label{which-counties-reduced-travel-the-most}}

In areas where public officials have resisted or delayed stay-at-home
orders, people changed their habits far less. Though travel distances in
those places have fallen drastically, during the week of March 23 they
were still typically more than three times those in areas that had
imposed lockdown orders, the analysis shows.

A half-dozen of the most populous counties where residents were
traveling widely were in Florida, where Gov. Ron DeSantis resisted
calling for a statewide lockdown until April 1. People in Jacksonville,
Tampa, Daytona Beach, Lakeland and surrounding areas continued to travel
much more than people in other parts of the country, putting those areas
at a higher risk for outbreaks in the coming weeks.

\hypertarget{where-people-were-still-traveling-the-most-on-march-27}{%
\subsubsection{Where people were still traveling the most on March
27}\label{where-people-were-still-traveling-the-most-on-march-27}}

\hypertarget{counties-above-500000-people}{%
\paragraph{Counties above 500,000
people}\label{counties-above-500000-people}}

Less travel

More

County

Avg. travel

Feb. 28

Travel by day

Mar. 27

F

S

S

M

T

W

T

F

S

S

M

T

W

T

F

S

S

M

T

W

T

F

S

S

M

T

W

T

F

In Jacksonville, the sheriff's department had to send out officers over
the weekend to break up block parties. In Spartanburg, S.C., people were
still going to the hardware store to buy supplies for home-improvement
projects, and pictures from children's birthday parties and playdates
were being posted on Facebook. And along the shorelines in Florida and
Alabama, communities that rely on tourists to help drive the economy
instead looked with frustration at out-of-state license plates on the
street.

``I saw people this weekend shaking hands with each other,'' Lenny
Curry, the mayor of Jacksonville, Florida's largest city, told
residents. ``I understand, maybe it's just habit. But we've got to stop,
folks. We've really got to stop this.''

The location data, from
\href{https://www.cuebiq.com/visitation-insights-covid19/?utm_source=nyt\&utm_medium=article\&utm_campaign=organic}{Cuebiq},
a data intelligence firm, measures the range that people travel each
day. It cannot predict where outbreaks will spread, and it does not
track how many interactions people had while they were traveling. Not
all travel is problematic: A person driving for a few miles to pick up
groceries would not be violating stay-at-home orders. And people in
cities can infect others without traveling far.

But broadly higher levels of travel suggest more contact with others and
more chances to spread or contract the disease, researchers said.
Counties with lax travel policies risk not only becoming the next hot
spots of the disease, but also acting as reservoirs for the virus that
reignite infection in places that have tamped it down, they said.

Florida waited so long to shut down travel that it will struggle to
control local outbreaks even if people immediately change their behavior
significantly, said Thomas Hladish, an infectious disease modeler at the
University of Florida. People who now sequester themselves at home still
risk having brought the virus home to their families, he said.

"A lockdown order right now is not going to be a silver bullet with
containing this," Mr. Hladish said. "It will absolutely save lives. But
in order to really have a big effect with social-distancing measures,
you would have had to move it back in time.''

In Spartanburg, a city of 37,000 people in the northwest part of South
Carolina, the City Council delayed putting into place any official
order. "I'd like to see us wait at least another couple of weeks for
this discussion," one councilwoman, Ruth Littlejohn, said.

Erica Brown, another councilwoman in Spartanburg, has pushed for
enacting limits, saying they were necessary because issuing simple
cautions had not worked. ``They're going about their days as if nothing
has changed,'' she said of residents.

Ms. Brown acknowledged that the city --- which has undergone an economic
rebirth in recent years --- and particularly locally owned businesses,
would surely suffer by keeping people home. But she said she feared that
waiting any longer may mean devastation down the line.

``Now everyone is just trying to crawl their way back to the starting
line,'' she said, ``and we're already behind in our response.''

On March 31, Gov. Henry McMaster of South Carolina issued an order
closing businesses deemed nonessential, including entertainment venues,
bingo halls, gyms and salons.

Other areas reduced travel weeks ago, the data show, especially in
California, New York and Washington, which were the first to experience
large outbreaks. Most people have essentially stopped traveling in those
places for weeks, the data show, a sign that they are taking the
measures seriously. Anthony S. Fauci, the nation's leading infectious
disease expert, said on March 31 that changes in behavior were already
bringing down the level of new cases in those places.

\hypertarget{when-average-distance-traveled-first-fell-below-2-miles}{%
\subsubsection{When average distance traveled first fell below 2
miles}\label{when-average-distance-traveled-first-fell-below-2-miles}}

By Mar. 16

Mar. 19

Mar. 24

Mar. 26

~

Not by Mar. 26

\hypertarget{data-is-through-march-26-only-weekdays-were-counted-because-almost-everyone-traveled-less-on-weekends}{%
\paragraph{Data is through March 26. Only weekdays were counted, because
almost everyone traveled less on
weekends.}\label{data-is-through-march-26-only-weekdays-were-counted-because-almost-everyone-traveled-less-on-weekends}}

Disease experts said that a tenfold reduction in travel in many counties
in the United States gave them hope that other places could have similar
success, as long as the reductions remain in effect for an extended
period and are implemented nationwide.

But even states and cities that were relatively early in setting
restrictions did not significantly reduce travel until mid- to late
March, leaving a path for the virus to make its way around the country.
Control measures typically take two to three weeks to show an effect
because of the time it takes for infected people to develop symptoms,
seek medical care and get tested.

In New Orleans, where more than 2,000 people have been infected, travel
was not reduced significantly until March 20, weeks after Mardi Gras
brought huge crowds together. In Albany, Ga., a town of 75,000, travel
reductions in recent weeks were too late to stop
\href{https://www.nytimes.com/2020/03/30/us/coronavirus-funeral-albany-georgia.html}{a
cluster of illnesses} that apparently emerged after a funeral in late
February. Travel started declining only about two weeks ago in
Indianapolis, Chicago, Miami and Phoenix, leaving room for continued
growth last month.

How travel changed between Feb. 28 ...

... and March 27

Travel fell dramatically in Seattle, from 3.8 miles...

... to an average of 61 feet.

University of

Washington

SEATTLE

Mar. 27

61 feet

Downtown

3.8 miles

But in Daytona Beach, Fla. travel only fell from 4.4 miles...

... to an average of 1.9 miles.

DAYTONA BEACH

Mar. 27

Daytona

Speedway

Downtown

1.9 miles

4.4 miles

How travel changed between Feb. 28 ...

... and March 27

Travel fell dramatically in Seattle, from 3.8 miles...

... to an average of 61 feet.

University of

Washington

SEATTLE

Mar. 27

61 feet

Downtown

3.8 miles

But in Daytona Beach, Fla. travel fell from 4.4 miles...

... to 1.9 miles.

DAYTONA BEACH

Daytona

Speedway

Downtown

Mar. 27

1.9 miles

4.4 miles

Travel fell dramatically

in Seattle, from 3.8 miles...

3.8 miles

Feb. 28

... to an average of 61 feet.

Mar. 27

SEATTLE

But in Daytona Beach, Fla.

travel fell from 4.4 miles...

4.4 miles

Feb. 28

... to 1.9 miles.

Mar. 27

DAYTONA

BEACH

Source: Aerial imagery from Google

The coronavirus outbreak is unprecedented in scale in recent history,
and it is hard to know the exact relationship between changes in travel
patterns and how quickly the virus spreads. Other factors play a big
role, including how quickly sick people are tested and isolated, how
closely people tend to congregate --- and luck.

Sheltering in place is protective and clearly reduces people's contact
with others, but the existing evidence that the policy can effectively
contain an epidemic within a large population is uncertain, experts
said.

``It's unprecedented, an extraordinarily disruptive natural experiment
like this,'' Mr. Hladish said. ``We have to look back to the influenza
pandemic of 1918 for anything similar, and then of course the field of
epidemiology was less well-developed, and we had nothing like the
technology we do now.''

Working on the fly, scientists are assessing the results of different
containment policies and their timing day by day --- and using those
findings to advise regions that are not yet overwhelmed.

On March 26, Gov. Kay Ivey of Alabama resisted issuing a stay-at-home
order as the outbreak appeared more acute and immediate in other states.
``We are not California,'' she said. ``We're not New York. We aren't
even Louisiana.'' But as the number of cases rose, she encouraged
residents to limit social gatherings and ordered the closure of
nonessential businesses. ``These are uncertain times, for sure,'' she
said in a televised speech ``So now, and for the foreseeable future,
please, please consider staying safe at home.''

The inconsistencies in policies --- and in when they are imposed --- may
create new problems, even for places that set limits weeks ago.

``Let's assume that we flatten the curve, that we push transmission down
in the Bay Area and we walk away with 1 percent immunity,'' said Dr.
George Rutherford, a professor of epidemiology at the University of
California, San Francisco. ``Then, people visit from regions that have
not sheltered in place, and we have another run of cases. This is going
to happen.''

Even in Florida, where limits were announced on April 1, weeks after
other places enacted them, some residents said their biggest worry was
not locals, but visitors from other places.

Before the limits came into effect, Quanita J. May, a city commissioner
in Daytona Beach, said her area was exceptionally quiet. Along the
beaches and downtown, she said, ``It feels like it's Christmas or
Thanksgiving, the three or four days when everybody is at home with
their families.''

``The residents are self-quarantining,'' she said, adding that her real
concern was people coming from other places. They were the ones, she
said, who were not staying put. ``People are listening to the newscasts,
they're reading, and they're paying attention," she said. ``This is here
now.''

Note: Cuebiq calculated distance traveled by measuring a line between
opposite corners of a box drawn around the locations observed for each
person on each day. The travel for each county is the median of these
per-person distances. Many states and counties have taken control
measures, such as closing restaurants or beaches, that were not included
in this analysis.

Sarah Mervosh and Vanessa Swales contributed reporting.

Read 3 Comments

\begin{itemize}
\item
\item
\item
\item
\end{itemize}

Advertisement

\protect\hyperlink{after-bottom}{Continue reading the main story}

\hypertarget{site-index}{%
\subsection{Site Index}\label{site-index}}

\hypertarget{site-information-navigation}{%
\subsection{Site Information
Navigation}\label{site-information-navigation}}

\begin{itemize}
\tightlist
\item
  \href{https://help.nytimes.com/hc/en-us/articles/115014792127-Copyright-notice}{©~2020~The
  New York Times Company}
\end{itemize}

\begin{itemize}
\tightlist
\item
  \href{https://www.nytco.com/}{NYTCo}
\item
  \href{https://help.nytimes.com/hc/en-us/articles/115015385887-Contact-Us}{Contact
  Us}
\item
  \href{https://www.nytco.com/careers/}{Work with us}
\item
  \href{https://nytmediakit.com/}{Advertise}
\item
  \href{http://www.tbrandstudio.com/}{T Brand Studio}
\item
  \href{https://www.nytimes.com/privacy/cookie-policy\#how-do-i-manage-trackers}{Your
  Ad Choices}
\item
  \href{https://www.nytimes.com/privacy}{Privacy}
\item
  \href{https://help.nytimes.com/hc/en-us/articles/115014893428-Terms-of-service}{Terms
  of Service}
\item
  \href{https://help.nytimes.com/hc/en-us/articles/115014893968-Terms-of-sale}{Terms
  of Sale}
\item
  \href{https://spiderbites.nytimes.com}{Site Map}
\item
  \href{https://help.nytimes.com/hc/en-us}{Help}
\item
  \href{https://www.nytimes.com/subscription?campaignId=37WXW}{Subscriptions}
\end{itemize}
