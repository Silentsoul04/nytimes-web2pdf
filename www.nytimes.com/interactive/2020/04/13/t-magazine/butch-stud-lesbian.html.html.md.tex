\hypertarget{comments}{%
\subsection{\texorpdfstring{\protect\hyperlink{commentsContainer}{Comments}}{Comments}}\label{comments}}

\href{}{The Butches and Studs Who've Defied the Male Gaze and Redefined
Culture}\href{}{Skip to Comments}

The comments section is closed. To submit a letter to the editor for
publication, write to
\href{mailto:letters@nytimes.com}{\nolinkurl{letters@nytimes.com}}.

\hypertarget{the-butches-and-studs-whove-defied-the-male-gaze-and-redefined-culture}{%
\section{The Butches and Studs Who've Defied the Male Gaze and Redefined
Culture}\label{the-butches-and-studs-whove-defied-the-male-gaze-and-redefined-culture}}

April 13, 2020

\begin{itemize}
\item
\item
\item
\item
\item
  \emph{+}
\end{itemize}

Without their presence and contributions, queer aesthetics --- and the
arts at large --- would be far less rich.

\href{https://www.nytimes.com/interactive/2020/04/13/t-magazine/culture-issue-2020.html}{We
Are Family}

\hypertarget{chapter-3-legends-pioneers-and-survivors}{%
\subparagraph{Chapter 3: Legends Pioneers and
Survivors}\label{chapter-3-legends-pioneers-and-survivors}}

\hypertarget{previous}{%
\subparagraph{Previous}\label{previous}}

\hypertarget{next}{%
\subparagraph{Next}\label{next}}

\hypertarget{the-butches-and-studs-whove-defied-the-male-gaze-and-redefined-culture-1}{%
\section{The Butches and Studs Who've Defied the Male Gaze and Redefined
Culture}\label{the-butches-and-studs-whove-defied-the-male-gaze-and-redefined-culture-1}}

\hypertarget{the-renegades}{%
\subsection{The Renegades}\label{the-renegades}}

Queer culture and the arts would be much poorer without the presence and
contribution of butch and stud lesbians, whose identity is both its own
aesthetic and a defiant repudiation of the male gaze.

By Kerry Manders

April 13, 2020

SHARE

Eileen Myles, Roxane Gay, Lea DeLaria and others discuss their
aesthetic, their identity and their place within L.G.B.T.Q. history.
Video by Caroline Berler

``BUTCH'' HAS LONG been the name we've given a certain kind ---
\emph{that} kind --- of lesbian. The old adage applies: You know her
when you see her. She wears men's clothing, short hair, no makeup. Butch
is an aesthetic, but it also conveys an attitude and energy. Both a
gender and a sexuality, butchness is about the body but also transcends
it: ``We exist in this realm of masculinity that has nothing to do with
cis men --- that's the part only we {[}butches{]} know how to talk
about,'' says the 42-year-old writer, former Olympic swimmer and men's
wear model
\href{https://www.nytimes.com/2018/08/08/style/casey-legler-olympic-swimmer-model.html}{Casey
Legler}. ``Many people don't even know how to ask questions about who we
are, or about what it means to be us.''

Many of us wear the butch label with a certain self-consciousness,
fearing the term doesn't quite fit --- like a new pair of jeans, it's
either too loose or too tight. The graphic novelist
\href{https://www.nytimes.com/2016/02/05/t-magazine/entertainment/my-10-favorite-books-alison-bechdel.html}{Alison
Bechdel}, 59, doesn't refer to herself as butch but understands why
others do. ``It's a lovely word, `butch': I'll take it, if you give it
to me,'' she says. ``But I'm afraid I'm not butch enough to really claim
it. Because part of being butch is \emph{owning} it, the whole aura
around it.''

What does owning it look like? Decades before genderless fashion became
\href{https://www.nytimes.com/2019/08/05/t-magazine/fall-fashion-androgyny.html}{its
own style}, butches were wearing denim and white tees, leather jackets
and work boots, wallet chains and gold necklaces. It isn't just about
what you're wearing, though, but how: Butchness embodies a certain
swagger, a 1950s-inspired
``\href{https://www.nytimes.com/watching/recommendations/watching-film-rebel-without-a-cause}{Rebel
Without a Cause}'' confidence. In doing so, these women --- and butches
who don't identify as women --- created something new and distinct, an
identity you could recognize even if you didn't know what to call it.

By refuting conventionally gendered aesthetics, butchness expands the
possibilities for women of all sizes, races, ethnicities and abilities.
``I always think of the first butch lesbian I ever saw,'' says the
33-year-old actor
\href{https://artsbeat.blogs.nytimes.com/2014/08/07/fun-home-will-reach-broadway-just-before-tonys-deadline/}{Roberta
Colindrez}. ``This beautiful butch came into the grocery store and she
was built like a brick house. Short hair, polo shirt, cargo pants and
that ring of keys \ldots{} It was the first time I saw the possibility
of who I was.'' And yet, to many people, ``butch style'' remains an
oxymoron: There's a prevalent assumption that we're all fat, frumpy
fashion disasters --- our baseball caps and baggy pants suggest to
others that we don't care about self-presentation. But it's not that
we're careless; it's that unlike, say, the gay white men who have been
given all too much credit for influencing contemporary visual culture,
we're simply not out to appease the male gaze. We disregard and reject
the confines of a sexualized and commodified femininity.

ETYMOLOGICALLY, ``butch'' is believed to be an abbreviation of
``butcher,'' American slang for ``tough kid'' in the early 20th century
and likely inspired by the outlaw
\href{https://www.nytimes.com/watching/recommendations/watching-film-butch-cassidy-and-the-sundance-kid}{Butch
Cassidy}. By the early 1940s, the word was used as a pejorative to
describe ``aggressive'' or ``macho'' women, but lesbians reclaimed it
almost immediately, using it with pride at 1950s-era bars such as
Manhattan's
\href{https://www.nyclgbtsites.org/site/mad-hatter-pony-stable-inn/}{Pony
Stable Inn} and Peg's Place in San Francisco. At these spots, where
cocktails cost 10 cents and police raids were a regular occurrence,
identifying yourself as either butch or femme was a prerequisite for
participating in the scene.

These butches were, in part, inspired by 19th-century cross-dressers ---
then called male impersonators or transvestites --- who presented and
lived fully as men in an era when passing was a crucial survival tactic.
We can also trace butchness back to the androgynous female artists of
early 20th-century Paris, including the writer
\href{https://www.nytimes.com/topic/person/gertrude-stein}{Gertrude
Stein} and the painter
\href{https://americanart.si.edu/artist/romaine-brooks-599}{Romaine
Brooks}. But it wasn't until the 1960s and early 1970s that butches,
themselves at the intersection of the burgeoning civil, gay and women's
rights movements, became a more visible and viable community.

From their earliest incarnations, butches faced brutal discrimination
and oppression, not only from outside their community but also from
within. A certain brand of (mostly white) lesbian feminism dominant in
the late '70s and early '80s marginalized certain sorts of ``otherness''
--- working-class lesbians, lesbians of color and masculine-of-center
women. They pilloried butchness as inextricably misogynist and
butch-femme relationships as dangerous replications of heteronormative
roles. (Such rhetoric has resurfaced, as trans men are regularly accused
of being anti-feminist in their desire to become the so-called enemy.)
Challenged yet again to defend their existence and further define
themselves, butches emerged from this debate emboldened, thriving in the
late '80s and early '90s as women's studies programs --- and, later,
gender and queer studies departments --- gained traction on North
American and European college campuses.

\begin{quote}
``It's a lovely word, `butch': I'll take it, if you give it to me,''
says Alison Bechdel. ``But I'm afraid I'm not butch enough to really
claim it. Because part of being butch is \emph{owning} it, the whole
aura around it.''
\end{quote}

The '90s were in fact a transformative decade for the butch community.
In 1990, the American philosopher Judith Butler published her
groundbreaking
``\href{https://www.routledge.com/Gender-Trouble-Feminism-and-the-Subversion-of-Identity-1st-Edition/Butler/p/book/9780415389556}{Gender
Trouble: Feminism and the Subversion of Identity},'' and her theories
about gender were soon translated and popularized for the masses. In her
academic work, Butler argues that gender and sexuality are both
constructed and performative; butch identity, as female masculinity,
subverts the notion that masculinity is the natural and exclusive
purview of the male body. Soon after, butch imagery infiltrated the
culture at large. The August 1993 issue of Vanity Fair featured the
straight supermodel
\href{https://www.nytimes.com/interactive/2015/10/21/t-magazine/21taketwo-degrasse-crawford.html}{Cindy
Crawford}, in a black maillot, straddling and shaving the butch icon
\href{https://www.nytimes.com/topic/person/k-d-lang}{K.D. Lang}. That
same year, the writer
\href{https://www.nytimes.com/2014/11/25/nyregion/leslie-feinberg-writer-and-transgender-activist-dies-at-65.html}{Leslie
Feinberg} published
``\href{https://microcosmpublishing.com/catalog/books/10392}{Stone Butch
Blues},'' a now classic novel about butch life in 1970s-era New York. In
Manhattan, comedians such as
\href{https://www.nytimes.com/2006/04/30/theater/girl-youre-a-man-now.html}{Lea
DeLaria} and drag kings such as Murray Hill took to the stage; it was
also the heyday of Bechdel's
``\href{https://www.indiebound.org/book/9780618968800}{Dykes to Watch
Out For},'' the serialized comic strip she started in 1983. In 1997,
\href{https://www.nytimes.com/topic/person/ellen-degeneres}{Ellen
DeGeneres}, still the most famous of butches, came out. Two years later,
Judith ``Jack'' Halberstam and Del LaGrace Volcano published
``\href{https://www.abebooks.com/9781852426071/Drag-King-Book-Halberstam-Judith-1852426071/plp}{The
Drag King Book}'' and the director
\href{https://www.nytimes.com/2013/09/29/magazine/carrie-is-back-so-is-kimberly-peirce.html}{Kimberly
Peirce} released her breakthrough film,
``\href{https://www.nytimes.com/watching/titles/movies/1000037931}{Boys
Don't Cry}''; its straight cisgender star, Hilary Swank, went on to win
an Oscar for her portrayal of Brandon Teena, a role that still
\href{https://www.nytimes.com/2019/10/09/movies/boys-dont-cry-anniversary.html}{incites
contentious debates} about the nebulous boundaries between butch and
\href{https://www.nytimes.com/2020/02/04/t-magazine/trans-actors.html}{trans
identity}. These artists and their legacies are the cornerstones of our
community. As Legler says, ``This is where we've come from, and the
folks we look back to. If you identify with that lineage, then we'd love
to have you.''

LIKE ANY QUEER subculture, butchness is vastly different now than it was
three decades ago --- though the codes have been tweaked and refined
over the years, younger butches continue to take them in new and varied
directions: They may experiment with their personas from day to day,
switching fluidly between masculine and feminine presentation. There are
``stone butches,'' a label that doesn't refer to coldness, as is often
assumed, but to a desire to touch rather than to be touched --- to give
rather than receive --- and is considered slightly more masculine than
``soft butch'' on the Futch Scale, a meme born in 2018 that attempted to
parse the gradations from ``high femme'' to ``stone butch.'' (``Futch,''
for ``femme/butch,'' is square in the middle.) And while there remains
some truth to butch stereotypes --- give us a plaid flannel shirt any
day of the week --- that once-static portrait falls apart under scrutiny
and reflection. Not every butch has short hair, can change a tire,
desires a femme. Some butches are bottoms. Some butches are bi. Some
butches are boys.

Different bodies own their butchness differently, but even a singular
body might do or be butch differently over time. We move between poles
as our feelings about --- and language for --- ourselves change. ``In my
early 20s, I identified as a stone butch,'' says the 45-year-old writer
\href{https://www.nytimes.com/column/roxane-gay}{Roxane Gay}. ``In
adulthood, I've come back to butch in terms of how I see myself in the
world and in my relationship, so I think of myself as soft butch now.''
Peirce, 52, adds that this continuum is as much an internal as an
external sliding scale: ``I've never aspired to a binary,'' she says.
``From day one, the idea of being a boy or a girl never made sense. The
ever-shifting signifiers of neither or both are what create meaning and
complexity.''

\begin{quote}
We rarely see butches depicted \emph{in} or \emph{as} community \ldots{}
but when you talk to butches, a more nuanced story emerges, one of deep
and abiding camaraderie and connection.
\end{quote}

Indeed, butch fluidity is especially resonant in our era of widespread
transphobia. Legler, who uses they/them pronouns, is a ``trans-butch
identified person --- no surgery, no hormones.'' Today, the
interconnected spectrums of gender and queerness are as vibrant and
diverse in language as they are in expression --- genderqueer,
transmasc, nonbinary, gender-nonconforming. Yet butches have always
called themselves and been called by many names: bull dyke, diesel dyke,
bulldagger, boi, daddy and so on. Language evolves, ``flowing in time
and changing constantly as new generations come along and social
structures shift,'' Bechdel says.

If it's necessary to think historically, it's also imperative to think
contextually. Compounding the usual homophobia and misogyny, black and
brown butches must contend with racist assumptions: ``Black women often
get read as butch whether they are butch or not,'' Gay says. ``Black
women in general are not seen, so black butchness tends to be doubly
invisible. Except for studs: They're very visible,'' she adds, referring
to a separate but related term used predominantly by black or Latinx
butches (though, unsurprisingly, white butches have appropriated it) who
are seen as ``harder'' in their heightened masculinity and attitude. Gay
notes that ``people tend to assume if you're a black butch, you're a
stud and that's it,'' which is ultimately untrue. Still, butch
legibility remains a paradox: As the most identifiable of lesbians ---
femmes often ``pass'' as straight, whether they want to or not --- we
are nonetheless maligned and erased for our failure of femininity, our
refusal to be the right kind of woman.

ANOTHER LINGERING stereotype, one born from ``Stone Butch Blues'' and
its more coded literary forebears, particularly
\href{https://www.britannica.com/biography/Radclyffe-Hall}{Radclyffe
Hall}'s
``\href{https://www.penguinrandomhouse.com/books/73965/the-well-of-loneliness-by-radclyffe-hall/}{The
Well of Loneliness}'' (1928), is the butch as a tragic and isolated
figure. She is either cast out by a dominant society that does not ---
will not --- ever see her or accept her, or she self-isolates as a
protective response to a world that continually and unrelentingly
disparages her.

When a butch woman \emph{does} appear in mainstream culture, it's
usually alongside her other: the femme lesbian. Without the femme and
the contrast she underscores, the butch is ``inherently
uncommodifiable,'' Bechdel says, since two butches together is just a
step ``too queer.'' We rarely see butches depicted \emph{in} or
\emph{as} community, an especially sobering observation given the
closure of so many lesbian bars over the past two decades. But when you
talk to butches, a more nuanced story emerges, one of deep and abiding
camaraderie and connection. Despite the dearth of representation, butch
love thrives --- in the anonymous, knowing glances across the subway
platform when we recognize someone like us, and in the bedroom, too.
``Many of my longest friendships are with people who register somewhere
on the butch scale,'' Peirce says. ``We're like married couples who fell
in love with each other as friends.''

Legler, for their part, recognizes a ``lone wolf'' effect, one in which
some young queers initially love ``being the only butch in the room.''
In organizing the group portrait that accompanies this essay over the
past months, Legler was curious ``what it would be like for butches to
just show up together and to be able to display all of their power, all
of their sexiness, all of their charisma, without having it be mitigated
in some way.'' And not only for butches of an older generation, but for
those still figuring things out, transforming the scene in ways that
both defy and inspire their elders. ``It's been centuries in the making,
the fact that we are all O.K.,'' Legler adds. ``That our bodies get to
exist: We have to celebrate that. You can do more than just survive. You
can \emph{contribute}.''

Not pictured: Rhea Butcher, KNOXXY, Kate Moennig, Catherine Opie, Yvonne
Rainer, Siya, Jill Soloway, Christine Vachon and Lena Waithe.

Kerry Manders is a writer, editor and photographer whose personal work
focuses on queer memory and mourning. Collier Schorr shows with 303
Gallery in New York City. Hair by Tamas Tuzes at L'Atelier NYC and
Latisha Chong. Makeup by Yumi Lee at Streeters. Set design by Jesse
Kaufmann at Frank Reps. Photo production by Hen's Tooth. Manicure: Ada
Yeung at Bridge Artists. Photo assistants: Jarrod Turner, Ari Sadok and
Tre Cassetta. Digital tech: Stephanie Levy. Stylist's assistants: Sarah
Lequimener, Andy Polanco and Umi Jiang. Hair assistants: Rachel
Polycarpe and Lamesha Mosely. Makeup assistants: Elika Hilata and Wakana
Ichikawa. Set assistants: Tyler Day and JP Huckins.

\href{https://www.nytimes.com/2019/12/04/t-magazine/gay-artwork-history.html}{}

\hypertarget{how-todays-queer-artists-are-revising-historydec-4-2019}{%
\paragraph{How Today's Queer Artists Are Revising HistoryDec. 4,
2019}\label{how-todays-queer-artists-are-revising-historydec-4-2019}}

\includegraphics{https://static01.nyt.com/images/2019/12/05/t-magazine/05tmag-revisionists-slide-TUME/05tmag-revisionists-slide-TUME-mediumThreeByTwo210.jpg}
\href{https://www.nytimes.com/2019/02/07/t-magazine/gay-children-book-authors.html}{}

\hypertarget{the-gay-history-of-americas-classic-childrens-booksfeb-7-2019}{%
\paragraph{The Gay History of America's Classic Children's BooksFeb. 7,
2019}\label{the-gay-history-of-americas-classic-childrens-booksfeb-7-2019}}

\includegraphics{https://static01.nyt.com/images/2019/01/28/t-magazine/oakImage-1548715394323/oakImage-1548715394323-mediumThreeByTwo210.jpg}
\href{https://www.nytimes.com/2018/05/16/t-magazine/food/female-chefs-rita-sodi-jody-williams-erika-nakamura.html}{}

\hypertarget{the-female-couples-remaking-the-restaurant-industrymay-16-2018}{%
\paragraph{The Female Couples Remaking the Restaurant IndustryMay 16,
2018}\label{the-female-couples-remaking-the-restaurant-industrymay-16-2018}}

\includegraphics{https://static01.nyt.com/images/2018/04/27/t-magazine/27tmag-chefs-slide-6289/27tmag-chefs-slide-6289-mediumThreeByTwo210.jpg}

\hypertarget{we-are-family-1}{%
\subsubsection{We Are Family}\label{we-are-family-1}}

\hypertarget{chapter-1-heirs-and-alumni}{%
\paragraph{Chapter 1: Heirs and
Alumni}\label{chapter-1-heirs-and-alumni}}

\href{/interactive/2020/04/13/t-magazine/black-art-galleries.html}{}

\hypertarget{the-artists}{%
\subparagraph{The Artists}\label{the-artists}}

\href{/interactive/2020/04/13/t-magazine/italian-fashion-design-houses.html}{}

\hypertarget{the-dynasties}{%
\subparagraph{The Dynasties}\label{the-dynasties}}

\href{/interactive/2020/04/13/t-magazine/gordon-parks.html}{}

\hypertarget{the-directors}{%
\subparagraph{The Directors}\label{the-directors}}

\href{/interactive/2020/04/13/t-magazine/enrique-olvera-chef.html}{}

\hypertarget{the-disciples}{%
\subparagraph{The Disciples}\label{the-disciples}}

\href{/interactive/2020/04/13/t-magazine/royal-academy-antwerp.html}{}

\hypertarget{the-graduates}{%
\subparagraph{The Graduates}\label{the-graduates}}

\hypertarget{chapter-2-reunions-and-reconsiderations}{%
\paragraph{Chapter 2: Reunions and
Reconsiderations}\label{chapter-2-reunions-and-reconsiderations}}

\href{/interactive/2020/04/13/t-magazine/ninth-street-greenwich-village-neighbors.html}{}

\hypertarget{the-neighbors}{%
\subparagraph{The Neighbors}\label{the-neighbors}}

\href{/interactive/2020/04/13/t-magazine/omen-restaurant-nyc.html}{}

\hypertarget{the-regulars}{%
\subparagraph{The Regulars}\label{the-regulars}}

\href{/interactive/2020/04/13/t-magazine/hair-musical-broadway.html}{}

\hypertarget{hair-1967}{%
\subparagraph{Hair (1967)}\label{hair-1967}}

\href{/interactive/2020/04/13/t-magazine/sweeney-todd-revival.html}{}

\hypertarget{sweeney-todd-2005-revival}{%
\subparagraph{Sweeney Todd (2005
Revival)}\label{sweeney-todd-2005-revival}}

\href{/interactive/2020/04/13/t-magazine/daughters-of-the-dust.html}{}

\hypertarget{daughters-of-the-dust-1991}{%
\subparagraph{Daughters of the Dust
(1991)}\label{daughters-of-the-dust-1991}}

\hypertarget{chapter-3-legends-pioneers-and-survivors-1}{%
\paragraph{Chapter 3: Legends Pioneers and
Survivors}\label{chapter-3-legends-pioneers-and-survivors-1}}

\href{/interactive/2020/04/13/t-magazine/butch-stud-lesbian.html}{}

\hypertarget{the-renegades-1}{%
\subparagraph{The Renegades}\label{the-renegades-1}}

\href{/interactive/2020/04/13/t-magazine/act-up-aids.html}{}

\hypertarget{the-activists}{%
\subparagraph{The Activists}\label{the-activists}}

\href{/interactive/2020/04/13/t-magazine/artist-recluse.html}{}

\hypertarget{the-shadows}{%
\subparagraph{The Shadows}\label{the-shadows}}

\href{/interactive/2020/04/13/t-magazine/black-actresses-bassett-berry-blige-henson-whitfield-elise.html}{}

\hypertarget{the-veterans}{%
\subparagraph{The Veterans}\label{the-veterans}}

\hypertarget{chapter-4-the-new-guard}{%
\paragraph{Chapter 4: The New Guard}\label{chapter-4-the-new-guard}}

\href{/interactive/2020/04/13/t-magazine/asian-american-fashion-designers.html}{}

\hypertarget{the-designers}{%
\subparagraph{The Designers}\label{the-designers}}

\href{13tmag-beauties.html}{}

\hypertarget{the-beauties}{%
\subparagraph{The Beauties}\label{the-beauties}}

\href{/interactive/2020/04/13/t-magazine/nyc-downtown-nightlife-party-scene.html}{}

\hypertarget{the-scenemakers}{%
\subparagraph{The Scenemakers}\label{the-scenemakers}}

\href{/interactive/2020/04/13/t-magazine/maria-cornejo-olivier-rousteing-telfar-clemens-alessandro-michele.html\#olivier-rousteing-and-co}{}

\hypertarget{olivier-rousteing-and-co}{%
\subparagraph{Olivier Rousteing and
Co.}\label{olivier-rousteing-and-co}}

\href{/interactive/2020/04/13/t-magazine/maria-cornejo-olivier-rousteing-telfar-clemens-alessandro-michele.html\#maria-cornejo-and-co}{}

\hypertarget{maria-cornejo-and-co}{%
\subparagraph{Maria Cornejo and Co.}\label{maria-cornejo-and-co}}

\href{/interactive/2020/04/13/t-magazine/maria-cornejo-olivier-rousteing-telfar-clemens-alessandro-michele.html\#telfar-clemens-and-co}{}

\hypertarget{telfar-clemens-and-co}{%
\subparagraph{Telfar Clemens and Co.}\label{telfar-clemens-and-co}}

\href{/interactive/2020/04/13/t-magazine/maria-cornejo-olivier-rousteing-telfar-clemens-alessandro-michele.html\#alessandro-michele-and-co}{}

\hypertarget{alessandro-michele-and-co}{%
\subparagraph{Alessandro Michele and
Co.}\label{alessandro-michele-and-co}}

\href{/interactive/2020/04/13/t-magazine/foreign-correspondents.html}{}

\hypertarget{the-journalists}{%
\subparagraph{The Journalists}\label{the-journalists}}

Write a comment

\begin{itemize}
\item
\item
\item
\item
\end{itemize}

Advertisement

\protect\hyperlink{after-bottom}{Continue reading the main story}

\hypertarget{site-index}{%
\subsection{Site Index}\label{site-index}}

\hypertarget{site-information-navigation}{%
\subsection{Site Information
Navigation}\label{site-information-navigation}}

\begin{itemize}
\tightlist
\item
  \href{https://help.nytimes.com/hc/en-us/articles/115014792127-Copyright-notice}{©~2020~The
  New York Times Company}
\end{itemize}

\begin{itemize}
\tightlist
\item
  \href{https://www.nytco.com/}{NYTCo}
\item
  \href{https://help.nytimes.com/hc/en-us/articles/115015385887-Contact-Us}{Contact
  Us}
\item
  \href{https://www.nytco.com/careers/}{Work with us}
\item
  \href{https://nytmediakit.com/}{Advertise}
\item
  \href{http://www.tbrandstudio.com/}{T Brand Studio}
\item
  \href{https://www.nytimes.com/privacy/cookie-policy\#how-do-i-manage-trackers}{Your
  Ad Choices}
\item
  \href{https://www.nytimes.com/privacy}{Privacy}
\item
  \href{https://help.nytimes.com/hc/en-us/articles/115014893428-Terms-of-service}{Terms
  of Service}
\item
  \href{https://help.nytimes.com/hc/en-us/articles/115014893968-Terms-of-sale}{Terms
  of Sale}
\item
  \href{https://spiderbites.nytimes.com}{Site Map}
\item
  \href{https://help.nytimes.com/hc/en-us}{Help}
\item
  \href{https://www.nytimes.com/subscription?campaignId=37WXW}{Subscriptions}
\end{itemize}
