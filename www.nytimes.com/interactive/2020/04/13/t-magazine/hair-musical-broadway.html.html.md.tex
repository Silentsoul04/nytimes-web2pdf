\hypertarget{comments}{%
\subsection{\texorpdfstring{\protect\hyperlink{commentsContainer}{Comments}}{Comments}}\label{comments}}

\href{}{Why `Hair' Has Endured}\href{}{Skip to Comments}

The comments section is closed. To submit a letter to the editor for
publication, write to
\href{mailto:letters@nytimes.com}{\nolinkurl{letters@nytimes.com}}.

\hypertarget{why-hair-has-endured}{%
\section{Why `Hair' Has Endured}\label{why-hair-has-endured}}

April 13, 2020

\begin{itemize}
\item
\item
\item
\item
\item
  \emph{+}
\end{itemize}

An appreciation of the 1967 love-rock musical, which, against the odds,
won over audiences across the world.

\href{https://www.nytimes.com/interactive/2020/04/13/t-magazine/culture-issue-2020.html}{We
Are Family}

\hypertarget{chapter-2-reunions-and-reconsiderations}{%
\subparagraph{Chapter 2: Reunions and
Reconsiderations}\label{chapter-2-reunions-and-reconsiderations}}

\hypertarget{previous}{%
\subparagraph{Previous}\label{previous}}

\hypertarget{next}{%
\subparagraph{Next}\label{next}}

\textbf{The forerunners}

\hypertarget{why-hair-has-endured-1}{%
\section{Why `Hair' Has Endured}\label{why-hair-has-endured-1}}

Every now and then, a piece of American performance is so memorable that
it both redefines its medium and reframes the culture at large. Here, an
appraisal of one such enduring and heavily referenced work --- a
youth-inflected 1967 musical that captured the popular (and political)
consciousness --- alongside a gathering of the stars who not only made
it but were made by it, too.

By \href{https://www.nytimes.com/by/ben-brantley}{Ben Brantley}

April 13, 2020

SHARE

ITS REPUTATION REMAINS that of a dangerous young renegade, pumped to the
point of explosion with anger, hormones and mind-altering substances.
But in truth, ``Hair,'' which opened in 1967 at New York's Public
Theater, was always a sweetheart: open-armed, open-minded, as ready to
comfort as to party. Among American musicals of the late 1960s, it was
the cool kid you wanted to cuddle with, even if you were no longer a kid
yourself.

True, as the fame of this self-labeled ``tribal love-rock musical''
spread after its successful transfer to Broadway in 1968, it trailed a
heady perfume of notoriety. This, after all, was a work that featured
pot smoking, draft-card burning, references to a Kama Sutra of sexual
practices and a host of unkempt young things singing in the nude for its
first-act finale. The Acapulco, Mexico, 1969 premiere was closed by
government order after its first performance. The show's London
producers cannily waited until there was a change in censorship laws to
open it in 1968 in the West End. And just last year, ``Hair'' was
removed from the
\href{https://www.nytimes.com/2018/05/25/arts/television/hair-live-nbc.html}{schedule
of} NBC's series of live televised musicals, suggesting it still wasn't
ready for prime time.

The T List \textbar{}

Sign up here

Yet the liberating breeze emanating from this portrait of an improvised
family of acid-dropping dropouts in New York City would be felt
throughout the world, with productions popping up like mushrooms in
Europe, South America and Japan. And the largely middle-class audiences
who might have been alarmed by the prospect of rebellious youth at the
barricades discovered that it was,
\href{https://www.nytimes.com/1968/04/30/archives/theater-hair-its-fresh-and-frank-likable-rock-musical-moves-to.html}{in
the words} of the New York Times critic Clive Barnes ``so likable.
\ldots{} So new, so fresh, and so unassuming.''

In fact, the cast album of ``Hair'' was one that, as young teenagers, my
friends and I were allowed to play --- and dance to --- in our living
rooms and even on church retreats (as long as we skipped the track
called ``Sodomy''). Compared to the acid rock that was then flooding the
airwaves,
\href{https://www.nytimes.com/2018/12/18/obituaries/galt-macdermot-dead.html}{Galt
MacDermot}'s score --- even allowing for expletive-laced lyrics by the
show's creators,
\href{https://www.nytimes.com/1991/07/13/obituaries/gerome-ragni-48-a-stage-actor-co-author-of-broadway-s-hair.html}{Gerome
Ragni} and James Rado --- sounded as melodic as Rodgers and Hammerstein.
Its songs became Top-40 hits, covered by the likes of the
\href{https://www.nytimes.com/2006/02/21/arts/music/william-cowsill-58-leader-of-family-poprock-band-dies.html}{Cowsills}
(the title song) and the
\href{https://www.nytimes.com/2001/08/04/arts/ron-townson-68-singer-in-fifth-dimension.html}{5th
Dimension} (a medley).

It had been a while since songs from Broadway shows featured on pop
radio, and there was hope that ``Hair'' might inaugurate a new age of
hipness for the American musical. That never quite happened. Its
imitations (including a couple of flops involving its original creators)
didn't have the same impact, and a 1977 revival died after only 43
performances. Still, ``Hair'' suggested that what was regarded as an
aging genre could be surprisingly accommodating to new styles and
voices, even if it would be decades before a contemporary, teen-appeal
musical achieved a similarly hopeful, global impact: first with
``\href{https://www.nytimes.com/2008/09/21/theater/21ishe.html}{Rent},''
in 1996, and later, most spectacularly, with
``\href{https://www.nytimes.com/2016/06/12/theater/hamilton-inc-the-path-to-a-billion-dollar-show.html}{Hamilton},''
in 2015.

IN A 1993 INTERVIEW, MacDermot said that ``Hair'' was not ultimately a
``political show'' but one ``about kids having fun and making fun of
things.'' Though it had a loose string of a plot, about a young man
afraid of being drafted for the Vietnam War, it was mostly a series of
sketches in which the characters satirized their convention-bound elders
and extolled the joys of sexual and pharmaceutical highs. Referring to a
wildly popular 1938 revue that combined topical sketch humor with antic
musical numbers, MacDermot called ``Hair'' the ```Hellzapoppin' of its
generation.''

And like ``Hellzapoppin,'' ``Hair'' seemed destined to fade into that
bright oblivion reserved for period novelties like Monkees albums and
troll dolls. Yet when I went to see the director
\href{https://www.nytimes.com/2016/05/15/realestate/tony-nominee-diane-paulus-at-home-on-the-upper-west-side.html}{Diane
Paulus}'s 2008 revival of the show in Central Park (which subsequently
transferred to Broadway), I was surprised to discover how moved I was by
it, and not just for nostalgic reasons. It was the tribal aspect of the
``tribal love-rock'' equation that got to me all those years later ---
its sense of vulnerable people banding together on the threshold of
adulthood, trying to postpone their entry into the scary world that
their elders had created. What little story the show \emph{did} have,
after all, pivoted on whether one of its characters would be drafted
into a conflict that made the United States as rancorously divided as it
has ever been in my lifetime --- until now.

The divisions then often included estrangements of teenagers from their
parents. So some young people wound up forming alternative clans in
which you chose your own family. It's a design for living that has been
translated into both the murderous darkness of the 1960s-era Charles
Manson cult and into the anodyne sitcom blitheness of the 1990s
television series
``\href{https://www.nytimes.com/watching/recommendations/watching-tv-friends}{Friends}''
or the messier 2010s-era
``\href{https://www.nytimes.com/watching/recommendations/watching-tv-girls}{Girls}.''
The clan members of ``Hair'' got high together and slept together, yes,
but they also nurtured, consoled and entertained one another. And for
much of the show, it's that reciprocally supportive camaraderie that
makes the musical feel so alive.

What I'd forgotten, though --- until I saw Paulus's production --- was
the show's awareness of its own ephemerality, its sense that tribes of
youth are destined to last only as long as youth itself. ``They'll never
get me,'' says Berger, the de facto leader of the show's nomadic gang.
``I'm gonna stay high forever.''

But the final number in ``Hair'' isn't one of its odes to defiant
individualism --- the title song, or ``I Got Life,'' or the twinkly
``Good Morning Starshine.'' It's called ``The Flesh Failures,'' and
though its subtitle is ``Let the Sunshine In,'' it leaves you with the
feeling that darkness is fast descending --- and that the tightly bound
coterie at the center of ``Hair'' may well have scattered forever by
daybreak.

Ben Brantley has been The New York Times's co-chief theater critic since
1996. Nicholas Calcott is working on a collection of portraits of New
York City artists. Photo assistants: Carlos Jaramillo and Maeve
Fitzhoward. Digital tech: Chen Xiangyun.

\href{https://www.nytimes.com/2018/04/16/t-magazine/broadway-1980s-actors-sarah-jessica-parker-willem-dafoe.html}{}

\hypertarget{the-stars-who-got-their-start-on-the-80s-new-york-stageapril-16-2018}{%
\paragraph{The Stars Who Got Their Start on the '80s New York StageApril
16,
2018}\label{the-stars-who-got-their-start-on-the-80s-new-york-stageapril-16-2018}}

\includegraphics{https://static01.nyt.com/images/2018/04/05/t-magazine/05tmag-group-slide-QT9P/05tmag-group-slide-QT9P-mediumThreeByTwo210.jpg}
\href{https://www.nytimes.com/2018/08/09/t-magazine/allison-janney-idina-menzel-female-actresses.html}{}

\hypertarget{these-actresses-dominate-not-just-the-stage-but-the-screen-tooaug-9-2018}{%
\paragraph{These Actresses Dominate Not Just the Stage, but the Screen
TooAug. 9,
2018}\label{these-actresses-dominate-not-just-the-stage-but-the-screen-tooaug-9-2018}}

\includegraphics{https://static01.nyt.com/images/2018/08/09/t-magazine/09tmag-women/09tmag-women-mediumThreeByTwo210.jpg}

\hypertarget{we-are-family-1}{%
\subsubsection{We Are Family}\label{we-are-family-1}}

\hypertarget{chapter-1-heirs-and-alumni}{%
\paragraph{Chapter 1: Heirs and
Alumni}\label{chapter-1-heirs-and-alumni}}

\href{/interactive/2020/04/13/t-magazine/black-art-galleries.html}{}

\hypertarget{the-artists}{%
\subparagraph{The Artists}\label{the-artists}}

\href{/interactive/2020/04/13/t-magazine/italian-fashion-design-houses.html}{}

\hypertarget{the-dynasties}{%
\subparagraph{The Dynasties}\label{the-dynasties}}

\href{/interactive/2020/04/13/t-magazine/gordon-parks.html}{}

\hypertarget{the-directors}{%
\subparagraph{The Directors}\label{the-directors}}

\href{/interactive/2020/04/13/t-magazine/enrique-olvera-chef.html}{}

\hypertarget{the-disciples}{%
\subparagraph{The Disciples}\label{the-disciples}}

\href{/interactive/2020/04/13/t-magazine/royal-academy-antwerp.html}{}

\hypertarget{the-graduates}{%
\subparagraph{The Graduates}\label{the-graduates}}

\hypertarget{chapter-2-reunions-and-reconsiderations-1}{%
\paragraph{Chapter 2: Reunions and
Reconsiderations}\label{chapter-2-reunions-and-reconsiderations-1}}

\href{/interactive/2020/04/13/t-magazine/ninth-street-greenwich-village-neighbors.html}{}

\hypertarget{the-neighbors}{%
\subparagraph{The Neighbors}\label{the-neighbors}}

\href{/interactive/2020/04/13/t-magazine/omen-restaurant-nyc.html}{}

\hypertarget{the-regulars}{%
\subparagraph{The Regulars}\label{the-regulars}}

\href{/interactive/2020/04/13/t-magazine/hair-musical-broadway.html}{}

\hypertarget{hair-1967}{%
\subparagraph{Hair (1967)}\label{hair-1967}}

\href{/interactive/2020/04/13/t-magazine/sweeney-todd-revival.html}{}

\hypertarget{sweeney-todd-2005-revival}{%
\subparagraph{Sweeney Todd (2005
Revival)}\label{sweeney-todd-2005-revival}}

\href{/interactive/2020/04/13/t-magazine/daughters-of-the-dust.html}{}

\hypertarget{daughters-of-the-dust-1991}{%
\subparagraph{Daughters of the Dust
(1991)}\label{daughters-of-the-dust-1991}}

\hypertarget{chapter-3-legends-pioneers-and-survivors}{%
\paragraph{Chapter 3: Legends Pioneers and
Survivors}\label{chapter-3-legends-pioneers-and-survivors}}

\href{/interactive/2020/04/13/t-magazine/butch-stud-lesbian.html}{}

\hypertarget{the-renegades}{%
\subparagraph{The Renegades}\label{the-renegades}}

\href{/interactive/2020/04/13/t-magazine/act-up-aids.html}{}

\hypertarget{the-activists}{%
\subparagraph{The Activists}\label{the-activists}}

\href{/interactive/2020/04/13/t-magazine/artist-recluse.html}{}

\hypertarget{the-shadows}{%
\subparagraph{The Shadows}\label{the-shadows}}

\href{/interactive/2020/04/13/t-magazine/black-actresses-bassett-berry-blige-henson-whitfield-elise.html}{}

\hypertarget{the-veterans}{%
\subparagraph{The Veterans}\label{the-veterans}}

\hypertarget{chapter-4-the-new-guard}{%
\paragraph{Chapter 4: The New Guard}\label{chapter-4-the-new-guard}}

\href{/interactive/2020/04/13/t-magazine/asian-american-fashion-designers.html}{}

\hypertarget{the-designers}{%
\subparagraph{The Designers}\label{the-designers}}

\href{13tmag-beauties.html}{}

\hypertarget{the-beauties}{%
\subparagraph{The Beauties}\label{the-beauties}}

\href{/interactive/2020/04/13/t-magazine/nyc-downtown-nightlife-party-scene.html}{}

\hypertarget{the-scenemakers}{%
\subparagraph{The Scenemakers}\label{the-scenemakers}}

\href{/interactive/2020/04/13/t-magazine/maria-cornejo-olivier-rousteing-telfar-clemens-alessandro-michele.html\#olivier-rousteing-and-co}{}

\hypertarget{olivier-rousteing-and-co}{%
\subparagraph{Olivier Rousteing and
Co.}\label{olivier-rousteing-and-co}}

\href{/interactive/2020/04/13/t-magazine/maria-cornejo-olivier-rousteing-telfar-clemens-alessandro-michele.html\#maria-cornejo-and-co}{}

\hypertarget{maria-cornejo-and-co}{%
\subparagraph{Maria Cornejo and Co.}\label{maria-cornejo-and-co}}

\href{/interactive/2020/04/13/t-magazine/maria-cornejo-olivier-rousteing-telfar-clemens-alessandro-michele.html\#telfar-clemens-and-co}{}

\hypertarget{telfar-clemens-and-co}{%
\subparagraph{Telfar Clemens and Co.}\label{telfar-clemens-and-co}}

\href{/interactive/2020/04/13/t-magazine/maria-cornejo-olivier-rousteing-telfar-clemens-alessandro-michele.html\#alessandro-michele-and-co}{}

\hypertarget{alessandro-michele-and-co}{%
\subparagraph{Alessandro Michele and
Co.}\label{alessandro-michele-and-co}}

\href{/interactive/2020/04/13/t-magazine/foreign-correspondents.html}{}

\hypertarget{the-journalists}{%
\subparagraph{The Journalists}\label{the-journalists}}

Write a comment

\begin{itemize}
\item
\item
\item
\item
\end{itemize}

Advertisement

\protect\hyperlink{after-bottom}{Continue reading the main story}

\hypertarget{site-index}{%
\subsection{Site Index}\label{site-index}}

\hypertarget{site-information-navigation}{%
\subsection{Site Information
Navigation}\label{site-information-navigation}}

\begin{itemize}
\tightlist
\item
  \href{https://help.nytimes.com/hc/en-us/articles/115014792127-Copyright-notice}{©~2020~The
  New York Times Company}
\end{itemize}

\begin{itemize}
\tightlist
\item
  \href{https://www.nytco.com/}{NYTCo}
\item
  \href{https://help.nytimes.com/hc/en-us/articles/115015385887-Contact-Us}{Contact
  Us}
\item
  \href{https://www.nytco.com/careers/}{Work with us}
\item
  \href{https://nytmediakit.com/}{Advertise}
\item
  \href{http://www.tbrandstudio.com/}{T Brand Studio}
\item
  \href{https://www.nytimes.com/privacy/cookie-policy\#how-do-i-manage-trackers}{Your
  Ad Choices}
\item
  \href{https://www.nytimes.com/privacy}{Privacy}
\item
  \href{https://help.nytimes.com/hc/en-us/articles/115014893428-Terms-of-service}{Terms
  of Service}
\item
  \href{https://help.nytimes.com/hc/en-us/articles/115014893968-Terms-of-sale}{Terms
  of Sale}
\item
  \href{https://spiderbites.nytimes.com}{Site Map}
\item
  \href{https://help.nytimes.com/hc/en-us}{Help}
\item
  \href{https://www.nytimes.com/subscription?campaignId=37WXW}{Subscriptions}
\end{itemize}
