\hypertarget{the-neighbors-of-ninth-street}{%
\section{The Neighbors of Ninth
Street}\label{the-neighbors-of-ninth-street}}

April 13, 2020

\begin{itemize}
\item
\item
\item
\item
\end{itemize}

For decades, two blocks in Greenwich Village have been home to a
disproportionate number of New York City's writers, artists, actors and
designers.

\href{https://www.nytimes.com/interactive/2020/04/13/t-magazine/culture-issue-2020.html}{We
Are Family}

\hypertarget{chapter-2-reunions-and-reconsiderations}{%
\subparagraph{Chapter 2: Reunions and
Reconsiderations}\label{chapter-2-reunions-and-reconsiderations}}

\hypertarget{previous}{%
\subparagraph{Previous}\label{previous}}

\hypertarget{next}{%
\subparagraph{Next}\label{next}}

\hypertarget{the-neighbors-of-ninth-street-1}{%
\section{The Neighbors of Ninth
Street}\label{the-neighbors-of-ninth-street-1}}

The two blocks in Greenwich Village that have been home to a
disproportionate number of New York City's writers, artists, actors and
designers for decades.

April 13, 2020

SHARE

In New York City, your neighborhood is not just where you live: It
represents who you are. ``People are community beings. And while you can
think of the entire city as your community in abstract terms, you can't
really enact that,'' says the writer
\href{https://www.penguinrandomhouse.com/authors/20823/susan-minot}{Susan
Minot}, who has lived on West Ninth Street in Greenwich Village since
1991. Throughout its many lives --- as a rural hamlet in the 17th
century, a bohemian haven for much of the 20th and, more recently, home
to a seven-block stretch that real-estate agents sometimes refer to as
Manhattan's Gold Coast --- the Village has remained essentially
residential and thus never without community.

Within this neighborhood is West Ninth Street. It runs for a single
block --- between Fifth and Sixth Avenues --- and has a sort of
companion block in East Ninth Street between Broadway and University
Place. Together, these two blocks, lined with ginkgo trees
(``Greenwich'' came from \emph{Groenwijck}, which is Dutch for ``Green
District''), stately apartment buildings and the occasional restaurant,
can feel like a world apart, one that, despite the inevitable closures
and transformations, has proved relatively immune to change. It's not
just the landscape that recalls old New York, though. Even in 2020, when
artists and writers are more likely to live in Brooklyn, or maybe
upriver in Hudson, these two blocks are occupied by an outsize number of
creative people --- including the actors Lee Pace and
\href{https://www.nytimes.com/2018/08/22/arts/television/amy-sedaris-emmys.html}{Amy
Sedaris}, and the artist
\href{https://tmagazine.blogs.nytimes.com/2011/07/22/asked-answered-helmut-lang/}{Helmut
Lang}, as well as the people pictured here --- just as they were 70
years ago. In
\href{https://www.nytimes.com/topic/person/alfred-hitchcock}{Alfred
Hitchcock}'s
``\href{https://www.nytimes.com/watching/recommendations/watching-film-rear-window}{Rear
Window}'' (1954), the photographer L.B. Jefferies looks out from his
perch on West 10th Street onto the buildings of an imagined West Ninth,
observing as their tenants, who include a dancer, a sculptor and a
composer, go about their daily routines. The film's genre aside, there
is an undeniable coziness to their setup --- clearly these are
characters who are gambling on their talents, without doubt or apology.

The T List \textbar{}

Sign up here

The same could be said of
\href{https://www.nytimes.com/topic/person/helen-frankenthaler}{Helen
Frankenthaler} and
\href{https://www.nytimes.com/1992/10/31/arts/joan-mitchell-abstract-artist-is-dead-at-66.html}{Joan
Mitchell}, who were among the artists who lived in derelict Greenwich
Village lofts and showed work in the watershed 1951 exhibition ``The
Ninth Street Show,'' held in a rented storefront at 60 East Ninth Street
(Gertrude Vanderbilt Whitney had opened her museum just a couple blocks
away, at 8 West Eighth Street, 20 years prior). ``If artists were drawn
here, they just had good taste,'' says Minot. ``The buildings aren't too
high, you're kind of near the river, the streets go this way and that
--- it's got a human scale to it.'' But with that human scale comes
actual humans, the people you live alongside, and all the better if
they're as interesting as these neighbors are. --- KATE GUADAGNINO

Kate Guadagnino is the deputy digital editor of T Magazine. Sean Donnola
works in photography, film and video. His work is in the permanent
collection of the Metropolitan Museum of Art.

Sean Donnola

\hypertarget{here-residents-share-how-they-ended-up-on-ninth-street-and-whats-kept-them-there-read-a-behind-the-scenes-account-of-the-making-of-the-group-portrait-taken-on-a-sunny-morning-in-january}{%
\subsubsection{Here, residents share how they ended up on Ninth Street
and what's kept them there. Read a behind-the-scenes account of the
making of the group portrait, taken on a sunny morning in
January.}\label{here-residents-share-how-they-ended-up-on-ninth-street-and-whats-kept-them-there-read-a-behind-the-scenes-account-of-the-making-of-the-group-portrait-taken-on-a-sunny-morning-in-january}}

\hypertarget{jay-mcinerney}{%
\paragraph{JAY McINERNEY}\label{jay-mcinerney}}

\hypertarget{writer-on-ninth-street-since-2001}{%
\subparagraph{\texorpdfstring{\textbf{Writer, on Ninth Street since
2001.}}{Writer, on Ninth Street since 2001.}}\label{writer-on-ninth-street-since-2001}}

At the end of 2001, I was living in Chelsea, and I'd just watched the
Twin Towers fall from my apartment. It was too depressing to stay there.
I'd always loved this stretch of the Village and was visiting a friend
near here when I saw a ``For Rent'' sign outside of 20 Fifth Avenue. I
was recently divorced and kind of at loose ends, and I just impulsively
decided to do it. My friend
\href{https://www.nytimes.com/2019/08/02/nyregion/candace-bushnell.html}{Candace
Bushnell} moved to Ninth Street not long after. We'd often get together
at my place or hers, where there was a rotating cast of characters. It
was pretty raucous. I remember Candace once throwing some glasses out
the window because, she said, they were misbehaving. Before all that,
Marylou's, the Italian restaurant on the north side of Ninth, was one of
my hangouts. I hesitate to mention it --- it was a real den of iniquity.
If you got there at 11 and stayed till 2, you would see \emph{tout le
monde} --- movie stars, writers, gangsters. I miss it, but my life span
will probably be longer because it closed. Still, Ninth Street feels
like a bit of a time capsule. I like to think it looks something like it
did when the poet Marianne Moore lived here in the 1960s.

\hypertarget{terrence-mcnally}{%
\paragraph{TERRENCE McNALLY}\label{terrence-mcnally}}

\hypertarget{playwright-on-ninth-street-since-1996}{%
\subparagraph{\texorpdfstring{\textbf{Playwright, on Ninth Street since
1996.}}{Playwright, on Ninth Street since 1996.}}\label{playwright-on-ninth-street-since-1996}}

Ninth Street always struck me as very lovely, and I'd think, ``Gee, one
day I'd like to live there.'' I fell in love with my apartment the
moment I opened the door. I didn't know at the time that it had so many
theatrical reverberations:
\href{https://www.nytimes.com/1979/11/16/archives/jed-harris-broadway-producer-and-director-for-30-years-dead.html}{Jed
Harris}, a very controversial producer in his day, is said to have lived
in the building;
\href{https://www.nytimes.com/1993/10/08/obituaries/agnes-de-mille-88-dance-visionary-is-dead.html}{Agnes
de Mille}, too. I often call this neighborhood ``Lower Fifth.'' It's a
unique part of New York, these few little blocks. They have this
elegance, and I've always loved seeing the older women, who I'm sure are
widowed by now, on the street. They're always made up and beautifully
dressed --- some of them still wear gloves --- and at Christmas, they
make cookies for all the doormen. They're the women of the old
\href{https://www.nytimes.com/1968/02/23/archives/peter-arno-cartoonist-64-dies-with-the-new-yorker-43-years-peter.html}{Peter
Arno} New Yorker cartoons, the women who wouldn't be caught dead on the
street in slacks or with their hair in a scarf. There are very few of
them left; each year, there's one fewer. It's very contemporary, too,
though. It's New York as a place for doers, achievers, people who are
involved. Not nostalgic. Not lazy. There's a briskness to everybody's
step. I feel at home here --- you know it when you've found a place you
belong.

\textbf{EDITORS' NOTE:} \emph{Mr. McNally died on March 24, 2020, the
day before this issue went to press.}

\hypertarget{wendy-goodman}{%
\paragraph{WENDY GOODMAN}\label{wendy-goodman}}

\hypertarget{writer-and-editor-on-ninth-street-since-1997}{%
\subparagraph{\texorpdfstring{\textbf{Writer and editor, on Ninth Street
since
1997.}}{Writer and editor, on Ninth Street since 1997.}}\label{writer-and-editor-on-ninth-street-since-1997}}

I was raised on the Upper East Side, but I always said to myself, ``When
I grow up, I'm moving downtown.'' And that's what I did --- I've lived
in the Village most of my adult life, though I didn't move to Ninth
Street until 1997. I'd just split up with my boyfriend and was desperate
for an apartment. So that was the situation as I stood in a phone booth
and called a friend who told me I had to see this place on Ninth. The
catch was it wasn't available yet, but as soon as I saw it from the
outside, I knew I'd wait it out no matter what. It turned out to be one
of those marvelous places in the Village with a fireplace, but the best
thing about it were the landlords. The place belonged to the artist
Sarai Sherman and her husband, David Jaffe, a psychiatrist, and they
lived upstairs. My apartment faces north, so I can see into the gardens
and kitchens of the brownstones behind our building, including, I
realized after moving in, that of my friend, the writer
\href{https://www.nytimes.com/2016/08/05/t-magazine/entertainment/andrew-solomon-favorite-books.html}{Andrew
Solomon}. We could have used tin cans and a string to talk. Ninth Street
is one of those magical blocks, and thankfully the neighbors are banding
together to ward off encroaching development.

\hypertarget{simon-doonan}{%
\paragraph{SIMON DOONAN}\label{simon-doonan}}

\hypertarget{writer-on-ninth-street-since-1994}{%
\subparagraph{\texorpdfstring{\textbf{Writer, on Ninth Street since
1994.}}{Writer, on Ninth Street since 1994.}}\label{writer-on-ninth-street-since-1994}}

When I was a kid, I went to see
``\href{https://www.nytimes.com/watching/recommendations/watching-film-rosemarys-baby}{Rosemary's
Baby}'' and fell in love with the apartment in the movie, which was
filmed at the Dakota on Central Park West. It was seared into my head as
some sort of optimal living experience. Later, I used to walk by 35 East
Ninth Street and think of it as the ``downtown Dakota,'' because it had
a little of that same mystique --- it's somber looking but also
fabulous. When I met Jonny {[}Jonathan Adler, a designer and Doonan's
husband{]}, I had just decided to buy an apartment in that building. I
think he took one look and thought, ``Oh, this is a good proposition. A
nice bit of real estate is a good aphrodisiac.'' Not long after, Jonny
moved in. He was 28 (to my 42) and was just starting to get some
recognition. He was essentially a production potter and would
Rollerblade into the apartment with lumps of clay cascading off him.
Eventually, he bought the apartment next door, and we combined them to
create one sprawling unit. I was born after the war in a two-room flat
with no kitchen or bathroom, so the idea that I ended up here, it's sort
of a dream. I think I'll be carried out of my apartment in a box. And I
love the block. Everybody walks their dog, and there are so many people
that I stop and kibbitz with --- I always run into the art director
\href{https://www.nytimes.com/2015/03/17/t-magazine/behind-sam-shahids-closed-doors.html}{Sam
Shahid} and the jewelry designer
\href{http://anndexterjonesdesign.com/About/}{Ann Dexter-Jones} when I'm
out picking up poop.

\hypertarget{susan-minot}{%
\paragraph{SUSAN MINOT}\label{susan-minot}}

\hypertarget{writer-on-ninth-street-since-1991}{%
\subparagraph{\texorpdfstring{\textbf{Writer, on Ninth Street since
1991.}}{Writer, on Ninth Street since 1991.}}\label{writer-on-ninth-street-since-1991}}

I'd just gotten a book contract and bought my place by the skin of my
teeth. I completely don't belong on this street. It's so far beyond and
I can't really afford the restaurants around here. Somehow, though, I've
managed to hang on. Even when I spent a chunk of the aughts living on an
island in Maine, I kept my apartment and was so grateful to have it to
come back to. It's in a building built in 1885 with units intended as
``gentleman's quarters'' and was a crumbling wreck when I got it, but to
me it was heaven. I turned the small back room that was once the kitchen
into my office, and then it became my daughter's bedroom. As in any
urban spot, the construction can drive you crazy. But then you remember
the things that are wonderful here. For a little while each afternoon,
sunlight comes in from the south, though it's mainly the variety of the
people and the depths of the friendships. You can spend a long time not
knowing your neighbors, and then start to. One of the things I love
about New York is that you don't have to put on a face --- you know
you're just going to be one in the stream. I let people in slowly, and
having that familiarity, it's comforting. After 30-some years of
thinking, ``Well, I'm just passing through and I don't really live
here,'' I will admit that oh yeah, this is my neighborhood.

\hypertarget{fabien-baron}{%
\paragraph{FABIEN BARON}\label{fabien-baron}}

\hypertarget{creative-director-on-ninth-street-since-2014}{%
\subparagraph{\texorpdfstring{\textbf{Creative director, on Ninth Street
since
2014.}}{Creative director, on Ninth Street since 2014.}}\label{creative-director-on-ninth-street-since-2014}}

Right before we bought a building on Ninth Street, my wife and I had
been living on Bond Street in NoHo. We'd always had lofts and apartments
up until that point, so we were excited to be in a brownstone. Usually,
I go all the way in and make my place either all classical or all
modern. With this one, though, the exterior is completely classic while
inside it's very modern. We put in oak floors like you might find in
Paris, and a Louis XVI fireplace, which is not in tune with, and really
not respectful of the Village --- it's too elevated and European. Ninth
Street is very private, and I don't go out much or see anyone --- I was
surprised to learn just how creative my neighbors are. But I do like
that I can walk to work, and to Washington Square. You have everything
in this neighborhood --- if you want a lightbulb, you got it. You need a
grapefruit, you can find it. It's all right here.

\hypertarget{not-pictured}{%
\subsubsection{Not Pictured:}\label{not-pictured}}

\hypertarget{amy-sedaris}{%
\paragraph{AMY SEDARIS}\label{amy-sedaris}}

\hypertarget{actress-on-ninth-street-since-2007}{%
\subparagraph{\texorpdfstring{\textbf{Actress, on Ninth Street since
2007.}}{Actress, on Ninth Street since 2007.}}\label{actress-on-ninth-street-since-2007}}

For 18 years I'd been living in the area around Christopher and Bleecker
Streets, but then I moved here, where it just feels a little different.
The moment I saw my place, I knew that was it --- it was the sense that
this was a real neighborhood with Christmas decorations and everything.
I grew up in North Carolina, so community is important to me, and Ninth
Street proves you can have that even in New York City. It's the sort of
street where if you go for a walk (sometimes I make it all the way down
to \href{https://www.astorwines.com/}{Astor Wines and Spirits}, in the
De Vinne Press Building), you run into people you know. There are lots
of therapists on the block, too, so that's why you always see a lot of
celebrities on the street --- they're going to see their shrinks. There
are block parties, and the Knickerbocker on University and Ninth is a
really great place to go, especially on Saturday nights after 10.
There's great jazz, and you can always get a table. Over the years, I've
had plenty of big parties, but mostly I like to entertain smaller groups
of friends now. The people in my building get together every once in a
while; we have a dinner party and everybody brings something.

\hypertarget{helmut-lang}{%
\paragraph{HELMUT LANG}\label{helmut-lang}}

\hypertarget{artist-on-ninth-street-since-1996}{%
\subparagraph{\texorpdfstring{\textbf{Artist, on Ninth Street since
1996.}}{Artist, on Ninth Street since 1996.}}\label{artist-on-ninth-street-since-1996}}

I had been traveling back and forth between New York and Austria for
many years, but I bought my place on Ninth Street in the summer of 1996
and settled there for good at the end of 1997. Before, I'd always stayed
in hotels, but when I found my current spot, which is a rooftop
apartment with a big terrace, the living situation felt similar to what
I'd had in Europe. I wasn't aware at the time that Ninth Street has a
such a particular history --- my building is full of artists, writers
and creatives, but I had no idea this was true of the whole street. Over
the years, the neighborhood hasn't changed much optically, and the
relaxed feeling retains a ``New York meets Paris'' vibe. These days, I
run into my neighbors on the street or in the elevator more than working
a proper social schedule. Jenny Holzer used to live for a while on Ninth
Street, and we became very close friends, visiting each other regularly.

\hypertarget{maggie-betts}{%
\paragraph{MAGGIE BETTS}\label{maggie-betts}}

\hypertarget{filmmaker-on-ninth-street-since-2006}{%
\subparagraph{\texorpdfstring{\textbf{Filmmaker, on Ninth Street since
2006.}}{Filmmaker, on Ninth Street since 2006.}}\label{filmmaker-on-ninth-street-since-2006}}

My place on Ninth Street is the first place I ever bought, and I think
it might also be my last. It's a townhouse that's part of a pair of
matching buildings dating from the 1860s built by a father for his twin
daughters. My property hadn't been changed much at all, so over the
years I've been able to restore it, rather than renovate it. It has nine
original marble fireplaces, which is a massive amount. An antiques
dealer once told me if I'm ever facing financial ruin, I could sell them
off one by one. When I moved here, there was a super cool vibe in the
neighborhood. Lots of artists were around, like Ross Bleckner, and
Patricia Clarkson lived down the street. There were lots of chichi
restaurants back then, too, and while New York has changed over the
years, as you get older, you change, too. Though I still spend far too
much time at the Knickerbocker. I've also gotten really into landscaping
from May to October, when the front garden is filled with flowers and I
can create my own aesthetic. I would say I have no block envy for any
other block in the city.

\hypertarget{as-told-to-merrell-hambleton-and-samuel-rutter-quotes-have-been-edited-and-condensed}{%
\subparagraph{As told to Merrell Hambleton and Samuel Rutter. Quotes
have been edited and
condensed.}\label{as-told-to-merrell-hambleton-and-samuel-rutter-quotes-have-been-edited-and-condensed}}

\begin{center}\rule{0.5\linewidth}{\linethickness}\end{center}

\emph{\textbf{A behind-the-scenes account of the making of the group
portrait:}}

On a mild, bright morning at the end of January,
\href{http://www.casaapicii.com/}{Casa Apicii}, a restaurant on West
Ninth Street in downtown Manhattan, was open early for a neighborhood
reunion of sorts. One by one, and in pairs, people descended from street
level. As their eyes adjusted to the dim light, the guests warmed with
recognition, and a feeling of neighborly intimacy pervaded the space.
``Are the renovations going well?'' inquired one in between sips of
coffee. ``I was in Paris for a while,'' said another, explaining a
recent absence. ``I still have your pot!'' said a third, in reference to
a not-long-past dinner party.

The group, composed of Ninth Street residents gathering for a photo
shoot for T's Culture issue, was casually dressed. But though it might
not have been immediately apparent, its members are, in a way, the heirs
of the artistic sorts who lived and worked and gathered along this
stretch of Greenwich Village in eras past. As early as the mid-1800s,
the Hotel Griffou, a boarding house at 21 West Ninth Street, drew a
literary crowd --- Mark Twain and Oscar Wilde dined there. In the 1950s,
the Cedar Tavern, around the corner on University Place, was a haunt for
artists like Grace Hartigan, Joan Mitchell, Willem de Kooning and
Jackson Pollock, and thus an incubator of the Abstract Expressionist
movement. Decades later, the Italian restaurant Marylou's played host to
the debauchery of Jack Nicholson, Robert De Niro and actual gangsters.
And long before 62 West Ninth housed Casa Apicii, it was a gay bar where
a then unknown Barbra Streisand once won a weekly talent contest.

The street is still home to artists and writers, though they tend to be
already established talents, from the writer Michael Wolff to the
designer Jonathan Adler. Once everyone had arrived on this Friday
morning, it was difficult to usher the group, now 23 strong and knotted
together in ardent conversation, back out to the street for a
photograph. Some started calling out the years they arrived on Ninth
Street --- ``ninety-seven,'' ``ninety-two,'' ``eighty-seven'' ---
enthusiastic as alumni at a homecoming bonfire. As they stood together
on a south-facing stoop, the neighbors reflected on the magic of their
spot in the city. ``It had some sizzle, that old New York glamour,''
said the writer Simon Doonan, recalling his early impressions of Ninth
Street. ``It does seem to have an inordinate number of people involved
in the arts,'' said Jay McInerney, who moved to Ninth Street in 2001.
``I guess like attracts like.''

For Terrence McNally, the beloved playwright who died of coronavirus
complications last month, the pleasures of Ninth Street were simple. He
described a favorite photograph of his block, taken in spring, that he
kept as a screen saver on his computer. ``Our building fills the planter
boxes with tulip bulbs --- tulips are my favorite flower --- and for one
week each year, it's the most brilliant blaze of color,'' he said. ``You
can see someone walking their dog, a truck delivering a washing machine,
someone with a baby carriage. I don't know anyone in the picture. It's
just home to me. It's New York to me.'' --- MERRELL HAMBLETON

\href{https://www.nytimes.com/interactive/2019/04/10/t-magazine/terrence-mcnally-interview.html}{}

\hypertarget{a-conversation-with-terrence-mcnally-the-bard-of-american-theaterapril-10-2019}{%
\paragraph{A Conversation With Terrence McNally, the Bard of American
TheaterApril 10,
2019}\label{a-conversation-with-terrence-mcnally-the-bard-of-american-theaterapril-10-2019}}

\includegraphics{https://static01.nyt.com/images/2019/03/25/t-magazine/oakImage-1553548061806/oakImage-1553548061806-mediumThreeByTwo210.jpg}
\href{https://www.nytimes.com/2017/04/17/t-magazine/greenwich-village-artists-townhouse.html}{}

\hypertarget{the-new-york-building-inhabited-entirely-by-creative-peopleapril-17-2017}{%
\paragraph{The New York Building Inhabited Entirely by Creative
PeopleApril 17,
2017}\label{the-new-york-building-inhabited-entirely-by-creative-peopleapril-17-2017}}

\includegraphics{https://static01.nyt.com/images/2017/04/17/t-magazine/townhouse-slide-1QFP/townhouse-slide-1QFP-mediumThreeByTwo210-v3.jpg}
\href{https://www.nytimes.com/2019/11/25/t-magazine/three-lives-bookstore.html}{}

\hypertarget{in-greenwich-village-the-perfect-new-york-bookstore-lives-onnov-25-2019}{%
\paragraph{In Greenwich Village, the Perfect New York Bookstore Lives
OnNov. 25,
2019}\label{in-greenwich-village-the-perfect-new-york-bookstore-lives-onnov-25-2019}}

\includegraphics{https://static01.nyt.com/images/2019/11/25/t-magazine/art/35tmag-3lives-slide-FY4X/35tmag-3lives-slide-FY4X-mediumThreeByTwo210.jpg}

\hypertarget{we-are-family-1}{%
\subsubsection{We Are Family}\label{we-are-family-1}}

\hypertarget{chapter-1-heirs-and-alumni}{%
\paragraph{Chapter 1: Heirs and
Alumni}\label{chapter-1-heirs-and-alumni}}

\href{/interactive/2020/04/13/t-magazine/black-art-galleries.html}{}

\hypertarget{the-artists}{%
\subparagraph{The Artists}\label{the-artists}}

\href{/interactive/2020/04/13/t-magazine/italian-fashion-design-houses.html}{}

\hypertarget{the-dynasties}{%
\subparagraph{The Dynasties}\label{the-dynasties}}

\href{/interactive/2020/04/13/t-magazine/gordon-parks.html}{}

\hypertarget{the-directors}{%
\subparagraph{The Directors}\label{the-directors}}

\href{/interactive/2020/04/13/t-magazine/enrique-olvera-chef.html}{}

\hypertarget{the-disciples}{%
\subparagraph{The Disciples}\label{the-disciples}}

\href{/interactive/2020/04/13/t-magazine/royal-academy-antwerp.html}{}

\hypertarget{the-graduates}{%
\subparagraph{The Graduates}\label{the-graduates}}

\hypertarget{chapter-2-reunions-and-reconsiderations-1}{%
\paragraph{Chapter 2: Reunions and
Reconsiderations}\label{chapter-2-reunions-and-reconsiderations-1}}

\href{/interactive/2020/04/13/t-magazine/ninth-street-greenwich-village-neighbors.html}{}

\hypertarget{the-neighbors}{%
\subparagraph{The Neighbors}\label{the-neighbors}}

\href{/interactive/2020/04/13/t-magazine/omen-restaurant-nyc.html}{}

\hypertarget{the-regulars}{%
\subparagraph{The Regulars}\label{the-regulars}}

\href{/interactive/2020/04/13/t-magazine/hair-musical-broadway.html}{}

\hypertarget{hair-1967}{%
\subparagraph{Hair (1967)}\label{hair-1967}}

\href{/interactive/2020/04/13/t-magazine/sweeney-todd-revival.html}{}

\hypertarget{sweeney-todd-2005-revival}{%
\subparagraph{Sweeney Todd (2005
Revival)}\label{sweeney-todd-2005-revival}}

\href{/interactive/2020/04/13/t-magazine/daughters-of-the-dust.html}{}

\hypertarget{daughters-of-the-dust-1991}{%
\subparagraph{Daughters of the Dust
(1991)}\label{daughters-of-the-dust-1991}}

\hypertarget{chapter-3-legends-pioneers-and-survivors}{%
\paragraph{Chapter 3: Legends Pioneers and
Survivors}\label{chapter-3-legends-pioneers-and-survivors}}

\href{/interactive/2020/04/13/t-magazine/butch-stud-lesbian.html}{}

\hypertarget{the-renegades}{%
\subparagraph{The Renegades}\label{the-renegades}}

\href{/interactive/2020/04/13/t-magazine/act-up-aids.html}{}

\hypertarget{the-activists}{%
\subparagraph{The Activists}\label{the-activists}}

\href{/interactive/2020/04/13/t-magazine/artist-recluse.html}{}

\hypertarget{the-shadows}{%
\subparagraph{The Shadows}\label{the-shadows}}

\href{/interactive/2020/04/13/t-magazine/black-actresses-bassett-berry-blige-henson-whitfield-elise.html}{}

\hypertarget{the-veterans}{%
\subparagraph{The Veterans}\label{the-veterans}}

\hypertarget{chapter-4-the-new-guard}{%
\paragraph{Chapter 4: The New Guard}\label{chapter-4-the-new-guard}}

\href{/interactive/2020/04/13/t-magazine/asian-american-fashion-designers.html}{}

\hypertarget{the-designers}{%
\subparagraph{The Designers}\label{the-designers}}

\href{13tmag-beauties.html}{}

\hypertarget{the-beauties}{%
\subparagraph{The Beauties}\label{the-beauties}}

\href{/interactive/2020/04/13/t-magazine/nyc-downtown-nightlife-party-scene.html}{}

\hypertarget{the-scenemakers}{%
\subparagraph{The Scenemakers}\label{the-scenemakers}}

\href{/interactive/2020/04/13/t-magazine/maria-cornejo-olivier-rousteing-telfar-clemens-alessandro-michele.html\#olivier-rousteing-and-co}{}

\hypertarget{olivier-rousteing-and-co}{%
\subparagraph{Olivier Rousteing and
Co.}\label{olivier-rousteing-and-co}}

\href{/interactive/2020/04/13/t-magazine/maria-cornejo-olivier-rousteing-telfar-clemens-alessandro-michele.html\#maria-cornejo-and-co}{}

\hypertarget{maria-cornejo-and-co}{%
\subparagraph{Maria Cornejo and Co.}\label{maria-cornejo-and-co}}

\href{/interactive/2020/04/13/t-magazine/maria-cornejo-olivier-rousteing-telfar-clemens-alessandro-michele.html\#telfar-clemens-and-co}{}

\hypertarget{telfar-clemens-and-co}{%
\subparagraph{Telfar Clemens and Co.}\label{telfar-clemens-and-co}}

\href{/interactive/2020/04/13/t-magazine/maria-cornejo-olivier-rousteing-telfar-clemens-alessandro-michele.html\#alessandro-michele-and-co}{}

\hypertarget{alessandro-michele-and-co}{%
\subparagraph{Alessandro Michele and
Co.}\label{alessandro-michele-and-co}}

\href{/interactive/2020/04/13/t-magazine/foreign-correspondents.html}{}

\hypertarget{the-journalists}{%
\subparagraph{The Journalists}\label{the-journalists}}

\begin{itemize}
\item
\item
\item
\item
\end{itemize}

Advertisement

\protect\hyperlink{after-bottom}{Continue reading the main story}

\hypertarget{site-index}{%
\subsection{Site Index}\label{site-index}}

\hypertarget{site-information-navigation}{%
\subsection{Site Information
Navigation}\label{site-information-navigation}}

\begin{itemize}
\tightlist
\item
  \href{https://help.nytimes.com/hc/en-us/articles/115014792127-Copyright-notice}{©~2020~The
  New York Times Company}
\end{itemize}

\begin{itemize}
\tightlist
\item
  \href{https://www.nytco.com/}{NYTCo}
\item
  \href{https://help.nytimes.com/hc/en-us/articles/115015385887-Contact-Us}{Contact
  Us}
\item
  \href{https://www.nytco.com/careers/}{Work with us}
\item
  \href{https://nytmediakit.com/}{Advertise}
\item
  \href{http://www.tbrandstudio.com/}{T Brand Studio}
\item
  \href{https://www.nytimes.com/privacy/cookie-policy\#how-do-i-manage-trackers}{Your
  Ad Choices}
\item
  \href{https://www.nytimes.com/privacy}{Privacy}
\item
  \href{https://help.nytimes.com/hc/en-us/articles/115014893428-Terms-of-service}{Terms
  of Service}
\item
  \href{https://help.nytimes.com/hc/en-us/articles/115014893968-Terms-of-sale}{Terms
  of Sale}
\item
  \href{https://spiderbites.nytimes.com}{Site Map}
\item
  \href{https://help.nytimes.com/hc/en-us}{Help}
\item
  \href{https://www.nytimes.com/subscription?campaignId=37WXW}{Subscriptions}
\end{itemize}
