\hypertarget{how-antwerps-royal-academy-of-fine-arts-birthed-the-contemporary-avant-garde}{%
\section{How Antwerp's Royal Academy of Fine Arts Birthed the
Contemporary
Avant-Garde}\label{how-antwerps-royal-academy-of-fine-arts-birthed-the-contemporary-avant-garde}}

April 13, 2020

\begin{itemize}
\item
\item
\item
\item
\end{itemize}

A city poised on the edge of Europe and the rest of the world became the
incubator for talents like Dries Van Noten, Luc Tuymans and Ann
Demeulemeester.

\href{https://www.nytimes.com/interactive/2020/04/13/t-magazine/culture-issue-2020.html}{We
Are Family}

\hypertarget{chapter-1-heirs-and-alumni}{%
\subparagraph{Chapter 1: Heirs and
Alumni}\label{chapter-1-heirs-and-alumni}}

\hypertarget{previous}{%
\subparagraph{Previous}\label{previous}}

\hypertarget{next}{%
\subparagraph{Next}\label{next}}

\hypertarget{how-antwerps-royal-academy-of-fine-arts-birthed-the-contemporary-avant-garde-1}{%
\section{How Antwerp's Royal Academy of Fine Arts Birthed the
Contemporary
Avant-Garde}\label{how-antwerps-royal-academy-of-fine-arts-birthed-the-contemporary-avant-garde-1}}

\hypertarget{the-graduates}{%
\subsection{The Graduates}\label{the-graduates}}

How the Royal Academy of Fine Arts in Antwerp --- a city poised on the
edge of Europe and the rest of the world --- became the incubator for
the contemporary avant-garde.

By \href{https://www.nytimes.com/by/alice-newell-hanson}{Alice
Newell-Hanson}

April 13, 2020

SHARE

KRIS VAN ASSCHEFashion designer, class of '98

ANNE-MIE VAN KERCKHOVENArtist, class of '74

DRIES VAN NOTENFashion designer, class of '80

KATRIN WOUTERSJewelry designer, class of '83

PETER PILOTTOFashion designer, class of '04

KATI HECKArtist, class of '03

MARINA YEEFashion designer, class of '80

PETER PHILIPSMakeup artist, class of '93

BEN SLEDSENSArtist, class of '15

LUC TUYMANSArtist, '78-79

HAIDER ACKERMANNFashion designer, '94-97

WALTER VAN BEIRENDONCKFashion designer and teacher, class of '80

DIRK VAN SAENEFashion designer and teacher, class of '81

JAN-JAN VAN ESSCHEFashion designer, class of '03

KAREN HENDRIXJewelry designer, class of '82

FILIP ARICKXFashion designer, class of '91

PATRICK VAN OMMESLAEGHEFashion designer, class of '90

AN VANDEVORSTFashion designer, class of '91

ANN DEMEULEMEESTERFashion designer, class of '81

Front row, from left: the fashion designer \textbf{KRIS VAN ASSCHE}
(class of '98), the artist \textbf{ANNE-MIE VAN KERCKHOVEN} ('74), the
fashion designer \textbf{DRIES VAN NOTEN} ('80), the jewelry designer
\textbf{KATRIN WOUTERS} ('83) and the fashion designer \textbf{PETER
PILOTTO} ('04). Back row: the artist \textbf{KATI HECK} ('03), the
fashion designer \textbf{MARINA YEE} ('80), the makeup artist
\textbf{PETER PHILIPS} ('93), the artist \textbf{BEN SLEDSENS} ('15),
the artist \textbf{LUC TUYMANS} ('78-79), the fashion designer
\textbf{HAIDER ACKERMANN} ('94-97), the fashion designer and teacher
\textbf{WALTER VAN BEIRENDONCK} ('80), the fashion designer and teacher
\textbf{DIRK VAN SAENE} ('81), the fashion designer \textbf{JAN-JAN VAN
ESSCHE} ('03), the jewelry designer \textbf{KAREN HENDRIX} ('82), the
fashion designer \textbf{FILIP ARICKX} ('91), the fashion designer
\textbf{PATRICK VAN OMMESLAEGHE} ('90), the fashion designer \textbf{AN
VANDEVORST} ('91) and the fashion designer \textbf{ANN DEMEULEMEESTER}
('81). Photographed at the Royal Academy of Fine Arts in Antwerp,
Belgium, on Dec. 13, 2019. Pascal Gambarte

THE CITY OF ANTWERP in northern Belgium is, somewhat confoundingly, both
Europe's second busiest seaport and located 50 miles inland from the
nearest sea. For much of the past half millennium, its character has
been defined by this geographical quirk. Sheltered in a nook of the
Scheldt River estuary, it is at once quintessentially European --- a
city of medieval cobblestone streets and Gothic Flemish Renaissance
buildings --- and directly connected to the globe-spanning shipping
routes entered by the North Sea. Accordingly, it has been one of the
continent's essential trade hubs as well as a prolific incubator of the
avant-garde.

These two identities developed in tandem in the early 1500s, when
Antwerp became, briefly, one of the centers of the West's rapidly
expanding known world. As Dutch, Spanish and Portuguese merchants began
traveling to and from newly encountered territories during the age of
exploration, up to 40 percent of the globe's trade passed through
Antwerp's docks, and along with silk from Turkey, peppercorns and
diamonds from Africa and silver from America came immigrants and new
ideas. ``We always know in Antwerp news of everything that goes through
the rest of the provinces of the universe,'' the Italian merchant
\href{https://www.oxfordartonline.com/groveart/view/10.1093/gao/9781884446054.001.0001/oao-9781884446054-e-7000035507}{Lodovico
Guicciardini} boasted in 1567. Though this golden age was short lived
(the Dutch blockaded the Scheldt from 1585, when Antwerp surrendered to
Spain during the Eighty Years' War, until 1795), it forged the city's
rich and enduring cultural life: The economic boom created a market for
art, allowing for printing shops, coffee houses and guilds to form. In
1663,
\href{https://www.nationalgallery.org.uk/artists/david-teniers-the-younger}{David
Teniers the Younger}, the master of the Guild of Saint Luke, an
association that had been vital to Antwerp's artistic development during
the 16th century, founded the
\href{https://www.ap.be/en/royal-academy-fine-arts-antwerp}{Royal
Academy of Fine Arts}, a school whose alumni, ranging from
\href{https://www.nytimes.com/topic/person/vincent-van-gogh}{Vincent van
Gogh} (1885-86) to the fashion designer
\href{https://www.nytimes.com/2016/04/11/t-magazine/gucci-alessandro-michele-balenciaga-vetements-demna-gvasalia.html}{Demna
Gvasalia} (2002-06), have continued to shape the world's culture.

The T List \textbar{}

Sign up here

ONCE EVERY FEW generations, there is a college or a workshop, a theater
company or a lab that produces a seemingly disproportionate number of
influential graduates: In the 1940s, the artists
\href{https://www.nytimes.com/topic/person/alex-katz}{Alex Katz},
\href{https://www.nytimes.com/topic/person/roy-decarava}{Roy DeCarava}
and Lois Dodd passed through
\href{https://www.nytimes.com/topic/organization/cooper-union-for-the-advancement-of-science-and-art}{Cooper
Union} in New York; in the mid 1990s, Central Saint Martins in London
nurtured
\href{https://www.nytimes.com/2014/02/14/t-magazine/phoebe-philo-celine-prophetic-fashion.html}{Phoebe
Philo}, Stella McCartney and Hussein Chalayan. What is less recognized
is the influence of the Royal Academy in the early '80s, when its
students made work that would transform the worlds of fashion, art and
theater.

Even now, despite its ongoing economic might (to this day, the city is
wealthy from its port and the diamond trade), Antwerp ``remains a sort
of village,'' says the Belgian artist
\href{https://www.nytimes.com/2019/03/21/arts/design/luc-tuymans-palazzo-grassi-mosaic-la-pelle.html}{Luc
Tuymans}, who studied painting at the Royal Academy from 1978 to 1979
and still lives in the city. Perhaps because its growth abruptly ended
in 1585, Antwerp never became a metropolis like London or Paris. ``There
is an atmosphere here of a rather small city in which a lot of people
are working together,'' says the fashion designer
\href{https://www.nytimes.com/2017/10/16/t-magazine/dries-van-noten.html}{Dries
Van Noten}. This collaborative spirit, as well as the competition
inherent in a close-knit community, is partly what enabled the designers
known as the Antwerp Six --- Van Noten,
\href{https://www.nytimes.com/2006/08/27/style/tmagazine/27ann.html}{Ann
Demeulemeester}, Dirk van Saene,
\href{https://www.nytimes.com/2019/09/03/t-magazine/walter-van-beirendonck.html}{Walter
Van Beirendonck}, Dirk Bikkembergs and Marina Yee --- along with fellow
alumni
\href{https://www.nytimes.com/2018/03/08/t-magazine/fashion/martin-margiela-history-fall-winter-2000-show.html}{Martin
Margiela}, to transform Antwerp into a capital of intellectual,
unconventional fashion. While the six designers were fiercely
individualistic --- ``We each wanted to be different from and better
than the others,'' recalls Demeulemeester --- they were also friends who
exchanged ideas from their own spheres of interest and from nearby
cities. Antwerp in the early '80s was, as in the city's 16th-century
heyday, a breeding ground for creativity: It was close enough to the
exploding youth subcultures of Berlin and London to feel their echoes
but far enough away that those influences could ferment into new forms.
``We look at everything with a certain distance,'' says Van Noten of his
city's residents. ``Not a very far distance, but just one step back. In
that way, we can rethink certain things and do things our own way.''

Since its belle epoque, Antwerp's mercantile and artistic communities
have converged in its social life. In the '80s, the city's abundant
ware­houses helped cultivate a thriving underground scene.
Demeulemeester recalls a favorite nightclub called Cinderella where
``people from the academy, from the harbor and punks'' danced together.
Tuymans remembers seeing Belgium's first punk group, the Kids, perform
in Antwerp not long after they formed there in the late '70s. Within
this milieu, the pioneering theater director
\href{https://www.nytimes.com/2018/11/20/t-magazine/rufus-wainwright-ivo-van-hove-conversation.html}{Ivo
van Hove}, who was then a student in the city, met his future partner
and collaborator, the set designer Jan Versweyveld, who was enrolled at
the Royal Academy at the time; they later opened a brasserie to fund
their experimental joint productions, reinforcing Antwerp's historic
marriage of art and commerce, and making use of the city's array of
affordable, nontraditional venues (in 1981, they staged their first
play, ``Rumours,'' in a disused laundromat). And it was in Antwerp's
cafes at the start of the next decade that the designer
\href{https://www.nytimes.com/2016/03/06/t-magazine/raf-simons-interview.html}{Raf
Simons} met many of his lifelong collaborators, including the
photographer
\href{https://www.nytimes.com/2020/02/20/t-magazine/spring-fashion-cover.html}{Willy
Vanderperre} and the stylist
\href{https://www.nytimes.com/2019/08/05/t-magazine/fall-fashion-androgyny.html}{Olivier
Rizzo}. Today, the Antwerp-based artist
\href{https://www.saatchigallery.com/artists/kati_heck.htm}{Kati Heck}
captures the vibrancy of the city's social world in her large-scale
paintings, such as ``Trinklied vom Jammer der Erde'' (``The Drinking
Song of the Earth's Sorrow,'' 2017), in which a group of her friends,
including the genre-defying cellist Simon Lenski, are depicted drinking
in their real-life local haunt, the Cafe De Kat.

``There's something free and artistic about the life here, if one looks
for it, perhaps more than anywhere else,'' van Gogh wrote to his brother
Theo from Antwerp in 1886: ``There's gusto and people enjoy
themselves.'' For a city whose culture has depended on trade since its
inception, on an openness to whatever innovation comes through its
docks, the burden of tradition is less stifling in Antwerp than it might
be for young artists in less changeable cities. While there is an
artistic legacy ---
\href{https://www.nytimes.com/2018/11/23/arts/design/bruegel-kunsthistorisches-museum-technology-layers.html}{Pieter
Bruegel} and Peter Paul Rubens both made some of their most important
works here --- as well as the legend of the Six to contend with, there
is also a desire for knowledge that is outward-facing rather than
self-contented. Indeed, Van Beirendonck, who has overseen the Royal
Academy's fashion department since 2007, says the most important quality
an applicant can possess is that they are, above all, ``culturally
hungry.''

Alice Newell-Hanson is the senior digital features editor of T Magazine.
Pascal Gambarte specializes in editorial fashion work. Hair: Louis Ghewy
at Management \& Artists Group. Makeup: Florence Teerlinck. Production:
Mindbox. Photo assistants: Joe Reddy and Samir Dari. Hair assistant:
Marlien Echelpoels. Makeup assistants: Marie Corbeel and Evelien de
Keukeleire.

\href{https://www.nytimes.com/2019/09/03/t-magazine/walter-van-beirendonck.html}{}

\hypertarget{the-strange-and-beautiful-universe-of-walter-van-beirendoncksept-3-2019}{%
\paragraph{The Strange and Beautiful Universe of Walter Van
BeirendonckSept. 3,
2019}\label{the-strange-and-beautiful-universe-of-walter-van-beirendoncksept-3-2019}}

\includegraphics{https://static01.nyt.com/images/2019/09/03/t-magazine/03tmag-beirendonck-slide-WJNK-copy/03tmag-beirendonck-slide-WJNK-copy-mediumThreeByTwo210-v2.jpg}
\href{https://www.nytimes.com/2017/10/16/t-magazine/dries-van-noten.html}{}

\hypertarget{dries-van-noten-icon-of-creative-freedomoct-16-2017}{%
\paragraph{Dries Van Noten, Icon of Creative FreedomOct. 16,
2017}\label{dries-van-noten-icon-of-creative-freedomoct-16-2017}}

\includegraphics{https://static01.nyt.com/images/2017/10/16/t-magazine/dries-van-noten-slide-RG7Z-copy/dries-van-noten-slide-RG7Z-copy-mediumThreeByTwo210-v2.jpg}
\href{https://www.nytimes.com/2018/09/11/t-magazine/charlotte-de-geyter-designer-bernadette-ben-sledsens-art.html}{}

\hypertarget{the-dress-in-his-painting-she-designed-itsept-11-2018}{%
\paragraph{The Dress in His Painting? She Designed ItSept. 11,
2018}\label{the-dress-in-his-painting-she-designed-itsept-11-2018}}

\includegraphics{https://static01.nyt.com/images/2018/09/06/t-magazine/art/Antwerp-couple-slide-R0ZO/Antwerp-couple-slide-R0ZO-mediumThreeByTwo210.jpg}
\href{https://www.nytimes.com/2019/11/29/t-magazine/kris-van-assche-berluti.html}{}

\hypertarget{the-mens-wear-designer-keeping-a-heritage-brand-coolnov-29-2019}{%
\paragraph{The Men's Wear Designer Keeping a Heritage Brand CoolNov. 29,
2019}\label{the-mens-wear-designer-keeping-a-heritage-brand-coolnov-29-2019}}

\includegraphics{https://static01.nyt.com/images/2019/12/08/t-magazine/08tmag-profile-slide-D13V/08tmag-profile-slide-D13V-mediumThreeByTwo210.jpg}

\hypertarget{correction-april-17-2020}{%
\subparagraph{\texorpdfstring{\textbf{Correction} April 17,
2020}{Correction April 17, 2020}}\label{correction-april-17-2020}}

An earlier version of this article misstated the years that Luc Tuymans
attended the Royal Academy; he studied there from 1978 to 1979, not 1980
to 1982.

\hypertarget{we-are-family-1}{%
\subsubsection{We Are Family}\label{we-are-family-1}}

\hypertarget{chapter-1-heirs-and-alumni-1}{%
\paragraph{Chapter 1: Heirs and
Alumni}\label{chapter-1-heirs-and-alumni-1}}

\href{/interactive/2020/04/13/t-magazine/black-art-galleries.html}{}

\hypertarget{the-artists}{%
\subparagraph{The Artists}\label{the-artists}}

\href{/interactive/2020/04/13/t-magazine/italian-fashion-design-houses.html}{}

\hypertarget{the-dynasties}{%
\subparagraph{The Dynasties}\label{the-dynasties}}

\href{/interactive/2020/04/13/t-magazine/gordon-parks.html}{}

\hypertarget{the-directors}{%
\subparagraph{The Directors}\label{the-directors}}

\href{/interactive/2020/04/13/t-magazine/enrique-olvera-chef.html}{}

\hypertarget{the-disciples}{%
\subparagraph{The Disciples}\label{the-disciples}}

\href{/interactive/2020/04/13/t-magazine/royal-academy-antwerp.html}{}

\hypertarget{the-graduates-1}{%
\subparagraph{The Graduates}\label{the-graduates-1}}

\hypertarget{chapter-2-reunions-and-reconsiderations}{%
\paragraph{Chapter 2: Reunions and
Reconsiderations}\label{chapter-2-reunions-and-reconsiderations}}

\href{/interactive/2020/04/13/t-magazine/ninth-street-greenwich-village-neighbors.html}{}

\hypertarget{the-neighbors}{%
\subparagraph{The Neighbors}\label{the-neighbors}}

\href{/interactive/2020/04/13/t-magazine/omen-restaurant-nyc.html}{}

\hypertarget{the-regulars}{%
\subparagraph{The Regulars}\label{the-regulars}}

\href{/interactive/2020/04/13/t-magazine/hair-musical-broadway.html}{}

\hypertarget{hair-1967}{%
\subparagraph{Hair (1967)}\label{hair-1967}}

\href{/interactive/2020/04/13/t-magazine/sweeney-todd-revival.html}{}

\hypertarget{sweeney-todd-2005-revival}{%
\subparagraph{Sweeney Todd (2005
Revival)}\label{sweeney-todd-2005-revival}}

\href{/interactive/2020/04/13/t-magazine/daughters-of-the-dust.html}{}

\hypertarget{daughters-of-the-dust-1991}{%
\subparagraph{Daughters of the Dust
(1991)}\label{daughters-of-the-dust-1991}}

\hypertarget{chapter-3-legends-pioneers-and-survivors}{%
\paragraph{Chapter 3: Legends Pioneers and
Survivors}\label{chapter-3-legends-pioneers-and-survivors}}

\href{/interactive/2020/04/13/t-magazine/butch-stud-lesbian.html}{}

\hypertarget{the-renegades}{%
\subparagraph{The Renegades}\label{the-renegades}}

\href{/interactive/2020/04/13/t-magazine/act-up-aids.html}{}

\hypertarget{the-activists}{%
\subparagraph{The Activists}\label{the-activists}}

\href{/interactive/2020/04/13/t-magazine/artist-recluse.html}{}

\hypertarget{the-shadows}{%
\subparagraph{The Shadows}\label{the-shadows}}

\href{/interactive/2020/04/13/t-magazine/black-actresses-bassett-berry-blige-henson-whitfield-elise.html}{}

\hypertarget{the-veterans}{%
\subparagraph{The Veterans}\label{the-veterans}}

\hypertarget{chapter-4-the-new-guard}{%
\paragraph{Chapter 4: The New Guard}\label{chapter-4-the-new-guard}}

\href{/interactive/2020/04/13/t-magazine/asian-american-fashion-designers.html}{}

\hypertarget{the-designers}{%
\subparagraph{The Designers}\label{the-designers}}

\href{13tmag-beauties.html}{}

\hypertarget{the-beauties}{%
\subparagraph{The Beauties}\label{the-beauties}}

\href{/interactive/2020/04/13/t-magazine/nyc-downtown-nightlife-party-scene.html}{}

\hypertarget{the-scenemakers}{%
\subparagraph{The Scenemakers}\label{the-scenemakers}}

\href{/interactive/2020/04/13/t-magazine/maria-cornejo-olivier-rousteing-telfar-clemens-alessandro-michele.html\#olivier-rousteing-and-co}{}

\hypertarget{olivier-rousteing-and-co}{%
\subparagraph{Olivier Rousteing and
Co.}\label{olivier-rousteing-and-co}}

\href{/interactive/2020/04/13/t-magazine/maria-cornejo-olivier-rousteing-telfar-clemens-alessandro-michele.html\#maria-cornejo-and-co}{}

\hypertarget{maria-cornejo-and-co}{%
\subparagraph{Maria Cornejo and Co.}\label{maria-cornejo-and-co}}

\href{/interactive/2020/04/13/t-magazine/maria-cornejo-olivier-rousteing-telfar-clemens-alessandro-michele.html\#telfar-clemens-and-co}{}

\hypertarget{telfar-clemens-and-co}{%
\subparagraph{Telfar Clemens and Co.}\label{telfar-clemens-and-co}}

\href{/interactive/2020/04/13/t-magazine/maria-cornejo-olivier-rousteing-telfar-clemens-alessandro-michele.html\#alessandro-michele-and-co}{}

\hypertarget{alessandro-michele-and-co}{%
\subparagraph{Alessandro Michele and
Co.}\label{alessandro-michele-and-co}}

\href{/interactive/2020/04/13/t-magazine/foreign-correspondents.html}{}

\hypertarget{the-journalists}{%
\subparagraph{The Journalists}\label{the-journalists}}

\begin{itemize}
\item
\item
\item
\item
\end{itemize}

Advertisement

\protect\hyperlink{after-bottom}{Continue reading the main story}

\hypertarget{site-index}{%
\subsection{Site Index}\label{site-index}}

\hypertarget{site-information-navigation}{%
\subsection{Site Information
Navigation}\label{site-information-navigation}}

\begin{itemize}
\tightlist
\item
  \href{https://help.nytimes.com/hc/en-us/articles/115014792127-Copyright-notice}{©~2020~The
  New York Times Company}
\end{itemize}

\begin{itemize}
\tightlist
\item
  \href{https://www.nytco.com/}{NYTCo}
\item
  \href{https://help.nytimes.com/hc/en-us/articles/115015385887-Contact-Us}{Contact
  Us}
\item
  \href{https://www.nytco.com/careers/}{Work with us}
\item
  \href{https://nytmediakit.com/}{Advertise}
\item
  \href{http://www.tbrandstudio.com/}{T Brand Studio}
\item
  \href{https://www.nytimes.com/privacy/cookie-policy\#how-do-i-manage-trackers}{Your
  Ad Choices}
\item
  \href{https://www.nytimes.com/privacy}{Privacy}
\item
  \href{https://help.nytimes.com/hc/en-us/articles/115014893428-Terms-of-service}{Terms
  of Service}
\item
  \href{https://help.nytimes.com/hc/en-us/articles/115014893968-Terms-of-sale}{Terms
  of Sale}
\item
  \href{https://spiderbites.nytimes.com}{Site Map}
\item
  \href{https://help.nytimes.com/hc/en-us}{Help}
\item
  \href{https://www.nytimes.com/subscription?campaignId=37WXW}{Subscriptions}
\end{itemize}
