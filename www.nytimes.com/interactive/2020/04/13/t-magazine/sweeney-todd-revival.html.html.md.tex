\hypertarget{how-the-2005-revival-of-sweeney-todd-inspired-a-new-wave-in-theater}{%
\section{How the 2005 Revival of `Sweeney Todd' Inspired a New Wave in
Theater}\label{how-the-2005-revival-of-sweeney-todd-inspired-a-new-wave-in-theater}}

April 13, 2020

\begin{itemize}
\item
\item
\item
\item
\end{itemize}

A minimalist staging by John Doyle of the tale of the barber of Fleet
Street emphasized the raw talents of its cast.

\href{https://www.nytimes.com/interactive/2020/04/13/t-magazine/culture-issue-2020.html}{We
Are Family}

\hypertarget{chapter-2-reunions-and-reconsiderations}{%
\subparagraph{Chapter 2: Reunions and
Reconsiderations}\label{chapter-2-reunions-and-reconsiderations}}

\hypertarget{previous}{%
\subparagraph{Previous}\label{previous}}

\hypertarget{next}{%
\subparagraph{Next}\label{next}}

\textbf{The forerunners}

\hypertarget{how-the-2005-revival-of-sweeney-todd-inspired-a-new-wave-in-theater-1}{%
\section{How the 2005 Revival of `Sweeney Todd' Inspired a New Wave in
Theater}\label{how-the-2005-revival-of-sweeney-todd-inspired-a-new-wave-in-theater-1}}

Every now and then, a piece of American performance is so memorable that
it both redefines its medium and reframes the culture at large. Here, an
appraisal of one such enduring and heavily referenced work --- a 2005
Broadway revival that brought a much-needed austerity and economy to the
stage --- alongside a gathering of the stars who not only made it but
were made by it, too.

By \href{https://www.nytimes.com/by/patricia-cohen}{Patricia Cohen}

April 13, 2020

SHARE

FOR MUCH OF the 20th century, theatrical lore was built upon flashy
pyrotechnics that elicited stunned gasps from audience members: In the
'80s, a hydraulically powered tire airlifted a feline in
``\href{https://www.nytimes.com/2016/07/24/theater/just-say-cats-and-watch-the-fur-fly.html}{Cats},''
while a 1,500-pound crystal chandelier was crashed onto the stage in
``\href{https://www.nytimes.com/2012/02/11/theater/phantom-of-the-opera-reaches-10000th-broadway-performance.html}{The
Phantom of the Opera}.'' In the '90s, a whirring rotary helicopter
ascended to the rafters in ``Miss Saigon.''

Broadway still loves extravaganzas, of course, and often provides a home
for even ill-fated ones. But over the past decade or so, it has leaned
in the opposite direction --- toward streamlined productions that forgo
razzle-dazzle to instead focus on a small ensemble cast, who not only
recite the lines and sing the score but frequently play the instruments,
too.

The T List \textbar{}

Sign up here

This shift can be traced to the revelatory 2005 Broadway revival of
\href{https://www.nytimes.com/topic/person/stephen-sondheim}{Stephen
Sondheim} and
\href{https://www.nytimes.com/1987/07/28/obituaries/hugh-wheeler-award-winning-playwright.html}{Hugh
Wheeler}'s 1979 masterwork,
``\href{https://www.nytimes.com/2005/12/11/theater/sweeney-todds-game-of-musical-chairs.html}{Sweeney
Todd: The Demon Barber of Fleet Street}'' --- a Grand Guignol tale about
a murderous barber who bakes his customers into meat pies. Directed by
John Doyle and starring
\href{https://www.nytimes.com/2016/05/29/realestate/michael-cerveriss-fun-home-in-chelsea.html}{Michael
Cerveris} as the title character and
\href{https://www.nytimes.com/topic/person/patti-lupone}{Patti LuPone}
as his blood-hungry accomplice, Mrs. Lovett, that show not only incited
a new era of theatrical minimalism but encouraged a new generation of
composers, directors and musicians to experiment on Broadway. ``It
opened the door for other shows,'' says Doyle, who has recently been
\href{https://www.nytimes.com/2019/03/20/theater/john-doyle-assassins.html}{at
work on} another Sondheim revival, 1990's ``Assassins,'' at New York's
Classic Stage Company, where he is the artistic director. ``Sweeney'' is
one of several Sondheim shows he has reinterpreted over the years --- a
production, he says, that ``allowed for the fact that musicals could be
something other than spectacle,'' where the actors, using the full range
of their skills, could ``capture the essence of storytelling.'' In this
case, every performer also served as an orchestra member, playing
Sondheim's complex score while simultaneously portraying a character.
Donna Lynne Champlin, as the competing barber, Pirelli, learned
accordion for the show, and several cast members took turns on the
center-stage piano. LuPone, who played the tuba in an all-girls marching
band while in high school on Long Island in the 1960s, pumped out each
horn blast as if it were steam escaping a pie. Cerveris strummed the
guitar, an instrument he was introduced to in grade school in West
Virginia, and says the experience was ``an opportunity to be a whole
artist for the first time,'' adding that it was, in fact, a return to
theater as it was practiced in Shakespearean times: ``Doyle engages the
audience in a way that harks back to childhood play.''

Economic necessity helped inspire the show's asceticism, which
characterizes much of Doyle's work. The 67-year-old Scottish director
spent much of his career in the United Kingdom at regional theaters,
where penny-pinching is as much of an art as speaking in iambic
pentameter; he had initially modernized ``Sweeney Todd'' in 2004 for the
Watermill Theater, a 220-seat house in the English countryside. ``I was
trying to find the least expensive way of doing this enormous piece with
only nine or 10 people,'' he says. The handful of actors played all the
characters and instruments on a mostly empty stage with a single coffin
to serve as the bow of a ship, a table for rolling out pie crusts or the
entrance to a madhouse. (By contrast, the director
\href{https://www.nytimes.com/2019/07/31/theater/hal-prince-dead.html}{Harold
Prince}'s original Broadway production had a 27-piece orchestra and a
27-person cast.) When Doyle learned the producers wanted to transfer his
revival to Broadway after it opened on London's West End later that
year, ``I really, truly thought they had lost their minds,'' he recalls.

Yet the minimalism of Doyle's production accentuated the play's
subtleties. Sweeney's brutish world is one in which everyone must make
do with what's on hand, whether it's a barber's knife to slit a throat
or a corpse to fill a meat pie. ``Well, waste not, want not, as I always
say,'' Mrs. Lovett remarks when she realizes the slumped body in
Sweeney's barber's chair could be converted into the next day's supper.
Such economy likewise governed the production. As Anthony and Johanna,
the actors \href{https://www.imdb.com/name/nm3735219/}{Benjamin
Magnuson} and \href{http://laurenmolina.com/}{Lauren Molina} not only
sang the romantic ballad ``Johanna'' but accompanied themselves on
cellos. Two arms and two bows swept across the strings, their movements
in tandem. Later, the lovers' cello movements transformed into violent
slashing that mirrored Sweeney's razor strokes, and later still, into
the self-flagellating whip cracks administered by Judge Turpin
(\href{https://www.imdb.com/name/nm0415002/}{Mark Jacoby}), the morally
depraved engineer of Sweeney's downfall. Nothing's wasted, and the
effect is spellbinding. Of Doyle's production, Sondheim told a
journalist at the time, ``John's, for me, is the most intense.''

AN EXUBERANT REVIVAL of ``The Pajama Game''
\href{https://www.nytimes.com/2006/06/12/theater/theaterspecial/12tony.html}{won
the Tony} that year. But Doyle's critical and commercial success had
lasting influence: It helped persuade audiences and producers that
artistic innovation was just as commanding as lavish scenery and
effects, and cleared the way for shows in the same vein, many of them
new works developed Off Broadway, where scarcity regularly prompts
reinvention. The spartan productions that followed not only redefined
what a Broadway show could be but also became symbolic of the challenges
and triumphs of making art in a New York chastened by the 2008 financial
crisis. Money was tight, funders were hard to secure. And as fortunes
and jobs melted away, glitzy opulence, sleight-of-hand stunts and ornate
effects began to seem like flimflammery, out of step with the changing
times. On Broadway, like everywhere else after the global recession,
people had to make do. As Mrs. Lovett sings, ``Business needs a lift
\ldots{} Think of it as thrift.''

Doyle's ``Sweeney'' also proved the enduring power of austerity. In
2012, ``Once,'' a modest Broadway production about two songwriters in a
bar that featured instrument-playing actors,
\href{https://www.nytimes.com/2012/06/11/theater/theaterspecial/musical-once-receives-8-tony-awards.html}{won
the Tony} for best musical. More soon followed: the Josh Groban-starring
experimental ensemble production
``\href{https://www.nytimes.com/2017/05/02/theater/tony-awards-nominations.html}{Natasha,
Pierre and the Great Comet of 1812}'' --- a musicalized Broadway version
of Leo Tolstoy's ``War and Peace'' --- in 2016, and, the following year,
``\href{https://www.nytimes.com/2018/06/10/theater/tony-awards-live.html}{The
Band's Visit},'' about a group of Egyptian musicians stuck overnight in
an Israeli village. Last year,
``\href{https://www.nytimes.com/2019/03/27/theater/rachel-chavkin-hadestown-great-comet.html}{Hadestown},''
a retelling of the Orpheus and Eurydice myth that featured a trio of
Fates playing the accordion, tambourine and fiddle,
\href{https://www.nytimes.com/2019/06/09/theater/tony-awards.html\#link-21590072}{won
the best musical Tony}. Best musical revival went to
``\href{https://www.nytimes.com/2018/12/11/theater/oklahoma-broadway-musical-revival.html}{Oklahoma!,}''
which replaced some of Rodgers and Hammerstein's lushest orchestrations
with a single guitar-strumming cowboy, played by
\href{https://www.nytimes.com/2019/02/22/theater/hadestown-oklahoma-amber-gray-damon-daunno.html}{Damon
Daunno}. That production eliminated the show's standard overture,
chorus, elaborate costumes and arrangements to expose the show's
\href{https://www.nytimes.com/2020/01/20/theater/patrick-vaill-oklahoma-broadway.html}{darker
undercurrents} and sexuality, transforming what could be a stodgy bit of
Americana into a vital story for our times.

Doyle, for his part, often hears from people who tell him his ``Sweeney
Todd'' was their first Broadway experience. ``It introduced a new
generation to something they'd never seen,'' he says. Transformation,
after all, is what drives theater, and these pared-down productions
encouraged creators with different stories to tell and different ways of
telling them to share their vision on a larger platform. That, in turn,
has inspired new, more diverse audiences, as well as Broadway veterans.
The cycle continues today, challenging the conceptions not only of what
theater can be but \emph{should} be. ``It doesn't have to be big to
affect an audience, it just has to be exciting,'' LuPone says. ``And, in
our case, very scary.''

Patricia Cohen is a domestic correspondent for The New York Times,
covering the national economy. She was the theater editor from 2004 to
2008. Jennifer Livingston specializes in editorial fashion photography.
Photo assistants: Mike O'Shea and Matt Labarbiera. Digital tech: Matthew
Willkens.

\href{https://www.nytimes.com/2017/10/16/t-magazine/lin-manuel-miranda-stephen-sondheim.html}{}

\hypertarget{stephen-sondheim-theaters-greatest-lyricistoct-16-2017}{%
\paragraph{Stephen Sondheim, Theater's Greatest LyricistOct. 16,
2017}\label{stephen-sondheim-theaters-greatest-lyricistoct-16-2017}}

\includegraphics{https://static01.nyt.com/images/2017/10/16/t-magazine/sondheim-slide-SB33-copy/sondheim-slide-SB33-copy-mediumThreeByTwo210-v2.jpg}
\href{https://www.nytimes.com/2018/08/09/t-magazine/allison-janney-idina-menzel-female-actresses.html}{}

\hypertarget{these-actresses-dominate-not-just-the-stage-but-the-screen-tooaug-9-2018}{%
\paragraph{These Actresses Dominate Not Just the Stage, but the Screen
TooAug. 9,
2018}\label{these-actresses-dominate-not-just-the-stage-but-the-screen-tooaug-9-2018}}

\includegraphics{https://static01.nyt.com/images/2018/08/09/t-magazine/09tmag-women/09tmag-women-mediumThreeByTwo210.jpg}
\href{https://www.nytimes.com/2018/04/16/t-magazine/broadway-1980s-actors-sarah-jessica-parker-willem-dafoe.html}{}

\hypertarget{the-stars-who-got-their-start-on-the-80s-new-york-stageapril-16-2018}{%
\paragraph{The Stars Who Got Their Start on the '80s New York StageApril
16,
2018}\label{the-stars-who-got-their-start-on-the-80s-new-york-stageapril-16-2018}}

\includegraphics{https://static01.nyt.com/images/2018/04/05/t-magazine/05tmag-group-slide-QT9P/05tmag-group-slide-QT9P-mediumThreeByTwo210.jpg}

\hypertarget{we-are-family-1}{%
\subsubsection{We Are Family}\label{we-are-family-1}}

\hypertarget{chapter-1-heirs-and-alumni}{%
\paragraph{Chapter 1: Heirs and
Alumni}\label{chapter-1-heirs-and-alumni}}

\href{/interactive/2020/04/13/t-magazine/black-art-galleries.html}{}

\hypertarget{the-artists}{%
\subparagraph{The Artists}\label{the-artists}}

\href{/interactive/2020/04/13/t-magazine/italian-fashion-design-houses.html}{}

\hypertarget{the-dynasties}{%
\subparagraph{The Dynasties}\label{the-dynasties}}

\href{/interactive/2020/04/13/t-magazine/gordon-parks.html}{}

\hypertarget{the-directors}{%
\subparagraph{The Directors}\label{the-directors}}

\href{/interactive/2020/04/13/t-magazine/enrique-olvera-chef.html}{}

\hypertarget{the-disciples}{%
\subparagraph{The Disciples}\label{the-disciples}}

\href{/interactive/2020/04/13/t-magazine/royal-academy-antwerp.html}{}

\hypertarget{the-graduates}{%
\subparagraph{The Graduates}\label{the-graduates}}

\hypertarget{chapter-2-reunions-and-reconsiderations-1}{%
\paragraph{Chapter 2: Reunions and
Reconsiderations}\label{chapter-2-reunions-and-reconsiderations-1}}

\href{/interactive/2020/04/13/t-magazine/ninth-street-greenwich-village-neighbors.html}{}

\hypertarget{the-neighbors}{%
\subparagraph{The Neighbors}\label{the-neighbors}}

\href{/interactive/2020/04/13/t-magazine/omen-restaurant-nyc.html}{}

\hypertarget{the-regulars}{%
\subparagraph{The Regulars}\label{the-regulars}}

\href{/interactive/2020/04/13/t-magazine/hair-musical-broadway.html}{}

\hypertarget{hair-1967}{%
\subparagraph{Hair (1967)}\label{hair-1967}}

\href{/interactive/2020/04/13/t-magazine/sweeney-todd-revival.html}{}

\hypertarget{sweeney-todd-2005-revival}{%
\subparagraph{Sweeney Todd (2005
Revival)}\label{sweeney-todd-2005-revival}}

\href{/interactive/2020/04/13/t-magazine/daughters-of-the-dust.html}{}

\hypertarget{daughters-of-the-dust-1991}{%
\subparagraph{Daughters of the Dust
(1991)}\label{daughters-of-the-dust-1991}}

\hypertarget{chapter-3-legends-pioneers-and-survivors}{%
\paragraph{Chapter 3: Legends Pioneers and
Survivors}\label{chapter-3-legends-pioneers-and-survivors}}

\href{/interactive/2020/04/13/t-magazine/butch-stud-lesbian.html}{}

\hypertarget{the-renegades}{%
\subparagraph{The Renegades}\label{the-renegades}}

\href{/interactive/2020/04/13/t-magazine/act-up-aids.html}{}

\hypertarget{the-activists}{%
\subparagraph{The Activists}\label{the-activists}}

\href{/interactive/2020/04/13/t-magazine/artist-recluse.html}{}

\hypertarget{the-shadows}{%
\subparagraph{The Shadows}\label{the-shadows}}

\href{/interactive/2020/04/13/t-magazine/black-actresses-bassett-berry-blige-henson-whitfield-elise.html}{}

\hypertarget{the-veterans}{%
\subparagraph{The Veterans}\label{the-veterans}}

\hypertarget{chapter-4-the-new-guard}{%
\paragraph{Chapter 4: The New Guard}\label{chapter-4-the-new-guard}}

\href{/interactive/2020/04/13/t-magazine/asian-american-fashion-designers.html}{}

\hypertarget{the-designers}{%
\subparagraph{The Designers}\label{the-designers}}

\href{13tmag-beauties.html}{}

\hypertarget{the-beauties}{%
\subparagraph{The Beauties}\label{the-beauties}}

\href{/interactive/2020/04/13/t-magazine/nyc-downtown-nightlife-party-scene.html}{}

\hypertarget{the-scenemakers}{%
\subparagraph{The Scenemakers}\label{the-scenemakers}}

\href{/interactive/2020/04/13/t-magazine/maria-cornejo-olivier-rousteing-telfar-clemens-alessandro-michele.html\#olivier-rousteing-and-co}{}

\hypertarget{olivier-rousteing-and-co}{%
\subparagraph{Olivier Rousteing and
Co.}\label{olivier-rousteing-and-co}}

\href{/interactive/2020/04/13/t-magazine/maria-cornejo-olivier-rousteing-telfar-clemens-alessandro-michele.html\#maria-cornejo-and-co}{}

\hypertarget{maria-cornejo-and-co}{%
\subparagraph{Maria Cornejo and Co.}\label{maria-cornejo-and-co}}

\href{/interactive/2020/04/13/t-magazine/maria-cornejo-olivier-rousteing-telfar-clemens-alessandro-michele.html\#telfar-clemens-and-co}{}

\hypertarget{telfar-clemens-and-co}{%
\subparagraph{Telfar Clemens and Co.}\label{telfar-clemens-and-co}}

\href{/interactive/2020/04/13/t-magazine/maria-cornejo-olivier-rousteing-telfar-clemens-alessandro-michele.html\#alessandro-michele-and-co}{}

\hypertarget{alessandro-michele-and-co}{%
\subparagraph{Alessandro Michele and
Co.}\label{alessandro-michele-and-co}}

\href{/interactive/2020/04/13/t-magazine/foreign-correspondents.html}{}

\hypertarget{the-journalists}{%
\subparagraph{The Journalists}\label{the-journalists}}

\begin{itemize}
\item
\item
\item
\item
\end{itemize}

Advertisement

\protect\hyperlink{after-bottom}{Continue reading the main story}

\hypertarget{site-index}{%
\subsection{Site Index}\label{site-index}}

\hypertarget{site-information-navigation}{%
\subsection{Site Information
Navigation}\label{site-information-navigation}}

\begin{itemize}
\tightlist
\item
  \href{https://help.nytimes.com/hc/en-us/articles/115014792127-Copyright-notice}{©~2020~The
  New York Times Company}
\end{itemize}

\begin{itemize}
\tightlist
\item
  \href{https://www.nytco.com/}{NYTCo}
\item
  \href{https://help.nytimes.com/hc/en-us/articles/115015385887-Contact-Us}{Contact
  Us}
\item
  \href{https://www.nytco.com/careers/}{Work with us}
\item
  \href{https://nytmediakit.com/}{Advertise}
\item
  \href{http://www.tbrandstudio.com/}{T Brand Studio}
\item
  \href{https://www.nytimes.com/privacy/cookie-policy\#how-do-i-manage-trackers}{Your
  Ad Choices}
\item
  \href{https://www.nytimes.com/privacy}{Privacy}
\item
  \href{https://help.nytimes.com/hc/en-us/articles/115014893428-Terms-of-service}{Terms
  of Service}
\item
  \href{https://help.nytimes.com/hc/en-us/articles/115014893968-Terms-of-sale}{Terms
  of Sale}
\item
  \href{https://spiderbites.nytimes.com}{Site Map}
\item
  \href{https://help.nytimes.com/hc/en-us}{Help}
\item
  \href{https://www.nytimes.com/subscription?campaignId=37WXW}{Subscriptions}
\end{itemize}
