Sections

SEARCH

\protect\hyperlink{site-content}{Skip to
content}\protect\hyperlink{site-index}{Skip to site index}

\hypertarget{how-we-juneteenth}{%
\section{How We Juneteenth}\label{how-we-juneteenth}}

By \href{https://www.nytimes.com/by/veronica-chambers}{Veronica
Chambers}June 18, 2020

\begin{itemize}
\item
\item
\item
\item
\end{itemize}

Friday, June 19, 2020

Juneteenth

\includegraphics{https://static01.nyt.com/newsgraphics/2020/06/12/juneteenth-essays/ca339de5b1338f770caeab3b773d6ef6287c40e9/freedom-karnak-white.svg}

Is

\includegraphics{https://static01.nyt.com/packages/flash/multimedia/ICONS/transparent.png}

\includegraphics{https://static01.nyt.com/images/2020/06/18/fashion/18juneteenth-intro/18juneteenth-intro-master1050.jpg}

Brooklyn, N.Y., in the 1990s.Eli Reed/Magnum Photos

In

The \emph{Claiming}

Gov. Andrew Cuomo of New York signed an executive order on Wednesday
making Juneteenth a holiday for state employees; the same goes for tech
companies like Twitter, and even where I work, at The New York Times.
This
year,\href{https://www.nytimes.com/article/juneteenth-day-celebration.html}{}\href{https://www.nytimes.com/article/juneteenth-day-celebration.html}{Juneteenth,}
a holiday that celebrates the arrival of the news of emancipation from
slavery, seems to be a bigger deal across the nation.

But there's a conversation I've been having with my friends: Is
celebrating this holiday enough to begin to fix all that's so very
broken? And, one tick further, is the national embrace of what has been
known as the African-American Independence Day a dangerous idea? Some
people wonder --- if we sip on
our\href{https://cooking.nytimes.com/recipes/1021139-edouardo-jordans-juneteenth-red-punch?action=click\&module=Collection\%20Page\%20Recipe\%20Card\&region=Juneteenth\%20Recipes\%20Curated\%20by\%20Nicole\%20Taylor\&pgType=collection\&rank=1}{}\href{https://cooking.nytimes.com/recipes/1021139-edouardo-jordans-juneteenth-red-punch?action=click\&module=Collection\%20Page\%20Recipe\%20Card\&region=Juneteenth\%20Recipes\%20Curated\%20by\%20Nicole\%20Taylor\&pgType=collection\&rank=1}{traditional
red drinks} as we socially distance on screens and porches --- will we
be lulled into feeling more free than we really are?

Saidiya Hartman, the author of ``Wayward Lives, Beautiful Experiments''
and a 2019 MacArthur ``genius'' grant winner whose work explores the
``afterlife of slavery in modern American society,'' said: ``How to live
a free life, how one can live, is the pressing question for black folks
in the wake of slavery's formal end.'' Ms. Hartman said that imagining a
freer life and a more just society has been the purpose of generations
of black people since the days of Reconstruction.

``Recently, I heard Angela Davis talk about the radical imagination,''
Ms. Hartman said. ``And a fundamental requirement is believing that the
world you want to come into existence can happen. I think that that is
how black folks have engaged with and invested in and articulated
freedom, as an ideal and as an everyday practice.''

I couldn't agree more. As someone who has celebrated Juneteenth for a
long time, I think we need it now --- not in lieu of the freedom,
justice and equality we are still fighting for --- but in addition,
because we have been fighting for so very long.

The elemental sermon embedded into the history and lore of Juneteenth
has always been one of hope. The gifts of the holiday are the moments of
connection, renewal and joy for a people who have had to endure so much,
for so long.

To me, Juneteenth matters because it says: Keep going, the future you
want is coming. \emph{--- Veronica Chambers}

``Words of Emancipation didn't arrive until the middle of June so they
called it Juneteenth. \emph{So that was it, the night of Juneteenth
celebration}, his mind went on. \emph{The celebration of a gaudy
illusion}.'' --- Ralph Ellison, ``Juneteenth''

\href{https://www.nytimes.com/article/juneteenth-celebration-history.html}{}

\includegraphics{https://static01.nyt.com/newsgraphics/2020/06/12/juneteenth-essays/ca339de5b1338f770caeab3b773d6ef6287c40e9/freedom-karnak.svg}

Is

\includegraphics{https://static01.nyt.com/packages/flash/multimedia/ICONS/transparent.png}

\includegraphics{https://static01.nyt.com/images/2020/06/18/fashion/18Juneteenth-celebrate-topper/18Juneteenth-celebrate-topper-master1050-v2.jpg}

Emancipation Day celebrations in 1900 in Austin, Texas; right: a
Juneteenth parade in Minneapolis in 1995.via Austin Public Library;
Getty Images

a

Celebration

How We Juneteenth

Gina Cherelus

Read Story

\href{https://www.nytimes.com/2020/06/18/style/self-care/sojourner-truth-harriet-tubman-slavery-names.html}{}

\includegraphics{https://static01.nyt.com/newsgraphics/2020/06/12/juneteenth-essays/ca339de5b1338f770caeab3b773d6ef6287c40e9/freedom-karnak.svg}

Is

\includegraphics{https://static01.nyt.com/packages/flash/multimedia/ICONS/transparent.png}

\includegraphics{https://static01.nyt.com/images/2020/06/18/fashion/18juneteenth-naming-01/18juneteenth-naming-01-master1050-v2.png}

Mary Church Terrell.Smithsonian

in

Our Names

Ida, Maya, Rosa, Harriet: The Power in Our Names

Martha S. Jones

Read Story

\includegraphics{https://static01.nyt.com/packages/flash/multimedia/ICONS/transparent.png}

\includegraphics{https://static01.nyt.com/newsgraphics/2020/06/12/juneteenth-essays/ca339de5b1338f770caeab3b773d6ef6287c40e9/Mangum_Stacke_009_detail.jpg}

Portraits by Hugh Mangum, circa 1900.From ``Photos Day or Night: The
Archive of Hugh Mangum, by Sarah Stacke.

The Stuff of Astounding: A Poem for Juneteenth\\
By PATRICIA SMITH

Unless you spring from a history that is smug and reckless,
\textbf{unless}

you've vowed yourself blind to a ceaseless light, you see us.
\textbf{We}

are a shea-shined toddler writhing through Sunday sermon, we
\textbf{are}

the grizzled elder gingerly unfolding his last body. And we are
\textbf{intent}

and insistent upon the human in ourselves. We are the doctor \textbf{on}

another day at the edge of reason, coaxing a wrong hope,
\textbf{ripping}

open a gasping body to find air. We are five men dripping from
\textbf{the}

burly branches of young trees, which is to say that we dare a
\textbf{world}

that is both predictable and impossible. What else can we learn
\textbf{from}

suicides of the cuffed, the soft targets black backs be? Stuck in
\textbf{its}

rhythmic unreel, time keeps including us, even as our aged \textbf{root}

is doggedly plucked and trampled, cursed by ham-fisted spitters
\textbf{in}

the throes of a particular fever. See how we push on as enigma,
\textbf{the}

free out loud, the audaciously unleashed, how slyly we scan the
\textbf{sky}---

all that wet voltage and scatters of furious star---to realize that
\textbf{we}

are the recipients of an ancient grace. No, we didn't \emph{begin} to
\textbf{live}

when, on the 19th June day of that awkward, ordinary
spring---\textbf{with}

no joy, in a monotone still flecked with deceit---\emph{Seems you and}
\emph{\textbf{these}}

\emph{others are free.} That moment did not begin our breath. Our
\textbf{truths---}

the ones we'd been birthed with---had already met reckoning in
\textbf{the}

fields as we muttered tangled nouns of home. We reveled in
\textbf{black}

from there to now, our rampant hue and nap, the unbridled
\textbf{breath}

that resides in the rafters, from then to here, everything we are
\textbf{is}

the stuff of astounding. We are a mother who hums snippets of
\textbf{gospel}

into the silk curls of her newborn, we are the harried sister on
\textbf{the}

elevator to the weekly paycheck mama dreamed for her. We are
\textbf{black}

in every way there is---perm and kink, upstart and elder, wide
\textbf{voice,}

fervent whisper. We heft our clumsy homemade placards, we \textbf{will}

curl small in the gloom weeping to old blues ballads. We swear
\textbf{not}

to be anybody else's idea of free, lining up precisely, waiting to
\textbf{be}

freed again and again. We are breach and bellow, resisting a
\textbf{silent}

consent as we claim our much of America, its burden and snarl,
\textbf{the}

stink and hallelujah of it, its sicknesses and safe words, all its
\textbf{black}

and otherwise. Only those feigning blindness fail to see the
\textbf{body}

of work we are, and the work of body we have done. Everything
\textbf{is}

what it is because of us. It is misunderstanding to believe that
\textbf{free}

fell upon us like a blessing, that it was granted by a signature
\textbf{and}

an abruptly opened door. Listen to the thousand ways to say
\emph{\textbf{black}}

out loud. Hear a whole people celebrate their free and fragile
\textbf{lives,}

then find your own place inside that song. Make the singing
\textbf{matter.}

\includegraphics{https://static01.nyt.com/packages/flash/multimedia/ICONS/transparent.png}

\includegraphics{https://static01.nyt.com/images/2020/06/18/fashion/18Juneteenth-poem-02/merlin_173674188_60afa09a-e668-45a1-aaf2-859d7eb01a2b-master1050.jpg}

Martha Yates Jones and Pinkie Yates sitting in a buggy decorated with
flowers for the annual Juneteenth Celebration in 1908.The African
American Library at the Gregory School, Houston Public Library

\includegraphics{https://static01.nyt.com/packages/flash/multimedia/ICONS/transparent.png}

\includegraphics{https://static01.nyt.com/newsgraphics/2020/06/12/juneteenth-essays/assets/images/8c05678u-2000.jpg}

The hands of Henry Brooks, a formerly enslaved man, in 1941.Library of
Congress

\href{https://www.nytimes.com/2020/06/18/style/aunt-jemima-black-cooking.html}{}

\includegraphics{https://static01.nyt.com/newsgraphics/2020/06/12/juneteenth-essays/ca339de5b1338f770caeab3b773d6ef6287c40e9/freedom-karnak.svg}

Is a Gift

\includegraphics{https://static01.nyt.com/packages/flash/multimedia/ICONS/transparent.png}

\includegraphics{https://static01.nyt.com/images/2020/06/18/fashion/18JUNETEENTH-Cooking-SUB/18JUNETEENTH-Cooking-SUB-master1050-v2.jpg}

Annie Dunbar, Toni Tipton-Martin's grandmother.via Toni Tipton-Martin

How The Women of the Jemima Code Freed Me

Toni Tipton-Martin

Read Story

\href{https://www.nytimes.com/2020/06/18/us/politics/reparations-slavery.html}{}

\includegraphics{https://static01.nyt.com/newsgraphics/2020/06/12/juneteenth-essays/ca339de5b1338f770caeab3b773d6ef6287c40e9/freedom-karnak.svg}

Is

\includegraphics{https://static01.nyt.com/packages/flash/multimedia/ICONS/transparent.png}

\includegraphics{https://static01.nyt.com/images/2020/06/18/fashion/18juneteenth-reparations-01/18juneteenth-reparations-01-master1050-v2.jpg}

A group of free men, women and children in Richmond, Va., in
1865.Library of Congress

Still

Overdue

How Reparations for Slavery Became a 2020 Campaign Issue

Emma Goldberg

Read Story

``Like a lot of Black women, I have always had to invent the power my
freedom requires.'' --- June Jordan, from ``On Call''

\includegraphics{https://static01.nyt.com/packages/flash/multimedia/ICONS/transparent.png}

\includegraphics{https://static01.nyt.com/newsgraphics/2020/06/12/juneteenth-essays/assets/images/e_rch_0060_pub-2000.jpg}

Texas, 1958.R.C. Hickman Photographic Archive/UT Austin Briscoe Center
for American History

``All you need in the world is love and laughter. That's all anybody
needs. To have love in one hand and laughter in the other.'' --- August
Wilson, ``Joe Turner's Come and Gone''

\includegraphics{https://static01.nyt.com/packages/flash/multimedia/ICONS/transparent.png}

\includegraphics{https://static01.nyt.com/newsgraphics/2020/06/12/juneteenth-essays/assets/images/e_rch_0087_pub-2000.jpg}

Texas, 1956.R.C. Hickman Photographic Archive/UT Austin Briscoe Center
for American History

\includegraphics{https://static01.nyt.com/packages/flash/multimedia/ICONS/transparent.png}

\includegraphics{https://static01.nyt.com/newsgraphics/2020/06/12/juneteenth-essays/assets/images/e_rch_0041_pub-2000.jpg}

Swimmers at pool in Exline Park in Dallas, Texas, in 1957.R.C. Hickman
Photographic Archive/UT Austin Briscoe Center for American History

``The function of freedom is to free someone else.'' --- Toni Morrison,
commencement address at Barnard College

``The world is before you and you need not take it or leave it as it was
when you came in.'' --- James Baldwin, ``Nobody Knows My Name''

\includegraphics{https://static01.nyt.com/packages/flash/multimedia/ICONS/transparent.png}

\includegraphics{https://static01.nyt.com/newsgraphics/2020/06/12/juneteenth-essays/assets/images/e_rch_0271_pub-2000.jpg}

Texas, 1956.R.C. Hickman Photographic Archive/UT Austin Briscoe Center
for American History

\includegraphics{https://static01.nyt.com/packages/flash/multimedia/ICONS/transparent.png}

\includegraphics{https://static01.nyt.com/newsgraphics/2020/06/12/juneteenth-essays/assets/images/e_rch_0010_pub-2000.jpg}

A car on parade route at Texas State Fair. Dallas, Texas, 1955.R.C.
Hickman Photographic Archive/UT Austin Briscoe Center for American
History

\includegraphics{https://static01.nyt.com/packages/flash/multimedia/ICONS/transparent.png}

\includegraphics{https://static01.nyt.com/newsgraphics/2020/06/12/juneteenth-essays/assets/images/di_07761_pub-2000.jpg}

R.C. Hickman, a photojournalist who took the photos in this collection,
posing with his camera.R.C. Hickman Photographic Archive/UT Austin
Briscoe Center for American History

\href{https://www.nytimes.com/2020/06/18/style/self-care/rachel-cargle-anti-racism-ally.html}{}

\includegraphics{https://static01.nyt.com/newsgraphics/2020/06/12/juneteenth-essays/ca339de5b1338f770caeab3b773d6ef6287c40e9/freedom-karnak.svg}

\includegraphics{https://static01.nyt.com/packages/flash/multimedia/ICONS/transparent.png}

\includegraphics{https://static01.nyt.com/images/2020/06/18/fashion/18Juneteenth-cargle-1/merlin_173591088_bc14f455-b7e3-4732-87a6-eeea367ca8a0-master1050.jpg}

Rachel Cargle.Maiya Imani for The New York Times

Is Education

`Dear White Women': The Public Classroom of Rachel Cargle

Siraad Dirshe

Read Story

\href{https://www.nytimes.com/2020/06/18/style/self-care/healing-trauma-racism-wellness.html}{}

\includegraphics{https://static01.nyt.com/newsgraphics/2020/06/12/juneteenth-essays/ca339de5b1338f770caeab3b773d6ef6287c40e9/freedom-karnak.svg}

Is

\includegraphics{https://static01.nyt.com/packages/flash/multimedia/ICONS/transparent.png}

\includegraphics{https://static01.nyt.com/images/2020/06/15/fashion/00-juneteenth-generational1/00-juneteenth-generational1-master1050-v2.jpg}

Rikkí Wright

Self-Care

Rest as Reparations

Sandra E. Garcia

Read Story

\emph{A project by} Veronica Chambers, \emph{with} Tracy Ma, Joanna
Nikas, Choire Sicha \emph{and friends}. \emph{Photo editing by} Beth
Bristow, Anika Burgess, Nakyung Han, Eve Lyons, JuliAnna Patino
\emph{and} James Pomerantz

\begin{itemize}
\item
\item
\item
\item
\end{itemize}

Advertisement

\protect\hyperlink{after-bottom}{Continue reading the main story}

\hypertarget{site-index}{%
\subsection{Site Index}\label{site-index}}

\hypertarget{site-information-navigation}{%
\subsection{Site Information
Navigation}\label{site-information-navigation}}

\begin{itemize}
\tightlist
\item
  \href{https://help.nytimes.com/hc/en-us/articles/115014792127-Copyright-notice}{©~2020~The
  New York Times Company}
\end{itemize}

\begin{itemize}
\tightlist
\item
  \href{https://www.nytco.com/}{NYTCo}
\item
  \href{https://help.nytimes.com/hc/en-us/articles/115015385887-Contact-Us}{Contact
  Us}
\item
  \href{https://www.nytco.com/careers/}{Work with us}
\item
  \href{https://nytmediakit.com/}{Advertise}
\item
  \href{http://www.tbrandstudio.com/}{T Brand Studio}
\item
  \href{https://www.nytimes.com/privacy/cookie-policy\#how-do-i-manage-trackers}{Your
  Ad Choices}
\item
  \href{https://www.nytimes.com/privacy}{Privacy}
\item
  \href{https://help.nytimes.com/hc/en-us/articles/115014893428-Terms-of-service}{Terms
  of Service}
\item
  \href{https://help.nytimes.com/hc/en-us/articles/115014893968-Terms-of-sale}{Terms
  of Sale}
\item
  \href{https://spiderbites.nytimes.com}{Site Map}
\item
  \href{https://help.nytimes.com/hc/en-us}{Help}
\item
  \href{https://www.nytimes.com/subscription?campaignId=37WXW}{Subscriptions}
\end{itemize}
