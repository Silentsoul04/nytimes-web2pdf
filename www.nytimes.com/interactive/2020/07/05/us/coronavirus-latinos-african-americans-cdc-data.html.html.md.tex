Sections

SEARCH

\protect\hyperlink{site-content}{Skip to
content}\protect\hyperlink{site-index}{Skip to site index}

\href{https://www.nytimes.com/section/us}{U.S.}

\href{https://myaccount.nytimes.com/auth/login?response_type=cookie\&client_id=vi}{}

\href{https://www.nytimes.com/section/todayspaper}{Today's Paper}

\hypertarget{comments}{%
\subsection{\texorpdfstring{\protect\hyperlink{commentsContainer}{Comments}}{Comments}}\label{comments}}

\href{}{The Fullest Look Yet at the Racial Inequity of
Coronavirus}\href{}{Skip to Comments}

The comments section is closed. To submit a letter to the editor for
publication, write to
\href{mailto:letters@nytimes.com}{\nolinkurl{letters@nytimes.com}}.

\hypertarget{the-fullest-look-yet-at-the-racial-inequity-of-coronavirus}{%
\section{The Fullest Look Yet at the Racial Inequity of
Coronavirus}\label{the-fullest-look-yet-at-the-racial-inequity-of-coronavirus}}

By \href{https://www.nytimes.com/by/richard-a-oppel-jr}{Richard A. Oppel
Jr.}, \href{https://www.nytimes.com/by/robert-gebeloff}{Robert
Gebeloff}, \href{https://www.nytimes.com/by/kk-rebecca-lai}{K.K. Rebecca
Lai}, Will Wright and
\href{https://www.nytimes.com/by/mitch-smith}{Mitch Smith}July 5, 2020

\href{https://www.nytimes.com/es/interactive/2020/07/09/espanol/mundo/coronavirus-latinos-africanoamericanos-datos.html}{Leer
en español}

\begin{itemize}
\item
\item
\item
\item
\item
  \emph{303}
\end{itemize}

\hypertarget{coronavirus-cases-per-10000-people}{%
\subsubsection{Coronavirus cases per 10,000
people}\label{coronavirus-cases-per-10000-people}}

White 23

All 38

Black 62

Latino 73

Teresa and Marvin Bradley can't say for sure how they got the
coronavirus. Maybe Ms. Bradley, a Michigan nurse, brought it from her
hospital. Maybe it came from a visiting relative. Maybe it was something
else entirely.

What is certain --- according to new federal data that provides the most
comprehensive look to date on nearly 1.5 million coronavirus patients in
America --- is that the Bradleys are not outliers.

Racial disparities in who contracts the virus have played out in big
cities like Milwaukee and New York, but also in smaller metropolitan
areas like Grand Rapids, Mich., where the Bradleys live. Those
inequities became painfully apparent when Ms. Bradley, who is Black, was
wheeled through the emergency room.

``Everybody in there was African-American,'' she said. ``Everybody
was.''

Early numbers had shown that Black and Latino people were being harmed
by the virus at higher rates. But the new federal data --- made
available after The New York Times sued the Centers for Disease Control
and Prevention --- reveals a clearer and more complete picture: ****
Black and Latino people have been disproportionately affected by the
coronavirus in a widespread manner that spans the country, throughout
hundreds of counties in urban, suburban and rural areas, and across all
age groups.

\hypertarget{race-or-ethnicity-with-the-highest-coronavirus-rate-in-each-county}{%
\subsubsection{Race or ethnicity with the highest coronavirus rate in
each
county}\label{race-or-ethnicity-with-the-highest-coronavirus-rate-in-each-county}}

25 people

per 10,000

50

White

Black

Latino

Asian

Native American

No race data

White

Black

Latino

Asian

Native American

No race data

25

50

people per 10,000

Source: Centers for Disease Control and Prevention \textbar{} Note: Data
is through May 28 and includes only cases for which the race/ethnicity
and home county of the infected person was known. Only groups that make
up at least 1 percent of a county's population are considered in
determining the highlight color on the map. Sparsely populated areas in
counties are not highlighted. The C.D.C. data included race/ethnicity
information, but no county location, for infected people in eight
additional states: Hawaii, Maryland, Missouri, Nebraska, New Hampshire,
New Mexico, Texas and Vermont.

Latino and African-American residents of the United States have been
three times as likely to become infected as their white neighbors,
according to the new data, which provides detailed characteristics of
640,000 infections detected in nearly 1,000 U.S. counties. And Black and
Latino people have been nearly twice as likely to die from the virus as
white people, the data shows.

\hypertarget{rate-of-black-and-latino-coronavirus-cases-compared-with-white-cases}{%
\subsubsection{Rate of Black and Latino coronavirus cases, compared with
white
cases}\label{rate-of-black-and-latino-coronavirus-cases-compared-with-white-cases}}

Source: Centers for Disease Control and Prevention \textbar{} Note: Data
is through May 28.

The disparities persist across state lines and regions. They exist in
rural towns on the Great Plains, in suburban counties, like Fairfax
County, Va., and in many of the country's biggest cities.

``Systemic racism doesn't just evidence itself in the criminal justice
system,'' said Quinton Lucas, who is the third Black mayor of Kansas
City, Mo., which is in a state where 40 percent of those infected are
Black or Latino even though those groups make up just 16 percent of the
state's population. ``It's something that we're seeing taking lives in
not just urban America, but rural America, and all types of parts where,
frankly, people deserve an equal opportunity to live --- to get health
care, to get testing, to get tracing.''

The data also showed several pockets of disparity involving Native
American people. In much of Arizona and in several other counties, they
were far more likely to become infected than white people. For people
who are Asian, the disparities were generally not as large, though they
were 1.3 times as likely as their white neighbors to become infected.

The new federal data, which is a major component of the agency's disease
surveillance efforts, is far from complete. Not only is race and
ethnicity information missing from more than half the cases, but so are
other epidemiologically important clues --- such as how the person might
have become infected.

And because it includes only cases through the end of May, it doesn't
reflect the recent surge in infections that has gripped parts of the
nation.

Still, the data is more comprehensive than anything the agency has
released to date, and The Times was able to analyze the racial disparity
in infection rates across 974 counties representing more than half the
U.S. population, a far more extensive survey than was previously
possible.

\hypertarget{disparities-in-the-suburbs}{%
\subsection{Disparities in the
suburbs}\label{disparities-in-the-suburbs}}

For the Bradleys, both in their early 60s, the symptoms didn't seem like
much at first. A tickle at the back of the throat.

But soon came fevers and trouble breathing, and when the pair went to
the hospital, they were separated. Ms. Bradley was admitted while Mr.
Bradley was sent home. He said he felt too sick to leave, but that he
had no choice. When he got home, he felt alone and uncertain about how
to treat the illness.

\includegraphics{https://static01.nyt.com/packages/flash/multimedia/ICONS/transparent.png}

\includegraphics{https://static01.nyt.com/newsgraphics/2020/06/17/coronavirus-race/assets/images/grand-rapids-2000.jpg}

Teresa Bradley, 60, and her husband, Marvin Bradley, 61, both had
Covid-19 earlier this year.Elaine Cromie for The New York Times

It took weeks, but eventually they both recovered. When Mr. Bradley
returned to work in the engineering department of a factory several
weeks later, a white co-worker told Mr. Bradley that he was the only
person he knew who contracted the virus.

By contrast, Mr. Bradley said he knew quite a few people who had gotten
sick. A few of them have died.

``We're most vulnerable to this thing,'' Mr. Bradley said.

In Kent County, which includes Grand Rapids and its suburbs, Black and
Latino residents account for 63 percent of infections, though they make
up just 20 percent of the county's population. Public health officials
and elected leaders in Michigan said there was no clear reason Black and
Latino people in Kent County were even more adversely affected than in
other parts of the country.

Among the 249 counties with at least 5,000 Black residents for which The
Times obtained detailed data, the infection rate for African-American
residents is higher than the rate for white residents in all but 14 of
those counties. Similarly, for the 206 counties with at least 5,000
Latino residents analyzed by The Times, 178 have higher infection rates
for Latino residents than for white residents.

\hypertarget{coronavirus-cases-per-10000-black-residents}{%
\subsubsection{Coronavirus cases per 10,000 Black
residents}\label{coronavirus-cases-per-10000-black-residents}}

Insufficient or

no race data

2 times the rate

of white cases

4 times

Source: Centers for Disease Control and Prevention \textbar{} Notes: Map
shows counties that have more than 5,000 Black people, that have more
than 50 cases and that have case data for both Black and white
residents. Sparsely populated areas in counties are not highlighted.
Data is through May 28.

``As an African-American woman, it's just such an emotional toll,'' said
Teresa Branson, the deputy administrative health officer in Kent County,
whose agency has coordinated with Black pastors and ramped up testing in
hard-hit neighborhoods.

Experts point to circumstances that have made Black and Latino people
more likely than white people to be exposed to the virus: Many of them
have front-line jobs that keep them from working at home; rely on public
transportation; or live in cramped apartments or multigenerational
homes.

``You literally can't isolate with one bathroom,'' said Lt. Gov. Garlin
Gilchrist II, who leads Michigan's task force on coronavirus racial
disparities.

\hypertarget{we-just-have-to-keep-working}{%
\subsection{`We just have to keep
working'}\label{we-just-have-to-keep-working}}

Latino people have also been infected at a jarringly disparate rate
compared with white people. One of the most alarming hot spots is also
one of the wealthiest: Fairfax County, just outside of Washington, D.C.

Three times as many white people live there as Latinos. Yet through the
end of May, four times as many Latino residents had tested positive for
the virus, according to the C.D.C. data.

\hypertarget{coronavirus-cases-per-10000-latinos}{%
\subsubsection{Coronavirus cases per 10,000
Latinos}\label{coronavirus-cases-per-10000-latinos}}

2 times the rate

of white cases

4 times

Insufficient or

no race data

Source: Centers for Disease Control and Prevention \textbar{} Notes: Map
shows counties that have more than 5,000 Latino residents, that have
more than 50 cases and that have case data for both Latino and white
residents. Sparsely populated areas in counties are not highlighted.
Data is through May 28.

With the
\href{https://www.census.gov/quickfacts/fairfaxcountyvirginia}{median
household income} in Fairfax twice the
\href{https://www.census.gov/quickfacts/fact/table/US/LFE305218\#LFE305218}{national
average} of about \$60,000, housing is expensive, leaving those with
modest incomes piling into apartments, where social distancing is an
impossibility. In 2017, it took an annual income of
\href{https://www.fairfaxcounty.gov/news2/who-we-are-in-fairfax-county-in-2018-annual-demographics-report/}{almost
\$64,000 to afford a typical one-bedroom apartment}, according to county
data. And many have had to keep commuting to jobs.

Diana, who is 26 and did not want her last name used out of fear for her
husband's job, said her husband got sick at a construction site in
April. She and her brother, who also works construction, soon fell ill,
too. With three children between them, the six family members live in a
two-bedroom apartment.

Diana, who was born in the United States but moved to Guatemala with her
parents as a small child before returning to this country five years
ago, is still battling symptoms. ``We have to go out to work,'' she
said. ``We have to pay our rent. We have to pay our utilities. We just
have to keep working.''

\includegraphics{https://static01.nyt.com/packages/flash/multimedia/ICONS/transparent.png}

\includegraphics{https://static01.nyt.com/newsgraphics/2020/06/17/coronavirus-race/assets/images/fairfax-1-2000.jpg}

Diana, with her 3-year-old son. She was sick with the coronavirus in
April.Hector Emanuel for The New York Times

At Culmore Clinic, an interfaith free clinic serving low-income adults
in Fairfax, about half of the 79 Latino patients who tested for the
virus have been positive.

``This is a very wealthy county, but their needs are invisible,'' said
Terry O'Hara Lavoie, a co-founder of the clinic. The risk of getting
sick from tight living quarters, she added, is compounded by the
pressure to keep working or quickly return to work, even in risky
settings.

The risks are borne out by demographic data. Across the country, 43
percent of Black and Latino workers are employed in service or
production jobs that for the most part cannot be done remotely, census
data from 2018 shows. Only about one in four white workers held such
jobs.

Also, Latino people are twice as likely to reside in a crowded dwelling
--- less than 500 square feet per person --- as white people, according
to the American Housing Survey.

The national figures for infections and deaths from the virus understate
the disparity to a certain extent, since the virus is far more prevalent
among older Americans, who are disproportionately white compared with
younger Americans. When comparing infections and deaths just within
groups who are around the same ages, the disparities are even more
extreme.

\hypertarget{coronavirus-cases-per-10000-people-by-age-and-race}{%
\subsubsection{Coronavirus cases per 10,000 people, by age and
race}\label{coronavirus-cases-per-10000-people-by-age-and-race}}

Source: Centers for Disease Control and Prevention \textbar{} Note: Data
is through May 28.

Latino people between the ages of 40 and 59 have been infected at five
times the rate of white people in the same age group, the new C.D.C.
data shows. The differences are even more stark when it comes to deaths:
Of Latino people who died, more than a quarter were younger than 60.
Among white people who died, only 6 percent were that young.

Jarvis Chen, a researcher and lecturer at the Harvard T. H. Chan School
of Public Health, said that the wide racial and ethnic disparities found
in suburban and exurban areas as revealed in the new C.D.C data should
not come as a surprise. The discrepancies in how people of different
races, ethnicities and socioeconomic statuses live and work may be even
more pronounced outside of urban centers than they are in big cities,
Dr. Chen said.

``As the epidemic moves into suburban areas, there are good reasons to
think that the disparities will grow larger,'' he said.

\hypertarget{the-shortfalls-of-the-governments-data}{%
\subsection{The shortfalls of the government's
data}\label{the-shortfalls-of-the-governments-data}}

The Times obtained the C.D.C. data after filing a Freedom of Information
Act lawsuit to force the agency to release the information.

To date, the agency has released nearly 1.5 million case records. The
Times asked for information about the race, ethnicity and county of
residence of every person who tested positive, but that data was missing
for hundreds of thousands of cases.

C.D.C. officials said the gaps in their data are because of the nature
of the national surveillance system, which depends on local agencies.
They said that the C.D.C. has asked state and local health agencies to
collect detailed information about every person who tests positive, but
that it cannot force local officials to do so. Many state and local
authorities have been overwhelmed by the volume of cases and lack the
resources to investigate the characteristics of every individual who
falls ill, C.D.C. officials said.

Even with the missing information, agency scientists said, they can
still find important patterns in the data, especially when combining the
records about individual cases with aggregated data from local agencies.

Still, some say the initial lack of transparency and the gaps in
information highlight a key weakness in the U.S. disease surveillance
system.

``You need all this information so that public health officials can make
adequate decisions,'' said Andre M. Perry, a fellow in the Metropolitan
Policy Program at The Brookings Institution. ``If they're not getting
this information, then municipalities and neighborhoods and families are
essentially operating in the dark.''

\hypertarget{higher-cases-higher-deaths}{%
\subsection{Higher cases, higher
deaths}\label{higher-cases-higher-deaths}}

The higher rate in deaths from the virus among Black and Latino people
has been explained, in part, by a higher prevalence of underlying health
problems, including diabetes and obesity. But the new C.D.C. data
reveals a significant imbalance in the number of virus cases, not just
deaths --- a fact that scientists say underscores inequities unrelated
to other health issues.

The focus on comorbidities ``makes me angry, because this really is
about who still has to leave their home to work, who has to leave a
crowded apartment, get on crowded transport, and go to a crowded
workplace, and we just haven't acknowledged that those of us who have
the privilege of continuing to work from our homes aren't facing those
risks,'' said Dr. Mary Bassett, the Director of the FXB Center for
Health and Human Rights at Harvard University.

Dr. Bassett, a former New York City health commissioner, said there is
no question that underlying health problems --- often caused by factors
that people cannot control, such as lack of access to healthy food
options and health care --- play a major role in Covid-19 deaths.

But she also said a big determinant of who dies is who gets sick in the
first place, and that infections have been far more prevalent among
people who can't work from home. ``Many of us also have problems with
obesity and diabetes, but we're not getting exposed, so we're not
getting sick,'' she said.

The differences in infection case rates are striking, said Jennifer
Nuzzo, an epidemiologist and professor at the Johns Hopkins Bloomberg
School of Public Health.

``Some people have kind of waved away the disparities by saying, `Oh,
that's just underlying health conditions,''' Dr. Nuzzo said. ``That's
much harder to do with the case data.''

In June, C.D.C. officials estimated that the true tally of virus cases
was 10 times the number of reported cases. They said they could not
determine whether these unreported cases had racial and ethnic
disparities similar to those seen in the reported infections.

But they said that more-severe infections --- which are more often
associated with underlying health conditions, and with people seeking
medical care --- are more likely to be recorded as cases.

That difference in the reporting of cases might explain some portion of
the race and ethnicity disparities in the number of documented
infections, C.D.C. officials said. But they said that it was also clear
that there have been significant disparities in the number of both
deaths and cases.

Methodology

To measure how the coronavirus pandemic is affecting various demographic
groups in the United States, The New York Times obtained a database of
individual confirmed cases along with characteristics of each infected
person from the Centers for Disease Control and Prevention.

The data was acquired after The Times filed a Freedom of Information Act
suit. The C.D.C. provided data on 1.45 million cases reported to the
agency by states through the end of May. Many of the records were
missing critical information The Times requested, like the race and home
county of an infected person, so the analysis was based on the nearly
640,000 cases for which the race, ethnicity and home county of a patient
was known.

The data allowed The Times to measure racial disparities across 974
counties, which account for about 55 percent of the nation's population,
a far wider look than had been possible previously. Infection and death
rates were calculated by grouping cases in the C.D.C. data by race,
ethnicity and age group, and comparing the totals with the most recent
Census Bureau population estimates for each county.

For national totals, The Times calculated rates based on both the actual
population and the age-adjusted population of each county. The age
adjustment accounts for the higher prevalence of the virus among older
U.S. residents and the varying age patterns of different racial and
ethnic groups. The national totals exclude data for eight states for
which county-level information was not provided, but each of those
states also showed a racial disparity in case rates.

Read 303 Comments

\begin{itemize}
\item
\item
\item
\item
\end{itemize}

Advertisement

\protect\hyperlink{after-bottom}{Continue reading the main story}

\hypertarget{site-index}{%
\subsection{Site Index}\label{site-index}}

\hypertarget{site-information-navigation}{%
\subsection{Site Information
Navigation}\label{site-information-navigation}}

\begin{itemize}
\tightlist
\item
  \href{https://help.nytimes.com/hc/en-us/articles/115014792127-Copyright-notice}{©~2020~The
  New York Times Company}
\end{itemize}

\begin{itemize}
\tightlist
\item
  \href{https://www.nytco.com/}{NYTCo}
\item
  \href{https://help.nytimes.com/hc/en-us/articles/115015385887-Contact-Us}{Contact
  Us}
\item
  \href{https://www.nytco.com/careers/}{Work with us}
\item
  \href{https://nytmediakit.com/}{Advertise}
\item
  \href{http://www.tbrandstudio.com/}{T Brand Studio}
\item
  \href{https://www.nytimes.com/privacy/cookie-policy\#how-do-i-manage-trackers}{Your
  Ad Choices}
\item
  \href{https://www.nytimes.com/privacy}{Privacy}
\item
  \href{https://help.nytimes.com/hc/en-us/articles/115014893428-Terms-of-service}{Terms
  of Service}
\item
  \href{https://help.nytimes.com/hc/en-us/articles/115014893968-Terms-of-sale}{Terms
  of Sale}
\item
  \href{https://spiderbites.nytimes.com}{Site Map}
\item
  \href{https://help.nytimes.com/hc/en-us}{Help}
\item
  \href{https://www.nytimes.com/subscription?campaignId=37WXW}{Subscriptions}
\end{itemize}
