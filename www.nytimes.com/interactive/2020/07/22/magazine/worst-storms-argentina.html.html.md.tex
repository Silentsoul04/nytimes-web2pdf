Sections

SEARCH

\protect\hyperlink{site-content}{Skip to
content}\protect\hyperlink{site-index}{Skip to site index}

\hypertarget{comments}{%
\subsection{\texorpdfstring{\protect\hyperlink{commentsContainer}{Comments}}{Comments}}\label{comments}}

\href{}{What's Going on Inside the Fearsome Thunderstorms of Córdoba
Province?}\href{}{Skip to Comments}

The comments section is closed. To submit a letter to the editor for
publication, write to
\href{mailto:letters@nytimes.com}{\nolinkurl{letters@nytimes.com}}.

\hypertarget{whats-going-on-inside-the-fearsome-thunderstorms-of-cuxf3rdoba-province}{%
\section{What's Going on Inside the Fearsome Thunderstorms of Córdoba
Province?}\label{whats-going-on-inside-the-fearsome-thunderstorms-of-cuxf3rdoba-province}}

By Noah Gallagher ShannonJuly 22, 2020

\href{https://www.nytimes.com/es/interactive/2020/07/23/espanol/ciencia-y-tecnologia/cordoba-tormentas-argentina.html}{Leer
en español}

\begin{itemize}
\item
\item
\item
\item
\item
  \emph{211}
\end{itemize}

Scientists are studying the extreme weather in northern Argentina to see
how it works --- and what it can tell us about the monster storms in our
future.

\hypertarget{whats-going-on-inside-the-fearsome-thunderstorms-of-cuxf3rdoba-province-1}{%
\section{What's Going on Inside the Fearsome Thunderstorms of Córdoba
Province?}\label{whats-going-on-inside-the-fearsome-thunderstorms-of-cuxf3rdoba-province-1}}

Scientists are studying the extreme weather in northern Argentina to see
how it works --- and what it can tell us about the monster storms in our
future.

By Noah Gallagher Shannon

July 22, 2020

SHARE

\hypertarget{listen-to-this-article}{%
\subsection{Listen to This Article}\label{listen-to-this-article}}

\emph{To hear more audio stories from publishers like The New York
Times,
download\href{https://www.audm.com/?utm_source=nytmag\&utm_medium=embed\&utm_campaign=left_behind_draper}{}\href{https://www.audm.com/?utm_source=nytmag\&utm_medium=embed\&utm_campaign=relampago_thunderstorms_argentina}{Audm
for iPhone or Android}.}

\href{https://www.nytimes.com/es/interactive/2020/07/23/espanol/ciencia-y-tecnologia/cordoba-tormentas-argentina.html}{Leer
en español}

When he thought back to the late-December morning when Berrotarán was
entombed in hail, it was the memory of fog that brought Matias Lenardon
the greatest dread. He remembered that it had drifted into the scattered
farming settlement in north-central Argentina sometime after dawn. Soon
it had grown thicker than almost any fog the young farmer had seen
before. It cloaked the corn and soybean fields ringing the town and
obscured the restaurants and carnicerias that line the main
thoroughfare. He remembered that the fog bore with it the cool mountain
air of the nearby Sierras de Córdoba, a mountain range whose tallest
peaks rise abruptly from the plains just to the town's northwest. Like
any lone feature in flat country, the sierras had long served as
lodestar to the local agricultural community, who kept a close watch on
them for signs of approaching weather. But if Lenardon or anyone else in
Berrotarán thought much of the fog that morning in 2015, it was only
that it obscured their usual view of the peaks.

At the time, Lenardon was at the local radio station, where he
moonlighted as the town's weather forecaster. It was a role the
22-year-old had inherited, in some sense, from his grandfather Eduardo
Malpassi, who began recording daily weather observations in a family
almanac almost 50 years before. Like many farmers in Córdoba Province,
Lenardon had learned from older generations how to read the day's
advancing weather according to a complex taxonomy of winds and clouds
that migrated across the pampas --- the vast pale grasslands that
blanket much of the country's interior. If the winds turned cool as the
day wore on, Lenardon knew it meant rain, brought north from Patagonia.
More troubling were the winds that blew in wet and hot from the
northwest --- off the sierras.

As forecaster, Lenardon's chief concern was identifying weather patterns
that might breed a thunderstorm, which on the pampas are notoriously
swift and violent. Few official records are kept in Córdoba and the
surrounding regions, but over the previous two years alone, newspapers
reported that hail, flooding and tornadoes had damaged or razed
thousands of acres of cropland, displaced more than five thousand people
and killed about a dozen. Locals described barbed hailstones, shaped
like medieval flails, destroying buildings and burying cars up to the
hoods. Lenardon's own family had lost their entire harvest to flooding
three of the last five years, forcing them at one point onto state
assistance. People in Berrotarán spent much of their summer bracing for
the atmosphere to explode; the fire department had recently taken to
standing at the ready with rescue equipment and heavy machinery, in
hopes of getting a jump on digging people out of debris. Even so,
Lenardon didn't think much of the fog when he first saw it. The cool,
moist air didn't indicate anything, as far as he knew, except a welcome
relief from the heat.

As Lenardon prepared to leave the station, he pulled up the feed from
the region's lone radar dish in the nearby city of Córdoba, more out of
habit than anything else. When the radar completed its 15-minute sweep,
a massive red splotch flashed on the screen --- a powerful storm
appeared to be bearing down on them. Convinced it was a glitch, Lenardon
raced outside to check the sky --- forgetting in his panic that it was
shrouded by fog. While the fog had little meteorological effect on the
storm, it had nonetheless ensured that it would be maximally
destructive. ``No one could feel the wind,'' he said. ``No one could see
the sierras.'' Though he rushed to go live on the radio, it was already
9 a.m. by the time he issued a severe storm warning for 9:15.

The storm descended quickly. It engulfed the western side of Berrotarán,
where winds began gusting at over 80 m.p.h. Soon, hail poured down,
caving in the roof of a machine shop and shattering windshields. In 20
minutes, so much ice had begun to accumulate that it stood in the street
in mounds, like snowdrifts. As the hail and rain continued to intensify,
they gradually mixed into a thick white slurry, encasing cars, icing
over fields and freezing the town's main canal. With the drainage
ditches filled in and frozen, parts of the town flooded, transforming
the dirt roads into surging muddy rivers. Residents watched as their
homes filled with icy water.

At home, Lenardon went back over his forecast, searching for what he had
missed. ``When you don't have a sophisticated forecast system,'' he
said, ``everyone is afraid of future storms.''

Lenardon and I met in early December 2018, at the height of summer storm
season, in the resort town of Villa Carlos Paz, about a two-hour drive
north of Berrotarán. A short and friendly man with large, inquisitive
black eyes and the molded frame of a rugby player, he wore a polo shirt
and carried with him a backpack full of weather books and records. We
were seated together in a hotel suite, where Lenardon was spending the
day meeting with a group of government and university scientists who are
funded by the National Science Foundation, NASA and the Department of
Energy. The group was in the midst of a two-month field campaign chasing
the storms of the Sierras de Córdoba, and asked for Lenardon to join
them.

The invitation had come specifically from the study's leader, a
43-year-old severe-weather expert named Steve Nesbitt, who after
learning of Lenardon's story had driven several hours to meet him. A
veteran of storm-chasing campaigns in Nepal, India and the Pacific,
Nesbitt had developed a habit over the years of enlisting local sources.
He found their stories often contained information that satellites
missed or couldn't perceive --- how the contour of the land influenced
clouds, how a storm might suddenly change directions in open country. In
the case of the sierras, Nesbitt also knew that stories like Lenardon's
represented some of the only existing in-situ data on the storms. Few,
if any, scientists had ever observed them up close.

Nesbitt, who is a professor at the University of Illinois at
Urbana-Champaign, had dedicated much of the last 15 years to studying
the freakish storms of this sleepy agricultural region. He first became
fascinated by them in the early 2000s, when a NASA satellite tentatively
identified them as the largest and most violent on Earth. ``We knew
about the Great Plains, the Sahel,'' Nesbitt said. But this appeared to
be another world. Radar images suggested cloud structures dwarfing those
of Tornado Alley or Ganges Plain, many of them materializing in as
little as 30 minutes. (Thunderstorms typically develop over the course
of several hours.) And yet in the years since, little reliable data had
emerged. Many in the meteorological community felt the storms were
simply too remote and too dangerous for controlled study. ``The only
thing the science community knew for certain,'' Nesbitt said, ``was that
these things were monsters.''

Nesbitt had traveled to Córdoba Province because he felt the weather
patterns might offer clues into the enduring riddle of why certain
storms grew unexpectedly into cataclysms. In the United States, which is
home to the most extensive weather forecasting infrastructure in the
world, around a third of severe weather predictions still prove wrong
--- not only about timing and location but also size, duration and
intensity. The false-alarm rate for tornadoes continues to hover at
about 70 percent, while the average warning time has only increased from
about 10 minutes in the mid-1990s to 15 minutes today. Satellites and
supercomputer modeling have greatly improved the detection of
large-scale phenomena --- uncertainty about a hurricane's path at 48
hours out, for example, has decreased by 30 percent since Katrina ---
but the more routine, and nevertheless destructive, storms that impact
rural provinces and towns continue to erupt with little warning. Today
few countries outside the United States and Western Europe even attempt
to forecast extreme weather. In a place like Córdoba, prediction has
often fallen to amateurs like Lenardon, who, tasked with the safety of
their communities, must puzzle from the air what the sparse and
unreliable infrastructure misses.

But it was a job that had grown considerably more difficult in recent
years. As Lenardon explained to Nesbitt, the region was beginning to see
ever more storms escalate in both size and intensity. ``Before, it was
impossible for me to imagine more than one damaging storm a year,'' he
said. ``Now I expect three or four.'' For Nesbitt, it was exactly these
abnormal qualities of growth and destructiveness that made the sierras
instructive. He believed that if he could chance a closer look inside
one of the superstorms --- mapping its internal wind structure and the
conditions that gave it life --- he might be able to produce a blueprint
for predicting others like it, in Argentina and worldwide.
``Climate-change models are predicting all this bad weather,'' Nesbitt
said. ``But no one knows exactly what that weather will look like.'' In
Córdoba, he thought he'd discovered a laboratory for studying it --- a
rugged, poorly mapped swath of ground the size of Wisconsin, which might
offer a glimpse of the storms to come.

\textbf{If storm forecasting} may seem the province of banal TV
broadcasts, it's only because its routine accuracy now underpins so much
of modern civilization's stability and abundance --- not just in the
evasion of disasters but also the preservation of the mundane. The World
Meteorological Organization estimates that preventive road closures,
supply-chain rerouting and the like save the world economy more than
\$100 billion annually. At any given moment, our expansive global
infrastructure of satellites and weather stations is working to predict
around 2,000 or more storms. It is a system that, at its best, promises
some semblance of order amid chaos.

Every storm is composed of the same fundamental DNA --- in this case,
moisture, unstable air and something to ignite the two skyward, often
heat. When the earth warms in the spring and summer months, hot wet air
rushes upward in columns, where it collides with cool dry air, forming
volatile cumulus clouds that can begin to swell against the top of the
troposphere, at times carrying as much as a million tons of water. If
one of these budding cells manages to punch through the tropopause, as
the boundary between the troposphere and stratosphere is called, the
storm mushrooms, feeding on the energy-rich air of the upper atmosphere.
As it continues to grow, inhaling up more moisture and breathing it back
down as rain and hail, this vast vertical lung can sprout into a
self-sustaining system that takes on many different forms. Predicting
exactly what form this DNA will arrange itself into, however, turns out
to be a puzzle on par with biological diversity. Composed of millions of
micro air currents, electrical pulses and unfathomably complex networks
of ice crystals, every storm is a singular creature, growing and
behaving differently based on its geography and climate.

With so many variables at play, it became apparent to modern
meteorologists that predicting storms required sampling as many as
possible. The perfect repository, as it turned out, existed in the Great
Plains, where many of the world's most dangerous storms are born. Here,
in the spring and summer months, moist air off the Gulf of Mexico pools
with dry air from the Arctic and southwestern deserts, which is all then
corralled by the Rocky Mountains, forming a massive eddy. For
meteorologists, this sustained volatility has made the plains the de
facto national laboratory, where about 30 National Weather Service
offices, tens of thousands of private radars and weather stations and
hundreds of airports are sampling the air conditions before, during and
after storms. Each sample, whether taken by radar or wind gauge, is a
snapshot of that particular storm's behavior and composition --- such as
air density, pressure, temperature, humidity and wind velocity ---
providing meteorologists a profile to look for in the future.

Until the launch of global weather satellites in the 1990s, this level
of sampling and detection wasn't widely available outside North America.
When NASA deployed its Tropical Rainfall Measuring Mission in 1997, the
satellite offered the first comprehensive look at the entire world's
weather. And part of what it revealed was an enormous regional
variability in the size and intensity of storms. In Argentina, in
particular, around the Sierras de Córdoba's sliver of peaks, T.R.M.M.
data detected anomalous cloud formations on a scale never seen before:
225 lightning flashes a minute, enormous hail and thunderheads reaching
almost 70,000 feet.

But data from T.R.M.M. and other satellites also revealed that storms
throughout the world shared many of the same microphysical properties
--- some of which appeared to be changing. In the last few decades, as
humans have poured more and more carbon into the atmosphere, heating the
land and oceans, the air has become infused with greater levels of
evaporated moisture, wind shear and what meteorologists call
``convective available potential energy,'' or CAPE --- a measure of how
much raw fuel for storms the sky contains. And with ever more heat,
moisture and unstable air available to feed on, storms in many parts of
the world have begun to exhibit increasingly erratic behavior. Since
1980, the number of storms with winds topping 155 m.p.h. --- the speed
at which wind starts to tear walls from buildings --- has tripled; over
the last few years, parts of India and the American South have flooded,
with anywhere from 275 to 500 percent more rain than usual. In the
oceans, where there is now 5 percent more water aloft than there was in
the middle of last century, the odds of a storm spinning into a major
hurricane have shot up substantially in the last 40 years. In the
Eastern United States, which is projected to see a 15 percent increase
in days with high CAPE values over the next century, the 2011 ``super
outbreak'' saw 362 tornadoes kill an estimated 321 people in four days.

Still, the most disturbing trend for meteorologists isn't the violence
of these supercharged storms; it's the deeper concern that entire
weather patterns are becoming distorted as storms stray into new
latitudes and seasons. When Cyclone Idai hit Mozambique in March 2019,
hundreds of thousands were caught unprepared by its late arrival in the
season. Six weeks later, when Cyclone Kenneth slammed into the same
coast, becoming perhaps the strongest storm to hit Mozambique,
evacuation routes and shelters were still choked with people.

But if meteorologists could broadly infer that a wetter, hotter planet
was contributing to these outbreaks, what they struggled to grasp was
how each storm was reacting to it. Some storms appear to metabolize
changes in the climate as faster sustained wind speeds, which is why
researchers at M.I.T. and Princeton now consider a Category Six
hurricane a realistic possibility; others as heavier deluges of rain.
Even if some basic trends appeared to be emerging, the relative rarity
of extreme events, coupled with their remoteness and the fact that
usable satellite data dates to only 1960 or so, meant that it was still
mostly impossible to project what extremes might materialize from place
to place --- much less in the years to come. In 2019, a study conducted
by Stockholm University found that one of the only uniform impacts of
climate change was on forecasting, which has become more difficult. It
all of a sudden seemed possible that humankind was losing not only the
comfort of a future that looked dependably like the present, but the
reliability of a stable tomorrow.

For Nesbitt and a growing cohort of young meteorologists, the chaos
wrought by climate change requires radically rethinking some of
meteorology's core concepts. As a discipline, meteorology is based on
the idea that the climate is a constant; within each year, season or
day, only a certain number and range of variable weather events are
possible. But because that constant has itself become a variable,
Nesbitt thinks the field needs to take a big step back and begin again
with the basics: close observations of how storms develop and behave.
``We thought we knew how the climate and weather operated,'' he told me.
``But now we have to think more like astronomers --- like we don't know
what's out there.''

\textbf{The makeshift headquarters} of the study --- named RELAMPAGO, an
English acronym that also means ``lightning'' in Spanish --- occupied an
array of outbuildings and conference rooms spread between a sprawling
white estate and high-rise hotel in downtown Villa Carlos Paz. The
sierras, which loom over the west end of town, are visible from almost
anywhere on the study's two sites, impeding the horizon. When I arrived
at the hotel ops-center, one afternoon in mid-December, I found Nesbitt
hunched over a swirling computer model in a narrow glass-enclosed room.
He is tall and thickset, with a round, dimpled chin and boyish flop of
hair, and he wore cargo shorts, a short-sleeve tropical shirt and
sandals. He led me through a crowded office lined with servers and
computers, where grad students stood monitoring satellite images, and
into a crumbling courtyard that served as an office. It had now been
four or five weeks since the last rash of major storms, and the sky
above us stood huge and empty, save an occasional, lonely cumulus cloud
that came drifting over the sierras, carried on the unseasonably
pleasant breeze.

Nesbitt had come to Argentina with the goal of chasing the region's
storms so he could get advanced imaging technology deep within them.
``In every storm there are fingerprints you can see of changing
processes,'' he said, and if he could find them, he could begin
assessing how the storms are transforming in a warmer climate. But as he
began scouting the study around 2012, he quickly realized that sampling
one of the most dangerous and unpredictable phenomena on Earth, in a
faraway region of scattered farm towns and mountain forests, would
require as much of an infrastructural endeavor as a scientific one. The
National Science Foundation had at various times funded armored
airplanes to penetrate storms, but its most recent iteration was plagued
by technological problems, and the project was eventually scuttled; the
interior dimensions of these storms remained essentially unmapped. When
Nesbitt started to think about what else might be able to get him close
enough to the innermost abyss of one of the sierra's superstorms, the
name of one organization came immediately to mind: the Center for Severe
Weather Research.

Founded in the 1990s, by the meteorologist Joshua Wurman, C.S.W.R. is a
seminomadic 11-person research institution that over the years has
earned a reputation for pushing boundaries in chasing technology. In the
mid-90s, Wurman built the first truck-mounted doppler radar system,
nicknamed the ``doppler on wheels,'' or DOW. By 1999, a DOW had recorded
the fastest wind speed in history within a tornado, in Moore, Okla., at
301 m.p.h. Since then, perhaps no other organization has ventured as far
into the world's deadliest tempests as C.S.W.R., whose fleet of four
trucks has now transmitted data from inside 15 hurricanes and about 250
tornadoes. Piloted directly into the path of a storm, the DOWs work as
any other radar does, like atmospheric flashlights: An antenna casts a
conical beam outward, inching upward typically one degree at a time, to
eventually produce a 3-D image of the surrounding storm, like a
spelunker lighting up a cave. Raised off the ground on hydraulic feet,
the trucks are able to scan in winds that might otherwise peel asphalt
off a road.

As technologically advanced as the DOWs are, however, Wurman and his
team are still subject to the mercurial whims of each storm; he likened
the work, at times, to a wildlife biologist scouting the best time and
place for an encounter with a rare species. One of Wurman's most
significant contributions to the field, in fact, happened one night in
Kansas when something went wrong and one of his DOWs was hit by a
tornado, exploding one of its windows. It was one of the best data sets
they'd ever collected. In the sierras, Wurman and Nesbitt didn't know if
they would be so lucky. Given the limited information about conditions
upstream in the Pacific, South Atlantic and Amazon --- which are all
relative blank spots on the weather map --- the chasers were left
somewhat blind downstream. It was a challenge that, while complicated
and potentially dangerous, didn't necessarily faze the seasoned Wurman.
``If we could forecast these storms perfectly,'' he said, ``there'd be
no point in chasing them.''

A few days later, the doldrums finally relented. The forecasters began
to pick up on something promising in the Pacific: For the last several
days, a trough of low-pressure air had been amassing, rolling steadily
eastward toward the Andes. At the same time, humidity levels from
weather balloons in the province indicated a low-level jet stream was
bringing moisture out of the Amazon. On the morning of Dec. 12, the
study forecasters reported that the two systems, along with another
pocket of dry air moving north from Patagonia, seemed poised to converge
over Córdoba sometime in the next few days. By the evening, values of
CAPE and humidity started to spike in ominous ways. With many of the
scientists getting ready to head home, the coming storm would in all
likelihood be the study's last big chase. That evening, as many retired
for the long day ahead, a few drank wine and watched ``Twister.''

In the morning, teams were on the road well before 7 o'clock, headed for
a rural grid of farm roads four or five hours south of Villa Carlos Paz.
The three DOWs stationed themselves at the points of a roughly
1,500-square-mile triangle --- the hope being that their overlapping
scans would form a vast enough atmospheric net to catch the storm. The
remaining six trucks fanned out, positioning to launch weather balloons
and drop off pods: ruggedized weather stations that resemble an
air-conditioning unit. Most parked in dirt pull-offs along irrigation
ditches, or in vacant gravel lots, careful to avoid depressions that
might flood, as well as silos and trees, which might block radars, snag
balloons or splinter into debris. With little to do but wait, the teams
passed the next hours texting photos of clouds and making runs for
gas-station empanadas.

Around 6 p.m., Angela Rowe, an assistant professor at the University of
Wisconsin-Madison who was running the day's operations, radioed from the
ops center that several storms were tracking on a northeast bearing
toward the triangle. Soon those of us who were in the field watched as
the skies before us transformed. Clouds along the leading edge of the
northernmost storm flattened, sending down graying tendrils of haze that
brushed along the ground. Far above, the blackening core of the storm
started bubbling, roiling skyward like an overflowing pot of pasta. The
temperature plummeted and spiked wildly, the air detonating with erratic
blasts of dust and rain. As night fell, lightning began coursing through
the approaching sky, outlining the storm's contorting shape in stenciled
flashes. By 9 p.m., the wind began to pitch team members sideways,
forcing them to dart back and forth between trucks, screaming to be
heard as they wrestled to inflate balloons and place pods.

For the next few hours, as the teams worked to stay ahead of the wind
and hail, all the storms appeared to push steadily northward, as
predicted. But at some point, currents of swollen black clouds overtook
us, rippling outward in every direction. Soon no one could tell exactly
where each storm began or ended, or in what direction they were moving.
Parts of the sky seemed to be eddying in place, flashing a ghostly pale
green, the color of a dirty aquarium; while others appeared to be
streaming back the way we came, pouring rain in steady, even sheets. By
11 p.m., the power in much of the province had gone out, and the sky's
seething black mass had all but collapsed the horizon, making it
impossible to navigate except during the most brilliant flashes. At one
point, we sped away from a tangle of lightning, which lit up the forest
around us in noonday light, only to find another road impassable with
windblown debris, another with standing water.

An hour or so later, we were on an empty four-lane highway, making our
way to another team, when it was suddenly raining and hailing much
harder. The whirling core of the storm appeared to be bearing down on
us: The corkscrewing center had been drawing up millions of pounds of
moisture until, around 30,000 feet, it froze, eventually hurtling back
to earth as mammoth hail. The stones started reporting on the vehicle's
steel frame so loudly they momentarily drowned out the wind in a
concussive drumming. Then another massive downpour erupted, obscuring
even the nearest taillights. It sounded like an airliner and, when it
subsided, a stream of murky water was rushing over the highway. Inching
along, I watched as the blinking shapes of floating cars, like ducks,
were swept into the median and shoulder.

At 1 a.m., the order came to evacuate. One of the support trucks had
already been winched out of a field in the mountains; another's antenna
bent 90 degrees. Over the next four hours, the teams made their way
carefully over roads washed-out and clotted with debris. Downed
electrical wires whipped frantically. A roof lay upside down in a
cornfield. People stood huddled under tollbooth awnings warning of
stones falling from the sky. As we passed over a bridge in Córdoba, the
sky lit up, illuminating a neighborhood heaped with fallen trees.
Further out in the province, a hospital and three schools had been
damaged by a tornado, which also threw two trucks into an outbuilding.
One woman, who was 23 and eight months pregnant, was later reported to
have died in her flooded home. In our vehicle, we hardly spoke. There
was the sense, after witnessing the unforeseen, of the unimaginable
expanding.

\textbf{In the hours} after the storm passed, Nesbitt, Wurman and the
others tried to figure out what they had seen. By the time the last
trucks pulled in, around 5:30 a.m., the storm had raged unabated for
more than six hours. At its peak, it stretched from the Andes to the
Atlantic. Parts of it, now already drifting into Brazil, were so
powerful they'd briefly become self-sustaining, the clouds feeding on
their own heat and moisture --- a destructive phenomenon meteorologists
call ``back-building.'' Local agencies would spend the next few months
trying to assess the extent of the damage, but it appeared to already
include entire neighborhoods across the province. In the hotel, the mood
among the meteorologists, many of whom were in their 24th hour of
monitoring, was delirious. Unable to return to their flooded rooms, a
few retired to the hotel restaurant, where distant lightning fields
stood visible out the windows.

One event in particular drew the meteorologists' attention. For most of
the evening, scans had shown a staggered line of storms marching
steadily northward. Then, around 11:15 or so, something strange flashed
on the satellite feed: a single, bulbous mass, which appeared suddenly,
covering much of the image field. ``This whole huge line just popped
up,'' said Kristen Rasmussen, one of the principal investigators of
RELAMPAGO and an assistant professor at Colorado State University. ``It
could tell us a lot,'' she said. ``It was exactly what we were hoping
for.''

To elaborate, Nesbitt explained that as a storm travels along hot,
saturated ground, its base tends to spread out and flatten, sucking up
all available energy. The more it draws in, the faster and stronger the
vacuum becomes, forming a narrow shaft of rushing air at the center of
the storm, or updraft. An updraft, as Nesbitt went on, is essentially
the storm's piston, drawing heat and moisture in like gas into a
crankshaft, before firing it upward, fueling the storm's growth and
movement. From what the team could gather, each of the storms had
generated such large, powerful updrafts that they'd eventually merged
together and begun to spawn other, smaller updrafts, creating what's
called a ``mesoscale convective system'' --- in short, a giant,
organized complex of perhaps 50 or more updrafts, which becomes
self-sustaining as it germinates more and more offspring. Most M.C.S.s
on the Great Plains take about four or five hours to form; this one,
according to time stamps, materialized in less than 30 minutes.

When Nesbitt and the others began combing through the scans and data,
they found that several of the other storms they'd observed in Argentina
had formed similarly strong updrafts --- many of them as much as 60
percent larger than those in North American storms. One had reached over
69,000 feet, among the tallest ever documented. Others covered more than
15 square miles --- a massive plume of air surging upward at more than
150 m.p.h. Based on the initial DOW scans, Nesbitt could infer that the
scale and strength of the updrafts were a major source of the storms'
violence. As winds within the updrafts began to widen and intensify,
they not only gathered more moisture and heat, feeding the storms'
growth, but also held that volatile mixture aloft, potentially turning
it deadly. Suspended this way, at 30,000 feet or so, for several minutes
or longer, the mixture froze, forming vast fields of tumbling ice
crystals, which, given enough space and time, collided repeatedly,
sparking lightning, or gradually congealing into enormous hailstones.

This finding seemed to suggest that something in the atmosphere was
supercharging updrafts --- wrenching heat and moisture off the ground so
violently that it spun into unusually broad and towering pillars of air.
To Nesbitt, the obvious culprit, at least in theory, was the heat and
moisture itself --- the storm's fuel. As the atmosphere has continued to
warm, lofting ever more moisture into the air, it has also begun to
expand, increasing the air's capacity to absorb ever greater volumes of
moisture, not unlike a gas tank that grows in size as you pump more gas
into it. And because water produces heat as it condenses at altitude,
the added moisture accelerates the process further. Based on the study's
local weather stations --- one of which was erected on the farmer
Lenardon's land --- Nesbitt knew that the atmosphere in the province was
already demonstrating signs of this cycle, including spikes in
evaporative moisture. But as he pointed out, moisture and heat are
merely values of potential energy. They tell us that the sky, like our
drying forests, is rapidly becoming an ocean of fuel, but they don't
tell us where and when it might ignite --- much less what, exactly,
might spark it.

Finding answers to those questions, as Nesbitt saw it, required mapping
updrafts in much more intricate detail. For years, the most prevalent
models used to forecast global weather patterns, he explained, had
relied on relatively simple mathematic calculations --- or
``parameterizations'' --- to predict where and when a storm might form.
Programmed to predict some of the largest and most damaging effects of a
storm, such as wind and rain, the parameters often failed to render the
full complexity of a storm's development, including the formation of its
updraft, resulting in a loss of overall accuracy. ``Now we're having to
go back,'' said Nesbitt, ``and try to add some additional realism to the
calculations, so they can represent the full stages of a storm's life
cycle.''

By the time RELAMPAGO left Argentina, the study had collected nearly 100
terabytes of data from 19 separate chases. To begin the process of
improving how storms are represented in models, the scientists would
first have to create a profile of each storm they studied, along with
all its minute microphysical features, digging through millions of
points of data to separate out the effects of the landscape and natural
fluctuations of weather from those features that might be unique to the
storm. What the work amounted to was the rough meteorological equivalent
of the parable of the blind men and the elephant: By July 2020, some 20
papers were in various stages of publication, each of them offering
insights into different aspects of Córdoba's storms. Ultimately, by
looking at them in aggregate, the goal for Nesbitt would be to isolate
what amounted to a fingerprint from a few molecules of air --- air that,
heated by the sun and bonded with evaporation, became the first
disastrous breaths of an updraft.

Already, a simple version of RELAMPAGO's model had helped Servicio
Meteorológico Nacional open the predictive window in the Córdoba
Province by roughly 48 hours, Nesbitt says. Eventually, he hoped a
higher-resolution version could provide similar warnings throughout the
warming world --- especially in the United States, where air conditions
are poised to resemble those in the province in the next few decades.
But for now, he contented himself with having provided families like
Lenardon's a few more hours of readiness --- though he wondered how long
it would be until these models were rendered, once more, obsolete.

\textbf{One day shortly} before the end of the study, the meteorologists
took me into the foothills of Villa Carlos Paz to visit a woman named
Maria Natividad Garay, who had in her possession what may be one of the
largest hailstones ever recovered. Her residence, which lay wedged
between an apartment complex and repair shop, included a modest ranch
home as well as several apartments and guesthouses, a few of which were
rented to Argentine meteorologists affiliated with the study. When we
arrived, Garay was sitting out back in a chair, her door left slightly
ajar to the cooling breeze.

Garay is a carefully spoken woman in her mid-50s, with short brown hair
and the mild, composed smile of someone long conversant with the
punctuated boredom of life on the plains. Asked about the storm that
produced the hail, she called up the precise date --- Feb. 8, 2018 ---
and told me that the storm had lasted exactly 15 minutes; it was etched
in her mind. She had lived in the area for nearly 30 years now, she
explained, and though the region was known for storms, that was merely a
thing people knew. ``You have to experience it firsthand,'' she said.

She pointed out several long scars on the building next door, places
where whole columns of bricks had been peeled away. ``That was the first
thing I saw,'' she said; ``hail was hitting the wall sideways.'' The
next instant, her skylights shattered, ice pouring into the house. The
noise was incredible, she said, like a train coming through your yard
--- thin and distant at first, then roaring overtop of you. After the
deluge stopped, she peered outside to find the yard blanketed in what
looked like shards of milky glass. ``It didn't rain at all until the
hail stopped,'' she said, still surprised by the observation a year
later. The meteorologists guessed this was why the stone had been so
remarkably well preserved.

She held it before us. It was spherical and nearly the size of a
grapefruit. She'd kept it wrapped in a Ziploc bag at the rear of her
freezer. She couldn't say why, exactly, only that it had struck her as
an object worthy of preservation. Its frightening size and appearance,
buried there in her yard --- it seemed of unearthly provenance. She
leaned in and showed us the many thousands of crystals spidering through
the stone, some of which were already beginning to fracture and melt in
her hand.

But then again, she continued, it was just air and water. It was, in
other words, composed of the same things we breathe.

\hypertarget{the-great-climate-migrationthe-teenagers-at-the-end-of-the-worlddestroying-a-way-of-life-to-save-louisianathe-fearsome-thunderstorms-of-cuxf3rdoba-provincelearning-from-the-kariba-dam}{%
\paragraph{\texorpdfstring{\href{https://www.nytimes.com/interactive/2020/07/23/magazine/climate-migration.html}{The
Great Climate
Migration}\href{https://www.nytimes.com/interactive/2020/07/21/magazine/teenage-activist-climate-change.html}{The
Teenagers at the End of the
World}\href{https://www.nytimes.com/interactive/2020/07/21/magazine/louisiana-coast-engineering.html}{Destroying
a Way of Life to Save
Louisiana}\href{https://www.nytimes.com/interactive/2020/07/22/magazine/worst-storms-argentina.html}{The
Fearsome Thunderstorms of Córdoba
Province}\href{https://www.nytimes.com/interactive/2020/07/22/magazine/zambia-kariba-dam.html}{Learning
From the Kariba
Dam}}{The Great Climate MigrationThe Teenagers at the End of the WorldDestroying a Way of Life to Save LouisianaThe Fearsome Thunderstorms of Córdoba ProvinceLearning From the Kariba Dam}}\label{the-great-climate-migrationthe-teenagers-at-the-end-of-the-worlddestroying-a-way-of-life-to-save-louisianathe-fearsome-thunderstorms-of-cuxf3rdoba-provincelearning-from-the-kariba-dam}}

\begin{center}\rule{0.5\linewidth}{\linethickness}\end{center}

Noah Gallagher Shannon is a writer from Northern Colorado who now lives
in New York. His last feature for the magazine was on the Pinkertons.

The Climate Issue

\begin{itemize}
\tightlist
\item
  \href{https://www.nytimes.com/interactive/2020/07/23/magazine/climate-migration.html}{The
  Great Climate Migration}
\item
  The Teenagers at the End of the World
\item
  \href{https://www.nytimes.com/interactive/2020/07/21/magazine/louisiana-coast-engineering.html}{Destroying
  a Way of Life to Save Louisiana}
\item
  \href{https://www.nytimes.com/interactive/2020/07/22/magazine/zambia-kariba-dam.html}{Learning
  From the Kariba Dam}
\item
  \href{https://www.nytimes.com/interactive/2020/07/22/magazine/worst-storms-argentina.html}{What's
  Going on Inside the Fearsome Thunderstorms of Córdoba Province?}
\end{itemize}

\protect\hyperlink{}{} \protect\hyperlink{}{}

\includegraphics{https://static01.nyt.com/newsgraphics/2020/07/26/climate/fbd0a5f2a975dc16f0d1a24d64f70f4d843e50a3/caret.svg}

Read 211 Comments

\begin{itemize}
\item
\item
\item
\item
\end{itemize}

Advertisement

\protect\hyperlink{after-bottom}{Continue reading the main story}

\hypertarget{site-index}{%
\subsection{Site Index}\label{site-index}}

\hypertarget{site-information-navigation}{%
\subsection{Site Information
Navigation}\label{site-information-navigation}}

\begin{itemize}
\tightlist
\item
  \href{https://help.nytimes.com/hc/en-us/articles/115014792127-Copyright-notice}{©~2020~The
  New York Times Company}
\end{itemize}

\begin{itemize}
\tightlist
\item
  \href{https://www.nytco.com/}{NYTCo}
\item
  \href{https://help.nytimes.com/hc/en-us/articles/115015385887-Contact-Us}{Contact
  Us}
\item
  \href{https://www.nytco.com/careers/}{Work with us}
\item
  \href{https://nytmediakit.com/}{Advertise}
\item
  \href{http://www.tbrandstudio.com/}{T Brand Studio}
\item
  \href{https://www.nytimes.com/privacy/cookie-policy\#how-do-i-manage-trackers}{Your
  Ad Choices}
\item
  \href{https://www.nytimes.com/privacy}{Privacy}
\item
  \href{https://help.nytimes.com/hc/en-us/articles/115014893428-Terms-of-service}{Terms
  of Service}
\item
  \href{https://help.nytimes.com/hc/en-us/articles/115014893968-Terms-of-sale}{Terms
  of Sale}
\item
  \href{https://spiderbites.nytimes.com}{Site Map}
\item
  \href{https://help.nytimes.com/hc/en-us}{Help}
\item
  \href{https://www.nytimes.com/subscription?campaignId=37WXW}{Subscriptions}
\end{itemize}
