Sections

SEARCH

\protect\hyperlink{site-content}{Skip to
content}\protect\hyperlink{site-index}{Skip to site index}

\hypertarget{comments}{%
\subsection{\texorpdfstring{\protect\hyperlink{commentsContainer}{Comments}}{Comments}}\label{comments}}

\href{}{Ben \& Jerry's Radical Ice Cream Dreams}\href{}{Skip to
Comments}

The comments section is closed. To submit a letter to the editor for
publication, write to
\href{mailto:letters@nytimes.com}{\nolinkurl{letters@nytimes.com}}.

Talk

\hypertarget{ben--jerrys-radical-ice-cream-dreams}{%
\section{Ben \& Jerry's Radical Ice Cream
Dreams}\label{ben--jerrys-radical-ice-cream-dreams}}

By David MarcheseJuly 27, 2020

\begin{itemize}
\item
\item
\item
\item
\item
  \emph{291}
\end{itemize}

``There wasn't any other business talking about dismantling white
supremacy.''

\includegraphics{https://static01.nyt.com/newsgraphics/2020/07/09/talk/a28b9c874788a9e2b60f2798087577317201a315/close.svg}

\textbf{Talk} July 27, 2020

\hypertarget{ben--jerrys-radical-ice-cream-dreams-1}{%
\section{Ben \& Jerry's Radical Ice Cream
Dreams}\label{ben--jerrys-radical-ice-cream-dreams-1}}

By David Marchese

SHARE

It has now become trendy for corporations to take a stance on
social-justice issues without fear of hurting the bottom line. In fact,
it's widely seen as a way to do the opposite, and that has a lot to do
with groundwork laid by Ben Cohen and Jerry Greenfield, the founders of
Ben \& Jerry's Homemade ice cream. The two childhood friends started out
making ice cream in 1978, opening their first shop in a remodeled gas
station in Burlington, Vt., and turned their business into one known
around the world --- both for its distinct chunk-filled product as well
as for its determination to be a force for good via community
involvement, environmentally sustainable practices and the creation of a
still-lucrative alternative to the simple pursuit of profit. Though
Cohen and Greenfield, both 69, long ago ceded control of the company,
they're still involved in promoting its social-justice campaigns, and
their spirit is still found in the company's progressive practices.
(Maybe you saw its
\href{https://www.benjerry.com/about-us/media-center/dismantle-white-supremacy}{corporate
statement}, released in response to George Floyd's killing, about
dismantling white supremacy.) ``Using ice cream to talk about difficult
issues creates an opening,'' Cohen said. ``You can talk in a way that's
tinged with lightness, which makes it much more palatable.'' Which is
something he and Greenfield know a little bit about.

\textbf{If you look at the history of Ben \& Jerry's, you always had
noble goals: You wanted to keep control of the company. You wanted the
highest-paid employee to never make more than five times that of the
lowest-paid employee. You didn't want to keep growing for the sake of
growth. And yet you had to sell the company, that pay ratio wasn't
maintained and the company just}
\textbf{\href{http://nytimes.com\#tooltip-1}{kept growing.}}
\textbf{What does that show us about how capitalism can subsume good
intentions?} \emph{Cohen:} The end result of capitalism is not unlike a
Monopoly game. One guy gets all the bucks, and everybody else loses.
What we have in America is a democracy that's run for the benefit of
corporations. That's a disaster. We're looking at it, we're living it
and it continues to get worse. Does that answer your question?

\textbf{It didn't really, but I can't tell if you went off on a tangent
or were being evasive.} \emph{Cohen:} I'm not trying to evade. I was
going off on a tangent. There's no doubt that Ben \& Jerry's has
influenced capitalism more than capitalism influenced Ben \& Jerry's.
Ben \& Jerry's and a few others --- The Body Shop, Patagonia --- were
pioneers in creating a model of business that saw making profits to be
coequal with its purpose of improving society beyond just providing
jobs. There's a bunch of corporations genuinely starting to see the
light. Granted, capitalism subsumed the concept of socially responsible
business. Every major corporation now has a corporate social
responsibility officer. The biggest problem in terms of Ben \& Jerry's
being subsumed is that if you see the major problem in our society as
being the continuing concentration of wealth into fewer hands, that we
ended up getting owned by a \href{http://nytimes.com\#tooltip-2}{huge
multinational} works against what I believe is needed to create a more
equitable society. But with that exception, the company continues to do
as much as it can to heal the wounds of capitalism.

\textbf{Is there anything that makes you squeamish about Ben \& Jerry's
making ice cream flavors called Pecan Resist, which is a reference to
resisting certain Trump administration's policies, or Justice ReMix'd,
whose name alludes to the company's work in criminal-justice reform?
Coming up with politically driven flavor names was not something you did
much of when you were running things. Isn't there something glib about
it?} \emph{Greenfield}: It doesn't make me squeamish if the initiative
is genuine. If you talk about Justice ReMix'd, the flavor is there to
call attention to the issue of criminal-justice reform, and the
activities the company has done --- one of them was closing down the
Workhouse jail in St. Louis. \emph{Cohen:}
\href{http://nytimes.com\#tooltip-3}{We won!} \emph{Greenfield:} Right,
and another was changing the budget in the school system in Miami and
hiring counselors instead of police officers.
\emph{Cohen:}\href{http://nytimes.com\#tooltip-4}{We half won!}
\emph{Greenfield:} When those flavors are part of real action that the
company is undertaking in partnership with nonprofits, I think it's
great to be tying ice cream into social action. \emph{Cohen:} You know,
the company once came out with a flavor called
\href{http://nytimes.com\#tooltip-5}{American Pie.} The packaging showed
the pie chart of the federal discretionary budget; it was advocating
shifting money out of nuclear weapons into children's services.
\emph{Greenfield:} A pie chart of the federal discretionary budget is a
well-known marketing technique for selling ice cream.

\textbf{I'm salivating just thinking about it. On the subject of flavors
---} \emph{Greenfield:} I want to interrupt you for a second. Ben and I
are always talking about the mission of the company, and people always
want to talk about flavors. People are \emph{fascinated} by flavors.

\textbf{So am I. What would go in a Joe Biden flavor, and what would go
in a Donald Trump flavor?} \emph{Cohen:} A Trump flavor, it's not
palatable. You can't make Trump into ice cream. You could make him into
coal.

\textbf{What about Biden?} \emph{Cohen:} It's an interesting question.
{[}Sighs.{]} You know, it'd be better than nothing.

\textbf{Ben, I know you're}
\textbf{\href{http://nytimes.com\#tooltip-6}{a Bernie Sanders guy,}}
\textbf{but what would it take to get you excited about Biden? As a
presidential candidate, not as a flavor of ice cream.} \emph{Cohen:} If
he would essentially adopt Bernie's platform. They talk about the
Hillary wing of the party and the Bernie wing of the party. Biden
epitomizes the Hillary wing, the wing in which he went to a group of
big-money donors and said: \href{http://nytimes.com\#tooltip-7}{``Elect
me. I guarantee nothing will change for you.''} We could go back to a
\emph{pre}-Trump country, and we would still need all the change
represented by Bernie's platform. Going back to a pre-Trump country will
not address systemic problems that our country faces in terms of
fairness, equality, and justice.

\textbf{Now I have an ice cream etiquette question. You know how some
people dig the chunks out of Ben \& Jerry's?} \emph{Cohen:} Marriages
have split because of that.

\textbf{What's your position? I think it's selfish.} \emph{Cohen:} If
your partner also likes the chunks, it's inconsiderate. But if it's
yourself who's doing it, it's fine. \emph{Greenfield:} The term for this
is ``mining.'' Mining for chunks. I've never been tempted to do it. I
don't see the point. Although recently Ben \& Jerry's started selling
chocolate-chip-cookie-dough pellets separately from ice cream for those
people who wanted to dig them out. \emph{Cohen:} Who wanted to mainline.
\emph{Greenfield:} \emph{Mainline}? No, Ben.

\textbf{And why did Ben \& Jerry's never sell gallons?} \textbf{Would
they have been prohibitively expensive?} \emph{Cohen:} Yes, the expense.
The other reason is that as ice cream hangs out in your modern
self-defrosting freezer, it degrades. If it's in a small package, you
finish it quicker, and there's less chance of it degrading in quality.
\emph{Greenfield:} There's an opposing theory, though, Ben. It's that
the more ice cream people have in their freezer, the more they'll
actually eat. But that only matters if one were concerned about selling
more ice cream, which Ben and I are not anymore. Now, there was a time
when Ben and I were absolutely trying to sell ice cream. We were out
there on the road hawking it. \emph{Cohen:} It was like an adventure in
the wilderness. \emph{Greenfield:} We have incredible memories of going
to restaurants that were going out of business, and some auction company
was selling off their old stuff, and we were bidding on things and
loading them up in a truck and driving them home. That's what I remember
more fondly than any business things.

\textbf{Is it right that before you guys started in ice cream, you had
some goofy idea for a business involving delivering bagels and lox and
The New York Times to people?} \emph{Greenfield:} Come on, man. What's
so goofy about that? \emph{Cohen:} We were calling that business U.B.S.,
United Bagel Service. But we wanted to locate our business in a rural
college town, because that's the kind of place where
\href{http://nytimes.com\#tooltip-8}{Jerry and I wanted to live.}
Eventually we realized that there weren't that many people in rural
college towns looking to have the Sunday New York Times and bagels,
cream cheese and lox delivered to their door.

\textbf{Maybe I'm too uptight about money, but it's surprising to me
that you've managed to stay such good friends after being in business
together all these years. Money was never an issue?} \emph{Greenfield:}
I don't think we ever had a disagreement about money. The most famous
disagreement was about the size of the chunks in the ice cream. Ben is
well known for his inability to smell and therefore his inability to
taste. So he was always focused on texture in ice cream. He liked big
chunks of cookies and candies. But I was the one making the ice cream,
and it's hard to put big chunks in ice cream, which is why no other ice
cream companies do it. I was advocating that a larger number of smaller
chunks be well distributed throughout the ice cream. Ben was insisting
on bigger chunks. Ben was right. \emph{Cohen:} I was eating Coffee
Toffee Bar Crunch last night and was tunneling around for the big
chunks.

\textbf{I believe the term is mining.} \emph{Greenfield:} Tunneling
works also. \emph{Cohen:} Tunneling works great. But, you know, it's
disappointing when you keep tunneling around and you never run into what
you're aiming for. I still think we ought to put a golden cone inside
some pint --- do the Willy Wonka golden-ticket thing. I can't understand
how we have yet to do that. \emph{Greenfield:} Ben, in case you hadn't
figured this out, did all the marketing for the company.

\textbf{Jerry, there has been a proliferation of other premium and}
\textbf{\href{http://nytimes.com\#tooltip-9}{ultra-premium}} \textbf{ice
cream brands. There's Ample Hills. There's Van Leeuwen. There's Jeni's
Splendid. And a result is that it's not uncommon to go into a grocery
store and see a \$9 pint of ice cream. What's your perspective on that
change in the market?} \emph{Greenfield:} It's kind of crazy for me to
say this, but it seems like a lot of money to pay for a pint of ice
cream. Ben and I remember when Ben \& Jerry's pints of ice cream first
started going over \$2 a pint. We were terrified that nobody was going
to buy it. \emph{Cohen:} There's a bunch of artisanal guys now, and one
of the great things about the ice cream business, which is one of the
reasons we got into it, is that there's a very low barrier to entry. The
equipment to make very high quality ice cream on a small scale is not
very expensive.

\textbf{Do either of you have non-Ben \& Jerry's ice creams in your
freezers?} \emph{Cohen:} No. \emph{Greenfield:} No. But if any of those
other ice cream companies wanted to give me some ice cream, I would be
all for it.

\textbf{To get back to a couple of bigger ideas ---} \emph{Greenfield:}
The other thing I want to mention is that Ben and I are sometimes asked,
``Why has Ben \& Jerry's been successful?'' We usually say it's because
of three things: really high quality ice cream, great ingredients, very
unusual flavors -- and also the activist social mission of the company.
Some other company \emph{could} start making ice cream with big chunks
the same way Ben \& Jerry's does, but Ben \& Jerry's having this
activist, outspoken social mission --- other companies can't copy that.
It's not something you can just say. It has to be who the people are.

\textbf{How close of a connection do you feel to Ben \& Jerry's today?}
\emph{Greenfield:} You may know that Ben and I both still work at the
company. But as we always tell people, we're not involved in management
or operations. I'm proud of the mission of the company and how it's
being actualized. Sometimes people ask me, ``How do you feel seeing your
name on ice cream containers in stores everywhere?'' I don't feel
anything from that. \emph{Cohen:} It's like the company is a child who
has moved out of the house and is now on their own. You hope that your
child will have the values that you tried to instill. I'm amazed to see
that the values are there. The regret I have is that the overwhelming
problem in the world is the increasing concentration of wealth in the
hands of fewer entities, and having Ben \& Jerry's owned by one of those
is, to me, unfortunate. When the company was sold, something I resisted,
there were people trying to comfort me by saying, ``Now Ben \& Jerry's
can influence Unilever.'' I thought that was a bunch of {[}expletive{]}.
But I think that it \emph{has} had a positive influence on Unilever. I
certainly wouldn't say Unilever is values-led, but it is starting to
integrate more \href{http://nytimes.com\#tooltip-10}{social benefits
into how it does business.} That's good.

\textbf{If you two are not in operations and not in management, what
exactly do you do? Are you mascots?} \emph{Cohen:} We have no
responsibilities and no authority, but no, I don't regard myself as a
mascot. I regard myself as a person
\href{http://nytimes.com\#tooltip-11}{who is focused on justice.} When
Ben \& Jerry's does something that aligns with my belief in justice, I
do everything I can to support that. **** \emph{Greenfield:} When the
company introduced Justice ReMix'd, Ben and I were involved in that.
Last year, the company introduced a flavor at a United Nations forum in
Geneva that was called Cone Together that was related to refugee rights.
Ben and I were involved in that, too. We also go to the
\href{http://nytimes.com\#tooltip-12}{franchise meeting} every year and
hang out with the shop owners and talk about our hopes for the company.
Even though we don't really influence things, people like to hear what
we're thinking. So, David, can I ask you a question?

\textbf{Of course.} \emph{Greenfield:} You've done a lot of reading
about Ben \& Jerry's. Is this what you were expecting? Where have we let
you down?

\textbf{You haven't let me down. But I wonder if there's more you could
be saying about what Ben \& Jerry's being bought by Unilever ultimately
meant for the values you originally tried to instill in the company.}
\emph{Greenfield:} Well, so, Ben \& Jerry's has been part of Unilever
for about 20 years. For the first number of years, I think Unilever did
not appreciate the mission of Ben \& Jerry's, and its energy went into
integrating Ben \& Jerry's into the Unilever system. During that time,
the social mission of the company suffered. The company as a brand also
suffered. About ten years ago, Unilever named
\href{http://nytimes.com\#tooltip-13}{a new chief executive for Ben \&
Jerry's, Jostein Solheim,} who told us that his assignment was to
re-radicalize Ben \& Jerry's. And during that time, Ben \& Jerry's
rediscovered its soul. Ben \& Jerry's publicly supported Occupy Wall
Street. Ben \& Jerry's publicly supported Black Lives Matter before most
other companies. Now within Unilever, there's an incredible amount of
respect for what Ben \& Jerry's has done. I mean, this last statement by
Ben \& Jerry's after the George Floyd killing: There wasn't any other
business talking about dismantling white supremacy.

\textbf{How skeptical, though, should we be of the intentions behind
statements like that? So many socially progressive statements that
companies are making these days obviously also double as marketing.}
\emph{Cohen:} The deal about Ben \& Jerry's is that when your company is
acting on its values and those values resonate with your consumers'
values, it's an incredibly deep connection based on justice, fairness,
equality --- the stuff that we thought the country is supposed to be
about when they taught us in elementary school. The other thing is that
businesses are the most powerful force in our society, and things have
gotten to such a state with Trumpism that businesses --- which had
always said, ``We're not going to take political stands'' --- have to
make their voice heard because there's no other powerful actor doing it.
Money talks.

\textbf{Do you ever meet people who are surprised that Ben and Jerry are
real guys? And that you're them? And that you're still alive?}
\emph{Cohen:} Maybe two years after we started, when the business was
this little homemade ice cream parlor in an old gas station in
Burlington, Jerry and I were hanging around outside the store, and a boy
and his father were walking in. The little boy said, ``Daddy, is there
really a Ben and a Jerry?'' And the father said, ``Maybe many, many
years ago.'' \emph{Greenfield:} I've had people ask me, ``Are you the
original Jerry?'' I say: ``There used to be another Jerry. I got hired
to be the next one.'' My wife gets a kick out of that.

\textbf{After all this time in and around the ice cream business, what
have you learned about what ice cream means to Americans?} \emph{Cohen:}
It's about happiness. Ice cream is present at most any celebration,
birthday, wedding, bar mitzvah. And Americans stock ice cream in their
freezers as a staple. That is very unusual compared with other
countries. Around the world, a huge amount of ice cream is sold in
single-unit servings.

\textbf{Do you guys ever get sick of ice cream? I worked in an ice cream
shop one summer when I was a teenager, and it put me off ice cream for a
solid year.} \emph{Greenfield:} You were in the industry! You've been
holding back on us! But no, I never had that. I tended to eat ice cream
more recreationally than Ben. Ben was in charge of quality control,
which meant eating a lot of ice cream. Once we started packaging ice
cream into pints, Ben felt that he had to eat all the way to the bottom.
Any ice cream flavor tastes good for the first couple of spoonfuls. The
real test is how it tastes when you get down to the bottom.
\emph{Cohen:} Yeah, I was sick of it. But now that I'm no longer eating
it because my job requires me to, I don't get sick of it. I eat a
reasonable amount. Every once in a while I go overboard.
\emph{Greenfield:} We both still eat a \emph{lot} of ice cream.

\begin{center}\rule{0.5\linewidth}{\linethickness}\end{center}

\emph{This interview has been edited and condensed for clarity from
three conversations.}

Read 291 Comments

\begin{itemize}
\item
\item
\item
\item
\end{itemize}

Advertisement

\protect\hyperlink{after-bottom}{Continue reading the main story}

\hypertarget{site-index}{%
\subsection{Site Index}\label{site-index}}

\hypertarget{site-information-navigation}{%
\subsection{Site Information
Navigation}\label{site-information-navigation}}

\begin{itemize}
\tightlist
\item
  \href{https://help.nytimes.com/hc/en-us/articles/115014792127-Copyright-notice}{©~2020~The
  New York Times Company}
\end{itemize}

\begin{itemize}
\tightlist
\item
  \href{https://www.nytco.com/}{NYTCo}
\item
  \href{https://help.nytimes.com/hc/en-us/articles/115015385887-Contact-Us}{Contact
  Us}
\item
  \href{https://www.nytco.com/careers/}{Work with us}
\item
  \href{https://nytmediakit.com/}{Advertise}
\item
  \href{http://www.tbrandstudio.com/}{T Brand Studio}
\item
  \href{https://www.nytimes.com/privacy/cookie-policy\#how-do-i-manage-trackers}{Your
  Ad Choices}
\item
  \href{https://www.nytimes.com/privacy}{Privacy}
\item
  \href{https://help.nytimes.com/hc/en-us/articles/115014893428-Terms-of-service}{Terms
  of Service}
\item
  \href{https://help.nytimes.com/hc/en-us/articles/115014893968-Terms-of-sale}{Terms
  of Sale}
\item
  \href{https://spiderbites.nytimes.com}{Site Map}
\item
  \href{https://help.nytimes.com/hc/en-us}{Help}
\item
  \href{https://www.nytimes.com/subscription?campaignId=37WXW}{Subscriptions}
\end{itemize}
