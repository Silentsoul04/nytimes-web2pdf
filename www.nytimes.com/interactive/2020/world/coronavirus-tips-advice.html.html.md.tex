Sections

SEARCH

\protect\hyperlink{site-content}{Skip to
content}\protect\hyperlink{site-index}{Skip to site index}

\hypertarget{answers-to-your-new-coronavirus-questions}{%
\section{Answers to Your New Coronavirus
Questions}\label{answers-to-your-new-coronavirus-questions}}

By The New York TimesUpdated July 27, 2020

\begin{itemize}
\item
\item
\item
\item
\end{itemize}

When the coronavirus pandemic began, we were figuring out how to stay
safe, how to stay home and how this was going to affect our lives in the
long run. Now that we have better answers to many of these questions,
new ones have surfaced. How can I stay healthy while interacting with
others? How can I prevent the spread of this disease? What can I do if
I've lost a job? How can I explain this to my kids? Simply, how can I
live in this new normal?

\hypertarget{what-do-you-want-to-know-about-coronavirus}{%
\section{What do you want to know about
coronavirus?}\label{what-do-you-want-to-know-about-coronavirus}}

\hypertarget{do-you-have-a-question-about-the-coronavirus-or-life-during-this-pandemic-try-this-experimental-feature-that-uses-machine-learning-to-help-answer-your-questions-if-we-dont-have-an-answer-available-we-will-do-our-best-to-address-it-in-the-future}{%
\subsection{\texorpdfstring{Do you have a question about the coronavirus
or life during this pandemic? Try this experimental feature that uses
\href{https://www.nytimes.com/2018/11/18/technology/artificial-intelligence-language.html?searchResultPosition=4}{machine
learning} to help answer your questions. If we don't have an answer
available, we will do our best to address it in the
future.}{Do you have a question about the coronavirus or life during this pandemic? Try this experimental feature that uses machine learning to help answer your questions. If we don't have an answer available, we will do our best to address it in the future.}}\label{do-you-have-a-question-about-the-coronavirus-or-life-during-this-pandemic-try-this-experimental-feature-that-uses-machine-learning-to-help-answer-your-questions-if-we-dont-have-an-answer-available-we-will-do-our-best-to-address-it-in-the-future}}

ASK

\hypertarget{four-big-questions}{%
\subsection{Four Big Questions}\label{four-big-questions}}

\hypertarget{where-are-we-in-developing-a-vaccine}{%
\subsubsection{Where are we in developing a
vaccine?}\label{where-are-we-in-developing-a-vaccine}}

+

The race to develop a vaccine for the coronavirus is, perhaps, one of
the most important efforts in the contemporary history of science and
medicine. In some labs across the world, there is now a cautious
optimism that there will be a vaccine for the coronavirus sometime next
year.

As of July 28, there were 25 vaccines under clinical evaluation, and 139
others in early evaluation stages, according to
\href{https://www.who.int/publications/m/item/draft-landscape-of-covid-19-candidate-vaccines}{a
list} released by the World Health Organization. Scientists are working
overtime and
\href{https://www.nytimes.com/interactive/2020/05/20/science/coronavirus-vaccine-development.html}{trying
many different methods} to prevent the disease.

An experimental
\href{https://www.nytimes.com/interactive/2020/science/coronavirus-vaccine-tracker.html}{coronavirus
vaccine} made by the biotech company Moderna provoked a
\href{https://www.nytimes.com/2020/07/14/health/cornavirus-vaccine-moderna.html}{promising
immune response} against the virus and appeared safe in the first 45
people who received it, researchers reported in
\href{https://www.nejm.org/coronavirus?query=RP}{The New England Journal
of Medicine} in mid-July.

Governments and major pharmaceutical companies are also wading in. The
United States government, as part of a project it's calling
\href{https://www.nytimes.com/2020/06/03/us/politics/coronavirus-vaccine-trump-moderna.html}{``Operation
Warp Speed,''} is on a quest to have a vaccine by the end of the year.

Still, there is little consensus in the scientific community whether
such accelerated timelines --- the end of this year, the start of next
--- are feasible. Even after a vaccine is researched and tested,
manufacturing vaccines is complex. The process typically requires large,
sterilized vats and the mobilization of hundreds of pharmaceutical
factories. The record for
\href{https://www.nytimes.com/interactive/2020/06/09/magazine/covid-vaccine.html}{developing
a vaccine is four years for the mumps vaccine,} and a decade is not an
unusual timeline.

But, optimists point out, there has also never been a time when so many
experts in so many countries
\href{https://www.nytimes.com/2020/04/01/world/europe/coronavirus-science-research-cooperation.html}{focused
simultaneously on a single topic} and with such urgency. That, alone,
gives some researchers
\href{https://www.nytimes.com/2020/05/20/health/coronavirus-vaccines.html}{hope.}

 Read More
\href{https://www.nytimes.com/interactive/2020/science/coronavirus-vaccine-tracker.html}{Coronavirus
Vaccine Tracker}

\hypertarget{how-can-i-help}{%
\subsubsection{How can I help?}\label{how-can-i-help}}

+

Great question! We're glad you asked.

There's a full section on helping others further down this list, with
specific suggestions for donating blood, helping neighbors and
reallocating resources.

But the most helpful thing that you, an individual, can do is wear a
mask and maintain six feet of physical distance from other people. We
know it's a pain, and we know a mask can be sweaty and uncomfortable.

Frankly, though, it's the simplest thing you can do to help. And no one
is really exempt. Researchers believe that
\href{https://www.nytimes.com/2020/06/27/world/europe/coronavirus-spread-asymptomatic.html}{symptomless
transmission} is a factor in the spread of the virus. You might feel
fine, but you could be infecting people around you.

\href{https://www.nytimes.com/2020/06/02/health/coronavirus-face-masks-surveys.html}{Not
yet convinced}? We'll keep going. A mask is
\href{https://www.cdc.gov/coronavirus/2019-ncov/prevent-getting-sick/cloth-face-cover-guidance.html}{proven
to be effective} in slowing the virus's spread. That also makes it a way
to kickstart the sagging economy again; a
\href{https://www.forbes.com/sites/sarahhansen/2020/06/30/a-national-mask-mandate-could-save-the-us-economy-1-trillion-goldman-sachs-says/\#1083f1e56f18}{Goldman
Sachs analysis} found that if all Americans were required to wear masks
in public --- as people are in dozens of other countries --- it could
save the U.S. economy \$1 trillion. That's trillion, with a T.

A minor discomfort and inconvenience could save someone's life. In fact,
it could save quite a few lives.

Our colleagues
\href{https://www.nytimes.com/2020/07/02/opinion/coronavirus-masks.html}{Jesse
Wegman} and
\href{https://www.nytimes.com/2020/07/01/opinion/coronavirus-face-masks.html?action=click\&module=RelatedLinks\&pgtype=Article}{Nicholas
Kristof} in the Opinion section offered their arguments for mask
wearing.

 Read More
\href{https://www.nytimes.com/interactive/2020/06/25/burst/how-to-get-the-most-out-of-your-mask.html}{Tips
for Making Your Mask Work}

\hypertarget{i-have-antibodies-am-i-now-immune}{%
\subsubsection{I have antibodies. Am I now
immune?}\label{i-have-antibodies-am-i-now-immune}}

+

As of right now, that seems likely, for at least several months.

There have been frightening accounts of people suffering what seems to
be a second bout of Covid-19. But experts say these patients may have a
drawn-out course of infection, with the virus taking a slow toll weeks
to months after initial exposure.

People infected with the coronavirus typically
\href{https://www.nature.com/articles/s41586-020-2456-9}{produce immune
molecules} called antibodies, which are
\href{https://www.nytimes.com/2020/05/07/health/coronavirus-antibody-prevalence.html}{protective
proteins made in response to an infection.} These antibodies may last in
the body \href{https://www.nature.com/articles/s41591-020-0965-6}{only
two to three months,} which may seem worrisome, but that's perfectly
normal after an acute infection subsides, said Dr. Michael Mina, an
immunologist at Harvard University.

It may be possible to get the coronavirus again, but it's highly
unlikely that it would be possible in a short window of time from
initial infection or make people sicker the second time.

 Read More
\href{https://www.nytimes.com/2020/07/22/health/covid-antibodies-herd-immunity.html}{You
May Have Antibodies After Coronavirus Infection. But Not for Long.}

\hypertarget{what-will-the-world-look-like-in-a-few-months}{%
\subsubsection{What will the world look like in a few
months?}\label{what-will-the-world-look-like-in-a-few-months}}

+

Here's the short answer: Everything is moving quickly. The only things
we can say with confidence are that nothing is certain, and that many
more Americans are likely to die before this is over.

There are a few big questions to follow.

First, will the virus return in full force with the cold weather? There
has been some speculation as to whether
\href{https://www.nytimes.com/2020/03/22/health/warm-weather-coronavirus.html}{warm
weather} has had an effect on slowing the virus. So far, we do know that
you are safer outside, provided you wear a mask and stay far apart from
other people. When winter comes, it will be harder to spend time
outside, especially if you are in a cold climate. If you do need to move
inside, only spend time in places with
\href{https://www.nytimes.com/2020/07/06/health/coronavirus-airborne-aerosols.html}{good
ventilation}, as the virus seems to
\href{https://www.nytimes.com/2020/07/04/health/239-experts-with-one-big-claim-the-coronavirus-is-airborne.html}{hang
in the air} for hours. But be prepared in case of another spike in cases
where you live.

Second, what's the deal with schools? We have a section on this below,
but expect significant disruption. Some elementary through high school
students will be in classrooms
\href{https://www.nytimes.com/2020/06/26/us/coronavirus-schools-reopen-fall.html}{part-time}
or
\href{https://www.nytimes.com/2020/07/13/us/lausd-san-diego-school-reopening.html}{not
at all.} Others might be going back to school like normal, but they
could be in greater danger of sudden outbreaks and shutdowns.

Some colleges,
\href{https://www.nytimes.com/2020/05/12/us/cal-state-online-classes.html}{including
California State University}, have announced that they will reopen in
the fall, but with most classes moved online. Others are trying to limit
the term length. Some professors are
\href{https://www.nytimes.com/2020/07/03/us/coronavirus-college-professors.html}{rebelling
against in-person teaching}. Everything is basically on a ``pencil it
in, pending an outbreak'' basis.

And third, what about work? Some non-essential workers are already back
in offices and others will soon be returning to plexiglass dividers and
alternating schedules, even though cases continue to rise. For many
companies, however, the return-to-office date keeps getting pushed out
later and later. We have more information about this below, but
\href{https://www.cdc.gov/coronavirus/2019-ncov/community/guidance-business-response.html}{as
an employee}, you have
\href{https://www.osha.gov/SLTC/covid-19/standards.html}{some rights}.
That choice might be made for you, though.

 Read More
\href{https://www.nytimes.com/2020/04/18/health/coronavirus-america-future.html}{The
Coronavirus in America: The Year Ahead}

\hypertarget{science-and-health}{%
\subsection{Science and Health}\label{science-and-health}}

\hypertarget{what-is-the-difference-between-the-coronavirus-and-covid-19}{%
\subsubsection{What is the difference between the coronavirus and
Covid-19?}\label{what-is-the-difference-between-the-coronavirus-and-covid-19}}

+

A coronavirus is a type of virus. There are lots of common
coronaviruses, which are
\href{https://www.cdc.gov/coronavirus/general-information.html}{basically
just forms of a cold.}

This coronavirus --- the coronavirus --- is novel. It causes Covid-19, a
disease. The 19 signifies 2019, the year when it started. The two terms
are used interchangeably in conversation, but actually mean slightly
different things.

 Read More
\href{https://www.nytimes.com/2020/03/18/us/coronavirus-terms-glossary.html}{From
Flattening the Curve to Pandemic: A Coronavirus Glossary}

\hypertarget{i-keep-hearing-about-droplets-aerosol-clouds-and-surface-contamination-what-does-this-thing-attach-to-how-does-it-get-around-and-how-would-i-get-it}{%
\subsubsection{I keep hearing about droplets, aerosol clouds and surface
contamination. What does this thing attach to, how does it get around
and how would I get
it?}\label{i-keep-hearing-about-droplets-aerosol-clouds-and-surface-contamination-what-does-this-thing-attach-to-how-does-it-get-around-and-how-would-i-get-it}}

+

If you were to share saliva with someone --- kiss, drink from the same
cup --- you could obviously get the disease. But it's not just direct
contact that can cause transmission. After pressure from researchers,
the W.H.O. conceded that the virus can be transmitted through
\href{https://www.nytimes.com/2020/07/09/health/virus-aerosols-who.html}{particles
that can linger in the air} for hours after an infected person has left.

The nastiest recent revelation is, perhaps, that
\href{https://www.nytimes.com/2020/06/16/health/coronavirus-toilets-flushing.html}{flushing
a toilet} can also spread the virus throughout a bathroom.

There is some good news, though. After months of sanitizing our
groceries, packages and mail,
\href{https://www.nytimes.com/2020/05/22/health/cdc-coronavirus-touching-surfaces.html}{the
Centers for Disease Control and Prevention said} that the risk of
catching the coronavirus from a surface is low.

 Read More
\href{https://www.nytimes.com/2020/07/06/health/coronavirus-airborne-aerosols.html}{Airborne
Coronavirus: What You Should Do Now}

\hypertarget{what-does-the-virus-do-to-my-body}{%
\subsubsection{What does the virus do to my
body?}\label{what-does-the-virus-do-to-my-body}}

+

In the beginning, the coronavirus seemed like it was primarily a
respiratory illness --- many patients had fever and chills, were weak
and tired, and coughed a lot. Those who seemed sickest had pneumonia or
acute respiratory distress syndrome --- which caused their blood oxygen
levels to plummet --- and received supplemental oxygen. In severe cases,
they were placed on ventilators to help them breathe.

By now, doctors have identified many more symptoms and syndromes. (And
some people don't show many symptoms at all.) In April,
\href{https://www.nytimes.com/2020/04/27/health/coronavirus-symptoms-cdc.html}{the
C.D.C. added to the list of early signs} sore throat, fever, chills and
muscle aches. Gastrointestinal upset, such as diarrhea and nausea, has
also been observed.

Another telltale sign of infection may be a sudden, profound diminution
of one's
\href{https://www.nytimes.com/2020/03/22/health/coronavirus-symptoms-smell-taste.html}{sense
of smell and taste.} Teenagers and young adults in some cases have
developed painful red and purple lesions on their fingers and toes ---
nicknamed ``Covid toe'' --- but few other serious symptoms.

More serious cases can lead to inflammation and organ damage, even
without difficulty breathing. There have been cases of dangerous blood
clots, strokes and brain impairments.

 Read More
\href{https://www.nytimes.com/article/coronavirus-facts-history.html\#link-6817bab5}{Six
Months of Coronavirus: Here's Some of What We've Learned}

\hypertarget{what-are-my-chances-of-getting-sick}{%
\subsubsection{What are my chances of getting
sick?}\label{what-are-my-chances-of-getting-sick}}

+

It's a moving target, and depends a lot on where you live and when. If
you're somewhere with an outbreak, you're at higher risk. If your state
is reopening and cases are climbing, your risk is highest.

Lots of factors determine how sick you actually get. Younger people tend
to do better than older people, and having underlying conditions
increases your risk. And if you were
\href{https://www.nytimes.com/2020/05/29/health/coronavirus-transmission-dose.html}{exposed
to a lot of the virus} ---
\href{https://www.nytimes.com/2020/04/24/magazine/surgeon-covid-diary.html}{as
many essential workers were} --- you have a higher risk of getting a
severe case.

Black and Latino people have been
\href{https://www.nytimes.com/interactive/2020/07/05/us/coronavirus-latinos-african-americans-cdc-data.html}{disproportionately
affected} by the coronavirus in a widespread manner that spans the
country, throughout hundreds of counties in urban, suburban and rural
areas, and across all age groups.

However, a better question might be what your chances are of getting
someone else sick. People in their 20s, 30s and 40s account for a
growing proportion of the cases in many places, raising fears that
\href{https://www.nytimes.com/2020/06/25/us/coronavirus-cases-young-people.html}{asymptomatic
young people} are helping to fuel the virus's spread.

Most people have a low chance of getting a severe case. But if you're
not careful, you have a very high chance of infecting someone who is
vulnerable. It's sort of like drunken driving. You could harm yourself,
but you could also harm other people, too. That's why we don't do it.

 Read More
\href{https://www.nytimes.com/2020/05/20/us/coronavirus-reopening-50-states.html}{All
50 States Are Now Reopening. But at What Cost?}

\hypertarget{if-i-am-sick-how-will-i-know}{%
\subsubsection{If I am sick, how will I
know?}\label{if-i-am-sick-how-will-i-know}}

+

If you've been exposed to the coronavirus or think you have, and have a
fever or symptoms like a cough or difficulty breathing, call a doctor.
They should give you advice on how to get tested and how to seek medical
treatment without potentially infecting or exposing others.

The Centers for Disease Control and Prevention says that if you're sick
or think you're sick, but only mildly ill, you should
\href{https://www.cdc.gov/coronavirus/2019-ncov/about/steps-when-sick.html}{isolate
yourself}, and you shouldn't leave your house except to go to the
doctor.

But a lot of people get the coronavirus without showing many, or any,
symptoms. That's why you should always wear a mask outside your home ---
the only surefire way to know you don't have it is to get a test. And
tests are still not reliable.

 Read More
\href{https://www.nytimes.com/article/symptoms-coronavirus.html}{`What
Are the Symptoms?' `What Should I Do if I Feel Sick?' and Other
Coronavirus Questions}

\hypertarget{how-do-i-get-tested-how-long-do-tests-take-to-come-back}{%
\subsubsection{How do I get tested? How long do tests take to come
back?}\label{how-do-i-get-tested-how-long-do-tests-take-to-come-back}}

+

The situation is constantly changing, and each case is unique.

One thing is consistent
\href{https://www.nytimes.com/interactive/2020/us/coronavirus-testing.html}{across
the country}, though: Testing is a mess.

As of July 28, the number of daily coronavirus tests being conducted in
the United States is only 38 percent of the level considered necessary
to mitigate the spread of the virus. People take a test, only to get
results days, a week or more later. That's too long for most people,
especially those who cannot quarantine even if they wanted to. (Other
countries are doing a lot better.)

Why? Well.
\href{https://www.nytimes.com/2020/07/13/upshot/coronavirus-response-fax-machines.html}{Fax
machines} are a surprising bottleneck.
\href{https://www.nytimes.com/2020/05/11/technology/coronavirus-worker-testing-privacy.html}{Temperature
checks} don't say much, because lots of people have the coronavirus
without a fever. And even once you get results,
\href{https://www.nytimes.com/2020/07/13/well/live/coronavirus-smart-testing.html}{they
may not be reliable}.

If you want to get tested, prepare for a
\href{https://www.nytimes.com/2020/04/13/nyregion/coronavirus-testing.html}{sore
nose} and a long wait time.

 Read More
\href{https://www.nytimes.com/article/test-for-coronavirus.html}{The
Experience of Getting Tested for Coronavirus}

\hypertarget{when-should-i-go-to-the-hospital}{%
\subsubsection{When should I go to the
hospital?}\label{when-should-i-go-to-the-hospital}}

+

Going to the hospital should not be your first interaction with a
doctor.

If you're feeling sick, speak with a medical professional over the phone
and get a test. If you can, consult with a doctor before you go to a
medical center or the hospital --- they might have insights or
suggestions.

As soon as symptoms start, mark the days. While most patients recover in
about a week, a significant minority of patients enter ``a very nasty
second wave'' of illness, said Dr. Ilan Schwartz, assistant professor of
infectious disease at the University of Alberta. ``After the initial
symptoms, things plateau and maybe even improve a little bit, and then
there is a secondary worsening.''

 Read More
\href{https://www.nytimes.com/2020/04/30/well/live/coronavirus-days-5-through-10.html}{Why
Days 5 to 10 Are So Important When You Have Coronavirus}

\hypertarget{preventative-measures-and-treatment}{%
\subsection{Preventative Measures and
Treatment}\label{preventative-measures-and-treatment}}

\hypertarget{why-do-masks-work}{%
\subsubsection{Why do masks work?}\label{why-do-masks-work}}

+

The coronavirus clings to wetness and enters and exits the body through
any wet tissue (your mouth, your eyes, the inside of your nose). That's
why people are wearing masks and eyeshields. They're like an umbrella
for your body; they keep your droplets in and other people's droplets
out.

But masks only work if you are wearing them properly. The mask should
cover your face from the bridge of your nose to under your chin, and
should stretch almost to your ears. Be sure there are no gaps --- that
sort of defeats the purpose, no?

 Read More
\href{https://www.nytimes.com/interactive/2020/06/25/burst/how-to-get-the-most-out-of-your-mask.html}{Tips
for Making Your Mask Work}

\hypertarget{what-should-i-consider-when-choosing-a-mask}{%
\subsubsection{What should I consider when choosing a
mask?}\label{what-should-i-consider-when-choosing-a-mask}}

+

There are masks everywhere. There are N95s and respirators, homemade
masks and bandanas. Respirator valves should be avoided, but they do
look cool. Even
\href{https://www.etonline.com/kim-kardashians-skims-face-masks-are-back-in-stock-shop-before-they-sell-out-again-146705}{Kim
Kardashian} has waded in.

There are a few basic things to consider when you're wearing a mask.

Does it have at least two layers? Good.

If you hold it up to the light, can you see through it? Bad.

Can you blow a candle out through your mask? Bad.

Do you feel mostly OK wearing it for hours at a time? Good.

The most important thing, after finding a mask that fits well without
gapping, is to find a mask that you will wear.

Every brand in the history of clothing, it seems, has
\href{https://www.nytimes.com/2020/04/22/fashion/coronavirus-fashion-face-masks.html}{entered
the mask market}. Spend some time picking out your mask, and find
something that works with your personal style. You should be wearing it
whenever you're out in public for the foreseeable future.

 Read More
\href{https://www.nytimes.com/article/coronavirus-homemade-mask-material-DIY-face-mask-ppe.html}{What's
the Best Material for a Mask?}

\hypertarget{what-about-gloves-should-i-wear-them-too}{%
\subsubsection{What about gloves? Should I wear them
too?}\label{what-about-gloves-should-i-wear-them-too}}

+

Not really. Gloves become a second skin. They themselves could become
contaminated (don't touch your face!) and depending on how they're made,
they might have holes.

Gloves may be helpful if someone in your household becomes sick, so you
can reduce the amount of times you have to wash your hands. But you will
have to change the gloves every time you leave their room or interact
with the sick person.

The most effective intervention is still washing your hands thoroughly,
for at least 20 seconds, every time you enter your home.

 Read More
\href{https://www.nytimes.com/article/coronavirus-myths.html}{Don't Fall
for These Myths About Coronavirus}

\hypertarget{why-is-six-feet-away-the-right-distance}{%
\subsubsection{Why is six feet away the right
distance?}\label{why-is-six-feet-away-the-right-distance}}

+

The coronavirus spreads primarily through droplets from your mouth and
nose, especially when you cough or sneeze. Six feet has never been a
magic number that guarantees complete protection. The C.D.C., one of the
organizations using that measure, bases its recommendation on the idea
that most large droplets that people expel when they cough or sneeze
will fall to the ground within six feet.

But some scientists have looked at studies of air flow and are concerned
about smaller particles called aerosols. They suggest that people
consider a number of factors, including their own vulnerability and
whether they are outdoors or in an enclosed room, when deciding whether
six feet is enough distance.

Sneezes, for instance, can launch droplets a lot farther than six feet,
\href{https://jamanetwork.com/journals/jama/fullarticle/2763852}{according
to a recent study}.

It's a rule of thumb: You should be safest standing six feet apart
outside, especially when it's windy. But keep a mask on at all times,
even when you think you're far enough apart.

 Read More
\href{https://www.nytimes.com/2020/04/14/health/coronavirus-six-feet.html}{Stay
6 Feet Apart, We're Told. But How Far Can Air Carry Coronavirus?}

\hypertarget{does-the-virus-live-in-clothes-and-hair}{%
\subsubsection{Does the virus live in clothes and
hair?}\label{does-the-virus-live-in-clothes-and-hair}}

+

Probably not. If you are practicing social distancing and making only
occasional trips to the grocery store or pharmacy, experts say that it's
not necessary to change clothes or take a shower when you return home.
You should, however, always wash your hands upon entering your home.

The same advice goes for
\href{https://www.nytimes.com/2020/04/21/smarter-living/maybe-consider-shaving-that-pandemic-beard.html}{head
and facial hair}: If you practice social distancing and wash your hands
frequently, you probably don't need to worry.

 Read More
\href{https://www.nytimes.com/2020/04/17/well/live/coronavirus-contagion-spead-clothes-shoes-hair-newspaper-packages-mail-infectious.html}{Is
the Virus on My Clothes? My Shoes? My Hair? My Newspaper?}

\hypertarget{president-trump-talked-about-sunlight-and-ingesting-disinfectants-helping-to-kill-this-coronavirus-whats-the-basis-of-that}{%
\subsubsection{President Trump talked about sunlight and ingesting
disinfectants helping to kill this coronavirus. What's the basis of
that?}\label{president-trump-talked-about-sunlight-and-ingesting-disinfectants-helping-to-kill-this-coronavirus-whats-the-basis-of-that}}

+

There is no evidence that sunlight can cure coronavirus on the human
body. Same for disinfectant and bleach. So, do not drink bleach. Do not
inject disinfectant. And do not believe there is some cure for
coronavirus coming from ultraviolet light.

Some forms of ultraviolet light have
\href{https://www.nytimes.com/2020/03/20/health/coronavirus-masks-reuse.html}{killed
the virus on surfaces}, and some businesses are using the
\href{https://www.nytimes.com/2020/05/07/science/ultraviolet-light-coronavirus.html}{purple
glow as a disinfectant} for surfaces, to good effect. But surfaces are
not the human body, and sunlight or concentrated ultraviolet light will
not un-infect you if you are sick.

 Read More
\href{https://www.nytimes.com/article/coronavirus-disinfectant-inject-ingest.html}{Please
Do Not Eat Disinfectant}

\hypertarget{can-hydroxychloroquine-treat-the-virus}{%
\subsubsection{Can hydroxychloroquine treat the
virus?}\label{can-hydroxychloroquine-treat-the-virus}}

+

There has been a lot of talk about hydroxychloroquine,
\href{https://www.nytimes.com/article/hydroxychloroquine-coronavirus.html}{a
drug used to treat malaria and autoimmune diseases} like rheumatoid
arthritis and lupus, but a National Institutes of Health panel has
specifically advised against using the combination of hydroxychloroquine
and the antibiotic azithromycin outside of clinical trials.

A
\href{https://www.medrxiv.org/content/10.1101/2020.04.16.20065920v2}{study
of the records of 368 patients in the Veterans Affairs system}, which
has yet to be peer-reviewed, found that hydroxychloroquine, with or
without azithromycin, did not help patients avoid the need for
ventilators. And hydroxychloroquine alone was associated with an
increased risk of death.

The study was not a controlled trial, and patients who received the
drugs were sicker to begin with. The authors wrote: ``These findings
highlight the importance of awaiting the results of ongoing prospective,
randomized, controlled studies before widespread adoption of these
drugs.''

 Read More
\href{https://www.nytimes.com/2020/04/21/health/nih-covid-19-treatment.html}{New
U.S. Treatment Guidelines for Covid-19 Don't See Much Progress}

\hypertarget{i-have-the-coronavirus-what-should-i-expect-my-recovery-to-look-like}{%
\subsubsection{I have the coronavirus. What should I expect my recovery
to look
like?}\label{i-have-the-coronavirus-what-should-i-expect-my-recovery-to-look-like}}

+

First off, we hope you get better soon.

Recovery is sort of a case-by-case basis. Some people show none or few
symptoms, and only find out they have been infected when they test
positive for antibodies. Other people are left with
\href{https://www.nytimes.com/2020/05/10/world/europe/coronavirus-italy-recovery.html}{continuing
shortness of breath, muscle weakness, flashbacks, mental fogginess and
other symptoms months after they first got sick}. And there's
\href{https://www.nytimes.com/2020/03/04/us/coronavirus-recovery.html}{a
range of experiences} in between.

Mara Gay, a member of The Times's editorial board,
\href{https://www.nytimes.com/2020/05/14/opinion/coronavirus-young-people.html}{wrote
candidly} about her experience in recovery.

``I am one of the lucky ones. I never needed a ventilator. I survived.
But 27 days later, I still have lingering pneumonia. I use two inhalers,
twice a day. I can't walk more than a few blocks without stopping,'' she
wrote.

Many people recover, but it may take a while. Go easy on yourself, and
be forgiving if your body isn't performing the way it used to. You
survived a pandemic. That alone is cause for celebration.

 Read More
\href{https://www.nytimes.com/2020/07/01/health/coronavirus-recovery-survivors.html}{Here's
What Recovery From Covid-19 Looks Like for Many Survivors}

\hypertarget{how-i-can-help}{%
\subsection{How I Can Help}\label{how-i-can-help}}

\hypertarget{what-can-i-do-to-stop-the-spread}{%
\subsubsection{What can I do to stop the
spread?}\label{what-can-i-do-to-stop-the-spread}}

+

Wear a mask and stay far away from other people. That's it. That's the
tweet.

 Read More
\href{https://www.nytimes.com/2020/04/08/well/live/coronavirus-face-mask-mistakes.html}{How
NOT to Wear a Mask}

\hypertarget{can-i-give-blood}{%
\subsubsection{Can I give blood?}\label{can-i-give-blood}}

+

You can. Transfusions are
\href{https://docs.google.com/document/d/1Ep_tZsoeCO6fLGDmD58CetFzaKk1uUwkAEoa_CQ1p0U/edit?ts=5f0dc4a0}{still
needed} for cases like organ transplants or complications of childbirth.

The American Red Cross is collecting donations at blood banks, which
have enacted new safety measures to prevent the spread of the
coronavirus. Those measures include checking the temperatures of staff
members and donors before they enter a drive facility, providing hand
sanitizer for use before and during the donation process, enhancing
their disinfection of surfaces and equipment, and spacing beds --- when
possible --- to enable social distancing between donors.

And even if you're under a stay-at-home order, donating blood is an
essential need, so public health officials have made an exception for
your trip to the center.

To find a donation center, check the American Association of Blood Banks
\href{http://www.aabb.org/tm/donation/Pages/Blood-Bank-Locator.aspx}{locator,}
visit the Red Cross \href{https://www.redcrossblood.org/}{website} or
call 1-800-RED-CROSS.

 Read More
\href{https://www.nytimes.com/2020/03/19/well/live/coronavirus-blood-donation.html}{How
to Donate Blood During the Crisis}

\hypertarget{ive-recovered-from-covid-19-how-do-i-donate-plasma}{%
\subsubsection{I've recovered from Covid-19. How do I donate
plasma?}\label{ive-recovered-from-covid-19-how-do-i-donate-plasma}}

+

Plasma is, basically, the liquid part of your blood: it's light yellow
and about 92 percent water. Any antibodies that your body creates are
contained in the plasma. Once you've recovered, or convalesced, from a
given virus, those antibodies stick around in your plasma for a certain
amount of time, ready to fight that virus if it comes back.

If you have antibodies, your convalescent plasma can be transfused into
a patient still battling the disease.

To qualify, donors must pass normal blood-donation requirements and be
symptom-free of Covid-19 for at least 14 days, and, in most cases, must
have positive results from a test. (Other restrictions may apply,
depending on the organization.)

Many health care institutions nationwide are involved in plasma
donations, including the Red Cross, so to find a location near you go to
the website for the \href{https://ccpp19.org/}{National Covid-19
Convalescent Plasma Project} or visit the
\href{https://www.redcrossblood.org/donate-blood/dlp/plasma-donations-from-recovered-covid-19-patients.html}{Red
Cross's website}.

 Read More
\href{https://www.nytimes.com/2020/04/24/smarter-living/coronavirus-convalescent-plasma-antibodies.html}{What
Is Convalescent Plasma, and Why Do We Care About It?}

\hypertarget{where-should-i-donate-money}{%
\subsubsection{Where should I donate
money?}\label{where-should-i-donate-money}}

+

If you are looking to give money to those who need it most, you may want
to consider an organization that provides food or helps with medical
efforts.

Be sure to take care when choosing a charity. The watchdogs
\href{https://www.charitynavigator.org/index.cfm?bay=content.view\&cpid=7779\&mod=article_inline}{Charity
Navigator} or
\href{https://www.charitywatch.org/charity-donating-articles/coronavirus-outbreak?mod=article_inline}{CharityWatch}
vet organizations.

You can also consider giving to local businesses and families in need
directly. Or helping your neighbors in ways that are
\href{https://www.nytimes.com/2020/03/15/smarter-living/wirecutter/5-ways-to-help-during-coronavirus-while-social-distancing.html}{not
necessarily monetary}.

 Read More
\href{https://www.nytimes.com/article/coronavirus-how-to-help-donations-charities.html}{How
You Can Help Victims of the Coronavirus Pandemic}

\hypertarget{reopening}{%
\subsection{Reopening}\label{reopening}}

\hypertarget{my-state-is-reopening-is-it-safe-to-return-to-life-as-normal}{%
\subsubsection{My state is reopening. Is it safe to return to life as
normal?}\label{my-state-is-reopening-is-it-safe-to-return-to-life-as-normal}}

+

Please practice caution no matter where you live. At this time, it is
just not safe to jump back into the world with two feet, especially if
you are in an area where cases are surging.

And in many states, that's the situation. Many re-openings have been
followed by
\href{https://www.nytimes.com/interactive/2020/07/09/us/coronavirus-cases-reopening-trends.html}{a
surge of new cases.}

Now a number of states are
\href{https://www.nytimes.com/2020/06/25/us/texas-coronavirus-cases-reopening-Greg-Abbott.html}{pausing
plans to re-open} or even reimposing restrictions they had lifted
earlier.

When you do visit businesses or are otherwise out and about, maintain
your social distance and keep your mask on. Just because Americans are
bored of the pandemic, it doesn't mean it's over.

 Read More
\href{https://www.nytimes.com/interactive/2020/us/states-reopen-map-coronavirus.html}{See
How All 50 States Are Reopening (and Closing Again)}

\hypertarget{what-businesses-are-safe-to-frequent}{%
\subsubsection{What businesses are safe to
frequent?}\label{what-businesses-are-safe-to-frequent}}

+

This is a hard question to answer. Different phases of reopening target
different businesses --- retail stores, restaurants and bars, salons and
barber shops, houses of worship, gyms, etc.

The increasing moves to lift restrictions --- or at least to open up
outdoor spaces like beaches and state parks --- reflect the immense
political and societal pressures weighing on the nation's governors,
even as epidemiologists remain cautious and warn of a second wave of
cases.

In general, think of businesses in terms of temptation. Where are people
going to feel most at ease, and where will they most want to get close
to one another and remove their masks? At a store, you can sort of get
in and get out.

But at businesses that exist for socialization, it's a lot harder. Bars
can be loud, and alcohol is flowing. Gyms are inherently germy, and
working out in a mask is hard. Outdoor dining in restaurants seems
feasible, but can be tricky: Tables are close, you're eating so you
cannot keep your mask on, and waiters keep dropping by to check up on
you.

Consider your comfort level with visiting a particular business before
you head out. There may be ways to ease your anxieties, such as calling
in advance to ask about crowds or cleaning procedures.

 Read More
\href{https://www.nytimes.com/interactive/2020/06/03/burst/coronavirus-risk-gym-surfaces-bike.html}{A
Quick Coronavirus Risk Assessment}

\hypertarget{how-should-i-decide-where-and-how-to-go-to-stores}{%
\subsubsection{How should I decide where and how to go to
stores?}\label{how-should-i-decide-where-and-how-to-go-to-stores}}

+

This is a hard question, especially because a stagnant economy also
damages the lives and livelihoods of unemployed people. No one wants
another recession, and spending money at
\href{https://www.nytimes.com/2020/06/18/business/small-business-reopening-coronavirus.html}{local
businesses} is a good way to re-grease the wheels. But no one wants to
get sick or contribute in any way to a second wave either.

\href{https://www.nytimes.com/2020/04/08/dining/takeout-restaurant-ethics-coronavirus.html}{Weigh
the gains against the losses} --- not just for you, but for everyone.
You can find your way back into a new kind of normal if you approach
reopening thoughtfully and apply ethical guidelines to your behavior.

Before you plan an errand, ask yourself
\href{https://www.nytimes.com/interactive/2020/05/06/opinion/coronavirus-us-reopen.html}{a
few questions}.

1. Is this necessary? If you're going to a store for food or supplies,
the answer is yes. But if you are dealing with your emotional health and
the health of your relationships by seeing the people you love, that
answer might also be yes.

2. Is this necessary now? Again, this might also be a yes. But separate
needs from wants. You might want a haircut. Do you need one? You want to
go sit in a restaurant. Must you?

3. Can I do this thing in a safer way? This requires some creative
thinking, but the answer is also, probably, yes. You might be missing
your religious community, and your house of worship is back open. Is
there a way to move the service outside, instead of sitting in the
crowded room? You're desperate to see your friends. And you should ---
mental health matters. But can you see them in their backyard, instead
of their living room?

 Read More
\href{https://www.nytimes.com/2020/04/10/magazine/coronavirus-economy-debate.html}{Restarting
America Means People Will Die. So When Do We Do It?}

\hypertarget{what-about-getting-a-haircut}{%
\subsubsection{What about getting a
haircut?}\label{what-about-getting-a-haircut}}

+

You could, if only for the story,
\href{https://www.nytimes.com/wirecutter/reviews/how-to-cut-your-own-hair/}{do
it yourself}. A
\href{https://www.nytimes.com/2020/04/15/style/self-care/buzz-cut-your-own-hair.html}{buzz
cut} is \ldots{} possible, although perhaps ill-advised. And the
\href{https://www.nytimes.com/2020/04/21/smarter-living/maybe-consider-shaving-that-pandemic-beard.html}{beard}?
Chances are that it's
\href{https://www.nytimes.com/2020/07/03/at-home/coronavirus-beards.html}{not
nearly as cute as you think it is}.

But fixing a 'do is one of the most beloved types of self-care there is.
Even when you're visible only on Zoom, split ends and straggly edges can
crimp your confidence. When barbershops reopened in New York City,
people came
\href{https://www.nytimes.com/2020/06/25/nyregion/nyc-barber-shops-coronavirus.html}{rushing
back}.

If you go to
\href{https://www.nytimes.com/2020/06/12/fashion/haircut-salon-reopening.html}{a
salon}, make sure you and your hairdresser or barber are both wearing
masks. They have to get close to you, so
\href{https://www.nytimes.com/2020/05/06/style/coronavirus-haircuts-barbers.html}{be
extra careful}. Also, unless you're getting your hair done outside, make
sure they have fans running and windows open, to keep air moving through
the space.

To keep yourself safe, shampoo your hair at home, so your stylist
doesn't have to bend down over you. Wait for the appointment outside and
keep the haircut time short. Skip the blowdry.

That's true for getting your hair colored, too. Step outside while the
chemicals are taking, and wash your own hair out.

And, ah, masks. But what about the hair behind my ears? That's an easy
fix. Either settle for a not-so-close shave this time, or unloop one ear
at a time, holding your mask to your face. Safety first. Vanity second.

 Read More
\href{https://www.nytimes.com/interactive/2020/06/17/burst/5-swipes-for-a-low-risk-salon-visit.html}{5
Swipes for a Low-Risk Salon Visit}

\hypertarget{what-about-the-dentist}{%
\subsubsection{What about the dentist?}\label{what-about-the-dentist}}

+

Taking care of your pearly whites is not just a good beauty habit; it's
also essential for your health. With all the baking and drinking we've
been doing during lockdown, you're going to need a visit sooner rather
than later. In fact, that's part of the reason economists are using
dentists as a
\href{https://www.nytimes.com/2020/06/10/upshot/dentists-coronavirus-economic-indicator.html}{bellwether
for comfort about reopening writ large}.

\href{https://www.nytimes.com/2020/06/08/well/live/dental-care-dentistry-teeth-coronavirus.html}{Dentists
have mobilized} and are ready to accommodate visitors. (Telehealth, for
obvious reasons, isn't an option for most dental cases.) Dentists and
hygienists should wear head-to-toe personal protective equipment and
change between appointments.

Getting the first appointment of the day may also limit risk, though
many dentists said they are seeing fewer patients so they have more time
to disinfect rooms between visits. But the risk might be more to the
providers than it is to the patients; they keep their mask on while you
take yours off.

Before you go, speak with your dentist about how crucial your check-up
is right now. If cases are rising where you live, maybe check back in
next month if your dental issue can wait.

 Read More
\href{https://www.nytimes.com/2020/06/25/health/dentist-coronavirus-safe.html}{Is
It Safe to Go Back to the Dentist?}

\hypertarget{what-about-laundromats}{%
\subsubsection{What about laundromats?}\label{what-about-laundromats}}

+

In confined spaces, droplets spread. The
\href{https://www.cdc.gov/coronavirus/2019-ncov/prevent-getting-sick/cleaning-disinfection.html}{C.D.C.
suggests that}, while doing the laundry of a sick person, you should
wear gloves and try not to shake the clothing, to minimize the
possibility of dispersing the virus through the air. If possible, wash
items using the warmest water setting, and dry thoroughly. But you can
wash their laundry with everyone else's.

That being said,
\href{https://www.nytimes.com/2020/05/22/health/cdc-coronavirus-touching-surfaces.html}{the
C.D.C. also says} that surfaces contaminated with droplets of the virus
can infect people, it is ``not thought to be the main way the virus
spreads.'' We know that's contradictory information. Safe is always
better than sorry, though, and if you're living with someone sick, it's
always good to take precautions.

At the laundromat, if you leave your home to wash your clothes, don't
hang out in the room between cycles. Sit in your car or lounge outside
as you wait, just for social distancing purposes.

 Read More
\href{https://www.nytimes.com/2020/03/26/style/how-to-do-laundry-coronavirus.html}{How
Should I Do Laundry Now?}

\hypertarget{how-about-gyms}{%
\subsubsection{How about gyms?}\label{how-about-gyms}}

+

By their very nature, athletic facilities like gyms tend to be germy,
and gym equipment can be difficult to sanitize. In
\href{https://pubmed.ncbi.nlm.nih.gov/31660785/}{a study published
earlier this year}, researchers found drug-resistant bacteria, flu virus
and other pathogens on about 25 percent of the surfaces they tested in
four different athletic training facilities.

If your gym is open and you plan to go, disinfect any surfaces that you
touch. Wash your hands frequently, or use hand sanitizer. When spraying
a disinfectant,
\href{https://www.nytimes.com/2020/05/06/well/live/coronavirus-cleaning-cleaners-disinfectants-home.html}{give
it a fews minutes to kill germs} before wiping. Clean any grime or dust
off surfaces first. Once you're done with that machine, disinfect it
again for the next person.

But, it's summer. Maybe just exercise outside instead? Or, there are
many exercise classes streaming online. Set up a corner of your home,
lay out a yoga mat, and stream those instead.

 Read More
\href{https://www.nytimes.com/2020/05/13/well/move/coronavirus-gym-safety.html}{Is
It Safe to Go Back to the Gym?}

\hypertarget{socializing-and-friends}{%
\subsection{Socializing and Friends}\label{socializing-and-friends}}

\hypertarget{why-is-it-safer-to-spend-time-together-outside}{%
\subsubsection{Why is it safer to spend time together
outside?}\label{why-is-it-safer-to-spend-time-together-outside}}

+

\href{https://www.nytimes.com/2020/05/15/us/coronavirus-what-to-do-outside.html}{Outdoor
gatherings} lower risk because wind disperses viral droplets, and
sunlight can kill some of the virus. Open spaces prevent the virus from
building up in concentrated amounts and being inhaled, which can happen
when infected people exhale in a confined space for long stretches of
time, said Dr. Julian W. Tang, a virologist at the University of
Leicester.

And, since it's summer, outside is fun.
\href{https://www.nytimes.com/2020/05/09/dining/coronavirus-how-to-have-a-picnic-safely.html}{Picnics}
are a great way to share a meal with friends --- just bring separate
blankets and utensils, and sit far enough apart. If you don't want to
cook,
\href{https://www.nytimes.com/2020/05/27/dining/takeout-delivery-safety-coronavirus.html}{takeout
and delivery are safe}, and will support the local economy. Getting
around can be tough without a car, but there are ways to
\href{https://www.nytimes.com/2020/06/08/nyregion/mta-subway-riding-health-coronavirus.html}{protect
yourself on public transportation}, if you need to take it.

 Read More
\href{https://www.nytimes.com/2020/07/03/well/live/coronavirus-spread-outdoors-party.html}{How
Safe Are Outdoor Gatherings?}

\hypertarget{what-are-superspreaders}{%
\subsubsection{What are superspreaders?}\label{what-are-superspreaders}}

+

A superspreader is someone who is really good at passing on the
coronavirus to others.

The good news is: they're rare.

But superspreaders exist. There was that
\href{https://apnews.com/d9a6ca7eef083315648003509d07515a}{Texas
birthday party}, where one man reportedly infected 17 members of his
family. What about
\href{https://www.nytimes.com/2020/04/12/us/coronavirus-biogen-boston-superspreader.html}{Biogen
employees}, who infected people after a healthcare conference? And there
was that
\href{https://www.nytimes.com/2020/03/23/us/coronavirus-westport-connecticut-party-zero.html}{Connecticut
soirée}, dubbed ``Party Zero.''

Now researchers are trying to figure out why so few people spread the
virus to so many. For the most part, they're trying to answer three
questions: Who are the superspreaders? When does superspreading take
place? And where?

The answer to these questions is, for the most part,
\href{https://www.nytimes.com/2020/06/30/science/how-coronavirus-spreads.html}{unknown}.
As with everything coronavirus-related, assume the worst and be
hypervigilant.

 Read More
\href{https://www.nytimes.com/2020/06/30/science/how-coronavirus-spreads.html}{Most
People With Coronavirus Won't Spread It. Why Do a Few Infect Many?}

\hypertarget{how-should-i-throw-a-party-now-that-reopening-is-possible}{%
\subsubsection{How should I throw a party now that reopening is
possible?}\label{how-should-i-throw-a-party-now-that-reopening-is-possible}}

+

Start with: Should you throw a party? Is it necessary? Now? And is a
party the best way to celebrate together?

Again, the answers to these questions may very well be: yes. An elderly
relative has a big birthday. You are getting married (more on this
later).

So. Invite a few people that you trust. Will they keep their masks on?
And, if they forget, will they be open to gentle reminders?

As for food and activities, keep it outside. Suggest that everyone bring
their own blanket, food and utensils. A ``bring your own dinner'' is
better than communal food, but if you go up one at a time to serve
yourselves, you should be OK. You can even
\href{https://www.nytimes.com/2020/06/06/at-home/coronavirus-cool-off-even-without-a-deep-end.html}{splash
around}, albeit at a safe distance.

If these restrictions still give you anxiety,
\href{https://www.nytimes.com/2020/05/02/smarter-living/zoom-birthday-party.html}{try
a party on Zoom}. It's honestly pretty fun.

 Read More
\href{https://www.nytimes.com/2020/05/09/dining/coronavirus-how-to-have-a-picnic-safely.html}{How
to Have a Picnic, Safely}

\hypertarget{what-if-i-have-to-be-indoors-in-a-small-space-like-an-elevator}{%
\subsubsection{What if I have to be indoors in a small space, like an
elevator?}\label{what-if-i-have-to-be-indoors-in-a-small-space-like-an-elevator}}

+

Are there stairs? Can you climb them? If not, or it's too many flights,
wait until the next car with no one in it.

Wait until you can ride the elevator on your own. But since droplets can
hang in the air long after you leave, wear your mask at all times.

And, no talking. Keeping your mouth shut lowers the risk of you spewing
coronavirus droplets into the contained space.

 Read More
\href{https://www.nytimes.com/2020/06/26/health/coronavirus-elevator-reopen.html}{Going
Up? Not So Fast: Strict New Rules to Govern Elevator Culture}

\hypertarget{i-actually--like--not-seeing-people-help}{%
\subsubsection{I actually \ldots{} like \ldots{} not seeing people.
Help?}\label{i-actually--like--not-seeing-people-help}}

+

That's really normal. For many people, the coronavirus has offered a
strange respite from the pace of the world. Many of us, also, are more
introverted than we previously thought.

Instead of self-flagellating, use this as an opportunity to reset.
\href{https://www.nytimes.com/2020/05/20/smarter-living/coronavirus-zoom-facetime-fatigue.html}{Politely
decline Zoom calls} if you're tired of them. Don't feel like you need to
eat outside with someone
\href{https://www.nytimes.com/2020/07/09/style/coronavirus-backyard-entertaining.html}{until
you're ready}. It is a good time to take stock: Who and what is
important to me, and who and what do I want in my life after all of
this?

Also, it has never been easier to politely decline plans. Just tell
someone you're wiped out from life right now. Real friends will
understand.

 Read More
\href{https://www.nytimes.com/2020/06/27/at-home/manage-your-coronavirus-anxiety.html}{Gotten
Used to Quarantine? Us Too}

\hypertarget{eating-and-drinking}{%
\subsection{Eating and Drinking}\label{eating-and-drinking}}

\hypertarget{i-am-not-really-eating-normally-what-should-i-do}{%
\subsubsection{I am not really eating normally. What should I
do?}\label{i-am-not-really-eating-normally-what-should-i-do}}

+

If you are eating differently, you might be able to manage the change on
your own. If you are snacking a lot, that's OK. But if it bothers you,
try to assess whether you are actually hungry and, if so,
\href{https://www.nytimes.com/2020/04/20/well/eat/coronavirus-diet-metabolic-health.html}{try
to add fruits and vegetables into your routine}, if you can.

If you're regularly not eating enough, find something you can tolerate.
Schedule your mealtimes like any other essential appointment. You can
also set alarms as reminders to drink some water, or to eat something.

But eating disorders can be
\href{https://www.nytimes.com/2020/05/03/sports/athletes-eating-disorders.html}{serious},
even life threatening. Roughly
\href{http://eatingdisorderscoalition.org.s208556.gridserver.com/couch/uploads/file/Eating\%20Disorders\%20Fact\%20Sheet.pdf}{one
in 10 Americans} struggle with disordered eating, and the pandemic has
created new hurdles for those managing difficult
\href{https://www.nytimes.com/2020/05/12/well/mind/i-have-an-eating-disorder-but-cant-escape-the-kitchen.html}{relationships
with food}.

``When the world feels out of control, people want to have control over
something,'' said Jessica Gold, a psychiatrist at Washington University
in St. Louis who treats patients with eating and other mental health
disorders. ``Often, it's what you put in your mouth.''

So if you are struggling, speak with people you trust and consider
seeking professional help.

 Read More
\href{https://www.nytimes.com/2020/06/05/health/eating-disorders-coronavirus.html}{Disordered
Eating in a Disordered Time}

\hypertarget{i-am-drinking-a-lot-should-i-be-worried}{%
\subsubsection{I am drinking a lot. Should I be
worried?}\label{i-am-drinking-a-lot-should-i-be-worried}}

+

Drinking is a hard one. If you imbibe, it makes a lot of sense that you
are drinking more now. The coronavirus can be both stressful and boring,
and drinking can dull the stress of waiting. New Yorkers, for example,
\href{https://www.nytimes.com/2020/04/23/nyregion/coronavirus-liquor-stores-nyc.html}{flocked
to buy cheap wine}.

And although you might have normally had wine with friends,
\href{https://www.nytimes.com/2020/03/16/dining/drinks/drinking-alone.html}{drinking
alone} --- especially if you live alone --- is not inherently
frightening.

But substance abuse experts say they are worried that the pandemic could
also trigger more serious drinking problems and even create new ones for
people who have never struggled with dependency before. And
\href{https://www.nytimes.com/2020/04/02/nyregion/coronavirus-alcoholics-anonymous-online.html}{without
regular meetings}, alcoholics are
\href{https://www.nytimes.com/2020/03/26/health/coronavirus-alcoholics-drugs-online.html}{struggling
to stay sober} without the help of counselors and group support.

There is no right answer as to when it's time for you to be worried.
Some questions, though, might help you get a clearer sense of the
problem. These are drawn from several self-administered quizzes put out
by rehabilitation facilities and medical centers, but are mostly common
sense gut checks.

First, just try to quantify your drinking. How many days a week do you
drink? When you do drink, how many drinks do you have? Is there a time
of day that you most want to drink? Can you sleep without a drink?

What about your attitude toward drinking? Do you get angry or irritable
when someone tries to take your alcohol away, or suggest you might have
a problem? Are you secretive about drinking?

And, primarily, have your drinking patterns changed dramatically since
the pandemic started? Have you had major life changes --- beyond
lockdown --- like losing your job or the death of a relative?

If your answer to some, or many of these questions, is yes, it might be
time to step back and reconsider. If it continues to escalate, talk with
friends, family and doctors about what professional help might look
like.

 Read More
\href{https://www.nytimes.com/2020/04/30/us/30IHW-drinking-women-coronavirus-quarantine-habit.html}{Quarantini
Anyone? When Everyday Drinking Becomes a Problem}

\hypertarget{restaurants-are-open-and-i-really-do-not-want-to-cook-anymore-what-should-i-consider}{%
\subsubsection{Restaurants are open and I really do not want to cook
anymore. What should I
consider?}\label{restaurants-are-open-and-i-really-do-not-want-to-cook-anymore-what-should-i-consider}}

+

Tantalizing, right? And
\href{https://www.nytimes.com/2020/06/23/dining/outdoor-restaurants-nyc-coronavirus.html}{spending
money in local businesses} is a great way to jumpstart the economy.

However, an open restaurant doesn't mean a
\href{https://www.nytimes.com/2020/05/15/dining/restaurant-opening-safety-coronavirus.html}{safe
restaurant}, either for you or for other people. And a quick look at
cities and states that reopened their public spaces earlier suggests
that it's not safe to go without a mask, or to be around those who
aren't wearing them.

Already, open restaurants have been linked to viral spread. The food
itself if not the problem. It's customers, whose laughter and talking
can spew viral droplets throughout the area and all over waiters and
staff.

If you want to eat restaurant food, consider getting it as take out.
That way you can support a local business, see your friends and avoid
having to cook --- without as many ethical or safety concerns.

 Read More
\href{https://www.nytimes.com/2020/06/30/dining/restaurant-risks-coronavirus.html}{I'm
Not Ready to Go Back to Restaurants. Is Anyone?}

\hypertarget{why-are-bars-linked-to-outbreaks}{%
\subsubsection{Why are bars linked to
outbreaks?}\label{why-are-bars-linked-to-outbreaks}}

+

Think about a bar. Alcohol is flowing. It can be loud, but it's
definitely intimate, and you often need to lean in close to hear your
friend. And strangers have way, way fewer reservations about coming up
to people in a bar. That's sort of the point of a bar. Feeling good and
close to strangers.

It's no surprise, then, that
\href{https://www.nytimes.com/2020/06/25/well/live/coronavirus-spread-bars-transmission.html}{bars
have been linked to outbreaks in several states}. Louisiana health
officials have tied
\href{https://www.nytimes.com/2020/06/22/us/new-coronavirus-phase.html}{at
least 100 coronavirus cases} to bars in the Tigerland nightlife district
in Baton Rouge. Minnesota has traced 328 recent cases to bars across the
state.
\href{https://www.boisestatepublicradio.org/post/bars-large-venues-close-ada-county-after-surge-coronavirus-prompts-rollback\#stream/0}{In
Idaho}, health officials shut down bars in Ada County after reporting
clusters of infections among young adults who had visited several bars
in downtown Boise.

Governors in
\href{https://www.nytimes.com/2020/07/01/us/california-coronavirus-reopening.html}{California},
\href{https://www.nytimes.com/2020/06/14/us/coronavirus-united-states.html}{Texas
and Arizona}, where coronavirus cases are soaring, have ordered hundreds
of newly reopened bars to shut down. Less than two weeks after
Colorado's bars reopened at limited capacity, Gov. Jared Polis
\href{https://www.denverpost.com/2020/06/30/colorado-bars-closed-coronavirus/}{ordered
them to close}.

To understand the connection between bars and outbreaks, consider a
simple question: Are you your most vigilant self after a few drinks?

 Read More
\href{https://www.nytimes.com/2020/07/02/us/coronavirus-bars.html}{All
Eyes on Bars as Virus Surges and Americans Go Drinking}

\hypertarget{ok-ill-bite-what-is-sourdough-and-how-do-i-start-it}{%
\subsubsection{Ok, I'll bite. What is sourdough and how do I start
it?}\label{ok-ill-bite-what-is-sourdough-and-how-do-i-start-it}}

+

Baking more? We see you. During the lockdown,
\href{https://www.nytimes.com/2020/03/30/style/bread-baking-coronavirus.html}{stress
baking skyrocketed}. Everyone and their mother was suddenly a bread
aficionado. Baking necessities
\href{https://www.washingtonpost.com/news/voraciously/wp/2020/03/24/people-are-baking-bread-like-crazy-and-now-were-running-out-of-flour-and-yeast/}{like
flour and yeast} were in short supply and bread makers
\href{https://people.com/food/bread-makers-viral-amazon/}{sold out}
across the internet.

People made
\href{https://www.nytimes.com/2020/04/24/dining/focaccia-bread.html}{focaccia
gardens}. They made
\href{https://www.nytimes.com/2020/05/09/at-home/virus-make-ice-cream-in-a-mason-jar.html}{mason
jar ice cream}. And, yes, they made sourdough.

Basically, sourdough is a type of bread that rises due to natural
fermentation. Instead of adding instant commercial yeast to your dough,
you add parts of a live bacterial culture that you feed and keep in your
fridge. That's called the ``starter.'' Once a baker has a starter, she
can make endless loaves, provided she has flour and a can-do attitude.

By this point, you probably know someone who has a starter: put a blast
out to your followers on social media. If not, that's an easy fix. You
can buy one online,
\href{https://www.kingarthurflour.com/recipes/sourdough-starter-recipe}{or
make one yourself}. Just leave some flour and water out for a few days,
and yeast will start to ferment. You'll know it's ready when it smells,
well, sour.

 Read More
\href{https://cooking.nytimes.com/guides/59-how-to-make-sourdough-bread}{How
to Make Sourdough Bread}

\hypertarget{what-should-i-consider-when-ordering-food}{%
\subsubsection{What should I consider when ordering
food?}\label{what-should-i-consider-when-ordering-food}}

+

Both takeout and delivery are safer than eating in restaurants, because
social distancing is possible. And, good news, packaging has a low risk
of transmitting the disease. Although the C.D.C. says that surfaces
contaminated with droplets of the virus can infect people, the agency
notes that it is ``not thought to be the main way the virus spreads.''

Delivery, though, is slightly safer because of contactless delivery,
which lets workers leave food at your door, said Ben Chapman, a
professor and food safety specialist at North Carolina State University.
Since the ordering and payment can be done electronically, customers and
workers never need to touch.

If you choose to pick your food up, ask the restaurant staff member to
put the food down and walk away before you pick it up. Wear a mask.
Stand far apart from other patrons. And whether you choose takeout or
delivery, try to pay in advance. You can do it electronically, which
will keep both you and the workers safer.

 Read More
\href{https://www.nytimes.com/2020/05/27/dining/takeout-delivery-safety-coronavirus.html}{Is
Takeout and Delivery Food Safe?}

\hypertarget{mental-health}{%
\subsection{Mental Health}\label{mental-health}}

\hypertarget{as-the-world-is-reopening-im-anxious-what-should-i-do}{%
\subsubsection{As the world is reopening, I'm anxious. What should I
do?}\label{as-the-world-is-reopening-im-anxious-what-should-i-do}}

+

Lots of people with anxiety are struggling more than usual, and lots of
people who haven't been anxious before are dealing with symptoms. With
different messages coming from varying levels of the government, people
can be left feeling as if there are few reliable answers about what
precautions they should take or without a clear sense of whether things
are under control.

``Uncertainty drives anxiety,'' said Dr. Ellen Hendriksen, a clinical
psychologist at Boston University and the author of
\href{https://us.macmillan.com/books/9781250161703}{``How to Be
Yourself: Quiet Your Inner Critic and Rise Above Social Anxiety.}''
``Anxiety is rooted in not knowing what is going to happen.''

Just because the world is reopening, you don't have to start living your
life as if it's before the pandemic. Instead of focusing on what
frightens you, think of the things that you want to have back in your
life that would enrich and fulfill you. Ease back in with activities
that you actually want to do, and view this time as an opportunity,
weighing whether you want to continue past relationships and activities.

Feel free, also, to tell people what your boundaries are, or name the
awkwardness. You might just ask what the other person feels comfortable
with. ``Should we elbow bump? Do you want to go first? I guess I'll take
the next elevator, right?''

``A problem shared is a problem halved,'' Dr. Hendriksen said. ``Naming
the uncertainty is helpful because it shatters the illusion that there
is a right way to do this.''

 Read More
\href{https://www.nytimes.com/2020/06/27/at-home/manage-your-coronavirus-anxiety.html}{Gotten
Used to Quarantine? Us Too}

\hypertarget{am-i-depressed-or-just-in-a-bad-mood}{%
\subsubsection{Am I depressed? Or just in a bad
mood?}\label{am-i-depressed-or-just-in-a-bad-mood}}

+

Most experts expect to see rates of depression and other psychological
disorders increase in the
\href{https://www.nytimes.com/2020/04/18/health/coronavirus-america-future.html}{coming
months}, as the pandemic continues. And yet,
\href{https://www.sciencedaily.com/releases/2015/03/150311160240.htm}{a
majority of those} who seek treatment for depression will improve if
they persist.

``I don't know anyone right now that's not having depression-like
symptoms,'' said Luana Marques, a psychologist at Harvard Medical School
and the president of the \href{https://adaa.org/}{Anxiety and Depression
Association of America}. ``It's hard to keep going when our brains are
constantly on fight or flight. It makes people really tired. If you're
having trouble concentrating or getting out of bed, it's not abnormal.
It's an evolutionary response to a threat.''

Familiarize yourself with depression's
\href{https://www.mayoclinic.org/diseases-conditions/depression/expert-answers/clinical-depression/faq-20057770}{physical
and mental markers}. Or take a
\href{https://www.med.umich.edu/1info/FHP/practiceguides/depress/phq-9.pdf}{self-diagnostic
test}. When depression isn't severe, a self-care routine may be enough.

If you are having thoughts of suicide, call the National Suicide
Prevention Lifeline at 1-800-273-TALK (1-800-273-8255) or go to
\href{http://speakingofsuicide.com/resources}{SpeakingOfSuicide.com/resources}
for a list of additional resources.

 Read More
\href{https://www.nytimes.com/2020/05/21/well/coronavirus-depression.html}{How
to Tell if It's More Than Just a Bad Mood}

\hypertarget{what-about-meditation}{%
\subsubsection{What about meditation?}\label{what-about-meditation}}

+

Meditating is a great way to take care of your stress. If you're a
novice
\href{https://www.nytimes.com/2020/06/22/at-home/how-to-start-meditating.html}{looking
to start}, try for just five minutes every day. There's no right way to
do it, but a consistent practice is a good way to build your relaxation
muscles. Apps can be helpful:
\href{https://www.nytimes.com/wirecutter/}{Wirecutter}, a product
recommendation site that's owned by The New York Times, recently named
the meditation app \href{https://www.headspace.com/}{Headspace} (which
costs \$69.99 a year, after a free two-week trial) as its top choice.

Or, just try to find a comfortable space and breathe through your
distress. Breathe from your belly, imagining your breath moving up from
the depths of your body. Slow it down, close your eyes, and feel your
lungs expanding and deflating.

 Read More
\href{https://www.nytimes.com/2020/06/22/at-home/how-to-start-meditating.html}{How
to Start Meditating}

\hypertarget{how-do-i-find-a-therapist}{%
\subsubsection{How do I find a
therapist?}\label{how-do-i-find-a-therapist}}

+

Finding a therapist can be difficult,
\href{https://www.nytimes.com/2017/07/17/smarter-living/how-to-find-the-right-therapist.html}{even
under normal circumstances}.

First, figure out what your insurance is and how you can pay for it. The
logistical structures will make it a lot easier to move forward, and
narrow down your choices for you. Of course, that's not always an option
--- therapy can be time-intensive and expensive. You may not have
insurance. If it is possible for you, though, here's a quick guide.

Then,
\href{https://www.nytimes.com/2020/05/13/well/mind/prospective-therapist-interview-questions-online-virus.html}{determine
what type of professional you need}. If you're mostly looking to talk, a
psychologist is a safe bet. Keep in mind that they do not prescribe
medication.

If you're suffering from a specific ailment --- panic attacks,
depression, post-traumatic stress disorder or obsessive-compulsive
disorder, bipolar disorder, major depressive disorder, sociopathy,
borderline personality disorder or
\href{https://www.nytimes.com/2016/01/28/health/schizophrenia-cause-synaptic-pruning-brain-psychiatry.html}{schizophrenia}
--- see a psychiatrist or a psychologist with considerable experience in
that specialty.

Then, set up a few preliminary calls and consultations. The
getting-to-know-you session is often free, and you should take it as an
opportunity to talk about what you are looking for from therapy, and how
they approach their practice.

You might click with someone, the way you would with a friend. And
there's nothing wrong with trying a few initial appointments to learn
about your own wants and desires.
\href{https://www.nytimes.com/2020/05/13/well/mind/prospective-therapist-interview-questions-online-virus.html}{Remote
options are now widely available}, so you can still attend sessions
while
\href{https://www.nytimes.com/2020/07/09/well/mind/teletherapy-mental-health-coronavirus.html}{staying
distant}.

 Read More
\href{https://www.nytimes.com/2017/07/17/smarter-living/how-to-find-the-right-therapist.html}{How
to Find the Right Therapist}

\hypertarget{im-having-trouble-sleeping-what-should-i-do}{%
\subsubsection{I'm having trouble sleeping. What should I
do?}\label{im-having-trouble-sleeping-what-should-i-do}}

+

While there's no single trick that works for everyone, one thing you can
try now to
\href{https://www.nytimes.com/2020/03/25/style/self-care/sleep-tips-benefits-coronavirus.html}{improve
your sleep} is setting up a consistent schedule. That means sticking to
a regular bedtime and wake time.
\href{https://www.nytimes.com/2020/04/24/well/melatonin-sleep-aid-coronavirus.html}{Melatonin}
is also worth considering.

Try to go 90 minutes without any screentime before bed --- that means no
emails, social media or watching TV. (Read a book, take a nice shower,
etc.) You can also take baby steps, beginning with 15 minutes, then
working your way up.

Working out, whether that means taking a socially distant walk or doing
an at-home routine, can help tire you out and dispel extra nervous
energy. Taking a warm shower or bath 90 minutes before lights out may
also help prime you for a better night's sleep. And be sure to nix the
coffee, alcohol and any food before bed --- these things don't promote
quality rest.

 Read More
\href{https://www.nytimes.com/2020/03/25/style/self-care/sleep-tips-benefits-coronavirus.html}{How
to Get More Sleep Tonight}

\hypertarget{how-can-i-stay-motivated-to-work}{%
\subsubsection{How can I stay motivated to
work?}\label{how-can-i-stay-motivated-to-work}}

+

As the weeks morph into months, the ennui of coronavirus-induced
isolation can undermine our enthusiasm for getting anything done.

If that sounds familiar, remember: Doing what's meaningful --- acting on
what really matters to a person --- is the antidote to burnout.

Motivation might best be fostered by dividing large goals into small,
specific tasks that are more easily accomplished, but not so simple that
they are boring and soon abandoned. Avoid perfectionism --- the ultimate
goal could become an insurmountable challenge. As each task is
completed, reward yourself with virtual brownie points (not chips or
cookies!), then go on to the next one.

 Read More
\href{https://www.nytimes.com/2020/05/18/well/mind/motivation-pandemic-coronavirus.html}{How
to Maintain Motivation in a Pandemic}

\hypertarget{relationships}{%
\subsection{Relationships}\label{relationships}}

\hypertarget{help-im-frustrated-with-my-partner}{%
\subsubsection{Help! I'm frustrated with my
partner.}\label{help-im-frustrated-with-my-partner}}

+

That makes sense. If you're
\href{https://www.nytimes.com/2020/05/26/style/coronavirus-living-together-callout.html}{cohabitating},
you could be in a tough spot.
\href{https://www.nytimes.com/2020/04/30/smarter-living/coronavirus-long-distance-relationships.html}{If
you're not}, you've probably had to figure out ways to see each other
--- or not --- that feel safe and nurturing for you and the people
around you.

Or as Eric Spiegelman, a podcasting executive based in Los Angeles,
\href{https://twitter.com/ericspiegelman/status/1246488909221003264?lang=en}{tweeted
in April}, ``My wife and I play this fun game during quarantine, it's
called `Why Are You Doing It That Way?' and there are no winners.''

No matter who you are or how you are interacting, make sure you take
care of yourself first. Then, make a plan. If you are both working from
home, when can you take time to care for each other and feel normal and
romantic again? What rituals can you implement to separate day from
night, roommates from lovers? You might just have to impose boundaries.

If you were both at work before the pandemic, separately, during the
day, you probably didn't talk for 14 waking hours straight. Build in
time apart, as if you really were at the office, so seeing each other is
a nice break at the end of the day,
\href{https://www.nytimes.com/2020/07/08/parenting/coronavirus-marriage-relationships.html}{rather
than a droning grind}.

``The traditional marriage vows are `for better or for worse,''' said
Jean Fitzpatrick, a \href{https://therapistnyc.com/}{relationship
therapist} based in Manhattan. ``This is for worse. And so how do we
navigate a time like this? Our relationships will either grow as a
result, or they will be harmed.''

But this time together might actually make you closer. So many parts of
our lives have changed without our consent, and we may be feeling a kind
of grief about it. Some people may not want to complain to their
partners about these bad feelings, but if you don't honestly share these
feelings, you two might feel a sense of disconnect. Lean on each other
right now.

And, you can still do nice, intentional, romantic things. Get dressed
up. Foot rubs. Chocolate. You'll be OK.

 Read More
\href{https://www.nytimes.com/2020/04/03/smarter-living/coronavirus-relationship-advice.html}{How
to Help Your Relationship Survive a Lockdown}

\hypertarget{my-partner-wants-to-go-out-more-than-i-do-what-should-we-do}{%
\subsubsection{My partner wants to go out more than I do. What should we
do?}\label{my-partner-wants-to-go-out-more-than-i-do-what-should-we-do}}

+

You are going to have to work together to weigh each other's needs. One
partner might have parents who are older and at higher risk of
complications from the coronavirus; the other might be an extrovert who
thrives on being around other people and is, emotionally, at a breaking
point.

Remember that you're on the same team. ``It's not, `My needs versus your
needs, and let's negotiate,' but asking the question and having the
posture of: `What is best for our relationship?''' said Jennifer
Bullock, a \href{https://www.letsdevelopphilly.com/}{psychotherapist
based in Philadelphia}.

Several psychologists and counselors recommended presenting a united
front when explaining shared decisions to friends and family. Any sort
of ``I would, but he's afraid'' seeds resentment and can amplify the
problem far past the boundaries of your own home.

In any fraught situation, sit down with your partner and listen. Instead
of offering rebuttals, try to treat it more like an interview about
where he or she is coming from. Ask open-ended questions --- which can't
be answered with a simple ``yes'' or ``no.'' You could try: ``How should
we approach our safety when I go back to the office?''

But remember, this is a pandemic. If one person is seriously scared
about their health or the health of their family, their needs might take
priority. Ultimately, your well-being and others' should take
precedence.

 Read More
\href{https://www.nytimes.com/2020/07/11/at-home/couples-coronavirus-pandemic.html}{Tackle
Reopening Choices as a Couple}

\hypertarget{how-do-i-date-during-this-time}{%
\subsubsection{How do I date during this
time?}\label{how-do-i-date-during-this-time}}

+

That's a tough one.

By nature, pandemic dating is much more serious than any other type of
dating. Even if you feel willing to take the risk, you're basically
co-mingling your
\href{https://www.nytimes.com/2020/06/09/parenting/coronavirus-pod-family.html}{quarantine
pods} when you make out. That doesn't mean that you have to marry the
person, but it does reframe the pace of everything.

There's something sort of nice, though, about
\href{https://www.nytimes.com/2020/05/21/style/first-date-during-quarantine-coronavirus.html}{taking
it slow}. Start with a socially distanced walk with masks or a
\href{https://www.nytimes.com/2020/04/18/nyregion/coronavirus-dating-video.html}{video
chat}, maybe. Get to know them. If you are going to be physically
intimate, you'll need to have frank and honest conversations about your
limits and your living situations.

Now, though,
\href{https://www.nytimes.com/2020/06/11/well/live/coronavirus-sex-dating-masks.html}{sex
is way more complicated}. A number of public health agencies have
offered
\href{https://www.nytimes.com/2020/05/20/world/netherlands-sex-buddies-coronavirus.html}{tips
for dating and sex during the pandemic}, but the New York City health
department has recently updated its
\href{https://www1.nyc.gov/assets/doh/downloads/pdf/imm/covid-sex-guidance.pdf}{Safer
Sex and Covid-19 fact sheet} with more-detailed and descriptive advice.
The new guidelines still say that ``you are your safest sex partner,''
and that the ``next safest partner'' is someone in your household.

Some couples are wearing masks during the deed; others are getting
tested before getting intimate. The only sure thing is that this is
going to be a bumpy year for single people.

One of the good things about dating during quarantine, though, is that
it's cheaper. Two drinks might be \$25. A video chat with your own wine
is much less.

 Read More
\href{https://www.nytimes.com/2020/05/07/well/mind/dating-coronavirus-love-relationships.html}{How
Coronavirus Is Changing the Dating Game for the Better}

\hypertarget{im-worried-that-i-or-my-friend-might-be-suffering-domestic-abuse-what-should-i-do}{%
\subsubsection{I'm worried that I, or my friend, might be suffering
domestic abuse. What should I
do?}\label{im-worried-that-i-or-my-friend-might-be-suffering-domestic-abuse-what-should-i-do}}

+

If you are in danger, call the National Domestic Violence Hotline:
1-800-799-7233.

This is a problem that is affecting people
\href{https://www.nytimes.com/2020/04/06/world/coronavirus-domestic-violence.html}{across
the world}. Since the pandemic started, abusers and their families are
often stuck in the same house, and tensions are running high. Doctors
and advocates for victims are seeing signs of an increase in violence at
home. They are hearing accounts of people lashing out, particularly at
women and children.

``No one can leave,'' said Kim Foxx, the chief prosecutor in Chicago.
``You're literally mandating that people who probably should not be
together in the same space stay.''

Across the country, hotlines are seeing increased traffic, with calls
and text messages mounting. But in many places,
\href{https://www.nytimes.com/2020/04/17/nyregion/new-york-city-domestic-violence-coronavirus.html}{reports
are dropping}. That's not because abuse is falling, officials say.
Instead, it's because people are having a harder time reporting.
\href{https://www.nytimes.com/2020/06/09/nyregion/coronavirus-nyc-child-abuse.html}{Reports
of child abuse cases are falling}, which is also worrisome.

Be careful when reaching out for help, especially if your partner
monitors your phone. Maybe ask a friend to call for you, or use their
online chat tool.

 Read More
\href{https://www.nytimes.com/2020/05/15/us/domestic-violence-coronavirus.html}{Domestic
Violence Calls Mount as Restrictions Linger: `No One Can Leave'}

\hypertarget{parenting-help}{%
\subsection{Parenting Help}\label{parenting-help}}

\hypertarget{how-should-i-explain-this-to-my-kids-how-is-their-mental-health}{%
\subsubsection{How should I explain this to my kids? How is their mental
health?}\label{how-should-i-explain-this-to-my-kids-how-is-their-mental-health}}

+

Before you talk to your children, it's important to
\href{https://www.nytimes.com/2020/03/17/parenting/coronavirus-kids-talk.html}{understand
your own anxiety} and keep it in check. If your child is worried about
the coronavirus, listen to him or her, rather than respond with comments
like, ``It'll be fine.'' Dismissive reactions can make children feel
like they're not being heard, said Abi Gewirtz, a clinical psychologist
and professor at the University of Minnesota.

Talking about the pandemic and its impact may help avoid the
\href{https://www.nytimes.com/2020/05/20/us/coronavirus-young-people-emotional-toll.html}{harmful
effects of stress}. Help your kids understand that there are things they
can do to help others --- like staying home whenever possible and
wearing a mask when they go out. Kids feel good when they know they are
helping solve a problem.

Also, avoid making assumptions. While we may be struggling with schools
being closed, kids could be rejoicing in it. We might assume our kids
miss their friends, but they may appreciate having more time with us.
And some who were dealing with bullying or social challenges at school
may be relieved to not have to see other kids.

 Read More
\href{https://www.nytimes.com/2020/05/07/well/family/coronavirus-children-stress-parents.html}{How
to Keep Children's Stress From Turning Into Trauma}

\hypertarget{how-does-this-affect-kids-physically-can-they-transmit-coronavirus}{%
\subsubsection{How does this affect kids, physically? Can they transmit
coronavirus?}\label{how-does-this-affect-kids-physically-can-they-transmit-coronavirus}}

+

Fewer children seem to get infected by the coronavirus than adults, and
their symptoms tend to be mild. But it can be serious,
\href{https://www.nytimes.com/2020/05/12/well/family/coronavirus-children-covid-19.html}{contrary
to earlier belief}.

A
\href{https://www.nytimes.com/2020/05/19/parenting/pmis-coronavirus-children.html}{very
small number of children} with the virus have shown
\href{https://www.nytimes.com/article/coronavirus-symptoms.html}{symptoms}
associated with toxic shock or
\href{https://www.nytimes.com/article/kawasaki-disease-coronavirus-children.html}{Kawasaki
disease}. This is a rare illness in children, according to the New York
City health department, that involves inflammation of the blood vessels,
including coronary arteries.

\href{https://www.nytimes.com/2020/05/05/health/coronavirus-children-transmission-school.html}{Two
studies} offer compelling evidence that children can transmit the virus.
While neither proved it, many epidemiologists who were not involved in
the research said that the evidence was strong enough to suggest that
schools should be kept closed. (We have more about schools in a later
section.)

If your child develops severe symptoms, such as trouble breathing, an
inability to eat or drink, or a change in behavior, you should contact a
doctor.

 Read More
\href{https://www.nytimes.com/2020/05/05/health/coronavirus-children-transmission-school.html}{New
Studies Add to Evidence that Children May Transmit the Coronavirus}

\hypertarget{my-partner-and-i-both-work-from-home-and-now-both-parent-from-home-what-should-we-do}{%
\subsubsection{My partner and I both work from home, and now both parent
from home. What should we
do?}\label{my-partner-and-i-both-work-from-home-and-now-both-parent-from-home-what-should-we-do}}

+

Honestly, the first step is to acknowledge this is totally bonkers. The
writer Deb Perelman, who created the food blog
\href{https://smittenkitchen.com/}{Smitten Kitchen}, laid it out plainly
in her essay:
"\href{https://www.nytimes.com/2020/07/02/business/covid-economy-parents-kids-career-homeschooling.html}{In
the Covid-19 Economy, You Can Have a Kid or a Job. You Can't Have
Both.}''

``Why are we not hearing a primal scream so deafening that no plodding
policy can be implemented without addressing the people buried by it?''
she asks.

There are a few strategies to try. Most importantly, work with your
partner and present a united front. Division will create disorder, which
will create chaos, which will drive you all up the wall.

However you can, try to create boundaries. That might be a physical
boundary (``This is the adult room'') or a time boundary (``You are not
allowed to talk to Daddy until 2 p.m. today''). To do that, you might
need to use time more effectively --- sleep whenever you can and use
noise-canceling technology, if need be. Having a daily schedule also
helps.

But most important, be kind to yourself, and to one another. This is not
like anything that has ever happened before. That's OK. If the kids are
\href{https://www.nytimes.com/2020/04/13/parenting/manage-screen-time-coronavirus.html}{spending
more time on their screens}, or if you're not whipping up super healthy
organic meals every night, whatever. You're getting through it, and
you're loving one another. That's what counts, right?

 Read More
\href{https://www.nytimes.com/2020/03/19/us/work-from-home-mothers-coronavirus-covid19.html}{Figuring
Out Work and Family in the Age of Coronavirus}

\hypertarget{what-about-grandkids-and-grandparents}{%
\subsubsection{What about grandkids and
grandparents?}\label{what-about-grandkids-and-grandparents}}

+

Older people are more likely to have a severe case of the coronavirus,
so visiting grandparents or grandchildren is
\href{https://www.nytimes.com/2020/03/20/parenting/grandparents-visit-safety-coronavirus.html}{risky}.

That being said, staying apart indefinitely is not feasible.
Grandchildren are
\href{https://www.nytimes.com/2020/06/16/parenting/baby/grandparents-meet-newborn-coronavirus.html}{being
born} and grandchildren are turning 10.
\href{https://www.nytimes.com/2020/04/29/well/family/coronavirus-grandchildren-compromised-immunity-cancer.html}{Life
is slipping by}. So find a way forward.

In the nice weather, you could spend time together outdoors. Don't
forget a mask.

Video chat does work, although you might need to walk your older parents
through the twists and turns a few times before it's second nature.

If you live close, perhaps you conjoin your quarantine pods. Get tested,
quarantine while you're getting results back, and then spend time
together normally. That might be necessary: Parents, stressed from
working at home, desperately need reliable child care. Some grandparents
who have become the primary caregivers are
\href{https://www.nytimes.com/2020/05/12/parenting/coronavirus-grandparents-childcare.html?type=roundup\&link=intro}{delighted,
and exhausted}.

That might be your best option. Social distancing is hard to explain to
kids, and often harder to explain to grandparents. You want kids to have
their grandparents around at high school graduations, first jobs and
weddings. If you can't see each other safely, maybe you should wait
until you can.

 Read More
\href{https://www.nytimes.com/2020/05/20/well/family/coronavirus-grandchildren-grandparents-when-can-i-see-my-grandkids.html}{When
Can I See My Grandkids?}

\hypertarget{how-do-i-get-my-kid-to-wear-a-mask}{%
\subsubsection{How do I get my kid to wear a
mask?}\label{how-do-i-get-my-kid-to-wear-a-mask}}

+

Older children can be a little cranky about adapting to life with masks,
but younger children are perfectly positioned to learn a new drill.

Most children enjoy the chance to feel morally superior to adults (and
adults often make this all too easy); go ahead and encourage a little
righteousness. They can be the family monitors, reminding their parents
not to forget their face coverings when they leave the house.

For some children, though, even the humblest of
\href{https://www.nytimes.com/2020/04/13/well/family/coronavirus-children-masks-fear.html?action=click\&module=RelatedLinks\&pgtype=Article}{masks
can be scary} --- scary in themselves, and scary as reminders of the
threat of infection and the generally frightening moment. By making it
into a game, or buying
\href{https://www.nytimes.com/2020/06/23/style/face-mask-emotion-coronavirus.html}{see-through
or emotive masks} for the family, you can alleviate their anxiety. Or,
maybe it's an opportunity to play dress up?

 Read More
\href{https://www.nytimes.com/2020/06/23/well/family/children-masks-coronavirus.html}{How
to Help Kids Embrace Mask-Wearing}

\hypertarget{how-do-i-explain-social-distancing-to-my-kids}{%
\subsubsection{How do I explain social distancing to my
kids?}\label{how-do-i-explain-social-distancing-to-my-kids}}

+

Use their imaginations and tell stories. To explain why social
distancing is important, one mother in Los Angeles compared it to
\href{https://www.instagram.com/p/B9wqC44nYKC/?utm_source=ig_embed\&utm_campaign=embed_video_watch_again}{pulling
to the side of the road} to let an ambulance pass.

You could also try to make it into a game. (Think hot potato, but
people.) You win points if you're far away.

Debates about using
\href{https://slate.com/human-interest/2017/08/rewards-systems-for-kids-are-effective-if-you-use-them-correctly.html}{rewards
to motivate children} are endless, but parents
\href{https://www.verywellfamily.com/concerns-about-giving-kids-rewards-1094886}{trade
favors for obedience} all the time. Even the C.D.C. signs off on
rewarding good behavior (say, wearing a mask outside without fussing)
with
\href{https://www.cdc.gov/parents/essentials/consequences/rewards.html}{praise,
a board game or an extra book at bedtime}.

Don't threaten them, though. Child psychology experts say that threats
hurt
\href{https://keepingyourcoolparenting.com/stop-with-the-threats-7-reasons-threats-dont-work-and-what-to-do-instead/}{motivation}
and undermine parent-child relationships. But you can still
\href{https://www.verywellfamily.com/taking-away-privileges-to-discipline-children-1094759}{take
away privileges} for not following the rules (like wandering too close
to strangers without a mask). Just make sure you explain the
consequences beforehand and make the punishment fit the infraction,
psychologists say.

 Read More
\href{https://www.nytimes.com/2020/06/13/style/kids-children-social-distance-coronavirus.html?action=click\&module=RelatedLinks\&pgtype=Article}{How
to Get Your Kids to Stay 6 Feet Away\ldots{} From Everything}

\hypertarget{how-do-i-help-my-kids-entertain-themselves-without-socializing}{%
\subsubsection{How do I help my kids entertain themselves without
socializing?}\label{how-do-i-help-my-kids-entertain-themselves-without-socializing}}

+

\href{https://www.nytimes.com/2020/06/18/arts/kids-summer-activities-virus.html}{Maybe
don't}. Let them amuse themselves.

That's tough: Research has shown that our heavily scheduled lives have
contributed to a significant decrease in the amount of free time kids
have, so their independent play skills may not be ready for the moment
we are facing.

But it might really be necessary. ``Independent play encourages time
management, executive function and organizational skills, and emotional
and physical awareness and regulation,'' said Dana Rosenbloom, a parent
and child educator in Manhattan. ``All skills that help us be successful
individuals as adults.''

You can help them play on their own. Start slow --- let younger kids
direct play for 20 minutes, 35, then an hour. Keep toys out of rotation,
said Avital Schreiber-Levy, a parenting performance coach in New Jersey
who has created a
\href{https://www.theparentingjunkie.com/wp-content/uploads/2020/03/PlayPandemic-Newer.pdf}{play
guide} for parents on lockdown. ``When toys sit out too long, they go
stale,'' she said. ``It's about making them novel again, either because
we take them away or we set them up in a new way.''

Let kids get messy, and keep play in a specific zone where they can
wiggle and jump. And, be patient. If your kid is used to socializing and
screentime, self-directed play might be tough for them. Letting them
work out the kinks on their own will develop strength and resiliency
that can last them a lifetime.

 Read More
\href{https://www.nytimes.com/2020/04/03/parenting/kids-independent-play-coronavirus-quarantine.html}{Now's
a Good Time to Teach Your Kids to Play on Their Own}

\hypertarget{what-about-socializing-with-other-children-can-we-do-it-safely}{%
\subsubsection{What about socializing with other children? Can we do it
safely?}\label{what-about-socializing-with-other-children-can-we-do-it-safely}}

+

Maybe.

If your family or friends live nearby, consider linking up
\href{https://www.nytimes.com/2020/06/09/parenting/coronavirus-pod-family.html}{quar-pods}.
If you as parents agree to community norms and guidelines, you can
probably let your kids play together as if the world were normal.

But if you're unable to do that, or you want to see someone for a play
date, know that there are risks.
\href{https://www.nytimes.com/2020/06/11/parenting/playgrounds-reopen-safety-coronavirus.html}{Playgrounds
are open} in some states, and young children are being asked to stay six
feet apart and wear masks, an often unrealistic expectation.

If you're trying for a play date,
\href{https://www.nytimes.com/2020/06/29/parenting/family-socially-distancing-coronavirus.html}{good
communication} with a parent is essential.

``A start would be, `Hi, our kids have been asking about getting
together, and as you know, this is a complicated conversation right
now,''' said Dr. Dipesh Navsaria, an associate professor of pediatrics
at the University of Wisconsin School of Medicine and Public Health. A
parent could continue, ``I wanted to start with an open conversation,
see where you are, tell you where I am, and see if it's possible to send
a consistent message to our kids.''

Enlist your kids for help, too. If they feel like it's a project that
they're doing together, they might find some way to enjoy it. You could
make spaces --- all of the kids get their own blanket --- if you're
playing outside.

But there's no good answer to this. If you can keep your kids away from
other kids, you're reducing your family's risk of infection. That might
not be possible, but it might also be the only way to do it safely.

 Read More
\href{https://www.nytimes.com/2020/06/01/well/family/coronavirus-reopening-children-parents-play-dates.html}{How
Do You Decide if Children Can Play Together Again?}

\hypertarget{but-will-not-socializing-with-other-children-stunt-my-kids-development}{%
\subsubsection{But will not socializing with other children stunt my
kid's
development?}\label{but-will-not-socializing-with-other-children-stunt-my-kids-development}}

+

Probably not, although it does make sense to worry.

Social interactions are an important part of development throughout
childhood, and spending time with peers is typically part of that
process. But try not to fret too much about what kids are missing right
now. Children are resilient and adaptable, and they're not entirely cut
off from other people.

There is much to be gained from interactions with parents, siblings and
even pets. Time alone is valuable, too. And connection through
technology, like hanging out or playing games through video chats, can
fill in some of the blanks. Even without peer interaction for a while,
kids can still develop socially and emotionally in ways that will
prepare them to pursue real-world friendships when those can resume.

Friendship is like riding a bicycle. You don't really forget how, even
if you have some time away.

 Read More
\href{https://www.nytimes.com/2020/06/18/parenting/kids-social-needs-quarantine.html}{Worried
About Your Kids' Social Skills Post-Lockdown?}

\hypertarget{schools}{%
\subsection{Schools}\label{schools}}

\hypertarget{for-children-what-is-school-going-to-look-like}{%
\subsubsection{For children, what is school going to look
like?}\label{for-children-what-is-school-going-to-look-like}}

+

As school districts across the United States consider whether and how to
restart in-person classes, their challenge is complicated by a pair of
fundamental uncertainties: No nation has tried to send children back to
school with the virus raging at levels like America's, and the
scientific research about transmission in classrooms is limited.

``It feels like we're playing Russian roulette with our kids and our
staff,'' said Robin Cogan, a nurse at the Yorkship School in Camden,
N.J., who serves on the state's committee on reopening schools.

It is unlikely that many schools will return to a normal schedule this
fall, requiring the grind of
\href{https://www.nytimes.com/2020/06/05/us/coronavirus-education-lost-learning.html}{online
learning,}
\href{https://www.nytimes.com/2020/05/29/us/coronavirus-child-care-centers.html}{makeshift
child care} and
\href{https://www.nytimes.com/2020/06/03/business/economy/coronavirus-working-women.html}{stunted
workdays} to continue.

California's two largest public school districts --- Los Angeles and San
Diego --- said on July 13, that
\href{https://www.nytimes.com/2020/07/13/us/lausd-san-diego-school-reopening.html}{instruction
will be remote-only in the fall,} citing concerns that surging
coronavirus infections in their areas pose too dire a risk for students
and teachers. Together, the two districts enroll some 825,000 students.
They are the largest in the country so far to abandon plans for even a
partial physical return to classrooms when they reopen in August.

For other districts, the solution won't be an all-or-nothing approach.
\href{https://bioethics.jhu.edu/research-and-outreach/projects/eschool-initiative/school-policy-tracker/}{Many
systems,} including the nation's largest, New York City, are devising
\href{https://www.nytimes.com/2020/06/26/us/coronavirus-schools-reopen-fall.html}{hybrid
plans} that involve spending some days in classrooms and other days
online. There's no national policy on this yet, so check with your
municipal school system regularly to see what is happening in your
community.

Plan for the worst and, frankly, assume the worst. An outbreak at a
school could slow operations to a halt for at least two weeks, and would
be enormously disruptive. Many scientists and researchers think school
reopenings might just be
\href{https://www.nytimes.com/2020/06/12/upshot/epidemiologists-decisions-children-school-coronavirus.html}{wishful
thinking.}

 Read More
\href{https://www.nytimes.com/2020/07/11/health/coronavirus-schools-reopen.html}{How
to Reopen Schools: What Science and Other Countries Teach Us}

\hypertarget{what-about-colleges}{%
\subsubsection{What about colleges?}\label{what-about-colleges}}

+

That depends on the college or college system, and is still pretty up in
the air. College is,
\href{https://www.nytimes.com/2020/06/15/opinion/coronavirus-college-safe.html}{by
its nature,} not a socially distant period of life. Dorm living is
close; shared bathrooms are closer. So much you do for fun in college is
predicated on touch.

Still,
\href{https://www.chronicle.com/article/Here-s-a-List-of-Colleges-/248626}{more
than three-quarters of colleges and universities} have decided students
can return to campus this fall, to attend in-class and remote studies.
But many
\href{https://www.nytimes.com/2020/07/03/us/coronavirus-college-professors.html}{professors
are concerned} about coming in to teach. Their students are at lower
risk of getting seriously ill from the virus, and are probably taking
many more risks outside the classroom.

``Until there's a vaccine, I'm not setting foot on campus,'' said Dana
Ward, 70, an emeritus professor of political studies at Pitzer College
in Claremont, Calif., who teaches a class in anarchist history and
thought.

Administrators are trying for creative solutions.
\href{https://www.nytimes.com/2020/05/19/us/coronavirus-college-fall-semester.html}{Calendars
are changing:} Campuses across the country are forgoing fall break to
zip through academic work and get students home before Thanksgiving. Or,
they're allowing only
\href{https://www.nytimes.com/2020/07/06/us/coronavirus-universities-colleges-reopening.html}{some
of their students} back to campus. In order to achieve social
distancing, many colleges are saying they will allow 40 to 60 percent of
their students to return to campus and live in the residence halls at
any one time, often divided by class year.

Sports are also a big question mark.
\href{https://www.nytimes.com/2020/07/01/sports/ncaafootball/coronavirus-college-football-hbcus-clemson.html}{College
football} is
\href{https://www.nytimes.com/2020/07/10/sports/ncaafootball/coronavirus-college-football-season-canceled.html}{teetering}
and the Ivy League has
\href{https://www.nytimes.com/2020/07/08/sports/ncaafootball/ivy-league-fall-sports-football-coronavirus.html}{suspended
all sports} until January. Expect more leagues to follow.

 Read More
\href{https://www.nytimes.com/2020/06/03/magazine/covid-college-fall.html}{What
Will College Be Like in the Fall?}

\hypertarget{im-an-international-student-what-now}{%
\subsubsection{I'm an international student. What
now?}\label{im-an-international-student-what-now}}

+

Already,
\href{https://www.nytimes.com/2020/04/25/us/coronavirus-international-foreign-students-universities.html}{international
students were in a tough spot.} More than a million students from
different countries were stranded during the lockdown, many unable to
stay on campus and some struggled to find a way back home.

After
\href{https://www.nytimes.com/aponline/2020/07/06/us/politics/ap-us-virus-outbreak-international-students.html}{a
directive by the Trump administration} that students whose classes were
moving entirely online for the fall would be
\href{https://www.nytimes.com/2020/07/07/us/student-visas-coronavirus.html}{stripped
of their visas} and required to leave the United States, many were
wracked with
\href{https://www.nytimes.com/2020/07/09/world/international-students-visa-reaction.html}{uncertainty.}

On July 14, the Trump administration
\href{https://www.nytimes.com/2020/07/14/world/coronavirus-update.html}{walked
back} the policy, ending a proposed plan that had thrown the higher
education world into turmoil. For now, things seem clear.

 Read More
\href{https://www.nytimes.com/2020/07/07/us/student-visas-coronavirus.html}{Trump
Visa Rules Seen as Way to Pressure Colleges on Reopening}

\hypertarget{im-an-incoming-first-year-student-can-i-defer-or-take-a-gap-year}{%
\subsubsection{I'm an incoming first-year student. Can I defer? Or take
a gap
year?}\label{im-an-incoming-first-year-student-can-i-defer-or-take-a-gap-year}}

+

Many colleges are allowing you to do so, but check with your admissions
counselor before you make a decision. You might be unpleasantly
surprised.

Already, some colleges are
\href{https://www.nytimes.com/2020/05/01/us/coronavirus-college-enrollment.html}{bleeding
students,} as a large number of high school seniors have put off a
decision about where to go to college in the fall --- or whether to go
at all.

If you can take
\href{https://www.nytimes.com/2020/04/23/well/family/college-students-gap-year-coronavirus.html}{a
gap year,} though, it might not be a terrible idea.

For some families, lost income makes it impossible to consider paying
tuition right now. For others, it's more about a refusal to forgo the
yearned-for experiences of university life outside the classroom ---
giggling through late nights with roommates, exploring new interests
through campus clubs, playing or cheering for sports teams --- for a
pale, Zoom-fueled imitation. They'd rather take a year off and try again
in 2021.

If you take a year off without a clear plan, try to set clear goals for
yourself and design your own curriculum, even if you are holding down a
job to support your family. Read what you want to read, try your hand at
a new language and enlist a mentor to help you find your way through
this strange time.

 Read More
\href{https://www.nytimes.com/2020/07/11/style/harvard-students-coronavirus.html}{What's
the Value of Harvard Without a Campus?}

\hypertarget{milestones}{%
\subsection{Milestones}\label{milestones}}

\hypertarget{we-have-things-to-celebrate-what-should-we-do}{%
\subsubsection{We have things to celebrate. What should we
do?}\label{we-have-things-to-celebrate-what-should-we-do}}

+

This is going to be hard, and there's no right way to do it. But the
coronavirus is happening right now, and it's not specifically happening
to you or your family. There are
\href{https://www.nytimes.com/2020/04/15/style/self-care/birthday-party-coronavirus-online-zoom.html}{still
ways to celebrate.}

With events, you have three options. You could postpone them
indefinitely. You could host them over video chat. Or you could have a
much smaller celebration --- masked up and distanced --- with the people
who matter the most.

\hypertarget{lets-start-with-birthdays-what-should-we-do}{%
\subsubsection{Let's start with birthdays. What should we
do?}\label{lets-start-with-birthdays-what-should-we-do}}

+

You could skip a year, and no one would be the wiser. You could try for
a socially distant park hang with friends who live near you. Or, you
could host a party over
\href{https://www.nytimes.com/2020/04/02/us/coronavirus-birthday-party.html}{video
conferencing,} which is really more fun than it sounds. Set a time
limit, call on people the way you would if you were leading a meeting
and decorate your own place.

You could also ask everyone to bake the same easy mug cake, or
something, so it feels like you're all together. Get creative, and take
the long view. Next year, you'll be a year older. Maybe you can host one
birthday party with two times the fun?

This is, of course, harder with kids. Start by talking to them about how
this year is going to be different. You might want to try for a
split-level birthday: in-person activities at home with the members of
your household, and online activities with friends.

At home, you can build a fort at home together, or try for an ambitious
craft project. Try some funky baking project, or give them ``magic
powers'' for the day.

For the virtual party, keep it short and sweet. One hour is probably the
maximum amount of time, so make sure you give your guests both a start
time and end time. The other parents on the call will appreciate it.

Not sure how to fill time online? You could hire virtual entertainment.
There are still music classes and magicians and clowns, and they offer
virtual shows. (The website Mommy Poppins has
\href{https://mommypoppins.com/birthday-parties/best-virtual-party-performers-and-kids-birthday-parties-at-home}{a
good list.})

 Read More
\href{https://www.nytimes.com/2020/05/02/smarter-living/zoom-birthday-party.html}{Go
Ahead, Blow Out the Candles on Zoom}

\hypertarget{what-about-our-wedding-what-should-we-do}{%
\subsubsection{What about our wedding? What should we
do?}\label{what-about-our-wedding-what-should-we-do}}

+

You could
\href{https://www.nytimes.com/2020/07/04/fashion/weddings/no-zoom-wedding-next-year.html}{postpone.}
Lots of people are. If you really want to be able to celebrate with
everyone you love in person, be flexible about choosing a date --- you
wouldn't want to have to re-cancel.

Long-held wedding decorum may no longer be applicable, especially when
it comes to uninviting guests. To avoid having to retract invitations,
Steve Moore, an owner of
\href{https://www.sinclairandmoore.com/}{Sinclair and Moore Events in
Seattle,} says couples should either skip the save-the-date cards or
include a disclaimer noting they plan to adhere to state and federal
guidelines for gatherings. ``We'll ask in advance for your flexibility,
understanding and grace,'' he offered as a suggestion.

You could try to
\href{https://www.nytimes.com/2020/04/25/fashion/weddings/how-to-livestream-your-wedding.html}{host
a wedding over Zoom} (or
\href{https://www.nytimes.com/2020/05/25/fashion/weddings/pass-the-popcorn-and-shhh-the-wedding-is-about-to-start.html}{a
drive-in theater}). Maybe your first or second anniversary is the big
party instead, which you can host without the pressure.

Or, just invite your
\href{https://www.nytimes.com/2017/10/12/smarter-living/what-to-know-about-having-a-microwedding.html}{very
closest people} to celebrate in person, making sure they all get tested
and swabbed before you all get together. Your immediate family, your
partner's immediate family, your best friends. There will be other
moments in your lives when you can get a bigger group together to
celebrate.

 Read More
\href{https://www.nytimes.com/2020/06/25/fashion/weddings/what-will-happen-with-weddings.html}{What
Will Happen With Weddings?}

\hypertarget{what-about-a-funeral}{%
\subsubsection{What about a funeral?}\label{what-about-a-funeral}}

+

There are steps you can take
\href{https://www.nytimes.com/2020/05/16/opinion/coronavirus-jewish-funeral.html}{to
treat the dead with dignity.}

If someone you love is in hospice or nearing the end, start planning.
Death is always logistical. Now, it's even more so. Call a few funeral
homes to ask about pricing and procedure, and check your state laws on
the size of the gathering.

If your religion mandates you bury someone within a specific period of
time, turn to a religious leader for help. They will have worked with
other people in your situation, and should have suggestions about how to
cope.

Understand that as much as you would like to honor your loved one in
your traditional ways, the changed world has also affected funeral
services. Consider the risk of a funeral gathering, and the comfort
levels and health concerns of those who would like to attend. You may
choose to organize a smaller in-person memorial, schedule a service at a
later date or hold a Zoom memorial.

As for grieving?
\href{https://www.nytimes.com/2020/05/28/us/politics/coronavirus-100000-trump-biden.html}{It's
going to be hard.}
\href{https://www.nytimes.com/2020/04/29/opinion/letters/coronavirus-mourning.html}{You
cannot hug your friends,} and you cannot have them over. But with
technology, you can lean on your community
\href{https://www.nytimes.com/2020/04/29/opinion/coronavirus-funerals-mourning.html}{virtually.}
Maybe that's just regular phone calls to your support network over the
first few weeks. Or maybe that's a Zoom shiva or wake. If you go the
videoconferencing route, ask everyone to show up with a memory or a poem
to share. Call on people one by one.

If your friend lost someone they loved,
\href{https://www.nytimes.com/2020/04/27/business/coronavirus-sympathy-cards.html}{send
a condolence note or sympathy card.} It means a lot,
\href{https://www.nytimes.com/2020/04/09/style/best-way-to-write-a-condolence-note-coronavirus.html}{even
over email.}

 Read More
\href{https://www.nytimes.com/2020/05/14/magazine/funeral-home-covid.html}{How
Do You Maintain Dignity for the Dead in a Pandemic?}

\hypertarget{staying-in}{%
\subsection{Staying in}\label{staying-in}}

\hypertarget{do-you-have-any-strategies-for-staving-off-boredom-at-home}{%
\subsubsection{Do you have any strategies for staving off boredom at
home?}\label{do-you-have-any-strategies-for-staving-off-boredom-at-home}}

+

Our section \href{https://www.nytimes.com/spotlight/at-home}{At Home} is
your one-stop guide to riding out the pandemic. In it, you'll find
advice and guidance on everything you'll need to pass the time: what to
watch, cook, read, listen to and more, along with tips for dealing with
the psychological and emotional effects of quarantine.

\hypertarget{what-should-i-do-to-keep-my-home-safe-should-i-still-be-disinfecting-everything}{%
\subsubsection{What should I do to keep my home safe? Should I still be
disinfecting
everything?}\label{what-should-i-do-to-keep-my-home-safe-should-i-still-be-disinfecting-everything}}

+

Always remember the basics, like washing your hands frequently (and
reminding members of your household to do the same).

The coronavirus isn't really transmitted on surfaces. Or, at least,
surfaces are a much lower risk than person-to-person transmission. So
although disinfecting isn't a bad idea, you probably don't need to be
quite as vigilant now as we initially thought.

Mostly, keep people who might be infected out of your house. But the
safest thing you can do to keep your home safe is to keep yourself safe
when you're outside of your home. That means wearing a mask and socially
distancing whenever possible.

 Read More
\href{https://www.nytimes.com/2020/03/27/smarter-living/how-you-can-make-your-home-safer.html}{How
You Can Make Your Home Safer}

\hypertarget{im-working-at-home-and-stiff-as-heck-tips}{%
\subsubsection{I'm working at home and stiff as heck.
Tips?}\label{im-working-at-home-and-stiff-as-heck-tips}}

+

You'll want to focus on your wrists and forearms, your shoulders, and
your chest and back.
\href{https://www.nytimes.com/2020/05/19/well/three-stretches-to-tend-to-the-aches-and-pains-of-working-from-home.html}{Read
here for three stretches that'll help,} but for now let's try this one
for your wrists:

* Stand with one arm out in front of you and your palm facing the
ground.

* With your other hand, gently pull the fingers of the outstretched arm
back. You should feel the stretch in the underside of your wrist.

* Hold for a moment or two, then release.

* Next, keeping your arm straight, use your other hand to push your
fingers and palm down and toward your body. Hold a few seconds, then
release.

* Repeat with your other arm.

Don't forget to stretch and strengthen your whole body, too. See The
Times's
\href{https://www.nytimes.com/guides/well/strength-training-plyometrics}{9-Minute
Strength Workout,} which includes exercises you can do in a pinch.

 Read More
\href{https://www.nytimes.com/2020/05/19/well/three-stretches-to-tend-to-the-aches-and-pains-of-working-from-home.html}{Three
Stretches to Tend to the Aches and Pains of Working From Home}

\hypertarget{im-trying-to-exercise-again-its-really-hard-tips}{%
\subsubsection{I'm trying to exercise again. It's really hard.
Tips?}\label{im-trying-to-exercise-again-its-really-hard-tips}}

+

If you've had the coronavirus, expect
\href{https://www.nytimes.com/2020/05/06/opinion/coronavirus-recovery.html}{a
long recovery.} Even if you were an athlete beforehand,
\href{https://www.nytimes.com/2020/05/20/well/returning-to-exercise-training-recovery-coronavirus.html}{you
are not going to be back at your peak capacity for a while.} The usual
return-to-play criteria for sick athletes do not apply, said Dr. James
Hull, a sports pulmonologist at Royal Brompton Hospital in London.

``We have seen people have some mild symptoms to start with and seem to
improve,'' he said, ``only to then deteriorate really badly at seven
days following their first symptoms.''

Even if you haven't been symptomatic, most everyone took at least a few
weeks off of their normal workout routine in the early part of the
lockdown. A lot of people are a lot less toned than they used to be.

If that's you,
\href{https://www.nytimes.com/2020/06/01/well/move/coronavirus-exercise-lockdown-quarantine-sports-weights-running-injuries.html}{start
slow.} Focus on strength, rather than cardio, because inactivity eats
away at muscle mass.

Dawdle through any resumption of weight training, said Brad Schoenfeld,
an associate professor of exercise science at Lehman College in New
York, who researches resistance exercise. ``If you have been doing
almost no training,'' he said, ``plan to start at 50 percent of the
volume and intensity of your prior workouts'' when you return to the
gym.

If you're running outside, wear a mask, even if it's uncomfortable.
You're breathing heavily, and you could be spitting out --- or sucking
in --- virus droplets. No one wants that.

 Read More
\href{https://www.nytimes.com/2020/07/02/well/move/the-well-summer-workout-challenge.html}{The
Well Summer Workout Challenge}

\hypertarget{what-about-pets-can-they-get-the-coronavirus-can-they-transmit-it}{%
\subsubsection{What about pets? Can they get the coronavirus? Can they
transmit
it?}\label{what-about-pets-can-they-get-the-coronavirus-can-they-transmit-it}}

+

In April,
\href{https://www.nytimes.com/2020/04/28/us/dog-coronavirus-positive-test.html}{a
dog in North Carolina tested positive for the coronavirus,} as did
\href{https://www.nytimes.com/2020/04/22/health/cats-pets-coronavirus.html}{two
pet cats in New York} and
\href{https://newsroom.wcs.org/News-Releases/articleType/ArticleView/articleId/14084/Update-Bronx-Zoo-Tigers-and-Lions-Recovering-from-COVID-19.aspx}{eight
tigers at the Bronx Zoo.} However, there has been no evidence that pets
such as dogs or cats can spread the coronavirus, according to
\href{https://www.who.int/docs/default-source/inaugural-who-partners-forum/coronavirus-poster-english-srilanka.pdf?sfvrsn=289dedc3_0}{the
World Health Organization} and
\href{https://www.cdc.gov/coronavirus/2019-ncov/daily-life-coping/animals.html?CDC_AA_refVal=https\%3A\%2F\%2Fwww.cdc.gov\%2Fcoronavirus\%2F2019-ncov\%2Fprepare\%2Fanimals.html}{the
C.D.C. (The animals showed mostly mild symptoms --- though} a tiger was
``visibly sick'' ---
\href{https://www.cdc.gov/coronavirus/2019-ncov/daily-life-coping/animals.html?CDC_AA_refVal=https\%3A\%2F\%2Fwww.cdc.gov\%2Fcoronavirus\%2F2019-ncov\%2Fprepare\%2Fanimals.html}{and
have recovered.})

Still, if you are sick with the coronavirus, it is best not to pet your
family dog. And avoid petting other people's dogs as well.

But do
\href{https://www.nytimes.com/2020/04/21/well/coronavirus-pet-care-grooming.html}{take
care of your animal's mental health} --- this is stressful for them,
too. Staying at home for months on end can
\href{https://www.nytimes.com/2020/05/27/smarter-living/how-to-prepare-your-dog-to-be-left-at-home-alone-again.html}{acclimate
your dog to an unrealistic reality,} experts said. Start leaving for a
bit, even to walk, so they get used to the idea that you won't always be
around.

If you know
\href{https://www.nytimes.com/2020/04/19/world/europe/coronavirus-spain-pets.html}{someone
is sick and has pets,} check in on them. Some animals are being
\href{https://www.nytimes.com/2020/06/23/nyregion/coronavirus-pets.html}{left
behind} by the virus.

 Read More
\href{https://www.nytimes.com/2020/03/04/science/animals-pets-coronavirus.html}{Coronavirus
and Your Dog: No Need to Panic Yet}

\hypertarget{i-want-to-renovate-my-home-what-should-i-do}{%
\subsubsection{I want to renovate my home. What should I
do?}\label{i-want-to-renovate-my-home-what-should-i-do}}

+

You're in luck. Home renovation is the new sourdough bread.

Almost a month after the phase 1 reopening of New York City that allowed
contractors and their crews back into residential buildings, ``we're
redefining what `full steam ahead' means,'' said Steve Mark, the chief
executive of SMI Construction. ``It's not going to mean what it used
to.''

It's a tough balance as contractors and homeowners try to navigate
restrictions, but hammers are banging once more.

Other than pandemic-related changes, renovating a home is a pretty
standard process. You budget and design, work with a contractor and/or a
designer, and watch as mishaps and mayhem happen around you.

 Read More
\href{https://www.nytimes.com/2018/03/07/style/home-renovation-advice.html}{Advice
for Renovating Your Home}

\hypertarget{i-miss-looking-at-art-what-do-i-do}{%
\subsubsection{I miss looking at art. What do I
do?}\label{i-miss-looking-at-art-what-do-i-do}}

+

Some museums are
\href{https://www.nytimes.com/2020/05/29/arts/design/museums-interactive-coronavirus.html}{trying
to reopen.} Science museums designed for kids are opting for no-touch
exhibits, and the Metropolitan Museum of Art, in New York, plans to open
in late August. Some museums have been
\href{https://www.nytimes.com/2020/04/29/arts/design/how-do-you-close-a-museum.html}{forced
to close.}

Some
\href{https://www.nytimes.com/2020/03/26/arts/design/new-york-art-galleries.html}{galleries
and museums} are uploading their collections, so you can
\href{https://www.nytimes.com/2020/04/09/arts/design/virtual-art-galleries.html}{``visit''
virtually.} The Arts section ran a series of reviews of
\href{https://www.nytimes.com/2020/04/23/arts/design/best-virtual-museum-guides.html}{virtual
shows.} It's, of course, not the real thing. But by zooming in, you can
``get up close'' to the surface of canvases in a way that you might
never have been able to before.

 Read More
\href{https://www.nytimes.com/2020/06/15/arts/design/art-therapy-museums-virus.html}{Museums
Embrace Art Therapy Techniques for Unsettled Times}

\hypertarget{i-miss-getting-dressed-up-and-i-have-hot-new-threads-should-i--wear-them}{%
\subsubsection{I miss getting dressed up. And I have hot new threads.
Should I \ldots{} wear
them?}\label{i-miss-getting-dressed-up-and-i-have-hot-new-threads-should-i--wear-them}}

+

Um, yes.

Extravagant purchases innocently made in February and March, before the
extent of the pandemic was known, have become markers of a fast-receding
era of freedom. Some purchasers have even saved their sales receipts, as
if they were historic documents. Many of these items now languish in
closets. Others are put to good use.

If that's you, wear them. Beautiful clothes are beautiful clothes. You
deserve to be stunning and extra, even if you've got nowhere to be. Wear
\href{https://www.thecut.com/2020/07/i-dressed-like-catherine-the-great-for-a-week.html\#comments}{a
gown} and a mask to walk the dog. It's a pandemic. There are no rules.
You're fierce no matter where you are.

 Read More
\href{https://www.nytimes.com/2020/07/08/style/why-did-i-buy-this.html}{They
Splurged on Fancy Clothes Before Quarantine. Now What?}

\hypertarget{money-and-work}{%
\subsection{Money and Work}\label{money-and-work}}

\hypertarget{help-me-understand-the-economic-fallout}{%
\subsubsection{Help me understand the economic
fallout.}\label{help-me-understand-the-economic-fallout}}

+

It feels a lot like 2008, doesn't it? Peter S. Goodman, an economics
correspondent for The Times, says there's a key difference between that
economic turmoil and today's: the utter unpredictability of the
outbreak's spread.

``The disaster feels eerily familiar, with trillions of dollars in
wealth annihilated near-daily and deepening fears that businesses will
fail,''
\href{https://www.nytimes.com/2020/03/13/business/coronavirus-global-economy.html}{he
writes.} ``Yet the traditional policy prescriptions seem no match for
the affliction at hand.''

But even the 2008 crisis pales in comparison to the scale of the
destruction wrought by the coronavirus. The United States lost 20
million jobs in a single month, a fall that one expert called
\href{https://www.nytimes.com/interactive/2020/05/08/business/economy/april-jobs-report.html}{"literally
off the charts."} And the very means of controlling the health crisis
--- keeping workers home, limiting travel and disrupting commerce ---
risk making the economic crisis worse.

Even though markets seem to be rebounding, the
\href{https://www.nytimes.com/2020/07/09/business/economy/unemployment-claims-coronavirus.html}{hiring
outlook is dim.} Total new unemployment claims have edged up from their
mid-June lows.

 Read More
\href{https://www.nytimes.com/2020/05/10/business/stock-market-economy-coronavirus.html}{Repeat
After Me: The Markets Are Not the Economy}

\hypertarget{im-a-small-business-owner-can-i-get-relief}{%
\subsubsection{I'm a small-business owner. Can I get
relief?}\label{im-a-small-business-owner-can-i-get-relief}}

+

Small-business owners are struggling. Some are
\href{https://www.nytimes.com/2020/07/13/business/small-businesses-coronavirus.html}{giving
up} or
\href{https://www.nytimes.com/2020/06/26/business/small-business-coronavirus-survival.html}{adapting}
to the point of being
\href{https://www.nytimes.com/2020/05/18/business/small-business-coronavirus-pandemic-mikes-organic.html}{almost
unrecognizable.} Some owners have
\href{https://www.nytimes.com/2020/06/23/business/coronavirus-great-recession-2008-lessons.html}{learned
from the 2008 recession,} but they're still struggling.

The stimulus bills enacted in March offer help for the millions of
American small businesses. Those eligible for aid are businesses and
nonprofit organizations with fewer than 500 workers, including sole
proprietorships, independent contractors and freelancers. Some larger
companies in some industries are also eligible. The help being offered,
which is being managed by the Small Business Administration, includes
the Paycheck Protection Program and the Economic Injury Disaster Loan
program.

But lots of folks have
\href{https://www.nytimes.com/interactive/2020/05/07/business/small-business-loans-coronavirus.html}{not
yet seen payouts.} Even those who have received help are confused: The
rules are draconian, and some are stuck sitting on
\href{https://www.nytimes.com/2020/05/02/business/economy/loans-coronavirus-small-business.html}{money
they don't know how to use.} Many small-business owners are getting less
than they expected or
\href{https://www.nytimes.com/2020/06/10/business/Small-business-loans-ppp.html}{not
hearing anything at all.}

 Read More
\href{https://www.nytimes.com/article/small-business-loans-stimulus-grants-freelancers-coronavirus.html}{F.A.Q.
on Coronavirus Relief for Small Businesses, Freelancers and More}

\hypertarget{what-are-my-rights-if-i-am-worried-about-going-back-to-work}{%
\subsubsection{What are my rights if I am worried about going back to
work?}\label{what-are-my-rights-if-i-am-worried-about-going-back-to-work}}

+

Employers have to provide
\href{https://www.osha.gov/SLTC/covid-19/standards.html}{a safe
workplace} with policies that protect everyone equally. And if one of
your co-workers tests positive for the coronavirus, the C.D.C. has said
that
\href{https://www.cdc.gov/coronavirus/2019-ncov/community/guidance-business-response.html}{employers
should tell their employees} --- without giving you the sick employee's
name --- that they may have been exposed to the virus.

 Read More
\href{https://www.nytimes.com/article/coronavirus-money-unemployment.html}{A
Hub for Help During the Coronavirus Crisis}

\hypertarget{i-have-questions-about-my-stimulus-check}{%
\subsubsection{I have questions about my stimulus
check.}\label{i-have-questions-about-my-stimulus-check}}

+

President Trump signed
\href{https://www.nytimes.com/2020/03/26/us/coronavirus-senate-stimulus-package.html}{a
bipartisan \$2 trillion economic relief plan} to offer assistance to
tens of millions of American households affected by the
\href{https://www.nytimes.com/2020/05/07/us/coronavirus-stimulus-package.html}{coronavirus}
pandemic. Its components include
\href{https://www.nytimes.com/2020/06/25/us/politics/coronavirus-stimulus-dead-people.html}{stimulus
payments} to individuals, expanded unemployment coverage, student loan
changes, different retirement account rules and more. Most adults
received \$1,200, although some received less, according to income.

But as with most coronavirus relief packages, questions remain. Some
people received them on
\href{https://www.washingtonpost.com/business/2020/06/01/faq-stimulus-debit-card/}{debit
cards,} and the envelopes look
\href{https://twitter.com/zackstanton/status/1263908922899009536}{like
junk mail.} Don't throw them out.

Think you received a payment by mistake? Say, for
\href{https://www.nytimes.com/2020/06/25/us/politics/coronavirus-stimulus-dead-people.html}{a
deceased relative?} Don't spend the money. The Internal Revenue Service
will likely ask for it back come tax time in 2021.

If you haven't received yours yet, don't panic. Many people have already
received their payments, but many others are still waiting or wondering.
There are a lot of reasons you could be among them, even if the
government has
\href{https://www.nytimes.com/2020/04/01/business/coronavirus-stimulus-social-security.html}{removed
some of the hurdles} it initially set up.

Try the I.R.S. \href{https://www.irs.gov/coronavirus/get-my-payment}{Get
My Payment} tool again. It's supposed to help people figure out when and
how their money might be arriving. The I.R.S. is updating the
information once each day, usually in the middle of the night.

Make sure you filed the right paperwork. If you haven't had to file a
return because your gross income did not exceed \$12,200 (\$24,400 for
married couples), you still qualify for a payment. But if you're not a
recipient of S.S.I. or V.A. benefits, you should fill out a special form
for non-filers.

Also, be careful of
\href{https://www.nytimes.com/2020/04/22/technology/stimulus-checks-hackers-coronavirus.html}{scammers.}

 Read More
\href{https://www.nytimes.com/article/where-is-my-stimulus-payment.html}{Didn't
Get Your Stimulus Payment Yet? Here's What to Do}

\hypertarget{how-does-unemployment-insurance-work}{%
\subsubsection{How does unemployment insurance
work?}\label{how-does-unemployment-insurance-work}}

+

The
\href{https://www.nytimes.com/2020/03/25/us/politics/whats-in-coronavirus-stimulus-bill.html}{stimulus
plan} includes a significant expansion of unemployment benefits that
would extend jobless insurance by 13 weeks. The program is aimed to help
freelancers, part-time workers, furloughed employees and gig workers,
such as Uber drivers, as well as full-time employees who have lost their
jobs.

States have set their own rules for eligibility and benefits, which are
generally calculated as a percentage of your income over the past year,
up to a certain maximum. Some states are more generous than others, but
unemployment typically replaces about 45 percent of your lost income.
Eligible workers had been getting an extra \$600 per week on top of
their state benefit, but that expired at the end of July

One important note: You might not have to lose your job to qualify. If
you're quarantined or have been furloughed --- and you're not being paid
but expect to return to your job eventually --- you may be able to get
unemployment benefits.

 Read More
\href{https://www.nytimes.com/article/coronavirus-money-unemployment.html}{A
Hub for Help During the Coronavirus Crisis}

\hypertarget{i-need-help-paying-my-bills-what-should-i-do}{%
\subsubsection{I need help paying my bills. What should I
do?}\label{i-need-help-paying-my-bills-what-should-i-do}}

+

Student loans: The Department of Education has granted a payment waiver
of at least 60 days to many people. But it's not necessarily automatic.
In general, you have to call your loan servicer to request a waiver and
to make sure that your loan is eligible. The waiver does not apply to
private student loans. One big private lender, Sallie Mae, said it was
offering suspension of payment for up to three months, with no damage to
a borrower's credit.

With the stimulus package, there will be
\href{https://www.nytimes.com/article/coronavirus-stimulus-package-questions-answers.html}{automatic
payment suspensions} for any student loan held by the federal government
until Sept. 30. It may be hard to contact many of the loan servicers
right now, so check your account online in the coming weeks.

Utility bills: Some utility providers are offering to stop cutting
people off for nonpayment. A number of large internet companies have
agreed not to terminate residential or small-business customers who
can't pay their bills. Exact policies and requirements vary, though, so
if you need help, you should call and ask.

Housing: There's also a good chance you can delay your mortgage payment
if the outbreak has left you short of money. The Federal Housing Finance
Agency has instructed mortgage servicers to allow borrowers whose
mortgages are owned by Fannie Mae or Freddie Mac to delay payments. If
you rent, the best national resource we've found so far is the
search-by-state function on
\href{https://justshelter.org/community-resources/}{Justshelter.org.}
This offers information on local organizations that can provide advice
to renters in distress.

Loans and credit: Many consumer lenders are offering affected customers
help if they can't make payments. The American Bankers Association has a
list of lenders here, although the help they're offering varies. Some
say they will allow borrowers to skip payments; some are offering other
accommodations.

 Read More
\href{https://www.nytimes.com/article/coronavirus-money-unemployment.html}{A
Hub for Help During the Coronavirus Crisis}

\hypertarget{i-dont-have-health-insurance-help}{%
\subsubsection{I don't have health insurance.
Help?}\label{i-dont-have-health-insurance-help}}

+

If you lost your job (and with it your health insurance), or even if you
never signed up for a policy, you may have options. Some states and
employers have loosened the rules around when you can sign up for
insurance because of the pandemic.

There's also the possibility of continuing your job-based coverage under
a law known as the Consolidated Omnibus Budget Reconciliation Act, or
COBRA; applying for Medicaid with your state agency; or getting an
individual plan at Healthcare.gov.
\href{https://www.nytimes.com/2020/03/25/upshot/coronavirus-health-insurance-faq.html}{It
depends on your circumstances} which route makes sense.

The government has also
\href{https://www.nytimes.com/2020/05/12/business/employer-health-plans-coronavirus.html}{made
it easier} for people to join or change their employer-provided health
insurance plans, if your company decides to offer the option.

 Read More
\href{https://www.nytimes.com/article/coronavirus-money-unemployment.html}{A
Hub for Help During the Coronavirus Crisis}

\hypertarget{what-about-my-401k}{%
\subsubsection{What about my 401(k)?}\label{what-about-my-401k}}

+

Watching your balance bounce around can be scary. You may be wondering
if you should decrease your contributions ---
\href{https://www.nytimes.com/2020/03/12/business/coronavirus-401k-contributions.html}{don't!}
In fact, if your employer matches any part of your contributions, make
sure you're at least saving as much as you can to get that ``free
money.''

Many components of the stimulus package include changes to rules
regarding retirement account withdrawals, borrowing and penalties.

 Read More
\href{https://www.nytimes.com/article/coronavirus-money-unemployment.html}{A
Hub for Help During the Coronavirus Crisis}

\hypertarget{should-i-be-putting-money-in-a-mattress}{%
\subsubsection{Should I be putting money in a
mattress?}\label{should-i-be-putting-money-in-a-mattress}}

+

That's not a good idea. Even if you're retired, having a balanced
portfolio of stocks and bonds so that your money keeps up with
inflation, or even grows, makes sense. But you may want to think about
having enough cash
\href{https://www.nytimes.com/article/coronavirus-money-advice.html}{to
cover basic needs} if you're retired. One financial planner recommended
that retirees set aside enough cash for a year's worth of living
expenses and big payments needed over the next five years.

 Read More
\href{https://www.nytimes.com/article/coronavirus-money-unemployment.html}{A
Hub for Help During the Coronavirus Crisis}

\hypertarget{should-i-refinance-my-mortgage}{%
\subsubsection{Should I refinance my
mortgage?}\label{should-i-refinance-my-mortgage}}

+

It could be a good idea, because mortgage rates have
\href{https://www.nytimes.com/2020/07/16/business/mortgage-rates-below-3-percent.html}{never
been lower.} Refinancing requests have pushed mortgage applications to
some of the highest levels since 2008, so be prepared to get in line.

But defaults are also up, so if you're thinking about buying a home, be
aware that some lenders have tightened their standards.

 Read More
\href{https://www.nytimes.com/article/coronavirus-money-unemployment.html}{A
Hub for Help During the Coronavirus Crisis}

\hypertarget{what-about-sick-leave}{%
\subsubsection{What about sick leave?}\label{what-about-sick-leave}}

+

A coronavirus emergency relief package, passed by Congress in mid-March,
gives qualified workers
\href{https://www.nytimes.com/2020/03/19/upshot/coronavirus-paid-leave-guide.html}{two
weeks of paid sick leave} if they are ill, quarantined or seeking a
diagnosis or preventive care for the coronavirus, or if they are caring
for sick family members. It gives 12 weeks of paid leave to people
caring for children whose schools are closed or whose child care
provider is unavailable because of the coronavirus.

Many workers at small and midsize companies and nonprofits can get the
paid leave, as can government employees, as long as they've been
employed at least 30 days. The law includes people who don't typically
get such benefits, like part-time and gig-economy workers. But it
excludes those at companies with more than 500 people, including many
front-line employees at many major retailers and fast-food restaurants.

Workers at places with fewer than 50 employees are included, but the
Labor Department could exempt small businesses if providing leave would
put them out of business. Employers can also decline to give leave to
workers on the front lines of the crisis: health care providers and
emergency responders.

 Read More
\href{https://www.nytimes.com/2020/03/19/upshot/coronavirus-paid-leave-guide.html}{Who
Qualifies for Paid Leave Under the New Coronavirus Law}

\hypertarget{how-do-i-get-a-new-job-right-now}{%
\subsubsection{How do I get a new job right
now?}\label{how-do-i-get-a-new-job-right-now}}

+

First, if you were laid off, recognize that
\href{https://www.nytimes.com/2020/06/21/smarter-living/coronavirus-laid-off-career-advice.html}{you
might be grieving.} Even if it wasn't your fault --- and it probably
wasn't your fault right now --- that's a hard thing to swallow. Give
yourself some time before jumping right back in.

If you're just starting out, you're not alone.
\href{https://www.nytimes.com/2020/03/27/business/coronavirus-class-of-2020-jobs.html}{People
in their early 20s} are entering a terrible job market, and might do
well to manage their expectations, at least for a few months. No matter
who you are, or what you were doing before, you're probably going to
have to be willing to be flexible. Lots of industries are laying off
people for the foreseeable future, and
\href{https://www.nytimes.com/2020/05/27/business/coronavirus-careers-on-hold.html}{many
careers are on hold.} Still, others are hiring in droves.

Becoming
\href{https://www.nytimes.com/2020/05/18/health/coronavirus-contact-tracing-jobs.html}{a
contract tracer} --- sort of a private investigator for viral spread ---
is one option. Taking a lower-wage service job ---
\href{https://www.nytimes.com/2020/03/22/business/coronavirus-hiring-jobs.html}{grocery
stores, pizza delivery} --- might also work.

This is going to be a slog. But the American economy might be changing
for the better. Training skilled workers went out of fashion. Now, it
may be coming back.

 Read More
\href{https://www.nytimes.com/2020/07/13/business/coronavirus-retraining-workers.html}{The
Pandemic Has Accelerated Demands for a More Skilled Work Force}

\hypertarget{i-have-a-work-visa-whats-going-on-with-that}{%
\subsubsection{I have a work visa. What's going on with
that?}\label{i-have-a-work-visa-whats-going-on-with-that}}

+

Again, no one really knows. The Trump administration is
\href{https://www.nytimes.com/2020/05/12/us/foreign-workers-visas-immigrants.html}{unlikely
to allow laid-off holders of H-1B and other work visas to extend their
stay in the country} amid the coronavirus pandemic. Many have been
waiting in a backlog for several years to obtain permanent legal
residency through their employer, and now face the prospect of
deportation.

\href{https://www.nytimes.com/2020/06/22/world/coronavirus-updates.html?action=click\&module=Top\%20Stories\&pgtype=Homepage\#link-2dd2e3cb}{The
temporary suspension of work visas} is opposed by a lot of
\href{https://www.nytimes.com/2020/06/19/us/foreign-worker-visas-trump-coronavirus.html}{businesses}
--- including tech companies in Silicon Valley and manufacturers ---
whose leaders say it will block their ability to recruit workers for
jobs that Americans are not willing or able to perform.
\href{https://www.nytimes.com/2020/04/13/us/coronavirus-foreign-doctors-nurses-visas.html}{Foreign
doctors} are willing to come help, but most cannot get into the country.

If you can, get in touch with an immigration lawyer and your employer.
There may be legal recourse, and some attorneys are working on such
cases for free.

 Read More
\href{https://www.nytimes.com/2020/06/12/us/politics/coronavirus-trump-immigration-policies.html}{Trump
Administration Moves to Solidify Restrictive Immigration Policies}

\hypertarget{travel}{%
\subsection{Travel}\label{travel}}

\hypertarget{is-flying-safe}{%
\subsubsection{Is flying safe?}\label{is-flying-safe}}

+

Before you book your trip (and there have been some great deals), ask
whether flying is absolutely necessary. Do you have to go? Or do you
just want to?

If you do have to fly, the most important thing you can do is keep your
mask on throughout your travel. It's the best way you can keep yourself
and the people around you safe. Bring hand sanitizer too.

Before you go, plan out every step of the process.

Packing: In your carry-on, be extra careful. Separate out your
\href{https://www.nytimes.com/2020/05/21/business/memorial-day-travel-2020-coronavirus.html}{food
and liquids,} and double check that you don't have anything sharp or
dangerous. Any backlog in a security line might compromise social
distancing, and opening the valise creates unnecessary opportunities for
exposure.

Getting to the airport: If you're in a rideshare, keep your windows down
and your mask on. If someone in your household can drive you, that's
safer.

Check in: Most airlines suggest that travelers download their app for
touchless boarding, which will minimize the number of times you have to
hand over documents or touch screens.

On board: Industry and government policies are shifting, almost daily,
but most airlines now require passengers to wear masks while boarding
and during flights.

Even boarding has changed. Southwest, for example, has had passengers
board in groups of 10, with people only lining up on one side of its
boarding poles. United is boarding people by row, with people sitting in
the back of the plane boarding after pre-boarding groups. Most airlines
are boarding fewer people at a time to keep crowds from forming at the
gate, on the jet bridge and as people get on the plane.

On the flight, things get different. Some airlines are spacing their
passengers out by leaving the middle seat open. Some are not. If you are
concerned about sitting next to someone and have a choice of airlines,
consider their different policies. Call and ask before you book a flight
with a specific company.

When the plane lands, maybe wait in your seat as passengers get their
luggage from overhead bins, to keep social distance. At baggage claim,
try to do the same.

 Read More
\href{https://www.nytimes.com/2020/06/01/business/coronavirus-airports-airlines.html}{Airlines
Say It's Safe to Travel. But Is It? and How The Coronavirus Changed Air
Travel: A Visual Diary of a Flight}

\hypertarget{im-american-can-i-leave-the-country}{%
\subsubsection{I'm American. Can I leave the
country?}\label{im-american-can-i-leave-the-country}}

+

That depends on where you want to go.

In recent weeks, many countries around the world, including the United
States, have imposed
\href{https://www.nytimes.com/2020/07/10/travel/state-travel-restrictions.html}{travel
restrictions} to help curb the spread of the coronavirus. Airport
closures, the suspension of all incoming and outgoing flights, and
nationwide lockdowns are just some of the measures countries are
adopting in an effort to help contain the pandemic.

Even as many countries remain off-limits to American visitors because of
the high rate of coronavirus within the United States, about two dozen
others have started to welcome U.S. citizens.

 Read More
\href{https://www.nytimes.com/2020/07/07/world/europe/american-passport-privilege-coronavirus.html}{Travel
Restrictions on Americans Erode a Sense of Passport Privilege}

\hypertarget{flights-to-some-tourist-destinations-are-cheap-but-is-it-ethical}{%
\subsubsection{Flights to some tourist destinations are cheap. But is it
ethical?}\label{flights-to-some-tourist-destinations-are-cheap-but-is-it-ethical}}

+

That depends. Questions about the morality of traveling are not new. For
years, people have raised concerns about
\href{https://www.nytimes.com/2019/06/13/opinion/letters/travel-climate-change.html}{climate
change} and traveling to countries with
\href{https://www.nytimes.com/2019/05/16/t-magazine/ethical-travel-reporting.html}{oppressive
governments.} Certain travel activities, like
\href{https://www.nytimes.com/2019/06/19/travel/thailand-elephant-tourism-humane.html}{elephant
riding,} also raise eyebrows.

On the one hand, the
\href{https://www.nytimes.com/2020/07/08/travel/travel-discounts-vouchers-coronavirus.html}{economy
of many vacation destinations,} like the Caribbean, often depend on
tourism. And cheap flights
\href{https://www.nytimes.com/2020/03/21/travel/budget-travel-coronavirus.html}{are
tempting.} But many countries have been spared outbreaks, and have much
weaker health systems than we do. As with everything these days, assess
the risks.

 Read More
\href{https://www.nytimes.com/2020/05/07/world/europe/coronavirus-reopening-costs.html}{Reopenings
Mark a New Phase: Global `Trial-and-Error' Played Out in Lives}

\hypertarget{what-are-other-strategies-for-getting-away}{%
\subsubsection{What are other strategies for getting
away?}\label{what-are-other-strategies-for-getting-away}}

+

Even domestic travel is hard right now: Some states have imposed
\href{https://www.nytimes.com/2020/07/10/travel/state-travel-restrictions.html}{travel
restrictions} on out-of-town folks, and the list changes every day. The
day before you go, check restrictions. But it might cause the least
stress and hassle just to stay close to home this time around.

But that doesn't mean you can't have fun. You could buy or rent
\href{https://www.nytimes.com/2020/07/01/style/camper-vans-mercedes.html}{a
van.} Or
\href{https://www.nytimes.com/2020/07/02/style/boat-sales-summer.html}{a
boat.} Both industries are booming. National parks might be crowded, but
you could take a road trip to
\href{https://www.nytimes.com/2020/06/04/travel/national-parks-social-distancing-coronavirus.html}{an
alternative destination.} Nature can be healing for your soul. Go bask.

 Read More
\href{https://www.nytimes.com/2020/06/14/business/coronavirus-road-trip.html}{Life
on the Road in Pandemic America}

\hypertarget{my-trip-was-canceled-how-do-i-get-a-refund}{%
\subsubsection{My trip was canceled. How do I get a
refund?}\label{my-trip-was-canceled-how-do-i-get-a-refund}}

+

Refunds are
\href{https://www.nytimes.com/2020/04/18/your-money/coronavirus-refunds.html}{a
pain in the neck.}
\href{https://www.nytimes.com/2020/06/12/travel/virus-airlines-private-arbitration.html}{Companies
are going to be annoying about it.} You have to be annoying about it
right back.

If your flight was canceled, you're in for an uphill battle. The airline
has to give you a refund, but they might drag their feet. And if you
booked with
\href{https://www.nytimes.com/2020/04/03/travel/coronavirus-refund-travel-ota.html}{an
agency,} it may be more complicated. Also,
\href{https://www.nytimes.com/2020/05/12/travel/refunds-or-credits-travelers-and-businesses-face-off.html}{don't
let them talk you into a credit.} You're entitled to a refund. Get a
refund. Here's how:

First, try to resolve it on your own --- send a few emails to establish
a paper trail and put on your best ``I'd like to speak to the manager''
voice. If you get nowhere,
\href{https://airconsumer.dot.gov/escomplaint/ConsumerForm.cfm}{file a
complaint with the Department of Transportation.} Be as specific as
possible. Then, let the company know you filed a complaint.

Sarah Firshein, a travel writer for The Times, has been writing
\href{https://www.nytimes.com/2019/09/15/reader-center/travel-sarah-firshein-tripped-up-columnist.html}{helpful
columns} on how to get your refund.

Are you having a hard time getting a refund from a \ldots{}
\href{https://www.nytimes.com/2020/05/25/travel/coronavirus-refunds-overseas-adventure-travel.html}{Travel
company?}
\href{https://www.nytimes.com/2020/07/07/travel/virus-refunds-hotel-franchises.html}{Hotel?}
\href{https://www.nytimes.com/2020/04/11/travel/coronavirus-travel-trip-refunds.html}{Tour?}
\href{https://www.nytimes.com/2020/05/01/travel/trip-refund-airlines.html}{Airline?}
\href{https://www.nytimes.com/2020/06/18/travel/travel-refunds-airlines.html}{Anything?}
If you're stuck, shoot us a note:
\href{mailto:travel@nytimes.com}{\nolinkurl{travel@nytimes.com}}

\hypertarget{politics-and-government}{%
\subsection{Politics and Government}\label{politics-and-government}}

\hypertarget{how-will-this-affect-the-2020-election}{%
\subsubsection{How will this affect the 2020
election?}\label{how-will-this-affect-the-2020-election}}

+

Although the general election has yet to begin, it is already
\href{https://www.nytimes.com/2020/03/12/us/politics/coronavirus-2020-campaign.html}{being
profoundly shaped by the coronavirus.} The pandemic's effects on the
economy have
\href{https://www.nytimes.com/2020/05/29/us/politics/trump-coronavirus-polls.html}{challenged
President Trump's core message:} that Americans are better off than
before he took office.

Former Vice President Joseph R. Biden Jr. has proposed a detailed plan
for hospitals and research, echoing the language of presidents during
past crises.
\href{https://www.nytimes.com/2020/03/29/us/2020-elections-coronavirus.html}{Campaigning
looks different.} The Democratic National Convention
\href{https://www.nytimes.com/2020/06/24/us/politics/democratic-convention-milwaukee-coronavirus.html}{was
moved} from July to August, and will take place in a smaller venue and
will feature live broadcasts. President Trump recently
\href{https://www.nytimes.com/2020/07/23/us/politics/jacksonville-rnc.html}{cancelled
the portion of the Republican National Convention to be held in
Jacksonville, Fla.} The rest of the convention is unclear.

It would be enormously
\href{https://www.nytimes.com/2020/03/24/us/politics/coronavirus-2020-election-primary.html}{difficult
to cancel or postpone} the November general election, though.

 Read More
\href{https://www.nytimes.com/2020/03/14/us/politics/election-postponed-canceled.html}{Could
the 2020 Election Be Postponed? Only With Great Difficulty. Here's Why.}

\hypertarget{what-is-the-deal-with-voting-by-mail}{%
\subsubsection{What is the deal with voting by
mail?}\label{what-is-the-deal-with-voting-by-mail}}

+

Many experts and ethicists agree: The only way to enable everyone to
vote is to turn to a mail-in ballot system, because
\href{https://www.nytimes.com/2020/05/05/magazine/voting-by-mail-2020-covid.html}{going
to the polls could be a matter of life or death.}

President Trump alleged that voting by mail could lead to voter fraud
--- despite being used for years in Democratic and Republican states
without controversy. But states are
\href{https://www.nytimes.com/2020/05/21/us/vote-by-mail-trump.html}{opening
the door} to the system, and some opposition to letting
\href{https://www.nytimes.com/2020/06/10/us/politics/voting-by-mail-georgia.html}{voters
mail in their ballots} has dissipated.

Republican opposition to the system seemed driven by the conviction that
an increase in mail voting would benefit Democrats. But
\href{https://www.nytimes.com/2020/05/25/us/vote-by-mail-coronavirus.html}{the
reality is far less clear.} The pandemic is also
\href{https://www.nytimes.com/2020/06/14/us/voter-registration-coronavirus-2020-election.html}{affecting
who can vote} --- millions of people who would ordinarily register or
update their registrations in a presidential election year have not been
able to do so.

The drop in registration has the potential to depress participation in a
November presidential election that has been widely expected to break
all records for turnout.

 Read More
\href{https://www.nytimes.com/2020/03/19/us/politics/voting-by-mail-coronavirus.html}{Voting
by Mail Is the Hot New Idea. Is There Time to Make It Work?}

Design and production by Rebecca Lieberman.

Research \& Development contributions by Jimmy Chion, Jon Cohrs, Jack
Cook, Or Fleisher, Allison Ing, Amelia Pisapia, Dalit Shalom and Yuri
Victor.

Written by Amelia Nierenberg.

Reporting was contributed by Reed Abelson, Rachel Abrams, Karen Barrow
Katrin Bennhold, Pam Belluck, Chris Berdik, Matthew Bloch, Dani Blum,
Nicholas Bogel-Burroughs, Jane E. Brody, Larry Buchanan, Benedict Carey,
Elaine Chen, Keith Collins, Stacy Cowley, Johnny Diaz, Caitlin
Dickerson, Catie Edmondson, David Enrich, Sarah Firshein, Jacey Fortin,
James Glanz, Anna Goldfarb, James Gorman, Jessica Grose, Jenny Gross,
Alisha Haridasani Gupta, Christine Hauser, Tim Herrera, Laura M. Holson,
Julia Jacobs, Zolan Kanno-Youngs, David D. Kirkpatrick, Gina Kolata,
Steven Kurutz, K.K. Rebecca Lai, Lauren Leatherby, Michael Levenson,
Ernesto Londoño, Ron Lieber, Neil MacFarquhar, Apoorva Mandavilli, Erin
McCann, Claire Cain Miller, Talya Minsberg, Melinda Wenner Moyer,
Heather Murphy, Amelia Nierenberg, Aimee Ortiz, Tara Parker-Pope, Roni
Caryn Rabin, Gretchen Reynolds, Kate Rope, Katherine Rosman, Andrea
Salcedo, Dan Saltzstein, David E. Sanger, Margot Sanger-Katz, Michael D.
Shear, Knvul Sheikh, A.C. Shilton, Tara Siegel Bernard, Lauren Sloss,
Mitch Smith, Jennifer Taitz, Neil Vigdor, Kathleen Walsh, Nancy Wartik,
Sui-Lee Wee, Pete Wells, Noah Weiland, Jin Wu, Alan Yuhas and Carl
Zimmer.

\begin{itemize}
\item
\item
\item
\item
\end{itemize}

Advertisement

\protect\hyperlink{after-bottom}{Continue reading the main story}

\hypertarget{site-index}{%
\subsection{Site Index}\label{site-index}}

\hypertarget{site-information-navigation}{%
\subsection{Site Information
Navigation}\label{site-information-navigation}}

\begin{itemize}
\tightlist
\item
  \href{https://help.nytimes.com/hc/en-us/articles/115014792127-Copyright-notice}{©~2020~The
  New York Times Company}
\end{itemize}

\begin{itemize}
\tightlist
\item
  \href{https://www.nytco.com/}{NYTCo}
\item
  \href{https://help.nytimes.com/hc/en-us/articles/115015385887-Contact-Us}{Contact
  Us}
\item
  \href{https://www.nytco.com/careers/}{Work with us}
\item
  \href{https://nytmediakit.com/}{Advertise}
\item
  \href{http://www.tbrandstudio.com/}{T Brand Studio}
\item
  \href{https://www.nytimes.com/privacy/cookie-policy\#how-do-i-manage-trackers}{Your
  Ad Choices}
\item
  \href{https://www.nytimes.com/privacy}{Privacy}
\item
  \href{https://help.nytimes.com/hc/en-us/articles/115014893428-Terms-of-service}{Terms
  of Service}
\item
  \href{https://help.nytimes.com/hc/en-us/articles/115014893968-Terms-of-sale}{Terms
  of Sale}
\item
  \href{https://spiderbites.nytimes.com}{Site Map}
\item
  \href{https://help.nytimes.com/hc/en-us}{Help}
\item
  \href{https://www.nytimes.com/subscription?campaignId=37WXW}{Subscriptions}
\end{itemize}
