Sections

SEARCH

\protect\hyperlink{site-content}{Skip to
content}\protect\hyperlink{site-index}{Skip to site index}

\href{https://myaccount.nytimes.com/auth/login?response_type=cookie\&client_id=vi}{}

\href{https://www.nytimes.com/section/todayspaper}{Today's Paper}

\href{https://www.nytimes.com/news-event/coronavirus}{The Coronavirus
Outbreak}

\begin{itemize}
\tightlist
\item
  live\href{https://www.nytimes.com/2020/08/01/world/coronavirus-covid-19.html}{Latest
  Updates}
\item
  \href{https://www.nytimes.com/interactive/2020/us/coronavirus-us-cases.html}{Maps
  and Cases}
\item
  \href{https://www.nytimes.com/interactive/2020/science/coronavirus-vaccine-tracker.html}{Vaccine
  Tracker}
\item
  \href{https://www.nytimes.com/interactive/2020/07/29/us/schools-reopening-coronavirus.html}{What
  School May Look Like}
\item
  \href{https://www.nytimes.com/live/2020/07/31/business/stock-market-today-coronavirus}{Economy}
\end{itemize}

Last Updated

July 23, 2020, 2:40 p.m. ET

July 23, 2020, 2:40 p.m. ET

\hypertarget{tesla-reaches-a-milestone-with-another-quarterly-profit}{%
\section{Tesla Reaches a Milestone With Another Quarterly
Profit}\label{tesla-reaches-a-milestone-with-another-quarterly-profit}}

This briefing is no longer being updated. Follow live updates
\href{https://www.nytimes.com/live/2020/07/23/business/stock-market-today-coronavirus}{here}.

\hypertarget{heres-what-you-need-to-know}{%
\subsubsection{Here's what you need to
know:}\label{heres-what-you-need-to-know}}

\begin{itemize}
\item
  \protect\hyperlink{tesla-earnings}{}

  Tesla earned \$104 million in the second quarter, surprising analysts.
\item
  \protect\hyperlink{elon-musk-salary}{}

  Elon Musk stands to get a \$2 billion stock award --- his second big
  payday this year.
\item
  \protect\hyperlink{tesla-s-and-p-500}{}

  Tesla could join the S\&P 500, and give its soaring stock price
  another jolt.
\item
  \protect\hyperlink{tesla-is-building-a-fourth-factory-in-texas}{}

  Tesla is building a fourth factory, in Texas.
\item
  \protect\hyperlink{microsoft-earnings}{}

  Microsoft revenue rose 13 percent despite the pandemic.
\item
  \protect\hyperlink{us-employment-has-declined-sharply-a-new-report-shows}{}

  U.S. employment has declined sharply, a new report shows.
\end{itemize}

\hypertarget{tesla-earned-104-million-in-the-second-quarter-surprising-analysts}{%
\subsection{\texorpdfstring{\protect\hyperlink{tesla-earnings}{Tesla
earned \$104 million in the second quarter, surprising
analysts.}}{Tesla earned \$104 million in the second quarter, surprising analysts.}}\label{tesla-earned-104-million-in-the-second-quarter-surprising-analysts}}

Copied to clipboard.

\includegraphics{https://static01.nyt.com/images/2020/07/22/business/22markets-brf-teslaearnings3/merlin_168394380_2ba1942c-587f-4305-b92e-f766f8812575-articleLarge.jpg?quality=75\&auto=webp\&disable=upscale}

\textbf{Tesla} on Wednesday reported a profit of \$104 million for the
three months ending in June.

The profit surprised analysts who were expecting the electric carmaker
to lose money because it was forced to halt production at its main plant
in Fremont, Calif., for nearly two months, from late March until the
middle of May. Sales also slowed as much of the economy shut down and as
millions of people lost their jobs and cut back on spending.

The profit was achieved ``despite tremendous difficulties in the
quarter,'' the company's chief executive, Elon Musk, said in a
conference call with analysts. ``We were able to achieve a fourth
consecutive profitable quarter. Although the auto industry was down
about 30 percent year-over-year, we managed to grow deliveries in the
first half of the year.''

In a statement, Tesla said
\href{https://ir.tesla.com/static-files/f41f4254-f1cc-4929-a0b6-6623b00475a6}{revenue
in the second quarter fell 5 percent}, to \$6 billion. Total sales of
automobiles declined 5 percent, to about 91,000 cars, an update to
preliminary figures it released earlier this month. Growing sales in
China and Europe helped cushion the pandemic's negative impact on sales
in the United States.

The company's profit was also made possible by the sales of \$428
million in emissions credits to other automakers who need them to meet
regulatory standards. That's nearly four times as many credits as it
sold in the same quarter a year earlier.

Tesla said it ended the quarter with \$8.6 billion in cash, up \$535
million from the end of March.

The company added that it now has the capacity to produce more than
500,000 cars a year. ``While achieving this goal has become more
difficult, delivering half a million vehicles in 2020 remains our
target,'' Tesla said.

Tesla appears to be weathering the pandemic better than some other
automakers. In China, the world's largest market for electric vehicles,
the company has benefited from a new factory near Shanghai that began
production late last year. The plant enables Tesla to avoid the tariffs
China imposes on imported vehicles and has made its cars more affordable
to Chinese consumers.

The company has also added to its lineup a fourth car, the Model Y
sport-utility vehicle, which is made in Fremont. Mr. Musk has said that
he expects the car to become its biggest seller.

--- \href{https://www.nytimes.com/by/neal-e-boudette}{Neal E. Boudette}

\hypertarget{elon-musk-stands-to-get-a-2-billion-stock-award--his-second-big-payday-this-year}{%
\subsection{\texorpdfstring{\protect\hyperlink{elon-musk-salary}{Elon
Musk stands to get a \$2 billion stock award --- his second big payday
this
year.}}{Elon Musk stands to get a \$2 billion stock award --- his second big payday this year.}}\label{elon-musk-stands-to-get-a-2-billion-stock-award--his-second-big-payday-this-year}}

Copied to clipboard.

\includegraphics{https://static01.nyt.com/images/2020/07/22/business/22markets-brf-telsacomp/22markets-brf-telsacomp-articleLarge.jpg?quality=75\&auto=webp\&disable=upscale}

Elon Musk, the chief executive of \textbf{Tesla}, could soon qualify for
his second giant payday of the year.

Mr. Musk's compensation is driven largely by the performance of Tesla's
stock. And as the carmaker's share price has soared in recent weeks, he
stands to receive a stock award worth roughly \$2 billion. The awards
are part of an unusual compensation package,
\href{https://www.nytimes.com/2018/01/23/business/dealbook/tesla-elon-musk-pay.html}{set
up in 2018}. The
\href{https://www.nytimes.com/2020/01/22/business/elon-musk-tesla-bonus.html}{first
payout} under that plan occurred
\href{https://www.sec.gov/Archives/edgar/data/1318605/000156459020027321/tsla-def14a_20200707.htm}{in
May} and now is also worth close to \$2 billion.

Tesla's stock has risen just over 275 percent this year, as investors
have become increasingly convinced that the company will have a
\href{https://www.nytimes.com/2020/07/22/business/tesla-electric-car-audi-polestar.html}{dominant
foothold} in the global market for electric vehicles. The rise in the
stock has bolstered Tesla's stock market value to around \$290 billion,
or \$100 billion more than Toyota's market value.

Mr. Musk's 2018 compensation package was designed to release shares in
12 installments as certain milestones are met. The first goal was for
Tesla's market value to be at least \$100 billion on average over two
different time periods --- that was achieved in January. Tesla also had
to hit operational milestones. Over 12 months, the company has to have
brought in a certain amount of revenue or a measure of profits called
earnings before interest, taxes, depreciation and amortization. In May,
Tesla said it had used the lowest revenue hurdle, \$20 billion, to
release the first tranche of shares.

Tesla's market value recently exceeded \$150 billion on average over the
past six months and over the last 30 trading days, the threshold for the
release of the second batch of shares. Tesla has already hit the lowest
profit goal without considering the company's second quarter results
released on Wednesday, in theory giving him the operational achievement
he needs to get the shares, though the board still has to release the
award. If Tesla's share price stays close to current levels, Mr. Musk
might even qualify for the third tranche of his stock awards this year.

Critics of the 2018 compensation package questioned why it was
necessary. Before the award, Mr. Musk already owned a large chunk of
Tesla --- shares that today are worth around \$60 billion. That's more
than twice the \$25 billion worth of shares available to Mr. Musk
through the 2018 package. Amazon's Jeff Bezos, another visionary chief
executive, has not needed multibillion compensation packages to motivate
him as he has led his company to become a dominant force in the American
economy.

--- \href{https://www.nytimes.com/by/peter-eavis}{Peter Eavis}

\hypertarget{advertisement}{%
\subsubsection{Advertisement}\label{advertisement}}

\protect\hyperlink{after-dfp-ad-mid1}{Continue reading the main story}

\hypertarget{tesla-could-join-the-sp-500-and-give-its-soaring-stock-price-another-jolt}{%
\subsection{\texorpdfstring{\protect\hyperlink{tesla-s-and-p-500}{Tesla
could join the S\&P 500, and give its soaring stock price another
jolt.}}{Tesla could join the S\&P 500, and give its soaring stock price another jolt.}}\label{tesla-could-join-the-sp-500-and-give-its-soaring-stock-price-another-jolt}}

Copied to clipboard.

A surprise profit in the second quarter has set Tesla up for another
major milestone: potential inclusion in the S\&P 500 index. The index is
one the most widely followed measures of the performance American stock
market, with more than \$11 trillion worth of mutual funds and other
investments measured against it.

The company said on Wednesday it earned \$104 million in the three
months through June, in its fourth consecutive quarter of profitability.

It's unusual for companies with market values as large as Tesla ---
roughly \$290 billion --- not to be included in the S\&P 500. But the
company's inability to consistently generate profits has made it
ineligible so far. (Criteria for inclusion require the sum of the
company's fully audited profits in the four most recent quarters to be
positive.)

The lack of profits hasn't bothered investors. Tesla's share price has
logged an astounding gain of more than 275 percent this year. But if
Tesla were to be included in the index, it could trigger another upward
push by stimulating a surge in demand for the shares by institutional
investors.

Index-based funds --- low cost investment vehicles designed to mirror
the performance of indexes like the S\&P 500, rather than trying to
``beat the market'' --- must buy any stock included in the index,
creating a rush for the shares of companies that are newly added.

``When a company goes in that means there's a lot of buying there,''
said Howard Silverblatt, senior index analyst at S\&P Dow Jones Indices,
the company that publishes the S\&P 500.

Changes to the index can, and do, occur regularly. For instance, when a
company is removed from the S\&P 500 after a merger or bankruptcy,
requiring a new addition. The additions can occur at any time and are
kept especially close to the vest by S\&P, because of the money making
opportunity someone could have if they learned about an inclusion before
everybody else.

``Nobody is supposed to know. The company isn't supposed to know
themselves,'' Mr. Silverblatt said. ``Nobody even calls them.''

--- \href{https://www.nytimes.com/by/matt-phillips}{Matt Phillips}

\hypertarget{tesla-is-building-a-fourth-factory-in-texas}{%
\subsection{\texorpdfstring{\protect\hyperlink{tesla-is-building-a-fourth-factory-in-texas}{Tesla
is building a fourth factory, in
Texas.}}{Tesla is building a fourth factory, in Texas.}}\label{tesla-is-building-a-fourth-factory-in-texas}}

Copied to clipboard.

\includegraphics{https://static01.nyt.com/images/2020/07/22/business/22markets-brf-teslafactory2/merlin_164920488_1ff5194b-9158-4baf-ad81-38a364d83a7d-articleLarge.jpg?quality=75\&auto=webp\&disable=upscale}

\textbf{Tesla} has started work on its fourth car factory at a site near
Austin, Texas, the company's chief executive, Elon Musk told analysts on
Wednesday.

The factory will produce a new electric pickup truck and a new semi
truck, along with the Model 3 and Model Y, which it already makes at a
factory in the San Francisco Bay Area. The new factory represents a
substantial investment for Tesla, which is already expanding a plant in
Shanghai and building another one near Berlin.

``We will be creating a massive
\href{https://www.nytimes.com/2019/11/21/business/tesla-cybertruck-pickup-truck.html}{Cybertruck}
and semi factory in Texas,'' Mr. Musk said, adding that the plant would
be open to the public and have a boardwalk, biking trails and bird
sanctuary.

Officials in Travis County, which includes Austin, this month approved a
tax break to recruit Tesla, which was also being courted by Oklahoma and
other states.

--- \href{https://www.nytimes.com/by/neal-e-boudette}{Neal E. Boudette}

\hypertarget{american-and-southwest-airlines-remove-medical-exceptions-to-rules-requiring-masks}{%
\subsection{\texorpdfstring{\protect\hyperlink{southwest-airlines-mask-policy}{American
and Southwest airlines remove medical exceptions to rules requiring
masks.}}{American and Southwest airlines remove medical exceptions to rules requiring masks.}}\label{american-and-southwest-airlines-remove-medical-exceptions-to-rules-requiring-masks}}

Copied to clipboard.

\includegraphics{https://static01.nyt.com/images/2020/07/22/business/22markets-brf-southwest/merlin_174169380_b81ce50f-254b-445a-80d3-2e903e37376a-articleLarge.jpg?quality=75\&auto=webp\&disable=upscale}

\textbf{American Airlines} and \textbf{Southwest Airlines} on Wednesday
became the first major airlines in the United States to broaden their
mask requirements to include passengers with a medical condition or
disability that would otherwise prevent them from wearing one.

``If a customer is unable to wear a face covering or mask for any
reason, Southwest regrets that we will be unable to transport the
individual,'' the airline said in a statement, noting that the virus can
be spread by individuals who are unaware that they have been infected.

American followed suit, saying: ``All customers must wear a face
covering from the time they enter their departure airport and not remove
it until they exit their arrival airport.''

The airlines said that children under the age of 2 will still be allowed
to fly without a mask, a policy in line with other major carriers.
\textbf{Delta Air Lines} still allows exceptions for individuals with a
disability or medical condition that prevents them from wearing a mask,
and \textbf{United Airlines} says individuals seeking exemptions should
reach out to its staff.

The Centers for Disease Control and Prevention
\href{https://www.cdc.gov/coronavirus/2019-ncov/prevent-getting-sick/how-to-wear-cloth-face-coverings.html}{recommends
against masks} for children under 2 years old and people with a medical
condition or disability.
\href{https://www.nytimes.com/2020/07/18/health/coronavirus-children-schools.html}{A
recent study} indicated that children under 10 are less likely to
transmit the virus than adults and older children, though the risk is
not zero.

Southwest's policy goes into effect on Monday, while American's will
start on July 29. Earlier in the day, United
\href{https://www.nytimes.com/live/2020/07/22/business/stock-market-today-coronavirus/united-airlines-to-require-masks-in-airports-too}{said
it would extend its requirement} for masks on planes to any area it
operated in an airport. Southwest and Delta already had such policies in
place.

Southwest and American are also expected to release financial results
from the industry's devastating second quarter on Thursday morning.
United said on Tuesday that its operating revenue
\href{https://www.nytimes.com/live/2020/07/21/business/stock-market-today-coronavirus/united-earnings}{declined
88 percent during the quarter} from a year earlier, leading to a \$1.6
billion loss. Delta said last week that its quarterly revenue had
dropped 87 percent, resulting in a \$5.7 billion loss.

--- \href{https://www.nytimes.com/by/niraj-chokshi}{Niraj Chokshi}

\hypertarget{advertisement-1}{%
\subsubsection{Advertisement}\label{advertisement-1}}

\protect\hyperlink{after-dfp-ad-mid2}{Continue reading the main story}

\hypertarget{microsoft-revenue-rose-13-percent-despite-the-pandemic}{%
\subsection{\texorpdfstring{\protect\hyperlink{microsoft-earnings}{Microsoft
revenue rose 13 percent despite the
pandemic.}}{Microsoft revenue rose 13 percent despite the pandemic.}}\label{microsoft-revenue-rose-13-percent-despite-the-pandemic}}

Copied to clipboard.

\includegraphics{https://static01.nyt.com/images/2020/07/22/business/22markets-brf-microsoft/merlin_154461804_6aa7ee21-05ee-47ef-911f-76b8f98a45d7-articleLarge.jpg?quality=75\&auto=webp\&disable=upscale}

\textbf{Microsoft} on Wednesday said its revenue rose 13 percent in the
last quarter despite the slump in the economy. The company's growth was
led by big gains in its cloud software offerings as more people work
from home.

The company is not immune to shocks from the pandemic. Technology
spending has fallen in industries like travel and retail. LinkedIn, the
hiring and professional networking site owned by Microsoft, said on
Tuesday that it was cutting 962 jobs, or 6 percent of its work force,
partly because hiring has fallen sharply. In June, Microsoft announced
it was shutting down its 83 retail stores, taking a \$450 million charge
against earnings, or 5 cents a share.

But the weaknesses were more than offset by higher demand for its cloud
businesses including its cloud processing and storage services, known as
Azure, and its Office 365 productivity programs.

For the three months ended in June, its fiscal fourth quarter, Microsoft
generated revenue of \$38 billion. Its operating profit increased 8
percent to \$13.4 billion, or \$1.46 a share. Both the company's sales
and earnings per share surpassed Wall Street estimates.

--- \href{https://www.nytimes.com/by/steve-lohr}{Steve Lohr}

\hypertarget{the-paycheck-protection-program-saved-jobs-but-at-a-high-cost-a-study-shows}{%
\subsection{\texorpdfstring{\protect\hyperlink{the-paycheck-protection-program-saved-jobs-but-at-a-high-cost-a-study-shows}{The
Paycheck Protection Program saved jobs, but at a high cost, a study
shows.}}{The Paycheck Protection Program saved jobs, but at a high cost, a study shows.}}\label{the-paycheck-protection-program-saved-jobs-but-at-a-high-cost-a-study-shows}}

Copied to clipboard.

\includegraphics{https://static01.nyt.com/images/2020/07/22/business/22markets-brf-ppp/merlin_174644982_89ba8fb5-9bc9-4dbb-bb3b-d6cea11b42c3-articleLarge.jpg?quality=75\&auto=webp\&disable=upscale}

As Congress struggles to advance negotiations over a new round of
federal spending to help people and businesses endure the
pandemic-induced recession, new research suggests a previous round of
aid helped save millions of jobs --- but at high cost.

The Paycheck Protection Program, which lawmakers created in March, saved
1.4 million to 3.2 million jobs in small businesses through the
beginning of June, according to research released on Wednesday by
economists from the Massachusetts Institute of Technology, the Federal
Reserve and ADP Research Institute. That works out to a cost of
\$162,000 to \$381,000 per job.

Some companies that accepted assistance through the program have emerged
in better financial condition and no longer need federal help. But many
companies have not seen the uptick in consumer demand that Republicans
were counting on reopening plans to deliver --- and thus could be
vulnerable to closure and layoffs without more aid.

``It's plausible that, when the money runs out, some of those firms
would downsize again,'' said David Autor, an M.I.T. economist and a lead
author of the study.

``This was actually a very aggressive policy,'' Mr. Autor said in an
interview. ``It's useful to know that when Congress sets out and spends
a half a trillion dollars, it can get something done.''

The new findings appear to run counter to another recent paper from a
team of prominent economists at Harvard and Brown, which concluded that
``the P.P.P. had little material impact on employment at small
businesses.''

That paper used similar methods to those employed by Mr. Autor and his
co-authors. But it used a different, much smaller set of data, from a
financial management application used primarily by low-wage workers. Mr.
Autor said it was possible that the federal loan program did not do much
to help those people, most of whom cannot work from home, even as it
succeeded in bringing back jobs elsewhere in the economy.

Trump administration officials had claimed in a news release earlier
this month that the program supported ``over 51 million jobs.'' Mr.
Autor said that the idea that a much larger number of jobs saved was
``not right,'' because much of the money appears to have gone to help
businesses that would not likely have folded shop without aid.

--- \href{https://www.nytimes.com/by/jim-tankersley}{Jim Tankersley} and
\href{https://www.nytimes.com/by/ben-casselman}{Ben Casselman}

\hypertarget{us-employment-has-declined-sharply-a-new-report-shows}{%
\subsection{\texorpdfstring{\protect\hyperlink{us-employment-has-declined-sharply-a-new-report-shows}{U.S.
employment has declined sharply, a new report
shows.}}{U.S. employment has declined sharply, a new report shows.}}\label{us-employment-has-declined-sharply-a-new-report-shows}}

Copied to clipboard.

For several weeks, real-time data has suggested that the U.S. economic
recovery could be stalling. Now there is evidence it could be going in
reverse.

\href{https://www.census.gov/data/tables/2020/demo/hhp/hhp11.html}{Data
from the Census Bureau} on Wednesday showed that the number of employed
people fell by more than four million last week, the fourth-straight
weekly decline. Taken literally, the results indicate that the economy
has given up all the job gains since mid-May, before the recent surge in
coronavirus cases.

Just under 52 percent of American adults were employed last week,
according the survey, down from 54 percent in June.

The data comes from the bureau's weekly Household Pulse Survey, an
experimental effort to track the pandemic's economic impact. The survey
has a brief track record, but a
\href{https://twitter.com/ernietedeschi/status/1285940771892482050}{good
one}: It correctly signaled the big increase in employment in the
\href{https://www.nytimes.com/2020/07/02/business/economy/jobs-unemployment-coronavirus.html}{jobs
report} for June.

The latest data corresponds to the survey week for the July report,
which will be released in early August. If the results hold up again, it
suggests that report could show a loss of millions of jobs, just as
enhanced unemployment benefits from the federal government are
\href{https://www.nytimes.com/2020/07/02/business/economy/jobs-unemployment-coronavirus.html}{in
danger of expiring}.

--- \href{https://www.nytimes.com/by/ben-casselman}{Ben Casselman}

\hypertarget{advertisement-2}{%
\subsubsection{Advertisement}\label{advertisement-2}}

\protect\hyperlink{after-dfp-ad-mid3}{Continue reading the main story}

\hypertarget{ending-the-600-weekly-unemployment-benefit-could-hurt-more-than-the-jobless}{%
\subsection{\texorpdfstring{\protect\hyperlink{ending-the-600-weekly-unemployment-benefit-could-hurt-more-than-the-jobless}{Ending
the \$600 weekly unemployment benefit could hurt more than the
jobless.}}{Ending the \$600 weekly unemployment benefit could hurt more than the jobless.}}\label{ending-the-600-weekly-unemployment-benefit-could-hurt-more-than-the-jobless}}

Copied to clipboard.

\includegraphics{https://static01.nyt.com/images/2020/07/21/business/21markets-brf-cliff-01/merlin_174779925_25f5846e-caf4-4a94-ad43-e28f9a67a9b7-articleLarge.jpg?quality=75\&auto=webp\&disable=upscale}

If the \$600 weekly federal supplement to unemployment benefits expires,
more than 20 million Americans could soon see their weekly income fall
by half. But it won't just be individual recipients who will suffer,
\href{https://www.nytimes.com/2020/07/21/business/economy/coronavirus-unemployment-benefits.html}{Ben
Casselman reports}:

\begin{quote}
The federal payments are injecting billions of dollars into the economy
each week, money that flows to landlords, grocery stores, retailers and
countless other businesses.

Ernie Tedeschi, a former Treasury Department official and an economist
at Evercore ISI Research, has estimated that if the payments ceased, the
United States gross domestic product would be 2 percent smaller at the
end of 2020 and there would be 1.7 million fewer jobs nationwide.

Congress returned from recess this week to consider a new relief
package,
which\href{https://www.nytimes.com/2020/07/20/us/politics/congress-coronavirus-aid-package.html}{could
include at least a partial extension} of the extra unemployment
benefits. Senate Republicans and the White House are considering a
roughly \$1 trillion package that would retain the program but scale it
back. Democrats are pressing to continue paying the full \$600 per week.

But Congress seems unlikely to act before benefits lapse.

``These unemployment benefit checks are really doing a large job in
propping up spending by these unemployed households,'' said Joseph
Vavra, a University of Chicago economist. If they expire, he said,
``there's a good chance that what is now an unemployment problem becomes
a foreclosure crisis and eviction crisis.''
\end{quote}

\hypertarget{wall-street-continues-its-rally-shaking-off-rising-tension-between-us-and-china}{%
\subsection{\texorpdfstring{\protect\hyperlink{wall-street-continues-its-rally-shaking-off-rising-tension-between-us-and-china}{Wall
Street continues its rally, shaking off rising tension between U.S. and
China.}}{Wall Street continues its rally, shaking off rising tension between U.S. and China.}}\label{wall-street-continues-its-rally-shaking-off-rising-tension-between-us-and-china}}

Copied to clipboard.

\includegraphics{https://static01.nyt.com/images/2020/07/22/world/22markets-brf-markets-china/22china-diplo-articleLarge.jpg?quality=75\&auto=webp\&disable=upscale}

Stocks on Wall Street rose on Wednesday, but the gains were constrained
by rising tension between the United States and China.

After an early dip, the S\&P 500 rose more than half a percent. Shares
in Europe and Asia were mostly lower.

Relations between the United States and China, two giant trading
partners, have been worsening recent weeks, as the Trump administration
has tightened the reins on Chinese diplomats, journalists, scholars and
others in the United States. In the latest action, the
\href{https://www.nytimes.com/2020/07/22/world/asia/us-china-houston-consulate.html?action=click\&module=Top\%20Stories\&pgtype=Homepage}{White
House told China}to leave the its consulate in Houston by Friday. China
warned that it might retaliate.

News of the consulate's closure had an immediate impact in financial
markets, with stock futures falling and trading in Treasury notes, gold
and oil also reflecting a jolt of nervousness.

But investors on Wall Street have shaken off a number of concerns
lately, including about the surge in coronavirus cases and deaths in the
United States. On Tuesday,
\href{https://www.nytimes.com/2020/07/22/world/coronavirus-covid-19.html}{President
Trump, in a shift from his usual rosy forecasts, told reporters} that
the outbreak would probably ``get worse before it gets better.'' And a
valuable economic lifeline for millions of Americans --- \$600 a week in
extra unemployment benefits ---
\href{https://www.nytimes.com/2020/07/21/business/economy/coronavirus-unemployment-benefits.html}{is
about to expire}if Congress doesn't extend it.

Recent gains have come as lawmakers in Washington haggle over another
economic aid package, and as some large businesses have reported better
than expected results, or signs of improvement.

On Wednesday, for example, shares of \textbf{Best Buy} jumped almost 8
percent after the electronics retailer
\href{https://www.nytimes.com/live/2020/07/21/business/stock-market-today-coronavirus/best-buy-to-join-retailers-paying-a-15-minimum-wage}{reported
that sales were rebounding as stores reopened}. Also sharply higher
Wednesday was \textbf{Pfizer}, which rose more than 5 percent after the
Trump administration said it would pay nearly
\href{https://www.nytimes.com/2020/07/22/us/politics/pfizer-gets-1-95-billion-to-produce-covid-19-vaccine-by-years-end.html}{\$2
billion}for up to 600 million doses of a Covid-19 vaccine. Shares of
\textbf{BioNTech,} a German company that is developing the vaccine with
Pfizer, were up nearly 14 percent.

--- \href{https://www.nytimes.com/by/kevin-granville}{Kevin Granville}

\hypertarget{were-tracking-which-chains-are-requiring-customers-to-wear-masks}{%
\subsection{\texorpdfstring{\protect\hyperlink{were-tracking-which-chains-are-requiring-customers-to-wear-masks}{We're
tracking which chains are requiring customers to wear
masks.}}{We're tracking which chains are requiring customers to wear masks.}}\label{were-tracking-which-chains-are-requiring-customers-to-wear-masks}}

Copied to clipboard.

\includegraphics{https://static01.nyt.com/images/2020/07/21/business/21virus-retailermasks2-copyForBriefing/merlin_172644567_eec8b61b-057c-4cc2-a093-fca5d5bdafc8-articleLarge.jpg?quality=75\&auto=webp\&disable=upscale}

After \textbf{Walmart}, America's largest retailer, announced on
\href{https://corporate.walmart.com/newsroom/2020/07/15/a-simple-step-to-help-keep-you-safe-walmart-and-sams-club-require-shoppers-to-wear-face-coverings}{July
15} that it would mandate in-store mask-wearing, a flurry of other
companies, including \textbf{Kroger, Target} and \textbf{Walgreens},
followed suit. This means that customers will be required to wear face
masks in stores even in places without local mask ordinances.

The
\href{https://nrf.com/media-center/press-releases/nrf-calls-retailers-set-nationwide-mask-policy}{National
Retail Federation} has encouraged companies to set nationwide mask
policies to protect employees and shoppers.

Some chains, however, have moved in the opposite direction. After
putting in place a customer mask requirement nearly two weeks ago,
\textbf{Dollar Tree} and \textbf{Family Dollar} reversed course on July
20, saying they would require masks only if mandated by state or local
rules.

--- Gillian Friedman

\hypertarget{catch-up-for-profit-hospitals-make-profits-and-united-airlines-expects-a-slow-recovery}{%
\subsection{\texorpdfstring{\protect\hyperlink{catch-up-for-profit-hospitals-make-profits-and-united-airlines-expects-a-slow-recovery}{Catch
up: For-profit hospitals make profits, and United Airlines expects a
slow
recovery.}}{Catch up: For-profit hospitals make profits, and United Airlines expects a slow recovery.}}\label{catch-up-for-profit-hospitals-make-profits-and-united-airlines-expects-a-slow-recovery}}

Copied to clipboard.

\begin{itemize}
\item
  💉 With profit bolstered by hundreds of millions of dollars in federal
  stimulus money, \textbf{HCA Healthcare}, the giant for-profit hospital
  chain,
  reported\href{https://investor.hcahealthcare.com/news/news-details/2020/HCA-Healthcare-Reports-Second-Quarter-2020-Results/default.aspx}{much
  higher second-quarter earnings} on Wednesday, even as its revenue fell
  when its huge network of hospitals treated fewer patients during the
  pandemic. The company reported \$1.1 billion in net income for the
  three months that ended June 30, a 38 percent jump from the same
  period in 2019, on lower revenue of \$11.1 billion. HCA, already
  \href{https://www.nytimes.com/2020/06/08/business/hospitals-bailouts-ceo-pay.html}{a
  major beneficiary} of hospital bailout money, said it had received a
  total of \$1.7 billion from the federal government so far.
\item
  🛬 \textbf{United Airlines} said its revenue will max out at about 50
  percent of last year's haul if a vaccine does not become available.
  The airline expects passenger revenue in July, August and September to
  be down about 83 percent from the same period last year, a slight
  improvement over the nearly 94 percent decline the airline
  \href{https://www.nytimes.com/live/2020/07/21/business/stock-market-today-coronavirus\#united-earnings}{reported
  for the second quarter on Tuesday}.
\end{itemize}

\hypertarget{site-index}{%
\subsection{Site Index}\label{site-index}}

\hypertarget{site-information-navigation}{%
\subsection{Site Information
Navigation}\label{site-information-navigation}}

\begin{itemize}
\tightlist
\item
  \href{https://help.nytimes.com/hc/en-us/articles/115014792127-Copyright-notice}{©~2020~The
  New York Times Company}
\end{itemize}

\begin{itemize}
\tightlist
\item
  \href{https://www.nytco.com/}{NYTCo}
\item
  \href{https://help.nytimes.com/hc/en-us/articles/115015385887-Contact-Us}{Contact
  Us}
\item
  \href{https://www.nytco.com/careers/}{Work with us}
\item
  \href{https://nytmediakit.com/}{Advertise}
\item
  \href{http://www.tbrandstudio.com/}{T Brand Studio}
\item
  \href{https://www.nytimes.com/privacy/cookie-policy\#how-do-i-manage-trackers}{Your
  Ad Choices}
\item
  \href{https://www.nytimes.com/privacy}{Privacy}
\item
  \href{https://help.nytimes.com/hc/en-us/articles/115014893428-Terms-of-service}{Terms
  of Service}
\item
  \href{https://help.nytimes.com/hc/en-us/articles/115014893968-Terms-of-sale}{Terms
  of Sale}
\item
  \href{https://spiderbites.nytimes.com}{Site Map}
\item
  \href{https://help.nytimes.com/hc/en-us}{Help}
\item
  \href{https://www.nytimes.com/subscription?campaignId=37WXW}{Subscriptions}
\end{itemize}
